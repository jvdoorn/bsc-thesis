% !TEX root = ../../thesis.tex
\chapter{Introduction}
Josephson junctions have a wide variety of applications. Notable applications are qubits\cite{placeNewMaterialPlatform2021,pechenezhskiySuperconductingQuasichargeQubit2020}, dissipationless electronics through Josephson diodes\cite{zhangReconfigurableMagneticfieldfreeSuperconducting2023a,ciacciaGateTunableJosephson2023} and microscopic imaging techniques\cite{clarkeSQUIDHandbook2004,rogSQUIDontipMagneticMicroscopy2022,pranceSensitivityDCSQUID2023}. The behaviour of a Josephson junctions is governed by their current-phase relation (CPR). Probing the CPR can lead to new insights and applications. By measuring the CPR it is possible to if the junction's behaviour is ballistic or diffusive\cite{endresCurrentPhaseRelation2023,kayyalhaHighlySkewedCurrent2020}. Additionally, it has proven the existence of $0$-$\pi$ and $\varphi_0$ junctions\cite{frolovMeasurementCurrentPhaseRelation2004,muraniBallisticEdgeStates2017,strambiniJosephsonPhaseBattery2020,szombatiJosephsonPh0junctionNanowire2016} as well as non-$2\pi$ periodic CPRs\cite{endresCurrentPhaseRelation2023}.

In our group there is an interest in the CPR of rings of \ce{Sr2RuO4}. Recent work by Lahabi \textit{et al.} provides evidence for the existence of chiral domain walls in homogenous rings of \ce{Sr2RuO4}\cite{lahabiSpintripletSupercurrentsOdd2018} that act as Josephson junctions. As such \ce{Sr2RuO4} rings show dc-SQUID like behaviour without the presence of constrictions, grain boundaries or an interface with a different material. More definitive proof for chiral domain walls could be found by measuring the Josephson energy\cite{lahabiSpintripletSupercurrentsOdd2018,sigristRoleDomainWalls1999}. The most elegant way to determine the Josephson energy is to measure the CPR.

This thesis utilizes a method based on the work of Frolov \textit{et al.}. The reader is referred to~\cite{frolovMeasurementCurrentPhaseRelation2004,frolovCurrentphaseRelationsJosephson2005} for their work. We explore a method to measure the current-phase relation of a single Josephson junction which, if successful, can be extended in later studies to measure the current-phase relation of \ce{Sr2RuO4} rings.

The next chapter will lay a theoretical foundation for our method. Chapter~\ref{chapter:method} delves deeper into the method and presents numerical calculations to guide our expectations. Then our results are presented on a per sample. Finally a conclusion is drawn and we sketch an outlook for future research.