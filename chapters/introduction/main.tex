% !TEX root = ../../thesis.tex
\chapter{Introduction}
Josephson junctions have a wide variety of applications such as qubits\cite{placeNewMaterialPlatform2021,pechenezhskiySuperconductingQuasichargeQubit2020}; superconducting electronics through Josephson diodes\cite{zhangReconfigurableMagneticfieldfreeSuperconducting2023a,ciacciaGateTunableJosephson2023}; and microscopic imaging techniques\cite{clarkeSQUIDHandbook2004,rogSQUIDontipMagneticMicroscopy2022,pranceSensitivityDCSQUID2023}. The behaviour of a Josephson junction is governed by its current-phase relation (CPR). Probing the CPR can lead to new insights and applications. Par example by measuring the CPR it is possible to if the junction's behaviour is ballistic or diffusive\cite{muraniBallisticEdgeStates2017,endresCurrentPhaseRelation2023,kayyalhaHighlySkewedCurrent2020} and it has shown the existence of $0$-$\pi$ and $\varphi_0$ junctions\cite{frolovMeasurementCurrentPhaseRelation2004,muraniBallisticEdgeStates2017,strambiniJosephsonPhaseBattery2020,szombatiJosephsonPh0junctionNanowire2016} as well as non-$2\pi$ periodic CPRs\cite{endresCurrentPhaseRelation2023}.

In our group there is an additional interest in the CPR of homogenous \ce{Sr2RuO4} rings. Recent work by Lahabi \textit{et al.} provides evidence for the existence of chiral domain walls in these rings that act as Josephson junctions.\cite{lahabiSpintripletSupercurrentsOdd2018} As such, homogenous \ce{Sr2RuO4} rings show dc-SQUID like behaviour without the presence of constrictions, grain boundaries or an interface with a different material. Definitive proof for chiral domain walls can be found by measuring the Josephson energy.\cite{lahabiSpintripletSupercurrentsOdd2018,sigristRoleDomainWalls1999} The most elegant way to determine the Josephson energy is by measuring the CPR.

Furthermore, our group has been characterising the behaviour of Josephson junctions containing \ce{La_{0.7}Sr_{0.3}MnO3} (LSMO). LSMO is a half-metallic ferromagnet and  spin triplet supercurrents were observed. Transport measurements have been performed and the magnetic field dependence of the critical current was determined. However, there are still several open questions regarding the observed behaviours and their origins.\cite{yaoSpinTransportSuperconductivity2023} New insights might be acquired by measuring the current-phase relation.

Two of the key benefits of our method is that it directly measures the full CPR and only needs a simple analysis. Details on the analysis can be found in Section~\ref{sec:analysis-method}. An alternative method uses a strongly asymmetric dc-SQUID where the junction with a much smaller critical current dominates the behaviour of the dc-SQUID.\cite{muraniBallisticEdgeStates2017,dellaroccaMeasurementCurrentPhaseRelation2007} A downside of this method is that it is much more difficult to produce the samples. Microwave measurements have been used to determine the skewness of the CPR but cannot probe the CPR directly.\cite{schmidtProbingCurrentphaseRelation2020}

%This thesis utilises a method based on the work of Frolov \textit{et al.}. The reader is referred to~\cite{frolovMeasurementCurrentPhaseRelation2004,frolovCurrentphaseRelationsJosephson2005} for their work. We explore a method to measure the current-phase relation of a single Josephson junction which, if successful, can be extended in later studies to measure the current-phase relation of \ce{Sr2RuO4} rings.

The next chapter will lay a theoretical foundation for our method. Chapter~\ref{chapter:method} delves deeper into our method, presents numerical calculations to guide our expectations. Then our results are presented on a per sample. Finally a conclusion is drawn and we sketch an outlook for future research.