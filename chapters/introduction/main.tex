% !TEX root = ../../thesis.tex
Recent work by Lahabi \textit{et al.} provides evidence for the existence of chiral domain walls in rings of \ce{Sr2RuO4}\cite{lahabiSpintripletSupercurrentsOdd2018}. Due to these domain walls, homogenous \ce{Sr2Ru04} rings show dc-SQUID like behaviour without Josephson junctions\footnote{There are no constrictions or different materials in the ring, hence it is homogenous. In this case the hypothesis is that the domain wall acts as a Josephson junction.}. More definitive proof for chiral domain walls could be found by measuring the Josephson energy. A prerequisite to determining the Josephson energy is measuring the current-phase relation\footnote{The phase here refers to the phase order parameter in Ginzburg-Landau theory.}.

This thesis presents a method based on \citeauthor{frolovMeasurementCurrentPhaseRelation2004} (\citeyear{frolovMeasurementCurrentPhaseRelation2004}). We explore a method to measure the current-phase relation of a single Josephson junction which, if successful, can be extended in later studies to measure the current-phase relation of a \ce{Sr2RuO4} ring.

In this chapter we will present relevant theory for our experiment. In the next chapter we will apply this theory and explain our methods.

\section{Superconductors}
The most well known property of superconductors are their perfect conductivity\footnote{Discovered by H.K. Onnes in 1911.}. Later it was discovered that they are also a perfect diamagnet and expel magnetic fields\footnote{Discovered by W. Meissner and R. Ochsenfeld in 1933.}. A microscopic description of the effect is given by Bardeen, Cooper and Schieffer (BCS theory) and phenomenologically by the Ginzburg-Landau theory\cite{tinkhamIntroductionSuperconductivity}. This section will highlight the relevant parts of these theories for our research.

BCS theory describes the formation of Cooper pairs. These form from an attractive interaction that overcomes the Coulomb repulsion between two electrons\cite{bardeenTheorySuperconductivity1957}. Electrons have half-integer spin, which means a Cooper pair, consisting of two electrons, has integer spin. Hence Cooper pairs are bosons.

Bosons, contrary to fermions, can occupy the same quantum state. At low temperatures bosons can condense into a condensate. That  means that Cooper pairs too can form a condensate. All bosons in the condensate are described by the same wavefunction\footnote{This is by definition of a `condensate' the case.}.
\begin{equation}
	\Psi = \left|\Psi\right| \exp(i\phi)
	\label{eqn:GL-wavefunction}
\end{equation}
Both $\left|\Psi\right|$ and $\phi$ are functions of position. The behaviour of this wave function is described by the Ginzburg-Landau theory. In this theory we can view $|\Psi|^2$ as the density of Cooper pairs (units of \unit{\per\cubic\meter}). The gradient in $\phi$ causes a supercurrent to flow, it will become important for the current-phase relation introduced in Sec. \ref{sec:josephson-effect}.

\section{Josephson effect}
\label{sec:josephson-effect}
When two superconductors are separated by a (thin) barrier\footnote{This can be an insulator, normal metal, different superconductor or a constriction.} a supercurrent can flow between them. Josephson showed in 1962 that for two superconductors separated by an insulator the current is given by\cite{tinkhamIntroductionSuperconductivity}:
\begin{equation}
	I_s = I_c \sin(\Delta \phi)
\end{equation}
Where $\Delta \phi$ is the difference in phase between the two condensates as described by Ginzburg-Landau theory, see Eq. \ref{eqn:GL-wavefunction}. Furthermore $I_c$ is the critical current which is a junction property. This equation is generally known as the first Josephson equation. The relation between $I_s$ and the phase difference is the current-phase relation. In general this does not have to be purely sinusoidal\cite{golubovCurrentphaseRelationJosephson2004a}.

In the more general case we first define the gauge invariant phase\footnote{See \citetitle{tinkhamIntroductionSuperconductivity} equation 6.11, this equation is valid in both Gaussian and SI units, this is because the conversion factor for $\Phi_0$ cancels with the conversion factor for $\vec{A}$.}:
\begin{equation}
	\gamma = \Delta \varphi - \frac{2\pi}{\Phi_0}\int \vec{A} \cdot d\vec{l}
	\label{eqn:gauge-invariant-phase}
\end{equation}
We are required to do so as $\Delta \phi$ is not uniquely determined for a given physical situation whilst $I_s$ is\cite{tinkhamIntroductionSuperconductivity}. It simply transforms $I_c \sin(\Delta \phi) \to I_c \sin(\gamma)$. To now generalize our current-phase relation we write:
\begin{equation}
	I_s = I_c f(\gamma)
\end{equation}
Here we have defined $f(\gamma)$ which is the current-phase relation. In general it has the following properties\cite{golubovCurrentphaseRelationJosephson2004a}:
\begin{equation}
	f(\gamma) = f(\gamma + 2\pi) \quad f(\gamma) = -f(-\gamma) \quad f(2\pi n) = f(\pi m) = 0
\end{equation}
With $m,n \in \mathcal{N}$.

\subsection{Characteristic length scales}
\label{sec:characteristic-length-scales}
There are two important length scales for superconductors. We will focus on these length scales mainly in relation to the Ginzburg-Landau theory. See Figure \ref{fig:characteristic-lengths} for a schematic depiction.

\begin{figure}[h]
	\centering
	\def\svgwidth{\textwidth}
	\import{figures}{characterstic_lengths.pdf_tex}
	\caption{Schematic depiction of the characteristic lengths $\xi$ and $\lambda$. The Cooper-pair density $|\Psi(x)|^2$, also referred to as $n_s$, falls off on a scale $\xi$. The magnetic field gets shielded by the superconductor using a shielding current and falls off on a scale $\lambda$. The S and N denote the `superconducting' and `normal' regimes respectively.}
	\label{fig:characteristic-lengths}
\end{figure}

The first is the scale over which the Cooper-pair density $|\Psi|^2$ can change. This is the so called coherence length. In the Ginzburg-Landau theory it is given by\cite{tinkhamIntroductionSuperconductivity}:
\begin{equation}
	\xi(T) = \frac{\hbar}{|4m_e\alpha(T)|^{1/2}}
\end{equation}
With $\alpha \propto 1 - T/T_c$. The coherence length plays an important role in constriction junctions.

The second length scale determines how deep the magnetic field penetrates into the superconductor. We denote this penetration depth using $\lambda$. The expulsion of magnetic fields is called the Meissner effect\cite{tinkhamIntroductionSuperconductivity}. The penetration depth in Ginzburg-Landau theory at \qty{0}{\kelvin} is given by\footnote{See \citetitle{tinkhamIntroductionSuperconductivity} equation 4.8, the equation has been converted to SI units, $|\psi|^2$ has units of \unit{\per\cubic\meter}.}:
\begin{align}
	\lambda(0) &= \sqrt{\frac{m^*c^2}{4\pi|\psi|^2e^{*2}}} \stackrel{\text{SI}}{=} \sqrt{\frac{m_e}{2|\psi|^2e^2\mu_0}}
	\label{eqn:london-penetration-depth}
\end{align}
Furthermore $\lambda(T) = \lambda(0) (1-(T/T_c)^4)^{-1/2}$\cite{tinkhamIntroductionSuperconductivity}. The penetration depth is indirectly also a measure for how deep the screening current occur in the superconductor. For a sufficiently thick superconductor this means that there is no screen current on the inside, similar to surface charge on a normal conductor. This will become important later on for certain assumptions in our method.

\section{dc-SQUID magnetometers}
% - Basic description of what they are
% - Basic explanation how they can be used to measure magnetic fields
A dc-SQUID magnetometer consists of a superconducting loop with two Josephson junctions. See Figure \ref{fig:schematic-dc-SQUID}. The basic idea behind a dc-SQUID is to run a bias-current $I_B$ through it. This causes a voltage $V_s$ across the device which also depends on the flux $\Phi_s$\cite{rogSQUIDontipMagneticMicroscopy2022}. $I_B$ is typically just above $2I_c$ where $I_c$ is the critical current of a single junction.

\begin{figure}
	\centering
	\begin{circuitikz}
		% Loop with a dc-SQUID.
		\draw (0, 0) to [josephsonjunction, i=$I_1$] (2, 0)
		to [short] (2, -2)
		to [josephsonjunction, i=$I_2$] (0, -2)
		to [short] (0, 0);
		% Wires to the sides of the dc-SQUID.
		\draw (-1, -1) to [short, *-, i=$I_s$] (0, -1);
		\draw (2, -1) to [short, -*, i=$I_s$] (3, -1);

		% Annotate flux through the loop
		\node[] at (1,-1) {$\Phi_s$};
		% Annotate V+ and V-
		\node[] at (-1, -1.4) {$V_+$};
		\node[] at (3, -1.4) {$V_-$};
	\end{circuitikz}

	\caption{Schematic depiction of a dc-SQUID. The loop contains two Josephson junctions (denoted with the crosses). The total current $I_B = I_1 - I_2$ and a voltage $V_s = V_+ - V_-$ can be measured between the two contacts.}
	\label{fig:schematic-dc-SQUID}
\end{figure}

