% !TEX root = ../../thesis.tex
TODO: reference to SRO rings, explain that our method will provide the possibility to measure its CPR and thus determine which theory is correct as multiple theories say something else.

\section{Superconductors}
The most well known property of superconductors are their perfect conductivity\footnote{Discovered by H.K. Onnes in 1911.}. Later it was discovered that they are also a perfect diamagnet and expel magnetic fields\footnote{Discovered by W. Meissner and R. Ochsenfeld in 1933.}. A microscopic description of the effect is given by Bardeen, Cooper and Schieffer (BCS theory) and phenomenologically by the Ginzburg-Landau theory\cite{tinkhamIntroductionSuperconductivity}. This section will highlight the relevant parts of these theories for our research.

BCS theory describes the formation of Cooper pairs. These form from an attractive interaction that overcomes the Coulomb repulsion between two electrons\cite{bardeenTheorySuperconductivity1957}. Electrons have half-integer spin, which means a Cooper pair, consisting of two electrons, has integer spin. Hence Cooper pairs are bosons.

Bosons, contrary to fermions, can occupy the same quantum state. At low temperatures bosons can condense into a condensate. That  means that Cooper pairs too can form a condensate. All bosons in the condensate are described by the same wavefunction\footnote{This is by definition the case.}.
\begin{equation}
	\Psi = \left|\Psi\right| \exp(i\phi)
	\label{eqn:GL-wavefunction}
\end{equation}
Both $\left|\Psi\right|$ and $\phi$ are functions of position. The behaviour of this wave function is described by the Ginzburg-Landau theory. In this theory we can view $|\Psi|^2$ as the density of Cooper pairs (units of \unit{\per\cubic\meter}). The gradient in $\phi$ causes a supercurrent to flow, it will become important for the current-phase relation introduced in Sec. \ref{sec:josephson-effect}.

\subsection{Meissner effect}
As mentioned before, superconductors expel magnetic fields\footnote{This is true for bulk superconductors.}. They do so by creating a screening current on the outside of the superconductor. The associated length scale for how deep those screen currents occur in the material is the so called London penetration depth\footnote{See \citetitle{tinkhamIntroductionSuperconductivity} equation 4.8, the equation has been converted to SI units, $|\psi|^2$ has units of \unit{\per\cubic\meter}.}\cite{tinkhamIntroductionSuperconductivity}:
% - Superconductors expell magnetic field
% - Explained by London equations
% - Give rise to London penetration depth
\begin{align}
	\lambda &= \lambda (T) = \lambda(0) \left(1 - \left(\frac{T}{T_c}\right)^4\right)^{-1/2} \nonumber \\
	\lambda(0) &= \sqrt{\frac{m^*c^2}{4\pi|\psi|^2e^{*2}}} \stackrel{\text{SI}}{=} \sqrt{\frac{m_e}{2|\psi|^2e^2\mu_0}}
	\label{eqn:london-penetration-depth}
\end{align}
That means that if a superconductor is thick enough that there will be no current in the interior. This will become important later on when we integrate a current over a path enclosing a hole in a superconductor. Under the right circumstances this current (almost) vanishes and we can neglect the integral.

\section{Josephson effect}
\label{sec:josephson-effect}
When two superconductors are separated by a (thin) barrier\footnote{This can be an insulator, normal metal, different superconductor or a constriction.} a supercurrent can flow between them. Josephson showed in 1962 that for two superconductors separated by an insulator the current is given by\cite{tinkhamIntroductionSuperconductivity}:
\begin{equation}
	I_s = I_c \sin(\Delta \phi)
\end{equation}
Where $\Delta \phi$ is the difference in phase between the two condensates as described by Ginzburg-Landau theory, see Eq. \ref{eqn:GL-wavefunction}. Furthermore $I_c$ is the critical current which is a junction property. This equation is generally known as the first Josephson equation. The relation between $I_s$ and the phase difference is the current-phase relation. In general this does not have to be purely sinusoidal\cite{golubovCurrentphaseRelationJosephson2004a}.

In the more general case we first define the gauge invariant phase\footnote{See \citetitle{tinkhamIntroductionSuperconductivity} equation 6.11, this equation is valid in both Gaussian and SI units, this is because the conversion factor for $\Phi_0$ cancels with the conversion factor for $\vec{A}$.}:
\begin{equation}
	\gamma = \Delta \varphi - \frac{2\pi}{\Phi_0}\int \vec{A} \cdot d\vec{l}
	\label{eqn:gauge-invariant-phase}
\end{equation}
We are required to do so as $\Delta \phi$ is not uniquely determined for a given physical situation whilst $I_s$ is\cite{tinkhamIntroductionSuperconductivity}. It simply transforms $I_c \sin(\Delta \phi) \to I_c \sin(\gamma)$. To now generalize our current-phase relation we write:
\begin{equation}
	I_s = I_c f(\gamma)
\end{equation}
Here we have defined $f(\gamma)$ which is the current-phase relation. In general it has the following properties\cite{golubovCurrentphaseRelationJosephson2004a}:
\begin{equation}
	f(\gamma) = f(\gamma + 2\pi) \quad f(\gamma) = -f(-\gamma) \quad f(2\pi n) = f(\pi m) = 0
\end{equation}
With $m,n \in \mathcal{N}$.

\section{dc-SQUID magnetometers}
% - Basic description of what they are
% - Basic explanation how they can be used to measure magnetic fields
TODO