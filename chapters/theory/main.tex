% !TEX root = ../../thesis.tex

\section{Superconductors}
% Basic introduction
% - Cooper pairs
% - Condensation
% - Wave function

\subsection{Meissner effect}
% - Superconductors expell magnetic field
% - Explained by London equations
% - Give rise to London penetration depth
\begin{equation}
	\lambda^2 = \frac{m^*c^2}{4\pi|\psi|^2e^{*2}} \stackrel{\text{SI}}{=} \frac{m_e}{2|\psi|^2e^2\mu_0}\footnote{See \citetitle{tinkhamIntroductionSuperconductivity} equation 4.8, the equation has been converted to SI units, $|\psi|^2$ has units of \unit{\per\cubic\meter}.}
	\label{eqn:london-penetration-depth}
\end{equation}

\subsection{Josephson effect}
% - Occurs when two superconductors are close to each other
% - Josephson effect produces a supercurrent that flows through a Josephson junction (superconductors coupled by a weak link)
% - Most basic solution is I_s = I_c sin(gamma)
\begin{equation}
	\gamma = \Delta \varphi - \frac{2\pi}{\Phi_0}\int \vec{A} \cdot d\vec{l}\footnote{See \citetitle{tinkhamIntroductionSuperconductivity} equation 6.11, this equation is valid in both Gaussian and SI units, this is because the conversion factor for $\Phi_0$ cancels with the conversion factor for $\vec{A}$.}
	\label{eqn:gauge-invariant-phase}
\end{equation}
\section{dc-SQUID magnetometers}
% - Basic description of what they are
% - Basic explanation how they can be used to measure magnetic fields