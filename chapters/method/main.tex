% !TEX root = ../../thesis.tex
% Method largely based on Frolov 2004.
% Uses a superconducting loop with a junction inductively coupled to a dc-SQUID magnetometer

Our method is mostly inspired by \Citeauthor{frolovMeasurementCurrentPhaseRelation2004} \citeyear{frolovMeasurementCurrentPhaseRelation2004}. We incorporate the junction whose CPR we want to measure into a superconducting loop. This loop is then inductively coupled to a dc-SQUID magnetometer. See Fig. \ref{fig:schematic-setup} for a schematic of the setup. This chapter will outline the ideas behind the setup and provide arguments for the chosen geometries.

\begin{figure}
	\centering
	\begin{circuitikz}
		% Main loop with single Josephson Junction (openbarrier) and an inductor for clarity.
		\draw (0,0) to [short, *-, i=$I_t$] (2,0)
		to [josephsonjunction, i=$I_s$, l_=$JJ$] (2, -2)
		to [short, -*, i=$I_t$] (0, -2);
		\draw (2,0) to [short, i=$I_l$] (4, 0)
		to [inductor, l_=$L_l$] (4, -2)
		to [short] (2, -2);

		% Secondary loop with a dc-SQUID.
		\draw (5, 0) to [josephsonjunction] (7, 0)
		to [short] (7, -2)
		to [josephsonjunction] (5, -2)
		to [inductor, l_=$L_s$] (5, 0);

		% Annotate flux through the loops
		\node[] at (3,-1) {$\Phi_l$};
		\node[] at (6,-1) {$\Phi_s$};
	\end{circuitikz}

	\caption{Schematic depiction of the system. The left loop is inductively coupled to the dc-SQUID on the right. This is illustrated by $L_l$ and $L_s$. The junction itself has an inductance $L_{JJ}$. The current $I_t$ is controlled externally. The flux through the two loops is denoted by $\Phi_l$ and $\Phi_s$. The junction whose CPR we want to measure is part of the left loop. Please note that the 4 contacts used for the dc-SQUID readout are not shown.}
	\label{fig:schematic-setup}
\end{figure}

\section{Relation between flux and phase}
Using flux quantization and the gauge-invariant phase we can derive a relation between the flux $\Phi_l$ and $\gamma$. For details please see Section \ref{app:derivation-phase-flux-relation}.
\begin{equation}
	\gamma = \frac{2\pi}{\Phi_0}\left(\Phi_l + \lambda^2\mu_0 \int \vec{J}\cdot d \vec{l} \right)
	\label{eqn:phase-flux-relation}
\end{equation}
We can express $\Phi_l$ in terms of $I_s$ and $I_l$:
\begin{equation}
	\Phi_l = I_sL_{JJ}  + I_lL_l
\end{equation}
Where $L_{JJ}$ and $L_l$ are the inductances of the Josephson junction and loop respectively\footnote{This is not the so called Josephson inductance but purely a magnetic inductance and not a kinetic inductance.}.

\subsection{Figure of merit}
\label{sec:figure-of-merit}
In order for $\gamma$ to be proportional to $\Phi_l$ the second term must be negligible in Eq. \ref{eqn:phase-flux-relation}. We define a figure of merit $\delta$ which we require to be $\ll 1$.
\begin{equation}
	\delta = \frac{\lambda^2\mu_0 \int \vec{J}\cdot d \vec{l}}{\Phi_l}
\end{equation}
We will assume that $L_{JJ} \ll L_l$ such that $\Phi_l \approx I_lL_l$. By approximating the loop containing the junction as a perfect loop with radius $r$, thickness (height) $d$ and width $w$ the approximate inductance is given by\cite{eewebCoilInductanceCalculator}:
\begin{equation}
	L_l \approx \mu_0 r \left( \ln\left(\frac{16r}{w}\right) - 2\right)
	\label{eq:loop-inductance}
\end{equation}
We can furthermore rewrite $\int \vec{J} \cdot d \vec{l} \approx 2\pi r I_l \tilde{j}$ with $\tilde{j} = \tilde{j}(d, w, \lambda)$ a geometric factor in \unit{\per\square\meter}. We thus obtain our final figure of merit:
\begin{equation}
	\delta \approx \frac{2\pi \lambda^2 \tilde{j}}{\ln\left(\frac{16r}{w}\right) - 2}
	\label{eqn:figure-of-merit}
\end{equation}
The value of $\tilde{j}$ can be calculated numerically and details can be found in Section \ref{app:derivation-geometric-factor-j}.

\subsection{Magnetic coupling}
\label{sec:magnetic-coupling}
There is not a $1:1$ relation between $\Phi_l$ and $\Phi_s$. We can calculate the mutual inductance between the two numerically. To do so we calculate the magnetic field created by a current through the junction loop and determining the magnetic flux through the dc-SQUID loop. This gives us the mutual inductance $M$. We define the coupling factor $\kappa$ such that:
\begin{equation}
	M = \kappa \sqrt{L_lL_s}
\end{equation}
Here we again assume that $L_{JJ} \ll L_l$. We now see that the magnetic fluxes are connected through the following relation:
\begin{equation}
	\Phi_l = \kappa \sqrt{\frac{L_l}{L_s}}\Phi_s \Leftrightarrow \Phi_s = \kappa \sqrt{\frac{L_s}{L_l}}\Phi_l
\end{equation}
This is a naive approach that does not take into account the possibility of magnetic lensing\cite{prigozhin3DSimulationSuperconducting2018}. Thus $\kappa$ is a `minimal' coupling between the two. As long as $\kappa$ is sufficiently large enough our method should work.

\section{Sample geometries}
Our method will make use of constriction junctions (ScS)\footnote{Also called a Dayem bridge, see \citeauthor{likharevSuperconductingWeakLinks1979} \citeyear{likharevSuperconductingWeakLinks1979}.} as these are fairly easy to construct. Following our figure of merit in Eq. \ref{eqn:figure-of-merit}, we benefit from a small London penetration depth. The size of our junction is determined by the value of $\xi$ and should be on the order of $3\xi$\cite{likharevSuperconductingWeakLinks1979} to make a single-valued CPR. Based on this and the available superconductors we have chosen for \ce{Nb}\footnote{$\lambda = \qty{39}{\nano\meter}$, $\xi = \qty{38}{\nano\meter}$ and $T_c \approx \qty{7.5}{\kelvin}$. Characteristic lengths attributed to R. Meservey and B. B. Schwartz and critical temperature based on measurements on a \ce{Nb}-thin film in our own lab.}. See Tab. \ref{tab:target-geometries} for the used geometries. For our calculations we used a thickness (height) of \qty{100}{\nano\meter} and a spacing between the two loops of \qty{200}{\nano\meter}. See Tab. \ref{tab:estimated-factors} for the results of the calculations.

\begin{table}
	\centering
	\begin{tabular}{@{}lrr@{}}
		\toprule
		Loop & Inner diameter [\unit{\micro\meter}] &  Outer diameter [\unit{\micro\meter}] \\ \midrule
		main  & 1.40  & 2.00 \\
		dc-SQUID & 1.15 & 1.38 \\
		\bottomrule
	\end{tabular}
	\caption{Geometries used in the calculation of the figure of merit and coupling factor. The geometries of the dc-SQUID are based on earlier work by \citeauthor{rogSQUIDontipMagneticMicroscopy2022} \citeyear{rogSQUIDontipMagneticMicroscopy2022}.}
	\label{tab:target-geometries}
\end{table}

\begin{table}
	\centering
	\begin{tabular}{@{}lr@{}}
		\toprule
		Parameter & Value \\
		\midrule
		mutual inductance ($M$) & \qty{-6.834e-02}{\pico\henry} \\
		main loop inductance ($L_l$) & \qty{1.938e+00}{\pico\henry} \\
		dc-SQUID inductance ($L_s$) & \qty{1.969e+00}{\pico\henry} \\
		coupling constant ($\kappa$) & -0.035 \\
		figure of merit ($\delta$) & 0.00015 \\
		\bottomrule
	\end{tabular}
	\caption{Estimated parameters of the system based on the geometries in Tab. \ref{tab:target-geometries} and using a thickness (height) of \qty{100}{\nano\meter} and a spacing between the two loops of \qty{200}{\nano\meter}.}
	\label{tab:estimated-factors}
\end{table}

\subsection{Sample fabrication}
The course structures such as contact pads are created using lithography. We use dc-sputtering to deposit a \qty{\pm 90}{\nano\meter} layer of \ce{Nb} on top of a resist coated \ce{Si} wafer. As \ce{Nb} is very sensitive to contamination we additionally cap it using dc-sputtered \qty{7}{\nano\meter} layer of \ce{Au}.

The junctions and loops are created by a focussed ion beam (FIB) that etches away the layers of \ce{Nb} and \ce{Au}. The advantage of this method over creating these fine structures using lithography is the higher resolution of the FIB and less contamination of the \ce{Nb} from the resist.

The junction type we use is a constriction junction. The relevant parameter is $\xi$. By choosing the width of the constriction on the order of $3\xi$ we can control the junction behaviour due to the temperature dependence of $\xi$\cite{tinkhamIntroductionSuperconductivity}. Details on the sample geometries will be given on a per sample basis in later sections as these are highly dependent on alignments and drift.

\section{Analysis method}
Our method requires little to no analysis, which immediately highlights one of the benefits of this method. This section mainly recaps how to retrieve the CPR, $f(\gamma)$, from the measured quantities $I_t$ (which is controlled in the experiment) and $\Phi_s$.

Starting from our Eq. \ref{eqn:phase-flux-relation}, the expression for $\Phi_l$ and applying Ohms laws to the circuit in Fig. \ref{fig:schematic-setup} we have:
\begin{equation}
	\gamma = \frac{2\pi}{\Phi_0}\Phi_l \qquad \Phi_l = I_sL_{JJ} + I_lL_l \qquad I_t = I_s + I_l
\end{equation}
Here we have neglected the term $\lambda^2\mu_0 \int \vec{J}\cdot d \vec{l}$ assuming our figure or merit is sufficiently small as highlighted in Sec. \ref{sec:figure-of-merit}. Using these relations and recalling $I_s = I_cf(\gamma)$ we see that:
\begin{align}
	I_t &= I_s + \frac{1}{L_l}\left(\Phi_l - I_sL_{JJ}\right) + \frac{\Phi_l}{L_l} \nonumber \\
	    &= I_c \left(1 - \frac{L_{JJ}}{L_l}\right)f(\gamma) + \frac{\Phi_0}{2\pi L_l}\gamma
\end{align}
This means that a linear trend is added on top of our CPR. Furthermore we have a final step where $\gamma$ is related to $\Phi_s$ based on Sec. \ref{sec:magnetic-coupling}.
\begin{equation}
	\gamma = \frac{2\pi}{\Phi_0}\Phi_l = \kappa \sqrt{\frac{L_l}{L_s}} \frac{2\pi}{\Phi_0} \Phi_s
\end{equation}
Since $f(\gamma)$ must be $2\pi$-periodic we can easily map these two even if $\kappa$, $L_l$ and $L_s$ are not know exactly. This is the only data manipulation we need to do based on unknown parameters.