% !TEX root = ../../thesis.tex
% Method largely based on Frolov 2004.
% Uses a superconducting loop with a junction inductively coupled to a dc-SQUID magnetometer

Our method is mostly inspired by \Citeauthor{frolovMeasurementCurrentPhaseRelation2004} \citeyear{frolovMeasurementCurrentPhaseRelation2004}. We incorporate the junction whose CPR we want to measure into a superconducting loop. This loop is then inductively coupled to a dc-SQUID magnetometer. See Fig. \ref{fig:schematic-setup} for a schematic of the setup. This chapter will outline the ideas behind the setup and provide arguments for the chosen geometries.

\begin{figure}
	\centering
	\begin{circuitikz}
		% Main loop with single Josephson Junction (openbarrier) and an inductor for clarity.
		\draw (0,0) to [short, *-, i=$I_t$] (2,0)
		to [josephsonjunction, i=$I_s$, l_=$JJ$] (2, -2)
		to [short, -*, i=$I_t$] (0, -2);
		\draw (2,0) to [short, i=$I_l$] (4, 0)
		to [inductor, l_=$L_l$] (4, -2)
		to [short] (2, -2);

		% Secondary loop with a dc-SQUID.
		\draw (5, 0) to [josephsonjunction] (7, 0)
		to [short] (7, -2)
		to [josephsonjunction] (5, -2)
		to [inductor, l_=$L_s$] (5, 0);

		% Annotate flux through the loops
		\node[] at (3,-1) {$\Phi_l$};
		\node[] at (6,-1) {$\Phi_s$};
	\end{circuitikz}

	\caption{Schematic depiction of the system. The left loop is inductively coupled to the dc-SQUID on the right. This is illustrated by $L_l$ and $L_s$. The junction itself has an inductance $L_{JJ}$. The current $I_t$ is controlled externally. The flux through the two loops is denoted by $\Phi_l$ and $\Phi_s$. The junction whose CPR we want to measure is part of the left loop. Please note that the 4 contacts used for the dc-SQUID readout are not shown.}
	\label{fig:schematic-setup}
\end{figure}

\section{Relation between flux and phase}
Using flux quantization and the gauge-invariant phase we can derive a relation between the flux $\Phi_l$ and $\gamma$. For details please see Section \ref{app:derivation-phase-flux-relation}.
\begin{equation}
	\gamma = \frac{2\pi}{\Phi_0}\left(\Phi_l + \lambda^2\mu_0 \int \vec{J}\cdot d \vec{l} \right)
	\label{eqn:phase-flux-relation}
\end{equation}
We can express $\Phi_l$ in terms of $I_s$ and $I_l$:
\begin{equation}
	\Phi_l = I_sL_{JJ}  + I_lL_l
\end{equation}
Where $L_{JJ}$ and $L_l$ are the inductances of the Josephson junction and loop respectively\footnote{This is not the so called Josephson inductance but purely a magnetic inductance and not a kinetic inductance.}.

\subsection{Figure of merit}
In order for $\gamma$ to be proportional to $\Phi_l$ the second term must be negligible in Eq. \ref{eqn:phase-flux-relation}. We define a figure of merit $\delta$ which we require to be $\ll 1$.
\begin{equation}
	\delta = \frac{\lambda^2\mu_0 \int \vec{J}\cdot d \vec{l}}{\Phi_l}
\end{equation}
We will assume that $L_{JJ} \ll L_l$ such that $\Phi_l \approx I_lL_l$. By approximating the loop containing the junction as a perfect loop with radius $r$, thickness (height) $d$ and width $w$ the approximate inductance is given by\cite{eewebCoilInductanceCalculator}:
\begin{equation}
	L_l \approx \mu_0 r \left( \ln\left(\frac{16r}{w}\right) - 2\right)
	\label{eq:loop-inductance}
\end{equation}
We can furthermore rewrite $\int \vec{J} \cdot d \vec{l} \approx 2\pi r I_l \tilde{j}$ with $\tilde{j} = \tilde{j}(d, w, \lambda)$ a geometric factor in \unit{\per\square\meter}. We thus obtain our final figure of merit:
\begin{equation}
	\delta \approx \frac{2\pi \lambda^2 \tilde{j}}{\ln\left(\frac{16r}{w}\right) - 2}
	\label{eqn:figure-of-merit}
\end{equation}
The value of $\tilde{j}$ can be calculated numerically and details can be found in Section \ref{app:derivation-geometric-factor-j}.

\subsection{Magnetic coupling}
There is not a $1:1$ relation between $\Phi_l$ and $\Phi_s$. We can calculate the mutual inductance between the two numerically. To do so we calculate the magnetic field created by a current through the junction loop and determining the magnetic flux through the dc-SQUID loop. This gives us the mutual inductance $M$. We define the coupling factor $\kappa$ such that:
\begin{equation}
	M = \kappa \sqrt{L_lL_s}
\end{equation}
Here we again assume that $L_{JJ} \ll L_l$. We now see that the magnetic fluxes are connected through the following relation:
\begin{equation}
	\Phi_l = \kappa \sqrt{\frac{L_l}{L_s}}\Phi_s \Leftrightarrow \Phi_s = \kappa \sqrt{\frac{L_s}{L_l}}\Phi_l
\end{equation}
This is a naive approach that does not take into account the possibility of magnetic lensing\cite{prigozhin3DSimulationSuperconducting2018}. Thus $\kappa$ is a `minimal' coupling between the two. As long as $\kappa$ is sufficiently large enough our method should work.

% Relation between flux and phase
% Figure of merit
% Magnetic coupling