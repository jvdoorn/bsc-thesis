% !TEX root = ../../thesis.tex
% Method largely based on Frolov 2004.
% Uses a superconducting loop with a junction inductively coupled to a dc-SQUID magnetometer
Our method is mostly inspired by \Citeauthor{frolovMeasurementCurrentPhaseRelation2004} \citeyear{frolovMeasurementCurrentPhaseRelation2004}. We incorporate the junction whose CPR we want to measure into a superconducting loop. This loop is then inductively coupled to a dc-SQUID magnetometer. See Fig. \ref{fig:schematic-setup} for a schematic of the setup. This chapter will outline the ideas behind the setup and provide arguments for the chosen geometries.

\begin{figure}
	\centering
	\begin{circuitikz}
		% Main loop with single Josephson Junction (openbarrier) and an inductor for clarity.
		\draw (0,0) to [short, *-, i=$I_t$] (2,0)
		to [josephsonjunction, i=$I_s$, l_=$JJ$] (2, -2)
		to [short, -*, i=$I_t$] (0, -2);
		\draw (2,0) to [short, i=$I_l$] (4, 0)
		to [inductor, l_=$L_l$] (4, -2)
		to [short] (2, -2);

		% Secondary loop with a dc-SQUID.
		\draw (5, 0) to [josephsonjunction] (7, 0)
		to [short] (7, -2)
		to [josephsonjunction] (5, -2)
		to [inductor, l_=$L_s$] (5, 0);

		% Annotate flux through the loops
		\node[] at (3,-1) {$\Phi_l$};
		\node[] at (6,-1) {$\Phi_s$};
	\end{circuitikz}

	\caption{Schematic depiction of the system. The left loop is inductively coupled to the dc-SQUID on the right. This is illustrated by $L_l$ and $L_s$. The junction itself has an inductance $L_{JJ}$. The current $I_t$ is controlled externally. The flux through the two loops is denoted by $\Phi_l$ and $\Phi_s$. The junction whose CPR we want to measure is part of the left loop. Please note that the 4 contacts used for the dc-SQUID readout are not shown.}
	\label{fig:schematic-setup}
\end{figure}

\section{Relation between flux and phase}
Using flux quantization and the gauge-invariant phase we can derive a relation between the flux $\Phi_l$ and $\gamma$. For details please see Section \ref{app:derivation-phase-flux-relation}.
\begin{equation}
	\gamma = \frac{2\pi}{\Phi_0}\left(\Phi_l + \lambda^2\mu_0 \int \vec{J}\cdot d \vec{l} \right)
	\label{eqn:phase-flux-relation}
\end{equation}
We can express $\Phi_l$ in terms of $I_s$ and $I_l$:
\begin{equation}
	\Phi_l = I_sL_{JJ}  + I_lL_l
\end{equation}
Where $L_{JJ}$ and $L_l$ are the inductances of the Josephson junction and loop respectively\footnote{This is not the so called Josephson inductance but purely a magnetic inductance and not a kinetic inductance.}. We will assume that $L_{JJ} \ll L_l$, this is also done by \citeauthor{frolovMeasurementCurrentPhaseRelation2004} (see \cite{frolovCurrentphaseRelationsJosephson2005,frolovMeasurementCurrentPhaseRelation2004}) and often in literature when discussing dc-SQUIDs (see par example \cite{clarkeSQUIDHandbook2004}). This means our final equation for the loop flux becomes:
\begin{equation}
	\Phi_l = I_lL_l
\end{equation}

\subsection{Figure of merit}
\label{sec:figure-of-merit}
In order for $\gamma$ to be proportional to $\Phi_l$ the second term must be negligible in Eq. \ref{eqn:phase-flux-relation}. We define a figure of merit $\delta$ which we require to be $\ll 1$.
\begin{equation}
	\delta = \frac{\lambda^2\mu_0 \int \vec{J}\cdot d \vec{l}}{\Phi_l}
\end{equation}
For a square and round geometry we can rewrite the integral part in the following way:
\begin{equation}
	\int \vec{J}\cdot d \vec{l} = \begin{cases}
		2\pi r \tilde{j} I_l, &\text{round} \\
		8r \tilde{j} I_l, &\text{square}
	\end{cases}
\end{equation}
Here $r$ is the radius (half the diameter) and $\tilde{j} = \tilde{j}(d, w, \lambda)$ a geometric factor in \unit{\per\square\meter}. The value of $\tilde{j}$ can be calculated numerically and details can be found in Section \ref{app:derivation-geometric-factor-j}.

The loop flux can simply be rewritten as $\Phi_l = I_lL_l$, this allows us to rewrite the figure of merit as:
\begin{equation}
	\delta = \frac{\lambda^2\mu_0\tilde{j}}{L_l} \cdot \begin{cases}
		2\pi r, &\text{round} \\
		8r, &\text{square}
	\end{cases}
	\label{eqn:figure-of-merit}
\end{equation}

It is important to note that the figure of merit is independent of the amount of current ($I_l$) passed through the loop, as such it is always valid. The inductance of the loop can be determined using a simulation done in SuperScreen\cite{bishop-vanhornSuperScreenOpensourcePackage2022}.

\subsection{Magnetic coupling}
\label{sec:magnetic-coupling}
There is not a $1:1$ relation between $\Phi_l$ and $\Phi_s$. We can calculate the mutual inductance between the two numerically. To do so we calculate the magnetic field created by a current through the junction loop and determining the magnetic flux through the dc-SQUID loop. This gives us the mutual inductance $M$. We define the coupling factor $\kappa$ such that:
\begin{equation}
	M = \kappa \sqrt{L_lL_s}
\end{equation}
Here we again assume that $L_{JJ} \ll L_l$. We now see that the magnetic fluxes are connected through the following relation:
\begin{equation}
	\Phi_l = \kappa \sqrt{\frac{L_l}{L_s}}\Phi_s \Leftrightarrow \Phi_s = \kappa \sqrt{\frac{L_s}{L_l}}\Phi_l
\end{equation}
This is a naive approach that does not take into account the possibility of magnetic lensing\cite{prigozhin3DSimulationSuperconducting2018}. Thus $\kappa$ is a `minimal' coupling between the two. As long as $\kappa$ is sufficiently large enough our method should work.

\section{Sample geometries}
The diameter of the dc-SQUID is chosen such that the periodicity of the SQUID interference pattern is on the order of a few \unit{\milli\tesla}. This means the effective diameter should be around \qtyrange{1}{2}{\micro\meter}. Furthermore, the width of the loop together with the thickness of the superconductor determine the geometric factor $\tilde{j}$, they are chosen such that the figure of merit is sufficiently small. In practice this means that the width is around \qty{0.3}{\micro\meter} and the thickness around \qty{100}{\nano\meter}. Details on this can be found on a per sample basis later as well as details on the type of junction.

\subsection{Sample fabrication}
\label{sec:method-sample-fabrication}
We use \ce{Si} wafer of approximately \qty{1}{\square\centi\meter}. The wafer is cleaned by dusting it off using a pressurised nitrogen gas and rinsed for \qty{30}{\minute} in acetone in an ultrasonic bath. To get rid of acetone residue we finally rinse it using IPA for \qty{5}{\minute} in ultrasonic bath. The wafer then spin coated with PMMA 600K, baked at \qty{180}{\celsius} for 5 minutes, coated with PMMA 950K and then baked again. We then use electron beam lithography to etch our patterns. The PMMA is a negative resist, so exposed regions can later be removed. After this process we use dc-sputtering to deposit around \qty{100}{\nano\meter} of \ce{Nb} and cap it using \qty{7}{\nano\meter} of \ce{Au}. The remaining PMMA is finally removed using acetone in an ultrasonic bath and rinsed with IPA. See Figure~\ref{fig:lithography}. Using this process we create all the coarse structures, these include the contact pads, leads to the contact pad and a central square in which we will make our fine geometries.

\begin{figure}[ht!]
	\centering
	\def\svgwidth{\textwidth}
	\import{figures}{lithography.pdf_tex}
	\caption{Schematic depiction of the lithography and sputtering process. \textbf{a)} We start with a \ce{Si} wafer (dark grey) with a layer of PMMA 600K (\qty{0.4}{\micro\meter}) and on top of that PMMA 950K (\qty{0.4}{\micro\meter}) (yellow) and expose it to an electron beam. \textbf{b)} After exposure, we can remove the exposed regions using MIBK / IPA (1:3). \textbf{c)} We sputter the \ce{Nb} and \ce{Au} capping layer (light grey) on top the sample. \textbf{d)} In the final step we use acetone to remove the left-over PMMA leaving us with just the \ce{Nb} and \ce{Au} on top of the wafer in the desired pattern. The figure is not to scale.}
	\label{fig:lithography}
\end{figure}

The finer geometries in the central square are created using a beam of \ce{Ga+} (focussed ion beam (FIB)). They are used to cut away the \ce{Nb} and \ce{Au}, leaving just the wafer. The advantage of use the FIB is that we can achieve a higher resolution compared to lithography which is important for our junctions.

\section{Analysis method}
Our method requires little to no analysis, which immediately highlights one of the benefits of this method. This section mainly recaps how to retrieve the CPR, $f(\gamma)$, from the measured quantities $I_t$ (which is controlled in the experiment) and $\Phi_s$.

Starting from our Equation~\ref{eqn:phase-flux-relation}, the expression for $\Phi_l$ and applying Ohms laws to the circuit in Figure~\ref{fig:schematic-setup} we have:
\begin{equation}
	\gamma = \frac{2\pi}{\Phi_0}\Phi_l \qquad \Phi_l = I_lL_l \qquad I_t = I_s + I_l
\end{equation}
Here we have neglected the term $\lambda^2\mu_0 \int \vec{J}\cdot d \vec{l}$ assuming our figure or merit is sufficiently small as highlighted in Sec. \ref{sec:figure-of-merit}. Using these relations and recalling $I_s = I_cf(\gamma)$ we see that:
\begin{align}
	I_t &= I_s + \frac{\Phi_l}{L_l} \nonumber \\
	    &= I_c f(\gamma) + \frac{\Phi_0}{2\pi L_l}\gamma
\end{align}
This means that a linear trend is added on top of our CPR. Furthermore we have a final step where $\gamma$ is related to $\Phi_s$ based on Sec. \ref{sec:magnetic-coupling}.

\begin{equation}
	\gamma = \frac{2\pi}{\Phi_0}\Phi_l = \kappa \sqrt{\frac{L_l}{L_s}} \frac{2\pi}{\Phi_0} \Phi_s
\end{equation}
Since $f(\gamma)$ must be $2\pi$-periodic we can easily map these two even if $\kappa$, $L_l$ and $L_s$ are not know exactly. This is the only data analysis we need to do based on unknown parameters.