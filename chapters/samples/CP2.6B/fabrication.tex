% !TEX root = ../../../thesis.tex
%TODO: fix values.
In order to make SNS junctions we first sputtered \qty{1}{\nano\meter} of \ce{Cu} on our \ce{SiO} wafer, on top of this we added our \qty{1}{\nano\meter} of \ce{Nb} and capped it with \qty{7}{\nano\meter} of \ce{Au}. We again used the FIB to create the fine structures, see Figure~\ref{fig:CP2.6B-SEM-images}.

\begin{figure}[ht]
	\begin{subfigure}[t]{0.3\textwidth}
		\centering
		\includegraphics[width=\textwidth]{figures/samples/CP2/CP2.6B_SEM_overview.jpg}
		\subcaption{Overview of the device. The top loop shows the dc-SQUID and the bottom is the junction loop.}
	\end{subfigure}
	\hfill
	\begin{subfigure}[t]{0.3\textwidth}
		\centering
		\includegraphics[width=\textwidth]{figures/samples/CP2/CP2.6B_SEM_junction.jpg}
		\subcaption{Zoomed in view of the junction, the width of the junction is \qty{12}{\nano\meter}.}
	\end{subfigure}
	\hfill
	\begin{subfigure}[t]{0.3\textwidth}
		\centering
		\includegraphics[width=\textwidth]{figures/samples/CP2/CP2.6B_SEM_SQUID.jpg}
		\subcaption{Zoomed in view of the dc-SQUID. The width of the junctions is \qty{22}{\nano\meter}.}
	\end{subfigure}

	\caption{Fine structures of sample CP2.6B after the FIB. See Table~\ref{tab:CP2.6B-geometries} for the exact geometries of the sample.}
	\label{fig:CP2.6B-SEM-images}
\end{figure}

\begin{table}
	\begin{subtable}{.5\linewidth}
		\centering
		\begin{tabular}{@{}lrr@{}}
			\toprule
			Parameter & Value \\ \midrule
			Junction loop $\diameter_{\text{outer}}$ & \qty{1.9}{\micro\meter} \\
			Junction loop $\diameter_{\text{inner}}$ & \qty{1.2}{\micro\meter} \\
			dc-SQUID $\diameter_{\text{outer}}$ & \qty{1.6}{\micro\meter} \\
			dc-SQUID $\diameter_{\text{inner}}$ & \qty{1.1}{\micro\meter} \\
			spacing & \qty{0.1}{\micro\meter} \\
			$d_{\ce{Cu}}$ & \qty{25}{\nano\meter} \\
			$d_{\ce{Nb}}$ & \qty{70}{\nano\meter} \\
			$d_{\ce{Au}}$ & \qty{7}{\nano\meter} \\
			\bottomrule
		\end{tabular}
    \end{subtable}
    \begin{subtable}{.5\linewidth}
    	\centering
    	\begin{tabular}{@{}lrr@{}}
    		\toprule
    		Parameter & Value \\ \midrule
    		$L_{l}$ & \qty{3.0}{\pico\henry} \\
			$L_{s}$ & \qty{3.0}{\pico\henry} \\
			$M$ & \qty{-0.2}{\pico\henry} \\
			$\delta$ & \num{0.001} \\
			$\kappa$ & \num{-0.058} \\
    		\bottomrule
    	\end{tabular}
    \end{subtable}
    \caption{The \textbf{left} table provides an overview of the geometries of CP2.6B as determined by SEM imaging and sputtering rates. The \textbf{right} table gives an overview of parameters found using a simulation based on the geometries. The geometries are based on sample CP1.2H.}
    \label{tab:CP2.6B-geometries}
\end{table}