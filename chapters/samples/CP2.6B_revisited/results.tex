% !TEX root = ../../../thesis.tex
The sample had not been permanently stored in a desiccator. An important check was thus if the sample had degraded. The RT-curve of the dc-SQUID showed no significant changes. As such it is unlikely that the sample degraded significantly. It also confirmed that all the contacts are still good.

Previously the CPR was measured at \qty{3}{\kelvin}. An increase in temperature decreases the sensitivity of our dc-SQUID and the critical current of the junction's loop. Our previous measurement was approximately \qty{200}{\micro\ampere} periodic. In order to see multiple periods it is preferable to change $I_t$ between \qty{\pm400}{\micro\ampere}. This allows us to see around 4 to 5 periods. Through trial and error we found the highest usable temperature to be \qty{3.6}{\kelvin}.

Similarly to last time SQIs were measured for several bias currents in the range of \qtyrange{300}{350}{\micro\ampere}. The SQIs shown in Figure~\ref{fig:CP2.6B_revisited_SQIs}. Whilst the SQIs are far from perfect, they are usable. The curved background is due to the magnetic field dependence of the critical current of the bulk. Near zero field the periodicity as well as the horizontal offset is reproducible. The amplitude of the oscillations is stable as well. Since the junction's loop will only create a small flux this should be sufficient. Around zero field we can approximate the oscillations as sinusoidal. After fitting it gives a sensitivity of \qtylist{491.37;189.73;326.49;410.49;419.73}{\micro\volt\per\fluxquantum} for \qtylist{2.8;3.0;3.2;3.4;3.6}{\kelvin} respectively. These are comparable to previous results. Please note that the first sensitivity is significantly higher due to a higher bias current.

\begin{figure}[ht!]
	\centering
	%% Creator: Matplotlib, PGF backend
%%
%% To include the figure in your LaTeX document, write
%%   \input{<filename>.pgf}
%%
%% Make sure the required packages are loaded in your preamble
%%   \usepackage{pgf}
%%
%% Also ensure that all the required font packages are loaded; for instance,
%% the lmodern package is sometimes necessary when using math font.
%%   \usepackage{lmodern}
%%
%% Figures using additional raster images can only be included by \input if
%% they are in the same directory as the main LaTeX file. For loading figures
%% from other directories you can use the `import` package
%%   \usepackage{import}
%%
%% and then include the figures with
%%   \import{<path to file>}{<filename>.pgf}
%%
%% Matplotlib used the following preamble
%%   \usepackage{siunitx}
%%   \usepackage{fontspec}
%%   \setmainfont{Times New Roman.ttf}[Path=\detokenize{/System/Library/Fonts/Supplemental/}]
%%   \setsansfont{DejaVuSans.ttf}[Path=\detokenize{/Users/julian/UL-BRP-analysis/venv/lib/python3.10/site-packages/matplotlib/mpl-data/fonts/ttf/}]
%%   \setmonofont{DejaVuSansMono.ttf}[Path=\detokenize{/Users/julian/UL-BRP-analysis/venv/lib/python3.10/site-packages/matplotlib/mpl-data/fonts/ttf/}]
%%   \makeatletter\@ifpackageloaded{underscore}{}{\usepackage[strings]{underscore}}\makeatother
%%
\begingroup%
\makeatletter%
\begin{pgfpicture}%
\pgfpathrectangle{\pgfpointorigin}{\pgfqpoint{4.726167in}{4.728881in}}%
\pgfusepath{use as bounding box, clip}%
\begin{pgfscope}%
\pgfsetbuttcap%
\pgfsetmiterjoin%
\definecolor{currentfill}{rgb}{1.000000,1.000000,1.000000}%
\pgfsetfillcolor{currentfill}%
\pgfsetlinewidth{0.000000pt}%
\definecolor{currentstroke}{rgb}{1.000000,1.000000,1.000000}%
\pgfsetstrokecolor{currentstroke}%
\pgfsetdash{}{0pt}%
\pgfpathmoveto{\pgfqpoint{0.000000in}{0.000000in}}%
\pgfpathlineto{\pgfqpoint{4.726167in}{0.000000in}}%
\pgfpathlineto{\pgfqpoint{4.726167in}{4.728881in}}%
\pgfpathlineto{\pgfqpoint{0.000000in}{4.728881in}}%
\pgfpathlineto{\pgfqpoint{0.000000in}{0.000000in}}%
\pgfpathclose%
\pgfusepath{fill}%
\end{pgfscope}%
\begin{pgfscope}%
\pgfsetbuttcap%
\pgfsetmiterjoin%
\definecolor{currentfill}{rgb}{1.000000,1.000000,1.000000}%
\pgfsetfillcolor{currentfill}%
\pgfsetlinewidth{0.000000pt}%
\definecolor{currentstroke}{rgb}{0.000000,0.000000,0.000000}%
\pgfsetstrokecolor{currentstroke}%
\pgfsetstrokeopacity{0.000000}%
\pgfsetdash{}{0pt}%
\pgfpathmoveto{\pgfqpoint{0.444748in}{3.403703in}}%
\pgfpathlineto{\pgfqpoint{4.676167in}{3.403703in}}%
\pgfpathlineto{\pgfqpoint{4.676167in}{4.479825in}}%
\pgfpathlineto{\pgfqpoint{0.444748in}{4.479825in}}%
\pgfpathlineto{\pgfqpoint{0.444748in}{3.403703in}}%
\pgfpathclose%
\pgfusepath{fill}%
\end{pgfscope}%
\begin{pgfscope}%
\pgfsetbuttcap%
\pgfsetroundjoin%
\definecolor{currentfill}{rgb}{0.000000,0.000000,0.000000}%
\pgfsetfillcolor{currentfill}%
\pgfsetlinewidth{0.501875pt}%
\definecolor{currentstroke}{rgb}{0.000000,0.000000,0.000000}%
\pgfsetstrokecolor{currentstroke}%
\pgfsetdash{}{0pt}%
\pgfsys@defobject{currentmarker}{\pgfqpoint{0.000000in}{0.000000in}}{\pgfqpoint{0.000000in}{0.041667in}}{%
\pgfpathmoveto{\pgfqpoint{0.000000in}{0.000000in}}%
\pgfpathlineto{\pgfqpoint{0.000000in}{0.041667in}}%
\pgfusepath{stroke,fill}%
}%
\begin{pgfscope}%
\pgfsys@transformshift{0.643182in}{3.403703in}%
\pgfsys@useobject{currentmarker}{}%
\end{pgfscope}%
\end{pgfscope}%
\begin{pgfscope}%
\pgfsetbuttcap%
\pgfsetroundjoin%
\definecolor{currentfill}{rgb}{0.000000,0.000000,0.000000}%
\pgfsetfillcolor{currentfill}%
\pgfsetlinewidth{0.501875pt}%
\definecolor{currentstroke}{rgb}{0.000000,0.000000,0.000000}%
\pgfsetstrokecolor{currentstroke}%
\pgfsetdash{}{0pt}%
\pgfsys@defobject{currentmarker}{\pgfqpoint{0.000000in}{-0.041667in}}{\pgfqpoint{0.000000in}{0.000000in}}{%
\pgfpathmoveto{\pgfqpoint{0.000000in}{0.000000in}}%
\pgfpathlineto{\pgfqpoint{0.000000in}{-0.041667in}}%
\pgfusepath{stroke,fill}%
}%
\begin{pgfscope}%
\pgfsys@transformshift{0.643182in}{4.479825in}%
\pgfsys@useobject{currentmarker}{}%
\end{pgfscope}%
\end{pgfscope}%
\begin{pgfscope}%
\pgfsetbuttcap%
\pgfsetroundjoin%
\definecolor{currentfill}{rgb}{0.000000,0.000000,0.000000}%
\pgfsetfillcolor{currentfill}%
\pgfsetlinewidth{0.501875pt}%
\definecolor{currentstroke}{rgb}{0.000000,0.000000,0.000000}%
\pgfsetstrokecolor{currentstroke}%
\pgfsetdash{}{0pt}%
\pgfsys@defobject{currentmarker}{\pgfqpoint{0.000000in}{0.000000in}}{\pgfqpoint{0.000000in}{0.041667in}}{%
\pgfpathmoveto{\pgfqpoint{0.000000in}{0.000000in}}%
\pgfpathlineto{\pgfqpoint{0.000000in}{0.041667in}}%
\pgfusepath{stroke,fill}%
}%
\begin{pgfscope}%
\pgfsys@transformshift{1.123645in}{3.403703in}%
\pgfsys@useobject{currentmarker}{}%
\end{pgfscope}%
\end{pgfscope}%
\begin{pgfscope}%
\pgfsetbuttcap%
\pgfsetroundjoin%
\definecolor{currentfill}{rgb}{0.000000,0.000000,0.000000}%
\pgfsetfillcolor{currentfill}%
\pgfsetlinewidth{0.501875pt}%
\definecolor{currentstroke}{rgb}{0.000000,0.000000,0.000000}%
\pgfsetstrokecolor{currentstroke}%
\pgfsetdash{}{0pt}%
\pgfsys@defobject{currentmarker}{\pgfqpoint{0.000000in}{-0.041667in}}{\pgfqpoint{0.000000in}{0.000000in}}{%
\pgfpathmoveto{\pgfqpoint{0.000000in}{0.000000in}}%
\pgfpathlineto{\pgfqpoint{0.000000in}{-0.041667in}}%
\pgfusepath{stroke,fill}%
}%
\begin{pgfscope}%
\pgfsys@transformshift{1.123645in}{4.479825in}%
\pgfsys@useobject{currentmarker}{}%
\end{pgfscope}%
\end{pgfscope}%
\begin{pgfscope}%
\pgfsetbuttcap%
\pgfsetroundjoin%
\definecolor{currentfill}{rgb}{0.000000,0.000000,0.000000}%
\pgfsetfillcolor{currentfill}%
\pgfsetlinewidth{0.501875pt}%
\definecolor{currentstroke}{rgb}{0.000000,0.000000,0.000000}%
\pgfsetstrokecolor{currentstroke}%
\pgfsetdash{}{0pt}%
\pgfsys@defobject{currentmarker}{\pgfqpoint{0.000000in}{0.000000in}}{\pgfqpoint{0.000000in}{0.041667in}}{%
\pgfpathmoveto{\pgfqpoint{0.000000in}{0.000000in}}%
\pgfpathlineto{\pgfqpoint{0.000000in}{0.041667in}}%
\pgfusepath{stroke,fill}%
}%
\begin{pgfscope}%
\pgfsys@transformshift{1.604109in}{3.403703in}%
\pgfsys@useobject{currentmarker}{}%
\end{pgfscope}%
\end{pgfscope}%
\begin{pgfscope}%
\pgfsetbuttcap%
\pgfsetroundjoin%
\definecolor{currentfill}{rgb}{0.000000,0.000000,0.000000}%
\pgfsetfillcolor{currentfill}%
\pgfsetlinewidth{0.501875pt}%
\definecolor{currentstroke}{rgb}{0.000000,0.000000,0.000000}%
\pgfsetstrokecolor{currentstroke}%
\pgfsetdash{}{0pt}%
\pgfsys@defobject{currentmarker}{\pgfqpoint{0.000000in}{-0.041667in}}{\pgfqpoint{0.000000in}{0.000000in}}{%
\pgfpathmoveto{\pgfqpoint{0.000000in}{0.000000in}}%
\pgfpathlineto{\pgfqpoint{0.000000in}{-0.041667in}}%
\pgfusepath{stroke,fill}%
}%
\begin{pgfscope}%
\pgfsys@transformshift{1.604109in}{4.479825in}%
\pgfsys@useobject{currentmarker}{}%
\end{pgfscope}%
\end{pgfscope}%
\begin{pgfscope}%
\pgfsetbuttcap%
\pgfsetroundjoin%
\definecolor{currentfill}{rgb}{0.000000,0.000000,0.000000}%
\pgfsetfillcolor{currentfill}%
\pgfsetlinewidth{0.501875pt}%
\definecolor{currentstroke}{rgb}{0.000000,0.000000,0.000000}%
\pgfsetstrokecolor{currentstroke}%
\pgfsetdash{}{0pt}%
\pgfsys@defobject{currentmarker}{\pgfqpoint{0.000000in}{0.000000in}}{\pgfqpoint{0.000000in}{0.041667in}}{%
\pgfpathmoveto{\pgfqpoint{0.000000in}{0.000000in}}%
\pgfpathlineto{\pgfqpoint{0.000000in}{0.041667in}}%
\pgfusepath{stroke,fill}%
}%
\begin{pgfscope}%
\pgfsys@transformshift{2.084572in}{3.403703in}%
\pgfsys@useobject{currentmarker}{}%
\end{pgfscope}%
\end{pgfscope}%
\begin{pgfscope}%
\pgfsetbuttcap%
\pgfsetroundjoin%
\definecolor{currentfill}{rgb}{0.000000,0.000000,0.000000}%
\pgfsetfillcolor{currentfill}%
\pgfsetlinewidth{0.501875pt}%
\definecolor{currentstroke}{rgb}{0.000000,0.000000,0.000000}%
\pgfsetstrokecolor{currentstroke}%
\pgfsetdash{}{0pt}%
\pgfsys@defobject{currentmarker}{\pgfqpoint{0.000000in}{-0.041667in}}{\pgfqpoint{0.000000in}{0.000000in}}{%
\pgfpathmoveto{\pgfqpoint{0.000000in}{0.000000in}}%
\pgfpathlineto{\pgfqpoint{0.000000in}{-0.041667in}}%
\pgfusepath{stroke,fill}%
}%
\begin{pgfscope}%
\pgfsys@transformshift{2.084572in}{4.479825in}%
\pgfsys@useobject{currentmarker}{}%
\end{pgfscope}%
\end{pgfscope}%
\begin{pgfscope}%
\pgfsetbuttcap%
\pgfsetroundjoin%
\definecolor{currentfill}{rgb}{0.000000,0.000000,0.000000}%
\pgfsetfillcolor{currentfill}%
\pgfsetlinewidth{0.501875pt}%
\definecolor{currentstroke}{rgb}{0.000000,0.000000,0.000000}%
\pgfsetstrokecolor{currentstroke}%
\pgfsetdash{}{0pt}%
\pgfsys@defobject{currentmarker}{\pgfqpoint{0.000000in}{0.000000in}}{\pgfqpoint{0.000000in}{0.041667in}}{%
\pgfpathmoveto{\pgfqpoint{0.000000in}{0.000000in}}%
\pgfpathlineto{\pgfqpoint{0.000000in}{0.041667in}}%
\pgfusepath{stroke,fill}%
}%
\begin{pgfscope}%
\pgfsys@transformshift{2.565036in}{3.403703in}%
\pgfsys@useobject{currentmarker}{}%
\end{pgfscope}%
\end{pgfscope}%
\begin{pgfscope}%
\pgfsetbuttcap%
\pgfsetroundjoin%
\definecolor{currentfill}{rgb}{0.000000,0.000000,0.000000}%
\pgfsetfillcolor{currentfill}%
\pgfsetlinewidth{0.501875pt}%
\definecolor{currentstroke}{rgb}{0.000000,0.000000,0.000000}%
\pgfsetstrokecolor{currentstroke}%
\pgfsetdash{}{0pt}%
\pgfsys@defobject{currentmarker}{\pgfqpoint{0.000000in}{-0.041667in}}{\pgfqpoint{0.000000in}{0.000000in}}{%
\pgfpathmoveto{\pgfqpoint{0.000000in}{0.000000in}}%
\pgfpathlineto{\pgfqpoint{0.000000in}{-0.041667in}}%
\pgfusepath{stroke,fill}%
}%
\begin{pgfscope}%
\pgfsys@transformshift{2.565036in}{4.479825in}%
\pgfsys@useobject{currentmarker}{}%
\end{pgfscope}%
\end{pgfscope}%
\begin{pgfscope}%
\pgfsetbuttcap%
\pgfsetroundjoin%
\definecolor{currentfill}{rgb}{0.000000,0.000000,0.000000}%
\pgfsetfillcolor{currentfill}%
\pgfsetlinewidth{0.501875pt}%
\definecolor{currentstroke}{rgb}{0.000000,0.000000,0.000000}%
\pgfsetstrokecolor{currentstroke}%
\pgfsetdash{}{0pt}%
\pgfsys@defobject{currentmarker}{\pgfqpoint{0.000000in}{0.000000in}}{\pgfqpoint{0.000000in}{0.041667in}}{%
\pgfpathmoveto{\pgfqpoint{0.000000in}{0.000000in}}%
\pgfpathlineto{\pgfqpoint{0.000000in}{0.041667in}}%
\pgfusepath{stroke,fill}%
}%
\begin{pgfscope}%
\pgfsys@transformshift{3.045499in}{3.403703in}%
\pgfsys@useobject{currentmarker}{}%
\end{pgfscope}%
\end{pgfscope}%
\begin{pgfscope}%
\pgfsetbuttcap%
\pgfsetroundjoin%
\definecolor{currentfill}{rgb}{0.000000,0.000000,0.000000}%
\pgfsetfillcolor{currentfill}%
\pgfsetlinewidth{0.501875pt}%
\definecolor{currentstroke}{rgb}{0.000000,0.000000,0.000000}%
\pgfsetstrokecolor{currentstroke}%
\pgfsetdash{}{0pt}%
\pgfsys@defobject{currentmarker}{\pgfqpoint{0.000000in}{-0.041667in}}{\pgfqpoint{0.000000in}{0.000000in}}{%
\pgfpathmoveto{\pgfqpoint{0.000000in}{0.000000in}}%
\pgfpathlineto{\pgfqpoint{0.000000in}{-0.041667in}}%
\pgfusepath{stroke,fill}%
}%
\begin{pgfscope}%
\pgfsys@transformshift{3.045499in}{4.479825in}%
\pgfsys@useobject{currentmarker}{}%
\end{pgfscope}%
\end{pgfscope}%
\begin{pgfscope}%
\pgfsetbuttcap%
\pgfsetroundjoin%
\definecolor{currentfill}{rgb}{0.000000,0.000000,0.000000}%
\pgfsetfillcolor{currentfill}%
\pgfsetlinewidth{0.501875pt}%
\definecolor{currentstroke}{rgb}{0.000000,0.000000,0.000000}%
\pgfsetstrokecolor{currentstroke}%
\pgfsetdash{}{0pt}%
\pgfsys@defobject{currentmarker}{\pgfqpoint{0.000000in}{0.000000in}}{\pgfqpoint{0.000000in}{0.041667in}}{%
\pgfpathmoveto{\pgfqpoint{0.000000in}{0.000000in}}%
\pgfpathlineto{\pgfqpoint{0.000000in}{0.041667in}}%
\pgfusepath{stroke,fill}%
}%
\begin{pgfscope}%
\pgfsys@transformshift{3.525963in}{3.403703in}%
\pgfsys@useobject{currentmarker}{}%
\end{pgfscope}%
\end{pgfscope}%
\begin{pgfscope}%
\pgfsetbuttcap%
\pgfsetroundjoin%
\definecolor{currentfill}{rgb}{0.000000,0.000000,0.000000}%
\pgfsetfillcolor{currentfill}%
\pgfsetlinewidth{0.501875pt}%
\definecolor{currentstroke}{rgb}{0.000000,0.000000,0.000000}%
\pgfsetstrokecolor{currentstroke}%
\pgfsetdash{}{0pt}%
\pgfsys@defobject{currentmarker}{\pgfqpoint{0.000000in}{-0.041667in}}{\pgfqpoint{0.000000in}{0.000000in}}{%
\pgfpathmoveto{\pgfqpoint{0.000000in}{0.000000in}}%
\pgfpathlineto{\pgfqpoint{0.000000in}{-0.041667in}}%
\pgfusepath{stroke,fill}%
}%
\begin{pgfscope}%
\pgfsys@transformshift{3.525963in}{4.479825in}%
\pgfsys@useobject{currentmarker}{}%
\end{pgfscope}%
\end{pgfscope}%
\begin{pgfscope}%
\pgfsetbuttcap%
\pgfsetroundjoin%
\definecolor{currentfill}{rgb}{0.000000,0.000000,0.000000}%
\pgfsetfillcolor{currentfill}%
\pgfsetlinewidth{0.501875pt}%
\definecolor{currentstroke}{rgb}{0.000000,0.000000,0.000000}%
\pgfsetstrokecolor{currentstroke}%
\pgfsetdash{}{0pt}%
\pgfsys@defobject{currentmarker}{\pgfqpoint{0.000000in}{0.000000in}}{\pgfqpoint{0.000000in}{0.041667in}}{%
\pgfpathmoveto{\pgfqpoint{0.000000in}{0.000000in}}%
\pgfpathlineto{\pgfqpoint{0.000000in}{0.041667in}}%
\pgfusepath{stroke,fill}%
}%
\begin{pgfscope}%
\pgfsys@transformshift{4.006426in}{3.403703in}%
\pgfsys@useobject{currentmarker}{}%
\end{pgfscope}%
\end{pgfscope}%
\begin{pgfscope}%
\pgfsetbuttcap%
\pgfsetroundjoin%
\definecolor{currentfill}{rgb}{0.000000,0.000000,0.000000}%
\pgfsetfillcolor{currentfill}%
\pgfsetlinewidth{0.501875pt}%
\definecolor{currentstroke}{rgb}{0.000000,0.000000,0.000000}%
\pgfsetstrokecolor{currentstroke}%
\pgfsetdash{}{0pt}%
\pgfsys@defobject{currentmarker}{\pgfqpoint{0.000000in}{-0.041667in}}{\pgfqpoint{0.000000in}{0.000000in}}{%
\pgfpathmoveto{\pgfqpoint{0.000000in}{0.000000in}}%
\pgfpathlineto{\pgfqpoint{0.000000in}{-0.041667in}}%
\pgfusepath{stroke,fill}%
}%
\begin{pgfscope}%
\pgfsys@transformshift{4.006426in}{4.479825in}%
\pgfsys@useobject{currentmarker}{}%
\end{pgfscope}%
\end{pgfscope}%
\begin{pgfscope}%
\pgfsetbuttcap%
\pgfsetroundjoin%
\definecolor{currentfill}{rgb}{0.000000,0.000000,0.000000}%
\pgfsetfillcolor{currentfill}%
\pgfsetlinewidth{0.501875pt}%
\definecolor{currentstroke}{rgb}{0.000000,0.000000,0.000000}%
\pgfsetstrokecolor{currentstroke}%
\pgfsetdash{}{0pt}%
\pgfsys@defobject{currentmarker}{\pgfqpoint{0.000000in}{0.000000in}}{\pgfqpoint{0.000000in}{0.041667in}}{%
\pgfpathmoveto{\pgfqpoint{0.000000in}{0.000000in}}%
\pgfpathlineto{\pgfqpoint{0.000000in}{0.041667in}}%
\pgfusepath{stroke,fill}%
}%
\begin{pgfscope}%
\pgfsys@transformshift{4.486890in}{3.403703in}%
\pgfsys@useobject{currentmarker}{}%
\end{pgfscope}%
\end{pgfscope}%
\begin{pgfscope}%
\pgfsetbuttcap%
\pgfsetroundjoin%
\definecolor{currentfill}{rgb}{0.000000,0.000000,0.000000}%
\pgfsetfillcolor{currentfill}%
\pgfsetlinewidth{0.501875pt}%
\definecolor{currentstroke}{rgb}{0.000000,0.000000,0.000000}%
\pgfsetstrokecolor{currentstroke}%
\pgfsetdash{}{0pt}%
\pgfsys@defobject{currentmarker}{\pgfqpoint{0.000000in}{-0.041667in}}{\pgfqpoint{0.000000in}{0.000000in}}{%
\pgfpathmoveto{\pgfqpoint{0.000000in}{0.000000in}}%
\pgfpathlineto{\pgfqpoint{0.000000in}{-0.041667in}}%
\pgfusepath{stroke,fill}%
}%
\begin{pgfscope}%
\pgfsys@transformshift{4.486890in}{4.479825in}%
\pgfsys@useobject{currentmarker}{}%
\end{pgfscope}%
\end{pgfscope}%
\begin{pgfscope}%
\pgfsetbuttcap%
\pgfsetroundjoin%
\definecolor{currentfill}{rgb}{0.000000,0.000000,0.000000}%
\pgfsetfillcolor{currentfill}%
\pgfsetlinewidth{0.501875pt}%
\definecolor{currentstroke}{rgb}{0.000000,0.000000,0.000000}%
\pgfsetstrokecolor{currentstroke}%
\pgfsetdash{}{0pt}%
\pgfsys@defobject{currentmarker}{\pgfqpoint{0.000000in}{0.000000in}}{\pgfqpoint{0.000000in}{0.020833in}}{%
\pgfpathmoveto{\pgfqpoint{0.000000in}{0.000000in}}%
\pgfpathlineto{\pgfqpoint{0.000000in}{0.020833in}}%
\pgfusepath{stroke,fill}%
}%
\begin{pgfscope}%
\pgfsys@transformshift{0.450996in}{3.403703in}%
\pgfsys@useobject{currentmarker}{}%
\end{pgfscope}%
\end{pgfscope}%
\begin{pgfscope}%
\pgfsetbuttcap%
\pgfsetroundjoin%
\definecolor{currentfill}{rgb}{0.000000,0.000000,0.000000}%
\pgfsetfillcolor{currentfill}%
\pgfsetlinewidth{0.501875pt}%
\definecolor{currentstroke}{rgb}{0.000000,0.000000,0.000000}%
\pgfsetstrokecolor{currentstroke}%
\pgfsetdash{}{0pt}%
\pgfsys@defobject{currentmarker}{\pgfqpoint{0.000000in}{-0.020833in}}{\pgfqpoint{0.000000in}{0.000000in}}{%
\pgfpathmoveto{\pgfqpoint{0.000000in}{0.000000in}}%
\pgfpathlineto{\pgfqpoint{0.000000in}{-0.020833in}}%
\pgfusepath{stroke,fill}%
}%
\begin{pgfscope}%
\pgfsys@transformshift{0.450996in}{4.479825in}%
\pgfsys@useobject{currentmarker}{}%
\end{pgfscope}%
\end{pgfscope}%
\begin{pgfscope}%
\pgfsetbuttcap%
\pgfsetroundjoin%
\definecolor{currentfill}{rgb}{0.000000,0.000000,0.000000}%
\pgfsetfillcolor{currentfill}%
\pgfsetlinewidth{0.501875pt}%
\definecolor{currentstroke}{rgb}{0.000000,0.000000,0.000000}%
\pgfsetstrokecolor{currentstroke}%
\pgfsetdash{}{0pt}%
\pgfsys@defobject{currentmarker}{\pgfqpoint{0.000000in}{0.000000in}}{\pgfqpoint{0.000000in}{0.020833in}}{%
\pgfpathmoveto{\pgfqpoint{0.000000in}{0.000000in}}%
\pgfpathlineto{\pgfqpoint{0.000000in}{0.020833in}}%
\pgfusepath{stroke,fill}%
}%
\begin{pgfscope}%
\pgfsys@transformshift{0.547089in}{3.403703in}%
\pgfsys@useobject{currentmarker}{}%
\end{pgfscope}%
\end{pgfscope}%
\begin{pgfscope}%
\pgfsetbuttcap%
\pgfsetroundjoin%
\definecolor{currentfill}{rgb}{0.000000,0.000000,0.000000}%
\pgfsetfillcolor{currentfill}%
\pgfsetlinewidth{0.501875pt}%
\definecolor{currentstroke}{rgb}{0.000000,0.000000,0.000000}%
\pgfsetstrokecolor{currentstroke}%
\pgfsetdash{}{0pt}%
\pgfsys@defobject{currentmarker}{\pgfqpoint{0.000000in}{-0.020833in}}{\pgfqpoint{0.000000in}{0.000000in}}{%
\pgfpathmoveto{\pgfqpoint{0.000000in}{0.000000in}}%
\pgfpathlineto{\pgfqpoint{0.000000in}{-0.020833in}}%
\pgfusepath{stroke,fill}%
}%
\begin{pgfscope}%
\pgfsys@transformshift{0.547089in}{4.479825in}%
\pgfsys@useobject{currentmarker}{}%
\end{pgfscope}%
\end{pgfscope}%
\begin{pgfscope}%
\pgfsetbuttcap%
\pgfsetroundjoin%
\definecolor{currentfill}{rgb}{0.000000,0.000000,0.000000}%
\pgfsetfillcolor{currentfill}%
\pgfsetlinewidth{0.501875pt}%
\definecolor{currentstroke}{rgb}{0.000000,0.000000,0.000000}%
\pgfsetstrokecolor{currentstroke}%
\pgfsetdash{}{0pt}%
\pgfsys@defobject{currentmarker}{\pgfqpoint{0.000000in}{0.000000in}}{\pgfqpoint{0.000000in}{0.020833in}}{%
\pgfpathmoveto{\pgfqpoint{0.000000in}{0.000000in}}%
\pgfpathlineto{\pgfqpoint{0.000000in}{0.020833in}}%
\pgfusepath{stroke,fill}%
}%
\begin{pgfscope}%
\pgfsys@transformshift{0.739275in}{3.403703in}%
\pgfsys@useobject{currentmarker}{}%
\end{pgfscope}%
\end{pgfscope}%
\begin{pgfscope}%
\pgfsetbuttcap%
\pgfsetroundjoin%
\definecolor{currentfill}{rgb}{0.000000,0.000000,0.000000}%
\pgfsetfillcolor{currentfill}%
\pgfsetlinewidth{0.501875pt}%
\definecolor{currentstroke}{rgb}{0.000000,0.000000,0.000000}%
\pgfsetstrokecolor{currentstroke}%
\pgfsetdash{}{0pt}%
\pgfsys@defobject{currentmarker}{\pgfqpoint{0.000000in}{-0.020833in}}{\pgfqpoint{0.000000in}{0.000000in}}{%
\pgfpathmoveto{\pgfqpoint{0.000000in}{0.000000in}}%
\pgfpathlineto{\pgfqpoint{0.000000in}{-0.020833in}}%
\pgfusepath{stroke,fill}%
}%
\begin{pgfscope}%
\pgfsys@transformshift{0.739275in}{4.479825in}%
\pgfsys@useobject{currentmarker}{}%
\end{pgfscope}%
\end{pgfscope}%
\begin{pgfscope}%
\pgfsetbuttcap%
\pgfsetroundjoin%
\definecolor{currentfill}{rgb}{0.000000,0.000000,0.000000}%
\pgfsetfillcolor{currentfill}%
\pgfsetlinewidth{0.501875pt}%
\definecolor{currentstroke}{rgb}{0.000000,0.000000,0.000000}%
\pgfsetstrokecolor{currentstroke}%
\pgfsetdash{}{0pt}%
\pgfsys@defobject{currentmarker}{\pgfqpoint{0.000000in}{0.000000in}}{\pgfqpoint{0.000000in}{0.020833in}}{%
\pgfpathmoveto{\pgfqpoint{0.000000in}{0.000000in}}%
\pgfpathlineto{\pgfqpoint{0.000000in}{0.020833in}}%
\pgfusepath{stroke,fill}%
}%
\begin{pgfscope}%
\pgfsys@transformshift{0.835367in}{3.403703in}%
\pgfsys@useobject{currentmarker}{}%
\end{pgfscope}%
\end{pgfscope}%
\begin{pgfscope}%
\pgfsetbuttcap%
\pgfsetroundjoin%
\definecolor{currentfill}{rgb}{0.000000,0.000000,0.000000}%
\pgfsetfillcolor{currentfill}%
\pgfsetlinewidth{0.501875pt}%
\definecolor{currentstroke}{rgb}{0.000000,0.000000,0.000000}%
\pgfsetstrokecolor{currentstroke}%
\pgfsetdash{}{0pt}%
\pgfsys@defobject{currentmarker}{\pgfqpoint{0.000000in}{-0.020833in}}{\pgfqpoint{0.000000in}{0.000000in}}{%
\pgfpathmoveto{\pgfqpoint{0.000000in}{0.000000in}}%
\pgfpathlineto{\pgfqpoint{0.000000in}{-0.020833in}}%
\pgfusepath{stroke,fill}%
}%
\begin{pgfscope}%
\pgfsys@transformshift{0.835367in}{4.479825in}%
\pgfsys@useobject{currentmarker}{}%
\end{pgfscope}%
\end{pgfscope}%
\begin{pgfscope}%
\pgfsetbuttcap%
\pgfsetroundjoin%
\definecolor{currentfill}{rgb}{0.000000,0.000000,0.000000}%
\pgfsetfillcolor{currentfill}%
\pgfsetlinewidth{0.501875pt}%
\definecolor{currentstroke}{rgb}{0.000000,0.000000,0.000000}%
\pgfsetstrokecolor{currentstroke}%
\pgfsetdash{}{0pt}%
\pgfsys@defobject{currentmarker}{\pgfqpoint{0.000000in}{0.000000in}}{\pgfqpoint{0.000000in}{0.020833in}}{%
\pgfpathmoveto{\pgfqpoint{0.000000in}{0.000000in}}%
\pgfpathlineto{\pgfqpoint{0.000000in}{0.020833in}}%
\pgfusepath{stroke,fill}%
}%
\begin{pgfscope}%
\pgfsys@transformshift{0.931460in}{3.403703in}%
\pgfsys@useobject{currentmarker}{}%
\end{pgfscope}%
\end{pgfscope}%
\begin{pgfscope}%
\pgfsetbuttcap%
\pgfsetroundjoin%
\definecolor{currentfill}{rgb}{0.000000,0.000000,0.000000}%
\pgfsetfillcolor{currentfill}%
\pgfsetlinewidth{0.501875pt}%
\definecolor{currentstroke}{rgb}{0.000000,0.000000,0.000000}%
\pgfsetstrokecolor{currentstroke}%
\pgfsetdash{}{0pt}%
\pgfsys@defobject{currentmarker}{\pgfqpoint{0.000000in}{-0.020833in}}{\pgfqpoint{0.000000in}{0.000000in}}{%
\pgfpathmoveto{\pgfqpoint{0.000000in}{0.000000in}}%
\pgfpathlineto{\pgfqpoint{0.000000in}{-0.020833in}}%
\pgfusepath{stroke,fill}%
}%
\begin{pgfscope}%
\pgfsys@transformshift{0.931460in}{4.479825in}%
\pgfsys@useobject{currentmarker}{}%
\end{pgfscope}%
\end{pgfscope}%
\begin{pgfscope}%
\pgfsetbuttcap%
\pgfsetroundjoin%
\definecolor{currentfill}{rgb}{0.000000,0.000000,0.000000}%
\pgfsetfillcolor{currentfill}%
\pgfsetlinewidth{0.501875pt}%
\definecolor{currentstroke}{rgb}{0.000000,0.000000,0.000000}%
\pgfsetstrokecolor{currentstroke}%
\pgfsetdash{}{0pt}%
\pgfsys@defobject{currentmarker}{\pgfqpoint{0.000000in}{0.000000in}}{\pgfqpoint{0.000000in}{0.020833in}}{%
\pgfpathmoveto{\pgfqpoint{0.000000in}{0.000000in}}%
\pgfpathlineto{\pgfqpoint{0.000000in}{0.020833in}}%
\pgfusepath{stroke,fill}%
}%
\begin{pgfscope}%
\pgfsys@transformshift{1.027553in}{3.403703in}%
\pgfsys@useobject{currentmarker}{}%
\end{pgfscope}%
\end{pgfscope}%
\begin{pgfscope}%
\pgfsetbuttcap%
\pgfsetroundjoin%
\definecolor{currentfill}{rgb}{0.000000,0.000000,0.000000}%
\pgfsetfillcolor{currentfill}%
\pgfsetlinewidth{0.501875pt}%
\definecolor{currentstroke}{rgb}{0.000000,0.000000,0.000000}%
\pgfsetstrokecolor{currentstroke}%
\pgfsetdash{}{0pt}%
\pgfsys@defobject{currentmarker}{\pgfqpoint{0.000000in}{-0.020833in}}{\pgfqpoint{0.000000in}{0.000000in}}{%
\pgfpathmoveto{\pgfqpoint{0.000000in}{0.000000in}}%
\pgfpathlineto{\pgfqpoint{0.000000in}{-0.020833in}}%
\pgfusepath{stroke,fill}%
}%
\begin{pgfscope}%
\pgfsys@transformshift{1.027553in}{4.479825in}%
\pgfsys@useobject{currentmarker}{}%
\end{pgfscope}%
\end{pgfscope}%
\begin{pgfscope}%
\pgfsetbuttcap%
\pgfsetroundjoin%
\definecolor{currentfill}{rgb}{0.000000,0.000000,0.000000}%
\pgfsetfillcolor{currentfill}%
\pgfsetlinewidth{0.501875pt}%
\definecolor{currentstroke}{rgb}{0.000000,0.000000,0.000000}%
\pgfsetstrokecolor{currentstroke}%
\pgfsetdash{}{0pt}%
\pgfsys@defobject{currentmarker}{\pgfqpoint{0.000000in}{0.000000in}}{\pgfqpoint{0.000000in}{0.020833in}}{%
\pgfpathmoveto{\pgfqpoint{0.000000in}{0.000000in}}%
\pgfpathlineto{\pgfqpoint{0.000000in}{0.020833in}}%
\pgfusepath{stroke,fill}%
}%
\begin{pgfscope}%
\pgfsys@transformshift{1.219738in}{3.403703in}%
\pgfsys@useobject{currentmarker}{}%
\end{pgfscope}%
\end{pgfscope}%
\begin{pgfscope}%
\pgfsetbuttcap%
\pgfsetroundjoin%
\definecolor{currentfill}{rgb}{0.000000,0.000000,0.000000}%
\pgfsetfillcolor{currentfill}%
\pgfsetlinewidth{0.501875pt}%
\definecolor{currentstroke}{rgb}{0.000000,0.000000,0.000000}%
\pgfsetstrokecolor{currentstroke}%
\pgfsetdash{}{0pt}%
\pgfsys@defobject{currentmarker}{\pgfqpoint{0.000000in}{-0.020833in}}{\pgfqpoint{0.000000in}{0.000000in}}{%
\pgfpathmoveto{\pgfqpoint{0.000000in}{0.000000in}}%
\pgfpathlineto{\pgfqpoint{0.000000in}{-0.020833in}}%
\pgfusepath{stroke,fill}%
}%
\begin{pgfscope}%
\pgfsys@transformshift{1.219738in}{4.479825in}%
\pgfsys@useobject{currentmarker}{}%
\end{pgfscope}%
\end{pgfscope}%
\begin{pgfscope}%
\pgfsetbuttcap%
\pgfsetroundjoin%
\definecolor{currentfill}{rgb}{0.000000,0.000000,0.000000}%
\pgfsetfillcolor{currentfill}%
\pgfsetlinewidth{0.501875pt}%
\definecolor{currentstroke}{rgb}{0.000000,0.000000,0.000000}%
\pgfsetstrokecolor{currentstroke}%
\pgfsetdash{}{0pt}%
\pgfsys@defobject{currentmarker}{\pgfqpoint{0.000000in}{0.000000in}}{\pgfqpoint{0.000000in}{0.020833in}}{%
\pgfpathmoveto{\pgfqpoint{0.000000in}{0.000000in}}%
\pgfpathlineto{\pgfqpoint{0.000000in}{0.020833in}}%
\pgfusepath{stroke,fill}%
}%
\begin{pgfscope}%
\pgfsys@transformshift{1.315831in}{3.403703in}%
\pgfsys@useobject{currentmarker}{}%
\end{pgfscope}%
\end{pgfscope}%
\begin{pgfscope}%
\pgfsetbuttcap%
\pgfsetroundjoin%
\definecolor{currentfill}{rgb}{0.000000,0.000000,0.000000}%
\pgfsetfillcolor{currentfill}%
\pgfsetlinewidth{0.501875pt}%
\definecolor{currentstroke}{rgb}{0.000000,0.000000,0.000000}%
\pgfsetstrokecolor{currentstroke}%
\pgfsetdash{}{0pt}%
\pgfsys@defobject{currentmarker}{\pgfqpoint{0.000000in}{-0.020833in}}{\pgfqpoint{0.000000in}{0.000000in}}{%
\pgfpathmoveto{\pgfqpoint{0.000000in}{0.000000in}}%
\pgfpathlineto{\pgfqpoint{0.000000in}{-0.020833in}}%
\pgfusepath{stroke,fill}%
}%
\begin{pgfscope}%
\pgfsys@transformshift{1.315831in}{4.479825in}%
\pgfsys@useobject{currentmarker}{}%
\end{pgfscope}%
\end{pgfscope}%
\begin{pgfscope}%
\pgfsetbuttcap%
\pgfsetroundjoin%
\definecolor{currentfill}{rgb}{0.000000,0.000000,0.000000}%
\pgfsetfillcolor{currentfill}%
\pgfsetlinewidth{0.501875pt}%
\definecolor{currentstroke}{rgb}{0.000000,0.000000,0.000000}%
\pgfsetstrokecolor{currentstroke}%
\pgfsetdash{}{0pt}%
\pgfsys@defobject{currentmarker}{\pgfqpoint{0.000000in}{0.000000in}}{\pgfqpoint{0.000000in}{0.020833in}}{%
\pgfpathmoveto{\pgfqpoint{0.000000in}{0.000000in}}%
\pgfpathlineto{\pgfqpoint{0.000000in}{0.020833in}}%
\pgfusepath{stroke,fill}%
}%
\begin{pgfscope}%
\pgfsys@transformshift{1.411923in}{3.403703in}%
\pgfsys@useobject{currentmarker}{}%
\end{pgfscope}%
\end{pgfscope}%
\begin{pgfscope}%
\pgfsetbuttcap%
\pgfsetroundjoin%
\definecolor{currentfill}{rgb}{0.000000,0.000000,0.000000}%
\pgfsetfillcolor{currentfill}%
\pgfsetlinewidth{0.501875pt}%
\definecolor{currentstroke}{rgb}{0.000000,0.000000,0.000000}%
\pgfsetstrokecolor{currentstroke}%
\pgfsetdash{}{0pt}%
\pgfsys@defobject{currentmarker}{\pgfqpoint{0.000000in}{-0.020833in}}{\pgfqpoint{0.000000in}{0.000000in}}{%
\pgfpathmoveto{\pgfqpoint{0.000000in}{0.000000in}}%
\pgfpathlineto{\pgfqpoint{0.000000in}{-0.020833in}}%
\pgfusepath{stroke,fill}%
}%
\begin{pgfscope}%
\pgfsys@transformshift{1.411923in}{4.479825in}%
\pgfsys@useobject{currentmarker}{}%
\end{pgfscope}%
\end{pgfscope}%
\begin{pgfscope}%
\pgfsetbuttcap%
\pgfsetroundjoin%
\definecolor{currentfill}{rgb}{0.000000,0.000000,0.000000}%
\pgfsetfillcolor{currentfill}%
\pgfsetlinewidth{0.501875pt}%
\definecolor{currentstroke}{rgb}{0.000000,0.000000,0.000000}%
\pgfsetstrokecolor{currentstroke}%
\pgfsetdash{}{0pt}%
\pgfsys@defobject{currentmarker}{\pgfqpoint{0.000000in}{0.000000in}}{\pgfqpoint{0.000000in}{0.020833in}}{%
\pgfpathmoveto{\pgfqpoint{0.000000in}{0.000000in}}%
\pgfpathlineto{\pgfqpoint{0.000000in}{0.020833in}}%
\pgfusepath{stroke,fill}%
}%
\begin{pgfscope}%
\pgfsys@transformshift{1.508016in}{3.403703in}%
\pgfsys@useobject{currentmarker}{}%
\end{pgfscope}%
\end{pgfscope}%
\begin{pgfscope}%
\pgfsetbuttcap%
\pgfsetroundjoin%
\definecolor{currentfill}{rgb}{0.000000,0.000000,0.000000}%
\pgfsetfillcolor{currentfill}%
\pgfsetlinewidth{0.501875pt}%
\definecolor{currentstroke}{rgb}{0.000000,0.000000,0.000000}%
\pgfsetstrokecolor{currentstroke}%
\pgfsetdash{}{0pt}%
\pgfsys@defobject{currentmarker}{\pgfqpoint{0.000000in}{-0.020833in}}{\pgfqpoint{0.000000in}{0.000000in}}{%
\pgfpathmoveto{\pgfqpoint{0.000000in}{0.000000in}}%
\pgfpathlineto{\pgfqpoint{0.000000in}{-0.020833in}}%
\pgfusepath{stroke,fill}%
}%
\begin{pgfscope}%
\pgfsys@transformshift{1.508016in}{4.479825in}%
\pgfsys@useobject{currentmarker}{}%
\end{pgfscope}%
\end{pgfscope}%
\begin{pgfscope}%
\pgfsetbuttcap%
\pgfsetroundjoin%
\definecolor{currentfill}{rgb}{0.000000,0.000000,0.000000}%
\pgfsetfillcolor{currentfill}%
\pgfsetlinewidth{0.501875pt}%
\definecolor{currentstroke}{rgb}{0.000000,0.000000,0.000000}%
\pgfsetstrokecolor{currentstroke}%
\pgfsetdash{}{0pt}%
\pgfsys@defobject{currentmarker}{\pgfqpoint{0.000000in}{0.000000in}}{\pgfqpoint{0.000000in}{0.020833in}}{%
\pgfpathmoveto{\pgfqpoint{0.000000in}{0.000000in}}%
\pgfpathlineto{\pgfqpoint{0.000000in}{0.020833in}}%
\pgfusepath{stroke,fill}%
}%
\begin{pgfscope}%
\pgfsys@transformshift{1.700201in}{3.403703in}%
\pgfsys@useobject{currentmarker}{}%
\end{pgfscope}%
\end{pgfscope}%
\begin{pgfscope}%
\pgfsetbuttcap%
\pgfsetroundjoin%
\definecolor{currentfill}{rgb}{0.000000,0.000000,0.000000}%
\pgfsetfillcolor{currentfill}%
\pgfsetlinewidth{0.501875pt}%
\definecolor{currentstroke}{rgb}{0.000000,0.000000,0.000000}%
\pgfsetstrokecolor{currentstroke}%
\pgfsetdash{}{0pt}%
\pgfsys@defobject{currentmarker}{\pgfqpoint{0.000000in}{-0.020833in}}{\pgfqpoint{0.000000in}{0.000000in}}{%
\pgfpathmoveto{\pgfqpoint{0.000000in}{0.000000in}}%
\pgfpathlineto{\pgfqpoint{0.000000in}{-0.020833in}}%
\pgfusepath{stroke,fill}%
}%
\begin{pgfscope}%
\pgfsys@transformshift{1.700201in}{4.479825in}%
\pgfsys@useobject{currentmarker}{}%
\end{pgfscope}%
\end{pgfscope}%
\begin{pgfscope}%
\pgfsetbuttcap%
\pgfsetroundjoin%
\definecolor{currentfill}{rgb}{0.000000,0.000000,0.000000}%
\pgfsetfillcolor{currentfill}%
\pgfsetlinewidth{0.501875pt}%
\definecolor{currentstroke}{rgb}{0.000000,0.000000,0.000000}%
\pgfsetstrokecolor{currentstroke}%
\pgfsetdash{}{0pt}%
\pgfsys@defobject{currentmarker}{\pgfqpoint{0.000000in}{0.000000in}}{\pgfqpoint{0.000000in}{0.020833in}}{%
\pgfpathmoveto{\pgfqpoint{0.000000in}{0.000000in}}%
\pgfpathlineto{\pgfqpoint{0.000000in}{0.020833in}}%
\pgfusepath{stroke,fill}%
}%
\begin{pgfscope}%
\pgfsys@transformshift{1.796294in}{3.403703in}%
\pgfsys@useobject{currentmarker}{}%
\end{pgfscope}%
\end{pgfscope}%
\begin{pgfscope}%
\pgfsetbuttcap%
\pgfsetroundjoin%
\definecolor{currentfill}{rgb}{0.000000,0.000000,0.000000}%
\pgfsetfillcolor{currentfill}%
\pgfsetlinewidth{0.501875pt}%
\definecolor{currentstroke}{rgb}{0.000000,0.000000,0.000000}%
\pgfsetstrokecolor{currentstroke}%
\pgfsetdash{}{0pt}%
\pgfsys@defobject{currentmarker}{\pgfqpoint{0.000000in}{-0.020833in}}{\pgfqpoint{0.000000in}{0.000000in}}{%
\pgfpathmoveto{\pgfqpoint{0.000000in}{0.000000in}}%
\pgfpathlineto{\pgfqpoint{0.000000in}{-0.020833in}}%
\pgfusepath{stroke,fill}%
}%
\begin{pgfscope}%
\pgfsys@transformshift{1.796294in}{4.479825in}%
\pgfsys@useobject{currentmarker}{}%
\end{pgfscope}%
\end{pgfscope}%
\begin{pgfscope}%
\pgfsetbuttcap%
\pgfsetroundjoin%
\definecolor{currentfill}{rgb}{0.000000,0.000000,0.000000}%
\pgfsetfillcolor{currentfill}%
\pgfsetlinewidth{0.501875pt}%
\definecolor{currentstroke}{rgb}{0.000000,0.000000,0.000000}%
\pgfsetstrokecolor{currentstroke}%
\pgfsetdash{}{0pt}%
\pgfsys@defobject{currentmarker}{\pgfqpoint{0.000000in}{0.000000in}}{\pgfqpoint{0.000000in}{0.020833in}}{%
\pgfpathmoveto{\pgfqpoint{0.000000in}{0.000000in}}%
\pgfpathlineto{\pgfqpoint{0.000000in}{0.020833in}}%
\pgfusepath{stroke,fill}%
}%
\begin{pgfscope}%
\pgfsys@transformshift{1.892387in}{3.403703in}%
\pgfsys@useobject{currentmarker}{}%
\end{pgfscope}%
\end{pgfscope}%
\begin{pgfscope}%
\pgfsetbuttcap%
\pgfsetroundjoin%
\definecolor{currentfill}{rgb}{0.000000,0.000000,0.000000}%
\pgfsetfillcolor{currentfill}%
\pgfsetlinewidth{0.501875pt}%
\definecolor{currentstroke}{rgb}{0.000000,0.000000,0.000000}%
\pgfsetstrokecolor{currentstroke}%
\pgfsetdash{}{0pt}%
\pgfsys@defobject{currentmarker}{\pgfqpoint{0.000000in}{-0.020833in}}{\pgfqpoint{0.000000in}{0.000000in}}{%
\pgfpathmoveto{\pgfqpoint{0.000000in}{0.000000in}}%
\pgfpathlineto{\pgfqpoint{0.000000in}{-0.020833in}}%
\pgfusepath{stroke,fill}%
}%
\begin{pgfscope}%
\pgfsys@transformshift{1.892387in}{4.479825in}%
\pgfsys@useobject{currentmarker}{}%
\end{pgfscope}%
\end{pgfscope}%
\begin{pgfscope}%
\pgfsetbuttcap%
\pgfsetroundjoin%
\definecolor{currentfill}{rgb}{0.000000,0.000000,0.000000}%
\pgfsetfillcolor{currentfill}%
\pgfsetlinewidth{0.501875pt}%
\definecolor{currentstroke}{rgb}{0.000000,0.000000,0.000000}%
\pgfsetstrokecolor{currentstroke}%
\pgfsetdash{}{0pt}%
\pgfsys@defobject{currentmarker}{\pgfqpoint{0.000000in}{0.000000in}}{\pgfqpoint{0.000000in}{0.020833in}}{%
\pgfpathmoveto{\pgfqpoint{0.000000in}{0.000000in}}%
\pgfpathlineto{\pgfqpoint{0.000000in}{0.020833in}}%
\pgfusepath{stroke,fill}%
}%
\begin{pgfscope}%
\pgfsys@transformshift{1.988480in}{3.403703in}%
\pgfsys@useobject{currentmarker}{}%
\end{pgfscope}%
\end{pgfscope}%
\begin{pgfscope}%
\pgfsetbuttcap%
\pgfsetroundjoin%
\definecolor{currentfill}{rgb}{0.000000,0.000000,0.000000}%
\pgfsetfillcolor{currentfill}%
\pgfsetlinewidth{0.501875pt}%
\definecolor{currentstroke}{rgb}{0.000000,0.000000,0.000000}%
\pgfsetstrokecolor{currentstroke}%
\pgfsetdash{}{0pt}%
\pgfsys@defobject{currentmarker}{\pgfqpoint{0.000000in}{-0.020833in}}{\pgfqpoint{0.000000in}{0.000000in}}{%
\pgfpathmoveto{\pgfqpoint{0.000000in}{0.000000in}}%
\pgfpathlineto{\pgfqpoint{0.000000in}{-0.020833in}}%
\pgfusepath{stroke,fill}%
}%
\begin{pgfscope}%
\pgfsys@transformshift{1.988480in}{4.479825in}%
\pgfsys@useobject{currentmarker}{}%
\end{pgfscope}%
\end{pgfscope}%
\begin{pgfscope}%
\pgfsetbuttcap%
\pgfsetroundjoin%
\definecolor{currentfill}{rgb}{0.000000,0.000000,0.000000}%
\pgfsetfillcolor{currentfill}%
\pgfsetlinewidth{0.501875pt}%
\definecolor{currentstroke}{rgb}{0.000000,0.000000,0.000000}%
\pgfsetstrokecolor{currentstroke}%
\pgfsetdash{}{0pt}%
\pgfsys@defobject{currentmarker}{\pgfqpoint{0.000000in}{0.000000in}}{\pgfqpoint{0.000000in}{0.020833in}}{%
\pgfpathmoveto{\pgfqpoint{0.000000in}{0.000000in}}%
\pgfpathlineto{\pgfqpoint{0.000000in}{0.020833in}}%
\pgfusepath{stroke,fill}%
}%
\begin{pgfscope}%
\pgfsys@transformshift{2.180665in}{3.403703in}%
\pgfsys@useobject{currentmarker}{}%
\end{pgfscope}%
\end{pgfscope}%
\begin{pgfscope}%
\pgfsetbuttcap%
\pgfsetroundjoin%
\definecolor{currentfill}{rgb}{0.000000,0.000000,0.000000}%
\pgfsetfillcolor{currentfill}%
\pgfsetlinewidth{0.501875pt}%
\definecolor{currentstroke}{rgb}{0.000000,0.000000,0.000000}%
\pgfsetstrokecolor{currentstroke}%
\pgfsetdash{}{0pt}%
\pgfsys@defobject{currentmarker}{\pgfqpoint{0.000000in}{-0.020833in}}{\pgfqpoint{0.000000in}{0.000000in}}{%
\pgfpathmoveto{\pgfqpoint{0.000000in}{0.000000in}}%
\pgfpathlineto{\pgfqpoint{0.000000in}{-0.020833in}}%
\pgfusepath{stroke,fill}%
}%
\begin{pgfscope}%
\pgfsys@transformshift{2.180665in}{4.479825in}%
\pgfsys@useobject{currentmarker}{}%
\end{pgfscope}%
\end{pgfscope}%
\begin{pgfscope}%
\pgfsetbuttcap%
\pgfsetroundjoin%
\definecolor{currentfill}{rgb}{0.000000,0.000000,0.000000}%
\pgfsetfillcolor{currentfill}%
\pgfsetlinewidth{0.501875pt}%
\definecolor{currentstroke}{rgb}{0.000000,0.000000,0.000000}%
\pgfsetstrokecolor{currentstroke}%
\pgfsetdash{}{0pt}%
\pgfsys@defobject{currentmarker}{\pgfqpoint{0.000000in}{0.000000in}}{\pgfqpoint{0.000000in}{0.020833in}}{%
\pgfpathmoveto{\pgfqpoint{0.000000in}{0.000000in}}%
\pgfpathlineto{\pgfqpoint{0.000000in}{0.020833in}}%
\pgfusepath{stroke,fill}%
}%
\begin{pgfscope}%
\pgfsys@transformshift{2.276758in}{3.403703in}%
\pgfsys@useobject{currentmarker}{}%
\end{pgfscope}%
\end{pgfscope}%
\begin{pgfscope}%
\pgfsetbuttcap%
\pgfsetroundjoin%
\definecolor{currentfill}{rgb}{0.000000,0.000000,0.000000}%
\pgfsetfillcolor{currentfill}%
\pgfsetlinewidth{0.501875pt}%
\definecolor{currentstroke}{rgb}{0.000000,0.000000,0.000000}%
\pgfsetstrokecolor{currentstroke}%
\pgfsetdash{}{0pt}%
\pgfsys@defobject{currentmarker}{\pgfqpoint{0.000000in}{-0.020833in}}{\pgfqpoint{0.000000in}{0.000000in}}{%
\pgfpathmoveto{\pgfqpoint{0.000000in}{0.000000in}}%
\pgfpathlineto{\pgfqpoint{0.000000in}{-0.020833in}}%
\pgfusepath{stroke,fill}%
}%
\begin{pgfscope}%
\pgfsys@transformshift{2.276758in}{4.479825in}%
\pgfsys@useobject{currentmarker}{}%
\end{pgfscope}%
\end{pgfscope}%
\begin{pgfscope}%
\pgfsetbuttcap%
\pgfsetroundjoin%
\definecolor{currentfill}{rgb}{0.000000,0.000000,0.000000}%
\pgfsetfillcolor{currentfill}%
\pgfsetlinewidth{0.501875pt}%
\definecolor{currentstroke}{rgb}{0.000000,0.000000,0.000000}%
\pgfsetstrokecolor{currentstroke}%
\pgfsetdash{}{0pt}%
\pgfsys@defobject{currentmarker}{\pgfqpoint{0.000000in}{0.000000in}}{\pgfqpoint{0.000000in}{0.020833in}}{%
\pgfpathmoveto{\pgfqpoint{0.000000in}{0.000000in}}%
\pgfpathlineto{\pgfqpoint{0.000000in}{0.020833in}}%
\pgfusepath{stroke,fill}%
}%
\begin{pgfscope}%
\pgfsys@transformshift{2.372850in}{3.403703in}%
\pgfsys@useobject{currentmarker}{}%
\end{pgfscope}%
\end{pgfscope}%
\begin{pgfscope}%
\pgfsetbuttcap%
\pgfsetroundjoin%
\definecolor{currentfill}{rgb}{0.000000,0.000000,0.000000}%
\pgfsetfillcolor{currentfill}%
\pgfsetlinewidth{0.501875pt}%
\definecolor{currentstroke}{rgb}{0.000000,0.000000,0.000000}%
\pgfsetstrokecolor{currentstroke}%
\pgfsetdash{}{0pt}%
\pgfsys@defobject{currentmarker}{\pgfqpoint{0.000000in}{-0.020833in}}{\pgfqpoint{0.000000in}{0.000000in}}{%
\pgfpathmoveto{\pgfqpoint{0.000000in}{0.000000in}}%
\pgfpathlineto{\pgfqpoint{0.000000in}{-0.020833in}}%
\pgfusepath{stroke,fill}%
}%
\begin{pgfscope}%
\pgfsys@transformshift{2.372850in}{4.479825in}%
\pgfsys@useobject{currentmarker}{}%
\end{pgfscope}%
\end{pgfscope}%
\begin{pgfscope}%
\pgfsetbuttcap%
\pgfsetroundjoin%
\definecolor{currentfill}{rgb}{0.000000,0.000000,0.000000}%
\pgfsetfillcolor{currentfill}%
\pgfsetlinewidth{0.501875pt}%
\definecolor{currentstroke}{rgb}{0.000000,0.000000,0.000000}%
\pgfsetstrokecolor{currentstroke}%
\pgfsetdash{}{0pt}%
\pgfsys@defobject{currentmarker}{\pgfqpoint{0.000000in}{0.000000in}}{\pgfqpoint{0.000000in}{0.020833in}}{%
\pgfpathmoveto{\pgfqpoint{0.000000in}{0.000000in}}%
\pgfpathlineto{\pgfqpoint{0.000000in}{0.020833in}}%
\pgfusepath{stroke,fill}%
}%
\begin{pgfscope}%
\pgfsys@transformshift{2.468943in}{3.403703in}%
\pgfsys@useobject{currentmarker}{}%
\end{pgfscope}%
\end{pgfscope}%
\begin{pgfscope}%
\pgfsetbuttcap%
\pgfsetroundjoin%
\definecolor{currentfill}{rgb}{0.000000,0.000000,0.000000}%
\pgfsetfillcolor{currentfill}%
\pgfsetlinewidth{0.501875pt}%
\definecolor{currentstroke}{rgb}{0.000000,0.000000,0.000000}%
\pgfsetstrokecolor{currentstroke}%
\pgfsetdash{}{0pt}%
\pgfsys@defobject{currentmarker}{\pgfqpoint{0.000000in}{-0.020833in}}{\pgfqpoint{0.000000in}{0.000000in}}{%
\pgfpathmoveto{\pgfqpoint{0.000000in}{0.000000in}}%
\pgfpathlineto{\pgfqpoint{0.000000in}{-0.020833in}}%
\pgfusepath{stroke,fill}%
}%
\begin{pgfscope}%
\pgfsys@transformshift{2.468943in}{4.479825in}%
\pgfsys@useobject{currentmarker}{}%
\end{pgfscope}%
\end{pgfscope}%
\begin{pgfscope}%
\pgfsetbuttcap%
\pgfsetroundjoin%
\definecolor{currentfill}{rgb}{0.000000,0.000000,0.000000}%
\pgfsetfillcolor{currentfill}%
\pgfsetlinewidth{0.501875pt}%
\definecolor{currentstroke}{rgb}{0.000000,0.000000,0.000000}%
\pgfsetstrokecolor{currentstroke}%
\pgfsetdash{}{0pt}%
\pgfsys@defobject{currentmarker}{\pgfqpoint{0.000000in}{0.000000in}}{\pgfqpoint{0.000000in}{0.020833in}}{%
\pgfpathmoveto{\pgfqpoint{0.000000in}{0.000000in}}%
\pgfpathlineto{\pgfqpoint{0.000000in}{0.020833in}}%
\pgfusepath{stroke,fill}%
}%
\begin{pgfscope}%
\pgfsys@transformshift{2.661128in}{3.403703in}%
\pgfsys@useobject{currentmarker}{}%
\end{pgfscope}%
\end{pgfscope}%
\begin{pgfscope}%
\pgfsetbuttcap%
\pgfsetroundjoin%
\definecolor{currentfill}{rgb}{0.000000,0.000000,0.000000}%
\pgfsetfillcolor{currentfill}%
\pgfsetlinewidth{0.501875pt}%
\definecolor{currentstroke}{rgb}{0.000000,0.000000,0.000000}%
\pgfsetstrokecolor{currentstroke}%
\pgfsetdash{}{0pt}%
\pgfsys@defobject{currentmarker}{\pgfqpoint{0.000000in}{-0.020833in}}{\pgfqpoint{0.000000in}{0.000000in}}{%
\pgfpathmoveto{\pgfqpoint{0.000000in}{0.000000in}}%
\pgfpathlineto{\pgfqpoint{0.000000in}{-0.020833in}}%
\pgfusepath{stroke,fill}%
}%
\begin{pgfscope}%
\pgfsys@transformshift{2.661128in}{4.479825in}%
\pgfsys@useobject{currentmarker}{}%
\end{pgfscope}%
\end{pgfscope}%
\begin{pgfscope}%
\pgfsetbuttcap%
\pgfsetroundjoin%
\definecolor{currentfill}{rgb}{0.000000,0.000000,0.000000}%
\pgfsetfillcolor{currentfill}%
\pgfsetlinewidth{0.501875pt}%
\definecolor{currentstroke}{rgb}{0.000000,0.000000,0.000000}%
\pgfsetstrokecolor{currentstroke}%
\pgfsetdash{}{0pt}%
\pgfsys@defobject{currentmarker}{\pgfqpoint{0.000000in}{0.000000in}}{\pgfqpoint{0.000000in}{0.020833in}}{%
\pgfpathmoveto{\pgfqpoint{0.000000in}{0.000000in}}%
\pgfpathlineto{\pgfqpoint{0.000000in}{0.020833in}}%
\pgfusepath{stroke,fill}%
}%
\begin{pgfscope}%
\pgfsys@transformshift{2.757221in}{3.403703in}%
\pgfsys@useobject{currentmarker}{}%
\end{pgfscope}%
\end{pgfscope}%
\begin{pgfscope}%
\pgfsetbuttcap%
\pgfsetroundjoin%
\definecolor{currentfill}{rgb}{0.000000,0.000000,0.000000}%
\pgfsetfillcolor{currentfill}%
\pgfsetlinewidth{0.501875pt}%
\definecolor{currentstroke}{rgb}{0.000000,0.000000,0.000000}%
\pgfsetstrokecolor{currentstroke}%
\pgfsetdash{}{0pt}%
\pgfsys@defobject{currentmarker}{\pgfqpoint{0.000000in}{-0.020833in}}{\pgfqpoint{0.000000in}{0.000000in}}{%
\pgfpathmoveto{\pgfqpoint{0.000000in}{0.000000in}}%
\pgfpathlineto{\pgfqpoint{0.000000in}{-0.020833in}}%
\pgfusepath{stroke,fill}%
}%
\begin{pgfscope}%
\pgfsys@transformshift{2.757221in}{4.479825in}%
\pgfsys@useobject{currentmarker}{}%
\end{pgfscope}%
\end{pgfscope}%
\begin{pgfscope}%
\pgfsetbuttcap%
\pgfsetroundjoin%
\definecolor{currentfill}{rgb}{0.000000,0.000000,0.000000}%
\pgfsetfillcolor{currentfill}%
\pgfsetlinewidth{0.501875pt}%
\definecolor{currentstroke}{rgb}{0.000000,0.000000,0.000000}%
\pgfsetstrokecolor{currentstroke}%
\pgfsetdash{}{0pt}%
\pgfsys@defobject{currentmarker}{\pgfqpoint{0.000000in}{0.000000in}}{\pgfqpoint{0.000000in}{0.020833in}}{%
\pgfpathmoveto{\pgfqpoint{0.000000in}{0.000000in}}%
\pgfpathlineto{\pgfqpoint{0.000000in}{0.020833in}}%
\pgfusepath{stroke,fill}%
}%
\begin{pgfscope}%
\pgfsys@transformshift{2.853314in}{3.403703in}%
\pgfsys@useobject{currentmarker}{}%
\end{pgfscope}%
\end{pgfscope}%
\begin{pgfscope}%
\pgfsetbuttcap%
\pgfsetroundjoin%
\definecolor{currentfill}{rgb}{0.000000,0.000000,0.000000}%
\pgfsetfillcolor{currentfill}%
\pgfsetlinewidth{0.501875pt}%
\definecolor{currentstroke}{rgb}{0.000000,0.000000,0.000000}%
\pgfsetstrokecolor{currentstroke}%
\pgfsetdash{}{0pt}%
\pgfsys@defobject{currentmarker}{\pgfqpoint{0.000000in}{-0.020833in}}{\pgfqpoint{0.000000in}{0.000000in}}{%
\pgfpathmoveto{\pgfqpoint{0.000000in}{0.000000in}}%
\pgfpathlineto{\pgfqpoint{0.000000in}{-0.020833in}}%
\pgfusepath{stroke,fill}%
}%
\begin{pgfscope}%
\pgfsys@transformshift{2.853314in}{4.479825in}%
\pgfsys@useobject{currentmarker}{}%
\end{pgfscope}%
\end{pgfscope}%
\begin{pgfscope}%
\pgfsetbuttcap%
\pgfsetroundjoin%
\definecolor{currentfill}{rgb}{0.000000,0.000000,0.000000}%
\pgfsetfillcolor{currentfill}%
\pgfsetlinewidth{0.501875pt}%
\definecolor{currentstroke}{rgb}{0.000000,0.000000,0.000000}%
\pgfsetstrokecolor{currentstroke}%
\pgfsetdash{}{0pt}%
\pgfsys@defobject{currentmarker}{\pgfqpoint{0.000000in}{0.000000in}}{\pgfqpoint{0.000000in}{0.020833in}}{%
\pgfpathmoveto{\pgfqpoint{0.000000in}{0.000000in}}%
\pgfpathlineto{\pgfqpoint{0.000000in}{0.020833in}}%
\pgfusepath{stroke,fill}%
}%
\begin{pgfscope}%
\pgfsys@transformshift{2.949407in}{3.403703in}%
\pgfsys@useobject{currentmarker}{}%
\end{pgfscope}%
\end{pgfscope}%
\begin{pgfscope}%
\pgfsetbuttcap%
\pgfsetroundjoin%
\definecolor{currentfill}{rgb}{0.000000,0.000000,0.000000}%
\pgfsetfillcolor{currentfill}%
\pgfsetlinewidth{0.501875pt}%
\definecolor{currentstroke}{rgb}{0.000000,0.000000,0.000000}%
\pgfsetstrokecolor{currentstroke}%
\pgfsetdash{}{0pt}%
\pgfsys@defobject{currentmarker}{\pgfqpoint{0.000000in}{-0.020833in}}{\pgfqpoint{0.000000in}{0.000000in}}{%
\pgfpathmoveto{\pgfqpoint{0.000000in}{0.000000in}}%
\pgfpathlineto{\pgfqpoint{0.000000in}{-0.020833in}}%
\pgfusepath{stroke,fill}%
}%
\begin{pgfscope}%
\pgfsys@transformshift{2.949407in}{4.479825in}%
\pgfsys@useobject{currentmarker}{}%
\end{pgfscope}%
\end{pgfscope}%
\begin{pgfscope}%
\pgfsetbuttcap%
\pgfsetroundjoin%
\definecolor{currentfill}{rgb}{0.000000,0.000000,0.000000}%
\pgfsetfillcolor{currentfill}%
\pgfsetlinewidth{0.501875pt}%
\definecolor{currentstroke}{rgb}{0.000000,0.000000,0.000000}%
\pgfsetstrokecolor{currentstroke}%
\pgfsetdash{}{0pt}%
\pgfsys@defobject{currentmarker}{\pgfqpoint{0.000000in}{0.000000in}}{\pgfqpoint{0.000000in}{0.020833in}}{%
\pgfpathmoveto{\pgfqpoint{0.000000in}{0.000000in}}%
\pgfpathlineto{\pgfqpoint{0.000000in}{0.020833in}}%
\pgfusepath{stroke,fill}%
}%
\begin{pgfscope}%
\pgfsys@transformshift{3.141592in}{3.403703in}%
\pgfsys@useobject{currentmarker}{}%
\end{pgfscope}%
\end{pgfscope}%
\begin{pgfscope}%
\pgfsetbuttcap%
\pgfsetroundjoin%
\definecolor{currentfill}{rgb}{0.000000,0.000000,0.000000}%
\pgfsetfillcolor{currentfill}%
\pgfsetlinewidth{0.501875pt}%
\definecolor{currentstroke}{rgb}{0.000000,0.000000,0.000000}%
\pgfsetstrokecolor{currentstroke}%
\pgfsetdash{}{0pt}%
\pgfsys@defobject{currentmarker}{\pgfqpoint{0.000000in}{-0.020833in}}{\pgfqpoint{0.000000in}{0.000000in}}{%
\pgfpathmoveto{\pgfqpoint{0.000000in}{0.000000in}}%
\pgfpathlineto{\pgfqpoint{0.000000in}{-0.020833in}}%
\pgfusepath{stroke,fill}%
}%
\begin{pgfscope}%
\pgfsys@transformshift{3.141592in}{4.479825in}%
\pgfsys@useobject{currentmarker}{}%
\end{pgfscope}%
\end{pgfscope}%
\begin{pgfscope}%
\pgfsetbuttcap%
\pgfsetroundjoin%
\definecolor{currentfill}{rgb}{0.000000,0.000000,0.000000}%
\pgfsetfillcolor{currentfill}%
\pgfsetlinewidth{0.501875pt}%
\definecolor{currentstroke}{rgb}{0.000000,0.000000,0.000000}%
\pgfsetstrokecolor{currentstroke}%
\pgfsetdash{}{0pt}%
\pgfsys@defobject{currentmarker}{\pgfqpoint{0.000000in}{0.000000in}}{\pgfqpoint{0.000000in}{0.020833in}}{%
\pgfpathmoveto{\pgfqpoint{0.000000in}{0.000000in}}%
\pgfpathlineto{\pgfqpoint{0.000000in}{0.020833in}}%
\pgfusepath{stroke,fill}%
}%
\begin{pgfscope}%
\pgfsys@transformshift{3.237685in}{3.403703in}%
\pgfsys@useobject{currentmarker}{}%
\end{pgfscope}%
\end{pgfscope}%
\begin{pgfscope}%
\pgfsetbuttcap%
\pgfsetroundjoin%
\definecolor{currentfill}{rgb}{0.000000,0.000000,0.000000}%
\pgfsetfillcolor{currentfill}%
\pgfsetlinewidth{0.501875pt}%
\definecolor{currentstroke}{rgb}{0.000000,0.000000,0.000000}%
\pgfsetstrokecolor{currentstroke}%
\pgfsetdash{}{0pt}%
\pgfsys@defobject{currentmarker}{\pgfqpoint{0.000000in}{-0.020833in}}{\pgfqpoint{0.000000in}{0.000000in}}{%
\pgfpathmoveto{\pgfqpoint{0.000000in}{0.000000in}}%
\pgfpathlineto{\pgfqpoint{0.000000in}{-0.020833in}}%
\pgfusepath{stroke,fill}%
}%
\begin{pgfscope}%
\pgfsys@transformshift{3.237685in}{4.479825in}%
\pgfsys@useobject{currentmarker}{}%
\end{pgfscope}%
\end{pgfscope}%
\begin{pgfscope}%
\pgfsetbuttcap%
\pgfsetroundjoin%
\definecolor{currentfill}{rgb}{0.000000,0.000000,0.000000}%
\pgfsetfillcolor{currentfill}%
\pgfsetlinewidth{0.501875pt}%
\definecolor{currentstroke}{rgb}{0.000000,0.000000,0.000000}%
\pgfsetstrokecolor{currentstroke}%
\pgfsetdash{}{0pt}%
\pgfsys@defobject{currentmarker}{\pgfqpoint{0.000000in}{0.000000in}}{\pgfqpoint{0.000000in}{0.020833in}}{%
\pgfpathmoveto{\pgfqpoint{0.000000in}{0.000000in}}%
\pgfpathlineto{\pgfqpoint{0.000000in}{0.020833in}}%
\pgfusepath{stroke,fill}%
}%
\begin{pgfscope}%
\pgfsys@transformshift{3.333777in}{3.403703in}%
\pgfsys@useobject{currentmarker}{}%
\end{pgfscope}%
\end{pgfscope}%
\begin{pgfscope}%
\pgfsetbuttcap%
\pgfsetroundjoin%
\definecolor{currentfill}{rgb}{0.000000,0.000000,0.000000}%
\pgfsetfillcolor{currentfill}%
\pgfsetlinewidth{0.501875pt}%
\definecolor{currentstroke}{rgb}{0.000000,0.000000,0.000000}%
\pgfsetstrokecolor{currentstroke}%
\pgfsetdash{}{0pt}%
\pgfsys@defobject{currentmarker}{\pgfqpoint{0.000000in}{-0.020833in}}{\pgfqpoint{0.000000in}{0.000000in}}{%
\pgfpathmoveto{\pgfqpoint{0.000000in}{0.000000in}}%
\pgfpathlineto{\pgfqpoint{0.000000in}{-0.020833in}}%
\pgfusepath{stroke,fill}%
}%
\begin{pgfscope}%
\pgfsys@transformshift{3.333777in}{4.479825in}%
\pgfsys@useobject{currentmarker}{}%
\end{pgfscope}%
\end{pgfscope}%
\begin{pgfscope}%
\pgfsetbuttcap%
\pgfsetroundjoin%
\definecolor{currentfill}{rgb}{0.000000,0.000000,0.000000}%
\pgfsetfillcolor{currentfill}%
\pgfsetlinewidth{0.501875pt}%
\definecolor{currentstroke}{rgb}{0.000000,0.000000,0.000000}%
\pgfsetstrokecolor{currentstroke}%
\pgfsetdash{}{0pt}%
\pgfsys@defobject{currentmarker}{\pgfqpoint{0.000000in}{0.000000in}}{\pgfqpoint{0.000000in}{0.020833in}}{%
\pgfpathmoveto{\pgfqpoint{0.000000in}{0.000000in}}%
\pgfpathlineto{\pgfqpoint{0.000000in}{0.020833in}}%
\pgfusepath{stroke,fill}%
}%
\begin{pgfscope}%
\pgfsys@transformshift{3.429870in}{3.403703in}%
\pgfsys@useobject{currentmarker}{}%
\end{pgfscope}%
\end{pgfscope}%
\begin{pgfscope}%
\pgfsetbuttcap%
\pgfsetroundjoin%
\definecolor{currentfill}{rgb}{0.000000,0.000000,0.000000}%
\pgfsetfillcolor{currentfill}%
\pgfsetlinewidth{0.501875pt}%
\definecolor{currentstroke}{rgb}{0.000000,0.000000,0.000000}%
\pgfsetstrokecolor{currentstroke}%
\pgfsetdash{}{0pt}%
\pgfsys@defobject{currentmarker}{\pgfqpoint{0.000000in}{-0.020833in}}{\pgfqpoint{0.000000in}{0.000000in}}{%
\pgfpathmoveto{\pgfqpoint{0.000000in}{0.000000in}}%
\pgfpathlineto{\pgfqpoint{0.000000in}{-0.020833in}}%
\pgfusepath{stroke,fill}%
}%
\begin{pgfscope}%
\pgfsys@transformshift{3.429870in}{4.479825in}%
\pgfsys@useobject{currentmarker}{}%
\end{pgfscope}%
\end{pgfscope}%
\begin{pgfscope}%
\pgfsetbuttcap%
\pgfsetroundjoin%
\definecolor{currentfill}{rgb}{0.000000,0.000000,0.000000}%
\pgfsetfillcolor{currentfill}%
\pgfsetlinewidth{0.501875pt}%
\definecolor{currentstroke}{rgb}{0.000000,0.000000,0.000000}%
\pgfsetstrokecolor{currentstroke}%
\pgfsetdash{}{0pt}%
\pgfsys@defobject{currentmarker}{\pgfqpoint{0.000000in}{0.000000in}}{\pgfqpoint{0.000000in}{0.020833in}}{%
\pgfpathmoveto{\pgfqpoint{0.000000in}{0.000000in}}%
\pgfpathlineto{\pgfqpoint{0.000000in}{0.020833in}}%
\pgfusepath{stroke,fill}%
}%
\begin{pgfscope}%
\pgfsys@transformshift{3.622055in}{3.403703in}%
\pgfsys@useobject{currentmarker}{}%
\end{pgfscope}%
\end{pgfscope}%
\begin{pgfscope}%
\pgfsetbuttcap%
\pgfsetroundjoin%
\definecolor{currentfill}{rgb}{0.000000,0.000000,0.000000}%
\pgfsetfillcolor{currentfill}%
\pgfsetlinewidth{0.501875pt}%
\definecolor{currentstroke}{rgb}{0.000000,0.000000,0.000000}%
\pgfsetstrokecolor{currentstroke}%
\pgfsetdash{}{0pt}%
\pgfsys@defobject{currentmarker}{\pgfqpoint{0.000000in}{-0.020833in}}{\pgfqpoint{0.000000in}{0.000000in}}{%
\pgfpathmoveto{\pgfqpoint{0.000000in}{0.000000in}}%
\pgfpathlineto{\pgfqpoint{0.000000in}{-0.020833in}}%
\pgfusepath{stroke,fill}%
}%
\begin{pgfscope}%
\pgfsys@transformshift{3.622055in}{4.479825in}%
\pgfsys@useobject{currentmarker}{}%
\end{pgfscope}%
\end{pgfscope}%
\begin{pgfscope}%
\pgfsetbuttcap%
\pgfsetroundjoin%
\definecolor{currentfill}{rgb}{0.000000,0.000000,0.000000}%
\pgfsetfillcolor{currentfill}%
\pgfsetlinewidth{0.501875pt}%
\definecolor{currentstroke}{rgb}{0.000000,0.000000,0.000000}%
\pgfsetstrokecolor{currentstroke}%
\pgfsetdash{}{0pt}%
\pgfsys@defobject{currentmarker}{\pgfqpoint{0.000000in}{0.000000in}}{\pgfqpoint{0.000000in}{0.020833in}}{%
\pgfpathmoveto{\pgfqpoint{0.000000in}{0.000000in}}%
\pgfpathlineto{\pgfqpoint{0.000000in}{0.020833in}}%
\pgfusepath{stroke,fill}%
}%
\begin{pgfscope}%
\pgfsys@transformshift{3.718148in}{3.403703in}%
\pgfsys@useobject{currentmarker}{}%
\end{pgfscope}%
\end{pgfscope}%
\begin{pgfscope}%
\pgfsetbuttcap%
\pgfsetroundjoin%
\definecolor{currentfill}{rgb}{0.000000,0.000000,0.000000}%
\pgfsetfillcolor{currentfill}%
\pgfsetlinewidth{0.501875pt}%
\definecolor{currentstroke}{rgb}{0.000000,0.000000,0.000000}%
\pgfsetstrokecolor{currentstroke}%
\pgfsetdash{}{0pt}%
\pgfsys@defobject{currentmarker}{\pgfqpoint{0.000000in}{-0.020833in}}{\pgfqpoint{0.000000in}{0.000000in}}{%
\pgfpathmoveto{\pgfqpoint{0.000000in}{0.000000in}}%
\pgfpathlineto{\pgfqpoint{0.000000in}{-0.020833in}}%
\pgfusepath{stroke,fill}%
}%
\begin{pgfscope}%
\pgfsys@transformshift{3.718148in}{4.479825in}%
\pgfsys@useobject{currentmarker}{}%
\end{pgfscope}%
\end{pgfscope}%
\begin{pgfscope}%
\pgfsetbuttcap%
\pgfsetroundjoin%
\definecolor{currentfill}{rgb}{0.000000,0.000000,0.000000}%
\pgfsetfillcolor{currentfill}%
\pgfsetlinewidth{0.501875pt}%
\definecolor{currentstroke}{rgb}{0.000000,0.000000,0.000000}%
\pgfsetstrokecolor{currentstroke}%
\pgfsetdash{}{0pt}%
\pgfsys@defobject{currentmarker}{\pgfqpoint{0.000000in}{0.000000in}}{\pgfqpoint{0.000000in}{0.020833in}}{%
\pgfpathmoveto{\pgfqpoint{0.000000in}{0.000000in}}%
\pgfpathlineto{\pgfqpoint{0.000000in}{0.020833in}}%
\pgfusepath{stroke,fill}%
}%
\begin{pgfscope}%
\pgfsys@transformshift{3.814241in}{3.403703in}%
\pgfsys@useobject{currentmarker}{}%
\end{pgfscope}%
\end{pgfscope}%
\begin{pgfscope}%
\pgfsetbuttcap%
\pgfsetroundjoin%
\definecolor{currentfill}{rgb}{0.000000,0.000000,0.000000}%
\pgfsetfillcolor{currentfill}%
\pgfsetlinewidth{0.501875pt}%
\definecolor{currentstroke}{rgb}{0.000000,0.000000,0.000000}%
\pgfsetstrokecolor{currentstroke}%
\pgfsetdash{}{0pt}%
\pgfsys@defobject{currentmarker}{\pgfqpoint{0.000000in}{-0.020833in}}{\pgfqpoint{0.000000in}{0.000000in}}{%
\pgfpathmoveto{\pgfqpoint{0.000000in}{0.000000in}}%
\pgfpathlineto{\pgfqpoint{0.000000in}{-0.020833in}}%
\pgfusepath{stroke,fill}%
}%
\begin{pgfscope}%
\pgfsys@transformshift{3.814241in}{4.479825in}%
\pgfsys@useobject{currentmarker}{}%
\end{pgfscope}%
\end{pgfscope}%
\begin{pgfscope}%
\pgfsetbuttcap%
\pgfsetroundjoin%
\definecolor{currentfill}{rgb}{0.000000,0.000000,0.000000}%
\pgfsetfillcolor{currentfill}%
\pgfsetlinewidth{0.501875pt}%
\definecolor{currentstroke}{rgb}{0.000000,0.000000,0.000000}%
\pgfsetstrokecolor{currentstroke}%
\pgfsetdash{}{0pt}%
\pgfsys@defobject{currentmarker}{\pgfqpoint{0.000000in}{0.000000in}}{\pgfqpoint{0.000000in}{0.020833in}}{%
\pgfpathmoveto{\pgfqpoint{0.000000in}{0.000000in}}%
\pgfpathlineto{\pgfqpoint{0.000000in}{0.020833in}}%
\pgfusepath{stroke,fill}%
}%
\begin{pgfscope}%
\pgfsys@transformshift{3.910334in}{3.403703in}%
\pgfsys@useobject{currentmarker}{}%
\end{pgfscope}%
\end{pgfscope}%
\begin{pgfscope}%
\pgfsetbuttcap%
\pgfsetroundjoin%
\definecolor{currentfill}{rgb}{0.000000,0.000000,0.000000}%
\pgfsetfillcolor{currentfill}%
\pgfsetlinewidth{0.501875pt}%
\definecolor{currentstroke}{rgb}{0.000000,0.000000,0.000000}%
\pgfsetstrokecolor{currentstroke}%
\pgfsetdash{}{0pt}%
\pgfsys@defobject{currentmarker}{\pgfqpoint{0.000000in}{-0.020833in}}{\pgfqpoint{0.000000in}{0.000000in}}{%
\pgfpathmoveto{\pgfqpoint{0.000000in}{0.000000in}}%
\pgfpathlineto{\pgfqpoint{0.000000in}{-0.020833in}}%
\pgfusepath{stroke,fill}%
}%
\begin{pgfscope}%
\pgfsys@transformshift{3.910334in}{4.479825in}%
\pgfsys@useobject{currentmarker}{}%
\end{pgfscope}%
\end{pgfscope}%
\begin{pgfscope}%
\pgfsetbuttcap%
\pgfsetroundjoin%
\definecolor{currentfill}{rgb}{0.000000,0.000000,0.000000}%
\pgfsetfillcolor{currentfill}%
\pgfsetlinewidth{0.501875pt}%
\definecolor{currentstroke}{rgb}{0.000000,0.000000,0.000000}%
\pgfsetstrokecolor{currentstroke}%
\pgfsetdash{}{0pt}%
\pgfsys@defobject{currentmarker}{\pgfqpoint{0.000000in}{0.000000in}}{\pgfqpoint{0.000000in}{0.020833in}}{%
\pgfpathmoveto{\pgfqpoint{0.000000in}{0.000000in}}%
\pgfpathlineto{\pgfqpoint{0.000000in}{0.020833in}}%
\pgfusepath{stroke,fill}%
}%
\begin{pgfscope}%
\pgfsys@transformshift{4.102519in}{3.403703in}%
\pgfsys@useobject{currentmarker}{}%
\end{pgfscope}%
\end{pgfscope}%
\begin{pgfscope}%
\pgfsetbuttcap%
\pgfsetroundjoin%
\definecolor{currentfill}{rgb}{0.000000,0.000000,0.000000}%
\pgfsetfillcolor{currentfill}%
\pgfsetlinewidth{0.501875pt}%
\definecolor{currentstroke}{rgb}{0.000000,0.000000,0.000000}%
\pgfsetstrokecolor{currentstroke}%
\pgfsetdash{}{0pt}%
\pgfsys@defobject{currentmarker}{\pgfqpoint{0.000000in}{-0.020833in}}{\pgfqpoint{0.000000in}{0.000000in}}{%
\pgfpathmoveto{\pgfqpoint{0.000000in}{0.000000in}}%
\pgfpathlineto{\pgfqpoint{0.000000in}{-0.020833in}}%
\pgfusepath{stroke,fill}%
}%
\begin{pgfscope}%
\pgfsys@transformshift{4.102519in}{4.479825in}%
\pgfsys@useobject{currentmarker}{}%
\end{pgfscope}%
\end{pgfscope}%
\begin{pgfscope}%
\pgfsetbuttcap%
\pgfsetroundjoin%
\definecolor{currentfill}{rgb}{0.000000,0.000000,0.000000}%
\pgfsetfillcolor{currentfill}%
\pgfsetlinewidth{0.501875pt}%
\definecolor{currentstroke}{rgb}{0.000000,0.000000,0.000000}%
\pgfsetstrokecolor{currentstroke}%
\pgfsetdash{}{0pt}%
\pgfsys@defobject{currentmarker}{\pgfqpoint{0.000000in}{0.000000in}}{\pgfqpoint{0.000000in}{0.020833in}}{%
\pgfpathmoveto{\pgfqpoint{0.000000in}{0.000000in}}%
\pgfpathlineto{\pgfqpoint{0.000000in}{0.020833in}}%
\pgfusepath{stroke,fill}%
}%
\begin{pgfscope}%
\pgfsys@transformshift{4.198612in}{3.403703in}%
\pgfsys@useobject{currentmarker}{}%
\end{pgfscope}%
\end{pgfscope}%
\begin{pgfscope}%
\pgfsetbuttcap%
\pgfsetroundjoin%
\definecolor{currentfill}{rgb}{0.000000,0.000000,0.000000}%
\pgfsetfillcolor{currentfill}%
\pgfsetlinewidth{0.501875pt}%
\definecolor{currentstroke}{rgb}{0.000000,0.000000,0.000000}%
\pgfsetstrokecolor{currentstroke}%
\pgfsetdash{}{0pt}%
\pgfsys@defobject{currentmarker}{\pgfqpoint{0.000000in}{-0.020833in}}{\pgfqpoint{0.000000in}{0.000000in}}{%
\pgfpathmoveto{\pgfqpoint{0.000000in}{0.000000in}}%
\pgfpathlineto{\pgfqpoint{0.000000in}{-0.020833in}}%
\pgfusepath{stroke,fill}%
}%
\begin{pgfscope}%
\pgfsys@transformshift{4.198612in}{4.479825in}%
\pgfsys@useobject{currentmarker}{}%
\end{pgfscope}%
\end{pgfscope}%
\begin{pgfscope}%
\pgfsetbuttcap%
\pgfsetroundjoin%
\definecolor{currentfill}{rgb}{0.000000,0.000000,0.000000}%
\pgfsetfillcolor{currentfill}%
\pgfsetlinewidth{0.501875pt}%
\definecolor{currentstroke}{rgb}{0.000000,0.000000,0.000000}%
\pgfsetstrokecolor{currentstroke}%
\pgfsetdash{}{0pt}%
\pgfsys@defobject{currentmarker}{\pgfqpoint{0.000000in}{0.000000in}}{\pgfqpoint{0.000000in}{0.020833in}}{%
\pgfpathmoveto{\pgfqpoint{0.000000in}{0.000000in}}%
\pgfpathlineto{\pgfqpoint{0.000000in}{0.020833in}}%
\pgfusepath{stroke,fill}%
}%
\begin{pgfscope}%
\pgfsys@transformshift{4.294704in}{3.403703in}%
\pgfsys@useobject{currentmarker}{}%
\end{pgfscope}%
\end{pgfscope}%
\begin{pgfscope}%
\pgfsetbuttcap%
\pgfsetroundjoin%
\definecolor{currentfill}{rgb}{0.000000,0.000000,0.000000}%
\pgfsetfillcolor{currentfill}%
\pgfsetlinewidth{0.501875pt}%
\definecolor{currentstroke}{rgb}{0.000000,0.000000,0.000000}%
\pgfsetstrokecolor{currentstroke}%
\pgfsetdash{}{0pt}%
\pgfsys@defobject{currentmarker}{\pgfqpoint{0.000000in}{-0.020833in}}{\pgfqpoint{0.000000in}{0.000000in}}{%
\pgfpathmoveto{\pgfqpoint{0.000000in}{0.000000in}}%
\pgfpathlineto{\pgfqpoint{0.000000in}{-0.020833in}}%
\pgfusepath{stroke,fill}%
}%
\begin{pgfscope}%
\pgfsys@transformshift{4.294704in}{4.479825in}%
\pgfsys@useobject{currentmarker}{}%
\end{pgfscope}%
\end{pgfscope}%
\begin{pgfscope}%
\pgfsetbuttcap%
\pgfsetroundjoin%
\definecolor{currentfill}{rgb}{0.000000,0.000000,0.000000}%
\pgfsetfillcolor{currentfill}%
\pgfsetlinewidth{0.501875pt}%
\definecolor{currentstroke}{rgb}{0.000000,0.000000,0.000000}%
\pgfsetstrokecolor{currentstroke}%
\pgfsetdash{}{0pt}%
\pgfsys@defobject{currentmarker}{\pgfqpoint{0.000000in}{0.000000in}}{\pgfqpoint{0.000000in}{0.020833in}}{%
\pgfpathmoveto{\pgfqpoint{0.000000in}{0.000000in}}%
\pgfpathlineto{\pgfqpoint{0.000000in}{0.020833in}}%
\pgfusepath{stroke,fill}%
}%
\begin{pgfscope}%
\pgfsys@transformshift{4.390797in}{3.403703in}%
\pgfsys@useobject{currentmarker}{}%
\end{pgfscope}%
\end{pgfscope}%
\begin{pgfscope}%
\pgfsetbuttcap%
\pgfsetroundjoin%
\definecolor{currentfill}{rgb}{0.000000,0.000000,0.000000}%
\pgfsetfillcolor{currentfill}%
\pgfsetlinewidth{0.501875pt}%
\definecolor{currentstroke}{rgb}{0.000000,0.000000,0.000000}%
\pgfsetstrokecolor{currentstroke}%
\pgfsetdash{}{0pt}%
\pgfsys@defobject{currentmarker}{\pgfqpoint{0.000000in}{-0.020833in}}{\pgfqpoint{0.000000in}{0.000000in}}{%
\pgfpathmoveto{\pgfqpoint{0.000000in}{0.000000in}}%
\pgfpathlineto{\pgfqpoint{0.000000in}{-0.020833in}}%
\pgfusepath{stroke,fill}%
}%
\begin{pgfscope}%
\pgfsys@transformshift{4.390797in}{4.479825in}%
\pgfsys@useobject{currentmarker}{}%
\end{pgfscope}%
\end{pgfscope}%
\begin{pgfscope}%
\pgfsetbuttcap%
\pgfsetroundjoin%
\definecolor{currentfill}{rgb}{0.000000,0.000000,0.000000}%
\pgfsetfillcolor{currentfill}%
\pgfsetlinewidth{0.501875pt}%
\definecolor{currentstroke}{rgb}{0.000000,0.000000,0.000000}%
\pgfsetstrokecolor{currentstroke}%
\pgfsetdash{}{0pt}%
\pgfsys@defobject{currentmarker}{\pgfqpoint{0.000000in}{0.000000in}}{\pgfqpoint{0.000000in}{0.020833in}}{%
\pgfpathmoveto{\pgfqpoint{0.000000in}{0.000000in}}%
\pgfpathlineto{\pgfqpoint{0.000000in}{0.020833in}}%
\pgfusepath{stroke,fill}%
}%
\begin{pgfscope}%
\pgfsys@transformshift{4.582982in}{3.403703in}%
\pgfsys@useobject{currentmarker}{}%
\end{pgfscope}%
\end{pgfscope}%
\begin{pgfscope}%
\pgfsetbuttcap%
\pgfsetroundjoin%
\definecolor{currentfill}{rgb}{0.000000,0.000000,0.000000}%
\pgfsetfillcolor{currentfill}%
\pgfsetlinewidth{0.501875pt}%
\definecolor{currentstroke}{rgb}{0.000000,0.000000,0.000000}%
\pgfsetstrokecolor{currentstroke}%
\pgfsetdash{}{0pt}%
\pgfsys@defobject{currentmarker}{\pgfqpoint{0.000000in}{-0.020833in}}{\pgfqpoint{0.000000in}{0.000000in}}{%
\pgfpathmoveto{\pgfqpoint{0.000000in}{0.000000in}}%
\pgfpathlineto{\pgfqpoint{0.000000in}{-0.020833in}}%
\pgfusepath{stroke,fill}%
}%
\begin{pgfscope}%
\pgfsys@transformshift{4.582982in}{4.479825in}%
\pgfsys@useobject{currentmarker}{}%
\end{pgfscope}%
\end{pgfscope}%
\begin{pgfscope}%
\pgfsetbuttcap%
\pgfsetroundjoin%
\definecolor{currentfill}{rgb}{0.000000,0.000000,0.000000}%
\pgfsetfillcolor{currentfill}%
\pgfsetlinewidth{0.501875pt}%
\definecolor{currentstroke}{rgb}{0.000000,0.000000,0.000000}%
\pgfsetstrokecolor{currentstroke}%
\pgfsetdash{}{0pt}%
\pgfsys@defobject{currentmarker}{\pgfqpoint{0.000000in}{0.000000in}}{\pgfqpoint{0.041667in}{0.000000in}}{%
\pgfpathmoveto{\pgfqpoint{0.000000in}{0.000000in}}%
\pgfpathlineto{\pgfqpoint{0.041667in}{0.000000in}}%
\pgfusepath{stroke,fill}%
}%
\begin{pgfscope}%
\pgfsys@transformshift{0.444748in}{3.476703in}%
\pgfsys@useobject{currentmarker}{}%
\end{pgfscope}%
\end{pgfscope}%
\begin{pgfscope}%
\pgfsetbuttcap%
\pgfsetroundjoin%
\definecolor{currentfill}{rgb}{0.000000,0.000000,0.000000}%
\pgfsetfillcolor{currentfill}%
\pgfsetlinewidth{0.501875pt}%
\definecolor{currentstroke}{rgb}{0.000000,0.000000,0.000000}%
\pgfsetstrokecolor{currentstroke}%
\pgfsetdash{}{0pt}%
\pgfsys@defobject{currentmarker}{\pgfqpoint{-0.041667in}{0.000000in}}{\pgfqpoint{-0.000000in}{0.000000in}}{%
\pgfpathmoveto{\pgfqpoint{-0.000000in}{0.000000in}}%
\pgfpathlineto{\pgfqpoint{-0.041667in}{0.000000in}}%
\pgfusepath{stroke,fill}%
}%
\begin{pgfscope}%
\pgfsys@transformshift{4.676167in}{3.476703in}%
\pgfsys@useobject{currentmarker}{}%
\end{pgfscope}%
\end{pgfscope}%
\begin{pgfscope}%
\definecolor{textcolor}{rgb}{0.000000,0.000000,0.000000}%
\pgfsetstrokecolor{textcolor}%
\pgfsetfillcolor{textcolor}%
\pgftext[x=0.326693in, y=3.428485in, left, base]{\color{textcolor}\rmfamily\fontsize{10.000000}{12.000000}\selectfont \(\displaystyle {0}\)}%
\end{pgfscope}%
\begin{pgfscope}%
\pgfsetbuttcap%
\pgfsetroundjoin%
\definecolor{currentfill}{rgb}{0.000000,0.000000,0.000000}%
\pgfsetfillcolor{currentfill}%
\pgfsetlinewidth{0.501875pt}%
\definecolor{currentstroke}{rgb}{0.000000,0.000000,0.000000}%
\pgfsetstrokecolor{currentstroke}%
\pgfsetdash{}{0pt}%
\pgfsys@defobject{currentmarker}{\pgfqpoint{0.000000in}{0.000000in}}{\pgfqpoint{0.041667in}{0.000000in}}{%
\pgfpathmoveto{\pgfqpoint{0.000000in}{0.000000in}}%
\pgfpathlineto{\pgfqpoint{0.041667in}{0.000000in}}%
\pgfusepath{stroke,fill}%
}%
\begin{pgfscope}%
\pgfsys@transformshift{0.444748in}{4.005347in}%
\pgfsys@useobject{currentmarker}{}%
\end{pgfscope}%
\end{pgfscope}%
\begin{pgfscope}%
\pgfsetbuttcap%
\pgfsetroundjoin%
\definecolor{currentfill}{rgb}{0.000000,0.000000,0.000000}%
\pgfsetfillcolor{currentfill}%
\pgfsetlinewidth{0.501875pt}%
\definecolor{currentstroke}{rgb}{0.000000,0.000000,0.000000}%
\pgfsetstrokecolor{currentstroke}%
\pgfsetdash{}{0pt}%
\pgfsys@defobject{currentmarker}{\pgfqpoint{-0.041667in}{0.000000in}}{\pgfqpoint{-0.000000in}{0.000000in}}{%
\pgfpathmoveto{\pgfqpoint{-0.000000in}{0.000000in}}%
\pgfpathlineto{\pgfqpoint{-0.041667in}{0.000000in}}%
\pgfusepath{stroke,fill}%
}%
\begin{pgfscope}%
\pgfsys@transformshift{4.676167in}{4.005347in}%
\pgfsys@useobject{currentmarker}{}%
\end{pgfscope}%
\end{pgfscope}%
\begin{pgfscope}%
\definecolor{textcolor}{rgb}{0.000000,0.000000,0.000000}%
\pgfsetstrokecolor{textcolor}%
\pgfsetfillcolor{textcolor}%
\pgftext[x=0.257248in, y=3.957130in, left, base]{\color{textcolor}\rmfamily\fontsize{10.000000}{12.000000}\selectfont \(\displaystyle {20}\)}%
\end{pgfscope}%
\begin{pgfscope}%
\pgfsetbuttcap%
\pgfsetroundjoin%
\definecolor{currentfill}{rgb}{0.000000,0.000000,0.000000}%
\pgfsetfillcolor{currentfill}%
\pgfsetlinewidth{0.501875pt}%
\definecolor{currentstroke}{rgb}{0.000000,0.000000,0.000000}%
\pgfsetstrokecolor{currentstroke}%
\pgfsetdash{}{0pt}%
\pgfsys@defobject{currentmarker}{\pgfqpoint{0.000000in}{0.000000in}}{\pgfqpoint{0.020833in}{0.000000in}}{%
\pgfpathmoveto{\pgfqpoint{0.000000in}{0.000000in}}%
\pgfpathlineto{\pgfqpoint{0.020833in}{0.000000in}}%
\pgfusepath{stroke,fill}%
}%
\begin{pgfscope}%
\pgfsys@transformshift{0.444748in}{3.608864in}%
\pgfsys@useobject{currentmarker}{}%
\end{pgfscope}%
\end{pgfscope}%
\begin{pgfscope}%
\pgfsetbuttcap%
\pgfsetroundjoin%
\definecolor{currentfill}{rgb}{0.000000,0.000000,0.000000}%
\pgfsetfillcolor{currentfill}%
\pgfsetlinewidth{0.501875pt}%
\definecolor{currentstroke}{rgb}{0.000000,0.000000,0.000000}%
\pgfsetstrokecolor{currentstroke}%
\pgfsetdash{}{0pt}%
\pgfsys@defobject{currentmarker}{\pgfqpoint{-0.020833in}{0.000000in}}{\pgfqpoint{-0.000000in}{0.000000in}}{%
\pgfpathmoveto{\pgfqpoint{-0.000000in}{0.000000in}}%
\pgfpathlineto{\pgfqpoint{-0.020833in}{0.000000in}}%
\pgfusepath{stroke,fill}%
}%
\begin{pgfscope}%
\pgfsys@transformshift{4.676167in}{3.608864in}%
\pgfsys@useobject{currentmarker}{}%
\end{pgfscope}%
\end{pgfscope}%
\begin{pgfscope}%
\pgfsetbuttcap%
\pgfsetroundjoin%
\definecolor{currentfill}{rgb}{0.000000,0.000000,0.000000}%
\pgfsetfillcolor{currentfill}%
\pgfsetlinewidth{0.501875pt}%
\definecolor{currentstroke}{rgb}{0.000000,0.000000,0.000000}%
\pgfsetstrokecolor{currentstroke}%
\pgfsetdash{}{0pt}%
\pgfsys@defobject{currentmarker}{\pgfqpoint{0.000000in}{0.000000in}}{\pgfqpoint{0.020833in}{0.000000in}}{%
\pgfpathmoveto{\pgfqpoint{0.000000in}{0.000000in}}%
\pgfpathlineto{\pgfqpoint{0.020833in}{0.000000in}}%
\pgfusepath{stroke,fill}%
}%
\begin{pgfscope}%
\pgfsys@transformshift{0.444748in}{3.741025in}%
\pgfsys@useobject{currentmarker}{}%
\end{pgfscope}%
\end{pgfscope}%
\begin{pgfscope}%
\pgfsetbuttcap%
\pgfsetroundjoin%
\definecolor{currentfill}{rgb}{0.000000,0.000000,0.000000}%
\pgfsetfillcolor{currentfill}%
\pgfsetlinewidth{0.501875pt}%
\definecolor{currentstroke}{rgb}{0.000000,0.000000,0.000000}%
\pgfsetstrokecolor{currentstroke}%
\pgfsetdash{}{0pt}%
\pgfsys@defobject{currentmarker}{\pgfqpoint{-0.020833in}{0.000000in}}{\pgfqpoint{-0.000000in}{0.000000in}}{%
\pgfpathmoveto{\pgfqpoint{-0.000000in}{0.000000in}}%
\pgfpathlineto{\pgfqpoint{-0.020833in}{0.000000in}}%
\pgfusepath{stroke,fill}%
}%
\begin{pgfscope}%
\pgfsys@transformshift{4.676167in}{3.741025in}%
\pgfsys@useobject{currentmarker}{}%
\end{pgfscope}%
\end{pgfscope}%
\begin{pgfscope}%
\pgfsetbuttcap%
\pgfsetroundjoin%
\definecolor{currentfill}{rgb}{0.000000,0.000000,0.000000}%
\pgfsetfillcolor{currentfill}%
\pgfsetlinewidth{0.501875pt}%
\definecolor{currentstroke}{rgb}{0.000000,0.000000,0.000000}%
\pgfsetstrokecolor{currentstroke}%
\pgfsetdash{}{0pt}%
\pgfsys@defobject{currentmarker}{\pgfqpoint{0.000000in}{0.000000in}}{\pgfqpoint{0.020833in}{0.000000in}}{%
\pgfpathmoveto{\pgfqpoint{0.000000in}{0.000000in}}%
\pgfpathlineto{\pgfqpoint{0.020833in}{0.000000in}}%
\pgfusepath{stroke,fill}%
}%
\begin{pgfscope}%
\pgfsys@transformshift{0.444748in}{3.873186in}%
\pgfsys@useobject{currentmarker}{}%
\end{pgfscope}%
\end{pgfscope}%
\begin{pgfscope}%
\pgfsetbuttcap%
\pgfsetroundjoin%
\definecolor{currentfill}{rgb}{0.000000,0.000000,0.000000}%
\pgfsetfillcolor{currentfill}%
\pgfsetlinewidth{0.501875pt}%
\definecolor{currentstroke}{rgb}{0.000000,0.000000,0.000000}%
\pgfsetstrokecolor{currentstroke}%
\pgfsetdash{}{0pt}%
\pgfsys@defobject{currentmarker}{\pgfqpoint{-0.020833in}{0.000000in}}{\pgfqpoint{-0.000000in}{0.000000in}}{%
\pgfpathmoveto{\pgfqpoint{-0.000000in}{0.000000in}}%
\pgfpathlineto{\pgfqpoint{-0.020833in}{0.000000in}}%
\pgfusepath{stroke,fill}%
}%
\begin{pgfscope}%
\pgfsys@transformshift{4.676167in}{3.873186in}%
\pgfsys@useobject{currentmarker}{}%
\end{pgfscope}%
\end{pgfscope}%
\begin{pgfscope}%
\pgfsetbuttcap%
\pgfsetroundjoin%
\definecolor{currentfill}{rgb}{0.000000,0.000000,0.000000}%
\pgfsetfillcolor{currentfill}%
\pgfsetlinewidth{0.501875pt}%
\definecolor{currentstroke}{rgb}{0.000000,0.000000,0.000000}%
\pgfsetstrokecolor{currentstroke}%
\pgfsetdash{}{0pt}%
\pgfsys@defobject{currentmarker}{\pgfqpoint{0.000000in}{0.000000in}}{\pgfqpoint{0.020833in}{0.000000in}}{%
\pgfpathmoveto{\pgfqpoint{0.000000in}{0.000000in}}%
\pgfpathlineto{\pgfqpoint{0.020833in}{0.000000in}}%
\pgfusepath{stroke,fill}%
}%
\begin{pgfscope}%
\pgfsys@transformshift{0.444748in}{4.137509in}%
\pgfsys@useobject{currentmarker}{}%
\end{pgfscope}%
\end{pgfscope}%
\begin{pgfscope}%
\pgfsetbuttcap%
\pgfsetroundjoin%
\definecolor{currentfill}{rgb}{0.000000,0.000000,0.000000}%
\pgfsetfillcolor{currentfill}%
\pgfsetlinewidth{0.501875pt}%
\definecolor{currentstroke}{rgb}{0.000000,0.000000,0.000000}%
\pgfsetstrokecolor{currentstroke}%
\pgfsetdash{}{0pt}%
\pgfsys@defobject{currentmarker}{\pgfqpoint{-0.020833in}{0.000000in}}{\pgfqpoint{-0.000000in}{0.000000in}}{%
\pgfpathmoveto{\pgfqpoint{-0.000000in}{0.000000in}}%
\pgfpathlineto{\pgfqpoint{-0.020833in}{0.000000in}}%
\pgfusepath{stroke,fill}%
}%
\begin{pgfscope}%
\pgfsys@transformshift{4.676167in}{4.137509in}%
\pgfsys@useobject{currentmarker}{}%
\end{pgfscope}%
\end{pgfscope}%
\begin{pgfscope}%
\pgfsetbuttcap%
\pgfsetroundjoin%
\definecolor{currentfill}{rgb}{0.000000,0.000000,0.000000}%
\pgfsetfillcolor{currentfill}%
\pgfsetlinewidth{0.501875pt}%
\definecolor{currentstroke}{rgb}{0.000000,0.000000,0.000000}%
\pgfsetstrokecolor{currentstroke}%
\pgfsetdash{}{0pt}%
\pgfsys@defobject{currentmarker}{\pgfqpoint{0.000000in}{0.000000in}}{\pgfqpoint{0.020833in}{0.000000in}}{%
\pgfpathmoveto{\pgfqpoint{0.000000in}{0.000000in}}%
\pgfpathlineto{\pgfqpoint{0.020833in}{0.000000in}}%
\pgfusepath{stroke,fill}%
}%
\begin{pgfscope}%
\pgfsys@transformshift{0.444748in}{4.269670in}%
\pgfsys@useobject{currentmarker}{}%
\end{pgfscope}%
\end{pgfscope}%
\begin{pgfscope}%
\pgfsetbuttcap%
\pgfsetroundjoin%
\definecolor{currentfill}{rgb}{0.000000,0.000000,0.000000}%
\pgfsetfillcolor{currentfill}%
\pgfsetlinewidth{0.501875pt}%
\definecolor{currentstroke}{rgb}{0.000000,0.000000,0.000000}%
\pgfsetstrokecolor{currentstroke}%
\pgfsetdash{}{0pt}%
\pgfsys@defobject{currentmarker}{\pgfqpoint{-0.020833in}{0.000000in}}{\pgfqpoint{-0.000000in}{0.000000in}}{%
\pgfpathmoveto{\pgfqpoint{-0.000000in}{0.000000in}}%
\pgfpathlineto{\pgfqpoint{-0.020833in}{0.000000in}}%
\pgfusepath{stroke,fill}%
}%
\begin{pgfscope}%
\pgfsys@transformshift{4.676167in}{4.269670in}%
\pgfsys@useobject{currentmarker}{}%
\end{pgfscope}%
\end{pgfscope}%
\begin{pgfscope}%
\pgfsetbuttcap%
\pgfsetroundjoin%
\definecolor{currentfill}{rgb}{0.000000,0.000000,0.000000}%
\pgfsetfillcolor{currentfill}%
\pgfsetlinewidth{0.501875pt}%
\definecolor{currentstroke}{rgb}{0.000000,0.000000,0.000000}%
\pgfsetstrokecolor{currentstroke}%
\pgfsetdash{}{0pt}%
\pgfsys@defobject{currentmarker}{\pgfqpoint{0.000000in}{0.000000in}}{\pgfqpoint{0.020833in}{0.000000in}}{%
\pgfpathmoveto{\pgfqpoint{0.000000in}{0.000000in}}%
\pgfpathlineto{\pgfqpoint{0.020833in}{0.000000in}}%
\pgfusepath{stroke,fill}%
}%
\begin{pgfscope}%
\pgfsys@transformshift{0.444748in}{4.401831in}%
\pgfsys@useobject{currentmarker}{}%
\end{pgfscope}%
\end{pgfscope}%
\begin{pgfscope}%
\pgfsetbuttcap%
\pgfsetroundjoin%
\definecolor{currentfill}{rgb}{0.000000,0.000000,0.000000}%
\pgfsetfillcolor{currentfill}%
\pgfsetlinewidth{0.501875pt}%
\definecolor{currentstroke}{rgb}{0.000000,0.000000,0.000000}%
\pgfsetstrokecolor{currentstroke}%
\pgfsetdash{}{0pt}%
\pgfsys@defobject{currentmarker}{\pgfqpoint{-0.020833in}{0.000000in}}{\pgfqpoint{-0.000000in}{0.000000in}}{%
\pgfpathmoveto{\pgfqpoint{-0.000000in}{0.000000in}}%
\pgfpathlineto{\pgfqpoint{-0.020833in}{0.000000in}}%
\pgfusepath{stroke,fill}%
}%
\begin{pgfscope}%
\pgfsys@transformshift{4.676167in}{4.401831in}%
\pgfsys@useobject{currentmarker}{}%
\end{pgfscope}%
\end{pgfscope}%
\begin{pgfscope}%
\definecolor{textcolor}{rgb}{0.000000,0.000000,0.000000}%
\pgfsetstrokecolor{textcolor}%
\pgfsetfillcolor{textcolor}%
\pgftext[x=0.201692in,y=3.941764in,,bottom,rotate=90.000000]{\color{textcolor}\rmfamily\fontsize{12.000000}{14.400000}\selectfont \(\displaystyle V_s\) (\unit{\micro\volt})}%
\end{pgfscope}%
\begin{pgfscope}%
\pgfpathrectangle{\pgfqpoint{0.444748in}{3.403703in}}{\pgfqpoint{4.231419in}{1.076123in}}%
\pgfusepath{clip}%
\pgfsetbuttcap%
\pgfsetroundjoin%
\pgfsetlinewidth{1.003750pt}%
\definecolor{currentstroke}{rgb}{0.047059,0.364706,0.647059}%
\pgfsetstrokecolor{currentstroke}%
\pgfsetdash{{3.700000pt}{1.600000pt}}{0.000000pt}%
\pgfpathmoveto{\pgfqpoint{0.637086in}{3.959921in}}%
\pgfpathlineto{\pgfqpoint{0.656101in}{3.613686in}}%
\pgfpathlineto{\pgfqpoint{0.676526in}{3.504229in}}%
\pgfpathlineto{\pgfqpoint{0.693665in}{3.472149in}}%
\pgfpathlineto{\pgfqpoint{0.714559in}{3.554874in}}%
\pgfpathlineto{\pgfqpoint{0.731699in}{3.738341in}}%
\pgfpathlineto{\pgfqpoint{0.753533in}{4.247934in}}%
\pgfpathlineto{\pgfqpoint{0.771375in}{4.363963in}}%
\pgfpathlineto{\pgfqpoint{0.790626in}{4.047141in}}%
\pgfpathlineto{\pgfqpoint{0.808234in}{3.663840in}}%
\pgfpathlineto{\pgfqpoint{0.828424in}{3.516453in}}%
\pgfpathlineto{\pgfqpoint{0.849083in}{3.465506in}}%
\pgfpathlineto{\pgfqpoint{0.868100in}{3.467743in}}%
\pgfpathlineto{\pgfqpoint{0.887117in}{3.514401in}}%
\pgfpathlineto{\pgfqpoint{0.905196in}{3.646135in}}%
\pgfpathlineto{\pgfqpoint{0.925150in}{4.041211in}}%
\pgfpathlineto{\pgfqpoint{0.943227in}{4.343342in}}%
\pgfpathlineto{\pgfqpoint{0.964358in}{4.166315in}}%
\pgfpathlineto{\pgfqpoint{0.984078in}{3.731858in}}%
\pgfpathlineto{\pgfqpoint{1.000747in}{3.549029in}}%
\pgfpathlineto{\pgfqpoint{1.020469in}{3.472392in}}%
\pgfpathlineto{\pgfqpoint{1.041129in}{3.458605in}}%
\pgfpathlineto{\pgfqpoint{1.060380in}{3.483286in}}%
\pgfpathlineto{\pgfqpoint{1.079162in}{3.553907in}}%
\pgfpathlineto{\pgfqpoint{1.095361in}{3.601151in}}%
\pgfpathlineto{\pgfqpoint{1.118368in}{3.910242in}}%
\pgfpathlineto{\pgfqpoint{1.136210in}{4.286363in}}%
\pgfpathlineto{\pgfqpoint{1.155932in}{4.188654in}}%
\pgfpathlineto{\pgfqpoint{1.174713in}{3.811895in}}%
\pgfpathlineto{\pgfqpoint{1.193495in}{3.561928in}}%
\pgfpathlineto{\pgfqpoint{1.212512in}{3.478314in}}%
\pgfpathlineto{\pgfqpoint{1.234580in}{3.457777in}}%
\pgfpathlineto{\pgfqpoint{1.250545in}{3.467398in}}%
\pgfpathlineto{\pgfqpoint{1.269562in}{3.511427in}}%
\pgfpathlineto{\pgfqpoint{1.292334in}{3.688503in}}%
\pgfpathlineto{\pgfqpoint{1.311116in}{4.060932in}}%
\pgfpathlineto{\pgfqpoint{1.329428in}{4.279277in}}%
\pgfpathlineto{\pgfqpoint{1.349384in}{4.125644in}}%
\pgfpathlineto{\pgfqpoint{1.368167in}{3.787904in}}%
\pgfpathlineto{\pgfqpoint{1.386949in}{3.554141in}}%
\pgfpathlineto{\pgfqpoint{1.405729in}{3.479316in}}%
\pgfpathlineto{\pgfqpoint{1.424512in}{3.460337in}}%
\pgfpathlineto{\pgfqpoint{1.446346in}{3.466192in}}%
\pgfpathlineto{\pgfqpoint{1.463248in}{3.495873in}}%
\pgfpathlineto{\pgfqpoint{1.482265in}{3.586479in}}%
\pgfpathlineto{\pgfqpoint{1.500342in}{3.826334in}}%
\pgfpathlineto{\pgfqpoint{1.522882in}{4.267717in}}%
\pgfpathlineto{\pgfqpoint{1.538141in}{4.229920in}}%
\pgfpathlineto{\pgfqpoint{1.580401in}{3.623428in}}%
\pgfpathlineto{\pgfqpoint{1.596129in}{3.513253in}}%
\pgfpathlineto{\pgfqpoint{1.615382in}{3.468195in}}%
\pgfpathlineto{\pgfqpoint{1.638623in}{3.456536in}}%
\pgfpathlineto{\pgfqpoint{1.654119in}{3.466428in}}%
\pgfpathlineto{\pgfqpoint{1.676188in}{3.515054in}}%
\pgfpathlineto{\pgfqpoint{1.695674in}{3.628517in}}%
\pgfpathlineto{\pgfqpoint{1.713282in}{3.950676in}}%
\pgfpathlineto{\pgfqpoint{1.733236in}{4.201583in}}%
\pgfpathlineto{\pgfqpoint{1.749436in}{4.262891in}}%
\pgfpathlineto{\pgfqpoint{1.770332in}{4.054820in}}%
\pgfpathlineto{\pgfqpoint{1.792164in}{3.650282in}}%
\pgfpathlineto{\pgfqpoint{1.809069in}{3.523851in}}%
\pgfpathlineto{\pgfqpoint{1.825503in}{3.485140in}}%
\pgfpathlineto{\pgfqpoint{1.825971in}{3.464246in}}%
\pgfpathlineto{\pgfqpoint{1.848040in}{3.456684in}}%
\pgfpathlineto{\pgfqpoint{1.867997in}{3.458472in}}%
\pgfpathlineto{\pgfqpoint{1.884899in}{3.481943in}}%
\pgfpathlineto{\pgfqpoint{1.906734in}{3.509365in}}%
\pgfpathlineto{\pgfqpoint{1.925985in}{3.647399in}}%
\pgfpathlineto{\pgfqpoint{1.945941in}{3.907107in}}%
\pgfpathlineto{\pgfqpoint{1.964724in}{4.230334in}}%
\pgfpathlineto{\pgfqpoint{1.986087in}{4.193742in}}%
\pgfpathlineto{\pgfqpoint{2.000877in}{3.956923in}}%
\pgfpathlineto{\pgfqpoint{2.022243in}{3.696507in}}%
\pgfpathlineto{\pgfqpoint{2.039145in}{3.526818in}}%
\pgfpathlineto{\pgfqpoint{2.060745in}{3.468579in}}%
\pgfpathlineto{\pgfqpoint{2.076710in}{3.456571in}}%
\pgfpathlineto{\pgfqpoint{2.114273in}{3.464037in}}%
\pgfpathlineto{\pgfqpoint{2.138690in}{3.505832in}}%
\pgfpathlineto{\pgfqpoint{2.153715in}{3.591088in}}%
\pgfpathlineto{\pgfqpoint{2.174375in}{3.881646in}}%
\pgfpathlineto{\pgfqpoint{2.192452in}{4.189895in}}%
\pgfpathlineto{\pgfqpoint{2.213346in}{4.245687in}}%
\pgfpathlineto{\pgfqpoint{2.231659in}{4.146148in}}%
\pgfpathlineto{\pgfqpoint{2.253963in}{3.797347in}}%
\pgfpathlineto{\pgfqpoint{2.271571in}{3.591330in}}%
\pgfpathlineto{\pgfqpoint{2.291291in}{3.506081in}}%
\pgfpathlineto{\pgfqpoint{2.310776in}{3.468001in}}%
\pgfpathlineto{\pgfqpoint{2.329324in}{3.454791in}}%
\pgfpathlineto{\pgfqpoint{2.347872in}{3.461178in}}%
\pgfpathlineto{\pgfqpoint{2.366418in}{3.471822in}}%
\pgfpathlineto{\pgfqpoint{2.384026in}{3.515357in}}%
\pgfpathlineto{\pgfqpoint{2.405626in}{3.643719in}}%
\pgfpathlineto{\pgfqpoint{2.422294in}{3.491589in}}%
\pgfpathlineto{\pgfqpoint{2.442954in}{3.568644in}}%
\pgfpathlineto{\pgfqpoint{2.461971in}{3.773719in}}%
\pgfpathlineto{\pgfqpoint{2.500004in}{4.248652in}}%
\pgfpathlineto{\pgfqpoint{2.521367in}{4.122258in}}%
\pgfpathlineto{\pgfqpoint{2.539210in}{3.771269in}}%
\pgfpathlineto{\pgfqpoint{2.557054in}{3.600369in}}%
\pgfpathlineto{\pgfqpoint{2.578418in}{3.488433in}}%
\pgfpathlineto{\pgfqpoint{2.596495in}{3.471427in}}%
\pgfpathlineto{\pgfqpoint{2.617860in}{3.455290in}}%
\pgfpathlineto{\pgfqpoint{2.634528in}{3.462200in}}%
\pgfpathlineto{\pgfqpoint{2.655188in}{3.486958in}}%
\pgfpathlineto{\pgfqpoint{2.674439in}{3.576200in}}%
\pgfpathlineto{\pgfqpoint{2.692282in}{3.781937in}}%
\pgfpathlineto{\pgfqpoint{2.712473in}{4.097736in}}%
\pgfpathlineto{\pgfqpoint{2.730550in}{4.252320in}}%
\pgfpathlineto{\pgfqpoint{2.748629in}{4.145009in}}%
\pgfpathlineto{\pgfqpoint{2.769992in}{3.803925in}}%
\pgfpathlineto{\pgfqpoint{2.788540in}{3.612876in}}%
\pgfpathlineto{\pgfqpoint{2.809669in}{3.497371in}}%
\pgfpathlineto{\pgfqpoint{2.827277in}{3.468034in}}%
\pgfpathlineto{\pgfqpoint{2.845119in}{3.456935in}}%
\pgfpathlineto{\pgfqpoint{2.866953in}{3.460972in}}%
\pgfpathlineto{\pgfqpoint{2.884092in}{3.486304in}}%
\pgfpathlineto{\pgfqpoint{2.903107in}{3.544801in}}%
\pgfpathlineto{\pgfqpoint{2.923767in}{3.725164in}}%
\pgfpathlineto{\pgfqpoint{2.941846in}{4.064003in}}%
\pgfpathlineto{\pgfqpoint{2.960863in}{4.240509in}}%
\pgfpathlineto{\pgfqpoint{2.980114in}{4.225256in}}%
\pgfpathlineto{\pgfqpoint{3.000774in}{3.921574in}}%
\pgfpathlineto{\pgfqpoint{3.018616in}{3.677668in}}%
\pgfpathlineto{\pgfqpoint{3.038102in}{3.536928in}}%
\pgfpathlineto{\pgfqpoint{3.057822in}{3.478577in}}%
\pgfpathlineto{\pgfqpoint{3.076370in}{3.463001in}}%
\pgfpathlineto{\pgfqpoint{3.094447in}{3.457766in}}%
\pgfpathlineto{\pgfqpoint{3.116516in}{3.470625in}}%
\pgfpathlineto{\pgfqpoint{3.134358in}{3.491249in}}%
\pgfpathlineto{\pgfqpoint{3.152906in}{3.553392in}}%
\pgfpathlineto{\pgfqpoint{3.173566in}{3.733322in}}%
\pgfpathlineto{\pgfqpoint{3.193991in}{4.072999in}}%
\pgfpathlineto{\pgfqpoint{3.211834in}{4.235586in}}%
\pgfpathlineto{\pgfqpoint{3.231788in}{4.277186in}}%
\pgfpathlineto{\pgfqpoint{3.249867in}{4.105389in}}%
\pgfpathlineto{\pgfqpoint{3.269353in}{3.886103in}}%
\pgfpathlineto{\pgfqpoint{3.290716in}{3.626397in}}%
\pgfpathlineto{\pgfqpoint{3.308795in}{3.526154in}}%
\pgfpathlineto{\pgfqpoint{3.326403in}{3.482134in}}%
\pgfpathlineto{\pgfqpoint{3.346358in}{3.459951in}}%
\pgfpathlineto{\pgfqpoint{3.364671in}{3.460304in}}%
\pgfpathlineto{\pgfqpoint{3.384391in}{3.474227in}}%
\pgfpathlineto{\pgfqpoint{3.405286in}{3.477630in}}%
\pgfpathlineto{\pgfqpoint{3.422425in}{3.460715in}}%
\pgfpathlineto{\pgfqpoint{3.440502in}{3.463508in}}%
\pgfpathlineto{\pgfqpoint{3.460458in}{3.475914in}}%
\pgfpathlineto{\pgfqpoint{3.481587in}{3.522548in}}%
\pgfpathlineto{\pgfqpoint{3.502482in}{3.655077in}}%
\pgfpathlineto{\pgfqpoint{3.517743in}{3.876119in}}%
\pgfpathlineto{\pgfqpoint{3.536994in}{4.252835in}}%
\pgfpathlineto{\pgfqpoint{3.559532in}{4.286190in}}%
\pgfpathlineto{\pgfqpoint{3.577609in}{4.108273in}}%
\pgfpathlineto{\pgfqpoint{3.595217in}{3.797155in}}%
\pgfpathlineto{\pgfqpoint{3.616346in}{3.590737in}}%
\pgfpathlineto{\pgfqpoint{3.635599in}{3.506195in}}%
\pgfpathlineto{\pgfqpoint{3.652736in}{3.484767in}}%
\pgfpathlineto{\pgfqpoint{3.673867in}{3.461214in}}%
\pgfpathlineto{\pgfqpoint{3.691709in}{3.462170in}}%
\pgfpathlineto{\pgfqpoint{3.709786in}{3.472360in}}%
\pgfpathlineto{\pgfqpoint{3.730681in}{3.510755in}}%
\pgfpathlineto{\pgfqpoint{3.752280in}{3.600534in}}%
\pgfpathlineto{\pgfqpoint{3.768009in}{3.518603in}}%
\pgfpathlineto{\pgfqpoint{3.789140in}{3.657272in}}%
\pgfpathlineto{\pgfqpoint{3.826233in}{4.309921in}}%
\pgfpathlineto{\pgfqpoint{3.844310in}{4.321475in}}%
\pgfpathlineto{\pgfqpoint{3.866379in}{4.072294in}}%
\pgfpathlineto{\pgfqpoint{3.883752in}{3.787838in}}%
\pgfpathlineto{\pgfqpoint{3.904412in}{3.570058in}}%
\pgfpathlineto{\pgfqpoint{3.923429in}{3.502883in}}%
\pgfpathlineto{\pgfqpoint{3.941506in}{3.472233in}}%
\pgfpathlineto{\pgfqpoint{3.962400in}{3.461348in}}%
\pgfpathlineto{\pgfqpoint{3.980479in}{3.470196in}}%
\pgfpathlineto{\pgfqpoint{3.997616in}{3.489360in}}%
\pgfpathlineto{\pgfqpoint{4.019216in}{3.549564in}}%
\pgfpathlineto{\pgfqpoint{4.037762in}{3.672647in}}%
\pgfpathlineto{\pgfqpoint{4.060771in}{4.052474in}}%
\pgfpathlineto{\pgfqpoint{4.077439in}{4.273823in}}%
\pgfpathlineto{\pgfqpoint{4.097864in}{4.388442in}}%
\pgfpathlineto{\pgfqpoint{4.112655in}{4.289429in}}%
\pgfpathlineto{\pgfqpoint{4.133786in}{4.039789in}}%
\pgfpathlineto{\pgfqpoint{4.155854in}{3.735379in}}%
\pgfpathlineto{\pgfqpoint{4.172991in}{3.588929in}}%
\pgfpathlineto{\pgfqpoint{4.193886in}{3.508602in}}%
\pgfpathlineto{\pgfqpoint{4.212199in}{3.473223in}}%
\pgfpathlineto{\pgfqpoint{4.229807in}{3.462807in}}%
\pgfpathlineto{\pgfqpoint{4.250936in}{3.476535in}}%
\pgfpathlineto{\pgfqpoint{4.270658in}{3.505124in}}%
\pgfpathlineto{\pgfqpoint{4.287326in}{3.570788in}}%
\pgfpathlineto{\pgfqpoint{4.308455in}{3.699983in}}%
\pgfpathlineto{\pgfqpoint{4.326063in}{3.956619in}}%
\pgfpathlineto{\pgfqpoint{4.344611in}{4.265867in}}%
\pgfpathlineto{\pgfqpoint{4.365505in}{4.430911in}}%
\pgfpathlineto{\pgfqpoint{4.384757in}{4.322130in}}%
\pgfpathlineto{\pgfqpoint{4.401659in}{4.091915in}}%
\pgfpathlineto{\pgfqpoint{4.422319in}{3.777529in}}%
\pgfpathlineto{\pgfqpoint{4.441807in}{3.608692in}}%
\pgfpathlineto{\pgfqpoint{4.463876in}{3.511973in}}%
\pgfpathlineto{\pgfqpoint{4.480075in}{3.480241in}}%
\pgfpathlineto{\pgfqpoint{4.473266in}{3.495331in}}%
\pgfpathlineto{\pgfqpoint{4.454954in}{3.646930in}}%
\pgfpathlineto{\pgfqpoint{4.433355in}{4.069740in}}%
\pgfpathlineto{\pgfqpoint{4.417624in}{4.353450in}}%
\pgfpathlineto{\pgfqpoint{4.396730in}{4.355841in}}%
\pgfpathlineto{\pgfqpoint{4.379827in}{3.930535in}}%
\pgfpathlineto{\pgfqpoint{4.360576in}{3.516100in}}%
\pgfpathlineto{\pgfqpoint{4.338271in}{3.464185in}}%
\pgfpathlineto{\pgfqpoint{4.319959in}{3.468705in}}%
\pgfpathlineto{\pgfqpoint{4.299300in}{3.527524in}}%
\pgfpathlineto{\pgfqpoint{4.283569in}{3.646901in}}%
\pgfpathlineto{\pgfqpoint{4.262675in}{4.074053in}}%
\pgfpathlineto{\pgfqpoint{4.245067in}{4.365175in}}%
\pgfpathlineto{\pgfqpoint{4.223703in}{4.249359in}}%
\pgfpathlineto{\pgfqpoint{4.206095in}{3.801191in}}%
\pgfpathlineto{\pgfqpoint{4.186139in}{3.558774in}}%
\pgfpathlineto{\pgfqpoint{4.167827in}{3.483032in}}%
\pgfpathlineto{\pgfqpoint{4.147167in}{3.460522in}}%
\pgfpathlineto{\pgfqpoint{4.129089in}{3.473682in}}%
\pgfpathlineto{\pgfqpoint{4.108194in}{3.557442in}}%
\pgfpathlineto{\pgfqpoint{4.085422in}{3.864575in}}%
\pgfpathlineto{\pgfqpoint{4.071806in}{4.199149in}}%
\pgfpathlineto{\pgfqpoint{4.050909in}{4.354038in}}%
\pgfpathlineto{\pgfqpoint{4.033538in}{4.034168in}}%
\pgfpathlineto{\pgfqpoint{4.013816in}{3.628269in}}%
\pgfpathlineto{\pgfqpoint{3.993156in}{3.497136in}}%
\pgfpathlineto{\pgfqpoint{3.975313in}{3.465729in}}%
\pgfpathlineto{\pgfqpoint{3.956062in}{3.461271in}}%
\pgfpathlineto{\pgfqpoint{3.938220in}{3.492558in}}%
\pgfpathlineto{\pgfqpoint{3.917091in}{3.619889in}}%
\pgfpathlineto{\pgfqpoint{3.896665in}{3.993787in}}%
\pgfpathlineto{\pgfqpoint{3.878823in}{4.290326in}}%
\pgfpathlineto{\pgfqpoint{3.857927in}{4.216716in}}%
\pgfpathlineto{\pgfqpoint{3.840789in}{3.848956in}}%
\pgfpathlineto{\pgfqpoint{3.822476in}{3.565570in}}%
\pgfpathlineto{\pgfqpoint{3.802756in}{3.484680in}}%
\pgfpathlineto{\pgfqpoint{3.782330in}{3.457571in}}%
\pgfpathlineto{\pgfqpoint{3.763548in}{3.463686in}}%
\pgfpathlineto{\pgfqpoint{3.742888in}{3.509575in}}%
\pgfpathlineto{\pgfqpoint{3.725046in}{3.633999in}}%
\pgfpathlineto{\pgfqpoint{3.703682in}{4.069157in}}%
\pgfpathlineto{\pgfqpoint{3.685840in}{4.291661in}}%
\pgfpathlineto{\pgfqpoint{3.667527in}{4.203974in}}%
\pgfpathlineto{\pgfqpoint{3.650389in}{3.826873in}}%
\pgfpathlineto{\pgfqpoint{3.626441in}{3.557979in}}%
\pgfpathlineto{\pgfqpoint{3.609069in}{3.488407in}}%
\pgfpathlineto{\pgfqpoint{3.592165in}{3.462139in}}%
\pgfpathlineto{\pgfqpoint{3.572210in}{3.456711in}}%
\pgfpathlineto{\pgfqpoint{3.553428in}{3.479556in}}%
\pgfpathlineto{\pgfqpoint{3.532534in}{3.523712in}}%
\pgfpathlineto{\pgfqpoint{3.514220in}{3.528283in}}%
\pgfpathlineto{\pgfqpoint{3.493560in}{3.743513in}}%
\pgfpathlineto{\pgfqpoint{3.473606in}{4.128954in}}%
\pgfpathlineto{\pgfqpoint{3.455292in}{4.283983in}}%
\pgfpathlineto{\pgfqpoint{3.438155in}{4.211663in}}%
\pgfpathlineto{\pgfqpoint{3.415616in}{3.766171in}}%
\pgfpathlineto{\pgfqpoint{3.397304in}{3.554365in}}%
\pgfpathlineto{\pgfqpoint{3.379696in}{3.495233in}}%
\pgfpathlineto{\pgfqpoint{3.361383in}{3.463761in}}%
\pgfpathlineto{\pgfqpoint{3.339551in}{3.456014in}}%
\pgfpathlineto{\pgfqpoint{3.321472in}{3.471169in}}%
\pgfpathlineto{\pgfqpoint{3.300577in}{3.528405in}}%
\pgfpathlineto{\pgfqpoint{3.282969in}{3.614651in}}%
\pgfpathlineto{\pgfqpoint{3.262309in}{3.953320in}}%
\pgfpathlineto{\pgfqpoint{3.245172in}{4.195554in}}%
\pgfpathlineto{\pgfqpoint{3.226624in}{4.268959in}}%
\pgfpathlineto{\pgfqpoint{3.205261in}{3.966718in}}%
\pgfpathlineto{\pgfqpoint{3.188122in}{3.682959in}}%
\pgfpathlineto{\pgfqpoint{3.167462in}{3.511304in}}%
\pgfpathlineto{\pgfqpoint{3.146568in}{3.466852in}}%
\pgfpathlineto{\pgfqpoint{3.128725in}{3.462480in}}%
\pgfpathlineto{\pgfqpoint{3.110177in}{3.455053in}}%
\pgfpathlineto{\pgfqpoint{3.092335in}{3.466611in}}%
\pgfpathlineto{\pgfqpoint{3.071440in}{3.511940in}}%
\pgfpathlineto{\pgfqpoint{3.053362in}{3.646274in}}%
\pgfpathlineto{\pgfqpoint{3.031058in}{4.023727in}}%
\pgfpathlineto{\pgfqpoint{3.011807in}{4.241096in}}%
\pgfpathlineto{\pgfqpoint{2.993965in}{3.971229in}}%
\pgfpathlineto{\pgfqpoint{2.975888in}{4.248594in}}%
\pgfpathlineto{\pgfqpoint{2.958045in}{4.206234in}}%
\pgfpathlineto{\pgfqpoint{2.938323in}{3.829533in}}%
\pgfpathlineto{\pgfqpoint{2.913203in}{3.557242in}}%
\pgfpathlineto{\pgfqpoint{2.900761in}{3.521972in}}%
\pgfpathlineto{\pgfqpoint{2.879395in}{3.468553in}}%
\pgfpathlineto{\pgfqpoint{2.860615in}{3.490407in}}%
\pgfpathlineto{\pgfqpoint{2.836903in}{3.456720in}}%
\pgfpathlineto{\pgfqpoint{2.822816in}{3.457130in}}%
\pgfpathlineto{\pgfqpoint{2.801451in}{3.483551in}}%
\pgfpathlineto{\pgfqpoint{2.783374in}{3.554088in}}%
\pgfpathlineto{\pgfqpoint{2.763654in}{3.774362in}}%
\pgfpathlineto{\pgfqpoint{2.745106in}{4.143773in}}%
\pgfpathlineto{\pgfqpoint{2.722803in}{4.249159in}}%
\pgfpathlineto{\pgfqpoint{2.685943in}{3.715326in}}%
\pgfpathlineto{\pgfqpoint{2.669275in}{3.551321in}}%
\pgfpathlineto{\pgfqpoint{2.648144in}{3.484864in}}%
\pgfpathlineto{\pgfqpoint{2.630536in}{3.459458in}}%
\pgfpathlineto{\pgfqpoint{2.611285in}{3.456798in}}%
\pgfpathlineto{\pgfqpoint{2.590391in}{3.470342in}}%
\pgfpathlineto{\pgfqpoint{2.572783in}{3.507309in}}%
\pgfpathlineto{\pgfqpoint{2.550949in}{3.660168in}}%
\pgfpathlineto{\pgfqpoint{2.534515in}{3.966657in}}%
\pgfpathlineto{\pgfqpoint{2.515734in}{4.216678in}}%
\pgfpathlineto{\pgfqpoint{2.496481in}{4.225690in}}%
\pgfpathlineto{\pgfqpoint{2.474884in}{3.884541in}}%
\pgfpathlineto{\pgfqpoint{2.456570in}{3.629274in}}%
\pgfpathlineto{\pgfqpoint{2.438024in}{3.518593in}}%
\pgfpathlineto{\pgfqpoint{2.412904in}{3.468455in}}%
\pgfpathlineto{\pgfqpoint{2.397642in}{3.456884in}}%
\pgfpathlineto{\pgfqpoint{2.378626in}{3.460629in}}%
\pgfpathlineto{\pgfqpoint{2.359845in}{3.469424in}}%
\pgfpathlineto{\pgfqpoint{2.342003in}{3.471189in}}%
\pgfpathlineto{\pgfqpoint{2.323455in}{3.527272in}}%
\pgfpathlineto{\pgfqpoint{2.301621in}{3.735993in}}%
\pgfpathlineto{\pgfqpoint{2.282838in}{4.085307in}}%
\pgfpathlineto{\pgfqpoint{2.264058in}{4.256882in}}%
\pgfpathlineto{\pgfqpoint{2.241050in}{4.088335in}}%
\pgfpathlineto{\pgfqpoint{2.223911in}{3.799447in}}%
\pgfpathlineto{\pgfqpoint{2.205365in}{3.591306in}}%
\pgfpathlineto{\pgfqpoint{2.186113in}{3.511370in}}%
\pgfpathlineto{\pgfqpoint{2.168036in}{3.469919in}}%
\pgfpathlineto{\pgfqpoint{2.149958in}{3.457390in}}%
\pgfpathlineto{\pgfqpoint{2.128123in}{3.462170in}}%
\pgfpathlineto{\pgfqpoint{2.110986in}{3.478485in}}%
\pgfpathlineto{\pgfqpoint{2.092438in}{3.527186in}}%
\pgfpathlineto{\pgfqpoint{2.073658in}{3.626751in}}%
\pgfpathlineto{\pgfqpoint{2.050650in}{3.987192in}}%
\pgfpathlineto{\pgfqpoint{2.032573in}{4.237500in}}%
\pgfpathlineto{\pgfqpoint{2.013321in}{4.233109in}}%
\pgfpathlineto{\pgfqpoint{1.994068in}{3.934929in}}%
\pgfpathlineto{\pgfqpoint{1.976226in}{3.675941in}}%
\pgfpathlineto{\pgfqpoint{1.957445in}{3.544795in}}%
\pgfpathlineto{\pgfqpoint{1.938663in}{3.490429in}}%
\pgfpathlineto{\pgfqpoint{1.917769in}{3.465690in}}%
\pgfpathlineto{\pgfqpoint{1.898987in}{3.458159in}}%
\pgfpathlineto{\pgfqpoint{1.880910in}{3.465606in}}%
\pgfpathlineto{\pgfqpoint{1.861658in}{3.483882in}}%
\pgfpathlineto{\pgfqpoint{1.841702in}{3.561563in}}%
\pgfpathlineto{\pgfqpoint{1.822685in}{3.672514in}}%
\pgfpathlineto{\pgfqpoint{1.803199in}{3.815869in}}%
\pgfpathlineto{\pgfqpoint{1.780896in}{4.010043in}}%
\pgfpathlineto{\pgfqpoint{1.766106in}{4.230552in}}%
\pgfpathlineto{\pgfqpoint{1.746855in}{4.270981in}}%
\pgfpathlineto{\pgfqpoint{1.725020in}{3.996623in}}%
\pgfpathlineto{\pgfqpoint{1.709761in}{3.779403in}}%
\pgfpathlineto{\pgfqpoint{1.688396in}{3.574512in}}%
\pgfpathlineto{\pgfqpoint{1.669379in}{3.501506in}}%
\pgfpathlineto{\pgfqpoint{1.644727in}{3.462260in}}%
\pgfpathlineto{\pgfqpoint{1.628764in}{3.458059in}}%
\pgfpathlineto{\pgfqpoint{1.610685in}{3.465228in}}%
\pgfpathlineto{\pgfqpoint{1.592608in}{3.753057in}}%
\pgfpathlineto{\pgfqpoint{1.570305in}{3.539810in}}%
\pgfpathlineto{\pgfqpoint{1.552463in}{3.480732in}}%
\pgfpathlineto{\pgfqpoint{1.533446in}{3.461073in}}%
\pgfpathlineto{\pgfqpoint{1.512317in}{3.459268in}}%
\pgfpathlineto{\pgfqpoint{1.491657in}{3.482030in}}%
\pgfpathlineto{\pgfqpoint{1.475927in}{3.536713in}}%
\pgfpathlineto{\pgfqpoint{1.455736in}{3.675428in}}%
\pgfpathlineto{\pgfqpoint{1.437659in}{3.981630in}}%
\pgfpathlineto{\pgfqpoint{1.419111in}{4.250214in}}%
\pgfpathlineto{\pgfqpoint{1.400094in}{4.309084in}}%
\pgfpathlineto{\pgfqpoint{1.382486in}{4.095579in}}%
\pgfpathlineto{\pgfqpoint{1.360889in}{3.719071in}}%
\pgfpathlineto{\pgfqpoint{1.341403in}{3.562191in}}%
\pgfpathlineto{\pgfqpoint{1.323558in}{3.504975in}}%
\pgfpathlineto{\pgfqpoint{1.301961in}{3.468568in}}%
\pgfpathlineto{\pgfqpoint{1.283884in}{3.459442in}}%
\pgfpathlineto{\pgfqpoint{1.265570in}{3.466680in}}%
\pgfpathlineto{\pgfqpoint{1.246319in}{3.493387in}}%
\pgfpathlineto{\pgfqpoint{1.227537in}{3.556151in}}%
\pgfpathlineto{\pgfqpoint{1.205703in}{3.764301in}}%
\pgfpathlineto{\pgfqpoint{1.187157in}{4.086210in}}%
\pgfpathlineto{\pgfqpoint{1.168843in}{4.282156in}}%
\pgfpathlineto{\pgfqpoint{1.151001in}{4.317978in}}%
\pgfpathlineto{\pgfqpoint{1.128698in}{4.061370in}}%
\pgfpathlineto{\pgfqpoint{1.109681in}{3.770280in}}%
\pgfpathlineto{\pgfqpoint{1.092073in}{3.588215in}}%
\pgfpathlineto{\pgfqpoint{1.072587in}{3.511966in}}%
\pgfpathlineto{\pgfqpoint{1.054042in}{3.477970in}}%
\pgfpathlineto{\pgfqpoint{1.033616in}{3.461051in}}%
\pgfpathlineto{\pgfqpoint{1.014365in}{3.463060in}}%
\pgfpathlineto{\pgfqpoint{0.995583in}{3.480740in}}%
\pgfpathlineto{\pgfqpoint{0.977740in}{3.526963in}}%
\pgfpathlineto{\pgfqpoint{0.956375in}{3.675191in}}%
\pgfpathlineto{\pgfqpoint{0.937594in}{3.951402in}}%
\pgfpathlineto{\pgfqpoint{0.919047in}{4.140234in}}%
\pgfpathlineto{\pgfqpoint{0.900030in}{4.251009in}}%
\pgfpathlineto{\pgfqpoint{0.878196in}{4.367212in}}%
\pgfpathlineto{\pgfqpoint{0.859884in}{4.287005in}}%
\pgfpathlineto{\pgfqpoint{0.842276in}{3.882134in}}%
\pgfpathlineto{\pgfqpoint{0.823025in}{4.157838in}}%
\pgfpathlineto{\pgfqpoint{0.804712in}{4.357827in}}%
\pgfpathlineto{\pgfqpoint{0.782409in}{4.250936in}}%
\pgfpathlineto{\pgfqpoint{0.765975in}{3.918544in}}%
\pgfpathlineto{\pgfqpoint{0.744610in}{3.627738in}}%
\pgfpathlineto{\pgfqpoint{0.722777in}{3.539430in}}%
\pgfpathlineto{\pgfqpoint{0.705638in}{3.493433in}}%
\pgfpathlineto{\pgfqpoint{0.686856in}{3.464785in}}%
\pgfpathlineto{\pgfqpoint{0.669013in}{3.462571in}}%
\pgfpathlineto{\pgfqpoint{0.650936in}{3.482239in}}%
\pgfpathlineto{\pgfqpoint{0.650468in}{3.484676in}}%
\pgfpathlineto{\pgfqpoint{0.658918in}{3.468665in}}%
\pgfpathlineto{\pgfqpoint{0.673945in}{3.461878in}}%
\pgfpathlineto{\pgfqpoint{0.696248in}{3.499793in}}%
\pgfpathlineto{\pgfqpoint{0.713151in}{3.593639in}}%
\pgfpathlineto{\pgfqpoint{0.732402in}{3.850221in}}%
\pgfpathlineto{\pgfqpoint{0.753533in}{4.344457in}}%
\pgfpathlineto{\pgfqpoint{0.771141in}{4.298412in}}%
\pgfpathlineto{\pgfqpoint{0.790626in}{3.889722in}}%
\pgfpathlineto{\pgfqpoint{0.810581in}{3.562331in}}%
\pgfpathlineto{\pgfqpoint{0.828424in}{3.480794in}}%
\pgfpathlineto{\pgfqpoint{0.847206in}{3.458579in}}%
\pgfpathlineto{\pgfqpoint{0.866223in}{3.474695in}}%
\pgfpathlineto{\pgfqpoint{0.888291in}{3.557590in}}%
\pgfpathlineto{\pgfqpoint{0.908011in}{3.773167in}}%
\pgfpathlineto{\pgfqpoint{0.923507in}{4.142775in}}%
\pgfpathlineto{\pgfqpoint{0.942524in}{4.330782in}}%
\pgfpathlineto{\pgfqpoint{0.980558in}{3.791809in}}%
\pgfpathlineto{\pgfqpoint{1.003798in}{3.525842in}}%
\pgfpathlineto{\pgfqpoint{1.022112in}{3.466618in}}%
\pgfpathlineto{\pgfqpoint{1.041832in}{3.458211in}}%
\pgfpathlineto{\pgfqpoint{1.059909in}{3.480402in}}%
\pgfpathlineto{\pgfqpoint{1.079631in}{3.553556in}}%
\pgfpathlineto{\pgfqpoint{1.098882in}{3.802142in}}%
\pgfpathlineto{\pgfqpoint{1.115550in}{4.199676in}}%
\pgfpathlineto{\pgfqpoint{1.136210in}{4.264613in}}%
\pgfpathlineto{\pgfqpoint{1.155932in}{3.950333in}}%
\pgfpathlineto{\pgfqpoint{1.174949in}{3.591700in}}%
\pgfpathlineto{\pgfqpoint{1.194200in}{3.488846in}}%
\pgfpathlineto{\pgfqpoint{1.212981in}{3.459663in}}%
\pgfpathlineto{\pgfqpoint{1.232937in}{3.461991in}}%
\pgfpathlineto{\pgfqpoint{1.251485in}{3.493200in}}%
\pgfpathlineto{\pgfqpoint{1.270031in}{3.594473in}}%
\pgfpathlineto{\pgfqpoint{1.289753in}{3.908619in}}%
\pgfpathlineto{\pgfqpoint{1.308299in}{4.266062in}}%
\pgfpathlineto{\pgfqpoint{1.326847in}{4.216183in}}%
\pgfpathlineto{\pgfqpoint{1.347507in}{3.942308in}}%
\pgfpathlineto{\pgfqpoint{1.368870in}{3.628457in}}%
\pgfpathlineto{\pgfqpoint{1.388121in}{3.497013in}}%
\pgfpathlineto{\pgfqpoint{1.406435in}{3.460800in}}%
\pgfpathlineto{\pgfqpoint{1.425451in}{3.456488in}}%
\pgfpathlineto{\pgfqpoint{1.444703in}{3.473851in}}%
\pgfpathlineto{\pgfqpoint{1.467240in}{3.544970in}}%
\pgfpathlineto{\pgfqpoint{1.481562in}{3.684335in}}%
\pgfpathlineto{\pgfqpoint{1.502925in}{4.089439in}}%
\pgfpathlineto{\pgfqpoint{1.520299in}{4.260904in}}%
\pgfpathlineto{\pgfqpoint{1.537907in}{4.085785in}}%
\pgfpathlineto{\pgfqpoint{1.560915in}{3.765314in}}%
\pgfpathlineto{\pgfqpoint{1.579695in}{3.565004in}}%
\pgfpathlineto{\pgfqpoint{1.598478in}{3.481559in}}%
\pgfpathlineto{\pgfqpoint{1.617729in}{3.458066in}}%
\pgfpathlineto{\pgfqpoint{1.635337in}{3.459673in}}%
\pgfpathlineto{\pgfqpoint{1.655057in}{3.479283in}}%
\pgfpathlineto{\pgfqpoint{1.674308in}{3.538044in}}%
\pgfpathlineto{\pgfqpoint{1.692387in}{3.635104in}}%
\pgfpathlineto{\pgfqpoint{1.714690in}{3.988584in}}%
\pgfpathlineto{\pgfqpoint{1.732533in}{4.235929in}}%
\pgfpathlineto{\pgfqpoint{1.751315in}{4.237470in}}%
\pgfpathlineto{\pgfqpoint{1.770095in}{4.006107in}}%
\pgfpathlineto{\pgfqpoint{1.789112in}{3.678507in}}%
\pgfpathlineto{\pgfqpoint{1.808129in}{3.522809in}}%
\pgfpathlineto{\pgfqpoint{1.827851in}{3.476192in}}%
\pgfpathlineto{\pgfqpoint{1.848277in}{3.458964in}}%
\pgfpathlineto{\pgfqpoint{1.867057in}{3.457642in}}%
\pgfpathlineto{\pgfqpoint{1.885136in}{3.468533in}}%
\pgfpathlineto{\pgfqpoint{1.904151in}{3.515370in}}%
\pgfpathlineto{\pgfqpoint{1.922933in}{3.582097in}}%
\pgfpathlineto{\pgfqpoint{1.945472in}{3.889338in}}%
\pgfpathlineto{\pgfqpoint{1.963315in}{4.190748in}}%
\pgfpathlineto{\pgfqpoint{1.981626in}{4.242759in}}%
\pgfpathlineto{\pgfqpoint{1.999000in}{4.101431in}}%
\pgfpathlineto{\pgfqpoint{2.018954in}{3.772808in}}%
\pgfpathlineto{\pgfqpoint{2.041259in}{3.558411in}}%
\pgfpathlineto{\pgfqpoint{2.060040in}{3.486011in}}%
\pgfpathlineto{\pgfqpoint{2.078119in}{3.466474in}}%
\pgfpathlineto{\pgfqpoint{2.098779in}{3.456244in}}%
\pgfpathlineto{\pgfqpoint{2.117793in}{3.460981in}}%
\pgfpathlineto{\pgfqpoint{2.135167in}{3.483480in}}%
\pgfpathlineto{\pgfqpoint{2.153246in}{3.531631in}}%
\pgfpathlineto{\pgfqpoint{2.173435in}{3.717096in}}%
\pgfpathlineto{\pgfqpoint{2.194566in}{4.002072in}}%
\pgfpathlineto{\pgfqpoint{2.212408in}{4.226408in}}%
\pgfpathlineto{\pgfqpoint{2.230485in}{4.187179in}}%
\pgfpathlineto{\pgfqpoint{2.251849in}{3.864467in}}%
\pgfpathlineto{\pgfqpoint{2.268753in}{3.590988in}}%
\pgfpathlineto{\pgfqpoint{2.287770in}{3.924362in}}%
\pgfpathlineto{\pgfqpoint{2.309368in}{3.570883in}}%
\pgfpathlineto{\pgfqpoint{2.330498in}{3.494783in}}%
\pgfpathlineto{\pgfqpoint{2.346932in}{3.465951in}}%
\pgfpathlineto{\pgfqpoint{2.365480in}{3.455332in}}%
\pgfpathlineto{\pgfqpoint{2.386374in}{3.465716in}}%
\pgfpathlineto{\pgfqpoint{2.403982in}{3.498077in}}%
\pgfpathlineto{\pgfqpoint{2.427929in}{3.649243in}}%
\pgfpathlineto{\pgfqpoint{2.444363in}{3.872645in}}%
\pgfpathlineto{\pgfqpoint{2.461502in}{4.186685in}}%
\pgfpathlineto{\pgfqpoint{2.482162in}{4.190191in}}%
\pgfpathlineto{\pgfqpoint{2.503056in}{3.969466in}}%
\pgfpathlineto{\pgfqpoint{2.517847in}{3.685875in}}%
\pgfpathlineto{\pgfqpoint{2.539681in}{3.513309in}}%
\pgfpathlineto{\pgfqpoint{2.560341in}{3.467293in}}%
\pgfpathlineto{\pgfqpoint{2.577714in}{3.457965in}}%
\pgfpathlineto{\pgfqpoint{2.597903in}{3.459600in}}%
\pgfpathlineto{\pgfqpoint{2.617155in}{3.479098in}}%
\pgfpathlineto{\pgfqpoint{2.634059in}{3.528249in}}%
\pgfpathlineto{\pgfqpoint{2.656362in}{3.630806in}}%
\pgfpathlineto{\pgfqpoint{2.673736in}{3.857667in}}%
\pgfpathlineto{\pgfqpoint{2.691813in}{4.162155in}}%
\pgfpathlineto{\pgfqpoint{2.715525in}{4.194330in}}%
\pgfpathlineto{\pgfqpoint{2.731021in}{3.969328in}}%
\pgfpathlineto{\pgfqpoint{2.749801in}{3.813467in}}%
\pgfpathlineto{\pgfqpoint{2.770697in}{3.589976in}}%
\pgfpathlineto{\pgfqpoint{2.788774in}{3.497740in}}%
\pgfpathlineto{\pgfqpoint{2.808494in}{3.466571in}}%
\pgfpathlineto{\pgfqpoint{2.850520in}{3.456069in}}%
\pgfpathlineto{\pgfqpoint{2.865310in}{3.465225in}}%
\pgfpathlineto{\pgfqpoint{2.886673in}{3.510264in}}%
\pgfpathlineto{\pgfqpoint{2.904516in}{3.591059in}}%
\pgfpathlineto{\pgfqpoint{2.921889in}{3.741965in}}%
\pgfpathlineto{\pgfqpoint{2.942784in}{4.129040in}}%
\pgfpathlineto{\pgfqpoint{2.962506in}{4.193614in}}%
\pgfpathlineto{\pgfqpoint{2.983869in}{4.221041in}}%
\pgfpathlineto{\pgfqpoint{3.002183in}{3.974586in}}%
\pgfpathlineto{\pgfqpoint{3.019791in}{3.704435in}}%
\pgfpathlineto{\pgfqpoint{3.040685in}{3.522369in}}%
\pgfpathlineto{\pgfqpoint{3.058996in}{3.475672in}}%
\pgfpathlineto{\pgfqpoint{3.075901in}{3.464399in}}%
\pgfpathlineto{\pgfqpoint{3.097735in}{3.456124in}}%
\pgfpathlineto{\pgfqpoint{3.097264in}{3.462519in}}%
\pgfpathlineto{\pgfqpoint{3.115107in}{3.468728in}}%
\pgfpathlineto{\pgfqpoint{3.133655in}{3.493656in}}%
\pgfpathlineto{\pgfqpoint{3.154549in}{3.567392in}}%
\pgfpathlineto{\pgfqpoint{3.172626in}{3.736162in}}%
\pgfpathlineto{\pgfqpoint{3.194460in}{4.098299in}}%
\pgfpathlineto{\pgfqpoint{3.211599in}{4.251905in}}%
\pgfpathlineto{\pgfqpoint{3.229442in}{4.223260in}}%
\pgfpathlineto{\pgfqpoint{3.250805in}{4.008097in}}%
\pgfpathlineto{\pgfqpoint{3.271231in}{3.687066in}}%
\pgfpathlineto{\pgfqpoint{3.288839in}{3.550304in}}%
\pgfpathlineto{\pgfqpoint{3.306681in}{3.487222in}}%
\pgfpathlineto{\pgfqpoint{3.328515in}{3.464366in}}%
\pgfpathlineto{\pgfqpoint{3.346123in}{3.457251in}}%
\pgfpathlineto{\pgfqpoint{3.367018in}{3.464902in}}%
\pgfpathlineto{\pgfqpoint{3.385566in}{3.479765in}}%
\pgfpathlineto{\pgfqpoint{3.403174in}{3.513912in}}%
\pgfpathlineto{\pgfqpoint{3.422425in}{3.607606in}}%
\pgfpathlineto{\pgfqpoint{3.443788in}{3.824492in}}%
\pgfpathlineto{\pgfqpoint{3.461631in}{4.113628in}}%
\pgfpathlineto{\pgfqpoint{3.479004in}{4.286805in}}%
\pgfpathlineto{\pgfqpoint{3.501307in}{4.167815in}}%
\pgfpathlineto{\pgfqpoint{3.520324in}{4.010193in}}%
\pgfpathlineto{\pgfqpoint{3.537932in}{3.766444in}}%
\pgfpathlineto{\pgfqpoint{3.559297in}{3.549315in}}%
\pgfpathlineto{\pgfqpoint{3.577374in}{3.713018in}}%
\pgfpathlineto{\pgfqpoint{3.595451in}{3.991220in}}%
\pgfpathlineto{\pgfqpoint{3.616346in}{4.293371in}}%
\pgfpathlineto{\pgfqpoint{3.634425in}{4.285161in}}%
\pgfpathlineto{\pgfqpoint{3.651798in}{4.065956in}}%
\pgfpathlineto{\pgfqpoint{3.674570in}{3.756912in}}%
\pgfpathlineto{\pgfqpoint{3.692178in}{3.561355in}}%
\pgfpathlineto{\pgfqpoint{3.709786in}{3.495707in}}%
\pgfpathlineto{\pgfqpoint{3.730681in}{3.462455in}}%
\pgfpathlineto{\pgfqpoint{3.749228in}{3.460744in}}%
\pgfpathlineto{\pgfqpoint{3.768009in}{3.476447in}}%
\pgfpathlineto{\pgfqpoint{3.788669in}{3.538799in}}%
\pgfpathlineto{\pgfqpoint{3.808860in}{3.653760in}}%
\pgfpathlineto{\pgfqpoint{3.826937in}{3.862922in}}%
\pgfpathlineto{\pgfqpoint{3.844545in}{4.232499in}}%
\pgfpathlineto{\pgfqpoint{3.862624in}{4.333531in}}%
\pgfpathlineto{\pgfqpoint{3.886804in}{4.124746in}}%
\pgfpathlineto{\pgfqpoint{3.905821in}{3.793696in}}%
\pgfpathlineto{\pgfqpoint{3.923195in}{3.593925in}}%
\pgfpathlineto{\pgfqpoint{3.942211in}{3.506867in}}%
\pgfpathlineto{\pgfqpoint{3.958645in}{3.472471in}}%
\pgfpathlineto{\pgfqpoint{3.980243in}{3.459374in}}%
\pgfpathlineto{\pgfqpoint{4.000668in}{3.466774in}}%
\pgfpathlineto{\pgfqpoint{4.019451in}{3.494209in}}%
\pgfpathlineto{\pgfqpoint{4.037293in}{3.535647in}}%
\pgfpathlineto{\pgfqpoint{4.055136in}{3.640962in}}%
\pgfpathlineto{\pgfqpoint{4.076735in}{3.965979in}}%
\pgfpathlineto{\pgfqpoint{4.094812in}{4.308521in}}%
\pgfpathlineto{\pgfqpoint{4.115472in}{4.351224in}}%
\pgfpathlineto{\pgfqpoint{4.137072in}{4.195754in}}%
\pgfpathlineto{\pgfqpoint{4.153271in}{3.935484in}}%
\pgfpathlineto{\pgfqpoint{4.172757in}{3.673472in}}%
\pgfpathlineto{\pgfqpoint{4.191070in}{3.538589in}}%
\pgfpathlineto{\pgfqpoint{4.212668in}{3.487252in}}%
\pgfpathlineto{\pgfqpoint{4.230511in}{3.465294in}}%
\pgfpathlineto{\pgfqpoint{4.247884in}{3.460998in}}%
\pgfpathlineto{\pgfqpoint{4.265727in}{3.473219in}}%
\pgfpathlineto{\pgfqpoint{4.289673in}{3.519147in}}%
\pgfpathlineto{\pgfqpoint{4.308221in}{3.552216in}}%
\pgfpathlineto{\pgfqpoint{4.325594in}{3.685331in}}%
\pgfpathlineto{\pgfqpoint{4.365974in}{4.366688in}}%
\pgfpathlineto{\pgfqpoint{4.383113in}{4.410149in}}%
\pgfpathlineto{\pgfqpoint{4.403773in}{4.190519in}}%
\pgfpathlineto{\pgfqpoint{4.419738in}{3.911133in}}%
\pgfpathlineto{\pgfqpoint{4.440867in}{3.711918in}}%
\pgfpathlineto{\pgfqpoint{4.461996in}{3.557513in}}%
\pgfpathlineto{\pgfqpoint{4.480075in}{3.495425in}}%
\pgfpathlineto{\pgfqpoint{4.483830in}{3.490389in}}%
\pgfpathlineto{\pgfqpoint{4.475378in}{3.521671in}}%
\pgfpathlineto{\pgfqpoint{4.455189in}{3.684284in}}%
\pgfpathlineto{\pgfqpoint{4.437112in}{4.063367in}}%
\pgfpathlineto{\pgfqpoint{4.418798in}{4.380005in}}%
\pgfpathlineto{\pgfqpoint{4.396495in}{4.252139in}}%
\pgfpathlineto{\pgfqpoint{4.377948in}{3.783093in}}%
\pgfpathlineto{\pgfqpoint{4.359636in}{3.562864in}}%
\pgfpathlineto{\pgfqpoint{4.338271in}{3.482189in}}%
\pgfpathlineto{\pgfqpoint{4.320194in}{3.459375in}}%
\pgfpathlineto{\pgfqpoint{4.301648in}{3.474245in}}%
\pgfpathlineto{\pgfqpoint{4.284509in}{3.533694in}}%
\pgfpathlineto{\pgfqpoint{4.263380in}{3.744762in}}%
\pgfpathlineto{\pgfqpoint{4.245772in}{4.140144in}}%
\pgfpathlineto{\pgfqpoint{4.225581in}{4.377140in}}%
\pgfpathlineto{\pgfqpoint{4.204452in}{4.065751in}}%
\pgfpathlineto{\pgfqpoint{4.186373in}{3.677122in}}%
\pgfpathlineto{\pgfqpoint{4.164541in}{3.509560in}}%
\pgfpathlineto{\pgfqpoint{4.147402in}{3.488504in}}%
\pgfpathlineto{\pgfqpoint{4.128854in}{3.581944in}}%
\pgfpathlineto{\pgfqpoint{4.107960in}{3.929745in}}%
\pgfpathlineto{\pgfqpoint{4.090117in}{4.267991in}}%
\pgfpathlineto{\pgfqpoint{4.070163in}{4.292144in}}%
\pgfpathlineto{\pgfqpoint{4.052789in}{3.875246in}}%
\pgfpathlineto{\pgfqpoint{4.032833in}{3.568236in}}%
\pgfpathlineto{\pgfqpoint{4.012407in}{3.482043in}}%
\pgfpathlineto{\pgfqpoint{3.994565in}{3.458709in}}%
\pgfpathlineto{\pgfqpoint{3.974139in}{3.467447in}}%
\pgfpathlineto{\pgfqpoint{3.953479in}{3.521093in}}%
\pgfpathlineto{\pgfqpoint{3.936342in}{3.673994in}}%
\pgfpathlineto{\pgfqpoint{3.918500in}{4.041502in}}%
\pgfpathlineto{\pgfqpoint{3.899483in}{4.303555in}}%
\pgfpathlineto{\pgfqpoint{3.880232in}{4.163891in}}%
\pgfpathlineto{\pgfqpoint{3.856754in}{3.673255in}}%
\pgfpathlineto{\pgfqpoint{3.838675in}{3.528806in}}%
\pgfpathlineto{\pgfqpoint{3.821067in}{3.473546in}}%
\pgfpathlineto{\pgfqpoint{3.801113in}{3.457286in}}%
\pgfpathlineto{\pgfqpoint{3.783036in}{3.467772in}}%
\pgfpathlineto{\pgfqpoint{3.761436in}{3.538521in}}%
\pgfpathlineto{\pgfqpoint{3.745706in}{3.669512in}}%
\pgfpathlineto{\pgfqpoint{3.721994in}{4.112577in}}%
\pgfpathlineto{\pgfqpoint{3.707909in}{4.273908in}}%
\pgfpathlineto{\pgfqpoint{3.685840in}{4.159863in}}%
\pgfpathlineto{\pgfqpoint{3.666118in}{3.732880in}}%
\pgfpathlineto{\pgfqpoint{3.648510in}{3.550012in}}%
\pgfpathlineto{\pgfqpoint{3.628321in}{3.495186in}}%
\pgfpathlineto{\pgfqpoint{3.610007in}{3.462921in}}%
\pgfpathlineto{\pgfqpoint{3.593574in}{3.456812in}}%
\pgfpathlineto{\pgfqpoint{3.570096in}{3.472015in}}%
\pgfpathlineto{\pgfqpoint{3.552254in}{3.507719in}}%
\pgfpathlineto{\pgfqpoint{3.532063in}{3.635885in}}%
\pgfpathlineto{\pgfqpoint{3.515394in}{3.900914in}}%
\pgfpathlineto{\pgfqpoint{3.494969in}{4.180452in}}%
\pgfpathlineto{\pgfqpoint{3.477595in}{4.264365in}}%
\pgfpathlineto{\pgfqpoint{3.454823in}{3.948351in}}%
\pgfpathlineto{\pgfqpoint{3.436276in}{3.633132in}}%
\pgfpathlineto{\pgfqpoint{3.418668in}{3.552657in}}%
\pgfpathlineto{\pgfqpoint{3.397539in}{3.487407in}}%
\pgfpathlineto{\pgfqpoint{3.377819in}{3.460297in}}%
\pgfpathlineto{\pgfqpoint{3.361148in}{3.456423in}}%
\pgfpathlineto{\pgfqpoint{3.340254in}{3.473305in}}%
\pgfpathlineto{\pgfqpoint{3.321708in}{3.531391in}}%
\pgfpathlineto{\pgfqpoint{3.301283in}{3.739087in}}%
\pgfpathlineto{\pgfqpoint{3.283204in}{4.052667in}}%
\pgfpathlineto{\pgfqpoint{3.263015in}{4.250086in}}%
\pgfpathlineto{\pgfqpoint{3.242355in}{4.075522in}}%
\pgfpathlineto{\pgfqpoint{3.224041in}{3.739484in}}%
\pgfpathlineto{\pgfqpoint{3.205025in}{4.072723in}}%
\pgfpathlineto{\pgfqpoint{3.186948in}{3.719222in}}%
\pgfpathlineto{\pgfqpoint{3.168636in}{3.543680in}}%
\pgfpathlineto{\pgfqpoint{3.168871in}{3.500396in}}%
\pgfpathlineto{\pgfqpoint{3.150323in}{3.487609in}}%
\pgfpathlineto{\pgfqpoint{3.129897in}{3.460337in}}%
\pgfpathlineto{\pgfqpoint{3.108769in}{3.457435in}}%
\pgfpathlineto{\pgfqpoint{3.091395in}{3.476078in}}%
\pgfpathlineto{\pgfqpoint{3.070970in}{3.526120in}}%
\pgfpathlineto{\pgfqpoint{3.052893in}{3.685132in}}%
\pgfpathlineto{\pgfqpoint{3.031998in}{4.069478in}}%
\pgfpathlineto{\pgfqpoint{3.013216in}{4.245940in}}%
\pgfpathlineto{\pgfqpoint{2.995139in}{4.145584in}}%
\pgfpathlineto{\pgfqpoint{2.973305in}{3.743163in}}%
\pgfpathlineto{\pgfqpoint{2.954759in}{3.552891in}}%
\pgfpathlineto{\pgfqpoint{2.935977in}{3.484808in}}%
\pgfpathlineto{\pgfqpoint{2.920715in}{3.467458in}}%
\pgfpathlineto{\pgfqpoint{2.898647in}{3.456474in}}%
\pgfpathlineto{\pgfqpoint{2.880335in}{3.454848in}}%
\pgfpathlineto{\pgfqpoint{2.857798in}{3.454302in}}%
\pgfpathlineto{\pgfqpoint{2.840424in}{3.464815in}}%
\pgfpathlineto{\pgfqpoint{2.820938in}{3.504016in}}%
\pgfpathlineto{\pgfqpoint{2.804739in}{3.591293in}}%
\pgfpathlineto{\pgfqpoint{2.784314in}{3.933307in}}%
\pgfpathlineto{\pgfqpoint{2.762948in}{4.193518in}}%
\pgfpathlineto{\pgfqpoint{2.744402in}{4.196653in}}%
\pgfpathlineto{\pgfqpoint{2.725151in}{3.882354in}}%
\pgfpathlineto{\pgfqpoint{2.705195in}{3.637906in}}%
\pgfpathlineto{\pgfqpoint{2.686649in}{3.515148in}}%
\pgfpathlineto{\pgfqpoint{2.668335in}{3.471532in}}%
\pgfpathlineto{\pgfqpoint{2.649790in}{3.456254in}}%
\pgfpathlineto{\pgfqpoint{2.631242in}{3.459363in}}%
\pgfpathlineto{\pgfqpoint{2.609876in}{3.480412in}}%
\pgfpathlineto{\pgfqpoint{2.590156in}{3.536795in}}%
\pgfpathlineto{\pgfqpoint{2.589453in}{3.624875in}}%
\pgfpathlineto{\pgfqpoint{2.571609in}{3.705928in}}%
\pgfpathlineto{\pgfqpoint{2.552357in}{4.056410in}}%
\pgfpathlineto{\pgfqpoint{2.532872in}{4.226793in}}%
\pgfpathlineto{\pgfqpoint{2.512212in}{4.136493in}}%
\pgfpathlineto{\pgfqpoint{2.495778in}{3.789839in}}%
\pgfpathlineto{\pgfqpoint{2.476527in}{3.569211in}}%
\pgfpathlineto{\pgfqpoint{2.459153in}{3.512779in}}%
\pgfpathlineto{\pgfqpoint{2.436381in}{3.466519in}}%
\pgfpathlineto{\pgfqpoint{2.416659in}{3.454903in}}%
\pgfpathlineto{\pgfqpoint{2.397408in}{3.455986in}}%
\pgfpathlineto{\pgfqpoint{2.379096in}{3.468643in}}%
\pgfpathlineto{\pgfqpoint{2.359845in}{3.512238in}}%
\pgfpathlineto{\pgfqpoint{2.341766in}{3.648285in}}%
\pgfpathlineto{\pgfqpoint{2.319229in}{3.996942in}}%
\pgfpathlineto{\pgfqpoint{2.302092in}{4.203602in}}%
\pgfpathlineto{\pgfqpoint{2.285187in}{4.201589in}}%
\pgfpathlineto{\pgfqpoint{2.263824in}{3.956673in}}%
\pgfpathlineto{\pgfqpoint{2.244336in}{3.665171in}}%
\pgfpathlineto{\pgfqpoint{2.226494in}{3.539271in}}%
\pgfpathlineto{\pgfqpoint{2.205130in}{3.478497in}}%
\pgfpathlineto{\pgfqpoint{2.186582in}{3.462885in}}%
\pgfpathlineto{\pgfqpoint{2.168505in}{3.454805in}}%
\pgfpathlineto{\pgfqpoint{2.148785in}{3.464645in}}%
\pgfpathlineto{\pgfqpoint{2.127420in}{3.500813in}}%
\pgfpathlineto{\pgfqpoint{2.108872in}{3.584098in}}%
\pgfpathlineto{\pgfqpoint{2.090326in}{3.807676in}}%
\pgfpathlineto{\pgfqpoint{2.072013in}{4.053736in}}%
\pgfpathlineto{\pgfqpoint{2.053233in}{4.227856in}}%
\pgfpathlineto{\pgfqpoint{2.031633in}{4.176213in}}%
\pgfpathlineto{\pgfqpoint{2.013085in}{4.023054in}}%
\pgfpathlineto{\pgfqpoint{1.997826in}{3.722803in}}%
\pgfpathlineto{\pgfqpoint{1.975288in}{3.542595in}}%
\pgfpathlineto{\pgfqpoint{1.957209in}{3.492477in}}%
\pgfpathlineto{\pgfqpoint{1.934672in}{3.463615in}}%
\pgfpathlineto{\pgfqpoint{1.919412in}{3.456220in}}%
\pgfpathlineto{\pgfqpoint{1.898283in}{3.460719in}}%
\pgfpathlineto{\pgfqpoint{1.879501in}{3.473747in}}%
\pgfpathlineto{\pgfqpoint{1.858606in}{3.475961in}}%
\pgfpathlineto{\pgfqpoint{1.839824in}{3.519306in}}%
\pgfpathlineto{\pgfqpoint{1.822451in}{3.639504in}}%
\pgfpathlineto{\pgfqpoint{1.801322in}{3.968321in}}%
\pgfpathlineto{\pgfqpoint{1.784183in}{4.194713in}}%
\pgfpathlineto{\pgfqpoint{1.762114in}{4.246058in}}%
\pgfpathlineto{\pgfqpoint{1.746620in}{4.110166in}}%
\pgfpathlineto{\pgfqpoint{1.724786in}{3.784546in}}%
\pgfpathlineto{\pgfqpoint{1.706004in}{3.594892in}}%
\pgfpathlineto{\pgfqpoint{1.687927in}{3.505801in}}%
\pgfpathlineto{\pgfqpoint{1.669144in}{3.472376in}}%
\pgfpathlineto{\pgfqpoint{1.647781in}{3.457721in}}%
\pgfpathlineto{\pgfqpoint{1.628528in}{3.457631in}}%
\pgfpathlineto{\pgfqpoint{1.610216in}{3.464865in}}%
\pgfpathlineto{\pgfqpoint{1.590965in}{4.206462in}}%
\pgfpathlineto{\pgfqpoint{1.569600in}{3.796545in}}%
\pgfpathlineto{\pgfqpoint{1.547768in}{3.548770in}}%
\pgfpathlineto{\pgfqpoint{1.514429in}{3.475766in}}%
\pgfpathlineto{\pgfqpoint{1.495647in}{3.456232in}}%
\pgfpathlineto{\pgfqpoint{1.476865in}{3.460522in}}%
\pgfpathlineto{\pgfqpoint{1.455501in}{3.487597in}}%
\pgfpathlineto{\pgfqpoint{1.436485in}{3.563667in}}%
\pgfpathlineto{\pgfqpoint{1.418173in}{3.771823in}}%
\pgfpathlineto{\pgfqpoint{1.399625in}{4.123181in}}%
\pgfpathlineto{\pgfqpoint{1.381549in}{4.271258in}}%
\pgfpathlineto{\pgfqpoint{1.359480in}{4.146973in}}%
\pgfpathlineto{\pgfqpoint{1.340698in}{3.777270in}}%
\pgfpathlineto{\pgfqpoint{1.322855in}{3.587881in}}%
\pgfpathlineto{\pgfqpoint{1.301255in}{3.489787in}}%
\pgfpathlineto{\pgfqpoint{1.282709in}{3.468109in}}%
\pgfpathlineto{\pgfqpoint{1.245616in}{3.456938in}}%
\pgfpathlineto{\pgfqpoint{1.226833in}{3.462008in}}%
\pgfpathlineto{\pgfqpoint{1.205234in}{3.489614in}}%
\pgfpathlineto{\pgfqpoint{1.187157in}{3.561414in}}%
\pgfpathlineto{\pgfqpoint{1.168375in}{3.700113in}}%
\pgfpathlineto{\pgfqpoint{1.147011in}{4.081890in}}%
\pgfpathlineto{\pgfqpoint{1.129638in}{4.297568in}}%
\pgfpathlineto{\pgfqpoint{1.112030in}{4.276530in}}%
\pgfpathlineto{\pgfqpoint{1.090430in}{3.931748in}}%
\pgfpathlineto{\pgfqpoint{1.069536in}{3.629673in}}%
\pgfpathlineto{\pgfqpoint{1.054276in}{3.534552in}}%
\pgfpathlineto{\pgfqpoint{1.035494in}{3.484864in}}%
\pgfpathlineto{\pgfqpoint{1.016946in}{3.467546in}}%
\pgfpathlineto{\pgfqpoint{0.995348in}{3.458627in}}%
\pgfpathlineto{\pgfqpoint{0.975861in}{3.459871in}}%
\pgfpathlineto{\pgfqpoint{0.958489in}{3.479005in}}%
\pgfpathlineto{\pgfqpoint{0.936420in}{3.559293in}}%
\pgfpathlineto{\pgfqpoint{0.917403in}{3.732125in}}%
\pgfpathlineto{\pgfqpoint{0.899795in}{4.043904in}}%
\pgfpathlineto{\pgfqpoint{0.877727in}{3.861973in}}%
\pgfpathlineto{\pgfqpoint{0.859650in}{4.202126in}}%
\pgfpathlineto{\pgfqpoint{0.843919in}{4.338142in}}%
\pgfpathlineto{\pgfqpoint{0.822791in}{4.247701in}}%
\pgfpathlineto{\pgfqpoint{0.803774in}{3.857875in}}%
\pgfpathlineto{\pgfqpoint{0.784052in}{3.628234in}}%
\pgfpathlineto{\pgfqpoint{0.763157in}{3.533133in}}%
\pgfpathlineto{\pgfqpoint{0.745080in}{3.489288in}}%
\pgfpathlineto{\pgfqpoint{0.726064in}{3.464485in}}%
\pgfpathlineto{\pgfqpoint{0.708925in}{3.459306in}}%
\pgfpathlineto{\pgfqpoint{0.686153in}{3.472123in}}%
\pgfpathlineto{\pgfqpoint{0.669013in}{3.515455in}}%
\pgfpathlineto{\pgfqpoint{0.649059in}{3.616566in}}%
\pgfpathlineto{\pgfqpoint{0.648822in}{3.608478in}}%
\pgfpathlineto{\pgfqpoint{0.654694in}{3.558327in}}%
\pgfpathlineto{\pgfqpoint{0.675588in}{3.475068in}}%
\pgfpathlineto{\pgfqpoint{0.694605in}{3.458436in}}%
\pgfpathlineto{\pgfqpoint{0.711977in}{3.474718in}}%
\pgfpathlineto{\pgfqpoint{0.733576in}{3.543158in}}%
\pgfpathlineto{\pgfqpoint{0.751184in}{3.732279in}}%
\pgfpathlineto{\pgfqpoint{0.769732in}{4.159471in}}%
\pgfpathlineto{\pgfqpoint{0.790156in}{4.324019in}}%
\pgfpathlineto{\pgfqpoint{0.808469in}{4.049214in}}%
\pgfpathlineto{\pgfqpoint{0.830069in}{3.633638in}}%
\pgfpathlineto{\pgfqpoint{0.848380in}{3.494500in}}%
\pgfpathlineto{\pgfqpoint{0.867631in}{3.462840in}}%
\pgfpathlineto{\pgfqpoint{0.887351in}{3.464794in}}%
\pgfpathlineto{\pgfqpoint{0.905899in}{3.712184in}}%
\pgfpathlineto{\pgfqpoint{0.923976in}{4.150535in}}%
\pgfpathlineto{\pgfqpoint{0.943227in}{4.295270in}}%
\pgfpathlineto{\pgfqpoint{0.962715in}{3.966547in}}%
\pgfpathlineto{\pgfqpoint{0.984313in}{3.630720in}}%
\pgfpathlineto{\pgfqpoint{1.002861in}{3.494826in}}%
\pgfpathlineto{\pgfqpoint{1.021643in}{3.460207in}}%
\pgfpathlineto{\pgfqpoint{1.041363in}{3.459410in}}%
\pgfpathlineto{\pgfqpoint{1.059909in}{3.488348in}}%
\pgfpathlineto{\pgfqpoint{1.079162in}{3.567744in}}%
\pgfpathlineto{\pgfqpoint{1.097942in}{3.865071in}}%
\pgfpathlineto{\pgfqpoint{1.117899in}{4.255393in}}%
\pgfpathlineto{\pgfqpoint{1.133629in}{4.174593in}}%
\pgfpathlineto{\pgfqpoint{1.175184in}{3.537417in}}%
\pgfpathlineto{\pgfqpoint{1.192086in}{3.477876in}}%
\pgfpathlineto{\pgfqpoint{1.213217in}{3.455576in}}%
\pgfpathlineto{\pgfqpoint{1.233172in}{3.462940in}}%
\pgfpathlineto{\pgfqpoint{1.252189in}{3.499757in}}%
\pgfpathlineto{\pgfqpoint{1.271440in}{3.620468in}}%
\pgfpathlineto{\pgfqpoint{1.290456in}{3.957771in}}%
\pgfpathlineto{\pgfqpoint{1.309004in}{4.250131in}}%
\pgfpathlineto{\pgfqpoint{1.328255in}{4.128748in}}%
\pgfpathlineto{\pgfqpoint{1.346801in}{3.768742in}}%
\pgfpathlineto{\pgfqpoint{1.365584in}{3.532899in}}%
\pgfpathlineto{\pgfqpoint{1.385069in}{3.470199in}}%
\pgfpathlineto{\pgfqpoint{1.407372in}{3.453732in}}%
\pgfpathlineto{\pgfqpoint{1.423103in}{3.458603in}}%
\pgfpathlineto{\pgfqpoint{1.443294in}{3.492162in}}%
\pgfpathlineto{\pgfqpoint{1.463248in}{3.576536in}}%
\pgfpathlineto{\pgfqpoint{1.481562in}{3.833944in}}%
\pgfpathlineto{\pgfqpoint{1.501987in}{4.175226in}}%
\pgfpathlineto{\pgfqpoint{1.520768in}{4.218703in}}%
\pgfpathlineto{\pgfqpoint{1.540724in}{3.946211in}}%
\pgfpathlineto{\pgfqpoint{1.558567in}{3.666668in}}%
\pgfpathlineto{\pgfqpoint{1.579461in}{3.509140in}}%
\pgfpathlineto{\pgfqpoint{1.595895in}{3.467883in}}%
\pgfpathlineto{\pgfqpoint{1.619607in}{3.453759in}}%
\pgfpathlineto{\pgfqpoint{1.636980in}{3.456243in}}%
\pgfpathlineto{\pgfqpoint{1.655997in}{3.476491in}}%
\pgfpathlineto{\pgfqpoint{1.671962in}{3.532314in}}%
\pgfpathlineto{\pgfqpoint{1.694030in}{3.632560in}}%
\pgfpathlineto{\pgfqpoint{1.712342in}{3.998536in}}%
\pgfpathlineto{\pgfqpoint{1.730655in}{4.217761in}}%
\pgfpathlineto{\pgfqpoint{1.749201in}{4.157154in}}%
\pgfpathlineto{\pgfqpoint{1.772210in}{3.794834in}}%
\pgfpathlineto{\pgfqpoint{1.791461in}{3.561492in}}%
\pgfpathlineto{\pgfqpoint{1.810477in}{3.479465in}}%
\pgfpathlineto{\pgfqpoint{1.828554in}{3.459907in}}%
\pgfpathlineto{\pgfqpoint{1.847571in}{3.452995in}}%
\pgfpathlineto{\pgfqpoint{1.866822in}{3.456663in}}%
\pgfpathlineto{\pgfqpoint{1.884430in}{3.475388in}}%
\pgfpathlineto{\pgfqpoint{1.903682in}{3.535982in}}%
\pgfpathlineto{\pgfqpoint{1.924576in}{3.746778in}}%
\pgfpathlineto{\pgfqpoint{1.943827in}{4.100894in}}%
\pgfpathlineto{\pgfqpoint{1.962375in}{4.196820in}}%
\pgfpathlineto{\pgfqpoint{1.981392in}{4.001579in}}%
\pgfpathlineto{\pgfqpoint{2.002992in}{3.759695in}}%
\pgfpathlineto{\pgfqpoint{2.018954in}{3.553706in}}%
\pgfpathlineto{\pgfqpoint{2.039614in}{3.869264in}}%
\pgfpathlineto{\pgfqpoint{2.058397in}{3.582988in}}%
\pgfpathlineto{\pgfqpoint{2.076944in}{3.488269in}}%
\pgfpathlineto{\pgfqpoint{2.097839in}{3.457030in}}%
\pgfpathlineto{\pgfqpoint{2.116387in}{3.453311in}}%
\pgfpathlineto{\pgfqpoint{2.134698in}{3.463772in}}%
\pgfpathlineto{\pgfqpoint{2.156532in}{3.502307in}}%
\pgfpathlineto{\pgfqpoint{2.174140in}{3.538854in}}%
\pgfpathlineto{\pgfqpoint{2.191512in}{3.685352in}}%
\pgfpathlineto{\pgfqpoint{2.212174in}{4.070095in}}%
\pgfpathlineto{\pgfqpoint{2.235415in}{4.189052in}}%
\pgfpathlineto{\pgfqpoint{2.252319in}{4.076099in}}%
\pgfpathlineto{\pgfqpoint{2.269693in}{3.752230in}}%
\pgfpathlineto{\pgfqpoint{2.288473in}{3.705223in}}%
\pgfpathlineto{\pgfqpoint{2.308664in}{3.513649in}}%
\pgfpathlineto{\pgfqpoint{2.326507in}{3.502160in}}%
\pgfpathlineto{\pgfqpoint{2.348341in}{3.462738in}}%
\pgfpathlineto{\pgfqpoint{2.365480in}{3.452939in}}%
\pgfpathlineto{\pgfqpoint{2.386374in}{3.461118in}}%
\pgfpathlineto{\pgfqpoint{2.404686in}{3.487864in}}%
\pgfpathlineto{\pgfqpoint{2.425346in}{3.583819in}}%
\pgfpathlineto{\pgfqpoint{2.442250in}{3.815561in}}%
\pgfpathlineto{\pgfqpoint{2.460327in}{4.034965in}}%
\pgfpathlineto{\pgfqpoint{2.481927in}{4.186842in}}%
\pgfpathlineto{\pgfqpoint{2.499299in}{4.002698in}}%
\pgfpathlineto{\pgfqpoint{2.518786in}{3.660348in}}%
\pgfpathlineto{\pgfqpoint{2.538741in}{3.509079in}}%
\pgfpathlineto{\pgfqpoint{2.559166in}{3.471984in}}%
\pgfpathlineto{\pgfqpoint{2.577478in}{3.473774in}}%
\pgfpathlineto{\pgfqpoint{2.597669in}{3.454416in}}%
\pgfpathlineto{\pgfqpoint{2.615277in}{3.454646in}}%
\pgfpathlineto{\pgfqpoint{2.633825in}{3.471303in}}%
\pgfpathlineto{\pgfqpoint{2.655422in}{3.521150in}}%
\pgfpathlineto{\pgfqpoint{2.672796in}{3.594801in}}%
\pgfpathlineto{\pgfqpoint{2.690873in}{3.840342in}}%
\pgfpathlineto{\pgfqpoint{2.711769in}{4.162984in}}%
\pgfpathlineto{\pgfqpoint{2.732898in}{4.142333in}}%
\pgfpathlineto{\pgfqpoint{2.750741in}{3.902389in}}%
\pgfpathlineto{\pgfqpoint{2.769992in}{3.608667in}}%
\pgfpathlineto{\pgfqpoint{2.790886in}{3.519127in}}%
\pgfpathlineto{\pgfqpoint{2.808025in}{3.472842in}}%
\pgfpathlineto{\pgfqpoint{2.829860in}{3.455363in}}%
\pgfpathlineto{\pgfqpoint{2.848405in}{3.456588in}}%
\pgfpathlineto{\pgfqpoint{2.866719in}{3.465120in}}%
\pgfpathlineto{\pgfqpoint{2.887144in}{3.504355in}}%
\pgfpathlineto{\pgfqpoint{2.908977in}{3.628561in}}%
\pgfpathlineto{\pgfqpoint{2.923298in}{3.773733in}}%
\pgfpathlineto{\pgfqpoint{2.941375in}{4.095270in}}%
\pgfpathlineto{\pgfqpoint{2.962271in}{4.223653in}}%
\pgfpathlineto{\pgfqpoint{2.979879in}{4.094923in}}%
\pgfpathlineto{\pgfqpoint{3.000774in}{3.721387in}}%
\pgfpathlineto{\pgfqpoint{3.022137in}{3.529454in}}%
\pgfpathlineto{\pgfqpoint{3.039980in}{3.482470in}}%
\pgfpathlineto{\pgfqpoint{3.058762in}{3.464701in}}%
\pgfpathlineto{\pgfqpoint{3.078482in}{3.455370in}}%
\pgfpathlineto{\pgfqpoint{3.114404in}{3.455298in}}%
\pgfpathlineto{\pgfqpoint{3.136003in}{3.471875in}}%
\pgfpathlineto{\pgfqpoint{3.155254in}{3.526046in}}%
\pgfpathlineto{\pgfqpoint{3.172157in}{3.634089in}}%
\pgfpathlineto{\pgfqpoint{3.212068in}{4.197188in}}%
\pgfpathlineto{\pgfqpoint{3.235311in}{4.195078in}}%
\pgfpathlineto{\pgfqpoint{3.249633in}{4.126710in}}%
\pgfpathlineto{\pgfqpoint{3.267944in}{3.866996in}}%
\pgfpathlineto{\pgfqpoint{3.289310in}{3.586550in}}%
\pgfpathlineto{\pgfqpoint{3.308324in}{3.500059in}}%
\pgfpathlineto{\pgfqpoint{3.327107in}{3.464037in}}%
\pgfpathlineto{\pgfqpoint{3.347532in}{3.455194in}}%
\pgfpathlineto{\pgfqpoint{3.365844in}{3.459968in}}%
\pgfpathlineto{\pgfqpoint{3.385097in}{3.474755in}}%
\pgfpathlineto{\pgfqpoint{3.403643in}{3.504665in}}%
\pgfpathlineto{\pgfqpoint{3.422190in}{3.581547in}}%
\pgfpathlineto{\pgfqpoint{3.441910in}{3.776474in}}%
\pgfpathlineto{\pgfqpoint{3.460693in}{4.079928in}}%
\pgfpathlineto{\pgfqpoint{3.481118in}{4.267232in}}%
\pgfpathlineto{\pgfqpoint{3.499195in}{4.198415in}}%
\pgfpathlineto{\pgfqpoint{3.513517in}{3.912554in}}%
\pgfpathlineto{\pgfqpoint{3.537698in}{3.774585in}}%
\pgfpathlineto{\pgfqpoint{3.558123in}{3.572574in}}%
\pgfpathlineto{\pgfqpoint{3.575497in}{3.513094in}}%
\pgfpathlineto{\pgfqpoint{3.596625in}{3.473263in}}%
\pgfpathlineto{\pgfqpoint{3.612825in}{3.458595in}}%
\pgfpathlineto{\pgfqpoint{3.632545in}{3.498697in}}%
\pgfpathlineto{\pgfqpoint{3.653910in}{3.466444in}}%
\pgfpathlineto{\pgfqpoint{3.673161in}{3.458296in}}%
\pgfpathlineto{\pgfqpoint{3.691238in}{3.460217in}}%
\pgfpathlineto{\pgfqpoint{3.713073in}{3.488331in}}%
\pgfpathlineto{\pgfqpoint{3.730446in}{3.543148in}}%
\pgfpathlineto{\pgfqpoint{3.747820in}{3.646970in}}%
\pgfpathlineto{\pgfqpoint{3.789608in}{4.289622in}}%
\pgfpathlineto{\pgfqpoint{3.808860in}{4.268962in}}%
\pgfpathlineto{\pgfqpoint{3.825999in}{4.103231in}}%
\pgfpathlineto{\pgfqpoint{3.846893in}{3.769187in}}%
\pgfpathlineto{\pgfqpoint{3.865675in}{3.623372in}}%
\pgfpathlineto{\pgfqpoint{3.886335in}{3.511071in}}%
\pgfpathlineto{\pgfqpoint{3.902769in}{3.475188in}}%
\pgfpathlineto{\pgfqpoint{3.922020in}{3.460001in}}%
\pgfpathlineto{\pgfqpoint{3.941975in}{3.459964in}}%
\pgfpathlineto{\pgfqpoint{3.958880in}{3.473892in}}%
\pgfpathlineto{\pgfqpoint{3.979774in}{3.504003in}}%
\pgfpathlineto{\pgfqpoint{3.997616in}{3.554416in}}%
\pgfpathlineto{\pgfqpoint{4.018511in}{3.727758in}}%
\pgfpathlineto{\pgfqpoint{4.037059in}{4.023682in}}%
\pgfpathlineto{\pgfqpoint{4.058188in}{4.327919in}}%
\pgfpathlineto{\pgfqpoint{4.075561in}{3.507449in}}%
\pgfpathlineto{\pgfqpoint{4.093874in}{3.604429in}}%
\pgfpathlineto{\pgfqpoint{4.134960in}{4.266208in}}%
\pgfpathlineto{\pgfqpoint{4.153740in}{4.356160in}}%
\pgfpathlineto{\pgfqpoint{4.172288in}{4.219967in}}%
\pgfpathlineto{\pgfqpoint{4.192948in}{3.861395in}}%
\pgfpathlineto{\pgfqpoint{4.209147in}{3.636474in}}%
\pgfpathlineto{\pgfqpoint{4.229573in}{3.519461in}}%
\pgfpathlineto{\pgfqpoint{4.250702in}{3.487326in}}%
\pgfpathlineto{\pgfqpoint{4.268544in}{3.468278in}}%
\pgfpathlineto{\pgfqpoint{4.289438in}{3.459616in}}%
\pgfpathlineto{\pgfqpoint{4.306812in}{3.471095in}}%
\pgfpathlineto{\pgfqpoint{4.326298in}{3.498896in}}%
\pgfpathlineto{\pgfqpoint{4.346958in}{3.591070in}}%
\pgfpathlineto{\pgfqpoint{4.363628in}{3.773047in}}%
\pgfpathlineto{\pgfqpoint{4.384757in}{4.107542in}}%
\pgfpathlineto{\pgfqpoint{4.404008in}{4.385119in}}%
\pgfpathlineto{\pgfqpoint{4.422085in}{4.357960in}}%
\pgfpathlineto{\pgfqpoint{4.444153in}{4.353942in}}%
\pgfpathlineto{\pgfqpoint{4.461996in}{4.074679in}}%
\pgfpathlineto{\pgfqpoint{4.479840in}{3.753025in}}%
\pgfpathlineto{\pgfqpoint{4.479840in}{3.754235in}}%
\pgfpathlineto{\pgfqpoint{4.454483in}{4.300002in}}%
\pgfpathlineto{\pgfqpoint{4.434529in}{4.378441in}}%
\pgfpathlineto{\pgfqpoint{4.414338in}{3.906874in}}%
\pgfpathlineto{\pgfqpoint{4.396495in}{3.593918in}}%
\pgfpathlineto{\pgfqpoint{4.377010in}{3.493366in}}%
\pgfpathlineto{\pgfqpoint{4.359167in}{3.461118in}}%
\pgfpathlineto{\pgfqpoint{4.340150in}{3.465527in}}%
\pgfpathlineto{\pgfqpoint{4.318551in}{3.515105in}}%
\pgfpathlineto{\pgfqpoint{4.300474in}{3.661140in}}%
\pgfpathlineto{\pgfqpoint{4.281457in}{4.042453in}}%
\pgfpathlineto{\pgfqpoint{4.264789in}{4.329149in}}%
\pgfpathlineto{\pgfqpoint{4.244129in}{4.204331in}}%
\pgfpathlineto{\pgfqpoint{4.223233in}{3.716399in}}%
\pgfpathlineto{\pgfqpoint{4.205390in}{3.536302in}}%
\pgfpathlineto{\pgfqpoint{4.186844in}{3.473944in}}%
\pgfpathlineto{\pgfqpoint{4.167357in}{3.457425in}}%
\pgfpathlineto{\pgfqpoint{4.146933in}{3.478456in}}%
\pgfpathlineto{\pgfqpoint{4.130968in}{3.528196in}}%
\pgfpathlineto{\pgfqpoint{4.109134in}{3.843372in}}%
\pgfpathlineto{\pgfqpoint{4.090586in}{4.194813in}}%
\pgfpathlineto{\pgfqpoint{4.069926in}{4.304665in}}%
\pgfpathlineto{\pgfqpoint{4.049737in}{3.898244in}}%
\pgfpathlineto{\pgfqpoint{4.033538in}{3.600367in}}%
\pgfpathlineto{\pgfqpoint{4.011938in}{3.487770in}}%
\pgfpathlineto{\pgfqpoint{3.995034in}{3.459897in}}%
\pgfpathlineto{\pgfqpoint{3.974610in}{3.460455in}}%
\pgfpathlineto{\pgfqpoint{3.954185in}{3.464743in}}%
\pgfpathlineto{\pgfqpoint{3.934462in}{3.480679in}}%
\pgfpathlineto{\pgfqpoint{3.916151in}{3.558241in}}%
\pgfpathlineto{\pgfqpoint{3.896194in}{3.876934in}}%
\pgfpathlineto{\pgfqpoint{3.877178in}{4.210792in}}%
\pgfpathlineto{\pgfqpoint{3.858632in}{4.237216in}}%
\pgfpathlineto{\pgfqpoint{3.840555in}{3.889720in}}%
\pgfpathlineto{\pgfqpoint{3.821304in}{3.572206in}}%
\pgfpathlineto{\pgfqpoint{3.797355in}{3.474832in}}%
\pgfpathlineto{\pgfqpoint{3.783036in}{3.458570in}}%
\pgfpathlineto{\pgfqpoint{3.761202in}{3.459318in}}%
\pgfpathlineto{\pgfqpoint{3.744297in}{3.483770in}}%
\pgfpathlineto{\pgfqpoint{3.723871in}{3.587211in}}%
\pgfpathlineto{\pgfqpoint{3.706500in}{3.808812in}}%
\pgfpathlineto{\pgfqpoint{3.685840in}{4.144317in}}%
\pgfpathlineto{\pgfqpoint{3.666589in}{4.249413in}}%
\pgfpathlineto{\pgfqpoint{3.648275in}{3.952092in}}%
\pgfpathlineto{\pgfqpoint{3.627381in}{3.590671in}}%
\pgfpathlineto{\pgfqpoint{3.609538in}{3.496145in}}%
\pgfpathlineto{\pgfqpoint{3.592399in}{3.463316in}}%
\pgfpathlineto{\pgfqpoint{3.573383in}{3.455487in}}%
\pgfpathlineto{\pgfqpoint{3.551079in}{3.468718in}}%
\pgfpathlineto{\pgfqpoint{3.530654in}{3.507944in}}%
\pgfpathlineto{\pgfqpoint{3.512577in}{3.624131in}}%
\pgfpathlineto{\pgfqpoint{3.496378in}{3.911073in}}%
\pgfpathlineto{\pgfqpoint{3.474544in}{4.214024in}}%
\pgfpathlineto{\pgfqpoint{3.454823in}{4.202805in}}%
\pgfpathlineto{\pgfqpoint{3.437215in}{3.815604in}}%
\pgfpathlineto{\pgfqpoint{3.418668in}{3.590010in}}%
\pgfpathlineto{\pgfqpoint{3.398008in}{3.498153in}}%
\pgfpathlineto{\pgfqpoint{3.375939in}{3.461781in}}%
\pgfpathlineto{\pgfqpoint{3.358802in}{3.454443in}}%
\pgfpathlineto{\pgfqpoint{3.340488in}{3.458423in}}%
\pgfpathlineto{\pgfqpoint{3.319829in}{3.470626in}}%
\pgfpathlineto{\pgfqpoint{3.298465in}{3.537023in}}%
\pgfpathlineto{\pgfqpoint{3.283440in}{3.640725in}}%
\pgfpathlineto{\pgfqpoint{3.263484in}{4.027586in}}%
\pgfpathlineto{\pgfqpoint{3.243998in}{4.232023in}}%
\pgfpathlineto{\pgfqpoint{3.226155in}{4.151041in}}%
\pgfpathlineto{\pgfqpoint{3.207608in}{4.016814in}}%
\pgfpathlineto{\pgfqpoint{3.186008in}{3.664482in}}%
\pgfpathlineto{\pgfqpoint{3.165584in}{3.513353in}}%
\pgfpathlineto{\pgfqpoint{3.147271in}{3.468698in}}%
\pgfpathlineto{\pgfqpoint{3.130134in}{3.454964in}}%
\pgfpathlineto{\pgfqpoint{3.106891in}{3.459677in}}%
\pgfpathlineto{\pgfqpoint{3.090221in}{3.470576in}}%
\pgfpathlineto{\pgfqpoint{3.069561in}{3.522934in}}%
\pgfpathlineto{\pgfqpoint{3.051015in}{3.702696in}}%
\pgfpathlineto{\pgfqpoint{3.033641in}{4.008750in}}%
\pgfpathlineto{\pgfqpoint{3.012278in}{4.220167in}}%
\pgfpathlineto{\pgfqpoint{2.991618in}{4.043693in}}%
\pgfpathlineto{\pgfqpoint{2.973305in}{3.686527in}}%
\pgfpathlineto{\pgfqpoint{2.955697in}{3.537422in}}%
\pgfpathlineto{\pgfqpoint{2.933628in}{3.472307in}}%
\pgfpathlineto{\pgfqpoint{2.917898in}{3.458609in}}%
\pgfpathlineto{\pgfqpoint{2.896769in}{3.455665in}}%
\pgfpathlineto{\pgfqpoint{2.878457in}{3.467554in}}%
\pgfpathlineto{\pgfqpoint{2.859206in}{3.498632in}}%
\pgfpathlineto{\pgfqpoint{2.840893in}{3.609314in}}%
\pgfpathlineto{\pgfqpoint{2.823519in}{3.864899in}}%
\pgfpathlineto{\pgfqpoint{2.803799in}{4.117796in}}%
\pgfpathlineto{\pgfqpoint{2.781496in}{4.203050in}}%
\pgfpathlineto{\pgfqpoint{2.762479in}{4.041931in}}%
\pgfpathlineto{\pgfqpoint{2.743228in}{3.692370in}}%
\pgfpathlineto{\pgfqpoint{2.725151in}{3.540728in}}%
\pgfpathlineto{\pgfqpoint{2.706603in}{4.068967in}}%
\pgfpathlineto{\pgfqpoint{2.688292in}{4.037055in}}%
\pgfpathlineto{\pgfqpoint{2.666223in}{3.655603in}}%
\pgfpathlineto{\pgfqpoint{2.647207in}{3.513400in}}%
\pgfpathlineto{\pgfqpoint{2.629128in}{3.471523in}}%
\pgfpathlineto{\pgfqpoint{2.610113in}{3.458302in}}%
\pgfpathlineto{\pgfqpoint{2.590625in}{3.453202in}}%
\pgfpathlineto{\pgfqpoint{2.568793in}{3.469016in}}%
\pgfpathlineto{\pgfqpoint{2.550714in}{3.463825in}}%
\pgfpathlineto{\pgfqpoint{2.532168in}{3.453406in}}%
\pgfpathlineto{\pgfqpoint{2.514795in}{3.461758in}}%
\pgfpathlineto{\pgfqpoint{2.496481in}{3.497077in}}%
\pgfpathlineto{\pgfqpoint{2.476527in}{3.551503in}}%
\pgfpathlineto{\pgfqpoint{2.458684in}{3.780777in}}%
\pgfpathlineto{\pgfqpoint{2.435910in}{4.138861in}}%
\pgfpathlineto{\pgfqpoint{2.418539in}{4.181070in}}%
\pgfpathlineto{\pgfqpoint{2.399051in}{3.847141in}}%
\pgfpathlineto{\pgfqpoint{2.378157in}{3.568990in}}%
\pgfpathlineto{\pgfqpoint{2.359140in}{3.491375in}}%
\pgfpathlineto{\pgfqpoint{2.339654in}{3.460404in}}%
\pgfpathlineto{\pgfqpoint{2.321106in}{3.452777in}}%
\pgfpathlineto{\pgfqpoint{2.303264in}{3.460152in}}%
\pgfpathlineto{\pgfqpoint{2.284718in}{3.484915in}}%
\pgfpathlineto{\pgfqpoint{2.261006in}{3.620428in}}%
\pgfpathlineto{\pgfqpoint{2.244102in}{3.904747in}}%
\pgfpathlineto{\pgfqpoint{2.224850in}{4.173747in}}%
\pgfpathlineto{\pgfqpoint{2.207008in}{4.189127in}}%
\pgfpathlineto{\pgfqpoint{2.187757in}{3.887365in}}%
\pgfpathlineto{\pgfqpoint{2.165688in}{3.615321in}}%
\pgfpathlineto{\pgfqpoint{2.147845in}{3.509472in}}%
\pgfpathlineto{\pgfqpoint{2.128594in}{3.477754in}}%
\pgfpathlineto{\pgfqpoint{2.110752in}{3.460056in}}%
\pgfpathlineto{\pgfqpoint{2.091970in}{3.454262in}}%
\pgfpathlineto{\pgfqpoint{2.070370in}{3.464656in}}%
\pgfpathlineto{\pgfqpoint{2.052058in}{3.501730in}}%
\pgfpathlineto{\pgfqpoint{2.033745in}{3.617693in}}%
\pgfpathlineto{\pgfqpoint{2.014259in}{3.914311in}}%
\pgfpathlineto{\pgfqpoint{1.996182in}{4.168292in}}%
\pgfpathlineto{\pgfqpoint{1.974114in}{4.169481in}}%
\pgfpathlineto{\pgfqpoint{1.937254in}{3.613201in}}%
\pgfpathlineto{\pgfqpoint{1.918238in}{3.508751in}}%
\pgfpathlineto{\pgfqpoint{1.897343in}{3.467015in}}%
\pgfpathlineto{\pgfqpoint{1.878327in}{3.455132in}}%
\pgfpathlineto{\pgfqpoint{1.861422in}{3.457219in}}%
\pgfpathlineto{\pgfqpoint{1.843111in}{3.469313in}}%
\pgfpathlineto{\pgfqpoint{1.821042in}{3.512766in}}%
\pgfpathlineto{\pgfqpoint{1.802025in}{3.641011in}}%
\pgfpathlineto{\pgfqpoint{1.782774in}{3.927941in}}%
\pgfpathlineto{\pgfqpoint{1.764931in}{4.170728in}}%
\pgfpathlineto{\pgfqpoint{1.745915in}{4.236220in}}%
\pgfpathlineto{\pgfqpoint{1.727369in}{4.099642in}}%
\pgfpathlineto{\pgfqpoint{1.708116in}{3.798417in}}%
\pgfpathlineto{\pgfqpoint{1.686987in}{3.632946in}}%
\pgfpathlineto{\pgfqpoint{1.667970in}{3.514735in}}%
\pgfpathlineto{\pgfqpoint{1.649659in}{3.477528in}}%
\pgfpathlineto{\pgfqpoint{1.631111in}{3.458084in}}%
\pgfpathlineto{\pgfqpoint{1.606696in}{3.458443in}}%
\pgfpathlineto{\pgfqpoint{1.590731in}{3.470991in}}%
\pgfpathlineto{\pgfqpoint{1.574297in}{3.499553in}}%
\pgfpathlineto{\pgfqpoint{1.554575in}{3.585422in}}%
\pgfpathlineto{\pgfqpoint{1.535089in}{3.782815in}}%
\pgfpathlineto{\pgfqpoint{1.514195in}{4.124005in}}%
\pgfpathlineto{\pgfqpoint{1.495647in}{4.266808in}}%
\pgfpathlineto{\pgfqpoint{1.475692in}{4.157558in}}%
\pgfpathlineto{\pgfqpoint{1.455267in}{4.086611in}}%
\pgfpathlineto{\pgfqpoint{1.437190in}{3.759182in}}%
\pgfpathlineto{\pgfqpoint{1.418408in}{3.594678in}}%
\pgfpathlineto{\pgfqpoint{1.397748in}{3.506929in}}%
\pgfpathlineto{\pgfqpoint{1.378966in}{3.483872in}}%
\pgfpathlineto{\pgfqpoint{1.360889in}{3.482647in}}%
\pgfpathlineto{\pgfqpoint{1.342106in}{3.460681in}}%
\pgfpathlineto{\pgfqpoint{1.323795in}{3.457994in}}%
\pgfpathlineto{\pgfqpoint{1.299378in}{3.476273in}}%
\pgfpathlineto{\pgfqpoint{1.284353in}{3.510261in}}%
\pgfpathlineto{\pgfqpoint{1.265101in}{3.609589in}}%
\pgfpathlineto{\pgfqpoint{1.245379in}{3.538478in}}%
\pgfpathlineto{\pgfqpoint{1.228242in}{3.669614in}}%
\pgfpathlineto{\pgfqpoint{1.205703in}{4.023249in}}%
\pgfpathlineto{\pgfqpoint{1.187860in}{4.244131in}}%
\pgfpathlineto{\pgfqpoint{1.167671in}{4.285872in}}%
\pgfpathlineto{\pgfqpoint{1.150298in}{4.074572in}}%
\pgfpathlineto{\pgfqpoint{1.129169in}{3.710254in}}%
\pgfpathlineto{\pgfqpoint{1.107569in}{3.562674in}}%
\pgfpathlineto{\pgfqpoint{1.091839in}{3.506657in}}%
\pgfpathlineto{\pgfqpoint{1.073527in}{3.474976in}}%
\pgfpathlineto{\pgfqpoint{1.051693in}{3.458856in}}%
\pgfpathlineto{\pgfqpoint{1.035025in}{3.460069in}}%
\pgfpathlineto{\pgfqpoint{1.012485in}{3.483087in}}%
\pgfpathlineto{\pgfqpoint{0.994408in}{3.520781in}}%
\pgfpathlineto{\pgfqpoint{0.974688in}{3.457958in}}%
\pgfpathlineto{\pgfqpoint{0.958018in}{3.470995in}}%
\pgfpathlineto{\pgfqpoint{0.938298in}{3.510003in}}%
\pgfpathlineto{\pgfqpoint{0.919986in}{3.606397in}}%
\pgfpathlineto{\pgfqpoint{0.900970in}{3.825051in}}%
\pgfpathlineto{\pgfqpoint{0.879604in}{4.192752in}}%
\pgfpathlineto{\pgfqpoint{0.859884in}{4.317610in}}%
\pgfpathlineto{\pgfqpoint{0.841807in}{4.305639in}}%
\pgfpathlineto{\pgfqpoint{0.824668in}{4.024781in}}%
\pgfpathlineto{\pgfqpoint{0.801894in}{3.729107in}}%
\pgfpathlineto{\pgfqpoint{0.784052in}{3.571221in}}%
\pgfpathlineto{\pgfqpoint{0.765506in}{3.500573in}}%
\pgfpathlineto{\pgfqpoint{0.746489in}{3.471710in}}%
\pgfpathlineto{\pgfqpoint{0.723246in}{3.460475in}}%
\pgfpathlineto{\pgfqpoint{0.706107in}{3.466693in}}%
\pgfpathlineto{\pgfqpoint{0.687561in}{3.483750in}}%
\pgfpathlineto{\pgfqpoint{0.669248in}{3.537613in}}%
\pgfpathlineto{\pgfqpoint{0.650702in}{3.669178in}}%
\pgfpathlineto{\pgfqpoint{0.650468in}{3.481633in}}%
\pgfpathlineto{\pgfqpoint{0.656101in}{3.470864in}}%
\pgfpathlineto{\pgfqpoint{0.673474in}{3.461037in}}%
\pgfpathlineto{\pgfqpoint{0.696951in}{3.495963in}}%
\pgfpathlineto{\pgfqpoint{0.715028in}{3.596109in}}%
\pgfpathlineto{\pgfqpoint{0.731699in}{3.888976in}}%
\pgfpathlineto{\pgfqpoint{0.750715in}{4.324112in}}%
\pgfpathlineto{\pgfqpoint{0.769732in}{4.266313in}}%
\pgfpathlineto{\pgfqpoint{0.790156in}{3.860283in}}%
\pgfpathlineto{\pgfqpoint{0.810816in}{3.551609in}}%
\pgfpathlineto{\pgfqpoint{0.829598in}{3.481301in}}%
\pgfpathlineto{\pgfqpoint{0.849789in}{3.458625in}}%
\pgfpathlineto{\pgfqpoint{0.868100in}{3.469233in}}%
\pgfpathlineto{\pgfqpoint{0.888526in}{3.518537in}}%
\pgfpathlineto{\pgfqpoint{0.905430in}{3.689987in}}%
\pgfpathlineto{\pgfqpoint{0.925150in}{4.105156in}}%
\pgfpathlineto{\pgfqpoint{0.943933in}{4.307397in}}%
\pgfpathlineto{\pgfqpoint{0.963418in}{4.048394in}}%
\pgfpathlineto{\pgfqpoint{0.982670in}{3.650329in}}%
\pgfpathlineto{\pgfqpoint{1.002155in}{3.505786in}}%
\pgfpathlineto{\pgfqpoint{1.020938in}{3.464503in}}%
\pgfpathlineto{\pgfqpoint{1.040658in}{3.458361in}}%
\pgfpathlineto{\pgfqpoint{1.061083in}{3.486314in}}%
\pgfpathlineto{\pgfqpoint{1.080100in}{3.571712in}}%
\pgfpathlineto{\pgfqpoint{1.098179in}{3.816533in}}%
\pgfpathlineto{\pgfqpoint{1.116725in}{4.242178in}}%
\pgfpathlineto{\pgfqpoint{1.138793in}{4.183127in}}%
\pgfpathlineto{\pgfqpoint{1.155932in}{3.931701in}}%
\pgfpathlineto{\pgfqpoint{1.174009in}{3.622150in}}%
\pgfpathlineto{\pgfqpoint{1.193261in}{3.495572in}}%
\pgfpathlineto{\pgfqpoint{1.212043in}{3.461454in}}%
\pgfpathlineto{\pgfqpoint{1.231060in}{3.457564in}}%
\pgfpathlineto{\pgfqpoint{1.249137in}{3.475334in}}%
\pgfpathlineto{\pgfqpoint{1.269091in}{3.546249in}}%
\pgfpathlineto{\pgfqpoint{1.292100in}{3.801670in}}%
\pgfpathlineto{\pgfqpoint{1.309942in}{4.194046in}}%
\pgfpathlineto{\pgfqpoint{1.325673in}{4.250927in}}%
\pgfpathlineto{\pgfqpoint{1.347507in}{3.937192in}}%
\pgfpathlineto{\pgfqpoint{1.366053in}{3.631648in}}%
\pgfpathlineto{\pgfqpoint{1.387887in}{3.487981in}}%
\pgfpathlineto{\pgfqpoint{1.404321in}{3.462244in}}%
\pgfpathlineto{\pgfqpoint{1.423572in}{3.456044in}}%
\pgfpathlineto{\pgfqpoint{1.442823in}{3.471245in}}%
\pgfpathlineto{\pgfqpoint{1.463483in}{3.464824in}}%
\pgfpathlineto{\pgfqpoint{1.479448in}{3.490371in}}%
\pgfpathlineto{\pgfqpoint{1.501751in}{3.575083in}}%
\pgfpathlineto{\pgfqpoint{1.521002in}{3.845605in}}%
\pgfpathlineto{\pgfqpoint{1.539315in}{4.197172in}}%
\pgfpathlineto{\pgfqpoint{1.557627in}{4.206909in}}%
\pgfpathlineto{\pgfqpoint{1.601530in}{3.561457in}}%
\pgfpathlineto{\pgfqpoint{1.617260in}{3.500447in}}%
\pgfpathlineto{\pgfqpoint{1.633459in}{3.467000in}}%
\pgfpathlineto{\pgfqpoint{1.659283in}{3.455019in}}%
\pgfpathlineto{\pgfqpoint{1.674074in}{3.461177in}}%
\pgfpathlineto{\pgfqpoint{1.694970in}{3.502448in}}%
\pgfpathlineto{\pgfqpoint{1.713751in}{3.609587in}}%
\pgfpathlineto{\pgfqpoint{1.732533in}{3.861243in}}%
\pgfpathlineto{\pgfqpoint{1.751550in}{4.207023in}}%
\pgfpathlineto{\pgfqpoint{1.770095in}{4.151830in}}%
\pgfpathlineto{\pgfqpoint{1.790052in}{3.962174in}}%
\pgfpathlineto{\pgfqpoint{1.811181in}{3.597181in}}%
\pgfpathlineto{\pgfqpoint{1.828789in}{3.497325in}}%
\pgfpathlineto{\pgfqpoint{1.848040in}{3.469351in}}%
\pgfpathlineto{\pgfqpoint{1.865885in}{3.455015in}}%
\pgfpathlineto{\pgfqpoint{1.886544in}{3.457593in}}%
\pgfpathlineto{\pgfqpoint{1.903916in}{3.477059in}}%
\pgfpathlineto{\pgfqpoint{1.925281in}{3.552054in}}%
\pgfpathlineto{\pgfqpoint{1.943358in}{3.744913in}}%
\pgfpathlineto{\pgfqpoint{1.962375in}{4.117783in}}%
\pgfpathlineto{\pgfqpoint{1.983975in}{4.211412in}}%
\pgfpathlineto{\pgfqpoint{2.001817in}{4.158553in}}%
\pgfpathlineto{\pgfqpoint{2.021068in}{3.917361in}}%
\pgfpathlineto{\pgfqpoint{2.041023in}{3.593180in}}%
\pgfpathlineto{\pgfqpoint{2.058866in}{3.491666in}}%
\pgfpathlineto{\pgfqpoint{2.081640in}{3.459261in}}%
\pgfpathlineto{\pgfqpoint{2.098308in}{3.454129in}}%
\pgfpathlineto{\pgfqpoint{2.116150in}{3.461578in}}%
\pgfpathlineto{\pgfqpoint{2.136341in}{3.491703in}}%
\pgfpathlineto{\pgfqpoint{2.155592in}{3.570144in}}%
\pgfpathlineto{\pgfqpoint{2.173669in}{3.774740in}}%
\pgfpathlineto{\pgfqpoint{2.193157in}{4.156052in}}%
\pgfpathlineto{\pgfqpoint{2.210765in}{4.191777in}}%
\pgfpathlineto{\pgfqpoint{2.232128in}{4.014319in}}%
\pgfpathlineto{\pgfqpoint{2.249971in}{3.699595in}}%
\pgfpathlineto{\pgfqpoint{2.267579in}{3.531855in}}%
\pgfpathlineto{\pgfqpoint{2.288944in}{3.470636in}}%
\pgfpathlineto{\pgfqpoint{2.309604in}{3.464635in}}%
\pgfpathlineto{\pgfqpoint{2.327916in}{3.455534in}}%
\pgfpathlineto{\pgfqpoint{2.345524in}{3.464835in}}%
\pgfpathlineto{\pgfqpoint{2.366652in}{3.453590in}}%
\pgfpathlineto{\pgfqpoint{2.385904in}{3.458930in}}%
\pgfpathlineto{\pgfqpoint{2.406095in}{3.480052in}}%
\pgfpathlineto{\pgfqpoint{2.423703in}{3.531468in}}%
\pgfpathlineto{\pgfqpoint{2.448354in}{3.787780in}}%
\pgfpathlineto{\pgfqpoint{2.461736in}{4.075920in}}%
\pgfpathlineto{\pgfqpoint{2.480518in}{4.203967in}}%
\pgfpathlineto{\pgfqpoint{2.501882in}{4.112163in}}%
\pgfpathlineto{\pgfqpoint{2.519021in}{3.782983in}}%
\pgfpathlineto{\pgfqpoint{2.539681in}{3.555865in}}%
\pgfpathlineto{\pgfqpoint{2.557758in}{3.482518in}}%
\pgfpathlineto{\pgfqpoint{2.582175in}{3.456145in}}%
\pgfpathlineto{\pgfqpoint{2.597200in}{3.452868in}}%
\pgfpathlineto{\pgfqpoint{2.614573in}{3.462662in}}%
\pgfpathlineto{\pgfqpoint{2.635702in}{3.501025in}}%
\pgfpathlineto{\pgfqpoint{2.653545in}{3.594925in}}%
\pgfpathlineto{\pgfqpoint{2.677022in}{3.803930in}}%
\pgfpathlineto{\pgfqpoint{2.692753in}{4.115890in}}%
\pgfpathlineto{\pgfqpoint{2.713647in}{4.208418in}}%
\pgfpathlineto{\pgfqpoint{2.731958in}{4.055255in}}%
\pgfpathlineto{\pgfqpoint{2.749332in}{3.754211in}}%
\pgfpathlineto{\pgfqpoint{2.767409in}{3.599641in}}%
\pgfpathlineto{\pgfqpoint{2.787365in}{3.757070in}}%
\pgfpathlineto{\pgfqpoint{2.808025in}{4.185883in}}%
\pgfpathlineto{\pgfqpoint{2.829389in}{3.853677in}}%
\pgfpathlineto{\pgfqpoint{2.845588in}{3.624463in}}%
\pgfpathlineto{\pgfqpoint{2.868128in}{3.509009in}}%
\pgfpathlineto{\pgfqpoint{2.885736in}{3.478616in}}%
\pgfpathlineto{\pgfqpoint{2.905221in}{3.462249in}}%
\pgfpathlineto{\pgfqpoint{2.924472in}{3.455127in}}%
\pgfpathlineto{\pgfqpoint{2.942080in}{3.464961in}}%
\pgfpathlineto{\pgfqpoint{2.961801in}{3.513112in}}%
\pgfpathlineto{\pgfqpoint{2.981757in}{3.649385in}}%
\pgfpathlineto{\pgfqpoint{2.997956in}{3.820478in}}%
\pgfpathlineto{\pgfqpoint{3.021903in}{4.224478in}}%
\pgfpathlineto{\pgfqpoint{3.037399in}{4.210895in}}%
\pgfpathlineto{\pgfqpoint{3.059231in}{3.908661in}}%
\pgfpathlineto{\pgfqpoint{3.077075in}{3.632554in}}%
\pgfpathlineto{\pgfqpoint{3.097499in}{3.498677in}}%
\pgfpathlineto{\pgfqpoint{3.115578in}{3.465563in}}%
\pgfpathlineto{\pgfqpoint{3.134124in}{3.454925in}}%
\pgfpathlineto{\pgfqpoint{3.154315in}{3.462893in}}%
\pgfpathlineto{\pgfqpoint{3.172626in}{3.485005in}}%
\pgfpathlineto{\pgfqpoint{3.194226in}{3.561931in}}%
\pgfpathlineto{\pgfqpoint{3.213008in}{3.704696in}}%
\pgfpathlineto{\pgfqpoint{3.229911in}{4.062852in}}%
\pgfpathlineto{\pgfqpoint{3.251042in}{4.234400in}}%
\pgfpathlineto{\pgfqpoint{3.267004in}{4.216310in}}%
\pgfpathlineto{\pgfqpoint{3.285552in}{3.941596in}}%
\pgfpathlineto{\pgfqpoint{3.307621in}{3.643439in}}%
\pgfpathlineto{\pgfqpoint{3.325698in}{3.518069in}}%
\pgfpathlineto{\pgfqpoint{3.343540in}{3.472435in}}%
\pgfpathlineto{\pgfqpoint{3.368192in}{3.456731in}}%
\pgfpathlineto{\pgfqpoint{3.385566in}{3.458330in}}%
\pgfpathlineto{\pgfqpoint{3.404111in}{3.468278in}}%
\pgfpathlineto{\pgfqpoint{3.422425in}{3.496098in}}%
\pgfpathlineto{\pgfqpoint{3.443319in}{3.572901in}}%
\pgfpathlineto{\pgfqpoint{3.463745in}{3.736410in}}%
\pgfpathlineto{\pgfqpoint{3.502247in}{4.265099in}}%
\pgfpathlineto{\pgfqpoint{3.520795in}{4.197343in}}%
\pgfpathlineto{\pgfqpoint{3.539575in}{4.039578in}}%
\pgfpathlineto{\pgfqpoint{3.556480in}{3.751339in}}%
\pgfpathlineto{\pgfqpoint{3.576905in}{3.547586in}}%
\pgfpathlineto{\pgfqpoint{3.596157in}{3.484971in}}%
\pgfpathlineto{\pgfqpoint{3.616111in}{3.463939in}}%
\pgfpathlineto{\pgfqpoint{3.636302in}{3.456137in}}%
\pgfpathlineto{\pgfqpoint{3.655319in}{3.460182in}}%
\pgfpathlineto{\pgfqpoint{3.672458in}{3.478484in}}%
\pgfpathlineto{\pgfqpoint{3.690064in}{3.516738in}}%
\pgfpathlineto{\pgfqpoint{3.710726in}{3.614511in}}%
\pgfpathlineto{\pgfqpoint{3.731855in}{3.838425in}}%
\pgfpathlineto{\pgfqpoint{3.750872in}{4.160263in}}%
\pgfpathlineto{\pgfqpoint{3.770123in}{4.301815in}}%
\pgfpathlineto{\pgfqpoint{3.788200in}{4.213249in}}%
\pgfpathlineto{\pgfqpoint{3.805808in}{3.990624in}}%
\pgfpathlineto{\pgfqpoint{3.826233in}{3.689761in}}%
\pgfpathlineto{\pgfqpoint{3.846188in}{3.587711in}}%
\pgfpathlineto{\pgfqpoint{3.863092in}{3.515872in}}%
\pgfpathlineto{\pgfqpoint{3.884692in}{3.469526in}}%
\pgfpathlineto{\pgfqpoint{3.908638in}{3.457475in}}%
\pgfpathlineto{\pgfqpoint{3.924132in}{3.460327in}}%
\pgfpathlineto{\pgfqpoint{3.941740in}{3.473531in}}%
\pgfpathlineto{\pgfqpoint{3.962400in}{3.659215in}}%
\pgfpathlineto{\pgfqpoint{3.979774in}{3.519860in}}%
\pgfpathlineto{\pgfqpoint{4.001843in}{3.471409in}}%
\pgfpathlineto{\pgfqpoint{4.022503in}{3.458416in}}%
\pgfpathlineto{\pgfqpoint{4.037528in}{3.462576in}}%
\pgfpathlineto{\pgfqpoint{4.059127in}{3.486194in}}%
\pgfpathlineto{\pgfqpoint{4.077204in}{3.532098in}}%
\pgfpathlineto{\pgfqpoint{4.094578in}{3.637710in}}%
\pgfpathlineto{\pgfqpoint{4.119229in}{4.003633in}}%
\pgfpathlineto{\pgfqpoint{4.133551in}{4.269580in}}%
\pgfpathlineto{\pgfqpoint{4.153740in}{4.349128in}}%
\pgfpathlineto{\pgfqpoint{4.171583in}{4.342894in}}%
\pgfpathlineto{\pgfqpoint{4.196000in}{4.056721in}}%
\pgfpathlineto{\pgfqpoint{4.211025in}{3.762044in}}%
\pgfpathlineto{\pgfqpoint{4.231919in}{3.556941in}}%
\pgfpathlineto{\pgfqpoint{4.250936in}{3.490641in}}%
\pgfpathlineto{\pgfqpoint{4.269484in}{3.466986in}}%
\pgfpathlineto{\pgfqpoint{4.290144in}{3.461399in}}%
\pgfpathlineto{\pgfqpoint{4.306812in}{3.475096in}}%
\pgfpathlineto{\pgfqpoint{4.327237in}{3.519919in}}%
\pgfpathlineto{\pgfqpoint{4.346958in}{3.598620in}}%
\pgfpathlineto{\pgfqpoint{4.366914in}{3.841848in}}%
\pgfpathlineto{\pgfqpoint{4.385696in}{4.091768in}}%
\pgfpathlineto{\pgfqpoint{4.403304in}{4.380193in}}%
\pgfpathlineto{\pgfqpoint{4.419267in}{4.375449in}}%
\pgfpathlineto{\pgfqpoint{4.439693in}{4.179910in}}%
\pgfpathlineto{\pgfqpoint{4.461996in}{3.794050in}}%
\pgfpathlineto{\pgfqpoint{4.479135in}{3.599636in}}%
\pgfpathlineto{\pgfqpoint{4.475849in}{3.637997in}}%
\pgfpathlineto{\pgfqpoint{4.454249in}{4.079984in}}%
\pgfpathlineto{\pgfqpoint{4.435467in}{4.375168in}}%
\pgfpathlineto{\pgfqpoint{4.417859in}{4.281032in}}%
\pgfpathlineto{\pgfqpoint{4.398844in}{3.802845in}}%
\pgfpathlineto{\pgfqpoint{4.377010in}{3.549971in}}%
\pgfpathlineto{\pgfqpoint{4.358696in}{3.484336in}}%
\pgfpathlineto{\pgfqpoint{4.341325in}{3.460334in}}%
\pgfpathlineto{\pgfqpoint{4.319020in}{3.476461in}}%
\pgfpathlineto{\pgfqpoint{4.301177in}{3.537108in}}%
\pgfpathlineto{\pgfqpoint{4.283335in}{3.737188in}}%
\pgfpathlineto{\pgfqpoint{4.262909in}{4.192011in}}%
\pgfpathlineto{\pgfqpoint{4.246006in}{4.355745in}}%
\pgfpathlineto{\pgfqpoint{4.225347in}{4.071024in}}%
\pgfpathlineto{\pgfqpoint{4.205624in}{3.625451in}}%
\pgfpathlineto{\pgfqpoint{4.187079in}{3.506242in}}%
\pgfpathlineto{\pgfqpoint{4.166653in}{3.464610in}}%
\pgfpathlineto{\pgfqpoint{4.167357in}{3.459262in}}%
\pgfpathlineto{\pgfqpoint{4.149514in}{3.459862in}}%
\pgfpathlineto{\pgfqpoint{4.129089in}{3.494063in}}%
\pgfpathlineto{\pgfqpoint{4.110777in}{3.587264in}}%
\pgfpathlineto{\pgfqpoint{4.072275in}{4.273037in}}%
\pgfpathlineto{\pgfqpoint{4.051146in}{4.226568in}}%
\pgfpathlineto{\pgfqpoint{4.031424in}{3.724242in}}%
\pgfpathlineto{\pgfqpoint{4.013347in}{3.542499in}}%
\pgfpathlineto{\pgfqpoint{3.993861in}{3.474498in}}%
\pgfpathlineto{\pgfqpoint{3.974139in}{3.457385in}}%
\pgfpathlineto{\pgfqpoint{3.956297in}{3.467661in}}%
\pgfpathlineto{\pgfqpoint{3.935402in}{3.525947in}}%
\pgfpathlineto{\pgfqpoint{3.918500in}{3.699003in}}%
\pgfpathlineto{\pgfqpoint{3.899012in}{4.132187in}}%
\pgfpathlineto{\pgfqpoint{3.878352in}{4.288099in}}%
\pgfpathlineto{\pgfqpoint{3.838675in}{3.614122in}}%
\pgfpathlineto{\pgfqpoint{3.821067in}{3.501402in}}%
\pgfpathlineto{\pgfqpoint{3.803696in}{3.468319in}}%
\pgfpathlineto{\pgfqpoint{3.783036in}{3.457229in}}%
\pgfpathlineto{\pgfqpoint{3.762845in}{3.473640in}}%
\pgfpathlineto{\pgfqpoint{3.747349in}{3.556194in}}%
\pgfpathlineto{\pgfqpoint{3.725280in}{3.724510in}}%
\pgfpathlineto{\pgfqpoint{3.708377in}{4.090879in}}%
\pgfpathlineto{\pgfqpoint{3.686778in}{4.174591in}}%
\pgfpathlineto{\pgfqpoint{3.669170in}{4.257913in}}%
\pgfpathlineto{\pgfqpoint{3.647806in}{3.862755in}}%
\pgfpathlineto{\pgfqpoint{3.627381in}{3.568609in}}%
\pgfpathlineto{\pgfqpoint{3.609069in}{3.486107in}}%
\pgfpathlineto{\pgfqpoint{3.588173in}{3.462942in}}%
\pgfpathlineto{\pgfqpoint{3.572445in}{3.456000in}}%
\pgfpathlineto{\pgfqpoint{3.550142in}{3.469900in}}%
\pgfpathlineto{\pgfqpoint{3.531828in}{3.510451in}}%
\pgfpathlineto{\pgfqpoint{3.514689in}{3.634640in}}%
\pgfpathlineto{\pgfqpoint{3.472900in}{4.209619in}}%
\pgfpathlineto{\pgfqpoint{3.455763in}{4.218549in}}%
\pgfpathlineto{\pgfqpoint{3.437919in}{3.901037in}}%
\pgfpathlineto{\pgfqpoint{3.418668in}{3.596182in}}%
\pgfpathlineto{\pgfqpoint{3.393547in}{3.484546in}}%
\pgfpathlineto{\pgfqpoint{3.377819in}{3.462394in}}%
\pgfpathlineto{\pgfqpoint{3.360211in}{3.455263in}}%
\pgfpathlineto{\pgfqpoint{3.342368in}{3.460693in}}%
\pgfpathlineto{\pgfqpoint{3.320534in}{3.492686in}}%
\pgfpathlineto{\pgfqpoint{3.304335in}{3.565708in}}%
\pgfpathlineto{\pgfqpoint{3.303395in}{3.730693in}}%
\pgfpathlineto{\pgfqpoint{3.282266in}{3.615854in}}%
\pgfpathlineto{\pgfqpoint{3.264187in}{3.482118in}}%
\pgfpathlineto{\pgfqpoint{3.241415in}{3.595610in}}%
\pgfpathlineto{\pgfqpoint{3.222164in}{3.897811in}}%
\pgfpathlineto{\pgfqpoint{3.205025in}{4.187892in}}%
\pgfpathlineto{\pgfqpoint{3.186713in}{4.205542in}}%
\pgfpathlineto{\pgfqpoint{3.168165in}{3.825160in}}%
\pgfpathlineto{\pgfqpoint{3.147037in}{3.559683in}}%
\pgfpathlineto{\pgfqpoint{3.131777in}{3.499057in}}%
\pgfpathlineto{\pgfqpoint{3.108769in}{3.465638in}}%
\pgfpathlineto{\pgfqpoint{3.090926in}{3.454264in}}%
\pgfpathlineto{\pgfqpoint{3.070501in}{3.467221in}}%
\pgfpathlineto{\pgfqpoint{3.054301in}{3.485212in}}%
\pgfpathlineto{\pgfqpoint{3.032233in}{3.577442in}}%
\pgfpathlineto{\pgfqpoint{3.014625in}{3.800448in}}%
\pgfpathlineto{\pgfqpoint{2.993965in}{4.150668in}}%
\pgfpathlineto{\pgfqpoint{2.975888in}{4.211991in}}%
\pgfpathlineto{\pgfqpoint{2.954523in}{3.880689in}}%
\pgfpathlineto{\pgfqpoint{2.935977in}{3.675575in}}%
\pgfpathlineto{\pgfqpoint{2.918603in}{3.532270in}}%
\pgfpathlineto{\pgfqpoint{2.897474in}{3.471013in}}%
\pgfpathlineto{\pgfqpoint{2.878692in}{3.455141in}}%
\pgfpathlineto{\pgfqpoint{2.857327in}{3.457380in}}%
\pgfpathlineto{\pgfqpoint{2.839015in}{3.478119in}}%
\pgfpathlineto{\pgfqpoint{2.821407in}{3.532074in}}%
\pgfpathlineto{\pgfqpoint{2.801922in}{3.704094in}}%
\pgfpathlineto{\pgfqpoint{2.783139in}{4.006201in}}%
\pgfpathlineto{\pgfqpoint{2.763888in}{4.210067in}}%
\pgfpathlineto{\pgfqpoint{2.745811in}{4.138789in}}%
\pgfpathlineto{\pgfqpoint{2.726324in}{3.805198in}}%
\pgfpathlineto{\pgfqpoint{2.704726in}{3.617834in}}%
\pgfpathlineto{\pgfqpoint{2.685709in}{3.505051in}}%
\pgfpathlineto{\pgfqpoint{2.667161in}{3.466911in}}%
\pgfpathlineto{\pgfqpoint{2.648850in}{3.453390in}}%
\pgfpathlineto{\pgfqpoint{2.630067in}{3.455363in}}%
\pgfpathlineto{\pgfqpoint{2.611051in}{3.469213in}}%
\pgfpathlineto{\pgfqpoint{2.590156in}{3.487825in}}%
\pgfpathlineto{\pgfqpoint{2.570671in}{3.575451in}}%
\pgfpathlineto{\pgfqpoint{2.530525in}{4.135220in}}%
\pgfpathlineto{\pgfqpoint{2.514560in}{4.200481in}}%
\pgfpathlineto{\pgfqpoint{2.493195in}{4.067919in}}%
\pgfpathlineto{\pgfqpoint{2.473944in}{3.713494in}}%
\pgfpathlineto{\pgfqpoint{2.455398in}{3.810602in}}%
\pgfpathlineto{\pgfqpoint{2.437085in}{3.578503in}}%
\pgfpathlineto{\pgfqpoint{2.415487in}{3.484466in}}%
\pgfpathlineto{\pgfqpoint{2.396704in}{3.458320in}}%
\pgfpathlineto{\pgfqpoint{2.380740in}{3.453384in}}%
\pgfpathlineto{\pgfqpoint{2.359374in}{3.456828in}}%
\pgfpathlineto{\pgfqpoint{2.340828in}{3.473874in}}%
\pgfpathlineto{\pgfqpoint{2.322046in}{3.534991in}}%
\pgfpathlineto{\pgfqpoint{2.302795in}{3.705450in}}%
\pgfpathlineto{\pgfqpoint{2.284013in}{4.072311in}}%
\pgfpathlineto{\pgfqpoint{2.265701in}{4.205486in}}%
\pgfpathlineto{\pgfqpoint{2.243398in}{4.023483in}}%
\pgfpathlineto{\pgfqpoint{2.228373in}{3.692447in}}%
\pgfpathlineto{\pgfqpoint{2.206773in}{3.538872in}}%
\pgfpathlineto{\pgfqpoint{2.182122in}{3.476727in}}%
\pgfpathlineto{\pgfqpoint{2.169445in}{3.461442in}}%
\pgfpathlineto{\pgfqpoint{2.150428in}{3.452617in}}%
\pgfpathlineto{\pgfqpoint{2.130237in}{3.456882in}}%
\pgfpathlineto{\pgfqpoint{2.107700in}{3.482386in}}%
\pgfpathlineto{\pgfqpoint{2.093613in}{3.523531in}}%
\pgfpathlineto{\pgfqpoint{2.070841in}{3.705092in}}%
\pgfpathlineto{\pgfqpoint{2.051119in}{4.038819in}}%
\pgfpathlineto{\pgfqpoint{2.032573in}{4.197753in}}%
\pgfpathlineto{\pgfqpoint{2.013556in}{4.117687in}}%
\pgfpathlineto{\pgfqpoint{1.993365in}{3.985588in}}%
\pgfpathlineto{\pgfqpoint{1.974348in}{3.704793in}}%
\pgfpathlineto{\pgfqpoint{1.958383in}{3.566329in}}%
\pgfpathlineto{\pgfqpoint{1.936080in}{3.482560in}}%
\pgfpathlineto{\pgfqpoint{1.917298in}{3.460739in}}%
\pgfpathlineto{\pgfqpoint{1.900864in}{3.453389in}}%
\pgfpathlineto{\pgfqpoint{1.879970in}{3.459054in}}%
\pgfpathlineto{\pgfqpoint{1.860015in}{3.479241in}}%
\pgfpathlineto{\pgfqpoint{1.841702in}{3.509026in}}%
\pgfpathlineto{\pgfqpoint{1.819868in}{3.511305in}}%
\pgfpathlineto{\pgfqpoint{1.800851in}{3.629239in}}%
\pgfpathlineto{\pgfqpoint{1.781834in}{3.934567in}}%
\pgfpathlineto{\pgfqpoint{1.763288in}{4.181225in}}%
\pgfpathlineto{\pgfqpoint{1.744037in}{4.221587in}}%
\pgfpathlineto{\pgfqpoint{1.725255in}{3.953405in}}%
\pgfpathlineto{\pgfqpoint{1.704126in}{3.674493in}}%
\pgfpathlineto{\pgfqpoint{1.686047in}{3.544809in}}%
\pgfpathlineto{\pgfqpoint{1.670553in}{3.491819in}}%
\pgfpathlineto{\pgfqpoint{1.649659in}{3.463336in}}%
\pgfpathlineto{\pgfqpoint{1.628999in}{3.456321in}}%
\pgfpathlineto{\pgfqpoint{1.611391in}{3.454325in}}%
\pgfpathlineto{\pgfqpoint{1.592843in}{3.464356in}}%
\pgfpathlineto{\pgfqpoint{1.574061in}{3.486473in}}%
\pgfpathlineto{\pgfqpoint{1.574061in}{3.486473in}}%
\pgfusepath{stroke}%
\end{pgfscope}%
\begin{pgfscope}%
\pgfpathrectangle{\pgfqpoint{0.444748in}{3.403703in}}{\pgfqpoint{4.231419in}{1.076123in}}%
\pgfusepath{clip}%
\pgfsetbuttcap%
\pgfsetroundjoin%
\definecolor{currentfill}{rgb}{0.047059,0.364706,0.647059}%
\pgfsetfillcolor{currentfill}%
\pgfsetlinewidth{1.003750pt}%
\definecolor{currentstroke}{rgb}{0.047059,0.364706,0.647059}%
\pgfsetstrokecolor{currentstroke}%
\pgfsetdash{}{0pt}%
\pgfsys@defobject{currentmarker}{\pgfqpoint{-0.010417in}{-0.010417in}}{\pgfqpoint{0.010417in}{0.010417in}}{%
\pgfpathmoveto{\pgfqpoint{0.000000in}{-0.010417in}}%
\pgfpathcurveto{\pgfqpoint{0.002763in}{-0.010417in}}{\pgfqpoint{0.005412in}{-0.009319in}}{\pgfqpoint{0.007366in}{-0.007366in}}%
\pgfpathcurveto{\pgfqpoint{0.009319in}{-0.005412in}}{\pgfqpoint{0.010417in}{-0.002763in}}{\pgfqpoint{0.010417in}{0.000000in}}%
\pgfpathcurveto{\pgfqpoint{0.010417in}{0.002763in}}{\pgfqpoint{0.009319in}{0.005412in}}{\pgfqpoint{0.007366in}{0.007366in}}%
\pgfpathcurveto{\pgfqpoint{0.005412in}{0.009319in}}{\pgfqpoint{0.002763in}{0.010417in}}{\pgfqpoint{0.000000in}{0.010417in}}%
\pgfpathcurveto{\pgfqpoint{-0.002763in}{0.010417in}}{\pgfqpoint{-0.005412in}{0.009319in}}{\pgfqpoint{-0.007366in}{0.007366in}}%
\pgfpathcurveto{\pgfqpoint{-0.009319in}{0.005412in}}{\pgfqpoint{-0.010417in}{0.002763in}}{\pgfqpoint{-0.010417in}{0.000000in}}%
\pgfpathcurveto{\pgfqpoint{-0.010417in}{-0.002763in}}{\pgfqpoint{-0.009319in}{-0.005412in}}{\pgfqpoint{-0.007366in}{-0.007366in}}%
\pgfpathcurveto{\pgfqpoint{-0.005412in}{-0.009319in}}{\pgfqpoint{-0.002763in}{-0.010417in}}{\pgfqpoint{0.000000in}{-0.010417in}}%
\pgfpathlineto{\pgfqpoint{0.000000in}{-0.010417in}}%
\pgfpathclose%
\pgfusepath{stroke,fill}%
}%
\begin{pgfscope}%
\pgfsys@transformshift{0.637086in}{3.959921in}%
\pgfsys@useobject{currentmarker}{}%
\end{pgfscope}%
\begin{pgfscope}%
\pgfsys@transformshift{0.656101in}{3.613686in}%
\pgfsys@useobject{currentmarker}{}%
\end{pgfscope}%
\begin{pgfscope}%
\pgfsys@transformshift{0.676526in}{3.504229in}%
\pgfsys@useobject{currentmarker}{}%
\end{pgfscope}%
\begin{pgfscope}%
\pgfsys@transformshift{0.693665in}{3.472149in}%
\pgfsys@useobject{currentmarker}{}%
\end{pgfscope}%
\begin{pgfscope}%
\pgfsys@transformshift{0.714559in}{3.554874in}%
\pgfsys@useobject{currentmarker}{}%
\end{pgfscope}%
\begin{pgfscope}%
\pgfsys@transformshift{0.731699in}{3.738341in}%
\pgfsys@useobject{currentmarker}{}%
\end{pgfscope}%
\begin{pgfscope}%
\pgfsys@transformshift{0.753533in}{4.247934in}%
\pgfsys@useobject{currentmarker}{}%
\end{pgfscope}%
\begin{pgfscope}%
\pgfsys@transformshift{0.771375in}{4.363963in}%
\pgfsys@useobject{currentmarker}{}%
\end{pgfscope}%
\begin{pgfscope}%
\pgfsys@transformshift{0.790626in}{4.047141in}%
\pgfsys@useobject{currentmarker}{}%
\end{pgfscope}%
\begin{pgfscope}%
\pgfsys@transformshift{0.808234in}{3.663840in}%
\pgfsys@useobject{currentmarker}{}%
\end{pgfscope}%
\begin{pgfscope}%
\pgfsys@transformshift{0.828424in}{3.516453in}%
\pgfsys@useobject{currentmarker}{}%
\end{pgfscope}%
\begin{pgfscope}%
\pgfsys@transformshift{0.849083in}{3.465506in}%
\pgfsys@useobject{currentmarker}{}%
\end{pgfscope}%
\begin{pgfscope}%
\pgfsys@transformshift{0.868100in}{3.467743in}%
\pgfsys@useobject{currentmarker}{}%
\end{pgfscope}%
\begin{pgfscope}%
\pgfsys@transformshift{0.887117in}{3.514401in}%
\pgfsys@useobject{currentmarker}{}%
\end{pgfscope}%
\begin{pgfscope}%
\pgfsys@transformshift{0.905196in}{3.646135in}%
\pgfsys@useobject{currentmarker}{}%
\end{pgfscope}%
\begin{pgfscope}%
\pgfsys@transformshift{0.925150in}{4.041211in}%
\pgfsys@useobject{currentmarker}{}%
\end{pgfscope}%
\begin{pgfscope}%
\pgfsys@transformshift{0.943227in}{4.343342in}%
\pgfsys@useobject{currentmarker}{}%
\end{pgfscope}%
\begin{pgfscope}%
\pgfsys@transformshift{0.964358in}{4.166315in}%
\pgfsys@useobject{currentmarker}{}%
\end{pgfscope}%
\begin{pgfscope}%
\pgfsys@transformshift{0.984078in}{3.731858in}%
\pgfsys@useobject{currentmarker}{}%
\end{pgfscope}%
\begin{pgfscope}%
\pgfsys@transformshift{1.000747in}{3.549029in}%
\pgfsys@useobject{currentmarker}{}%
\end{pgfscope}%
\begin{pgfscope}%
\pgfsys@transformshift{1.020469in}{3.472392in}%
\pgfsys@useobject{currentmarker}{}%
\end{pgfscope}%
\begin{pgfscope}%
\pgfsys@transformshift{1.041129in}{3.458605in}%
\pgfsys@useobject{currentmarker}{}%
\end{pgfscope}%
\begin{pgfscope}%
\pgfsys@transformshift{1.060380in}{3.483286in}%
\pgfsys@useobject{currentmarker}{}%
\end{pgfscope}%
\begin{pgfscope}%
\pgfsys@transformshift{1.079162in}{3.553907in}%
\pgfsys@useobject{currentmarker}{}%
\end{pgfscope}%
\begin{pgfscope}%
\pgfsys@transformshift{1.095361in}{3.601151in}%
\pgfsys@useobject{currentmarker}{}%
\end{pgfscope}%
\begin{pgfscope}%
\pgfsys@transformshift{1.118368in}{3.910242in}%
\pgfsys@useobject{currentmarker}{}%
\end{pgfscope}%
\begin{pgfscope}%
\pgfsys@transformshift{1.136210in}{4.286363in}%
\pgfsys@useobject{currentmarker}{}%
\end{pgfscope}%
\begin{pgfscope}%
\pgfsys@transformshift{1.155932in}{4.188654in}%
\pgfsys@useobject{currentmarker}{}%
\end{pgfscope}%
\begin{pgfscope}%
\pgfsys@transformshift{1.174713in}{3.811895in}%
\pgfsys@useobject{currentmarker}{}%
\end{pgfscope}%
\begin{pgfscope}%
\pgfsys@transformshift{1.193495in}{3.561928in}%
\pgfsys@useobject{currentmarker}{}%
\end{pgfscope}%
\begin{pgfscope}%
\pgfsys@transformshift{1.212512in}{3.478314in}%
\pgfsys@useobject{currentmarker}{}%
\end{pgfscope}%
\begin{pgfscope}%
\pgfsys@transformshift{1.234580in}{3.457777in}%
\pgfsys@useobject{currentmarker}{}%
\end{pgfscope}%
\begin{pgfscope}%
\pgfsys@transformshift{1.250545in}{3.467398in}%
\pgfsys@useobject{currentmarker}{}%
\end{pgfscope}%
\begin{pgfscope}%
\pgfsys@transformshift{1.269562in}{3.511427in}%
\pgfsys@useobject{currentmarker}{}%
\end{pgfscope}%
\begin{pgfscope}%
\pgfsys@transformshift{1.292334in}{3.688503in}%
\pgfsys@useobject{currentmarker}{}%
\end{pgfscope}%
\begin{pgfscope}%
\pgfsys@transformshift{1.311116in}{4.060932in}%
\pgfsys@useobject{currentmarker}{}%
\end{pgfscope}%
\begin{pgfscope}%
\pgfsys@transformshift{1.329428in}{4.279277in}%
\pgfsys@useobject{currentmarker}{}%
\end{pgfscope}%
\begin{pgfscope}%
\pgfsys@transformshift{1.349384in}{4.125644in}%
\pgfsys@useobject{currentmarker}{}%
\end{pgfscope}%
\begin{pgfscope}%
\pgfsys@transformshift{1.368167in}{3.787904in}%
\pgfsys@useobject{currentmarker}{}%
\end{pgfscope}%
\begin{pgfscope}%
\pgfsys@transformshift{1.386949in}{3.554141in}%
\pgfsys@useobject{currentmarker}{}%
\end{pgfscope}%
\begin{pgfscope}%
\pgfsys@transformshift{1.405729in}{3.479316in}%
\pgfsys@useobject{currentmarker}{}%
\end{pgfscope}%
\begin{pgfscope}%
\pgfsys@transformshift{1.424512in}{3.460337in}%
\pgfsys@useobject{currentmarker}{}%
\end{pgfscope}%
\begin{pgfscope}%
\pgfsys@transformshift{1.446346in}{3.466192in}%
\pgfsys@useobject{currentmarker}{}%
\end{pgfscope}%
\begin{pgfscope}%
\pgfsys@transformshift{1.463248in}{3.495873in}%
\pgfsys@useobject{currentmarker}{}%
\end{pgfscope}%
\begin{pgfscope}%
\pgfsys@transformshift{1.482265in}{3.586479in}%
\pgfsys@useobject{currentmarker}{}%
\end{pgfscope}%
\begin{pgfscope}%
\pgfsys@transformshift{1.500342in}{3.826334in}%
\pgfsys@useobject{currentmarker}{}%
\end{pgfscope}%
\begin{pgfscope}%
\pgfsys@transformshift{1.522882in}{4.267717in}%
\pgfsys@useobject{currentmarker}{}%
\end{pgfscope}%
\begin{pgfscope}%
\pgfsys@transformshift{1.538141in}{4.229920in}%
\pgfsys@useobject{currentmarker}{}%
\end{pgfscope}%
\begin{pgfscope}%
\pgfsys@transformshift{1.563496in}{3.873773in}%
\pgfsys@useobject{currentmarker}{}%
\end{pgfscope}%
\begin{pgfscope}%
\pgfsys@transformshift{1.580401in}{3.623428in}%
\pgfsys@useobject{currentmarker}{}%
\end{pgfscope}%
\begin{pgfscope}%
\pgfsys@transformshift{1.596129in}{3.513253in}%
\pgfsys@useobject{currentmarker}{}%
\end{pgfscope}%
\begin{pgfscope}%
\pgfsys@transformshift{1.615382in}{3.468195in}%
\pgfsys@useobject{currentmarker}{}%
\end{pgfscope}%
\begin{pgfscope}%
\pgfsys@transformshift{1.638623in}{3.456536in}%
\pgfsys@useobject{currentmarker}{}%
\end{pgfscope}%
\begin{pgfscope}%
\pgfsys@transformshift{1.654119in}{3.466428in}%
\pgfsys@useobject{currentmarker}{}%
\end{pgfscope}%
\begin{pgfscope}%
\pgfsys@transformshift{1.676188in}{3.515054in}%
\pgfsys@useobject{currentmarker}{}%
\end{pgfscope}%
\begin{pgfscope}%
\pgfsys@transformshift{1.695674in}{3.628517in}%
\pgfsys@useobject{currentmarker}{}%
\end{pgfscope}%
\begin{pgfscope}%
\pgfsys@transformshift{1.713282in}{3.950676in}%
\pgfsys@useobject{currentmarker}{}%
\end{pgfscope}%
\begin{pgfscope}%
\pgfsys@transformshift{1.733236in}{4.201583in}%
\pgfsys@useobject{currentmarker}{}%
\end{pgfscope}%
\begin{pgfscope}%
\pgfsys@transformshift{1.749436in}{4.262891in}%
\pgfsys@useobject{currentmarker}{}%
\end{pgfscope}%
\begin{pgfscope}%
\pgfsys@transformshift{1.770332in}{4.054820in}%
\pgfsys@useobject{currentmarker}{}%
\end{pgfscope}%
\begin{pgfscope}%
\pgfsys@transformshift{1.792164in}{3.650282in}%
\pgfsys@useobject{currentmarker}{}%
\end{pgfscope}%
\begin{pgfscope}%
\pgfsys@transformshift{1.809069in}{3.523851in}%
\pgfsys@useobject{currentmarker}{}%
\end{pgfscope}%
\begin{pgfscope}%
\pgfsys@transformshift{1.825503in}{3.485140in}%
\pgfsys@useobject{currentmarker}{}%
\end{pgfscope}%
\begin{pgfscope}%
\pgfsys@transformshift{1.825971in}{3.464246in}%
\pgfsys@useobject{currentmarker}{}%
\end{pgfscope}%
\begin{pgfscope}%
\pgfsys@transformshift{1.848040in}{3.456684in}%
\pgfsys@useobject{currentmarker}{}%
\end{pgfscope}%
\begin{pgfscope}%
\pgfsys@transformshift{1.867997in}{3.458472in}%
\pgfsys@useobject{currentmarker}{}%
\end{pgfscope}%
\begin{pgfscope}%
\pgfsys@transformshift{1.884899in}{3.481943in}%
\pgfsys@useobject{currentmarker}{}%
\end{pgfscope}%
\begin{pgfscope}%
\pgfsys@transformshift{1.906734in}{3.509365in}%
\pgfsys@useobject{currentmarker}{}%
\end{pgfscope}%
\begin{pgfscope}%
\pgfsys@transformshift{1.925985in}{3.647399in}%
\pgfsys@useobject{currentmarker}{}%
\end{pgfscope}%
\begin{pgfscope}%
\pgfsys@transformshift{1.945941in}{3.907107in}%
\pgfsys@useobject{currentmarker}{}%
\end{pgfscope}%
\begin{pgfscope}%
\pgfsys@transformshift{1.964724in}{4.230334in}%
\pgfsys@useobject{currentmarker}{}%
\end{pgfscope}%
\begin{pgfscope}%
\pgfsys@transformshift{1.986087in}{4.193742in}%
\pgfsys@useobject{currentmarker}{}%
\end{pgfscope}%
\begin{pgfscope}%
\pgfsys@transformshift{2.000877in}{3.956923in}%
\pgfsys@useobject{currentmarker}{}%
\end{pgfscope}%
\begin{pgfscope}%
\pgfsys@transformshift{2.022243in}{3.696507in}%
\pgfsys@useobject{currentmarker}{}%
\end{pgfscope}%
\begin{pgfscope}%
\pgfsys@transformshift{2.039145in}{3.526818in}%
\pgfsys@useobject{currentmarker}{}%
\end{pgfscope}%
\begin{pgfscope}%
\pgfsys@transformshift{2.060745in}{3.468579in}%
\pgfsys@useobject{currentmarker}{}%
\end{pgfscope}%
\begin{pgfscope}%
\pgfsys@transformshift{2.076710in}{3.456571in}%
\pgfsys@useobject{currentmarker}{}%
\end{pgfscope}%
\begin{pgfscope}%
\pgfsys@transformshift{2.098073in}{3.460808in}%
\pgfsys@useobject{currentmarker}{}%
\end{pgfscope}%
\begin{pgfscope}%
\pgfsys@transformshift{2.114273in}{3.464037in}%
\pgfsys@useobject{currentmarker}{}%
\end{pgfscope}%
\begin{pgfscope}%
\pgfsys@transformshift{2.138690in}{3.505832in}%
\pgfsys@useobject{currentmarker}{}%
\end{pgfscope}%
\begin{pgfscope}%
\pgfsys@transformshift{2.153715in}{3.591088in}%
\pgfsys@useobject{currentmarker}{}%
\end{pgfscope}%
\begin{pgfscope}%
\pgfsys@transformshift{2.174375in}{3.881646in}%
\pgfsys@useobject{currentmarker}{}%
\end{pgfscope}%
\begin{pgfscope}%
\pgfsys@transformshift{2.192452in}{4.189895in}%
\pgfsys@useobject{currentmarker}{}%
\end{pgfscope}%
\begin{pgfscope}%
\pgfsys@transformshift{2.213346in}{4.245687in}%
\pgfsys@useobject{currentmarker}{}%
\end{pgfscope}%
\begin{pgfscope}%
\pgfsys@transformshift{2.231659in}{4.146148in}%
\pgfsys@useobject{currentmarker}{}%
\end{pgfscope}%
\begin{pgfscope}%
\pgfsys@transformshift{2.253963in}{3.797347in}%
\pgfsys@useobject{currentmarker}{}%
\end{pgfscope}%
\begin{pgfscope}%
\pgfsys@transformshift{2.271571in}{3.591330in}%
\pgfsys@useobject{currentmarker}{}%
\end{pgfscope}%
\begin{pgfscope}%
\pgfsys@transformshift{2.291291in}{3.506081in}%
\pgfsys@useobject{currentmarker}{}%
\end{pgfscope}%
\begin{pgfscope}%
\pgfsys@transformshift{2.310776in}{3.468001in}%
\pgfsys@useobject{currentmarker}{}%
\end{pgfscope}%
\begin{pgfscope}%
\pgfsys@transformshift{2.329324in}{3.454791in}%
\pgfsys@useobject{currentmarker}{}%
\end{pgfscope}%
\begin{pgfscope}%
\pgfsys@transformshift{2.347872in}{3.461178in}%
\pgfsys@useobject{currentmarker}{}%
\end{pgfscope}%
\begin{pgfscope}%
\pgfsys@transformshift{2.366418in}{3.471822in}%
\pgfsys@useobject{currentmarker}{}%
\end{pgfscope}%
\begin{pgfscope}%
\pgfsys@transformshift{2.384026in}{3.515357in}%
\pgfsys@useobject{currentmarker}{}%
\end{pgfscope}%
\begin{pgfscope}%
\pgfsys@transformshift{2.405626in}{3.643719in}%
\pgfsys@useobject{currentmarker}{}%
\end{pgfscope}%
\begin{pgfscope}%
\pgfsys@transformshift{2.422294in}{3.491589in}%
\pgfsys@useobject{currentmarker}{}%
\end{pgfscope}%
\begin{pgfscope}%
\pgfsys@transformshift{2.442954in}{3.568644in}%
\pgfsys@useobject{currentmarker}{}%
\end{pgfscope}%
\begin{pgfscope}%
\pgfsys@transformshift{2.461971in}{3.773719in}%
\pgfsys@useobject{currentmarker}{}%
\end{pgfscope}%
\begin{pgfscope}%
\pgfsys@transformshift{2.480048in}{4.001430in}%
\pgfsys@useobject{currentmarker}{}%
\end{pgfscope}%
\begin{pgfscope}%
\pgfsys@transformshift{2.500004in}{4.248652in}%
\pgfsys@useobject{currentmarker}{}%
\end{pgfscope}%
\begin{pgfscope}%
\pgfsys@transformshift{2.521367in}{4.122258in}%
\pgfsys@useobject{currentmarker}{}%
\end{pgfscope}%
\begin{pgfscope}%
\pgfsys@transformshift{2.539210in}{3.771269in}%
\pgfsys@useobject{currentmarker}{}%
\end{pgfscope}%
\begin{pgfscope}%
\pgfsys@transformshift{2.557054in}{3.600369in}%
\pgfsys@useobject{currentmarker}{}%
\end{pgfscope}%
\begin{pgfscope}%
\pgfsys@transformshift{2.578418in}{3.488433in}%
\pgfsys@useobject{currentmarker}{}%
\end{pgfscope}%
\begin{pgfscope}%
\pgfsys@transformshift{2.596495in}{3.471427in}%
\pgfsys@useobject{currentmarker}{}%
\end{pgfscope}%
\begin{pgfscope}%
\pgfsys@transformshift{2.617860in}{3.455290in}%
\pgfsys@useobject{currentmarker}{}%
\end{pgfscope}%
\begin{pgfscope}%
\pgfsys@transformshift{2.634528in}{3.462200in}%
\pgfsys@useobject{currentmarker}{}%
\end{pgfscope}%
\begin{pgfscope}%
\pgfsys@transformshift{2.655188in}{3.486958in}%
\pgfsys@useobject{currentmarker}{}%
\end{pgfscope}%
\begin{pgfscope}%
\pgfsys@transformshift{2.674439in}{3.576200in}%
\pgfsys@useobject{currentmarker}{}%
\end{pgfscope}%
\begin{pgfscope}%
\pgfsys@transformshift{2.692282in}{3.781937in}%
\pgfsys@useobject{currentmarker}{}%
\end{pgfscope}%
\begin{pgfscope}%
\pgfsys@transformshift{2.712473in}{4.097736in}%
\pgfsys@useobject{currentmarker}{}%
\end{pgfscope}%
\begin{pgfscope}%
\pgfsys@transformshift{2.730550in}{4.252320in}%
\pgfsys@useobject{currentmarker}{}%
\end{pgfscope}%
\begin{pgfscope}%
\pgfsys@transformshift{2.748629in}{4.145009in}%
\pgfsys@useobject{currentmarker}{}%
\end{pgfscope}%
\begin{pgfscope}%
\pgfsys@transformshift{2.769992in}{3.803925in}%
\pgfsys@useobject{currentmarker}{}%
\end{pgfscope}%
\begin{pgfscope}%
\pgfsys@transformshift{2.788540in}{3.612876in}%
\pgfsys@useobject{currentmarker}{}%
\end{pgfscope}%
\begin{pgfscope}%
\pgfsys@transformshift{2.809669in}{3.497371in}%
\pgfsys@useobject{currentmarker}{}%
\end{pgfscope}%
\begin{pgfscope}%
\pgfsys@transformshift{2.827277in}{3.468034in}%
\pgfsys@useobject{currentmarker}{}%
\end{pgfscope}%
\begin{pgfscope}%
\pgfsys@transformshift{2.845119in}{3.456935in}%
\pgfsys@useobject{currentmarker}{}%
\end{pgfscope}%
\begin{pgfscope}%
\pgfsys@transformshift{2.866953in}{3.460972in}%
\pgfsys@useobject{currentmarker}{}%
\end{pgfscope}%
\begin{pgfscope}%
\pgfsys@transformshift{2.884092in}{3.486304in}%
\pgfsys@useobject{currentmarker}{}%
\end{pgfscope}%
\begin{pgfscope}%
\pgfsys@transformshift{2.903107in}{3.544801in}%
\pgfsys@useobject{currentmarker}{}%
\end{pgfscope}%
\begin{pgfscope}%
\pgfsys@transformshift{2.923767in}{3.725164in}%
\pgfsys@useobject{currentmarker}{}%
\end{pgfscope}%
\begin{pgfscope}%
\pgfsys@transformshift{2.941846in}{4.064003in}%
\pgfsys@useobject{currentmarker}{}%
\end{pgfscope}%
\begin{pgfscope}%
\pgfsys@transformshift{2.960863in}{4.240509in}%
\pgfsys@useobject{currentmarker}{}%
\end{pgfscope}%
\begin{pgfscope}%
\pgfsys@transformshift{2.980114in}{4.225256in}%
\pgfsys@useobject{currentmarker}{}%
\end{pgfscope}%
\begin{pgfscope}%
\pgfsys@transformshift{3.000774in}{3.921574in}%
\pgfsys@useobject{currentmarker}{}%
\end{pgfscope}%
\begin{pgfscope}%
\pgfsys@transformshift{3.018616in}{3.677668in}%
\pgfsys@useobject{currentmarker}{}%
\end{pgfscope}%
\begin{pgfscope}%
\pgfsys@transformshift{3.038102in}{3.536928in}%
\pgfsys@useobject{currentmarker}{}%
\end{pgfscope}%
\begin{pgfscope}%
\pgfsys@transformshift{3.057822in}{3.478577in}%
\pgfsys@useobject{currentmarker}{}%
\end{pgfscope}%
\begin{pgfscope}%
\pgfsys@transformshift{3.076370in}{3.463001in}%
\pgfsys@useobject{currentmarker}{}%
\end{pgfscope}%
\begin{pgfscope}%
\pgfsys@transformshift{3.094447in}{3.457766in}%
\pgfsys@useobject{currentmarker}{}%
\end{pgfscope}%
\begin{pgfscope}%
\pgfsys@transformshift{3.116516in}{3.470625in}%
\pgfsys@useobject{currentmarker}{}%
\end{pgfscope}%
\begin{pgfscope}%
\pgfsys@transformshift{3.134358in}{3.491249in}%
\pgfsys@useobject{currentmarker}{}%
\end{pgfscope}%
\begin{pgfscope}%
\pgfsys@transformshift{3.152906in}{3.553392in}%
\pgfsys@useobject{currentmarker}{}%
\end{pgfscope}%
\begin{pgfscope}%
\pgfsys@transformshift{3.173566in}{3.733322in}%
\pgfsys@useobject{currentmarker}{}%
\end{pgfscope}%
\begin{pgfscope}%
\pgfsys@transformshift{3.193991in}{4.072999in}%
\pgfsys@useobject{currentmarker}{}%
\end{pgfscope}%
\begin{pgfscope}%
\pgfsys@transformshift{3.211834in}{4.235586in}%
\pgfsys@useobject{currentmarker}{}%
\end{pgfscope}%
\begin{pgfscope}%
\pgfsys@transformshift{3.231788in}{4.277186in}%
\pgfsys@useobject{currentmarker}{}%
\end{pgfscope}%
\begin{pgfscope}%
\pgfsys@transformshift{3.249867in}{4.105389in}%
\pgfsys@useobject{currentmarker}{}%
\end{pgfscope}%
\begin{pgfscope}%
\pgfsys@transformshift{3.269353in}{3.886103in}%
\pgfsys@useobject{currentmarker}{}%
\end{pgfscope}%
\begin{pgfscope}%
\pgfsys@transformshift{3.290716in}{3.626397in}%
\pgfsys@useobject{currentmarker}{}%
\end{pgfscope}%
\begin{pgfscope}%
\pgfsys@transformshift{3.308795in}{3.526154in}%
\pgfsys@useobject{currentmarker}{}%
\end{pgfscope}%
\begin{pgfscope}%
\pgfsys@transformshift{3.326403in}{3.482134in}%
\pgfsys@useobject{currentmarker}{}%
\end{pgfscope}%
\begin{pgfscope}%
\pgfsys@transformshift{3.346358in}{3.459951in}%
\pgfsys@useobject{currentmarker}{}%
\end{pgfscope}%
\begin{pgfscope}%
\pgfsys@transformshift{3.364671in}{3.460304in}%
\pgfsys@useobject{currentmarker}{}%
\end{pgfscope}%
\begin{pgfscope}%
\pgfsys@transformshift{3.384391in}{3.474227in}%
\pgfsys@useobject{currentmarker}{}%
\end{pgfscope}%
\begin{pgfscope}%
\pgfsys@transformshift{3.405286in}{3.477630in}%
\pgfsys@useobject{currentmarker}{}%
\end{pgfscope}%
\begin{pgfscope}%
\pgfsys@transformshift{3.422425in}{3.460715in}%
\pgfsys@useobject{currentmarker}{}%
\end{pgfscope}%
\begin{pgfscope}%
\pgfsys@transformshift{3.440502in}{3.463508in}%
\pgfsys@useobject{currentmarker}{}%
\end{pgfscope}%
\begin{pgfscope}%
\pgfsys@transformshift{3.460458in}{3.475914in}%
\pgfsys@useobject{currentmarker}{}%
\end{pgfscope}%
\begin{pgfscope}%
\pgfsys@transformshift{3.481587in}{3.522548in}%
\pgfsys@useobject{currentmarker}{}%
\end{pgfscope}%
\begin{pgfscope}%
\pgfsys@transformshift{3.502482in}{3.655077in}%
\pgfsys@useobject{currentmarker}{}%
\end{pgfscope}%
\begin{pgfscope}%
\pgfsys@transformshift{3.517743in}{3.876119in}%
\pgfsys@useobject{currentmarker}{}%
\end{pgfscope}%
\begin{pgfscope}%
\pgfsys@transformshift{3.536994in}{4.252835in}%
\pgfsys@useobject{currentmarker}{}%
\end{pgfscope}%
\begin{pgfscope}%
\pgfsys@transformshift{3.559532in}{4.286190in}%
\pgfsys@useobject{currentmarker}{}%
\end{pgfscope}%
\begin{pgfscope}%
\pgfsys@transformshift{3.577609in}{4.108273in}%
\pgfsys@useobject{currentmarker}{}%
\end{pgfscope}%
\begin{pgfscope}%
\pgfsys@transformshift{3.595217in}{3.797155in}%
\pgfsys@useobject{currentmarker}{}%
\end{pgfscope}%
\begin{pgfscope}%
\pgfsys@transformshift{3.616346in}{3.590737in}%
\pgfsys@useobject{currentmarker}{}%
\end{pgfscope}%
\begin{pgfscope}%
\pgfsys@transformshift{3.635599in}{3.506195in}%
\pgfsys@useobject{currentmarker}{}%
\end{pgfscope}%
\begin{pgfscope}%
\pgfsys@transformshift{3.652736in}{3.484767in}%
\pgfsys@useobject{currentmarker}{}%
\end{pgfscope}%
\begin{pgfscope}%
\pgfsys@transformshift{3.673867in}{3.461214in}%
\pgfsys@useobject{currentmarker}{}%
\end{pgfscope}%
\begin{pgfscope}%
\pgfsys@transformshift{3.691709in}{3.462170in}%
\pgfsys@useobject{currentmarker}{}%
\end{pgfscope}%
\begin{pgfscope}%
\pgfsys@transformshift{3.709786in}{3.472360in}%
\pgfsys@useobject{currentmarker}{}%
\end{pgfscope}%
\begin{pgfscope}%
\pgfsys@transformshift{3.730681in}{3.510755in}%
\pgfsys@useobject{currentmarker}{}%
\end{pgfscope}%
\begin{pgfscope}%
\pgfsys@transformshift{3.752280in}{3.600534in}%
\pgfsys@useobject{currentmarker}{}%
\end{pgfscope}%
\begin{pgfscope}%
\pgfsys@transformshift{3.768009in}{3.518603in}%
\pgfsys@useobject{currentmarker}{}%
\end{pgfscope}%
\begin{pgfscope}%
\pgfsys@transformshift{3.789140in}{3.657272in}%
\pgfsys@useobject{currentmarker}{}%
\end{pgfscope}%
\begin{pgfscope}%
\pgfsys@transformshift{3.808156in}{3.998364in}%
\pgfsys@useobject{currentmarker}{}%
\end{pgfscope}%
\begin{pgfscope}%
\pgfsys@transformshift{3.826233in}{4.309921in}%
\pgfsys@useobject{currentmarker}{}%
\end{pgfscope}%
\begin{pgfscope}%
\pgfsys@transformshift{3.844310in}{4.321475in}%
\pgfsys@useobject{currentmarker}{}%
\end{pgfscope}%
\begin{pgfscope}%
\pgfsys@transformshift{3.866379in}{4.072294in}%
\pgfsys@useobject{currentmarker}{}%
\end{pgfscope}%
\begin{pgfscope}%
\pgfsys@transformshift{3.883752in}{3.787838in}%
\pgfsys@useobject{currentmarker}{}%
\end{pgfscope}%
\begin{pgfscope}%
\pgfsys@transformshift{3.904412in}{3.570058in}%
\pgfsys@useobject{currentmarker}{}%
\end{pgfscope}%
\begin{pgfscope}%
\pgfsys@transformshift{3.923429in}{3.502883in}%
\pgfsys@useobject{currentmarker}{}%
\end{pgfscope}%
\begin{pgfscope}%
\pgfsys@transformshift{3.941506in}{3.472233in}%
\pgfsys@useobject{currentmarker}{}%
\end{pgfscope}%
\begin{pgfscope}%
\pgfsys@transformshift{3.962400in}{3.461348in}%
\pgfsys@useobject{currentmarker}{}%
\end{pgfscope}%
\begin{pgfscope}%
\pgfsys@transformshift{3.980479in}{3.470196in}%
\pgfsys@useobject{currentmarker}{}%
\end{pgfscope}%
\begin{pgfscope}%
\pgfsys@transformshift{3.997616in}{3.489360in}%
\pgfsys@useobject{currentmarker}{}%
\end{pgfscope}%
\begin{pgfscope}%
\pgfsys@transformshift{4.019216in}{3.549564in}%
\pgfsys@useobject{currentmarker}{}%
\end{pgfscope}%
\begin{pgfscope}%
\pgfsys@transformshift{4.037762in}{3.672647in}%
\pgfsys@useobject{currentmarker}{}%
\end{pgfscope}%
\begin{pgfscope}%
\pgfsys@transformshift{4.060771in}{4.052474in}%
\pgfsys@useobject{currentmarker}{}%
\end{pgfscope}%
\begin{pgfscope}%
\pgfsys@transformshift{4.077439in}{4.273823in}%
\pgfsys@useobject{currentmarker}{}%
\end{pgfscope}%
\begin{pgfscope}%
\pgfsys@transformshift{4.097864in}{4.388442in}%
\pgfsys@useobject{currentmarker}{}%
\end{pgfscope}%
\begin{pgfscope}%
\pgfsys@transformshift{4.112655in}{4.289429in}%
\pgfsys@useobject{currentmarker}{}%
\end{pgfscope}%
\begin{pgfscope}%
\pgfsys@transformshift{4.133786in}{4.039789in}%
\pgfsys@useobject{currentmarker}{}%
\end{pgfscope}%
\begin{pgfscope}%
\pgfsys@transformshift{4.155854in}{3.735379in}%
\pgfsys@useobject{currentmarker}{}%
\end{pgfscope}%
\begin{pgfscope}%
\pgfsys@transformshift{4.172991in}{3.588929in}%
\pgfsys@useobject{currentmarker}{}%
\end{pgfscope}%
\begin{pgfscope}%
\pgfsys@transformshift{4.193886in}{3.508602in}%
\pgfsys@useobject{currentmarker}{}%
\end{pgfscope}%
\begin{pgfscope}%
\pgfsys@transformshift{4.212199in}{3.473223in}%
\pgfsys@useobject{currentmarker}{}%
\end{pgfscope}%
\begin{pgfscope}%
\pgfsys@transformshift{4.229807in}{3.462807in}%
\pgfsys@useobject{currentmarker}{}%
\end{pgfscope}%
\begin{pgfscope}%
\pgfsys@transformshift{4.250936in}{3.476535in}%
\pgfsys@useobject{currentmarker}{}%
\end{pgfscope}%
\begin{pgfscope}%
\pgfsys@transformshift{4.270658in}{3.505124in}%
\pgfsys@useobject{currentmarker}{}%
\end{pgfscope}%
\begin{pgfscope}%
\pgfsys@transformshift{4.287326in}{3.570788in}%
\pgfsys@useobject{currentmarker}{}%
\end{pgfscope}%
\begin{pgfscope}%
\pgfsys@transformshift{4.308455in}{3.699983in}%
\pgfsys@useobject{currentmarker}{}%
\end{pgfscope}%
\begin{pgfscope}%
\pgfsys@transformshift{4.326063in}{3.956619in}%
\pgfsys@useobject{currentmarker}{}%
\end{pgfscope}%
\begin{pgfscope}%
\pgfsys@transformshift{4.344611in}{4.265867in}%
\pgfsys@useobject{currentmarker}{}%
\end{pgfscope}%
\begin{pgfscope}%
\pgfsys@transformshift{4.365505in}{4.430911in}%
\pgfsys@useobject{currentmarker}{}%
\end{pgfscope}%
\begin{pgfscope}%
\pgfsys@transformshift{4.384757in}{4.322130in}%
\pgfsys@useobject{currentmarker}{}%
\end{pgfscope}%
\begin{pgfscope}%
\pgfsys@transformshift{4.401659in}{4.091915in}%
\pgfsys@useobject{currentmarker}{}%
\end{pgfscope}%
\begin{pgfscope}%
\pgfsys@transformshift{4.422319in}{3.777529in}%
\pgfsys@useobject{currentmarker}{}%
\end{pgfscope}%
\begin{pgfscope}%
\pgfsys@transformshift{4.441807in}{3.608692in}%
\pgfsys@useobject{currentmarker}{}%
\end{pgfscope}%
\begin{pgfscope}%
\pgfsys@transformshift{4.463876in}{3.511973in}%
\pgfsys@useobject{currentmarker}{}%
\end{pgfscope}%
\begin{pgfscope}%
\pgfsys@transformshift{4.480075in}{3.480241in}%
\pgfsys@useobject{currentmarker}{}%
\end{pgfscope}%
\begin{pgfscope}%
\pgfsys@transformshift{4.479840in}{3.480903in}%
\pgfsys@useobject{currentmarker}{}%
\end{pgfscope}%
\begin{pgfscope}%
\pgfsys@transformshift{4.473266in}{3.495331in}%
\pgfsys@useobject{currentmarker}{}%
\end{pgfscope}%
\begin{pgfscope}%
\pgfsys@transformshift{4.454954in}{3.646930in}%
\pgfsys@useobject{currentmarker}{}%
\end{pgfscope}%
\begin{pgfscope}%
\pgfsys@transformshift{4.433355in}{4.069740in}%
\pgfsys@useobject{currentmarker}{}%
\end{pgfscope}%
\begin{pgfscope}%
\pgfsys@transformshift{4.417624in}{4.353450in}%
\pgfsys@useobject{currentmarker}{}%
\end{pgfscope}%
\begin{pgfscope}%
\pgfsys@transformshift{4.396730in}{4.355841in}%
\pgfsys@useobject{currentmarker}{}%
\end{pgfscope}%
\begin{pgfscope}%
\pgfsys@transformshift{4.379827in}{3.930535in}%
\pgfsys@useobject{currentmarker}{}%
\end{pgfscope}%
\begin{pgfscope}%
\pgfsys@transformshift{4.360576in}{3.516100in}%
\pgfsys@useobject{currentmarker}{}%
\end{pgfscope}%
\begin{pgfscope}%
\pgfsys@transformshift{4.338271in}{3.464185in}%
\pgfsys@useobject{currentmarker}{}%
\end{pgfscope}%
\begin{pgfscope}%
\pgfsys@transformshift{4.319959in}{3.468705in}%
\pgfsys@useobject{currentmarker}{}%
\end{pgfscope}%
\begin{pgfscope}%
\pgfsys@transformshift{4.299300in}{3.527524in}%
\pgfsys@useobject{currentmarker}{}%
\end{pgfscope}%
\begin{pgfscope}%
\pgfsys@transformshift{4.283569in}{3.646901in}%
\pgfsys@useobject{currentmarker}{}%
\end{pgfscope}%
\begin{pgfscope}%
\pgfsys@transformshift{4.262675in}{4.074053in}%
\pgfsys@useobject{currentmarker}{}%
\end{pgfscope}%
\begin{pgfscope}%
\pgfsys@transformshift{4.245067in}{4.365175in}%
\pgfsys@useobject{currentmarker}{}%
\end{pgfscope}%
\begin{pgfscope}%
\pgfsys@transformshift{4.223703in}{4.249359in}%
\pgfsys@useobject{currentmarker}{}%
\end{pgfscope}%
\begin{pgfscope}%
\pgfsys@transformshift{4.206095in}{3.801191in}%
\pgfsys@useobject{currentmarker}{}%
\end{pgfscope}%
\begin{pgfscope}%
\pgfsys@transformshift{4.186139in}{3.558774in}%
\pgfsys@useobject{currentmarker}{}%
\end{pgfscope}%
\begin{pgfscope}%
\pgfsys@transformshift{4.167827in}{3.483032in}%
\pgfsys@useobject{currentmarker}{}%
\end{pgfscope}%
\begin{pgfscope}%
\pgfsys@transformshift{4.147167in}{3.460522in}%
\pgfsys@useobject{currentmarker}{}%
\end{pgfscope}%
\begin{pgfscope}%
\pgfsys@transformshift{4.129089in}{3.473682in}%
\pgfsys@useobject{currentmarker}{}%
\end{pgfscope}%
\begin{pgfscope}%
\pgfsys@transformshift{4.108194in}{3.557442in}%
\pgfsys@useobject{currentmarker}{}%
\end{pgfscope}%
\begin{pgfscope}%
\pgfsys@transformshift{4.085422in}{3.864575in}%
\pgfsys@useobject{currentmarker}{}%
\end{pgfscope}%
\begin{pgfscope}%
\pgfsys@transformshift{4.071806in}{4.199149in}%
\pgfsys@useobject{currentmarker}{}%
\end{pgfscope}%
\begin{pgfscope}%
\pgfsys@transformshift{4.050909in}{4.354038in}%
\pgfsys@useobject{currentmarker}{}%
\end{pgfscope}%
\begin{pgfscope}%
\pgfsys@transformshift{4.033538in}{4.034168in}%
\pgfsys@useobject{currentmarker}{}%
\end{pgfscope}%
\begin{pgfscope}%
\pgfsys@transformshift{4.013816in}{3.628269in}%
\pgfsys@useobject{currentmarker}{}%
\end{pgfscope}%
\begin{pgfscope}%
\pgfsys@transformshift{3.993156in}{3.497136in}%
\pgfsys@useobject{currentmarker}{}%
\end{pgfscope}%
\begin{pgfscope}%
\pgfsys@transformshift{3.975313in}{3.465729in}%
\pgfsys@useobject{currentmarker}{}%
\end{pgfscope}%
\begin{pgfscope}%
\pgfsys@transformshift{3.956062in}{3.461271in}%
\pgfsys@useobject{currentmarker}{}%
\end{pgfscope}%
\begin{pgfscope}%
\pgfsys@transformshift{3.938220in}{3.492558in}%
\pgfsys@useobject{currentmarker}{}%
\end{pgfscope}%
\begin{pgfscope}%
\pgfsys@transformshift{3.917091in}{3.619889in}%
\pgfsys@useobject{currentmarker}{}%
\end{pgfscope}%
\begin{pgfscope}%
\pgfsys@transformshift{3.896665in}{3.993787in}%
\pgfsys@useobject{currentmarker}{}%
\end{pgfscope}%
\begin{pgfscope}%
\pgfsys@transformshift{3.878823in}{4.290326in}%
\pgfsys@useobject{currentmarker}{}%
\end{pgfscope}%
\begin{pgfscope}%
\pgfsys@transformshift{3.857927in}{4.216716in}%
\pgfsys@useobject{currentmarker}{}%
\end{pgfscope}%
\begin{pgfscope}%
\pgfsys@transformshift{3.840789in}{3.848956in}%
\pgfsys@useobject{currentmarker}{}%
\end{pgfscope}%
\begin{pgfscope}%
\pgfsys@transformshift{3.822476in}{3.565570in}%
\pgfsys@useobject{currentmarker}{}%
\end{pgfscope}%
\begin{pgfscope}%
\pgfsys@transformshift{3.802756in}{3.484680in}%
\pgfsys@useobject{currentmarker}{}%
\end{pgfscope}%
\begin{pgfscope}%
\pgfsys@transformshift{3.782330in}{3.457571in}%
\pgfsys@useobject{currentmarker}{}%
\end{pgfscope}%
\begin{pgfscope}%
\pgfsys@transformshift{3.763548in}{3.463686in}%
\pgfsys@useobject{currentmarker}{}%
\end{pgfscope}%
\begin{pgfscope}%
\pgfsys@transformshift{3.742888in}{3.509575in}%
\pgfsys@useobject{currentmarker}{}%
\end{pgfscope}%
\begin{pgfscope}%
\pgfsys@transformshift{3.725046in}{3.633999in}%
\pgfsys@useobject{currentmarker}{}%
\end{pgfscope}%
\begin{pgfscope}%
\pgfsys@transformshift{3.703682in}{4.069157in}%
\pgfsys@useobject{currentmarker}{}%
\end{pgfscope}%
\begin{pgfscope}%
\pgfsys@transformshift{3.685840in}{4.291661in}%
\pgfsys@useobject{currentmarker}{}%
\end{pgfscope}%
\begin{pgfscope}%
\pgfsys@transformshift{3.667527in}{4.203974in}%
\pgfsys@useobject{currentmarker}{}%
\end{pgfscope}%
\begin{pgfscope}%
\pgfsys@transformshift{3.650389in}{3.826873in}%
\pgfsys@useobject{currentmarker}{}%
\end{pgfscope}%
\begin{pgfscope}%
\pgfsys@transformshift{3.626441in}{3.557979in}%
\pgfsys@useobject{currentmarker}{}%
\end{pgfscope}%
\begin{pgfscope}%
\pgfsys@transformshift{3.609069in}{3.488407in}%
\pgfsys@useobject{currentmarker}{}%
\end{pgfscope}%
\begin{pgfscope}%
\pgfsys@transformshift{3.592165in}{3.462139in}%
\pgfsys@useobject{currentmarker}{}%
\end{pgfscope}%
\begin{pgfscope}%
\pgfsys@transformshift{3.572210in}{3.456711in}%
\pgfsys@useobject{currentmarker}{}%
\end{pgfscope}%
\begin{pgfscope}%
\pgfsys@transformshift{3.553428in}{3.479556in}%
\pgfsys@useobject{currentmarker}{}%
\end{pgfscope}%
\begin{pgfscope}%
\pgfsys@transformshift{3.532534in}{3.523712in}%
\pgfsys@useobject{currentmarker}{}%
\end{pgfscope}%
\begin{pgfscope}%
\pgfsys@transformshift{3.514220in}{3.528283in}%
\pgfsys@useobject{currentmarker}{}%
\end{pgfscope}%
\begin{pgfscope}%
\pgfsys@transformshift{3.493560in}{3.743513in}%
\pgfsys@useobject{currentmarker}{}%
\end{pgfscope}%
\begin{pgfscope}%
\pgfsys@transformshift{3.473606in}{4.128954in}%
\pgfsys@useobject{currentmarker}{}%
\end{pgfscope}%
\begin{pgfscope}%
\pgfsys@transformshift{3.455292in}{4.283983in}%
\pgfsys@useobject{currentmarker}{}%
\end{pgfscope}%
\begin{pgfscope}%
\pgfsys@transformshift{3.438155in}{4.211663in}%
\pgfsys@useobject{currentmarker}{}%
\end{pgfscope}%
\begin{pgfscope}%
\pgfsys@transformshift{3.415616in}{3.766171in}%
\pgfsys@useobject{currentmarker}{}%
\end{pgfscope}%
\begin{pgfscope}%
\pgfsys@transformshift{3.397304in}{3.554365in}%
\pgfsys@useobject{currentmarker}{}%
\end{pgfscope}%
\begin{pgfscope}%
\pgfsys@transformshift{3.379696in}{3.495233in}%
\pgfsys@useobject{currentmarker}{}%
\end{pgfscope}%
\begin{pgfscope}%
\pgfsys@transformshift{3.361383in}{3.463761in}%
\pgfsys@useobject{currentmarker}{}%
\end{pgfscope}%
\begin{pgfscope}%
\pgfsys@transformshift{3.339551in}{3.456014in}%
\pgfsys@useobject{currentmarker}{}%
\end{pgfscope}%
\begin{pgfscope}%
\pgfsys@transformshift{3.321472in}{3.471169in}%
\pgfsys@useobject{currentmarker}{}%
\end{pgfscope}%
\begin{pgfscope}%
\pgfsys@transformshift{3.300577in}{3.528405in}%
\pgfsys@useobject{currentmarker}{}%
\end{pgfscope}%
\begin{pgfscope}%
\pgfsys@transformshift{3.282969in}{3.614651in}%
\pgfsys@useobject{currentmarker}{}%
\end{pgfscope}%
\begin{pgfscope}%
\pgfsys@transformshift{3.262309in}{3.953320in}%
\pgfsys@useobject{currentmarker}{}%
\end{pgfscope}%
\begin{pgfscope}%
\pgfsys@transformshift{3.245172in}{4.195554in}%
\pgfsys@useobject{currentmarker}{}%
\end{pgfscope}%
\begin{pgfscope}%
\pgfsys@transformshift{3.226624in}{4.268959in}%
\pgfsys@useobject{currentmarker}{}%
\end{pgfscope}%
\begin{pgfscope}%
\pgfsys@transformshift{3.205261in}{3.966718in}%
\pgfsys@useobject{currentmarker}{}%
\end{pgfscope}%
\begin{pgfscope}%
\pgfsys@transformshift{3.188122in}{3.682959in}%
\pgfsys@useobject{currentmarker}{}%
\end{pgfscope}%
\begin{pgfscope}%
\pgfsys@transformshift{3.167462in}{3.511304in}%
\pgfsys@useobject{currentmarker}{}%
\end{pgfscope}%
\begin{pgfscope}%
\pgfsys@transformshift{3.146568in}{3.466852in}%
\pgfsys@useobject{currentmarker}{}%
\end{pgfscope}%
\begin{pgfscope}%
\pgfsys@transformshift{3.128725in}{3.462480in}%
\pgfsys@useobject{currentmarker}{}%
\end{pgfscope}%
\begin{pgfscope}%
\pgfsys@transformshift{3.110177in}{3.455053in}%
\pgfsys@useobject{currentmarker}{}%
\end{pgfscope}%
\begin{pgfscope}%
\pgfsys@transformshift{3.092335in}{3.466611in}%
\pgfsys@useobject{currentmarker}{}%
\end{pgfscope}%
\begin{pgfscope}%
\pgfsys@transformshift{3.071440in}{3.511940in}%
\pgfsys@useobject{currentmarker}{}%
\end{pgfscope}%
\begin{pgfscope}%
\pgfsys@transformshift{3.053362in}{3.646274in}%
\pgfsys@useobject{currentmarker}{}%
\end{pgfscope}%
\begin{pgfscope}%
\pgfsys@transformshift{3.031058in}{4.023727in}%
\pgfsys@useobject{currentmarker}{}%
\end{pgfscope}%
\begin{pgfscope}%
\pgfsys@transformshift{3.011807in}{4.241096in}%
\pgfsys@useobject{currentmarker}{}%
\end{pgfscope}%
\begin{pgfscope}%
\pgfsys@transformshift{2.993965in}{3.971229in}%
\pgfsys@useobject{currentmarker}{}%
\end{pgfscope}%
\begin{pgfscope}%
\pgfsys@transformshift{2.975888in}{4.248594in}%
\pgfsys@useobject{currentmarker}{}%
\end{pgfscope}%
\begin{pgfscope}%
\pgfsys@transformshift{2.958045in}{4.206234in}%
\pgfsys@useobject{currentmarker}{}%
\end{pgfscope}%
\begin{pgfscope}%
\pgfsys@transformshift{2.938323in}{3.829533in}%
\pgfsys@useobject{currentmarker}{}%
\end{pgfscope}%
\begin{pgfscope}%
\pgfsys@transformshift{2.913203in}{3.557242in}%
\pgfsys@useobject{currentmarker}{}%
\end{pgfscope}%
\begin{pgfscope}%
\pgfsys@transformshift{2.900761in}{3.521972in}%
\pgfsys@useobject{currentmarker}{}%
\end{pgfscope}%
\begin{pgfscope}%
\pgfsys@transformshift{2.879395in}{3.468553in}%
\pgfsys@useobject{currentmarker}{}%
\end{pgfscope}%
\begin{pgfscope}%
\pgfsys@transformshift{2.860615in}{3.490407in}%
\pgfsys@useobject{currentmarker}{}%
\end{pgfscope}%
\begin{pgfscope}%
\pgfsys@transformshift{2.836903in}{3.456720in}%
\pgfsys@useobject{currentmarker}{}%
\end{pgfscope}%
\begin{pgfscope}%
\pgfsys@transformshift{2.822816in}{3.457130in}%
\pgfsys@useobject{currentmarker}{}%
\end{pgfscope}%
\begin{pgfscope}%
\pgfsys@transformshift{2.801451in}{3.483551in}%
\pgfsys@useobject{currentmarker}{}%
\end{pgfscope}%
\begin{pgfscope}%
\pgfsys@transformshift{2.783374in}{3.554088in}%
\pgfsys@useobject{currentmarker}{}%
\end{pgfscope}%
\begin{pgfscope}%
\pgfsys@transformshift{2.763654in}{3.774362in}%
\pgfsys@useobject{currentmarker}{}%
\end{pgfscope}%
\begin{pgfscope}%
\pgfsys@transformshift{2.745106in}{4.143773in}%
\pgfsys@useobject{currentmarker}{}%
\end{pgfscope}%
\begin{pgfscope}%
\pgfsys@transformshift{2.722803in}{4.249159in}%
\pgfsys@useobject{currentmarker}{}%
\end{pgfscope}%
\begin{pgfscope}%
\pgfsys@transformshift{2.704491in}{3.986634in}%
\pgfsys@useobject{currentmarker}{}%
\end{pgfscope}%
\begin{pgfscope}%
\pgfsys@transformshift{2.685943in}{3.715326in}%
\pgfsys@useobject{currentmarker}{}%
\end{pgfscope}%
\begin{pgfscope}%
\pgfsys@transformshift{2.669275in}{3.551321in}%
\pgfsys@useobject{currentmarker}{}%
\end{pgfscope}%
\begin{pgfscope}%
\pgfsys@transformshift{2.648144in}{3.484864in}%
\pgfsys@useobject{currentmarker}{}%
\end{pgfscope}%
\begin{pgfscope}%
\pgfsys@transformshift{2.630536in}{3.459458in}%
\pgfsys@useobject{currentmarker}{}%
\end{pgfscope}%
\begin{pgfscope}%
\pgfsys@transformshift{2.611285in}{3.456798in}%
\pgfsys@useobject{currentmarker}{}%
\end{pgfscope}%
\begin{pgfscope}%
\pgfsys@transformshift{2.590391in}{3.470342in}%
\pgfsys@useobject{currentmarker}{}%
\end{pgfscope}%
\begin{pgfscope}%
\pgfsys@transformshift{2.572783in}{3.507309in}%
\pgfsys@useobject{currentmarker}{}%
\end{pgfscope}%
\begin{pgfscope}%
\pgfsys@transformshift{2.550949in}{3.660168in}%
\pgfsys@useobject{currentmarker}{}%
\end{pgfscope}%
\begin{pgfscope}%
\pgfsys@transformshift{2.534515in}{3.966657in}%
\pgfsys@useobject{currentmarker}{}%
\end{pgfscope}%
\begin{pgfscope}%
\pgfsys@transformshift{2.515734in}{4.216678in}%
\pgfsys@useobject{currentmarker}{}%
\end{pgfscope}%
\begin{pgfscope}%
\pgfsys@transformshift{2.496481in}{4.225690in}%
\pgfsys@useobject{currentmarker}{}%
\end{pgfscope}%
\begin{pgfscope}%
\pgfsys@transformshift{2.474884in}{3.884541in}%
\pgfsys@useobject{currentmarker}{}%
\end{pgfscope}%
\begin{pgfscope}%
\pgfsys@transformshift{2.456570in}{3.629274in}%
\pgfsys@useobject{currentmarker}{}%
\end{pgfscope}%
\begin{pgfscope}%
\pgfsys@transformshift{2.438024in}{3.518593in}%
\pgfsys@useobject{currentmarker}{}%
\end{pgfscope}%
\begin{pgfscope}%
\pgfsys@transformshift{2.412904in}{3.468455in}%
\pgfsys@useobject{currentmarker}{}%
\end{pgfscope}%
\begin{pgfscope}%
\pgfsys@transformshift{2.397642in}{3.456884in}%
\pgfsys@useobject{currentmarker}{}%
\end{pgfscope}%
\begin{pgfscope}%
\pgfsys@transformshift{2.378626in}{3.460629in}%
\pgfsys@useobject{currentmarker}{}%
\end{pgfscope}%
\begin{pgfscope}%
\pgfsys@transformshift{2.359845in}{3.469424in}%
\pgfsys@useobject{currentmarker}{}%
\end{pgfscope}%
\begin{pgfscope}%
\pgfsys@transformshift{2.342003in}{3.471189in}%
\pgfsys@useobject{currentmarker}{}%
\end{pgfscope}%
\begin{pgfscope}%
\pgfsys@transformshift{2.323455in}{3.527272in}%
\pgfsys@useobject{currentmarker}{}%
\end{pgfscope}%
\begin{pgfscope}%
\pgfsys@transformshift{2.301621in}{3.735993in}%
\pgfsys@useobject{currentmarker}{}%
\end{pgfscope}%
\begin{pgfscope}%
\pgfsys@transformshift{2.282838in}{4.085307in}%
\pgfsys@useobject{currentmarker}{}%
\end{pgfscope}%
\begin{pgfscope}%
\pgfsys@transformshift{2.264058in}{4.256882in}%
\pgfsys@useobject{currentmarker}{}%
\end{pgfscope}%
\begin{pgfscope}%
\pgfsys@transformshift{2.241050in}{4.088335in}%
\pgfsys@useobject{currentmarker}{}%
\end{pgfscope}%
\begin{pgfscope}%
\pgfsys@transformshift{2.223911in}{3.799447in}%
\pgfsys@useobject{currentmarker}{}%
\end{pgfscope}%
\begin{pgfscope}%
\pgfsys@transformshift{2.205365in}{3.591306in}%
\pgfsys@useobject{currentmarker}{}%
\end{pgfscope}%
\begin{pgfscope}%
\pgfsys@transformshift{2.186113in}{3.511370in}%
\pgfsys@useobject{currentmarker}{}%
\end{pgfscope}%
\begin{pgfscope}%
\pgfsys@transformshift{2.168036in}{3.469919in}%
\pgfsys@useobject{currentmarker}{}%
\end{pgfscope}%
\begin{pgfscope}%
\pgfsys@transformshift{2.149958in}{3.457390in}%
\pgfsys@useobject{currentmarker}{}%
\end{pgfscope}%
\begin{pgfscope}%
\pgfsys@transformshift{2.128123in}{3.462170in}%
\pgfsys@useobject{currentmarker}{}%
\end{pgfscope}%
\begin{pgfscope}%
\pgfsys@transformshift{2.110986in}{3.478485in}%
\pgfsys@useobject{currentmarker}{}%
\end{pgfscope}%
\begin{pgfscope}%
\pgfsys@transformshift{2.092438in}{3.527186in}%
\pgfsys@useobject{currentmarker}{}%
\end{pgfscope}%
\begin{pgfscope}%
\pgfsys@transformshift{2.073658in}{3.626751in}%
\pgfsys@useobject{currentmarker}{}%
\end{pgfscope}%
\begin{pgfscope}%
\pgfsys@transformshift{2.050650in}{3.987192in}%
\pgfsys@useobject{currentmarker}{}%
\end{pgfscope}%
\begin{pgfscope}%
\pgfsys@transformshift{2.032573in}{4.237500in}%
\pgfsys@useobject{currentmarker}{}%
\end{pgfscope}%
\begin{pgfscope}%
\pgfsys@transformshift{2.013321in}{4.233109in}%
\pgfsys@useobject{currentmarker}{}%
\end{pgfscope}%
\begin{pgfscope}%
\pgfsys@transformshift{1.994068in}{3.934929in}%
\pgfsys@useobject{currentmarker}{}%
\end{pgfscope}%
\begin{pgfscope}%
\pgfsys@transformshift{1.976226in}{3.675941in}%
\pgfsys@useobject{currentmarker}{}%
\end{pgfscope}%
\begin{pgfscope}%
\pgfsys@transformshift{1.957445in}{3.544795in}%
\pgfsys@useobject{currentmarker}{}%
\end{pgfscope}%
\begin{pgfscope}%
\pgfsys@transformshift{1.938663in}{3.490429in}%
\pgfsys@useobject{currentmarker}{}%
\end{pgfscope}%
\begin{pgfscope}%
\pgfsys@transformshift{1.917769in}{3.465690in}%
\pgfsys@useobject{currentmarker}{}%
\end{pgfscope}%
\begin{pgfscope}%
\pgfsys@transformshift{1.898987in}{3.458159in}%
\pgfsys@useobject{currentmarker}{}%
\end{pgfscope}%
\begin{pgfscope}%
\pgfsys@transformshift{1.880910in}{3.465606in}%
\pgfsys@useobject{currentmarker}{}%
\end{pgfscope}%
\begin{pgfscope}%
\pgfsys@transformshift{1.861658in}{3.483882in}%
\pgfsys@useobject{currentmarker}{}%
\end{pgfscope}%
\begin{pgfscope}%
\pgfsys@transformshift{1.841702in}{3.561563in}%
\pgfsys@useobject{currentmarker}{}%
\end{pgfscope}%
\begin{pgfscope}%
\pgfsys@transformshift{1.822685in}{3.672514in}%
\pgfsys@useobject{currentmarker}{}%
\end{pgfscope}%
\begin{pgfscope}%
\pgfsys@transformshift{1.803199in}{3.815869in}%
\pgfsys@useobject{currentmarker}{}%
\end{pgfscope}%
\begin{pgfscope}%
\pgfsys@transformshift{1.780896in}{4.010043in}%
\pgfsys@useobject{currentmarker}{}%
\end{pgfscope}%
\begin{pgfscope}%
\pgfsys@transformshift{1.766106in}{4.230552in}%
\pgfsys@useobject{currentmarker}{}%
\end{pgfscope}%
\begin{pgfscope}%
\pgfsys@transformshift{1.746855in}{4.270981in}%
\pgfsys@useobject{currentmarker}{}%
\end{pgfscope}%
\begin{pgfscope}%
\pgfsys@transformshift{1.725020in}{3.996623in}%
\pgfsys@useobject{currentmarker}{}%
\end{pgfscope}%
\begin{pgfscope}%
\pgfsys@transformshift{1.709761in}{3.779403in}%
\pgfsys@useobject{currentmarker}{}%
\end{pgfscope}%
\begin{pgfscope}%
\pgfsys@transformshift{1.688396in}{3.574512in}%
\pgfsys@useobject{currentmarker}{}%
\end{pgfscope}%
\begin{pgfscope}%
\pgfsys@transformshift{1.669379in}{3.501506in}%
\pgfsys@useobject{currentmarker}{}%
\end{pgfscope}%
\begin{pgfscope}%
\pgfsys@transformshift{1.644727in}{3.462260in}%
\pgfsys@useobject{currentmarker}{}%
\end{pgfscope}%
\begin{pgfscope}%
\pgfsys@transformshift{1.628764in}{3.458059in}%
\pgfsys@useobject{currentmarker}{}%
\end{pgfscope}%
\begin{pgfscope}%
\pgfsys@transformshift{1.610685in}{3.465228in}%
\pgfsys@useobject{currentmarker}{}%
\end{pgfscope}%
\begin{pgfscope}%
\pgfsys@transformshift{1.592608in}{3.753057in}%
\pgfsys@useobject{currentmarker}{}%
\end{pgfscope}%
\begin{pgfscope}%
\pgfsys@transformshift{1.570305in}{3.539810in}%
\pgfsys@useobject{currentmarker}{}%
\end{pgfscope}%
\begin{pgfscope}%
\pgfsys@transformshift{1.552463in}{3.480732in}%
\pgfsys@useobject{currentmarker}{}%
\end{pgfscope}%
\begin{pgfscope}%
\pgfsys@transformshift{1.533446in}{3.461073in}%
\pgfsys@useobject{currentmarker}{}%
\end{pgfscope}%
\begin{pgfscope}%
\pgfsys@transformshift{1.512317in}{3.459268in}%
\pgfsys@useobject{currentmarker}{}%
\end{pgfscope}%
\begin{pgfscope}%
\pgfsys@transformshift{1.491657in}{3.482030in}%
\pgfsys@useobject{currentmarker}{}%
\end{pgfscope}%
\begin{pgfscope}%
\pgfsys@transformshift{1.475927in}{3.536713in}%
\pgfsys@useobject{currentmarker}{}%
\end{pgfscope}%
\begin{pgfscope}%
\pgfsys@transformshift{1.455736in}{3.675428in}%
\pgfsys@useobject{currentmarker}{}%
\end{pgfscope}%
\begin{pgfscope}%
\pgfsys@transformshift{1.437659in}{3.981630in}%
\pgfsys@useobject{currentmarker}{}%
\end{pgfscope}%
\begin{pgfscope}%
\pgfsys@transformshift{1.419111in}{4.250214in}%
\pgfsys@useobject{currentmarker}{}%
\end{pgfscope}%
\begin{pgfscope}%
\pgfsys@transformshift{1.400094in}{4.309084in}%
\pgfsys@useobject{currentmarker}{}%
\end{pgfscope}%
\begin{pgfscope}%
\pgfsys@transformshift{1.382486in}{4.095579in}%
\pgfsys@useobject{currentmarker}{}%
\end{pgfscope}%
\begin{pgfscope}%
\pgfsys@transformshift{1.360889in}{3.719071in}%
\pgfsys@useobject{currentmarker}{}%
\end{pgfscope}%
\begin{pgfscope}%
\pgfsys@transformshift{1.341403in}{3.562191in}%
\pgfsys@useobject{currentmarker}{}%
\end{pgfscope}%
\begin{pgfscope}%
\pgfsys@transformshift{1.323558in}{3.504975in}%
\pgfsys@useobject{currentmarker}{}%
\end{pgfscope}%
\begin{pgfscope}%
\pgfsys@transformshift{1.301961in}{3.468568in}%
\pgfsys@useobject{currentmarker}{}%
\end{pgfscope}%
\begin{pgfscope}%
\pgfsys@transformshift{1.283884in}{3.459442in}%
\pgfsys@useobject{currentmarker}{}%
\end{pgfscope}%
\begin{pgfscope}%
\pgfsys@transformshift{1.265570in}{3.466680in}%
\pgfsys@useobject{currentmarker}{}%
\end{pgfscope}%
\begin{pgfscope}%
\pgfsys@transformshift{1.246319in}{3.493387in}%
\pgfsys@useobject{currentmarker}{}%
\end{pgfscope}%
\begin{pgfscope}%
\pgfsys@transformshift{1.227537in}{3.556151in}%
\pgfsys@useobject{currentmarker}{}%
\end{pgfscope}%
\begin{pgfscope}%
\pgfsys@transformshift{1.205703in}{3.764301in}%
\pgfsys@useobject{currentmarker}{}%
\end{pgfscope}%
\begin{pgfscope}%
\pgfsys@transformshift{1.187157in}{4.086210in}%
\pgfsys@useobject{currentmarker}{}%
\end{pgfscope}%
\begin{pgfscope}%
\pgfsys@transformshift{1.168843in}{4.282156in}%
\pgfsys@useobject{currentmarker}{}%
\end{pgfscope}%
\begin{pgfscope}%
\pgfsys@transformshift{1.151001in}{4.317978in}%
\pgfsys@useobject{currentmarker}{}%
\end{pgfscope}%
\begin{pgfscope}%
\pgfsys@transformshift{1.128698in}{4.061370in}%
\pgfsys@useobject{currentmarker}{}%
\end{pgfscope}%
\begin{pgfscope}%
\pgfsys@transformshift{1.109681in}{3.770280in}%
\pgfsys@useobject{currentmarker}{}%
\end{pgfscope}%
\begin{pgfscope}%
\pgfsys@transformshift{1.092073in}{3.588215in}%
\pgfsys@useobject{currentmarker}{}%
\end{pgfscope}%
\begin{pgfscope}%
\pgfsys@transformshift{1.072587in}{3.511966in}%
\pgfsys@useobject{currentmarker}{}%
\end{pgfscope}%
\begin{pgfscope}%
\pgfsys@transformshift{1.054042in}{3.477970in}%
\pgfsys@useobject{currentmarker}{}%
\end{pgfscope}%
\begin{pgfscope}%
\pgfsys@transformshift{1.033616in}{3.461051in}%
\pgfsys@useobject{currentmarker}{}%
\end{pgfscope}%
\begin{pgfscope}%
\pgfsys@transformshift{1.014365in}{3.463060in}%
\pgfsys@useobject{currentmarker}{}%
\end{pgfscope}%
\begin{pgfscope}%
\pgfsys@transformshift{0.995583in}{3.480740in}%
\pgfsys@useobject{currentmarker}{}%
\end{pgfscope}%
\begin{pgfscope}%
\pgfsys@transformshift{0.977740in}{3.526963in}%
\pgfsys@useobject{currentmarker}{}%
\end{pgfscope}%
\begin{pgfscope}%
\pgfsys@transformshift{0.956375in}{3.675191in}%
\pgfsys@useobject{currentmarker}{}%
\end{pgfscope}%
\begin{pgfscope}%
\pgfsys@transformshift{0.937594in}{3.951402in}%
\pgfsys@useobject{currentmarker}{}%
\end{pgfscope}%
\begin{pgfscope}%
\pgfsys@transformshift{0.919047in}{4.140234in}%
\pgfsys@useobject{currentmarker}{}%
\end{pgfscope}%
\begin{pgfscope}%
\pgfsys@transformshift{0.900030in}{4.251009in}%
\pgfsys@useobject{currentmarker}{}%
\end{pgfscope}%
\begin{pgfscope}%
\pgfsys@transformshift{0.878196in}{4.367212in}%
\pgfsys@useobject{currentmarker}{}%
\end{pgfscope}%
\begin{pgfscope}%
\pgfsys@transformshift{0.859884in}{4.287005in}%
\pgfsys@useobject{currentmarker}{}%
\end{pgfscope}%
\begin{pgfscope}%
\pgfsys@transformshift{0.842276in}{3.882134in}%
\pgfsys@useobject{currentmarker}{}%
\end{pgfscope}%
\begin{pgfscope}%
\pgfsys@transformshift{0.823025in}{4.157838in}%
\pgfsys@useobject{currentmarker}{}%
\end{pgfscope}%
\begin{pgfscope}%
\pgfsys@transformshift{0.804712in}{4.357827in}%
\pgfsys@useobject{currentmarker}{}%
\end{pgfscope}%
\begin{pgfscope}%
\pgfsys@transformshift{0.782409in}{4.250936in}%
\pgfsys@useobject{currentmarker}{}%
\end{pgfscope}%
\begin{pgfscope}%
\pgfsys@transformshift{0.765975in}{3.918544in}%
\pgfsys@useobject{currentmarker}{}%
\end{pgfscope}%
\begin{pgfscope}%
\pgfsys@transformshift{0.744610in}{3.627738in}%
\pgfsys@useobject{currentmarker}{}%
\end{pgfscope}%
\begin{pgfscope}%
\pgfsys@transformshift{0.722777in}{3.539430in}%
\pgfsys@useobject{currentmarker}{}%
\end{pgfscope}%
\begin{pgfscope}%
\pgfsys@transformshift{0.705638in}{3.493433in}%
\pgfsys@useobject{currentmarker}{}%
\end{pgfscope}%
\begin{pgfscope}%
\pgfsys@transformshift{0.686856in}{3.464785in}%
\pgfsys@useobject{currentmarker}{}%
\end{pgfscope}%
\begin{pgfscope}%
\pgfsys@transformshift{0.669013in}{3.462571in}%
\pgfsys@useobject{currentmarker}{}%
\end{pgfscope}%
\begin{pgfscope}%
\pgfsys@transformshift{0.650936in}{3.482239in}%
\pgfsys@useobject{currentmarker}{}%
\end{pgfscope}%
\begin{pgfscope}%
\pgfsys@transformshift{0.650468in}{3.484676in}%
\pgfsys@useobject{currentmarker}{}%
\end{pgfscope}%
\begin{pgfscope}%
\pgfsys@transformshift{0.658918in}{3.468665in}%
\pgfsys@useobject{currentmarker}{}%
\end{pgfscope}%
\begin{pgfscope}%
\pgfsys@transformshift{0.673945in}{3.461878in}%
\pgfsys@useobject{currentmarker}{}%
\end{pgfscope}%
\begin{pgfscope}%
\pgfsys@transformshift{0.696248in}{3.499793in}%
\pgfsys@useobject{currentmarker}{}%
\end{pgfscope}%
\begin{pgfscope}%
\pgfsys@transformshift{0.713151in}{3.593639in}%
\pgfsys@useobject{currentmarker}{}%
\end{pgfscope}%
\begin{pgfscope}%
\pgfsys@transformshift{0.732402in}{3.850221in}%
\pgfsys@useobject{currentmarker}{}%
\end{pgfscope}%
\begin{pgfscope}%
\pgfsys@transformshift{0.753533in}{4.344457in}%
\pgfsys@useobject{currentmarker}{}%
\end{pgfscope}%
\begin{pgfscope}%
\pgfsys@transformshift{0.771141in}{4.298412in}%
\pgfsys@useobject{currentmarker}{}%
\end{pgfscope}%
\begin{pgfscope}%
\pgfsys@transformshift{0.790626in}{3.889722in}%
\pgfsys@useobject{currentmarker}{}%
\end{pgfscope}%
\begin{pgfscope}%
\pgfsys@transformshift{0.810581in}{3.562331in}%
\pgfsys@useobject{currentmarker}{}%
\end{pgfscope}%
\begin{pgfscope}%
\pgfsys@transformshift{0.828424in}{3.480794in}%
\pgfsys@useobject{currentmarker}{}%
\end{pgfscope}%
\begin{pgfscope}%
\pgfsys@transformshift{0.847206in}{3.458579in}%
\pgfsys@useobject{currentmarker}{}%
\end{pgfscope}%
\begin{pgfscope}%
\pgfsys@transformshift{0.866223in}{3.474695in}%
\pgfsys@useobject{currentmarker}{}%
\end{pgfscope}%
\begin{pgfscope}%
\pgfsys@transformshift{0.888291in}{3.557590in}%
\pgfsys@useobject{currentmarker}{}%
\end{pgfscope}%
\begin{pgfscope}%
\pgfsys@transformshift{0.908011in}{3.773167in}%
\pgfsys@useobject{currentmarker}{}%
\end{pgfscope}%
\begin{pgfscope}%
\pgfsys@transformshift{0.923507in}{4.142775in}%
\pgfsys@useobject{currentmarker}{}%
\end{pgfscope}%
\begin{pgfscope}%
\pgfsys@transformshift{0.942524in}{4.330782in}%
\pgfsys@useobject{currentmarker}{}%
\end{pgfscope}%
\begin{pgfscope}%
\pgfsys@transformshift{0.962010in}{4.054682in}%
\pgfsys@useobject{currentmarker}{}%
\end{pgfscope}%
\begin{pgfscope}%
\pgfsys@transformshift{0.980558in}{3.791809in}%
\pgfsys@useobject{currentmarker}{}%
\end{pgfscope}%
\begin{pgfscope}%
\pgfsys@transformshift{1.003798in}{3.525842in}%
\pgfsys@useobject{currentmarker}{}%
\end{pgfscope}%
\begin{pgfscope}%
\pgfsys@transformshift{1.022112in}{3.466618in}%
\pgfsys@useobject{currentmarker}{}%
\end{pgfscope}%
\begin{pgfscope}%
\pgfsys@transformshift{1.041832in}{3.458211in}%
\pgfsys@useobject{currentmarker}{}%
\end{pgfscope}%
\begin{pgfscope}%
\pgfsys@transformshift{1.059909in}{3.480402in}%
\pgfsys@useobject{currentmarker}{}%
\end{pgfscope}%
\begin{pgfscope}%
\pgfsys@transformshift{1.079631in}{3.553556in}%
\pgfsys@useobject{currentmarker}{}%
\end{pgfscope}%
\begin{pgfscope}%
\pgfsys@transformshift{1.098882in}{3.802142in}%
\pgfsys@useobject{currentmarker}{}%
\end{pgfscope}%
\begin{pgfscope}%
\pgfsys@transformshift{1.115550in}{4.199676in}%
\pgfsys@useobject{currentmarker}{}%
\end{pgfscope}%
\begin{pgfscope}%
\pgfsys@transformshift{1.136210in}{4.264613in}%
\pgfsys@useobject{currentmarker}{}%
\end{pgfscope}%
\begin{pgfscope}%
\pgfsys@transformshift{1.155932in}{3.950333in}%
\pgfsys@useobject{currentmarker}{}%
\end{pgfscope}%
\begin{pgfscope}%
\pgfsys@transformshift{1.174949in}{3.591700in}%
\pgfsys@useobject{currentmarker}{}%
\end{pgfscope}%
\begin{pgfscope}%
\pgfsys@transformshift{1.194200in}{3.488846in}%
\pgfsys@useobject{currentmarker}{}%
\end{pgfscope}%
\begin{pgfscope}%
\pgfsys@transformshift{1.212981in}{3.459663in}%
\pgfsys@useobject{currentmarker}{}%
\end{pgfscope}%
\begin{pgfscope}%
\pgfsys@transformshift{1.232937in}{3.461991in}%
\pgfsys@useobject{currentmarker}{}%
\end{pgfscope}%
\begin{pgfscope}%
\pgfsys@transformshift{1.251485in}{3.493200in}%
\pgfsys@useobject{currentmarker}{}%
\end{pgfscope}%
\begin{pgfscope}%
\pgfsys@transformshift{1.270031in}{3.594473in}%
\pgfsys@useobject{currentmarker}{}%
\end{pgfscope}%
\begin{pgfscope}%
\pgfsys@transformshift{1.289753in}{3.908619in}%
\pgfsys@useobject{currentmarker}{}%
\end{pgfscope}%
\begin{pgfscope}%
\pgfsys@transformshift{1.308299in}{4.266062in}%
\pgfsys@useobject{currentmarker}{}%
\end{pgfscope}%
\begin{pgfscope}%
\pgfsys@transformshift{1.326847in}{4.216183in}%
\pgfsys@useobject{currentmarker}{}%
\end{pgfscope}%
\begin{pgfscope}%
\pgfsys@transformshift{1.347507in}{3.942308in}%
\pgfsys@useobject{currentmarker}{}%
\end{pgfscope}%
\begin{pgfscope}%
\pgfsys@transformshift{1.368870in}{3.628457in}%
\pgfsys@useobject{currentmarker}{}%
\end{pgfscope}%
\begin{pgfscope}%
\pgfsys@transformshift{1.388121in}{3.497013in}%
\pgfsys@useobject{currentmarker}{}%
\end{pgfscope}%
\begin{pgfscope}%
\pgfsys@transformshift{1.406435in}{3.460800in}%
\pgfsys@useobject{currentmarker}{}%
\end{pgfscope}%
\begin{pgfscope}%
\pgfsys@transformshift{1.425451in}{3.456488in}%
\pgfsys@useobject{currentmarker}{}%
\end{pgfscope}%
\begin{pgfscope}%
\pgfsys@transformshift{1.444703in}{3.473851in}%
\pgfsys@useobject{currentmarker}{}%
\end{pgfscope}%
\begin{pgfscope}%
\pgfsys@transformshift{1.467240in}{3.544970in}%
\pgfsys@useobject{currentmarker}{}%
\end{pgfscope}%
\begin{pgfscope}%
\pgfsys@transformshift{1.481562in}{3.684335in}%
\pgfsys@useobject{currentmarker}{}%
\end{pgfscope}%
\begin{pgfscope}%
\pgfsys@transformshift{1.502925in}{4.089439in}%
\pgfsys@useobject{currentmarker}{}%
\end{pgfscope}%
\begin{pgfscope}%
\pgfsys@transformshift{1.520299in}{4.260904in}%
\pgfsys@useobject{currentmarker}{}%
\end{pgfscope}%
\begin{pgfscope}%
\pgfsys@transformshift{1.537907in}{4.085785in}%
\pgfsys@useobject{currentmarker}{}%
\end{pgfscope}%
\begin{pgfscope}%
\pgfsys@transformshift{1.560915in}{3.765314in}%
\pgfsys@useobject{currentmarker}{}%
\end{pgfscope}%
\begin{pgfscope}%
\pgfsys@transformshift{1.579695in}{3.565004in}%
\pgfsys@useobject{currentmarker}{}%
\end{pgfscope}%
\begin{pgfscope}%
\pgfsys@transformshift{1.598478in}{3.481559in}%
\pgfsys@useobject{currentmarker}{}%
\end{pgfscope}%
\begin{pgfscope}%
\pgfsys@transformshift{1.617729in}{3.458066in}%
\pgfsys@useobject{currentmarker}{}%
\end{pgfscope}%
\begin{pgfscope}%
\pgfsys@transformshift{1.635337in}{3.459673in}%
\pgfsys@useobject{currentmarker}{}%
\end{pgfscope}%
\begin{pgfscope}%
\pgfsys@transformshift{1.655057in}{3.479283in}%
\pgfsys@useobject{currentmarker}{}%
\end{pgfscope}%
\begin{pgfscope}%
\pgfsys@transformshift{1.674308in}{3.538044in}%
\pgfsys@useobject{currentmarker}{}%
\end{pgfscope}%
\begin{pgfscope}%
\pgfsys@transformshift{1.692387in}{3.635104in}%
\pgfsys@useobject{currentmarker}{}%
\end{pgfscope}%
\begin{pgfscope}%
\pgfsys@transformshift{1.714690in}{3.988584in}%
\pgfsys@useobject{currentmarker}{}%
\end{pgfscope}%
\begin{pgfscope}%
\pgfsys@transformshift{1.732533in}{4.235929in}%
\pgfsys@useobject{currentmarker}{}%
\end{pgfscope}%
\begin{pgfscope}%
\pgfsys@transformshift{1.751315in}{4.237470in}%
\pgfsys@useobject{currentmarker}{}%
\end{pgfscope}%
\begin{pgfscope}%
\pgfsys@transformshift{1.770095in}{4.006107in}%
\pgfsys@useobject{currentmarker}{}%
\end{pgfscope}%
\begin{pgfscope}%
\pgfsys@transformshift{1.789112in}{3.678507in}%
\pgfsys@useobject{currentmarker}{}%
\end{pgfscope}%
\begin{pgfscope}%
\pgfsys@transformshift{1.808129in}{3.522809in}%
\pgfsys@useobject{currentmarker}{}%
\end{pgfscope}%
\begin{pgfscope}%
\pgfsys@transformshift{1.827851in}{3.476192in}%
\pgfsys@useobject{currentmarker}{}%
\end{pgfscope}%
\begin{pgfscope}%
\pgfsys@transformshift{1.848277in}{3.458964in}%
\pgfsys@useobject{currentmarker}{}%
\end{pgfscope}%
\begin{pgfscope}%
\pgfsys@transformshift{1.867057in}{3.457642in}%
\pgfsys@useobject{currentmarker}{}%
\end{pgfscope}%
\begin{pgfscope}%
\pgfsys@transformshift{1.885136in}{3.468533in}%
\pgfsys@useobject{currentmarker}{}%
\end{pgfscope}%
\begin{pgfscope}%
\pgfsys@transformshift{1.904151in}{3.515370in}%
\pgfsys@useobject{currentmarker}{}%
\end{pgfscope}%
\begin{pgfscope}%
\pgfsys@transformshift{1.922933in}{3.582097in}%
\pgfsys@useobject{currentmarker}{}%
\end{pgfscope}%
\begin{pgfscope}%
\pgfsys@transformshift{1.945472in}{3.889338in}%
\pgfsys@useobject{currentmarker}{}%
\end{pgfscope}%
\begin{pgfscope}%
\pgfsys@transformshift{1.963315in}{4.190748in}%
\pgfsys@useobject{currentmarker}{}%
\end{pgfscope}%
\begin{pgfscope}%
\pgfsys@transformshift{1.981626in}{4.242759in}%
\pgfsys@useobject{currentmarker}{}%
\end{pgfscope}%
\begin{pgfscope}%
\pgfsys@transformshift{1.999000in}{4.101431in}%
\pgfsys@useobject{currentmarker}{}%
\end{pgfscope}%
\begin{pgfscope}%
\pgfsys@transformshift{2.018954in}{3.772808in}%
\pgfsys@useobject{currentmarker}{}%
\end{pgfscope}%
\begin{pgfscope}%
\pgfsys@transformshift{2.041259in}{3.558411in}%
\pgfsys@useobject{currentmarker}{}%
\end{pgfscope}%
\begin{pgfscope}%
\pgfsys@transformshift{2.060040in}{3.486011in}%
\pgfsys@useobject{currentmarker}{}%
\end{pgfscope}%
\begin{pgfscope}%
\pgfsys@transformshift{2.078119in}{3.466474in}%
\pgfsys@useobject{currentmarker}{}%
\end{pgfscope}%
\begin{pgfscope}%
\pgfsys@transformshift{2.098779in}{3.456244in}%
\pgfsys@useobject{currentmarker}{}%
\end{pgfscope}%
\begin{pgfscope}%
\pgfsys@transformshift{2.117793in}{3.460981in}%
\pgfsys@useobject{currentmarker}{}%
\end{pgfscope}%
\begin{pgfscope}%
\pgfsys@transformshift{2.135167in}{3.483480in}%
\pgfsys@useobject{currentmarker}{}%
\end{pgfscope}%
\begin{pgfscope}%
\pgfsys@transformshift{2.153246in}{3.531631in}%
\pgfsys@useobject{currentmarker}{}%
\end{pgfscope}%
\begin{pgfscope}%
\pgfsys@transformshift{2.173435in}{3.717096in}%
\pgfsys@useobject{currentmarker}{}%
\end{pgfscope}%
\begin{pgfscope}%
\pgfsys@transformshift{2.194566in}{4.002072in}%
\pgfsys@useobject{currentmarker}{}%
\end{pgfscope}%
\begin{pgfscope}%
\pgfsys@transformshift{2.212408in}{4.226408in}%
\pgfsys@useobject{currentmarker}{}%
\end{pgfscope}%
\begin{pgfscope}%
\pgfsys@transformshift{2.230485in}{4.187179in}%
\pgfsys@useobject{currentmarker}{}%
\end{pgfscope}%
\begin{pgfscope}%
\pgfsys@transformshift{2.251849in}{3.864467in}%
\pgfsys@useobject{currentmarker}{}%
\end{pgfscope}%
\begin{pgfscope}%
\pgfsys@transformshift{2.268753in}{3.590988in}%
\pgfsys@useobject{currentmarker}{}%
\end{pgfscope}%
\begin{pgfscope}%
\pgfsys@transformshift{2.287770in}{3.924362in}%
\pgfsys@useobject{currentmarker}{}%
\end{pgfscope}%
\begin{pgfscope}%
\pgfsys@transformshift{2.309368in}{3.570883in}%
\pgfsys@useobject{currentmarker}{}%
\end{pgfscope}%
\begin{pgfscope}%
\pgfsys@transformshift{2.330498in}{3.494783in}%
\pgfsys@useobject{currentmarker}{}%
\end{pgfscope}%
\begin{pgfscope}%
\pgfsys@transformshift{2.346932in}{3.465951in}%
\pgfsys@useobject{currentmarker}{}%
\end{pgfscope}%
\begin{pgfscope}%
\pgfsys@transformshift{2.365480in}{3.455332in}%
\pgfsys@useobject{currentmarker}{}%
\end{pgfscope}%
\begin{pgfscope}%
\pgfsys@transformshift{2.386374in}{3.465716in}%
\pgfsys@useobject{currentmarker}{}%
\end{pgfscope}%
\begin{pgfscope}%
\pgfsys@transformshift{2.403982in}{3.498077in}%
\pgfsys@useobject{currentmarker}{}%
\end{pgfscope}%
\begin{pgfscope}%
\pgfsys@transformshift{2.427929in}{3.649243in}%
\pgfsys@useobject{currentmarker}{}%
\end{pgfscope}%
\begin{pgfscope}%
\pgfsys@transformshift{2.444363in}{3.872645in}%
\pgfsys@useobject{currentmarker}{}%
\end{pgfscope}%
\begin{pgfscope}%
\pgfsys@transformshift{2.461502in}{4.186685in}%
\pgfsys@useobject{currentmarker}{}%
\end{pgfscope}%
\begin{pgfscope}%
\pgfsys@transformshift{2.482162in}{4.190191in}%
\pgfsys@useobject{currentmarker}{}%
\end{pgfscope}%
\begin{pgfscope}%
\pgfsys@transformshift{2.503056in}{3.969466in}%
\pgfsys@useobject{currentmarker}{}%
\end{pgfscope}%
\begin{pgfscope}%
\pgfsys@transformshift{2.517847in}{3.685875in}%
\pgfsys@useobject{currentmarker}{}%
\end{pgfscope}%
\begin{pgfscope}%
\pgfsys@transformshift{2.539681in}{3.513309in}%
\pgfsys@useobject{currentmarker}{}%
\end{pgfscope}%
\begin{pgfscope}%
\pgfsys@transformshift{2.560341in}{3.467293in}%
\pgfsys@useobject{currentmarker}{}%
\end{pgfscope}%
\begin{pgfscope}%
\pgfsys@transformshift{2.577714in}{3.457965in}%
\pgfsys@useobject{currentmarker}{}%
\end{pgfscope}%
\begin{pgfscope}%
\pgfsys@transformshift{2.597903in}{3.459600in}%
\pgfsys@useobject{currentmarker}{}%
\end{pgfscope}%
\begin{pgfscope}%
\pgfsys@transformshift{2.617155in}{3.479098in}%
\pgfsys@useobject{currentmarker}{}%
\end{pgfscope}%
\begin{pgfscope}%
\pgfsys@transformshift{2.634059in}{3.528249in}%
\pgfsys@useobject{currentmarker}{}%
\end{pgfscope}%
\begin{pgfscope}%
\pgfsys@transformshift{2.656362in}{3.630806in}%
\pgfsys@useobject{currentmarker}{}%
\end{pgfscope}%
\begin{pgfscope}%
\pgfsys@transformshift{2.673736in}{3.857667in}%
\pgfsys@useobject{currentmarker}{}%
\end{pgfscope}%
\begin{pgfscope}%
\pgfsys@transformshift{2.691813in}{4.162155in}%
\pgfsys@useobject{currentmarker}{}%
\end{pgfscope}%
\begin{pgfscope}%
\pgfsys@transformshift{2.715525in}{4.194330in}%
\pgfsys@useobject{currentmarker}{}%
\end{pgfscope}%
\begin{pgfscope}%
\pgfsys@transformshift{2.731021in}{3.969328in}%
\pgfsys@useobject{currentmarker}{}%
\end{pgfscope}%
\begin{pgfscope}%
\pgfsys@transformshift{2.749801in}{3.813467in}%
\pgfsys@useobject{currentmarker}{}%
\end{pgfscope}%
\begin{pgfscope}%
\pgfsys@transformshift{2.770697in}{3.589976in}%
\pgfsys@useobject{currentmarker}{}%
\end{pgfscope}%
\begin{pgfscope}%
\pgfsys@transformshift{2.788774in}{3.497740in}%
\pgfsys@useobject{currentmarker}{}%
\end{pgfscope}%
\begin{pgfscope}%
\pgfsys@transformshift{2.808494in}{3.466571in}%
\pgfsys@useobject{currentmarker}{}%
\end{pgfscope}%
\begin{pgfscope}%
\pgfsys@transformshift{2.825868in}{3.462079in}%
\pgfsys@useobject{currentmarker}{}%
\end{pgfscope}%
\begin{pgfscope}%
\pgfsys@transformshift{2.850520in}{3.456069in}%
\pgfsys@useobject{currentmarker}{}%
\end{pgfscope}%
\begin{pgfscope}%
\pgfsys@transformshift{2.865310in}{3.465225in}%
\pgfsys@useobject{currentmarker}{}%
\end{pgfscope}%
\begin{pgfscope}%
\pgfsys@transformshift{2.886673in}{3.510264in}%
\pgfsys@useobject{currentmarker}{}%
\end{pgfscope}%
\begin{pgfscope}%
\pgfsys@transformshift{2.904516in}{3.591059in}%
\pgfsys@useobject{currentmarker}{}%
\end{pgfscope}%
\begin{pgfscope}%
\pgfsys@transformshift{2.921889in}{3.741965in}%
\pgfsys@useobject{currentmarker}{}%
\end{pgfscope}%
\begin{pgfscope}%
\pgfsys@transformshift{2.942784in}{4.129040in}%
\pgfsys@useobject{currentmarker}{}%
\end{pgfscope}%
\begin{pgfscope}%
\pgfsys@transformshift{2.962506in}{4.193614in}%
\pgfsys@useobject{currentmarker}{}%
\end{pgfscope}%
\begin{pgfscope}%
\pgfsys@transformshift{2.983869in}{4.221041in}%
\pgfsys@useobject{currentmarker}{}%
\end{pgfscope}%
\begin{pgfscope}%
\pgfsys@transformshift{3.002183in}{3.974586in}%
\pgfsys@useobject{currentmarker}{}%
\end{pgfscope}%
\begin{pgfscope}%
\pgfsys@transformshift{3.019791in}{3.704435in}%
\pgfsys@useobject{currentmarker}{}%
\end{pgfscope}%
\begin{pgfscope}%
\pgfsys@transformshift{3.040685in}{3.522369in}%
\pgfsys@useobject{currentmarker}{}%
\end{pgfscope}%
\begin{pgfscope}%
\pgfsys@transformshift{3.058996in}{3.475672in}%
\pgfsys@useobject{currentmarker}{}%
\end{pgfscope}%
\begin{pgfscope}%
\pgfsys@transformshift{3.075901in}{3.464399in}%
\pgfsys@useobject{currentmarker}{}%
\end{pgfscope}%
\begin{pgfscope}%
\pgfsys@transformshift{3.097735in}{3.456124in}%
\pgfsys@useobject{currentmarker}{}%
\end{pgfscope}%
\begin{pgfscope}%
\pgfsys@transformshift{3.097264in}{3.462519in}%
\pgfsys@useobject{currentmarker}{}%
\end{pgfscope}%
\begin{pgfscope}%
\pgfsys@transformshift{3.115107in}{3.468728in}%
\pgfsys@useobject{currentmarker}{}%
\end{pgfscope}%
\begin{pgfscope}%
\pgfsys@transformshift{3.133655in}{3.493656in}%
\pgfsys@useobject{currentmarker}{}%
\end{pgfscope}%
\begin{pgfscope}%
\pgfsys@transformshift{3.154549in}{3.567392in}%
\pgfsys@useobject{currentmarker}{}%
\end{pgfscope}%
\begin{pgfscope}%
\pgfsys@transformshift{3.172626in}{3.736162in}%
\pgfsys@useobject{currentmarker}{}%
\end{pgfscope}%
\begin{pgfscope}%
\pgfsys@transformshift{3.194460in}{4.098299in}%
\pgfsys@useobject{currentmarker}{}%
\end{pgfscope}%
\begin{pgfscope}%
\pgfsys@transformshift{3.211599in}{4.251905in}%
\pgfsys@useobject{currentmarker}{}%
\end{pgfscope}%
\begin{pgfscope}%
\pgfsys@transformshift{3.229442in}{4.223260in}%
\pgfsys@useobject{currentmarker}{}%
\end{pgfscope}%
\begin{pgfscope}%
\pgfsys@transformshift{3.250805in}{4.008097in}%
\pgfsys@useobject{currentmarker}{}%
\end{pgfscope}%
\begin{pgfscope}%
\pgfsys@transformshift{3.271231in}{3.687066in}%
\pgfsys@useobject{currentmarker}{}%
\end{pgfscope}%
\begin{pgfscope}%
\pgfsys@transformshift{3.288839in}{3.550304in}%
\pgfsys@useobject{currentmarker}{}%
\end{pgfscope}%
\begin{pgfscope}%
\pgfsys@transformshift{3.306681in}{3.487222in}%
\pgfsys@useobject{currentmarker}{}%
\end{pgfscope}%
\begin{pgfscope}%
\pgfsys@transformshift{3.328515in}{3.464366in}%
\pgfsys@useobject{currentmarker}{}%
\end{pgfscope}%
\begin{pgfscope}%
\pgfsys@transformshift{3.346123in}{3.457251in}%
\pgfsys@useobject{currentmarker}{}%
\end{pgfscope}%
\begin{pgfscope}%
\pgfsys@transformshift{3.367018in}{3.464902in}%
\pgfsys@useobject{currentmarker}{}%
\end{pgfscope}%
\begin{pgfscope}%
\pgfsys@transformshift{3.385566in}{3.479765in}%
\pgfsys@useobject{currentmarker}{}%
\end{pgfscope}%
\begin{pgfscope}%
\pgfsys@transformshift{3.403174in}{3.513912in}%
\pgfsys@useobject{currentmarker}{}%
\end{pgfscope}%
\begin{pgfscope}%
\pgfsys@transformshift{3.422425in}{3.607606in}%
\pgfsys@useobject{currentmarker}{}%
\end{pgfscope}%
\begin{pgfscope}%
\pgfsys@transformshift{3.443788in}{3.824492in}%
\pgfsys@useobject{currentmarker}{}%
\end{pgfscope}%
\begin{pgfscope}%
\pgfsys@transformshift{3.461631in}{4.113628in}%
\pgfsys@useobject{currentmarker}{}%
\end{pgfscope}%
\begin{pgfscope}%
\pgfsys@transformshift{3.479004in}{4.286805in}%
\pgfsys@useobject{currentmarker}{}%
\end{pgfscope}%
\begin{pgfscope}%
\pgfsys@transformshift{3.501307in}{4.167815in}%
\pgfsys@useobject{currentmarker}{}%
\end{pgfscope}%
\begin{pgfscope}%
\pgfsys@transformshift{3.520324in}{4.010193in}%
\pgfsys@useobject{currentmarker}{}%
\end{pgfscope}%
\begin{pgfscope}%
\pgfsys@transformshift{3.537932in}{3.766444in}%
\pgfsys@useobject{currentmarker}{}%
\end{pgfscope}%
\begin{pgfscope}%
\pgfsys@transformshift{3.559297in}{3.549315in}%
\pgfsys@useobject{currentmarker}{}%
\end{pgfscope}%
\begin{pgfscope}%
\pgfsys@transformshift{3.577374in}{3.713018in}%
\pgfsys@useobject{currentmarker}{}%
\end{pgfscope}%
\begin{pgfscope}%
\pgfsys@transformshift{3.595451in}{3.991220in}%
\pgfsys@useobject{currentmarker}{}%
\end{pgfscope}%
\begin{pgfscope}%
\pgfsys@transformshift{3.616346in}{4.293371in}%
\pgfsys@useobject{currentmarker}{}%
\end{pgfscope}%
\begin{pgfscope}%
\pgfsys@transformshift{3.634425in}{4.285161in}%
\pgfsys@useobject{currentmarker}{}%
\end{pgfscope}%
\begin{pgfscope}%
\pgfsys@transformshift{3.651798in}{4.065956in}%
\pgfsys@useobject{currentmarker}{}%
\end{pgfscope}%
\begin{pgfscope}%
\pgfsys@transformshift{3.674570in}{3.756912in}%
\pgfsys@useobject{currentmarker}{}%
\end{pgfscope}%
\begin{pgfscope}%
\pgfsys@transformshift{3.692178in}{3.561355in}%
\pgfsys@useobject{currentmarker}{}%
\end{pgfscope}%
\begin{pgfscope}%
\pgfsys@transformshift{3.709786in}{3.495707in}%
\pgfsys@useobject{currentmarker}{}%
\end{pgfscope}%
\begin{pgfscope}%
\pgfsys@transformshift{3.730681in}{3.462455in}%
\pgfsys@useobject{currentmarker}{}%
\end{pgfscope}%
\begin{pgfscope}%
\pgfsys@transformshift{3.749228in}{3.460744in}%
\pgfsys@useobject{currentmarker}{}%
\end{pgfscope}%
\begin{pgfscope}%
\pgfsys@transformshift{3.768009in}{3.476447in}%
\pgfsys@useobject{currentmarker}{}%
\end{pgfscope}%
\begin{pgfscope}%
\pgfsys@transformshift{3.788669in}{3.538799in}%
\pgfsys@useobject{currentmarker}{}%
\end{pgfscope}%
\begin{pgfscope}%
\pgfsys@transformshift{3.808860in}{3.653760in}%
\pgfsys@useobject{currentmarker}{}%
\end{pgfscope}%
\begin{pgfscope}%
\pgfsys@transformshift{3.826937in}{3.862922in}%
\pgfsys@useobject{currentmarker}{}%
\end{pgfscope}%
\begin{pgfscope}%
\pgfsys@transformshift{3.844545in}{4.232499in}%
\pgfsys@useobject{currentmarker}{}%
\end{pgfscope}%
\begin{pgfscope}%
\pgfsys@transformshift{3.862624in}{4.333531in}%
\pgfsys@useobject{currentmarker}{}%
\end{pgfscope}%
\begin{pgfscope}%
\pgfsys@transformshift{3.886804in}{4.124746in}%
\pgfsys@useobject{currentmarker}{}%
\end{pgfscope}%
\begin{pgfscope}%
\pgfsys@transformshift{3.905821in}{3.793696in}%
\pgfsys@useobject{currentmarker}{}%
\end{pgfscope}%
\begin{pgfscope}%
\pgfsys@transformshift{3.923195in}{3.593925in}%
\pgfsys@useobject{currentmarker}{}%
\end{pgfscope}%
\begin{pgfscope}%
\pgfsys@transformshift{3.942211in}{3.506867in}%
\pgfsys@useobject{currentmarker}{}%
\end{pgfscope}%
\begin{pgfscope}%
\pgfsys@transformshift{3.958645in}{3.472471in}%
\pgfsys@useobject{currentmarker}{}%
\end{pgfscope}%
\begin{pgfscope}%
\pgfsys@transformshift{3.980243in}{3.459374in}%
\pgfsys@useobject{currentmarker}{}%
\end{pgfscope}%
\begin{pgfscope}%
\pgfsys@transformshift{4.000668in}{3.466774in}%
\pgfsys@useobject{currentmarker}{}%
\end{pgfscope}%
\begin{pgfscope}%
\pgfsys@transformshift{4.019451in}{3.494209in}%
\pgfsys@useobject{currentmarker}{}%
\end{pgfscope}%
\begin{pgfscope}%
\pgfsys@transformshift{4.037293in}{3.535647in}%
\pgfsys@useobject{currentmarker}{}%
\end{pgfscope}%
\begin{pgfscope}%
\pgfsys@transformshift{4.055136in}{3.640962in}%
\pgfsys@useobject{currentmarker}{}%
\end{pgfscope}%
\begin{pgfscope}%
\pgfsys@transformshift{4.076735in}{3.965979in}%
\pgfsys@useobject{currentmarker}{}%
\end{pgfscope}%
\begin{pgfscope}%
\pgfsys@transformshift{4.094812in}{4.308521in}%
\pgfsys@useobject{currentmarker}{}%
\end{pgfscope}%
\begin{pgfscope}%
\pgfsys@transformshift{4.115472in}{4.351224in}%
\pgfsys@useobject{currentmarker}{}%
\end{pgfscope}%
\begin{pgfscope}%
\pgfsys@transformshift{4.137072in}{4.195754in}%
\pgfsys@useobject{currentmarker}{}%
\end{pgfscope}%
\begin{pgfscope}%
\pgfsys@transformshift{4.153271in}{3.935484in}%
\pgfsys@useobject{currentmarker}{}%
\end{pgfscope}%
\begin{pgfscope}%
\pgfsys@transformshift{4.172757in}{3.673472in}%
\pgfsys@useobject{currentmarker}{}%
\end{pgfscope}%
\begin{pgfscope}%
\pgfsys@transformshift{4.191070in}{3.538589in}%
\pgfsys@useobject{currentmarker}{}%
\end{pgfscope}%
\begin{pgfscope}%
\pgfsys@transformshift{4.212668in}{3.487252in}%
\pgfsys@useobject{currentmarker}{}%
\end{pgfscope}%
\begin{pgfscope}%
\pgfsys@transformshift{4.230511in}{3.465294in}%
\pgfsys@useobject{currentmarker}{}%
\end{pgfscope}%
\begin{pgfscope}%
\pgfsys@transformshift{4.247884in}{3.460998in}%
\pgfsys@useobject{currentmarker}{}%
\end{pgfscope}%
\begin{pgfscope}%
\pgfsys@transformshift{4.265727in}{3.473219in}%
\pgfsys@useobject{currentmarker}{}%
\end{pgfscope}%
\begin{pgfscope}%
\pgfsys@transformshift{4.289673in}{3.519147in}%
\pgfsys@useobject{currentmarker}{}%
\end{pgfscope}%
\begin{pgfscope}%
\pgfsys@transformshift{4.308221in}{3.552216in}%
\pgfsys@useobject{currentmarker}{}%
\end{pgfscope}%
\begin{pgfscope}%
\pgfsys@transformshift{4.325594in}{3.685331in}%
\pgfsys@useobject{currentmarker}{}%
\end{pgfscope}%
\begin{pgfscope}%
\pgfsys@transformshift{4.346958in}{4.050014in}%
\pgfsys@useobject{currentmarker}{}%
\end{pgfscope}%
\begin{pgfscope}%
\pgfsys@transformshift{4.365974in}{4.366688in}%
\pgfsys@useobject{currentmarker}{}%
\end{pgfscope}%
\begin{pgfscope}%
\pgfsys@transformshift{4.383113in}{4.410149in}%
\pgfsys@useobject{currentmarker}{}%
\end{pgfscope}%
\begin{pgfscope}%
\pgfsys@transformshift{4.403773in}{4.190519in}%
\pgfsys@useobject{currentmarker}{}%
\end{pgfscope}%
\begin{pgfscope}%
\pgfsys@transformshift{4.419738in}{3.911133in}%
\pgfsys@useobject{currentmarker}{}%
\end{pgfscope}%
\begin{pgfscope}%
\pgfsys@transformshift{4.440867in}{3.711918in}%
\pgfsys@useobject{currentmarker}{}%
\end{pgfscope}%
\begin{pgfscope}%
\pgfsys@transformshift{4.461996in}{3.557513in}%
\pgfsys@useobject{currentmarker}{}%
\end{pgfscope}%
\begin{pgfscope}%
\pgfsys@transformshift{4.480075in}{3.495425in}%
\pgfsys@useobject{currentmarker}{}%
\end{pgfscope}%
\begin{pgfscope}%
\pgfsys@transformshift{4.483830in}{3.490389in}%
\pgfsys@useobject{currentmarker}{}%
\end{pgfscope}%
\begin{pgfscope}%
\pgfsys@transformshift{4.475378in}{3.521671in}%
\pgfsys@useobject{currentmarker}{}%
\end{pgfscope}%
\begin{pgfscope}%
\pgfsys@transformshift{4.455189in}{3.684284in}%
\pgfsys@useobject{currentmarker}{}%
\end{pgfscope}%
\begin{pgfscope}%
\pgfsys@transformshift{4.437112in}{4.063367in}%
\pgfsys@useobject{currentmarker}{}%
\end{pgfscope}%
\begin{pgfscope}%
\pgfsys@transformshift{4.418798in}{4.380005in}%
\pgfsys@useobject{currentmarker}{}%
\end{pgfscope}%
\begin{pgfscope}%
\pgfsys@transformshift{4.396495in}{4.252139in}%
\pgfsys@useobject{currentmarker}{}%
\end{pgfscope}%
\begin{pgfscope}%
\pgfsys@transformshift{4.377948in}{3.783093in}%
\pgfsys@useobject{currentmarker}{}%
\end{pgfscope}%
\begin{pgfscope}%
\pgfsys@transformshift{4.359636in}{3.562864in}%
\pgfsys@useobject{currentmarker}{}%
\end{pgfscope}%
\begin{pgfscope}%
\pgfsys@transformshift{4.338271in}{3.482189in}%
\pgfsys@useobject{currentmarker}{}%
\end{pgfscope}%
\begin{pgfscope}%
\pgfsys@transformshift{4.320194in}{3.459375in}%
\pgfsys@useobject{currentmarker}{}%
\end{pgfscope}%
\begin{pgfscope}%
\pgfsys@transformshift{4.301648in}{3.474245in}%
\pgfsys@useobject{currentmarker}{}%
\end{pgfscope}%
\begin{pgfscope}%
\pgfsys@transformshift{4.284509in}{3.533694in}%
\pgfsys@useobject{currentmarker}{}%
\end{pgfscope}%
\begin{pgfscope}%
\pgfsys@transformshift{4.263380in}{3.744762in}%
\pgfsys@useobject{currentmarker}{}%
\end{pgfscope}%
\begin{pgfscope}%
\pgfsys@transformshift{4.245772in}{4.140144in}%
\pgfsys@useobject{currentmarker}{}%
\end{pgfscope}%
\begin{pgfscope}%
\pgfsys@transformshift{4.225581in}{4.377140in}%
\pgfsys@useobject{currentmarker}{}%
\end{pgfscope}%
\begin{pgfscope}%
\pgfsys@transformshift{4.204452in}{4.065751in}%
\pgfsys@useobject{currentmarker}{}%
\end{pgfscope}%
\begin{pgfscope}%
\pgfsys@transformshift{4.186373in}{3.677122in}%
\pgfsys@useobject{currentmarker}{}%
\end{pgfscope}%
\begin{pgfscope}%
\pgfsys@transformshift{4.164541in}{3.509560in}%
\pgfsys@useobject{currentmarker}{}%
\end{pgfscope}%
\begin{pgfscope}%
\pgfsys@transformshift{4.147402in}{3.488504in}%
\pgfsys@useobject{currentmarker}{}%
\end{pgfscope}%
\begin{pgfscope}%
\pgfsys@transformshift{4.128854in}{3.581944in}%
\pgfsys@useobject{currentmarker}{}%
\end{pgfscope}%
\begin{pgfscope}%
\pgfsys@transformshift{4.107960in}{3.929745in}%
\pgfsys@useobject{currentmarker}{}%
\end{pgfscope}%
\begin{pgfscope}%
\pgfsys@transformshift{4.090117in}{4.267991in}%
\pgfsys@useobject{currentmarker}{}%
\end{pgfscope}%
\begin{pgfscope}%
\pgfsys@transformshift{4.070163in}{4.292144in}%
\pgfsys@useobject{currentmarker}{}%
\end{pgfscope}%
\begin{pgfscope}%
\pgfsys@transformshift{4.052789in}{3.875246in}%
\pgfsys@useobject{currentmarker}{}%
\end{pgfscope}%
\begin{pgfscope}%
\pgfsys@transformshift{4.032833in}{3.568236in}%
\pgfsys@useobject{currentmarker}{}%
\end{pgfscope}%
\begin{pgfscope}%
\pgfsys@transformshift{4.012407in}{3.482043in}%
\pgfsys@useobject{currentmarker}{}%
\end{pgfscope}%
\begin{pgfscope}%
\pgfsys@transformshift{3.994565in}{3.458709in}%
\pgfsys@useobject{currentmarker}{}%
\end{pgfscope}%
\begin{pgfscope}%
\pgfsys@transformshift{3.974139in}{3.467447in}%
\pgfsys@useobject{currentmarker}{}%
\end{pgfscope}%
\begin{pgfscope}%
\pgfsys@transformshift{3.953479in}{3.521093in}%
\pgfsys@useobject{currentmarker}{}%
\end{pgfscope}%
\begin{pgfscope}%
\pgfsys@transformshift{3.936342in}{3.673994in}%
\pgfsys@useobject{currentmarker}{}%
\end{pgfscope}%
\begin{pgfscope}%
\pgfsys@transformshift{3.918500in}{4.041502in}%
\pgfsys@useobject{currentmarker}{}%
\end{pgfscope}%
\begin{pgfscope}%
\pgfsys@transformshift{3.899483in}{4.303555in}%
\pgfsys@useobject{currentmarker}{}%
\end{pgfscope}%
\begin{pgfscope}%
\pgfsys@transformshift{3.880232in}{4.163891in}%
\pgfsys@useobject{currentmarker}{}%
\end{pgfscope}%
\begin{pgfscope}%
\pgfsys@transformshift{3.856754in}{3.673255in}%
\pgfsys@useobject{currentmarker}{}%
\end{pgfscope}%
\begin{pgfscope}%
\pgfsys@transformshift{3.838675in}{3.528806in}%
\pgfsys@useobject{currentmarker}{}%
\end{pgfscope}%
\begin{pgfscope}%
\pgfsys@transformshift{3.821067in}{3.473546in}%
\pgfsys@useobject{currentmarker}{}%
\end{pgfscope}%
\begin{pgfscope}%
\pgfsys@transformshift{3.801113in}{3.457286in}%
\pgfsys@useobject{currentmarker}{}%
\end{pgfscope}%
\begin{pgfscope}%
\pgfsys@transformshift{3.783036in}{3.467772in}%
\pgfsys@useobject{currentmarker}{}%
\end{pgfscope}%
\begin{pgfscope}%
\pgfsys@transformshift{3.761436in}{3.538521in}%
\pgfsys@useobject{currentmarker}{}%
\end{pgfscope}%
\begin{pgfscope}%
\pgfsys@transformshift{3.745706in}{3.669512in}%
\pgfsys@useobject{currentmarker}{}%
\end{pgfscope}%
\begin{pgfscope}%
\pgfsys@transformshift{3.721994in}{4.112577in}%
\pgfsys@useobject{currentmarker}{}%
\end{pgfscope}%
\begin{pgfscope}%
\pgfsys@transformshift{3.707909in}{4.273908in}%
\pgfsys@useobject{currentmarker}{}%
\end{pgfscope}%
\begin{pgfscope}%
\pgfsys@transformshift{3.685840in}{4.159863in}%
\pgfsys@useobject{currentmarker}{}%
\end{pgfscope}%
\begin{pgfscope}%
\pgfsys@transformshift{3.666118in}{3.732880in}%
\pgfsys@useobject{currentmarker}{}%
\end{pgfscope}%
\begin{pgfscope}%
\pgfsys@transformshift{3.648510in}{3.550012in}%
\pgfsys@useobject{currentmarker}{}%
\end{pgfscope}%
\begin{pgfscope}%
\pgfsys@transformshift{3.628321in}{3.495186in}%
\pgfsys@useobject{currentmarker}{}%
\end{pgfscope}%
\begin{pgfscope}%
\pgfsys@transformshift{3.610007in}{3.462921in}%
\pgfsys@useobject{currentmarker}{}%
\end{pgfscope}%
\begin{pgfscope}%
\pgfsys@transformshift{3.593574in}{3.456812in}%
\pgfsys@useobject{currentmarker}{}%
\end{pgfscope}%
\begin{pgfscope}%
\pgfsys@transformshift{3.570096in}{3.472015in}%
\pgfsys@useobject{currentmarker}{}%
\end{pgfscope}%
\begin{pgfscope}%
\pgfsys@transformshift{3.552254in}{3.507719in}%
\pgfsys@useobject{currentmarker}{}%
\end{pgfscope}%
\begin{pgfscope}%
\pgfsys@transformshift{3.532063in}{3.635885in}%
\pgfsys@useobject{currentmarker}{}%
\end{pgfscope}%
\begin{pgfscope}%
\pgfsys@transformshift{3.515394in}{3.900914in}%
\pgfsys@useobject{currentmarker}{}%
\end{pgfscope}%
\begin{pgfscope}%
\pgfsys@transformshift{3.494969in}{4.180452in}%
\pgfsys@useobject{currentmarker}{}%
\end{pgfscope}%
\begin{pgfscope}%
\pgfsys@transformshift{3.477595in}{4.264365in}%
\pgfsys@useobject{currentmarker}{}%
\end{pgfscope}%
\begin{pgfscope}%
\pgfsys@transformshift{3.454823in}{3.948351in}%
\pgfsys@useobject{currentmarker}{}%
\end{pgfscope}%
\begin{pgfscope}%
\pgfsys@transformshift{3.436276in}{3.633132in}%
\pgfsys@useobject{currentmarker}{}%
\end{pgfscope}%
\begin{pgfscope}%
\pgfsys@transformshift{3.418668in}{3.552657in}%
\pgfsys@useobject{currentmarker}{}%
\end{pgfscope}%
\begin{pgfscope}%
\pgfsys@transformshift{3.397539in}{3.487407in}%
\pgfsys@useobject{currentmarker}{}%
\end{pgfscope}%
\begin{pgfscope}%
\pgfsys@transformshift{3.377819in}{3.460297in}%
\pgfsys@useobject{currentmarker}{}%
\end{pgfscope}%
\begin{pgfscope}%
\pgfsys@transformshift{3.361148in}{3.456423in}%
\pgfsys@useobject{currentmarker}{}%
\end{pgfscope}%
\begin{pgfscope}%
\pgfsys@transformshift{3.340254in}{3.473305in}%
\pgfsys@useobject{currentmarker}{}%
\end{pgfscope}%
\begin{pgfscope}%
\pgfsys@transformshift{3.321708in}{3.531391in}%
\pgfsys@useobject{currentmarker}{}%
\end{pgfscope}%
\begin{pgfscope}%
\pgfsys@transformshift{3.301283in}{3.739087in}%
\pgfsys@useobject{currentmarker}{}%
\end{pgfscope}%
\begin{pgfscope}%
\pgfsys@transformshift{3.283204in}{4.052667in}%
\pgfsys@useobject{currentmarker}{}%
\end{pgfscope}%
\begin{pgfscope}%
\pgfsys@transformshift{3.263015in}{4.250086in}%
\pgfsys@useobject{currentmarker}{}%
\end{pgfscope}%
\begin{pgfscope}%
\pgfsys@transformshift{3.242355in}{4.075522in}%
\pgfsys@useobject{currentmarker}{}%
\end{pgfscope}%
\begin{pgfscope}%
\pgfsys@transformshift{3.224041in}{3.739484in}%
\pgfsys@useobject{currentmarker}{}%
\end{pgfscope}%
\begin{pgfscope}%
\pgfsys@transformshift{3.205025in}{4.072723in}%
\pgfsys@useobject{currentmarker}{}%
\end{pgfscope}%
\begin{pgfscope}%
\pgfsys@transformshift{3.186948in}{3.719222in}%
\pgfsys@useobject{currentmarker}{}%
\end{pgfscope}%
\begin{pgfscope}%
\pgfsys@transformshift{3.168636in}{3.543680in}%
\pgfsys@useobject{currentmarker}{}%
\end{pgfscope}%
\begin{pgfscope}%
\pgfsys@transformshift{3.168871in}{3.500396in}%
\pgfsys@useobject{currentmarker}{}%
\end{pgfscope}%
\begin{pgfscope}%
\pgfsys@transformshift{3.150323in}{3.487609in}%
\pgfsys@useobject{currentmarker}{}%
\end{pgfscope}%
\begin{pgfscope}%
\pgfsys@transformshift{3.129897in}{3.460337in}%
\pgfsys@useobject{currentmarker}{}%
\end{pgfscope}%
\begin{pgfscope}%
\pgfsys@transformshift{3.108769in}{3.457435in}%
\pgfsys@useobject{currentmarker}{}%
\end{pgfscope}%
\begin{pgfscope}%
\pgfsys@transformshift{3.091395in}{3.476078in}%
\pgfsys@useobject{currentmarker}{}%
\end{pgfscope}%
\begin{pgfscope}%
\pgfsys@transformshift{3.070970in}{3.526120in}%
\pgfsys@useobject{currentmarker}{}%
\end{pgfscope}%
\begin{pgfscope}%
\pgfsys@transformshift{3.052893in}{3.685132in}%
\pgfsys@useobject{currentmarker}{}%
\end{pgfscope}%
\begin{pgfscope}%
\pgfsys@transformshift{3.031998in}{4.069478in}%
\pgfsys@useobject{currentmarker}{}%
\end{pgfscope}%
\begin{pgfscope}%
\pgfsys@transformshift{3.013216in}{4.245940in}%
\pgfsys@useobject{currentmarker}{}%
\end{pgfscope}%
\begin{pgfscope}%
\pgfsys@transformshift{2.995139in}{4.145584in}%
\pgfsys@useobject{currentmarker}{}%
\end{pgfscope}%
\begin{pgfscope}%
\pgfsys@transformshift{2.973305in}{3.743163in}%
\pgfsys@useobject{currentmarker}{}%
\end{pgfscope}%
\begin{pgfscope}%
\pgfsys@transformshift{2.954759in}{3.552891in}%
\pgfsys@useobject{currentmarker}{}%
\end{pgfscope}%
\begin{pgfscope}%
\pgfsys@transformshift{2.935977in}{3.484808in}%
\pgfsys@useobject{currentmarker}{}%
\end{pgfscope}%
\begin{pgfscope}%
\pgfsys@transformshift{2.920715in}{3.467458in}%
\pgfsys@useobject{currentmarker}{}%
\end{pgfscope}%
\begin{pgfscope}%
\pgfsys@transformshift{2.898647in}{3.456474in}%
\pgfsys@useobject{currentmarker}{}%
\end{pgfscope}%
\begin{pgfscope}%
\pgfsys@transformshift{2.880335in}{3.454848in}%
\pgfsys@useobject{currentmarker}{}%
\end{pgfscope}%
\begin{pgfscope}%
\pgfsys@transformshift{2.857798in}{3.454302in}%
\pgfsys@useobject{currentmarker}{}%
\end{pgfscope}%
\begin{pgfscope}%
\pgfsys@transformshift{2.840424in}{3.464815in}%
\pgfsys@useobject{currentmarker}{}%
\end{pgfscope}%
\begin{pgfscope}%
\pgfsys@transformshift{2.820938in}{3.504016in}%
\pgfsys@useobject{currentmarker}{}%
\end{pgfscope}%
\begin{pgfscope}%
\pgfsys@transformshift{2.804739in}{3.591293in}%
\pgfsys@useobject{currentmarker}{}%
\end{pgfscope}%
\begin{pgfscope}%
\pgfsys@transformshift{2.784314in}{3.933307in}%
\pgfsys@useobject{currentmarker}{}%
\end{pgfscope}%
\begin{pgfscope}%
\pgfsys@transformshift{2.762948in}{4.193518in}%
\pgfsys@useobject{currentmarker}{}%
\end{pgfscope}%
\begin{pgfscope}%
\pgfsys@transformshift{2.744402in}{4.196653in}%
\pgfsys@useobject{currentmarker}{}%
\end{pgfscope}%
\begin{pgfscope}%
\pgfsys@transformshift{2.725151in}{3.882354in}%
\pgfsys@useobject{currentmarker}{}%
\end{pgfscope}%
\begin{pgfscope}%
\pgfsys@transformshift{2.705195in}{3.637906in}%
\pgfsys@useobject{currentmarker}{}%
\end{pgfscope}%
\begin{pgfscope}%
\pgfsys@transformshift{2.686649in}{3.515148in}%
\pgfsys@useobject{currentmarker}{}%
\end{pgfscope}%
\begin{pgfscope}%
\pgfsys@transformshift{2.668335in}{3.471532in}%
\pgfsys@useobject{currentmarker}{}%
\end{pgfscope}%
\begin{pgfscope}%
\pgfsys@transformshift{2.649790in}{3.456254in}%
\pgfsys@useobject{currentmarker}{}%
\end{pgfscope}%
\begin{pgfscope}%
\pgfsys@transformshift{2.631242in}{3.459363in}%
\pgfsys@useobject{currentmarker}{}%
\end{pgfscope}%
\begin{pgfscope}%
\pgfsys@transformshift{2.609876in}{3.480412in}%
\pgfsys@useobject{currentmarker}{}%
\end{pgfscope}%
\begin{pgfscope}%
\pgfsys@transformshift{2.590156in}{3.536795in}%
\pgfsys@useobject{currentmarker}{}%
\end{pgfscope}%
\begin{pgfscope}%
\pgfsys@transformshift{2.589453in}{3.624875in}%
\pgfsys@useobject{currentmarker}{}%
\end{pgfscope}%
\begin{pgfscope}%
\pgfsys@transformshift{2.571609in}{3.705928in}%
\pgfsys@useobject{currentmarker}{}%
\end{pgfscope}%
\begin{pgfscope}%
\pgfsys@transformshift{2.552357in}{4.056410in}%
\pgfsys@useobject{currentmarker}{}%
\end{pgfscope}%
\begin{pgfscope}%
\pgfsys@transformshift{2.532872in}{4.226793in}%
\pgfsys@useobject{currentmarker}{}%
\end{pgfscope}%
\begin{pgfscope}%
\pgfsys@transformshift{2.512212in}{4.136493in}%
\pgfsys@useobject{currentmarker}{}%
\end{pgfscope}%
\begin{pgfscope}%
\pgfsys@transformshift{2.495778in}{3.789839in}%
\pgfsys@useobject{currentmarker}{}%
\end{pgfscope}%
\begin{pgfscope}%
\pgfsys@transformshift{2.476527in}{3.569211in}%
\pgfsys@useobject{currentmarker}{}%
\end{pgfscope}%
\begin{pgfscope}%
\pgfsys@transformshift{2.459153in}{3.512779in}%
\pgfsys@useobject{currentmarker}{}%
\end{pgfscope}%
\begin{pgfscope}%
\pgfsys@transformshift{2.436381in}{3.466519in}%
\pgfsys@useobject{currentmarker}{}%
\end{pgfscope}%
\begin{pgfscope}%
\pgfsys@transformshift{2.416659in}{3.454903in}%
\pgfsys@useobject{currentmarker}{}%
\end{pgfscope}%
\begin{pgfscope}%
\pgfsys@transformshift{2.397408in}{3.455986in}%
\pgfsys@useobject{currentmarker}{}%
\end{pgfscope}%
\begin{pgfscope}%
\pgfsys@transformshift{2.379096in}{3.468643in}%
\pgfsys@useobject{currentmarker}{}%
\end{pgfscope}%
\begin{pgfscope}%
\pgfsys@transformshift{2.359845in}{3.512238in}%
\pgfsys@useobject{currentmarker}{}%
\end{pgfscope}%
\begin{pgfscope}%
\pgfsys@transformshift{2.341766in}{3.648285in}%
\pgfsys@useobject{currentmarker}{}%
\end{pgfscope}%
\begin{pgfscope}%
\pgfsys@transformshift{2.319229in}{3.996942in}%
\pgfsys@useobject{currentmarker}{}%
\end{pgfscope}%
\begin{pgfscope}%
\pgfsys@transformshift{2.302092in}{4.203602in}%
\pgfsys@useobject{currentmarker}{}%
\end{pgfscope}%
\begin{pgfscope}%
\pgfsys@transformshift{2.285187in}{4.201589in}%
\pgfsys@useobject{currentmarker}{}%
\end{pgfscope}%
\begin{pgfscope}%
\pgfsys@transformshift{2.263824in}{3.956673in}%
\pgfsys@useobject{currentmarker}{}%
\end{pgfscope}%
\begin{pgfscope}%
\pgfsys@transformshift{2.244336in}{3.665171in}%
\pgfsys@useobject{currentmarker}{}%
\end{pgfscope}%
\begin{pgfscope}%
\pgfsys@transformshift{2.226494in}{3.539271in}%
\pgfsys@useobject{currentmarker}{}%
\end{pgfscope}%
\begin{pgfscope}%
\pgfsys@transformshift{2.205130in}{3.478497in}%
\pgfsys@useobject{currentmarker}{}%
\end{pgfscope}%
\begin{pgfscope}%
\pgfsys@transformshift{2.186582in}{3.462885in}%
\pgfsys@useobject{currentmarker}{}%
\end{pgfscope}%
\begin{pgfscope}%
\pgfsys@transformshift{2.168505in}{3.454805in}%
\pgfsys@useobject{currentmarker}{}%
\end{pgfscope}%
\begin{pgfscope}%
\pgfsys@transformshift{2.148785in}{3.464645in}%
\pgfsys@useobject{currentmarker}{}%
\end{pgfscope}%
\begin{pgfscope}%
\pgfsys@transformshift{2.127420in}{3.500813in}%
\pgfsys@useobject{currentmarker}{}%
\end{pgfscope}%
\begin{pgfscope}%
\pgfsys@transformshift{2.108872in}{3.584098in}%
\pgfsys@useobject{currentmarker}{}%
\end{pgfscope}%
\begin{pgfscope}%
\pgfsys@transformshift{2.090326in}{3.807676in}%
\pgfsys@useobject{currentmarker}{}%
\end{pgfscope}%
\begin{pgfscope}%
\pgfsys@transformshift{2.072013in}{4.053736in}%
\pgfsys@useobject{currentmarker}{}%
\end{pgfscope}%
\begin{pgfscope}%
\pgfsys@transformshift{2.053233in}{4.227856in}%
\pgfsys@useobject{currentmarker}{}%
\end{pgfscope}%
\begin{pgfscope}%
\pgfsys@transformshift{2.031633in}{4.176213in}%
\pgfsys@useobject{currentmarker}{}%
\end{pgfscope}%
\begin{pgfscope}%
\pgfsys@transformshift{2.013085in}{4.023054in}%
\pgfsys@useobject{currentmarker}{}%
\end{pgfscope}%
\begin{pgfscope}%
\pgfsys@transformshift{1.997826in}{3.722803in}%
\pgfsys@useobject{currentmarker}{}%
\end{pgfscope}%
\begin{pgfscope}%
\pgfsys@transformshift{1.975288in}{3.542595in}%
\pgfsys@useobject{currentmarker}{}%
\end{pgfscope}%
\begin{pgfscope}%
\pgfsys@transformshift{1.957209in}{3.492477in}%
\pgfsys@useobject{currentmarker}{}%
\end{pgfscope}%
\begin{pgfscope}%
\pgfsys@transformshift{1.934672in}{3.463615in}%
\pgfsys@useobject{currentmarker}{}%
\end{pgfscope}%
\begin{pgfscope}%
\pgfsys@transformshift{1.919412in}{3.456220in}%
\pgfsys@useobject{currentmarker}{}%
\end{pgfscope}%
\begin{pgfscope}%
\pgfsys@transformshift{1.898283in}{3.460719in}%
\pgfsys@useobject{currentmarker}{}%
\end{pgfscope}%
\begin{pgfscope}%
\pgfsys@transformshift{1.879501in}{3.473747in}%
\pgfsys@useobject{currentmarker}{}%
\end{pgfscope}%
\begin{pgfscope}%
\pgfsys@transformshift{1.858606in}{3.475961in}%
\pgfsys@useobject{currentmarker}{}%
\end{pgfscope}%
\begin{pgfscope}%
\pgfsys@transformshift{1.839824in}{3.519306in}%
\pgfsys@useobject{currentmarker}{}%
\end{pgfscope}%
\begin{pgfscope}%
\pgfsys@transformshift{1.822451in}{3.639504in}%
\pgfsys@useobject{currentmarker}{}%
\end{pgfscope}%
\begin{pgfscope}%
\pgfsys@transformshift{1.801322in}{3.968321in}%
\pgfsys@useobject{currentmarker}{}%
\end{pgfscope}%
\begin{pgfscope}%
\pgfsys@transformshift{1.784183in}{4.194713in}%
\pgfsys@useobject{currentmarker}{}%
\end{pgfscope}%
\begin{pgfscope}%
\pgfsys@transformshift{1.762114in}{4.246058in}%
\pgfsys@useobject{currentmarker}{}%
\end{pgfscope}%
\begin{pgfscope}%
\pgfsys@transformshift{1.746620in}{4.110166in}%
\pgfsys@useobject{currentmarker}{}%
\end{pgfscope}%
\begin{pgfscope}%
\pgfsys@transformshift{1.724786in}{3.784546in}%
\pgfsys@useobject{currentmarker}{}%
\end{pgfscope}%
\begin{pgfscope}%
\pgfsys@transformshift{1.706004in}{3.594892in}%
\pgfsys@useobject{currentmarker}{}%
\end{pgfscope}%
\begin{pgfscope}%
\pgfsys@transformshift{1.687927in}{3.505801in}%
\pgfsys@useobject{currentmarker}{}%
\end{pgfscope}%
\begin{pgfscope}%
\pgfsys@transformshift{1.669144in}{3.472376in}%
\pgfsys@useobject{currentmarker}{}%
\end{pgfscope}%
\begin{pgfscope}%
\pgfsys@transformshift{1.647781in}{3.457721in}%
\pgfsys@useobject{currentmarker}{}%
\end{pgfscope}%
\begin{pgfscope}%
\pgfsys@transformshift{1.628528in}{3.457631in}%
\pgfsys@useobject{currentmarker}{}%
\end{pgfscope}%
\begin{pgfscope}%
\pgfsys@transformshift{1.610216in}{3.464865in}%
\pgfsys@useobject{currentmarker}{}%
\end{pgfscope}%
\begin{pgfscope}%
\pgfsys@transformshift{1.590965in}{4.206462in}%
\pgfsys@useobject{currentmarker}{}%
\end{pgfscope}%
\begin{pgfscope}%
\pgfsys@transformshift{1.569600in}{3.796545in}%
\pgfsys@useobject{currentmarker}{}%
\end{pgfscope}%
\begin{pgfscope}%
\pgfsys@transformshift{1.547768in}{3.548770in}%
\pgfsys@useobject{currentmarker}{}%
\end{pgfscope}%
\begin{pgfscope}%
\pgfsys@transformshift{1.533915in}{3.517819in}%
\pgfsys@useobject{currentmarker}{}%
\end{pgfscope}%
\begin{pgfscope}%
\pgfsys@transformshift{1.514429in}{3.475766in}%
\pgfsys@useobject{currentmarker}{}%
\end{pgfscope}%
\begin{pgfscope}%
\pgfsys@transformshift{1.495647in}{3.456232in}%
\pgfsys@useobject{currentmarker}{}%
\end{pgfscope}%
\begin{pgfscope}%
\pgfsys@transformshift{1.476865in}{3.460522in}%
\pgfsys@useobject{currentmarker}{}%
\end{pgfscope}%
\begin{pgfscope}%
\pgfsys@transformshift{1.455501in}{3.487597in}%
\pgfsys@useobject{currentmarker}{}%
\end{pgfscope}%
\begin{pgfscope}%
\pgfsys@transformshift{1.436485in}{3.563667in}%
\pgfsys@useobject{currentmarker}{}%
\end{pgfscope}%
\begin{pgfscope}%
\pgfsys@transformshift{1.418173in}{3.771823in}%
\pgfsys@useobject{currentmarker}{}%
\end{pgfscope}%
\begin{pgfscope}%
\pgfsys@transformshift{1.399625in}{4.123181in}%
\pgfsys@useobject{currentmarker}{}%
\end{pgfscope}%
\begin{pgfscope}%
\pgfsys@transformshift{1.381549in}{4.271258in}%
\pgfsys@useobject{currentmarker}{}%
\end{pgfscope}%
\begin{pgfscope}%
\pgfsys@transformshift{1.359480in}{4.146973in}%
\pgfsys@useobject{currentmarker}{}%
\end{pgfscope}%
\begin{pgfscope}%
\pgfsys@transformshift{1.340698in}{3.777270in}%
\pgfsys@useobject{currentmarker}{}%
\end{pgfscope}%
\begin{pgfscope}%
\pgfsys@transformshift{1.322855in}{3.587881in}%
\pgfsys@useobject{currentmarker}{}%
\end{pgfscope}%
\begin{pgfscope}%
\pgfsys@transformshift{1.301255in}{3.489787in}%
\pgfsys@useobject{currentmarker}{}%
\end{pgfscope}%
\begin{pgfscope}%
\pgfsys@transformshift{1.282709in}{3.468109in}%
\pgfsys@useobject{currentmarker}{}%
\end{pgfscope}%
\begin{pgfscope}%
\pgfsys@transformshift{1.263927in}{3.462189in}%
\pgfsys@useobject{currentmarker}{}%
\end{pgfscope}%
\begin{pgfscope}%
\pgfsys@transformshift{1.245616in}{3.456938in}%
\pgfsys@useobject{currentmarker}{}%
\end{pgfscope}%
\begin{pgfscope}%
\pgfsys@transformshift{1.226833in}{3.462008in}%
\pgfsys@useobject{currentmarker}{}%
\end{pgfscope}%
\begin{pgfscope}%
\pgfsys@transformshift{1.205234in}{3.489614in}%
\pgfsys@useobject{currentmarker}{}%
\end{pgfscope}%
\begin{pgfscope}%
\pgfsys@transformshift{1.187157in}{3.561414in}%
\pgfsys@useobject{currentmarker}{}%
\end{pgfscope}%
\begin{pgfscope}%
\pgfsys@transformshift{1.168375in}{3.700113in}%
\pgfsys@useobject{currentmarker}{}%
\end{pgfscope}%
\begin{pgfscope}%
\pgfsys@transformshift{1.147011in}{4.081890in}%
\pgfsys@useobject{currentmarker}{}%
\end{pgfscope}%
\begin{pgfscope}%
\pgfsys@transformshift{1.129638in}{4.297568in}%
\pgfsys@useobject{currentmarker}{}%
\end{pgfscope}%
\begin{pgfscope}%
\pgfsys@transformshift{1.112030in}{4.276530in}%
\pgfsys@useobject{currentmarker}{}%
\end{pgfscope}%
\begin{pgfscope}%
\pgfsys@transformshift{1.090430in}{3.931748in}%
\pgfsys@useobject{currentmarker}{}%
\end{pgfscope}%
\begin{pgfscope}%
\pgfsys@transformshift{1.069536in}{3.629673in}%
\pgfsys@useobject{currentmarker}{}%
\end{pgfscope}%
\begin{pgfscope}%
\pgfsys@transformshift{1.054276in}{3.534552in}%
\pgfsys@useobject{currentmarker}{}%
\end{pgfscope}%
\begin{pgfscope}%
\pgfsys@transformshift{1.035494in}{3.484864in}%
\pgfsys@useobject{currentmarker}{}%
\end{pgfscope}%
\begin{pgfscope}%
\pgfsys@transformshift{1.016946in}{3.467546in}%
\pgfsys@useobject{currentmarker}{}%
\end{pgfscope}%
\begin{pgfscope}%
\pgfsys@transformshift{0.995348in}{3.458627in}%
\pgfsys@useobject{currentmarker}{}%
\end{pgfscope}%
\begin{pgfscope}%
\pgfsys@transformshift{0.975861in}{3.459871in}%
\pgfsys@useobject{currentmarker}{}%
\end{pgfscope}%
\begin{pgfscope}%
\pgfsys@transformshift{0.958489in}{3.479005in}%
\pgfsys@useobject{currentmarker}{}%
\end{pgfscope}%
\begin{pgfscope}%
\pgfsys@transformshift{0.936420in}{3.559293in}%
\pgfsys@useobject{currentmarker}{}%
\end{pgfscope}%
\begin{pgfscope}%
\pgfsys@transformshift{0.917403in}{3.732125in}%
\pgfsys@useobject{currentmarker}{}%
\end{pgfscope}%
\begin{pgfscope}%
\pgfsys@transformshift{0.899795in}{4.043904in}%
\pgfsys@useobject{currentmarker}{}%
\end{pgfscope}%
\begin{pgfscope}%
\pgfsys@transformshift{0.877727in}{3.861973in}%
\pgfsys@useobject{currentmarker}{}%
\end{pgfscope}%
\begin{pgfscope}%
\pgfsys@transformshift{0.859650in}{4.202126in}%
\pgfsys@useobject{currentmarker}{}%
\end{pgfscope}%
\begin{pgfscope}%
\pgfsys@transformshift{0.843919in}{4.338142in}%
\pgfsys@useobject{currentmarker}{}%
\end{pgfscope}%
\begin{pgfscope}%
\pgfsys@transformshift{0.822791in}{4.247701in}%
\pgfsys@useobject{currentmarker}{}%
\end{pgfscope}%
\begin{pgfscope}%
\pgfsys@transformshift{0.803774in}{3.857875in}%
\pgfsys@useobject{currentmarker}{}%
\end{pgfscope}%
\begin{pgfscope}%
\pgfsys@transformshift{0.784052in}{3.628234in}%
\pgfsys@useobject{currentmarker}{}%
\end{pgfscope}%
\begin{pgfscope}%
\pgfsys@transformshift{0.763157in}{3.533133in}%
\pgfsys@useobject{currentmarker}{}%
\end{pgfscope}%
\begin{pgfscope}%
\pgfsys@transformshift{0.745080in}{3.489288in}%
\pgfsys@useobject{currentmarker}{}%
\end{pgfscope}%
\begin{pgfscope}%
\pgfsys@transformshift{0.726064in}{3.464485in}%
\pgfsys@useobject{currentmarker}{}%
\end{pgfscope}%
\begin{pgfscope}%
\pgfsys@transformshift{0.708925in}{3.459306in}%
\pgfsys@useobject{currentmarker}{}%
\end{pgfscope}%
\begin{pgfscope}%
\pgfsys@transformshift{0.686153in}{3.472123in}%
\pgfsys@useobject{currentmarker}{}%
\end{pgfscope}%
\begin{pgfscope}%
\pgfsys@transformshift{0.669013in}{3.515455in}%
\pgfsys@useobject{currentmarker}{}%
\end{pgfscope}%
\begin{pgfscope}%
\pgfsys@transformshift{0.649059in}{3.616566in}%
\pgfsys@useobject{currentmarker}{}%
\end{pgfscope}%
\begin{pgfscope}%
\pgfsys@transformshift{0.648822in}{3.608478in}%
\pgfsys@useobject{currentmarker}{}%
\end{pgfscope}%
\begin{pgfscope}%
\pgfsys@transformshift{0.654694in}{3.558327in}%
\pgfsys@useobject{currentmarker}{}%
\end{pgfscope}%
\begin{pgfscope}%
\pgfsys@transformshift{0.675588in}{3.475068in}%
\pgfsys@useobject{currentmarker}{}%
\end{pgfscope}%
\begin{pgfscope}%
\pgfsys@transformshift{0.694605in}{3.458436in}%
\pgfsys@useobject{currentmarker}{}%
\end{pgfscope}%
\begin{pgfscope}%
\pgfsys@transformshift{0.711977in}{3.474718in}%
\pgfsys@useobject{currentmarker}{}%
\end{pgfscope}%
\begin{pgfscope}%
\pgfsys@transformshift{0.733576in}{3.543158in}%
\pgfsys@useobject{currentmarker}{}%
\end{pgfscope}%
\begin{pgfscope}%
\pgfsys@transformshift{0.751184in}{3.732279in}%
\pgfsys@useobject{currentmarker}{}%
\end{pgfscope}%
\begin{pgfscope}%
\pgfsys@transformshift{0.769732in}{4.159471in}%
\pgfsys@useobject{currentmarker}{}%
\end{pgfscope}%
\begin{pgfscope}%
\pgfsys@transformshift{0.790156in}{4.324019in}%
\pgfsys@useobject{currentmarker}{}%
\end{pgfscope}%
\begin{pgfscope}%
\pgfsys@transformshift{0.808469in}{4.049214in}%
\pgfsys@useobject{currentmarker}{}%
\end{pgfscope}%
\begin{pgfscope}%
\pgfsys@transformshift{0.830069in}{3.633638in}%
\pgfsys@useobject{currentmarker}{}%
\end{pgfscope}%
\begin{pgfscope}%
\pgfsys@transformshift{0.848380in}{3.494500in}%
\pgfsys@useobject{currentmarker}{}%
\end{pgfscope}%
\begin{pgfscope}%
\pgfsys@transformshift{0.867631in}{3.462840in}%
\pgfsys@useobject{currentmarker}{}%
\end{pgfscope}%
\begin{pgfscope}%
\pgfsys@transformshift{0.887351in}{3.464794in}%
\pgfsys@useobject{currentmarker}{}%
\end{pgfscope}%
\begin{pgfscope}%
\pgfsys@transformshift{0.905899in}{3.712184in}%
\pgfsys@useobject{currentmarker}{}%
\end{pgfscope}%
\begin{pgfscope}%
\pgfsys@transformshift{0.923976in}{4.150535in}%
\pgfsys@useobject{currentmarker}{}%
\end{pgfscope}%
\begin{pgfscope}%
\pgfsys@transformshift{0.943227in}{4.295270in}%
\pgfsys@useobject{currentmarker}{}%
\end{pgfscope}%
\begin{pgfscope}%
\pgfsys@transformshift{0.962715in}{3.966547in}%
\pgfsys@useobject{currentmarker}{}%
\end{pgfscope}%
\begin{pgfscope}%
\pgfsys@transformshift{0.984313in}{3.630720in}%
\pgfsys@useobject{currentmarker}{}%
\end{pgfscope}%
\begin{pgfscope}%
\pgfsys@transformshift{1.002861in}{3.494826in}%
\pgfsys@useobject{currentmarker}{}%
\end{pgfscope}%
\begin{pgfscope}%
\pgfsys@transformshift{1.021643in}{3.460207in}%
\pgfsys@useobject{currentmarker}{}%
\end{pgfscope}%
\begin{pgfscope}%
\pgfsys@transformshift{1.041363in}{3.459410in}%
\pgfsys@useobject{currentmarker}{}%
\end{pgfscope}%
\begin{pgfscope}%
\pgfsys@transformshift{1.059909in}{3.488348in}%
\pgfsys@useobject{currentmarker}{}%
\end{pgfscope}%
\begin{pgfscope}%
\pgfsys@transformshift{1.079162in}{3.567744in}%
\pgfsys@useobject{currentmarker}{}%
\end{pgfscope}%
\begin{pgfscope}%
\pgfsys@transformshift{1.097942in}{3.865071in}%
\pgfsys@useobject{currentmarker}{}%
\end{pgfscope}%
\begin{pgfscope}%
\pgfsys@transformshift{1.117899in}{4.255393in}%
\pgfsys@useobject{currentmarker}{}%
\end{pgfscope}%
\begin{pgfscope}%
\pgfsys@transformshift{1.133629in}{4.174593in}%
\pgfsys@useobject{currentmarker}{}%
\end{pgfscope}%
\begin{pgfscope}%
\pgfsys@transformshift{1.153584in}{3.869529in}%
\pgfsys@useobject{currentmarker}{}%
\end{pgfscope}%
\begin{pgfscope}%
\pgfsys@transformshift{1.175184in}{3.537417in}%
\pgfsys@useobject{currentmarker}{}%
\end{pgfscope}%
\begin{pgfscope}%
\pgfsys@transformshift{1.192086in}{3.477876in}%
\pgfsys@useobject{currentmarker}{}%
\end{pgfscope}%
\begin{pgfscope}%
\pgfsys@transformshift{1.213217in}{3.455576in}%
\pgfsys@useobject{currentmarker}{}%
\end{pgfscope}%
\begin{pgfscope}%
\pgfsys@transformshift{1.233172in}{3.462940in}%
\pgfsys@useobject{currentmarker}{}%
\end{pgfscope}%
\begin{pgfscope}%
\pgfsys@transformshift{1.252189in}{3.499757in}%
\pgfsys@useobject{currentmarker}{}%
\end{pgfscope}%
\begin{pgfscope}%
\pgfsys@transformshift{1.271440in}{3.620468in}%
\pgfsys@useobject{currentmarker}{}%
\end{pgfscope}%
\begin{pgfscope}%
\pgfsys@transformshift{1.290456in}{3.957771in}%
\pgfsys@useobject{currentmarker}{}%
\end{pgfscope}%
\begin{pgfscope}%
\pgfsys@transformshift{1.309004in}{4.250131in}%
\pgfsys@useobject{currentmarker}{}%
\end{pgfscope}%
\begin{pgfscope}%
\pgfsys@transformshift{1.328255in}{4.128748in}%
\pgfsys@useobject{currentmarker}{}%
\end{pgfscope}%
\begin{pgfscope}%
\pgfsys@transformshift{1.346801in}{3.768742in}%
\pgfsys@useobject{currentmarker}{}%
\end{pgfscope}%
\begin{pgfscope}%
\pgfsys@transformshift{1.365584in}{3.532899in}%
\pgfsys@useobject{currentmarker}{}%
\end{pgfscope}%
\begin{pgfscope}%
\pgfsys@transformshift{1.385069in}{3.470199in}%
\pgfsys@useobject{currentmarker}{}%
\end{pgfscope}%
\begin{pgfscope}%
\pgfsys@transformshift{1.407372in}{3.453732in}%
\pgfsys@useobject{currentmarker}{}%
\end{pgfscope}%
\begin{pgfscope}%
\pgfsys@transformshift{1.423103in}{3.458603in}%
\pgfsys@useobject{currentmarker}{}%
\end{pgfscope}%
\begin{pgfscope}%
\pgfsys@transformshift{1.443294in}{3.492162in}%
\pgfsys@useobject{currentmarker}{}%
\end{pgfscope}%
\begin{pgfscope}%
\pgfsys@transformshift{1.463248in}{3.576536in}%
\pgfsys@useobject{currentmarker}{}%
\end{pgfscope}%
\begin{pgfscope}%
\pgfsys@transformshift{1.481562in}{3.833944in}%
\pgfsys@useobject{currentmarker}{}%
\end{pgfscope}%
\begin{pgfscope}%
\pgfsys@transformshift{1.501987in}{4.175226in}%
\pgfsys@useobject{currentmarker}{}%
\end{pgfscope}%
\begin{pgfscope}%
\pgfsys@transformshift{1.520768in}{4.218703in}%
\pgfsys@useobject{currentmarker}{}%
\end{pgfscope}%
\begin{pgfscope}%
\pgfsys@transformshift{1.540724in}{3.946211in}%
\pgfsys@useobject{currentmarker}{}%
\end{pgfscope}%
\begin{pgfscope}%
\pgfsys@transformshift{1.558567in}{3.666668in}%
\pgfsys@useobject{currentmarker}{}%
\end{pgfscope}%
\begin{pgfscope}%
\pgfsys@transformshift{1.579461in}{3.509140in}%
\pgfsys@useobject{currentmarker}{}%
\end{pgfscope}%
\begin{pgfscope}%
\pgfsys@transformshift{1.595895in}{3.467883in}%
\pgfsys@useobject{currentmarker}{}%
\end{pgfscope}%
\begin{pgfscope}%
\pgfsys@transformshift{1.619607in}{3.453759in}%
\pgfsys@useobject{currentmarker}{}%
\end{pgfscope}%
\begin{pgfscope}%
\pgfsys@transformshift{1.636980in}{3.456243in}%
\pgfsys@useobject{currentmarker}{}%
\end{pgfscope}%
\begin{pgfscope}%
\pgfsys@transformshift{1.655997in}{3.476491in}%
\pgfsys@useobject{currentmarker}{}%
\end{pgfscope}%
\begin{pgfscope}%
\pgfsys@transformshift{1.671962in}{3.532314in}%
\pgfsys@useobject{currentmarker}{}%
\end{pgfscope}%
\begin{pgfscope}%
\pgfsys@transformshift{1.694030in}{3.632560in}%
\pgfsys@useobject{currentmarker}{}%
\end{pgfscope}%
\begin{pgfscope}%
\pgfsys@transformshift{1.712342in}{3.998536in}%
\pgfsys@useobject{currentmarker}{}%
\end{pgfscope}%
\begin{pgfscope}%
\pgfsys@transformshift{1.730655in}{4.217761in}%
\pgfsys@useobject{currentmarker}{}%
\end{pgfscope}%
\begin{pgfscope}%
\pgfsys@transformshift{1.749201in}{4.157154in}%
\pgfsys@useobject{currentmarker}{}%
\end{pgfscope}%
\begin{pgfscope}%
\pgfsys@transformshift{1.772210in}{3.794834in}%
\pgfsys@useobject{currentmarker}{}%
\end{pgfscope}%
\begin{pgfscope}%
\pgfsys@transformshift{1.791461in}{3.561492in}%
\pgfsys@useobject{currentmarker}{}%
\end{pgfscope}%
\begin{pgfscope}%
\pgfsys@transformshift{1.810477in}{3.479465in}%
\pgfsys@useobject{currentmarker}{}%
\end{pgfscope}%
\begin{pgfscope}%
\pgfsys@transformshift{1.828554in}{3.459907in}%
\pgfsys@useobject{currentmarker}{}%
\end{pgfscope}%
\begin{pgfscope}%
\pgfsys@transformshift{1.847571in}{3.452995in}%
\pgfsys@useobject{currentmarker}{}%
\end{pgfscope}%
\begin{pgfscope}%
\pgfsys@transformshift{1.866822in}{3.456663in}%
\pgfsys@useobject{currentmarker}{}%
\end{pgfscope}%
\begin{pgfscope}%
\pgfsys@transformshift{1.884430in}{3.475388in}%
\pgfsys@useobject{currentmarker}{}%
\end{pgfscope}%
\begin{pgfscope}%
\pgfsys@transformshift{1.903682in}{3.535982in}%
\pgfsys@useobject{currentmarker}{}%
\end{pgfscope}%
\begin{pgfscope}%
\pgfsys@transformshift{1.924576in}{3.746778in}%
\pgfsys@useobject{currentmarker}{}%
\end{pgfscope}%
\begin{pgfscope}%
\pgfsys@transformshift{1.943827in}{4.100894in}%
\pgfsys@useobject{currentmarker}{}%
\end{pgfscope}%
\begin{pgfscope}%
\pgfsys@transformshift{1.962375in}{4.196820in}%
\pgfsys@useobject{currentmarker}{}%
\end{pgfscope}%
\begin{pgfscope}%
\pgfsys@transformshift{1.981392in}{4.001579in}%
\pgfsys@useobject{currentmarker}{}%
\end{pgfscope}%
\begin{pgfscope}%
\pgfsys@transformshift{2.002992in}{3.759695in}%
\pgfsys@useobject{currentmarker}{}%
\end{pgfscope}%
\begin{pgfscope}%
\pgfsys@transformshift{2.018954in}{3.553706in}%
\pgfsys@useobject{currentmarker}{}%
\end{pgfscope}%
\begin{pgfscope}%
\pgfsys@transformshift{2.039614in}{3.869264in}%
\pgfsys@useobject{currentmarker}{}%
\end{pgfscope}%
\begin{pgfscope}%
\pgfsys@transformshift{2.058397in}{3.582988in}%
\pgfsys@useobject{currentmarker}{}%
\end{pgfscope}%
\begin{pgfscope}%
\pgfsys@transformshift{2.076944in}{3.488269in}%
\pgfsys@useobject{currentmarker}{}%
\end{pgfscope}%
\begin{pgfscope}%
\pgfsys@transformshift{2.097839in}{3.457030in}%
\pgfsys@useobject{currentmarker}{}%
\end{pgfscope}%
\begin{pgfscope}%
\pgfsys@transformshift{2.116387in}{3.453311in}%
\pgfsys@useobject{currentmarker}{}%
\end{pgfscope}%
\begin{pgfscope}%
\pgfsys@transformshift{2.134698in}{3.463772in}%
\pgfsys@useobject{currentmarker}{}%
\end{pgfscope}%
\begin{pgfscope}%
\pgfsys@transformshift{2.156532in}{3.502307in}%
\pgfsys@useobject{currentmarker}{}%
\end{pgfscope}%
\begin{pgfscope}%
\pgfsys@transformshift{2.174140in}{3.538854in}%
\pgfsys@useobject{currentmarker}{}%
\end{pgfscope}%
\begin{pgfscope}%
\pgfsys@transformshift{2.191512in}{3.685352in}%
\pgfsys@useobject{currentmarker}{}%
\end{pgfscope}%
\begin{pgfscope}%
\pgfsys@transformshift{2.212174in}{4.070095in}%
\pgfsys@useobject{currentmarker}{}%
\end{pgfscope}%
\begin{pgfscope}%
\pgfsys@transformshift{2.235415in}{4.189052in}%
\pgfsys@useobject{currentmarker}{}%
\end{pgfscope}%
\begin{pgfscope}%
\pgfsys@transformshift{2.252319in}{4.076099in}%
\pgfsys@useobject{currentmarker}{}%
\end{pgfscope}%
\begin{pgfscope}%
\pgfsys@transformshift{2.269693in}{3.752230in}%
\pgfsys@useobject{currentmarker}{}%
\end{pgfscope}%
\begin{pgfscope}%
\pgfsys@transformshift{2.288473in}{3.705223in}%
\pgfsys@useobject{currentmarker}{}%
\end{pgfscope}%
\begin{pgfscope}%
\pgfsys@transformshift{2.308664in}{3.513649in}%
\pgfsys@useobject{currentmarker}{}%
\end{pgfscope}%
\begin{pgfscope}%
\pgfsys@transformshift{2.326507in}{3.502160in}%
\pgfsys@useobject{currentmarker}{}%
\end{pgfscope}%
\begin{pgfscope}%
\pgfsys@transformshift{2.348341in}{3.462738in}%
\pgfsys@useobject{currentmarker}{}%
\end{pgfscope}%
\begin{pgfscope}%
\pgfsys@transformshift{2.365480in}{3.452939in}%
\pgfsys@useobject{currentmarker}{}%
\end{pgfscope}%
\begin{pgfscope}%
\pgfsys@transformshift{2.386374in}{3.461118in}%
\pgfsys@useobject{currentmarker}{}%
\end{pgfscope}%
\begin{pgfscope}%
\pgfsys@transformshift{2.404686in}{3.487864in}%
\pgfsys@useobject{currentmarker}{}%
\end{pgfscope}%
\begin{pgfscope}%
\pgfsys@transformshift{2.425346in}{3.583819in}%
\pgfsys@useobject{currentmarker}{}%
\end{pgfscope}%
\begin{pgfscope}%
\pgfsys@transformshift{2.442250in}{3.815561in}%
\pgfsys@useobject{currentmarker}{}%
\end{pgfscope}%
\begin{pgfscope}%
\pgfsys@transformshift{2.460327in}{4.034965in}%
\pgfsys@useobject{currentmarker}{}%
\end{pgfscope}%
\begin{pgfscope}%
\pgfsys@transformshift{2.481927in}{4.186842in}%
\pgfsys@useobject{currentmarker}{}%
\end{pgfscope}%
\begin{pgfscope}%
\pgfsys@transformshift{2.499299in}{4.002698in}%
\pgfsys@useobject{currentmarker}{}%
\end{pgfscope}%
\begin{pgfscope}%
\pgfsys@transformshift{2.518786in}{3.660348in}%
\pgfsys@useobject{currentmarker}{}%
\end{pgfscope}%
\begin{pgfscope}%
\pgfsys@transformshift{2.538741in}{3.509079in}%
\pgfsys@useobject{currentmarker}{}%
\end{pgfscope}%
\begin{pgfscope}%
\pgfsys@transformshift{2.559166in}{3.471984in}%
\pgfsys@useobject{currentmarker}{}%
\end{pgfscope}%
\begin{pgfscope}%
\pgfsys@transformshift{2.577478in}{3.473774in}%
\pgfsys@useobject{currentmarker}{}%
\end{pgfscope}%
\begin{pgfscope}%
\pgfsys@transformshift{2.597669in}{3.454416in}%
\pgfsys@useobject{currentmarker}{}%
\end{pgfscope}%
\begin{pgfscope}%
\pgfsys@transformshift{2.615277in}{3.454646in}%
\pgfsys@useobject{currentmarker}{}%
\end{pgfscope}%
\begin{pgfscope}%
\pgfsys@transformshift{2.633825in}{3.471303in}%
\pgfsys@useobject{currentmarker}{}%
\end{pgfscope}%
\begin{pgfscope}%
\pgfsys@transformshift{2.655422in}{3.521150in}%
\pgfsys@useobject{currentmarker}{}%
\end{pgfscope}%
\begin{pgfscope}%
\pgfsys@transformshift{2.672796in}{3.594801in}%
\pgfsys@useobject{currentmarker}{}%
\end{pgfscope}%
\begin{pgfscope}%
\pgfsys@transformshift{2.690873in}{3.840342in}%
\pgfsys@useobject{currentmarker}{}%
\end{pgfscope}%
\begin{pgfscope}%
\pgfsys@transformshift{2.711769in}{4.162984in}%
\pgfsys@useobject{currentmarker}{}%
\end{pgfscope}%
\begin{pgfscope}%
\pgfsys@transformshift{2.732898in}{4.142333in}%
\pgfsys@useobject{currentmarker}{}%
\end{pgfscope}%
\begin{pgfscope}%
\pgfsys@transformshift{2.750741in}{3.902389in}%
\pgfsys@useobject{currentmarker}{}%
\end{pgfscope}%
\begin{pgfscope}%
\pgfsys@transformshift{2.769992in}{3.608667in}%
\pgfsys@useobject{currentmarker}{}%
\end{pgfscope}%
\begin{pgfscope}%
\pgfsys@transformshift{2.790886in}{3.519127in}%
\pgfsys@useobject{currentmarker}{}%
\end{pgfscope}%
\begin{pgfscope}%
\pgfsys@transformshift{2.808025in}{3.472842in}%
\pgfsys@useobject{currentmarker}{}%
\end{pgfscope}%
\begin{pgfscope}%
\pgfsys@transformshift{2.829860in}{3.455363in}%
\pgfsys@useobject{currentmarker}{}%
\end{pgfscope}%
\begin{pgfscope}%
\pgfsys@transformshift{2.848405in}{3.456588in}%
\pgfsys@useobject{currentmarker}{}%
\end{pgfscope}%
\begin{pgfscope}%
\pgfsys@transformshift{2.866719in}{3.465120in}%
\pgfsys@useobject{currentmarker}{}%
\end{pgfscope}%
\begin{pgfscope}%
\pgfsys@transformshift{2.887144in}{3.504355in}%
\pgfsys@useobject{currentmarker}{}%
\end{pgfscope}%
\begin{pgfscope}%
\pgfsys@transformshift{2.908977in}{3.628561in}%
\pgfsys@useobject{currentmarker}{}%
\end{pgfscope}%
\begin{pgfscope}%
\pgfsys@transformshift{2.923298in}{3.773733in}%
\pgfsys@useobject{currentmarker}{}%
\end{pgfscope}%
\begin{pgfscope}%
\pgfsys@transformshift{2.941375in}{4.095270in}%
\pgfsys@useobject{currentmarker}{}%
\end{pgfscope}%
\begin{pgfscope}%
\pgfsys@transformshift{2.962271in}{4.223653in}%
\pgfsys@useobject{currentmarker}{}%
\end{pgfscope}%
\begin{pgfscope}%
\pgfsys@transformshift{2.979879in}{4.094923in}%
\pgfsys@useobject{currentmarker}{}%
\end{pgfscope}%
\begin{pgfscope}%
\pgfsys@transformshift{3.000774in}{3.721387in}%
\pgfsys@useobject{currentmarker}{}%
\end{pgfscope}%
\begin{pgfscope}%
\pgfsys@transformshift{3.022137in}{3.529454in}%
\pgfsys@useobject{currentmarker}{}%
\end{pgfscope}%
\begin{pgfscope}%
\pgfsys@transformshift{3.039980in}{3.482470in}%
\pgfsys@useobject{currentmarker}{}%
\end{pgfscope}%
\begin{pgfscope}%
\pgfsys@transformshift{3.058762in}{3.464701in}%
\pgfsys@useobject{currentmarker}{}%
\end{pgfscope}%
\begin{pgfscope}%
\pgfsys@transformshift{3.078482in}{3.455370in}%
\pgfsys@useobject{currentmarker}{}%
\end{pgfscope}%
\begin{pgfscope}%
\pgfsys@transformshift{3.097264in}{3.455624in}%
\pgfsys@useobject{currentmarker}{}%
\end{pgfscope}%
\begin{pgfscope}%
\pgfsys@transformshift{3.097030in}{3.455441in}%
\pgfsys@useobject{currentmarker}{}%
\end{pgfscope}%
\begin{pgfscope}%
\pgfsys@transformshift{3.114404in}{3.455298in}%
\pgfsys@useobject{currentmarker}{}%
\end{pgfscope}%
\begin{pgfscope}%
\pgfsys@transformshift{3.136003in}{3.471875in}%
\pgfsys@useobject{currentmarker}{}%
\end{pgfscope}%
\begin{pgfscope}%
\pgfsys@transformshift{3.155254in}{3.526046in}%
\pgfsys@useobject{currentmarker}{}%
\end{pgfscope}%
\begin{pgfscope}%
\pgfsys@transformshift{3.172157in}{3.634089in}%
\pgfsys@useobject{currentmarker}{}%
\end{pgfscope}%
\begin{pgfscope}%
\pgfsys@transformshift{3.193991in}{3.942204in}%
\pgfsys@useobject{currentmarker}{}%
\end{pgfscope}%
\begin{pgfscope}%
\pgfsys@transformshift{3.212068in}{4.197188in}%
\pgfsys@useobject{currentmarker}{}%
\end{pgfscope}%
\begin{pgfscope}%
\pgfsys@transformshift{3.235311in}{4.195078in}%
\pgfsys@useobject{currentmarker}{}%
\end{pgfscope}%
\begin{pgfscope}%
\pgfsys@transformshift{3.249633in}{4.126710in}%
\pgfsys@useobject{currentmarker}{}%
\end{pgfscope}%
\begin{pgfscope}%
\pgfsys@transformshift{3.267944in}{3.866996in}%
\pgfsys@useobject{currentmarker}{}%
\end{pgfscope}%
\begin{pgfscope}%
\pgfsys@transformshift{3.289310in}{3.586550in}%
\pgfsys@useobject{currentmarker}{}%
\end{pgfscope}%
\begin{pgfscope}%
\pgfsys@transformshift{3.308324in}{3.500059in}%
\pgfsys@useobject{currentmarker}{}%
\end{pgfscope}%
\begin{pgfscope}%
\pgfsys@transformshift{3.327107in}{3.464037in}%
\pgfsys@useobject{currentmarker}{}%
\end{pgfscope}%
\begin{pgfscope}%
\pgfsys@transformshift{3.347532in}{3.455194in}%
\pgfsys@useobject{currentmarker}{}%
\end{pgfscope}%
\begin{pgfscope}%
\pgfsys@transformshift{3.365844in}{3.459968in}%
\pgfsys@useobject{currentmarker}{}%
\end{pgfscope}%
\begin{pgfscope}%
\pgfsys@transformshift{3.385097in}{3.474755in}%
\pgfsys@useobject{currentmarker}{}%
\end{pgfscope}%
\begin{pgfscope}%
\pgfsys@transformshift{3.403643in}{3.504665in}%
\pgfsys@useobject{currentmarker}{}%
\end{pgfscope}%
\begin{pgfscope}%
\pgfsys@transformshift{3.422190in}{3.581547in}%
\pgfsys@useobject{currentmarker}{}%
\end{pgfscope}%
\begin{pgfscope}%
\pgfsys@transformshift{3.441910in}{3.776474in}%
\pgfsys@useobject{currentmarker}{}%
\end{pgfscope}%
\begin{pgfscope}%
\pgfsys@transformshift{3.460693in}{4.079928in}%
\pgfsys@useobject{currentmarker}{}%
\end{pgfscope}%
\begin{pgfscope}%
\pgfsys@transformshift{3.481118in}{4.267232in}%
\pgfsys@useobject{currentmarker}{}%
\end{pgfscope}%
\begin{pgfscope}%
\pgfsys@transformshift{3.499195in}{4.198415in}%
\pgfsys@useobject{currentmarker}{}%
\end{pgfscope}%
\begin{pgfscope}%
\pgfsys@transformshift{3.513517in}{3.912554in}%
\pgfsys@useobject{currentmarker}{}%
\end{pgfscope}%
\begin{pgfscope}%
\pgfsys@transformshift{3.537698in}{3.774585in}%
\pgfsys@useobject{currentmarker}{}%
\end{pgfscope}%
\begin{pgfscope}%
\pgfsys@transformshift{3.558123in}{3.572574in}%
\pgfsys@useobject{currentmarker}{}%
\end{pgfscope}%
\begin{pgfscope}%
\pgfsys@transformshift{3.575497in}{3.513094in}%
\pgfsys@useobject{currentmarker}{}%
\end{pgfscope}%
\begin{pgfscope}%
\pgfsys@transformshift{3.596625in}{3.473263in}%
\pgfsys@useobject{currentmarker}{}%
\end{pgfscope}%
\begin{pgfscope}%
\pgfsys@transformshift{3.612825in}{3.458595in}%
\pgfsys@useobject{currentmarker}{}%
\end{pgfscope}%
\begin{pgfscope}%
\pgfsys@transformshift{3.632545in}{3.498697in}%
\pgfsys@useobject{currentmarker}{}%
\end{pgfscope}%
\begin{pgfscope}%
\pgfsys@transformshift{3.653910in}{3.466444in}%
\pgfsys@useobject{currentmarker}{}%
\end{pgfscope}%
\begin{pgfscope}%
\pgfsys@transformshift{3.673161in}{3.458296in}%
\pgfsys@useobject{currentmarker}{}%
\end{pgfscope}%
\begin{pgfscope}%
\pgfsys@transformshift{3.691238in}{3.460217in}%
\pgfsys@useobject{currentmarker}{}%
\end{pgfscope}%
\begin{pgfscope}%
\pgfsys@transformshift{3.713073in}{3.488331in}%
\pgfsys@useobject{currentmarker}{}%
\end{pgfscope}%
\begin{pgfscope}%
\pgfsys@transformshift{3.730446in}{3.543148in}%
\pgfsys@useobject{currentmarker}{}%
\end{pgfscope}%
\begin{pgfscope}%
\pgfsys@transformshift{3.747820in}{3.646970in}%
\pgfsys@useobject{currentmarker}{}%
\end{pgfscope}%
\begin{pgfscope}%
\pgfsys@transformshift{3.768949in}{3.978749in}%
\pgfsys@useobject{currentmarker}{}%
\end{pgfscope}%
\begin{pgfscope}%
\pgfsys@transformshift{3.789608in}{4.289622in}%
\pgfsys@useobject{currentmarker}{}%
\end{pgfscope}%
\begin{pgfscope}%
\pgfsys@transformshift{3.808860in}{4.268962in}%
\pgfsys@useobject{currentmarker}{}%
\end{pgfscope}%
\begin{pgfscope}%
\pgfsys@transformshift{3.825999in}{4.103231in}%
\pgfsys@useobject{currentmarker}{}%
\end{pgfscope}%
\begin{pgfscope}%
\pgfsys@transformshift{3.846893in}{3.769187in}%
\pgfsys@useobject{currentmarker}{}%
\end{pgfscope}%
\begin{pgfscope}%
\pgfsys@transformshift{3.865675in}{3.623372in}%
\pgfsys@useobject{currentmarker}{}%
\end{pgfscope}%
\begin{pgfscope}%
\pgfsys@transformshift{3.886335in}{3.511071in}%
\pgfsys@useobject{currentmarker}{}%
\end{pgfscope}%
\begin{pgfscope}%
\pgfsys@transformshift{3.902769in}{3.475188in}%
\pgfsys@useobject{currentmarker}{}%
\end{pgfscope}%
\begin{pgfscope}%
\pgfsys@transformshift{3.922020in}{3.460001in}%
\pgfsys@useobject{currentmarker}{}%
\end{pgfscope}%
\begin{pgfscope}%
\pgfsys@transformshift{3.941975in}{3.459964in}%
\pgfsys@useobject{currentmarker}{}%
\end{pgfscope}%
\begin{pgfscope}%
\pgfsys@transformshift{3.958880in}{3.473892in}%
\pgfsys@useobject{currentmarker}{}%
\end{pgfscope}%
\begin{pgfscope}%
\pgfsys@transformshift{3.979774in}{3.504003in}%
\pgfsys@useobject{currentmarker}{}%
\end{pgfscope}%
\begin{pgfscope}%
\pgfsys@transformshift{3.997616in}{3.554416in}%
\pgfsys@useobject{currentmarker}{}%
\end{pgfscope}%
\begin{pgfscope}%
\pgfsys@transformshift{4.018511in}{3.727758in}%
\pgfsys@useobject{currentmarker}{}%
\end{pgfscope}%
\begin{pgfscope}%
\pgfsys@transformshift{4.037059in}{4.023682in}%
\pgfsys@useobject{currentmarker}{}%
\end{pgfscope}%
\begin{pgfscope}%
\pgfsys@transformshift{4.058188in}{4.327919in}%
\pgfsys@useobject{currentmarker}{}%
\end{pgfscope}%
\begin{pgfscope}%
\pgfsys@transformshift{4.075561in}{3.507449in}%
\pgfsys@useobject{currentmarker}{}%
\end{pgfscope}%
\begin{pgfscope}%
\pgfsys@transformshift{4.093874in}{3.604429in}%
\pgfsys@useobject{currentmarker}{}%
\end{pgfscope}%
\begin{pgfscope}%
\pgfsys@transformshift{4.114064in}{3.929375in}%
\pgfsys@useobject{currentmarker}{}%
\end{pgfscope}%
\begin{pgfscope}%
\pgfsys@transformshift{4.134960in}{4.266208in}%
\pgfsys@useobject{currentmarker}{}%
\end{pgfscope}%
\begin{pgfscope}%
\pgfsys@transformshift{4.153740in}{4.356160in}%
\pgfsys@useobject{currentmarker}{}%
\end{pgfscope}%
\begin{pgfscope}%
\pgfsys@transformshift{4.172288in}{4.219967in}%
\pgfsys@useobject{currentmarker}{}%
\end{pgfscope}%
\begin{pgfscope}%
\pgfsys@transformshift{4.192948in}{3.861395in}%
\pgfsys@useobject{currentmarker}{}%
\end{pgfscope}%
\begin{pgfscope}%
\pgfsys@transformshift{4.209147in}{3.636474in}%
\pgfsys@useobject{currentmarker}{}%
\end{pgfscope}%
\begin{pgfscope}%
\pgfsys@transformshift{4.229573in}{3.519461in}%
\pgfsys@useobject{currentmarker}{}%
\end{pgfscope}%
\begin{pgfscope}%
\pgfsys@transformshift{4.250702in}{3.487326in}%
\pgfsys@useobject{currentmarker}{}%
\end{pgfscope}%
\begin{pgfscope}%
\pgfsys@transformshift{4.268544in}{3.468278in}%
\pgfsys@useobject{currentmarker}{}%
\end{pgfscope}%
\begin{pgfscope}%
\pgfsys@transformshift{4.289438in}{3.459616in}%
\pgfsys@useobject{currentmarker}{}%
\end{pgfscope}%
\begin{pgfscope}%
\pgfsys@transformshift{4.306812in}{3.471095in}%
\pgfsys@useobject{currentmarker}{}%
\end{pgfscope}%
\begin{pgfscope}%
\pgfsys@transformshift{4.326298in}{3.498896in}%
\pgfsys@useobject{currentmarker}{}%
\end{pgfscope}%
\begin{pgfscope}%
\pgfsys@transformshift{4.346958in}{3.591070in}%
\pgfsys@useobject{currentmarker}{}%
\end{pgfscope}%
\begin{pgfscope}%
\pgfsys@transformshift{4.363628in}{3.773047in}%
\pgfsys@useobject{currentmarker}{}%
\end{pgfscope}%
\begin{pgfscope}%
\pgfsys@transformshift{4.384757in}{4.107542in}%
\pgfsys@useobject{currentmarker}{}%
\end{pgfscope}%
\begin{pgfscope}%
\pgfsys@transformshift{4.404008in}{4.385119in}%
\pgfsys@useobject{currentmarker}{}%
\end{pgfscope}%
\begin{pgfscope}%
\pgfsys@transformshift{4.422085in}{4.357960in}%
\pgfsys@useobject{currentmarker}{}%
\end{pgfscope}%
\begin{pgfscope}%
\pgfsys@transformshift{4.444153in}{4.353942in}%
\pgfsys@useobject{currentmarker}{}%
\end{pgfscope}%
\begin{pgfscope}%
\pgfsys@transformshift{4.461996in}{4.074679in}%
\pgfsys@useobject{currentmarker}{}%
\end{pgfscope}%
\begin{pgfscope}%
\pgfsys@transformshift{4.479840in}{3.753025in}%
\pgfsys@useobject{currentmarker}{}%
\end{pgfscope}%
\begin{pgfscope}%
\pgfsys@transformshift{4.479840in}{3.754235in}%
\pgfsys@useobject{currentmarker}{}%
\end{pgfscope}%
\begin{pgfscope}%
\pgfsys@transformshift{4.470683in}{3.958242in}%
\pgfsys@useobject{currentmarker}{}%
\end{pgfscope}%
\begin{pgfscope}%
\pgfsys@transformshift{4.454483in}{4.300002in}%
\pgfsys@useobject{currentmarker}{}%
\end{pgfscope}%
\begin{pgfscope}%
\pgfsys@transformshift{4.434529in}{4.378441in}%
\pgfsys@useobject{currentmarker}{}%
\end{pgfscope}%
\begin{pgfscope}%
\pgfsys@transformshift{4.414338in}{3.906874in}%
\pgfsys@useobject{currentmarker}{}%
\end{pgfscope}%
\begin{pgfscope}%
\pgfsys@transformshift{4.396495in}{3.593918in}%
\pgfsys@useobject{currentmarker}{}%
\end{pgfscope}%
\begin{pgfscope}%
\pgfsys@transformshift{4.377010in}{3.493366in}%
\pgfsys@useobject{currentmarker}{}%
\end{pgfscope}%
\begin{pgfscope}%
\pgfsys@transformshift{4.359167in}{3.461118in}%
\pgfsys@useobject{currentmarker}{}%
\end{pgfscope}%
\begin{pgfscope}%
\pgfsys@transformshift{4.340150in}{3.465527in}%
\pgfsys@useobject{currentmarker}{}%
\end{pgfscope}%
\begin{pgfscope}%
\pgfsys@transformshift{4.318551in}{3.515105in}%
\pgfsys@useobject{currentmarker}{}%
\end{pgfscope}%
\begin{pgfscope}%
\pgfsys@transformshift{4.300474in}{3.661140in}%
\pgfsys@useobject{currentmarker}{}%
\end{pgfscope}%
\begin{pgfscope}%
\pgfsys@transformshift{4.281457in}{4.042453in}%
\pgfsys@useobject{currentmarker}{}%
\end{pgfscope}%
\begin{pgfscope}%
\pgfsys@transformshift{4.264789in}{4.329149in}%
\pgfsys@useobject{currentmarker}{}%
\end{pgfscope}%
\begin{pgfscope}%
\pgfsys@transformshift{4.244129in}{4.204331in}%
\pgfsys@useobject{currentmarker}{}%
\end{pgfscope}%
\begin{pgfscope}%
\pgfsys@transformshift{4.223233in}{3.716399in}%
\pgfsys@useobject{currentmarker}{}%
\end{pgfscope}%
\begin{pgfscope}%
\pgfsys@transformshift{4.205390in}{3.536302in}%
\pgfsys@useobject{currentmarker}{}%
\end{pgfscope}%
\begin{pgfscope}%
\pgfsys@transformshift{4.186844in}{3.473944in}%
\pgfsys@useobject{currentmarker}{}%
\end{pgfscope}%
\begin{pgfscope}%
\pgfsys@transformshift{4.167357in}{3.457425in}%
\pgfsys@useobject{currentmarker}{}%
\end{pgfscope}%
\begin{pgfscope}%
\pgfsys@transformshift{4.146933in}{3.478456in}%
\pgfsys@useobject{currentmarker}{}%
\end{pgfscope}%
\begin{pgfscope}%
\pgfsys@transformshift{4.130968in}{3.528196in}%
\pgfsys@useobject{currentmarker}{}%
\end{pgfscope}%
\begin{pgfscope}%
\pgfsys@transformshift{4.109134in}{3.843372in}%
\pgfsys@useobject{currentmarker}{}%
\end{pgfscope}%
\begin{pgfscope}%
\pgfsys@transformshift{4.090586in}{4.194813in}%
\pgfsys@useobject{currentmarker}{}%
\end{pgfscope}%
\begin{pgfscope}%
\pgfsys@transformshift{4.069926in}{4.304665in}%
\pgfsys@useobject{currentmarker}{}%
\end{pgfscope}%
\begin{pgfscope}%
\pgfsys@transformshift{4.049737in}{3.898244in}%
\pgfsys@useobject{currentmarker}{}%
\end{pgfscope}%
\begin{pgfscope}%
\pgfsys@transformshift{4.033538in}{3.600367in}%
\pgfsys@useobject{currentmarker}{}%
\end{pgfscope}%
\begin{pgfscope}%
\pgfsys@transformshift{4.011938in}{3.487770in}%
\pgfsys@useobject{currentmarker}{}%
\end{pgfscope}%
\begin{pgfscope}%
\pgfsys@transformshift{3.995034in}{3.459897in}%
\pgfsys@useobject{currentmarker}{}%
\end{pgfscope}%
\begin{pgfscope}%
\pgfsys@transformshift{3.974610in}{3.460455in}%
\pgfsys@useobject{currentmarker}{}%
\end{pgfscope}%
\begin{pgfscope}%
\pgfsys@transformshift{3.954185in}{3.464743in}%
\pgfsys@useobject{currentmarker}{}%
\end{pgfscope}%
\begin{pgfscope}%
\pgfsys@transformshift{3.934462in}{3.480679in}%
\pgfsys@useobject{currentmarker}{}%
\end{pgfscope}%
\begin{pgfscope}%
\pgfsys@transformshift{3.916151in}{3.558241in}%
\pgfsys@useobject{currentmarker}{}%
\end{pgfscope}%
\begin{pgfscope}%
\pgfsys@transformshift{3.896194in}{3.876934in}%
\pgfsys@useobject{currentmarker}{}%
\end{pgfscope}%
\begin{pgfscope}%
\pgfsys@transformshift{3.877178in}{4.210792in}%
\pgfsys@useobject{currentmarker}{}%
\end{pgfscope}%
\begin{pgfscope}%
\pgfsys@transformshift{3.858632in}{4.237216in}%
\pgfsys@useobject{currentmarker}{}%
\end{pgfscope}%
\begin{pgfscope}%
\pgfsys@transformshift{3.840555in}{3.889720in}%
\pgfsys@useobject{currentmarker}{}%
\end{pgfscope}%
\begin{pgfscope}%
\pgfsys@transformshift{3.821304in}{3.572206in}%
\pgfsys@useobject{currentmarker}{}%
\end{pgfscope}%
\begin{pgfscope}%
\pgfsys@transformshift{3.797355in}{3.474832in}%
\pgfsys@useobject{currentmarker}{}%
\end{pgfscope}%
\begin{pgfscope}%
\pgfsys@transformshift{3.783036in}{3.458570in}%
\pgfsys@useobject{currentmarker}{}%
\end{pgfscope}%
\begin{pgfscope}%
\pgfsys@transformshift{3.761202in}{3.459318in}%
\pgfsys@useobject{currentmarker}{}%
\end{pgfscope}%
\begin{pgfscope}%
\pgfsys@transformshift{3.744297in}{3.483770in}%
\pgfsys@useobject{currentmarker}{}%
\end{pgfscope}%
\begin{pgfscope}%
\pgfsys@transformshift{3.723871in}{3.587211in}%
\pgfsys@useobject{currentmarker}{}%
\end{pgfscope}%
\begin{pgfscope}%
\pgfsys@transformshift{3.706500in}{3.808812in}%
\pgfsys@useobject{currentmarker}{}%
\end{pgfscope}%
\begin{pgfscope}%
\pgfsys@transformshift{3.685840in}{4.144317in}%
\pgfsys@useobject{currentmarker}{}%
\end{pgfscope}%
\begin{pgfscope}%
\pgfsys@transformshift{3.666589in}{4.249413in}%
\pgfsys@useobject{currentmarker}{}%
\end{pgfscope}%
\begin{pgfscope}%
\pgfsys@transformshift{3.648275in}{3.952092in}%
\pgfsys@useobject{currentmarker}{}%
\end{pgfscope}%
\begin{pgfscope}%
\pgfsys@transformshift{3.627381in}{3.590671in}%
\pgfsys@useobject{currentmarker}{}%
\end{pgfscope}%
\begin{pgfscope}%
\pgfsys@transformshift{3.609538in}{3.496145in}%
\pgfsys@useobject{currentmarker}{}%
\end{pgfscope}%
\begin{pgfscope}%
\pgfsys@transformshift{3.592399in}{3.463316in}%
\pgfsys@useobject{currentmarker}{}%
\end{pgfscope}%
\begin{pgfscope}%
\pgfsys@transformshift{3.573383in}{3.455487in}%
\pgfsys@useobject{currentmarker}{}%
\end{pgfscope}%
\begin{pgfscope}%
\pgfsys@transformshift{3.551079in}{3.468718in}%
\pgfsys@useobject{currentmarker}{}%
\end{pgfscope}%
\begin{pgfscope}%
\pgfsys@transformshift{3.530654in}{3.507944in}%
\pgfsys@useobject{currentmarker}{}%
\end{pgfscope}%
\begin{pgfscope}%
\pgfsys@transformshift{3.512577in}{3.624131in}%
\pgfsys@useobject{currentmarker}{}%
\end{pgfscope}%
\begin{pgfscope}%
\pgfsys@transformshift{3.496378in}{3.911073in}%
\pgfsys@useobject{currentmarker}{}%
\end{pgfscope}%
\begin{pgfscope}%
\pgfsys@transformshift{3.474544in}{4.214024in}%
\pgfsys@useobject{currentmarker}{}%
\end{pgfscope}%
\begin{pgfscope}%
\pgfsys@transformshift{3.454823in}{4.202805in}%
\pgfsys@useobject{currentmarker}{}%
\end{pgfscope}%
\begin{pgfscope}%
\pgfsys@transformshift{3.437215in}{3.815604in}%
\pgfsys@useobject{currentmarker}{}%
\end{pgfscope}%
\begin{pgfscope}%
\pgfsys@transformshift{3.418668in}{3.590010in}%
\pgfsys@useobject{currentmarker}{}%
\end{pgfscope}%
\begin{pgfscope}%
\pgfsys@transformshift{3.398008in}{3.498153in}%
\pgfsys@useobject{currentmarker}{}%
\end{pgfscope}%
\begin{pgfscope}%
\pgfsys@transformshift{3.375939in}{3.461781in}%
\pgfsys@useobject{currentmarker}{}%
\end{pgfscope}%
\begin{pgfscope}%
\pgfsys@transformshift{3.358802in}{3.454443in}%
\pgfsys@useobject{currentmarker}{}%
\end{pgfscope}%
\begin{pgfscope}%
\pgfsys@transformshift{3.340488in}{3.458423in}%
\pgfsys@useobject{currentmarker}{}%
\end{pgfscope}%
\begin{pgfscope}%
\pgfsys@transformshift{3.319829in}{3.470626in}%
\pgfsys@useobject{currentmarker}{}%
\end{pgfscope}%
\begin{pgfscope}%
\pgfsys@transformshift{3.298465in}{3.537023in}%
\pgfsys@useobject{currentmarker}{}%
\end{pgfscope}%
\begin{pgfscope}%
\pgfsys@transformshift{3.283440in}{3.640725in}%
\pgfsys@useobject{currentmarker}{}%
\end{pgfscope}%
\begin{pgfscope}%
\pgfsys@transformshift{3.263484in}{4.027586in}%
\pgfsys@useobject{currentmarker}{}%
\end{pgfscope}%
\begin{pgfscope}%
\pgfsys@transformshift{3.243998in}{4.232023in}%
\pgfsys@useobject{currentmarker}{}%
\end{pgfscope}%
\begin{pgfscope}%
\pgfsys@transformshift{3.226155in}{4.151041in}%
\pgfsys@useobject{currentmarker}{}%
\end{pgfscope}%
\begin{pgfscope}%
\pgfsys@transformshift{3.207608in}{4.016814in}%
\pgfsys@useobject{currentmarker}{}%
\end{pgfscope}%
\begin{pgfscope}%
\pgfsys@transformshift{3.186008in}{3.664482in}%
\pgfsys@useobject{currentmarker}{}%
\end{pgfscope}%
\begin{pgfscope}%
\pgfsys@transformshift{3.165584in}{3.513353in}%
\pgfsys@useobject{currentmarker}{}%
\end{pgfscope}%
\begin{pgfscope}%
\pgfsys@transformshift{3.147271in}{3.468698in}%
\pgfsys@useobject{currentmarker}{}%
\end{pgfscope}%
\begin{pgfscope}%
\pgfsys@transformshift{3.130134in}{3.454964in}%
\pgfsys@useobject{currentmarker}{}%
\end{pgfscope}%
\begin{pgfscope}%
\pgfsys@transformshift{3.106891in}{3.459677in}%
\pgfsys@useobject{currentmarker}{}%
\end{pgfscope}%
\begin{pgfscope}%
\pgfsys@transformshift{3.090221in}{3.470576in}%
\pgfsys@useobject{currentmarker}{}%
\end{pgfscope}%
\begin{pgfscope}%
\pgfsys@transformshift{3.069561in}{3.522934in}%
\pgfsys@useobject{currentmarker}{}%
\end{pgfscope}%
\begin{pgfscope}%
\pgfsys@transformshift{3.051015in}{3.702696in}%
\pgfsys@useobject{currentmarker}{}%
\end{pgfscope}%
\begin{pgfscope}%
\pgfsys@transformshift{3.033641in}{4.008750in}%
\pgfsys@useobject{currentmarker}{}%
\end{pgfscope}%
\begin{pgfscope}%
\pgfsys@transformshift{3.012278in}{4.220167in}%
\pgfsys@useobject{currentmarker}{}%
\end{pgfscope}%
\begin{pgfscope}%
\pgfsys@transformshift{2.991618in}{4.043693in}%
\pgfsys@useobject{currentmarker}{}%
\end{pgfscope}%
\begin{pgfscope}%
\pgfsys@transformshift{2.973305in}{3.686527in}%
\pgfsys@useobject{currentmarker}{}%
\end{pgfscope}%
\begin{pgfscope}%
\pgfsys@transformshift{2.955697in}{3.537422in}%
\pgfsys@useobject{currentmarker}{}%
\end{pgfscope}%
\begin{pgfscope}%
\pgfsys@transformshift{2.933628in}{3.472307in}%
\pgfsys@useobject{currentmarker}{}%
\end{pgfscope}%
\begin{pgfscope}%
\pgfsys@transformshift{2.917898in}{3.458609in}%
\pgfsys@useobject{currentmarker}{}%
\end{pgfscope}%
\begin{pgfscope}%
\pgfsys@transformshift{2.896769in}{3.455665in}%
\pgfsys@useobject{currentmarker}{}%
\end{pgfscope}%
\begin{pgfscope}%
\pgfsys@transformshift{2.878457in}{3.467554in}%
\pgfsys@useobject{currentmarker}{}%
\end{pgfscope}%
\begin{pgfscope}%
\pgfsys@transformshift{2.859206in}{3.498632in}%
\pgfsys@useobject{currentmarker}{}%
\end{pgfscope}%
\begin{pgfscope}%
\pgfsys@transformshift{2.840893in}{3.609314in}%
\pgfsys@useobject{currentmarker}{}%
\end{pgfscope}%
\begin{pgfscope}%
\pgfsys@transformshift{2.823519in}{3.864899in}%
\pgfsys@useobject{currentmarker}{}%
\end{pgfscope}%
\begin{pgfscope}%
\pgfsys@transformshift{2.803799in}{4.117796in}%
\pgfsys@useobject{currentmarker}{}%
\end{pgfscope}%
\begin{pgfscope}%
\pgfsys@transformshift{2.781496in}{4.203050in}%
\pgfsys@useobject{currentmarker}{}%
\end{pgfscope}%
\begin{pgfscope}%
\pgfsys@transformshift{2.762479in}{4.041931in}%
\pgfsys@useobject{currentmarker}{}%
\end{pgfscope}%
\begin{pgfscope}%
\pgfsys@transformshift{2.743228in}{3.692370in}%
\pgfsys@useobject{currentmarker}{}%
\end{pgfscope}%
\begin{pgfscope}%
\pgfsys@transformshift{2.725151in}{3.540728in}%
\pgfsys@useobject{currentmarker}{}%
\end{pgfscope}%
\begin{pgfscope}%
\pgfsys@transformshift{2.706603in}{4.068967in}%
\pgfsys@useobject{currentmarker}{}%
\end{pgfscope}%
\begin{pgfscope}%
\pgfsys@transformshift{2.688292in}{4.037055in}%
\pgfsys@useobject{currentmarker}{}%
\end{pgfscope}%
\begin{pgfscope}%
\pgfsys@transformshift{2.666223in}{3.655603in}%
\pgfsys@useobject{currentmarker}{}%
\end{pgfscope}%
\begin{pgfscope}%
\pgfsys@transformshift{2.647207in}{3.513400in}%
\pgfsys@useobject{currentmarker}{}%
\end{pgfscope}%
\begin{pgfscope}%
\pgfsys@transformshift{2.629128in}{3.471523in}%
\pgfsys@useobject{currentmarker}{}%
\end{pgfscope}%
\begin{pgfscope}%
\pgfsys@transformshift{2.610113in}{3.458302in}%
\pgfsys@useobject{currentmarker}{}%
\end{pgfscope}%
\begin{pgfscope}%
\pgfsys@transformshift{2.590625in}{3.453202in}%
\pgfsys@useobject{currentmarker}{}%
\end{pgfscope}%
\begin{pgfscope}%
\pgfsys@transformshift{2.568793in}{3.469016in}%
\pgfsys@useobject{currentmarker}{}%
\end{pgfscope}%
\begin{pgfscope}%
\pgfsys@transformshift{2.550714in}{3.463825in}%
\pgfsys@useobject{currentmarker}{}%
\end{pgfscope}%
\begin{pgfscope}%
\pgfsys@transformshift{2.532168in}{3.453406in}%
\pgfsys@useobject{currentmarker}{}%
\end{pgfscope}%
\begin{pgfscope}%
\pgfsys@transformshift{2.514795in}{3.461758in}%
\pgfsys@useobject{currentmarker}{}%
\end{pgfscope}%
\begin{pgfscope}%
\pgfsys@transformshift{2.496481in}{3.497077in}%
\pgfsys@useobject{currentmarker}{}%
\end{pgfscope}%
\begin{pgfscope}%
\pgfsys@transformshift{2.476527in}{3.551503in}%
\pgfsys@useobject{currentmarker}{}%
\end{pgfscope}%
\begin{pgfscope}%
\pgfsys@transformshift{2.458684in}{3.780777in}%
\pgfsys@useobject{currentmarker}{}%
\end{pgfscope}%
\begin{pgfscope}%
\pgfsys@transformshift{2.435910in}{4.138861in}%
\pgfsys@useobject{currentmarker}{}%
\end{pgfscope}%
\begin{pgfscope}%
\pgfsys@transformshift{2.418539in}{4.181070in}%
\pgfsys@useobject{currentmarker}{}%
\end{pgfscope}%
\begin{pgfscope}%
\pgfsys@transformshift{2.399051in}{3.847141in}%
\pgfsys@useobject{currentmarker}{}%
\end{pgfscope}%
\begin{pgfscope}%
\pgfsys@transformshift{2.378157in}{3.568990in}%
\pgfsys@useobject{currentmarker}{}%
\end{pgfscope}%
\begin{pgfscope}%
\pgfsys@transformshift{2.359140in}{3.491375in}%
\pgfsys@useobject{currentmarker}{}%
\end{pgfscope}%
\begin{pgfscope}%
\pgfsys@transformshift{2.339654in}{3.460404in}%
\pgfsys@useobject{currentmarker}{}%
\end{pgfscope}%
\begin{pgfscope}%
\pgfsys@transformshift{2.321106in}{3.452777in}%
\pgfsys@useobject{currentmarker}{}%
\end{pgfscope}%
\begin{pgfscope}%
\pgfsys@transformshift{2.303264in}{3.460152in}%
\pgfsys@useobject{currentmarker}{}%
\end{pgfscope}%
\begin{pgfscope}%
\pgfsys@transformshift{2.284718in}{3.484915in}%
\pgfsys@useobject{currentmarker}{}%
\end{pgfscope}%
\begin{pgfscope}%
\pgfsys@transformshift{2.261006in}{3.620428in}%
\pgfsys@useobject{currentmarker}{}%
\end{pgfscope}%
\begin{pgfscope}%
\pgfsys@transformshift{2.244102in}{3.904747in}%
\pgfsys@useobject{currentmarker}{}%
\end{pgfscope}%
\begin{pgfscope}%
\pgfsys@transformshift{2.224850in}{4.173747in}%
\pgfsys@useobject{currentmarker}{}%
\end{pgfscope}%
\begin{pgfscope}%
\pgfsys@transformshift{2.207008in}{4.189127in}%
\pgfsys@useobject{currentmarker}{}%
\end{pgfscope}%
\begin{pgfscope}%
\pgfsys@transformshift{2.187757in}{3.887365in}%
\pgfsys@useobject{currentmarker}{}%
\end{pgfscope}%
\begin{pgfscope}%
\pgfsys@transformshift{2.165688in}{3.615321in}%
\pgfsys@useobject{currentmarker}{}%
\end{pgfscope}%
\begin{pgfscope}%
\pgfsys@transformshift{2.147845in}{3.509472in}%
\pgfsys@useobject{currentmarker}{}%
\end{pgfscope}%
\begin{pgfscope}%
\pgfsys@transformshift{2.128594in}{3.477754in}%
\pgfsys@useobject{currentmarker}{}%
\end{pgfscope}%
\begin{pgfscope}%
\pgfsys@transformshift{2.110752in}{3.460056in}%
\pgfsys@useobject{currentmarker}{}%
\end{pgfscope}%
\begin{pgfscope}%
\pgfsys@transformshift{2.091970in}{3.454262in}%
\pgfsys@useobject{currentmarker}{}%
\end{pgfscope}%
\begin{pgfscope}%
\pgfsys@transformshift{2.070370in}{3.464656in}%
\pgfsys@useobject{currentmarker}{}%
\end{pgfscope}%
\begin{pgfscope}%
\pgfsys@transformshift{2.052058in}{3.501730in}%
\pgfsys@useobject{currentmarker}{}%
\end{pgfscope}%
\begin{pgfscope}%
\pgfsys@transformshift{2.033745in}{3.617693in}%
\pgfsys@useobject{currentmarker}{}%
\end{pgfscope}%
\begin{pgfscope}%
\pgfsys@transformshift{2.014259in}{3.914311in}%
\pgfsys@useobject{currentmarker}{}%
\end{pgfscope}%
\begin{pgfscope}%
\pgfsys@transformshift{1.996182in}{4.168292in}%
\pgfsys@useobject{currentmarker}{}%
\end{pgfscope}%
\begin{pgfscope}%
\pgfsys@transformshift{1.974114in}{4.169481in}%
\pgfsys@useobject{currentmarker}{}%
\end{pgfscope}%
\begin{pgfscope}%
\pgfsys@transformshift{1.955566in}{3.894424in}%
\pgfsys@useobject{currentmarker}{}%
\end{pgfscope}%
\begin{pgfscope}%
\pgfsys@transformshift{1.937254in}{3.613201in}%
\pgfsys@useobject{currentmarker}{}%
\end{pgfscope}%
\begin{pgfscope}%
\pgfsys@transformshift{1.918238in}{3.508751in}%
\pgfsys@useobject{currentmarker}{}%
\end{pgfscope}%
\begin{pgfscope}%
\pgfsys@transformshift{1.897343in}{3.467015in}%
\pgfsys@useobject{currentmarker}{}%
\end{pgfscope}%
\begin{pgfscope}%
\pgfsys@transformshift{1.878327in}{3.455132in}%
\pgfsys@useobject{currentmarker}{}%
\end{pgfscope}%
\begin{pgfscope}%
\pgfsys@transformshift{1.861422in}{3.457219in}%
\pgfsys@useobject{currentmarker}{}%
\end{pgfscope}%
\begin{pgfscope}%
\pgfsys@transformshift{1.843111in}{3.469313in}%
\pgfsys@useobject{currentmarker}{}%
\end{pgfscope}%
\begin{pgfscope}%
\pgfsys@transformshift{1.821042in}{3.512766in}%
\pgfsys@useobject{currentmarker}{}%
\end{pgfscope}%
\begin{pgfscope}%
\pgfsys@transformshift{1.802025in}{3.641011in}%
\pgfsys@useobject{currentmarker}{}%
\end{pgfscope}%
\begin{pgfscope}%
\pgfsys@transformshift{1.782774in}{3.927941in}%
\pgfsys@useobject{currentmarker}{}%
\end{pgfscope}%
\begin{pgfscope}%
\pgfsys@transformshift{1.764931in}{4.170728in}%
\pgfsys@useobject{currentmarker}{}%
\end{pgfscope}%
\begin{pgfscope}%
\pgfsys@transformshift{1.745915in}{4.236220in}%
\pgfsys@useobject{currentmarker}{}%
\end{pgfscope}%
\begin{pgfscope}%
\pgfsys@transformshift{1.727369in}{4.099642in}%
\pgfsys@useobject{currentmarker}{}%
\end{pgfscope}%
\begin{pgfscope}%
\pgfsys@transformshift{1.708116in}{3.798417in}%
\pgfsys@useobject{currentmarker}{}%
\end{pgfscope}%
\begin{pgfscope}%
\pgfsys@transformshift{1.686987in}{3.632946in}%
\pgfsys@useobject{currentmarker}{}%
\end{pgfscope}%
\begin{pgfscope}%
\pgfsys@transformshift{1.667970in}{3.514735in}%
\pgfsys@useobject{currentmarker}{}%
\end{pgfscope}%
\begin{pgfscope}%
\pgfsys@transformshift{1.649659in}{3.477528in}%
\pgfsys@useobject{currentmarker}{}%
\end{pgfscope}%
\begin{pgfscope}%
\pgfsys@transformshift{1.631111in}{3.458084in}%
\pgfsys@useobject{currentmarker}{}%
\end{pgfscope}%
\begin{pgfscope}%
\pgfsys@transformshift{1.606696in}{3.458443in}%
\pgfsys@useobject{currentmarker}{}%
\end{pgfscope}%
\begin{pgfscope}%
\pgfsys@transformshift{1.590731in}{3.470991in}%
\pgfsys@useobject{currentmarker}{}%
\end{pgfscope}%
\begin{pgfscope}%
\pgfsys@transformshift{1.574297in}{3.499553in}%
\pgfsys@useobject{currentmarker}{}%
\end{pgfscope}%
\begin{pgfscope}%
\pgfsys@transformshift{1.554575in}{3.585422in}%
\pgfsys@useobject{currentmarker}{}%
\end{pgfscope}%
\begin{pgfscope}%
\pgfsys@transformshift{1.535089in}{3.782815in}%
\pgfsys@useobject{currentmarker}{}%
\end{pgfscope}%
\begin{pgfscope}%
\pgfsys@transformshift{1.514195in}{4.124005in}%
\pgfsys@useobject{currentmarker}{}%
\end{pgfscope}%
\begin{pgfscope}%
\pgfsys@transformshift{1.495647in}{4.266808in}%
\pgfsys@useobject{currentmarker}{}%
\end{pgfscope}%
\begin{pgfscope}%
\pgfsys@transformshift{1.475692in}{4.157558in}%
\pgfsys@useobject{currentmarker}{}%
\end{pgfscope}%
\begin{pgfscope}%
\pgfsys@transformshift{1.455267in}{4.086611in}%
\pgfsys@useobject{currentmarker}{}%
\end{pgfscope}%
\begin{pgfscope}%
\pgfsys@transformshift{1.437190in}{3.759182in}%
\pgfsys@useobject{currentmarker}{}%
\end{pgfscope}%
\begin{pgfscope}%
\pgfsys@transformshift{1.418408in}{3.594678in}%
\pgfsys@useobject{currentmarker}{}%
\end{pgfscope}%
\begin{pgfscope}%
\pgfsys@transformshift{1.397748in}{3.506929in}%
\pgfsys@useobject{currentmarker}{}%
\end{pgfscope}%
\begin{pgfscope}%
\pgfsys@transformshift{1.378966in}{3.483872in}%
\pgfsys@useobject{currentmarker}{}%
\end{pgfscope}%
\begin{pgfscope}%
\pgfsys@transformshift{1.360889in}{3.482647in}%
\pgfsys@useobject{currentmarker}{}%
\end{pgfscope}%
\begin{pgfscope}%
\pgfsys@transformshift{1.342106in}{3.460681in}%
\pgfsys@useobject{currentmarker}{}%
\end{pgfscope}%
\begin{pgfscope}%
\pgfsys@transformshift{1.323795in}{3.457994in}%
\pgfsys@useobject{currentmarker}{}%
\end{pgfscope}%
\begin{pgfscope}%
\pgfsys@transformshift{1.299378in}{3.476273in}%
\pgfsys@useobject{currentmarker}{}%
\end{pgfscope}%
\begin{pgfscope}%
\pgfsys@transformshift{1.284353in}{3.510261in}%
\pgfsys@useobject{currentmarker}{}%
\end{pgfscope}%
\begin{pgfscope}%
\pgfsys@transformshift{1.265101in}{3.609589in}%
\pgfsys@useobject{currentmarker}{}%
\end{pgfscope}%
\begin{pgfscope}%
\pgfsys@transformshift{1.245379in}{3.538478in}%
\pgfsys@useobject{currentmarker}{}%
\end{pgfscope}%
\begin{pgfscope}%
\pgfsys@transformshift{1.228242in}{3.669614in}%
\pgfsys@useobject{currentmarker}{}%
\end{pgfscope}%
\begin{pgfscope}%
\pgfsys@transformshift{1.205703in}{4.023249in}%
\pgfsys@useobject{currentmarker}{}%
\end{pgfscope}%
\begin{pgfscope}%
\pgfsys@transformshift{1.187860in}{4.244131in}%
\pgfsys@useobject{currentmarker}{}%
\end{pgfscope}%
\begin{pgfscope}%
\pgfsys@transformshift{1.167671in}{4.285872in}%
\pgfsys@useobject{currentmarker}{}%
\end{pgfscope}%
\begin{pgfscope}%
\pgfsys@transformshift{1.150298in}{4.074572in}%
\pgfsys@useobject{currentmarker}{}%
\end{pgfscope}%
\begin{pgfscope}%
\pgfsys@transformshift{1.129169in}{3.710254in}%
\pgfsys@useobject{currentmarker}{}%
\end{pgfscope}%
\begin{pgfscope}%
\pgfsys@transformshift{1.107569in}{3.562674in}%
\pgfsys@useobject{currentmarker}{}%
\end{pgfscope}%
\begin{pgfscope}%
\pgfsys@transformshift{1.091839in}{3.506657in}%
\pgfsys@useobject{currentmarker}{}%
\end{pgfscope}%
\begin{pgfscope}%
\pgfsys@transformshift{1.073527in}{3.474976in}%
\pgfsys@useobject{currentmarker}{}%
\end{pgfscope}%
\begin{pgfscope}%
\pgfsys@transformshift{1.051693in}{3.458856in}%
\pgfsys@useobject{currentmarker}{}%
\end{pgfscope}%
\begin{pgfscope}%
\pgfsys@transformshift{1.035025in}{3.460069in}%
\pgfsys@useobject{currentmarker}{}%
\end{pgfscope}%
\begin{pgfscope}%
\pgfsys@transformshift{1.012485in}{3.483087in}%
\pgfsys@useobject{currentmarker}{}%
\end{pgfscope}%
\begin{pgfscope}%
\pgfsys@transformshift{0.994408in}{3.520781in}%
\pgfsys@useobject{currentmarker}{}%
\end{pgfscope}%
\begin{pgfscope}%
\pgfsys@transformshift{0.974688in}{3.457958in}%
\pgfsys@useobject{currentmarker}{}%
\end{pgfscope}%
\begin{pgfscope}%
\pgfsys@transformshift{0.958018in}{3.470995in}%
\pgfsys@useobject{currentmarker}{}%
\end{pgfscope}%
\begin{pgfscope}%
\pgfsys@transformshift{0.938298in}{3.510003in}%
\pgfsys@useobject{currentmarker}{}%
\end{pgfscope}%
\begin{pgfscope}%
\pgfsys@transformshift{0.919986in}{3.606397in}%
\pgfsys@useobject{currentmarker}{}%
\end{pgfscope}%
\begin{pgfscope}%
\pgfsys@transformshift{0.900970in}{3.825051in}%
\pgfsys@useobject{currentmarker}{}%
\end{pgfscope}%
\begin{pgfscope}%
\pgfsys@transformshift{0.879604in}{4.192752in}%
\pgfsys@useobject{currentmarker}{}%
\end{pgfscope}%
\begin{pgfscope}%
\pgfsys@transformshift{0.859884in}{4.317610in}%
\pgfsys@useobject{currentmarker}{}%
\end{pgfscope}%
\begin{pgfscope}%
\pgfsys@transformshift{0.841807in}{4.305639in}%
\pgfsys@useobject{currentmarker}{}%
\end{pgfscope}%
\begin{pgfscope}%
\pgfsys@transformshift{0.824668in}{4.024781in}%
\pgfsys@useobject{currentmarker}{}%
\end{pgfscope}%
\begin{pgfscope}%
\pgfsys@transformshift{0.801894in}{3.729107in}%
\pgfsys@useobject{currentmarker}{}%
\end{pgfscope}%
\begin{pgfscope}%
\pgfsys@transformshift{0.784052in}{3.571221in}%
\pgfsys@useobject{currentmarker}{}%
\end{pgfscope}%
\begin{pgfscope}%
\pgfsys@transformshift{0.765506in}{3.500573in}%
\pgfsys@useobject{currentmarker}{}%
\end{pgfscope}%
\begin{pgfscope}%
\pgfsys@transformshift{0.746489in}{3.471710in}%
\pgfsys@useobject{currentmarker}{}%
\end{pgfscope}%
\begin{pgfscope}%
\pgfsys@transformshift{0.723246in}{3.460475in}%
\pgfsys@useobject{currentmarker}{}%
\end{pgfscope}%
\begin{pgfscope}%
\pgfsys@transformshift{0.706107in}{3.466693in}%
\pgfsys@useobject{currentmarker}{}%
\end{pgfscope}%
\begin{pgfscope}%
\pgfsys@transformshift{0.687561in}{3.483750in}%
\pgfsys@useobject{currentmarker}{}%
\end{pgfscope}%
\begin{pgfscope}%
\pgfsys@transformshift{0.669248in}{3.537613in}%
\pgfsys@useobject{currentmarker}{}%
\end{pgfscope}%
\begin{pgfscope}%
\pgfsys@transformshift{0.650702in}{3.669178in}%
\pgfsys@useobject{currentmarker}{}%
\end{pgfscope}%
\begin{pgfscope}%
\pgfsys@transformshift{0.650468in}{3.481633in}%
\pgfsys@useobject{currentmarker}{}%
\end{pgfscope}%
\begin{pgfscope}%
\pgfsys@transformshift{0.656101in}{3.470864in}%
\pgfsys@useobject{currentmarker}{}%
\end{pgfscope}%
\begin{pgfscope}%
\pgfsys@transformshift{0.673474in}{3.461037in}%
\pgfsys@useobject{currentmarker}{}%
\end{pgfscope}%
\begin{pgfscope}%
\pgfsys@transformshift{0.696951in}{3.495963in}%
\pgfsys@useobject{currentmarker}{}%
\end{pgfscope}%
\begin{pgfscope}%
\pgfsys@transformshift{0.715028in}{3.596109in}%
\pgfsys@useobject{currentmarker}{}%
\end{pgfscope}%
\begin{pgfscope}%
\pgfsys@transformshift{0.731699in}{3.888976in}%
\pgfsys@useobject{currentmarker}{}%
\end{pgfscope}%
\begin{pgfscope}%
\pgfsys@transformshift{0.750715in}{4.324112in}%
\pgfsys@useobject{currentmarker}{}%
\end{pgfscope}%
\begin{pgfscope}%
\pgfsys@transformshift{0.769732in}{4.266313in}%
\pgfsys@useobject{currentmarker}{}%
\end{pgfscope}%
\begin{pgfscope}%
\pgfsys@transformshift{0.790156in}{3.860283in}%
\pgfsys@useobject{currentmarker}{}%
\end{pgfscope}%
\begin{pgfscope}%
\pgfsys@transformshift{0.810816in}{3.551609in}%
\pgfsys@useobject{currentmarker}{}%
\end{pgfscope}%
\begin{pgfscope}%
\pgfsys@transformshift{0.829598in}{3.481301in}%
\pgfsys@useobject{currentmarker}{}%
\end{pgfscope}%
\begin{pgfscope}%
\pgfsys@transformshift{0.849789in}{3.458625in}%
\pgfsys@useobject{currentmarker}{}%
\end{pgfscope}%
\begin{pgfscope}%
\pgfsys@transformshift{0.868100in}{3.469233in}%
\pgfsys@useobject{currentmarker}{}%
\end{pgfscope}%
\begin{pgfscope}%
\pgfsys@transformshift{0.888526in}{3.518537in}%
\pgfsys@useobject{currentmarker}{}%
\end{pgfscope}%
\begin{pgfscope}%
\pgfsys@transformshift{0.905430in}{3.689987in}%
\pgfsys@useobject{currentmarker}{}%
\end{pgfscope}%
\begin{pgfscope}%
\pgfsys@transformshift{0.925150in}{4.105156in}%
\pgfsys@useobject{currentmarker}{}%
\end{pgfscope}%
\begin{pgfscope}%
\pgfsys@transformshift{0.943933in}{4.307397in}%
\pgfsys@useobject{currentmarker}{}%
\end{pgfscope}%
\begin{pgfscope}%
\pgfsys@transformshift{0.963418in}{4.048394in}%
\pgfsys@useobject{currentmarker}{}%
\end{pgfscope}%
\begin{pgfscope}%
\pgfsys@transformshift{0.982670in}{3.650329in}%
\pgfsys@useobject{currentmarker}{}%
\end{pgfscope}%
\begin{pgfscope}%
\pgfsys@transformshift{1.002155in}{3.505786in}%
\pgfsys@useobject{currentmarker}{}%
\end{pgfscope}%
\begin{pgfscope}%
\pgfsys@transformshift{1.020938in}{3.464503in}%
\pgfsys@useobject{currentmarker}{}%
\end{pgfscope}%
\begin{pgfscope}%
\pgfsys@transformshift{1.040658in}{3.458361in}%
\pgfsys@useobject{currentmarker}{}%
\end{pgfscope}%
\begin{pgfscope}%
\pgfsys@transformshift{1.061083in}{3.486314in}%
\pgfsys@useobject{currentmarker}{}%
\end{pgfscope}%
\begin{pgfscope}%
\pgfsys@transformshift{1.080100in}{3.571712in}%
\pgfsys@useobject{currentmarker}{}%
\end{pgfscope}%
\begin{pgfscope}%
\pgfsys@transformshift{1.098179in}{3.816533in}%
\pgfsys@useobject{currentmarker}{}%
\end{pgfscope}%
\begin{pgfscope}%
\pgfsys@transformshift{1.116725in}{4.242178in}%
\pgfsys@useobject{currentmarker}{}%
\end{pgfscope}%
\begin{pgfscope}%
\pgfsys@transformshift{1.138793in}{4.183127in}%
\pgfsys@useobject{currentmarker}{}%
\end{pgfscope}%
\begin{pgfscope}%
\pgfsys@transformshift{1.155932in}{3.931701in}%
\pgfsys@useobject{currentmarker}{}%
\end{pgfscope}%
\begin{pgfscope}%
\pgfsys@transformshift{1.174009in}{3.622150in}%
\pgfsys@useobject{currentmarker}{}%
\end{pgfscope}%
\begin{pgfscope}%
\pgfsys@transformshift{1.193261in}{3.495572in}%
\pgfsys@useobject{currentmarker}{}%
\end{pgfscope}%
\begin{pgfscope}%
\pgfsys@transformshift{1.212043in}{3.461454in}%
\pgfsys@useobject{currentmarker}{}%
\end{pgfscope}%
\begin{pgfscope}%
\pgfsys@transformshift{1.231060in}{3.457564in}%
\pgfsys@useobject{currentmarker}{}%
\end{pgfscope}%
\begin{pgfscope}%
\pgfsys@transformshift{1.249137in}{3.475334in}%
\pgfsys@useobject{currentmarker}{}%
\end{pgfscope}%
\begin{pgfscope}%
\pgfsys@transformshift{1.269091in}{3.546249in}%
\pgfsys@useobject{currentmarker}{}%
\end{pgfscope}%
\begin{pgfscope}%
\pgfsys@transformshift{1.292100in}{3.801670in}%
\pgfsys@useobject{currentmarker}{}%
\end{pgfscope}%
\begin{pgfscope}%
\pgfsys@transformshift{1.309942in}{4.194046in}%
\pgfsys@useobject{currentmarker}{}%
\end{pgfscope}%
\begin{pgfscope}%
\pgfsys@transformshift{1.325673in}{4.250927in}%
\pgfsys@useobject{currentmarker}{}%
\end{pgfscope}%
\begin{pgfscope}%
\pgfsys@transformshift{1.347507in}{3.937192in}%
\pgfsys@useobject{currentmarker}{}%
\end{pgfscope}%
\begin{pgfscope}%
\pgfsys@transformshift{1.366053in}{3.631648in}%
\pgfsys@useobject{currentmarker}{}%
\end{pgfscope}%
\begin{pgfscope}%
\pgfsys@transformshift{1.387887in}{3.487981in}%
\pgfsys@useobject{currentmarker}{}%
\end{pgfscope}%
\begin{pgfscope}%
\pgfsys@transformshift{1.404321in}{3.462244in}%
\pgfsys@useobject{currentmarker}{}%
\end{pgfscope}%
\begin{pgfscope}%
\pgfsys@transformshift{1.423572in}{3.456044in}%
\pgfsys@useobject{currentmarker}{}%
\end{pgfscope}%
\begin{pgfscope}%
\pgfsys@transformshift{1.442823in}{3.471245in}%
\pgfsys@useobject{currentmarker}{}%
\end{pgfscope}%
\begin{pgfscope}%
\pgfsys@transformshift{1.463483in}{3.464824in}%
\pgfsys@useobject{currentmarker}{}%
\end{pgfscope}%
\begin{pgfscope}%
\pgfsys@transformshift{1.479448in}{3.490371in}%
\pgfsys@useobject{currentmarker}{}%
\end{pgfscope}%
\begin{pgfscope}%
\pgfsys@transformshift{1.501751in}{3.575083in}%
\pgfsys@useobject{currentmarker}{}%
\end{pgfscope}%
\begin{pgfscope}%
\pgfsys@transformshift{1.521002in}{3.845605in}%
\pgfsys@useobject{currentmarker}{}%
\end{pgfscope}%
\begin{pgfscope}%
\pgfsys@transformshift{1.539315in}{4.197172in}%
\pgfsys@useobject{currentmarker}{}%
\end{pgfscope}%
\begin{pgfscope}%
\pgfsys@transformshift{1.557627in}{4.206909in}%
\pgfsys@useobject{currentmarker}{}%
\end{pgfscope}%
\begin{pgfscope}%
\pgfsys@transformshift{1.579930in}{3.875518in}%
\pgfsys@useobject{currentmarker}{}%
\end{pgfscope}%
\begin{pgfscope}%
\pgfsys@transformshift{1.601530in}{3.561457in}%
\pgfsys@useobject{currentmarker}{}%
\end{pgfscope}%
\begin{pgfscope}%
\pgfsys@transformshift{1.617260in}{3.500447in}%
\pgfsys@useobject{currentmarker}{}%
\end{pgfscope}%
\begin{pgfscope}%
\pgfsys@transformshift{1.633459in}{3.467000in}%
\pgfsys@useobject{currentmarker}{}%
\end{pgfscope}%
\begin{pgfscope}%
\pgfsys@transformshift{1.659283in}{3.455019in}%
\pgfsys@useobject{currentmarker}{}%
\end{pgfscope}%
\begin{pgfscope}%
\pgfsys@transformshift{1.674074in}{3.461177in}%
\pgfsys@useobject{currentmarker}{}%
\end{pgfscope}%
\begin{pgfscope}%
\pgfsys@transformshift{1.694970in}{3.502448in}%
\pgfsys@useobject{currentmarker}{}%
\end{pgfscope}%
\begin{pgfscope}%
\pgfsys@transformshift{1.713751in}{3.609587in}%
\pgfsys@useobject{currentmarker}{}%
\end{pgfscope}%
\begin{pgfscope}%
\pgfsys@transformshift{1.732533in}{3.861243in}%
\pgfsys@useobject{currentmarker}{}%
\end{pgfscope}%
\begin{pgfscope}%
\pgfsys@transformshift{1.751550in}{4.207023in}%
\pgfsys@useobject{currentmarker}{}%
\end{pgfscope}%
\begin{pgfscope}%
\pgfsys@transformshift{1.770095in}{4.151830in}%
\pgfsys@useobject{currentmarker}{}%
\end{pgfscope}%
\begin{pgfscope}%
\pgfsys@transformshift{1.790052in}{3.962174in}%
\pgfsys@useobject{currentmarker}{}%
\end{pgfscope}%
\begin{pgfscope}%
\pgfsys@transformshift{1.811181in}{3.597181in}%
\pgfsys@useobject{currentmarker}{}%
\end{pgfscope}%
\begin{pgfscope}%
\pgfsys@transformshift{1.828789in}{3.497325in}%
\pgfsys@useobject{currentmarker}{}%
\end{pgfscope}%
\begin{pgfscope}%
\pgfsys@transformshift{1.848040in}{3.469351in}%
\pgfsys@useobject{currentmarker}{}%
\end{pgfscope}%
\begin{pgfscope}%
\pgfsys@transformshift{1.865885in}{3.455015in}%
\pgfsys@useobject{currentmarker}{}%
\end{pgfscope}%
\begin{pgfscope}%
\pgfsys@transformshift{1.886544in}{3.457593in}%
\pgfsys@useobject{currentmarker}{}%
\end{pgfscope}%
\begin{pgfscope}%
\pgfsys@transformshift{1.903916in}{3.477059in}%
\pgfsys@useobject{currentmarker}{}%
\end{pgfscope}%
\begin{pgfscope}%
\pgfsys@transformshift{1.925281in}{3.552054in}%
\pgfsys@useobject{currentmarker}{}%
\end{pgfscope}%
\begin{pgfscope}%
\pgfsys@transformshift{1.943358in}{3.744913in}%
\pgfsys@useobject{currentmarker}{}%
\end{pgfscope}%
\begin{pgfscope}%
\pgfsys@transformshift{1.962375in}{4.117783in}%
\pgfsys@useobject{currentmarker}{}%
\end{pgfscope}%
\begin{pgfscope}%
\pgfsys@transformshift{1.983975in}{4.211412in}%
\pgfsys@useobject{currentmarker}{}%
\end{pgfscope}%
\begin{pgfscope}%
\pgfsys@transformshift{2.001817in}{4.158553in}%
\pgfsys@useobject{currentmarker}{}%
\end{pgfscope}%
\begin{pgfscope}%
\pgfsys@transformshift{2.021068in}{3.917361in}%
\pgfsys@useobject{currentmarker}{}%
\end{pgfscope}%
\begin{pgfscope}%
\pgfsys@transformshift{2.041023in}{3.593180in}%
\pgfsys@useobject{currentmarker}{}%
\end{pgfscope}%
\begin{pgfscope}%
\pgfsys@transformshift{2.058866in}{3.491666in}%
\pgfsys@useobject{currentmarker}{}%
\end{pgfscope}%
\begin{pgfscope}%
\pgfsys@transformshift{2.081640in}{3.459261in}%
\pgfsys@useobject{currentmarker}{}%
\end{pgfscope}%
\begin{pgfscope}%
\pgfsys@transformshift{2.098308in}{3.454129in}%
\pgfsys@useobject{currentmarker}{}%
\end{pgfscope}%
\begin{pgfscope}%
\pgfsys@transformshift{2.116150in}{3.461578in}%
\pgfsys@useobject{currentmarker}{}%
\end{pgfscope}%
\begin{pgfscope}%
\pgfsys@transformshift{2.136341in}{3.491703in}%
\pgfsys@useobject{currentmarker}{}%
\end{pgfscope}%
\begin{pgfscope}%
\pgfsys@transformshift{2.155592in}{3.570144in}%
\pgfsys@useobject{currentmarker}{}%
\end{pgfscope}%
\begin{pgfscope}%
\pgfsys@transformshift{2.173669in}{3.774740in}%
\pgfsys@useobject{currentmarker}{}%
\end{pgfscope}%
\begin{pgfscope}%
\pgfsys@transformshift{2.193157in}{4.156052in}%
\pgfsys@useobject{currentmarker}{}%
\end{pgfscope}%
\begin{pgfscope}%
\pgfsys@transformshift{2.210765in}{4.191777in}%
\pgfsys@useobject{currentmarker}{}%
\end{pgfscope}%
\begin{pgfscope}%
\pgfsys@transformshift{2.232128in}{4.014319in}%
\pgfsys@useobject{currentmarker}{}%
\end{pgfscope}%
\begin{pgfscope}%
\pgfsys@transformshift{2.249971in}{3.699595in}%
\pgfsys@useobject{currentmarker}{}%
\end{pgfscope}%
\begin{pgfscope}%
\pgfsys@transformshift{2.267579in}{3.531855in}%
\pgfsys@useobject{currentmarker}{}%
\end{pgfscope}%
\begin{pgfscope}%
\pgfsys@transformshift{2.288944in}{3.470636in}%
\pgfsys@useobject{currentmarker}{}%
\end{pgfscope}%
\begin{pgfscope}%
\pgfsys@transformshift{2.309604in}{3.464635in}%
\pgfsys@useobject{currentmarker}{}%
\end{pgfscope}%
\begin{pgfscope}%
\pgfsys@transformshift{2.327916in}{3.455534in}%
\pgfsys@useobject{currentmarker}{}%
\end{pgfscope}%
\begin{pgfscope}%
\pgfsys@transformshift{2.345524in}{3.464835in}%
\pgfsys@useobject{currentmarker}{}%
\end{pgfscope}%
\begin{pgfscope}%
\pgfsys@transformshift{2.366652in}{3.453590in}%
\pgfsys@useobject{currentmarker}{}%
\end{pgfscope}%
\begin{pgfscope}%
\pgfsys@transformshift{2.385904in}{3.458930in}%
\pgfsys@useobject{currentmarker}{}%
\end{pgfscope}%
\begin{pgfscope}%
\pgfsys@transformshift{2.406095in}{3.480052in}%
\pgfsys@useobject{currentmarker}{}%
\end{pgfscope}%
\begin{pgfscope}%
\pgfsys@transformshift{2.423703in}{3.531468in}%
\pgfsys@useobject{currentmarker}{}%
\end{pgfscope}%
\begin{pgfscope}%
\pgfsys@transformshift{2.448354in}{3.787780in}%
\pgfsys@useobject{currentmarker}{}%
\end{pgfscope}%
\begin{pgfscope}%
\pgfsys@transformshift{2.461736in}{4.075920in}%
\pgfsys@useobject{currentmarker}{}%
\end{pgfscope}%
\begin{pgfscope}%
\pgfsys@transformshift{2.480518in}{4.203967in}%
\pgfsys@useobject{currentmarker}{}%
\end{pgfscope}%
\begin{pgfscope}%
\pgfsys@transformshift{2.501882in}{4.112163in}%
\pgfsys@useobject{currentmarker}{}%
\end{pgfscope}%
\begin{pgfscope}%
\pgfsys@transformshift{2.519021in}{3.782983in}%
\pgfsys@useobject{currentmarker}{}%
\end{pgfscope}%
\begin{pgfscope}%
\pgfsys@transformshift{2.539681in}{3.555865in}%
\pgfsys@useobject{currentmarker}{}%
\end{pgfscope}%
\begin{pgfscope}%
\pgfsys@transformshift{2.557758in}{3.482518in}%
\pgfsys@useobject{currentmarker}{}%
\end{pgfscope}%
\begin{pgfscope}%
\pgfsys@transformshift{2.582175in}{3.456145in}%
\pgfsys@useobject{currentmarker}{}%
\end{pgfscope}%
\begin{pgfscope}%
\pgfsys@transformshift{2.597200in}{3.452868in}%
\pgfsys@useobject{currentmarker}{}%
\end{pgfscope}%
\begin{pgfscope}%
\pgfsys@transformshift{2.614573in}{3.462662in}%
\pgfsys@useobject{currentmarker}{}%
\end{pgfscope}%
\begin{pgfscope}%
\pgfsys@transformshift{2.635702in}{3.501025in}%
\pgfsys@useobject{currentmarker}{}%
\end{pgfscope}%
\begin{pgfscope}%
\pgfsys@transformshift{2.653545in}{3.594925in}%
\pgfsys@useobject{currentmarker}{}%
\end{pgfscope}%
\begin{pgfscope}%
\pgfsys@transformshift{2.677022in}{3.803930in}%
\pgfsys@useobject{currentmarker}{}%
\end{pgfscope}%
\begin{pgfscope}%
\pgfsys@transformshift{2.692753in}{4.115890in}%
\pgfsys@useobject{currentmarker}{}%
\end{pgfscope}%
\begin{pgfscope}%
\pgfsys@transformshift{2.713647in}{4.208418in}%
\pgfsys@useobject{currentmarker}{}%
\end{pgfscope}%
\begin{pgfscope}%
\pgfsys@transformshift{2.731958in}{4.055255in}%
\pgfsys@useobject{currentmarker}{}%
\end{pgfscope}%
\begin{pgfscope}%
\pgfsys@transformshift{2.749332in}{3.754211in}%
\pgfsys@useobject{currentmarker}{}%
\end{pgfscope}%
\begin{pgfscope}%
\pgfsys@transformshift{2.767409in}{3.599641in}%
\pgfsys@useobject{currentmarker}{}%
\end{pgfscope}%
\begin{pgfscope}%
\pgfsys@transformshift{2.787365in}{3.757070in}%
\pgfsys@useobject{currentmarker}{}%
\end{pgfscope}%
\begin{pgfscope}%
\pgfsys@transformshift{2.808025in}{4.185883in}%
\pgfsys@useobject{currentmarker}{}%
\end{pgfscope}%
\begin{pgfscope}%
\pgfsys@transformshift{2.829389in}{3.853677in}%
\pgfsys@useobject{currentmarker}{}%
\end{pgfscope}%
\begin{pgfscope}%
\pgfsys@transformshift{2.845588in}{3.624463in}%
\pgfsys@useobject{currentmarker}{}%
\end{pgfscope}%
\begin{pgfscope}%
\pgfsys@transformshift{2.868128in}{3.509009in}%
\pgfsys@useobject{currentmarker}{}%
\end{pgfscope}%
\begin{pgfscope}%
\pgfsys@transformshift{2.885736in}{3.478616in}%
\pgfsys@useobject{currentmarker}{}%
\end{pgfscope}%
\begin{pgfscope}%
\pgfsys@transformshift{2.905221in}{3.462249in}%
\pgfsys@useobject{currentmarker}{}%
\end{pgfscope}%
\begin{pgfscope}%
\pgfsys@transformshift{2.924472in}{3.455127in}%
\pgfsys@useobject{currentmarker}{}%
\end{pgfscope}%
\begin{pgfscope}%
\pgfsys@transformshift{2.942080in}{3.464961in}%
\pgfsys@useobject{currentmarker}{}%
\end{pgfscope}%
\begin{pgfscope}%
\pgfsys@transformshift{2.961801in}{3.513112in}%
\pgfsys@useobject{currentmarker}{}%
\end{pgfscope}%
\begin{pgfscope}%
\pgfsys@transformshift{2.981757in}{3.649385in}%
\pgfsys@useobject{currentmarker}{}%
\end{pgfscope}%
\begin{pgfscope}%
\pgfsys@transformshift{2.997956in}{3.820478in}%
\pgfsys@useobject{currentmarker}{}%
\end{pgfscope}%
\begin{pgfscope}%
\pgfsys@transformshift{3.021903in}{4.224478in}%
\pgfsys@useobject{currentmarker}{}%
\end{pgfscope}%
\begin{pgfscope}%
\pgfsys@transformshift{3.037399in}{4.210895in}%
\pgfsys@useobject{currentmarker}{}%
\end{pgfscope}%
\begin{pgfscope}%
\pgfsys@transformshift{3.059231in}{3.908661in}%
\pgfsys@useobject{currentmarker}{}%
\end{pgfscope}%
\begin{pgfscope}%
\pgfsys@transformshift{3.077075in}{3.632554in}%
\pgfsys@useobject{currentmarker}{}%
\end{pgfscope}%
\begin{pgfscope}%
\pgfsys@transformshift{3.097499in}{3.498677in}%
\pgfsys@useobject{currentmarker}{}%
\end{pgfscope}%
\begin{pgfscope}%
\pgfsys@transformshift{3.115578in}{3.465563in}%
\pgfsys@useobject{currentmarker}{}%
\end{pgfscope}%
\begin{pgfscope}%
\pgfsys@transformshift{3.134124in}{3.454925in}%
\pgfsys@useobject{currentmarker}{}%
\end{pgfscope}%
\begin{pgfscope}%
\pgfsys@transformshift{3.154315in}{3.462893in}%
\pgfsys@useobject{currentmarker}{}%
\end{pgfscope}%
\begin{pgfscope}%
\pgfsys@transformshift{3.172626in}{3.485005in}%
\pgfsys@useobject{currentmarker}{}%
\end{pgfscope}%
\begin{pgfscope}%
\pgfsys@transformshift{3.194226in}{3.561931in}%
\pgfsys@useobject{currentmarker}{}%
\end{pgfscope}%
\begin{pgfscope}%
\pgfsys@transformshift{3.213008in}{3.704696in}%
\pgfsys@useobject{currentmarker}{}%
\end{pgfscope}%
\begin{pgfscope}%
\pgfsys@transformshift{3.229911in}{4.062852in}%
\pgfsys@useobject{currentmarker}{}%
\end{pgfscope}%
\begin{pgfscope}%
\pgfsys@transformshift{3.251042in}{4.234400in}%
\pgfsys@useobject{currentmarker}{}%
\end{pgfscope}%
\begin{pgfscope}%
\pgfsys@transformshift{3.267004in}{4.216310in}%
\pgfsys@useobject{currentmarker}{}%
\end{pgfscope}%
\begin{pgfscope}%
\pgfsys@transformshift{3.285552in}{3.941596in}%
\pgfsys@useobject{currentmarker}{}%
\end{pgfscope}%
\begin{pgfscope}%
\pgfsys@transformshift{3.307621in}{3.643439in}%
\pgfsys@useobject{currentmarker}{}%
\end{pgfscope}%
\begin{pgfscope}%
\pgfsys@transformshift{3.325698in}{3.518069in}%
\pgfsys@useobject{currentmarker}{}%
\end{pgfscope}%
\begin{pgfscope}%
\pgfsys@transformshift{3.343540in}{3.472435in}%
\pgfsys@useobject{currentmarker}{}%
\end{pgfscope}%
\begin{pgfscope}%
\pgfsys@transformshift{3.368192in}{3.456731in}%
\pgfsys@useobject{currentmarker}{}%
\end{pgfscope}%
\begin{pgfscope}%
\pgfsys@transformshift{3.385566in}{3.458330in}%
\pgfsys@useobject{currentmarker}{}%
\end{pgfscope}%
\begin{pgfscope}%
\pgfsys@transformshift{3.404111in}{3.468278in}%
\pgfsys@useobject{currentmarker}{}%
\end{pgfscope}%
\begin{pgfscope}%
\pgfsys@transformshift{3.422425in}{3.496098in}%
\pgfsys@useobject{currentmarker}{}%
\end{pgfscope}%
\begin{pgfscope}%
\pgfsys@transformshift{3.443319in}{3.572901in}%
\pgfsys@useobject{currentmarker}{}%
\end{pgfscope}%
\begin{pgfscope}%
\pgfsys@transformshift{3.463745in}{3.736410in}%
\pgfsys@useobject{currentmarker}{}%
\end{pgfscope}%
\begin{pgfscope}%
\pgfsys@transformshift{3.479004in}{3.942452in}%
\pgfsys@useobject{currentmarker}{}%
\end{pgfscope}%
\begin{pgfscope}%
\pgfsys@transformshift{3.502247in}{4.265099in}%
\pgfsys@useobject{currentmarker}{}%
\end{pgfscope}%
\begin{pgfscope}%
\pgfsys@transformshift{3.520795in}{4.197343in}%
\pgfsys@useobject{currentmarker}{}%
\end{pgfscope}%
\begin{pgfscope}%
\pgfsys@transformshift{3.539575in}{4.039578in}%
\pgfsys@useobject{currentmarker}{}%
\end{pgfscope}%
\begin{pgfscope}%
\pgfsys@transformshift{3.556480in}{3.751339in}%
\pgfsys@useobject{currentmarker}{}%
\end{pgfscope}%
\begin{pgfscope}%
\pgfsys@transformshift{3.576905in}{3.547586in}%
\pgfsys@useobject{currentmarker}{}%
\end{pgfscope}%
\begin{pgfscope}%
\pgfsys@transformshift{3.596157in}{3.484971in}%
\pgfsys@useobject{currentmarker}{}%
\end{pgfscope}%
\begin{pgfscope}%
\pgfsys@transformshift{3.616111in}{3.463939in}%
\pgfsys@useobject{currentmarker}{}%
\end{pgfscope}%
\begin{pgfscope}%
\pgfsys@transformshift{3.636302in}{3.456137in}%
\pgfsys@useobject{currentmarker}{}%
\end{pgfscope}%
\begin{pgfscope}%
\pgfsys@transformshift{3.655319in}{3.460182in}%
\pgfsys@useobject{currentmarker}{}%
\end{pgfscope}%
\begin{pgfscope}%
\pgfsys@transformshift{3.672458in}{3.478484in}%
\pgfsys@useobject{currentmarker}{}%
\end{pgfscope}%
\begin{pgfscope}%
\pgfsys@transformshift{3.690064in}{3.516738in}%
\pgfsys@useobject{currentmarker}{}%
\end{pgfscope}%
\begin{pgfscope}%
\pgfsys@transformshift{3.710726in}{3.614511in}%
\pgfsys@useobject{currentmarker}{}%
\end{pgfscope}%
\begin{pgfscope}%
\pgfsys@transformshift{3.731855in}{3.838425in}%
\pgfsys@useobject{currentmarker}{}%
\end{pgfscope}%
\begin{pgfscope}%
\pgfsys@transformshift{3.750872in}{4.160263in}%
\pgfsys@useobject{currentmarker}{}%
\end{pgfscope}%
\begin{pgfscope}%
\pgfsys@transformshift{3.770123in}{4.301815in}%
\pgfsys@useobject{currentmarker}{}%
\end{pgfscope}%
\begin{pgfscope}%
\pgfsys@transformshift{3.788200in}{4.213249in}%
\pgfsys@useobject{currentmarker}{}%
\end{pgfscope}%
\begin{pgfscope}%
\pgfsys@transformshift{3.805808in}{3.990624in}%
\pgfsys@useobject{currentmarker}{}%
\end{pgfscope}%
\begin{pgfscope}%
\pgfsys@transformshift{3.826233in}{3.689761in}%
\pgfsys@useobject{currentmarker}{}%
\end{pgfscope}%
\begin{pgfscope}%
\pgfsys@transformshift{3.846188in}{3.587711in}%
\pgfsys@useobject{currentmarker}{}%
\end{pgfscope}%
\begin{pgfscope}%
\pgfsys@transformshift{3.863092in}{3.515872in}%
\pgfsys@useobject{currentmarker}{}%
\end{pgfscope}%
\begin{pgfscope}%
\pgfsys@transformshift{3.884692in}{3.469526in}%
\pgfsys@useobject{currentmarker}{}%
\end{pgfscope}%
\begin{pgfscope}%
\pgfsys@transformshift{3.908638in}{3.457475in}%
\pgfsys@useobject{currentmarker}{}%
\end{pgfscope}%
\begin{pgfscope}%
\pgfsys@transformshift{3.924132in}{3.460327in}%
\pgfsys@useobject{currentmarker}{}%
\end{pgfscope}%
\begin{pgfscope}%
\pgfsys@transformshift{3.941740in}{3.473531in}%
\pgfsys@useobject{currentmarker}{}%
\end{pgfscope}%
\begin{pgfscope}%
\pgfsys@transformshift{3.962400in}{3.659215in}%
\pgfsys@useobject{currentmarker}{}%
\end{pgfscope}%
\begin{pgfscope}%
\pgfsys@transformshift{3.979774in}{3.519860in}%
\pgfsys@useobject{currentmarker}{}%
\end{pgfscope}%
\begin{pgfscope}%
\pgfsys@transformshift{4.001843in}{3.471409in}%
\pgfsys@useobject{currentmarker}{}%
\end{pgfscope}%
\begin{pgfscope}%
\pgfsys@transformshift{4.022503in}{3.458416in}%
\pgfsys@useobject{currentmarker}{}%
\end{pgfscope}%
\begin{pgfscope}%
\pgfsys@transformshift{4.037528in}{3.462576in}%
\pgfsys@useobject{currentmarker}{}%
\end{pgfscope}%
\begin{pgfscope}%
\pgfsys@transformshift{4.059127in}{3.486194in}%
\pgfsys@useobject{currentmarker}{}%
\end{pgfscope}%
\begin{pgfscope}%
\pgfsys@transformshift{4.077204in}{3.532098in}%
\pgfsys@useobject{currentmarker}{}%
\end{pgfscope}%
\begin{pgfscope}%
\pgfsys@transformshift{4.094578in}{3.637710in}%
\pgfsys@useobject{currentmarker}{}%
\end{pgfscope}%
\begin{pgfscope}%
\pgfsys@transformshift{4.119229in}{4.003633in}%
\pgfsys@useobject{currentmarker}{}%
\end{pgfscope}%
\begin{pgfscope}%
\pgfsys@transformshift{4.133551in}{4.269580in}%
\pgfsys@useobject{currentmarker}{}%
\end{pgfscope}%
\begin{pgfscope}%
\pgfsys@transformshift{4.153740in}{4.349128in}%
\pgfsys@useobject{currentmarker}{}%
\end{pgfscope}%
\begin{pgfscope}%
\pgfsys@transformshift{4.171583in}{4.342894in}%
\pgfsys@useobject{currentmarker}{}%
\end{pgfscope}%
\begin{pgfscope}%
\pgfsys@transformshift{4.196000in}{4.056721in}%
\pgfsys@useobject{currentmarker}{}%
\end{pgfscope}%
\begin{pgfscope}%
\pgfsys@transformshift{4.211025in}{3.762044in}%
\pgfsys@useobject{currentmarker}{}%
\end{pgfscope}%
\begin{pgfscope}%
\pgfsys@transformshift{4.231919in}{3.556941in}%
\pgfsys@useobject{currentmarker}{}%
\end{pgfscope}%
\begin{pgfscope}%
\pgfsys@transformshift{4.250936in}{3.490641in}%
\pgfsys@useobject{currentmarker}{}%
\end{pgfscope}%
\begin{pgfscope}%
\pgfsys@transformshift{4.269484in}{3.466986in}%
\pgfsys@useobject{currentmarker}{}%
\end{pgfscope}%
\begin{pgfscope}%
\pgfsys@transformshift{4.290144in}{3.461399in}%
\pgfsys@useobject{currentmarker}{}%
\end{pgfscope}%
\begin{pgfscope}%
\pgfsys@transformshift{4.306812in}{3.475096in}%
\pgfsys@useobject{currentmarker}{}%
\end{pgfscope}%
\begin{pgfscope}%
\pgfsys@transformshift{4.327237in}{3.519919in}%
\pgfsys@useobject{currentmarker}{}%
\end{pgfscope}%
\begin{pgfscope}%
\pgfsys@transformshift{4.346958in}{3.598620in}%
\pgfsys@useobject{currentmarker}{}%
\end{pgfscope}%
\begin{pgfscope}%
\pgfsys@transformshift{4.366914in}{3.841848in}%
\pgfsys@useobject{currentmarker}{}%
\end{pgfscope}%
\begin{pgfscope}%
\pgfsys@transformshift{4.385696in}{4.091768in}%
\pgfsys@useobject{currentmarker}{}%
\end{pgfscope}%
\begin{pgfscope}%
\pgfsys@transformshift{4.403304in}{4.380193in}%
\pgfsys@useobject{currentmarker}{}%
\end{pgfscope}%
\begin{pgfscope}%
\pgfsys@transformshift{4.419267in}{4.375449in}%
\pgfsys@useobject{currentmarker}{}%
\end{pgfscope}%
\begin{pgfscope}%
\pgfsys@transformshift{4.439693in}{4.179910in}%
\pgfsys@useobject{currentmarker}{}%
\end{pgfscope}%
\begin{pgfscope}%
\pgfsys@transformshift{4.461996in}{3.794050in}%
\pgfsys@useobject{currentmarker}{}%
\end{pgfscope}%
\begin{pgfscope}%
\pgfsys@transformshift{4.479135in}{3.599636in}%
\pgfsys@useobject{currentmarker}{}%
\end{pgfscope}%
\begin{pgfscope}%
\pgfsys@transformshift{4.478901in}{3.602266in}%
\pgfsys@useobject{currentmarker}{}%
\end{pgfscope}%
\begin{pgfscope}%
\pgfsys@transformshift{4.475849in}{3.637997in}%
\pgfsys@useobject{currentmarker}{}%
\end{pgfscope}%
\begin{pgfscope}%
\pgfsys@transformshift{4.454249in}{4.079984in}%
\pgfsys@useobject{currentmarker}{}%
\end{pgfscope}%
\begin{pgfscope}%
\pgfsys@transformshift{4.435467in}{4.375168in}%
\pgfsys@useobject{currentmarker}{}%
\end{pgfscope}%
\begin{pgfscope}%
\pgfsys@transformshift{4.417859in}{4.281032in}%
\pgfsys@useobject{currentmarker}{}%
\end{pgfscope}%
\begin{pgfscope}%
\pgfsys@transformshift{4.398844in}{3.802845in}%
\pgfsys@useobject{currentmarker}{}%
\end{pgfscope}%
\begin{pgfscope}%
\pgfsys@transformshift{4.377010in}{3.549971in}%
\pgfsys@useobject{currentmarker}{}%
\end{pgfscope}%
\begin{pgfscope}%
\pgfsys@transformshift{4.358696in}{3.484336in}%
\pgfsys@useobject{currentmarker}{}%
\end{pgfscope}%
\begin{pgfscope}%
\pgfsys@transformshift{4.341325in}{3.460334in}%
\pgfsys@useobject{currentmarker}{}%
\end{pgfscope}%
\begin{pgfscope}%
\pgfsys@transformshift{4.319020in}{3.476461in}%
\pgfsys@useobject{currentmarker}{}%
\end{pgfscope}%
\begin{pgfscope}%
\pgfsys@transformshift{4.301177in}{3.537108in}%
\pgfsys@useobject{currentmarker}{}%
\end{pgfscope}%
\begin{pgfscope}%
\pgfsys@transformshift{4.283335in}{3.737188in}%
\pgfsys@useobject{currentmarker}{}%
\end{pgfscope}%
\begin{pgfscope}%
\pgfsys@transformshift{4.262909in}{4.192011in}%
\pgfsys@useobject{currentmarker}{}%
\end{pgfscope}%
\begin{pgfscope}%
\pgfsys@transformshift{4.246006in}{4.355745in}%
\pgfsys@useobject{currentmarker}{}%
\end{pgfscope}%
\begin{pgfscope}%
\pgfsys@transformshift{4.225347in}{4.071024in}%
\pgfsys@useobject{currentmarker}{}%
\end{pgfscope}%
\begin{pgfscope}%
\pgfsys@transformshift{4.205624in}{3.625451in}%
\pgfsys@useobject{currentmarker}{}%
\end{pgfscope}%
\begin{pgfscope}%
\pgfsys@transformshift{4.187079in}{3.506242in}%
\pgfsys@useobject{currentmarker}{}%
\end{pgfscope}%
\begin{pgfscope}%
\pgfsys@transformshift{4.166653in}{3.464610in}%
\pgfsys@useobject{currentmarker}{}%
\end{pgfscope}%
\begin{pgfscope}%
\pgfsys@transformshift{4.167357in}{3.459262in}%
\pgfsys@useobject{currentmarker}{}%
\end{pgfscope}%
\begin{pgfscope}%
\pgfsys@transformshift{4.149514in}{3.459862in}%
\pgfsys@useobject{currentmarker}{}%
\end{pgfscope}%
\begin{pgfscope}%
\pgfsys@transformshift{4.129089in}{3.494063in}%
\pgfsys@useobject{currentmarker}{}%
\end{pgfscope}%
\begin{pgfscope}%
\pgfsys@transformshift{4.110777in}{3.587264in}%
\pgfsys@useobject{currentmarker}{}%
\end{pgfscope}%
\begin{pgfscope}%
\pgfsys@transformshift{4.090352in}{3.959593in}%
\pgfsys@useobject{currentmarker}{}%
\end{pgfscope}%
\begin{pgfscope}%
\pgfsys@transformshift{4.072275in}{4.273037in}%
\pgfsys@useobject{currentmarker}{}%
\end{pgfscope}%
\begin{pgfscope}%
\pgfsys@transformshift{4.051146in}{4.226568in}%
\pgfsys@useobject{currentmarker}{}%
\end{pgfscope}%
\begin{pgfscope}%
\pgfsys@transformshift{4.031424in}{3.724242in}%
\pgfsys@useobject{currentmarker}{}%
\end{pgfscope}%
\begin{pgfscope}%
\pgfsys@transformshift{4.013347in}{3.542499in}%
\pgfsys@useobject{currentmarker}{}%
\end{pgfscope}%
\begin{pgfscope}%
\pgfsys@transformshift{3.993861in}{3.474498in}%
\pgfsys@useobject{currentmarker}{}%
\end{pgfscope}%
\begin{pgfscope}%
\pgfsys@transformshift{3.974139in}{3.457385in}%
\pgfsys@useobject{currentmarker}{}%
\end{pgfscope}%
\begin{pgfscope}%
\pgfsys@transformshift{3.956297in}{3.467661in}%
\pgfsys@useobject{currentmarker}{}%
\end{pgfscope}%
\begin{pgfscope}%
\pgfsys@transformshift{3.935402in}{3.525947in}%
\pgfsys@useobject{currentmarker}{}%
\end{pgfscope}%
\begin{pgfscope}%
\pgfsys@transformshift{3.918500in}{3.699003in}%
\pgfsys@useobject{currentmarker}{}%
\end{pgfscope}%
\begin{pgfscope}%
\pgfsys@transformshift{3.899012in}{4.132187in}%
\pgfsys@useobject{currentmarker}{}%
\end{pgfscope}%
\begin{pgfscope}%
\pgfsys@transformshift{3.878352in}{4.288099in}%
\pgfsys@useobject{currentmarker}{}%
\end{pgfscope}%
\begin{pgfscope}%
\pgfsys@transformshift{3.861215in}{4.003646in}%
\pgfsys@useobject{currentmarker}{}%
\end{pgfscope}%
\begin{pgfscope}%
\pgfsys@transformshift{3.838675in}{3.614122in}%
\pgfsys@useobject{currentmarker}{}%
\end{pgfscope}%
\begin{pgfscope}%
\pgfsys@transformshift{3.821067in}{3.501402in}%
\pgfsys@useobject{currentmarker}{}%
\end{pgfscope}%
\begin{pgfscope}%
\pgfsys@transformshift{3.803696in}{3.468319in}%
\pgfsys@useobject{currentmarker}{}%
\end{pgfscope}%
\begin{pgfscope}%
\pgfsys@transformshift{3.783036in}{3.457229in}%
\pgfsys@useobject{currentmarker}{}%
\end{pgfscope}%
\begin{pgfscope}%
\pgfsys@transformshift{3.762845in}{3.473640in}%
\pgfsys@useobject{currentmarker}{}%
\end{pgfscope}%
\begin{pgfscope}%
\pgfsys@transformshift{3.747349in}{3.556194in}%
\pgfsys@useobject{currentmarker}{}%
\end{pgfscope}%
\begin{pgfscope}%
\pgfsys@transformshift{3.725280in}{3.724510in}%
\pgfsys@useobject{currentmarker}{}%
\end{pgfscope}%
\begin{pgfscope}%
\pgfsys@transformshift{3.708377in}{4.090879in}%
\pgfsys@useobject{currentmarker}{}%
\end{pgfscope}%
\begin{pgfscope}%
\pgfsys@transformshift{3.686778in}{4.174591in}%
\pgfsys@useobject{currentmarker}{}%
\end{pgfscope}%
\begin{pgfscope}%
\pgfsys@transformshift{3.669170in}{4.257913in}%
\pgfsys@useobject{currentmarker}{}%
\end{pgfscope}%
\begin{pgfscope}%
\pgfsys@transformshift{3.647806in}{3.862755in}%
\pgfsys@useobject{currentmarker}{}%
\end{pgfscope}%
\begin{pgfscope}%
\pgfsys@transformshift{3.627381in}{3.568609in}%
\pgfsys@useobject{currentmarker}{}%
\end{pgfscope}%
\begin{pgfscope}%
\pgfsys@transformshift{3.609069in}{3.486107in}%
\pgfsys@useobject{currentmarker}{}%
\end{pgfscope}%
\begin{pgfscope}%
\pgfsys@transformshift{3.588173in}{3.462942in}%
\pgfsys@useobject{currentmarker}{}%
\end{pgfscope}%
\begin{pgfscope}%
\pgfsys@transformshift{3.572445in}{3.456000in}%
\pgfsys@useobject{currentmarker}{}%
\end{pgfscope}%
\begin{pgfscope}%
\pgfsys@transformshift{3.550142in}{3.469900in}%
\pgfsys@useobject{currentmarker}{}%
\end{pgfscope}%
\begin{pgfscope}%
\pgfsys@transformshift{3.531828in}{3.510451in}%
\pgfsys@useobject{currentmarker}{}%
\end{pgfscope}%
\begin{pgfscope}%
\pgfsys@transformshift{3.514689in}{3.634640in}%
\pgfsys@useobject{currentmarker}{}%
\end{pgfscope}%
\begin{pgfscope}%
\pgfsys@transformshift{3.494266in}{3.917782in}%
\pgfsys@useobject{currentmarker}{}%
\end{pgfscope}%
\begin{pgfscope}%
\pgfsys@transformshift{3.472900in}{4.209619in}%
\pgfsys@useobject{currentmarker}{}%
\end{pgfscope}%
\begin{pgfscope}%
\pgfsys@transformshift{3.455763in}{4.218549in}%
\pgfsys@useobject{currentmarker}{}%
\end{pgfscope}%
\begin{pgfscope}%
\pgfsys@transformshift{3.437919in}{3.901037in}%
\pgfsys@useobject{currentmarker}{}%
\end{pgfscope}%
\begin{pgfscope}%
\pgfsys@transformshift{3.418668in}{3.596182in}%
\pgfsys@useobject{currentmarker}{}%
\end{pgfscope}%
\begin{pgfscope}%
\pgfsys@transformshift{3.393547in}{3.484546in}%
\pgfsys@useobject{currentmarker}{}%
\end{pgfscope}%
\begin{pgfscope}%
\pgfsys@transformshift{3.377819in}{3.462394in}%
\pgfsys@useobject{currentmarker}{}%
\end{pgfscope}%
\begin{pgfscope}%
\pgfsys@transformshift{3.360211in}{3.455263in}%
\pgfsys@useobject{currentmarker}{}%
\end{pgfscope}%
\begin{pgfscope}%
\pgfsys@transformshift{3.342368in}{3.460693in}%
\pgfsys@useobject{currentmarker}{}%
\end{pgfscope}%
\begin{pgfscope}%
\pgfsys@transformshift{3.320534in}{3.492686in}%
\pgfsys@useobject{currentmarker}{}%
\end{pgfscope}%
\begin{pgfscope}%
\pgfsys@transformshift{3.304335in}{3.565708in}%
\pgfsys@useobject{currentmarker}{}%
\end{pgfscope}%
\begin{pgfscope}%
\pgfsys@transformshift{3.303395in}{3.730693in}%
\pgfsys@useobject{currentmarker}{}%
\end{pgfscope}%
\begin{pgfscope}%
\pgfsys@transformshift{3.282266in}{3.615854in}%
\pgfsys@useobject{currentmarker}{}%
\end{pgfscope}%
\begin{pgfscope}%
\pgfsys@transformshift{3.264187in}{3.482118in}%
\pgfsys@useobject{currentmarker}{}%
\end{pgfscope}%
\begin{pgfscope}%
\pgfsys@transformshift{3.241415in}{3.595610in}%
\pgfsys@useobject{currentmarker}{}%
\end{pgfscope}%
\begin{pgfscope}%
\pgfsys@transformshift{3.222164in}{3.897811in}%
\pgfsys@useobject{currentmarker}{}%
\end{pgfscope}%
\begin{pgfscope}%
\pgfsys@transformshift{3.205025in}{4.187892in}%
\pgfsys@useobject{currentmarker}{}%
\end{pgfscope}%
\begin{pgfscope}%
\pgfsys@transformshift{3.186713in}{4.205542in}%
\pgfsys@useobject{currentmarker}{}%
\end{pgfscope}%
\begin{pgfscope}%
\pgfsys@transformshift{3.168165in}{3.825160in}%
\pgfsys@useobject{currentmarker}{}%
\end{pgfscope}%
\begin{pgfscope}%
\pgfsys@transformshift{3.147037in}{3.559683in}%
\pgfsys@useobject{currentmarker}{}%
\end{pgfscope}%
\begin{pgfscope}%
\pgfsys@transformshift{3.131777in}{3.499057in}%
\pgfsys@useobject{currentmarker}{}%
\end{pgfscope}%
\begin{pgfscope}%
\pgfsys@transformshift{3.108769in}{3.465638in}%
\pgfsys@useobject{currentmarker}{}%
\end{pgfscope}%
\begin{pgfscope}%
\pgfsys@transformshift{3.090926in}{3.454264in}%
\pgfsys@useobject{currentmarker}{}%
\end{pgfscope}%
\begin{pgfscope}%
\pgfsys@transformshift{3.070501in}{3.467221in}%
\pgfsys@useobject{currentmarker}{}%
\end{pgfscope}%
\begin{pgfscope}%
\pgfsys@transformshift{3.054301in}{3.485212in}%
\pgfsys@useobject{currentmarker}{}%
\end{pgfscope}%
\begin{pgfscope}%
\pgfsys@transformshift{3.032233in}{3.577442in}%
\pgfsys@useobject{currentmarker}{}%
\end{pgfscope}%
\begin{pgfscope}%
\pgfsys@transformshift{3.014625in}{3.800448in}%
\pgfsys@useobject{currentmarker}{}%
\end{pgfscope}%
\begin{pgfscope}%
\pgfsys@transformshift{2.993965in}{4.150668in}%
\pgfsys@useobject{currentmarker}{}%
\end{pgfscope}%
\begin{pgfscope}%
\pgfsys@transformshift{2.975888in}{4.211991in}%
\pgfsys@useobject{currentmarker}{}%
\end{pgfscope}%
\begin{pgfscope}%
\pgfsys@transformshift{2.954523in}{3.880689in}%
\pgfsys@useobject{currentmarker}{}%
\end{pgfscope}%
\begin{pgfscope}%
\pgfsys@transformshift{2.935977in}{3.675575in}%
\pgfsys@useobject{currentmarker}{}%
\end{pgfscope}%
\begin{pgfscope}%
\pgfsys@transformshift{2.918603in}{3.532270in}%
\pgfsys@useobject{currentmarker}{}%
\end{pgfscope}%
\begin{pgfscope}%
\pgfsys@transformshift{2.897474in}{3.471013in}%
\pgfsys@useobject{currentmarker}{}%
\end{pgfscope}%
\begin{pgfscope}%
\pgfsys@transformshift{2.878692in}{3.455141in}%
\pgfsys@useobject{currentmarker}{}%
\end{pgfscope}%
\begin{pgfscope}%
\pgfsys@transformshift{2.857327in}{3.457380in}%
\pgfsys@useobject{currentmarker}{}%
\end{pgfscope}%
\begin{pgfscope}%
\pgfsys@transformshift{2.839015in}{3.478119in}%
\pgfsys@useobject{currentmarker}{}%
\end{pgfscope}%
\begin{pgfscope}%
\pgfsys@transformshift{2.821407in}{3.532074in}%
\pgfsys@useobject{currentmarker}{}%
\end{pgfscope}%
\begin{pgfscope}%
\pgfsys@transformshift{2.801922in}{3.704094in}%
\pgfsys@useobject{currentmarker}{}%
\end{pgfscope}%
\begin{pgfscope}%
\pgfsys@transformshift{2.783139in}{4.006201in}%
\pgfsys@useobject{currentmarker}{}%
\end{pgfscope}%
\begin{pgfscope}%
\pgfsys@transformshift{2.763888in}{4.210067in}%
\pgfsys@useobject{currentmarker}{}%
\end{pgfscope}%
\begin{pgfscope}%
\pgfsys@transformshift{2.745811in}{4.138789in}%
\pgfsys@useobject{currentmarker}{}%
\end{pgfscope}%
\begin{pgfscope}%
\pgfsys@transformshift{2.726324in}{3.805198in}%
\pgfsys@useobject{currentmarker}{}%
\end{pgfscope}%
\begin{pgfscope}%
\pgfsys@transformshift{2.704726in}{3.617834in}%
\pgfsys@useobject{currentmarker}{}%
\end{pgfscope}%
\begin{pgfscope}%
\pgfsys@transformshift{2.685709in}{3.505051in}%
\pgfsys@useobject{currentmarker}{}%
\end{pgfscope}%
\begin{pgfscope}%
\pgfsys@transformshift{2.667161in}{3.466911in}%
\pgfsys@useobject{currentmarker}{}%
\end{pgfscope}%
\begin{pgfscope}%
\pgfsys@transformshift{2.648850in}{3.453390in}%
\pgfsys@useobject{currentmarker}{}%
\end{pgfscope}%
\begin{pgfscope}%
\pgfsys@transformshift{2.630067in}{3.455363in}%
\pgfsys@useobject{currentmarker}{}%
\end{pgfscope}%
\begin{pgfscope}%
\pgfsys@transformshift{2.611051in}{3.469213in}%
\pgfsys@useobject{currentmarker}{}%
\end{pgfscope}%
\begin{pgfscope}%
\pgfsys@transformshift{2.590156in}{3.487825in}%
\pgfsys@useobject{currentmarker}{}%
\end{pgfscope}%
\begin{pgfscope}%
\pgfsys@transformshift{2.570671in}{3.575451in}%
\pgfsys@useobject{currentmarker}{}%
\end{pgfscope}%
\begin{pgfscope}%
\pgfsys@transformshift{2.553766in}{3.805392in}%
\pgfsys@useobject{currentmarker}{}%
\end{pgfscope}%
\begin{pgfscope}%
\pgfsys@transformshift{2.530525in}{4.135220in}%
\pgfsys@useobject{currentmarker}{}%
\end{pgfscope}%
\begin{pgfscope}%
\pgfsys@transformshift{2.514560in}{4.200481in}%
\pgfsys@useobject{currentmarker}{}%
\end{pgfscope}%
\begin{pgfscope}%
\pgfsys@transformshift{2.493195in}{4.067919in}%
\pgfsys@useobject{currentmarker}{}%
\end{pgfscope}%
\begin{pgfscope}%
\pgfsys@transformshift{2.473944in}{3.713494in}%
\pgfsys@useobject{currentmarker}{}%
\end{pgfscope}%
\begin{pgfscope}%
\pgfsys@transformshift{2.455398in}{3.810602in}%
\pgfsys@useobject{currentmarker}{}%
\end{pgfscope}%
\begin{pgfscope}%
\pgfsys@transformshift{2.437085in}{3.578503in}%
\pgfsys@useobject{currentmarker}{}%
\end{pgfscope}%
\begin{pgfscope}%
\pgfsys@transformshift{2.415487in}{3.484466in}%
\pgfsys@useobject{currentmarker}{}%
\end{pgfscope}%
\begin{pgfscope}%
\pgfsys@transformshift{2.396704in}{3.458320in}%
\pgfsys@useobject{currentmarker}{}%
\end{pgfscope}%
\begin{pgfscope}%
\pgfsys@transformshift{2.380740in}{3.453384in}%
\pgfsys@useobject{currentmarker}{}%
\end{pgfscope}%
\begin{pgfscope}%
\pgfsys@transformshift{2.359374in}{3.456828in}%
\pgfsys@useobject{currentmarker}{}%
\end{pgfscope}%
\begin{pgfscope}%
\pgfsys@transformshift{2.340828in}{3.473874in}%
\pgfsys@useobject{currentmarker}{}%
\end{pgfscope}%
\begin{pgfscope}%
\pgfsys@transformshift{2.322046in}{3.534991in}%
\pgfsys@useobject{currentmarker}{}%
\end{pgfscope}%
\begin{pgfscope}%
\pgfsys@transformshift{2.302795in}{3.705450in}%
\pgfsys@useobject{currentmarker}{}%
\end{pgfscope}%
\begin{pgfscope}%
\pgfsys@transformshift{2.284013in}{4.072311in}%
\pgfsys@useobject{currentmarker}{}%
\end{pgfscope}%
\begin{pgfscope}%
\pgfsys@transformshift{2.265701in}{4.205486in}%
\pgfsys@useobject{currentmarker}{}%
\end{pgfscope}%
\begin{pgfscope}%
\pgfsys@transformshift{2.243398in}{4.023483in}%
\pgfsys@useobject{currentmarker}{}%
\end{pgfscope}%
\begin{pgfscope}%
\pgfsys@transformshift{2.228373in}{3.692447in}%
\pgfsys@useobject{currentmarker}{}%
\end{pgfscope}%
\begin{pgfscope}%
\pgfsys@transformshift{2.206773in}{3.538872in}%
\pgfsys@useobject{currentmarker}{}%
\end{pgfscope}%
\begin{pgfscope}%
\pgfsys@transformshift{2.182122in}{3.476727in}%
\pgfsys@useobject{currentmarker}{}%
\end{pgfscope}%
\begin{pgfscope}%
\pgfsys@transformshift{2.169445in}{3.461442in}%
\pgfsys@useobject{currentmarker}{}%
\end{pgfscope}%
\begin{pgfscope}%
\pgfsys@transformshift{2.150428in}{3.452617in}%
\pgfsys@useobject{currentmarker}{}%
\end{pgfscope}%
\begin{pgfscope}%
\pgfsys@transformshift{2.130237in}{3.456882in}%
\pgfsys@useobject{currentmarker}{}%
\end{pgfscope}%
\begin{pgfscope}%
\pgfsys@transformshift{2.107700in}{3.482386in}%
\pgfsys@useobject{currentmarker}{}%
\end{pgfscope}%
\begin{pgfscope}%
\pgfsys@transformshift{2.093613in}{3.523531in}%
\pgfsys@useobject{currentmarker}{}%
\end{pgfscope}%
\begin{pgfscope}%
\pgfsys@transformshift{2.070841in}{3.705092in}%
\pgfsys@useobject{currentmarker}{}%
\end{pgfscope}%
\begin{pgfscope}%
\pgfsys@transformshift{2.051119in}{4.038819in}%
\pgfsys@useobject{currentmarker}{}%
\end{pgfscope}%
\begin{pgfscope}%
\pgfsys@transformshift{2.032573in}{4.197753in}%
\pgfsys@useobject{currentmarker}{}%
\end{pgfscope}%
\begin{pgfscope}%
\pgfsys@transformshift{2.013556in}{4.117687in}%
\pgfsys@useobject{currentmarker}{}%
\end{pgfscope}%
\begin{pgfscope}%
\pgfsys@transformshift{1.993365in}{3.985588in}%
\pgfsys@useobject{currentmarker}{}%
\end{pgfscope}%
\begin{pgfscope}%
\pgfsys@transformshift{1.974348in}{3.704793in}%
\pgfsys@useobject{currentmarker}{}%
\end{pgfscope}%
\begin{pgfscope}%
\pgfsys@transformshift{1.958383in}{3.566329in}%
\pgfsys@useobject{currentmarker}{}%
\end{pgfscope}%
\begin{pgfscope}%
\pgfsys@transformshift{1.936080in}{3.482560in}%
\pgfsys@useobject{currentmarker}{}%
\end{pgfscope}%
\begin{pgfscope}%
\pgfsys@transformshift{1.917298in}{3.460739in}%
\pgfsys@useobject{currentmarker}{}%
\end{pgfscope}%
\begin{pgfscope}%
\pgfsys@transformshift{1.900864in}{3.453389in}%
\pgfsys@useobject{currentmarker}{}%
\end{pgfscope}%
\begin{pgfscope}%
\pgfsys@transformshift{1.879970in}{3.459054in}%
\pgfsys@useobject{currentmarker}{}%
\end{pgfscope}%
\begin{pgfscope}%
\pgfsys@transformshift{1.860015in}{3.479241in}%
\pgfsys@useobject{currentmarker}{}%
\end{pgfscope}%
\begin{pgfscope}%
\pgfsys@transformshift{1.841702in}{3.509026in}%
\pgfsys@useobject{currentmarker}{}%
\end{pgfscope}%
\begin{pgfscope}%
\pgfsys@transformshift{1.819868in}{3.511305in}%
\pgfsys@useobject{currentmarker}{}%
\end{pgfscope}%
\begin{pgfscope}%
\pgfsys@transformshift{1.800851in}{3.629239in}%
\pgfsys@useobject{currentmarker}{}%
\end{pgfscope}%
\begin{pgfscope}%
\pgfsys@transformshift{1.781834in}{3.934567in}%
\pgfsys@useobject{currentmarker}{}%
\end{pgfscope}%
\begin{pgfscope}%
\pgfsys@transformshift{1.763288in}{4.181225in}%
\pgfsys@useobject{currentmarker}{}%
\end{pgfscope}%
\begin{pgfscope}%
\pgfsys@transformshift{1.744037in}{4.221587in}%
\pgfsys@useobject{currentmarker}{}%
\end{pgfscope}%
\begin{pgfscope}%
\pgfsys@transformshift{1.725255in}{3.953405in}%
\pgfsys@useobject{currentmarker}{}%
\end{pgfscope}%
\begin{pgfscope}%
\pgfsys@transformshift{1.704126in}{3.674493in}%
\pgfsys@useobject{currentmarker}{}%
\end{pgfscope}%
\begin{pgfscope}%
\pgfsys@transformshift{1.686047in}{3.544809in}%
\pgfsys@useobject{currentmarker}{}%
\end{pgfscope}%
\begin{pgfscope}%
\pgfsys@transformshift{1.670553in}{3.491819in}%
\pgfsys@useobject{currentmarker}{}%
\end{pgfscope}%
\begin{pgfscope}%
\pgfsys@transformshift{1.649659in}{3.463336in}%
\pgfsys@useobject{currentmarker}{}%
\end{pgfscope}%
\begin{pgfscope}%
\pgfsys@transformshift{1.628999in}{3.456321in}%
\pgfsys@useobject{currentmarker}{}%
\end{pgfscope}%
\begin{pgfscope}%
\pgfsys@transformshift{1.611391in}{3.454325in}%
\pgfsys@useobject{currentmarker}{}%
\end{pgfscope}%
\begin{pgfscope}%
\pgfsys@transformshift{1.592843in}{3.464356in}%
\pgfsys@useobject{currentmarker}{}%
\end{pgfscope}%
\begin{pgfscope}%
\pgfsys@transformshift{1.574061in}{3.486473in}%
\pgfsys@useobject{currentmarker}{}%
\end{pgfscope}%
\end{pgfscope}%
\begin{pgfscope}%
\pgfsetrectcap%
\pgfsetmiterjoin%
\pgfsetlinewidth{0.501875pt}%
\definecolor{currentstroke}{rgb}{0.000000,0.000000,0.000000}%
\pgfsetstrokecolor{currentstroke}%
\pgfsetdash{}{0pt}%
\pgfpathmoveto{\pgfqpoint{0.444748in}{3.403703in}}%
\pgfpathlineto{\pgfqpoint{0.444748in}{4.479825in}}%
\pgfusepath{stroke}%
\end{pgfscope}%
\begin{pgfscope}%
\pgfsetrectcap%
\pgfsetmiterjoin%
\pgfsetlinewidth{0.501875pt}%
\definecolor{currentstroke}{rgb}{0.000000,0.000000,0.000000}%
\pgfsetstrokecolor{currentstroke}%
\pgfsetdash{}{0pt}%
\pgfpathmoveto{\pgfqpoint{4.676167in}{3.403703in}}%
\pgfpathlineto{\pgfqpoint{4.676167in}{4.479825in}}%
\pgfusepath{stroke}%
\end{pgfscope}%
\begin{pgfscope}%
\pgfsetrectcap%
\pgfsetmiterjoin%
\pgfsetlinewidth{0.501875pt}%
\definecolor{currentstroke}{rgb}{0.000000,0.000000,0.000000}%
\pgfsetstrokecolor{currentstroke}%
\pgfsetdash{}{0pt}%
\pgfpathmoveto{\pgfqpoint{0.444748in}{3.403703in}}%
\pgfpathlineto{\pgfqpoint{4.676167in}{3.403703in}}%
\pgfusepath{stroke}%
\end{pgfscope}%
\begin{pgfscope}%
\pgfsetrectcap%
\pgfsetmiterjoin%
\pgfsetlinewidth{0.501875pt}%
\definecolor{currentstroke}{rgb}{0.000000,0.000000,0.000000}%
\pgfsetstrokecolor{currentstroke}%
\pgfsetdash{}{0pt}%
\pgfpathmoveto{\pgfqpoint{0.444748in}{4.479825in}}%
\pgfpathlineto{\pgfqpoint{4.676167in}{4.479825in}}%
\pgfusepath{stroke}%
\end{pgfscope}%
\begin{pgfscope}%
\definecolor{textcolor}{rgb}{0.000000,0.000000,0.000000}%
\pgfsetstrokecolor{textcolor}%
\pgfsetfillcolor{textcolor}%
\pgftext[x=2.560458in,y=4.563159in,,base]{\color{textcolor}\rmfamily\fontsize{12.000000}{14.400000}\selectfont T = \qty{3}{\kelvin}}%
\end{pgfscope}%
\begin{pgfscope}%
\pgfsetbuttcap%
\pgfsetmiterjoin%
\definecolor{currentfill}{rgb}{1.000000,1.000000,1.000000}%
\pgfsetfillcolor{currentfill}%
\pgfsetlinewidth{0.000000pt}%
\definecolor{currentstroke}{rgb}{0.000000,0.000000,0.000000}%
\pgfsetstrokecolor{currentstroke}%
\pgfsetstrokeopacity{0.000000}%
\pgfsetdash{}{0pt}%
\pgfpathmoveto{\pgfqpoint{0.444748in}{1.917688in}}%
\pgfpathlineto{\pgfqpoint{4.676167in}{1.917688in}}%
\pgfpathlineto{\pgfqpoint{4.676167in}{2.993810in}}%
\pgfpathlineto{\pgfqpoint{0.444748in}{2.993810in}}%
\pgfpathlineto{\pgfqpoint{0.444748in}{1.917688in}}%
\pgfpathclose%
\pgfusepath{fill}%
\end{pgfscope}%
\begin{pgfscope}%
\pgfsetbuttcap%
\pgfsetroundjoin%
\definecolor{currentfill}{rgb}{0.000000,0.000000,0.000000}%
\pgfsetfillcolor{currentfill}%
\pgfsetlinewidth{0.501875pt}%
\definecolor{currentstroke}{rgb}{0.000000,0.000000,0.000000}%
\pgfsetstrokecolor{currentstroke}%
\pgfsetdash{}{0pt}%
\pgfsys@defobject{currentmarker}{\pgfqpoint{0.000000in}{0.000000in}}{\pgfqpoint{0.000000in}{0.041667in}}{%
\pgfpathmoveto{\pgfqpoint{0.000000in}{0.000000in}}%
\pgfpathlineto{\pgfqpoint{0.000000in}{0.041667in}}%
\pgfusepath{stroke,fill}%
}%
\begin{pgfscope}%
\pgfsys@transformshift{0.643182in}{1.917688in}%
\pgfsys@useobject{currentmarker}{}%
\end{pgfscope}%
\end{pgfscope}%
\begin{pgfscope}%
\pgfsetbuttcap%
\pgfsetroundjoin%
\definecolor{currentfill}{rgb}{0.000000,0.000000,0.000000}%
\pgfsetfillcolor{currentfill}%
\pgfsetlinewidth{0.501875pt}%
\definecolor{currentstroke}{rgb}{0.000000,0.000000,0.000000}%
\pgfsetstrokecolor{currentstroke}%
\pgfsetdash{}{0pt}%
\pgfsys@defobject{currentmarker}{\pgfqpoint{0.000000in}{-0.041667in}}{\pgfqpoint{0.000000in}{0.000000in}}{%
\pgfpathmoveto{\pgfqpoint{0.000000in}{0.000000in}}%
\pgfpathlineto{\pgfqpoint{0.000000in}{-0.041667in}}%
\pgfusepath{stroke,fill}%
}%
\begin{pgfscope}%
\pgfsys@transformshift{0.643182in}{2.993810in}%
\pgfsys@useobject{currentmarker}{}%
\end{pgfscope}%
\end{pgfscope}%
\begin{pgfscope}%
\pgfsetbuttcap%
\pgfsetroundjoin%
\definecolor{currentfill}{rgb}{0.000000,0.000000,0.000000}%
\pgfsetfillcolor{currentfill}%
\pgfsetlinewidth{0.501875pt}%
\definecolor{currentstroke}{rgb}{0.000000,0.000000,0.000000}%
\pgfsetstrokecolor{currentstroke}%
\pgfsetdash{}{0pt}%
\pgfsys@defobject{currentmarker}{\pgfqpoint{0.000000in}{0.000000in}}{\pgfqpoint{0.000000in}{0.041667in}}{%
\pgfpathmoveto{\pgfqpoint{0.000000in}{0.000000in}}%
\pgfpathlineto{\pgfqpoint{0.000000in}{0.041667in}}%
\pgfusepath{stroke,fill}%
}%
\begin{pgfscope}%
\pgfsys@transformshift{1.123645in}{1.917688in}%
\pgfsys@useobject{currentmarker}{}%
\end{pgfscope}%
\end{pgfscope}%
\begin{pgfscope}%
\pgfsetbuttcap%
\pgfsetroundjoin%
\definecolor{currentfill}{rgb}{0.000000,0.000000,0.000000}%
\pgfsetfillcolor{currentfill}%
\pgfsetlinewidth{0.501875pt}%
\definecolor{currentstroke}{rgb}{0.000000,0.000000,0.000000}%
\pgfsetstrokecolor{currentstroke}%
\pgfsetdash{}{0pt}%
\pgfsys@defobject{currentmarker}{\pgfqpoint{0.000000in}{-0.041667in}}{\pgfqpoint{0.000000in}{0.000000in}}{%
\pgfpathmoveto{\pgfqpoint{0.000000in}{0.000000in}}%
\pgfpathlineto{\pgfqpoint{0.000000in}{-0.041667in}}%
\pgfusepath{stroke,fill}%
}%
\begin{pgfscope}%
\pgfsys@transformshift{1.123645in}{2.993810in}%
\pgfsys@useobject{currentmarker}{}%
\end{pgfscope}%
\end{pgfscope}%
\begin{pgfscope}%
\pgfsetbuttcap%
\pgfsetroundjoin%
\definecolor{currentfill}{rgb}{0.000000,0.000000,0.000000}%
\pgfsetfillcolor{currentfill}%
\pgfsetlinewidth{0.501875pt}%
\definecolor{currentstroke}{rgb}{0.000000,0.000000,0.000000}%
\pgfsetstrokecolor{currentstroke}%
\pgfsetdash{}{0pt}%
\pgfsys@defobject{currentmarker}{\pgfqpoint{0.000000in}{0.000000in}}{\pgfqpoint{0.000000in}{0.041667in}}{%
\pgfpathmoveto{\pgfqpoint{0.000000in}{0.000000in}}%
\pgfpathlineto{\pgfqpoint{0.000000in}{0.041667in}}%
\pgfusepath{stroke,fill}%
}%
\begin{pgfscope}%
\pgfsys@transformshift{1.604109in}{1.917688in}%
\pgfsys@useobject{currentmarker}{}%
\end{pgfscope}%
\end{pgfscope}%
\begin{pgfscope}%
\pgfsetbuttcap%
\pgfsetroundjoin%
\definecolor{currentfill}{rgb}{0.000000,0.000000,0.000000}%
\pgfsetfillcolor{currentfill}%
\pgfsetlinewidth{0.501875pt}%
\definecolor{currentstroke}{rgb}{0.000000,0.000000,0.000000}%
\pgfsetstrokecolor{currentstroke}%
\pgfsetdash{}{0pt}%
\pgfsys@defobject{currentmarker}{\pgfqpoint{0.000000in}{-0.041667in}}{\pgfqpoint{0.000000in}{0.000000in}}{%
\pgfpathmoveto{\pgfqpoint{0.000000in}{0.000000in}}%
\pgfpathlineto{\pgfqpoint{0.000000in}{-0.041667in}}%
\pgfusepath{stroke,fill}%
}%
\begin{pgfscope}%
\pgfsys@transformshift{1.604109in}{2.993810in}%
\pgfsys@useobject{currentmarker}{}%
\end{pgfscope}%
\end{pgfscope}%
\begin{pgfscope}%
\pgfsetbuttcap%
\pgfsetroundjoin%
\definecolor{currentfill}{rgb}{0.000000,0.000000,0.000000}%
\pgfsetfillcolor{currentfill}%
\pgfsetlinewidth{0.501875pt}%
\definecolor{currentstroke}{rgb}{0.000000,0.000000,0.000000}%
\pgfsetstrokecolor{currentstroke}%
\pgfsetdash{}{0pt}%
\pgfsys@defobject{currentmarker}{\pgfqpoint{0.000000in}{0.000000in}}{\pgfqpoint{0.000000in}{0.041667in}}{%
\pgfpathmoveto{\pgfqpoint{0.000000in}{0.000000in}}%
\pgfpathlineto{\pgfqpoint{0.000000in}{0.041667in}}%
\pgfusepath{stroke,fill}%
}%
\begin{pgfscope}%
\pgfsys@transformshift{2.084572in}{1.917688in}%
\pgfsys@useobject{currentmarker}{}%
\end{pgfscope}%
\end{pgfscope}%
\begin{pgfscope}%
\pgfsetbuttcap%
\pgfsetroundjoin%
\definecolor{currentfill}{rgb}{0.000000,0.000000,0.000000}%
\pgfsetfillcolor{currentfill}%
\pgfsetlinewidth{0.501875pt}%
\definecolor{currentstroke}{rgb}{0.000000,0.000000,0.000000}%
\pgfsetstrokecolor{currentstroke}%
\pgfsetdash{}{0pt}%
\pgfsys@defobject{currentmarker}{\pgfqpoint{0.000000in}{-0.041667in}}{\pgfqpoint{0.000000in}{0.000000in}}{%
\pgfpathmoveto{\pgfqpoint{0.000000in}{0.000000in}}%
\pgfpathlineto{\pgfqpoint{0.000000in}{-0.041667in}}%
\pgfusepath{stroke,fill}%
}%
\begin{pgfscope}%
\pgfsys@transformshift{2.084572in}{2.993810in}%
\pgfsys@useobject{currentmarker}{}%
\end{pgfscope}%
\end{pgfscope}%
\begin{pgfscope}%
\pgfsetbuttcap%
\pgfsetroundjoin%
\definecolor{currentfill}{rgb}{0.000000,0.000000,0.000000}%
\pgfsetfillcolor{currentfill}%
\pgfsetlinewidth{0.501875pt}%
\definecolor{currentstroke}{rgb}{0.000000,0.000000,0.000000}%
\pgfsetstrokecolor{currentstroke}%
\pgfsetdash{}{0pt}%
\pgfsys@defobject{currentmarker}{\pgfqpoint{0.000000in}{0.000000in}}{\pgfqpoint{0.000000in}{0.041667in}}{%
\pgfpathmoveto{\pgfqpoint{0.000000in}{0.000000in}}%
\pgfpathlineto{\pgfqpoint{0.000000in}{0.041667in}}%
\pgfusepath{stroke,fill}%
}%
\begin{pgfscope}%
\pgfsys@transformshift{2.565036in}{1.917688in}%
\pgfsys@useobject{currentmarker}{}%
\end{pgfscope}%
\end{pgfscope}%
\begin{pgfscope}%
\pgfsetbuttcap%
\pgfsetroundjoin%
\definecolor{currentfill}{rgb}{0.000000,0.000000,0.000000}%
\pgfsetfillcolor{currentfill}%
\pgfsetlinewidth{0.501875pt}%
\definecolor{currentstroke}{rgb}{0.000000,0.000000,0.000000}%
\pgfsetstrokecolor{currentstroke}%
\pgfsetdash{}{0pt}%
\pgfsys@defobject{currentmarker}{\pgfqpoint{0.000000in}{-0.041667in}}{\pgfqpoint{0.000000in}{0.000000in}}{%
\pgfpathmoveto{\pgfqpoint{0.000000in}{0.000000in}}%
\pgfpathlineto{\pgfqpoint{0.000000in}{-0.041667in}}%
\pgfusepath{stroke,fill}%
}%
\begin{pgfscope}%
\pgfsys@transformshift{2.565036in}{2.993810in}%
\pgfsys@useobject{currentmarker}{}%
\end{pgfscope}%
\end{pgfscope}%
\begin{pgfscope}%
\pgfsetbuttcap%
\pgfsetroundjoin%
\definecolor{currentfill}{rgb}{0.000000,0.000000,0.000000}%
\pgfsetfillcolor{currentfill}%
\pgfsetlinewidth{0.501875pt}%
\definecolor{currentstroke}{rgb}{0.000000,0.000000,0.000000}%
\pgfsetstrokecolor{currentstroke}%
\pgfsetdash{}{0pt}%
\pgfsys@defobject{currentmarker}{\pgfqpoint{0.000000in}{0.000000in}}{\pgfqpoint{0.000000in}{0.041667in}}{%
\pgfpathmoveto{\pgfqpoint{0.000000in}{0.000000in}}%
\pgfpathlineto{\pgfqpoint{0.000000in}{0.041667in}}%
\pgfusepath{stroke,fill}%
}%
\begin{pgfscope}%
\pgfsys@transformshift{3.045499in}{1.917688in}%
\pgfsys@useobject{currentmarker}{}%
\end{pgfscope}%
\end{pgfscope}%
\begin{pgfscope}%
\pgfsetbuttcap%
\pgfsetroundjoin%
\definecolor{currentfill}{rgb}{0.000000,0.000000,0.000000}%
\pgfsetfillcolor{currentfill}%
\pgfsetlinewidth{0.501875pt}%
\definecolor{currentstroke}{rgb}{0.000000,0.000000,0.000000}%
\pgfsetstrokecolor{currentstroke}%
\pgfsetdash{}{0pt}%
\pgfsys@defobject{currentmarker}{\pgfqpoint{0.000000in}{-0.041667in}}{\pgfqpoint{0.000000in}{0.000000in}}{%
\pgfpathmoveto{\pgfqpoint{0.000000in}{0.000000in}}%
\pgfpathlineto{\pgfqpoint{0.000000in}{-0.041667in}}%
\pgfusepath{stroke,fill}%
}%
\begin{pgfscope}%
\pgfsys@transformshift{3.045499in}{2.993810in}%
\pgfsys@useobject{currentmarker}{}%
\end{pgfscope}%
\end{pgfscope}%
\begin{pgfscope}%
\pgfsetbuttcap%
\pgfsetroundjoin%
\definecolor{currentfill}{rgb}{0.000000,0.000000,0.000000}%
\pgfsetfillcolor{currentfill}%
\pgfsetlinewidth{0.501875pt}%
\definecolor{currentstroke}{rgb}{0.000000,0.000000,0.000000}%
\pgfsetstrokecolor{currentstroke}%
\pgfsetdash{}{0pt}%
\pgfsys@defobject{currentmarker}{\pgfqpoint{0.000000in}{0.000000in}}{\pgfqpoint{0.000000in}{0.041667in}}{%
\pgfpathmoveto{\pgfqpoint{0.000000in}{0.000000in}}%
\pgfpathlineto{\pgfqpoint{0.000000in}{0.041667in}}%
\pgfusepath{stroke,fill}%
}%
\begin{pgfscope}%
\pgfsys@transformshift{3.525963in}{1.917688in}%
\pgfsys@useobject{currentmarker}{}%
\end{pgfscope}%
\end{pgfscope}%
\begin{pgfscope}%
\pgfsetbuttcap%
\pgfsetroundjoin%
\definecolor{currentfill}{rgb}{0.000000,0.000000,0.000000}%
\pgfsetfillcolor{currentfill}%
\pgfsetlinewidth{0.501875pt}%
\definecolor{currentstroke}{rgb}{0.000000,0.000000,0.000000}%
\pgfsetstrokecolor{currentstroke}%
\pgfsetdash{}{0pt}%
\pgfsys@defobject{currentmarker}{\pgfqpoint{0.000000in}{-0.041667in}}{\pgfqpoint{0.000000in}{0.000000in}}{%
\pgfpathmoveto{\pgfqpoint{0.000000in}{0.000000in}}%
\pgfpathlineto{\pgfqpoint{0.000000in}{-0.041667in}}%
\pgfusepath{stroke,fill}%
}%
\begin{pgfscope}%
\pgfsys@transformshift{3.525963in}{2.993810in}%
\pgfsys@useobject{currentmarker}{}%
\end{pgfscope}%
\end{pgfscope}%
\begin{pgfscope}%
\pgfsetbuttcap%
\pgfsetroundjoin%
\definecolor{currentfill}{rgb}{0.000000,0.000000,0.000000}%
\pgfsetfillcolor{currentfill}%
\pgfsetlinewidth{0.501875pt}%
\definecolor{currentstroke}{rgb}{0.000000,0.000000,0.000000}%
\pgfsetstrokecolor{currentstroke}%
\pgfsetdash{}{0pt}%
\pgfsys@defobject{currentmarker}{\pgfqpoint{0.000000in}{0.000000in}}{\pgfqpoint{0.000000in}{0.041667in}}{%
\pgfpathmoveto{\pgfqpoint{0.000000in}{0.000000in}}%
\pgfpathlineto{\pgfqpoint{0.000000in}{0.041667in}}%
\pgfusepath{stroke,fill}%
}%
\begin{pgfscope}%
\pgfsys@transformshift{4.006426in}{1.917688in}%
\pgfsys@useobject{currentmarker}{}%
\end{pgfscope}%
\end{pgfscope}%
\begin{pgfscope}%
\pgfsetbuttcap%
\pgfsetroundjoin%
\definecolor{currentfill}{rgb}{0.000000,0.000000,0.000000}%
\pgfsetfillcolor{currentfill}%
\pgfsetlinewidth{0.501875pt}%
\definecolor{currentstroke}{rgb}{0.000000,0.000000,0.000000}%
\pgfsetstrokecolor{currentstroke}%
\pgfsetdash{}{0pt}%
\pgfsys@defobject{currentmarker}{\pgfqpoint{0.000000in}{-0.041667in}}{\pgfqpoint{0.000000in}{0.000000in}}{%
\pgfpathmoveto{\pgfqpoint{0.000000in}{0.000000in}}%
\pgfpathlineto{\pgfqpoint{0.000000in}{-0.041667in}}%
\pgfusepath{stroke,fill}%
}%
\begin{pgfscope}%
\pgfsys@transformshift{4.006426in}{2.993810in}%
\pgfsys@useobject{currentmarker}{}%
\end{pgfscope}%
\end{pgfscope}%
\begin{pgfscope}%
\pgfsetbuttcap%
\pgfsetroundjoin%
\definecolor{currentfill}{rgb}{0.000000,0.000000,0.000000}%
\pgfsetfillcolor{currentfill}%
\pgfsetlinewidth{0.501875pt}%
\definecolor{currentstroke}{rgb}{0.000000,0.000000,0.000000}%
\pgfsetstrokecolor{currentstroke}%
\pgfsetdash{}{0pt}%
\pgfsys@defobject{currentmarker}{\pgfqpoint{0.000000in}{0.000000in}}{\pgfqpoint{0.000000in}{0.041667in}}{%
\pgfpathmoveto{\pgfqpoint{0.000000in}{0.000000in}}%
\pgfpathlineto{\pgfqpoint{0.000000in}{0.041667in}}%
\pgfusepath{stroke,fill}%
}%
\begin{pgfscope}%
\pgfsys@transformshift{4.486890in}{1.917688in}%
\pgfsys@useobject{currentmarker}{}%
\end{pgfscope}%
\end{pgfscope}%
\begin{pgfscope}%
\pgfsetbuttcap%
\pgfsetroundjoin%
\definecolor{currentfill}{rgb}{0.000000,0.000000,0.000000}%
\pgfsetfillcolor{currentfill}%
\pgfsetlinewidth{0.501875pt}%
\definecolor{currentstroke}{rgb}{0.000000,0.000000,0.000000}%
\pgfsetstrokecolor{currentstroke}%
\pgfsetdash{}{0pt}%
\pgfsys@defobject{currentmarker}{\pgfqpoint{0.000000in}{-0.041667in}}{\pgfqpoint{0.000000in}{0.000000in}}{%
\pgfpathmoveto{\pgfqpoint{0.000000in}{0.000000in}}%
\pgfpathlineto{\pgfqpoint{0.000000in}{-0.041667in}}%
\pgfusepath{stroke,fill}%
}%
\begin{pgfscope}%
\pgfsys@transformshift{4.486890in}{2.993810in}%
\pgfsys@useobject{currentmarker}{}%
\end{pgfscope}%
\end{pgfscope}%
\begin{pgfscope}%
\pgfsetbuttcap%
\pgfsetroundjoin%
\definecolor{currentfill}{rgb}{0.000000,0.000000,0.000000}%
\pgfsetfillcolor{currentfill}%
\pgfsetlinewidth{0.501875pt}%
\definecolor{currentstroke}{rgb}{0.000000,0.000000,0.000000}%
\pgfsetstrokecolor{currentstroke}%
\pgfsetdash{}{0pt}%
\pgfsys@defobject{currentmarker}{\pgfqpoint{0.000000in}{0.000000in}}{\pgfqpoint{0.000000in}{0.020833in}}{%
\pgfpathmoveto{\pgfqpoint{0.000000in}{0.000000in}}%
\pgfpathlineto{\pgfqpoint{0.000000in}{0.020833in}}%
\pgfusepath{stroke,fill}%
}%
\begin{pgfscope}%
\pgfsys@transformshift{0.450996in}{1.917688in}%
\pgfsys@useobject{currentmarker}{}%
\end{pgfscope}%
\end{pgfscope}%
\begin{pgfscope}%
\pgfsetbuttcap%
\pgfsetroundjoin%
\definecolor{currentfill}{rgb}{0.000000,0.000000,0.000000}%
\pgfsetfillcolor{currentfill}%
\pgfsetlinewidth{0.501875pt}%
\definecolor{currentstroke}{rgb}{0.000000,0.000000,0.000000}%
\pgfsetstrokecolor{currentstroke}%
\pgfsetdash{}{0pt}%
\pgfsys@defobject{currentmarker}{\pgfqpoint{0.000000in}{-0.020833in}}{\pgfqpoint{0.000000in}{0.000000in}}{%
\pgfpathmoveto{\pgfqpoint{0.000000in}{0.000000in}}%
\pgfpathlineto{\pgfqpoint{0.000000in}{-0.020833in}}%
\pgfusepath{stroke,fill}%
}%
\begin{pgfscope}%
\pgfsys@transformshift{0.450996in}{2.993810in}%
\pgfsys@useobject{currentmarker}{}%
\end{pgfscope}%
\end{pgfscope}%
\begin{pgfscope}%
\pgfsetbuttcap%
\pgfsetroundjoin%
\definecolor{currentfill}{rgb}{0.000000,0.000000,0.000000}%
\pgfsetfillcolor{currentfill}%
\pgfsetlinewidth{0.501875pt}%
\definecolor{currentstroke}{rgb}{0.000000,0.000000,0.000000}%
\pgfsetstrokecolor{currentstroke}%
\pgfsetdash{}{0pt}%
\pgfsys@defobject{currentmarker}{\pgfqpoint{0.000000in}{0.000000in}}{\pgfqpoint{0.000000in}{0.020833in}}{%
\pgfpathmoveto{\pgfqpoint{0.000000in}{0.000000in}}%
\pgfpathlineto{\pgfqpoint{0.000000in}{0.020833in}}%
\pgfusepath{stroke,fill}%
}%
\begin{pgfscope}%
\pgfsys@transformshift{0.547089in}{1.917688in}%
\pgfsys@useobject{currentmarker}{}%
\end{pgfscope}%
\end{pgfscope}%
\begin{pgfscope}%
\pgfsetbuttcap%
\pgfsetroundjoin%
\definecolor{currentfill}{rgb}{0.000000,0.000000,0.000000}%
\pgfsetfillcolor{currentfill}%
\pgfsetlinewidth{0.501875pt}%
\definecolor{currentstroke}{rgb}{0.000000,0.000000,0.000000}%
\pgfsetstrokecolor{currentstroke}%
\pgfsetdash{}{0pt}%
\pgfsys@defobject{currentmarker}{\pgfqpoint{0.000000in}{-0.020833in}}{\pgfqpoint{0.000000in}{0.000000in}}{%
\pgfpathmoveto{\pgfqpoint{0.000000in}{0.000000in}}%
\pgfpathlineto{\pgfqpoint{0.000000in}{-0.020833in}}%
\pgfusepath{stroke,fill}%
}%
\begin{pgfscope}%
\pgfsys@transformshift{0.547089in}{2.993810in}%
\pgfsys@useobject{currentmarker}{}%
\end{pgfscope}%
\end{pgfscope}%
\begin{pgfscope}%
\pgfsetbuttcap%
\pgfsetroundjoin%
\definecolor{currentfill}{rgb}{0.000000,0.000000,0.000000}%
\pgfsetfillcolor{currentfill}%
\pgfsetlinewidth{0.501875pt}%
\definecolor{currentstroke}{rgb}{0.000000,0.000000,0.000000}%
\pgfsetstrokecolor{currentstroke}%
\pgfsetdash{}{0pt}%
\pgfsys@defobject{currentmarker}{\pgfqpoint{0.000000in}{0.000000in}}{\pgfqpoint{0.000000in}{0.020833in}}{%
\pgfpathmoveto{\pgfqpoint{0.000000in}{0.000000in}}%
\pgfpathlineto{\pgfqpoint{0.000000in}{0.020833in}}%
\pgfusepath{stroke,fill}%
}%
\begin{pgfscope}%
\pgfsys@transformshift{0.739275in}{1.917688in}%
\pgfsys@useobject{currentmarker}{}%
\end{pgfscope}%
\end{pgfscope}%
\begin{pgfscope}%
\pgfsetbuttcap%
\pgfsetroundjoin%
\definecolor{currentfill}{rgb}{0.000000,0.000000,0.000000}%
\pgfsetfillcolor{currentfill}%
\pgfsetlinewidth{0.501875pt}%
\definecolor{currentstroke}{rgb}{0.000000,0.000000,0.000000}%
\pgfsetstrokecolor{currentstroke}%
\pgfsetdash{}{0pt}%
\pgfsys@defobject{currentmarker}{\pgfqpoint{0.000000in}{-0.020833in}}{\pgfqpoint{0.000000in}{0.000000in}}{%
\pgfpathmoveto{\pgfqpoint{0.000000in}{0.000000in}}%
\pgfpathlineto{\pgfqpoint{0.000000in}{-0.020833in}}%
\pgfusepath{stroke,fill}%
}%
\begin{pgfscope}%
\pgfsys@transformshift{0.739275in}{2.993810in}%
\pgfsys@useobject{currentmarker}{}%
\end{pgfscope}%
\end{pgfscope}%
\begin{pgfscope}%
\pgfsetbuttcap%
\pgfsetroundjoin%
\definecolor{currentfill}{rgb}{0.000000,0.000000,0.000000}%
\pgfsetfillcolor{currentfill}%
\pgfsetlinewidth{0.501875pt}%
\definecolor{currentstroke}{rgb}{0.000000,0.000000,0.000000}%
\pgfsetstrokecolor{currentstroke}%
\pgfsetdash{}{0pt}%
\pgfsys@defobject{currentmarker}{\pgfqpoint{0.000000in}{0.000000in}}{\pgfqpoint{0.000000in}{0.020833in}}{%
\pgfpathmoveto{\pgfqpoint{0.000000in}{0.000000in}}%
\pgfpathlineto{\pgfqpoint{0.000000in}{0.020833in}}%
\pgfusepath{stroke,fill}%
}%
\begin{pgfscope}%
\pgfsys@transformshift{0.835367in}{1.917688in}%
\pgfsys@useobject{currentmarker}{}%
\end{pgfscope}%
\end{pgfscope}%
\begin{pgfscope}%
\pgfsetbuttcap%
\pgfsetroundjoin%
\definecolor{currentfill}{rgb}{0.000000,0.000000,0.000000}%
\pgfsetfillcolor{currentfill}%
\pgfsetlinewidth{0.501875pt}%
\definecolor{currentstroke}{rgb}{0.000000,0.000000,0.000000}%
\pgfsetstrokecolor{currentstroke}%
\pgfsetdash{}{0pt}%
\pgfsys@defobject{currentmarker}{\pgfqpoint{0.000000in}{-0.020833in}}{\pgfqpoint{0.000000in}{0.000000in}}{%
\pgfpathmoveto{\pgfqpoint{0.000000in}{0.000000in}}%
\pgfpathlineto{\pgfqpoint{0.000000in}{-0.020833in}}%
\pgfusepath{stroke,fill}%
}%
\begin{pgfscope}%
\pgfsys@transformshift{0.835367in}{2.993810in}%
\pgfsys@useobject{currentmarker}{}%
\end{pgfscope}%
\end{pgfscope}%
\begin{pgfscope}%
\pgfsetbuttcap%
\pgfsetroundjoin%
\definecolor{currentfill}{rgb}{0.000000,0.000000,0.000000}%
\pgfsetfillcolor{currentfill}%
\pgfsetlinewidth{0.501875pt}%
\definecolor{currentstroke}{rgb}{0.000000,0.000000,0.000000}%
\pgfsetstrokecolor{currentstroke}%
\pgfsetdash{}{0pt}%
\pgfsys@defobject{currentmarker}{\pgfqpoint{0.000000in}{0.000000in}}{\pgfqpoint{0.000000in}{0.020833in}}{%
\pgfpathmoveto{\pgfqpoint{0.000000in}{0.000000in}}%
\pgfpathlineto{\pgfqpoint{0.000000in}{0.020833in}}%
\pgfusepath{stroke,fill}%
}%
\begin{pgfscope}%
\pgfsys@transformshift{0.931460in}{1.917688in}%
\pgfsys@useobject{currentmarker}{}%
\end{pgfscope}%
\end{pgfscope}%
\begin{pgfscope}%
\pgfsetbuttcap%
\pgfsetroundjoin%
\definecolor{currentfill}{rgb}{0.000000,0.000000,0.000000}%
\pgfsetfillcolor{currentfill}%
\pgfsetlinewidth{0.501875pt}%
\definecolor{currentstroke}{rgb}{0.000000,0.000000,0.000000}%
\pgfsetstrokecolor{currentstroke}%
\pgfsetdash{}{0pt}%
\pgfsys@defobject{currentmarker}{\pgfqpoint{0.000000in}{-0.020833in}}{\pgfqpoint{0.000000in}{0.000000in}}{%
\pgfpathmoveto{\pgfqpoint{0.000000in}{0.000000in}}%
\pgfpathlineto{\pgfqpoint{0.000000in}{-0.020833in}}%
\pgfusepath{stroke,fill}%
}%
\begin{pgfscope}%
\pgfsys@transformshift{0.931460in}{2.993810in}%
\pgfsys@useobject{currentmarker}{}%
\end{pgfscope}%
\end{pgfscope}%
\begin{pgfscope}%
\pgfsetbuttcap%
\pgfsetroundjoin%
\definecolor{currentfill}{rgb}{0.000000,0.000000,0.000000}%
\pgfsetfillcolor{currentfill}%
\pgfsetlinewidth{0.501875pt}%
\definecolor{currentstroke}{rgb}{0.000000,0.000000,0.000000}%
\pgfsetstrokecolor{currentstroke}%
\pgfsetdash{}{0pt}%
\pgfsys@defobject{currentmarker}{\pgfqpoint{0.000000in}{0.000000in}}{\pgfqpoint{0.000000in}{0.020833in}}{%
\pgfpathmoveto{\pgfqpoint{0.000000in}{0.000000in}}%
\pgfpathlineto{\pgfqpoint{0.000000in}{0.020833in}}%
\pgfusepath{stroke,fill}%
}%
\begin{pgfscope}%
\pgfsys@transformshift{1.027553in}{1.917688in}%
\pgfsys@useobject{currentmarker}{}%
\end{pgfscope}%
\end{pgfscope}%
\begin{pgfscope}%
\pgfsetbuttcap%
\pgfsetroundjoin%
\definecolor{currentfill}{rgb}{0.000000,0.000000,0.000000}%
\pgfsetfillcolor{currentfill}%
\pgfsetlinewidth{0.501875pt}%
\definecolor{currentstroke}{rgb}{0.000000,0.000000,0.000000}%
\pgfsetstrokecolor{currentstroke}%
\pgfsetdash{}{0pt}%
\pgfsys@defobject{currentmarker}{\pgfqpoint{0.000000in}{-0.020833in}}{\pgfqpoint{0.000000in}{0.000000in}}{%
\pgfpathmoveto{\pgfqpoint{0.000000in}{0.000000in}}%
\pgfpathlineto{\pgfqpoint{0.000000in}{-0.020833in}}%
\pgfusepath{stroke,fill}%
}%
\begin{pgfscope}%
\pgfsys@transformshift{1.027553in}{2.993810in}%
\pgfsys@useobject{currentmarker}{}%
\end{pgfscope}%
\end{pgfscope}%
\begin{pgfscope}%
\pgfsetbuttcap%
\pgfsetroundjoin%
\definecolor{currentfill}{rgb}{0.000000,0.000000,0.000000}%
\pgfsetfillcolor{currentfill}%
\pgfsetlinewidth{0.501875pt}%
\definecolor{currentstroke}{rgb}{0.000000,0.000000,0.000000}%
\pgfsetstrokecolor{currentstroke}%
\pgfsetdash{}{0pt}%
\pgfsys@defobject{currentmarker}{\pgfqpoint{0.000000in}{0.000000in}}{\pgfqpoint{0.000000in}{0.020833in}}{%
\pgfpathmoveto{\pgfqpoint{0.000000in}{0.000000in}}%
\pgfpathlineto{\pgfqpoint{0.000000in}{0.020833in}}%
\pgfusepath{stroke,fill}%
}%
\begin{pgfscope}%
\pgfsys@transformshift{1.219738in}{1.917688in}%
\pgfsys@useobject{currentmarker}{}%
\end{pgfscope}%
\end{pgfscope}%
\begin{pgfscope}%
\pgfsetbuttcap%
\pgfsetroundjoin%
\definecolor{currentfill}{rgb}{0.000000,0.000000,0.000000}%
\pgfsetfillcolor{currentfill}%
\pgfsetlinewidth{0.501875pt}%
\definecolor{currentstroke}{rgb}{0.000000,0.000000,0.000000}%
\pgfsetstrokecolor{currentstroke}%
\pgfsetdash{}{0pt}%
\pgfsys@defobject{currentmarker}{\pgfqpoint{0.000000in}{-0.020833in}}{\pgfqpoint{0.000000in}{0.000000in}}{%
\pgfpathmoveto{\pgfqpoint{0.000000in}{0.000000in}}%
\pgfpathlineto{\pgfqpoint{0.000000in}{-0.020833in}}%
\pgfusepath{stroke,fill}%
}%
\begin{pgfscope}%
\pgfsys@transformshift{1.219738in}{2.993810in}%
\pgfsys@useobject{currentmarker}{}%
\end{pgfscope}%
\end{pgfscope}%
\begin{pgfscope}%
\pgfsetbuttcap%
\pgfsetroundjoin%
\definecolor{currentfill}{rgb}{0.000000,0.000000,0.000000}%
\pgfsetfillcolor{currentfill}%
\pgfsetlinewidth{0.501875pt}%
\definecolor{currentstroke}{rgb}{0.000000,0.000000,0.000000}%
\pgfsetstrokecolor{currentstroke}%
\pgfsetdash{}{0pt}%
\pgfsys@defobject{currentmarker}{\pgfqpoint{0.000000in}{0.000000in}}{\pgfqpoint{0.000000in}{0.020833in}}{%
\pgfpathmoveto{\pgfqpoint{0.000000in}{0.000000in}}%
\pgfpathlineto{\pgfqpoint{0.000000in}{0.020833in}}%
\pgfusepath{stroke,fill}%
}%
\begin{pgfscope}%
\pgfsys@transformshift{1.315831in}{1.917688in}%
\pgfsys@useobject{currentmarker}{}%
\end{pgfscope}%
\end{pgfscope}%
\begin{pgfscope}%
\pgfsetbuttcap%
\pgfsetroundjoin%
\definecolor{currentfill}{rgb}{0.000000,0.000000,0.000000}%
\pgfsetfillcolor{currentfill}%
\pgfsetlinewidth{0.501875pt}%
\definecolor{currentstroke}{rgb}{0.000000,0.000000,0.000000}%
\pgfsetstrokecolor{currentstroke}%
\pgfsetdash{}{0pt}%
\pgfsys@defobject{currentmarker}{\pgfqpoint{0.000000in}{-0.020833in}}{\pgfqpoint{0.000000in}{0.000000in}}{%
\pgfpathmoveto{\pgfqpoint{0.000000in}{0.000000in}}%
\pgfpathlineto{\pgfqpoint{0.000000in}{-0.020833in}}%
\pgfusepath{stroke,fill}%
}%
\begin{pgfscope}%
\pgfsys@transformshift{1.315831in}{2.993810in}%
\pgfsys@useobject{currentmarker}{}%
\end{pgfscope}%
\end{pgfscope}%
\begin{pgfscope}%
\pgfsetbuttcap%
\pgfsetroundjoin%
\definecolor{currentfill}{rgb}{0.000000,0.000000,0.000000}%
\pgfsetfillcolor{currentfill}%
\pgfsetlinewidth{0.501875pt}%
\definecolor{currentstroke}{rgb}{0.000000,0.000000,0.000000}%
\pgfsetstrokecolor{currentstroke}%
\pgfsetdash{}{0pt}%
\pgfsys@defobject{currentmarker}{\pgfqpoint{0.000000in}{0.000000in}}{\pgfqpoint{0.000000in}{0.020833in}}{%
\pgfpathmoveto{\pgfqpoint{0.000000in}{0.000000in}}%
\pgfpathlineto{\pgfqpoint{0.000000in}{0.020833in}}%
\pgfusepath{stroke,fill}%
}%
\begin{pgfscope}%
\pgfsys@transformshift{1.411923in}{1.917688in}%
\pgfsys@useobject{currentmarker}{}%
\end{pgfscope}%
\end{pgfscope}%
\begin{pgfscope}%
\pgfsetbuttcap%
\pgfsetroundjoin%
\definecolor{currentfill}{rgb}{0.000000,0.000000,0.000000}%
\pgfsetfillcolor{currentfill}%
\pgfsetlinewidth{0.501875pt}%
\definecolor{currentstroke}{rgb}{0.000000,0.000000,0.000000}%
\pgfsetstrokecolor{currentstroke}%
\pgfsetdash{}{0pt}%
\pgfsys@defobject{currentmarker}{\pgfqpoint{0.000000in}{-0.020833in}}{\pgfqpoint{0.000000in}{0.000000in}}{%
\pgfpathmoveto{\pgfqpoint{0.000000in}{0.000000in}}%
\pgfpathlineto{\pgfqpoint{0.000000in}{-0.020833in}}%
\pgfusepath{stroke,fill}%
}%
\begin{pgfscope}%
\pgfsys@transformshift{1.411923in}{2.993810in}%
\pgfsys@useobject{currentmarker}{}%
\end{pgfscope}%
\end{pgfscope}%
\begin{pgfscope}%
\pgfsetbuttcap%
\pgfsetroundjoin%
\definecolor{currentfill}{rgb}{0.000000,0.000000,0.000000}%
\pgfsetfillcolor{currentfill}%
\pgfsetlinewidth{0.501875pt}%
\definecolor{currentstroke}{rgb}{0.000000,0.000000,0.000000}%
\pgfsetstrokecolor{currentstroke}%
\pgfsetdash{}{0pt}%
\pgfsys@defobject{currentmarker}{\pgfqpoint{0.000000in}{0.000000in}}{\pgfqpoint{0.000000in}{0.020833in}}{%
\pgfpathmoveto{\pgfqpoint{0.000000in}{0.000000in}}%
\pgfpathlineto{\pgfqpoint{0.000000in}{0.020833in}}%
\pgfusepath{stroke,fill}%
}%
\begin{pgfscope}%
\pgfsys@transformshift{1.508016in}{1.917688in}%
\pgfsys@useobject{currentmarker}{}%
\end{pgfscope}%
\end{pgfscope}%
\begin{pgfscope}%
\pgfsetbuttcap%
\pgfsetroundjoin%
\definecolor{currentfill}{rgb}{0.000000,0.000000,0.000000}%
\pgfsetfillcolor{currentfill}%
\pgfsetlinewidth{0.501875pt}%
\definecolor{currentstroke}{rgb}{0.000000,0.000000,0.000000}%
\pgfsetstrokecolor{currentstroke}%
\pgfsetdash{}{0pt}%
\pgfsys@defobject{currentmarker}{\pgfqpoint{0.000000in}{-0.020833in}}{\pgfqpoint{0.000000in}{0.000000in}}{%
\pgfpathmoveto{\pgfqpoint{0.000000in}{0.000000in}}%
\pgfpathlineto{\pgfqpoint{0.000000in}{-0.020833in}}%
\pgfusepath{stroke,fill}%
}%
\begin{pgfscope}%
\pgfsys@transformshift{1.508016in}{2.993810in}%
\pgfsys@useobject{currentmarker}{}%
\end{pgfscope}%
\end{pgfscope}%
\begin{pgfscope}%
\pgfsetbuttcap%
\pgfsetroundjoin%
\definecolor{currentfill}{rgb}{0.000000,0.000000,0.000000}%
\pgfsetfillcolor{currentfill}%
\pgfsetlinewidth{0.501875pt}%
\definecolor{currentstroke}{rgb}{0.000000,0.000000,0.000000}%
\pgfsetstrokecolor{currentstroke}%
\pgfsetdash{}{0pt}%
\pgfsys@defobject{currentmarker}{\pgfqpoint{0.000000in}{0.000000in}}{\pgfqpoint{0.000000in}{0.020833in}}{%
\pgfpathmoveto{\pgfqpoint{0.000000in}{0.000000in}}%
\pgfpathlineto{\pgfqpoint{0.000000in}{0.020833in}}%
\pgfusepath{stroke,fill}%
}%
\begin{pgfscope}%
\pgfsys@transformshift{1.700201in}{1.917688in}%
\pgfsys@useobject{currentmarker}{}%
\end{pgfscope}%
\end{pgfscope}%
\begin{pgfscope}%
\pgfsetbuttcap%
\pgfsetroundjoin%
\definecolor{currentfill}{rgb}{0.000000,0.000000,0.000000}%
\pgfsetfillcolor{currentfill}%
\pgfsetlinewidth{0.501875pt}%
\definecolor{currentstroke}{rgb}{0.000000,0.000000,0.000000}%
\pgfsetstrokecolor{currentstroke}%
\pgfsetdash{}{0pt}%
\pgfsys@defobject{currentmarker}{\pgfqpoint{0.000000in}{-0.020833in}}{\pgfqpoint{0.000000in}{0.000000in}}{%
\pgfpathmoveto{\pgfqpoint{0.000000in}{0.000000in}}%
\pgfpathlineto{\pgfqpoint{0.000000in}{-0.020833in}}%
\pgfusepath{stroke,fill}%
}%
\begin{pgfscope}%
\pgfsys@transformshift{1.700201in}{2.993810in}%
\pgfsys@useobject{currentmarker}{}%
\end{pgfscope}%
\end{pgfscope}%
\begin{pgfscope}%
\pgfsetbuttcap%
\pgfsetroundjoin%
\definecolor{currentfill}{rgb}{0.000000,0.000000,0.000000}%
\pgfsetfillcolor{currentfill}%
\pgfsetlinewidth{0.501875pt}%
\definecolor{currentstroke}{rgb}{0.000000,0.000000,0.000000}%
\pgfsetstrokecolor{currentstroke}%
\pgfsetdash{}{0pt}%
\pgfsys@defobject{currentmarker}{\pgfqpoint{0.000000in}{0.000000in}}{\pgfqpoint{0.000000in}{0.020833in}}{%
\pgfpathmoveto{\pgfqpoint{0.000000in}{0.000000in}}%
\pgfpathlineto{\pgfqpoint{0.000000in}{0.020833in}}%
\pgfusepath{stroke,fill}%
}%
\begin{pgfscope}%
\pgfsys@transformshift{1.796294in}{1.917688in}%
\pgfsys@useobject{currentmarker}{}%
\end{pgfscope}%
\end{pgfscope}%
\begin{pgfscope}%
\pgfsetbuttcap%
\pgfsetroundjoin%
\definecolor{currentfill}{rgb}{0.000000,0.000000,0.000000}%
\pgfsetfillcolor{currentfill}%
\pgfsetlinewidth{0.501875pt}%
\definecolor{currentstroke}{rgb}{0.000000,0.000000,0.000000}%
\pgfsetstrokecolor{currentstroke}%
\pgfsetdash{}{0pt}%
\pgfsys@defobject{currentmarker}{\pgfqpoint{0.000000in}{-0.020833in}}{\pgfqpoint{0.000000in}{0.000000in}}{%
\pgfpathmoveto{\pgfqpoint{0.000000in}{0.000000in}}%
\pgfpathlineto{\pgfqpoint{0.000000in}{-0.020833in}}%
\pgfusepath{stroke,fill}%
}%
\begin{pgfscope}%
\pgfsys@transformshift{1.796294in}{2.993810in}%
\pgfsys@useobject{currentmarker}{}%
\end{pgfscope}%
\end{pgfscope}%
\begin{pgfscope}%
\pgfsetbuttcap%
\pgfsetroundjoin%
\definecolor{currentfill}{rgb}{0.000000,0.000000,0.000000}%
\pgfsetfillcolor{currentfill}%
\pgfsetlinewidth{0.501875pt}%
\definecolor{currentstroke}{rgb}{0.000000,0.000000,0.000000}%
\pgfsetstrokecolor{currentstroke}%
\pgfsetdash{}{0pt}%
\pgfsys@defobject{currentmarker}{\pgfqpoint{0.000000in}{0.000000in}}{\pgfqpoint{0.000000in}{0.020833in}}{%
\pgfpathmoveto{\pgfqpoint{0.000000in}{0.000000in}}%
\pgfpathlineto{\pgfqpoint{0.000000in}{0.020833in}}%
\pgfusepath{stroke,fill}%
}%
\begin{pgfscope}%
\pgfsys@transformshift{1.892387in}{1.917688in}%
\pgfsys@useobject{currentmarker}{}%
\end{pgfscope}%
\end{pgfscope}%
\begin{pgfscope}%
\pgfsetbuttcap%
\pgfsetroundjoin%
\definecolor{currentfill}{rgb}{0.000000,0.000000,0.000000}%
\pgfsetfillcolor{currentfill}%
\pgfsetlinewidth{0.501875pt}%
\definecolor{currentstroke}{rgb}{0.000000,0.000000,0.000000}%
\pgfsetstrokecolor{currentstroke}%
\pgfsetdash{}{0pt}%
\pgfsys@defobject{currentmarker}{\pgfqpoint{0.000000in}{-0.020833in}}{\pgfqpoint{0.000000in}{0.000000in}}{%
\pgfpathmoveto{\pgfqpoint{0.000000in}{0.000000in}}%
\pgfpathlineto{\pgfqpoint{0.000000in}{-0.020833in}}%
\pgfusepath{stroke,fill}%
}%
\begin{pgfscope}%
\pgfsys@transformshift{1.892387in}{2.993810in}%
\pgfsys@useobject{currentmarker}{}%
\end{pgfscope}%
\end{pgfscope}%
\begin{pgfscope}%
\pgfsetbuttcap%
\pgfsetroundjoin%
\definecolor{currentfill}{rgb}{0.000000,0.000000,0.000000}%
\pgfsetfillcolor{currentfill}%
\pgfsetlinewidth{0.501875pt}%
\definecolor{currentstroke}{rgb}{0.000000,0.000000,0.000000}%
\pgfsetstrokecolor{currentstroke}%
\pgfsetdash{}{0pt}%
\pgfsys@defobject{currentmarker}{\pgfqpoint{0.000000in}{0.000000in}}{\pgfqpoint{0.000000in}{0.020833in}}{%
\pgfpathmoveto{\pgfqpoint{0.000000in}{0.000000in}}%
\pgfpathlineto{\pgfqpoint{0.000000in}{0.020833in}}%
\pgfusepath{stroke,fill}%
}%
\begin{pgfscope}%
\pgfsys@transformshift{1.988480in}{1.917688in}%
\pgfsys@useobject{currentmarker}{}%
\end{pgfscope}%
\end{pgfscope}%
\begin{pgfscope}%
\pgfsetbuttcap%
\pgfsetroundjoin%
\definecolor{currentfill}{rgb}{0.000000,0.000000,0.000000}%
\pgfsetfillcolor{currentfill}%
\pgfsetlinewidth{0.501875pt}%
\definecolor{currentstroke}{rgb}{0.000000,0.000000,0.000000}%
\pgfsetstrokecolor{currentstroke}%
\pgfsetdash{}{0pt}%
\pgfsys@defobject{currentmarker}{\pgfqpoint{0.000000in}{-0.020833in}}{\pgfqpoint{0.000000in}{0.000000in}}{%
\pgfpathmoveto{\pgfqpoint{0.000000in}{0.000000in}}%
\pgfpathlineto{\pgfqpoint{0.000000in}{-0.020833in}}%
\pgfusepath{stroke,fill}%
}%
\begin{pgfscope}%
\pgfsys@transformshift{1.988480in}{2.993810in}%
\pgfsys@useobject{currentmarker}{}%
\end{pgfscope}%
\end{pgfscope}%
\begin{pgfscope}%
\pgfsetbuttcap%
\pgfsetroundjoin%
\definecolor{currentfill}{rgb}{0.000000,0.000000,0.000000}%
\pgfsetfillcolor{currentfill}%
\pgfsetlinewidth{0.501875pt}%
\definecolor{currentstroke}{rgb}{0.000000,0.000000,0.000000}%
\pgfsetstrokecolor{currentstroke}%
\pgfsetdash{}{0pt}%
\pgfsys@defobject{currentmarker}{\pgfqpoint{0.000000in}{0.000000in}}{\pgfqpoint{0.000000in}{0.020833in}}{%
\pgfpathmoveto{\pgfqpoint{0.000000in}{0.000000in}}%
\pgfpathlineto{\pgfqpoint{0.000000in}{0.020833in}}%
\pgfusepath{stroke,fill}%
}%
\begin{pgfscope}%
\pgfsys@transformshift{2.180665in}{1.917688in}%
\pgfsys@useobject{currentmarker}{}%
\end{pgfscope}%
\end{pgfscope}%
\begin{pgfscope}%
\pgfsetbuttcap%
\pgfsetroundjoin%
\definecolor{currentfill}{rgb}{0.000000,0.000000,0.000000}%
\pgfsetfillcolor{currentfill}%
\pgfsetlinewidth{0.501875pt}%
\definecolor{currentstroke}{rgb}{0.000000,0.000000,0.000000}%
\pgfsetstrokecolor{currentstroke}%
\pgfsetdash{}{0pt}%
\pgfsys@defobject{currentmarker}{\pgfqpoint{0.000000in}{-0.020833in}}{\pgfqpoint{0.000000in}{0.000000in}}{%
\pgfpathmoveto{\pgfqpoint{0.000000in}{0.000000in}}%
\pgfpathlineto{\pgfqpoint{0.000000in}{-0.020833in}}%
\pgfusepath{stroke,fill}%
}%
\begin{pgfscope}%
\pgfsys@transformshift{2.180665in}{2.993810in}%
\pgfsys@useobject{currentmarker}{}%
\end{pgfscope}%
\end{pgfscope}%
\begin{pgfscope}%
\pgfsetbuttcap%
\pgfsetroundjoin%
\definecolor{currentfill}{rgb}{0.000000,0.000000,0.000000}%
\pgfsetfillcolor{currentfill}%
\pgfsetlinewidth{0.501875pt}%
\definecolor{currentstroke}{rgb}{0.000000,0.000000,0.000000}%
\pgfsetstrokecolor{currentstroke}%
\pgfsetdash{}{0pt}%
\pgfsys@defobject{currentmarker}{\pgfqpoint{0.000000in}{0.000000in}}{\pgfqpoint{0.000000in}{0.020833in}}{%
\pgfpathmoveto{\pgfqpoint{0.000000in}{0.000000in}}%
\pgfpathlineto{\pgfqpoint{0.000000in}{0.020833in}}%
\pgfusepath{stroke,fill}%
}%
\begin{pgfscope}%
\pgfsys@transformshift{2.276758in}{1.917688in}%
\pgfsys@useobject{currentmarker}{}%
\end{pgfscope}%
\end{pgfscope}%
\begin{pgfscope}%
\pgfsetbuttcap%
\pgfsetroundjoin%
\definecolor{currentfill}{rgb}{0.000000,0.000000,0.000000}%
\pgfsetfillcolor{currentfill}%
\pgfsetlinewidth{0.501875pt}%
\definecolor{currentstroke}{rgb}{0.000000,0.000000,0.000000}%
\pgfsetstrokecolor{currentstroke}%
\pgfsetdash{}{0pt}%
\pgfsys@defobject{currentmarker}{\pgfqpoint{0.000000in}{-0.020833in}}{\pgfqpoint{0.000000in}{0.000000in}}{%
\pgfpathmoveto{\pgfqpoint{0.000000in}{0.000000in}}%
\pgfpathlineto{\pgfqpoint{0.000000in}{-0.020833in}}%
\pgfusepath{stroke,fill}%
}%
\begin{pgfscope}%
\pgfsys@transformshift{2.276758in}{2.993810in}%
\pgfsys@useobject{currentmarker}{}%
\end{pgfscope}%
\end{pgfscope}%
\begin{pgfscope}%
\pgfsetbuttcap%
\pgfsetroundjoin%
\definecolor{currentfill}{rgb}{0.000000,0.000000,0.000000}%
\pgfsetfillcolor{currentfill}%
\pgfsetlinewidth{0.501875pt}%
\definecolor{currentstroke}{rgb}{0.000000,0.000000,0.000000}%
\pgfsetstrokecolor{currentstroke}%
\pgfsetdash{}{0pt}%
\pgfsys@defobject{currentmarker}{\pgfqpoint{0.000000in}{0.000000in}}{\pgfqpoint{0.000000in}{0.020833in}}{%
\pgfpathmoveto{\pgfqpoint{0.000000in}{0.000000in}}%
\pgfpathlineto{\pgfqpoint{0.000000in}{0.020833in}}%
\pgfusepath{stroke,fill}%
}%
\begin{pgfscope}%
\pgfsys@transformshift{2.372850in}{1.917688in}%
\pgfsys@useobject{currentmarker}{}%
\end{pgfscope}%
\end{pgfscope}%
\begin{pgfscope}%
\pgfsetbuttcap%
\pgfsetroundjoin%
\definecolor{currentfill}{rgb}{0.000000,0.000000,0.000000}%
\pgfsetfillcolor{currentfill}%
\pgfsetlinewidth{0.501875pt}%
\definecolor{currentstroke}{rgb}{0.000000,0.000000,0.000000}%
\pgfsetstrokecolor{currentstroke}%
\pgfsetdash{}{0pt}%
\pgfsys@defobject{currentmarker}{\pgfqpoint{0.000000in}{-0.020833in}}{\pgfqpoint{0.000000in}{0.000000in}}{%
\pgfpathmoveto{\pgfqpoint{0.000000in}{0.000000in}}%
\pgfpathlineto{\pgfqpoint{0.000000in}{-0.020833in}}%
\pgfusepath{stroke,fill}%
}%
\begin{pgfscope}%
\pgfsys@transformshift{2.372850in}{2.993810in}%
\pgfsys@useobject{currentmarker}{}%
\end{pgfscope}%
\end{pgfscope}%
\begin{pgfscope}%
\pgfsetbuttcap%
\pgfsetroundjoin%
\definecolor{currentfill}{rgb}{0.000000,0.000000,0.000000}%
\pgfsetfillcolor{currentfill}%
\pgfsetlinewidth{0.501875pt}%
\definecolor{currentstroke}{rgb}{0.000000,0.000000,0.000000}%
\pgfsetstrokecolor{currentstroke}%
\pgfsetdash{}{0pt}%
\pgfsys@defobject{currentmarker}{\pgfqpoint{0.000000in}{0.000000in}}{\pgfqpoint{0.000000in}{0.020833in}}{%
\pgfpathmoveto{\pgfqpoint{0.000000in}{0.000000in}}%
\pgfpathlineto{\pgfqpoint{0.000000in}{0.020833in}}%
\pgfusepath{stroke,fill}%
}%
\begin{pgfscope}%
\pgfsys@transformshift{2.468943in}{1.917688in}%
\pgfsys@useobject{currentmarker}{}%
\end{pgfscope}%
\end{pgfscope}%
\begin{pgfscope}%
\pgfsetbuttcap%
\pgfsetroundjoin%
\definecolor{currentfill}{rgb}{0.000000,0.000000,0.000000}%
\pgfsetfillcolor{currentfill}%
\pgfsetlinewidth{0.501875pt}%
\definecolor{currentstroke}{rgb}{0.000000,0.000000,0.000000}%
\pgfsetstrokecolor{currentstroke}%
\pgfsetdash{}{0pt}%
\pgfsys@defobject{currentmarker}{\pgfqpoint{0.000000in}{-0.020833in}}{\pgfqpoint{0.000000in}{0.000000in}}{%
\pgfpathmoveto{\pgfqpoint{0.000000in}{0.000000in}}%
\pgfpathlineto{\pgfqpoint{0.000000in}{-0.020833in}}%
\pgfusepath{stroke,fill}%
}%
\begin{pgfscope}%
\pgfsys@transformshift{2.468943in}{2.993810in}%
\pgfsys@useobject{currentmarker}{}%
\end{pgfscope}%
\end{pgfscope}%
\begin{pgfscope}%
\pgfsetbuttcap%
\pgfsetroundjoin%
\definecolor{currentfill}{rgb}{0.000000,0.000000,0.000000}%
\pgfsetfillcolor{currentfill}%
\pgfsetlinewidth{0.501875pt}%
\definecolor{currentstroke}{rgb}{0.000000,0.000000,0.000000}%
\pgfsetstrokecolor{currentstroke}%
\pgfsetdash{}{0pt}%
\pgfsys@defobject{currentmarker}{\pgfqpoint{0.000000in}{0.000000in}}{\pgfqpoint{0.000000in}{0.020833in}}{%
\pgfpathmoveto{\pgfqpoint{0.000000in}{0.000000in}}%
\pgfpathlineto{\pgfqpoint{0.000000in}{0.020833in}}%
\pgfusepath{stroke,fill}%
}%
\begin{pgfscope}%
\pgfsys@transformshift{2.661128in}{1.917688in}%
\pgfsys@useobject{currentmarker}{}%
\end{pgfscope}%
\end{pgfscope}%
\begin{pgfscope}%
\pgfsetbuttcap%
\pgfsetroundjoin%
\definecolor{currentfill}{rgb}{0.000000,0.000000,0.000000}%
\pgfsetfillcolor{currentfill}%
\pgfsetlinewidth{0.501875pt}%
\definecolor{currentstroke}{rgb}{0.000000,0.000000,0.000000}%
\pgfsetstrokecolor{currentstroke}%
\pgfsetdash{}{0pt}%
\pgfsys@defobject{currentmarker}{\pgfqpoint{0.000000in}{-0.020833in}}{\pgfqpoint{0.000000in}{0.000000in}}{%
\pgfpathmoveto{\pgfqpoint{0.000000in}{0.000000in}}%
\pgfpathlineto{\pgfqpoint{0.000000in}{-0.020833in}}%
\pgfusepath{stroke,fill}%
}%
\begin{pgfscope}%
\pgfsys@transformshift{2.661128in}{2.993810in}%
\pgfsys@useobject{currentmarker}{}%
\end{pgfscope}%
\end{pgfscope}%
\begin{pgfscope}%
\pgfsetbuttcap%
\pgfsetroundjoin%
\definecolor{currentfill}{rgb}{0.000000,0.000000,0.000000}%
\pgfsetfillcolor{currentfill}%
\pgfsetlinewidth{0.501875pt}%
\definecolor{currentstroke}{rgb}{0.000000,0.000000,0.000000}%
\pgfsetstrokecolor{currentstroke}%
\pgfsetdash{}{0pt}%
\pgfsys@defobject{currentmarker}{\pgfqpoint{0.000000in}{0.000000in}}{\pgfqpoint{0.000000in}{0.020833in}}{%
\pgfpathmoveto{\pgfqpoint{0.000000in}{0.000000in}}%
\pgfpathlineto{\pgfqpoint{0.000000in}{0.020833in}}%
\pgfusepath{stroke,fill}%
}%
\begin{pgfscope}%
\pgfsys@transformshift{2.757221in}{1.917688in}%
\pgfsys@useobject{currentmarker}{}%
\end{pgfscope}%
\end{pgfscope}%
\begin{pgfscope}%
\pgfsetbuttcap%
\pgfsetroundjoin%
\definecolor{currentfill}{rgb}{0.000000,0.000000,0.000000}%
\pgfsetfillcolor{currentfill}%
\pgfsetlinewidth{0.501875pt}%
\definecolor{currentstroke}{rgb}{0.000000,0.000000,0.000000}%
\pgfsetstrokecolor{currentstroke}%
\pgfsetdash{}{0pt}%
\pgfsys@defobject{currentmarker}{\pgfqpoint{0.000000in}{-0.020833in}}{\pgfqpoint{0.000000in}{0.000000in}}{%
\pgfpathmoveto{\pgfqpoint{0.000000in}{0.000000in}}%
\pgfpathlineto{\pgfqpoint{0.000000in}{-0.020833in}}%
\pgfusepath{stroke,fill}%
}%
\begin{pgfscope}%
\pgfsys@transformshift{2.757221in}{2.993810in}%
\pgfsys@useobject{currentmarker}{}%
\end{pgfscope}%
\end{pgfscope}%
\begin{pgfscope}%
\pgfsetbuttcap%
\pgfsetroundjoin%
\definecolor{currentfill}{rgb}{0.000000,0.000000,0.000000}%
\pgfsetfillcolor{currentfill}%
\pgfsetlinewidth{0.501875pt}%
\definecolor{currentstroke}{rgb}{0.000000,0.000000,0.000000}%
\pgfsetstrokecolor{currentstroke}%
\pgfsetdash{}{0pt}%
\pgfsys@defobject{currentmarker}{\pgfqpoint{0.000000in}{0.000000in}}{\pgfqpoint{0.000000in}{0.020833in}}{%
\pgfpathmoveto{\pgfqpoint{0.000000in}{0.000000in}}%
\pgfpathlineto{\pgfqpoint{0.000000in}{0.020833in}}%
\pgfusepath{stroke,fill}%
}%
\begin{pgfscope}%
\pgfsys@transformshift{2.853314in}{1.917688in}%
\pgfsys@useobject{currentmarker}{}%
\end{pgfscope}%
\end{pgfscope}%
\begin{pgfscope}%
\pgfsetbuttcap%
\pgfsetroundjoin%
\definecolor{currentfill}{rgb}{0.000000,0.000000,0.000000}%
\pgfsetfillcolor{currentfill}%
\pgfsetlinewidth{0.501875pt}%
\definecolor{currentstroke}{rgb}{0.000000,0.000000,0.000000}%
\pgfsetstrokecolor{currentstroke}%
\pgfsetdash{}{0pt}%
\pgfsys@defobject{currentmarker}{\pgfqpoint{0.000000in}{-0.020833in}}{\pgfqpoint{0.000000in}{0.000000in}}{%
\pgfpathmoveto{\pgfqpoint{0.000000in}{0.000000in}}%
\pgfpathlineto{\pgfqpoint{0.000000in}{-0.020833in}}%
\pgfusepath{stroke,fill}%
}%
\begin{pgfscope}%
\pgfsys@transformshift{2.853314in}{2.993810in}%
\pgfsys@useobject{currentmarker}{}%
\end{pgfscope}%
\end{pgfscope}%
\begin{pgfscope}%
\pgfsetbuttcap%
\pgfsetroundjoin%
\definecolor{currentfill}{rgb}{0.000000,0.000000,0.000000}%
\pgfsetfillcolor{currentfill}%
\pgfsetlinewidth{0.501875pt}%
\definecolor{currentstroke}{rgb}{0.000000,0.000000,0.000000}%
\pgfsetstrokecolor{currentstroke}%
\pgfsetdash{}{0pt}%
\pgfsys@defobject{currentmarker}{\pgfqpoint{0.000000in}{0.000000in}}{\pgfqpoint{0.000000in}{0.020833in}}{%
\pgfpathmoveto{\pgfqpoint{0.000000in}{0.000000in}}%
\pgfpathlineto{\pgfqpoint{0.000000in}{0.020833in}}%
\pgfusepath{stroke,fill}%
}%
\begin{pgfscope}%
\pgfsys@transformshift{2.949407in}{1.917688in}%
\pgfsys@useobject{currentmarker}{}%
\end{pgfscope}%
\end{pgfscope}%
\begin{pgfscope}%
\pgfsetbuttcap%
\pgfsetroundjoin%
\definecolor{currentfill}{rgb}{0.000000,0.000000,0.000000}%
\pgfsetfillcolor{currentfill}%
\pgfsetlinewidth{0.501875pt}%
\definecolor{currentstroke}{rgb}{0.000000,0.000000,0.000000}%
\pgfsetstrokecolor{currentstroke}%
\pgfsetdash{}{0pt}%
\pgfsys@defobject{currentmarker}{\pgfqpoint{0.000000in}{-0.020833in}}{\pgfqpoint{0.000000in}{0.000000in}}{%
\pgfpathmoveto{\pgfqpoint{0.000000in}{0.000000in}}%
\pgfpathlineto{\pgfqpoint{0.000000in}{-0.020833in}}%
\pgfusepath{stroke,fill}%
}%
\begin{pgfscope}%
\pgfsys@transformshift{2.949407in}{2.993810in}%
\pgfsys@useobject{currentmarker}{}%
\end{pgfscope}%
\end{pgfscope}%
\begin{pgfscope}%
\pgfsetbuttcap%
\pgfsetroundjoin%
\definecolor{currentfill}{rgb}{0.000000,0.000000,0.000000}%
\pgfsetfillcolor{currentfill}%
\pgfsetlinewidth{0.501875pt}%
\definecolor{currentstroke}{rgb}{0.000000,0.000000,0.000000}%
\pgfsetstrokecolor{currentstroke}%
\pgfsetdash{}{0pt}%
\pgfsys@defobject{currentmarker}{\pgfqpoint{0.000000in}{0.000000in}}{\pgfqpoint{0.000000in}{0.020833in}}{%
\pgfpathmoveto{\pgfqpoint{0.000000in}{0.000000in}}%
\pgfpathlineto{\pgfqpoint{0.000000in}{0.020833in}}%
\pgfusepath{stroke,fill}%
}%
\begin{pgfscope}%
\pgfsys@transformshift{3.141592in}{1.917688in}%
\pgfsys@useobject{currentmarker}{}%
\end{pgfscope}%
\end{pgfscope}%
\begin{pgfscope}%
\pgfsetbuttcap%
\pgfsetroundjoin%
\definecolor{currentfill}{rgb}{0.000000,0.000000,0.000000}%
\pgfsetfillcolor{currentfill}%
\pgfsetlinewidth{0.501875pt}%
\definecolor{currentstroke}{rgb}{0.000000,0.000000,0.000000}%
\pgfsetstrokecolor{currentstroke}%
\pgfsetdash{}{0pt}%
\pgfsys@defobject{currentmarker}{\pgfqpoint{0.000000in}{-0.020833in}}{\pgfqpoint{0.000000in}{0.000000in}}{%
\pgfpathmoveto{\pgfqpoint{0.000000in}{0.000000in}}%
\pgfpathlineto{\pgfqpoint{0.000000in}{-0.020833in}}%
\pgfusepath{stroke,fill}%
}%
\begin{pgfscope}%
\pgfsys@transformshift{3.141592in}{2.993810in}%
\pgfsys@useobject{currentmarker}{}%
\end{pgfscope}%
\end{pgfscope}%
\begin{pgfscope}%
\pgfsetbuttcap%
\pgfsetroundjoin%
\definecolor{currentfill}{rgb}{0.000000,0.000000,0.000000}%
\pgfsetfillcolor{currentfill}%
\pgfsetlinewidth{0.501875pt}%
\definecolor{currentstroke}{rgb}{0.000000,0.000000,0.000000}%
\pgfsetstrokecolor{currentstroke}%
\pgfsetdash{}{0pt}%
\pgfsys@defobject{currentmarker}{\pgfqpoint{0.000000in}{0.000000in}}{\pgfqpoint{0.000000in}{0.020833in}}{%
\pgfpathmoveto{\pgfqpoint{0.000000in}{0.000000in}}%
\pgfpathlineto{\pgfqpoint{0.000000in}{0.020833in}}%
\pgfusepath{stroke,fill}%
}%
\begin{pgfscope}%
\pgfsys@transformshift{3.237685in}{1.917688in}%
\pgfsys@useobject{currentmarker}{}%
\end{pgfscope}%
\end{pgfscope}%
\begin{pgfscope}%
\pgfsetbuttcap%
\pgfsetroundjoin%
\definecolor{currentfill}{rgb}{0.000000,0.000000,0.000000}%
\pgfsetfillcolor{currentfill}%
\pgfsetlinewidth{0.501875pt}%
\definecolor{currentstroke}{rgb}{0.000000,0.000000,0.000000}%
\pgfsetstrokecolor{currentstroke}%
\pgfsetdash{}{0pt}%
\pgfsys@defobject{currentmarker}{\pgfqpoint{0.000000in}{-0.020833in}}{\pgfqpoint{0.000000in}{0.000000in}}{%
\pgfpathmoveto{\pgfqpoint{0.000000in}{0.000000in}}%
\pgfpathlineto{\pgfqpoint{0.000000in}{-0.020833in}}%
\pgfusepath{stroke,fill}%
}%
\begin{pgfscope}%
\pgfsys@transformshift{3.237685in}{2.993810in}%
\pgfsys@useobject{currentmarker}{}%
\end{pgfscope}%
\end{pgfscope}%
\begin{pgfscope}%
\pgfsetbuttcap%
\pgfsetroundjoin%
\definecolor{currentfill}{rgb}{0.000000,0.000000,0.000000}%
\pgfsetfillcolor{currentfill}%
\pgfsetlinewidth{0.501875pt}%
\definecolor{currentstroke}{rgb}{0.000000,0.000000,0.000000}%
\pgfsetstrokecolor{currentstroke}%
\pgfsetdash{}{0pt}%
\pgfsys@defobject{currentmarker}{\pgfqpoint{0.000000in}{0.000000in}}{\pgfqpoint{0.000000in}{0.020833in}}{%
\pgfpathmoveto{\pgfqpoint{0.000000in}{0.000000in}}%
\pgfpathlineto{\pgfqpoint{0.000000in}{0.020833in}}%
\pgfusepath{stroke,fill}%
}%
\begin{pgfscope}%
\pgfsys@transformshift{3.333777in}{1.917688in}%
\pgfsys@useobject{currentmarker}{}%
\end{pgfscope}%
\end{pgfscope}%
\begin{pgfscope}%
\pgfsetbuttcap%
\pgfsetroundjoin%
\definecolor{currentfill}{rgb}{0.000000,0.000000,0.000000}%
\pgfsetfillcolor{currentfill}%
\pgfsetlinewidth{0.501875pt}%
\definecolor{currentstroke}{rgb}{0.000000,0.000000,0.000000}%
\pgfsetstrokecolor{currentstroke}%
\pgfsetdash{}{0pt}%
\pgfsys@defobject{currentmarker}{\pgfqpoint{0.000000in}{-0.020833in}}{\pgfqpoint{0.000000in}{0.000000in}}{%
\pgfpathmoveto{\pgfqpoint{0.000000in}{0.000000in}}%
\pgfpathlineto{\pgfqpoint{0.000000in}{-0.020833in}}%
\pgfusepath{stroke,fill}%
}%
\begin{pgfscope}%
\pgfsys@transformshift{3.333777in}{2.993810in}%
\pgfsys@useobject{currentmarker}{}%
\end{pgfscope}%
\end{pgfscope}%
\begin{pgfscope}%
\pgfsetbuttcap%
\pgfsetroundjoin%
\definecolor{currentfill}{rgb}{0.000000,0.000000,0.000000}%
\pgfsetfillcolor{currentfill}%
\pgfsetlinewidth{0.501875pt}%
\definecolor{currentstroke}{rgb}{0.000000,0.000000,0.000000}%
\pgfsetstrokecolor{currentstroke}%
\pgfsetdash{}{0pt}%
\pgfsys@defobject{currentmarker}{\pgfqpoint{0.000000in}{0.000000in}}{\pgfqpoint{0.000000in}{0.020833in}}{%
\pgfpathmoveto{\pgfqpoint{0.000000in}{0.000000in}}%
\pgfpathlineto{\pgfqpoint{0.000000in}{0.020833in}}%
\pgfusepath{stroke,fill}%
}%
\begin{pgfscope}%
\pgfsys@transformshift{3.429870in}{1.917688in}%
\pgfsys@useobject{currentmarker}{}%
\end{pgfscope}%
\end{pgfscope}%
\begin{pgfscope}%
\pgfsetbuttcap%
\pgfsetroundjoin%
\definecolor{currentfill}{rgb}{0.000000,0.000000,0.000000}%
\pgfsetfillcolor{currentfill}%
\pgfsetlinewidth{0.501875pt}%
\definecolor{currentstroke}{rgb}{0.000000,0.000000,0.000000}%
\pgfsetstrokecolor{currentstroke}%
\pgfsetdash{}{0pt}%
\pgfsys@defobject{currentmarker}{\pgfqpoint{0.000000in}{-0.020833in}}{\pgfqpoint{0.000000in}{0.000000in}}{%
\pgfpathmoveto{\pgfqpoint{0.000000in}{0.000000in}}%
\pgfpathlineto{\pgfqpoint{0.000000in}{-0.020833in}}%
\pgfusepath{stroke,fill}%
}%
\begin{pgfscope}%
\pgfsys@transformshift{3.429870in}{2.993810in}%
\pgfsys@useobject{currentmarker}{}%
\end{pgfscope}%
\end{pgfscope}%
\begin{pgfscope}%
\pgfsetbuttcap%
\pgfsetroundjoin%
\definecolor{currentfill}{rgb}{0.000000,0.000000,0.000000}%
\pgfsetfillcolor{currentfill}%
\pgfsetlinewidth{0.501875pt}%
\definecolor{currentstroke}{rgb}{0.000000,0.000000,0.000000}%
\pgfsetstrokecolor{currentstroke}%
\pgfsetdash{}{0pt}%
\pgfsys@defobject{currentmarker}{\pgfqpoint{0.000000in}{0.000000in}}{\pgfqpoint{0.000000in}{0.020833in}}{%
\pgfpathmoveto{\pgfqpoint{0.000000in}{0.000000in}}%
\pgfpathlineto{\pgfqpoint{0.000000in}{0.020833in}}%
\pgfusepath{stroke,fill}%
}%
\begin{pgfscope}%
\pgfsys@transformshift{3.622055in}{1.917688in}%
\pgfsys@useobject{currentmarker}{}%
\end{pgfscope}%
\end{pgfscope}%
\begin{pgfscope}%
\pgfsetbuttcap%
\pgfsetroundjoin%
\definecolor{currentfill}{rgb}{0.000000,0.000000,0.000000}%
\pgfsetfillcolor{currentfill}%
\pgfsetlinewidth{0.501875pt}%
\definecolor{currentstroke}{rgb}{0.000000,0.000000,0.000000}%
\pgfsetstrokecolor{currentstroke}%
\pgfsetdash{}{0pt}%
\pgfsys@defobject{currentmarker}{\pgfqpoint{0.000000in}{-0.020833in}}{\pgfqpoint{0.000000in}{0.000000in}}{%
\pgfpathmoveto{\pgfqpoint{0.000000in}{0.000000in}}%
\pgfpathlineto{\pgfqpoint{0.000000in}{-0.020833in}}%
\pgfusepath{stroke,fill}%
}%
\begin{pgfscope}%
\pgfsys@transformshift{3.622055in}{2.993810in}%
\pgfsys@useobject{currentmarker}{}%
\end{pgfscope}%
\end{pgfscope}%
\begin{pgfscope}%
\pgfsetbuttcap%
\pgfsetroundjoin%
\definecolor{currentfill}{rgb}{0.000000,0.000000,0.000000}%
\pgfsetfillcolor{currentfill}%
\pgfsetlinewidth{0.501875pt}%
\definecolor{currentstroke}{rgb}{0.000000,0.000000,0.000000}%
\pgfsetstrokecolor{currentstroke}%
\pgfsetdash{}{0pt}%
\pgfsys@defobject{currentmarker}{\pgfqpoint{0.000000in}{0.000000in}}{\pgfqpoint{0.000000in}{0.020833in}}{%
\pgfpathmoveto{\pgfqpoint{0.000000in}{0.000000in}}%
\pgfpathlineto{\pgfqpoint{0.000000in}{0.020833in}}%
\pgfusepath{stroke,fill}%
}%
\begin{pgfscope}%
\pgfsys@transformshift{3.718148in}{1.917688in}%
\pgfsys@useobject{currentmarker}{}%
\end{pgfscope}%
\end{pgfscope}%
\begin{pgfscope}%
\pgfsetbuttcap%
\pgfsetroundjoin%
\definecolor{currentfill}{rgb}{0.000000,0.000000,0.000000}%
\pgfsetfillcolor{currentfill}%
\pgfsetlinewidth{0.501875pt}%
\definecolor{currentstroke}{rgb}{0.000000,0.000000,0.000000}%
\pgfsetstrokecolor{currentstroke}%
\pgfsetdash{}{0pt}%
\pgfsys@defobject{currentmarker}{\pgfqpoint{0.000000in}{-0.020833in}}{\pgfqpoint{0.000000in}{0.000000in}}{%
\pgfpathmoveto{\pgfqpoint{0.000000in}{0.000000in}}%
\pgfpathlineto{\pgfqpoint{0.000000in}{-0.020833in}}%
\pgfusepath{stroke,fill}%
}%
\begin{pgfscope}%
\pgfsys@transformshift{3.718148in}{2.993810in}%
\pgfsys@useobject{currentmarker}{}%
\end{pgfscope}%
\end{pgfscope}%
\begin{pgfscope}%
\pgfsetbuttcap%
\pgfsetroundjoin%
\definecolor{currentfill}{rgb}{0.000000,0.000000,0.000000}%
\pgfsetfillcolor{currentfill}%
\pgfsetlinewidth{0.501875pt}%
\definecolor{currentstroke}{rgb}{0.000000,0.000000,0.000000}%
\pgfsetstrokecolor{currentstroke}%
\pgfsetdash{}{0pt}%
\pgfsys@defobject{currentmarker}{\pgfqpoint{0.000000in}{0.000000in}}{\pgfqpoint{0.000000in}{0.020833in}}{%
\pgfpathmoveto{\pgfqpoint{0.000000in}{0.000000in}}%
\pgfpathlineto{\pgfqpoint{0.000000in}{0.020833in}}%
\pgfusepath{stroke,fill}%
}%
\begin{pgfscope}%
\pgfsys@transformshift{3.814241in}{1.917688in}%
\pgfsys@useobject{currentmarker}{}%
\end{pgfscope}%
\end{pgfscope}%
\begin{pgfscope}%
\pgfsetbuttcap%
\pgfsetroundjoin%
\definecolor{currentfill}{rgb}{0.000000,0.000000,0.000000}%
\pgfsetfillcolor{currentfill}%
\pgfsetlinewidth{0.501875pt}%
\definecolor{currentstroke}{rgb}{0.000000,0.000000,0.000000}%
\pgfsetstrokecolor{currentstroke}%
\pgfsetdash{}{0pt}%
\pgfsys@defobject{currentmarker}{\pgfqpoint{0.000000in}{-0.020833in}}{\pgfqpoint{0.000000in}{0.000000in}}{%
\pgfpathmoveto{\pgfqpoint{0.000000in}{0.000000in}}%
\pgfpathlineto{\pgfqpoint{0.000000in}{-0.020833in}}%
\pgfusepath{stroke,fill}%
}%
\begin{pgfscope}%
\pgfsys@transformshift{3.814241in}{2.993810in}%
\pgfsys@useobject{currentmarker}{}%
\end{pgfscope}%
\end{pgfscope}%
\begin{pgfscope}%
\pgfsetbuttcap%
\pgfsetroundjoin%
\definecolor{currentfill}{rgb}{0.000000,0.000000,0.000000}%
\pgfsetfillcolor{currentfill}%
\pgfsetlinewidth{0.501875pt}%
\definecolor{currentstroke}{rgb}{0.000000,0.000000,0.000000}%
\pgfsetstrokecolor{currentstroke}%
\pgfsetdash{}{0pt}%
\pgfsys@defobject{currentmarker}{\pgfqpoint{0.000000in}{0.000000in}}{\pgfqpoint{0.000000in}{0.020833in}}{%
\pgfpathmoveto{\pgfqpoint{0.000000in}{0.000000in}}%
\pgfpathlineto{\pgfqpoint{0.000000in}{0.020833in}}%
\pgfusepath{stroke,fill}%
}%
\begin{pgfscope}%
\pgfsys@transformshift{3.910334in}{1.917688in}%
\pgfsys@useobject{currentmarker}{}%
\end{pgfscope}%
\end{pgfscope}%
\begin{pgfscope}%
\pgfsetbuttcap%
\pgfsetroundjoin%
\definecolor{currentfill}{rgb}{0.000000,0.000000,0.000000}%
\pgfsetfillcolor{currentfill}%
\pgfsetlinewidth{0.501875pt}%
\definecolor{currentstroke}{rgb}{0.000000,0.000000,0.000000}%
\pgfsetstrokecolor{currentstroke}%
\pgfsetdash{}{0pt}%
\pgfsys@defobject{currentmarker}{\pgfqpoint{0.000000in}{-0.020833in}}{\pgfqpoint{0.000000in}{0.000000in}}{%
\pgfpathmoveto{\pgfqpoint{0.000000in}{0.000000in}}%
\pgfpathlineto{\pgfqpoint{0.000000in}{-0.020833in}}%
\pgfusepath{stroke,fill}%
}%
\begin{pgfscope}%
\pgfsys@transformshift{3.910334in}{2.993810in}%
\pgfsys@useobject{currentmarker}{}%
\end{pgfscope}%
\end{pgfscope}%
\begin{pgfscope}%
\pgfsetbuttcap%
\pgfsetroundjoin%
\definecolor{currentfill}{rgb}{0.000000,0.000000,0.000000}%
\pgfsetfillcolor{currentfill}%
\pgfsetlinewidth{0.501875pt}%
\definecolor{currentstroke}{rgb}{0.000000,0.000000,0.000000}%
\pgfsetstrokecolor{currentstroke}%
\pgfsetdash{}{0pt}%
\pgfsys@defobject{currentmarker}{\pgfqpoint{0.000000in}{0.000000in}}{\pgfqpoint{0.000000in}{0.020833in}}{%
\pgfpathmoveto{\pgfqpoint{0.000000in}{0.000000in}}%
\pgfpathlineto{\pgfqpoint{0.000000in}{0.020833in}}%
\pgfusepath{stroke,fill}%
}%
\begin{pgfscope}%
\pgfsys@transformshift{4.102519in}{1.917688in}%
\pgfsys@useobject{currentmarker}{}%
\end{pgfscope}%
\end{pgfscope}%
\begin{pgfscope}%
\pgfsetbuttcap%
\pgfsetroundjoin%
\definecolor{currentfill}{rgb}{0.000000,0.000000,0.000000}%
\pgfsetfillcolor{currentfill}%
\pgfsetlinewidth{0.501875pt}%
\definecolor{currentstroke}{rgb}{0.000000,0.000000,0.000000}%
\pgfsetstrokecolor{currentstroke}%
\pgfsetdash{}{0pt}%
\pgfsys@defobject{currentmarker}{\pgfqpoint{0.000000in}{-0.020833in}}{\pgfqpoint{0.000000in}{0.000000in}}{%
\pgfpathmoveto{\pgfqpoint{0.000000in}{0.000000in}}%
\pgfpathlineto{\pgfqpoint{0.000000in}{-0.020833in}}%
\pgfusepath{stroke,fill}%
}%
\begin{pgfscope}%
\pgfsys@transformshift{4.102519in}{2.993810in}%
\pgfsys@useobject{currentmarker}{}%
\end{pgfscope}%
\end{pgfscope}%
\begin{pgfscope}%
\pgfsetbuttcap%
\pgfsetroundjoin%
\definecolor{currentfill}{rgb}{0.000000,0.000000,0.000000}%
\pgfsetfillcolor{currentfill}%
\pgfsetlinewidth{0.501875pt}%
\definecolor{currentstroke}{rgb}{0.000000,0.000000,0.000000}%
\pgfsetstrokecolor{currentstroke}%
\pgfsetdash{}{0pt}%
\pgfsys@defobject{currentmarker}{\pgfqpoint{0.000000in}{0.000000in}}{\pgfqpoint{0.000000in}{0.020833in}}{%
\pgfpathmoveto{\pgfqpoint{0.000000in}{0.000000in}}%
\pgfpathlineto{\pgfqpoint{0.000000in}{0.020833in}}%
\pgfusepath{stroke,fill}%
}%
\begin{pgfscope}%
\pgfsys@transformshift{4.198612in}{1.917688in}%
\pgfsys@useobject{currentmarker}{}%
\end{pgfscope}%
\end{pgfscope}%
\begin{pgfscope}%
\pgfsetbuttcap%
\pgfsetroundjoin%
\definecolor{currentfill}{rgb}{0.000000,0.000000,0.000000}%
\pgfsetfillcolor{currentfill}%
\pgfsetlinewidth{0.501875pt}%
\definecolor{currentstroke}{rgb}{0.000000,0.000000,0.000000}%
\pgfsetstrokecolor{currentstroke}%
\pgfsetdash{}{0pt}%
\pgfsys@defobject{currentmarker}{\pgfqpoint{0.000000in}{-0.020833in}}{\pgfqpoint{0.000000in}{0.000000in}}{%
\pgfpathmoveto{\pgfqpoint{0.000000in}{0.000000in}}%
\pgfpathlineto{\pgfqpoint{0.000000in}{-0.020833in}}%
\pgfusepath{stroke,fill}%
}%
\begin{pgfscope}%
\pgfsys@transformshift{4.198612in}{2.993810in}%
\pgfsys@useobject{currentmarker}{}%
\end{pgfscope}%
\end{pgfscope}%
\begin{pgfscope}%
\pgfsetbuttcap%
\pgfsetroundjoin%
\definecolor{currentfill}{rgb}{0.000000,0.000000,0.000000}%
\pgfsetfillcolor{currentfill}%
\pgfsetlinewidth{0.501875pt}%
\definecolor{currentstroke}{rgb}{0.000000,0.000000,0.000000}%
\pgfsetstrokecolor{currentstroke}%
\pgfsetdash{}{0pt}%
\pgfsys@defobject{currentmarker}{\pgfqpoint{0.000000in}{0.000000in}}{\pgfqpoint{0.000000in}{0.020833in}}{%
\pgfpathmoveto{\pgfqpoint{0.000000in}{0.000000in}}%
\pgfpathlineto{\pgfqpoint{0.000000in}{0.020833in}}%
\pgfusepath{stroke,fill}%
}%
\begin{pgfscope}%
\pgfsys@transformshift{4.294704in}{1.917688in}%
\pgfsys@useobject{currentmarker}{}%
\end{pgfscope}%
\end{pgfscope}%
\begin{pgfscope}%
\pgfsetbuttcap%
\pgfsetroundjoin%
\definecolor{currentfill}{rgb}{0.000000,0.000000,0.000000}%
\pgfsetfillcolor{currentfill}%
\pgfsetlinewidth{0.501875pt}%
\definecolor{currentstroke}{rgb}{0.000000,0.000000,0.000000}%
\pgfsetstrokecolor{currentstroke}%
\pgfsetdash{}{0pt}%
\pgfsys@defobject{currentmarker}{\pgfqpoint{0.000000in}{-0.020833in}}{\pgfqpoint{0.000000in}{0.000000in}}{%
\pgfpathmoveto{\pgfqpoint{0.000000in}{0.000000in}}%
\pgfpathlineto{\pgfqpoint{0.000000in}{-0.020833in}}%
\pgfusepath{stroke,fill}%
}%
\begin{pgfscope}%
\pgfsys@transformshift{4.294704in}{2.993810in}%
\pgfsys@useobject{currentmarker}{}%
\end{pgfscope}%
\end{pgfscope}%
\begin{pgfscope}%
\pgfsetbuttcap%
\pgfsetroundjoin%
\definecolor{currentfill}{rgb}{0.000000,0.000000,0.000000}%
\pgfsetfillcolor{currentfill}%
\pgfsetlinewidth{0.501875pt}%
\definecolor{currentstroke}{rgb}{0.000000,0.000000,0.000000}%
\pgfsetstrokecolor{currentstroke}%
\pgfsetdash{}{0pt}%
\pgfsys@defobject{currentmarker}{\pgfqpoint{0.000000in}{0.000000in}}{\pgfqpoint{0.000000in}{0.020833in}}{%
\pgfpathmoveto{\pgfqpoint{0.000000in}{0.000000in}}%
\pgfpathlineto{\pgfqpoint{0.000000in}{0.020833in}}%
\pgfusepath{stroke,fill}%
}%
\begin{pgfscope}%
\pgfsys@transformshift{4.390797in}{1.917688in}%
\pgfsys@useobject{currentmarker}{}%
\end{pgfscope}%
\end{pgfscope}%
\begin{pgfscope}%
\pgfsetbuttcap%
\pgfsetroundjoin%
\definecolor{currentfill}{rgb}{0.000000,0.000000,0.000000}%
\pgfsetfillcolor{currentfill}%
\pgfsetlinewidth{0.501875pt}%
\definecolor{currentstroke}{rgb}{0.000000,0.000000,0.000000}%
\pgfsetstrokecolor{currentstroke}%
\pgfsetdash{}{0pt}%
\pgfsys@defobject{currentmarker}{\pgfqpoint{0.000000in}{-0.020833in}}{\pgfqpoint{0.000000in}{0.000000in}}{%
\pgfpathmoveto{\pgfqpoint{0.000000in}{0.000000in}}%
\pgfpathlineto{\pgfqpoint{0.000000in}{-0.020833in}}%
\pgfusepath{stroke,fill}%
}%
\begin{pgfscope}%
\pgfsys@transformshift{4.390797in}{2.993810in}%
\pgfsys@useobject{currentmarker}{}%
\end{pgfscope}%
\end{pgfscope}%
\begin{pgfscope}%
\pgfsetbuttcap%
\pgfsetroundjoin%
\definecolor{currentfill}{rgb}{0.000000,0.000000,0.000000}%
\pgfsetfillcolor{currentfill}%
\pgfsetlinewidth{0.501875pt}%
\definecolor{currentstroke}{rgb}{0.000000,0.000000,0.000000}%
\pgfsetstrokecolor{currentstroke}%
\pgfsetdash{}{0pt}%
\pgfsys@defobject{currentmarker}{\pgfqpoint{0.000000in}{0.000000in}}{\pgfqpoint{0.000000in}{0.020833in}}{%
\pgfpathmoveto{\pgfqpoint{0.000000in}{0.000000in}}%
\pgfpathlineto{\pgfqpoint{0.000000in}{0.020833in}}%
\pgfusepath{stroke,fill}%
}%
\begin{pgfscope}%
\pgfsys@transformshift{4.582982in}{1.917688in}%
\pgfsys@useobject{currentmarker}{}%
\end{pgfscope}%
\end{pgfscope}%
\begin{pgfscope}%
\pgfsetbuttcap%
\pgfsetroundjoin%
\definecolor{currentfill}{rgb}{0.000000,0.000000,0.000000}%
\pgfsetfillcolor{currentfill}%
\pgfsetlinewidth{0.501875pt}%
\definecolor{currentstroke}{rgb}{0.000000,0.000000,0.000000}%
\pgfsetstrokecolor{currentstroke}%
\pgfsetdash{}{0pt}%
\pgfsys@defobject{currentmarker}{\pgfqpoint{0.000000in}{-0.020833in}}{\pgfqpoint{0.000000in}{0.000000in}}{%
\pgfpathmoveto{\pgfqpoint{0.000000in}{0.000000in}}%
\pgfpathlineto{\pgfqpoint{0.000000in}{-0.020833in}}%
\pgfusepath{stroke,fill}%
}%
\begin{pgfscope}%
\pgfsys@transformshift{4.582982in}{2.993810in}%
\pgfsys@useobject{currentmarker}{}%
\end{pgfscope}%
\end{pgfscope}%
\begin{pgfscope}%
\pgfsetbuttcap%
\pgfsetroundjoin%
\definecolor{currentfill}{rgb}{0.000000,0.000000,0.000000}%
\pgfsetfillcolor{currentfill}%
\pgfsetlinewidth{0.501875pt}%
\definecolor{currentstroke}{rgb}{0.000000,0.000000,0.000000}%
\pgfsetstrokecolor{currentstroke}%
\pgfsetdash{}{0pt}%
\pgfsys@defobject{currentmarker}{\pgfqpoint{0.000000in}{0.000000in}}{\pgfqpoint{0.041667in}{0.000000in}}{%
\pgfpathmoveto{\pgfqpoint{0.000000in}{0.000000in}}%
\pgfpathlineto{\pgfqpoint{0.041667in}{0.000000in}}%
\pgfusepath{stroke,fill}%
}%
\begin{pgfscope}%
\pgfsys@transformshift{0.444748in}{1.959217in}%
\pgfsys@useobject{currentmarker}{}%
\end{pgfscope}%
\end{pgfscope}%
\begin{pgfscope}%
\pgfsetbuttcap%
\pgfsetroundjoin%
\definecolor{currentfill}{rgb}{0.000000,0.000000,0.000000}%
\pgfsetfillcolor{currentfill}%
\pgfsetlinewidth{0.501875pt}%
\definecolor{currentstroke}{rgb}{0.000000,0.000000,0.000000}%
\pgfsetstrokecolor{currentstroke}%
\pgfsetdash{}{0pt}%
\pgfsys@defobject{currentmarker}{\pgfqpoint{-0.041667in}{0.000000in}}{\pgfqpoint{-0.000000in}{0.000000in}}{%
\pgfpathmoveto{\pgfqpoint{-0.000000in}{0.000000in}}%
\pgfpathlineto{\pgfqpoint{-0.041667in}{0.000000in}}%
\pgfusepath{stroke,fill}%
}%
\begin{pgfscope}%
\pgfsys@transformshift{4.676167in}{1.959217in}%
\pgfsys@useobject{currentmarker}{}%
\end{pgfscope}%
\end{pgfscope}%
\begin{pgfscope}%
\definecolor{textcolor}{rgb}{0.000000,0.000000,0.000000}%
\pgfsetstrokecolor{textcolor}%
\pgfsetfillcolor{textcolor}%
\pgftext[x=0.326693in, y=1.910999in, left, base]{\color{textcolor}\rmfamily\fontsize{10.000000}{12.000000}\selectfont \(\displaystyle {0}\)}%
\end{pgfscope}%
\begin{pgfscope}%
\pgfsetbuttcap%
\pgfsetroundjoin%
\definecolor{currentfill}{rgb}{0.000000,0.000000,0.000000}%
\pgfsetfillcolor{currentfill}%
\pgfsetlinewidth{0.501875pt}%
\definecolor{currentstroke}{rgb}{0.000000,0.000000,0.000000}%
\pgfsetstrokecolor{currentstroke}%
\pgfsetdash{}{0pt}%
\pgfsys@defobject{currentmarker}{\pgfqpoint{0.000000in}{0.000000in}}{\pgfqpoint{0.041667in}{0.000000in}}{%
\pgfpathmoveto{\pgfqpoint{0.000000in}{0.000000in}}%
\pgfpathlineto{\pgfqpoint{0.041667in}{0.000000in}}%
\pgfusepath{stroke,fill}%
}%
\begin{pgfscope}%
\pgfsys@transformshift{0.444748in}{2.371662in}%
\pgfsys@useobject{currentmarker}{}%
\end{pgfscope}%
\end{pgfscope}%
\begin{pgfscope}%
\pgfsetbuttcap%
\pgfsetroundjoin%
\definecolor{currentfill}{rgb}{0.000000,0.000000,0.000000}%
\pgfsetfillcolor{currentfill}%
\pgfsetlinewidth{0.501875pt}%
\definecolor{currentstroke}{rgb}{0.000000,0.000000,0.000000}%
\pgfsetstrokecolor{currentstroke}%
\pgfsetdash{}{0pt}%
\pgfsys@defobject{currentmarker}{\pgfqpoint{-0.041667in}{0.000000in}}{\pgfqpoint{-0.000000in}{0.000000in}}{%
\pgfpathmoveto{\pgfqpoint{-0.000000in}{0.000000in}}%
\pgfpathlineto{\pgfqpoint{-0.041667in}{0.000000in}}%
\pgfusepath{stroke,fill}%
}%
\begin{pgfscope}%
\pgfsys@transformshift{4.676167in}{2.371662in}%
\pgfsys@useobject{currentmarker}{}%
\end{pgfscope}%
\end{pgfscope}%
\begin{pgfscope}%
\definecolor{textcolor}{rgb}{0.000000,0.000000,0.000000}%
\pgfsetstrokecolor{textcolor}%
\pgfsetfillcolor{textcolor}%
\pgftext[x=0.257248in, y=2.323444in, left, base]{\color{textcolor}\rmfamily\fontsize{10.000000}{12.000000}\selectfont \(\displaystyle {25}\)}%
\end{pgfscope}%
\begin{pgfscope}%
\pgfsetbuttcap%
\pgfsetroundjoin%
\definecolor{currentfill}{rgb}{0.000000,0.000000,0.000000}%
\pgfsetfillcolor{currentfill}%
\pgfsetlinewidth{0.501875pt}%
\definecolor{currentstroke}{rgb}{0.000000,0.000000,0.000000}%
\pgfsetstrokecolor{currentstroke}%
\pgfsetdash{}{0pt}%
\pgfsys@defobject{currentmarker}{\pgfqpoint{0.000000in}{0.000000in}}{\pgfqpoint{0.041667in}{0.000000in}}{%
\pgfpathmoveto{\pgfqpoint{0.000000in}{0.000000in}}%
\pgfpathlineto{\pgfqpoint{0.041667in}{0.000000in}}%
\pgfusepath{stroke,fill}%
}%
\begin{pgfscope}%
\pgfsys@transformshift{0.444748in}{2.784107in}%
\pgfsys@useobject{currentmarker}{}%
\end{pgfscope}%
\end{pgfscope}%
\begin{pgfscope}%
\pgfsetbuttcap%
\pgfsetroundjoin%
\definecolor{currentfill}{rgb}{0.000000,0.000000,0.000000}%
\pgfsetfillcolor{currentfill}%
\pgfsetlinewidth{0.501875pt}%
\definecolor{currentstroke}{rgb}{0.000000,0.000000,0.000000}%
\pgfsetstrokecolor{currentstroke}%
\pgfsetdash{}{0pt}%
\pgfsys@defobject{currentmarker}{\pgfqpoint{-0.041667in}{0.000000in}}{\pgfqpoint{-0.000000in}{0.000000in}}{%
\pgfpathmoveto{\pgfqpoint{-0.000000in}{0.000000in}}%
\pgfpathlineto{\pgfqpoint{-0.041667in}{0.000000in}}%
\pgfusepath{stroke,fill}%
}%
\begin{pgfscope}%
\pgfsys@transformshift{4.676167in}{2.784107in}%
\pgfsys@useobject{currentmarker}{}%
\end{pgfscope}%
\end{pgfscope}%
\begin{pgfscope}%
\definecolor{textcolor}{rgb}{0.000000,0.000000,0.000000}%
\pgfsetstrokecolor{textcolor}%
\pgfsetfillcolor{textcolor}%
\pgftext[x=0.257248in, y=2.735889in, left, base]{\color{textcolor}\rmfamily\fontsize{10.000000}{12.000000}\selectfont \(\displaystyle {50}\)}%
\end{pgfscope}%
\begin{pgfscope}%
\pgfsetbuttcap%
\pgfsetroundjoin%
\definecolor{currentfill}{rgb}{0.000000,0.000000,0.000000}%
\pgfsetfillcolor{currentfill}%
\pgfsetlinewidth{0.501875pt}%
\definecolor{currentstroke}{rgb}{0.000000,0.000000,0.000000}%
\pgfsetstrokecolor{currentstroke}%
\pgfsetdash{}{0pt}%
\pgfsys@defobject{currentmarker}{\pgfqpoint{0.000000in}{0.000000in}}{\pgfqpoint{0.020833in}{0.000000in}}{%
\pgfpathmoveto{\pgfqpoint{0.000000in}{0.000000in}}%
\pgfpathlineto{\pgfqpoint{0.020833in}{0.000000in}}%
\pgfusepath{stroke,fill}%
}%
\begin{pgfscope}%
\pgfsys@transformshift{0.444748in}{2.041706in}%
\pgfsys@useobject{currentmarker}{}%
\end{pgfscope}%
\end{pgfscope}%
\begin{pgfscope}%
\pgfsetbuttcap%
\pgfsetroundjoin%
\definecolor{currentfill}{rgb}{0.000000,0.000000,0.000000}%
\pgfsetfillcolor{currentfill}%
\pgfsetlinewidth{0.501875pt}%
\definecolor{currentstroke}{rgb}{0.000000,0.000000,0.000000}%
\pgfsetstrokecolor{currentstroke}%
\pgfsetdash{}{0pt}%
\pgfsys@defobject{currentmarker}{\pgfqpoint{-0.020833in}{0.000000in}}{\pgfqpoint{-0.000000in}{0.000000in}}{%
\pgfpathmoveto{\pgfqpoint{-0.000000in}{0.000000in}}%
\pgfpathlineto{\pgfqpoint{-0.020833in}{0.000000in}}%
\pgfusepath{stroke,fill}%
}%
\begin{pgfscope}%
\pgfsys@transformshift{4.676167in}{2.041706in}%
\pgfsys@useobject{currentmarker}{}%
\end{pgfscope}%
\end{pgfscope}%
\begin{pgfscope}%
\pgfsetbuttcap%
\pgfsetroundjoin%
\definecolor{currentfill}{rgb}{0.000000,0.000000,0.000000}%
\pgfsetfillcolor{currentfill}%
\pgfsetlinewidth{0.501875pt}%
\definecolor{currentstroke}{rgb}{0.000000,0.000000,0.000000}%
\pgfsetstrokecolor{currentstroke}%
\pgfsetdash{}{0pt}%
\pgfsys@defobject{currentmarker}{\pgfqpoint{0.000000in}{0.000000in}}{\pgfqpoint{0.020833in}{0.000000in}}{%
\pgfpathmoveto{\pgfqpoint{0.000000in}{0.000000in}}%
\pgfpathlineto{\pgfqpoint{0.020833in}{0.000000in}}%
\pgfusepath{stroke,fill}%
}%
\begin{pgfscope}%
\pgfsys@transformshift{0.444748in}{2.124195in}%
\pgfsys@useobject{currentmarker}{}%
\end{pgfscope}%
\end{pgfscope}%
\begin{pgfscope}%
\pgfsetbuttcap%
\pgfsetroundjoin%
\definecolor{currentfill}{rgb}{0.000000,0.000000,0.000000}%
\pgfsetfillcolor{currentfill}%
\pgfsetlinewidth{0.501875pt}%
\definecolor{currentstroke}{rgb}{0.000000,0.000000,0.000000}%
\pgfsetstrokecolor{currentstroke}%
\pgfsetdash{}{0pt}%
\pgfsys@defobject{currentmarker}{\pgfqpoint{-0.020833in}{0.000000in}}{\pgfqpoint{-0.000000in}{0.000000in}}{%
\pgfpathmoveto{\pgfqpoint{-0.000000in}{0.000000in}}%
\pgfpathlineto{\pgfqpoint{-0.020833in}{0.000000in}}%
\pgfusepath{stroke,fill}%
}%
\begin{pgfscope}%
\pgfsys@transformshift{4.676167in}{2.124195in}%
\pgfsys@useobject{currentmarker}{}%
\end{pgfscope}%
\end{pgfscope}%
\begin{pgfscope}%
\pgfsetbuttcap%
\pgfsetroundjoin%
\definecolor{currentfill}{rgb}{0.000000,0.000000,0.000000}%
\pgfsetfillcolor{currentfill}%
\pgfsetlinewidth{0.501875pt}%
\definecolor{currentstroke}{rgb}{0.000000,0.000000,0.000000}%
\pgfsetstrokecolor{currentstroke}%
\pgfsetdash{}{0pt}%
\pgfsys@defobject{currentmarker}{\pgfqpoint{0.000000in}{0.000000in}}{\pgfqpoint{0.020833in}{0.000000in}}{%
\pgfpathmoveto{\pgfqpoint{0.000000in}{0.000000in}}%
\pgfpathlineto{\pgfqpoint{0.020833in}{0.000000in}}%
\pgfusepath{stroke,fill}%
}%
\begin{pgfscope}%
\pgfsys@transformshift{0.444748in}{2.206684in}%
\pgfsys@useobject{currentmarker}{}%
\end{pgfscope}%
\end{pgfscope}%
\begin{pgfscope}%
\pgfsetbuttcap%
\pgfsetroundjoin%
\definecolor{currentfill}{rgb}{0.000000,0.000000,0.000000}%
\pgfsetfillcolor{currentfill}%
\pgfsetlinewidth{0.501875pt}%
\definecolor{currentstroke}{rgb}{0.000000,0.000000,0.000000}%
\pgfsetstrokecolor{currentstroke}%
\pgfsetdash{}{0pt}%
\pgfsys@defobject{currentmarker}{\pgfqpoint{-0.020833in}{0.000000in}}{\pgfqpoint{-0.000000in}{0.000000in}}{%
\pgfpathmoveto{\pgfqpoint{-0.000000in}{0.000000in}}%
\pgfpathlineto{\pgfqpoint{-0.020833in}{0.000000in}}%
\pgfusepath{stroke,fill}%
}%
\begin{pgfscope}%
\pgfsys@transformshift{4.676167in}{2.206684in}%
\pgfsys@useobject{currentmarker}{}%
\end{pgfscope}%
\end{pgfscope}%
\begin{pgfscope}%
\pgfsetbuttcap%
\pgfsetroundjoin%
\definecolor{currentfill}{rgb}{0.000000,0.000000,0.000000}%
\pgfsetfillcolor{currentfill}%
\pgfsetlinewidth{0.501875pt}%
\definecolor{currentstroke}{rgb}{0.000000,0.000000,0.000000}%
\pgfsetstrokecolor{currentstroke}%
\pgfsetdash{}{0pt}%
\pgfsys@defobject{currentmarker}{\pgfqpoint{0.000000in}{0.000000in}}{\pgfqpoint{0.020833in}{0.000000in}}{%
\pgfpathmoveto{\pgfqpoint{0.000000in}{0.000000in}}%
\pgfpathlineto{\pgfqpoint{0.020833in}{0.000000in}}%
\pgfusepath{stroke,fill}%
}%
\begin{pgfscope}%
\pgfsys@transformshift{0.444748in}{2.289173in}%
\pgfsys@useobject{currentmarker}{}%
\end{pgfscope}%
\end{pgfscope}%
\begin{pgfscope}%
\pgfsetbuttcap%
\pgfsetroundjoin%
\definecolor{currentfill}{rgb}{0.000000,0.000000,0.000000}%
\pgfsetfillcolor{currentfill}%
\pgfsetlinewidth{0.501875pt}%
\definecolor{currentstroke}{rgb}{0.000000,0.000000,0.000000}%
\pgfsetstrokecolor{currentstroke}%
\pgfsetdash{}{0pt}%
\pgfsys@defobject{currentmarker}{\pgfqpoint{-0.020833in}{0.000000in}}{\pgfqpoint{-0.000000in}{0.000000in}}{%
\pgfpathmoveto{\pgfqpoint{-0.000000in}{0.000000in}}%
\pgfpathlineto{\pgfqpoint{-0.020833in}{0.000000in}}%
\pgfusepath{stroke,fill}%
}%
\begin{pgfscope}%
\pgfsys@transformshift{4.676167in}{2.289173in}%
\pgfsys@useobject{currentmarker}{}%
\end{pgfscope}%
\end{pgfscope}%
\begin{pgfscope}%
\pgfsetbuttcap%
\pgfsetroundjoin%
\definecolor{currentfill}{rgb}{0.000000,0.000000,0.000000}%
\pgfsetfillcolor{currentfill}%
\pgfsetlinewidth{0.501875pt}%
\definecolor{currentstroke}{rgb}{0.000000,0.000000,0.000000}%
\pgfsetstrokecolor{currentstroke}%
\pgfsetdash{}{0pt}%
\pgfsys@defobject{currentmarker}{\pgfqpoint{0.000000in}{0.000000in}}{\pgfqpoint{0.020833in}{0.000000in}}{%
\pgfpathmoveto{\pgfqpoint{0.000000in}{0.000000in}}%
\pgfpathlineto{\pgfqpoint{0.020833in}{0.000000in}}%
\pgfusepath{stroke,fill}%
}%
\begin{pgfscope}%
\pgfsys@transformshift{0.444748in}{2.454151in}%
\pgfsys@useobject{currentmarker}{}%
\end{pgfscope}%
\end{pgfscope}%
\begin{pgfscope}%
\pgfsetbuttcap%
\pgfsetroundjoin%
\definecolor{currentfill}{rgb}{0.000000,0.000000,0.000000}%
\pgfsetfillcolor{currentfill}%
\pgfsetlinewidth{0.501875pt}%
\definecolor{currentstroke}{rgb}{0.000000,0.000000,0.000000}%
\pgfsetstrokecolor{currentstroke}%
\pgfsetdash{}{0pt}%
\pgfsys@defobject{currentmarker}{\pgfqpoint{-0.020833in}{0.000000in}}{\pgfqpoint{-0.000000in}{0.000000in}}{%
\pgfpathmoveto{\pgfqpoint{-0.000000in}{0.000000in}}%
\pgfpathlineto{\pgfqpoint{-0.020833in}{0.000000in}}%
\pgfusepath{stroke,fill}%
}%
\begin{pgfscope}%
\pgfsys@transformshift{4.676167in}{2.454151in}%
\pgfsys@useobject{currentmarker}{}%
\end{pgfscope}%
\end{pgfscope}%
\begin{pgfscope}%
\pgfsetbuttcap%
\pgfsetroundjoin%
\definecolor{currentfill}{rgb}{0.000000,0.000000,0.000000}%
\pgfsetfillcolor{currentfill}%
\pgfsetlinewidth{0.501875pt}%
\definecolor{currentstroke}{rgb}{0.000000,0.000000,0.000000}%
\pgfsetstrokecolor{currentstroke}%
\pgfsetdash{}{0pt}%
\pgfsys@defobject{currentmarker}{\pgfqpoint{0.000000in}{0.000000in}}{\pgfqpoint{0.020833in}{0.000000in}}{%
\pgfpathmoveto{\pgfqpoint{0.000000in}{0.000000in}}%
\pgfpathlineto{\pgfqpoint{0.020833in}{0.000000in}}%
\pgfusepath{stroke,fill}%
}%
\begin{pgfscope}%
\pgfsys@transformshift{0.444748in}{2.536640in}%
\pgfsys@useobject{currentmarker}{}%
\end{pgfscope}%
\end{pgfscope}%
\begin{pgfscope}%
\pgfsetbuttcap%
\pgfsetroundjoin%
\definecolor{currentfill}{rgb}{0.000000,0.000000,0.000000}%
\pgfsetfillcolor{currentfill}%
\pgfsetlinewidth{0.501875pt}%
\definecolor{currentstroke}{rgb}{0.000000,0.000000,0.000000}%
\pgfsetstrokecolor{currentstroke}%
\pgfsetdash{}{0pt}%
\pgfsys@defobject{currentmarker}{\pgfqpoint{-0.020833in}{0.000000in}}{\pgfqpoint{-0.000000in}{0.000000in}}{%
\pgfpathmoveto{\pgfqpoint{-0.000000in}{0.000000in}}%
\pgfpathlineto{\pgfqpoint{-0.020833in}{0.000000in}}%
\pgfusepath{stroke,fill}%
}%
\begin{pgfscope}%
\pgfsys@transformshift{4.676167in}{2.536640in}%
\pgfsys@useobject{currentmarker}{}%
\end{pgfscope}%
\end{pgfscope}%
\begin{pgfscope}%
\pgfsetbuttcap%
\pgfsetroundjoin%
\definecolor{currentfill}{rgb}{0.000000,0.000000,0.000000}%
\pgfsetfillcolor{currentfill}%
\pgfsetlinewidth{0.501875pt}%
\definecolor{currentstroke}{rgb}{0.000000,0.000000,0.000000}%
\pgfsetstrokecolor{currentstroke}%
\pgfsetdash{}{0pt}%
\pgfsys@defobject{currentmarker}{\pgfqpoint{0.000000in}{0.000000in}}{\pgfqpoint{0.020833in}{0.000000in}}{%
\pgfpathmoveto{\pgfqpoint{0.000000in}{0.000000in}}%
\pgfpathlineto{\pgfqpoint{0.020833in}{0.000000in}}%
\pgfusepath{stroke,fill}%
}%
\begin{pgfscope}%
\pgfsys@transformshift{0.444748in}{2.619129in}%
\pgfsys@useobject{currentmarker}{}%
\end{pgfscope}%
\end{pgfscope}%
\begin{pgfscope}%
\pgfsetbuttcap%
\pgfsetroundjoin%
\definecolor{currentfill}{rgb}{0.000000,0.000000,0.000000}%
\pgfsetfillcolor{currentfill}%
\pgfsetlinewidth{0.501875pt}%
\definecolor{currentstroke}{rgb}{0.000000,0.000000,0.000000}%
\pgfsetstrokecolor{currentstroke}%
\pgfsetdash{}{0pt}%
\pgfsys@defobject{currentmarker}{\pgfqpoint{-0.020833in}{0.000000in}}{\pgfqpoint{-0.000000in}{0.000000in}}{%
\pgfpathmoveto{\pgfqpoint{-0.000000in}{0.000000in}}%
\pgfpathlineto{\pgfqpoint{-0.020833in}{0.000000in}}%
\pgfusepath{stroke,fill}%
}%
\begin{pgfscope}%
\pgfsys@transformshift{4.676167in}{2.619129in}%
\pgfsys@useobject{currentmarker}{}%
\end{pgfscope}%
\end{pgfscope}%
\begin{pgfscope}%
\pgfsetbuttcap%
\pgfsetroundjoin%
\definecolor{currentfill}{rgb}{0.000000,0.000000,0.000000}%
\pgfsetfillcolor{currentfill}%
\pgfsetlinewidth{0.501875pt}%
\definecolor{currentstroke}{rgb}{0.000000,0.000000,0.000000}%
\pgfsetstrokecolor{currentstroke}%
\pgfsetdash{}{0pt}%
\pgfsys@defobject{currentmarker}{\pgfqpoint{0.000000in}{0.000000in}}{\pgfqpoint{0.020833in}{0.000000in}}{%
\pgfpathmoveto{\pgfqpoint{0.000000in}{0.000000in}}%
\pgfpathlineto{\pgfqpoint{0.020833in}{0.000000in}}%
\pgfusepath{stroke,fill}%
}%
\begin{pgfscope}%
\pgfsys@transformshift{0.444748in}{2.701618in}%
\pgfsys@useobject{currentmarker}{}%
\end{pgfscope}%
\end{pgfscope}%
\begin{pgfscope}%
\pgfsetbuttcap%
\pgfsetroundjoin%
\definecolor{currentfill}{rgb}{0.000000,0.000000,0.000000}%
\pgfsetfillcolor{currentfill}%
\pgfsetlinewidth{0.501875pt}%
\definecolor{currentstroke}{rgb}{0.000000,0.000000,0.000000}%
\pgfsetstrokecolor{currentstroke}%
\pgfsetdash{}{0pt}%
\pgfsys@defobject{currentmarker}{\pgfqpoint{-0.020833in}{0.000000in}}{\pgfqpoint{-0.000000in}{0.000000in}}{%
\pgfpathmoveto{\pgfqpoint{-0.000000in}{0.000000in}}%
\pgfpathlineto{\pgfqpoint{-0.020833in}{0.000000in}}%
\pgfusepath{stroke,fill}%
}%
\begin{pgfscope}%
\pgfsys@transformshift{4.676167in}{2.701618in}%
\pgfsys@useobject{currentmarker}{}%
\end{pgfscope}%
\end{pgfscope}%
\begin{pgfscope}%
\pgfsetbuttcap%
\pgfsetroundjoin%
\definecolor{currentfill}{rgb}{0.000000,0.000000,0.000000}%
\pgfsetfillcolor{currentfill}%
\pgfsetlinewidth{0.501875pt}%
\definecolor{currentstroke}{rgb}{0.000000,0.000000,0.000000}%
\pgfsetstrokecolor{currentstroke}%
\pgfsetdash{}{0pt}%
\pgfsys@defobject{currentmarker}{\pgfqpoint{0.000000in}{0.000000in}}{\pgfqpoint{0.020833in}{0.000000in}}{%
\pgfpathmoveto{\pgfqpoint{0.000000in}{0.000000in}}%
\pgfpathlineto{\pgfqpoint{0.020833in}{0.000000in}}%
\pgfusepath{stroke,fill}%
}%
\begin{pgfscope}%
\pgfsys@transformshift{0.444748in}{2.866596in}%
\pgfsys@useobject{currentmarker}{}%
\end{pgfscope}%
\end{pgfscope}%
\begin{pgfscope}%
\pgfsetbuttcap%
\pgfsetroundjoin%
\definecolor{currentfill}{rgb}{0.000000,0.000000,0.000000}%
\pgfsetfillcolor{currentfill}%
\pgfsetlinewidth{0.501875pt}%
\definecolor{currentstroke}{rgb}{0.000000,0.000000,0.000000}%
\pgfsetstrokecolor{currentstroke}%
\pgfsetdash{}{0pt}%
\pgfsys@defobject{currentmarker}{\pgfqpoint{-0.020833in}{0.000000in}}{\pgfqpoint{-0.000000in}{0.000000in}}{%
\pgfpathmoveto{\pgfqpoint{-0.000000in}{0.000000in}}%
\pgfpathlineto{\pgfqpoint{-0.020833in}{0.000000in}}%
\pgfusepath{stroke,fill}%
}%
\begin{pgfscope}%
\pgfsys@transformshift{4.676167in}{2.866596in}%
\pgfsys@useobject{currentmarker}{}%
\end{pgfscope}%
\end{pgfscope}%
\begin{pgfscope}%
\pgfsetbuttcap%
\pgfsetroundjoin%
\definecolor{currentfill}{rgb}{0.000000,0.000000,0.000000}%
\pgfsetfillcolor{currentfill}%
\pgfsetlinewidth{0.501875pt}%
\definecolor{currentstroke}{rgb}{0.000000,0.000000,0.000000}%
\pgfsetstrokecolor{currentstroke}%
\pgfsetdash{}{0pt}%
\pgfsys@defobject{currentmarker}{\pgfqpoint{0.000000in}{0.000000in}}{\pgfqpoint{0.020833in}{0.000000in}}{%
\pgfpathmoveto{\pgfqpoint{0.000000in}{0.000000in}}%
\pgfpathlineto{\pgfqpoint{0.020833in}{0.000000in}}%
\pgfusepath{stroke,fill}%
}%
\begin{pgfscope}%
\pgfsys@transformshift{0.444748in}{2.949085in}%
\pgfsys@useobject{currentmarker}{}%
\end{pgfscope}%
\end{pgfscope}%
\begin{pgfscope}%
\pgfsetbuttcap%
\pgfsetroundjoin%
\definecolor{currentfill}{rgb}{0.000000,0.000000,0.000000}%
\pgfsetfillcolor{currentfill}%
\pgfsetlinewidth{0.501875pt}%
\definecolor{currentstroke}{rgb}{0.000000,0.000000,0.000000}%
\pgfsetstrokecolor{currentstroke}%
\pgfsetdash{}{0pt}%
\pgfsys@defobject{currentmarker}{\pgfqpoint{-0.020833in}{0.000000in}}{\pgfqpoint{-0.000000in}{0.000000in}}{%
\pgfpathmoveto{\pgfqpoint{-0.000000in}{0.000000in}}%
\pgfpathlineto{\pgfqpoint{-0.020833in}{0.000000in}}%
\pgfusepath{stroke,fill}%
}%
\begin{pgfscope}%
\pgfsys@transformshift{4.676167in}{2.949085in}%
\pgfsys@useobject{currentmarker}{}%
\end{pgfscope}%
\end{pgfscope}%
\begin{pgfscope}%
\definecolor{textcolor}{rgb}{0.000000,0.000000,0.000000}%
\pgfsetstrokecolor{textcolor}%
\pgfsetfillcolor{textcolor}%
\pgftext[x=0.201692in,y=2.455749in,,bottom,rotate=90.000000]{\color{textcolor}\rmfamily\fontsize{12.000000}{14.400000}\selectfont \(\displaystyle V_s\) (\unit{\micro\volt})}%
\end{pgfscope}%
\begin{pgfscope}%
\pgfpathrectangle{\pgfqpoint{0.444748in}{1.917688in}}{\pgfqpoint{4.231419in}{1.076123in}}%
\pgfusepath{clip}%
\pgfsetbuttcap%
\pgfsetroundjoin%
\pgfsetlinewidth{1.003750pt}%
\definecolor{currentstroke}{rgb}{0.047059,0.364706,0.647059}%
\pgfsetstrokecolor{currentstroke}%
\pgfsetdash{{3.700000pt}{1.600000pt}}{0.000000pt}%
\pgfpathmoveto{\pgfqpoint{0.645536in}{2.407326in}}%
\pgfpathlineto{\pgfqpoint{0.655866in}{2.230851in}}%
\pgfpathlineto{\pgfqpoint{0.674179in}{2.068339in}}%
\pgfpathlineto{\pgfqpoint{0.694368in}{1.990209in}}%
\pgfpathlineto{\pgfqpoint{0.714091in}{2.011323in}}%
\pgfpathlineto{\pgfqpoint{0.734045in}{2.103865in}}%
\pgfpathlineto{\pgfqpoint{0.754705in}{2.325021in}}%
\pgfpathlineto{\pgfqpoint{0.771141in}{2.687404in}}%
\pgfpathlineto{\pgfqpoint{0.788983in}{2.021212in}}%
\pgfpathlineto{\pgfqpoint{0.812930in}{1.986460in}}%
\pgfpathlineto{\pgfqpoint{0.828660in}{2.037250in}}%
\pgfpathlineto{\pgfqpoint{0.848615in}{2.175184in}}%
\pgfpathlineto{\pgfqpoint{0.886179in}{2.831267in}}%
\pgfpathlineto{\pgfqpoint{0.905665in}{2.855327in}}%
\pgfpathlineto{\pgfqpoint{0.945341in}{2.183368in}}%
\pgfpathlineto{\pgfqpoint{0.962010in}{2.039864in}}%
\pgfpathlineto{\pgfqpoint{0.980558in}{1.982610in}}%
\pgfpathlineto{\pgfqpoint{1.002861in}{2.004278in}}%
\pgfpathlineto{\pgfqpoint{1.024929in}{2.117066in}}%
\pgfpathlineto{\pgfqpoint{1.041363in}{2.294240in}}%
\pgfpathlineto{\pgfqpoint{1.057797in}{2.649578in}}%
\pgfpathlineto{\pgfqpoint{1.078926in}{2.872062in}}%
\pgfpathlineto{\pgfqpoint{1.096065in}{2.729227in}}%
\pgfpathlineto{\pgfqpoint{1.118133in}{2.307185in}}%
\pgfpathlineto{\pgfqpoint{1.138324in}{2.074045in}}%
\pgfpathlineto{\pgfqpoint{1.156636in}{1.999960in}}%
\pgfpathlineto{\pgfqpoint{1.175418in}{1.975937in}}%
\pgfpathlineto{\pgfqpoint{1.193026in}{2.015803in}}%
\pgfpathlineto{\pgfqpoint{1.214860in}{2.124182in}}%
\pgfpathlineto{\pgfqpoint{1.253597in}{2.614283in}}%
\pgfpathlineto{\pgfqpoint{1.269091in}{2.825224in}}%
\pgfpathlineto{\pgfqpoint{1.289517in}{2.788390in}}%
\pgfpathlineto{\pgfqpoint{1.310647in}{2.374836in}}%
\pgfpathlineto{\pgfqpoint{1.328959in}{2.160642in}}%
\pgfpathlineto{\pgfqpoint{1.345627in}{2.035637in}}%
\pgfpathlineto{\pgfqpoint{1.366758in}{1.982216in}}%
\pgfpathlineto{\pgfqpoint{1.386009in}{1.979267in}}%
\pgfpathlineto{\pgfqpoint{1.406904in}{2.038032in}}%
\pgfpathlineto{\pgfqpoint{1.423572in}{2.130434in}}%
\pgfpathlineto{\pgfqpoint{1.444703in}{2.373141in}}%
\pgfpathlineto{\pgfqpoint{1.464188in}{2.657275in}}%
\pgfpathlineto{\pgfqpoint{1.484379in}{2.842228in}}%
\pgfpathlineto{\pgfqpoint{1.506211in}{2.671599in}}%
\pgfpathlineto{\pgfqpoint{1.519830in}{2.454468in}}%
\pgfpathlineto{\pgfqpoint{1.540490in}{2.142989in}}%
\pgfpathlineto{\pgfqpoint{1.559270in}{2.034850in}}%
\pgfpathlineto{\pgfqpoint{1.579930in}{1.979545in}}%
\pgfpathlineto{\pgfqpoint{1.598712in}{1.970959in}}%
\pgfpathlineto{\pgfqpoint{1.615617in}{1.986728in}}%
\pgfpathlineto{\pgfqpoint{1.637215in}{2.049452in}}%
\pgfpathlineto{\pgfqpoint{1.655057in}{2.150032in}}%
\pgfpathlineto{\pgfqpoint{1.675483in}{2.366412in}}%
\pgfpathlineto{\pgfqpoint{1.691916in}{2.660631in}}%
\pgfpathlineto{\pgfqpoint{1.715865in}{2.829842in}}%
\pgfpathlineto{\pgfqpoint{1.731829in}{2.741833in}}%
\pgfpathlineto{\pgfqpoint{1.754367in}{2.390429in}}%
\pgfpathlineto{\pgfqpoint{1.771270in}{2.172407in}}%
\pgfpathlineto{\pgfqpoint{1.789349in}{2.062302in}}%
\pgfpathlineto{\pgfqpoint{1.807660in}{2.013221in}}%
\pgfpathlineto{\pgfqpoint{1.828320in}{1.975939in}}%
\pgfpathlineto{\pgfqpoint{1.846397in}{1.977224in}}%
\pgfpathlineto{\pgfqpoint{1.867291in}{2.028018in}}%
\pgfpathlineto{\pgfqpoint{1.884665in}{2.117561in}}%
\pgfpathlineto{\pgfqpoint{1.924342in}{2.393912in}}%
\pgfpathlineto{\pgfqpoint{1.945707in}{2.717630in}}%
\pgfpathlineto{\pgfqpoint{1.963315in}{2.820764in}}%
\pgfpathlineto{\pgfqpoint{1.981861in}{2.699328in}}%
\pgfpathlineto{\pgfqpoint{2.002052in}{2.422042in}}%
\pgfpathlineto{\pgfqpoint{2.019894in}{2.744822in}}%
\pgfpathlineto{\pgfqpoint{2.038206in}{2.826374in}}%
\pgfpathlineto{\pgfqpoint{2.061683in}{2.669395in}}%
\pgfpathlineto{\pgfqpoint{2.077179in}{2.382053in}}%
\pgfpathlineto{\pgfqpoint{2.099013in}{2.115465in}}%
\pgfpathlineto{\pgfqpoint{2.116150in}{2.017532in}}%
\pgfpathlineto{\pgfqpoint{2.133993in}{1.974975in}}%
\pgfpathlineto{\pgfqpoint{2.152306in}{1.977139in}}%
\pgfpathlineto{\pgfqpoint{2.173201in}{2.020890in}}%
\pgfpathlineto{\pgfqpoint{2.194329in}{2.137068in}}%
\pgfpathlineto{\pgfqpoint{2.213112in}{2.359919in}}%
\pgfpathlineto{\pgfqpoint{2.232363in}{2.669665in}}%
\pgfpathlineto{\pgfqpoint{2.250205in}{2.820256in}}%
\pgfpathlineto{\pgfqpoint{2.269222in}{2.712987in}}%
\pgfpathlineto{\pgfqpoint{2.290587in}{2.347478in}}%
\pgfpathlineto{\pgfqpoint{2.306552in}{2.124513in}}%
\pgfpathlineto{\pgfqpoint{2.327212in}{2.023630in}}%
\pgfpathlineto{\pgfqpoint{2.348107in}{1.974057in}}%
\pgfpathlineto{\pgfqpoint{2.365480in}{1.970945in}}%
\pgfpathlineto{\pgfqpoint{2.383792in}{2.003997in}}%
\pgfpathlineto{\pgfqpoint{2.405391in}{2.086875in}}%
\pgfpathlineto{\pgfqpoint{2.422763in}{2.244333in}}%
\pgfpathlineto{\pgfqpoint{2.461971in}{2.783672in}}%
\pgfpathlineto{\pgfqpoint{2.480048in}{2.819273in}}%
\pgfpathlineto{\pgfqpoint{2.500944in}{2.658588in}}%
\pgfpathlineto{\pgfqpoint{2.520195in}{2.345812in}}%
\pgfpathlineto{\pgfqpoint{2.540384in}{2.107981in}}%
\pgfpathlineto{\pgfqpoint{2.558698in}{2.023957in}}%
\pgfpathlineto{\pgfqpoint{2.576306in}{1.981556in}}%
\pgfpathlineto{\pgfqpoint{2.597903in}{1.970416in}}%
\pgfpathlineto{\pgfqpoint{2.615277in}{1.996696in}}%
\pgfpathlineto{\pgfqpoint{2.639929in}{2.048007in}}%
\pgfpathlineto{\pgfqpoint{2.654954in}{2.137074in}}%
\pgfpathlineto{\pgfqpoint{2.672093in}{2.292889in}}%
\pgfpathlineto{\pgfqpoint{2.693456in}{2.449427in}}%
\pgfpathlineto{\pgfqpoint{2.712473in}{2.732453in}}%
\pgfpathlineto{\pgfqpoint{2.733367in}{2.809344in}}%
\pgfpathlineto{\pgfqpoint{2.769523in}{2.451917in}}%
\pgfpathlineto{\pgfqpoint{2.792529in}{2.108469in}}%
\pgfpathlineto{\pgfqpoint{2.806148in}{2.048616in}}%
\pgfpathlineto{\pgfqpoint{2.827511in}{1.986469in}}%
\pgfpathlineto{\pgfqpoint{2.847233in}{1.972430in}}%
\pgfpathlineto{\pgfqpoint{2.865779in}{1.978697in}}%
\pgfpathlineto{\pgfqpoint{2.886204in}{2.036082in}}%
\pgfpathlineto{\pgfqpoint{2.906161in}{2.137003in}}%
\pgfpathlineto{\pgfqpoint{2.923533in}{2.322179in}}%
\pgfpathlineto{\pgfqpoint{2.942784in}{2.565938in}}%
\pgfpathlineto{\pgfqpoint{2.961801in}{2.717599in}}%
\pgfpathlineto{\pgfqpoint{2.982695in}{2.827738in}}%
\pgfpathlineto{\pgfqpoint{3.000069in}{2.662216in}}%
\pgfpathlineto{\pgfqpoint{3.018616in}{2.362562in}}%
\pgfpathlineto{\pgfqpoint{3.038807in}{2.137009in}}%
\pgfpathlineto{\pgfqpoint{3.056884in}{2.035355in}}%
\pgfpathlineto{\pgfqpoint{3.075196in}{1.992609in}}%
\pgfpathlineto{\pgfqpoint{3.095856in}{1.972950in}}%
\pgfpathlineto{\pgfqpoint{3.117221in}{2.002906in}}%
\pgfpathlineto{\pgfqpoint{3.135063in}{2.055500in}}%
\pgfpathlineto{\pgfqpoint{3.135298in}{2.120828in}}%
\pgfpathlineto{\pgfqpoint{3.153846in}{2.150494in}}%
\pgfpathlineto{\pgfqpoint{3.174975in}{2.350864in}}%
\pgfpathlineto{\pgfqpoint{3.191877in}{1.972984in}}%
\pgfpathlineto{\pgfqpoint{3.214182in}{2.008422in}}%
\pgfpathlineto{\pgfqpoint{3.231319in}{2.086892in}}%
\pgfpathlineto{\pgfqpoint{3.250805in}{2.250409in}}%
\pgfpathlineto{\pgfqpoint{3.270762in}{2.548300in}}%
\pgfpathlineto{\pgfqpoint{3.290716in}{2.796930in}}%
\pgfpathlineto{\pgfqpoint{3.309733in}{2.833148in}}%
\pgfpathlineto{\pgfqpoint{3.328281in}{2.610775in}}%
\pgfpathlineto{\pgfqpoint{3.348941in}{2.313848in}}%
\pgfpathlineto{\pgfqpoint{3.366549in}{2.122162in}}%
\pgfpathlineto{\pgfqpoint{3.384391in}{2.037815in}}%
\pgfpathlineto{\pgfqpoint{3.406694in}{2.005986in}}%
\pgfpathlineto{\pgfqpoint{3.424537in}{1.975585in}}%
\pgfpathlineto{\pgfqpoint{3.442616in}{1.985713in}}%
\pgfpathlineto{\pgfqpoint{3.462101in}{2.039269in}}%
\pgfpathlineto{\pgfqpoint{3.481822in}{2.129483in}}%
\pgfpathlineto{\pgfqpoint{3.517507in}{2.520969in}}%
\pgfpathlineto{\pgfqpoint{3.539812in}{2.602937in}}%
\pgfpathlineto{\pgfqpoint{3.557183in}{2.622314in}}%
\pgfpathlineto{\pgfqpoint{3.580895in}{2.859961in}}%
\pgfpathlineto{\pgfqpoint{3.595451in}{2.820823in}}%
\pgfpathlineto{\pgfqpoint{3.634893in}{2.277483in}}%
\pgfpathlineto{\pgfqpoint{3.652267in}{2.107103in}}%
\pgfpathlineto{\pgfqpoint{3.672222in}{2.024732in}}%
\pgfpathlineto{\pgfqpoint{3.691004in}{1.992562in}}%
\pgfpathlineto{\pgfqpoint{3.710960in}{1.979934in}}%
\pgfpathlineto{\pgfqpoint{3.729037in}{2.017789in}}%
\pgfpathlineto{\pgfqpoint{3.751341in}{2.101012in}}%
\pgfpathlineto{\pgfqpoint{3.769417in}{2.220126in}}%
\pgfpathlineto{\pgfqpoint{3.788905in}{2.460141in}}%
\pgfpathlineto{\pgfqpoint{3.807922in}{2.724642in}}%
\pgfpathlineto{\pgfqpoint{3.826702in}{2.878485in}}%
\pgfpathlineto{\pgfqpoint{3.845953in}{2.850300in}}%
\pgfpathlineto{\pgfqpoint{3.864501in}{2.671742in}}%
\pgfpathlineto{\pgfqpoint{3.883987in}{2.389851in}}%
\pgfpathlineto{\pgfqpoint{3.903004in}{2.216452in}}%
\pgfpathlineto{\pgfqpoint{3.925778in}{2.071254in}}%
\pgfpathlineto{\pgfqpoint{3.944323in}{2.003166in}}%
\pgfpathlineto{\pgfqpoint{3.964514in}{1.999013in}}%
\pgfpathlineto{\pgfqpoint{3.982122in}{1.981657in}}%
\pgfpathlineto{\pgfqpoint{4.001374in}{2.014491in}}%
\pgfpathlineto{\pgfqpoint{4.019685in}{2.081641in}}%
\pgfpathlineto{\pgfqpoint{4.040345in}{2.214750in}}%
\pgfpathlineto{\pgfqpoint{4.055607in}{2.402417in}}%
\pgfpathlineto{\pgfqpoint{4.077204in}{2.629711in}}%
\pgfpathlineto{\pgfqpoint{4.098099in}{2.823111in}}%
\pgfpathlineto{\pgfqpoint{4.117115in}{2.913261in}}%
\pgfpathlineto{\pgfqpoint{4.134960in}{2.815154in}}%
\pgfpathlineto{\pgfqpoint{4.173462in}{2.274233in}}%
\pgfpathlineto{\pgfqpoint{4.192948in}{2.110773in}}%
\pgfpathlineto{\pgfqpoint{4.211025in}{2.797752in}}%
\pgfpathlineto{\pgfqpoint{4.229573in}{2.550932in}}%
\pgfpathlineto{\pgfqpoint{4.248119in}{2.261677in}}%
\pgfpathlineto{\pgfqpoint{4.272301in}{2.080004in}}%
\pgfpathlineto{\pgfqpoint{4.288735in}{2.010401in}}%
\pgfpathlineto{\pgfqpoint{4.308690in}{1.986796in}}%
\pgfpathlineto{\pgfqpoint{4.329350in}{2.032404in}}%
\pgfpathlineto{\pgfqpoint{4.344611in}{2.092357in}}%
\pgfpathlineto{\pgfqpoint{4.363157in}{2.225806in}}%
\pgfpathlineto{\pgfqpoint{4.384991in}{2.504385in}}%
\pgfpathlineto{\pgfqpoint{4.404008in}{2.482324in}}%
\pgfpathlineto{\pgfqpoint{4.422319in}{2.778515in}}%
\pgfpathlineto{\pgfqpoint{4.442510in}{2.939749in}}%
\pgfpathlineto{\pgfqpoint{4.460824in}{2.908728in}}%
\pgfpathlineto{\pgfqpoint{4.480075in}{2.684013in}}%
\pgfpathlineto{\pgfqpoint{4.478901in}{2.738893in}}%
\pgfpathlineto{\pgfqpoint{4.474674in}{2.836014in}}%
\pgfpathlineto{\pgfqpoint{4.454249in}{2.944896in}}%
\pgfpathlineto{\pgfqpoint{4.436641in}{2.734186in}}%
\pgfpathlineto{\pgfqpoint{4.418095in}{2.334889in}}%
\pgfpathlineto{\pgfqpoint{4.396261in}{2.095842in}}%
\pgfpathlineto{\pgfqpoint{4.377010in}{2.002212in}}%
\pgfpathlineto{\pgfqpoint{4.358931in}{1.995567in}}%
\pgfpathlineto{\pgfqpoint{4.340619in}{2.067757in}}%
\pgfpathlineto{\pgfqpoint{4.319020in}{2.292313in}}%
\pgfpathlineto{\pgfqpoint{4.299768in}{2.679042in}}%
\pgfpathlineto{\pgfqpoint{4.281223in}{2.910914in}}%
\pgfpathlineto{\pgfqpoint{4.262675in}{2.823719in}}%
\pgfpathlineto{\pgfqpoint{4.243892in}{2.420513in}}%
\pgfpathlineto{\pgfqpoint{4.225347in}{2.158519in}}%
\pgfpathlineto{\pgfqpoint{4.207504in}{2.030385in}}%
\pgfpathlineto{\pgfqpoint{4.184496in}{1.981038in}}%
\pgfpathlineto{\pgfqpoint{4.166888in}{2.025662in}}%
\pgfpathlineto{\pgfqpoint{4.147871in}{2.145814in}}%
\pgfpathlineto{\pgfqpoint{4.128151in}{2.420822in}}%
\pgfpathlineto{\pgfqpoint{4.110308in}{2.765684in}}%
\pgfpathlineto{\pgfqpoint{4.089177in}{2.889475in}}%
\pgfpathlineto{\pgfqpoint{4.072509in}{2.705281in}}%
\pgfpathlineto{\pgfqpoint{4.051146in}{2.274643in}}%
\pgfpathlineto{\pgfqpoint{4.033538in}{2.088188in}}%
\pgfpathlineto{\pgfqpoint{4.010764in}{1.995920in}}%
\pgfpathlineto{\pgfqpoint{3.993627in}{1.983517in}}%
\pgfpathlineto{\pgfqpoint{3.973670in}{2.044632in}}%
\pgfpathlineto{\pgfqpoint{3.955122in}{2.183392in}}%
\pgfpathlineto{\pgfqpoint{3.936342in}{2.500118in}}%
\pgfpathlineto{\pgfqpoint{3.918263in}{2.826150in}}%
\pgfpathlineto{\pgfqpoint{3.897134in}{2.827863in}}%
\pgfpathlineto{\pgfqpoint{3.877178in}{2.486760in}}%
\pgfpathlineto{\pgfqpoint{3.858163in}{2.219760in}}%
\pgfpathlineto{\pgfqpoint{3.839615in}{2.063108in}}%
\pgfpathlineto{\pgfqpoint{3.821304in}{1.991762in}}%
\pgfpathlineto{\pgfqpoint{3.799469in}{1.983467in}}%
\pgfpathlineto{\pgfqpoint{3.780922in}{2.028070in}}%
\pgfpathlineto{\pgfqpoint{3.765662in}{2.122629in}}%
\pgfpathlineto{\pgfqpoint{3.741479in}{2.477731in}}%
\pgfpathlineto{\pgfqpoint{3.725046in}{2.677447in}}%
\pgfpathlineto{\pgfqpoint{3.704386in}{2.853300in}}%
\pgfpathlineto{\pgfqpoint{3.685135in}{2.740003in}}%
\pgfpathlineto{\pgfqpoint{3.666352in}{2.389991in}}%
\pgfpathlineto{\pgfqpoint{3.647806in}{2.145672in}}%
\pgfpathlineto{\pgfqpoint{3.628555in}{2.029728in}}%
\pgfpathlineto{\pgfqpoint{3.609773in}{1.978099in}}%
\pgfpathlineto{\pgfqpoint{3.590287in}{1.976983in}}%
\pgfpathlineto{\pgfqpoint{3.567750in}{2.024086in}}%
\pgfpathlineto{\pgfqpoint{3.550845in}{2.109948in}}%
\pgfpathlineto{\pgfqpoint{3.534411in}{2.303512in}}%
\pgfpathlineto{\pgfqpoint{3.513282in}{2.656322in}}%
\pgfpathlineto{\pgfqpoint{3.493795in}{2.832064in}}%
\pgfpathlineto{\pgfqpoint{3.475014in}{2.803788in}}%
\pgfpathlineto{\pgfqpoint{3.450832in}{2.638833in}}%
\pgfpathlineto{\pgfqpoint{3.436276in}{2.334019in}}%
\pgfpathlineto{\pgfqpoint{3.417024in}{2.128126in}}%
\pgfpathlineto{\pgfqpoint{3.401060in}{2.147596in}}%
\pgfpathlineto{\pgfqpoint{3.378756in}{2.831732in}}%
\pgfpathlineto{\pgfqpoint{3.357159in}{2.694689in}}%
\pgfpathlineto{\pgfqpoint{3.342837in}{2.403358in}}%
\pgfpathlineto{\pgfqpoint{3.320534in}{2.133940in}}%
\pgfpathlineto{\pgfqpoint{3.302455in}{2.020193in}}%
\pgfpathlineto{\pgfqpoint{3.282500in}{1.980666in}}%
\pgfpathlineto{\pgfqpoint{3.265832in}{1.975953in}}%
\pgfpathlineto{\pgfqpoint{3.243293in}{2.033489in}}%
\pgfpathlineto{\pgfqpoint{3.225216in}{2.153017in}}%
\pgfpathlineto{\pgfqpoint{3.209485in}{2.375217in}}%
\pgfpathlineto{\pgfqpoint{3.186244in}{2.756505in}}%
\pgfpathlineto{\pgfqpoint{3.166522in}{2.803016in}}%
\pgfpathlineto{\pgfqpoint{3.149149in}{2.610191in}}%
\pgfpathlineto{\pgfqpoint{3.127785in}{2.266039in}}%
\pgfpathlineto{\pgfqpoint{3.109238in}{2.098033in}}%
\pgfpathlineto{\pgfqpoint{3.089986in}{2.007027in}}%
\pgfpathlineto{\pgfqpoint{3.068623in}{1.970641in}}%
\pgfpathlineto{\pgfqpoint{3.053127in}{1.987857in}}%
\pgfpathlineto{\pgfqpoint{3.031529in}{2.049903in}}%
\pgfpathlineto{\pgfqpoint{3.013685in}{2.191340in}}%
\pgfpathlineto{\pgfqpoint{2.994670in}{2.471903in}}%
\pgfpathlineto{\pgfqpoint{2.972836in}{2.772673in}}%
\pgfpathlineto{\pgfqpoint{2.955462in}{2.815660in}}%
\pgfpathlineto{\pgfqpoint{2.936211in}{2.581801in}}%
\pgfpathlineto{\pgfqpoint{2.918369in}{2.392681in}}%
\pgfpathlineto{\pgfqpoint{2.899586in}{2.141858in}}%
\pgfpathlineto{\pgfqpoint{2.881275in}{2.024930in}}%
\pgfpathlineto{\pgfqpoint{2.852397in}{1.973058in}}%
\pgfpathlineto{\pgfqpoint{2.840658in}{1.973113in}}%
\pgfpathlineto{\pgfqpoint{2.821642in}{2.013896in}}%
\pgfpathlineto{\pgfqpoint{2.803330in}{2.086888in}}%
\pgfpathlineto{\pgfqpoint{2.782905in}{2.275112in}}%
\pgfpathlineto{\pgfqpoint{2.763888in}{2.507623in}}%
\pgfpathlineto{\pgfqpoint{2.743932in}{2.780648in}}%
\pgfpathlineto{\pgfqpoint{2.726324in}{2.795325in}}%
\pgfpathlineto{\pgfqpoint{2.707309in}{2.540607in}}%
\pgfpathlineto{\pgfqpoint{2.685709in}{2.235278in}}%
\pgfpathlineto{\pgfqpoint{2.667396in}{2.083811in}}%
\pgfpathlineto{\pgfqpoint{2.648850in}{2.011981in}}%
\pgfpathlineto{\pgfqpoint{2.626781in}{1.975335in}}%
\pgfpathlineto{\pgfqpoint{2.606356in}{1.978041in}}%
\pgfpathlineto{\pgfqpoint{2.588044in}{2.018321in}}%
\pgfpathlineto{\pgfqpoint{2.569262in}{2.109739in}}%
\pgfpathlineto{\pgfqpoint{2.553063in}{2.292734in}}%
\pgfpathlineto{\pgfqpoint{2.534749in}{2.622268in}}%
\pgfpathlineto{\pgfqpoint{2.513855in}{2.807291in}}%
\pgfpathlineto{\pgfqpoint{2.492726in}{2.700020in}}%
\pgfpathlineto{\pgfqpoint{2.474413in}{2.465935in}}%
\pgfpathlineto{\pgfqpoint{2.455161in}{2.205475in}}%
\pgfpathlineto{\pgfqpoint{2.438024in}{2.081633in}}%
\pgfpathlineto{\pgfqpoint{2.415487in}{2.000829in}}%
\pgfpathlineto{\pgfqpoint{2.401868in}{1.972875in}}%
\pgfpathlineto{\pgfqpoint{2.380271in}{1.975832in}}%
\pgfpathlineto{\pgfqpoint{2.360783in}{2.003982in}}%
\pgfpathlineto{\pgfqpoint{2.342003in}{2.080020in}}%
\pgfpathlineto{\pgfqpoint{2.317351in}{2.327964in}}%
\pgfpathlineto{\pgfqpoint{2.283309in}{2.767273in}}%
\pgfpathlineto{\pgfqpoint{2.265936in}{2.819816in}}%
\pgfpathlineto{\pgfqpoint{2.245979in}{2.650236in}}%
\pgfpathlineto{\pgfqpoint{2.224147in}{2.332998in}}%
\pgfpathlineto{\pgfqpoint{2.206068in}{2.211878in}}%
\pgfpathlineto{\pgfqpoint{2.187757in}{2.803923in}}%
\pgfpathlineto{\pgfqpoint{2.169209in}{2.558096in}}%
\pgfpathlineto{\pgfqpoint{2.145968in}{2.199085in}}%
\pgfpathlineto{\pgfqpoint{2.127420in}{2.064079in}}%
\pgfpathlineto{\pgfqpoint{2.111690in}{2.013084in}}%
\pgfpathlineto{\pgfqpoint{2.090561in}{1.975473in}}%
\pgfpathlineto{\pgfqpoint{2.071544in}{1.972936in}}%
\pgfpathlineto{\pgfqpoint{2.054170in}{2.000058in}}%
\pgfpathlineto{\pgfqpoint{2.035390in}{2.066465in}}%
\pgfpathlineto{\pgfqpoint{2.015199in}{2.213869in}}%
\pgfpathlineto{\pgfqpoint{1.995008in}{2.580289in}}%
\pgfpathlineto{\pgfqpoint{1.974114in}{2.808879in}}%
\pgfpathlineto{\pgfqpoint{1.956506in}{2.698206in}}%
\pgfpathlineto{\pgfqpoint{1.938429in}{2.827487in}}%
\pgfpathlineto{\pgfqpoint{1.916595in}{2.740746in}}%
\pgfpathlineto{\pgfqpoint{1.902038in}{2.505550in}}%
\pgfpathlineto{\pgfqpoint{1.878796in}{2.174185in}}%
\pgfpathlineto{\pgfqpoint{1.861893in}{2.050474in}}%
\pgfpathlineto{\pgfqpoint{1.842876in}{1.991993in}}%
\pgfpathlineto{\pgfqpoint{1.820573in}{1.974081in}}%
\pgfpathlineto{\pgfqpoint{1.802025in}{1.997083in}}%
\pgfpathlineto{\pgfqpoint{1.783948in}{2.057922in}}%
\pgfpathlineto{\pgfqpoint{1.763757in}{2.132815in}}%
\pgfpathlineto{\pgfqpoint{1.745211in}{2.330129in}}%
\pgfpathlineto{\pgfqpoint{1.724315in}{2.528027in}}%
\pgfpathlineto{\pgfqpoint{1.708116in}{2.808290in}}%
\pgfpathlineto{\pgfqpoint{1.688161in}{2.815975in}}%
\pgfpathlineto{\pgfqpoint{1.669144in}{2.629640in}}%
\pgfpathlineto{\pgfqpoint{1.649893in}{2.332614in}}%
\pgfpathlineto{\pgfqpoint{1.628293in}{2.148319in}}%
\pgfpathlineto{\pgfqpoint{1.611625in}{2.050564in}}%
\pgfpathlineto{\pgfqpoint{1.591434in}{1.989280in}}%
\pgfpathlineto{\pgfqpoint{1.574531in}{1.974551in}}%
\pgfpathlineto{\pgfqpoint{1.552932in}{2.007301in}}%
\pgfpathlineto{\pgfqpoint{1.533212in}{2.057072in}}%
\pgfpathlineto{\pgfqpoint{1.515133in}{2.165391in}}%
\pgfpathlineto{\pgfqpoint{1.496821in}{2.335643in}}%
\pgfpathlineto{\pgfqpoint{1.477101in}{2.621564in}}%
\pgfpathlineto{\pgfqpoint{1.457850in}{2.829531in}}%
\pgfpathlineto{\pgfqpoint{1.436485in}{2.852668in}}%
\pgfpathlineto{\pgfqpoint{1.419111in}{2.724949in}}%
\pgfpathlineto{\pgfqpoint{1.398922in}{2.412241in}}%
\pgfpathlineto{\pgfqpoint{1.379905in}{2.210541in}}%
\pgfpathlineto{\pgfqpoint{1.362063in}{2.092880in}}%
\pgfpathlineto{\pgfqpoint{1.341167in}{2.030368in}}%
\pgfpathlineto{\pgfqpoint{1.323090in}{1.998245in}}%
\pgfpathlineto{\pgfqpoint{1.301961in}{1.978355in}}%
\pgfpathlineto{\pgfqpoint{1.285056in}{1.998556in}}%
\pgfpathlineto{\pgfqpoint{1.264162in}{2.072118in}}%
\pgfpathlineto{\pgfqpoint{1.243736in}{2.209495in}}%
\pgfpathlineto{\pgfqpoint{1.189740in}{2.815133in}}%
\pgfpathlineto{\pgfqpoint{1.167435in}{2.880091in}}%
\pgfpathlineto{\pgfqpoint{1.147715in}{2.816158in}}%
\pgfpathlineto{\pgfqpoint{1.128463in}{2.584927in}}%
\pgfpathlineto{\pgfqpoint{1.108743in}{2.394751in}}%
\pgfpathlineto{\pgfqpoint{1.092544in}{2.202634in}}%
\pgfpathlineto{\pgfqpoint{1.072822in}{2.085712in}}%
\pgfpathlineto{\pgfqpoint{1.052396in}{2.017784in}}%
\pgfpathlineto{\pgfqpoint{1.034554in}{1.984278in}}%
\pgfpathlineto{\pgfqpoint{1.016477in}{2.125982in}}%
\pgfpathlineto{\pgfqpoint{0.993000in}{2.030209in}}%
\pgfpathlineto{\pgfqpoint{0.977506in}{1.992714in}}%
\pgfpathlineto{\pgfqpoint{0.957080in}{1.991600in}}%
\pgfpathlineto{\pgfqpoint{0.936420in}{2.037358in}}%
\pgfpathlineto{\pgfqpoint{0.919750in}{2.113840in}}%
\pgfpathlineto{\pgfqpoint{0.899325in}{2.311126in}}%
\pgfpathlineto{\pgfqpoint{0.880310in}{2.586953in}}%
\pgfpathlineto{\pgfqpoint{0.859650in}{2.829492in}}%
\pgfpathlineto{\pgfqpoint{0.842745in}{2.915959in}}%
\pgfpathlineto{\pgfqpoint{0.824903in}{2.842917in}}%
\pgfpathlineto{\pgfqpoint{0.804243in}{2.644654in}}%
\pgfpathlineto{\pgfqpoint{0.786166in}{2.390820in}}%
\pgfpathlineto{\pgfqpoint{0.764097in}{2.207212in}}%
\pgfpathlineto{\pgfqpoint{0.745784in}{2.098189in}}%
\pgfpathlineto{\pgfqpoint{0.725829in}{2.030897in}}%
\pgfpathlineto{\pgfqpoint{0.707987in}{1.991641in}}%
\pgfpathlineto{\pgfqpoint{0.689908in}{2.002370in}}%
\pgfpathlineto{\pgfqpoint{0.668545in}{2.070721in}}%
\pgfpathlineto{\pgfqpoint{0.650702in}{2.158906in}}%
\pgfpathlineto{\pgfqpoint{0.650936in}{2.153957in}}%
\pgfpathlineto{\pgfqpoint{0.657746in}{2.093417in}}%
\pgfpathlineto{\pgfqpoint{0.673005in}{2.023643in}}%
\pgfpathlineto{\pgfqpoint{0.694134in}{1.993552in}}%
\pgfpathlineto{\pgfqpoint{0.712682in}{2.055336in}}%
\pgfpathlineto{\pgfqpoint{0.733342in}{2.217128in}}%
\pgfpathlineto{\pgfqpoint{0.775836in}{2.901714in}}%
\pgfpathlineto{\pgfqpoint{0.790861in}{2.882803in}}%
\pgfpathlineto{\pgfqpoint{0.808469in}{2.600566in}}%
\pgfpathlineto{\pgfqpoint{0.829129in}{2.224177in}}%
\pgfpathlineto{\pgfqpoint{0.846971in}{2.066373in}}%
\pgfpathlineto{\pgfqpoint{0.865048in}{1.993112in}}%
\pgfpathlineto{\pgfqpoint{0.887351in}{2.007369in}}%
\pgfpathlineto{\pgfqpoint{0.906134in}{2.081856in}}%
\pgfpathlineto{\pgfqpoint{0.923038in}{2.254502in}}%
\pgfpathlineto{\pgfqpoint{0.943462in}{2.606176in}}%
\pgfpathlineto{\pgfqpoint{0.965296in}{2.888777in}}%
\pgfpathlineto{\pgfqpoint{0.980558in}{2.814300in}}%
\pgfpathlineto{\pgfqpoint{1.001217in}{2.446103in}}%
\pgfpathlineto{\pgfqpoint{1.020938in}{2.133782in}}%
\pgfpathlineto{\pgfqpoint{1.039251in}{2.024316in}}%
\pgfpathlineto{\pgfqpoint{1.060614in}{1.980084in}}%
\pgfpathlineto{\pgfqpoint{1.080571in}{2.027756in}}%
\pgfpathlineto{\pgfqpoint{1.098882in}{2.135357in}}%
\pgfpathlineto{\pgfqpoint{1.115785in}{2.355724in}}%
\pgfpathlineto{\pgfqpoint{1.136681in}{2.752321in}}%
\pgfpathlineto{\pgfqpoint{1.157107in}{2.864161in}}%
\pgfpathlineto{\pgfqpoint{1.174713in}{2.667529in}}%
\pgfpathlineto{\pgfqpoint{1.194435in}{2.290743in}}%
\pgfpathlineto{\pgfqpoint{1.214155in}{2.094566in}}%
\pgfpathlineto{\pgfqpoint{1.230825in}{2.003114in}}%
\pgfpathlineto{\pgfqpoint{1.251485in}{1.976865in}}%
\pgfpathlineto{\pgfqpoint{1.270031in}{2.020884in}}%
\pgfpathlineto{\pgfqpoint{1.289282in}{2.131024in}}%
\pgfpathlineto{\pgfqpoint{1.310882in}{2.362212in}}%
\pgfpathlineto{\pgfqpoint{1.328724in}{2.732662in}}%
\pgfpathlineto{\pgfqpoint{1.349384in}{2.852925in}}%
\pgfpathlineto{\pgfqpoint{1.366053in}{2.650424in}}%
\pgfpathlineto{\pgfqpoint{1.384366in}{2.359121in}}%
\pgfpathlineto{\pgfqpoint{1.406200in}{2.120989in}}%
\pgfpathlineto{\pgfqpoint{1.425451in}{2.015749in}}%
\pgfpathlineto{\pgfqpoint{1.442354in}{1.976120in}}%
\pgfpathlineto{\pgfqpoint{1.461840in}{1.985264in}}%
\pgfpathlineto{\pgfqpoint{1.483205in}{2.053000in}}%
\pgfpathlineto{\pgfqpoint{1.501751in}{2.088704in}}%
\pgfpathlineto{\pgfqpoint{1.518890in}{2.210046in}}%
\pgfpathlineto{\pgfqpoint{1.558332in}{2.818765in}}%
\pgfpathlineto{\pgfqpoint{1.580401in}{2.770476in}}%
\pgfpathlineto{\pgfqpoint{1.597538in}{2.450369in}}%
\pgfpathlineto{\pgfqpoint{1.614912in}{2.180374in}}%
\pgfpathlineto{\pgfqpoint{1.635571in}{2.036344in}}%
\pgfpathlineto{\pgfqpoint{1.653885in}{1.980621in}}%
\pgfpathlineto{\pgfqpoint{1.674545in}{1.981342in}}%
\pgfpathlineto{\pgfqpoint{1.692622in}{2.030974in}}%
\pgfpathlineto{\pgfqpoint{1.713751in}{2.154792in}}%
\pgfpathlineto{\pgfqpoint{1.731593in}{2.355636in}}%
\pgfpathlineto{\pgfqpoint{1.752253in}{2.744163in}}%
\pgfpathlineto{\pgfqpoint{1.772678in}{2.828543in}}%
\pgfpathlineto{\pgfqpoint{1.790286in}{2.651108in}}%
\pgfpathlineto{\pgfqpoint{1.808129in}{2.330514in}}%
\pgfpathlineto{\pgfqpoint{1.827146in}{2.182611in}}%
\pgfpathlineto{\pgfqpoint{1.848980in}{2.036481in}}%
\pgfpathlineto{\pgfqpoint{1.867057in}{1.990211in}}%
\pgfpathlineto{\pgfqpoint{1.885605in}{1.971047in}}%
\pgfpathlineto{\pgfqpoint{1.906265in}{2.012567in}}%
\pgfpathlineto{\pgfqpoint{1.922933in}{2.085632in}}%
\pgfpathlineto{\pgfqpoint{1.944533in}{2.290847in}}%
\pgfpathlineto{\pgfqpoint{1.962141in}{2.470453in}}%
\pgfpathlineto{\pgfqpoint{1.983035in}{2.737527in}}%
\pgfpathlineto{\pgfqpoint{2.001112in}{2.822364in}}%
\pgfpathlineto{\pgfqpoint{2.019660in}{2.738897in}}%
\pgfpathlineto{\pgfqpoint{2.038206in}{2.419257in}}%
\pgfpathlineto{\pgfqpoint{2.059102in}{2.204930in}}%
\pgfpathlineto{\pgfqpoint{2.080465in}{2.055761in}}%
\pgfpathlineto{\pgfqpoint{2.094787in}{1.998166in}}%
\pgfpathlineto{\pgfqpoint{2.117090in}{1.971363in}}%
\pgfpathlineto{\pgfqpoint{2.135638in}{1.995589in}}%
\pgfpathlineto{\pgfqpoint{2.155592in}{2.054447in}}%
\pgfpathlineto{\pgfqpoint{2.173201in}{2.178684in}}%
\pgfpathlineto{\pgfqpoint{2.193157in}{2.387384in}}%
\pgfpathlineto{\pgfqpoint{2.213817in}{2.726068in}}%
\pgfpathlineto{\pgfqpoint{2.232128in}{2.797084in}}%
\pgfpathlineto{\pgfqpoint{2.251614in}{2.785607in}}%
\pgfpathlineto{\pgfqpoint{2.270396in}{2.539541in}}%
\pgfpathlineto{\pgfqpoint{2.290587in}{2.192366in}}%
\pgfpathlineto{\pgfqpoint{2.305612in}{2.067258in}}%
\pgfpathlineto{\pgfqpoint{2.326976in}{2.008024in}}%
\pgfpathlineto{\pgfqpoint{2.347636in}{1.976352in}}%
\pgfpathlineto{\pgfqpoint{2.369941in}{1.977011in}}%
\pgfpathlineto{\pgfqpoint{2.384495in}{2.007446in}}%
\pgfpathlineto{\pgfqpoint{2.405860in}{2.098377in}}%
\pgfpathlineto{\pgfqpoint{2.422999in}{2.262378in}}%
\pgfpathlineto{\pgfqpoint{2.444128in}{2.618031in}}%
\pgfpathlineto{\pgfqpoint{2.461502in}{2.778719in}}%
\pgfpathlineto{\pgfqpoint{2.480282in}{2.806610in}}%
\pgfpathlineto{\pgfqpoint{2.499770in}{2.661950in}}%
\pgfpathlineto{\pgfqpoint{2.520195in}{2.329829in}}%
\pgfpathlineto{\pgfqpoint{2.540384in}{2.099943in}}%
\pgfpathlineto{\pgfqpoint{2.558227in}{2.021328in}}%
\pgfpathlineto{\pgfqpoint{2.597903in}{1.972996in}}%
\pgfpathlineto{\pgfqpoint{2.615511in}{1.983580in}}%
\pgfpathlineto{\pgfqpoint{2.634528in}{2.023866in}}%
\pgfpathlineto{\pgfqpoint{2.654954in}{2.138878in}}%
\pgfpathlineto{\pgfqpoint{2.672796in}{2.338469in}}%
\pgfpathlineto{\pgfqpoint{2.691109in}{2.118718in}}%
\pgfpathlineto{\pgfqpoint{2.711064in}{2.341529in}}%
\pgfpathlineto{\pgfqpoint{2.730550in}{2.666844in}}%
\pgfpathlineto{\pgfqpoint{2.750975in}{2.817709in}}%
\pgfpathlineto{\pgfqpoint{2.767645in}{2.750590in}}%
\pgfpathlineto{\pgfqpoint{2.787600in}{2.559628in}}%
\pgfpathlineto{\pgfqpoint{2.808494in}{2.228444in}}%
\pgfpathlineto{\pgfqpoint{2.830094in}{2.092004in}}%
\pgfpathlineto{\pgfqpoint{2.848171in}{2.011664in}}%
\pgfpathlineto{\pgfqpoint{2.865545in}{1.976141in}}%
\pgfpathlineto{\pgfqpoint{2.884092in}{1.979423in}}%
\pgfpathlineto{\pgfqpoint{2.904987in}{2.033316in}}%
\pgfpathlineto{\pgfqpoint{2.923533in}{2.128573in}}%
\pgfpathlineto{\pgfqpoint{2.943958in}{2.344832in}}%
\pgfpathlineto{\pgfqpoint{2.962975in}{2.623512in}}%
\pgfpathlineto{\pgfqpoint{2.980348in}{2.807723in}}%
\pgfpathlineto{\pgfqpoint{3.001008in}{2.808170in}}%
\pgfpathlineto{\pgfqpoint{3.017911in}{2.706511in}}%
\pgfpathlineto{\pgfqpoint{3.039511in}{2.393691in}}%
\pgfpathlineto{\pgfqpoint{3.058059in}{2.149613in}}%
\pgfpathlineto{\pgfqpoint{3.076839in}{2.050640in}}%
\pgfpathlineto{\pgfqpoint{3.094918in}{1.994923in}}%
\pgfpathlineto{\pgfqpoint{3.116047in}{1.972662in}}%
\pgfpathlineto{\pgfqpoint{3.136472in}{1.996998in}}%
\pgfpathlineto{\pgfqpoint{3.155489in}{2.060962in}}%
\pgfpathlineto{\pgfqpoint{3.174506in}{2.171431in}}%
\pgfpathlineto{\pgfqpoint{3.194460in}{2.437592in}}%
\pgfpathlineto{\pgfqpoint{3.212537in}{2.609018in}}%
\pgfpathlineto{\pgfqpoint{3.230382in}{2.820114in}}%
\pgfpathlineto{\pgfqpoint{3.251745in}{2.807908in}}%
\pgfpathlineto{\pgfqpoint{3.269587in}{2.625862in}}%
\pgfpathlineto{\pgfqpoint{3.288135in}{2.323425in}}%
\pgfpathlineto{\pgfqpoint{3.309733in}{2.135792in}}%
\pgfpathlineto{\pgfqpoint{3.326638in}{2.035855in}}%
\pgfpathlineto{\pgfqpoint{3.345889in}{1.986924in}}%
\pgfpathlineto{\pgfqpoint{3.367723in}{1.977157in}}%
\pgfpathlineto{\pgfqpoint{3.385097in}{2.010131in}}%
\pgfpathlineto{\pgfqpoint{3.404348in}{2.049968in}}%
\pgfpathlineto{\pgfqpoint{3.422190in}{2.129535in}}%
\pgfpathlineto{\pgfqpoint{3.441910in}{2.301251in}}%
\pgfpathlineto{\pgfqpoint{3.463510in}{2.508730in}}%
\pgfpathlineto{\pgfqpoint{3.479004in}{2.741866in}}%
\pgfpathlineto{\pgfqpoint{3.500838in}{2.859633in}}%
\pgfpathlineto{\pgfqpoint{3.522204in}{2.795295in}}%
\pgfpathlineto{\pgfqpoint{3.521733in}{2.667290in}}%
\pgfpathlineto{\pgfqpoint{3.537463in}{2.593522in}}%
\pgfpathlineto{\pgfqpoint{3.558592in}{2.284652in}}%
\pgfpathlineto{\pgfqpoint{3.578783in}{2.150749in}}%
\pgfpathlineto{\pgfqpoint{3.597565in}{2.046843in}}%
\pgfpathlineto{\pgfqpoint{3.612825in}{2.002108in}}%
\pgfpathlineto{\pgfqpoint{3.635362in}{1.976665in}}%
\pgfpathlineto{\pgfqpoint{3.651327in}{1.993608in}}%
\pgfpathlineto{\pgfqpoint{3.673867in}{2.029111in}}%
\pgfpathlineto{\pgfqpoint{3.693118in}{2.123748in}}%
\pgfpathlineto{\pgfqpoint{3.712369in}{2.225703in}}%
\pgfpathlineto{\pgfqpoint{3.732795in}{2.452022in}}%
\pgfpathlineto{\pgfqpoint{3.751341in}{2.624543in}}%
\pgfpathlineto{\pgfqpoint{3.769654in}{2.842908in}}%
\pgfpathlineto{\pgfqpoint{3.786557in}{2.877873in}}%
\pgfpathlineto{\pgfqpoint{3.809094in}{2.821817in}}%
\pgfpathlineto{\pgfqpoint{3.826937in}{2.557197in}}%
\pgfpathlineto{\pgfqpoint{3.847362in}{2.281061in}}%
\pgfpathlineto{\pgfqpoint{3.866144in}{2.148402in}}%
\pgfpathlineto{\pgfqpoint{3.883752in}{2.117483in}}%
\pgfpathlineto{\pgfqpoint{3.902535in}{2.171612in}}%
\pgfpathlineto{\pgfqpoint{3.922020in}{2.049015in}}%
\pgfpathlineto{\pgfqpoint{3.940803in}{1.998267in}}%
\pgfpathlineto{\pgfqpoint{3.962871in}{1.982719in}}%
\pgfpathlineto{\pgfqpoint{3.986349in}{2.035046in}}%
\pgfpathlineto{\pgfqpoint{3.997616in}{2.085757in}}%
\pgfpathlineto{\pgfqpoint{4.019451in}{2.205734in}}%
\pgfpathlineto{\pgfqpoint{4.039407in}{2.397949in}}%
\pgfpathlineto{\pgfqpoint{4.057484in}{2.696362in}}%
\pgfpathlineto{\pgfqpoint{4.076266in}{2.894121in}}%
\pgfpathlineto{\pgfqpoint{4.095518in}{2.903974in}}%
\pgfpathlineto{\pgfqpoint{4.117586in}{2.750698in}}%
\pgfpathlineto{\pgfqpoint{4.134723in}{2.462908in}}%
\pgfpathlineto{\pgfqpoint{4.154914in}{2.264942in}}%
\pgfpathlineto{\pgfqpoint{4.172991in}{2.171002in}}%
\pgfpathlineto{\pgfqpoint{4.192008in}{2.064345in}}%
\pgfpathlineto{\pgfqpoint{4.210790in}{2.005381in}}%
\pgfpathlineto{\pgfqpoint{4.234033in}{1.985930in}}%
\pgfpathlineto{\pgfqpoint{4.249293in}{2.006151in}}%
\pgfpathlineto{\pgfqpoint{4.267604in}{2.062449in}}%
\pgfpathlineto{\pgfqpoint{4.288501in}{2.150862in}}%
\pgfpathlineto{\pgfqpoint{4.308690in}{2.293481in}}%
\pgfpathlineto{\pgfqpoint{4.327472in}{2.545053in}}%
\pgfpathlineto{\pgfqpoint{4.345549in}{2.646434in}}%
\pgfpathlineto{\pgfqpoint{4.364800in}{2.888784in}}%
\pgfpathlineto{\pgfqpoint{4.383113in}{2.939481in}}%
\pgfpathlineto{\pgfqpoint{4.402599in}{2.852098in}}%
\pgfpathlineto{\pgfqpoint{4.420442in}{2.625128in}}%
\pgfpathlineto{\pgfqpoint{4.439458in}{2.341157in}}%
\pgfpathlineto{\pgfqpoint{4.463170in}{2.130293in}}%
\pgfpathlineto{\pgfqpoint{4.480778in}{2.043973in}}%
\pgfpathlineto{\pgfqpoint{4.474206in}{2.078242in}}%
\pgfpathlineto{\pgfqpoint{4.453546in}{2.288878in}}%
\pgfpathlineto{\pgfqpoint{4.434998in}{2.652037in}}%
\pgfpathlineto{\pgfqpoint{4.417155in}{2.915939in}}%
\pgfpathlineto{\pgfqpoint{4.398607in}{2.888876in}}%
\pgfpathlineto{\pgfqpoint{4.377479in}{2.474717in}}%
\pgfpathlineto{\pgfqpoint{4.357993in}{2.191736in}}%
\pgfpathlineto{\pgfqpoint{4.336393in}{2.037991in}}%
\pgfpathlineto{\pgfqpoint{4.336393in}{2.001421in}}%
\pgfpathlineto{\pgfqpoint{4.322542in}{1.988922in}}%
\pgfpathlineto{\pgfqpoint{4.300708in}{2.025211in}}%
\pgfpathlineto{\pgfqpoint{4.280752in}{2.135137in}}%
\pgfpathlineto{\pgfqpoint{4.261971in}{2.446948in}}%
\pgfpathlineto{\pgfqpoint{4.242249in}{2.807701in}}%
\pgfpathlineto{\pgfqpoint{4.223938in}{2.912298in}}%
\pgfpathlineto{\pgfqpoint{4.204687in}{2.688867in}}%
\pgfpathlineto{\pgfqpoint{4.186139in}{2.315358in}}%
\pgfpathlineto{\pgfqpoint{4.167357in}{2.103056in}}%
\pgfpathlineto{\pgfqpoint{4.148576in}{2.009317in}}%
\pgfpathlineto{\pgfqpoint{4.130028in}{1.984564in}}%
\pgfpathlineto{\pgfqpoint{4.106551in}{2.065305in}}%
\pgfpathlineto{\pgfqpoint{4.091526in}{2.204222in}}%
\pgfpathlineto{\pgfqpoint{4.070397in}{2.555961in}}%
\pgfpathlineto{\pgfqpoint{4.051615in}{2.826169in}}%
\pgfpathlineto{\pgfqpoint{4.034241in}{2.437535in}}%
\pgfpathlineto{\pgfqpoint{4.011938in}{2.129353in}}%
\pgfpathlineto{\pgfqpoint{3.994096in}{2.018736in}}%
\pgfpathlineto{\pgfqpoint{3.975782in}{1.978468in}}%
\pgfpathlineto{\pgfqpoint{3.955122in}{2.012610in}}%
\pgfpathlineto{\pgfqpoint{3.936811in}{2.119529in}}%
\pgfpathlineto{\pgfqpoint{3.917794in}{2.358138in}}%
\pgfpathlineto{\pgfqpoint{3.895726in}{2.781377in}}%
\pgfpathlineto{\pgfqpoint{3.877649in}{2.711712in}}%
\pgfpathlineto{\pgfqpoint{3.859101in}{2.869503in}}%
\pgfpathlineto{\pgfqpoint{3.839615in}{2.686681in}}%
\pgfpathlineto{\pgfqpoint{3.822476in}{2.326055in}}%
\pgfpathlineto{\pgfqpoint{3.799235in}{2.095291in}}%
\pgfpathlineto{\pgfqpoint{3.783505in}{2.008796in}}%
\pgfpathlineto{\pgfqpoint{3.762610in}{1.975606in}}%
\pgfpathlineto{\pgfqpoint{3.742888in}{2.016830in}}%
\pgfpathlineto{\pgfqpoint{3.724577in}{2.134454in}}%
\pgfpathlineto{\pgfqpoint{3.705794in}{2.369776in}}%
\pgfpathlineto{\pgfqpoint{3.685603in}{2.710779in}}%
\pgfpathlineto{\pgfqpoint{3.667292in}{2.854579in}}%
\pgfpathlineto{\pgfqpoint{3.648510in}{2.673050in}}%
\pgfpathlineto{\pgfqpoint{3.630433in}{2.348202in}}%
\pgfpathlineto{\pgfqpoint{3.606721in}{2.085464in}}%
\pgfpathlineto{\pgfqpoint{3.590756in}{2.010898in}}%
\pgfpathlineto{\pgfqpoint{3.571739in}{1.976557in}}%
\pgfpathlineto{\pgfqpoint{3.549671in}{1.990504in}}%
\pgfpathlineto{\pgfqpoint{3.527837in}{2.075199in}}%
\pgfpathlineto{\pgfqpoint{3.512812in}{2.200011in}}%
\pgfpathlineto{\pgfqpoint{3.496612in}{2.394343in}}%
\pgfpathlineto{\pgfqpoint{3.475249in}{2.731218in}}%
\pgfpathlineto{\pgfqpoint{3.456467in}{2.836315in}}%
\pgfpathlineto{\pgfqpoint{3.438624in}{2.722082in}}%
\pgfpathlineto{\pgfqpoint{3.415616in}{2.318759in}}%
\pgfpathlineto{\pgfqpoint{3.397304in}{2.134814in}}%
\pgfpathlineto{\pgfqpoint{3.376879in}{2.016321in}}%
\pgfpathlineto{\pgfqpoint{3.358802in}{1.975774in}}%
\pgfpathlineto{\pgfqpoint{3.341428in}{1.980924in}}%
\pgfpathlineto{\pgfqpoint{3.322177in}{2.032262in}}%
\pgfpathlineto{\pgfqpoint{3.301517in}{2.161254in}}%
\pgfpathlineto{\pgfqpoint{3.281795in}{2.297781in}}%
\pgfpathlineto{\pgfqpoint{3.263015in}{2.637543in}}%
\pgfpathlineto{\pgfqpoint{3.240946in}{2.814271in}}%
\pgfpathlineto{\pgfqpoint{3.226155in}{2.791694in}}%
\pgfpathlineto{\pgfqpoint{3.186008in}{2.199440in}}%
\pgfpathlineto{\pgfqpoint{3.170279in}{2.066658in}}%
\pgfpathlineto{\pgfqpoint{3.146097in}{1.993501in}}%
\pgfpathlineto{\pgfqpoint{3.128254in}{1.972011in}}%
\pgfpathlineto{\pgfqpoint{3.112055in}{1.982653in}}%
\pgfpathlineto{\pgfqpoint{3.090457in}{2.042315in}}%
\pgfpathlineto{\pgfqpoint{3.068623in}{2.194185in}}%
\pgfpathlineto{\pgfqpoint{3.053362in}{2.368877in}}%
\pgfpathlineto{\pgfqpoint{3.031998in}{2.739816in}}%
\pgfpathlineto{\pgfqpoint{3.030824in}{2.779694in}}%
\pgfpathlineto{\pgfqpoint{3.013450in}{2.816516in}}%
\pgfpathlineto{\pgfqpoint{2.993730in}{2.739000in}}%
\pgfpathlineto{\pgfqpoint{2.975653in}{2.426513in}}%
\pgfpathlineto{\pgfqpoint{2.956166in}{2.230640in}}%
\pgfpathlineto{\pgfqpoint{2.938089in}{2.077528in}}%
\pgfpathlineto{\pgfqpoint{2.918603in}{2.001482in}}%
\pgfpathlineto{\pgfqpoint{2.892074in}{1.970970in}}%
\pgfpathlineto{\pgfqpoint{2.880335in}{1.972878in}}%
\pgfpathlineto{\pgfqpoint{2.859441in}{2.014746in}}%
\pgfpathlineto{\pgfqpoint{2.841364in}{2.104246in}}%
\pgfpathlineto{\pgfqpoint{2.822347in}{2.178894in}}%
\pgfpathlineto{\pgfqpoint{2.801216in}{2.500061in}}%
\pgfpathlineto{\pgfqpoint{2.781262in}{2.779409in}}%
\pgfpathlineto{\pgfqpoint{2.764357in}{2.796295in}}%
\pgfpathlineto{\pgfqpoint{2.762479in}{2.712411in}}%
\pgfpathlineto{\pgfqpoint{2.745106in}{2.604767in}}%
\pgfpathlineto{\pgfqpoint{2.725151in}{2.333788in}}%
\pgfpathlineto{\pgfqpoint{2.705195in}{2.107938in}}%
\pgfpathlineto{\pgfqpoint{2.685240in}{2.252815in}}%
\pgfpathlineto{\pgfqpoint{2.667632in}{2.579278in}}%
\pgfpathlineto{\pgfqpoint{2.648381in}{2.818794in}}%
\pgfpathlineto{\pgfqpoint{2.629599in}{2.729523in}}%
\pgfpathlineto{\pgfqpoint{2.608468in}{2.374706in}}%
\pgfpathlineto{\pgfqpoint{2.590391in}{2.151891in}}%
\pgfpathlineto{\pgfqpoint{2.573957in}{2.042456in}}%
\pgfpathlineto{\pgfqpoint{2.552594in}{1.976762in}}%
\pgfpathlineto{\pgfqpoint{2.533577in}{1.971466in}}%
\pgfpathlineto{\pgfqpoint{2.512446in}{2.018723in}}%
\pgfpathlineto{\pgfqpoint{2.493195in}{2.068099in}}%
\pgfpathlineto{\pgfqpoint{2.493429in}{2.167219in}}%
\pgfpathlineto{\pgfqpoint{2.475352in}{2.244086in}}%
\pgfpathlineto{\pgfqpoint{2.456336in}{2.563779in}}%
\pgfpathlineto{\pgfqpoint{2.438493in}{2.725739in}}%
\pgfpathlineto{\pgfqpoint{2.419242in}{2.817533in}}%
\pgfpathlineto{\pgfqpoint{2.398113in}{2.570783in}}%
\pgfpathlineto{\pgfqpoint{2.378862in}{2.270376in}}%
\pgfpathlineto{\pgfqpoint{2.361019in}{2.136122in}}%
\pgfpathlineto{\pgfqpoint{2.343175in}{2.028211in}}%
\pgfpathlineto{\pgfqpoint{2.320403in}{1.974647in}}%
\pgfpathlineto{\pgfqpoint{2.302092in}{1.978993in}}%
\pgfpathlineto{\pgfqpoint{2.284952in}{2.017064in}}%
\pgfpathlineto{\pgfqpoint{2.264293in}{2.136865in}}%
\pgfpathlineto{\pgfqpoint{2.245510in}{2.384638in}}%
\pgfpathlineto{\pgfqpoint{2.224147in}{2.704608in}}%
\pgfpathlineto{\pgfqpoint{2.224850in}{2.804547in}}%
\pgfpathlineto{\pgfqpoint{2.205365in}{2.824151in}}%
\pgfpathlineto{\pgfqpoint{2.187991in}{2.669515in}}%
\pgfpathlineto{\pgfqpoint{2.169209in}{2.377914in}}%
\pgfpathlineto{\pgfqpoint{2.147377in}{2.162181in}}%
\pgfpathlineto{\pgfqpoint{2.128594in}{2.045591in}}%
\pgfpathlineto{\pgfqpoint{2.110517in}{1.988463in}}%
\pgfpathlineto{\pgfqpoint{2.093613in}{1.971689in}}%
\pgfpathlineto{\pgfqpoint{2.070604in}{2.004993in}}%
\pgfpathlineto{\pgfqpoint{2.052293in}{2.075576in}}%
\pgfpathlineto{\pgfqpoint{2.033042in}{2.239777in}}%
\pgfpathlineto{\pgfqpoint{2.012616in}{2.580365in}}%
\pgfpathlineto{\pgfqpoint{1.996651in}{2.791116in}}%
\pgfpathlineto{\pgfqpoint{1.971531in}{2.784457in}}%
\pgfpathlineto{\pgfqpoint{1.959089in}{2.652989in}}%
\pgfpathlineto{\pgfqpoint{1.937723in}{2.379821in}}%
\pgfpathlineto{\pgfqpoint{1.918707in}{2.167115in}}%
\pgfpathlineto{\pgfqpoint{1.898047in}{2.039925in}}%
\pgfpathlineto{\pgfqpoint{1.879735in}{1.987473in}}%
\pgfpathlineto{\pgfqpoint{1.859779in}{1.973392in}}%
\pgfpathlineto{\pgfqpoint{1.842171in}{1.988866in}}%
\pgfpathlineto{\pgfqpoint{1.821511in}{2.054341in}}%
\pgfpathlineto{\pgfqpoint{1.802025in}{2.144535in}}%
\pgfpathlineto{\pgfqpoint{1.781600in}{2.387931in}}%
\pgfpathlineto{\pgfqpoint{1.765166in}{2.680521in}}%
\pgfpathlineto{\pgfqpoint{1.744272in}{2.806000in}}%
\pgfpathlineto{\pgfqpoint{1.727369in}{2.840309in}}%
\pgfpathlineto{\pgfqpoint{1.707178in}{2.671188in}}%
\pgfpathlineto{\pgfqpoint{1.668205in}{2.177083in}}%
\pgfpathlineto{\pgfqpoint{1.645433in}{2.061466in}}%
\pgfpathlineto{\pgfqpoint{1.631111in}{2.015737in}}%
\pgfpathlineto{\pgfqpoint{1.609748in}{1.977699in}}%
\pgfpathlineto{\pgfqpoint{1.588617in}{1.988968in}}%
\pgfpathlineto{\pgfqpoint{1.572652in}{2.023456in}}%
\pgfpathlineto{\pgfqpoint{1.551992in}{2.090402in}}%
\pgfpathlineto{\pgfqpoint{1.533915in}{2.166676in}}%
\pgfpathlineto{\pgfqpoint{1.514664in}{2.025046in}}%
\pgfpathlineto{\pgfqpoint{1.497996in}{1.982370in}}%
\pgfpathlineto{\pgfqpoint{1.476630in}{1.988453in}}%
\pgfpathlineto{\pgfqpoint{1.455736in}{2.030245in}}%
\pgfpathlineto{\pgfqpoint{1.438128in}{2.119463in}}%
\pgfpathlineto{\pgfqpoint{1.418173in}{2.347217in}}%
\pgfpathlineto{\pgfqpoint{1.401269in}{2.623911in}}%
\pgfpathlineto{\pgfqpoint{1.379200in}{2.848420in}}%
\pgfpathlineto{\pgfqpoint{1.361826in}{2.843445in}}%
\pgfpathlineto{\pgfqpoint{1.321446in}{2.328727in}}%
\pgfpathlineto{\pgfqpoint{1.303369in}{2.147780in}}%
\pgfpathlineto{\pgfqpoint{1.284587in}{2.045822in}}%
\pgfpathlineto{\pgfqpoint{1.265805in}{2.020328in}}%
\pgfpathlineto{\pgfqpoint{1.246319in}{1.979095in}}%
\pgfpathlineto{\pgfqpoint{1.227302in}{1.992796in}}%
\pgfpathlineto{\pgfqpoint{1.206642in}{2.054231in}}%
\pgfpathlineto{\pgfqpoint{1.183165in}{2.216138in}}%
\pgfpathlineto{\pgfqpoint{1.168140in}{2.408461in}}%
\pgfpathlineto{\pgfqpoint{1.151001in}{2.644714in}}%
\pgfpathlineto{\pgfqpoint{1.133393in}{2.858660in}}%
\pgfpathlineto{\pgfqpoint{1.109212in}{2.849015in}}%
\pgfpathlineto{\pgfqpoint{1.093013in}{2.671143in}}%
\pgfpathlineto{\pgfqpoint{1.073056in}{2.324396in}}%
\pgfpathlineto{\pgfqpoint{1.052867in}{2.168781in}}%
\pgfpathlineto{\pgfqpoint{1.035259in}{2.081224in}}%
\pgfpathlineto{\pgfqpoint{1.014599in}{2.013185in}}%
\pgfpathlineto{\pgfqpoint{0.993705in}{1.982976in}}%
\pgfpathlineto{\pgfqpoint{0.976097in}{2.010241in}}%
\pgfpathlineto{\pgfqpoint{0.959192in}{2.047343in}}%
\pgfpathlineto{\pgfqpoint{0.940410in}{2.145801in}}%
\pgfpathlineto{\pgfqpoint{0.920455in}{2.277803in}}%
\pgfpathlineto{\pgfqpoint{0.895100in}{2.609455in}}%
\pgfpathlineto{\pgfqpoint{0.880779in}{2.815231in}}%
\pgfpathlineto{\pgfqpoint{0.859413in}{2.914807in}}%
\pgfpathlineto{\pgfqpoint{0.841336in}{2.866785in}}%
\pgfpathlineto{\pgfqpoint{0.823025in}{2.710395in}}%
\pgfpathlineto{\pgfqpoint{0.805652in}{2.438360in}}%
\pgfpathlineto{\pgfqpoint{0.784052in}{2.240641in}}%
\pgfpathlineto{\pgfqpoint{0.766209in}{2.121364in}}%
\pgfpathlineto{\pgfqpoint{0.743437in}{2.039382in}}%
\pgfpathlineto{\pgfqpoint{0.726064in}{1.994827in}}%
\pgfpathlineto{\pgfqpoint{0.708690in}{1.991306in}}%
\pgfpathlineto{\pgfqpoint{0.688499in}{2.038026in}}%
\pgfpathlineto{\pgfqpoint{0.669719in}{2.117516in}}%
\pgfpathlineto{\pgfqpoint{0.650468in}{2.214366in}}%
\pgfpathlineto{\pgfqpoint{0.650468in}{2.203978in}}%
\pgfpathlineto{\pgfqpoint{0.655163in}{2.180215in}}%
\pgfpathlineto{\pgfqpoint{0.674648in}{2.033976in}}%
\pgfpathlineto{\pgfqpoint{0.696248in}{1.988211in}}%
\pgfpathlineto{\pgfqpoint{0.714091in}{2.037666in}}%
\pgfpathlineto{\pgfqpoint{0.731933in}{2.147806in}}%
\pgfpathlineto{\pgfqpoint{0.752827in}{2.425479in}}%
\pgfpathlineto{\pgfqpoint{0.769967in}{2.792864in}}%
\pgfpathlineto{\pgfqpoint{0.789218in}{2.915272in}}%
\pgfpathlineto{\pgfqpoint{0.809643in}{2.679831in}}%
\pgfpathlineto{\pgfqpoint{0.827486in}{2.328313in}}%
\pgfpathlineto{\pgfqpoint{0.847911in}{2.085033in}}%
\pgfpathlineto{\pgfqpoint{0.865754in}{2.003717in}}%
\pgfpathlineto{\pgfqpoint{0.887117in}{1.993888in}}%
\pgfpathlineto{\pgfqpoint{0.904256in}{2.056122in}}%
\pgfpathlineto{\pgfqpoint{0.929377in}{2.444373in}}%
\pgfpathlineto{\pgfqpoint{0.946985in}{2.819400in}}%
\pgfpathlineto{\pgfqpoint{0.962010in}{2.889953in}}%
\pgfpathlineto{\pgfqpoint{0.982670in}{2.624763in}}%
\pgfpathlineto{\pgfqpoint{1.003564in}{2.236151in}}%
\pgfpathlineto{\pgfqpoint{1.022112in}{2.066283in}}%
\pgfpathlineto{\pgfqpoint{1.040189in}{1.995678in}}%
\pgfpathlineto{\pgfqpoint{1.061083in}{1.991114in}}%
\pgfpathlineto{\pgfqpoint{1.079162in}{2.059954in}}%
\pgfpathlineto{\pgfqpoint{1.095830in}{2.193155in}}%
\pgfpathlineto{\pgfqpoint{1.115316in}{2.512743in}}%
\pgfpathlineto{\pgfqpoint{1.139968in}{2.860256in}}%
\pgfpathlineto{\pgfqpoint{1.156167in}{2.800656in}}%
\pgfpathlineto{\pgfqpoint{1.173775in}{2.458007in}}%
\pgfpathlineto{\pgfqpoint{1.195138in}{2.153027in}}%
\pgfpathlineto{\pgfqpoint{1.212981in}{2.030850in}}%
\pgfpathlineto{\pgfqpoint{1.234580in}{1.978891in}}%
\pgfpathlineto{\pgfqpoint{1.252423in}{1.994417in}}%
\pgfpathlineto{\pgfqpoint{1.271205in}{2.047543in}}%
\pgfpathlineto{\pgfqpoint{1.291396in}{2.208403in}}%
\pgfpathlineto{\pgfqpoint{1.328255in}{2.794028in}}%
\pgfpathlineto{\pgfqpoint{1.348915in}{2.811223in}}%
\pgfpathlineto{\pgfqpoint{1.385304in}{2.222386in}}%
\pgfpathlineto{\pgfqpoint{1.405026in}{2.052822in}}%
\pgfpathlineto{\pgfqpoint{1.429441in}{1.978432in}}%
\pgfpathlineto{\pgfqpoint{1.444468in}{1.977937in}}%
\pgfpathlineto{\pgfqpoint{1.464423in}{2.044112in}}%
\pgfpathlineto{\pgfqpoint{1.482736in}{2.126486in}}%
\pgfpathlineto{\pgfqpoint{1.501047in}{2.307052in}}%
\pgfpathlineto{\pgfqpoint{1.518890in}{2.650038in}}%
\pgfpathlineto{\pgfqpoint{1.540253in}{2.842567in}}%
\pgfpathlineto{\pgfqpoint{1.559270in}{2.713705in}}%
\pgfpathlineto{\pgfqpoint{1.579461in}{2.487830in}}%
\pgfpathlineto{\pgfqpoint{1.597069in}{2.205906in}}%
\pgfpathlineto{\pgfqpoint{1.616789in}{2.040348in}}%
\pgfpathlineto{\pgfqpoint{1.635103in}{1.988504in}}%
\pgfpathlineto{\pgfqpoint{1.653179in}{1.972854in}}%
\pgfpathlineto{\pgfqpoint{1.674545in}{2.007779in}}%
\pgfpathlineto{\pgfqpoint{1.694970in}{2.076828in}}%
\pgfpathlineto{\pgfqpoint{1.711638in}{2.164390in}}%
\pgfpathlineto{\pgfqpoint{1.731124in}{2.480501in}}%
\pgfpathlineto{\pgfqpoint{1.753896in}{2.802848in}}%
\pgfpathlineto{\pgfqpoint{1.771270in}{2.830384in}}%
\pgfpathlineto{\pgfqpoint{1.790286in}{2.674118in}}%
\pgfpathlineto{\pgfqpoint{1.810243in}{2.319075in}}%
\pgfpathlineto{\pgfqpoint{1.827851in}{2.131229in}}%
\pgfpathlineto{\pgfqpoint{1.846868in}{2.031130in}}%
\pgfpathlineto{\pgfqpoint{1.866353in}{1.979369in}}%
\pgfpathlineto{\pgfqpoint{1.885370in}{1.975537in}}%
\pgfpathlineto{\pgfqpoint{1.905325in}{2.024651in}}%
\pgfpathlineto{\pgfqpoint{1.923404in}{2.106341in}}%
\pgfpathlineto{\pgfqpoint{1.945707in}{2.276669in}}%
\pgfpathlineto{\pgfqpoint{1.962610in}{2.393709in}}%
\pgfpathlineto{\pgfqpoint{1.983504in}{2.743104in}}%
\pgfpathlineto{\pgfqpoint{2.001583in}{2.820689in}}%
\pgfpathlineto{\pgfqpoint{2.019425in}{2.667313in}}%
\pgfpathlineto{\pgfqpoint{2.041023in}{2.294316in}}%
\pgfpathlineto{\pgfqpoint{2.058866in}{2.103668in}}%
\pgfpathlineto{\pgfqpoint{2.077413in}{2.017085in}}%
\pgfpathlineto{\pgfqpoint{2.096430in}{2.754403in}}%
\pgfpathlineto{\pgfqpoint{2.117090in}{2.367410in}}%
\pgfpathlineto{\pgfqpoint{2.136810in}{2.111960in}}%
\pgfpathlineto{\pgfqpoint{2.154418in}{2.016005in}}%
\pgfpathlineto{\pgfqpoint{2.176252in}{1.972361in}}%
\pgfpathlineto{\pgfqpoint{2.193860in}{1.986254in}}%
\pgfpathlineto{\pgfqpoint{2.211937in}{2.047174in}}%
\pgfpathlineto{\pgfqpoint{2.233772in}{2.168052in}}%
\pgfpathlineto{\pgfqpoint{2.251849in}{2.409865in}}%
\pgfpathlineto{\pgfqpoint{2.289413in}{2.822122in}}%
\pgfpathlineto{\pgfqpoint{2.307021in}{2.732178in}}%
\pgfpathlineto{\pgfqpoint{2.327447in}{2.389687in}}%
\pgfpathlineto{\pgfqpoint{2.348575in}{2.114257in}}%
\pgfpathlineto{\pgfqpoint{2.366889in}{2.026406in}}%
\pgfpathlineto{\pgfqpoint{2.386140in}{1.978562in}}%
\pgfpathlineto{\pgfqpoint{2.403982in}{1.973225in}}%
\pgfpathlineto{\pgfqpoint{2.423468in}{1.993870in}}%
\pgfpathlineto{\pgfqpoint{2.441780in}{2.055931in}}%
\pgfpathlineto{\pgfqpoint{2.460562in}{2.204830in}}%
\pgfpathlineto{\pgfqpoint{2.485917in}{2.589222in}}%
\pgfpathlineto{\pgfqpoint{2.501178in}{2.783238in}}%
\pgfpathlineto{\pgfqpoint{2.518316in}{2.803880in}}%
\pgfpathlineto{\pgfqpoint{2.539681in}{2.588078in}}%
\pgfpathlineto{\pgfqpoint{2.557992in}{2.275223in}}%
\pgfpathlineto{\pgfqpoint{2.578652in}{2.076856in}}%
\pgfpathlineto{\pgfqpoint{2.597903in}{2.005800in}}%
\pgfpathlineto{\pgfqpoint{2.617860in}{1.970774in}}%
\pgfpathlineto{\pgfqpoint{2.637346in}{1.978407in}}%
\pgfpathlineto{\pgfqpoint{2.655422in}{2.024695in}}%
\pgfpathlineto{\pgfqpoint{2.674439in}{2.086856in}}%
\pgfpathlineto{\pgfqpoint{2.692753in}{2.212793in}}%
\pgfpathlineto{\pgfqpoint{2.714350in}{2.492972in}}%
\pgfpathlineto{\pgfqpoint{2.732898in}{2.759943in}}%
\pgfpathlineto{\pgfqpoint{2.751915in}{2.824430in}}%
\pgfpathlineto{\pgfqpoint{2.768349in}{2.699345in}}%
\pgfpathlineto{\pgfqpoint{2.789243in}{2.337282in}}%
\pgfpathlineto{\pgfqpoint{2.806851in}{2.125224in}}%
\pgfpathlineto{\pgfqpoint{2.826102in}{2.024991in}}%
\pgfpathlineto{\pgfqpoint{2.847468in}{1.981164in}}%
\pgfpathlineto{\pgfqpoint{2.867188in}{1.978079in}}%
\pgfpathlineto{\pgfqpoint{2.885499in}{2.013713in}}%
\pgfpathlineto{\pgfqpoint{2.904281in}{2.083087in}}%
\pgfpathlineto{\pgfqpoint{2.920481in}{2.188717in}}%
\pgfpathlineto{\pgfqpoint{2.943255in}{2.472615in}}%
\pgfpathlineto{\pgfqpoint{2.959923in}{2.732129in}}%
\pgfpathlineto{\pgfqpoint{2.982226in}{2.833179in}}%
\pgfpathlineto{\pgfqpoint{3.000774in}{2.740188in}}%
\pgfpathlineto{\pgfqpoint{3.017911in}{2.528550in}}%
\pgfpathlineto{\pgfqpoint{3.039745in}{2.212363in}}%
\pgfpathlineto{\pgfqpoint{3.058762in}{2.085427in}}%
\pgfpathlineto{\pgfqpoint{3.078953in}{2.030195in}}%
\pgfpathlineto{\pgfqpoint{3.099613in}{1.977494in}}%
\pgfpathlineto{\pgfqpoint{3.115107in}{1.974240in}}%
\pgfpathlineto{\pgfqpoint{3.136472in}{2.007263in}}%
\pgfpathlineto{\pgfqpoint{3.154549in}{2.066495in}}%
\pgfpathlineto{\pgfqpoint{3.172157in}{2.175338in}}%
\pgfpathlineto{\pgfqpoint{3.194695in}{2.419338in}}%
\pgfpathlineto{\pgfqpoint{3.211365in}{2.709700in}}%
\pgfpathlineto{\pgfqpoint{3.230851in}{2.841599in}}%
\pgfpathlineto{\pgfqpoint{3.250571in}{2.184666in}}%
\pgfpathlineto{\pgfqpoint{3.268650in}{2.346963in}}%
\pgfpathlineto{\pgfqpoint{3.288604in}{2.681030in}}%
\pgfpathlineto{\pgfqpoint{3.306681in}{2.838760in}}%
\pgfpathlineto{\pgfqpoint{3.325932in}{2.846957in}}%
\pgfpathlineto{\pgfqpoint{3.348237in}{2.601999in}}%
\pgfpathlineto{\pgfqpoint{3.367252in}{2.289773in}}%
\pgfpathlineto{\pgfqpoint{3.383922in}{2.162514in}}%
\pgfpathlineto{\pgfqpoint{3.405520in}{2.031517in}}%
\pgfpathlineto{\pgfqpoint{3.425711in}{1.980464in}}%
\pgfpathlineto{\pgfqpoint{3.441442in}{1.976877in}}%
\pgfpathlineto{\pgfqpoint{3.463745in}{2.020277in}}%
\pgfpathlineto{\pgfqpoint{3.481822in}{2.105682in}}%
\pgfpathlineto{\pgfqpoint{3.501544in}{2.263503in}}%
\pgfpathlineto{\pgfqpoint{3.539575in}{2.813082in}}%
\pgfpathlineto{\pgfqpoint{3.558826in}{2.851374in}}%
\pgfpathlineto{\pgfqpoint{3.577843in}{2.689382in}}%
\pgfpathlineto{\pgfqpoint{3.598034in}{2.407388in}}%
\pgfpathlineto{\pgfqpoint{3.617991in}{2.207556in}}%
\pgfpathlineto{\pgfqpoint{3.635128in}{2.069391in}}%
\pgfpathlineto{\pgfqpoint{3.655084in}{2.007302in}}%
\pgfpathlineto{\pgfqpoint{3.673630in}{1.979127in}}%
\pgfpathlineto{\pgfqpoint{3.689126in}{1.984398in}}%
\pgfpathlineto{\pgfqpoint{3.712604in}{2.041376in}}%
\pgfpathlineto{\pgfqpoint{3.732089in}{2.095492in}}%
\pgfpathlineto{\pgfqpoint{3.750166in}{2.233841in}}%
\pgfpathlineto{\pgfqpoint{3.769888in}{2.437676in}}%
\pgfpathlineto{\pgfqpoint{3.787731in}{2.718606in}}%
\pgfpathlineto{\pgfqpoint{3.806982in}{2.855010in}}%
\pgfpathlineto{\pgfqpoint{3.826937in}{2.869987in}}%
\pgfpathlineto{\pgfqpoint{3.849242in}{2.622114in}}%
\pgfpathlineto{\pgfqpoint{3.864032in}{2.385368in}}%
\pgfpathlineto{\pgfqpoint{3.882578in}{2.177850in}}%
\pgfpathlineto{\pgfqpoint{3.905352in}{2.079390in}}%
\pgfpathlineto{\pgfqpoint{3.920846in}{2.019879in}}%
\pgfpathlineto{\pgfqpoint{3.942211in}{1.981690in}}%
\pgfpathlineto{\pgfqpoint{3.960992in}{1.992492in}}%
\pgfpathlineto{\pgfqpoint{3.980243in}{2.039300in}}%
\pgfpathlineto{\pgfqpoint{3.999260in}{2.104483in}}%
\pgfpathlineto{\pgfqpoint{4.021328in}{2.252454in}}%
\pgfpathlineto{\pgfqpoint{4.038233in}{2.475837in}}%
\pgfpathlineto{\pgfqpoint{4.076970in}{2.899236in}}%
\pgfpathlineto{\pgfqpoint{4.096221in}{2.876547in}}%
\pgfpathlineto{\pgfqpoint{4.117586in}{2.780210in}}%
\pgfpathlineto{\pgfqpoint{4.133786in}{2.533173in}}%
\pgfpathlineto{\pgfqpoint{4.152802in}{2.276947in}}%
\pgfpathlineto{\pgfqpoint{4.172054in}{2.134308in}}%
\pgfpathlineto{\pgfqpoint{4.193651in}{2.033941in}}%
\pgfpathlineto{\pgfqpoint{4.211965in}{1.999878in}}%
\pgfpathlineto{\pgfqpoint{4.230981in}{1.990280in}}%
\pgfpathlineto{\pgfqpoint{4.250702in}{2.026803in}}%
\pgfpathlineto{\pgfqpoint{4.270187in}{2.096043in}}%
\pgfpathlineto{\pgfqpoint{4.308926in}{2.347923in}}%
\pgfpathlineto{\pgfqpoint{4.328412in}{2.620392in}}%
\pgfpathlineto{\pgfqpoint{4.346958in}{2.844664in}}%
\pgfpathlineto{\pgfqpoint{4.365974in}{2.941410in}}%
\pgfpathlineto{\pgfqpoint{4.383582in}{2.870255in}}%
\pgfpathlineto{\pgfqpoint{4.402130in}{2.715905in}}%
\pgfpathlineto{\pgfqpoint{4.422085in}{2.421648in}}%
\pgfpathlineto{\pgfqpoint{4.443216in}{2.202903in}}%
\pgfpathlineto{\pgfqpoint{4.461996in}{2.089941in}}%
\pgfpathlineto{\pgfqpoint{4.477492in}{2.030770in}}%
\pgfpathlineto{\pgfqpoint{4.480309in}{2.042624in}}%
\pgfpathlineto{\pgfqpoint{4.474440in}{2.071267in}}%
\pgfpathlineto{\pgfqpoint{4.452840in}{2.290659in}}%
\pgfpathlineto{\pgfqpoint{4.434998in}{2.633812in}}%
\pgfpathlineto{\pgfqpoint{4.415981in}{2.914969in}}%
\pgfpathlineto{\pgfqpoint{4.396730in}{2.892379in}}%
\pgfpathlineto{\pgfqpoint{4.360105in}{2.224251in}}%
\pgfpathlineto{\pgfqpoint{4.340150in}{2.021225in}}%
\pgfpathlineto{\pgfqpoint{4.318785in}{1.990655in}}%
\pgfpathlineto{\pgfqpoint{4.302351in}{2.052251in}}%
\pgfpathlineto{\pgfqpoint{4.282160in}{2.208195in}}%
\pgfpathlineto{\pgfqpoint{4.263849in}{2.529623in}}%
\pgfpathlineto{\pgfqpoint{4.241311in}{2.887798in}}%
\pgfpathlineto{\pgfqpoint{4.223233in}{2.870632in}}%
\pgfpathlineto{\pgfqpoint{4.205390in}{2.537265in}}%
\pgfpathlineto{\pgfqpoint{4.185670in}{2.205556in}}%
\pgfpathlineto{\pgfqpoint{4.166888in}{2.057690in}}%
\pgfpathlineto{\pgfqpoint{4.148342in}{1.988938in}}%
\pgfpathlineto{\pgfqpoint{4.126976in}{2.007169in}}%
\pgfpathlineto{\pgfqpoint{4.108194in}{2.101037in}}%
\pgfpathlineto{\pgfqpoint{4.089648in}{2.325874in}}%
\pgfpathlineto{\pgfqpoint{4.071100in}{2.686609in}}%
\pgfpathlineto{\pgfqpoint{4.049266in}{2.891176in}}%
\pgfpathlineto{\pgfqpoint{4.029781in}{2.728657in}}%
\pgfpathlineto{\pgfqpoint{4.011235in}{2.338341in}}%
\pgfpathlineto{\pgfqpoint{3.992687in}{2.109681in}}%
\pgfpathlineto{\pgfqpoint{3.974610in}{2.013108in}}%
\pgfpathlineto{\pgfqpoint{3.956062in}{1.977188in}}%
\pgfpathlineto{\pgfqpoint{3.937280in}{2.001582in}}%
\pgfpathlineto{\pgfqpoint{3.914977in}{2.109149in}}%
\pgfpathlineto{\pgfqpoint{3.896665in}{2.348943in}}%
\pgfpathlineto{\pgfqpoint{3.878118in}{2.696893in}}%
\pgfpathlineto{\pgfqpoint{3.859335in}{2.869684in}}%
\pgfpathlineto{\pgfqpoint{3.839850in}{2.756027in}}%
\pgfpathlineto{\pgfqpoint{3.822007in}{2.389680in}}%
\pgfpathlineto{\pgfqpoint{3.800407in}{2.110911in}}%
\pgfpathlineto{\pgfqpoint{3.781861in}{2.018352in}}%
\pgfpathlineto{\pgfqpoint{3.764019in}{1.977042in}}%
\pgfpathlineto{\pgfqpoint{3.742185in}{2.019443in}}%
\pgfpathlineto{\pgfqpoint{3.725517in}{2.068470in}}%
\pgfpathlineto{\pgfqpoint{3.701568in}{2.318860in}}%
\pgfpathlineto{\pgfqpoint{3.685135in}{2.624219in}}%
\pgfpathlineto{\pgfqpoint{3.685840in}{2.783301in}}%
\pgfpathlineto{\pgfqpoint{3.667527in}{2.840490in}}%
\pgfpathlineto{\pgfqpoint{3.650624in}{2.816934in}}%
\pgfpathlineto{\pgfqpoint{3.607895in}{2.165784in}}%
\pgfpathlineto{\pgfqpoint{3.591461in}{2.093413in}}%
\pgfpathlineto{\pgfqpoint{3.572914in}{2.023371in}}%
\pgfpathlineto{\pgfqpoint{3.551550in}{1.973716in}}%
\pgfpathlineto{\pgfqpoint{3.531359in}{1.993212in}}%
\pgfpathlineto{\pgfqpoint{3.510699in}{2.073680in}}%
\pgfpathlineto{\pgfqpoint{3.494500in}{2.212869in}}%
\pgfpathlineto{\pgfqpoint{3.475952in}{2.471301in}}%
\pgfpathlineto{\pgfqpoint{3.456232in}{2.201403in}}%
\pgfpathlineto{\pgfqpoint{3.434632in}{2.581870in}}%
\pgfpathlineto{\pgfqpoint{3.418433in}{2.771164in}}%
\pgfpathlineto{\pgfqpoint{3.398479in}{2.808270in}}%
\pgfpathlineto{\pgfqpoint{3.375939in}{2.459936in}}%
\pgfpathlineto{\pgfqpoint{3.362323in}{2.199460in}}%
\pgfpathlineto{\pgfqpoint{3.340488in}{2.074257in}}%
\pgfpathlineto{\pgfqpoint{3.319125in}{1.990104in}}%
\pgfpathlineto{\pgfqpoint{3.300343in}{1.972943in}}%
\pgfpathlineto{\pgfqpoint{3.281326in}{2.011341in}}%
\pgfpathlineto{\pgfqpoint{3.262780in}{2.091303in}}%
\pgfpathlineto{\pgfqpoint{3.245172in}{2.192216in}}%
\pgfpathlineto{\pgfqpoint{3.225685in}{2.471023in}}%
\pgfpathlineto{\pgfqpoint{3.204556in}{2.803443in}}%
\pgfpathlineto{\pgfqpoint{3.185539in}{2.798553in}}%
\pgfpathlineto{\pgfqpoint{3.166993in}{2.489614in}}%
\pgfpathlineto{\pgfqpoint{3.147506in}{2.260048in}}%
\pgfpathlineto{\pgfqpoint{3.127551in}{2.091843in}}%
\pgfpathlineto{\pgfqpoint{3.108769in}{2.006380in}}%
\pgfpathlineto{\pgfqpoint{3.090457in}{1.971659in}}%
\pgfpathlineto{\pgfqpoint{3.071909in}{1.988103in}}%
\pgfpathlineto{\pgfqpoint{3.053127in}{2.048394in}}%
\pgfpathlineto{\pgfqpoint{3.031529in}{2.214586in}}%
\pgfpathlineto{\pgfqpoint{3.014390in}{2.455775in}}%
\pgfpathlineto{\pgfqpoint{2.995139in}{2.773845in}}%
\pgfpathlineto{\pgfqpoint{2.974714in}{2.818246in}}%
\pgfpathlineto{\pgfqpoint{2.956402in}{2.662513in}}%
\pgfpathlineto{\pgfqpoint{2.917429in}{2.150800in}}%
\pgfpathlineto{\pgfqpoint{2.898647in}{2.037750in}}%
\pgfpathlineto{\pgfqpoint{2.880101in}{1.983523in}}%
\pgfpathlineto{\pgfqpoint{2.860379in}{1.974343in}}%
\pgfpathlineto{\pgfqpoint{2.836198in}{2.007459in}}%
\pgfpathlineto{\pgfqpoint{2.820938in}{2.065668in}}%
\pgfpathlineto{\pgfqpoint{2.802156in}{2.199082in}}%
\pgfpathlineto{\pgfqpoint{2.783374in}{2.462665in}}%
\pgfpathlineto{\pgfqpoint{2.761776in}{2.784513in}}%
\pgfpathlineto{\pgfqpoint{2.742994in}{2.817621in}}%
\pgfpathlineto{\pgfqpoint{2.727969in}{2.676377in}}%
\pgfpathlineto{\pgfqpoint{2.687821in}{2.251211in}}%
\pgfpathlineto{\pgfqpoint{2.668804in}{2.080592in}}%
\pgfpathlineto{\pgfqpoint{2.648144in}{1.998766in}}%
\pgfpathlineto{\pgfqpoint{2.628190in}{1.973613in}}%
\pgfpathlineto{\pgfqpoint{2.609876in}{1.971924in}}%
\pgfpathlineto{\pgfqpoint{2.592739in}{2.001352in}}%
\pgfpathlineto{\pgfqpoint{2.573254in}{2.084051in}}%
\pgfpathlineto{\pgfqpoint{2.552123in}{2.230465in}}%
\pgfpathlineto{\pgfqpoint{2.532872in}{2.544462in}}%
\pgfpathlineto{\pgfqpoint{2.515734in}{2.785165in}}%
\pgfpathlineto{\pgfqpoint{2.496718in}{2.790655in}}%
\pgfpathlineto{\pgfqpoint{2.455632in}{2.242952in}}%
\pgfpathlineto{\pgfqpoint{2.437319in}{2.090605in}}%
\pgfpathlineto{\pgfqpoint{2.416425in}{2.007031in}}%
\pgfpathlineto{\pgfqpoint{2.397879in}{1.973613in}}%
\pgfpathlineto{\pgfqpoint{2.380974in}{1.976538in}}%
\pgfpathlineto{\pgfqpoint{2.360080in}{2.024745in}}%
\pgfpathlineto{\pgfqpoint{2.337071in}{2.129067in}}%
\pgfpathlineto{\pgfqpoint{2.322986in}{2.019800in}}%
\pgfpathlineto{\pgfqpoint{2.301855in}{2.104767in}}%
\pgfpathlineto{\pgfqpoint{2.280492in}{2.375902in}}%
\pgfpathlineto{\pgfqpoint{2.264527in}{2.659774in}}%
\pgfpathlineto{\pgfqpoint{2.244807in}{2.821951in}}%
\pgfpathlineto{\pgfqpoint{2.228842in}{2.739134in}}%
\pgfpathlineto{\pgfqpoint{2.186817in}{2.175542in}}%
\pgfpathlineto{\pgfqpoint{2.168505in}{2.111765in}}%
\pgfpathlineto{\pgfqpoint{2.147611in}{2.013300in}}%
\pgfpathlineto{\pgfqpoint{2.129769in}{1.975374in}}%
\pgfpathlineto{\pgfqpoint{2.111924in}{1.971839in}}%
\pgfpathlineto{\pgfqpoint{2.088683in}{2.007605in}}%
\pgfpathlineto{\pgfqpoint{2.074361in}{2.063641in}}%
\pgfpathlineto{\pgfqpoint{2.052762in}{2.225545in}}%
\pgfpathlineto{\pgfqpoint{2.033745in}{2.539610in}}%
\pgfpathlineto{\pgfqpoint{2.012147in}{2.784251in}}%
\pgfpathlineto{\pgfqpoint{1.996182in}{2.827341in}}%
\pgfpathlineto{\pgfqpoint{1.975288in}{2.671198in}}%
\pgfpathlineto{\pgfqpoint{1.955802in}{2.352245in}}%
\pgfpathlineto{\pgfqpoint{1.938898in}{2.159901in}}%
\pgfpathlineto{\pgfqpoint{1.919178in}{2.047908in}}%
\pgfpathlineto{\pgfqpoint{1.897578in}{1.985915in}}%
\pgfpathlineto{\pgfqpoint{1.875744in}{1.974008in}}%
\pgfpathlineto{\pgfqpoint{1.860015in}{2.014968in}}%
\pgfpathlineto{\pgfqpoint{1.843579in}{2.002286in}}%
\pgfpathlineto{\pgfqpoint{1.821747in}{1.974334in}}%
\pgfpathlineto{\pgfqpoint{1.803199in}{1.993912in}}%
\pgfpathlineto{\pgfqpoint{1.782539in}{2.067261in}}%
\pgfpathlineto{\pgfqpoint{1.764463in}{2.177183in}}%
\pgfpathlineto{\pgfqpoint{1.726429in}{2.686533in}}%
\pgfpathlineto{\pgfqpoint{1.703186in}{2.846075in}}%
\pgfpathlineto{\pgfqpoint{1.686518in}{2.761260in}}%
\pgfpathlineto{\pgfqpoint{1.670084in}{2.518079in}}%
\pgfpathlineto{\pgfqpoint{1.650128in}{2.269045in}}%
\pgfpathlineto{\pgfqpoint{1.629937in}{2.090883in}}%
\pgfpathlineto{\pgfqpoint{1.609982in}{2.025035in}}%
\pgfpathlineto{\pgfqpoint{1.592608in}{1.987562in}}%
\pgfpathlineto{\pgfqpoint{1.573592in}{1.976129in}}%
\pgfpathlineto{\pgfqpoint{1.534620in}{2.051119in}}%
\pgfpathlineto{\pgfqpoint{1.512317in}{2.211849in}}%
\pgfpathlineto{\pgfqpoint{1.495178in}{2.346129in}}%
\pgfpathlineto{\pgfqpoint{1.474518in}{2.669341in}}%
\pgfpathlineto{\pgfqpoint{1.456910in}{2.837059in}}%
\pgfpathlineto{\pgfqpoint{1.438599in}{2.851926in}}%
\pgfpathlineto{\pgfqpoint{1.420991in}{2.768299in}}%
\pgfpathlineto{\pgfqpoint{1.398686in}{2.606580in}}%
\pgfpathlineto{\pgfqpoint{1.379434in}{2.311192in}}%
\pgfpathlineto{\pgfqpoint{1.363472in}{2.136910in}}%
\pgfpathlineto{\pgfqpoint{1.342812in}{2.045229in}}%
\pgfpathlineto{\pgfqpoint{1.321212in}{1.998554in}}%
\pgfpathlineto{\pgfqpoint{1.303604in}{1.978047in}}%
\pgfpathlineto{\pgfqpoint{1.283647in}{2.004157in}}%
\pgfpathlineto{\pgfqpoint{1.266276in}{2.052704in}}%
\pgfpathlineto{\pgfqpoint{1.245145in}{2.159083in}}%
\pgfpathlineto{\pgfqpoint{1.226833in}{2.300576in}}%
\pgfpathlineto{\pgfqpoint{1.206877in}{2.551216in}}%
\pgfpathlineto{\pgfqpoint{1.188097in}{2.092962in}}%
\pgfpathlineto{\pgfqpoint{1.170252in}{2.013591in}}%
\pgfpathlineto{\pgfqpoint{1.148654in}{1.977565in}}%
\pgfpathlineto{\pgfqpoint{1.129872in}{2.002416in}}%
\pgfpathlineto{\pgfqpoint{1.110152in}{2.099709in}}%
\pgfpathlineto{\pgfqpoint{1.091839in}{2.225554in}}%
\pgfpathlineto{\pgfqpoint{1.071648in}{2.553462in}}%
\pgfpathlineto{\pgfqpoint{1.053571in}{2.820225in}}%
\pgfpathlineto{\pgfqpoint{1.031268in}{2.885302in}}%
\pgfpathlineto{\pgfqpoint{1.016242in}{2.740864in}}%
\pgfpathlineto{\pgfqpoint{0.994174in}{2.387709in}}%
\pgfpathlineto{\pgfqpoint{0.977269in}{2.224866in}}%
\pgfpathlineto{\pgfqpoint{0.955437in}{2.047576in}}%
\pgfpathlineto{\pgfqpoint{0.936889in}{2.001008in}}%
\pgfpathlineto{\pgfqpoint{0.919281in}{1.985302in}}%
\pgfpathlineto{\pgfqpoint{0.899090in}{2.024467in}}%
\pgfpathlineto{\pgfqpoint{0.880544in}{2.106330in}}%
\pgfpathlineto{\pgfqpoint{0.861293in}{2.306842in}}%
\pgfpathlineto{\pgfqpoint{0.842276in}{2.520701in}}%
\pgfpathlineto{\pgfqpoint{0.824199in}{2.803858in}}%
\pgfpathlineto{\pgfqpoint{0.803774in}{2.916361in}}%
\pgfpathlineto{\pgfqpoint{0.779826in}{2.737829in}}%
\pgfpathlineto{\pgfqpoint{0.763392in}{2.486183in}}%
\pgfpathlineto{\pgfqpoint{0.745315in}{2.246694in}}%
\pgfpathlineto{\pgfqpoint{0.728176in}{2.110381in}}%
\pgfpathlineto{\pgfqpoint{0.707516in}{2.024104in}}%
\pgfpathlineto{\pgfqpoint{0.685918in}{1.989775in}}%
\pgfpathlineto{\pgfqpoint{0.667839in}{2.000560in}}%
\pgfpathlineto{\pgfqpoint{0.650231in}{2.027898in}}%
\pgfpathlineto{\pgfqpoint{0.656806in}{2.005192in}}%
\pgfpathlineto{\pgfqpoint{0.675117in}{1.996972in}}%
\pgfpathlineto{\pgfqpoint{0.693196in}{2.064038in}}%
\pgfpathlineto{\pgfqpoint{0.712916in}{2.215060in}}%
\pgfpathlineto{\pgfqpoint{0.750950in}{2.883569in}}%
\pgfpathlineto{\pgfqpoint{0.771844in}{2.859083in}}%
\pgfpathlineto{\pgfqpoint{0.790626in}{2.543471in}}%
\pgfpathlineto{\pgfqpoint{0.808469in}{2.211846in}}%
\pgfpathlineto{\pgfqpoint{0.831007in}{2.041719in}}%
\pgfpathlineto{\pgfqpoint{0.847440in}{1.987445in}}%
\pgfpathlineto{\pgfqpoint{0.865519in}{2.009386in}}%
\pgfpathlineto{\pgfqpoint{0.885943in}{2.146250in}}%
\pgfpathlineto{\pgfqpoint{0.905430in}{2.327231in}}%
\pgfpathlineto{\pgfqpoint{0.922567in}{2.682411in}}%
\pgfpathlineto{\pgfqpoint{0.944871in}{2.893568in}}%
\pgfpathlineto{\pgfqpoint{0.961306in}{2.740527in}}%
\pgfpathlineto{\pgfqpoint{0.983375in}{2.316287in}}%
\pgfpathlineto{\pgfqpoint{1.001921in}{2.109004in}}%
\pgfpathlineto{\pgfqpoint{1.022112in}{2.000624in}}%
\pgfpathlineto{\pgfqpoint{1.039954in}{1.982302in}}%
\pgfpathlineto{\pgfqpoint{1.057328in}{2.034366in}}%
\pgfpathlineto{\pgfqpoint{1.079162in}{2.177152in}}%
\pgfpathlineto{\pgfqpoint{1.114613in}{2.804440in}}%
\pgfpathlineto{\pgfqpoint{1.135273in}{2.839396in}}%
\pgfpathlineto{\pgfqpoint{1.155698in}{2.498706in}}%
\pgfpathlineto{\pgfqpoint{1.174478in}{2.207932in}}%
\pgfpathlineto{\pgfqpoint{1.195138in}{2.036906in}}%
\pgfpathlineto{\pgfqpoint{1.213921in}{1.981532in}}%
\pgfpathlineto{\pgfqpoint{1.231529in}{1.991719in}}%
\pgfpathlineto{\pgfqpoint{1.251954in}{2.063063in}}%
\pgfpathlineto{\pgfqpoint{1.271205in}{2.214274in}}%
\pgfpathlineto{\pgfqpoint{1.287873in}{2.531010in}}%
\pgfpathlineto{\pgfqpoint{1.309239in}{2.825401in}}%
\pgfpathlineto{\pgfqpoint{1.331542in}{2.780585in}}%
\pgfpathlineto{\pgfqpoint{1.348210in}{2.485380in}}%
\pgfpathlineto{\pgfqpoint{1.365818in}{2.215638in}}%
\pgfpathlineto{\pgfqpoint{1.386949in}{2.045814in}}%
\pgfpathlineto{\pgfqpoint{1.405729in}{1.987301in}}%
\pgfpathlineto{\pgfqpoint{1.425920in}{1.979948in}}%
\pgfpathlineto{\pgfqpoint{1.443528in}{2.028412in}}%
\pgfpathlineto{\pgfqpoint{1.462311in}{2.130751in}}%
\pgfpathlineto{\pgfqpoint{1.482031in}{2.408009in}}%
\pgfpathlineto{\pgfqpoint{1.500342in}{2.737310in}}%
\pgfpathlineto{\pgfqpoint{1.521707in}{2.843776in}}%
\pgfpathlineto{\pgfqpoint{1.539784in}{2.752252in}}%
\pgfpathlineto{\pgfqpoint{1.556689in}{2.447606in}}%
\pgfpathlineto{\pgfqpoint{1.579461in}{2.156154in}}%
\pgfpathlineto{\pgfqpoint{1.596600in}{2.036226in}}%
\pgfpathlineto{\pgfqpoint{1.617260in}{1.990837in}}%
\pgfpathlineto{\pgfqpoint{1.635337in}{1.973166in}}%
\pgfpathlineto{\pgfqpoint{1.658814in}{2.025734in}}%
\pgfpathlineto{\pgfqpoint{1.673605in}{2.095639in}}%
\pgfpathlineto{\pgfqpoint{1.693796in}{2.284436in}}%
\pgfpathlineto{\pgfqpoint{1.712342in}{2.597726in}}%
\pgfpathlineto{\pgfqpoint{1.732064in}{2.815189in}}%
\pgfpathlineto{\pgfqpoint{1.751081in}{2.773358in}}%
\pgfpathlineto{\pgfqpoint{1.769627in}{2.528585in}}%
\pgfpathlineto{\pgfqpoint{1.791695in}{2.284255in}}%
\pgfpathlineto{\pgfqpoint{1.809303in}{2.119837in}}%
\pgfpathlineto{\pgfqpoint{1.827146in}{2.039505in}}%
\pgfpathlineto{\pgfqpoint{1.849214in}{1.980828in}}%
\pgfpathlineto{\pgfqpoint{1.867528in}{1.974485in}}%
\pgfpathlineto{\pgfqpoint{1.887717in}{2.005081in}}%
\pgfpathlineto{\pgfqpoint{1.903213in}{2.068911in}}%
\pgfpathlineto{\pgfqpoint{1.923404in}{2.211346in}}%
\pgfpathlineto{\pgfqpoint{1.944298in}{2.486484in}}%
\pgfpathlineto{\pgfqpoint{1.965427in}{2.052486in}}%
\pgfpathlineto{\pgfqpoint{1.981626in}{2.182353in}}%
\pgfpathlineto{\pgfqpoint{2.001817in}{2.507287in}}%
\pgfpathlineto{\pgfqpoint{2.019894in}{2.768600in}}%
\pgfpathlineto{\pgfqpoint{2.036797in}{2.818290in}}%
\pgfpathlineto{\pgfqpoint{2.058631in}{2.557296in}}%
\pgfpathlineto{\pgfqpoint{2.076474in}{2.264013in}}%
\pgfpathlineto{\pgfqpoint{2.096899in}{2.063133in}}%
\pgfpathlineto{\pgfqpoint{2.115212in}{1.998297in}}%
\pgfpathlineto{\pgfqpoint{2.137047in}{1.970370in}}%
\pgfpathlineto{\pgfqpoint{2.153949in}{1.992676in}}%
\pgfpathlineto{\pgfqpoint{2.171557in}{2.055789in}}%
\pgfpathlineto{\pgfqpoint{2.193157in}{2.200735in}}%
\pgfpathlineto{\pgfqpoint{2.229076in}{2.733373in}}%
\pgfpathlineto{\pgfqpoint{2.250911in}{2.814332in}}%
\pgfpathlineto{\pgfqpoint{2.271805in}{2.588768in}}%
\pgfpathlineto{\pgfqpoint{2.290587in}{2.266309in}}%
\pgfpathlineto{\pgfqpoint{2.307021in}{2.096069in}}%
\pgfpathlineto{\pgfqpoint{2.328384in}{2.007479in}}%
\pgfpathlineto{\pgfqpoint{2.346698in}{1.974718in}}%
\pgfpathlineto{\pgfqpoint{2.364306in}{1.976489in}}%
\pgfpathlineto{\pgfqpoint{2.386140in}{2.026095in}}%
\pgfpathlineto{\pgfqpoint{2.405391in}{2.079582in}}%
\pgfpathlineto{\pgfqpoint{2.423234in}{2.054730in}}%
\pgfpathlineto{\pgfqpoint{2.443659in}{2.157737in}}%
\pgfpathlineto{\pgfqpoint{2.462440in}{2.347383in}}%
\pgfpathlineto{\pgfqpoint{2.481691in}{2.693493in}}%
\pgfpathlineto{\pgfqpoint{2.502822in}{2.818396in}}%
\pgfpathlineto{\pgfqpoint{2.521133in}{2.664971in}}%
\pgfpathlineto{\pgfqpoint{2.538741in}{2.344366in}}%
\pgfpathlineto{\pgfqpoint{2.557054in}{2.134287in}}%
\pgfpathlineto{\pgfqpoint{2.577714in}{2.022581in}}%
\pgfpathlineto{\pgfqpoint{2.596260in}{1.977630in}}%
\pgfpathlineto{\pgfqpoint{2.615982in}{1.976019in}}%
\pgfpathlineto{\pgfqpoint{2.634997in}{2.020979in}}%
\pgfpathlineto{\pgfqpoint{2.653545in}{2.097618in}}%
\pgfpathlineto{\pgfqpoint{2.674205in}{2.271984in}}%
\pgfpathlineto{\pgfqpoint{2.695099in}{2.572380in}}%
\pgfpathlineto{\pgfqpoint{2.713178in}{2.708577in}}%
\pgfpathlineto{\pgfqpoint{2.730550in}{2.804563in}}%
\pgfpathlineto{\pgfqpoint{2.750975in}{2.759720in}}%
\pgfpathlineto{\pgfqpoint{2.769523in}{2.597007in}}%
\pgfpathlineto{\pgfqpoint{2.787834in}{2.275935in}}%
\pgfpathlineto{\pgfqpoint{2.810138in}{2.073952in}}%
\pgfpathlineto{\pgfqpoint{2.827746in}{2.024669in}}%
\pgfpathlineto{\pgfqpoint{2.850988in}{1.972785in}}%
\pgfpathlineto{\pgfqpoint{2.866248in}{1.980099in}}%
\pgfpathlineto{\pgfqpoint{2.884561in}{2.002194in}}%
\pgfpathlineto{\pgfqpoint{2.905456in}{2.087059in}}%
\pgfpathlineto{\pgfqpoint{2.924941in}{2.224091in}}%
\pgfpathlineto{\pgfqpoint{2.941846in}{2.493798in}}%
\pgfpathlineto{\pgfqpoint{2.962740in}{2.806868in}}%
\pgfpathlineto{\pgfqpoint{2.981992in}{2.819207in}}%
\pgfpathlineto{\pgfqpoint{2.998894in}{2.637481in}}%
\pgfpathlineto{\pgfqpoint{3.021434in}{2.289413in}}%
\pgfpathlineto{\pgfqpoint{3.038807in}{2.121512in}}%
\pgfpathlineto{\pgfqpoint{3.059467in}{2.028295in}}%
\pgfpathlineto{\pgfqpoint{3.077310in}{1.985493in}}%
\pgfpathlineto{\pgfqpoint{3.096561in}{1.974952in}}%
\pgfpathlineto{\pgfqpoint{3.118395in}{2.013893in}}%
\pgfpathlineto{\pgfqpoint{3.135298in}{2.077525in}}%
\pgfpathlineto{\pgfqpoint{3.154549in}{2.193801in}}%
\pgfpathlineto{\pgfqpoint{3.174269in}{2.410773in}}%
\pgfpathlineto{\pgfqpoint{3.191408in}{2.616901in}}%
\pgfpathlineto{\pgfqpoint{3.213243in}{2.804644in}}%
\pgfpathlineto{\pgfqpoint{3.230145in}{2.758818in}}%
\pgfpathlineto{\pgfqpoint{3.249162in}{2.841432in}}%
\pgfpathlineto{\pgfqpoint{3.266536in}{2.739494in}}%
\pgfpathlineto{\pgfqpoint{3.289073in}{2.392903in}}%
\pgfpathlineto{\pgfqpoint{3.309030in}{2.183944in}}%
\pgfpathlineto{\pgfqpoint{3.327107in}{2.081190in}}%
\pgfpathlineto{\pgfqpoint{3.345654in}{2.017536in}}%
\pgfpathlineto{\pgfqpoint{3.364437in}{1.975869in}}%
\pgfpathlineto{\pgfqpoint{3.382514in}{1.984599in}}%
\pgfpathlineto{\pgfqpoint{3.404582in}{2.013020in}}%
\pgfpathlineto{\pgfqpoint{3.423128in}{2.088307in}}%
\pgfpathlineto{\pgfqpoint{3.445197in}{2.251032in}}%
\pgfpathlineto{\pgfqpoint{3.460693in}{2.441261in}}%
\pgfpathlineto{\pgfqpoint{3.482527in}{2.728085in}}%
\pgfpathlineto{\pgfqpoint{3.498255in}{2.859756in}}%
\pgfpathlineto{\pgfqpoint{3.519152in}{2.832155in}}%
\pgfpathlineto{\pgfqpoint{3.538403in}{2.663760in}}%
\pgfpathlineto{\pgfqpoint{3.557183in}{2.406845in}}%
\pgfpathlineto{\pgfqpoint{3.576905in}{2.167362in}}%
\pgfpathlineto{\pgfqpoint{3.594748in}{2.675451in}}%
\pgfpathlineto{\pgfqpoint{3.615408in}{2.864836in}}%
\pgfpathlineto{\pgfqpoint{3.633719in}{2.819154in}}%
\pgfpathlineto{\pgfqpoint{3.673161in}{2.235548in}}%
\pgfpathlineto{\pgfqpoint{3.691709in}{2.081443in}}%
\pgfpathlineto{\pgfqpoint{3.710960in}{2.006728in}}%
\pgfpathlineto{\pgfqpoint{3.733498in}{1.978371in}}%
\pgfpathlineto{\pgfqpoint{3.750401in}{2.003762in}}%
\pgfpathlineto{\pgfqpoint{3.768480in}{2.064581in}}%
\pgfpathlineto{\pgfqpoint{3.787731in}{2.180269in}}%
\pgfpathlineto{\pgfqpoint{3.809329in}{2.411959in}}%
\pgfpathlineto{\pgfqpoint{3.825999in}{2.674143in}}%
\pgfpathlineto{\pgfqpoint{3.851354in}{2.886767in}}%
\pgfpathlineto{\pgfqpoint{3.867319in}{2.826277in}}%
\pgfpathlineto{\pgfqpoint{3.886335in}{2.584361in}}%
\pgfpathlineto{\pgfqpoint{3.901829in}{2.351762in}}%
\pgfpathlineto{\pgfqpoint{3.922489in}{2.151666in}}%
\pgfpathlineto{\pgfqpoint{3.941272in}{2.051856in}}%
\pgfpathlineto{\pgfqpoint{3.962871in}{1.996697in}}%
\pgfpathlineto{\pgfqpoint{3.982357in}{1.981403in}}%
\pgfpathlineto{\pgfqpoint{4.000903in}{2.008721in}}%
\pgfpathlineto{\pgfqpoint{4.019451in}{2.071600in}}%
\pgfpathlineto{\pgfqpoint{4.038702in}{2.181954in}}%
\pgfpathlineto{\pgfqpoint{4.057250in}{2.350550in}}%
\pgfpathlineto{\pgfqpoint{4.076030in}{2.564405in}}%
\pgfpathlineto{\pgfqpoint{4.096456in}{2.838547in}}%
\pgfpathlineto{\pgfqpoint{4.115003in}{2.908719in}}%
\pgfpathlineto{\pgfqpoint{4.136367in}{2.737380in}}%
\pgfpathlineto{\pgfqpoint{4.155618in}{2.626944in}}%
\pgfpathlineto{\pgfqpoint{4.177218in}{2.328790in}}%
\pgfpathlineto{\pgfqpoint{4.193182in}{2.170010in}}%
\pgfpathlineto{\pgfqpoint{4.209147in}{2.070808in}}%
\pgfpathlineto{\pgfqpoint{4.230747in}{2.012861in}}%
\pgfpathlineto{\pgfqpoint{4.249762in}{1.987449in}}%
\pgfpathlineto{\pgfqpoint{4.269013in}{2.016473in}}%
\pgfpathlineto{\pgfqpoint{4.290144in}{2.076098in}}%
\pgfpathlineto{\pgfqpoint{4.306343in}{2.163247in}}%
\pgfpathlineto{\pgfqpoint{4.330289in}{2.417794in}}%
\pgfpathlineto{\pgfqpoint{4.346254in}{2.662377in}}%
\pgfpathlineto{\pgfqpoint{4.366680in}{2.892775in}}%
\pgfpathlineto{\pgfqpoint{4.385226in}{2.942701in}}%
\pgfpathlineto{\pgfqpoint{4.404008in}{2.902506in}}%
\pgfpathlineto{\pgfqpoint{4.422319in}{2.704652in}}%
\pgfpathlineto{\pgfqpoint{4.444624in}{2.414671in}}%
\pgfpathlineto{\pgfqpoint{4.461293in}{2.255625in}}%
\pgfpathlineto{\pgfqpoint{4.479840in}{2.114526in}}%
\pgfpathlineto{\pgfqpoint{4.473971in}{2.164886in}}%
\pgfpathlineto{\pgfqpoint{4.454483in}{2.447379in}}%
\pgfpathlineto{\pgfqpoint{4.434529in}{2.824744in}}%
\pgfpathlineto{\pgfqpoint{4.417390in}{2.942807in}}%
\pgfpathlineto{\pgfqpoint{4.398844in}{2.701836in}}%
\pgfpathlineto{\pgfqpoint{4.377713in}{2.281115in}}%
\pgfpathlineto{\pgfqpoint{4.360105in}{2.100198in}}%
\pgfpathlineto{\pgfqpoint{4.338976in}{1.999276in}}%
\pgfpathlineto{\pgfqpoint{4.320899in}{1.999017in}}%
\pgfpathlineto{\pgfqpoint{4.302117in}{2.081213in}}%
\pgfpathlineto{\pgfqpoint{4.280988in}{2.317690in}}%
\pgfpathlineto{\pgfqpoint{4.261500in}{2.674859in}}%
\pgfpathlineto{\pgfqpoint{4.243189in}{2.907850in}}%
\pgfpathlineto{\pgfqpoint{4.225347in}{2.814338in}}%
\pgfpathlineto{\pgfqpoint{4.203043in}{2.370884in}}%
\pgfpathlineto{\pgfqpoint{4.186844in}{2.170831in}}%
\pgfpathlineto{\pgfqpoint{4.168296in}{2.041936in}}%
\pgfpathlineto{\pgfqpoint{4.147167in}{1.980881in}}%
\pgfpathlineto{\pgfqpoint{4.128854in}{2.020037in}}%
\pgfpathlineto{\pgfqpoint{4.109134in}{2.138065in}}%
\pgfpathlineto{\pgfqpoint{4.090117in}{2.390063in}}%
\pgfpathlineto{\pgfqpoint{4.068754in}{2.807960in}}%
\pgfpathlineto{\pgfqpoint{4.050441in}{2.882823in}}%
\pgfpathlineto{\pgfqpoint{4.031658in}{2.635713in}}%
\pgfpathlineto{\pgfqpoint{4.014756in}{2.284261in}}%
\pgfpathlineto{\pgfqpoint{3.992687in}{2.077867in}}%
\pgfpathlineto{\pgfqpoint{3.974610in}{1.992337in}}%
\pgfpathlineto{\pgfqpoint{3.954185in}{1.984121in}}%
\pgfpathlineto{\pgfqpoint{3.938220in}{2.031403in}}%
\pgfpathlineto{\pgfqpoint{3.916385in}{2.201877in}}%
\pgfpathlineto{\pgfqpoint{3.898074in}{2.486429in}}%
\pgfpathlineto{\pgfqpoint{3.878823in}{2.811413in}}%
\pgfpathlineto{\pgfqpoint{3.860041in}{2.844809in}}%
\pgfpathlineto{\pgfqpoint{3.819660in}{2.194500in}}%
\pgfpathlineto{\pgfqpoint{3.802052in}{2.053196in}}%
\pgfpathlineto{\pgfqpoint{3.782565in}{1.988026in}}%
\pgfpathlineto{\pgfqpoint{3.761905in}{1.983412in}}%
\pgfpathlineto{\pgfqpoint{3.741716in}{2.054931in}}%
\pgfpathlineto{\pgfqpoint{3.724577in}{2.199704in}}%
\pgfpathlineto{\pgfqpoint{3.706029in}{2.510191in}}%
\pgfpathlineto{\pgfqpoint{3.686074in}{2.789599in}}%
\pgfpathlineto{\pgfqpoint{3.667058in}{2.829552in}}%
\pgfpathlineto{\pgfqpoint{3.647101in}{2.594954in}}%
\pgfpathlineto{\pgfqpoint{3.628321in}{2.291885in}}%
\pgfpathlineto{\pgfqpoint{3.608599in}{2.092083in}}%
\pgfpathlineto{\pgfqpoint{3.593339in}{2.025170in}}%
\pgfpathlineto{\pgfqpoint{3.571270in}{1.984677in}}%
\pgfpathlineto{\pgfqpoint{3.552019in}{1.981108in}}%
\pgfpathlineto{\pgfqpoint{3.533003in}{2.025216in}}%
\pgfpathlineto{\pgfqpoint{3.512343in}{2.181693in}}%
\pgfpathlineto{\pgfqpoint{3.496612in}{2.354463in}}%
\pgfpathlineto{\pgfqpoint{3.475718in}{2.734890in}}%
\pgfpathlineto{\pgfqpoint{3.458344in}{2.843208in}}%
\pgfpathlineto{\pgfqpoint{3.438155in}{2.767560in}}%
\pgfpathlineto{\pgfqpoint{3.415850in}{2.396336in}}%
\pgfpathlineto{\pgfqpoint{3.398713in}{2.155086in}}%
\pgfpathlineto{\pgfqpoint{3.378522in}{2.037143in}}%
\pgfpathlineto{\pgfqpoint{3.359505in}{1.983433in}}%
\pgfpathlineto{\pgfqpoint{3.337907in}{1.978423in}}%
\pgfpathlineto{\pgfqpoint{3.322646in}{2.014682in}}%
\pgfpathlineto{\pgfqpoint{3.300577in}{2.075803in}}%
\pgfpathlineto{\pgfqpoint{3.283204in}{2.245210in}}%
\pgfpathlineto{\pgfqpoint{3.263484in}{2.504697in}}%
\pgfpathlineto{\pgfqpoint{3.244232in}{2.794338in}}%
\pgfpathlineto{\pgfqpoint{3.226859in}{2.815982in}}%
\pgfpathlineto{\pgfqpoint{3.205025in}{2.521253in}}%
\pgfpathlineto{\pgfqpoint{3.186244in}{2.231368in}}%
\pgfpathlineto{\pgfqpoint{3.167228in}{2.113328in}}%
\pgfpathlineto{\pgfqpoint{3.145159in}{2.007425in}}%
\pgfpathlineto{\pgfqpoint{3.131072in}{1.976301in}}%
\pgfpathlineto{\pgfqpoint{3.109003in}{1.984108in}}%
\pgfpathlineto{\pgfqpoint{3.090457in}{2.038927in}}%
\pgfpathlineto{\pgfqpoint{3.071675in}{2.157502in}}%
\pgfpathlineto{\pgfqpoint{3.052893in}{2.401583in}}%
\pgfpathlineto{\pgfqpoint{3.034347in}{2.719984in}}%
\pgfpathlineto{\pgfqpoint{3.014859in}{2.144244in}}%
\pgfpathlineto{\pgfqpoint{2.994199in}{2.455638in}}%
\pgfpathlineto{\pgfqpoint{2.975182in}{2.711049in}}%
\pgfpathlineto{\pgfqpoint{2.956871in}{2.828816in}}%
\pgfpathlineto{\pgfqpoint{2.936446in}{2.646071in}}%
\pgfpathlineto{\pgfqpoint{2.916960in}{2.320087in}}%
\pgfpathlineto{\pgfqpoint{2.898647in}{2.122484in}}%
\pgfpathlineto{\pgfqpoint{2.876343in}{2.009911in}}%
\pgfpathlineto{\pgfqpoint{2.860615in}{1.975802in}}%
\pgfpathlineto{\pgfqpoint{2.839719in}{1.984173in}}%
\pgfpathlineto{\pgfqpoint{2.817181in}{2.044402in}}%
\pgfpathlineto{\pgfqpoint{2.802391in}{2.129521in}}%
\pgfpathlineto{\pgfqpoint{2.783374in}{2.319107in}}%
\pgfpathlineto{\pgfqpoint{2.761776in}{2.685474in}}%
\pgfpathlineto{\pgfqpoint{2.746515in}{2.817530in}}%
\pgfpathlineto{\pgfqpoint{2.725855in}{2.743189in}}%
\pgfpathlineto{\pgfqpoint{2.705900in}{2.427476in}}%
\pgfpathlineto{\pgfqpoint{2.688056in}{2.180378in}}%
\pgfpathlineto{\pgfqpoint{2.669041in}{2.059373in}}%
\pgfpathlineto{\pgfqpoint{2.647676in}{1.986927in}}%
\pgfpathlineto{\pgfqpoint{2.631711in}{1.970274in}}%
\pgfpathlineto{\pgfqpoint{2.610347in}{1.978439in}}%
\pgfpathlineto{\pgfqpoint{2.591565in}{2.023142in}}%
\pgfpathlineto{\pgfqpoint{2.569731in}{2.132774in}}%
\pgfpathlineto{\pgfqpoint{2.551888in}{2.260169in}}%
\pgfpathlineto{\pgfqpoint{2.532872in}{2.586624in}}%
\pgfpathlineto{\pgfqpoint{2.513855in}{2.800222in}}%
\pgfpathlineto{\pgfqpoint{2.495543in}{2.767075in}}%
\pgfpathlineto{\pgfqpoint{2.474178in}{2.504748in}}%
\pgfpathlineto{\pgfqpoint{2.456101in}{2.250579in}}%
\pgfpathlineto{\pgfqpoint{2.437319in}{2.091042in}}%
\pgfpathlineto{\pgfqpoint{2.419008in}{2.009098in}}%
\pgfpathlineto{\pgfqpoint{2.397173in}{1.971415in}}%
\pgfpathlineto{\pgfqpoint{2.379096in}{1.980091in}}%
\pgfpathlineto{\pgfqpoint{2.360783in}{2.020049in}}%
\pgfpathlineto{\pgfqpoint{2.339185in}{2.138056in}}%
\pgfpathlineto{\pgfqpoint{2.317586in}{2.367698in}}%
\pgfpathlineto{\pgfqpoint{2.301621in}{2.591443in}}%
\pgfpathlineto{\pgfqpoint{2.283778in}{2.775101in}}%
\pgfpathlineto{\pgfqpoint{2.264527in}{2.814906in}}%
\pgfpathlineto{\pgfqpoint{2.225085in}{2.456907in}}%
\pgfpathlineto{\pgfqpoint{2.205130in}{2.248216in}}%
\pgfpathlineto{\pgfqpoint{2.187051in}{2.103088in}}%
\pgfpathlineto{\pgfqpoint{2.169209in}{2.020885in}}%
\pgfpathlineto{\pgfqpoint{2.147140in}{1.974024in}}%
\pgfpathlineto{\pgfqpoint{2.129532in}{1.985778in}}%
\pgfpathlineto{\pgfqpoint{2.110046in}{2.023406in}}%
\pgfpathlineto{\pgfqpoint{2.092204in}{2.094623in}}%
\pgfpathlineto{\pgfqpoint{2.070135in}{2.302287in}}%
\pgfpathlineto{\pgfqpoint{2.052293in}{2.595187in}}%
\pgfpathlineto{\pgfqpoint{2.033276in}{2.673128in}}%
\pgfpathlineto{\pgfqpoint{2.015434in}{2.006993in}}%
\pgfpathlineto{\pgfqpoint{1.996417in}{2.094331in}}%
\pgfpathlineto{\pgfqpoint{1.975757in}{2.286318in}}%
\pgfpathlineto{\pgfqpoint{1.954863in}{2.669069in}}%
\pgfpathlineto{\pgfqpoint{1.937020in}{2.823848in}}%
\pgfpathlineto{\pgfqpoint{1.916126in}{2.746438in}}%
\pgfpathlineto{\pgfqpoint{1.898518in}{2.541215in}}%
\pgfpathlineto{\pgfqpoint{1.875509in}{2.205758in}}%
\pgfpathlineto{\pgfqpoint{1.861188in}{2.081471in}}%
\pgfpathlineto{\pgfqpoint{1.841467in}{2.122953in}}%
\pgfpathlineto{\pgfqpoint{1.823625in}{2.025079in}}%
\pgfpathlineto{\pgfqpoint{1.802494in}{1.983298in}}%
\pgfpathlineto{\pgfqpoint{1.784183in}{1.977491in}}%
\pgfpathlineto{\pgfqpoint{1.764228in}{2.021373in}}%
\pgfpathlineto{\pgfqpoint{1.743803in}{2.123420in}}%
\pgfpathlineto{\pgfqpoint{1.725724in}{2.273273in}}%
\pgfpathlineto{\pgfqpoint{1.706707in}{2.586506in}}%
\pgfpathlineto{\pgfqpoint{1.688630in}{2.780691in}}%
\pgfpathlineto{\pgfqpoint{1.669848in}{2.847786in}}%
\pgfpathlineto{\pgfqpoint{1.648015in}{2.683946in}}%
\pgfpathlineto{\pgfqpoint{1.630407in}{2.378009in}}%
\pgfpathlineto{\pgfqpoint{1.609042in}{2.148337in}}%
\pgfpathlineto{\pgfqpoint{1.591905in}{2.054059in}}%
\pgfpathlineto{\pgfqpoint{1.571245in}{1.991772in}}%
\pgfpathlineto{\pgfqpoint{1.554106in}{1.975360in}}%
\pgfpathlineto{\pgfqpoint{1.531098in}{2.004633in}}%
\pgfpathlineto{\pgfqpoint{1.515133in}{2.053011in}}%
\pgfpathlineto{\pgfqpoint{1.494709in}{2.174896in}}%
\pgfpathlineto{\pgfqpoint{1.474284in}{2.444110in}}%
\pgfpathlineto{\pgfqpoint{1.456910in}{2.576245in}}%
\pgfpathlineto{\pgfqpoint{1.436954in}{2.815012in}}%
\pgfpathlineto{\pgfqpoint{1.418642in}{2.860957in}}%
\pgfpathlineto{\pgfqpoint{1.397982in}{2.735026in}}%
\pgfpathlineto{\pgfqpoint{1.380843in}{2.558005in}}%
\pgfpathlineto{\pgfqpoint{1.360654in}{2.275890in}}%
\pgfpathlineto{\pgfqpoint{1.336942in}{2.092482in}}%
\pgfpathlineto{\pgfqpoint{1.322621in}{2.031738in}}%
\pgfpathlineto{\pgfqpoint{1.301961in}{1.984764in}}%
\pgfpathlineto{\pgfqpoint{1.284822in}{1.980888in}}%
\pgfpathlineto{\pgfqpoint{1.264631in}{2.009729in}}%
\pgfpathlineto{\pgfqpoint{1.246554in}{2.050813in}}%
\pgfpathlineto{\pgfqpoint{1.225659in}{2.171386in}}%
\pgfpathlineto{\pgfqpoint{1.207348in}{2.439990in}}%
\pgfpathlineto{\pgfqpoint{1.187157in}{2.696912in}}%
\pgfpathlineto{\pgfqpoint{1.167435in}{2.868019in}}%
\pgfpathlineto{\pgfqpoint{1.149829in}{2.882425in}}%
\pgfpathlineto{\pgfqpoint{1.131515in}{2.744082in}}%
\pgfpathlineto{\pgfqpoint{1.111324in}{2.464563in}}%
\pgfpathlineto{\pgfqpoint{1.091370in}{2.212692in}}%
\pgfpathlineto{\pgfqpoint{1.073527in}{2.122672in}}%
\pgfpathlineto{\pgfqpoint{1.053336in}{2.048140in}}%
\pgfpathlineto{\pgfqpoint{1.032442in}{1.992204in}}%
\pgfpathlineto{\pgfqpoint{1.015068in}{1.983929in}}%
\pgfpathlineto{\pgfqpoint{0.996991in}{2.017685in}}%
\pgfpathlineto{\pgfqpoint{0.976566in}{2.037960in}}%
\pgfpathlineto{\pgfqpoint{0.955671in}{2.138939in}}%
\pgfpathlineto{\pgfqpoint{0.938298in}{2.307308in}}%
\pgfpathlineto{\pgfqpoint{0.902142in}{2.834281in}}%
\pgfpathlineto{\pgfqpoint{0.881953in}{2.909879in}}%
\pgfpathlineto{\pgfqpoint{0.861059in}{2.844980in}}%
\pgfpathlineto{\pgfqpoint{0.839224in}{2.693728in}}%
\pgfpathlineto{\pgfqpoint{0.822791in}{2.414325in}}%
\pgfpathlineto{\pgfqpoint{0.801894in}{2.188173in}}%
\pgfpathlineto{\pgfqpoint{0.784286in}{2.087331in}}%
\pgfpathlineto{\pgfqpoint{0.762688in}{2.018805in}}%
\pgfpathlineto{\pgfqpoint{0.745549in}{1.988673in}}%
\pgfpathlineto{\pgfqpoint{0.727707in}{2.074173in}}%
\pgfpathlineto{\pgfqpoint{0.706812in}{2.015428in}}%
\pgfpathlineto{\pgfqpoint{0.687561in}{1.988018in}}%
\pgfpathlineto{\pgfqpoint{0.669719in}{2.020456in}}%
\pgfpathlineto{\pgfqpoint{0.648588in}{2.084186in}}%
\pgfpathlineto{\pgfqpoint{0.649528in}{2.088671in}}%
\pgfpathlineto{\pgfqpoint{0.657040in}{2.035542in}}%
\pgfpathlineto{\pgfqpoint{0.674414in}{1.989263in}}%
\pgfpathlineto{\pgfqpoint{0.696248in}{2.043706in}}%
\pgfpathlineto{\pgfqpoint{0.712213in}{2.144068in}}%
\pgfpathlineto{\pgfqpoint{0.733811in}{2.421919in}}%
\pgfpathlineto{\pgfqpoint{0.752124in}{2.798305in}}%
\pgfpathlineto{\pgfqpoint{0.771375in}{2.914428in}}%
\pgfpathlineto{\pgfqpoint{0.791564in}{2.671830in}}%
\pgfpathlineto{\pgfqpoint{0.808938in}{2.325422in}}%
\pgfpathlineto{\pgfqpoint{0.827486in}{2.107150in}}%
\pgfpathlineto{\pgfqpoint{0.849554in}{2.000504in}}%
\pgfpathlineto{\pgfqpoint{0.867397in}{1.989403in}}%
\pgfpathlineto{\pgfqpoint{0.885708in}{2.052444in}}%
\pgfpathlineto{\pgfqpoint{0.904491in}{2.195070in}}%
\pgfpathlineto{\pgfqpoint{0.925385in}{2.532556in}}%
\pgfpathlineto{\pgfqpoint{0.943698in}{2.853016in}}%
\pgfpathlineto{\pgfqpoint{0.961306in}{2.860760in}}%
\pgfpathlineto{\pgfqpoint{0.979383in}{2.550413in}}%
\pgfpathlineto{\pgfqpoint{1.000983in}{2.219161in}}%
\pgfpathlineto{\pgfqpoint{1.021407in}{2.042725in}}%
\pgfpathlineto{\pgfqpoint{1.039954in}{1.983970in}}%
\pgfpathlineto{\pgfqpoint{1.057797in}{1.997718in}}%
\pgfpathlineto{\pgfqpoint{1.079631in}{2.088347in}}%
\pgfpathlineto{\pgfqpoint{1.096534in}{2.254535in}}%
\pgfpathlineto{\pgfqpoint{1.117194in}{2.621046in}}%
\pgfpathlineto{\pgfqpoint{1.135273in}{2.861254in}}%
\pgfpathlineto{\pgfqpoint{1.155698in}{2.751343in}}%
\pgfpathlineto{\pgfqpoint{1.174949in}{2.385873in}}%
\pgfpathlineto{\pgfqpoint{1.194669in}{2.124360in}}%
\pgfpathlineto{\pgfqpoint{1.212981in}{2.019342in}}%
\pgfpathlineto{\pgfqpoint{1.231529in}{1.976580in}}%
\pgfpathlineto{\pgfqpoint{1.251014in}{2.009014in}}%
\pgfpathlineto{\pgfqpoint{1.270500in}{2.088784in}}%
\pgfpathlineto{\pgfqpoint{1.289753in}{2.260542in}}%
\pgfpathlineto{\pgfqpoint{1.308299in}{2.587399in}}%
\pgfpathlineto{\pgfqpoint{1.328724in}{2.843121in}}%
\pgfpathlineto{\pgfqpoint{1.346801in}{2.805648in}}%
\pgfpathlineto{\pgfqpoint{1.367932in}{2.437236in}}%
\pgfpathlineto{\pgfqpoint{1.385538in}{2.173490in}}%
\pgfpathlineto{\pgfqpoint{1.404555in}{2.044625in}}%
\pgfpathlineto{\pgfqpoint{1.424980in}{1.982124in}}%
\pgfpathlineto{\pgfqpoint{1.445875in}{1.985698in}}%
\pgfpathlineto{\pgfqpoint{1.480856in}{2.086452in}}%
\pgfpathlineto{\pgfqpoint{1.501751in}{2.307272in}}%
\pgfpathlineto{\pgfqpoint{1.520064in}{2.650335in}}%
\pgfpathlineto{\pgfqpoint{1.542133in}{2.842349in}}%
\pgfpathlineto{\pgfqpoint{1.559741in}{2.737487in}}%
\pgfpathlineto{\pgfqpoint{1.596835in}{2.217628in}}%
\pgfpathlineto{\pgfqpoint{1.615146in}{2.064622in}}%
\pgfpathlineto{\pgfqpoint{1.636511in}{2.791804in}}%
\pgfpathlineto{\pgfqpoint{1.672902in}{2.177091in}}%
\pgfpathlineto{\pgfqpoint{1.690979in}{2.029625in}}%
\pgfpathlineto{\pgfqpoint{1.712576in}{1.980267in}}%
\pgfpathlineto{\pgfqpoint{1.732298in}{1.982163in}}%
\pgfpathlineto{\pgfqpoint{1.753193in}{2.047801in}}%
\pgfpathlineto{\pgfqpoint{1.771741in}{2.176794in}}%
\pgfpathlineto{\pgfqpoint{1.810946in}{2.741944in}}%
\pgfpathlineto{\pgfqpoint{1.828789in}{2.824147in}}%
\pgfpathlineto{\pgfqpoint{1.846631in}{2.670439in}}%
\pgfpathlineto{\pgfqpoint{1.867528in}{2.324740in}}%
\pgfpathlineto{\pgfqpoint{1.885370in}{2.117700in}}%
\pgfpathlineto{\pgfqpoint{1.903447in}{2.020943in}}%
\pgfpathlineto{\pgfqpoint{1.925047in}{1.971756in}}%
\pgfpathlineto{\pgfqpoint{1.942655in}{1.980852in}}%
\pgfpathlineto{\pgfqpoint{1.963784in}{2.046040in}}%
\pgfpathlineto{\pgfqpoint{1.981861in}{2.155967in}}%
\pgfpathlineto{\pgfqpoint{2.002755in}{2.448865in}}%
\pgfpathlineto{\pgfqpoint{2.020129in}{2.745203in}}%
\pgfpathlineto{\pgfqpoint{2.041259in}{2.810874in}}%
\pgfpathlineto{\pgfqpoint{2.060274in}{2.597848in}}%
\pgfpathlineto{\pgfqpoint{2.078119in}{2.284571in}}%
\pgfpathlineto{\pgfqpoint{2.095021in}{2.100809in}}%
\pgfpathlineto{\pgfqpoint{2.117325in}{2.026113in}}%
\pgfpathlineto{\pgfqpoint{2.137281in}{1.978821in}}%
\pgfpathlineto{\pgfqpoint{2.156061in}{1.972970in}}%
\pgfpathlineto{\pgfqpoint{2.173669in}{2.010356in}}%
\pgfpathlineto{\pgfqpoint{2.194800in}{2.109954in}}%
\pgfpathlineto{\pgfqpoint{2.212877in}{2.287864in}}%
\pgfpathlineto{\pgfqpoint{2.230720in}{2.561800in}}%
\pgfpathlineto{\pgfqpoint{2.251614in}{2.809672in}}%
\pgfpathlineto{\pgfqpoint{2.268753in}{2.764431in}}%
\pgfpathlineto{\pgfqpoint{2.309368in}{2.167799in}}%
\pgfpathlineto{\pgfqpoint{2.326741in}{2.047800in}}%
\pgfpathlineto{\pgfqpoint{2.347401in}{1.986980in}}%
\pgfpathlineto{\pgfqpoint{2.365480in}{1.966602in}}%
\pgfpathlineto{\pgfqpoint{2.387312in}{1.990493in}}%
\pgfpathlineto{\pgfqpoint{2.404686in}{2.047677in}}%
\pgfpathlineto{\pgfqpoint{2.422763in}{2.055356in}}%
\pgfpathlineto{\pgfqpoint{2.440136in}{2.131461in}}%
\pgfpathlineto{\pgfqpoint{2.461971in}{2.406733in}}%
\pgfpathlineto{\pgfqpoint{2.479110in}{2.707738in}}%
\pgfpathlineto{\pgfqpoint{2.503525in}{2.797143in}}%
\pgfpathlineto{\pgfqpoint{2.519255in}{2.754882in}}%
\pgfpathlineto{\pgfqpoint{2.543202in}{2.383102in}}%
\pgfpathlineto{\pgfqpoint{2.559401in}{2.167419in}}%
\pgfpathlineto{\pgfqpoint{2.576540in}{2.041407in}}%
\pgfpathlineto{\pgfqpoint{2.597669in}{1.981507in}}%
\pgfpathlineto{\pgfqpoint{2.615277in}{1.968915in}}%
\pgfpathlineto{\pgfqpoint{2.637814in}{2.007542in}}%
\pgfpathlineto{\pgfqpoint{2.652605in}{2.078845in}}%
\pgfpathlineto{\pgfqpoint{2.672562in}{2.213068in}}%
\pgfpathlineto{\pgfqpoint{2.711533in}{2.763908in}}%
\pgfpathlineto{\pgfqpoint{2.732429in}{2.801330in}}%
\pgfpathlineto{\pgfqpoint{2.750506in}{2.583437in}}%
\pgfpathlineto{\pgfqpoint{2.771166in}{2.815529in}}%
\pgfpathlineto{\pgfqpoint{2.789712in}{2.756020in}}%
\pgfpathlineto{\pgfqpoint{2.807320in}{2.464057in}}%
\pgfpathlineto{\pgfqpoint{2.828920in}{2.478586in}}%
\pgfpathlineto{\pgfqpoint{2.846997in}{2.225158in}}%
\pgfpathlineto{\pgfqpoint{2.865310in}{2.065497in}}%
\pgfpathlineto{\pgfqpoint{2.882684in}{1.993579in}}%
\pgfpathlineto{\pgfqpoint{2.905221in}{1.973145in}}%
\pgfpathlineto{\pgfqpoint{2.923298in}{1.977202in}}%
\pgfpathlineto{\pgfqpoint{2.943489in}{2.032226in}}%
\pgfpathlineto{\pgfqpoint{2.961801in}{2.131230in}}%
\pgfpathlineto{\pgfqpoint{2.982695in}{2.215746in}}%
\pgfpathlineto{\pgfqpoint{3.018851in}{2.751030in}}%
\pgfpathlineto{\pgfqpoint{3.040214in}{2.810932in}}%
\pgfpathlineto{\pgfqpoint{3.057353in}{2.586587in}}%
\pgfpathlineto{\pgfqpoint{3.075901in}{2.284188in}}%
\pgfpathlineto{\pgfqpoint{3.097030in}{2.078763in}}%
\pgfpathlineto{\pgfqpoint{3.115578in}{2.005452in}}%
\pgfpathlineto{\pgfqpoint{3.133889in}{1.981581in}}%
\pgfpathlineto{\pgfqpoint{3.154315in}{1.972786in}}%
\pgfpathlineto{\pgfqpoint{3.173097in}{2.010962in}}%
\pgfpathlineto{\pgfqpoint{3.194460in}{2.101149in}}%
\pgfpathlineto{\pgfqpoint{3.212303in}{2.248058in}}%
\pgfpathlineto{\pgfqpoint{3.232728in}{2.561242in}}%
\pgfpathlineto{\pgfqpoint{3.251276in}{2.792536in}}%
\pgfpathlineto{\pgfqpoint{3.269353in}{2.841531in}}%
\pgfpathlineto{\pgfqpoint{3.291187in}{2.670444in}}%
\pgfpathlineto{\pgfqpoint{3.308090in}{2.438834in}}%
\pgfpathlineto{\pgfqpoint{3.325698in}{2.214312in}}%
\pgfpathlineto{\pgfqpoint{3.347298in}{2.058903in}}%
\pgfpathlineto{\pgfqpoint{3.365844in}{1.998390in}}%
\pgfpathlineto{\pgfqpoint{3.386035in}{1.973498in}}%
\pgfpathlineto{\pgfqpoint{3.404817in}{1.987439in}}%
\pgfpathlineto{\pgfqpoint{3.422894in}{2.014764in}}%
\pgfpathlineto{\pgfqpoint{3.441442in}{2.091148in}}%
\pgfpathlineto{\pgfqpoint{3.461867in}{2.203190in}}%
\pgfpathlineto{\pgfqpoint{3.480413in}{2.427692in}}%
\pgfpathlineto{\pgfqpoint{3.498490in}{2.746311in}}%
\pgfpathlineto{\pgfqpoint{3.519855in}{2.858036in}}%
\pgfpathlineto{\pgfqpoint{3.538167in}{2.764817in}}%
\pgfpathlineto{\pgfqpoint{3.557183in}{2.510815in}}%
\pgfpathlineto{\pgfqpoint{3.579957in}{2.235621in}}%
\pgfpathlineto{\pgfqpoint{3.594513in}{2.108889in}}%
\pgfpathlineto{\pgfqpoint{3.615173in}{2.031030in}}%
\pgfpathlineto{\pgfqpoint{3.633485in}{1.991707in}}%
\pgfpathlineto{\pgfqpoint{3.653441in}{1.974772in}}%
\pgfpathlineto{\pgfqpoint{3.674336in}{2.000844in}}%
\pgfpathlineto{\pgfqpoint{3.693587in}{2.063402in}}%
\pgfpathlineto{\pgfqpoint{3.729272in}{2.291425in}}%
\pgfpathlineto{\pgfqpoint{3.751341in}{2.565407in}}%
\pgfpathlineto{\pgfqpoint{3.769654in}{2.810081in}}%
\pgfpathlineto{\pgfqpoint{3.789140in}{2.877458in}}%
\pgfpathlineto{\pgfqpoint{3.807922in}{2.785497in}}%
\pgfpathlineto{\pgfqpoint{3.827407in}{2.528908in}}%
\pgfpathlineto{\pgfqpoint{3.846893in}{2.606098in}}%
\pgfpathlineto{\pgfqpoint{3.866144in}{2.884906in}}%
\pgfpathlineto{\pgfqpoint{3.884692in}{2.763293in}}%
\pgfpathlineto{\pgfqpoint{3.904412in}{2.461076in}}%
\pgfpathlineto{\pgfqpoint{3.923429in}{2.209965in}}%
\pgfpathlineto{\pgfqpoint{3.942915in}{2.097385in}}%
\pgfpathlineto{\pgfqpoint{3.961228in}{2.017602in}}%
\pgfpathlineto{\pgfqpoint{3.980714in}{1.982940in}}%
\pgfpathlineto{\pgfqpoint{3.999496in}{1.988788in}}%
\pgfpathlineto{\pgfqpoint{4.019920in}{2.036002in}}%
\pgfpathlineto{\pgfqpoint{4.037528in}{2.114393in}}%
\pgfpathlineto{\pgfqpoint{4.056779in}{2.265672in}}%
\pgfpathlineto{\pgfqpoint{4.076030in}{2.478805in}}%
\pgfpathlineto{\pgfqpoint{4.094343in}{2.789206in}}%
\pgfpathlineto{\pgfqpoint{4.114534in}{2.910499in}}%
\pgfpathlineto{\pgfqpoint{4.132611in}{2.851060in}}%
\pgfpathlineto{\pgfqpoint{4.154914in}{2.595684in}}%
\pgfpathlineto{\pgfqpoint{4.170645in}{2.370203in}}%
\pgfpathlineto{\pgfqpoint{4.192008in}{2.144727in}}%
\pgfpathlineto{\pgfqpoint{4.211730in}{2.059452in}}%
\pgfpathlineto{\pgfqpoint{4.230276in}{2.014245in}}%
\pgfpathlineto{\pgfqpoint{4.248119in}{1.984981in}}%
\pgfpathlineto{\pgfqpoint{4.265961in}{2.002096in}}%
\pgfpathlineto{\pgfqpoint{4.290378in}{2.048298in}}%
\pgfpathlineto{\pgfqpoint{4.308926in}{2.147895in}}%
\pgfpathlineto{\pgfqpoint{4.327237in}{2.281444in}}%
\pgfpathlineto{\pgfqpoint{4.346489in}{2.472941in}}%
\pgfpathlineto{\pgfqpoint{4.365271in}{2.791534in}}%
\pgfpathlineto{\pgfqpoint{4.384288in}{2.929482in}}%
\pgfpathlineto{\pgfqpoint{4.402834in}{2.912287in}}%
\pgfpathlineto{\pgfqpoint{4.421850in}{2.781190in}}%
\pgfpathlineto{\pgfqpoint{4.441102in}{2.504664in}}%
\pgfpathlineto{\pgfqpoint{4.462467in}{2.213896in}}%
\pgfpathlineto{\pgfqpoint{4.482421in}{2.106069in}}%
\pgfpathlineto{\pgfqpoint{4.474674in}{2.166349in}}%
\pgfpathlineto{\pgfqpoint{4.455658in}{2.469153in}}%
\pgfpathlineto{\pgfqpoint{4.432180in}{2.890613in}}%
\pgfpathlineto{\pgfqpoint{4.416921in}{2.936309in}}%
\pgfpathlineto{\pgfqpoint{4.393678in}{2.576460in}}%
\pgfpathlineto{\pgfqpoint{4.377244in}{2.282903in}}%
\pgfpathlineto{\pgfqpoint{4.356350in}{2.078401in}}%
\pgfpathlineto{\pgfqpoint{4.338507in}{1.999800in}}%
\pgfpathlineto{\pgfqpoint{4.319959in}{1.993070in}}%
\pgfpathlineto{\pgfqpoint{4.302351in}{2.065375in}}%
\pgfpathlineto{\pgfqpoint{4.276291in}{2.360378in}}%
\pgfpathlineto{\pgfqpoint{4.261971in}{2.667585in}}%
\pgfpathlineto{\pgfqpoint{4.242484in}{2.905087in}}%
\pgfpathlineto{\pgfqpoint{4.224172in}{2.823330in}}%
\pgfpathlineto{\pgfqpoint{4.205156in}{2.445397in}}%
\pgfpathlineto{\pgfqpoint{4.184027in}{2.137838in}}%
\pgfpathlineto{\pgfqpoint{4.165244in}{2.020709in}}%
\pgfpathlineto{\pgfqpoint{4.146697in}{1.980344in}}%
\pgfpathlineto{\pgfqpoint{4.127916in}{2.018392in}}%
\pgfpathlineto{\pgfqpoint{4.108900in}{2.130032in}}%
\pgfpathlineto{\pgfqpoint{4.090821in}{2.380748in}}%
\pgfpathlineto{\pgfqpoint{4.069223in}{2.801391in}}%
\pgfpathlineto{\pgfqpoint{4.049972in}{2.883772in}}%
\pgfpathlineto{\pgfqpoint{4.034476in}{2.692796in}}%
\pgfpathlineto{\pgfqpoint{4.012878in}{2.276012in}}%
\pgfpathlineto{\pgfqpoint{3.991513in}{2.068770in}}%
\pgfpathlineto{\pgfqpoint{3.975548in}{2.001877in}}%
\pgfpathlineto{\pgfqpoint{3.957236in}{1.977739in}}%
\pgfpathlineto{\pgfqpoint{3.936342in}{2.037877in}}%
\pgfpathlineto{\pgfqpoint{3.916620in}{2.180386in}}%
\pgfpathlineto{\pgfqpoint{3.875535in}{2.842538in}}%
\pgfpathlineto{\pgfqpoint{3.857927in}{2.819853in}}%
\pgfpathlineto{\pgfqpoint{3.842198in}{2.541051in}}%
\pgfpathlineto{\pgfqpoint{3.817546in}{2.147333in}}%
\pgfpathlineto{\pgfqpoint{3.801347in}{2.057177in}}%
\pgfpathlineto{\pgfqpoint{3.783505in}{1.987538in}}%
\pgfpathlineto{\pgfqpoint{3.765897in}{1.978054in}}%
\pgfpathlineto{\pgfqpoint{3.743123in}{2.043593in}}%
\pgfpathlineto{\pgfqpoint{3.724108in}{2.182260in}}%
\pgfpathlineto{\pgfqpoint{3.687012in}{2.800152in}}%
\pgfpathlineto{\pgfqpoint{3.668701in}{2.850262in}}%
\pgfpathlineto{\pgfqpoint{3.646632in}{2.588113in}}%
\pgfpathlineto{\pgfqpoint{3.627615in}{2.269585in}}%
\pgfpathlineto{\pgfqpoint{3.610242in}{2.101143in}}%
\pgfpathlineto{\pgfqpoint{3.588410in}{2.013661in}}%
\pgfpathlineto{\pgfqpoint{3.572210in}{1.974027in}}%
\pgfpathlineto{\pgfqpoint{3.550845in}{1.981319in}}%
\pgfpathlineto{\pgfqpoint{3.531594in}{2.032930in}}%
\pgfpathlineto{\pgfqpoint{3.513046in}{2.147659in}}%
\pgfpathlineto{\pgfqpoint{3.494735in}{2.384712in}}%
\pgfpathlineto{\pgfqpoint{3.474309in}{2.770860in}}%
\pgfpathlineto{\pgfqpoint{3.455527in}{2.839872in}}%
\pgfpathlineto{\pgfqpoint{3.436510in}{2.662549in}}%
\pgfpathlineto{\pgfqpoint{3.417730in}{2.329913in}}%
\pgfpathlineto{\pgfqpoint{3.398713in}{2.134150in}}%
\pgfpathlineto{\pgfqpoint{3.379931in}{2.028097in}}%
\pgfpathlineto{\pgfqpoint{3.359271in}{1.972517in}}%
\pgfpathlineto{\pgfqpoint{3.339551in}{1.978233in}}%
\pgfpathlineto{\pgfqpoint{3.320299in}{2.029715in}}%
\pgfpathlineto{\pgfqpoint{3.302221in}{2.147548in}}%
\pgfpathlineto{\pgfqpoint{3.279917in}{2.442826in}}%
\pgfpathlineto{\pgfqpoint{3.261372in}{2.715644in}}%
\pgfpathlineto{\pgfqpoint{3.242589in}{2.827523in}}%
\pgfpathlineto{\pgfqpoint{3.224276in}{2.682577in}}%
\pgfpathlineto{\pgfqpoint{3.205496in}{2.393600in}}%
\pgfpathlineto{\pgfqpoint{3.186713in}{2.172414in}}%
\pgfpathlineto{\pgfqpoint{3.168636in}{2.070702in}}%
\pgfpathlineto{\pgfqpoint{3.146802in}{1.989842in}}%
\pgfpathlineto{\pgfqpoint{3.128489in}{1.968481in}}%
\pgfpathlineto{\pgfqpoint{3.108769in}{1.998210in}}%
\pgfpathlineto{\pgfqpoint{3.091161in}{2.066629in}}%
\pgfpathlineto{\pgfqpoint{3.072613in}{2.237218in}}%
\pgfpathlineto{\pgfqpoint{3.035754in}{2.806111in}}%
\pgfpathlineto{\pgfqpoint{3.013450in}{2.806679in}}%
\pgfpathlineto{\pgfqpoint{2.995373in}{2.768453in}}%
\pgfpathlineto{\pgfqpoint{2.976357in}{2.479192in}}%
\pgfpathlineto{\pgfqpoint{2.956871in}{2.206441in}}%
\pgfpathlineto{\pgfqpoint{2.938794in}{2.069816in}}%
\pgfpathlineto{\pgfqpoint{2.916491in}{1.992513in}}%
\pgfpathlineto{\pgfqpoint{2.899352in}{1.970919in}}%
\pgfpathlineto{\pgfqpoint{2.877518in}{1.977642in}}%
\pgfpathlineto{\pgfqpoint{2.859206in}{2.011348in}}%
\pgfpathlineto{\pgfqpoint{2.839719in}{2.062987in}}%
\pgfpathlineto{\pgfqpoint{2.822816in}{2.194806in}}%
\pgfpathlineto{\pgfqpoint{2.802625in}{2.517219in}}%
\pgfpathlineto{\pgfqpoint{2.782200in}{2.779161in}}%
\pgfpathlineto{\pgfqpoint{2.763419in}{2.796048in}}%
\pgfpathlineto{\pgfqpoint{2.746749in}{2.628643in}}%
\pgfpathlineto{\pgfqpoint{2.725151in}{2.300569in}}%
\pgfpathlineto{\pgfqpoint{2.706134in}{2.114755in}}%
\pgfpathlineto{\pgfqpoint{2.687587in}{2.090725in}}%
\pgfpathlineto{\pgfqpoint{2.687821in}{2.034539in}}%
\pgfpathlineto{\pgfqpoint{2.669979in}{2.014720in}}%
\pgfpathlineto{\pgfqpoint{2.647207in}{1.971209in}}%
\pgfpathlineto{\pgfqpoint{2.628424in}{1.979925in}}%
\pgfpathlineto{\pgfqpoint{2.610582in}{2.025873in}}%
\pgfpathlineto{\pgfqpoint{2.591800in}{2.103393in}}%
\pgfpathlineto{\pgfqpoint{2.574897in}{2.296187in}}%
\pgfpathlineto{\pgfqpoint{2.551185in}{2.519233in}}%
\pgfpathlineto{\pgfqpoint{2.532637in}{2.772056in}}%
\pgfpathlineto{\pgfqpoint{2.513855in}{2.786284in}}%
\pgfpathlineto{\pgfqpoint{2.497187in}{2.578502in}}%
\pgfpathlineto{\pgfqpoint{2.473709in}{2.227167in}}%
\pgfpathlineto{\pgfqpoint{2.457276in}{2.081266in}}%
\pgfpathlineto{\pgfqpoint{2.437085in}{2.001664in}}%
\pgfpathlineto{\pgfqpoint{2.418773in}{2.001393in}}%
\pgfpathlineto{\pgfqpoint{2.396939in}{1.974453in}}%
\pgfpathlineto{\pgfqpoint{2.378391in}{1.971331in}}%
\pgfpathlineto{\pgfqpoint{2.360783in}{2.008471in}}%
\pgfpathlineto{\pgfqpoint{2.338480in}{2.074109in}}%
\pgfpathlineto{\pgfqpoint{2.322515in}{2.206983in}}%
\pgfpathlineto{\pgfqpoint{2.300683in}{2.578304in}}%
\pgfpathlineto{\pgfqpoint{2.278614in}{2.789302in}}%
\pgfpathlineto{\pgfqpoint{2.263353in}{2.804336in}}%
\pgfpathlineto{\pgfqpoint{2.245510in}{2.666236in}}%
\pgfpathlineto{\pgfqpoint{2.226728in}{2.432060in}}%
\pgfpathlineto{\pgfqpoint{2.207948in}{2.180793in}}%
\pgfpathlineto{\pgfqpoint{2.186582in}{2.051895in}}%
\pgfpathlineto{\pgfqpoint{2.168271in}{1.998660in}}%
\pgfpathlineto{\pgfqpoint{2.149489in}{1.976983in}}%
\pgfpathlineto{\pgfqpoint{2.131412in}{1.973106in}}%
\pgfpathlineto{\pgfqpoint{2.110046in}{2.017394in}}%
\pgfpathlineto{\pgfqpoint{2.091735in}{2.096457in}}%
\pgfpathlineto{\pgfqpoint{2.072718in}{2.277796in}}%
\pgfpathlineto{\pgfqpoint{2.050415in}{2.558643in}}%
\pgfpathlineto{\pgfqpoint{2.032573in}{2.759119in}}%
\pgfpathlineto{\pgfqpoint{2.014494in}{2.824389in}}%
\pgfpathlineto{\pgfqpoint{1.995243in}{2.779921in}}%
\pgfpathlineto{\pgfqpoint{1.975288in}{2.529955in}}%
\pgfpathlineto{\pgfqpoint{1.956506in}{2.232357in}}%
\pgfpathlineto{\pgfqpoint{1.937723in}{2.079949in}}%
\pgfpathlineto{\pgfqpoint{1.920586in}{2.047075in}}%
\pgfpathlineto{\pgfqpoint{1.896638in}{2.490384in}}%
\pgfpathlineto{\pgfqpoint{1.878327in}{2.207283in}}%
\pgfpathlineto{\pgfqpoint{1.860250in}{2.069707in}}%
\pgfpathlineto{\pgfqpoint{1.841233in}{1.993790in}}%
\pgfpathlineto{\pgfqpoint{1.820339in}{1.970246in}}%
\pgfpathlineto{\pgfqpoint{1.801322in}{1.988473in}}%
\pgfpathlineto{\pgfqpoint{1.783479in}{2.045259in}}%
\pgfpathlineto{\pgfqpoint{1.765166in}{2.174727in}}%
\pgfpathlineto{\pgfqpoint{1.744037in}{2.461929in}}%
\pgfpathlineto{\pgfqpoint{1.725255in}{2.756671in}}%
\pgfpathlineto{\pgfqpoint{1.706473in}{2.825455in}}%
\pgfpathlineto{\pgfqpoint{1.687927in}{2.834248in}}%
\pgfpathlineto{\pgfqpoint{1.669848in}{2.679606in}}%
\pgfpathlineto{\pgfqpoint{1.650362in}{2.375676in}}%
\pgfpathlineto{\pgfqpoint{1.626885in}{2.151384in}}%
\pgfpathlineto{\pgfqpoint{1.612329in}{2.065851in}}%
\pgfpathlineto{\pgfqpoint{1.591200in}{2.014037in}}%
\pgfpathlineto{\pgfqpoint{1.573826in}{1.979483in}}%
\pgfpathlineto{\pgfqpoint{1.553637in}{1.977007in}}%
\pgfpathlineto{\pgfqpoint{1.533212in}{2.019448in}}%
\pgfpathlineto{\pgfqpoint{1.515369in}{2.098137in}}%
\pgfpathlineto{\pgfqpoint{1.495647in}{2.278181in}}%
\pgfpathlineto{\pgfqpoint{1.474518in}{2.599355in}}%
\pgfpathlineto{\pgfqpoint{1.457145in}{2.800440in}}%
\pgfpathlineto{\pgfqpoint{1.437659in}{2.856282in}}%
\pgfpathlineto{\pgfqpoint{1.419582in}{2.777506in}}%
\pgfpathlineto{\pgfqpoint{1.419816in}{2.563497in}}%
\pgfpathlineto{\pgfqpoint{1.395868in}{2.406146in}}%
\pgfpathlineto{\pgfqpoint{1.380374in}{2.228365in}}%
\pgfpathlineto{\pgfqpoint{1.360183in}{2.084578in}}%
\pgfpathlineto{\pgfqpoint{1.342575in}{2.026722in}}%
\pgfpathlineto{\pgfqpoint{1.323795in}{1.984683in}}%
\pgfpathlineto{\pgfqpoint{1.305013in}{1.976485in}}%
\pgfpathlineto{\pgfqpoint{1.283178in}{2.010075in}}%
\pgfpathlineto{\pgfqpoint{1.265336in}{2.071088in}}%
\pgfpathlineto{\pgfqpoint{1.244676in}{2.213046in}}%
\pgfpathlineto{\pgfqpoint{1.207817in}{2.682925in}}%
\pgfpathlineto{\pgfqpoint{1.187860in}{2.858775in}}%
\pgfpathlineto{\pgfqpoint{1.168140in}{2.876846in}}%
\pgfpathlineto{\pgfqpoint{1.146540in}{2.829174in}}%
\pgfpathlineto{\pgfqpoint{1.132455in}{2.637128in}}%
\pgfpathlineto{\pgfqpoint{1.094187in}{2.196154in}}%
\pgfpathlineto{\pgfqpoint{1.073056in}{2.074494in}}%
\pgfpathlineto{\pgfqpoint{1.053102in}{2.010687in}}%
\pgfpathlineto{\pgfqpoint{1.031973in}{1.979091in}}%
\pgfpathlineto{\pgfqpoint{1.014365in}{1.991230in}}%
\pgfpathlineto{\pgfqpoint{0.996991in}{1.997881in}}%
\pgfpathlineto{\pgfqpoint{0.976097in}{2.054286in}}%
\pgfpathlineto{\pgfqpoint{0.958489in}{2.142527in}}%
\pgfpathlineto{\pgfqpoint{0.937829in}{2.363134in}}%
\pgfpathlineto{\pgfqpoint{0.917169in}{2.678328in}}%
\pgfpathlineto{\pgfqpoint{0.898856in}{2.869185in}}%
\pgfpathlineto{\pgfqpoint{0.881482in}{2.906519in}}%
\pgfpathlineto{\pgfqpoint{0.861059in}{2.771574in}}%
\pgfpathlineto{\pgfqpoint{0.841571in}{2.575916in}}%
\pgfpathlineto{\pgfqpoint{0.824199in}{2.302768in}}%
\pgfpathlineto{\pgfqpoint{0.804712in}{2.207426in}}%
\pgfpathlineto{\pgfqpoint{0.783348in}{2.077404in}}%
\pgfpathlineto{\pgfqpoint{0.764332in}{2.019693in}}%
\pgfpathlineto{\pgfqpoint{0.745549in}{2.191232in}}%
\pgfpathlineto{\pgfqpoint{0.725595in}{2.093567in}}%
\pgfpathlineto{\pgfqpoint{0.707516in}{2.015555in}}%
\pgfpathlineto{\pgfqpoint{0.689908in}{1.985118in}}%
\pgfpathlineto{\pgfqpoint{0.668310in}{2.013760in}}%
\pgfpathlineto{\pgfqpoint{0.647885in}{2.086733in}}%
\pgfpathlineto{\pgfqpoint{0.650702in}{2.076469in}}%
\pgfpathlineto{\pgfqpoint{0.656101in}{2.040514in}}%
\pgfpathlineto{\pgfqpoint{0.675117in}{1.984557in}}%
\pgfpathlineto{\pgfqpoint{0.694839in}{2.024970in}}%
\pgfpathlineto{\pgfqpoint{0.714325in}{2.136871in}}%
\pgfpathlineto{\pgfqpoint{0.732167in}{2.363827in}}%
\pgfpathlineto{\pgfqpoint{0.751184in}{2.752224in}}%
\pgfpathlineto{\pgfqpoint{0.770670in}{2.914078in}}%
\pgfpathlineto{\pgfqpoint{0.788747in}{2.760318in}}%
\pgfpathlineto{\pgfqpoint{0.808938in}{2.333683in}}%
\pgfpathlineto{\pgfqpoint{0.828424in}{2.113051in}}%
\pgfpathlineto{\pgfqpoint{0.849789in}{1.999266in}}%
\pgfpathlineto{\pgfqpoint{0.866457in}{1.983954in}}%
\pgfpathlineto{\pgfqpoint{0.887588in}{2.052080in}}%
\pgfpathlineto{\pgfqpoint{0.905899in}{2.184570in}}%
\pgfpathlineto{\pgfqpoint{0.924447in}{2.466347in}}%
\pgfpathlineto{\pgfqpoint{0.945341in}{2.839895in}}%
\pgfpathlineto{\pgfqpoint{0.963418in}{2.863309in}}%
\pgfpathlineto{\pgfqpoint{0.981026in}{2.568080in}}%
\pgfpathlineto{\pgfqpoint{1.001921in}{2.197240in}}%
\pgfpathlineto{\pgfqpoint{1.021643in}{2.035904in}}%
\pgfpathlineto{\pgfqpoint{1.040658in}{1.980716in}}%
\pgfpathlineto{\pgfqpoint{1.058737in}{1.994918in}}%
\pgfpathlineto{\pgfqpoint{1.078926in}{2.086551in}}%
\pgfpathlineto{\pgfqpoint{1.097005in}{2.253024in}}%
\pgfpathlineto{\pgfqpoint{1.117664in}{2.636083in}}%
\pgfpathlineto{\pgfqpoint{1.135976in}{2.861745in}}%
\pgfpathlineto{\pgfqpoint{1.154289in}{2.782493in}}%
\pgfpathlineto{\pgfqpoint{1.175418in}{2.387325in}}%
\pgfpathlineto{\pgfqpoint{1.195609in}{2.127822in}}%
\pgfpathlineto{\pgfqpoint{1.214155in}{2.011212in}}%
\pgfpathlineto{\pgfqpoint{1.231998in}{1.972633in}}%
\pgfpathlineto{\pgfqpoint{1.250545in}{1.997285in}}%
\pgfpathlineto{\pgfqpoint{1.271205in}{2.088372in}}%
\pgfpathlineto{\pgfqpoint{1.289282in}{2.260342in}}%
\pgfpathlineto{\pgfqpoint{1.307125in}{2.598355in}}%
\pgfpathlineto{\pgfqpoint{1.329193in}{2.841518in}}%
\pgfpathlineto{\pgfqpoint{1.347272in}{2.762452in}}%
\pgfpathlineto{\pgfqpoint{1.367227in}{2.363310in}}%
\pgfpathlineto{\pgfqpoint{1.384366in}{2.152700in}}%
\pgfpathlineto{\pgfqpoint{1.402208in}{2.031721in}}%
\pgfpathlineto{\pgfqpoint{1.423572in}{1.976818in}}%
\pgfpathlineto{\pgfqpoint{1.444937in}{1.990480in}}%
\pgfpathlineto{\pgfqpoint{1.462074in}{2.053275in}}%
\pgfpathlineto{\pgfqpoint{1.482971in}{2.169937in}}%
\pgfpathlineto{\pgfqpoint{1.501516in}{2.419914in}}%
\pgfpathlineto{\pgfqpoint{1.522176in}{2.671013in}}%
\pgfpathlineto{\pgfqpoint{1.540724in}{2.838333in}}%
\pgfpathlineto{\pgfqpoint{1.558801in}{2.707529in}}%
\pgfpathlineto{\pgfqpoint{1.576644in}{2.373943in}}%
\pgfpathlineto{\pgfqpoint{1.597538in}{2.147924in}}%
\pgfpathlineto{\pgfqpoint{1.615382in}{2.045158in}}%
\pgfpathlineto{\pgfqpoint{1.635806in}{1.979203in}}%
\pgfpathlineto{\pgfqpoint{1.653179in}{1.970676in}}%
\pgfpathlineto{\pgfqpoint{1.675014in}{2.011684in}}%
\pgfpathlineto{\pgfqpoint{1.696143in}{2.097372in}}%
\pgfpathlineto{\pgfqpoint{1.710933in}{2.189813in}}%
\pgfpathlineto{\pgfqpoint{1.731593in}{2.502373in}}%
\pgfpathlineto{\pgfqpoint{1.752489in}{2.796619in}}%
\pgfpathlineto{\pgfqpoint{1.770332in}{2.813792in}}%
\pgfpathlineto{\pgfqpoint{1.791461in}{2.577464in}}%
\pgfpathlineto{\pgfqpoint{1.809772in}{2.445710in}}%
\pgfpathlineto{\pgfqpoint{1.827617in}{2.175397in}}%
\pgfpathlineto{\pgfqpoint{1.844988in}{2.037753in}}%
\pgfpathlineto{\pgfqpoint{1.867291in}{1.977886in}}%
\pgfpathlineto{\pgfqpoint{1.884899in}{1.971701in}}%
\pgfpathlineto{\pgfqpoint{1.905796in}{2.014827in}}%
\pgfpathlineto{\pgfqpoint{1.923404in}{2.091998in}}%
\pgfpathlineto{\pgfqpoint{1.944064in}{2.226268in}}%
\pgfpathlineto{\pgfqpoint{1.961906in}{2.529651in}}%
\pgfpathlineto{\pgfqpoint{1.983738in}{2.811457in}}%
\pgfpathlineto{\pgfqpoint{2.000877in}{2.779465in}}%
\pgfpathlineto{\pgfqpoint{2.018954in}{2.575733in}}%
\pgfpathlineto{\pgfqpoint{2.055579in}{2.066473in}}%
\pgfpathlineto{\pgfqpoint{2.077179in}{2.023250in}}%
\pgfpathlineto{\pgfqpoint{2.097134in}{1.974197in}}%
\pgfpathlineto{\pgfqpoint{2.097370in}{1.970294in}}%
\pgfpathlineto{\pgfqpoint{2.118264in}{1.977102in}}%
\pgfpathlineto{\pgfqpoint{2.137984in}{2.015918in}}%
\pgfpathlineto{\pgfqpoint{2.154418in}{2.088975in}}%
\pgfpathlineto{\pgfqpoint{2.175078in}{2.273445in}}%
\pgfpathlineto{\pgfqpoint{2.193392in}{2.538501in}}%
\pgfpathlineto{\pgfqpoint{2.211234in}{2.711025in}}%
\pgfpathlineto{\pgfqpoint{2.229780in}{2.812861in}}%
\pgfpathlineto{\pgfqpoint{2.250676in}{2.641130in}}%
\pgfpathlineto{\pgfqpoint{2.271571in}{2.316679in}}%
\pgfpathlineto{\pgfqpoint{2.290353in}{2.108029in}}%
\pgfpathlineto{\pgfqpoint{2.309133in}{2.018119in}}%
\pgfpathlineto{\pgfqpoint{2.326507in}{1.974026in}}%
\pgfpathlineto{\pgfqpoint{2.347636in}{1.972865in}}%
\pgfpathlineto{\pgfqpoint{2.369001in}{2.006196in}}%
\pgfpathlineto{\pgfqpoint{2.386609in}{2.042367in}}%
\pgfpathlineto{\pgfqpoint{2.404217in}{2.123471in}}%
\pgfpathlineto{\pgfqpoint{2.422528in}{2.305416in}}%
\pgfpathlineto{\pgfqpoint{2.442250in}{2.576490in}}%
\pgfpathlineto{\pgfqpoint{2.461031in}{2.799235in}}%
\pgfpathlineto{\pgfqpoint{2.482162in}{2.730055in}}%
\pgfpathlineto{\pgfqpoint{2.500473in}{2.437515in}}%
\pgfpathlineto{\pgfqpoint{2.517847in}{2.191904in}}%
\pgfpathlineto{\pgfqpoint{2.540150in}{2.048185in}}%
\pgfpathlineto{\pgfqpoint{2.559401in}{1.987981in}}%
\pgfpathlineto{\pgfqpoint{2.576774in}{1.975066in}}%
\pgfpathlineto{\pgfqpoint{2.598138in}{1.975487in}}%
\pgfpathlineto{\pgfqpoint{2.615982in}{2.006250in}}%
\pgfpathlineto{\pgfqpoint{2.634528in}{2.072095in}}%
\pgfpathlineto{\pgfqpoint{2.654014in}{2.213963in}}%
\pgfpathlineto{\pgfqpoint{2.674674in}{2.218785in}}%
\pgfpathlineto{\pgfqpoint{2.694396in}{2.512725in}}%
\pgfpathlineto{\pgfqpoint{2.713881in}{2.775349in}}%
\pgfpathlineto{\pgfqpoint{2.732664in}{2.812151in}}%
\pgfpathlineto{\pgfqpoint{2.750506in}{2.612814in}}%
\pgfpathlineto{\pgfqpoint{2.768583in}{2.319298in}}%
\pgfpathlineto{\pgfqpoint{2.789478in}{2.103192in}}%
\pgfpathlineto{\pgfqpoint{2.807320in}{2.017607in}}%
\pgfpathlineto{\pgfqpoint{2.826102in}{1.976127in}}%
\pgfpathlineto{\pgfqpoint{2.846997in}{1.976107in}}%
\pgfpathlineto{\pgfqpoint{2.866484in}{2.017116in}}%
\pgfpathlineto{\pgfqpoint{2.884327in}{2.091065in}}%
\pgfpathlineto{\pgfqpoint{2.908273in}{2.228465in}}%
\pgfpathlineto{\pgfqpoint{2.924004in}{2.451459in}}%
\pgfpathlineto{\pgfqpoint{2.944193in}{2.708897in}}%
\pgfpathlineto{\pgfqpoint{2.962271in}{2.827094in}}%
\pgfpathlineto{\pgfqpoint{2.981286in}{2.753137in}}%
\pgfpathlineto{\pgfqpoint{3.001948in}{2.441864in}}%
\pgfpathlineto{\pgfqpoint{3.019320in}{2.250648in}}%
\pgfpathlineto{\pgfqpoint{3.037399in}{2.100012in}}%
\pgfpathlineto{\pgfqpoint{3.037633in}{2.042056in}}%
\pgfpathlineto{\pgfqpoint{3.062519in}{2.024832in}}%
\pgfpathlineto{\pgfqpoint{3.077075in}{1.990780in}}%
\pgfpathlineto{\pgfqpoint{3.095152in}{2.623884in}}%
\pgfpathlineto{\pgfqpoint{3.116281in}{2.243495in}}%
\pgfpathlineto{\pgfqpoint{3.134358in}{2.085172in}}%
\pgfpathlineto{\pgfqpoint{3.152671in}{2.002815in}}%
\pgfpathlineto{\pgfqpoint{3.171217in}{1.972840in}}%
\pgfpathlineto{\pgfqpoint{3.191877in}{1.989981in}}%
\pgfpathlineto{\pgfqpoint{3.210191in}{2.040428in}}%
\pgfpathlineto{\pgfqpoint{3.232259in}{2.167889in}}%
\pgfpathlineto{\pgfqpoint{3.251979in}{2.416620in}}%
\pgfpathlineto{\pgfqpoint{3.271231in}{2.683738in}}%
\pgfpathlineto{\pgfqpoint{3.288839in}{2.839169in}}%
\pgfpathlineto{\pgfqpoint{3.306681in}{2.780310in}}%
\pgfpathlineto{\pgfqpoint{3.327812in}{2.556699in}}%
\pgfpathlineto{\pgfqpoint{3.348472in}{2.228552in}}%
\pgfpathlineto{\pgfqpoint{3.363966in}{2.102015in}}%
\pgfpathlineto{\pgfqpoint{3.384860in}{2.013721in}}%
\pgfpathlineto{\pgfqpoint{3.405286in}{1.977565in}}%
\pgfpathlineto{\pgfqpoint{3.424537in}{1.979704in}}%
\pgfpathlineto{\pgfqpoint{3.444259in}{2.027337in}}%
\pgfpathlineto{\pgfqpoint{3.459987in}{2.083900in}}%
\pgfpathlineto{\pgfqpoint{3.480647in}{2.256451in}}%
\pgfpathlineto{\pgfqpoint{3.500604in}{2.529846in}}%
\pgfpathlineto{\pgfqpoint{3.520795in}{2.824449in}}%
\pgfpathlineto{\pgfqpoint{3.540515in}{2.856996in}}%
\pgfpathlineto{\pgfqpoint{3.558358in}{2.727776in}}%
\pgfpathlineto{\pgfqpoint{3.577374in}{2.453201in}}%
\pgfpathlineto{\pgfqpoint{3.598269in}{2.181837in}}%
\pgfpathlineto{\pgfqpoint{3.616111in}{2.096488in}}%
\pgfpathlineto{\pgfqpoint{3.633954in}{2.040027in}}%
\pgfpathlineto{\pgfqpoint{3.652970in}{1.988931in}}%
\pgfpathlineto{\pgfqpoint{3.673630in}{1.981249in}}%
\pgfpathlineto{\pgfqpoint{3.693821in}{2.016784in}}%
\pgfpathlineto{\pgfqpoint{3.713073in}{2.086990in}}%
\pgfpathlineto{\pgfqpoint{3.733498in}{2.187836in}}%
\pgfpathlineto{\pgfqpoint{3.750637in}{2.381998in}}%
\pgfpathlineto{\pgfqpoint{3.769654in}{2.661559in}}%
\pgfpathlineto{\pgfqpoint{3.789140in}{2.854299in}}%
\pgfpathlineto{\pgfqpoint{3.808156in}{2.869246in}}%
\pgfpathlineto{\pgfqpoint{3.826702in}{2.712794in}}%
\pgfpathlineto{\pgfqpoint{3.846188in}{2.492686in}}%
\pgfpathlineto{\pgfqpoint{3.866850in}{2.236369in}}%
\pgfpathlineto{\pgfqpoint{3.885396in}{2.099013in}}%
\pgfpathlineto{\pgfqpoint{3.904647in}{2.024425in}}%
\pgfpathlineto{\pgfqpoint{3.922960in}{1.998602in}}%
\pgfpathlineto{\pgfqpoint{3.942680in}{1.979294in}}%
\pgfpathlineto{\pgfqpoint{3.960288in}{2.002863in}}%
\pgfpathlineto{\pgfqpoint{3.980479in}{2.033207in}}%
\pgfpathlineto{\pgfqpoint{3.999260in}{2.121491in}}%
\pgfpathlineto{\pgfqpoint{4.019216in}{2.270841in}}%
\pgfpathlineto{\pgfqpoint{4.037293in}{2.537020in}}%
\pgfpathlineto{\pgfqpoint{4.056544in}{2.803225in}}%
\pgfpathlineto{\pgfqpoint{4.075327in}{2.904597in}}%
\pgfpathlineto{\pgfqpoint{4.097630in}{2.880203in}}%
\pgfpathlineto{\pgfqpoint{4.116178in}{2.691189in}}%
\pgfpathlineto{\pgfqpoint{4.136367in}{2.379251in}}%
\pgfpathlineto{\pgfqpoint{4.154914in}{2.210299in}}%
\pgfpathlineto{\pgfqpoint{4.174400in}{2.084906in}}%
\pgfpathlineto{\pgfqpoint{4.192477in}{2.030860in}}%
\pgfpathlineto{\pgfqpoint{4.212199in}{1.985684in}}%
\pgfpathlineto{\pgfqpoint{4.229102in}{2.593701in}}%
\pgfpathlineto{\pgfqpoint{4.249998in}{2.882951in}}%
\pgfpathlineto{\pgfqpoint{4.268544in}{2.902532in}}%
\pgfpathlineto{\pgfqpoint{4.287561in}{2.696877in}}%
\pgfpathlineto{\pgfqpoint{4.308455in}{2.313021in}}%
\pgfpathlineto{\pgfqpoint{4.327237in}{2.137946in}}%
\pgfpathlineto{\pgfqpoint{4.346958in}{2.024202in}}%
\pgfpathlineto{\pgfqpoint{4.364331in}{1.985882in}}%
\pgfpathlineto{\pgfqpoint{4.388983in}{2.014195in}}%
\pgfpathlineto{\pgfqpoint{4.402365in}{2.079423in}}%
\pgfpathlineto{\pgfqpoint{4.421147in}{2.193590in}}%
\pgfpathlineto{\pgfqpoint{4.443919in}{2.506413in}}%
\pgfpathlineto{\pgfqpoint{4.462232in}{2.833791in}}%
\pgfpathlineto{\pgfqpoint{4.481013in}{2.943912in}}%
\pgfpathlineto{\pgfqpoint{4.473266in}{2.910139in}}%
\pgfpathlineto{\pgfqpoint{4.433824in}{2.209121in}}%
\pgfpathlineto{\pgfqpoint{4.415981in}{2.060825in}}%
\pgfpathlineto{\pgfqpoint{4.397904in}{1.992760in}}%
\pgfpathlineto{\pgfqpoint{4.377244in}{2.010457in}}%
\pgfpathlineto{\pgfqpoint{4.359402in}{2.104572in}}%
\pgfpathlineto{\pgfqpoint{4.342028in}{2.330684in}}%
\pgfpathlineto{\pgfqpoint{4.322072in}{2.727413in}}%
\pgfpathlineto{\pgfqpoint{4.300943in}{2.913953in}}%
\pgfpathlineto{\pgfqpoint{4.282866in}{2.690864in}}%
\pgfpathlineto{\pgfqpoint{4.260563in}{2.237691in}}%
\pgfpathlineto{\pgfqpoint{4.243892in}{2.078561in}}%
\pgfpathlineto{\pgfqpoint{4.224876in}{1.995341in}}%
\pgfpathlineto{\pgfqpoint{4.203043in}{1.989891in}}%
\pgfpathlineto{\pgfqpoint{4.185201in}{2.071119in}}%
\pgfpathlineto{\pgfqpoint{4.164541in}{2.276190in}}%
\pgfpathlineto{\pgfqpoint{4.146228in}{2.656041in}}%
\pgfpathlineto{\pgfqpoint{4.127445in}{2.886896in}}%
\pgfpathlineto{\pgfqpoint{4.105142in}{2.696345in}}%
\pgfpathlineto{\pgfqpoint{4.089883in}{2.359134in}}%
\pgfpathlineto{\pgfqpoint{4.071569in}{2.113965in}}%
\pgfpathlineto{\pgfqpoint{4.053024in}{2.014515in}}%
\pgfpathlineto{\pgfqpoint{4.031189in}{1.976406in}}%
\pgfpathlineto{\pgfqpoint{4.012173in}{2.010360in}}%
\pgfpathlineto{\pgfqpoint{3.993861in}{2.116406in}}%
\pgfpathlineto{\pgfqpoint{3.974844in}{2.372820in}}%
\pgfpathlineto{\pgfqpoint{3.953245in}{2.779074in}}%
\pgfpathlineto{\pgfqpoint{3.933994in}{2.863944in}}%
\pgfpathlineto{\pgfqpoint{3.915917in}{2.607188in}}%
\pgfpathlineto{\pgfqpoint{3.897134in}{2.250015in}}%
\pgfpathlineto{\pgfqpoint{3.878823in}{2.077000in}}%
\pgfpathlineto{\pgfqpoint{3.860041in}{2.000995in}}%
\pgfpathlineto{\pgfqpoint{3.840555in}{1.972444in}}%
\pgfpathlineto{\pgfqpoint{3.822476in}{2.009824in}}%
\pgfpathlineto{\pgfqpoint{3.800644in}{2.128556in}}%
\pgfpathlineto{\pgfqpoint{3.785148in}{2.282849in}}%
\pgfpathlineto{\pgfqpoint{3.765193in}{2.627410in}}%
\pgfpathlineto{\pgfqpoint{3.740307in}{2.853826in}}%
\pgfpathlineto{\pgfqpoint{3.723637in}{2.801485in}}%
\pgfpathlineto{\pgfqpoint{3.701099in}{2.399909in}}%
\pgfpathlineto{\pgfqpoint{3.686309in}{2.204350in}}%
\pgfpathlineto{\pgfqpoint{3.667527in}{2.064844in}}%
\pgfpathlineto{\pgfqpoint{3.649215in}{1.994193in}}%
\pgfpathlineto{\pgfqpoint{3.626912in}{2.830666in}}%
\pgfpathlineto{\pgfqpoint{3.608833in}{2.571949in}}%
\pgfpathlineto{\pgfqpoint{3.589818in}{2.242208in}}%
\pgfpathlineto{\pgfqpoint{3.568453in}{2.064753in}}%
\pgfpathlineto{\pgfqpoint{3.552957in}{2.003462in}}%
\pgfpathlineto{\pgfqpoint{3.531125in}{1.970055in}}%
\pgfpathlineto{\pgfqpoint{3.515160in}{1.986210in}}%
\pgfpathlineto{\pgfqpoint{3.493560in}{2.049273in}}%
\pgfpathlineto{\pgfqpoint{3.473840in}{2.190123in}}%
\pgfpathlineto{\pgfqpoint{3.452240in}{2.527995in}}%
\pgfpathlineto{\pgfqpoint{3.436746in}{2.789976in}}%
\pgfpathlineto{\pgfqpoint{3.417730in}{2.831449in}}%
\pgfpathlineto{\pgfqpoint{3.397539in}{2.620451in}}%
\pgfpathlineto{\pgfqpoint{3.379462in}{2.259378in}}%
\pgfpathlineto{\pgfqpoint{3.356922in}{2.073501in}}%
\pgfpathlineto{\pgfqpoint{3.340723in}{2.009523in}}%
\pgfpathlineto{\pgfqpoint{3.321708in}{1.969433in}}%
\pgfpathlineto{\pgfqpoint{3.301046in}{1.991500in}}%
\pgfpathlineto{\pgfqpoint{3.282031in}{2.056016in}}%
\pgfpathlineto{\pgfqpoint{3.264187in}{2.203332in}}%
\pgfpathlineto{\pgfqpoint{3.245407in}{2.498312in}}%
\pgfpathlineto{\pgfqpoint{3.223338in}{2.797823in}}%
\pgfpathlineto{\pgfqpoint{3.205730in}{2.807372in}}%
\pgfpathlineto{\pgfqpoint{3.186713in}{2.562230in}}%
\pgfpathlineto{\pgfqpoint{3.167462in}{2.258670in}}%
\pgfpathlineto{\pgfqpoint{3.148914in}{2.091964in}}%
\pgfpathlineto{\pgfqpoint{3.128960in}{2.003234in}}%
\pgfpathlineto{\pgfqpoint{3.110177in}{1.970327in}}%
\pgfpathlineto{\pgfqpoint{3.090457in}{1.983801in}}%
\pgfpathlineto{\pgfqpoint{3.071909in}{2.029797in}}%
\pgfpathlineto{\pgfqpoint{3.050075in}{2.113135in}}%
\pgfpathlineto{\pgfqpoint{3.031529in}{2.359379in}}%
\pgfpathlineto{\pgfqpoint{3.013685in}{2.682410in}}%
\pgfpathlineto{\pgfqpoint{2.993730in}{2.823597in}}%
\pgfpathlineto{\pgfqpoint{2.975419in}{2.708981in}}%
\pgfpathlineto{\pgfqpoint{2.957811in}{2.420710in}}%
\pgfpathlineto{\pgfqpoint{2.935271in}{2.161692in}}%
\pgfpathlineto{\pgfqpoint{2.916960in}{2.041235in}}%
\pgfpathlineto{\pgfqpoint{2.898647in}{1.985365in}}%
\pgfpathlineto{\pgfqpoint{2.879632in}{1.969949in}}%
\pgfpathlineto{\pgfqpoint{2.859441in}{1.990232in}}%
\pgfpathlineto{\pgfqpoint{2.839719in}{2.029189in}}%
\pgfpathlineto{\pgfqpoint{2.821407in}{2.139321in}}%
\pgfpathlineto{\pgfqpoint{2.802156in}{2.361400in}}%
\pgfpathlineto{\pgfqpoint{2.783845in}{2.655880in}}%
\pgfpathlineto{\pgfqpoint{2.765766in}{2.815734in}}%
\pgfpathlineto{\pgfqpoint{2.743463in}{2.673629in}}%
\pgfpathlineto{\pgfqpoint{2.724680in}{2.613114in}}%
\pgfpathlineto{\pgfqpoint{2.708012in}{2.312529in}}%
\pgfpathlineto{\pgfqpoint{2.686178in}{2.095619in}}%
\pgfpathlineto{\pgfqpoint{2.667632in}{2.022818in}}%
\pgfpathlineto{\pgfqpoint{2.648850in}{1.974708in}}%
\pgfpathlineto{\pgfqpoint{2.628893in}{1.971950in}}%
\pgfpathlineto{\pgfqpoint{2.609876in}{2.006732in}}%
\pgfpathlineto{\pgfqpoint{2.589217in}{2.099556in}}%
\pgfpathlineto{\pgfqpoint{2.573254in}{2.250341in}}%
\pgfpathlineto{\pgfqpoint{2.551419in}{2.609402in}}%
\pgfpathlineto{\pgfqpoint{2.532637in}{2.780200in}}%
\pgfpathlineto{\pgfqpoint{2.513855in}{2.785770in}}%
\pgfpathlineto{\pgfqpoint{2.496012in}{2.673783in}}%
\pgfpathlineto{\pgfqpoint{2.495778in}{2.786010in}}%
\pgfpathlineto{\pgfqpoint{2.474178in}{2.815841in}}%
\pgfpathlineto{\pgfqpoint{2.458684in}{2.722544in}}%
\pgfpathlineto{\pgfqpoint{2.432858in}{2.289383in}}%
\pgfpathlineto{\pgfqpoint{2.416894in}{2.111306in}}%
\pgfpathlineto{\pgfqpoint{2.399756in}{2.029313in}}%
\pgfpathlineto{\pgfqpoint{2.378626in}{1.976177in}}%
\pgfpathlineto{\pgfqpoint{2.360080in}{1.970514in}}%
\pgfpathlineto{\pgfqpoint{2.341063in}{2.003744in}}%
\pgfpathlineto{\pgfqpoint{2.322515in}{2.057416in}}%
\pgfpathlineto{\pgfqpoint{2.302560in}{2.194254in}}%
\pgfpathlineto{\pgfqpoint{2.282604in}{2.521339in}}%
\pgfpathlineto{\pgfqpoint{2.261944in}{2.806492in}}%
\pgfpathlineto{\pgfqpoint{2.245510in}{2.819504in}}%
\pgfpathlineto{\pgfqpoint{2.223676in}{2.626036in}}%
\pgfpathlineto{\pgfqpoint{2.207242in}{2.341624in}}%
\pgfpathlineto{\pgfqpoint{2.186817in}{2.135377in}}%
\pgfpathlineto{\pgfqpoint{2.168740in}{2.032537in}}%
\pgfpathlineto{\pgfqpoint{2.149723in}{1.983763in}}%
\pgfpathlineto{\pgfqpoint{2.126246in}{1.976394in}}%
\pgfpathlineto{\pgfqpoint{2.110281in}{1.989964in}}%
\pgfpathlineto{\pgfqpoint{2.088449in}{2.042450in}}%
\pgfpathlineto{\pgfqpoint{2.072718in}{2.141027in}}%
\pgfpathlineto{\pgfqpoint{2.050884in}{2.392809in}}%
\pgfpathlineto{\pgfqpoint{2.033511in}{2.613451in}}%
\pgfpathlineto{\pgfqpoint{2.014259in}{2.817319in}}%
\pgfpathlineto{\pgfqpoint{1.995948in}{2.767394in}}%
\pgfpathlineto{\pgfqpoint{1.974583in}{2.481447in}}%
\pgfpathlineto{\pgfqpoint{1.955802in}{2.272470in}}%
\pgfpathlineto{\pgfqpoint{1.937254in}{2.110181in}}%
\pgfpathlineto{\pgfqpoint{1.915420in}{2.011594in}}%
\pgfpathlineto{\pgfqpoint{1.901335in}{1.983238in}}%
\pgfpathlineto{\pgfqpoint{1.879030in}{1.973320in}}%
\pgfpathlineto{\pgfqpoint{1.860015in}{2.001076in}}%
\pgfpathlineto{\pgfqpoint{1.840293in}{2.061198in}}%
\pgfpathlineto{\pgfqpoint{1.821511in}{2.155535in}}%
\pgfpathlineto{\pgfqpoint{1.803199in}{2.319168in}}%
\pgfpathlineto{\pgfqpoint{1.782539in}{2.615100in}}%
\pgfpathlineto{\pgfqpoint{1.764697in}{2.803656in}}%
\pgfpathlineto{\pgfqpoint{1.742863in}{2.805908in}}%
\pgfpathlineto{\pgfqpoint{1.727369in}{2.680798in}}%
\pgfpathlineto{\pgfqpoint{1.705769in}{2.361976in}}%
\pgfpathlineto{\pgfqpoint{1.686987in}{2.156805in}}%
\pgfpathlineto{\pgfqpoint{1.668441in}{2.066643in}}%
\pgfpathlineto{\pgfqpoint{1.650128in}{2.004807in}}%
\pgfpathlineto{\pgfqpoint{1.630407in}{1.977721in}}%
\pgfpathlineto{\pgfqpoint{1.612799in}{1.979089in}}%
\pgfpathlineto{\pgfqpoint{1.592139in}{2.009612in}}%
\pgfpathlineto{\pgfqpoint{1.574061in}{2.081460in}}%
\pgfpathlineto{\pgfqpoint{1.550349in}{2.230842in}}%
\pgfpathlineto{\pgfqpoint{1.533212in}{2.414790in}}%
\pgfpathlineto{\pgfqpoint{1.512552in}{2.718568in}}%
\pgfpathlineto{\pgfqpoint{1.494709in}{2.842403in}}%
\pgfpathlineto{\pgfqpoint{1.476396in}{2.835991in}}%
\pgfpathlineto{\pgfqpoint{1.458553in}{2.674241in}}%
\pgfpathlineto{\pgfqpoint{1.438362in}{2.369011in}}%
\pgfpathlineto{\pgfqpoint{1.419111in}{2.202211in}}%
\pgfpathlineto{\pgfqpoint{1.396105in}{2.348555in}}%
\pgfpathlineto{\pgfqpoint{1.379200in}{2.172317in}}%
\pgfpathlineto{\pgfqpoint{1.358540in}{2.054511in}}%
\pgfpathlineto{\pgfqpoint{1.341872in}{2.000948in}}%
\pgfpathlineto{\pgfqpoint{1.323558in}{1.976164in}}%
\pgfpathlineto{\pgfqpoint{1.304073in}{1.987047in}}%
\pgfpathlineto{\pgfqpoint{1.283647in}{2.032249in}}%
\pgfpathlineto{\pgfqpoint{1.263458in}{2.118995in}}%
\pgfpathlineto{\pgfqpoint{1.246085in}{2.200294in}}%
\pgfpathlineto{\pgfqpoint{1.225190in}{2.482871in}}%
\pgfpathlineto{\pgfqpoint{1.206642in}{2.610037in}}%
\pgfpathlineto{\pgfqpoint{1.186922in}{2.836625in}}%
\pgfpathlineto{\pgfqpoint{1.169783in}{2.877010in}}%
\pgfpathlineto{\pgfqpoint{1.149123in}{2.872998in}}%
\pgfpathlineto{\pgfqpoint{1.131515in}{2.737042in}}%
\pgfpathlineto{\pgfqpoint{1.111561in}{2.409395in}}%
\pgfpathlineto{\pgfqpoint{1.091135in}{2.178199in}}%
\pgfpathlineto{\pgfqpoint{1.070475in}{2.070713in}}%
\pgfpathlineto{\pgfqpoint{1.052396in}{2.020366in}}%
\pgfpathlineto{\pgfqpoint{1.035963in}{1.985752in}}%
\pgfpathlineto{\pgfqpoint{1.015068in}{1.982739in}}%
\pgfpathlineto{\pgfqpoint{0.994877in}{2.026855in}}%
\pgfpathlineto{\pgfqpoint{0.978678in}{2.094696in}}%
\pgfpathlineto{\pgfqpoint{0.957549in}{2.239577in}}%
\pgfpathlineto{\pgfqpoint{0.939707in}{2.467360in}}%
\pgfpathlineto{\pgfqpoint{0.918107in}{2.769505in}}%
\pgfpathlineto{\pgfqpoint{0.897681in}{2.901833in}}%
\pgfpathlineto{\pgfqpoint{0.879604in}{2.885784in}}%
\pgfpathlineto{\pgfqpoint{0.861762in}{2.707330in}}%
\pgfpathlineto{\pgfqpoint{0.824199in}{2.267735in}}%
\pgfpathlineto{\pgfqpoint{0.785461in}{2.067562in}}%
\pgfpathlineto{\pgfqpoint{0.762454in}{2.004154in}}%
\pgfpathlineto{\pgfqpoint{0.744141in}{1.985315in}}%
\pgfpathlineto{\pgfqpoint{0.727003in}{1.998519in}}%
\pgfpathlineto{\pgfqpoint{0.706812in}{2.052236in}}%
\pgfpathlineto{\pgfqpoint{0.688499in}{2.122170in}}%
\pgfpathlineto{\pgfqpoint{0.667136in}{2.316550in}}%
\pgfpathlineto{\pgfqpoint{0.649293in}{2.530174in}}%
\pgfpathlineto{\pgfqpoint{0.650468in}{2.518926in}}%
\pgfpathlineto{\pgfqpoint{0.656337in}{2.393026in}}%
\pgfpathlineto{\pgfqpoint{0.676057in}{2.131602in}}%
\pgfpathlineto{\pgfqpoint{0.695777in}{2.015084in}}%
\pgfpathlineto{\pgfqpoint{0.713856in}{1.984300in}}%
\pgfpathlineto{\pgfqpoint{0.732167in}{2.033346in}}%
\pgfpathlineto{\pgfqpoint{0.754002in}{2.178000in}}%
\pgfpathlineto{\pgfqpoint{0.775836in}{2.521053in}}%
\pgfpathlineto{\pgfqpoint{0.791095in}{2.828361in}}%
\pgfpathlineto{\pgfqpoint{0.809643in}{2.899350in}}%
\pgfpathlineto{\pgfqpoint{0.828894in}{2.648521in}}%
\pgfpathlineto{\pgfqpoint{0.848146in}{2.280719in}}%
\pgfpathlineto{\pgfqpoint{0.865988in}{2.079233in}}%
\pgfpathlineto{\pgfqpoint{0.887588in}{1.990028in}}%
\pgfpathlineto{\pgfqpoint{0.905899in}{1.992021in}}%
\pgfpathlineto{\pgfqpoint{0.924682in}{2.056857in}}%
\pgfpathlineto{\pgfqpoint{0.943698in}{2.193635in}}%
\pgfpathlineto{\pgfqpoint{0.961070in}{2.491114in}}%
\pgfpathlineto{\pgfqpoint{0.982904in}{2.206688in}}%
\pgfpathlineto{\pgfqpoint{1.001217in}{2.050888in}}%
\pgfpathlineto{\pgfqpoint{1.022581in}{1.979130in}}%
\pgfpathlineto{\pgfqpoint{1.040658in}{1.993063in}}%
\pgfpathlineto{\pgfqpoint{1.058502in}{2.062380in}}%
\pgfpathlineto{\pgfqpoint{1.079631in}{2.263678in}}%
\pgfpathlineto{\pgfqpoint{1.100525in}{2.660036in}}%
\pgfpathlineto{\pgfqpoint{1.115550in}{2.854565in}}%
\pgfpathlineto{\pgfqpoint{1.136445in}{2.758194in}}%
\pgfpathlineto{\pgfqpoint{1.160393in}{2.309192in}}%
\pgfpathlineto{\pgfqpoint{1.171661in}{2.128115in}}%
\pgfpathlineto{\pgfqpoint{1.192792in}{2.010421in}}%
\pgfpathlineto{\pgfqpoint{1.216738in}{1.975101in}}%
\pgfpathlineto{\pgfqpoint{1.231529in}{2.012874in}}%
\pgfpathlineto{\pgfqpoint{1.249371in}{2.093932in}}%
\pgfpathlineto{\pgfqpoint{1.270265in}{2.321458in}}%
\pgfpathlineto{\pgfqpoint{1.288813in}{2.645695in}}%
\pgfpathlineto{\pgfqpoint{1.310177in}{2.850295in}}%
\pgfpathlineto{\pgfqpoint{1.330602in}{2.682788in}}%
\pgfpathlineto{\pgfqpoint{1.348210in}{2.330267in}}%
\pgfpathlineto{\pgfqpoint{1.366287in}{2.101907in}}%
\pgfpathlineto{\pgfqpoint{1.387183in}{2.000138in}}%
\pgfpathlineto{\pgfqpoint{1.405495in}{1.972096in}}%
\pgfpathlineto{\pgfqpoint{1.423806in}{1.985543in}}%
\pgfpathlineto{\pgfqpoint{1.443997in}{2.056585in}}%
\pgfpathlineto{\pgfqpoint{1.465362in}{2.223309in}}%
\pgfpathlineto{\pgfqpoint{1.483439in}{2.532117in}}%
\pgfpathlineto{\pgfqpoint{1.501282in}{2.764280in}}%
\pgfpathlineto{\pgfqpoint{1.521942in}{2.831984in}}%
\pgfpathlineto{\pgfqpoint{1.540019in}{2.625694in}}%
\pgfpathlineto{\pgfqpoint{1.558098in}{2.289387in}}%
\pgfpathlineto{\pgfqpoint{1.579227in}{2.065159in}}%
\pgfpathlineto{\pgfqpoint{1.597304in}{2.009537in}}%
\pgfpathlineto{\pgfqpoint{1.615382in}{1.972369in}}%
\pgfpathlineto{\pgfqpoint{1.636042in}{1.984273in}}%
\pgfpathlineto{\pgfqpoint{1.656702in}{2.038379in}}%
\pgfpathlineto{\pgfqpoint{1.674308in}{2.147602in}}%
\pgfpathlineto{\pgfqpoint{1.692387in}{2.374477in}}%
\pgfpathlineto{\pgfqpoint{1.713516in}{2.669815in}}%
\pgfpathlineto{\pgfqpoint{1.730890in}{2.828083in}}%
\pgfpathlineto{\pgfqpoint{1.750610in}{2.743660in}}%
\pgfpathlineto{\pgfqpoint{1.770566in}{2.510309in}}%
\pgfpathlineto{\pgfqpoint{1.790992in}{2.167563in}}%
\pgfpathlineto{\pgfqpoint{1.809772in}{2.035750in}}%
\pgfpathlineto{\pgfqpoint{1.827617in}{1.989661in}}%
\pgfpathlineto{\pgfqpoint{1.848040in}{1.970307in}}%
\pgfpathlineto{\pgfqpoint{1.866353in}{1.989534in}}%
\pgfpathlineto{\pgfqpoint{1.887482in}{2.040607in}}%
\pgfpathlineto{\pgfqpoint{1.903682in}{2.132962in}}%
\pgfpathlineto{\pgfqpoint{1.924342in}{2.363258in}}%
\pgfpathlineto{\pgfqpoint{1.942419in}{2.683335in}}%
\pgfpathlineto{\pgfqpoint{1.963315in}{2.818799in}}%
\pgfpathlineto{\pgfqpoint{1.982095in}{2.697033in}}%
\pgfpathlineto{\pgfqpoint{2.001817in}{2.471688in}}%
\pgfpathlineto{\pgfqpoint{2.019660in}{2.190087in}}%
\pgfpathlineto{\pgfqpoint{2.038676in}{2.063554in}}%
\pgfpathlineto{\pgfqpoint{2.058631in}{2.011548in}}%
\pgfpathlineto{\pgfqpoint{2.079996in}{1.977860in}}%
\pgfpathlineto{\pgfqpoint{2.096196in}{1.969724in}}%
\pgfpathlineto{\pgfqpoint{2.116621in}{2.006547in}}%
\pgfpathlineto{\pgfqpoint{2.133758in}{2.071795in}}%
\pgfpathlineto{\pgfqpoint{2.156532in}{2.255822in}}%
\pgfpathlineto{\pgfqpoint{2.172966in}{2.503795in}}%
\pgfpathlineto{\pgfqpoint{2.193860in}{2.781260in}}%
\pgfpathlineto{\pgfqpoint{2.211703in}{2.794586in}}%
\pgfpathlineto{\pgfqpoint{2.233068in}{2.529143in}}%
\pgfpathlineto{\pgfqpoint{2.251849in}{2.247322in}}%
\pgfpathlineto{\pgfqpoint{2.270396in}{2.085168in}}%
\pgfpathlineto{\pgfqpoint{2.291762in}{2.092541in}}%
\pgfpathlineto{\pgfqpoint{2.309839in}{2.010875in}}%
\pgfpathlineto{\pgfqpoint{2.330733in}{1.969922in}}%
\pgfpathlineto{\pgfqpoint{2.351158in}{1.981093in}}%
\pgfpathlineto{\pgfqpoint{2.366652in}{2.017960in}}%
\pgfpathlineto{\pgfqpoint{2.383323in}{2.104104in}}%
\pgfpathlineto{\pgfqpoint{2.404686in}{2.250215in}}%
\pgfpathlineto{\pgfqpoint{2.425580in}{2.494120in}}%
\pgfpathlineto{\pgfqpoint{2.443423in}{2.751561in}}%
\pgfpathlineto{\pgfqpoint{2.464788in}{2.811862in}}%
\pgfpathlineto{\pgfqpoint{2.482162in}{2.729269in}}%
\pgfpathlineto{\pgfqpoint{2.500239in}{2.534072in}}%
\pgfpathlineto{\pgfqpoint{2.518786in}{2.273538in}}%
\pgfpathlineto{\pgfqpoint{2.539446in}{2.122480in}}%
\pgfpathlineto{\pgfqpoint{2.557289in}{2.017865in}}%
\pgfpathlineto{\pgfqpoint{2.581938in}{1.971709in}}%
\pgfpathlineto{\pgfqpoint{2.597434in}{1.971921in}}%
\pgfpathlineto{\pgfqpoint{2.614573in}{2.009246in}}%
\pgfpathlineto{\pgfqpoint{2.636877in}{2.077683in}}%
\pgfpathlineto{\pgfqpoint{2.654250in}{2.237438in}}%
\pgfpathlineto{\pgfqpoint{2.675379in}{2.544383in}}%
\pgfpathlineto{\pgfqpoint{2.693690in}{2.779316in}}%
\pgfpathlineto{\pgfqpoint{2.711533in}{2.797954in}}%
\pgfpathlineto{\pgfqpoint{2.749801in}{2.311460in}}%
\pgfpathlineto{\pgfqpoint{2.768583in}{2.143878in}}%
\pgfpathlineto{\pgfqpoint{2.789478in}{2.034394in}}%
\pgfpathlineto{\pgfqpoint{2.807791in}{1.989334in}}%
\pgfpathlineto{\pgfqpoint{2.828920in}{1.968291in}}%
\pgfpathlineto{\pgfqpoint{2.846293in}{1.987145in}}%
\pgfpathlineto{\pgfqpoint{2.864605in}{2.020496in}}%
\pgfpathlineto{\pgfqpoint{2.886204in}{2.110964in}}%
\pgfpathlineto{\pgfqpoint{2.903813in}{2.285783in}}%
\pgfpathlineto{\pgfqpoint{2.921655in}{2.448680in}}%
\pgfpathlineto{\pgfqpoint{2.943724in}{2.765236in}}%
\pgfpathlineto{\pgfqpoint{2.963915in}{2.808092in}}%
\pgfpathlineto{\pgfqpoint{2.981523in}{2.721858in}}%
\pgfpathlineto{\pgfqpoint{3.000303in}{2.552611in}}%
\pgfpathlineto{\pgfqpoint{3.017677in}{2.275712in}}%
\pgfpathlineto{\pgfqpoint{3.039745in}{2.082380in}}%
\pgfpathlineto{\pgfqpoint{3.056413in}{2.023344in}}%
\pgfpathlineto{\pgfqpoint{3.078719in}{1.978287in}}%
\pgfpathlineto{\pgfqpoint{3.096561in}{1.973278in}}%
\pgfpathlineto{\pgfqpoint{3.117455in}{2.002180in}}%
\pgfpathlineto{\pgfqpoint{3.136238in}{2.056936in}}%
\pgfpathlineto{\pgfqpoint{3.154549in}{2.122355in}}%
\pgfpathlineto{\pgfqpoint{3.173097in}{2.320467in}}%
\pgfpathlineto{\pgfqpoint{3.191174in}{2.595821in}}%
\pgfpathlineto{\pgfqpoint{3.212303in}{2.819661in}}%
\pgfpathlineto{\pgfqpoint{3.231319in}{2.819953in}}%
\pgfpathlineto{\pgfqpoint{3.247050in}{2.609956in}}%
\pgfpathlineto{\pgfqpoint{3.272170in}{2.306543in}}%
\pgfpathlineto{\pgfqpoint{3.288839in}{2.166677in}}%
\pgfpathlineto{\pgfqpoint{3.307621in}{2.059657in}}%
\pgfpathlineto{\pgfqpoint{3.328046in}{2.015370in}}%
\pgfpathlineto{\pgfqpoint{3.346358in}{2.009114in}}%
\pgfpathlineto{\pgfqpoint{3.363966in}{1.974961in}}%
\pgfpathlineto{\pgfqpoint{3.385566in}{1.987632in}}%
\pgfpathlineto{\pgfqpoint{3.403643in}{2.030907in}}%
\pgfpathlineto{\pgfqpoint{3.427120in}{2.111721in}}%
\pgfpathlineto{\pgfqpoint{3.443319in}{2.273188in}}%
\pgfpathlineto{\pgfqpoint{3.463979in}{2.519675in}}%
\pgfpathlineto{\pgfqpoint{3.482291in}{2.778583in}}%
\pgfpathlineto{\pgfqpoint{3.499899in}{2.858115in}}%
\pgfpathlineto{\pgfqpoint{3.518446in}{2.783410in}}%
\pgfpathlineto{\pgfqpoint{3.538637in}{2.479268in}}%
\pgfpathlineto{\pgfqpoint{3.557889in}{2.226507in}}%
\pgfpathlineto{\pgfqpoint{3.577843in}{2.093087in}}%
\pgfpathlineto{\pgfqpoint{3.614939in}{1.989379in}}%
\pgfpathlineto{\pgfqpoint{3.635362in}{1.988951in}}%
\pgfpathlineto{\pgfqpoint{3.652970in}{1.980871in}}%
\pgfpathlineto{\pgfqpoint{3.674336in}{2.022530in}}%
\pgfpathlineto{\pgfqpoint{3.693118in}{2.095566in}}%
\pgfpathlineto{\pgfqpoint{3.711898in}{2.240403in}}%
\pgfpathlineto{\pgfqpoint{3.729272in}{2.458516in}}%
\pgfpathlineto{\pgfqpoint{3.748992in}{2.601563in}}%
\pgfpathlineto{\pgfqpoint{3.770123in}{2.854589in}}%
\pgfpathlineto{\pgfqpoint{3.789608in}{2.863432in}}%
\pgfpathlineto{\pgfqpoint{3.809565in}{2.701413in}}%
\pgfpathlineto{\pgfqpoint{3.828582in}{2.435400in}}%
\pgfpathlineto{\pgfqpoint{3.845016in}{2.231120in}}%
\pgfpathlineto{\pgfqpoint{3.864267in}{2.101563in}}%
\pgfpathlineto{\pgfqpoint{3.889387in}{2.005840in}}%
\pgfpathlineto{\pgfqpoint{3.905116in}{1.988898in}}%
\pgfpathlineto{\pgfqpoint{3.924838in}{1.977729in}}%
\pgfpathlineto{\pgfqpoint{3.939394in}{1.998874in}}%
\pgfpathlineto{\pgfqpoint{3.961931in}{2.056804in}}%
\pgfpathlineto{\pgfqpoint{3.980948in}{2.159039in}}%
\pgfpathlineto{\pgfqpoint{3.999965in}{2.283062in}}%
\pgfpathlineto{\pgfqpoint{4.019216in}{2.553934in}}%
\pgfpathlineto{\pgfqpoint{4.038233in}{2.797611in}}%
\pgfpathlineto{\pgfqpoint{4.057484in}{2.894975in}}%
\pgfpathlineto{\pgfqpoint{4.076030in}{2.895077in}}%
\pgfpathlineto{\pgfqpoint{4.098570in}{2.767263in}}%
\pgfpathlineto{\pgfqpoint{4.116412in}{2.568957in}}%
\pgfpathlineto{\pgfqpoint{4.134960in}{2.323196in}}%
\pgfpathlineto{\pgfqpoint{4.154446in}{2.161021in}}%
\pgfpathlineto{\pgfqpoint{4.173462in}{2.097365in}}%
\pgfpathlineto{\pgfqpoint{4.189896in}{2.027628in}}%
\pgfpathlineto{\pgfqpoint{4.211494in}{1.987096in}}%
\pgfpathlineto{\pgfqpoint{4.231919in}{1.994407in}}%
\pgfpathlineto{\pgfqpoint{4.250467in}{2.049812in}}%
\pgfpathlineto{\pgfqpoint{4.268075in}{2.901014in}}%
\pgfpathlineto{\pgfqpoint{4.290378in}{2.618440in}}%
\pgfpathlineto{\pgfqpoint{4.309629in}{2.305744in}}%
\pgfpathlineto{\pgfqpoint{4.324889in}{2.131432in}}%
\pgfpathlineto{\pgfqpoint{4.350010in}{2.023133in}}%
\pgfpathlineto{\pgfqpoint{4.365271in}{1.986391in}}%
\pgfpathlineto{\pgfqpoint{4.384991in}{2.004005in}}%
\pgfpathlineto{\pgfqpoint{4.404948in}{2.077282in}}%
\pgfpathlineto{\pgfqpoint{4.422790in}{2.212385in}}%
\pgfpathlineto{\pgfqpoint{4.441336in}{2.432721in}}%
\pgfpathlineto{\pgfqpoint{4.460353in}{2.764337in}}%
\pgfpathlineto{\pgfqpoint{4.479604in}{2.937732in}}%
\pgfpathlineto{\pgfqpoint{4.474206in}{2.905299in}}%
\pgfpathlineto{\pgfqpoint{4.436875in}{2.220759in}}%
\pgfpathlineto{\pgfqpoint{4.416686in}{2.059718in}}%
\pgfpathlineto{\pgfqpoint{4.396495in}{1.987488in}}%
\pgfpathlineto{\pgfqpoint{4.378184in}{2.011150in}}%
\pgfpathlineto{\pgfqpoint{4.357993in}{2.135570in}}%
\pgfpathlineto{\pgfqpoint{4.339211in}{2.389293in}}%
\pgfpathlineto{\pgfqpoint{4.320428in}{2.790040in}}%
\pgfpathlineto{\pgfqpoint{4.302586in}{2.915908in}}%
\pgfpathlineto{\pgfqpoint{4.281926in}{2.638309in}}%
\pgfpathlineto{\pgfqpoint{4.260797in}{2.226218in}}%
\pgfpathlineto{\pgfqpoint{4.242484in}{2.062589in}}%
\pgfpathlineto{\pgfqpoint{4.224641in}{1.990542in}}%
\pgfpathlineto{\pgfqpoint{4.206330in}{1.996090in}}%
\pgfpathlineto{\pgfqpoint{4.186844in}{2.077363in}}%
\pgfpathlineto{\pgfqpoint{4.166419in}{2.315392in}}%
\pgfpathlineto{\pgfqpoint{4.146697in}{2.694176in}}%
\pgfpathlineto{\pgfqpoint{4.128151in}{2.891472in}}%
\pgfpathlineto{\pgfqpoint{4.109837in}{2.724548in}}%
\pgfpathlineto{\pgfqpoint{4.092935in}{2.338832in}}%
\pgfpathlineto{\pgfqpoint{4.067580in}{2.072121in}}%
\pgfpathlineto{\pgfqpoint{4.051380in}{2.005764in}}%
\pgfpathlineto{\pgfqpoint{4.033301in}{1.979041in}}%
\pgfpathlineto{\pgfqpoint{4.013112in}{2.025119in}}%
\pgfpathlineto{\pgfqpoint{3.994096in}{2.134489in}}%
\pgfpathlineto{\pgfqpoint{3.972730in}{2.462153in}}%
\pgfpathlineto{\pgfqpoint{3.954185in}{2.764741in}}%
\pgfpathlineto{\pgfqpoint{3.938454in}{2.869477in}}%
\pgfpathlineto{\pgfqpoint{3.917091in}{2.609194in}}%
\pgfpathlineto{\pgfqpoint{3.899483in}{2.246425in}}%
\pgfpathlineto{\pgfqpoint{3.875535in}{2.065279in}}%
\pgfpathlineto{\pgfqpoint{3.860275in}{2.001501in}}%
\pgfpathlineto{\pgfqpoint{3.841024in}{1.973756in}}%
\pgfpathlineto{\pgfqpoint{3.822712in}{2.004008in}}%
\pgfpathlineto{\pgfqpoint{3.801816in}{2.064382in}}%
\pgfpathlineto{\pgfqpoint{3.779044in}{2.266332in}}%
\pgfpathlineto{\pgfqpoint{3.764722in}{2.529992in}}%
\pgfpathlineto{\pgfqpoint{3.741011in}{2.819303in}}%
\pgfpathlineto{\pgfqpoint{3.721525in}{2.806413in}}%
\pgfpathlineto{\pgfqpoint{3.684431in}{2.224935in}}%
\pgfpathlineto{\pgfqpoint{3.665883in}{2.093375in}}%
\pgfpathlineto{\pgfqpoint{3.647336in}{2.788207in}}%
\pgfpathlineto{\pgfqpoint{3.628321in}{2.818414in}}%
\pgfpathlineto{\pgfqpoint{3.610947in}{2.527283in}}%
\pgfpathlineto{\pgfqpoint{3.590522in}{2.206302in}}%
\pgfpathlineto{\pgfqpoint{3.571739in}{2.057030in}}%
\pgfpathlineto{\pgfqpoint{3.549436in}{1.988921in}}%
\pgfpathlineto{\pgfqpoint{3.532063in}{1.973822in}}%
\pgfpathlineto{\pgfqpoint{3.513751in}{2.006610in}}%
\pgfpathlineto{\pgfqpoint{3.494969in}{2.093686in}}%
\pgfpathlineto{\pgfqpoint{3.473369in}{2.355258in}}%
\pgfpathlineto{\pgfqpoint{3.454589in}{2.699562in}}%
\pgfpathlineto{\pgfqpoint{3.437450in}{2.836176in}}%
\pgfpathlineto{\pgfqpoint{3.417259in}{2.687491in}}%
\pgfpathlineto{\pgfqpoint{3.399887in}{2.375547in}}%
\pgfpathlineto{\pgfqpoint{3.374061in}{2.141980in}}%
\pgfpathlineto{\pgfqpoint{3.359740in}{2.052140in}}%
\pgfpathlineto{\pgfqpoint{3.338142in}{1.985466in}}%
\pgfpathlineto{\pgfqpoint{3.319829in}{1.972121in}}%
\pgfpathlineto{\pgfqpoint{3.300577in}{2.006743in}}%
\pgfpathlineto{\pgfqpoint{3.283204in}{2.063809in}}%
\pgfpathlineto{\pgfqpoint{3.263015in}{2.208635in}}%
\pgfpathlineto{\pgfqpoint{3.246110in}{2.419041in}}%
\pgfpathlineto{\pgfqpoint{3.224981in}{2.779190in}}%
\pgfpathlineto{\pgfqpoint{3.207842in}{2.817562in}}%
\pgfpathlineto{\pgfqpoint{3.185070in}{2.530275in}}%
\pgfpathlineto{\pgfqpoint{3.166053in}{2.243837in}}%
\pgfpathlineto{\pgfqpoint{3.147976in}{2.077844in}}%
\pgfpathlineto{\pgfqpoint{3.129663in}{2.006111in}}%
\pgfpathlineto{\pgfqpoint{3.107360in}{1.970287in}}%
\pgfpathlineto{\pgfqpoint{3.092569in}{1.982915in}}%
\pgfpathlineto{\pgfqpoint{3.069797in}{2.039760in}}%
\pgfpathlineto{\pgfqpoint{3.051015in}{2.144724in}}%
\pgfpathlineto{\pgfqpoint{3.032938in}{2.395679in}}%
\pgfpathlineto{\pgfqpoint{3.014156in}{2.590575in}}%
\pgfpathlineto{\pgfqpoint{2.995842in}{2.760584in}}%
\pgfpathlineto{\pgfqpoint{2.978471in}{2.800676in}}%
\pgfpathlineto{\pgfqpoint{2.955228in}{2.510632in}}%
\pgfpathlineto{\pgfqpoint{2.938089in}{2.239539in}}%
\pgfpathlineto{\pgfqpoint{2.915317in}{2.070266in}}%
\pgfpathlineto{\pgfqpoint{2.899586in}{2.004088in}}%
\pgfpathlineto{\pgfqpoint{2.881039in}{1.969233in}}%
\pgfpathlineto{\pgfqpoint{2.858970in}{1.974015in}}%
\pgfpathlineto{\pgfqpoint{2.840893in}{2.012124in}}%
\pgfpathlineto{\pgfqpoint{2.822111in}{2.101902in}}%
\pgfpathlineto{\pgfqpoint{2.800042in}{2.341519in}}%
\pgfpathlineto{\pgfqpoint{2.785017in}{2.596896in}}%
\pgfpathlineto{\pgfqpoint{2.759427in}{2.814698in}}%
\pgfpathlineto{\pgfqpoint{2.747454in}{2.769970in}}%
\pgfpathlineto{\pgfqpoint{2.725855in}{2.499751in}}%
\pgfpathlineto{\pgfqpoint{2.705900in}{2.220937in}}%
\pgfpathlineto{\pgfqpoint{2.685943in}{2.070775in}}%
\pgfpathlineto{\pgfqpoint{2.668101in}{1.998438in}}%
\pgfpathlineto{\pgfqpoint{2.645563in}{1.969951in}}%
\pgfpathlineto{\pgfqpoint{2.626547in}{1.969909in}}%
\pgfpathlineto{\pgfqpoint{2.608939in}{2.000923in}}%
\pgfpathlineto{\pgfqpoint{2.590391in}{2.055969in}}%
\pgfpathlineto{\pgfqpoint{2.571374in}{2.199888in}}%
\pgfpathlineto{\pgfqpoint{2.553297in}{2.483813in}}%
\pgfpathlineto{\pgfqpoint{2.533577in}{2.494535in}}%
\pgfpathlineto{\pgfqpoint{2.515734in}{2.771016in}}%
\pgfpathlineto{\pgfqpoint{2.494369in}{2.786041in}}%
\pgfpathlineto{\pgfqpoint{2.475118in}{2.570693in}}%
\pgfpathlineto{\pgfqpoint{2.458684in}{2.296148in}}%
\pgfpathlineto{\pgfqpoint{2.437085in}{2.095570in}}%
\pgfpathlineto{\pgfqpoint{2.417833in}{2.033647in}}%
\pgfpathlineto{\pgfqpoint{2.399051in}{1.982568in}}%
\pgfpathlineto{\pgfqpoint{2.379331in}{1.968425in}}%
\pgfpathlineto{\pgfqpoint{2.359845in}{2.004052in}}%
\pgfpathlineto{\pgfqpoint{2.338951in}{2.084674in}}%
\pgfpathlineto{\pgfqpoint{2.320637in}{2.262352in}}%
\pgfpathlineto{\pgfqpoint{2.303264in}{2.539077in}}%
\pgfpathlineto{\pgfqpoint{2.285421in}{2.791894in}}%
\pgfpathlineto{\pgfqpoint{2.265467in}{2.810792in}}%
\pgfpathlineto{\pgfqpoint{2.244572in}{2.593175in}}%
\pgfpathlineto{\pgfqpoint{2.222033in}{2.249364in}}%
\pgfpathlineto{\pgfqpoint{2.200199in}{2.131680in}}%
\pgfpathlineto{\pgfqpoint{2.189400in}{2.060512in}}%
\pgfpathlineto{\pgfqpoint{2.165922in}{1.999803in}}%
\pgfpathlineto{\pgfqpoint{2.147611in}{1.973097in}}%
\pgfpathlineto{\pgfqpoint{2.127654in}{1.992775in}}%
\pgfpathlineto{\pgfqpoint{2.111690in}{2.026875in}}%
\pgfpathlineto{\pgfqpoint{2.090326in}{2.132230in}}%
\pgfpathlineto{\pgfqpoint{2.070135in}{2.287466in}}%
\pgfpathlineto{\pgfqpoint{2.051824in}{2.585944in}}%
\pgfpathlineto{\pgfqpoint{2.033276in}{2.775121in}}%
\pgfpathlineto{\pgfqpoint{2.014730in}{2.818923in}}%
\pgfpathlineto{\pgfqpoint{1.996886in}{2.758027in}}%
\pgfpathlineto{\pgfqpoint{1.972471in}{2.361186in}}%
\pgfpathlineto{\pgfqpoint{1.956037in}{2.290228in}}%
\pgfpathlineto{\pgfqpoint{1.937020in}{2.130069in}}%
\pgfpathlineto{\pgfqpoint{1.918707in}{2.038020in}}%
\pgfpathlineto{\pgfqpoint{1.900864in}{1.985967in}}%
\pgfpathlineto{\pgfqpoint{1.878327in}{1.973963in}}%
\pgfpathlineto{\pgfqpoint{1.860250in}{2.000334in}}%
\pgfpathlineto{\pgfqpoint{1.844754in}{2.053577in}}%
\pgfpathlineto{\pgfqpoint{1.823625in}{2.176348in}}%
\pgfpathlineto{\pgfqpoint{1.801791in}{2.472650in}}%
\pgfpathlineto{\pgfqpoint{1.783243in}{2.681151in}}%
\pgfpathlineto{\pgfqpoint{1.765166in}{2.823250in}}%
\pgfpathlineto{\pgfqpoint{1.746620in}{2.795625in}}%
\pgfpathlineto{\pgfqpoint{1.726664in}{2.585812in}}%
\pgfpathlineto{\pgfqpoint{1.706943in}{2.289738in}}%
\pgfpathlineto{\pgfqpoint{1.688396in}{2.136496in}}%
\pgfpathlineto{\pgfqpoint{1.666327in}{2.067897in}}%
\pgfpathlineto{\pgfqpoint{1.651536in}{2.015007in}}%
\pgfpathlineto{\pgfqpoint{1.629233in}{1.976762in}}%
\pgfpathlineto{\pgfqpoint{1.607868in}{1.986094in}}%
\pgfpathlineto{\pgfqpoint{1.589791in}{2.009623in}}%
\pgfpathlineto{\pgfqpoint{1.571245in}{2.074246in}}%
\pgfpathlineto{\pgfqpoint{1.553166in}{2.203483in}}%
\pgfpathlineto{\pgfqpoint{1.534149in}{2.416903in}}%
\pgfpathlineto{\pgfqpoint{1.514664in}{2.720991in}}%
\pgfpathlineto{\pgfqpoint{1.497056in}{2.816159in}}%
\pgfpathlineto{\pgfqpoint{1.477101in}{2.851022in}}%
\pgfpathlineto{\pgfqpoint{1.457614in}{2.761425in}}%
\pgfpathlineto{\pgfqpoint{1.438128in}{2.467423in}}%
\pgfpathlineto{\pgfqpoint{1.417468in}{2.257471in}}%
\pgfpathlineto{\pgfqpoint{1.399391in}{2.116518in}}%
\pgfpathlineto{\pgfqpoint{1.359009in}{2.010774in}}%
\pgfpathlineto{\pgfqpoint{1.341403in}{1.979597in}}%
\pgfpathlineto{\pgfqpoint{1.324264in}{1.981602in}}%
\pgfpathlineto{\pgfqpoint{1.303838in}{1.988034in}}%
\pgfpathlineto{\pgfqpoint{1.281066in}{2.027167in}}%
\pgfpathlineto{\pgfqpoint{1.265805in}{2.091790in}}%
\pgfpathlineto{\pgfqpoint{1.246085in}{2.260813in}}%
\pgfpathlineto{\pgfqpoint{1.224485in}{2.574484in}}%
\pgfpathlineto{\pgfqpoint{1.207817in}{2.715572in}}%
\pgfpathlineto{\pgfqpoint{1.187391in}{2.872321in}}%
\pgfpathlineto{\pgfqpoint{1.163445in}{2.820228in}}%
\pgfpathlineto{\pgfqpoint{1.148654in}{2.667395in}}%
\pgfpathlineto{\pgfqpoint{1.131046in}{2.410859in}}%
\pgfpathlineto{\pgfqpoint{1.110855in}{2.181940in}}%
\pgfpathlineto{\pgfqpoint{1.090195in}{2.077554in}}%
\pgfpathlineto{\pgfqpoint{1.050519in}{1.995136in}}%
\pgfpathlineto{\pgfqpoint{1.029859in}{1.982069in}}%
\pgfpathlineto{\pgfqpoint{1.009433in}{2.009074in}}%
\pgfpathlineto{\pgfqpoint{0.994408in}{2.047964in}}%
\pgfpathlineto{\pgfqpoint{0.975861in}{2.114064in}}%
\pgfpathlineto{\pgfqpoint{0.954732in}{2.238083in}}%
\pgfpathlineto{\pgfqpoint{0.938063in}{2.434397in}}%
\pgfpathlineto{\pgfqpoint{0.920455in}{2.709580in}}%
\pgfpathlineto{\pgfqpoint{0.902378in}{2.867792in}}%
\pgfpathlineto{\pgfqpoint{0.881718in}{2.903400in}}%
\pgfpathlineto{\pgfqpoint{0.860588in}{2.729455in}}%
\pgfpathlineto{\pgfqpoint{0.842511in}{2.469451in}}%
\pgfpathlineto{\pgfqpoint{0.822554in}{2.230962in}}%
\pgfpathlineto{\pgfqpoint{0.802834in}{2.098009in}}%
\pgfpathlineto{\pgfqpoint{0.783583in}{2.044978in}}%
\pgfpathlineto{\pgfqpoint{0.765506in}{2.008445in}}%
\pgfpathlineto{\pgfqpoint{0.747427in}{1.986768in}}%
\pgfpathlineto{\pgfqpoint{0.725124in}{2.009994in}}%
\pgfpathlineto{\pgfqpoint{0.707047in}{2.057418in}}%
\pgfpathlineto{\pgfqpoint{0.690144in}{2.152731in}}%
\pgfpathlineto{\pgfqpoint{0.669013in}{2.302106in}}%
\pgfpathlineto{\pgfqpoint{0.651405in}{2.495201in}}%
\pgfpathlineto{\pgfqpoint{0.650702in}{2.483993in}}%
\pgfpathlineto{\pgfqpoint{0.653754in}{2.393474in}}%
\pgfpathlineto{\pgfqpoint{0.676057in}{2.108920in}}%
\pgfpathlineto{\pgfqpoint{0.695074in}{2.010136in}}%
\pgfpathlineto{\pgfqpoint{0.713620in}{1.988568in}}%
\pgfpathlineto{\pgfqpoint{0.733107in}{2.045973in}}%
\pgfpathlineto{\pgfqpoint{0.750244in}{2.176161in}}%
\pgfpathlineto{\pgfqpoint{0.769967in}{2.454364in}}%
\pgfpathlineto{\pgfqpoint{0.794616in}{2.887830in}}%
\pgfpathlineto{\pgfqpoint{0.810347in}{2.884590in}}%
\pgfpathlineto{\pgfqpoint{0.810816in}{2.753453in}}%
\pgfpathlineto{\pgfqpoint{0.830303in}{2.606321in}}%
\pgfpathlineto{\pgfqpoint{0.847440in}{2.254120in}}%
\pgfpathlineto{\pgfqpoint{0.866223in}{2.069539in}}%
\pgfpathlineto{\pgfqpoint{0.887588in}{1.987284in}}%
\pgfpathlineto{\pgfqpoint{0.906134in}{1.995042in}}%
\pgfpathlineto{\pgfqpoint{0.924447in}{2.061889in}}%
\pgfpathlineto{\pgfqpoint{0.943698in}{2.216405in}}%
\pgfpathlineto{\pgfqpoint{0.964593in}{2.579602in}}%
\pgfpathlineto{\pgfqpoint{0.982670in}{2.861885in}}%
\pgfpathlineto{\pgfqpoint{1.001217in}{2.829103in}}%
\pgfpathlineto{\pgfqpoint{1.021877in}{2.464559in}}%
\pgfpathlineto{\pgfqpoint{1.040189in}{2.179035in}}%
\pgfpathlineto{\pgfqpoint{1.057797in}{2.041615in}}%
\pgfpathlineto{\pgfqpoint{1.078691in}{1.979603in}}%
\pgfpathlineto{\pgfqpoint{1.097474in}{1.995529in}}%
\pgfpathlineto{\pgfqpoint{1.115082in}{2.060587in}}%
\pgfpathlineto{\pgfqpoint{1.134333in}{2.238770in}}%
\pgfpathlineto{\pgfqpoint{1.156636in}{2.640382in}}%
\pgfpathlineto{\pgfqpoint{1.174478in}{2.857347in}}%
\pgfpathlineto{\pgfqpoint{1.192321in}{2.780844in}}%
\pgfpathlineto{\pgfqpoint{1.212746in}{2.391751in}}%
\pgfpathlineto{\pgfqpoint{1.233172in}{2.130265in}}%
\pgfpathlineto{\pgfqpoint{1.249606in}{2.019358in}}%
\pgfpathlineto{\pgfqpoint{1.270971in}{1.974132in}}%
\pgfpathlineto{\pgfqpoint{1.291865in}{2.002111in}}%
\pgfpathlineto{\pgfqpoint{1.311351in}{2.090433in}}%
\pgfpathlineto{\pgfqpoint{1.327316in}{2.230472in}}%
\pgfpathlineto{\pgfqpoint{1.353140in}{2.683020in}}%
\pgfpathlineto{\pgfqpoint{1.365349in}{2.814212in}}%
\pgfpathlineto{\pgfqpoint{1.387887in}{2.778377in}}%
\pgfpathlineto{\pgfqpoint{1.404086in}{2.464966in}}%
\pgfpathlineto{\pgfqpoint{1.425451in}{2.146341in}}%
\pgfpathlineto{\pgfqpoint{1.444703in}{2.022418in}}%
\pgfpathlineto{\pgfqpoint{1.461605in}{1.981105in}}%
\pgfpathlineto{\pgfqpoint{1.481796in}{1.980458in}}%
\pgfpathlineto{\pgfqpoint{1.500579in}{2.029299in}}%
\pgfpathlineto{\pgfqpoint{1.518655in}{2.131601in}}%
\pgfpathlineto{\pgfqpoint{1.539550in}{2.388837in}}%
\pgfpathlineto{\pgfqpoint{1.557392in}{2.721147in}}%
\pgfpathlineto{\pgfqpoint{1.579930in}{2.822985in}}%
\pgfpathlineto{\pgfqpoint{1.598009in}{2.608405in}}%
\pgfpathlineto{\pgfqpoint{1.618903in}{2.315573in}}%
\pgfpathlineto{\pgfqpoint{1.636042in}{2.107566in}}%
\pgfpathlineto{\pgfqpoint{1.652476in}{2.017007in}}%
\pgfpathlineto{\pgfqpoint{1.673136in}{1.972699in}}%
\pgfpathlineto{\pgfqpoint{1.694265in}{1.988332in}}%
\pgfpathlineto{\pgfqpoint{1.712342in}{2.037028in}}%
\pgfpathlineto{\pgfqpoint{1.733473in}{2.130159in}}%
\pgfpathlineto{\pgfqpoint{1.752253in}{2.334537in}}%
\pgfpathlineto{\pgfqpoint{1.769392in}{2.577570in}}%
\pgfpathlineto{\pgfqpoint{1.790286in}{2.820342in}}%
\pgfpathlineto{\pgfqpoint{1.808600in}{2.791491in}}%
\pgfpathlineto{\pgfqpoint{1.826208in}{2.541905in}}%
\pgfpathlineto{\pgfqpoint{1.847337in}{2.195727in}}%
\pgfpathlineto{\pgfqpoint{1.865414in}{2.044709in}}%
\pgfpathlineto{\pgfqpoint{1.886074in}{1.992803in}}%
\pgfpathlineto{\pgfqpoint{1.905090in}{1.971195in}}%
\pgfpathlineto{\pgfqpoint{1.925047in}{1.989244in}}%
\pgfpathlineto{\pgfqpoint{1.944064in}{2.048315in}}%
\pgfpathlineto{\pgfqpoint{1.965192in}{2.199532in}}%
\pgfpathlineto{\pgfqpoint{1.979043in}{2.435794in}}%
\pgfpathlineto{\pgfqpoint{2.000643in}{2.663938in}}%
\pgfpathlineto{\pgfqpoint{2.021772in}{2.822087in}}%
\pgfpathlineto{\pgfqpoint{2.038911in}{2.778483in}}%
\pgfpathlineto{\pgfqpoint{2.056753in}{2.487223in}}%
\pgfpathlineto{\pgfqpoint{2.078822in}{2.183959in}}%
\pgfpathlineto{\pgfqpoint{2.099717in}{2.031919in}}%
\pgfpathlineto{\pgfqpoint{2.118030in}{1.982906in}}%
\pgfpathlineto{\pgfqpoint{2.135638in}{1.968467in}}%
\pgfpathlineto{\pgfqpoint{2.153715in}{1.999935in}}%
\pgfpathlineto{\pgfqpoint{2.175078in}{2.076602in}}%
\pgfpathlineto{\pgfqpoint{2.192452in}{2.226046in}}%
\pgfpathlineto{\pgfqpoint{2.213817in}{2.553589in}}%
\pgfpathlineto{\pgfqpoint{2.231189in}{2.627400in}}%
\pgfpathlineto{\pgfqpoint{2.255840in}{2.105618in}}%
\pgfpathlineto{\pgfqpoint{2.269693in}{2.265746in}}%
\pgfpathlineto{\pgfqpoint{2.288944in}{2.458728in}}%
\pgfpathlineto{\pgfqpoint{2.309133in}{2.789171in}}%
\pgfpathlineto{\pgfqpoint{2.326038in}{2.796245in}}%
\pgfpathlineto{\pgfqpoint{2.367592in}{2.163946in}}%
\pgfpathlineto{\pgfqpoint{2.384731in}{2.032186in}}%
\pgfpathlineto{\pgfqpoint{2.405626in}{1.978218in}}%
\pgfpathlineto{\pgfqpoint{2.423937in}{1.970605in}}%
\pgfpathlineto{\pgfqpoint{2.444597in}{2.011456in}}%
\pgfpathlineto{\pgfqpoint{2.462205in}{2.091938in}}%
\pgfpathlineto{\pgfqpoint{2.480753in}{2.273571in}}%
\pgfpathlineto{\pgfqpoint{2.518081in}{2.773615in}}%
\pgfpathlineto{\pgfqpoint{2.538272in}{2.790303in}}%
\pgfpathlineto{\pgfqpoint{2.559166in}{2.559836in}}%
\pgfpathlineto{\pgfqpoint{2.577243in}{2.257298in}}%
\pgfpathlineto{\pgfqpoint{2.595322in}{2.084901in}}%
\pgfpathlineto{\pgfqpoint{2.615982in}{1.999586in}}%
\pgfpathlineto{\pgfqpoint{2.633825in}{1.969879in}}%
\pgfpathlineto{\pgfqpoint{2.655659in}{1.984239in}}%
\pgfpathlineto{\pgfqpoint{2.673031in}{2.035474in}}%
\pgfpathlineto{\pgfqpoint{2.691109in}{2.111139in}}%
\pgfpathlineto{\pgfqpoint{2.712707in}{2.345418in}}%
\pgfpathlineto{\pgfqpoint{2.733838in}{2.664015in}}%
\pgfpathlineto{\pgfqpoint{2.751680in}{2.805955in}}%
\pgfpathlineto{\pgfqpoint{2.769992in}{2.764575in}}%
\pgfpathlineto{\pgfqpoint{2.812486in}{2.151977in}}%
\pgfpathlineto{\pgfqpoint{2.827277in}{2.055972in}}%
\pgfpathlineto{\pgfqpoint{2.848640in}{1.986101in}}%
\pgfpathlineto{\pgfqpoint{2.866484in}{1.973313in}}%
\pgfpathlineto{\pgfqpoint{2.884092in}{1.975190in}}%
\pgfpathlineto{\pgfqpoint{2.906161in}{2.029543in}}%
\pgfpathlineto{\pgfqpoint{2.922829in}{2.124600in}}%
\pgfpathlineto{\pgfqpoint{2.946072in}{2.274468in}}%
\pgfpathlineto{\pgfqpoint{2.959454in}{2.482235in}}%
\pgfpathlineto{\pgfqpoint{2.980817in}{2.743088in}}%
\pgfpathlineto{\pgfqpoint{3.002183in}{2.812765in}}%
\pgfpathlineto{\pgfqpoint{3.019085in}{2.668823in}}%
\pgfpathlineto{\pgfqpoint{3.037399in}{2.348715in}}%
\pgfpathlineto{\pgfqpoint{3.056179in}{2.132630in}}%
\pgfpathlineto{\pgfqpoint{3.075901in}{2.029880in}}%
\pgfpathlineto{\pgfqpoint{3.098204in}{1.984399in}}%
\pgfpathlineto{\pgfqpoint{3.115107in}{1.970638in}}%
\pgfpathlineto{\pgfqpoint{3.132949in}{1.990179in}}%
\pgfpathlineto{\pgfqpoint{3.154549in}{2.044372in}}%
\pgfpathlineto{\pgfqpoint{3.174740in}{2.150507in}}%
\pgfpathlineto{\pgfqpoint{3.193757in}{2.346471in}}%
\pgfpathlineto{\pgfqpoint{3.212068in}{2.651113in}}%
\pgfpathlineto{\pgfqpoint{3.232259in}{2.826136in}}%
\pgfpathlineto{\pgfqpoint{3.250571in}{2.816527in}}%
\pgfpathlineto{\pgfqpoint{3.272170in}{2.616766in}}%
\pgfpathlineto{\pgfqpoint{3.286961in}{2.383401in}}%
\pgfpathlineto{\pgfqpoint{3.307621in}{2.171369in}}%
\pgfpathlineto{\pgfqpoint{3.326872in}{2.055079in}}%
\pgfpathlineto{\pgfqpoint{3.345184in}{1.999905in}}%
\pgfpathlineto{\pgfqpoint{3.367018in}{1.976201in}}%
\pgfpathlineto{\pgfqpoint{3.384157in}{1.979637in}}%
\pgfpathlineto{\pgfqpoint{3.401530in}{2.016560in}}%
\pgfpathlineto{\pgfqpoint{3.423128in}{1.999241in}}%
\pgfpathlineto{\pgfqpoint{3.443319in}{2.071549in}}%
\pgfpathlineto{\pgfqpoint{3.461867in}{2.182269in}}%
\pgfpathlineto{\pgfqpoint{3.481587in}{2.353900in}}%
\pgfpathlineto{\pgfqpoint{3.499195in}{2.622887in}}%
\pgfpathlineto{\pgfqpoint{3.520090in}{2.843844in}}%
\pgfpathlineto{\pgfqpoint{3.537463in}{2.817704in}}%
\pgfpathlineto{\pgfqpoint{3.556245in}{2.603371in}}%
\pgfpathlineto{\pgfqpoint{3.577374in}{2.295388in}}%
\pgfpathlineto{\pgfqpoint{3.599443in}{2.112955in}}%
\pgfpathlineto{\pgfqpoint{3.616816in}{2.038232in}}%
\pgfpathlineto{\pgfqpoint{3.634893in}{1.989304in}}%
\pgfpathlineto{\pgfqpoint{3.656259in}{1.999525in}}%
\pgfpathlineto{\pgfqpoint{3.672222in}{1.975558in}}%
\pgfpathlineto{\pgfqpoint{3.691004in}{1.988950in}}%
\pgfpathlineto{\pgfqpoint{3.713073in}{2.017861in}}%
\pgfpathlineto{\pgfqpoint{3.732558in}{2.110612in}}%
\pgfpathlineto{\pgfqpoint{3.748523in}{2.218386in}}%
\pgfpathlineto{\pgfqpoint{3.766600in}{2.453123in}}%
\pgfpathlineto{\pgfqpoint{3.788905in}{2.801299in}}%
\pgfpathlineto{\pgfqpoint{3.808156in}{2.883043in}}%
\pgfpathlineto{\pgfqpoint{3.827173in}{2.824310in}}%
\pgfpathlineto{\pgfqpoint{3.846424in}{2.570072in}}%
\pgfpathlineto{\pgfqpoint{3.865441in}{2.288137in}}%
\pgfpathlineto{\pgfqpoint{3.884221in}{2.129652in}}%
\pgfpathlineto{\pgfqpoint{3.903709in}{2.035524in}}%
\pgfpathlineto{\pgfqpoint{3.922255in}{1.992316in}}%
\pgfpathlineto{\pgfqpoint{3.943384in}{1.978726in}}%
\pgfpathlineto{\pgfqpoint{3.962871in}{2.002456in}}%
\pgfpathlineto{\pgfqpoint{3.980948in}{2.056333in}}%
\pgfpathlineto{\pgfqpoint{3.999260in}{2.147157in}}%
\pgfpathlineto{\pgfqpoint{4.018511in}{2.325883in}}%
\pgfpathlineto{\pgfqpoint{4.037998in}{2.620692in}}%
\pgfpathlineto{\pgfqpoint{4.056544in}{2.824823in}}%
\pgfpathlineto{\pgfqpoint{4.077439in}{2.905900in}}%
\pgfpathlineto{\pgfqpoint{4.096456in}{2.798355in}}%
\pgfpathlineto{\pgfqpoint{4.117821in}{2.912224in}}%
\pgfpathlineto{\pgfqpoint{4.133786in}{2.860124in}}%
\pgfpathlineto{\pgfqpoint{4.151628in}{2.630096in}}%
\pgfpathlineto{\pgfqpoint{4.174635in}{2.352382in}}%
\pgfpathlineto{\pgfqpoint{4.189425in}{2.201753in}}%
\pgfpathlineto{\pgfqpoint{4.209616in}{2.080525in}}%
\pgfpathlineto{\pgfqpoint{4.231216in}{2.001494in}}%
\pgfpathlineto{\pgfqpoint{4.249998in}{1.983722in}}%
\pgfpathlineto{\pgfqpoint{4.269484in}{2.006986in}}%
\pgfpathlineto{\pgfqpoint{4.288735in}{2.066625in}}%
\pgfpathlineto{\pgfqpoint{4.307281in}{2.178792in}}%
\pgfpathlineto{\pgfqpoint{4.346020in}{2.601957in}}%
\pgfpathlineto{\pgfqpoint{4.346489in}{2.712394in}}%
\pgfpathlineto{\pgfqpoint{4.363862in}{2.793530in}}%
\pgfpathlineto{\pgfqpoint{4.385226in}{2.940548in}}%
\pgfpathlineto{\pgfqpoint{4.404242in}{2.858574in}}%
\pgfpathlineto{\pgfqpoint{4.423259in}{2.649402in}}%
\pgfpathlineto{\pgfqpoint{4.441807in}{2.388999in}}%
\pgfpathlineto{\pgfqpoint{4.460824in}{2.190943in}}%
\pgfpathlineto{\pgfqpoint{4.480309in}{2.081800in}}%
\pgfpathlineto{\pgfqpoint{4.473735in}{2.122851in}}%
\pgfpathlineto{\pgfqpoint{4.456832in}{2.339296in}}%
\pgfpathlineto{\pgfqpoint{4.435703in}{2.766239in}}%
\pgfpathlineto{\pgfqpoint{4.417859in}{2.027261in}}%
\pgfpathlineto{\pgfqpoint{4.396964in}{1.981908in}}%
\pgfpathlineto{\pgfqpoint{4.373252in}{2.060594in}}%
\pgfpathlineto{\pgfqpoint{4.360810in}{2.138453in}}%
\pgfpathlineto{\pgfqpoint{4.340385in}{2.377650in}}%
\pgfpathlineto{\pgfqpoint{4.322072in}{2.754466in}}%
\pgfpathlineto{\pgfqpoint{4.301882in}{2.914988in}}%
\pgfpathlineto{\pgfqpoint{4.284509in}{2.657105in}}%
\pgfpathlineto{\pgfqpoint{4.260328in}{2.228785in}}%
\pgfpathlineto{\pgfqpoint{4.245772in}{2.084132in}}%
\pgfpathlineto{\pgfqpoint{4.224876in}{1.993494in}}%
\pgfpathlineto{\pgfqpoint{4.203278in}{1.993677in}}%
\pgfpathlineto{\pgfqpoint{4.184965in}{2.065719in}}%
\pgfpathlineto{\pgfqpoint{4.167122in}{2.238555in}}%
\pgfpathlineto{\pgfqpoint{4.149514in}{2.582901in}}%
\pgfpathlineto{\pgfqpoint{4.128151in}{2.882358in}}%
\pgfpathlineto{\pgfqpoint{4.111012in}{2.816413in}}%
\pgfpathlineto{\pgfqpoint{4.088709in}{2.351924in}}%
\pgfpathlineto{\pgfqpoint{4.070866in}{2.122999in}}%
\pgfpathlineto{\pgfqpoint{4.052318in}{2.017414in}}%
\pgfpathlineto{\pgfqpoint{4.031893in}{1.975956in}}%
\pgfpathlineto{\pgfqpoint{4.013112in}{2.014936in}}%
\pgfpathlineto{\pgfqpoint{3.994330in}{2.126059in}}%
\pgfpathlineto{\pgfqpoint{3.972965in}{2.437716in}}%
\pgfpathlineto{\pgfqpoint{3.957940in}{2.738769in}}%
\pgfpathlineto{\pgfqpoint{3.935168in}{2.858378in}}%
\pgfpathlineto{\pgfqpoint{3.917091in}{2.641408in}}%
\pgfpathlineto{\pgfqpoint{3.897840in}{2.265862in}}%
\pgfpathlineto{\pgfqpoint{3.879526in}{2.084797in}}%
\pgfpathlineto{\pgfqpoint{3.857458in}{2.037972in}}%
\pgfpathlineto{\pgfqpoint{3.839615in}{1.978257in}}%
\pgfpathlineto{\pgfqpoint{3.820364in}{1.984478in}}%
\pgfpathlineto{\pgfqpoint{3.801113in}{2.048850in}}%
\pgfpathlineto{\pgfqpoint{3.783270in}{2.197327in}}%
\pgfpathlineto{\pgfqpoint{3.761436in}{2.480708in}}%
\pgfpathlineto{\pgfqpoint{3.743594in}{2.794125in}}%
\pgfpathlineto{\pgfqpoint{3.724577in}{2.839493in}}%
\pgfpathlineto{\pgfqpoint{3.708143in}{2.566684in}}%
\pgfpathlineto{\pgfqpoint{3.684666in}{2.222633in}}%
\pgfpathlineto{\pgfqpoint{3.667997in}{2.083328in}}%
\pgfpathlineto{\pgfqpoint{3.648744in}{2.018388in}}%
\pgfpathlineto{\pgfqpoint{3.629024in}{1.974577in}}%
\pgfpathlineto{\pgfqpoint{3.606955in}{1.989006in}}%
\pgfpathlineto{\pgfqpoint{3.585356in}{2.068672in}}%
\pgfpathlineto{\pgfqpoint{3.570096in}{2.153623in}}%
\pgfpathlineto{\pgfqpoint{3.549436in}{2.446952in}}%
\pgfpathlineto{\pgfqpoint{3.533003in}{2.700948in}}%
\pgfpathlineto{\pgfqpoint{3.514689in}{2.838092in}}%
\pgfpathlineto{\pgfqpoint{3.494266in}{2.816018in}}%
\pgfpathlineto{\pgfqpoint{3.474075in}{2.577899in}}%
\pgfpathlineto{\pgfqpoint{3.457170in}{2.298420in}}%
\pgfpathlineto{\pgfqpoint{3.438155in}{2.106850in}}%
\pgfpathlineto{\pgfqpoint{3.412798in}{2.001193in}}%
\pgfpathlineto{\pgfqpoint{3.396364in}{1.985281in}}%
\pgfpathlineto{\pgfqpoint{3.377819in}{1.971208in}}%
\pgfpathlineto{\pgfqpoint{3.358802in}{2.002035in}}%
\pgfpathlineto{\pgfqpoint{3.339551in}{2.078088in}}%
\pgfpathlineto{\pgfqpoint{3.320768in}{2.256512in}}%
\pgfpathlineto{\pgfqpoint{3.301986in}{2.541807in}}%
\pgfpathlineto{\pgfqpoint{3.280623in}{2.780268in}}%
\pgfpathlineto{\pgfqpoint{3.263953in}{2.816115in}}%
\pgfpathlineto{\pgfqpoint{3.243293in}{2.676156in}}%
\pgfpathlineto{\pgfqpoint{3.223338in}{2.317275in}}%
\pgfpathlineto{\pgfqpoint{3.205025in}{2.125276in}}%
\pgfpathlineto{\pgfqpoint{3.184365in}{2.020489in}}%
\pgfpathlineto{\pgfqpoint{3.165584in}{1.975746in}}%
\pgfpathlineto{\pgfqpoint{3.147271in}{2.442819in}}%
\pgfpathlineto{\pgfqpoint{3.129429in}{2.773885in}}%
\pgfpathlineto{\pgfqpoint{3.109708in}{2.799490in}}%
\pgfpathlineto{\pgfqpoint{3.066509in}{2.168285in}}%
\pgfpathlineto{\pgfqpoint{3.054301in}{2.070086in}}%
\pgfpathlineto{\pgfqpoint{3.035285in}{1.993179in}}%
\pgfpathlineto{\pgfqpoint{3.013450in}{1.969276in}}%
\pgfpathlineto{\pgfqpoint{2.995139in}{1.994975in}}%
\pgfpathlineto{\pgfqpoint{2.975419in}{2.086436in}}%
\pgfpathlineto{\pgfqpoint{2.955697in}{2.278229in}}%
\pgfpathlineto{\pgfqpoint{2.936211in}{2.624360in}}%
\pgfpathlineto{\pgfqpoint{2.918369in}{2.819553in}}%
\pgfpathlineto{\pgfqpoint{2.900055in}{2.733268in}}%
\pgfpathlineto{\pgfqpoint{2.879632in}{2.546170in}}%
\pgfpathlineto{\pgfqpoint{2.858501in}{2.215964in}}%
\pgfpathlineto{\pgfqpoint{2.840190in}{2.062911in}}%
\pgfpathlineto{\pgfqpoint{2.821407in}{1.993896in}}%
\pgfpathlineto{\pgfqpoint{2.803096in}{1.969280in}}%
\pgfpathlineto{\pgfqpoint{2.783374in}{1.990668in}}%
\pgfpathlineto{\pgfqpoint{2.762948in}{2.067132in}}%
\pgfpathlineto{\pgfqpoint{2.744402in}{2.235299in}}%
\pgfpathlineto{\pgfqpoint{2.743697in}{2.422114in}}%
\pgfpathlineto{\pgfqpoint{2.725386in}{2.555132in}}%
\pgfpathlineto{\pgfqpoint{2.707072in}{2.800475in}}%
\pgfpathlineto{\pgfqpoint{2.688761in}{2.807763in}}%
\pgfpathlineto{\pgfqpoint{2.666223in}{2.536938in}}%
\pgfpathlineto{\pgfqpoint{2.644624in}{2.198351in}}%
\pgfpathlineto{\pgfqpoint{2.629364in}{2.082409in}}%
\pgfpathlineto{\pgfqpoint{2.610113in}{2.004283in}}%
\pgfpathlineto{\pgfqpoint{2.589217in}{1.969476in}}%
\pgfpathlineto{\pgfqpoint{2.571140in}{1.984897in}}%
\pgfpathlineto{\pgfqpoint{2.551654in}{2.034867in}}%
\pgfpathlineto{\pgfqpoint{2.532403in}{2.150751in}}%
\pgfpathlineto{\pgfqpoint{2.493195in}{2.732682in}}%
\pgfpathlineto{\pgfqpoint{2.475352in}{2.811916in}}%
\pgfpathlineto{\pgfqpoint{2.456101in}{2.744372in}}%
\pgfpathlineto{\pgfqpoint{2.437085in}{2.510887in}}%
\pgfpathlineto{\pgfqpoint{2.418302in}{2.228965in}}%
\pgfpathlineto{\pgfqpoint{2.397173in}{2.055613in}}%
\pgfpathlineto{\pgfqpoint{2.378391in}{2.020867in}}%
\pgfpathlineto{\pgfqpoint{2.360549in}{1.975865in}}%
\pgfpathlineto{\pgfqpoint{2.341063in}{1.973067in}}%
\pgfpathlineto{\pgfqpoint{2.323220in}{2.002858in}}%
\pgfpathlineto{\pgfqpoint{2.301621in}{2.099400in}}%
\pgfpathlineto{\pgfqpoint{2.286127in}{2.253236in}}%
\pgfpathlineto{\pgfqpoint{2.263587in}{2.615733in}}%
\pgfpathlineto{\pgfqpoint{2.246216in}{2.802539in}}%
\pgfpathlineto{\pgfqpoint{2.223911in}{2.813676in}}%
\pgfpathlineto{\pgfqpoint{2.205130in}{2.633693in}}%
\pgfpathlineto{\pgfqpoint{2.186582in}{2.331698in}}%
\pgfpathlineto{\pgfqpoint{2.168505in}{2.145001in}}%
\pgfpathlineto{\pgfqpoint{2.147140in}{2.025157in}}%
\pgfpathlineto{\pgfqpoint{2.127889in}{1.986975in}}%
\pgfpathlineto{\pgfqpoint{2.111690in}{1.971137in}}%
\pgfpathlineto{\pgfqpoint{2.091501in}{1.997158in}}%
\pgfpathlineto{\pgfqpoint{2.069901in}{2.070743in}}%
\pgfpathlineto{\pgfqpoint{2.052527in}{2.225210in}}%
\pgfpathlineto{\pgfqpoint{2.033042in}{2.535468in}}%
\pgfpathlineto{\pgfqpoint{2.013556in}{2.786774in}}%
\pgfpathlineto{\pgfqpoint{1.995477in}{2.821735in}}%
\pgfpathlineto{\pgfqpoint{1.973645in}{2.773645in}}%
\pgfpathlineto{\pgfqpoint{1.955331in}{2.514831in}}%
\pgfpathlineto{\pgfqpoint{1.937958in}{2.805213in}}%
\pgfpathlineto{\pgfqpoint{1.914951in}{2.752375in}}%
\pgfpathlineto{\pgfqpoint{1.900161in}{2.553864in}}%
\pgfpathlineto{\pgfqpoint{1.881379in}{2.234190in}}%
\pgfpathlineto{\pgfqpoint{1.860015in}{2.072130in}}%
\pgfpathlineto{\pgfqpoint{1.840528in}{2.011261in}}%
\pgfpathlineto{\pgfqpoint{1.822451in}{1.976376in}}%
\pgfpathlineto{\pgfqpoint{1.803903in}{1.984215in}}%
\pgfpathlineto{\pgfqpoint{1.781834in}{2.046160in}}%
\pgfpathlineto{\pgfqpoint{1.766809in}{2.128542in}}%
\pgfpathlineto{\pgfqpoint{1.745211in}{2.377772in}}%
\pgfpathlineto{\pgfqpoint{1.726429in}{2.672968in}}%
\pgfpathlineto{\pgfqpoint{1.708352in}{2.833929in}}%
\pgfpathlineto{\pgfqpoint{1.686752in}{2.819310in}}%
\pgfpathlineto{\pgfqpoint{1.668441in}{2.608628in}}%
\pgfpathlineto{\pgfqpoint{1.650128in}{2.334728in}}%
\pgfpathlineto{\pgfqpoint{1.628999in}{2.151079in}}%
\pgfpathlineto{\pgfqpoint{1.609982in}{2.050406in}}%
\pgfpathlineto{\pgfqpoint{1.590965in}{1.993371in}}%
\pgfpathlineto{\pgfqpoint{1.572652in}{1.973290in}}%
\pgfpathlineto{\pgfqpoint{1.554340in}{1.987307in}}%
\pgfpathlineto{\pgfqpoint{1.533681in}{2.052144in}}%
\pgfpathlineto{\pgfqpoint{1.514898in}{2.167313in}}%
\pgfpathlineto{\pgfqpoint{1.495647in}{2.393741in}}%
\pgfpathlineto{\pgfqpoint{1.475927in}{2.715713in}}%
\pgfpathlineto{\pgfqpoint{1.456205in}{2.854131in}}%
\pgfpathlineto{\pgfqpoint{1.438128in}{2.832514in}}%
\pgfpathlineto{\pgfqpoint{1.420285in}{2.652549in}}%
\pgfpathlineto{\pgfqpoint{1.397748in}{2.374693in}}%
\pgfpathlineto{\pgfqpoint{1.378731in}{2.209992in}}%
\pgfpathlineto{\pgfqpoint{1.360654in}{2.096168in}}%
\pgfpathlineto{\pgfqpoint{1.339758in}{2.015496in}}%
\pgfpathlineto{\pgfqpoint{1.322621in}{2.195596in}}%
\pgfpathlineto{\pgfqpoint{1.303604in}{2.075298in}}%
\pgfpathlineto{\pgfqpoint{1.283647in}{2.009475in}}%
\pgfpathlineto{\pgfqpoint{1.266510in}{1.979757in}}%
\pgfpathlineto{\pgfqpoint{1.245616in}{1.982053in}}%
\pgfpathlineto{\pgfqpoint{1.225659in}{2.034997in}}%
\pgfpathlineto{\pgfqpoint{1.208757in}{2.120077in}}%
\pgfpathlineto{\pgfqpoint{1.188097in}{2.298658in}}%
\pgfpathlineto{\pgfqpoint{1.167671in}{2.538549in}}%
\pgfpathlineto{\pgfqpoint{1.150532in}{2.773182in}}%
\pgfpathlineto{\pgfqpoint{1.128932in}{2.884683in}}%
\pgfpathlineto{\pgfqpoint{1.108978in}{2.807712in}}%
\pgfpathlineto{\pgfqpoint{1.071413in}{2.270995in}}%
\pgfpathlineto{\pgfqpoint{1.053805in}{2.132796in}}%
\pgfpathlineto{\pgfqpoint{1.034319in}{2.056115in}}%
\pgfpathlineto{\pgfqpoint{1.016242in}{2.002482in}}%
\pgfpathlineto{\pgfqpoint{0.995348in}{1.981410in}}%
\pgfpathlineto{\pgfqpoint{0.977269in}{2.003269in}}%
\pgfpathlineto{\pgfqpoint{0.956140in}{2.053328in}}%
\pgfpathlineto{\pgfqpoint{0.939472in}{2.140438in}}%
\pgfpathlineto{\pgfqpoint{0.919047in}{2.303220in}}%
\pgfpathlineto{\pgfqpoint{0.899090in}{2.618444in}}%
\pgfpathlineto{\pgfqpoint{0.880779in}{2.842622in}}%
\pgfpathlineto{\pgfqpoint{0.860353in}{2.910902in}}%
\pgfpathlineto{\pgfqpoint{0.840868in}{2.828776in}}%
\pgfpathlineto{\pgfqpoint{0.823260in}{2.564559in}}%
\pgfpathlineto{\pgfqpoint{0.801894in}{2.283113in}}%
\pgfpathlineto{\pgfqpoint{0.783348in}{2.150053in}}%
\pgfpathlineto{\pgfqpoint{0.766444in}{2.069938in}}%
\pgfpathlineto{\pgfqpoint{0.746724in}{2.014753in}}%
\pgfpathlineto{\pgfqpoint{0.726767in}{1.985043in}}%
\pgfpathlineto{\pgfqpoint{0.707047in}{2.002811in}}%
\pgfpathlineto{\pgfqpoint{0.686621in}{2.056084in}}%
\pgfpathlineto{\pgfqpoint{0.668779in}{2.150454in}}%
\pgfpathlineto{\pgfqpoint{0.651640in}{2.295302in}}%
\pgfpathlineto{\pgfqpoint{0.651171in}{2.289913in}}%
\pgfpathlineto{\pgfqpoint{0.655163in}{2.239499in}}%
\pgfpathlineto{\pgfqpoint{0.673945in}{2.064223in}}%
\pgfpathlineto{\pgfqpoint{0.698360in}{1.986917in}}%
\pgfpathlineto{\pgfqpoint{0.711742in}{2.015616in}}%
\pgfpathlineto{\pgfqpoint{0.734516in}{2.142899in}}%
\pgfpathlineto{\pgfqpoint{0.754002in}{2.371397in}}%
\pgfpathlineto{\pgfqpoint{0.771610in}{2.761279in}}%
\pgfpathlineto{\pgfqpoint{0.790626in}{2.914565in}}%
\pgfpathlineto{\pgfqpoint{0.809172in}{2.736874in}}%
\pgfpathlineto{\pgfqpoint{0.827486in}{2.361101in}}%
\pgfpathlineto{\pgfqpoint{0.849554in}{2.093119in}}%
\pgfpathlineto{\pgfqpoint{0.867631in}{2.004332in}}%
\pgfpathlineto{\pgfqpoint{0.887117in}{1.986947in}}%
\pgfpathlineto{\pgfqpoint{0.905899in}{2.046148in}}%
\pgfpathlineto{\pgfqpoint{0.924447in}{2.181655in}}%
\pgfpathlineto{\pgfqpoint{0.943227in}{2.475650in}}%
\pgfpathlineto{\pgfqpoint{0.962010in}{2.819890in}}%
\pgfpathlineto{\pgfqpoint{0.983844in}{2.849901in}}%
\pgfpathlineto{\pgfqpoint{1.004035in}{2.545712in}}%
\pgfpathlineto{\pgfqpoint{1.021643in}{2.206663in}}%
\pgfpathlineto{\pgfqpoint{1.039954in}{2.051227in}}%
\pgfpathlineto{\pgfqpoint{1.057328in}{1.985225in}}%
\pgfpathlineto{\pgfqpoint{1.077048in}{1.991887in}}%
\pgfpathlineto{\pgfqpoint{1.095125in}{2.066790in}}%
\pgfpathlineto{\pgfqpoint{1.118837in}{2.248086in}}%
\pgfpathlineto{\pgfqpoint{1.135741in}{2.588251in}}%
\pgfpathlineto{\pgfqpoint{1.154524in}{2.850042in}}%
\pgfpathlineto{\pgfqpoint{1.173304in}{2.798802in}}%
\pgfpathlineto{\pgfqpoint{1.191383in}{2.459526in}}%
\pgfpathlineto{\pgfqpoint{1.213217in}{2.146331in}}%
\pgfpathlineto{\pgfqpoint{1.233172in}{2.019815in}}%
\pgfpathlineto{\pgfqpoint{1.251954in}{1.978162in}}%
\pgfpathlineto{\pgfqpoint{1.269562in}{1.988308in}}%
\pgfpathlineto{\pgfqpoint{1.289517in}{2.062248in}}%
\pgfpathlineto{\pgfqpoint{1.308064in}{2.210798in}}%
\pgfpathlineto{\pgfqpoint{1.325907in}{2.504845in}}%
\pgfpathlineto{\pgfqpoint{1.346801in}{2.830085in}}%
\pgfpathlineto{\pgfqpoint{1.366992in}{2.776420in}}%
\pgfpathlineto{\pgfqpoint{1.385538in}{2.439632in}}%
\pgfpathlineto{\pgfqpoint{1.403852in}{2.176773in}}%
\pgfpathlineto{\pgfqpoint{1.425686in}{2.026329in}}%
\pgfpathlineto{\pgfqpoint{1.442120in}{1.978555in}}%
\pgfpathlineto{\pgfqpoint{1.464188in}{1.991567in}}%
\pgfpathlineto{\pgfqpoint{1.480622in}{2.052238in}}%
\pgfpathlineto{\pgfqpoint{1.501751in}{2.202224in}}%
\pgfpathlineto{\pgfqpoint{1.541662in}{2.789839in}}%
\pgfpathlineto{\pgfqpoint{1.557627in}{2.826366in}}%
\pgfpathlineto{\pgfqpoint{1.578287in}{2.515962in}}%
\pgfpathlineto{\pgfqpoint{1.596129in}{2.801612in}}%
\pgfpathlineto{\pgfqpoint{1.617963in}{2.810995in}}%
\pgfpathlineto{\pgfqpoint{1.636277in}{2.639206in}}%
\pgfpathlineto{\pgfqpoint{1.655762in}{2.277796in}}%
\pgfpathlineto{\pgfqpoint{1.674779in}{2.064250in}}%
\pgfpathlineto{\pgfqpoint{1.693325in}{1.994010in}}%
\pgfpathlineto{\pgfqpoint{1.713985in}{1.973845in}}%
\pgfpathlineto{\pgfqpoint{1.731593in}{2.007029in}}%
\pgfpathlineto{\pgfqpoint{1.752253in}{2.089221in}}%
\pgfpathlineto{\pgfqpoint{1.770566in}{2.271106in}}%
\pgfpathlineto{\pgfqpoint{1.789112in}{2.575804in}}%
\pgfpathlineto{\pgfqpoint{1.809303in}{2.817958in}}%
\pgfpathlineto{\pgfqpoint{1.828320in}{2.764732in}}%
\pgfpathlineto{\pgfqpoint{1.849449in}{2.457404in}}%
\pgfpathlineto{\pgfqpoint{1.867762in}{2.169734in}}%
\pgfpathlineto{\pgfqpoint{1.867997in}{2.078696in}}%
\pgfpathlineto{\pgfqpoint{1.887951in}{2.034975in}}%
\pgfpathlineto{\pgfqpoint{1.907204in}{1.981303in}}%
\pgfpathlineto{\pgfqpoint{1.924107in}{1.970241in}}%
\pgfpathlineto{\pgfqpoint{1.944767in}{2.005600in}}%
\pgfpathlineto{\pgfqpoint{1.962610in}{2.073965in}}%
\pgfpathlineto{\pgfqpoint{1.981157in}{2.223034in}}%
\pgfpathlineto{\pgfqpoint{2.002052in}{2.578485in}}%
\pgfpathlineto{\pgfqpoint{2.023415in}{2.819047in}}%
\pgfpathlineto{\pgfqpoint{2.037502in}{2.777672in}}%
\pgfpathlineto{\pgfqpoint{2.059102in}{2.559253in}}%
\pgfpathlineto{\pgfqpoint{2.076944in}{2.258430in}}%
\pgfpathlineto{\pgfqpoint{2.097839in}{2.063688in}}%
\pgfpathlineto{\pgfqpoint{2.117090in}{1.995451in}}%
\pgfpathlineto{\pgfqpoint{2.140333in}{1.971075in}}%
\pgfpathlineto{\pgfqpoint{2.155124in}{1.981693in}}%
\pgfpathlineto{\pgfqpoint{2.175078in}{2.029038in}}%
\pgfpathlineto{\pgfqpoint{2.193157in}{2.130807in}}%
\pgfpathlineto{\pgfqpoint{2.214520in}{2.345732in}}%
\pgfpathlineto{\pgfqpoint{2.232363in}{2.670956in}}%
\pgfpathlineto{\pgfqpoint{2.250676in}{2.715659in}}%
\pgfpathlineto{\pgfqpoint{2.274623in}{2.792397in}}%
\pgfpathlineto{\pgfqpoint{2.289648in}{2.589647in}}%
\pgfpathlineto{\pgfqpoint{2.307490in}{2.277382in}}%
\pgfpathlineto{\pgfqpoint{2.327916in}{2.077590in}}%
\pgfpathlineto{\pgfqpoint{2.345524in}{2.021211in}}%
\pgfpathlineto{\pgfqpoint{2.367123in}{1.975578in}}%
\pgfpathlineto{\pgfqpoint{2.384495in}{1.972344in}}%
\pgfpathlineto{\pgfqpoint{2.406564in}{2.011846in}}%
\pgfpathlineto{\pgfqpoint{2.423937in}{2.092153in}}%
\pgfpathlineto{\pgfqpoint{2.443659in}{2.271823in}}%
\pgfpathlineto{\pgfqpoint{2.461971in}{2.541003in}}%
\pgfpathlineto{\pgfqpoint{2.480987in}{2.793542in}}%
\pgfpathlineto{\pgfqpoint{2.502587in}{2.790147in}}%
\pgfpathlineto{\pgfqpoint{2.521133in}{2.543619in}}%
\pgfpathlineto{\pgfqpoint{2.541793in}{2.204400in}}%
\pgfpathlineto{\pgfqpoint{2.558227in}{2.074032in}}%
\pgfpathlineto{\pgfqpoint{2.578418in}{1.995012in}}%
\pgfpathlineto{\pgfqpoint{2.596026in}{2.013894in}}%
\pgfpathlineto{\pgfqpoint{2.634997in}{1.968816in}}%
\pgfpathlineto{\pgfqpoint{2.653076in}{1.989389in}}%
\pgfpathlineto{\pgfqpoint{2.673736in}{2.059249in}}%
\pgfpathlineto{\pgfqpoint{2.692047in}{2.190820in}}%
\pgfpathlineto{\pgfqpoint{2.712473in}{2.482418in}}%
\pgfpathlineto{\pgfqpoint{2.731255in}{2.729354in}}%
\pgfpathlineto{\pgfqpoint{2.753324in}{2.811305in}}%
\pgfpathlineto{\pgfqpoint{2.773278in}{2.553753in}}%
\pgfpathlineto{\pgfqpoint{2.788540in}{2.537682in}}%
\pgfpathlineto{\pgfqpoint{2.810138in}{2.187921in}}%
\pgfpathlineto{\pgfqpoint{2.829154in}{2.047065in}}%
\pgfpathlineto{\pgfqpoint{2.844650in}{1.995958in}}%
\pgfpathlineto{\pgfqpoint{2.866248in}{1.971255in}}%
\pgfpathlineto{\pgfqpoint{2.888082in}{1.997306in}}%
\pgfpathlineto{\pgfqpoint{2.902873in}{2.044659in}}%
\pgfpathlineto{\pgfqpoint{2.924238in}{2.145512in}}%
\pgfpathlineto{\pgfqpoint{2.942080in}{2.339182in}}%
\pgfpathlineto{\pgfqpoint{2.963209in}{2.671295in}}%
\pgfpathlineto{\pgfqpoint{2.981286in}{2.784830in}}%
\pgfpathlineto{\pgfqpoint{2.998425in}{2.810881in}}%
\pgfpathlineto{\pgfqpoint{3.019320in}{2.552566in}}%
\pgfpathlineto{\pgfqpoint{3.037162in}{2.271180in}}%
\pgfpathlineto{\pgfqpoint{3.055945in}{2.107469in}}%
\pgfpathlineto{\pgfqpoint{3.076604in}{2.025950in}}%
\pgfpathlineto{\pgfqpoint{3.094681in}{1.981232in}}%
\pgfpathlineto{\pgfqpoint{3.116047in}{2.486862in}}%
\pgfpathlineto{\pgfqpoint{3.135063in}{2.622512in}}%
\pgfpathlineto{\pgfqpoint{3.155958in}{2.295020in}}%
\pgfpathlineto{\pgfqpoint{3.176149in}{2.080857in}}%
\pgfpathlineto{\pgfqpoint{3.191174in}{2.027702in}}%
\pgfpathlineto{\pgfqpoint{3.214651in}{1.972021in}}%
\pgfpathlineto{\pgfqpoint{3.229676in}{1.979140in}}%
\pgfpathlineto{\pgfqpoint{3.250805in}{2.028140in}}%
\pgfpathlineto{\pgfqpoint{3.269119in}{2.122907in}}%
\pgfpathlineto{\pgfqpoint{3.292830in}{2.457301in}}%
\pgfpathlineto{\pgfqpoint{3.307855in}{2.702809in}}%
\pgfpathlineto{\pgfqpoint{3.327107in}{2.844496in}}%
\pgfpathlineto{\pgfqpoint{3.345654in}{2.779494in}}%
\pgfpathlineto{\pgfqpoint{3.366080in}{2.531527in}}%
\pgfpathlineto{\pgfqpoint{3.383922in}{2.228888in}}%
\pgfpathlineto{\pgfqpoint{3.404817in}{2.063566in}}%
\pgfpathlineto{\pgfqpoint{3.423365in}{2.003960in}}%
\pgfpathlineto{\pgfqpoint{3.440971in}{1.974821in}}%
\pgfpathlineto{\pgfqpoint{3.462570in}{2.007478in}}%
\pgfpathlineto{\pgfqpoint{3.480178in}{2.075125in}}%
\pgfpathlineto{\pgfqpoint{3.500838in}{2.228136in}}%
\pgfpathlineto{\pgfqpoint{3.517272in}{2.329928in}}%
\pgfpathlineto{\pgfqpoint{3.536994in}{2.624672in}}%
\pgfpathlineto{\pgfqpoint{3.559063in}{2.854125in}}%
\pgfpathlineto{\pgfqpoint{3.579486in}{2.782457in}}%
\pgfpathlineto{\pgfqpoint{3.598034in}{2.548853in}}%
\pgfpathlineto{\pgfqpoint{3.615877in}{2.269835in}}%
\pgfpathlineto{\pgfqpoint{3.633485in}{2.125661in}}%
\pgfpathlineto{\pgfqpoint{3.654614in}{2.030312in}}%
\pgfpathlineto{\pgfqpoint{3.675744in}{1.983113in}}%
\pgfpathlineto{\pgfqpoint{3.690300in}{1.978450in}}%
\pgfpathlineto{\pgfqpoint{3.711898in}{2.028384in}}%
\pgfpathlineto{\pgfqpoint{3.731150in}{2.091615in}}%
\pgfpathlineto{\pgfqpoint{3.750872in}{2.236060in}}%
\pgfpathlineto{\pgfqpoint{3.769654in}{2.465120in}}%
\pgfpathlineto{\pgfqpoint{3.789843in}{2.764958in}}%
\pgfpathlineto{\pgfqpoint{3.808156in}{2.883296in}}%
\pgfpathlineto{\pgfqpoint{3.827407in}{2.832736in}}%
\pgfpathlineto{\pgfqpoint{3.845484in}{2.580508in}}%
\pgfpathlineto{\pgfqpoint{3.865675in}{2.259631in}}%
\pgfpathlineto{\pgfqpoint{3.884692in}{2.135986in}}%
\pgfpathlineto{\pgfqpoint{3.903004in}{2.046594in}}%
\pgfpathlineto{\pgfqpoint{3.922255in}{2.006209in}}%
\pgfpathlineto{\pgfqpoint{3.940332in}{1.981250in}}%
\pgfpathlineto{\pgfqpoint{3.961228in}{1.992374in}}%
\pgfpathlineto{\pgfqpoint{3.980479in}{2.041873in}}%
\pgfpathlineto{\pgfqpoint{4.002312in}{2.159895in}}%
\pgfpathlineto{\pgfqpoint{4.020156in}{2.323988in}}%
\pgfpathlineto{\pgfqpoint{4.038936in}{2.574999in}}%
\pgfpathlineto{\pgfqpoint{4.057719in}{2.809701in}}%
\pgfpathlineto{\pgfqpoint{4.076501in}{2.900362in}}%
\pgfpathlineto{\pgfqpoint{4.095518in}{2.901698in}}%
\pgfpathlineto{\pgfqpoint{4.117821in}{2.748215in}}%
\pgfpathlineto{\pgfqpoint{4.134960in}{2.500932in}}%
\pgfpathlineto{\pgfqpoint{4.153271in}{2.284355in}}%
\pgfpathlineto{\pgfqpoint{4.172288in}{2.123010in}}%
\pgfpathlineto{\pgfqpoint{4.190599in}{2.065343in}}%
\pgfpathlineto{\pgfqpoint{4.213139in}{1.998196in}}%
\pgfpathlineto{\pgfqpoint{4.232625in}{1.987543in}}%
\pgfpathlineto{\pgfqpoint{4.251171in}{2.005058in}}%
\pgfpathlineto{\pgfqpoint{4.269484in}{2.067171in}}%
\pgfpathlineto{\pgfqpoint{4.293196in}{2.199926in}}%
\pgfpathlineto{\pgfqpoint{4.308455in}{2.313277in}}%
\pgfpathlineto{\pgfqpoint{4.346723in}{2.807219in}}%
\pgfpathlineto{\pgfqpoint{4.365271in}{2.927817in}}%
\pgfpathlineto{\pgfqpoint{4.383817in}{2.904494in}}%
\pgfpathlineto{\pgfqpoint{4.403539in}{2.012348in}}%
\pgfpathlineto{\pgfqpoint{4.422085in}{2.087217in}}%
\pgfpathlineto{\pgfqpoint{4.440867in}{2.249327in}}%
\pgfpathlineto{\pgfqpoint{4.460353in}{2.539700in}}%
\pgfpathlineto{\pgfqpoint{4.479135in}{2.865087in}}%
\pgfpathlineto{\pgfqpoint{4.481013in}{2.876595in}}%
\pgfpathlineto{\pgfqpoint{4.475143in}{2.796452in}}%
\pgfpathlineto{\pgfqpoint{4.452137in}{2.308735in}}%
\pgfpathlineto{\pgfqpoint{4.434998in}{2.107016in}}%
\pgfpathlineto{\pgfqpoint{4.416686in}{2.009315in}}%
\pgfpathlineto{\pgfqpoint{4.398139in}{1.989658in}}%
\pgfpathlineto{\pgfqpoint{4.376775in}{2.065538in}}%
\pgfpathlineto{\pgfqpoint{4.359402in}{2.245149in}}%
\pgfpathlineto{\pgfqpoint{4.339916in}{2.633455in}}%
\pgfpathlineto{\pgfqpoint{4.317611in}{2.912156in}}%
\pgfpathlineto{\pgfqpoint{4.303057in}{2.848395in}}%
\pgfpathlineto{\pgfqpoint{4.278171in}{2.349305in}}%
\pgfpathlineto{\pgfqpoint{4.263849in}{2.149409in}}%
\pgfpathlineto{\pgfqpoint{4.244363in}{2.026773in}}%
\pgfpathlineto{\pgfqpoint{4.226050in}{1.981024in}}%
\pgfpathlineto{\pgfqpoint{4.205624in}{2.024880in}}%
\pgfpathlineto{\pgfqpoint{4.188016in}{2.142664in}}%
\pgfpathlineto{\pgfqpoint{4.166419in}{2.468646in}}%
\pgfpathlineto{\pgfqpoint{4.149750in}{2.806053in}}%
\pgfpathlineto{\pgfqpoint{4.127916in}{2.873978in}}%
\pgfpathlineto{\pgfqpoint{4.109368in}{2.569603in}}%
\pgfpathlineto{\pgfqpoint{4.088709in}{2.202213in}}%
\pgfpathlineto{\pgfqpoint{4.067814in}{2.047971in}}%
\pgfpathlineto{\pgfqpoint{4.050206in}{1.996062in}}%
\pgfpathlineto{\pgfqpoint{4.033067in}{1.981682in}}%
\pgfpathlineto{\pgfqpoint{4.014287in}{2.039066in}}%
\pgfpathlineto{\pgfqpoint{3.993861in}{2.173124in}}%
\pgfpathlineto{\pgfqpoint{3.975782in}{2.457828in}}%
\pgfpathlineto{\pgfqpoint{3.954653in}{2.824363in}}%
\pgfpathlineto{\pgfqpoint{3.933525in}{2.827832in}}%
\pgfpathlineto{\pgfqpoint{3.916151in}{2.510422in}}%
\pgfpathlineto{\pgfqpoint{3.897840in}{2.198605in}}%
\pgfpathlineto{\pgfqpoint{3.878823in}{2.055180in}}%
\pgfpathlineto{\pgfqpoint{3.854875in}{1.980113in}}%
\pgfpathlineto{\pgfqpoint{3.838912in}{1.980894in}}%
\pgfpathlineto{\pgfqpoint{3.819660in}{2.443570in}}%
\pgfpathlineto{\pgfqpoint{3.801347in}{2.162045in}}%
\pgfpathlineto{\pgfqpoint{3.782565in}{2.035278in}}%
\pgfpathlineto{\pgfqpoint{3.765428in}{1.983740in}}%
\pgfpathlineto{\pgfqpoint{3.743359in}{1.993586in}}%
\pgfpathlineto{\pgfqpoint{3.725517in}{2.064112in}}%
\pgfpathlineto{\pgfqpoint{3.705091in}{2.266710in}}%
\pgfpathlineto{\pgfqpoint{3.685840in}{2.627871in}}%
\pgfpathlineto{\pgfqpoint{3.666589in}{2.830312in}}%
\pgfpathlineto{\pgfqpoint{3.648275in}{2.789944in}}%
\pgfpathlineto{\pgfqpoint{3.629259in}{2.481476in}}%
\pgfpathlineto{\pgfqpoint{3.607190in}{2.161751in}}%
\pgfpathlineto{\pgfqpoint{3.591696in}{2.051618in}}%
\pgfpathlineto{\pgfqpoint{3.569393in}{1.984062in}}%
\pgfpathlineto{\pgfqpoint{3.551079in}{1.982660in}}%
\pgfpathlineto{\pgfqpoint{3.529245in}{2.031684in}}%
\pgfpathlineto{\pgfqpoint{3.514220in}{2.125445in}}%
\pgfpathlineto{\pgfqpoint{3.496143in}{2.319920in}}%
\pgfpathlineto{\pgfqpoint{3.474309in}{2.617359in}}%
\pgfpathlineto{\pgfqpoint{3.455292in}{2.831630in}}%
\pgfpathlineto{\pgfqpoint{3.436510in}{2.751040in}}%
\pgfpathlineto{\pgfqpoint{3.417964in}{2.556225in}}%
\pgfpathlineto{\pgfqpoint{3.398713in}{2.245132in}}%
\pgfpathlineto{\pgfqpoint{3.376644in}{2.063117in}}%
\pgfpathlineto{\pgfqpoint{3.358802in}{1.992541in}}%
\pgfpathlineto{\pgfqpoint{3.340254in}{1.970462in}}%
\pgfpathlineto{\pgfqpoint{3.321472in}{2.000352in}}%
\pgfpathlineto{\pgfqpoint{3.302455in}{2.075039in}}%
\pgfpathlineto{\pgfqpoint{3.278040in}{2.285190in}}%
\pgfpathlineto{\pgfqpoint{3.265127in}{2.505892in}}%
\pgfpathlineto{\pgfqpoint{3.243763in}{2.270877in}}%
\pgfpathlineto{\pgfqpoint{3.226390in}{2.546817in}}%
\pgfpathlineto{\pgfqpoint{3.203616in}{2.821476in}}%
\pgfpathlineto{\pgfqpoint{3.184836in}{2.744370in}}%
\pgfpathlineto{\pgfqpoint{3.169105in}{2.493152in}}%
\pgfpathlineto{\pgfqpoint{3.147740in}{2.186596in}}%
\pgfpathlineto{\pgfqpoint{3.127316in}{2.045639in}}%
\pgfpathlineto{\pgfqpoint{3.109003in}{1.987203in}}%
\pgfpathlineto{\pgfqpoint{3.091161in}{1.970730in}}%
\pgfpathlineto{\pgfqpoint{3.071440in}{2.005394in}}%
\pgfpathlineto{\pgfqpoint{3.051249in}{2.073606in}}%
\pgfpathlineto{\pgfqpoint{3.032702in}{2.243484in}}%
\pgfpathlineto{\pgfqpoint{3.013921in}{2.556463in}}%
\pgfpathlineto{\pgfqpoint{2.996782in}{2.801882in}}%
\pgfpathlineto{\pgfqpoint{2.978000in}{2.795595in}}%
\pgfpathlineto{\pgfqpoint{2.956871in}{2.579310in}}%
\pgfpathlineto{\pgfqpoint{2.936446in}{2.243688in}}%
\pgfpathlineto{\pgfqpoint{2.917194in}{2.100158in}}%
\pgfpathlineto{\pgfqpoint{2.900761in}{2.024006in}}%
\pgfpathlineto{\pgfqpoint{2.878457in}{1.974832in}}%
\pgfpathlineto{\pgfqpoint{2.856858in}{1.976529in}}%
\pgfpathlineto{\pgfqpoint{2.838781in}{2.015470in}}%
\pgfpathlineto{\pgfqpoint{2.819999in}{2.109967in}}%
\pgfpathlineto{\pgfqpoint{2.800513in}{2.305193in}}%
\pgfpathlineto{\pgfqpoint{2.785722in}{2.581820in}}%
\pgfpathlineto{\pgfqpoint{2.765062in}{2.800955in}}%
\pgfpathlineto{\pgfqpoint{2.744871in}{2.802044in}}%
\pgfpathlineto{\pgfqpoint{2.723742in}{2.556812in}}%
\pgfpathlineto{\pgfqpoint{2.701674in}{2.251908in}}%
\pgfpathlineto{\pgfqpoint{2.686883in}{2.139102in}}%
\pgfpathlineto{\pgfqpoint{2.668101in}{2.039367in}}%
\pgfpathlineto{\pgfqpoint{2.649084in}{2.001237in}}%
\pgfpathlineto{\pgfqpoint{2.627721in}{1.969449in}}%
\pgfpathlineto{\pgfqpoint{2.610582in}{1.982741in}}%
\pgfpathlineto{\pgfqpoint{2.590391in}{2.033659in}}%
\pgfpathlineto{\pgfqpoint{2.569027in}{2.105460in}}%
\pgfpathlineto{\pgfqpoint{2.550480in}{2.321593in}}%
\pgfpathlineto{\pgfqpoint{2.532168in}{2.611958in}}%
\pgfpathlineto{\pgfqpoint{2.515734in}{2.808689in}}%
\pgfpathlineto{\pgfqpoint{2.494604in}{2.733188in}}%
\pgfpathlineto{\pgfqpoint{2.455161in}{2.163590in}}%
\pgfpathlineto{\pgfqpoint{2.437319in}{2.060767in}}%
\pgfpathlineto{\pgfqpoint{2.420885in}{2.003552in}}%
\pgfpathlineto{\pgfqpoint{2.398582in}{1.969311in}}%
\pgfpathlineto{\pgfqpoint{2.376982in}{1.988073in}}%
\pgfpathlineto{\pgfqpoint{2.358436in}{2.042708in}}%
\pgfpathlineto{\pgfqpoint{2.342941in}{2.115908in}}%
\pgfpathlineto{\pgfqpoint{2.321343in}{2.366259in}}%
\pgfpathlineto{\pgfqpoint{2.303264in}{2.647824in}}%
\pgfpathlineto{\pgfqpoint{2.280961in}{2.816072in}}%
\pgfpathlineto{\pgfqpoint{2.263118in}{2.740929in}}%
\pgfpathlineto{\pgfqpoint{2.225556in}{2.263090in}}%
\pgfpathlineto{\pgfqpoint{2.207477in}{2.109227in}}%
\pgfpathlineto{\pgfqpoint{2.187991in}{2.026897in}}%
\pgfpathlineto{\pgfqpoint{2.167566in}{1.986821in}}%
\pgfpathlineto{\pgfqpoint{2.148314in}{1.970873in}}%
\pgfpathlineto{\pgfqpoint{2.126480in}{2.004524in}}%
\pgfpathlineto{\pgfqpoint{2.110986in}{2.050319in}}%
\pgfpathlineto{\pgfqpoint{2.089387in}{2.102254in}}%
\pgfpathlineto{\pgfqpoint{2.070841in}{2.316495in}}%
\pgfpathlineto{\pgfqpoint{2.052762in}{2.574576in}}%
\pgfpathlineto{\pgfqpoint{2.034216in}{2.694536in}}%
\pgfpathlineto{\pgfqpoint{2.015199in}{2.826358in}}%
\pgfpathlineto{\pgfqpoint{1.993834in}{2.704117in}}%
\pgfpathlineto{\pgfqpoint{1.957680in}{2.200917in}}%
\pgfpathlineto{\pgfqpoint{1.937489in}{2.074739in}}%
\pgfpathlineto{\pgfqpoint{1.920350in}{2.043625in}}%
\pgfpathlineto{\pgfqpoint{1.898518in}{1.985347in}}%
\pgfpathlineto{\pgfqpoint{1.879735in}{1.972332in}}%
\pgfpathlineto{\pgfqpoint{1.858136in}{2.004343in}}%
\pgfpathlineto{\pgfqpoint{1.839590in}{2.073092in}}%
\pgfpathlineto{\pgfqpoint{1.821042in}{2.210424in}}%
\pgfpathlineto{\pgfqpoint{1.783008in}{2.687312in}}%
\pgfpathlineto{\pgfqpoint{1.764463in}{2.828475in}}%
\pgfpathlineto{\pgfqpoint{1.743568in}{2.788442in}}%
\pgfpathlineto{\pgfqpoint{1.727603in}{2.586373in}}%
\pgfpathlineto{\pgfqpoint{1.710230in}{2.321823in}}%
\pgfpathlineto{\pgfqpoint{1.687221in}{2.181589in}}%
\pgfpathlineto{\pgfqpoint{1.669848in}{2.089817in}}%
\pgfpathlineto{\pgfqpoint{1.648484in}{2.016546in}}%
\pgfpathlineto{\pgfqpoint{1.629233in}{1.986455in}}%
\pgfpathlineto{\pgfqpoint{1.610920in}{1.976938in}}%
\pgfpathlineto{\pgfqpoint{1.592374in}{2.000312in}}%
\pgfpathlineto{\pgfqpoint{1.571009in}{2.072652in}}%
\pgfpathlineto{\pgfqpoint{1.554106in}{2.190538in}}%
\pgfpathlineto{\pgfqpoint{1.530629in}{2.163084in}}%
\pgfpathlineto{\pgfqpoint{1.511612in}{2.374498in}}%
\pgfpathlineto{\pgfqpoint{1.494004in}{2.649821in}}%
\pgfpathlineto{\pgfqpoint{1.478510in}{2.806264in}}%
\pgfpathlineto{\pgfqpoint{1.459022in}{2.851778in}}%
\pgfpathlineto{\pgfqpoint{1.438128in}{2.675255in}}%
\pgfpathlineto{\pgfqpoint{1.417702in}{2.373180in}}%
\pgfpathlineto{\pgfqpoint{1.395165in}{2.162051in}}%
\pgfpathlineto{\pgfqpoint{1.378497in}{2.103200in}}%
\pgfpathlineto{\pgfqpoint{1.359949in}{2.045695in}}%
\pgfpathlineto{\pgfqpoint{1.339758in}{1.988781in}}%
\pgfpathlineto{\pgfqpoint{1.321681in}{1.977786in}}%
\pgfpathlineto{\pgfqpoint{1.302899in}{1.998672in}}%
\pgfpathlineto{\pgfqpoint{1.283884in}{2.055182in}}%
\pgfpathlineto{\pgfqpoint{1.262753in}{2.155325in}}%
\pgfpathlineto{\pgfqpoint{1.244442in}{2.320586in}}%
\pgfpathlineto{\pgfqpoint{1.226363in}{2.546722in}}%
\pgfpathlineto{\pgfqpoint{1.207582in}{2.694421in}}%
\pgfpathlineto{\pgfqpoint{1.187860in}{2.861189in}}%
\pgfpathlineto{\pgfqpoint{1.167435in}{2.866539in}}%
\pgfpathlineto{\pgfqpoint{1.148889in}{2.702284in}}%
\pgfpathlineto{\pgfqpoint{1.129403in}{2.412735in}}%
\pgfpathlineto{\pgfqpoint{1.110855in}{2.211894in}}%
\pgfpathlineto{\pgfqpoint{1.090901in}{2.099352in}}%
\pgfpathlineto{\pgfqpoint{1.052867in}{1.994758in}}%
\pgfpathlineto{\pgfqpoint{1.035259in}{1.981884in}}%
\pgfpathlineto{\pgfqpoint{1.015537in}{2.018306in}}%
\pgfpathlineto{\pgfqpoint{0.994877in}{2.078711in}}%
\pgfpathlineto{\pgfqpoint{0.976097in}{2.265021in}}%
\pgfpathlineto{\pgfqpoint{0.956846in}{2.113302in}}%
\pgfpathlineto{\pgfqpoint{0.939707in}{2.038603in}}%
\pgfpathlineto{\pgfqpoint{0.917169in}{1.984115in}}%
\pgfpathlineto{\pgfqpoint{0.900499in}{1.991998in}}%
\pgfpathlineto{\pgfqpoint{0.879604in}{2.059333in}}%
\pgfpathlineto{\pgfqpoint{0.860119in}{2.189675in}}%
\pgfpathlineto{\pgfqpoint{0.843451in}{2.310101in}}%
\pgfpathlineto{\pgfqpoint{0.823025in}{2.595901in}}%
\pgfpathlineto{\pgfqpoint{0.802365in}{2.815133in}}%
\pgfpathlineto{\pgfqpoint{0.783348in}{2.916420in}}%
\pgfpathlineto{\pgfqpoint{0.762688in}{2.761078in}}%
\pgfpathlineto{\pgfqpoint{0.746489in}{2.495130in}}%
\pgfpathlineto{\pgfqpoint{0.726767in}{2.223490in}}%
\pgfpathlineto{\pgfqpoint{0.706344in}{2.083442in}}%
\pgfpathlineto{\pgfqpoint{0.688265in}{2.018157in}}%
\pgfpathlineto{\pgfqpoint{0.665258in}{1.987407in}}%
\pgfpathlineto{\pgfqpoint{0.649528in}{2.035791in}}%
\pgfpathlineto{\pgfqpoint{0.650231in}{2.038546in}}%
\pgfpathlineto{\pgfqpoint{0.655397in}{2.063862in}}%
\pgfpathlineto{\pgfqpoint{0.675823in}{2.247009in}}%
\pgfpathlineto{\pgfqpoint{0.711508in}{2.872678in}}%
\pgfpathlineto{\pgfqpoint{0.733811in}{2.832193in}}%
\pgfpathlineto{\pgfqpoint{0.752124in}{2.484941in}}%
\pgfpathlineto{\pgfqpoint{0.770670in}{2.241265in}}%
\pgfpathlineto{\pgfqpoint{0.789452in}{2.060542in}}%
\pgfpathlineto{\pgfqpoint{0.808938in}{1.989034in}}%
\pgfpathlineto{\pgfqpoint{0.827720in}{1.999512in}}%
\pgfpathlineto{\pgfqpoint{0.849789in}{2.110131in}}%
\pgfpathlineto{\pgfqpoint{0.868335in}{2.321608in}}%
\pgfpathlineto{\pgfqpoint{0.887117in}{2.704670in}}%
\pgfpathlineto{\pgfqpoint{0.906134in}{2.893796in}}%
\pgfpathlineto{\pgfqpoint{0.925150in}{2.729133in}}%
\pgfpathlineto{\pgfqpoint{0.946279in}{2.290744in}}%
\pgfpathlineto{\pgfqpoint{0.962949in}{2.107550in}}%
\pgfpathlineto{\pgfqpoint{0.982904in}{2.009095in}}%
\pgfpathlineto{\pgfqpoint{1.000278in}{1.979865in}}%
\pgfpathlineto{\pgfqpoint{1.025867in}{2.061063in}}%
\pgfpathlineto{\pgfqpoint{1.042303in}{2.164098in}}%
\pgfpathlineto{\pgfqpoint{1.060380in}{2.437889in}}%
\pgfpathlineto{\pgfqpoint{1.079631in}{2.800132in}}%
\pgfpathlineto{\pgfqpoint{1.097942in}{2.851846in}}%
\pgfpathlineto{\pgfqpoint{1.140437in}{2.173848in}}%
\pgfpathlineto{\pgfqpoint{1.154053in}{2.055184in}}%
\pgfpathlineto{\pgfqpoint{1.172835in}{1.984582in}}%
\pgfpathlineto{\pgfqpoint{1.194904in}{1.986647in}}%
\pgfpathlineto{\pgfqpoint{1.212512in}{2.049200in}}%
\pgfpathlineto{\pgfqpoint{1.232937in}{2.177518in}}%
\pgfpathlineto{\pgfqpoint{1.253363in}{2.432512in}}%
\pgfpathlineto{\pgfqpoint{1.271909in}{2.737024in}}%
\pgfpathlineto{\pgfqpoint{1.288108in}{2.853629in}}%
\pgfpathlineto{\pgfqpoint{1.309473in}{2.614610in}}%
\pgfpathlineto{\pgfqpoint{1.327550in}{2.277830in}}%
\pgfpathlineto{\pgfqpoint{1.349150in}{2.069731in}}%
\pgfpathlineto{\pgfqpoint{1.366523in}{1.996119in}}%
\pgfpathlineto{\pgfqpoint{1.387183in}{1.974257in}}%
\pgfpathlineto{\pgfqpoint{1.405964in}{2.012231in}}%
\pgfpathlineto{\pgfqpoint{1.423103in}{2.092855in}}%
\pgfpathlineto{\pgfqpoint{1.444232in}{2.312625in}}%
\pgfpathlineto{\pgfqpoint{1.462074in}{2.577581in}}%
\pgfpathlineto{\pgfqpoint{1.486491in}{2.840292in}}%
\pgfpathlineto{\pgfqpoint{1.501282in}{2.811321in}}%
\pgfpathlineto{\pgfqpoint{1.521473in}{2.784227in}}%
\pgfpathlineto{\pgfqpoint{1.539315in}{2.479755in}}%
\pgfpathlineto{\pgfqpoint{1.559270in}{2.166455in}}%
\pgfpathlineto{\pgfqpoint{1.578992in}{2.045618in}}%
\pgfpathlineto{\pgfqpoint{1.599418in}{1.982162in}}%
\pgfpathlineto{\pgfqpoint{1.617260in}{1.974434in}}%
\pgfpathlineto{\pgfqpoint{1.633928in}{2.005347in}}%
\pgfpathlineto{\pgfqpoint{1.655997in}{2.056631in}}%
\pgfpathlineto{\pgfqpoint{1.674074in}{2.151520in}}%
\pgfpathlineto{\pgfqpoint{1.693796in}{2.398793in}}%
\pgfpathlineto{\pgfqpoint{1.713751in}{2.387692in}}%
\pgfpathlineto{\pgfqpoint{1.729481in}{2.696380in}}%
\pgfpathlineto{\pgfqpoint{1.750141in}{2.828089in}}%
\pgfpathlineto{\pgfqpoint{1.770566in}{2.637079in}}%
\pgfpathlineto{\pgfqpoint{1.789112in}{2.300717in}}%
\pgfpathlineto{\pgfqpoint{1.809069in}{2.099114in}}%
\pgfpathlineto{\pgfqpoint{1.828789in}{2.086454in}}%
\pgfpathlineto{\pgfqpoint{1.845928in}{2.707679in}}%
\pgfpathlineto{\pgfqpoint{1.867057in}{2.815740in}}%
\pgfpathlineto{\pgfqpoint{1.888422in}{2.618060in}}%
\pgfpathlineto{\pgfqpoint{1.887717in}{2.399412in}}%
\pgfpathlineto{\pgfqpoint{1.906734in}{2.273315in}}%
\pgfpathlineto{\pgfqpoint{1.924107in}{2.080796in}}%
\pgfpathlineto{\pgfqpoint{1.942655in}{1.999890in}}%
\pgfpathlineto{\pgfqpoint{1.963078in}{1.969909in}}%
\pgfpathlineto{\pgfqpoint{1.982566in}{1.996097in}}%
\pgfpathlineto{\pgfqpoint{2.002286in}{2.073596in}}%
\pgfpathlineto{\pgfqpoint{2.019660in}{2.227794in}}%
\pgfpathlineto{\pgfqpoint{2.041023in}{2.596164in}}%
\pgfpathlineto{\pgfqpoint{2.058866in}{2.802483in}}%
\pgfpathlineto{\pgfqpoint{2.076944in}{2.774723in}}%
\pgfpathlineto{\pgfqpoint{2.097839in}{2.475990in}}%
\pgfpathlineto{\pgfqpoint{2.115681in}{2.186344in}}%
\pgfpathlineto{\pgfqpoint{2.139159in}{2.026044in}}%
\pgfpathlineto{\pgfqpoint{2.158175in}{1.977364in}}%
\pgfpathlineto{\pgfqpoint{2.175549in}{1.972430in}}%
\pgfpathlineto{\pgfqpoint{2.193157in}{1.994180in}}%
\pgfpathlineto{\pgfqpoint{2.211234in}{2.056100in}}%
\pgfpathlineto{\pgfqpoint{2.232597in}{2.223328in}}%
\pgfpathlineto{\pgfqpoint{2.250440in}{2.443118in}}%
\pgfpathlineto{\pgfqpoint{2.272274in}{2.771712in}}%
\pgfpathlineto{\pgfqpoint{2.289413in}{2.817617in}}%
\pgfpathlineto{\pgfqpoint{2.307961in}{2.672627in}}%
\pgfpathlineto{\pgfqpoint{2.329090in}{2.281682in}}%
\pgfpathlineto{\pgfqpoint{2.347636in}{2.098864in}}%
\pgfpathlineto{\pgfqpoint{2.365009in}{2.015434in}}%
\pgfpathlineto{\pgfqpoint{2.386843in}{1.971066in}}%
\pgfpathlineto{\pgfqpoint{2.403748in}{1.978167in}}%
\pgfpathlineto{\pgfqpoint{2.424408in}{2.033067in}}%
\pgfpathlineto{\pgfqpoint{2.442954in}{2.138132in}}%
\pgfpathlineto{\pgfqpoint{2.481456in}{2.563823in}}%
\pgfpathlineto{\pgfqpoint{2.499299in}{2.722888in}}%
\pgfpathlineto{\pgfqpoint{2.520430in}{2.811813in}}%
\pgfpathlineto{\pgfqpoint{2.539210in}{2.620168in}}%
\pgfpathlineto{\pgfqpoint{2.557289in}{2.323854in}}%
\pgfpathlineto{\pgfqpoint{2.577714in}{2.107917in}}%
\pgfpathlineto{\pgfqpoint{2.595791in}{2.015636in}}%
\pgfpathlineto{\pgfqpoint{2.617391in}{1.974643in}}%
\pgfpathlineto{\pgfqpoint{2.635233in}{1.975726in}}%
\pgfpathlineto{\pgfqpoint{2.653076in}{2.015123in}}%
\pgfpathlineto{\pgfqpoint{2.673736in}{2.102012in}}%
\pgfpathlineto{\pgfqpoint{2.691578in}{2.286152in}}%
\pgfpathlineto{\pgfqpoint{2.713647in}{2.607485in}}%
\pgfpathlineto{\pgfqpoint{2.731021in}{2.801580in}}%
\pgfpathlineto{\pgfqpoint{2.752149in}{2.773940in}}%
\pgfpathlineto{\pgfqpoint{2.788305in}{2.232256in}}%
\pgfpathlineto{\pgfqpoint{2.807556in}{2.078357in}}%
\pgfpathlineto{\pgfqpoint{2.828685in}{2.001370in}}%
\pgfpathlineto{\pgfqpoint{2.846059in}{1.972052in}}%
\pgfpathlineto{\pgfqpoint{2.866484in}{1.979343in}}%
\pgfpathlineto{\pgfqpoint{2.886673in}{2.010456in}}%
\pgfpathlineto{\pgfqpoint{2.904047in}{2.074976in}}%
\pgfpathlineto{\pgfqpoint{2.924707in}{2.253100in}}%
\pgfpathlineto{\pgfqpoint{2.942784in}{2.498748in}}%
\pgfpathlineto{\pgfqpoint{2.963209in}{2.793265in}}%
\pgfpathlineto{\pgfqpoint{2.979643in}{2.825691in}}%
\pgfpathlineto{\pgfqpoint{3.000774in}{2.758848in}}%
\pgfpathlineto{\pgfqpoint{3.022372in}{2.442936in}}%
\pgfpathlineto{\pgfqpoint{3.038807in}{2.558834in}}%
\pgfpathlineto{\pgfqpoint{3.057822in}{2.268756in}}%
\pgfpathlineto{\pgfqpoint{3.078013in}{2.084790in}}%
\pgfpathlineto{\pgfqpoint{3.097030in}{2.013239in}}%
\pgfpathlineto{\pgfqpoint{3.113464in}{1.977676in}}%
\pgfpathlineto{\pgfqpoint{3.134595in}{1.985102in}}%
\pgfpathlineto{\pgfqpoint{3.155958in}{2.016661in}}%
\pgfpathlineto{\pgfqpoint{3.174269in}{2.097821in}}%
\pgfpathlineto{\pgfqpoint{3.191643in}{2.218750in}}%
\pgfpathlineto{\pgfqpoint{3.212068in}{2.449843in}}%
\pgfpathlineto{\pgfqpoint{3.232025in}{2.698350in}}%
\pgfpathlineto{\pgfqpoint{3.250336in}{2.840665in}}%
\pgfpathlineto{\pgfqpoint{3.267475in}{2.758492in}}%
\pgfpathlineto{\pgfqpoint{3.287664in}{2.503496in}}%
\pgfpathlineto{\pgfqpoint{3.309499in}{2.333129in}}%
\pgfpathlineto{\pgfqpoint{3.324524in}{2.147873in}}%
\pgfpathlineto{\pgfqpoint{3.348001in}{2.024221in}}%
\pgfpathlineto{\pgfqpoint{3.364671in}{1.989834in}}%
\pgfpathlineto{\pgfqpoint{3.384626in}{1.977469in}}%
\pgfpathlineto{\pgfqpoint{3.403408in}{2.009734in}}%
\pgfpathlineto{\pgfqpoint{3.423365in}{2.089481in}}%
\pgfpathlineto{\pgfqpoint{3.442850in}{2.193893in}}%
\pgfpathlineto{\pgfqpoint{3.463276in}{2.336624in}}%
\pgfpathlineto{\pgfqpoint{3.481118in}{2.619353in}}%
\pgfpathlineto{\pgfqpoint{3.498961in}{2.832578in}}%
\pgfpathlineto{\pgfqpoint{3.520324in}{2.836975in}}%
\pgfpathlineto{\pgfqpoint{3.537698in}{2.656671in}}%
\pgfpathlineto{\pgfqpoint{3.577140in}{2.199234in}}%
\pgfpathlineto{\pgfqpoint{3.594982in}{2.089103in}}%
\pgfpathlineto{\pgfqpoint{3.616346in}{2.014445in}}%
\pgfpathlineto{\pgfqpoint{3.634190in}{1.983345in}}%
\pgfpathlineto{\pgfqpoint{3.653910in}{1.980838in}}%
\pgfpathlineto{\pgfqpoint{3.673630in}{2.023737in}}%
\pgfpathlineto{\pgfqpoint{3.692413in}{2.073496in}}%
\pgfpathlineto{\pgfqpoint{3.713778in}{2.172205in}}%
\pgfpathlineto{\pgfqpoint{3.728332in}{2.320249in}}%
\pgfpathlineto{\pgfqpoint{3.748992in}{2.598639in}}%
\pgfpathlineto{\pgfqpoint{3.769888in}{2.013315in}}%
\pgfpathlineto{\pgfqpoint{3.791486in}{2.095301in}}%
\pgfpathlineto{\pgfqpoint{3.804399in}{2.242767in}}%
\pgfpathlineto{\pgfqpoint{3.826937in}{2.476491in}}%
\pgfpathlineto{\pgfqpoint{3.846659in}{2.774062in}}%
\pgfpathlineto{\pgfqpoint{3.866379in}{2.885934in}}%
\pgfpathlineto{\pgfqpoint{3.885396in}{2.808468in}}%
\pgfpathlineto{\pgfqpoint{3.902300in}{2.600272in}}%
\pgfpathlineto{\pgfqpoint{3.924132in}{2.280679in}}%
\pgfpathlineto{\pgfqpoint{3.941975in}{2.128900in}}%
\pgfpathlineto{\pgfqpoint{3.960054in}{2.030699in}}%
\pgfpathlineto{\pgfqpoint{3.979540in}{1.988468in}}%
\pgfpathlineto{\pgfqpoint{3.997382in}{1.982899in}}%
\pgfpathlineto{\pgfqpoint{4.018276in}{2.027223in}}%
\pgfpathlineto{\pgfqpoint{4.037293in}{2.114971in}}%
\pgfpathlineto{\pgfqpoint{4.057719in}{2.251653in}}%
\pgfpathlineto{\pgfqpoint{4.079318in}{2.527041in}}%
\pgfpathlineto{\pgfqpoint{4.095047in}{2.775809in}}%
\pgfpathlineto{\pgfqpoint{4.113126in}{2.910659in}}%
\pgfpathlineto{\pgfqpoint{4.133315in}{2.831028in}}%
\pgfpathlineto{\pgfqpoint{4.152097in}{2.614778in}}%
\pgfpathlineto{\pgfqpoint{4.174635in}{2.291560in}}%
\pgfpathlineto{\pgfqpoint{4.192008in}{2.162138in}}%
\pgfpathlineto{\pgfqpoint{4.210556in}{2.056030in}}%
\pgfpathlineto{\pgfqpoint{4.229338in}{2.004086in}}%
\pgfpathlineto{\pgfqpoint{4.248824in}{1.987442in}}%
\pgfpathlineto{\pgfqpoint{4.267604in}{2.006755in}}%
\pgfpathlineto{\pgfqpoint{4.287795in}{2.062504in}}%
\pgfpathlineto{\pgfqpoint{4.308455in}{2.157932in}}%
\pgfpathlineto{\pgfqpoint{4.330993in}{2.407239in}}%
\pgfpathlineto{\pgfqpoint{4.344846in}{2.603080in}}%
\pgfpathlineto{\pgfqpoint{4.362688in}{2.826767in}}%
\pgfpathlineto{\pgfqpoint{4.385226in}{2.939003in}}%
\pgfpathlineto{\pgfqpoint{4.405886in}{2.860670in}}%
\pgfpathlineto{\pgfqpoint{4.423494in}{2.476261in}}%
\pgfpathlineto{\pgfqpoint{4.442745in}{2.773837in}}%
\pgfpathlineto{\pgfqpoint{4.461762in}{2.939419in}}%
\pgfpathlineto{\pgfqpoint{4.481484in}{2.894238in}}%
\pgfpathlineto{\pgfqpoint{4.481953in}{2.887247in}}%
\pgfpathlineto{\pgfqpoint{4.473735in}{2.941555in}}%
\pgfpathlineto{\pgfqpoint{4.455892in}{2.856702in}}%
\pgfpathlineto{\pgfqpoint{4.435232in}{2.425192in}}%
\pgfpathlineto{\pgfqpoint{4.417859in}{2.168514in}}%
\pgfpathlineto{\pgfqpoint{4.398373in}{2.035880in}}%
\pgfpathlineto{\pgfqpoint{4.377244in}{1.985019in}}%
\pgfpathlineto{\pgfqpoint{4.358931in}{2.033467in}}%
\pgfpathlineto{\pgfqpoint{4.338742in}{2.180092in}}%
\pgfpathlineto{\pgfqpoint{4.300943in}{2.861304in}}%
\pgfpathlineto{\pgfqpoint{4.281926in}{2.895143in}}%
\pgfpathlineto{\pgfqpoint{4.264318in}{2.613185in}}%
\pgfpathlineto{\pgfqpoint{4.243892in}{2.224019in}}%
\pgfpathlineto{\pgfqpoint{4.222998in}{2.051785in}}%
\pgfpathlineto{\pgfqpoint{4.205156in}{1.987127in}}%
\pgfpathlineto{\pgfqpoint{4.186610in}{1.999141in}}%
\pgfpathlineto{\pgfqpoint{4.166184in}{2.092264in}}%
\pgfpathlineto{\pgfqpoint{4.148105in}{2.290817in}}%
\pgfpathlineto{\pgfqpoint{4.127916in}{2.717775in}}%
\pgfpathlineto{\pgfqpoint{4.109368in}{2.893767in}}%
\pgfpathlineto{\pgfqpoint{4.089177in}{2.715118in}}%
\pgfpathlineto{\pgfqpoint{4.070866in}{2.356767in}}%
\pgfpathlineto{\pgfqpoint{4.052789in}{2.120927in}}%
\pgfpathlineto{\pgfqpoint{4.031658in}{2.002828in}}%
\pgfpathlineto{\pgfqpoint{4.012407in}{1.978569in}}%
\pgfpathlineto{\pgfqpoint{3.993390in}{2.028389in}}%
\pgfpathlineto{\pgfqpoint{3.971087in}{2.182216in}}%
\pgfpathlineto{\pgfqpoint{3.956062in}{2.413143in}}%
\pgfpathlineto{\pgfqpoint{3.935637in}{2.810116in}}%
\pgfpathlineto{\pgfqpoint{3.919672in}{2.861025in}}%
\pgfpathlineto{\pgfqpoint{3.897369in}{2.586537in}}%
\pgfpathlineto{\pgfqpoint{3.878586in}{2.252589in}}%
\pgfpathlineto{\pgfqpoint{3.856989in}{2.056884in}}%
\pgfpathlineto{\pgfqpoint{3.841258in}{2.000072in}}%
\pgfpathlineto{\pgfqpoint{3.819190in}{1.974495in}}%
\pgfpathlineto{\pgfqpoint{3.801113in}{2.000924in}}%
\pgfpathlineto{\pgfqpoint{3.783270in}{2.070516in}}%
\pgfpathlineto{\pgfqpoint{3.764019in}{2.271973in}}%
\pgfpathlineto{\pgfqpoint{3.744297in}{2.623593in}}%
\pgfpathlineto{\pgfqpoint{3.726689in}{2.846093in}}%
\pgfpathlineto{\pgfqpoint{3.706734in}{2.772154in}}%
\pgfpathlineto{\pgfqpoint{3.688186in}{2.502656in}}%
\pgfpathlineto{\pgfqpoint{3.666589in}{2.267958in}}%
\pgfpathlineto{\pgfqpoint{3.647806in}{2.238131in}}%
\pgfpathlineto{\pgfqpoint{3.628555in}{2.067728in}}%
\pgfpathlineto{\pgfqpoint{3.610713in}{1.996522in}}%
\pgfpathlineto{\pgfqpoint{3.588644in}{1.972896in}}%
\pgfpathlineto{\pgfqpoint{3.570331in}{2.013512in}}%
\pgfpathlineto{\pgfqpoint{3.554366in}{2.107578in}}%
\pgfpathlineto{\pgfqpoint{3.533471in}{2.349466in}}%
\pgfpathlineto{\pgfqpoint{3.514455in}{2.443467in}}%
\pgfpathlineto{\pgfqpoint{3.495674in}{2.753158in}}%
\pgfpathlineto{\pgfqpoint{3.477127in}{2.045169in}}%
\pgfpathlineto{\pgfqpoint{3.454589in}{2.211273in}}%
\pgfpathlineto{\pgfqpoint{3.436746in}{2.526113in}}%
\pgfpathlineto{\pgfqpoint{3.417964in}{2.811097in}}%
\pgfpathlineto{\pgfqpoint{3.397773in}{2.763116in}}%
\pgfpathlineto{\pgfqpoint{3.378053in}{2.517360in}}%
\pgfpathlineto{\pgfqpoint{3.359271in}{2.215958in}}%
\pgfpathlineto{\pgfqpoint{3.340959in}{2.068930in}}%
\pgfpathlineto{\pgfqpoint{3.321708in}{1.994177in}}%
\pgfpathlineto{\pgfqpoint{3.299874in}{1.971043in}}%
\pgfpathlineto{\pgfqpoint{3.280857in}{1.995102in}}%
\pgfpathlineto{\pgfqpoint{3.262544in}{2.062155in}}%
\pgfpathlineto{\pgfqpoint{3.243998in}{2.221405in}}%
\pgfpathlineto{\pgfqpoint{3.227564in}{2.510277in}}%
\pgfpathlineto{\pgfqpoint{3.205025in}{2.803587in}}%
\pgfpathlineto{\pgfqpoint{3.189296in}{2.809289in}}%
\pgfpathlineto{\pgfqpoint{3.167931in}{2.537967in}}%
\pgfpathlineto{\pgfqpoint{3.148445in}{2.229407in}}%
\pgfpathlineto{\pgfqpoint{3.126377in}{2.055346in}}%
\pgfpathlineto{\pgfqpoint{3.111821in}{2.003332in}}%
\pgfpathlineto{\pgfqpoint{3.088812in}{1.968873in}}%
\pgfpathlineto{\pgfqpoint{3.073553in}{1.986318in}}%
\pgfpathlineto{\pgfqpoint{3.052893in}{2.052411in}}%
\pgfpathlineto{\pgfqpoint{3.034347in}{2.148921in}}%
\pgfpathlineto{\pgfqpoint{3.012747in}{2.386655in}}%
\pgfpathlineto{\pgfqpoint{2.993025in}{2.736946in}}%
\pgfpathlineto{\pgfqpoint{2.974479in}{2.824579in}}%
\pgfpathlineto{\pgfqpoint{2.954288in}{2.717519in}}%
\pgfpathlineto{\pgfqpoint{2.938089in}{2.486367in}}%
\pgfpathlineto{\pgfqpoint{2.918603in}{2.184845in}}%
\pgfpathlineto{\pgfqpoint{2.899821in}{2.202071in}}%
\pgfpathlineto{\pgfqpoint{2.879161in}{2.038983in}}%
\pgfpathlineto{\pgfqpoint{2.860615in}{1.985525in}}%
\pgfpathlineto{\pgfqpoint{2.838781in}{1.972414in}}%
\pgfpathlineto{\pgfqpoint{2.819059in}{2.013348in}}%
\pgfpathlineto{\pgfqpoint{2.804739in}{2.086202in}}%
\pgfpathlineto{\pgfqpoint{2.804268in}{2.189129in}}%
\pgfpathlineto{\pgfqpoint{2.783139in}{2.243069in}}%
\pgfpathlineto{\pgfqpoint{2.764123in}{2.540034in}}%
\pgfpathlineto{\pgfqpoint{2.746280in}{2.778584in}}%
\pgfpathlineto{\pgfqpoint{2.724211in}{2.782520in}}%
\pgfpathlineto{\pgfqpoint{2.705664in}{2.546539in}}%
\pgfpathlineto{\pgfqpoint{2.683831in}{2.202771in}}%
\pgfpathlineto{\pgfqpoint{2.668335in}{2.087276in}}%
\pgfpathlineto{\pgfqpoint{2.648381in}{1.997753in}}%
\pgfpathlineto{\pgfqpoint{2.627250in}{1.969311in}}%
\pgfpathlineto{\pgfqpoint{2.611756in}{1.980771in}}%
\pgfpathlineto{\pgfqpoint{2.593208in}{2.037675in}}%
\pgfpathlineto{\pgfqpoint{2.566445in}{2.207038in}}%
\pgfpathlineto{\pgfqpoint{2.549540in}{2.439928in}}%
\pgfpathlineto{\pgfqpoint{2.531697in}{2.720301in}}%
\pgfpathlineto{\pgfqpoint{2.513151in}{2.814503in}}%
\pgfpathlineto{\pgfqpoint{2.494369in}{2.673701in}}%
\pgfpathlineto{\pgfqpoint{2.476058in}{2.458772in}}%
\pgfpathlineto{\pgfqpoint{2.455161in}{2.182170in}}%
\pgfpathlineto{\pgfqpoint{2.438259in}{2.067199in}}%
\pgfpathlineto{\pgfqpoint{2.416894in}{2.004753in}}%
\pgfpathlineto{\pgfqpoint{2.398817in}{1.982698in}}%
\pgfpathlineto{\pgfqpoint{2.380271in}{1.969951in}}%
\pgfpathlineto{\pgfqpoint{2.359140in}{2.002265in}}%
\pgfpathlineto{\pgfqpoint{2.339889in}{2.073114in}}%
\pgfpathlineto{\pgfqpoint{2.321577in}{2.208504in}}%
\pgfpathlineto{\pgfqpoint{2.303029in}{2.485488in}}%
\pgfpathlineto{\pgfqpoint{2.281901in}{2.746115in}}%
\pgfpathlineto{\pgfqpoint{2.265467in}{2.820446in}}%
\pgfpathlineto{\pgfqpoint{2.245745in}{2.647889in}}%
\pgfpathlineto{\pgfqpoint{2.225556in}{2.386322in}}%
\pgfpathlineto{\pgfqpoint{2.204659in}{2.210965in}}%
\pgfpathlineto{\pgfqpoint{2.185879in}{2.075106in}}%
\pgfpathlineto{\pgfqpoint{2.167566in}{2.037302in}}%
\pgfpathlineto{\pgfqpoint{2.147845in}{2.001771in}}%
\pgfpathlineto{\pgfqpoint{2.129769in}{1.984793in}}%
\pgfpathlineto{\pgfqpoint{2.111924in}{1.973487in}}%
\pgfpathlineto{\pgfqpoint{2.089621in}{2.006247in}}%
\pgfpathlineto{\pgfqpoint{2.070841in}{2.087306in}}%
\pgfpathlineto{\pgfqpoint{2.053467in}{2.269938in}}%
\pgfpathlineto{\pgfqpoint{2.036094in}{2.624421in}}%
\pgfpathlineto{\pgfqpoint{2.016137in}{2.808828in}}%
\pgfpathlineto{\pgfqpoint{1.993599in}{2.804660in}}%
\pgfpathlineto{\pgfqpoint{1.975522in}{2.597319in}}%
\pgfpathlineto{\pgfqpoint{1.956975in}{2.349356in}}%
\pgfpathlineto{\pgfqpoint{1.934906in}{2.126721in}}%
\pgfpathlineto{\pgfqpoint{1.917298in}{2.040262in}}%
\pgfpathlineto{\pgfqpoint{1.899690in}{2.000145in}}%
\pgfpathlineto{\pgfqpoint{1.875978in}{1.976563in}}%
\pgfpathlineto{\pgfqpoint{1.858606in}{1.994420in}}%
\pgfpathlineto{\pgfqpoint{1.840293in}{2.051277in}}%
\pgfpathlineto{\pgfqpoint{1.821747in}{2.147685in}}%
\pgfpathlineto{\pgfqpoint{1.802965in}{2.277686in}}%
\pgfpathlineto{\pgfqpoint{1.781834in}{2.586343in}}%
\pgfpathlineto{\pgfqpoint{1.763288in}{2.487237in}}%
\pgfpathlineto{\pgfqpoint{1.746620in}{2.759034in}}%
\pgfpathlineto{\pgfqpoint{1.726195in}{2.848582in}}%
\pgfpathlineto{\pgfqpoint{1.707178in}{2.741552in}}%
\pgfpathlineto{\pgfqpoint{1.686047in}{2.539256in}}%
\pgfpathlineto{\pgfqpoint{1.670319in}{2.350815in}}%
\pgfpathlineto{\pgfqpoint{1.648015in}{2.816623in}}%
\pgfpathlineto{\pgfqpoint{1.629937in}{2.823682in}}%
\pgfpathlineto{\pgfqpoint{1.611625in}{2.621634in}}%
\pgfpathlineto{\pgfqpoint{1.611625in}{2.621634in}}%
\pgfusepath{stroke}%
\end{pgfscope}%
\begin{pgfscope}%
\pgfpathrectangle{\pgfqpoint{0.444748in}{1.917688in}}{\pgfqpoint{4.231419in}{1.076123in}}%
\pgfusepath{clip}%
\pgfsetbuttcap%
\pgfsetroundjoin%
\definecolor{currentfill}{rgb}{0.047059,0.364706,0.647059}%
\pgfsetfillcolor{currentfill}%
\pgfsetlinewidth{1.003750pt}%
\definecolor{currentstroke}{rgb}{0.047059,0.364706,0.647059}%
\pgfsetstrokecolor{currentstroke}%
\pgfsetdash{}{0pt}%
\pgfsys@defobject{currentmarker}{\pgfqpoint{-0.010417in}{-0.010417in}}{\pgfqpoint{0.010417in}{0.010417in}}{%
\pgfpathmoveto{\pgfqpoint{0.000000in}{-0.010417in}}%
\pgfpathcurveto{\pgfqpoint{0.002763in}{-0.010417in}}{\pgfqpoint{0.005412in}{-0.009319in}}{\pgfqpoint{0.007366in}{-0.007366in}}%
\pgfpathcurveto{\pgfqpoint{0.009319in}{-0.005412in}}{\pgfqpoint{0.010417in}{-0.002763in}}{\pgfqpoint{0.010417in}{0.000000in}}%
\pgfpathcurveto{\pgfqpoint{0.010417in}{0.002763in}}{\pgfqpoint{0.009319in}{0.005412in}}{\pgfqpoint{0.007366in}{0.007366in}}%
\pgfpathcurveto{\pgfqpoint{0.005412in}{0.009319in}}{\pgfqpoint{0.002763in}{0.010417in}}{\pgfqpoint{0.000000in}{0.010417in}}%
\pgfpathcurveto{\pgfqpoint{-0.002763in}{0.010417in}}{\pgfqpoint{-0.005412in}{0.009319in}}{\pgfqpoint{-0.007366in}{0.007366in}}%
\pgfpathcurveto{\pgfqpoint{-0.009319in}{0.005412in}}{\pgfqpoint{-0.010417in}{0.002763in}}{\pgfqpoint{-0.010417in}{0.000000in}}%
\pgfpathcurveto{\pgfqpoint{-0.010417in}{-0.002763in}}{\pgfqpoint{-0.009319in}{-0.005412in}}{\pgfqpoint{-0.007366in}{-0.007366in}}%
\pgfpathcurveto{\pgfqpoint{-0.005412in}{-0.009319in}}{\pgfqpoint{-0.002763in}{-0.010417in}}{\pgfqpoint{0.000000in}{-0.010417in}}%
\pgfpathlineto{\pgfqpoint{0.000000in}{-0.010417in}}%
\pgfpathclose%
\pgfusepath{stroke,fill}%
}%
\begin{pgfscope}%
\pgfsys@transformshift{0.645536in}{2.407326in}%
\pgfsys@useobject{currentmarker}{}%
\end{pgfscope}%
\begin{pgfscope}%
\pgfsys@transformshift{0.655866in}{2.230851in}%
\pgfsys@useobject{currentmarker}{}%
\end{pgfscope}%
\begin{pgfscope}%
\pgfsys@transformshift{0.674179in}{2.068339in}%
\pgfsys@useobject{currentmarker}{}%
\end{pgfscope}%
\begin{pgfscope}%
\pgfsys@transformshift{0.694368in}{1.990209in}%
\pgfsys@useobject{currentmarker}{}%
\end{pgfscope}%
\begin{pgfscope}%
\pgfsys@transformshift{0.714091in}{2.011323in}%
\pgfsys@useobject{currentmarker}{}%
\end{pgfscope}%
\begin{pgfscope}%
\pgfsys@transformshift{0.734045in}{2.103865in}%
\pgfsys@useobject{currentmarker}{}%
\end{pgfscope}%
\begin{pgfscope}%
\pgfsys@transformshift{0.754705in}{2.325021in}%
\pgfsys@useobject{currentmarker}{}%
\end{pgfscope}%
\begin{pgfscope}%
\pgfsys@transformshift{0.771141in}{2.687404in}%
\pgfsys@useobject{currentmarker}{}%
\end{pgfscope}%
\begin{pgfscope}%
\pgfsys@transformshift{0.788983in}{2.021212in}%
\pgfsys@useobject{currentmarker}{}%
\end{pgfscope}%
\begin{pgfscope}%
\pgfsys@transformshift{0.812930in}{1.986460in}%
\pgfsys@useobject{currentmarker}{}%
\end{pgfscope}%
\begin{pgfscope}%
\pgfsys@transformshift{0.828660in}{2.037250in}%
\pgfsys@useobject{currentmarker}{}%
\end{pgfscope}%
\begin{pgfscope}%
\pgfsys@transformshift{0.848615in}{2.175184in}%
\pgfsys@useobject{currentmarker}{}%
\end{pgfscope}%
\begin{pgfscope}%
\pgfsys@transformshift{0.866928in}{2.487590in}%
\pgfsys@useobject{currentmarker}{}%
\end{pgfscope}%
\begin{pgfscope}%
\pgfsys@transformshift{0.886179in}{2.831267in}%
\pgfsys@useobject{currentmarker}{}%
\end{pgfscope}%
\begin{pgfscope}%
\pgfsys@transformshift{0.905665in}{2.855327in}%
\pgfsys@useobject{currentmarker}{}%
\end{pgfscope}%
\begin{pgfscope}%
\pgfsys@transformshift{0.923507in}{2.554414in}%
\pgfsys@useobject{currentmarker}{}%
\end{pgfscope}%
\begin{pgfscope}%
\pgfsys@transformshift{0.945341in}{2.183368in}%
\pgfsys@useobject{currentmarker}{}%
\end{pgfscope}%
\begin{pgfscope}%
\pgfsys@transformshift{0.962010in}{2.039864in}%
\pgfsys@useobject{currentmarker}{}%
\end{pgfscope}%
\begin{pgfscope}%
\pgfsys@transformshift{0.980558in}{1.982610in}%
\pgfsys@useobject{currentmarker}{}%
\end{pgfscope}%
\begin{pgfscope}%
\pgfsys@transformshift{1.002861in}{2.004278in}%
\pgfsys@useobject{currentmarker}{}%
\end{pgfscope}%
\begin{pgfscope}%
\pgfsys@transformshift{1.024929in}{2.117066in}%
\pgfsys@useobject{currentmarker}{}%
\end{pgfscope}%
\begin{pgfscope}%
\pgfsys@transformshift{1.041363in}{2.294240in}%
\pgfsys@useobject{currentmarker}{}%
\end{pgfscope}%
\begin{pgfscope}%
\pgfsys@transformshift{1.057797in}{2.649578in}%
\pgfsys@useobject{currentmarker}{}%
\end{pgfscope}%
\begin{pgfscope}%
\pgfsys@transformshift{1.078926in}{2.872062in}%
\pgfsys@useobject{currentmarker}{}%
\end{pgfscope}%
\begin{pgfscope}%
\pgfsys@transformshift{1.096065in}{2.729227in}%
\pgfsys@useobject{currentmarker}{}%
\end{pgfscope}%
\begin{pgfscope}%
\pgfsys@transformshift{1.118133in}{2.307185in}%
\pgfsys@useobject{currentmarker}{}%
\end{pgfscope}%
\begin{pgfscope}%
\pgfsys@transformshift{1.138324in}{2.074045in}%
\pgfsys@useobject{currentmarker}{}%
\end{pgfscope}%
\begin{pgfscope}%
\pgfsys@transformshift{1.156636in}{1.999960in}%
\pgfsys@useobject{currentmarker}{}%
\end{pgfscope}%
\begin{pgfscope}%
\pgfsys@transformshift{1.175418in}{1.975937in}%
\pgfsys@useobject{currentmarker}{}%
\end{pgfscope}%
\begin{pgfscope}%
\pgfsys@transformshift{1.193026in}{2.015803in}%
\pgfsys@useobject{currentmarker}{}%
\end{pgfscope}%
\begin{pgfscope}%
\pgfsys@transformshift{1.214860in}{2.124182in}%
\pgfsys@useobject{currentmarker}{}%
\end{pgfscope}%
\begin{pgfscope}%
\pgfsys@transformshift{1.232468in}{2.343020in}%
\pgfsys@useobject{currentmarker}{}%
\end{pgfscope}%
\begin{pgfscope}%
\pgfsys@transformshift{1.253597in}{2.614283in}%
\pgfsys@useobject{currentmarker}{}%
\end{pgfscope}%
\begin{pgfscope}%
\pgfsys@transformshift{1.269091in}{2.825224in}%
\pgfsys@useobject{currentmarker}{}%
\end{pgfscope}%
\begin{pgfscope}%
\pgfsys@transformshift{1.289517in}{2.788390in}%
\pgfsys@useobject{currentmarker}{}%
\end{pgfscope}%
\begin{pgfscope}%
\pgfsys@transformshift{1.310647in}{2.374836in}%
\pgfsys@useobject{currentmarker}{}%
\end{pgfscope}%
\begin{pgfscope}%
\pgfsys@transformshift{1.328959in}{2.160642in}%
\pgfsys@useobject{currentmarker}{}%
\end{pgfscope}%
\begin{pgfscope}%
\pgfsys@transformshift{1.345627in}{2.035637in}%
\pgfsys@useobject{currentmarker}{}%
\end{pgfscope}%
\begin{pgfscope}%
\pgfsys@transformshift{1.366758in}{1.982216in}%
\pgfsys@useobject{currentmarker}{}%
\end{pgfscope}%
\begin{pgfscope}%
\pgfsys@transformshift{1.386009in}{1.979267in}%
\pgfsys@useobject{currentmarker}{}%
\end{pgfscope}%
\begin{pgfscope}%
\pgfsys@transformshift{1.406904in}{2.038032in}%
\pgfsys@useobject{currentmarker}{}%
\end{pgfscope}%
\begin{pgfscope}%
\pgfsys@transformshift{1.423572in}{2.130434in}%
\pgfsys@useobject{currentmarker}{}%
\end{pgfscope}%
\begin{pgfscope}%
\pgfsys@transformshift{1.444703in}{2.373141in}%
\pgfsys@useobject{currentmarker}{}%
\end{pgfscope}%
\begin{pgfscope}%
\pgfsys@transformshift{1.464188in}{2.657275in}%
\pgfsys@useobject{currentmarker}{}%
\end{pgfscope}%
\begin{pgfscope}%
\pgfsys@transformshift{1.484379in}{2.842228in}%
\pgfsys@useobject{currentmarker}{}%
\end{pgfscope}%
\begin{pgfscope}%
\pgfsys@transformshift{1.506211in}{2.671599in}%
\pgfsys@useobject{currentmarker}{}%
\end{pgfscope}%
\begin{pgfscope}%
\pgfsys@transformshift{1.519830in}{2.454468in}%
\pgfsys@useobject{currentmarker}{}%
\end{pgfscope}%
\begin{pgfscope}%
\pgfsys@transformshift{1.540490in}{2.142989in}%
\pgfsys@useobject{currentmarker}{}%
\end{pgfscope}%
\begin{pgfscope}%
\pgfsys@transformshift{1.559270in}{2.034850in}%
\pgfsys@useobject{currentmarker}{}%
\end{pgfscope}%
\begin{pgfscope}%
\pgfsys@transformshift{1.579930in}{1.979545in}%
\pgfsys@useobject{currentmarker}{}%
\end{pgfscope}%
\begin{pgfscope}%
\pgfsys@transformshift{1.598712in}{1.970959in}%
\pgfsys@useobject{currentmarker}{}%
\end{pgfscope}%
\begin{pgfscope}%
\pgfsys@transformshift{1.615617in}{1.986728in}%
\pgfsys@useobject{currentmarker}{}%
\end{pgfscope}%
\begin{pgfscope}%
\pgfsys@transformshift{1.637215in}{2.049452in}%
\pgfsys@useobject{currentmarker}{}%
\end{pgfscope}%
\begin{pgfscope}%
\pgfsys@transformshift{1.655057in}{2.150032in}%
\pgfsys@useobject{currentmarker}{}%
\end{pgfscope}%
\begin{pgfscope}%
\pgfsys@transformshift{1.675483in}{2.366412in}%
\pgfsys@useobject{currentmarker}{}%
\end{pgfscope}%
\begin{pgfscope}%
\pgfsys@transformshift{1.691916in}{2.660631in}%
\pgfsys@useobject{currentmarker}{}%
\end{pgfscope}%
\begin{pgfscope}%
\pgfsys@transformshift{1.715865in}{2.829842in}%
\pgfsys@useobject{currentmarker}{}%
\end{pgfscope}%
\begin{pgfscope}%
\pgfsys@transformshift{1.731829in}{2.741833in}%
\pgfsys@useobject{currentmarker}{}%
\end{pgfscope}%
\begin{pgfscope}%
\pgfsys@transformshift{1.754367in}{2.390429in}%
\pgfsys@useobject{currentmarker}{}%
\end{pgfscope}%
\begin{pgfscope}%
\pgfsys@transformshift{1.771270in}{2.172407in}%
\pgfsys@useobject{currentmarker}{}%
\end{pgfscope}%
\begin{pgfscope}%
\pgfsys@transformshift{1.789349in}{2.062302in}%
\pgfsys@useobject{currentmarker}{}%
\end{pgfscope}%
\begin{pgfscope}%
\pgfsys@transformshift{1.807660in}{2.013221in}%
\pgfsys@useobject{currentmarker}{}%
\end{pgfscope}%
\begin{pgfscope}%
\pgfsys@transformshift{1.828320in}{1.975939in}%
\pgfsys@useobject{currentmarker}{}%
\end{pgfscope}%
\begin{pgfscope}%
\pgfsys@transformshift{1.846397in}{1.977224in}%
\pgfsys@useobject{currentmarker}{}%
\end{pgfscope}%
\begin{pgfscope}%
\pgfsys@transformshift{1.867291in}{2.028018in}%
\pgfsys@useobject{currentmarker}{}%
\end{pgfscope}%
\begin{pgfscope}%
\pgfsys@transformshift{1.884665in}{2.117561in}%
\pgfsys@useobject{currentmarker}{}%
\end{pgfscope}%
\begin{pgfscope}%
\pgfsys@transformshift{1.906265in}{2.270390in}%
\pgfsys@useobject{currentmarker}{}%
\end{pgfscope}%
\begin{pgfscope}%
\pgfsys@transformshift{1.924342in}{2.393912in}%
\pgfsys@useobject{currentmarker}{}%
\end{pgfscope}%
\begin{pgfscope}%
\pgfsys@transformshift{1.945707in}{2.717630in}%
\pgfsys@useobject{currentmarker}{}%
\end{pgfscope}%
\begin{pgfscope}%
\pgfsys@transformshift{1.963315in}{2.820764in}%
\pgfsys@useobject{currentmarker}{}%
\end{pgfscope}%
\begin{pgfscope}%
\pgfsys@transformshift{1.981861in}{2.699328in}%
\pgfsys@useobject{currentmarker}{}%
\end{pgfscope}%
\begin{pgfscope}%
\pgfsys@transformshift{2.002052in}{2.422042in}%
\pgfsys@useobject{currentmarker}{}%
\end{pgfscope}%
\begin{pgfscope}%
\pgfsys@transformshift{2.019894in}{2.744822in}%
\pgfsys@useobject{currentmarker}{}%
\end{pgfscope}%
\begin{pgfscope}%
\pgfsys@transformshift{2.038206in}{2.826374in}%
\pgfsys@useobject{currentmarker}{}%
\end{pgfscope}%
\begin{pgfscope}%
\pgfsys@transformshift{2.061683in}{2.669395in}%
\pgfsys@useobject{currentmarker}{}%
\end{pgfscope}%
\begin{pgfscope}%
\pgfsys@transformshift{2.077179in}{2.382053in}%
\pgfsys@useobject{currentmarker}{}%
\end{pgfscope}%
\begin{pgfscope}%
\pgfsys@transformshift{2.099013in}{2.115465in}%
\pgfsys@useobject{currentmarker}{}%
\end{pgfscope}%
\begin{pgfscope}%
\pgfsys@transformshift{2.116150in}{2.017532in}%
\pgfsys@useobject{currentmarker}{}%
\end{pgfscope}%
\begin{pgfscope}%
\pgfsys@transformshift{2.133993in}{1.974975in}%
\pgfsys@useobject{currentmarker}{}%
\end{pgfscope}%
\begin{pgfscope}%
\pgfsys@transformshift{2.152306in}{1.977139in}%
\pgfsys@useobject{currentmarker}{}%
\end{pgfscope}%
\begin{pgfscope}%
\pgfsys@transformshift{2.173201in}{2.020890in}%
\pgfsys@useobject{currentmarker}{}%
\end{pgfscope}%
\begin{pgfscope}%
\pgfsys@transformshift{2.194329in}{2.137068in}%
\pgfsys@useobject{currentmarker}{}%
\end{pgfscope}%
\begin{pgfscope}%
\pgfsys@transformshift{2.213112in}{2.359919in}%
\pgfsys@useobject{currentmarker}{}%
\end{pgfscope}%
\begin{pgfscope}%
\pgfsys@transformshift{2.232363in}{2.669665in}%
\pgfsys@useobject{currentmarker}{}%
\end{pgfscope}%
\begin{pgfscope}%
\pgfsys@transformshift{2.250205in}{2.820256in}%
\pgfsys@useobject{currentmarker}{}%
\end{pgfscope}%
\begin{pgfscope}%
\pgfsys@transformshift{2.269222in}{2.712987in}%
\pgfsys@useobject{currentmarker}{}%
\end{pgfscope}%
\begin{pgfscope}%
\pgfsys@transformshift{2.290587in}{2.347478in}%
\pgfsys@useobject{currentmarker}{}%
\end{pgfscope}%
\begin{pgfscope}%
\pgfsys@transformshift{2.306552in}{2.124513in}%
\pgfsys@useobject{currentmarker}{}%
\end{pgfscope}%
\begin{pgfscope}%
\pgfsys@transformshift{2.327212in}{2.023630in}%
\pgfsys@useobject{currentmarker}{}%
\end{pgfscope}%
\begin{pgfscope}%
\pgfsys@transformshift{2.348107in}{1.974057in}%
\pgfsys@useobject{currentmarker}{}%
\end{pgfscope}%
\begin{pgfscope}%
\pgfsys@transformshift{2.365480in}{1.970945in}%
\pgfsys@useobject{currentmarker}{}%
\end{pgfscope}%
\begin{pgfscope}%
\pgfsys@transformshift{2.383792in}{2.003997in}%
\pgfsys@useobject{currentmarker}{}%
\end{pgfscope}%
\begin{pgfscope}%
\pgfsys@transformshift{2.405391in}{2.086875in}%
\pgfsys@useobject{currentmarker}{}%
\end{pgfscope}%
\begin{pgfscope}%
\pgfsys@transformshift{2.422763in}{2.244333in}%
\pgfsys@useobject{currentmarker}{}%
\end{pgfscope}%
\begin{pgfscope}%
\pgfsys@transformshift{2.440842in}{2.498142in}%
\pgfsys@useobject{currentmarker}{}%
\end{pgfscope}%
\begin{pgfscope}%
\pgfsys@transformshift{2.461971in}{2.783672in}%
\pgfsys@useobject{currentmarker}{}%
\end{pgfscope}%
\begin{pgfscope}%
\pgfsys@transformshift{2.480048in}{2.819273in}%
\pgfsys@useobject{currentmarker}{}%
\end{pgfscope}%
\begin{pgfscope}%
\pgfsys@transformshift{2.500944in}{2.658588in}%
\pgfsys@useobject{currentmarker}{}%
\end{pgfscope}%
\begin{pgfscope}%
\pgfsys@transformshift{2.520195in}{2.345812in}%
\pgfsys@useobject{currentmarker}{}%
\end{pgfscope}%
\begin{pgfscope}%
\pgfsys@transformshift{2.540384in}{2.107981in}%
\pgfsys@useobject{currentmarker}{}%
\end{pgfscope}%
\begin{pgfscope}%
\pgfsys@transformshift{2.558698in}{2.023957in}%
\pgfsys@useobject{currentmarker}{}%
\end{pgfscope}%
\begin{pgfscope}%
\pgfsys@transformshift{2.576306in}{1.981556in}%
\pgfsys@useobject{currentmarker}{}%
\end{pgfscope}%
\begin{pgfscope}%
\pgfsys@transformshift{2.597903in}{1.970416in}%
\pgfsys@useobject{currentmarker}{}%
\end{pgfscope}%
\begin{pgfscope}%
\pgfsys@transformshift{2.615277in}{1.996696in}%
\pgfsys@useobject{currentmarker}{}%
\end{pgfscope}%
\begin{pgfscope}%
\pgfsys@transformshift{2.639929in}{2.048007in}%
\pgfsys@useobject{currentmarker}{}%
\end{pgfscope}%
\begin{pgfscope}%
\pgfsys@transformshift{2.654954in}{2.137074in}%
\pgfsys@useobject{currentmarker}{}%
\end{pgfscope}%
\begin{pgfscope}%
\pgfsys@transformshift{2.672093in}{2.292889in}%
\pgfsys@useobject{currentmarker}{}%
\end{pgfscope}%
\begin{pgfscope}%
\pgfsys@transformshift{2.693456in}{2.449427in}%
\pgfsys@useobject{currentmarker}{}%
\end{pgfscope}%
\begin{pgfscope}%
\pgfsys@transformshift{2.712473in}{2.732453in}%
\pgfsys@useobject{currentmarker}{}%
\end{pgfscope}%
\begin{pgfscope}%
\pgfsys@transformshift{2.733367in}{2.809344in}%
\pgfsys@useobject{currentmarker}{}%
\end{pgfscope}%
\begin{pgfscope}%
\pgfsys@transformshift{2.751446in}{2.635381in}%
\pgfsys@useobject{currentmarker}{}%
\end{pgfscope}%
\begin{pgfscope}%
\pgfsys@transformshift{2.769523in}{2.451917in}%
\pgfsys@useobject{currentmarker}{}%
\end{pgfscope}%
\begin{pgfscope}%
\pgfsys@transformshift{2.792529in}{2.108469in}%
\pgfsys@useobject{currentmarker}{}%
\end{pgfscope}%
\begin{pgfscope}%
\pgfsys@transformshift{2.806148in}{2.048616in}%
\pgfsys@useobject{currentmarker}{}%
\end{pgfscope}%
\begin{pgfscope}%
\pgfsys@transformshift{2.827511in}{1.986469in}%
\pgfsys@useobject{currentmarker}{}%
\end{pgfscope}%
\begin{pgfscope}%
\pgfsys@transformshift{2.847233in}{1.972430in}%
\pgfsys@useobject{currentmarker}{}%
\end{pgfscope}%
\begin{pgfscope}%
\pgfsys@transformshift{2.865779in}{1.978697in}%
\pgfsys@useobject{currentmarker}{}%
\end{pgfscope}%
\begin{pgfscope}%
\pgfsys@transformshift{2.886204in}{2.036082in}%
\pgfsys@useobject{currentmarker}{}%
\end{pgfscope}%
\begin{pgfscope}%
\pgfsys@transformshift{2.906161in}{2.137003in}%
\pgfsys@useobject{currentmarker}{}%
\end{pgfscope}%
\begin{pgfscope}%
\pgfsys@transformshift{2.923533in}{2.322179in}%
\pgfsys@useobject{currentmarker}{}%
\end{pgfscope}%
\begin{pgfscope}%
\pgfsys@transformshift{2.942784in}{2.565938in}%
\pgfsys@useobject{currentmarker}{}%
\end{pgfscope}%
\begin{pgfscope}%
\pgfsys@transformshift{2.961801in}{2.717599in}%
\pgfsys@useobject{currentmarker}{}%
\end{pgfscope}%
\begin{pgfscope}%
\pgfsys@transformshift{2.982695in}{2.827738in}%
\pgfsys@useobject{currentmarker}{}%
\end{pgfscope}%
\begin{pgfscope}%
\pgfsys@transformshift{3.000069in}{2.662216in}%
\pgfsys@useobject{currentmarker}{}%
\end{pgfscope}%
\begin{pgfscope}%
\pgfsys@transformshift{3.018616in}{2.362562in}%
\pgfsys@useobject{currentmarker}{}%
\end{pgfscope}%
\begin{pgfscope}%
\pgfsys@transformshift{3.038807in}{2.137009in}%
\pgfsys@useobject{currentmarker}{}%
\end{pgfscope}%
\begin{pgfscope}%
\pgfsys@transformshift{3.056884in}{2.035355in}%
\pgfsys@useobject{currentmarker}{}%
\end{pgfscope}%
\begin{pgfscope}%
\pgfsys@transformshift{3.075196in}{1.992609in}%
\pgfsys@useobject{currentmarker}{}%
\end{pgfscope}%
\begin{pgfscope}%
\pgfsys@transformshift{3.095856in}{1.972950in}%
\pgfsys@useobject{currentmarker}{}%
\end{pgfscope}%
\begin{pgfscope}%
\pgfsys@transformshift{3.117221in}{2.002906in}%
\pgfsys@useobject{currentmarker}{}%
\end{pgfscope}%
\begin{pgfscope}%
\pgfsys@transformshift{3.135063in}{2.055500in}%
\pgfsys@useobject{currentmarker}{}%
\end{pgfscope}%
\begin{pgfscope}%
\pgfsys@transformshift{3.135298in}{2.120828in}%
\pgfsys@useobject{currentmarker}{}%
\end{pgfscope}%
\begin{pgfscope}%
\pgfsys@transformshift{3.153846in}{2.150494in}%
\pgfsys@useobject{currentmarker}{}%
\end{pgfscope}%
\begin{pgfscope}%
\pgfsys@transformshift{3.174975in}{2.350864in}%
\pgfsys@useobject{currentmarker}{}%
\end{pgfscope}%
\begin{pgfscope}%
\pgfsys@transformshift{3.191877in}{1.972984in}%
\pgfsys@useobject{currentmarker}{}%
\end{pgfscope}%
\begin{pgfscope}%
\pgfsys@transformshift{3.214182in}{2.008422in}%
\pgfsys@useobject{currentmarker}{}%
\end{pgfscope}%
\begin{pgfscope}%
\pgfsys@transformshift{3.231319in}{2.086892in}%
\pgfsys@useobject{currentmarker}{}%
\end{pgfscope}%
\begin{pgfscope}%
\pgfsys@transformshift{3.250805in}{2.250409in}%
\pgfsys@useobject{currentmarker}{}%
\end{pgfscope}%
\begin{pgfscope}%
\pgfsys@transformshift{3.270762in}{2.548300in}%
\pgfsys@useobject{currentmarker}{}%
\end{pgfscope}%
\begin{pgfscope}%
\pgfsys@transformshift{3.290716in}{2.796930in}%
\pgfsys@useobject{currentmarker}{}%
\end{pgfscope}%
\begin{pgfscope}%
\pgfsys@transformshift{3.309733in}{2.833148in}%
\pgfsys@useobject{currentmarker}{}%
\end{pgfscope}%
\begin{pgfscope}%
\pgfsys@transformshift{3.328281in}{2.610775in}%
\pgfsys@useobject{currentmarker}{}%
\end{pgfscope}%
\begin{pgfscope}%
\pgfsys@transformshift{3.348941in}{2.313848in}%
\pgfsys@useobject{currentmarker}{}%
\end{pgfscope}%
\begin{pgfscope}%
\pgfsys@transformshift{3.366549in}{2.122162in}%
\pgfsys@useobject{currentmarker}{}%
\end{pgfscope}%
\begin{pgfscope}%
\pgfsys@transformshift{3.384391in}{2.037815in}%
\pgfsys@useobject{currentmarker}{}%
\end{pgfscope}%
\begin{pgfscope}%
\pgfsys@transformshift{3.406694in}{2.005986in}%
\pgfsys@useobject{currentmarker}{}%
\end{pgfscope}%
\begin{pgfscope}%
\pgfsys@transformshift{3.424537in}{1.975585in}%
\pgfsys@useobject{currentmarker}{}%
\end{pgfscope}%
\begin{pgfscope}%
\pgfsys@transformshift{3.442616in}{1.985713in}%
\pgfsys@useobject{currentmarker}{}%
\end{pgfscope}%
\begin{pgfscope}%
\pgfsys@transformshift{3.462101in}{2.039269in}%
\pgfsys@useobject{currentmarker}{}%
\end{pgfscope}%
\begin{pgfscope}%
\pgfsys@transformshift{3.481822in}{2.129483in}%
\pgfsys@useobject{currentmarker}{}%
\end{pgfscope}%
\begin{pgfscope}%
\pgfsys@transformshift{3.500135in}{2.325620in}%
\pgfsys@useobject{currentmarker}{}%
\end{pgfscope}%
\begin{pgfscope}%
\pgfsys@transformshift{3.517507in}{2.520969in}%
\pgfsys@useobject{currentmarker}{}%
\end{pgfscope}%
\begin{pgfscope}%
\pgfsys@transformshift{3.539812in}{2.602937in}%
\pgfsys@useobject{currentmarker}{}%
\end{pgfscope}%
\begin{pgfscope}%
\pgfsys@transformshift{3.557183in}{2.622314in}%
\pgfsys@useobject{currentmarker}{}%
\end{pgfscope}%
\begin{pgfscope}%
\pgfsys@transformshift{3.580895in}{2.859961in}%
\pgfsys@useobject{currentmarker}{}%
\end{pgfscope}%
\begin{pgfscope}%
\pgfsys@transformshift{3.595451in}{2.820823in}%
\pgfsys@useobject{currentmarker}{}%
\end{pgfscope}%
\begin{pgfscope}%
\pgfsys@transformshift{3.617754in}{2.508164in}%
\pgfsys@useobject{currentmarker}{}%
\end{pgfscope}%
\begin{pgfscope}%
\pgfsys@transformshift{3.634893in}{2.277483in}%
\pgfsys@useobject{currentmarker}{}%
\end{pgfscope}%
\begin{pgfscope}%
\pgfsys@transformshift{3.652267in}{2.107103in}%
\pgfsys@useobject{currentmarker}{}%
\end{pgfscope}%
\begin{pgfscope}%
\pgfsys@transformshift{3.672222in}{2.024732in}%
\pgfsys@useobject{currentmarker}{}%
\end{pgfscope}%
\begin{pgfscope}%
\pgfsys@transformshift{3.691004in}{1.992562in}%
\pgfsys@useobject{currentmarker}{}%
\end{pgfscope}%
\begin{pgfscope}%
\pgfsys@transformshift{3.710960in}{1.979934in}%
\pgfsys@useobject{currentmarker}{}%
\end{pgfscope}%
\begin{pgfscope}%
\pgfsys@transformshift{3.729037in}{2.017789in}%
\pgfsys@useobject{currentmarker}{}%
\end{pgfscope}%
\begin{pgfscope}%
\pgfsys@transformshift{3.751341in}{2.101012in}%
\pgfsys@useobject{currentmarker}{}%
\end{pgfscope}%
\begin{pgfscope}%
\pgfsys@transformshift{3.769417in}{2.220126in}%
\pgfsys@useobject{currentmarker}{}%
\end{pgfscope}%
\begin{pgfscope}%
\pgfsys@transformshift{3.788905in}{2.460141in}%
\pgfsys@useobject{currentmarker}{}%
\end{pgfscope}%
\begin{pgfscope}%
\pgfsys@transformshift{3.807922in}{2.724642in}%
\pgfsys@useobject{currentmarker}{}%
\end{pgfscope}%
\begin{pgfscope}%
\pgfsys@transformshift{3.826702in}{2.878485in}%
\pgfsys@useobject{currentmarker}{}%
\end{pgfscope}%
\begin{pgfscope}%
\pgfsys@transformshift{3.845953in}{2.850300in}%
\pgfsys@useobject{currentmarker}{}%
\end{pgfscope}%
\begin{pgfscope}%
\pgfsys@transformshift{3.864501in}{2.671742in}%
\pgfsys@useobject{currentmarker}{}%
\end{pgfscope}%
\begin{pgfscope}%
\pgfsys@transformshift{3.883987in}{2.389851in}%
\pgfsys@useobject{currentmarker}{}%
\end{pgfscope}%
\begin{pgfscope}%
\pgfsys@transformshift{3.903004in}{2.216452in}%
\pgfsys@useobject{currentmarker}{}%
\end{pgfscope}%
\begin{pgfscope}%
\pgfsys@transformshift{3.925778in}{2.071254in}%
\pgfsys@useobject{currentmarker}{}%
\end{pgfscope}%
\begin{pgfscope}%
\pgfsys@transformshift{3.944323in}{2.003166in}%
\pgfsys@useobject{currentmarker}{}%
\end{pgfscope}%
\begin{pgfscope}%
\pgfsys@transformshift{3.964514in}{1.999013in}%
\pgfsys@useobject{currentmarker}{}%
\end{pgfscope}%
\begin{pgfscope}%
\pgfsys@transformshift{3.982122in}{1.981657in}%
\pgfsys@useobject{currentmarker}{}%
\end{pgfscope}%
\begin{pgfscope}%
\pgfsys@transformshift{4.001374in}{2.014491in}%
\pgfsys@useobject{currentmarker}{}%
\end{pgfscope}%
\begin{pgfscope}%
\pgfsys@transformshift{4.019685in}{2.081641in}%
\pgfsys@useobject{currentmarker}{}%
\end{pgfscope}%
\begin{pgfscope}%
\pgfsys@transformshift{4.040345in}{2.214750in}%
\pgfsys@useobject{currentmarker}{}%
\end{pgfscope}%
\begin{pgfscope}%
\pgfsys@transformshift{4.055607in}{2.402417in}%
\pgfsys@useobject{currentmarker}{}%
\end{pgfscope}%
\begin{pgfscope}%
\pgfsys@transformshift{4.077204in}{2.629711in}%
\pgfsys@useobject{currentmarker}{}%
\end{pgfscope}%
\begin{pgfscope}%
\pgfsys@transformshift{4.098099in}{2.823111in}%
\pgfsys@useobject{currentmarker}{}%
\end{pgfscope}%
\begin{pgfscope}%
\pgfsys@transformshift{4.117115in}{2.913261in}%
\pgfsys@useobject{currentmarker}{}%
\end{pgfscope}%
\begin{pgfscope}%
\pgfsys@transformshift{4.134960in}{2.815154in}%
\pgfsys@useobject{currentmarker}{}%
\end{pgfscope}%
\begin{pgfscope}%
\pgfsys@transformshift{4.150923in}{2.593007in}%
\pgfsys@useobject{currentmarker}{}%
\end{pgfscope}%
\begin{pgfscope}%
\pgfsys@transformshift{4.173462in}{2.274233in}%
\pgfsys@useobject{currentmarker}{}%
\end{pgfscope}%
\begin{pgfscope}%
\pgfsys@transformshift{4.192948in}{2.110773in}%
\pgfsys@useobject{currentmarker}{}%
\end{pgfscope}%
\begin{pgfscope}%
\pgfsys@transformshift{4.211025in}{2.797752in}%
\pgfsys@useobject{currentmarker}{}%
\end{pgfscope}%
\begin{pgfscope}%
\pgfsys@transformshift{4.229573in}{2.550932in}%
\pgfsys@useobject{currentmarker}{}%
\end{pgfscope}%
\begin{pgfscope}%
\pgfsys@transformshift{4.248119in}{2.261677in}%
\pgfsys@useobject{currentmarker}{}%
\end{pgfscope}%
\begin{pgfscope}%
\pgfsys@transformshift{4.272301in}{2.080004in}%
\pgfsys@useobject{currentmarker}{}%
\end{pgfscope}%
\begin{pgfscope}%
\pgfsys@transformshift{4.288735in}{2.010401in}%
\pgfsys@useobject{currentmarker}{}%
\end{pgfscope}%
\begin{pgfscope}%
\pgfsys@transformshift{4.308690in}{1.986796in}%
\pgfsys@useobject{currentmarker}{}%
\end{pgfscope}%
\begin{pgfscope}%
\pgfsys@transformshift{4.329350in}{2.032404in}%
\pgfsys@useobject{currentmarker}{}%
\end{pgfscope}%
\begin{pgfscope}%
\pgfsys@transformshift{4.344611in}{2.092357in}%
\pgfsys@useobject{currentmarker}{}%
\end{pgfscope}%
\begin{pgfscope}%
\pgfsys@transformshift{4.363157in}{2.225806in}%
\pgfsys@useobject{currentmarker}{}%
\end{pgfscope}%
\begin{pgfscope}%
\pgfsys@transformshift{4.384991in}{2.504385in}%
\pgfsys@useobject{currentmarker}{}%
\end{pgfscope}%
\begin{pgfscope}%
\pgfsys@transformshift{4.404008in}{2.482324in}%
\pgfsys@useobject{currentmarker}{}%
\end{pgfscope}%
\begin{pgfscope}%
\pgfsys@transformshift{4.422319in}{2.778515in}%
\pgfsys@useobject{currentmarker}{}%
\end{pgfscope}%
\begin{pgfscope}%
\pgfsys@transformshift{4.442510in}{2.939749in}%
\pgfsys@useobject{currentmarker}{}%
\end{pgfscope}%
\begin{pgfscope}%
\pgfsys@transformshift{4.460824in}{2.908728in}%
\pgfsys@useobject{currentmarker}{}%
\end{pgfscope}%
\begin{pgfscope}%
\pgfsys@transformshift{4.480075in}{2.684013in}%
\pgfsys@useobject{currentmarker}{}%
\end{pgfscope}%
\begin{pgfscope}%
\pgfsys@transformshift{4.478901in}{2.738893in}%
\pgfsys@useobject{currentmarker}{}%
\end{pgfscope}%
\begin{pgfscope}%
\pgfsys@transformshift{4.474674in}{2.836014in}%
\pgfsys@useobject{currentmarker}{}%
\end{pgfscope}%
\begin{pgfscope}%
\pgfsys@transformshift{4.454249in}{2.944896in}%
\pgfsys@useobject{currentmarker}{}%
\end{pgfscope}%
\begin{pgfscope}%
\pgfsys@transformshift{4.436641in}{2.734186in}%
\pgfsys@useobject{currentmarker}{}%
\end{pgfscope}%
\begin{pgfscope}%
\pgfsys@transformshift{4.418095in}{2.334889in}%
\pgfsys@useobject{currentmarker}{}%
\end{pgfscope}%
\begin{pgfscope}%
\pgfsys@transformshift{4.396261in}{2.095842in}%
\pgfsys@useobject{currentmarker}{}%
\end{pgfscope}%
\begin{pgfscope}%
\pgfsys@transformshift{4.377010in}{2.002212in}%
\pgfsys@useobject{currentmarker}{}%
\end{pgfscope}%
\begin{pgfscope}%
\pgfsys@transformshift{4.358931in}{1.995567in}%
\pgfsys@useobject{currentmarker}{}%
\end{pgfscope}%
\begin{pgfscope}%
\pgfsys@transformshift{4.340619in}{2.067757in}%
\pgfsys@useobject{currentmarker}{}%
\end{pgfscope}%
\begin{pgfscope}%
\pgfsys@transformshift{4.319020in}{2.292313in}%
\pgfsys@useobject{currentmarker}{}%
\end{pgfscope}%
\begin{pgfscope}%
\pgfsys@transformshift{4.299768in}{2.679042in}%
\pgfsys@useobject{currentmarker}{}%
\end{pgfscope}%
\begin{pgfscope}%
\pgfsys@transformshift{4.281223in}{2.910914in}%
\pgfsys@useobject{currentmarker}{}%
\end{pgfscope}%
\begin{pgfscope}%
\pgfsys@transformshift{4.262675in}{2.823719in}%
\pgfsys@useobject{currentmarker}{}%
\end{pgfscope}%
\begin{pgfscope}%
\pgfsys@transformshift{4.243892in}{2.420513in}%
\pgfsys@useobject{currentmarker}{}%
\end{pgfscope}%
\begin{pgfscope}%
\pgfsys@transformshift{4.225347in}{2.158519in}%
\pgfsys@useobject{currentmarker}{}%
\end{pgfscope}%
\begin{pgfscope}%
\pgfsys@transformshift{4.207504in}{2.030385in}%
\pgfsys@useobject{currentmarker}{}%
\end{pgfscope}%
\begin{pgfscope}%
\pgfsys@transformshift{4.184496in}{1.981038in}%
\pgfsys@useobject{currentmarker}{}%
\end{pgfscope}%
\begin{pgfscope}%
\pgfsys@transformshift{4.166888in}{2.025662in}%
\pgfsys@useobject{currentmarker}{}%
\end{pgfscope}%
\begin{pgfscope}%
\pgfsys@transformshift{4.147871in}{2.145814in}%
\pgfsys@useobject{currentmarker}{}%
\end{pgfscope}%
\begin{pgfscope}%
\pgfsys@transformshift{4.128151in}{2.420822in}%
\pgfsys@useobject{currentmarker}{}%
\end{pgfscope}%
\begin{pgfscope}%
\pgfsys@transformshift{4.110308in}{2.765684in}%
\pgfsys@useobject{currentmarker}{}%
\end{pgfscope}%
\begin{pgfscope}%
\pgfsys@transformshift{4.089177in}{2.889475in}%
\pgfsys@useobject{currentmarker}{}%
\end{pgfscope}%
\begin{pgfscope}%
\pgfsys@transformshift{4.072509in}{2.705281in}%
\pgfsys@useobject{currentmarker}{}%
\end{pgfscope}%
\begin{pgfscope}%
\pgfsys@transformshift{4.051146in}{2.274643in}%
\pgfsys@useobject{currentmarker}{}%
\end{pgfscope}%
\begin{pgfscope}%
\pgfsys@transformshift{4.033538in}{2.088188in}%
\pgfsys@useobject{currentmarker}{}%
\end{pgfscope}%
\begin{pgfscope}%
\pgfsys@transformshift{4.010764in}{1.995920in}%
\pgfsys@useobject{currentmarker}{}%
\end{pgfscope}%
\begin{pgfscope}%
\pgfsys@transformshift{3.993627in}{1.983517in}%
\pgfsys@useobject{currentmarker}{}%
\end{pgfscope}%
\begin{pgfscope}%
\pgfsys@transformshift{3.973670in}{2.044632in}%
\pgfsys@useobject{currentmarker}{}%
\end{pgfscope}%
\begin{pgfscope}%
\pgfsys@transformshift{3.955122in}{2.183392in}%
\pgfsys@useobject{currentmarker}{}%
\end{pgfscope}%
\begin{pgfscope}%
\pgfsys@transformshift{3.936342in}{2.500118in}%
\pgfsys@useobject{currentmarker}{}%
\end{pgfscope}%
\begin{pgfscope}%
\pgfsys@transformshift{3.918263in}{2.826150in}%
\pgfsys@useobject{currentmarker}{}%
\end{pgfscope}%
\begin{pgfscope}%
\pgfsys@transformshift{3.897134in}{2.827863in}%
\pgfsys@useobject{currentmarker}{}%
\end{pgfscope}%
\begin{pgfscope}%
\pgfsys@transformshift{3.877178in}{2.486760in}%
\pgfsys@useobject{currentmarker}{}%
\end{pgfscope}%
\begin{pgfscope}%
\pgfsys@transformshift{3.858163in}{2.219760in}%
\pgfsys@useobject{currentmarker}{}%
\end{pgfscope}%
\begin{pgfscope}%
\pgfsys@transformshift{3.839615in}{2.063108in}%
\pgfsys@useobject{currentmarker}{}%
\end{pgfscope}%
\begin{pgfscope}%
\pgfsys@transformshift{3.821304in}{1.991762in}%
\pgfsys@useobject{currentmarker}{}%
\end{pgfscope}%
\begin{pgfscope}%
\pgfsys@transformshift{3.799469in}{1.983467in}%
\pgfsys@useobject{currentmarker}{}%
\end{pgfscope}%
\begin{pgfscope}%
\pgfsys@transformshift{3.780922in}{2.028070in}%
\pgfsys@useobject{currentmarker}{}%
\end{pgfscope}%
\begin{pgfscope}%
\pgfsys@transformshift{3.765662in}{2.122629in}%
\pgfsys@useobject{currentmarker}{}%
\end{pgfscope}%
\begin{pgfscope}%
\pgfsys@transformshift{3.741479in}{2.477731in}%
\pgfsys@useobject{currentmarker}{}%
\end{pgfscope}%
\begin{pgfscope}%
\pgfsys@transformshift{3.725046in}{2.677447in}%
\pgfsys@useobject{currentmarker}{}%
\end{pgfscope}%
\begin{pgfscope}%
\pgfsys@transformshift{3.704386in}{2.853300in}%
\pgfsys@useobject{currentmarker}{}%
\end{pgfscope}%
\begin{pgfscope}%
\pgfsys@transformshift{3.685135in}{2.740003in}%
\pgfsys@useobject{currentmarker}{}%
\end{pgfscope}%
\begin{pgfscope}%
\pgfsys@transformshift{3.666352in}{2.389991in}%
\pgfsys@useobject{currentmarker}{}%
\end{pgfscope}%
\begin{pgfscope}%
\pgfsys@transformshift{3.647806in}{2.145672in}%
\pgfsys@useobject{currentmarker}{}%
\end{pgfscope}%
\begin{pgfscope}%
\pgfsys@transformshift{3.628555in}{2.029728in}%
\pgfsys@useobject{currentmarker}{}%
\end{pgfscope}%
\begin{pgfscope}%
\pgfsys@transformshift{3.609773in}{1.978099in}%
\pgfsys@useobject{currentmarker}{}%
\end{pgfscope}%
\begin{pgfscope}%
\pgfsys@transformshift{3.590287in}{1.976983in}%
\pgfsys@useobject{currentmarker}{}%
\end{pgfscope}%
\begin{pgfscope}%
\pgfsys@transformshift{3.567750in}{2.024086in}%
\pgfsys@useobject{currentmarker}{}%
\end{pgfscope}%
\begin{pgfscope}%
\pgfsys@transformshift{3.550845in}{2.109948in}%
\pgfsys@useobject{currentmarker}{}%
\end{pgfscope}%
\begin{pgfscope}%
\pgfsys@transformshift{3.534411in}{2.303512in}%
\pgfsys@useobject{currentmarker}{}%
\end{pgfscope}%
\begin{pgfscope}%
\pgfsys@transformshift{3.513282in}{2.656322in}%
\pgfsys@useobject{currentmarker}{}%
\end{pgfscope}%
\begin{pgfscope}%
\pgfsys@transformshift{3.493795in}{2.832064in}%
\pgfsys@useobject{currentmarker}{}%
\end{pgfscope}%
\begin{pgfscope}%
\pgfsys@transformshift{3.475014in}{2.803788in}%
\pgfsys@useobject{currentmarker}{}%
\end{pgfscope}%
\begin{pgfscope}%
\pgfsys@transformshift{3.450832in}{2.638833in}%
\pgfsys@useobject{currentmarker}{}%
\end{pgfscope}%
\begin{pgfscope}%
\pgfsys@transformshift{3.436276in}{2.334019in}%
\pgfsys@useobject{currentmarker}{}%
\end{pgfscope}%
\begin{pgfscope}%
\pgfsys@transformshift{3.417024in}{2.128126in}%
\pgfsys@useobject{currentmarker}{}%
\end{pgfscope}%
\begin{pgfscope}%
\pgfsys@transformshift{3.401060in}{2.147596in}%
\pgfsys@useobject{currentmarker}{}%
\end{pgfscope}%
\begin{pgfscope}%
\pgfsys@transformshift{3.378756in}{2.831732in}%
\pgfsys@useobject{currentmarker}{}%
\end{pgfscope}%
\begin{pgfscope}%
\pgfsys@transformshift{3.357159in}{2.694689in}%
\pgfsys@useobject{currentmarker}{}%
\end{pgfscope}%
\begin{pgfscope}%
\pgfsys@transformshift{3.342837in}{2.403358in}%
\pgfsys@useobject{currentmarker}{}%
\end{pgfscope}%
\begin{pgfscope}%
\pgfsys@transformshift{3.320534in}{2.133940in}%
\pgfsys@useobject{currentmarker}{}%
\end{pgfscope}%
\begin{pgfscope}%
\pgfsys@transformshift{3.302455in}{2.020193in}%
\pgfsys@useobject{currentmarker}{}%
\end{pgfscope}%
\begin{pgfscope}%
\pgfsys@transformshift{3.282500in}{1.980666in}%
\pgfsys@useobject{currentmarker}{}%
\end{pgfscope}%
\begin{pgfscope}%
\pgfsys@transformshift{3.265832in}{1.975953in}%
\pgfsys@useobject{currentmarker}{}%
\end{pgfscope}%
\begin{pgfscope}%
\pgfsys@transformshift{3.243293in}{2.033489in}%
\pgfsys@useobject{currentmarker}{}%
\end{pgfscope}%
\begin{pgfscope}%
\pgfsys@transformshift{3.225216in}{2.153017in}%
\pgfsys@useobject{currentmarker}{}%
\end{pgfscope}%
\begin{pgfscope}%
\pgfsys@transformshift{3.209485in}{2.375217in}%
\pgfsys@useobject{currentmarker}{}%
\end{pgfscope}%
\begin{pgfscope}%
\pgfsys@transformshift{3.186244in}{2.756505in}%
\pgfsys@useobject{currentmarker}{}%
\end{pgfscope}%
\begin{pgfscope}%
\pgfsys@transformshift{3.166522in}{2.803016in}%
\pgfsys@useobject{currentmarker}{}%
\end{pgfscope}%
\begin{pgfscope}%
\pgfsys@transformshift{3.149149in}{2.610191in}%
\pgfsys@useobject{currentmarker}{}%
\end{pgfscope}%
\begin{pgfscope}%
\pgfsys@transformshift{3.127785in}{2.266039in}%
\pgfsys@useobject{currentmarker}{}%
\end{pgfscope}%
\begin{pgfscope}%
\pgfsys@transformshift{3.109238in}{2.098033in}%
\pgfsys@useobject{currentmarker}{}%
\end{pgfscope}%
\begin{pgfscope}%
\pgfsys@transformshift{3.089986in}{2.007027in}%
\pgfsys@useobject{currentmarker}{}%
\end{pgfscope}%
\begin{pgfscope}%
\pgfsys@transformshift{3.068623in}{1.970641in}%
\pgfsys@useobject{currentmarker}{}%
\end{pgfscope}%
\begin{pgfscope}%
\pgfsys@transformshift{3.053127in}{1.987857in}%
\pgfsys@useobject{currentmarker}{}%
\end{pgfscope}%
\begin{pgfscope}%
\pgfsys@transformshift{3.031529in}{2.049903in}%
\pgfsys@useobject{currentmarker}{}%
\end{pgfscope}%
\begin{pgfscope}%
\pgfsys@transformshift{3.013685in}{2.191340in}%
\pgfsys@useobject{currentmarker}{}%
\end{pgfscope}%
\begin{pgfscope}%
\pgfsys@transformshift{2.994670in}{2.471903in}%
\pgfsys@useobject{currentmarker}{}%
\end{pgfscope}%
\begin{pgfscope}%
\pgfsys@transformshift{2.972836in}{2.772673in}%
\pgfsys@useobject{currentmarker}{}%
\end{pgfscope}%
\begin{pgfscope}%
\pgfsys@transformshift{2.955462in}{2.815660in}%
\pgfsys@useobject{currentmarker}{}%
\end{pgfscope}%
\begin{pgfscope}%
\pgfsys@transformshift{2.936211in}{2.581801in}%
\pgfsys@useobject{currentmarker}{}%
\end{pgfscope}%
\begin{pgfscope}%
\pgfsys@transformshift{2.918369in}{2.392681in}%
\pgfsys@useobject{currentmarker}{}%
\end{pgfscope}%
\begin{pgfscope}%
\pgfsys@transformshift{2.899586in}{2.141858in}%
\pgfsys@useobject{currentmarker}{}%
\end{pgfscope}%
\begin{pgfscope}%
\pgfsys@transformshift{2.881275in}{2.024930in}%
\pgfsys@useobject{currentmarker}{}%
\end{pgfscope}%
\begin{pgfscope}%
\pgfsys@transformshift{2.852397in}{1.973058in}%
\pgfsys@useobject{currentmarker}{}%
\end{pgfscope}%
\begin{pgfscope}%
\pgfsys@transformshift{2.840658in}{1.973113in}%
\pgfsys@useobject{currentmarker}{}%
\end{pgfscope}%
\begin{pgfscope}%
\pgfsys@transformshift{2.821642in}{2.013896in}%
\pgfsys@useobject{currentmarker}{}%
\end{pgfscope}%
\begin{pgfscope}%
\pgfsys@transformshift{2.803330in}{2.086888in}%
\pgfsys@useobject{currentmarker}{}%
\end{pgfscope}%
\begin{pgfscope}%
\pgfsys@transformshift{2.782905in}{2.275112in}%
\pgfsys@useobject{currentmarker}{}%
\end{pgfscope}%
\begin{pgfscope}%
\pgfsys@transformshift{2.763888in}{2.507623in}%
\pgfsys@useobject{currentmarker}{}%
\end{pgfscope}%
\begin{pgfscope}%
\pgfsys@transformshift{2.743932in}{2.780648in}%
\pgfsys@useobject{currentmarker}{}%
\end{pgfscope}%
\begin{pgfscope}%
\pgfsys@transformshift{2.726324in}{2.795325in}%
\pgfsys@useobject{currentmarker}{}%
\end{pgfscope}%
\begin{pgfscope}%
\pgfsys@transformshift{2.707309in}{2.540607in}%
\pgfsys@useobject{currentmarker}{}%
\end{pgfscope}%
\begin{pgfscope}%
\pgfsys@transformshift{2.685709in}{2.235278in}%
\pgfsys@useobject{currentmarker}{}%
\end{pgfscope}%
\begin{pgfscope}%
\pgfsys@transformshift{2.667396in}{2.083811in}%
\pgfsys@useobject{currentmarker}{}%
\end{pgfscope}%
\begin{pgfscope}%
\pgfsys@transformshift{2.648850in}{2.011981in}%
\pgfsys@useobject{currentmarker}{}%
\end{pgfscope}%
\begin{pgfscope}%
\pgfsys@transformshift{2.626781in}{1.975335in}%
\pgfsys@useobject{currentmarker}{}%
\end{pgfscope}%
\begin{pgfscope}%
\pgfsys@transformshift{2.606356in}{1.978041in}%
\pgfsys@useobject{currentmarker}{}%
\end{pgfscope}%
\begin{pgfscope}%
\pgfsys@transformshift{2.588044in}{2.018321in}%
\pgfsys@useobject{currentmarker}{}%
\end{pgfscope}%
\begin{pgfscope}%
\pgfsys@transformshift{2.569262in}{2.109739in}%
\pgfsys@useobject{currentmarker}{}%
\end{pgfscope}%
\begin{pgfscope}%
\pgfsys@transformshift{2.553063in}{2.292734in}%
\pgfsys@useobject{currentmarker}{}%
\end{pgfscope}%
\begin{pgfscope}%
\pgfsys@transformshift{2.534749in}{2.622268in}%
\pgfsys@useobject{currentmarker}{}%
\end{pgfscope}%
\begin{pgfscope}%
\pgfsys@transformshift{2.513855in}{2.807291in}%
\pgfsys@useobject{currentmarker}{}%
\end{pgfscope}%
\begin{pgfscope}%
\pgfsys@transformshift{2.492726in}{2.700020in}%
\pgfsys@useobject{currentmarker}{}%
\end{pgfscope}%
\begin{pgfscope}%
\pgfsys@transformshift{2.474413in}{2.465935in}%
\pgfsys@useobject{currentmarker}{}%
\end{pgfscope}%
\begin{pgfscope}%
\pgfsys@transformshift{2.455161in}{2.205475in}%
\pgfsys@useobject{currentmarker}{}%
\end{pgfscope}%
\begin{pgfscope}%
\pgfsys@transformshift{2.438024in}{2.081633in}%
\pgfsys@useobject{currentmarker}{}%
\end{pgfscope}%
\begin{pgfscope}%
\pgfsys@transformshift{2.415487in}{2.000829in}%
\pgfsys@useobject{currentmarker}{}%
\end{pgfscope}%
\begin{pgfscope}%
\pgfsys@transformshift{2.401868in}{1.972875in}%
\pgfsys@useobject{currentmarker}{}%
\end{pgfscope}%
\begin{pgfscope}%
\pgfsys@transformshift{2.380271in}{1.975832in}%
\pgfsys@useobject{currentmarker}{}%
\end{pgfscope}%
\begin{pgfscope}%
\pgfsys@transformshift{2.360783in}{2.003982in}%
\pgfsys@useobject{currentmarker}{}%
\end{pgfscope}%
\begin{pgfscope}%
\pgfsys@transformshift{2.342003in}{2.080020in}%
\pgfsys@useobject{currentmarker}{}%
\end{pgfscope}%
\begin{pgfscope}%
\pgfsys@transformshift{2.317351in}{2.327964in}%
\pgfsys@useobject{currentmarker}{}%
\end{pgfscope}%
\begin{pgfscope}%
\pgfsys@transformshift{2.303264in}{2.507799in}%
\pgfsys@useobject{currentmarker}{}%
\end{pgfscope}%
\begin{pgfscope}%
\pgfsys@transformshift{2.283309in}{2.767273in}%
\pgfsys@useobject{currentmarker}{}%
\end{pgfscope}%
\begin{pgfscope}%
\pgfsys@transformshift{2.265936in}{2.819816in}%
\pgfsys@useobject{currentmarker}{}%
\end{pgfscope}%
\begin{pgfscope}%
\pgfsys@transformshift{2.245979in}{2.650236in}%
\pgfsys@useobject{currentmarker}{}%
\end{pgfscope}%
\begin{pgfscope}%
\pgfsys@transformshift{2.224147in}{2.332998in}%
\pgfsys@useobject{currentmarker}{}%
\end{pgfscope}%
\begin{pgfscope}%
\pgfsys@transformshift{2.206068in}{2.211878in}%
\pgfsys@useobject{currentmarker}{}%
\end{pgfscope}%
\begin{pgfscope}%
\pgfsys@transformshift{2.187757in}{2.803923in}%
\pgfsys@useobject{currentmarker}{}%
\end{pgfscope}%
\begin{pgfscope}%
\pgfsys@transformshift{2.169209in}{2.558096in}%
\pgfsys@useobject{currentmarker}{}%
\end{pgfscope}%
\begin{pgfscope}%
\pgfsys@transformshift{2.145968in}{2.199085in}%
\pgfsys@useobject{currentmarker}{}%
\end{pgfscope}%
\begin{pgfscope}%
\pgfsys@transformshift{2.127420in}{2.064079in}%
\pgfsys@useobject{currentmarker}{}%
\end{pgfscope}%
\begin{pgfscope}%
\pgfsys@transformshift{2.111690in}{2.013084in}%
\pgfsys@useobject{currentmarker}{}%
\end{pgfscope}%
\begin{pgfscope}%
\pgfsys@transformshift{2.090561in}{1.975473in}%
\pgfsys@useobject{currentmarker}{}%
\end{pgfscope}%
\begin{pgfscope}%
\pgfsys@transformshift{2.071544in}{1.972936in}%
\pgfsys@useobject{currentmarker}{}%
\end{pgfscope}%
\begin{pgfscope}%
\pgfsys@transformshift{2.054170in}{2.000058in}%
\pgfsys@useobject{currentmarker}{}%
\end{pgfscope}%
\begin{pgfscope}%
\pgfsys@transformshift{2.035390in}{2.066465in}%
\pgfsys@useobject{currentmarker}{}%
\end{pgfscope}%
\begin{pgfscope}%
\pgfsys@transformshift{2.015199in}{2.213869in}%
\pgfsys@useobject{currentmarker}{}%
\end{pgfscope}%
\begin{pgfscope}%
\pgfsys@transformshift{1.995008in}{2.580289in}%
\pgfsys@useobject{currentmarker}{}%
\end{pgfscope}%
\begin{pgfscope}%
\pgfsys@transformshift{1.974114in}{2.808879in}%
\pgfsys@useobject{currentmarker}{}%
\end{pgfscope}%
\begin{pgfscope}%
\pgfsys@transformshift{1.956506in}{2.698206in}%
\pgfsys@useobject{currentmarker}{}%
\end{pgfscope}%
\begin{pgfscope}%
\pgfsys@transformshift{1.938429in}{2.827487in}%
\pgfsys@useobject{currentmarker}{}%
\end{pgfscope}%
\begin{pgfscope}%
\pgfsys@transformshift{1.916595in}{2.740746in}%
\pgfsys@useobject{currentmarker}{}%
\end{pgfscope}%
\begin{pgfscope}%
\pgfsys@transformshift{1.902038in}{2.505550in}%
\pgfsys@useobject{currentmarker}{}%
\end{pgfscope}%
\begin{pgfscope}%
\pgfsys@transformshift{1.878796in}{2.174185in}%
\pgfsys@useobject{currentmarker}{}%
\end{pgfscope}%
\begin{pgfscope}%
\pgfsys@transformshift{1.861893in}{2.050474in}%
\pgfsys@useobject{currentmarker}{}%
\end{pgfscope}%
\begin{pgfscope}%
\pgfsys@transformshift{1.842876in}{1.991993in}%
\pgfsys@useobject{currentmarker}{}%
\end{pgfscope}%
\begin{pgfscope}%
\pgfsys@transformshift{1.820573in}{1.974081in}%
\pgfsys@useobject{currentmarker}{}%
\end{pgfscope}%
\begin{pgfscope}%
\pgfsys@transformshift{1.802025in}{1.997083in}%
\pgfsys@useobject{currentmarker}{}%
\end{pgfscope}%
\begin{pgfscope}%
\pgfsys@transformshift{1.783948in}{2.057922in}%
\pgfsys@useobject{currentmarker}{}%
\end{pgfscope}%
\begin{pgfscope}%
\pgfsys@transformshift{1.763757in}{2.132815in}%
\pgfsys@useobject{currentmarker}{}%
\end{pgfscope}%
\begin{pgfscope}%
\pgfsys@transformshift{1.745211in}{2.330129in}%
\pgfsys@useobject{currentmarker}{}%
\end{pgfscope}%
\begin{pgfscope}%
\pgfsys@transformshift{1.724315in}{2.528027in}%
\pgfsys@useobject{currentmarker}{}%
\end{pgfscope}%
\begin{pgfscope}%
\pgfsys@transformshift{1.708116in}{2.808290in}%
\pgfsys@useobject{currentmarker}{}%
\end{pgfscope}%
\begin{pgfscope}%
\pgfsys@transformshift{1.688161in}{2.815975in}%
\pgfsys@useobject{currentmarker}{}%
\end{pgfscope}%
\begin{pgfscope}%
\pgfsys@transformshift{1.669144in}{2.629640in}%
\pgfsys@useobject{currentmarker}{}%
\end{pgfscope}%
\begin{pgfscope}%
\pgfsys@transformshift{1.649893in}{2.332614in}%
\pgfsys@useobject{currentmarker}{}%
\end{pgfscope}%
\begin{pgfscope}%
\pgfsys@transformshift{1.628293in}{2.148319in}%
\pgfsys@useobject{currentmarker}{}%
\end{pgfscope}%
\begin{pgfscope}%
\pgfsys@transformshift{1.611625in}{2.050564in}%
\pgfsys@useobject{currentmarker}{}%
\end{pgfscope}%
\begin{pgfscope}%
\pgfsys@transformshift{1.591434in}{1.989280in}%
\pgfsys@useobject{currentmarker}{}%
\end{pgfscope}%
\begin{pgfscope}%
\pgfsys@transformshift{1.574531in}{1.974551in}%
\pgfsys@useobject{currentmarker}{}%
\end{pgfscope}%
\begin{pgfscope}%
\pgfsys@transformshift{1.552932in}{2.007301in}%
\pgfsys@useobject{currentmarker}{}%
\end{pgfscope}%
\begin{pgfscope}%
\pgfsys@transformshift{1.533212in}{2.057072in}%
\pgfsys@useobject{currentmarker}{}%
\end{pgfscope}%
\begin{pgfscope}%
\pgfsys@transformshift{1.515133in}{2.165391in}%
\pgfsys@useobject{currentmarker}{}%
\end{pgfscope}%
\begin{pgfscope}%
\pgfsys@transformshift{1.496821in}{2.335643in}%
\pgfsys@useobject{currentmarker}{}%
\end{pgfscope}%
\begin{pgfscope}%
\pgfsys@transformshift{1.477101in}{2.621564in}%
\pgfsys@useobject{currentmarker}{}%
\end{pgfscope}%
\begin{pgfscope}%
\pgfsys@transformshift{1.457850in}{2.829531in}%
\pgfsys@useobject{currentmarker}{}%
\end{pgfscope}%
\begin{pgfscope}%
\pgfsys@transformshift{1.436485in}{2.852668in}%
\pgfsys@useobject{currentmarker}{}%
\end{pgfscope}%
\begin{pgfscope}%
\pgfsys@transformshift{1.419111in}{2.724949in}%
\pgfsys@useobject{currentmarker}{}%
\end{pgfscope}%
\begin{pgfscope}%
\pgfsys@transformshift{1.398922in}{2.412241in}%
\pgfsys@useobject{currentmarker}{}%
\end{pgfscope}%
\begin{pgfscope}%
\pgfsys@transformshift{1.379905in}{2.210541in}%
\pgfsys@useobject{currentmarker}{}%
\end{pgfscope}%
\begin{pgfscope}%
\pgfsys@transformshift{1.362063in}{2.092880in}%
\pgfsys@useobject{currentmarker}{}%
\end{pgfscope}%
\begin{pgfscope}%
\pgfsys@transformshift{1.341167in}{2.030368in}%
\pgfsys@useobject{currentmarker}{}%
\end{pgfscope}%
\begin{pgfscope}%
\pgfsys@transformshift{1.323090in}{1.998245in}%
\pgfsys@useobject{currentmarker}{}%
\end{pgfscope}%
\begin{pgfscope}%
\pgfsys@transformshift{1.301961in}{1.978355in}%
\pgfsys@useobject{currentmarker}{}%
\end{pgfscope}%
\begin{pgfscope}%
\pgfsys@transformshift{1.285056in}{1.998556in}%
\pgfsys@useobject{currentmarker}{}%
\end{pgfscope}%
\begin{pgfscope}%
\pgfsys@transformshift{1.264162in}{2.072118in}%
\pgfsys@useobject{currentmarker}{}%
\end{pgfscope}%
\begin{pgfscope}%
\pgfsys@transformshift{1.243736in}{2.209495in}%
\pgfsys@useobject{currentmarker}{}%
\end{pgfscope}%
\begin{pgfscope}%
\pgfsys@transformshift{1.225659in}{2.412750in}%
\pgfsys@useobject{currentmarker}{}%
\end{pgfscope}%
\begin{pgfscope}%
\pgfsys@transformshift{1.204999in}{2.644444in}%
\pgfsys@useobject{currentmarker}{}%
\end{pgfscope}%
\begin{pgfscope}%
\pgfsys@transformshift{1.189740in}{2.815133in}%
\pgfsys@useobject{currentmarker}{}%
\end{pgfscope}%
\begin{pgfscope}%
\pgfsys@transformshift{1.167435in}{2.880091in}%
\pgfsys@useobject{currentmarker}{}%
\end{pgfscope}%
\begin{pgfscope}%
\pgfsys@transformshift{1.147715in}{2.816158in}%
\pgfsys@useobject{currentmarker}{}%
\end{pgfscope}%
\begin{pgfscope}%
\pgfsys@transformshift{1.128463in}{2.584927in}%
\pgfsys@useobject{currentmarker}{}%
\end{pgfscope}%
\begin{pgfscope}%
\pgfsys@transformshift{1.108743in}{2.394751in}%
\pgfsys@useobject{currentmarker}{}%
\end{pgfscope}%
\begin{pgfscope}%
\pgfsys@transformshift{1.092544in}{2.202634in}%
\pgfsys@useobject{currentmarker}{}%
\end{pgfscope}%
\begin{pgfscope}%
\pgfsys@transformshift{1.072822in}{2.085712in}%
\pgfsys@useobject{currentmarker}{}%
\end{pgfscope}%
\begin{pgfscope}%
\pgfsys@transformshift{1.052396in}{2.017784in}%
\pgfsys@useobject{currentmarker}{}%
\end{pgfscope}%
\begin{pgfscope}%
\pgfsys@transformshift{1.034554in}{1.984278in}%
\pgfsys@useobject{currentmarker}{}%
\end{pgfscope}%
\begin{pgfscope}%
\pgfsys@transformshift{1.016477in}{2.125982in}%
\pgfsys@useobject{currentmarker}{}%
\end{pgfscope}%
\begin{pgfscope}%
\pgfsys@transformshift{0.993000in}{2.030209in}%
\pgfsys@useobject{currentmarker}{}%
\end{pgfscope}%
\begin{pgfscope}%
\pgfsys@transformshift{0.977506in}{1.992714in}%
\pgfsys@useobject{currentmarker}{}%
\end{pgfscope}%
\begin{pgfscope}%
\pgfsys@transformshift{0.957080in}{1.991600in}%
\pgfsys@useobject{currentmarker}{}%
\end{pgfscope}%
\begin{pgfscope}%
\pgfsys@transformshift{0.936420in}{2.037358in}%
\pgfsys@useobject{currentmarker}{}%
\end{pgfscope}%
\begin{pgfscope}%
\pgfsys@transformshift{0.919750in}{2.113840in}%
\pgfsys@useobject{currentmarker}{}%
\end{pgfscope}%
\begin{pgfscope}%
\pgfsys@transformshift{0.899325in}{2.311126in}%
\pgfsys@useobject{currentmarker}{}%
\end{pgfscope}%
\begin{pgfscope}%
\pgfsys@transformshift{0.880310in}{2.586953in}%
\pgfsys@useobject{currentmarker}{}%
\end{pgfscope}%
\begin{pgfscope}%
\pgfsys@transformshift{0.859650in}{2.829492in}%
\pgfsys@useobject{currentmarker}{}%
\end{pgfscope}%
\begin{pgfscope}%
\pgfsys@transformshift{0.842745in}{2.915959in}%
\pgfsys@useobject{currentmarker}{}%
\end{pgfscope}%
\begin{pgfscope}%
\pgfsys@transformshift{0.824903in}{2.842917in}%
\pgfsys@useobject{currentmarker}{}%
\end{pgfscope}%
\begin{pgfscope}%
\pgfsys@transformshift{0.804243in}{2.644654in}%
\pgfsys@useobject{currentmarker}{}%
\end{pgfscope}%
\begin{pgfscope}%
\pgfsys@transformshift{0.786166in}{2.390820in}%
\pgfsys@useobject{currentmarker}{}%
\end{pgfscope}%
\begin{pgfscope}%
\pgfsys@transformshift{0.764097in}{2.207212in}%
\pgfsys@useobject{currentmarker}{}%
\end{pgfscope}%
\begin{pgfscope}%
\pgfsys@transformshift{0.745784in}{2.098189in}%
\pgfsys@useobject{currentmarker}{}%
\end{pgfscope}%
\begin{pgfscope}%
\pgfsys@transformshift{0.725829in}{2.030897in}%
\pgfsys@useobject{currentmarker}{}%
\end{pgfscope}%
\begin{pgfscope}%
\pgfsys@transformshift{0.707987in}{1.991641in}%
\pgfsys@useobject{currentmarker}{}%
\end{pgfscope}%
\begin{pgfscope}%
\pgfsys@transformshift{0.689908in}{2.002370in}%
\pgfsys@useobject{currentmarker}{}%
\end{pgfscope}%
\begin{pgfscope}%
\pgfsys@transformshift{0.668545in}{2.070721in}%
\pgfsys@useobject{currentmarker}{}%
\end{pgfscope}%
\begin{pgfscope}%
\pgfsys@transformshift{0.650702in}{2.158906in}%
\pgfsys@useobject{currentmarker}{}%
\end{pgfscope}%
\begin{pgfscope}%
\pgfsys@transformshift{0.650936in}{2.153957in}%
\pgfsys@useobject{currentmarker}{}%
\end{pgfscope}%
\begin{pgfscope}%
\pgfsys@transformshift{0.657746in}{2.093417in}%
\pgfsys@useobject{currentmarker}{}%
\end{pgfscope}%
\begin{pgfscope}%
\pgfsys@transformshift{0.673005in}{2.023643in}%
\pgfsys@useobject{currentmarker}{}%
\end{pgfscope}%
\begin{pgfscope}%
\pgfsys@transformshift{0.694134in}{1.993552in}%
\pgfsys@useobject{currentmarker}{}%
\end{pgfscope}%
\begin{pgfscope}%
\pgfsys@transformshift{0.712682in}{2.055336in}%
\pgfsys@useobject{currentmarker}{}%
\end{pgfscope}%
\begin{pgfscope}%
\pgfsys@transformshift{0.733342in}{2.217128in}%
\pgfsys@useobject{currentmarker}{}%
\end{pgfscope}%
\begin{pgfscope}%
\pgfsys@transformshift{0.751184in}{2.508104in}%
\pgfsys@useobject{currentmarker}{}%
\end{pgfscope}%
\begin{pgfscope}%
\pgfsys@transformshift{0.775836in}{2.901714in}%
\pgfsys@useobject{currentmarker}{}%
\end{pgfscope}%
\begin{pgfscope}%
\pgfsys@transformshift{0.790861in}{2.882803in}%
\pgfsys@useobject{currentmarker}{}%
\end{pgfscope}%
\begin{pgfscope}%
\pgfsys@transformshift{0.808469in}{2.600566in}%
\pgfsys@useobject{currentmarker}{}%
\end{pgfscope}%
\begin{pgfscope}%
\pgfsys@transformshift{0.829129in}{2.224177in}%
\pgfsys@useobject{currentmarker}{}%
\end{pgfscope}%
\begin{pgfscope}%
\pgfsys@transformshift{0.846971in}{2.066373in}%
\pgfsys@useobject{currentmarker}{}%
\end{pgfscope}%
\begin{pgfscope}%
\pgfsys@transformshift{0.865048in}{1.993112in}%
\pgfsys@useobject{currentmarker}{}%
\end{pgfscope}%
\begin{pgfscope}%
\pgfsys@transformshift{0.887351in}{2.007369in}%
\pgfsys@useobject{currentmarker}{}%
\end{pgfscope}%
\begin{pgfscope}%
\pgfsys@transformshift{0.906134in}{2.081856in}%
\pgfsys@useobject{currentmarker}{}%
\end{pgfscope}%
\begin{pgfscope}%
\pgfsys@transformshift{0.923038in}{2.254502in}%
\pgfsys@useobject{currentmarker}{}%
\end{pgfscope}%
\begin{pgfscope}%
\pgfsys@transformshift{0.943462in}{2.606176in}%
\pgfsys@useobject{currentmarker}{}%
\end{pgfscope}%
\begin{pgfscope}%
\pgfsys@transformshift{0.965296in}{2.888777in}%
\pgfsys@useobject{currentmarker}{}%
\end{pgfscope}%
\begin{pgfscope}%
\pgfsys@transformshift{0.980558in}{2.814300in}%
\pgfsys@useobject{currentmarker}{}%
\end{pgfscope}%
\begin{pgfscope}%
\pgfsys@transformshift{1.001217in}{2.446103in}%
\pgfsys@useobject{currentmarker}{}%
\end{pgfscope}%
\begin{pgfscope}%
\pgfsys@transformshift{1.020938in}{2.133782in}%
\pgfsys@useobject{currentmarker}{}%
\end{pgfscope}%
\begin{pgfscope}%
\pgfsys@transformshift{1.039251in}{2.024316in}%
\pgfsys@useobject{currentmarker}{}%
\end{pgfscope}%
\begin{pgfscope}%
\pgfsys@transformshift{1.060614in}{1.980084in}%
\pgfsys@useobject{currentmarker}{}%
\end{pgfscope}%
\begin{pgfscope}%
\pgfsys@transformshift{1.080571in}{2.027756in}%
\pgfsys@useobject{currentmarker}{}%
\end{pgfscope}%
\begin{pgfscope}%
\pgfsys@transformshift{1.098882in}{2.135357in}%
\pgfsys@useobject{currentmarker}{}%
\end{pgfscope}%
\begin{pgfscope}%
\pgfsys@transformshift{1.115785in}{2.355724in}%
\pgfsys@useobject{currentmarker}{}%
\end{pgfscope}%
\begin{pgfscope}%
\pgfsys@transformshift{1.136681in}{2.752321in}%
\pgfsys@useobject{currentmarker}{}%
\end{pgfscope}%
\begin{pgfscope}%
\pgfsys@transformshift{1.157107in}{2.864161in}%
\pgfsys@useobject{currentmarker}{}%
\end{pgfscope}%
\begin{pgfscope}%
\pgfsys@transformshift{1.174713in}{2.667529in}%
\pgfsys@useobject{currentmarker}{}%
\end{pgfscope}%
\begin{pgfscope}%
\pgfsys@transformshift{1.194435in}{2.290743in}%
\pgfsys@useobject{currentmarker}{}%
\end{pgfscope}%
\begin{pgfscope}%
\pgfsys@transformshift{1.214155in}{2.094566in}%
\pgfsys@useobject{currentmarker}{}%
\end{pgfscope}%
\begin{pgfscope}%
\pgfsys@transformshift{1.230825in}{2.003114in}%
\pgfsys@useobject{currentmarker}{}%
\end{pgfscope}%
\begin{pgfscope}%
\pgfsys@transformshift{1.251485in}{1.976865in}%
\pgfsys@useobject{currentmarker}{}%
\end{pgfscope}%
\begin{pgfscope}%
\pgfsys@transformshift{1.270031in}{2.020884in}%
\pgfsys@useobject{currentmarker}{}%
\end{pgfscope}%
\begin{pgfscope}%
\pgfsys@transformshift{1.289282in}{2.131024in}%
\pgfsys@useobject{currentmarker}{}%
\end{pgfscope}%
\begin{pgfscope}%
\pgfsys@transformshift{1.310882in}{2.362212in}%
\pgfsys@useobject{currentmarker}{}%
\end{pgfscope}%
\begin{pgfscope}%
\pgfsys@transformshift{1.328724in}{2.732662in}%
\pgfsys@useobject{currentmarker}{}%
\end{pgfscope}%
\begin{pgfscope}%
\pgfsys@transformshift{1.349384in}{2.852925in}%
\pgfsys@useobject{currentmarker}{}%
\end{pgfscope}%
\begin{pgfscope}%
\pgfsys@transformshift{1.366053in}{2.650424in}%
\pgfsys@useobject{currentmarker}{}%
\end{pgfscope}%
\begin{pgfscope}%
\pgfsys@transformshift{1.384366in}{2.359121in}%
\pgfsys@useobject{currentmarker}{}%
\end{pgfscope}%
\begin{pgfscope}%
\pgfsys@transformshift{1.406200in}{2.120989in}%
\pgfsys@useobject{currentmarker}{}%
\end{pgfscope}%
\begin{pgfscope}%
\pgfsys@transformshift{1.425451in}{2.015749in}%
\pgfsys@useobject{currentmarker}{}%
\end{pgfscope}%
\begin{pgfscope}%
\pgfsys@transformshift{1.442354in}{1.976120in}%
\pgfsys@useobject{currentmarker}{}%
\end{pgfscope}%
\begin{pgfscope}%
\pgfsys@transformshift{1.461840in}{1.985264in}%
\pgfsys@useobject{currentmarker}{}%
\end{pgfscope}%
\begin{pgfscope}%
\pgfsys@transformshift{1.483205in}{2.053000in}%
\pgfsys@useobject{currentmarker}{}%
\end{pgfscope}%
\begin{pgfscope}%
\pgfsys@transformshift{1.501751in}{2.088704in}%
\pgfsys@useobject{currentmarker}{}%
\end{pgfscope}%
\begin{pgfscope}%
\pgfsys@transformshift{1.518890in}{2.210046in}%
\pgfsys@useobject{currentmarker}{}%
\end{pgfscope}%
\begin{pgfscope}%
\pgfsys@transformshift{1.541193in}{2.553230in}%
\pgfsys@useobject{currentmarker}{}%
\end{pgfscope}%
\begin{pgfscope}%
\pgfsys@transformshift{1.558332in}{2.818765in}%
\pgfsys@useobject{currentmarker}{}%
\end{pgfscope}%
\begin{pgfscope}%
\pgfsys@transformshift{1.580401in}{2.770476in}%
\pgfsys@useobject{currentmarker}{}%
\end{pgfscope}%
\begin{pgfscope}%
\pgfsys@transformshift{1.597538in}{2.450369in}%
\pgfsys@useobject{currentmarker}{}%
\end{pgfscope}%
\begin{pgfscope}%
\pgfsys@transformshift{1.614912in}{2.180374in}%
\pgfsys@useobject{currentmarker}{}%
\end{pgfscope}%
\begin{pgfscope}%
\pgfsys@transformshift{1.635571in}{2.036344in}%
\pgfsys@useobject{currentmarker}{}%
\end{pgfscope}%
\begin{pgfscope}%
\pgfsys@transformshift{1.653885in}{1.980621in}%
\pgfsys@useobject{currentmarker}{}%
\end{pgfscope}%
\begin{pgfscope}%
\pgfsys@transformshift{1.674545in}{1.981342in}%
\pgfsys@useobject{currentmarker}{}%
\end{pgfscope}%
\begin{pgfscope}%
\pgfsys@transformshift{1.692622in}{2.030974in}%
\pgfsys@useobject{currentmarker}{}%
\end{pgfscope}%
\begin{pgfscope}%
\pgfsys@transformshift{1.713751in}{2.154792in}%
\pgfsys@useobject{currentmarker}{}%
\end{pgfscope}%
\begin{pgfscope}%
\pgfsys@transformshift{1.731593in}{2.355636in}%
\pgfsys@useobject{currentmarker}{}%
\end{pgfscope}%
\begin{pgfscope}%
\pgfsys@transformshift{1.752253in}{2.744163in}%
\pgfsys@useobject{currentmarker}{}%
\end{pgfscope}%
\begin{pgfscope}%
\pgfsys@transformshift{1.772678in}{2.828543in}%
\pgfsys@useobject{currentmarker}{}%
\end{pgfscope}%
\begin{pgfscope}%
\pgfsys@transformshift{1.790286in}{2.651108in}%
\pgfsys@useobject{currentmarker}{}%
\end{pgfscope}%
\begin{pgfscope}%
\pgfsys@transformshift{1.808129in}{2.330514in}%
\pgfsys@useobject{currentmarker}{}%
\end{pgfscope}%
\begin{pgfscope}%
\pgfsys@transformshift{1.827146in}{2.182611in}%
\pgfsys@useobject{currentmarker}{}%
\end{pgfscope}%
\begin{pgfscope}%
\pgfsys@transformshift{1.848980in}{2.036481in}%
\pgfsys@useobject{currentmarker}{}%
\end{pgfscope}%
\begin{pgfscope}%
\pgfsys@transformshift{1.867057in}{1.990211in}%
\pgfsys@useobject{currentmarker}{}%
\end{pgfscope}%
\begin{pgfscope}%
\pgfsys@transformshift{1.885605in}{1.971047in}%
\pgfsys@useobject{currentmarker}{}%
\end{pgfscope}%
\begin{pgfscope}%
\pgfsys@transformshift{1.906265in}{2.012567in}%
\pgfsys@useobject{currentmarker}{}%
\end{pgfscope}%
\begin{pgfscope}%
\pgfsys@transformshift{1.922933in}{2.085632in}%
\pgfsys@useobject{currentmarker}{}%
\end{pgfscope}%
\begin{pgfscope}%
\pgfsys@transformshift{1.944533in}{2.290847in}%
\pgfsys@useobject{currentmarker}{}%
\end{pgfscope}%
\begin{pgfscope}%
\pgfsys@transformshift{1.962141in}{2.470453in}%
\pgfsys@useobject{currentmarker}{}%
\end{pgfscope}%
\begin{pgfscope}%
\pgfsys@transformshift{1.983035in}{2.737527in}%
\pgfsys@useobject{currentmarker}{}%
\end{pgfscope}%
\begin{pgfscope}%
\pgfsys@transformshift{2.001112in}{2.822364in}%
\pgfsys@useobject{currentmarker}{}%
\end{pgfscope}%
\begin{pgfscope}%
\pgfsys@transformshift{2.019660in}{2.738897in}%
\pgfsys@useobject{currentmarker}{}%
\end{pgfscope}%
\begin{pgfscope}%
\pgfsys@transformshift{2.038206in}{2.419257in}%
\pgfsys@useobject{currentmarker}{}%
\end{pgfscope}%
\begin{pgfscope}%
\pgfsys@transformshift{2.059102in}{2.204930in}%
\pgfsys@useobject{currentmarker}{}%
\end{pgfscope}%
\begin{pgfscope}%
\pgfsys@transformshift{2.080465in}{2.055761in}%
\pgfsys@useobject{currentmarker}{}%
\end{pgfscope}%
\begin{pgfscope}%
\pgfsys@transformshift{2.094787in}{1.998166in}%
\pgfsys@useobject{currentmarker}{}%
\end{pgfscope}%
\begin{pgfscope}%
\pgfsys@transformshift{2.117090in}{1.971363in}%
\pgfsys@useobject{currentmarker}{}%
\end{pgfscope}%
\begin{pgfscope}%
\pgfsys@transformshift{2.135638in}{1.995589in}%
\pgfsys@useobject{currentmarker}{}%
\end{pgfscope}%
\begin{pgfscope}%
\pgfsys@transformshift{2.155592in}{2.054447in}%
\pgfsys@useobject{currentmarker}{}%
\end{pgfscope}%
\begin{pgfscope}%
\pgfsys@transformshift{2.173201in}{2.178684in}%
\pgfsys@useobject{currentmarker}{}%
\end{pgfscope}%
\begin{pgfscope}%
\pgfsys@transformshift{2.193157in}{2.387384in}%
\pgfsys@useobject{currentmarker}{}%
\end{pgfscope}%
\begin{pgfscope}%
\pgfsys@transformshift{2.213817in}{2.726068in}%
\pgfsys@useobject{currentmarker}{}%
\end{pgfscope}%
\begin{pgfscope}%
\pgfsys@transformshift{2.232128in}{2.797084in}%
\pgfsys@useobject{currentmarker}{}%
\end{pgfscope}%
\begin{pgfscope}%
\pgfsys@transformshift{2.251614in}{2.785607in}%
\pgfsys@useobject{currentmarker}{}%
\end{pgfscope}%
\begin{pgfscope}%
\pgfsys@transformshift{2.270396in}{2.539541in}%
\pgfsys@useobject{currentmarker}{}%
\end{pgfscope}%
\begin{pgfscope}%
\pgfsys@transformshift{2.290587in}{2.192366in}%
\pgfsys@useobject{currentmarker}{}%
\end{pgfscope}%
\begin{pgfscope}%
\pgfsys@transformshift{2.305612in}{2.067258in}%
\pgfsys@useobject{currentmarker}{}%
\end{pgfscope}%
\begin{pgfscope}%
\pgfsys@transformshift{2.326976in}{2.008024in}%
\pgfsys@useobject{currentmarker}{}%
\end{pgfscope}%
\begin{pgfscope}%
\pgfsys@transformshift{2.347636in}{1.976352in}%
\pgfsys@useobject{currentmarker}{}%
\end{pgfscope}%
\begin{pgfscope}%
\pgfsys@transformshift{2.369941in}{1.977011in}%
\pgfsys@useobject{currentmarker}{}%
\end{pgfscope}%
\begin{pgfscope}%
\pgfsys@transformshift{2.384495in}{2.007446in}%
\pgfsys@useobject{currentmarker}{}%
\end{pgfscope}%
\begin{pgfscope}%
\pgfsys@transformshift{2.405860in}{2.098377in}%
\pgfsys@useobject{currentmarker}{}%
\end{pgfscope}%
\begin{pgfscope}%
\pgfsys@transformshift{2.422999in}{2.262378in}%
\pgfsys@useobject{currentmarker}{}%
\end{pgfscope}%
\begin{pgfscope}%
\pgfsys@transformshift{2.444128in}{2.618031in}%
\pgfsys@useobject{currentmarker}{}%
\end{pgfscope}%
\begin{pgfscope}%
\pgfsys@transformshift{2.461502in}{2.778719in}%
\pgfsys@useobject{currentmarker}{}%
\end{pgfscope}%
\begin{pgfscope}%
\pgfsys@transformshift{2.480282in}{2.806610in}%
\pgfsys@useobject{currentmarker}{}%
\end{pgfscope}%
\begin{pgfscope}%
\pgfsys@transformshift{2.499770in}{2.661950in}%
\pgfsys@useobject{currentmarker}{}%
\end{pgfscope}%
\begin{pgfscope}%
\pgfsys@transformshift{2.520195in}{2.329829in}%
\pgfsys@useobject{currentmarker}{}%
\end{pgfscope}%
\begin{pgfscope}%
\pgfsys@transformshift{2.540384in}{2.099943in}%
\pgfsys@useobject{currentmarker}{}%
\end{pgfscope}%
\begin{pgfscope}%
\pgfsys@transformshift{2.558227in}{2.021328in}%
\pgfsys@useobject{currentmarker}{}%
\end{pgfscope}%
\begin{pgfscope}%
\pgfsys@transformshift{2.577478in}{1.997765in}%
\pgfsys@useobject{currentmarker}{}%
\end{pgfscope}%
\begin{pgfscope}%
\pgfsys@transformshift{2.597903in}{1.972996in}%
\pgfsys@useobject{currentmarker}{}%
\end{pgfscope}%
\begin{pgfscope}%
\pgfsys@transformshift{2.615511in}{1.983580in}%
\pgfsys@useobject{currentmarker}{}%
\end{pgfscope}%
\begin{pgfscope}%
\pgfsys@transformshift{2.634528in}{2.023866in}%
\pgfsys@useobject{currentmarker}{}%
\end{pgfscope}%
\begin{pgfscope}%
\pgfsys@transformshift{2.654954in}{2.138878in}%
\pgfsys@useobject{currentmarker}{}%
\end{pgfscope}%
\begin{pgfscope}%
\pgfsys@transformshift{2.672796in}{2.338469in}%
\pgfsys@useobject{currentmarker}{}%
\end{pgfscope}%
\begin{pgfscope}%
\pgfsys@transformshift{2.691109in}{2.118718in}%
\pgfsys@useobject{currentmarker}{}%
\end{pgfscope}%
\begin{pgfscope}%
\pgfsys@transformshift{2.711064in}{2.341529in}%
\pgfsys@useobject{currentmarker}{}%
\end{pgfscope}%
\begin{pgfscope}%
\pgfsys@transformshift{2.730550in}{2.666844in}%
\pgfsys@useobject{currentmarker}{}%
\end{pgfscope}%
\begin{pgfscope}%
\pgfsys@transformshift{2.750975in}{2.817709in}%
\pgfsys@useobject{currentmarker}{}%
\end{pgfscope}%
\begin{pgfscope}%
\pgfsys@transformshift{2.767645in}{2.750590in}%
\pgfsys@useobject{currentmarker}{}%
\end{pgfscope}%
\begin{pgfscope}%
\pgfsys@transformshift{2.787600in}{2.559628in}%
\pgfsys@useobject{currentmarker}{}%
\end{pgfscope}%
\begin{pgfscope}%
\pgfsys@transformshift{2.808494in}{2.228444in}%
\pgfsys@useobject{currentmarker}{}%
\end{pgfscope}%
\begin{pgfscope}%
\pgfsys@transformshift{2.830094in}{2.092004in}%
\pgfsys@useobject{currentmarker}{}%
\end{pgfscope}%
\begin{pgfscope}%
\pgfsys@transformshift{2.848171in}{2.011664in}%
\pgfsys@useobject{currentmarker}{}%
\end{pgfscope}%
\begin{pgfscope}%
\pgfsys@transformshift{2.865545in}{1.976141in}%
\pgfsys@useobject{currentmarker}{}%
\end{pgfscope}%
\begin{pgfscope}%
\pgfsys@transformshift{2.884092in}{1.979423in}%
\pgfsys@useobject{currentmarker}{}%
\end{pgfscope}%
\begin{pgfscope}%
\pgfsys@transformshift{2.904987in}{2.033316in}%
\pgfsys@useobject{currentmarker}{}%
\end{pgfscope}%
\begin{pgfscope}%
\pgfsys@transformshift{2.923533in}{2.128573in}%
\pgfsys@useobject{currentmarker}{}%
\end{pgfscope}%
\begin{pgfscope}%
\pgfsys@transformshift{2.943958in}{2.344832in}%
\pgfsys@useobject{currentmarker}{}%
\end{pgfscope}%
\begin{pgfscope}%
\pgfsys@transformshift{2.962975in}{2.623512in}%
\pgfsys@useobject{currentmarker}{}%
\end{pgfscope}%
\begin{pgfscope}%
\pgfsys@transformshift{2.980348in}{2.807723in}%
\pgfsys@useobject{currentmarker}{}%
\end{pgfscope}%
\begin{pgfscope}%
\pgfsys@transformshift{3.001008in}{2.808170in}%
\pgfsys@useobject{currentmarker}{}%
\end{pgfscope}%
\begin{pgfscope}%
\pgfsys@transformshift{3.017911in}{2.706511in}%
\pgfsys@useobject{currentmarker}{}%
\end{pgfscope}%
\begin{pgfscope}%
\pgfsys@transformshift{3.039511in}{2.393691in}%
\pgfsys@useobject{currentmarker}{}%
\end{pgfscope}%
\begin{pgfscope}%
\pgfsys@transformshift{3.058059in}{2.149613in}%
\pgfsys@useobject{currentmarker}{}%
\end{pgfscope}%
\begin{pgfscope}%
\pgfsys@transformshift{3.076839in}{2.050640in}%
\pgfsys@useobject{currentmarker}{}%
\end{pgfscope}%
\begin{pgfscope}%
\pgfsys@transformshift{3.094918in}{1.994923in}%
\pgfsys@useobject{currentmarker}{}%
\end{pgfscope}%
\begin{pgfscope}%
\pgfsys@transformshift{3.116047in}{1.972662in}%
\pgfsys@useobject{currentmarker}{}%
\end{pgfscope}%
\begin{pgfscope}%
\pgfsys@transformshift{3.136472in}{1.996998in}%
\pgfsys@useobject{currentmarker}{}%
\end{pgfscope}%
\begin{pgfscope}%
\pgfsys@transformshift{3.155489in}{2.060962in}%
\pgfsys@useobject{currentmarker}{}%
\end{pgfscope}%
\begin{pgfscope}%
\pgfsys@transformshift{3.174506in}{2.171431in}%
\pgfsys@useobject{currentmarker}{}%
\end{pgfscope}%
\begin{pgfscope}%
\pgfsys@transformshift{3.194460in}{2.437592in}%
\pgfsys@useobject{currentmarker}{}%
\end{pgfscope}%
\begin{pgfscope}%
\pgfsys@transformshift{3.212537in}{2.609018in}%
\pgfsys@useobject{currentmarker}{}%
\end{pgfscope}%
\begin{pgfscope}%
\pgfsys@transformshift{3.230382in}{2.820114in}%
\pgfsys@useobject{currentmarker}{}%
\end{pgfscope}%
\begin{pgfscope}%
\pgfsys@transformshift{3.251745in}{2.807908in}%
\pgfsys@useobject{currentmarker}{}%
\end{pgfscope}%
\begin{pgfscope}%
\pgfsys@transformshift{3.269587in}{2.625862in}%
\pgfsys@useobject{currentmarker}{}%
\end{pgfscope}%
\begin{pgfscope}%
\pgfsys@transformshift{3.288135in}{2.323425in}%
\pgfsys@useobject{currentmarker}{}%
\end{pgfscope}%
\begin{pgfscope}%
\pgfsys@transformshift{3.309733in}{2.135792in}%
\pgfsys@useobject{currentmarker}{}%
\end{pgfscope}%
\begin{pgfscope}%
\pgfsys@transformshift{3.326638in}{2.035855in}%
\pgfsys@useobject{currentmarker}{}%
\end{pgfscope}%
\begin{pgfscope}%
\pgfsys@transformshift{3.345889in}{1.986924in}%
\pgfsys@useobject{currentmarker}{}%
\end{pgfscope}%
\begin{pgfscope}%
\pgfsys@transformshift{3.367723in}{1.977157in}%
\pgfsys@useobject{currentmarker}{}%
\end{pgfscope}%
\begin{pgfscope}%
\pgfsys@transformshift{3.385097in}{2.010131in}%
\pgfsys@useobject{currentmarker}{}%
\end{pgfscope}%
\begin{pgfscope}%
\pgfsys@transformshift{3.404348in}{2.049968in}%
\pgfsys@useobject{currentmarker}{}%
\end{pgfscope}%
\begin{pgfscope}%
\pgfsys@transformshift{3.422190in}{2.129535in}%
\pgfsys@useobject{currentmarker}{}%
\end{pgfscope}%
\begin{pgfscope}%
\pgfsys@transformshift{3.441910in}{2.301251in}%
\pgfsys@useobject{currentmarker}{}%
\end{pgfscope}%
\begin{pgfscope}%
\pgfsys@transformshift{3.463510in}{2.508730in}%
\pgfsys@useobject{currentmarker}{}%
\end{pgfscope}%
\begin{pgfscope}%
\pgfsys@transformshift{3.479004in}{2.741866in}%
\pgfsys@useobject{currentmarker}{}%
\end{pgfscope}%
\begin{pgfscope}%
\pgfsys@transformshift{3.500838in}{2.859633in}%
\pgfsys@useobject{currentmarker}{}%
\end{pgfscope}%
\begin{pgfscope}%
\pgfsys@transformshift{3.522204in}{2.795295in}%
\pgfsys@useobject{currentmarker}{}%
\end{pgfscope}%
\begin{pgfscope}%
\pgfsys@transformshift{3.521733in}{2.667290in}%
\pgfsys@useobject{currentmarker}{}%
\end{pgfscope}%
\begin{pgfscope}%
\pgfsys@transformshift{3.537463in}{2.593522in}%
\pgfsys@useobject{currentmarker}{}%
\end{pgfscope}%
\begin{pgfscope}%
\pgfsys@transformshift{3.558592in}{2.284652in}%
\pgfsys@useobject{currentmarker}{}%
\end{pgfscope}%
\begin{pgfscope}%
\pgfsys@transformshift{3.578783in}{2.150749in}%
\pgfsys@useobject{currentmarker}{}%
\end{pgfscope}%
\begin{pgfscope}%
\pgfsys@transformshift{3.597565in}{2.046843in}%
\pgfsys@useobject{currentmarker}{}%
\end{pgfscope}%
\begin{pgfscope}%
\pgfsys@transformshift{3.612825in}{2.002108in}%
\pgfsys@useobject{currentmarker}{}%
\end{pgfscope}%
\begin{pgfscope}%
\pgfsys@transformshift{3.635362in}{1.976665in}%
\pgfsys@useobject{currentmarker}{}%
\end{pgfscope}%
\begin{pgfscope}%
\pgfsys@transformshift{3.651327in}{1.993608in}%
\pgfsys@useobject{currentmarker}{}%
\end{pgfscope}%
\begin{pgfscope}%
\pgfsys@transformshift{3.673867in}{2.029111in}%
\pgfsys@useobject{currentmarker}{}%
\end{pgfscope}%
\begin{pgfscope}%
\pgfsys@transformshift{3.693118in}{2.123748in}%
\pgfsys@useobject{currentmarker}{}%
\end{pgfscope}%
\begin{pgfscope}%
\pgfsys@transformshift{3.712369in}{2.225703in}%
\pgfsys@useobject{currentmarker}{}%
\end{pgfscope}%
\begin{pgfscope}%
\pgfsys@transformshift{3.732795in}{2.452022in}%
\pgfsys@useobject{currentmarker}{}%
\end{pgfscope}%
\begin{pgfscope}%
\pgfsys@transformshift{3.751341in}{2.624543in}%
\pgfsys@useobject{currentmarker}{}%
\end{pgfscope}%
\begin{pgfscope}%
\pgfsys@transformshift{3.769654in}{2.842908in}%
\pgfsys@useobject{currentmarker}{}%
\end{pgfscope}%
\begin{pgfscope}%
\pgfsys@transformshift{3.786557in}{2.877873in}%
\pgfsys@useobject{currentmarker}{}%
\end{pgfscope}%
\begin{pgfscope}%
\pgfsys@transformshift{3.809094in}{2.821817in}%
\pgfsys@useobject{currentmarker}{}%
\end{pgfscope}%
\begin{pgfscope}%
\pgfsys@transformshift{3.826937in}{2.557197in}%
\pgfsys@useobject{currentmarker}{}%
\end{pgfscope}%
\begin{pgfscope}%
\pgfsys@transformshift{3.847362in}{2.281061in}%
\pgfsys@useobject{currentmarker}{}%
\end{pgfscope}%
\begin{pgfscope}%
\pgfsys@transformshift{3.866144in}{2.148402in}%
\pgfsys@useobject{currentmarker}{}%
\end{pgfscope}%
\begin{pgfscope}%
\pgfsys@transformshift{3.883752in}{2.117483in}%
\pgfsys@useobject{currentmarker}{}%
\end{pgfscope}%
\begin{pgfscope}%
\pgfsys@transformshift{3.902535in}{2.171612in}%
\pgfsys@useobject{currentmarker}{}%
\end{pgfscope}%
\begin{pgfscope}%
\pgfsys@transformshift{3.922020in}{2.049015in}%
\pgfsys@useobject{currentmarker}{}%
\end{pgfscope}%
\begin{pgfscope}%
\pgfsys@transformshift{3.940803in}{1.998267in}%
\pgfsys@useobject{currentmarker}{}%
\end{pgfscope}%
\begin{pgfscope}%
\pgfsys@transformshift{3.962871in}{1.982719in}%
\pgfsys@useobject{currentmarker}{}%
\end{pgfscope}%
\begin{pgfscope}%
\pgfsys@transformshift{3.986349in}{2.035046in}%
\pgfsys@useobject{currentmarker}{}%
\end{pgfscope}%
\begin{pgfscope}%
\pgfsys@transformshift{3.997616in}{2.085757in}%
\pgfsys@useobject{currentmarker}{}%
\end{pgfscope}%
\begin{pgfscope}%
\pgfsys@transformshift{4.019451in}{2.205734in}%
\pgfsys@useobject{currentmarker}{}%
\end{pgfscope}%
\begin{pgfscope}%
\pgfsys@transformshift{4.039407in}{2.397949in}%
\pgfsys@useobject{currentmarker}{}%
\end{pgfscope}%
\begin{pgfscope}%
\pgfsys@transformshift{4.057484in}{2.696362in}%
\pgfsys@useobject{currentmarker}{}%
\end{pgfscope}%
\begin{pgfscope}%
\pgfsys@transformshift{4.076266in}{2.894121in}%
\pgfsys@useobject{currentmarker}{}%
\end{pgfscope}%
\begin{pgfscope}%
\pgfsys@transformshift{4.095518in}{2.903974in}%
\pgfsys@useobject{currentmarker}{}%
\end{pgfscope}%
\begin{pgfscope}%
\pgfsys@transformshift{4.117586in}{2.750698in}%
\pgfsys@useobject{currentmarker}{}%
\end{pgfscope}%
\begin{pgfscope}%
\pgfsys@transformshift{4.134723in}{2.462908in}%
\pgfsys@useobject{currentmarker}{}%
\end{pgfscope}%
\begin{pgfscope}%
\pgfsys@transformshift{4.154914in}{2.264942in}%
\pgfsys@useobject{currentmarker}{}%
\end{pgfscope}%
\begin{pgfscope}%
\pgfsys@transformshift{4.172991in}{2.171002in}%
\pgfsys@useobject{currentmarker}{}%
\end{pgfscope}%
\begin{pgfscope}%
\pgfsys@transformshift{4.192008in}{2.064345in}%
\pgfsys@useobject{currentmarker}{}%
\end{pgfscope}%
\begin{pgfscope}%
\pgfsys@transformshift{4.210790in}{2.005381in}%
\pgfsys@useobject{currentmarker}{}%
\end{pgfscope}%
\begin{pgfscope}%
\pgfsys@transformshift{4.234033in}{1.985930in}%
\pgfsys@useobject{currentmarker}{}%
\end{pgfscope}%
\begin{pgfscope}%
\pgfsys@transformshift{4.249293in}{2.006151in}%
\pgfsys@useobject{currentmarker}{}%
\end{pgfscope}%
\begin{pgfscope}%
\pgfsys@transformshift{4.267604in}{2.062449in}%
\pgfsys@useobject{currentmarker}{}%
\end{pgfscope}%
\begin{pgfscope}%
\pgfsys@transformshift{4.288501in}{2.150862in}%
\pgfsys@useobject{currentmarker}{}%
\end{pgfscope}%
\begin{pgfscope}%
\pgfsys@transformshift{4.308690in}{2.293481in}%
\pgfsys@useobject{currentmarker}{}%
\end{pgfscope}%
\begin{pgfscope}%
\pgfsys@transformshift{4.327472in}{2.545053in}%
\pgfsys@useobject{currentmarker}{}%
\end{pgfscope}%
\begin{pgfscope}%
\pgfsys@transformshift{4.345549in}{2.646434in}%
\pgfsys@useobject{currentmarker}{}%
\end{pgfscope}%
\begin{pgfscope}%
\pgfsys@transformshift{4.364800in}{2.888784in}%
\pgfsys@useobject{currentmarker}{}%
\end{pgfscope}%
\begin{pgfscope}%
\pgfsys@transformshift{4.383113in}{2.939481in}%
\pgfsys@useobject{currentmarker}{}%
\end{pgfscope}%
\begin{pgfscope}%
\pgfsys@transformshift{4.402599in}{2.852098in}%
\pgfsys@useobject{currentmarker}{}%
\end{pgfscope}%
\begin{pgfscope}%
\pgfsys@transformshift{4.420442in}{2.625128in}%
\pgfsys@useobject{currentmarker}{}%
\end{pgfscope}%
\begin{pgfscope}%
\pgfsys@transformshift{4.439458in}{2.341157in}%
\pgfsys@useobject{currentmarker}{}%
\end{pgfscope}%
\begin{pgfscope}%
\pgfsys@transformshift{4.463170in}{2.130293in}%
\pgfsys@useobject{currentmarker}{}%
\end{pgfscope}%
\begin{pgfscope}%
\pgfsys@transformshift{4.480778in}{2.043973in}%
\pgfsys@useobject{currentmarker}{}%
\end{pgfscope}%
\begin{pgfscope}%
\pgfsys@transformshift{4.479840in}{2.045948in}%
\pgfsys@useobject{currentmarker}{}%
\end{pgfscope}%
\begin{pgfscope}%
\pgfsys@transformshift{4.474206in}{2.078242in}%
\pgfsys@useobject{currentmarker}{}%
\end{pgfscope}%
\begin{pgfscope}%
\pgfsys@transformshift{4.453546in}{2.288878in}%
\pgfsys@useobject{currentmarker}{}%
\end{pgfscope}%
\begin{pgfscope}%
\pgfsys@transformshift{4.434998in}{2.652037in}%
\pgfsys@useobject{currentmarker}{}%
\end{pgfscope}%
\begin{pgfscope}%
\pgfsys@transformshift{4.417155in}{2.915939in}%
\pgfsys@useobject{currentmarker}{}%
\end{pgfscope}%
\begin{pgfscope}%
\pgfsys@transformshift{4.398607in}{2.888876in}%
\pgfsys@useobject{currentmarker}{}%
\end{pgfscope}%
\begin{pgfscope}%
\pgfsys@transformshift{4.377479in}{2.474717in}%
\pgfsys@useobject{currentmarker}{}%
\end{pgfscope}%
\begin{pgfscope}%
\pgfsys@transformshift{4.357993in}{2.191736in}%
\pgfsys@useobject{currentmarker}{}%
\end{pgfscope}%
\begin{pgfscope}%
\pgfsys@transformshift{4.336393in}{2.037991in}%
\pgfsys@useobject{currentmarker}{}%
\end{pgfscope}%
\begin{pgfscope}%
\pgfsys@transformshift{4.336393in}{2.001421in}%
\pgfsys@useobject{currentmarker}{}%
\end{pgfscope}%
\begin{pgfscope}%
\pgfsys@transformshift{4.322542in}{1.988922in}%
\pgfsys@useobject{currentmarker}{}%
\end{pgfscope}%
\begin{pgfscope}%
\pgfsys@transformshift{4.300708in}{2.025211in}%
\pgfsys@useobject{currentmarker}{}%
\end{pgfscope}%
\begin{pgfscope}%
\pgfsys@transformshift{4.280752in}{2.135137in}%
\pgfsys@useobject{currentmarker}{}%
\end{pgfscope}%
\begin{pgfscope}%
\pgfsys@transformshift{4.261971in}{2.446948in}%
\pgfsys@useobject{currentmarker}{}%
\end{pgfscope}%
\begin{pgfscope}%
\pgfsys@transformshift{4.242249in}{2.807701in}%
\pgfsys@useobject{currentmarker}{}%
\end{pgfscope}%
\begin{pgfscope}%
\pgfsys@transformshift{4.223938in}{2.912298in}%
\pgfsys@useobject{currentmarker}{}%
\end{pgfscope}%
\begin{pgfscope}%
\pgfsys@transformshift{4.204687in}{2.688867in}%
\pgfsys@useobject{currentmarker}{}%
\end{pgfscope}%
\begin{pgfscope}%
\pgfsys@transformshift{4.186139in}{2.315358in}%
\pgfsys@useobject{currentmarker}{}%
\end{pgfscope}%
\begin{pgfscope}%
\pgfsys@transformshift{4.167357in}{2.103056in}%
\pgfsys@useobject{currentmarker}{}%
\end{pgfscope}%
\begin{pgfscope}%
\pgfsys@transformshift{4.148576in}{2.009317in}%
\pgfsys@useobject{currentmarker}{}%
\end{pgfscope}%
\begin{pgfscope}%
\pgfsys@transformshift{4.130028in}{1.984564in}%
\pgfsys@useobject{currentmarker}{}%
\end{pgfscope}%
\begin{pgfscope}%
\pgfsys@transformshift{4.106551in}{2.065305in}%
\pgfsys@useobject{currentmarker}{}%
\end{pgfscope}%
\begin{pgfscope}%
\pgfsys@transformshift{4.091526in}{2.204222in}%
\pgfsys@useobject{currentmarker}{}%
\end{pgfscope}%
\begin{pgfscope}%
\pgfsys@transformshift{4.070397in}{2.555961in}%
\pgfsys@useobject{currentmarker}{}%
\end{pgfscope}%
\begin{pgfscope}%
\pgfsys@transformshift{4.051615in}{2.826169in}%
\pgfsys@useobject{currentmarker}{}%
\end{pgfscope}%
\begin{pgfscope}%
\pgfsys@transformshift{4.034241in}{2.437535in}%
\pgfsys@useobject{currentmarker}{}%
\end{pgfscope}%
\begin{pgfscope}%
\pgfsys@transformshift{4.011938in}{2.129353in}%
\pgfsys@useobject{currentmarker}{}%
\end{pgfscope}%
\begin{pgfscope}%
\pgfsys@transformshift{3.994096in}{2.018736in}%
\pgfsys@useobject{currentmarker}{}%
\end{pgfscope}%
\begin{pgfscope}%
\pgfsys@transformshift{3.975782in}{1.978468in}%
\pgfsys@useobject{currentmarker}{}%
\end{pgfscope}%
\begin{pgfscope}%
\pgfsys@transformshift{3.955122in}{2.012610in}%
\pgfsys@useobject{currentmarker}{}%
\end{pgfscope}%
\begin{pgfscope}%
\pgfsys@transformshift{3.936811in}{2.119529in}%
\pgfsys@useobject{currentmarker}{}%
\end{pgfscope}%
\begin{pgfscope}%
\pgfsys@transformshift{3.917794in}{2.358138in}%
\pgfsys@useobject{currentmarker}{}%
\end{pgfscope}%
\begin{pgfscope}%
\pgfsys@transformshift{3.895726in}{2.781377in}%
\pgfsys@useobject{currentmarker}{}%
\end{pgfscope}%
\begin{pgfscope}%
\pgfsys@transformshift{3.877649in}{2.711712in}%
\pgfsys@useobject{currentmarker}{}%
\end{pgfscope}%
\begin{pgfscope}%
\pgfsys@transformshift{3.859101in}{2.869503in}%
\pgfsys@useobject{currentmarker}{}%
\end{pgfscope}%
\begin{pgfscope}%
\pgfsys@transformshift{3.839615in}{2.686681in}%
\pgfsys@useobject{currentmarker}{}%
\end{pgfscope}%
\begin{pgfscope}%
\pgfsys@transformshift{3.822476in}{2.326055in}%
\pgfsys@useobject{currentmarker}{}%
\end{pgfscope}%
\begin{pgfscope}%
\pgfsys@transformshift{3.799235in}{2.095291in}%
\pgfsys@useobject{currentmarker}{}%
\end{pgfscope}%
\begin{pgfscope}%
\pgfsys@transformshift{3.783505in}{2.008796in}%
\pgfsys@useobject{currentmarker}{}%
\end{pgfscope}%
\begin{pgfscope}%
\pgfsys@transformshift{3.762610in}{1.975606in}%
\pgfsys@useobject{currentmarker}{}%
\end{pgfscope}%
\begin{pgfscope}%
\pgfsys@transformshift{3.742888in}{2.016830in}%
\pgfsys@useobject{currentmarker}{}%
\end{pgfscope}%
\begin{pgfscope}%
\pgfsys@transformshift{3.724577in}{2.134454in}%
\pgfsys@useobject{currentmarker}{}%
\end{pgfscope}%
\begin{pgfscope}%
\pgfsys@transformshift{3.705794in}{2.369776in}%
\pgfsys@useobject{currentmarker}{}%
\end{pgfscope}%
\begin{pgfscope}%
\pgfsys@transformshift{3.685603in}{2.710779in}%
\pgfsys@useobject{currentmarker}{}%
\end{pgfscope}%
\begin{pgfscope}%
\pgfsys@transformshift{3.667292in}{2.854579in}%
\pgfsys@useobject{currentmarker}{}%
\end{pgfscope}%
\begin{pgfscope}%
\pgfsys@transformshift{3.648510in}{2.673050in}%
\pgfsys@useobject{currentmarker}{}%
\end{pgfscope}%
\begin{pgfscope}%
\pgfsys@transformshift{3.630433in}{2.348202in}%
\pgfsys@useobject{currentmarker}{}%
\end{pgfscope}%
\begin{pgfscope}%
\pgfsys@transformshift{3.606721in}{2.085464in}%
\pgfsys@useobject{currentmarker}{}%
\end{pgfscope}%
\begin{pgfscope}%
\pgfsys@transformshift{3.590756in}{2.010898in}%
\pgfsys@useobject{currentmarker}{}%
\end{pgfscope}%
\begin{pgfscope}%
\pgfsys@transformshift{3.571739in}{1.976557in}%
\pgfsys@useobject{currentmarker}{}%
\end{pgfscope}%
\begin{pgfscope}%
\pgfsys@transformshift{3.549671in}{1.990504in}%
\pgfsys@useobject{currentmarker}{}%
\end{pgfscope}%
\begin{pgfscope}%
\pgfsys@transformshift{3.527837in}{2.075199in}%
\pgfsys@useobject{currentmarker}{}%
\end{pgfscope}%
\begin{pgfscope}%
\pgfsys@transformshift{3.512812in}{2.200011in}%
\pgfsys@useobject{currentmarker}{}%
\end{pgfscope}%
\begin{pgfscope}%
\pgfsys@transformshift{3.496612in}{2.394343in}%
\pgfsys@useobject{currentmarker}{}%
\end{pgfscope}%
\begin{pgfscope}%
\pgfsys@transformshift{3.475249in}{2.731218in}%
\pgfsys@useobject{currentmarker}{}%
\end{pgfscope}%
\begin{pgfscope}%
\pgfsys@transformshift{3.456467in}{2.836315in}%
\pgfsys@useobject{currentmarker}{}%
\end{pgfscope}%
\begin{pgfscope}%
\pgfsys@transformshift{3.438624in}{2.722082in}%
\pgfsys@useobject{currentmarker}{}%
\end{pgfscope}%
\begin{pgfscope}%
\pgfsys@transformshift{3.415616in}{2.318759in}%
\pgfsys@useobject{currentmarker}{}%
\end{pgfscope}%
\begin{pgfscope}%
\pgfsys@transformshift{3.397304in}{2.134814in}%
\pgfsys@useobject{currentmarker}{}%
\end{pgfscope}%
\begin{pgfscope}%
\pgfsys@transformshift{3.376879in}{2.016321in}%
\pgfsys@useobject{currentmarker}{}%
\end{pgfscope}%
\begin{pgfscope}%
\pgfsys@transformshift{3.358802in}{1.975774in}%
\pgfsys@useobject{currentmarker}{}%
\end{pgfscope}%
\begin{pgfscope}%
\pgfsys@transformshift{3.341428in}{1.980924in}%
\pgfsys@useobject{currentmarker}{}%
\end{pgfscope}%
\begin{pgfscope}%
\pgfsys@transformshift{3.322177in}{2.032262in}%
\pgfsys@useobject{currentmarker}{}%
\end{pgfscope}%
\begin{pgfscope}%
\pgfsys@transformshift{3.301517in}{2.161254in}%
\pgfsys@useobject{currentmarker}{}%
\end{pgfscope}%
\begin{pgfscope}%
\pgfsys@transformshift{3.281795in}{2.297781in}%
\pgfsys@useobject{currentmarker}{}%
\end{pgfscope}%
\begin{pgfscope}%
\pgfsys@transformshift{3.263015in}{2.637543in}%
\pgfsys@useobject{currentmarker}{}%
\end{pgfscope}%
\begin{pgfscope}%
\pgfsys@transformshift{3.240946in}{2.814271in}%
\pgfsys@useobject{currentmarker}{}%
\end{pgfscope}%
\begin{pgfscope}%
\pgfsys@transformshift{3.226155in}{2.791694in}%
\pgfsys@useobject{currentmarker}{}%
\end{pgfscope}%
\begin{pgfscope}%
\pgfsys@transformshift{3.204556in}{2.479139in}%
\pgfsys@useobject{currentmarker}{}%
\end{pgfscope}%
\begin{pgfscope}%
\pgfsys@transformshift{3.186008in}{2.199440in}%
\pgfsys@useobject{currentmarker}{}%
\end{pgfscope}%
\begin{pgfscope}%
\pgfsys@transformshift{3.170279in}{2.066658in}%
\pgfsys@useobject{currentmarker}{}%
\end{pgfscope}%
\begin{pgfscope}%
\pgfsys@transformshift{3.146097in}{1.993501in}%
\pgfsys@useobject{currentmarker}{}%
\end{pgfscope}%
\begin{pgfscope}%
\pgfsys@transformshift{3.128254in}{1.972011in}%
\pgfsys@useobject{currentmarker}{}%
\end{pgfscope}%
\begin{pgfscope}%
\pgfsys@transformshift{3.112055in}{1.982653in}%
\pgfsys@useobject{currentmarker}{}%
\end{pgfscope}%
\begin{pgfscope}%
\pgfsys@transformshift{3.090457in}{2.042315in}%
\pgfsys@useobject{currentmarker}{}%
\end{pgfscope}%
\begin{pgfscope}%
\pgfsys@transformshift{3.068623in}{2.194185in}%
\pgfsys@useobject{currentmarker}{}%
\end{pgfscope}%
\begin{pgfscope}%
\pgfsys@transformshift{3.053362in}{2.368877in}%
\pgfsys@useobject{currentmarker}{}%
\end{pgfscope}%
\begin{pgfscope}%
\pgfsys@transformshift{3.031998in}{2.739816in}%
\pgfsys@useobject{currentmarker}{}%
\end{pgfscope}%
\begin{pgfscope}%
\pgfsys@transformshift{3.030824in}{2.779694in}%
\pgfsys@useobject{currentmarker}{}%
\end{pgfscope}%
\begin{pgfscope}%
\pgfsys@transformshift{3.013450in}{2.816516in}%
\pgfsys@useobject{currentmarker}{}%
\end{pgfscope}%
\begin{pgfscope}%
\pgfsys@transformshift{2.993730in}{2.739000in}%
\pgfsys@useobject{currentmarker}{}%
\end{pgfscope}%
\begin{pgfscope}%
\pgfsys@transformshift{2.975653in}{2.426513in}%
\pgfsys@useobject{currentmarker}{}%
\end{pgfscope}%
\begin{pgfscope}%
\pgfsys@transformshift{2.956166in}{2.230640in}%
\pgfsys@useobject{currentmarker}{}%
\end{pgfscope}%
\begin{pgfscope}%
\pgfsys@transformshift{2.938089in}{2.077528in}%
\pgfsys@useobject{currentmarker}{}%
\end{pgfscope}%
\begin{pgfscope}%
\pgfsys@transformshift{2.918603in}{2.001482in}%
\pgfsys@useobject{currentmarker}{}%
\end{pgfscope}%
\begin{pgfscope}%
\pgfsys@transformshift{2.892074in}{1.970970in}%
\pgfsys@useobject{currentmarker}{}%
\end{pgfscope}%
\begin{pgfscope}%
\pgfsys@transformshift{2.880335in}{1.972878in}%
\pgfsys@useobject{currentmarker}{}%
\end{pgfscope}%
\begin{pgfscope}%
\pgfsys@transformshift{2.859441in}{2.014746in}%
\pgfsys@useobject{currentmarker}{}%
\end{pgfscope}%
\begin{pgfscope}%
\pgfsys@transformshift{2.841364in}{2.104246in}%
\pgfsys@useobject{currentmarker}{}%
\end{pgfscope}%
\begin{pgfscope}%
\pgfsys@transformshift{2.822347in}{2.178894in}%
\pgfsys@useobject{currentmarker}{}%
\end{pgfscope}%
\begin{pgfscope}%
\pgfsys@transformshift{2.801216in}{2.500061in}%
\pgfsys@useobject{currentmarker}{}%
\end{pgfscope}%
\begin{pgfscope}%
\pgfsys@transformshift{2.781262in}{2.779409in}%
\pgfsys@useobject{currentmarker}{}%
\end{pgfscope}%
\begin{pgfscope}%
\pgfsys@transformshift{2.764357in}{2.796295in}%
\pgfsys@useobject{currentmarker}{}%
\end{pgfscope}%
\begin{pgfscope}%
\pgfsys@transformshift{2.762479in}{2.712411in}%
\pgfsys@useobject{currentmarker}{}%
\end{pgfscope}%
\begin{pgfscope}%
\pgfsys@transformshift{2.745106in}{2.604767in}%
\pgfsys@useobject{currentmarker}{}%
\end{pgfscope}%
\begin{pgfscope}%
\pgfsys@transformshift{2.725151in}{2.333788in}%
\pgfsys@useobject{currentmarker}{}%
\end{pgfscope}%
\begin{pgfscope}%
\pgfsys@transformshift{2.705195in}{2.107938in}%
\pgfsys@useobject{currentmarker}{}%
\end{pgfscope}%
\begin{pgfscope}%
\pgfsys@transformshift{2.685240in}{2.252815in}%
\pgfsys@useobject{currentmarker}{}%
\end{pgfscope}%
\begin{pgfscope}%
\pgfsys@transformshift{2.667632in}{2.579278in}%
\pgfsys@useobject{currentmarker}{}%
\end{pgfscope}%
\begin{pgfscope}%
\pgfsys@transformshift{2.648381in}{2.818794in}%
\pgfsys@useobject{currentmarker}{}%
\end{pgfscope}%
\begin{pgfscope}%
\pgfsys@transformshift{2.629599in}{2.729523in}%
\pgfsys@useobject{currentmarker}{}%
\end{pgfscope}%
\begin{pgfscope}%
\pgfsys@transformshift{2.608468in}{2.374706in}%
\pgfsys@useobject{currentmarker}{}%
\end{pgfscope}%
\begin{pgfscope}%
\pgfsys@transformshift{2.590391in}{2.151891in}%
\pgfsys@useobject{currentmarker}{}%
\end{pgfscope}%
\begin{pgfscope}%
\pgfsys@transformshift{2.573957in}{2.042456in}%
\pgfsys@useobject{currentmarker}{}%
\end{pgfscope}%
\begin{pgfscope}%
\pgfsys@transformshift{2.552594in}{1.976762in}%
\pgfsys@useobject{currentmarker}{}%
\end{pgfscope}%
\begin{pgfscope}%
\pgfsys@transformshift{2.533577in}{1.971466in}%
\pgfsys@useobject{currentmarker}{}%
\end{pgfscope}%
\begin{pgfscope}%
\pgfsys@transformshift{2.512446in}{2.018723in}%
\pgfsys@useobject{currentmarker}{}%
\end{pgfscope}%
\begin{pgfscope}%
\pgfsys@transformshift{2.493195in}{2.068099in}%
\pgfsys@useobject{currentmarker}{}%
\end{pgfscope}%
\begin{pgfscope}%
\pgfsys@transformshift{2.493429in}{2.167219in}%
\pgfsys@useobject{currentmarker}{}%
\end{pgfscope}%
\begin{pgfscope}%
\pgfsys@transformshift{2.475352in}{2.244086in}%
\pgfsys@useobject{currentmarker}{}%
\end{pgfscope}%
\begin{pgfscope}%
\pgfsys@transformshift{2.456336in}{2.563779in}%
\pgfsys@useobject{currentmarker}{}%
\end{pgfscope}%
\begin{pgfscope}%
\pgfsys@transformshift{2.438493in}{2.725739in}%
\pgfsys@useobject{currentmarker}{}%
\end{pgfscope}%
\begin{pgfscope}%
\pgfsys@transformshift{2.419242in}{2.817533in}%
\pgfsys@useobject{currentmarker}{}%
\end{pgfscope}%
\begin{pgfscope}%
\pgfsys@transformshift{2.398113in}{2.570783in}%
\pgfsys@useobject{currentmarker}{}%
\end{pgfscope}%
\begin{pgfscope}%
\pgfsys@transformshift{2.378862in}{2.270376in}%
\pgfsys@useobject{currentmarker}{}%
\end{pgfscope}%
\begin{pgfscope}%
\pgfsys@transformshift{2.361019in}{2.136122in}%
\pgfsys@useobject{currentmarker}{}%
\end{pgfscope}%
\begin{pgfscope}%
\pgfsys@transformshift{2.343175in}{2.028211in}%
\pgfsys@useobject{currentmarker}{}%
\end{pgfscope}%
\begin{pgfscope}%
\pgfsys@transformshift{2.320403in}{1.974647in}%
\pgfsys@useobject{currentmarker}{}%
\end{pgfscope}%
\begin{pgfscope}%
\pgfsys@transformshift{2.302092in}{1.978993in}%
\pgfsys@useobject{currentmarker}{}%
\end{pgfscope}%
\begin{pgfscope}%
\pgfsys@transformshift{2.284952in}{2.017064in}%
\pgfsys@useobject{currentmarker}{}%
\end{pgfscope}%
\begin{pgfscope}%
\pgfsys@transformshift{2.264293in}{2.136865in}%
\pgfsys@useobject{currentmarker}{}%
\end{pgfscope}%
\begin{pgfscope}%
\pgfsys@transformshift{2.245510in}{2.384638in}%
\pgfsys@useobject{currentmarker}{}%
\end{pgfscope}%
\begin{pgfscope}%
\pgfsys@transformshift{2.224147in}{2.704608in}%
\pgfsys@useobject{currentmarker}{}%
\end{pgfscope}%
\begin{pgfscope}%
\pgfsys@transformshift{2.224850in}{2.804547in}%
\pgfsys@useobject{currentmarker}{}%
\end{pgfscope}%
\begin{pgfscope}%
\pgfsys@transformshift{2.205365in}{2.824151in}%
\pgfsys@useobject{currentmarker}{}%
\end{pgfscope}%
\begin{pgfscope}%
\pgfsys@transformshift{2.187991in}{2.669515in}%
\pgfsys@useobject{currentmarker}{}%
\end{pgfscope}%
\begin{pgfscope}%
\pgfsys@transformshift{2.169209in}{2.377914in}%
\pgfsys@useobject{currentmarker}{}%
\end{pgfscope}%
\begin{pgfscope}%
\pgfsys@transformshift{2.147377in}{2.162181in}%
\pgfsys@useobject{currentmarker}{}%
\end{pgfscope}%
\begin{pgfscope}%
\pgfsys@transformshift{2.128594in}{2.045591in}%
\pgfsys@useobject{currentmarker}{}%
\end{pgfscope}%
\begin{pgfscope}%
\pgfsys@transformshift{2.110517in}{1.988463in}%
\pgfsys@useobject{currentmarker}{}%
\end{pgfscope}%
\begin{pgfscope}%
\pgfsys@transformshift{2.093613in}{1.971689in}%
\pgfsys@useobject{currentmarker}{}%
\end{pgfscope}%
\begin{pgfscope}%
\pgfsys@transformshift{2.070604in}{2.004993in}%
\pgfsys@useobject{currentmarker}{}%
\end{pgfscope}%
\begin{pgfscope}%
\pgfsys@transformshift{2.052293in}{2.075576in}%
\pgfsys@useobject{currentmarker}{}%
\end{pgfscope}%
\begin{pgfscope}%
\pgfsys@transformshift{2.033042in}{2.239777in}%
\pgfsys@useobject{currentmarker}{}%
\end{pgfscope}%
\begin{pgfscope}%
\pgfsys@transformshift{2.012616in}{2.580365in}%
\pgfsys@useobject{currentmarker}{}%
\end{pgfscope}%
\begin{pgfscope}%
\pgfsys@transformshift{1.996651in}{2.791116in}%
\pgfsys@useobject{currentmarker}{}%
\end{pgfscope}%
\begin{pgfscope}%
\pgfsys@transformshift{1.971531in}{2.784457in}%
\pgfsys@useobject{currentmarker}{}%
\end{pgfscope}%
\begin{pgfscope}%
\pgfsys@transformshift{1.959089in}{2.652989in}%
\pgfsys@useobject{currentmarker}{}%
\end{pgfscope}%
\begin{pgfscope}%
\pgfsys@transformshift{1.937723in}{2.379821in}%
\pgfsys@useobject{currentmarker}{}%
\end{pgfscope}%
\begin{pgfscope}%
\pgfsys@transformshift{1.918707in}{2.167115in}%
\pgfsys@useobject{currentmarker}{}%
\end{pgfscope}%
\begin{pgfscope}%
\pgfsys@transformshift{1.898047in}{2.039925in}%
\pgfsys@useobject{currentmarker}{}%
\end{pgfscope}%
\begin{pgfscope}%
\pgfsys@transformshift{1.879735in}{1.987473in}%
\pgfsys@useobject{currentmarker}{}%
\end{pgfscope}%
\begin{pgfscope}%
\pgfsys@transformshift{1.859779in}{1.973392in}%
\pgfsys@useobject{currentmarker}{}%
\end{pgfscope}%
\begin{pgfscope}%
\pgfsys@transformshift{1.842171in}{1.988866in}%
\pgfsys@useobject{currentmarker}{}%
\end{pgfscope}%
\begin{pgfscope}%
\pgfsys@transformshift{1.821511in}{2.054341in}%
\pgfsys@useobject{currentmarker}{}%
\end{pgfscope}%
\begin{pgfscope}%
\pgfsys@transformshift{1.802025in}{2.144535in}%
\pgfsys@useobject{currentmarker}{}%
\end{pgfscope}%
\begin{pgfscope}%
\pgfsys@transformshift{1.781600in}{2.387931in}%
\pgfsys@useobject{currentmarker}{}%
\end{pgfscope}%
\begin{pgfscope}%
\pgfsys@transformshift{1.765166in}{2.680521in}%
\pgfsys@useobject{currentmarker}{}%
\end{pgfscope}%
\begin{pgfscope}%
\pgfsys@transformshift{1.744272in}{2.806000in}%
\pgfsys@useobject{currentmarker}{}%
\end{pgfscope}%
\begin{pgfscope}%
\pgfsys@transformshift{1.727369in}{2.840309in}%
\pgfsys@useobject{currentmarker}{}%
\end{pgfscope}%
\begin{pgfscope}%
\pgfsys@transformshift{1.707178in}{2.671188in}%
\pgfsys@useobject{currentmarker}{}%
\end{pgfscope}%
\begin{pgfscope}%
\pgfsys@transformshift{1.688161in}{2.430740in}%
\pgfsys@useobject{currentmarker}{}%
\end{pgfscope}%
\begin{pgfscope}%
\pgfsys@transformshift{1.668205in}{2.177083in}%
\pgfsys@useobject{currentmarker}{}%
\end{pgfscope}%
\begin{pgfscope}%
\pgfsys@transformshift{1.645433in}{2.061466in}%
\pgfsys@useobject{currentmarker}{}%
\end{pgfscope}%
\begin{pgfscope}%
\pgfsys@transformshift{1.631111in}{2.015737in}%
\pgfsys@useobject{currentmarker}{}%
\end{pgfscope}%
\begin{pgfscope}%
\pgfsys@transformshift{1.609748in}{1.977699in}%
\pgfsys@useobject{currentmarker}{}%
\end{pgfscope}%
\begin{pgfscope}%
\pgfsys@transformshift{1.588617in}{1.988968in}%
\pgfsys@useobject{currentmarker}{}%
\end{pgfscope}%
\begin{pgfscope}%
\pgfsys@transformshift{1.572652in}{2.023456in}%
\pgfsys@useobject{currentmarker}{}%
\end{pgfscope}%
\begin{pgfscope}%
\pgfsys@transformshift{1.551992in}{2.090402in}%
\pgfsys@useobject{currentmarker}{}%
\end{pgfscope}%
\begin{pgfscope}%
\pgfsys@transformshift{1.533915in}{2.166676in}%
\pgfsys@useobject{currentmarker}{}%
\end{pgfscope}%
\begin{pgfscope}%
\pgfsys@transformshift{1.514664in}{2.025046in}%
\pgfsys@useobject{currentmarker}{}%
\end{pgfscope}%
\begin{pgfscope}%
\pgfsys@transformshift{1.497996in}{1.982370in}%
\pgfsys@useobject{currentmarker}{}%
\end{pgfscope}%
\begin{pgfscope}%
\pgfsys@transformshift{1.476630in}{1.988453in}%
\pgfsys@useobject{currentmarker}{}%
\end{pgfscope}%
\begin{pgfscope}%
\pgfsys@transformshift{1.455736in}{2.030245in}%
\pgfsys@useobject{currentmarker}{}%
\end{pgfscope}%
\begin{pgfscope}%
\pgfsys@transformshift{1.438128in}{2.119463in}%
\pgfsys@useobject{currentmarker}{}%
\end{pgfscope}%
\begin{pgfscope}%
\pgfsys@transformshift{1.418173in}{2.347217in}%
\pgfsys@useobject{currentmarker}{}%
\end{pgfscope}%
\begin{pgfscope}%
\pgfsys@transformshift{1.401269in}{2.623911in}%
\pgfsys@useobject{currentmarker}{}%
\end{pgfscope}%
\begin{pgfscope}%
\pgfsys@transformshift{1.379200in}{2.848420in}%
\pgfsys@useobject{currentmarker}{}%
\end{pgfscope}%
\begin{pgfscope}%
\pgfsys@transformshift{1.361826in}{2.843445in}%
\pgfsys@useobject{currentmarker}{}%
\end{pgfscope}%
\begin{pgfscope}%
\pgfsys@transformshift{1.341403in}{2.586093in}%
\pgfsys@useobject{currentmarker}{}%
\end{pgfscope}%
\begin{pgfscope}%
\pgfsys@transformshift{1.321446in}{2.328727in}%
\pgfsys@useobject{currentmarker}{}%
\end{pgfscope}%
\begin{pgfscope}%
\pgfsys@transformshift{1.303369in}{2.147780in}%
\pgfsys@useobject{currentmarker}{}%
\end{pgfscope}%
\begin{pgfscope}%
\pgfsys@transformshift{1.284587in}{2.045822in}%
\pgfsys@useobject{currentmarker}{}%
\end{pgfscope}%
\begin{pgfscope}%
\pgfsys@transformshift{1.265805in}{2.020328in}%
\pgfsys@useobject{currentmarker}{}%
\end{pgfscope}%
\begin{pgfscope}%
\pgfsys@transformshift{1.246319in}{1.979095in}%
\pgfsys@useobject{currentmarker}{}%
\end{pgfscope}%
\begin{pgfscope}%
\pgfsys@transformshift{1.227302in}{1.992796in}%
\pgfsys@useobject{currentmarker}{}%
\end{pgfscope}%
\begin{pgfscope}%
\pgfsys@transformshift{1.206642in}{2.054231in}%
\pgfsys@useobject{currentmarker}{}%
\end{pgfscope}%
\begin{pgfscope}%
\pgfsys@transformshift{1.183165in}{2.216138in}%
\pgfsys@useobject{currentmarker}{}%
\end{pgfscope}%
\begin{pgfscope}%
\pgfsys@transformshift{1.168140in}{2.408461in}%
\pgfsys@useobject{currentmarker}{}%
\end{pgfscope}%
\begin{pgfscope}%
\pgfsys@transformshift{1.151001in}{2.644714in}%
\pgfsys@useobject{currentmarker}{}%
\end{pgfscope}%
\begin{pgfscope}%
\pgfsys@transformshift{1.133393in}{2.858660in}%
\pgfsys@useobject{currentmarker}{}%
\end{pgfscope}%
\begin{pgfscope}%
\pgfsys@transformshift{1.109212in}{2.849015in}%
\pgfsys@useobject{currentmarker}{}%
\end{pgfscope}%
\begin{pgfscope}%
\pgfsys@transformshift{1.093013in}{2.671143in}%
\pgfsys@useobject{currentmarker}{}%
\end{pgfscope}%
\begin{pgfscope}%
\pgfsys@transformshift{1.073056in}{2.324396in}%
\pgfsys@useobject{currentmarker}{}%
\end{pgfscope}%
\begin{pgfscope}%
\pgfsys@transformshift{1.052867in}{2.168781in}%
\pgfsys@useobject{currentmarker}{}%
\end{pgfscope}%
\begin{pgfscope}%
\pgfsys@transformshift{1.035259in}{2.081224in}%
\pgfsys@useobject{currentmarker}{}%
\end{pgfscope}%
\begin{pgfscope}%
\pgfsys@transformshift{1.014599in}{2.013185in}%
\pgfsys@useobject{currentmarker}{}%
\end{pgfscope}%
\begin{pgfscope}%
\pgfsys@transformshift{0.993705in}{1.982976in}%
\pgfsys@useobject{currentmarker}{}%
\end{pgfscope}%
\begin{pgfscope}%
\pgfsys@transformshift{0.976097in}{2.010241in}%
\pgfsys@useobject{currentmarker}{}%
\end{pgfscope}%
\begin{pgfscope}%
\pgfsys@transformshift{0.959192in}{2.047343in}%
\pgfsys@useobject{currentmarker}{}%
\end{pgfscope}%
\begin{pgfscope}%
\pgfsys@transformshift{0.940410in}{2.145801in}%
\pgfsys@useobject{currentmarker}{}%
\end{pgfscope}%
\begin{pgfscope}%
\pgfsys@transformshift{0.920455in}{2.277803in}%
\pgfsys@useobject{currentmarker}{}%
\end{pgfscope}%
\begin{pgfscope}%
\pgfsys@transformshift{0.895100in}{2.609455in}%
\pgfsys@useobject{currentmarker}{}%
\end{pgfscope}%
\begin{pgfscope}%
\pgfsys@transformshift{0.880779in}{2.815231in}%
\pgfsys@useobject{currentmarker}{}%
\end{pgfscope}%
\begin{pgfscope}%
\pgfsys@transformshift{0.859413in}{2.914807in}%
\pgfsys@useobject{currentmarker}{}%
\end{pgfscope}%
\begin{pgfscope}%
\pgfsys@transformshift{0.841336in}{2.866785in}%
\pgfsys@useobject{currentmarker}{}%
\end{pgfscope}%
\begin{pgfscope}%
\pgfsys@transformshift{0.823025in}{2.710395in}%
\pgfsys@useobject{currentmarker}{}%
\end{pgfscope}%
\begin{pgfscope}%
\pgfsys@transformshift{0.805652in}{2.438360in}%
\pgfsys@useobject{currentmarker}{}%
\end{pgfscope}%
\begin{pgfscope}%
\pgfsys@transformshift{0.784052in}{2.240641in}%
\pgfsys@useobject{currentmarker}{}%
\end{pgfscope}%
\begin{pgfscope}%
\pgfsys@transformshift{0.766209in}{2.121364in}%
\pgfsys@useobject{currentmarker}{}%
\end{pgfscope}%
\begin{pgfscope}%
\pgfsys@transformshift{0.743437in}{2.039382in}%
\pgfsys@useobject{currentmarker}{}%
\end{pgfscope}%
\begin{pgfscope}%
\pgfsys@transformshift{0.726064in}{1.994827in}%
\pgfsys@useobject{currentmarker}{}%
\end{pgfscope}%
\begin{pgfscope}%
\pgfsys@transformshift{0.708690in}{1.991306in}%
\pgfsys@useobject{currentmarker}{}%
\end{pgfscope}%
\begin{pgfscope}%
\pgfsys@transformshift{0.688499in}{2.038026in}%
\pgfsys@useobject{currentmarker}{}%
\end{pgfscope}%
\begin{pgfscope}%
\pgfsys@transformshift{0.669719in}{2.117516in}%
\pgfsys@useobject{currentmarker}{}%
\end{pgfscope}%
\begin{pgfscope}%
\pgfsys@transformshift{0.650468in}{2.214366in}%
\pgfsys@useobject{currentmarker}{}%
\end{pgfscope}%
\begin{pgfscope}%
\pgfsys@transformshift{0.650468in}{2.203978in}%
\pgfsys@useobject{currentmarker}{}%
\end{pgfscope}%
\begin{pgfscope}%
\pgfsys@transformshift{0.655163in}{2.180215in}%
\pgfsys@useobject{currentmarker}{}%
\end{pgfscope}%
\begin{pgfscope}%
\pgfsys@transformshift{0.674648in}{2.033976in}%
\pgfsys@useobject{currentmarker}{}%
\end{pgfscope}%
\begin{pgfscope}%
\pgfsys@transformshift{0.696248in}{1.988211in}%
\pgfsys@useobject{currentmarker}{}%
\end{pgfscope}%
\begin{pgfscope}%
\pgfsys@transformshift{0.714091in}{2.037666in}%
\pgfsys@useobject{currentmarker}{}%
\end{pgfscope}%
\begin{pgfscope}%
\pgfsys@transformshift{0.731933in}{2.147806in}%
\pgfsys@useobject{currentmarker}{}%
\end{pgfscope}%
\begin{pgfscope}%
\pgfsys@transformshift{0.752827in}{2.425479in}%
\pgfsys@useobject{currentmarker}{}%
\end{pgfscope}%
\begin{pgfscope}%
\pgfsys@transformshift{0.769967in}{2.792864in}%
\pgfsys@useobject{currentmarker}{}%
\end{pgfscope}%
\begin{pgfscope}%
\pgfsys@transformshift{0.789218in}{2.915272in}%
\pgfsys@useobject{currentmarker}{}%
\end{pgfscope}%
\begin{pgfscope}%
\pgfsys@transformshift{0.809643in}{2.679831in}%
\pgfsys@useobject{currentmarker}{}%
\end{pgfscope}%
\begin{pgfscope}%
\pgfsys@transformshift{0.827486in}{2.328313in}%
\pgfsys@useobject{currentmarker}{}%
\end{pgfscope}%
\begin{pgfscope}%
\pgfsys@transformshift{0.847911in}{2.085033in}%
\pgfsys@useobject{currentmarker}{}%
\end{pgfscope}%
\begin{pgfscope}%
\pgfsys@transformshift{0.865754in}{2.003717in}%
\pgfsys@useobject{currentmarker}{}%
\end{pgfscope}%
\begin{pgfscope}%
\pgfsys@transformshift{0.887117in}{1.993888in}%
\pgfsys@useobject{currentmarker}{}%
\end{pgfscope}%
\begin{pgfscope}%
\pgfsys@transformshift{0.904256in}{2.056122in}%
\pgfsys@useobject{currentmarker}{}%
\end{pgfscope}%
\begin{pgfscope}%
\pgfsys@transformshift{0.929377in}{2.444373in}%
\pgfsys@useobject{currentmarker}{}%
\end{pgfscope}%
\begin{pgfscope}%
\pgfsys@transformshift{0.946985in}{2.819400in}%
\pgfsys@useobject{currentmarker}{}%
\end{pgfscope}%
\begin{pgfscope}%
\pgfsys@transformshift{0.962010in}{2.889953in}%
\pgfsys@useobject{currentmarker}{}%
\end{pgfscope}%
\begin{pgfscope}%
\pgfsys@transformshift{0.982670in}{2.624763in}%
\pgfsys@useobject{currentmarker}{}%
\end{pgfscope}%
\begin{pgfscope}%
\pgfsys@transformshift{1.003564in}{2.236151in}%
\pgfsys@useobject{currentmarker}{}%
\end{pgfscope}%
\begin{pgfscope}%
\pgfsys@transformshift{1.022112in}{2.066283in}%
\pgfsys@useobject{currentmarker}{}%
\end{pgfscope}%
\begin{pgfscope}%
\pgfsys@transformshift{1.040189in}{1.995678in}%
\pgfsys@useobject{currentmarker}{}%
\end{pgfscope}%
\begin{pgfscope}%
\pgfsys@transformshift{1.061083in}{1.991114in}%
\pgfsys@useobject{currentmarker}{}%
\end{pgfscope}%
\begin{pgfscope}%
\pgfsys@transformshift{1.079162in}{2.059954in}%
\pgfsys@useobject{currentmarker}{}%
\end{pgfscope}%
\begin{pgfscope}%
\pgfsys@transformshift{1.095830in}{2.193155in}%
\pgfsys@useobject{currentmarker}{}%
\end{pgfscope}%
\begin{pgfscope}%
\pgfsys@transformshift{1.115316in}{2.512743in}%
\pgfsys@useobject{currentmarker}{}%
\end{pgfscope}%
\begin{pgfscope}%
\pgfsys@transformshift{1.139968in}{2.860256in}%
\pgfsys@useobject{currentmarker}{}%
\end{pgfscope}%
\begin{pgfscope}%
\pgfsys@transformshift{1.156167in}{2.800656in}%
\pgfsys@useobject{currentmarker}{}%
\end{pgfscope}%
\begin{pgfscope}%
\pgfsys@transformshift{1.173775in}{2.458007in}%
\pgfsys@useobject{currentmarker}{}%
\end{pgfscope}%
\begin{pgfscope}%
\pgfsys@transformshift{1.195138in}{2.153027in}%
\pgfsys@useobject{currentmarker}{}%
\end{pgfscope}%
\begin{pgfscope}%
\pgfsys@transformshift{1.212981in}{2.030850in}%
\pgfsys@useobject{currentmarker}{}%
\end{pgfscope}%
\begin{pgfscope}%
\pgfsys@transformshift{1.234580in}{1.978891in}%
\pgfsys@useobject{currentmarker}{}%
\end{pgfscope}%
\begin{pgfscope}%
\pgfsys@transformshift{1.252423in}{1.994417in}%
\pgfsys@useobject{currentmarker}{}%
\end{pgfscope}%
\begin{pgfscope}%
\pgfsys@transformshift{1.271205in}{2.047543in}%
\pgfsys@useobject{currentmarker}{}%
\end{pgfscope}%
\begin{pgfscope}%
\pgfsys@transformshift{1.291396in}{2.208403in}%
\pgfsys@useobject{currentmarker}{}%
\end{pgfscope}%
\begin{pgfscope}%
\pgfsys@transformshift{1.309708in}{2.492810in}%
\pgfsys@useobject{currentmarker}{}%
\end{pgfscope}%
\begin{pgfscope}%
\pgfsys@transformshift{1.328255in}{2.794028in}%
\pgfsys@useobject{currentmarker}{}%
\end{pgfscope}%
\begin{pgfscope}%
\pgfsys@transformshift{1.348915in}{2.811223in}%
\pgfsys@useobject{currentmarker}{}%
\end{pgfscope}%
\begin{pgfscope}%
\pgfsys@transformshift{1.366992in}{2.517304in}%
\pgfsys@useobject{currentmarker}{}%
\end{pgfscope}%
\begin{pgfscope}%
\pgfsys@transformshift{1.385304in}{2.222386in}%
\pgfsys@useobject{currentmarker}{}%
\end{pgfscope}%
\begin{pgfscope}%
\pgfsys@transformshift{1.405026in}{2.052822in}%
\pgfsys@useobject{currentmarker}{}%
\end{pgfscope}%
\begin{pgfscope}%
\pgfsys@transformshift{1.429441in}{1.978432in}%
\pgfsys@useobject{currentmarker}{}%
\end{pgfscope}%
\begin{pgfscope}%
\pgfsys@transformshift{1.444468in}{1.977937in}%
\pgfsys@useobject{currentmarker}{}%
\end{pgfscope}%
\begin{pgfscope}%
\pgfsys@transformshift{1.464423in}{2.044112in}%
\pgfsys@useobject{currentmarker}{}%
\end{pgfscope}%
\begin{pgfscope}%
\pgfsys@transformshift{1.482736in}{2.126486in}%
\pgfsys@useobject{currentmarker}{}%
\end{pgfscope}%
\begin{pgfscope}%
\pgfsys@transformshift{1.501047in}{2.307052in}%
\pgfsys@useobject{currentmarker}{}%
\end{pgfscope}%
\begin{pgfscope}%
\pgfsys@transformshift{1.518890in}{2.650038in}%
\pgfsys@useobject{currentmarker}{}%
\end{pgfscope}%
\begin{pgfscope}%
\pgfsys@transformshift{1.540253in}{2.842567in}%
\pgfsys@useobject{currentmarker}{}%
\end{pgfscope}%
\begin{pgfscope}%
\pgfsys@transformshift{1.559270in}{2.713705in}%
\pgfsys@useobject{currentmarker}{}%
\end{pgfscope}%
\begin{pgfscope}%
\pgfsys@transformshift{1.579461in}{2.487830in}%
\pgfsys@useobject{currentmarker}{}%
\end{pgfscope}%
\begin{pgfscope}%
\pgfsys@transformshift{1.597069in}{2.205906in}%
\pgfsys@useobject{currentmarker}{}%
\end{pgfscope}%
\begin{pgfscope}%
\pgfsys@transformshift{1.616789in}{2.040348in}%
\pgfsys@useobject{currentmarker}{}%
\end{pgfscope}%
\begin{pgfscope}%
\pgfsys@transformshift{1.635103in}{1.988504in}%
\pgfsys@useobject{currentmarker}{}%
\end{pgfscope}%
\begin{pgfscope}%
\pgfsys@transformshift{1.653179in}{1.972854in}%
\pgfsys@useobject{currentmarker}{}%
\end{pgfscope}%
\begin{pgfscope}%
\pgfsys@transformshift{1.674545in}{2.007779in}%
\pgfsys@useobject{currentmarker}{}%
\end{pgfscope}%
\begin{pgfscope}%
\pgfsys@transformshift{1.694970in}{2.076828in}%
\pgfsys@useobject{currentmarker}{}%
\end{pgfscope}%
\begin{pgfscope}%
\pgfsys@transformshift{1.711638in}{2.164390in}%
\pgfsys@useobject{currentmarker}{}%
\end{pgfscope}%
\begin{pgfscope}%
\pgfsys@transformshift{1.731124in}{2.480501in}%
\pgfsys@useobject{currentmarker}{}%
\end{pgfscope}%
\begin{pgfscope}%
\pgfsys@transformshift{1.753896in}{2.802848in}%
\pgfsys@useobject{currentmarker}{}%
\end{pgfscope}%
\begin{pgfscope}%
\pgfsys@transformshift{1.771270in}{2.830384in}%
\pgfsys@useobject{currentmarker}{}%
\end{pgfscope}%
\begin{pgfscope}%
\pgfsys@transformshift{1.790286in}{2.674118in}%
\pgfsys@useobject{currentmarker}{}%
\end{pgfscope}%
\begin{pgfscope}%
\pgfsys@transformshift{1.810243in}{2.319075in}%
\pgfsys@useobject{currentmarker}{}%
\end{pgfscope}%
\begin{pgfscope}%
\pgfsys@transformshift{1.827851in}{2.131229in}%
\pgfsys@useobject{currentmarker}{}%
\end{pgfscope}%
\begin{pgfscope}%
\pgfsys@transformshift{1.846868in}{2.031130in}%
\pgfsys@useobject{currentmarker}{}%
\end{pgfscope}%
\begin{pgfscope}%
\pgfsys@transformshift{1.866353in}{1.979369in}%
\pgfsys@useobject{currentmarker}{}%
\end{pgfscope}%
\begin{pgfscope}%
\pgfsys@transformshift{1.885370in}{1.975537in}%
\pgfsys@useobject{currentmarker}{}%
\end{pgfscope}%
\begin{pgfscope}%
\pgfsys@transformshift{1.905325in}{2.024651in}%
\pgfsys@useobject{currentmarker}{}%
\end{pgfscope}%
\begin{pgfscope}%
\pgfsys@transformshift{1.923404in}{2.106341in}%
\pgfsys@useobject{currentmarker}{}%
\end{pgfscope}%
\begin{pgfscope}%
\pgfsys@transformshift{1.945707in}{2.276669in}%
\pgfsys@useobject{currentmarker}{}%
\end{pgfscope}%
\begin{pgfscope}%
\pgfsys@transformshift{1.962610in}{2.393709in}%
\pgfsys@useobject{currentmarker}{}%
\end{pgfscope}%
\begin{pgfscope}%
\pgfsys@transformshift{1.983504in}{2.743104in}%
\pgfsys@useobject{currentmarker}{}%
\end{pgfscope}%
\begin{pgfscope}%
\pgfsys@transformshift{2.001583in}{2.820689in}%
\pgfsys@useobject{currentmarker}{}%
\end{pgfscope}%
\begin{pgfscope}%
\pgfsys@transformshift{2.019425in}{2.667313in}%
\pgfsys@useobject{currentmarker}{}%
\end{pgfscope}%
\begin{pgfscope}%
\pgfsys@transformshift{2.041023in}{2.294316in}%
\pgfsys@useobject{currentmarker}{}%
\end{pgfscope}%
\begin{pgfscope}%
\pgfsys@transformshift{2.058866in}{2.103668in}%
\pgfsys@useobject{currentmarker}{}%
\end{pgfscope}%
\begin{pgfscope}%
\pgfsys@transformshift{2.077413in}{2.017085in}%
\pgfsys@useobject{currentmarker}{}%
\end{pgfscope}%
\begin{pgfscope}%
\pgfsys@transformshift{2.096430in}{2.754403in}%
\pgfsys@useobject{currentmarker}{}%
\end{pgfscope}%
\begin{pgfscope}%
\pgfsys@transformshift{2.117090in}{2.367410in}%
\pgfsys@useobject{currentmarker}{}%
\end{pgfscope}%
\begin{pgfscope}%
\pgfsys@transformshift{2.136810in}{2.111960in}%
\pgfsys@useobject{currentmarker}{}%
\end{pgfscope}%
\begin{pgfscope}%
\pgfsys@transformshift{2.154418in}{2.016005in}%
\pgfsys@useobject{currentmarker}{}%
\end{pgfscope}%
\begin{pgfscope}%
\pgfsys@transformshift{2.176252in}{1.972361in}%
\pgfsys@useobject{currentmarker}{}%
\end{pgfscope}%
\begin{pgfscope}%
\pgfsys@transformshift{2.193860in}{1.986254in}%
\pgfsys@useobject{currentmarker}{}%
\end{pgfscope}%
\begin{pgfscope}%
\pgfsys@transformshift{2.211937in}{2.047174in}%
\pgfsys@useobject{currentmarker}{}%
\end{pgfscope}%
\begin{pgfscope}%
\pgfsys@transformshift{2.233772in}{2.168052in}%
\pgfsys@useobject{currentmarker}{}%
\end{pgfscope}%
\begin{pgfscope}%
\pgfsys@transformshift{2.251849in}{2.409865in}%
\pgfsys@useobject{currentmarker}{}%
\end{pgfscope}%
\begin{pgfscope}%
\pgfsys@transformshift{2.267813in}{2.584504in}%
\pgfsys@useobject{currentmarker}{}%
\end{pgfscope}%
\begin{pgfscope}%
\pgfsys@transformshift{2.289413in}{2.822122in}%
\pgfsys@useobject{currentmarker}{}%
\end{pgfscope}%
\begin{pgfscope}%
\pgfsys@transformshift{2.307021in}{2.732178in}%
\pgfsys@useobject{currentmarker}{}%
\end{pgfscope}%
\begin{pgfscope}%
\pgfsys@transformshift{2.327447in}{2.389687in}%
\pgfsys@useobject{currentmarker}{}%
\end{pgfscope}%
\begin{pgfscope}%
\pgfsys@transformshift{2.348575in}{2.114257in}%
\pgfsys@useobject{currentmarker}{}%
\end{pgfscope}%
\begin{pgfscope}%
\pgfsys@transformshift{2.366889in}{2.026406in}%
\pgfsys@useobject{currentmarker}{}%
\end{pgfscope}%
\begin{pgfscope}%
\pgfsys@transformshift{2.386140in}{1.978562in}%
\pgfsys@useobject{currentmarker}{}%
\end{pgfscope}%
\begin{pgfscope}%
\pgfsys@transformshift{2.403982in}{1.973225in}%
\pgfsys@useobject{currentmarker}{}%
\end{pgfscope}%
\begin{pgfscope}%
\pgfsys@transformshift{2.423468in}{1.993870in}%
\pgfsys@useobject{currentmarker}{}%
\end{pgfscope}%
\begin{pgfscope}%
\pgfsys@transformshift{2.441780in}{2.055931in}%
\pgfsys@useobject{currentmarker}{}%
\end{pgfscope}%
\begin{pgfscope}%
\pgfsys@transformshift{2.460562in}{2.204830in}%
\pgfsys@useobject{currentmarker}{}%
\end{pgfscope}%
\begin{pgfscope}%
\pgfsys@transformshift{2.485917in}{2.589222in}%
\pgfsys@useobject{currentmarker}{}%
\end{pgfscope}%
\begin{pgfscope}%
\pgfsys@transformshift{2.501178in}{2.783238in}%
\pgfsys@useobject{currentmarker}{}%
\end{pgfscope}%
\begin{pgfscope}%
\pgfsys@transformshift{2.518316in}{2.803880in}%
\pgfsys@useobject{currentmarker}{}%
\end{pgfscope}%
\begin{pgfscope}%
\pgfsys@transformshift{2.539681in}{2.588078in}%
\pgfsys@useobject{currentmarker}{}%
\end{pgfscope}%
\begin{pgfscope}%
\pgfsys@transformshift{2.557992in}{2.275223in}%
\pgfsys@useobject{currentmarker}{}%
\end{pgfscope}%
\begin{pgfscope}%
\pgfsys@transformshift{2.578652in}{2.076856in}%
\pgfsys@useobject{currentmarker}{}%
\end{pgfscope}%
\begin{pgfscope}%
\pgfsys@transformshift{2.597903in}{2.005800in}%
\pgfsys@useobject{currentmarker}{}%
\end{pgfscope}%
\begin{pgfscope}%
\pgfsys@transformshift{2.617860in}{1.970774in}%
\pgfsys@useobject{currentmarker}{}%
\end{pgfscope}%
\begin{pgfscope}%
\pgfsys@transformshift{2.637346in}{1.978407in}%
\pgfsys@useobject{currentmarker}{}%
\end{pgfscope}%
\begin{pgfscope}%
\pgfsys@transformshift{2.655422in}{2.024695in}%
\pgfsys@useobject{currentmarker}{}%
\end{pgfscope}%
\begin{pgfscope}%
\pgfsys@transformshift{2.674439in}{2.086856in}%
\pgfsys@useobject{currentmarker}{}%
\end{pgfscope}%
\begin{pgfscope}%
\pgfsys@transformshift{2.692753in}{2.212793in}%
\pgfsys@useobject{currentmarker}{}%
\end{pgfscope}%
\begin{pgfscope}%
\pgfsys@transformshift{2.714350in}{2.492972in}%
\pgfsys@useobject{currentmarker}{}%
\end{pgfscope}%
\begin{pgfscope}%
\pgfsys@transformshift{2.732898in}{2.759943in}%
\pgfsys@useobject{currentmarker}{}%
\end{pgfscope}%
\begin{pgfscope}%
\pgfsys@transformshift{2.751915in}{2.824430in}%
\pgfsys@useobject{currentmarker}{}%
\end{pgfscope}%
\begin{pgfscope}%
\pgfsys@transformshift{2.768349in}{2.699345in}%
\pgfsys@useobject{currentmarker}{}%
\end{pgfscope}%
\begin{pgfscope}%
\pgfsys@transformshift{2.789243in}{2.337282in}%
\pgfsys@useobject{currentmarker}{}%
\end{pgfscope}%
\begin{pgfscope}%
\pgfsys@transformshift{2.806851in}{2.125224in}%
\pgfsys@useobject{currentmarker}{}%
\end{pgfscope}%
\begin{pgfscope}%
\pgfsys@transformshift{2.826102in}{2.024991in}%
\pgfsys@useobject{currentmarker}{}%
\end{pgfscope}%
\begin{pgfscope}%
\pgfsys@transformshift{2.847468in}{1.981164in}%
\pgfsys@useobject{currentmarker}{}%
\end{pgfscope}%
\begin{pgfscope}%
\pgfsys@transformshift{2.867188in}{1.978079in}%
\pgfsys@useobject{currentmarker}{}%
\end{pgfscope}%
\begin{pgfscope}%
\pgfsys@transformshift{2.885499in}{2.013713in}%
\pgfsys@useobject{currentmarker}{}%
\end{pgfscope}%
\begin{pgfscope}%
\pgfsys@transformshift{2.904281in}{2.083087in}%
\pgfsys@useobject{currentmarker}{}%
\end{pgfscope}%
\begin{pgfscope}%
\pgfsys@transformshift{2.920481in}{2.188717in}%
\pgfsys@useobject{currentmarker}{}%
\end{pgfscope}%
\begin{pgfscope}%
\pgfsys@transformshift{2.943255in}{2.472615in}%
\pgfsys@useobject{currentmarker}{}%
\end{pgfscope}%
\begin{pgfscope}%
\pgfsys@transformshift{2.959923in}{2.732129in}%
\pgfsys@useobject{currentmarker}{}%
\end{pgfscope}%
\begin{pgfscope}%
\pgfsys@transformshift{2.982226in}{2.833179in}%
\pgfsys@useobject{currentmarker}{}%
\end{pgfscope}%
\begin{pgfscope}%
\pgfsys@transformshift{3.000774in}{2.740188in}%
\pgfsys@useobject{currentmarker}{}%
\end{pgfscope}%
\begin{pgfscope}%
\pgfsys@transformshift{3.017911in}{2.528550in}%
\pgfsys@useobject{currentmarker}{}%
\end{pgfscope}%
\begin{pgfscope}%
\pgfsys@transformshift{3.039745in}{2.212363in}%
\pgfsys@useobject{currentmarker}{}%
\end{pgfscope}%
\begin{pgfscope}%
\pgfsys@transformshift{3.058762in}{2.085427in}%
\pgfsys@useobject{currentmarker}{}%
\end{pgfscope}%
\begin{pgfscope}%
\pgfsys@transformshift{3.078953in}{2.030195in}%
\pgfsys@useobject{currentmarker}{}%
\end{pgfscope}%
\begin{pgfscope}%
\pgfsys@transformshift{3.099613in}{1.977494in}%
\pgfsys@useobject{currentmarker}{}%
\end{pgfscope}%
\begin{pgfscope}%
\pgfsys@transformshift{3.115107in}{1.974240in}%
\pgfsys@useobject{currentmarker}{}%
\end{pgfscope}%
\begin{pgfscope}%
\pgfsys@transformshift{3.136472in}{2.007263in}%
\pgfsys@useobject{currentmarker}{}%
\end{pgfscope}%
\begin{pgfscope}%
\pgfsys@transformshift{3.154549in}{2.066495in}%
\pgfsys@useobject{currentmarker}{}%
\end{pgfscope}%
\begin{pgfscope}%
\pgfsys@transformshift{3.172157in}{2.175338in}%
\pgfsys@useobject{currentmarker}{}%
\end{pgfscope}%
\begin{pgfscope}%
\pgfsys@transformshift{3.194695in}{2.419338in}%
\pgfsys@useobject{currentmarker}{}%
\end{pgfscope}%
\begin{pgfscope}%
\pgfsys@transformshift{3.211365in}{2.709700in}%
\pgfsys@useobject{currentmarker}{}%
\end{pgfscope}%
\begin{pgfscope}%
\pgfsys@transformshift{3.230851in}{2.841599in}%
\pgfsys@useobject{currentmarker}{}%
\end{pgfscope}%
\begin{pgfscope}%
\pgfsys@transformshift{3.250571in}{2.184666in}%
\pgfsys@useobject{currentmarker}{}%
\end{pgfscope}%
\begin{pgfscope}%
\pgfsys@transformshift{3.268650in}{2.346963in}%
\pgfsys@useobject{currentmarker}{}%
\end{pgfscope}%
\begin{pgfscope}%
\pgfsys@transformshift{3.288604in}{2.681030in}%
\pgfsys@useobject{currentmarker}{}%
\end{pgfscope}%
\begin{pgfscope}%
\pgfsys@transformshift{3.306681in}{2.838760in}%
\pgfsys@useobject{currentmarker}{}%
\end{pgfscope}%
\begin{pgfscope}%
\pgfsys@transformshift{3.325932in}{2.846957in}%
\pgfsys@useobject{currentmarker}{}%
\end{pgfscope}%
\begin{pgfscope}%
\pgfsys@transformshift{3.348237in}{2.601999in}%
\pgfsys@useobject{currentmarker}{}%
\end{pgfscope}%
\begin{pgfscope}%
\pgfsys@transformshift{3.367252in}{2.289773in}%
\pgfsys@useobject{currentmarker}{}%
\end{pgfscope}%
\begin{pgfscope}%
\pgfsys@transformshift{3.383922in}{2.162514in}%
\pgfsys@useobject{currentmarker}{}%
\end{pgfscope}%
\begin{pgfscope}%
\pgfsys@transformshift{3.405520in}{2.031517in}%
\pgfsys@useobject{currentmarker}{}%
\end{pgfscope}%
\begin{pgfscope}%
\pgfsys@transformshift{3.425711in}{1.980464in}%
\pgfsys@useobject{currentmarker}{}%
\end{pgfscope}%
\begin{pgfscope}%
\pgfsys@transformshift{3.441442in}{1.976877in}%
\pgfsys@useobject{currentmarker}{}%
\end{pgfscope}%
\begin{pgfscope}%
\pgfsys@transformshift{3.463745in}{2.020277in}%
\pgfsys@useobject{currentmarker}{}%
\end{pgfscope}%
\begin{pgfscope}%
\pgfsys@transformshift{3.481822in}{2.105682in}%
\pgfsys@useobject{currentmarker}{}%
\end{pgfscope}%
\begin{pgfscope}%
\pgfsys@transformshift{3.501544in}{2.263503in}%
\pgfsys@useobject{currentmarker}{}%
\end{pgfscope}%
\begin{pgfscope}%
\pgfsys@transformshift{3.521264in}{2.543356in}%
\pgfsys@useobject{currentmarker}{}%
\end{pgfscope}%
\begin{pgfscope}%
\pgfsys@transformshift{3.539575in}{2.813082in}%
\pgfsys@useobject{currentmarker}{}%
\end{pgfscope}%
\begin{pgfscope}%
\pgfsys@transformshift{3.558826in}{2.851374in}%
\pgfsys@useobject{currentmarker}{}%
\end{pgfscope}%
\begin{pgfscope}%
\pgfsys@transformshift{3.577843in}{2.689382in}%
\pgfsys@useobject{currentmarker}{}%
\end{pgfscope}%
\begin{pgfscope}%
\pgfsys@transformshift{3.598034in}{2.407388in}%
\pgfsys@useobject{currentmarker}{}%
\end{pgfscope}%
\begin{pgfscope}%
\pgfsys@transformshift{3.617991in}{2.207556in}%
\pgfsys@useobject{currentmarker}{}%
\end{pgfscope}%
\begin{pgfscope}%
\pgfsys@transformshift{3.635128in}{2.069391in}%
\pgfsys@useobject{currentmarker}{}%
\end{pgfscope}%
\begin{pgfscope}%
\pgfsys@transformshift{3.655084in}{2.007302in}%
\pgfsys@useobject{currentmarker}{}%
\end{pgfscope}%
\begin{pgfscope}%
\pgfsys@transformshift{3.673630in}{1.979127in}%
\pgfsys@useobject{currentmarker}{}%
\end{pgfscope}%
\begin{pgfscope}%
\pgfsys@transformshift{3.689126in}{1.984398in}%
\pgfsys@useobject{currentmarker}{}%
\end{pgfscope}%
\begin{pgfscope}%
\pgfsys@transformshift{3.712604in}{2.041376in}%
\pgfsys@useobject{currentmarker}{}%
\end{pgfscope}%
\begin{pgfscope}%
\pgfsys@transformshift{3.732089in}{2.095492in}%
\pgfsys@useobject{currentmarker}{}%
\end{pgfscope}%
\begin{pgfscope}%
\pgfsys@transformshift{3.750166in}{2.233841in}%
\pgfsys@useobject{currentmarker}{}%
\end{pgfscope}%
\begin{pgfscope}%
\pgfsys@transformshift{3.769888in}{2.437676in}%
\pgfsys@useobject{currentmarker}{}%
\end{pgfscope}%
\begin{pgfscope}%
\pgfsys@transformshift{3.787731in}{2.718606in}%
\pgfsys@useobject{currentmarker}{}%
\end{pgfscope}%
\begin{pgfscope}%
\pgfsys@transformshift{3.806982in}{2.855010in}%
\pgfsys@useobject{currentmarker}{}%
\end{pgfscope}%
\begin{pgfscope}%
\pgfsys@transformshift{3.826937in}{2.869987in}%
\pgfsys@useobject{currentmarker}{}%
\end{pgfscope}%
\begin{pgfscope}%
\pgfsys@transformshift{3.849242in}{2.622114in}%
\pgfsys@useobject{currentmarker}{}%
\end{pgfscope}%
\begin{pgfscope}%
\pgfsys@transformshift{3.864032in}{2.385368in}%
\pgfsys@useobject{currentmarker}{}%
\end{pgfscope}%
\begin{pgfscope}%
\pgfsys@transformshift{3.882578in}{2.177850in}%
\pgfsys@useobject{currentmarker}{}%
\end{pgfscope}%
\begin{pgfscope}%
\pgfsys@transformshift{3.905352in}{2.079390in}%
\pgfsys@useobject{currentmarker}{}%
\end{pgfscope}%
\begin{pgfscope}%
\pgfsys@transformshift{3.920846in}{2.019879in}%
\pgfsys@useobject{currentmarker}{}%
\end{pgfscope}%
\begin{pgfscope}%
\pgfsys@transformshift{3.942211in}{1.981690in}%
\pgfsys@useobject{currentmarker}{}%
\end{pgfscope}%
\begin{pgfscope}%
\pgfsys@transformshift{3.960992in}{1.992492in}%
\pgfsys@useobject{currentmarker}{}%
\end{pgfscope}%
\begin{pgfscope}%
\pgfsys@transformshift{3.980243in}{2.039300in}%
\pgfsys@useobject{currentmarker}{}%
\end{pgfscope}%
\begin{pgfscope}%
\pgfsys@transformshift{3.999260in}{2.104483in}%
\pgfsys@useobject{currentmarker}{}%
\end{pgfscope}%
\begin{pgfscope}%
\pgfsys@transformshift{4.021328in}{2.252454in}%
\pgfsys@useobject{currentmarker}{}%
\end{pgfscope}%
\begin{pgfscope}%
\pgfsys@transformshift{4.038233in}{2.475837in}%
\pgfsys@useobject{currentmarker}{}%
\end{pgfscope}%
\begin{pgfscope}%
\pgfsys@transformshift{4.060302in}{2.712364in}%
\pgfsys@useobject{currentmarker}{}%
\end{pgfscope}%
\begin{pgfscope}%
\pgfsys@transformshift{4.076970in}{2.899236in}%
\pgfsys@useobject{currentmarker}{}%
\end{pgfscope}%
\begin{pgfscope}%
\pgfsys@transformshift{4.096221in}{2.876547in}%
\pgfsys@useobject{currentmarker}{}%
\end{pgfscope}%
\begin{pgfscope}%
\pgfsys@transformshift{4.117586in}{2.780210in}%
\pgfsys@useobject{currentmarker}{}%
\end{pgfscope}%
\begin{pgfscope}%
\pgfsys@transformshift{4.133786in}{2.533173in}%
\pgfsys@useobject{currentmarker}{}%
\end{pgfscope}%
\begin{pgfscope}%
\pgfsys@transformshift{4.152802in}{2.276947in}%
\pgfsys@useobject{currentmarker}{}%
\end{pgfscope}%
\begin{pgfscope}%
\pgfsys@transformshift{4.172054in}{2.134308in}%
\pgfsys@useobject{currentmarker}{}%
\end{pgfscope}%
\begin{pgfscope}%
\pgfsys@transformshift{4.193651in}{2.033941in}%
\pgfsys@useobject{currentmarker}{}%
\end{pgfscope}%
\begin{pgfscope}%
\pgfsys@transformshift{4.211965in}{1.999878in}%
\pgfsys@useobject{currentmarker}{}%
\end{pgfscope}%
\begin{pgfscope}%
\pgfsys@transformshift{4.230981in}{1.990280in}%
\pgfsys@useobject{currentmarker}{}%
\end{pgfscope}%
\begin{pgfscope}%
\pgfsys@transformshift{4.250702in}{2.026803in}%
\pgfsys@useobject{currentmarker}{}%
\end{pgfscope}%
\begin{pgfscope}%
\pgfsys@transformshift{4.270187in}{2.096043in}%
\pgfsys@useobject{currentmarker}{}%
\end{pgfscope}%
\begin{pgfscope}%
\pgfsys@transformshift{4.288264in}{2.212530in}%
\pgfsys@useobject{currentmarker}{}%
\end{pgfscope}%
\begin{pgfscope}%
\pgfsys@transformshift{4.308926in}{2.347923in}%
\pgfsys@useobject{currentmarker}{}%
\end{pgfscope}%
\begin{pgfscope}%
\pgfsys@transformshift{4.328412in}{2.620392in}%
\pgfsys@useobject{currentmarker}{}%
\end{pgfscope}%
\begin{pgfscope}%
\pgfsys@transformshift{4.346958in}{2.844664in}%
\pgfsys@useobject{currentmarker}{}%
\end{pgfscope}%
\begin{pgfscope}%
\pgfsys@transformshift{4.365974in}{2.941410in}%
\pgfsys@useobject{currentmarker}{}%
\end{pgfscope}%
\begin{pgfscope}%
\pgfsys@transformshift{4.383582in}{2.870255in}%
\pgfsys@useobject{currentmarker}{}%
\end{pgfscope}%
\begin{pgfscope}%
\pgfsys@transformshift{4.402130in}{2.715905in}%
\pgfsys@useobject{currentmarker}{}%
\end{pgfscope}%
\begin{pgfscope}%
\pgfsys@transformshift{4.422085in}{2.421648in}%
\pgfsys@useobject{currentmarker}{}%
\end{pgfscope}%
\begin{pgfscope}%
\pgfsys@transformshift{4.443216in}{2.202903in}%
\pgfsys@useobject{currentmarker}{}%
\end{pgfscope}%
\begin{pgfscope}%
\pgfsys@transformshift{4.461996in}{2.089941in}%
\pgfsys@useobject{currentmarker}{}%
\end{pgfscope}%
\begin{pgfscope}%
\pgfsys@transformshift{4.477492in}{2.030770in}%
\pgfsys@useobject{currentmarker}{}%
\end{pgfscope}%
\begin{pgfscope}%
\pgfsys@transformshift{4.480309in}{2.042624in}%
\pgfsys@useobject{currentmarker}{}%
\end{pgfscope}%
\begin{pgfscope}%
\pgfsys@transformshift{4.474440in}{2.071267in}%
\pgfsys@useobject{currentmarker}{}%
\end{pgfscope}%
\begin{pgfscope}%
\pgfsys@transformshift{4.452840in}{2.290659in}%
\pgfsys@useobject{currentmarker}{}%
\end{pgfscope}%
\begin{pgfscope}%
\pgfsys@transformshift{4.434998in}{2.633812in}%
\pgfsys@useobject{currentmarker}{}%
\end{pgfscope}%
\begin{pgfscope}%
\pgfsys@transformshift{4.415981in}{2.914969in}%
\pgfsys@useobject{currentmarker}{}%
\end{pgfscope}%
\begin{pgfscope}%
\pgfsys@transformshift{4.396730in}{2.892379in}%
\pgfsys@useobject{currentmarker}{}%
\end{pgfscope}%
\begin{pgfscope}%
\pgfsys@transformshift{4.377948in}{2.548285in}%
\pgfsys@useobject{currentmarker}{}%
\end{pgfscope}%
\begin{pgfscope}%
\pgfsys@transformshift{4.360105in}{2.224251in}%
\pgfsys@useobject{currentmarker}{}%
\end{pgfscope}%
\begin{pgfscope}%
\pgfsys@transformshift{4.340150in}{2.021225in}%
\pgfsys@useobject{currentmarker}{}%
\end{pgfscope}%
\begin{pgfscope}%
\pgfsys@transformshift{4.318785in}{1.990655in}%
\pgfsys@useobject{currentmarker}{}%
\end{pgfscope}%
\begin{pgfscope}%
\pgfsys@transformshift{4.302351in}{2.052251in}%
\pgfsys@useobject{currentmarker}{}%
\end{pgfscope}%
\begin{pgfscope}%
\pgfsys@transformshift{4.282160in}{2.208195in}%
\pgfsys@useobject{currentmarker}{}%
\end{pgfscope}%
\begin{pgfscope}%
\pgfsys@transformshift{4.263849in}{2.529623in}%
\pgfsys@useobject{currentmarker}{}%
\end{pgfscope}%
\begin{pgfscope}%
\pgfsys@transformshift{4.241311in}{2.887798in}%
\pgfsys@useobject{currentmarker}{}%
\end{pgfscope}%
\begin{pgfscope}%
\pgfsys@transformshift{4.223233in}{2.870632in}%
\pgfsys@useobject{currentmarker}{}%
\end{pgfscope}%
\begin{pgfscope}%
\pgfsys@transformshift{4.205390in}{2.537265in}%
\pgfsys@useobject{currentmarker}{}%
\end{pgfscope}%
\begin{pgfscope}%
\pgfsys@transformshift{4.185670in}{2.205556in}%
\pgfsys@useobject{currentmarker}{}%
\end{pgfscope}%
\begin{pgfscope}%
\pgfsys@transformshift{4.166888in}{2.057690in}%
\pgfsys@useobject{currentmarker}{}%
\end{pgfscope}%
\begin{pgfscope}%
\pgfsys@transformshift{4.148342in}{1.988938in}%
\pgfsys@useobject{currentmarker}{}%
\end{pgfscope}%
\begin{pgfscope}%
\pgfsys@transformshift{4.126976in}{2.007169in}%
\pgfsys@useobject{currentmarker}{}%
\end{pgfscope}%
\begin{pgfscope}%
\pgfsys@transformshift{4.108194in}{2.101037in}%
\pgfsys@useobject{currentmarker}{}%
\end{pgfscope}%
\begin{pgfscope}%
\pgfsys@transformshift{4.089648in}{2.325874in}%
\pgfsys@useobject{currentmarker}{}%
\end{pgfscope}%
\begin{pgfscope}%
\pgfsys@transformshift{4.071100in}{2.686609in}%
\pgfsys@useobject{currentmarker}{}%
\end{pgfscope}%
\begin{pgfscope}%
\pgfsys@transformshift{4.049266in}{2.891176in}%
\pgfsys@useobject{currentmarker}{}%
\end{pgfscope}%
\begin{pgfscope}%
\pgfsys@transformshift{4.029781in}{2.728657in}%
\pgfsys@useobject{currentmarker}{}%
\end{pgfscope}%
\begin{pgfscope}%
\pgfsys@transformshift{4.011235in}{2.338341in}%
\pgfsys@useobject{currentmarker}{}%
\end{pgfscope}%
\begin{pgfscope}%
\pgfsys@transformshift{3.992687in}{2.109681in}%
\pgfsys@useobject{currentmarker}{}%
\end{pgfscope}%
\begin{pgfscope}%
\pgfsys@transformshift{3.974610in}{2.013108in}%
\pgfsys@useobject{currentmarker}{}%
\end{pgfscope}%
\begin{pgfscope}%
\pgfsys@transformshift{3.956062in}{1.977188in}%
\pgfsys@useobject{currentmarker}{}%
\end{pgfscope}%
\begin{pgfscope}%
\pgfsys@transformshift{3.937280in}{2.001582in}%
\pgfsys@useobject{currentmarker}{}%
\end{pgfscope}%
\begin{pgfscope}%
\pgfsys@transformshift{3.914977in}{2.109149in}%
\pgfsys@useobject{currentmarker}{}%
\end{pgfscope}%
\begin{pgfscope}%
\pgfsys@transformshift{3.896665in}{2.348943in}%
\pgfsys@useobject{currentmarker}{}%
\end{pgfscope}%
\begin{pgfscope}%
\pgfsys@transformshift{3.878118in}{2.696893in}%
\pgfsys@useobject{currentmarker}{}%
\end{pgfscope}%
\begin{pgfscope}%
\pgfsys@transformshift{3.859335in}{2.869684in}%
\pgfsys@useobject{currentmarker}{}%
\end{pgfscope}%
\begin{pgfscope}%
\pgfsys@transformshift{3.839850in}{2.756027in}%
\pgfsys@useobject{currentmarker}{}%
\end{pgfscope}%
\begin{pgfscope}%
\pgfsys@transformshift{3.822007in}{2.389680in}%
\pgfsys@useobject{currentmarker}{}%
\end{pgfscope}%
\begin{pgfscope}%
\pgfsys@transformshift{3.800407in}{2.110911in}%
\pgfsys@useobject{currentmarker}{}%
\end{pgfscope}%
\begin{pgfscope}%
\pgfsys@transformshift{3.781861in}{2.018352in}%
\pgfsys@useobject{currentmarker}{}%
\end{pgfscope}%
\begin{pgfscope}%
\pgfsys@transformshift{3.764019in}{1.977042in}%
\pgfsys@useobject{currentmarker}{}%
\end{pgfscope}%
\begin{pgfscope}%
\pgfsys@transformshift{3.742185in}{2.019443in}%
\pgfsys@useobject{currentmarker}{}%
\end{pgfscope}%
\begin{pgfscope}%
\pgfsys@transformshift{3.725517in}{2.068470in}%
\pgfsys@useobject{currentmarker}{}%
\end{pgfscope}%
\begin{pgfscope}%
\pgfsys@transformshift{3.701568in}{2.318860in}%
\pgfsys@useobject{currentmarker}{}%
\end{pgfscope}%
\begin{pgfscope}%
\pgfsys@transformshift{3.685135in}{2.624219in}%
\pgfsys@useobject{currentmarker}{}%
\end{pgfscope}%
\begin{pgfscope}%
\pgfsys@transformshift{3.685840in}{2.783301in}%
\pgfsys@useobject{currentmarker}{}%
\end{pgfscope}%
\begin{pgfscope}%
\pgfsys@transformshift{3.667527in}{2.840490in}%
\pgfsys@useobject{currentmarker}{}%
\end{pgfscope}%
\begin{pgfscope}%
\pgfsys@transformshift{3.650624in}{2.816934in}%
\pgfsys@useobject{currentmarker}{}%
\end{pgfscope}%
\begin{pgfscope}%
\pgfsys@transformshift{3.627381in}{2.461712in}%
\pgfsys@useobject{currentmarker}{}%
\end{pgfscope}%
\begin{pgfscope}%
\pgfsys@transformshift{3.607895in}{2.165784in}%
\pgfsys@useobject{currentmarker}{}%
\end{pgfscope}%
\begin{pgfscope}%
\pgfsys@transformshift{3.591461in}{2.093413in}%
\pgfsys@useobject{currentmarker}{}%
\end{pgfscope}%
\begin{pgfscope}%
\pgfsys@transformshift{3.572914in}{2.023371in}%
\pgfsys@useobject{currentmarker}{}%
\end{pgfscope}%
\begin{pgfscope}%
\pgfsys@transformshift{3.551550in}{1.973716in}%
\pgfsys@useobject{currentmarker}{}%
\end{pgfscope}%
\begin{pgfscope}%
\pgfsys@transformshift{3.531359in}{1.993212in}%
\pgfsys@useobject{currentmarker}{}%
\end{pgfscope}%
\begin{pgfscope}%
\pgfsys@transformshift{3.510699in}{2.073680in}%
\pgfsys@useobject{currentmarker}{}%
\end{pgfscope}%
\begin{pgfscope}%
\pgfsys@transformshift{3.494500in}{2.212869in}%
\pgfsys@useobject{currentmarker}{}%
\end{pgfscope}%
\begin{pgfscope}%
\pgfsys@transformshift{3.475952in}{2.471301in}%
\pgfsys@useobject{currentmarker}{}%
\end{pgfscope}%
\begin{pgfscope}%
\pgfsys@transformshift{3.456232in}{2.201403in}%
\pgfsys@useobject{currentmarker}{}%
\end{pgfscope}%
\begin{pgfscope}%
\pgfsys@transformshift{3.434632in}{2.581870in}%
\pgfsys@useobject{currentmarker}{}%
\end{pgfscope}%
\begin{pgfscope}%
\pgfsys@transformshift{3.418433in}{2.771164in}%
\pgfsys@useobject{currentmarker}{}%
\end{pgfscope}%
\begin{pgfscope}%
\pgfsys@transformshift{3.398479in}{2.808270in}%
\pgfsys@useobject{currentmarker}{}%
\end{pgfscope}%
\begin{pgfscope}%
\pgfsys@transformshift{3.375939in}{2.459936in}%
\pgfsys@useobject{currentmarker}{}%
\end{pgfscope}%
\begin{pgfscope}%
\pgfsys@transformshift{3.362323in}{2.199460in}%
\pgfsys@useobject{currentmarker}{}%
\end{pgfscope}%
\begin{pgfscope}%
\pgfsys@transformshift{3.340488in}{2.074257in}%
\pgfsys@useobject{currentmarker}{}%
\end{pgfscope}%
\begin{pgfscope}%
\pgfsys@transformshift{3.319125in}{1.990104in}%
\pgfsys@useobject{currentmarker}{}%
\end{pgfscope}%
\begin{pgfscope}%
\pgfsys@transformshift{3.300343in}{1.972943in}%
\pgfsys@useobject{currentmarker}{}%
\end{pgfscope}%
\begin{pgfscope}%
\pgfsys@transformshift{3.281326in}{2.011341in}%
\pgfsys@useobject{currentmarker}{}%
\end{pgfscope}%
\begin{pgfscope}%
\pgfsys@transformshift{3.262780in}{2.091303in}%
\pgfsys@useobject{currentmarker}{}%
\end{pgfscope}%
\begin{pgfscope}%
\pgfsys@transformshift{3.245172in}{2.192216in}%
\pgfsys@useobject{currentmarker}{}%
\end{pgfscope}%
\begin{pgfscope}%
\pgfsys@transformshift{3.225685in}{2.471023in}%
\pgfsys@useobject{currentmarker}{}%
\end{pgfscope}%
\begin{pgfscope}%
\pgfsys@transformshift{3.204556in}{2.803443in}%
\pgfsys@useobject{currentmarker}{}%
\end{pgfscope}%
\begin{pgfscope}%
\pgfsys@transformshift{3.185539in}{2.798553in}%
\pgfsys@useobject{currentmarker}{}%
\end{pgfscope}%
\begin{pgfscope}%
\pgfsys@transformshift{3.166993in}{2.489614in}%
\pgfsys@useobject{currentmarker}{}%
\end{pgfscope}%
\begin{pgfscope}%
\pgfsys@transformshift{3.147506in}{2.260048in}%
\pgfsys@useobject{currentmarker}{}%
\end{pgfscope}%
\begin{pgfscope}%
\pgfsys@transformshift{3.127551in}{2.091843in}%
\pgfsys@useobject{currentmarker}{}%
\end{pgfscope}%
\begin{pgfscope}%
\pgfsys@transformshift{3.108769in}{2.006380in}%
\pgfsys@useobject{currentmarker}{}%
\end{pgfscope}%
\begin{pgfscope}%
\pgfsys@transformshift{3.090457in}{1.971659in}%
\pgfsys@useobject{currentmarker}{}%
\end{pgfscope}%
\begin{pgfscope}%
\pgfsys@transformshift{3.071909in}{1.988103in}%
\pgfsys@useobject{currentmarker}{}%
\end{pgfscope}%
\begin{pgfscope}%
\pgfsys@transformshift{3.053127in}{2.048394in}%
\pgfsys@useobject{currentmarker}{}%
\end{pgfscope}%
\begin{pgfscope}%
\pgfsys@transformshift{3.031529in}{2.214586in}%
\pgfsys@useobject{currentmarker}{}%
\end{pgfscope}%
\begin{pgfscope}%
\pgfsys@transformshift{3.014390in}{2.455775in}%
\pgfsys@useobject{currentmarker}{}%
\end{pgfscope}%
\begin{pgfscope}%
\pgfsys@transformshift{2.995139in}{2.773845in}%
\pgfsys@useobject{currentmarker}{}%
\end{pgfscope}%
\begin{pgfscope}%
\pgfsys@transformshift{2.974714in}{2.818246in}%
\pgfsys@useobject{currentmarker}{}%
\end{pgfscope}%
\begin{pgfscope}%
\pgfsys@transformshift{2.956402in}{2.662513in}%
\pgfsys@useobject{currentmarker}{}%
\end{pgfscope}%
\begin{pgfscope}%
\pgfsys@transformshift{2.938560in}{2.422255in}%
\pgfsys@useobject{currentmarker}{}%
\end{pgfscope}%
\begin{pgfscope}%
\pgfsys@transformshift{2.917429in}{2.150800in}%
\pgfsys@useobject{currentmarker}{}%
\end{pgfscope}%
\begin{pgfscope}%
\pgfsys@transformshift{2.898647in}{2.037750in}%
\pgfsys@useobject{currentmarker}{}%
\end{pgfscope}%
\begin{pgfscope}%
\pgfsys@transformshift{2.880101in}{1.983523in}%
\pgfsys@useobject{currentmarker}{}%
\end{pgfscope}%
\begin{pgfscope}%
\pgfsys@transformshift{2.860379in}{1.974343in}%
\pgfsys@useobject{currentmarker}{}%
\end{pgfscope}%
\begin{pgfscope}%
\pgfsys@transformshift{2.836198in}{2.007459in}%
\pgfsys@useobject{currentmarker}{}%
\end{pgfscope}%
\begin{pgfscope}%
\pgfsys@transformshift{2.820938in}{2.065668in}%
\pgfsys@useobject{currentmarker}{}%
\end{pgfscope}%
\begin{pgfscope}%
\pgfsys@transformshift{2.802156in}{2.199082in}%
\pgfsys@useobject{currentmarker}{}%
\end{pgfscope}%
\begin{pgfscope}%
\pgfsys@transformshift{2.783374in}{2.462665in}%
\pgfsys@useobject{currentmarker}{}%
\end{pgfscope}%
\begin{pgfscope}%
\pgfsys@transformshift{2.761776in}{2.784513in}%
\pgfsys@useobject{currentmarker}{}%
\end{pgfscope}%
\begin{pgfscope}%
\pgfsys@transformshift{2.742994in}{2.817621in}%
\pgfsys@useobject{currentmarker}{}%
\end{pgfscope}%
\begin{pgfscope}%
\pgfsys@transformshift{2.727969in}{2.676377in}%
\pgfsys@useobject{currentmarker}{}%
\end{pgfscope}%
\begin{pgfscope}%
\pgfsys@transformshift{2.705429in}{2.434407in}%
\pgfsys@useobject{currentmarker}{}%
\end{pgfscope}%
\begin{pgfscope}%
\pgfsys@transformshift{2.687821in}{2.251211in}%
\pgfsys@useobject{currentmarker}{}%
\end{pgfscope}%
\begin{pgfscope}%
\pgfsys@transformshift{2.668804in}{2.080592in}%
\pgfsys@useobject{currentmarker}{}%
\end{pgfscope}%
\begin{pgfscope}%
\pgfsys@transformshift{2.648144in}{1.998766in}%
\pgfsys@useobject{currentmarker}{}%
\end{pgfscope}%
\begin{pgfscope}%
\pgfsys@transformshift{2.628190in}{1.973613in}%
\pgfsys@useobject{currentmarker}{}%
\end{pgfscope}%
\begin{pgfscope}%
\pgfsys@transformshift{2.609876in}{1.971924in}%
\pgfsys@useobject{currentmarker}{}%
\end{pgfscope}%
\begin{pgfscope}%
\pgfsys@transformshift{2.592739in}{2.001352in}%
\pgfsys@useobject{currentmarker}{}%
\end{pgfscope}%
\begin{pgfscope}%
\pgfsys@transformshift{2.573254in}{2.084051in}%
\pgfsys@useobject{currentmarker}{}%
\end{pgfscope}%
\begin{pgfscope}%
\pgfsys@transformshift{2.552123in}{2.230465in}%
\pgfsys@useobject{currentmarker}{}%
\end{pgfscope}%
\begin{pgfscope}%
\pgfsys@transformshift{2.532872in}{2.544462in}%
\pgfsys@useobject{currentmarker}{}%
\end{pgfscope}%
\begin{pgfscope}%
\pgfsys@transformshift{2.515734in}{2.785165in}%
\pgfsys@useobject{currentmarker}{}%
\end{pgfscope}%
\begin{pgfscope}%
\pgfsys@transformshift{2.496718in}{2.790655in}%
\pgfsys@useobject{currentmarker}{}%
\end{pgfscope}%
\begin{pgfscope}%
\pgfsys@transformshift{2.477230in}{2.530683in}%
\pgfsys@useobject{currentmarker}{}%
\end{pgfscope}%
\begin{pgfscope}%
\pgfsys@transformshift{2.455632in}{2.242952in}%
\pgfsys@useobject{currentmarker}{}%
\end{pgfscope}%
\begin{pgfscope}%
\pgfsys@transformshift{2.437319in}{2.090605in}%
\pgfsys@useobject{currentmarker}{}%
\end{pgfscope}%
\begin{pgfscope}%
\pgfsys@transformshift{2.416425in}{2.007031in}%
\pgfsys@useobject{currentmarker}{}%
\end{pgfscope}%
\begin{pgfscope}%
\pgfsys@transformshift{2.397879in}{1.973613in}%
\pgfsys@useobject{currentmarker}{}%
\end{pgfscope}%
\begin{pgfscope}%
\pgfsys@transformshift{2.380974in}{1.976538in}%
\pgfsys@useobject{currentmarker}{}%
\end{pgfscope}%
\begin{pgfscope}%
\pgfsys@transformshift{2.360080in}{2.024745in}%
\pgfsys@useobject{currentmarker}{}%
\end{pgfscope}%
\begin{pgfscope}%
\pgfsys@transformshift{2.337071in}{2.129067in}%
\pgfsys@useobject{currentmarker}{}%
\end{pgfscope}%
\begin{pgfscope}%
\pgfsys@transformshift{2.322986in}{2.019800in}%
\pgfsys@useobject{currentmarker}{}%
\end{pgfscope}%
\begin{pgfscope}%
\pgfsys@transformshift{2.301855in}{2.104767in}%
\pgfsys@useobject{currentmarker}{}%
\end{pgfscope}%
\begin{pgfscope}%
\pgfsys@transformshift{2.280492in}{2.375902in}%
\pgfsys@useobject{currentmarker}{}%
\end{pgfscope}%
\begin{pgfscope}%
\pgfsys@transformshift{2.264527in}{2.659774in}%
\pgfsys@useobject{currentmarker}{}%
\end{pgfscope}%
\begin{pgfscope}%
\pgfsys@transformshift{2.244807in}{2.821951in}%
\pgfsys@useobject{currentmarker}{}%
\end{pgfscope}%
\begin{pgfscope}%
\pgfsys@transformshift{2.228842in}{2.739134in}%
\pgfsys@useobject{currentmarker}{}%
\end{pgfscope}%
\begin{pgfscope}%
\pgfsys@transformshift{2.205365in}{2.415748in}%
\pgfsys@useobject{currentmarker}{}%
\end{pgfscope}%
\begin{pgfscope}%
\pgfsys@transformshift{2.186817in}{2.175542in}%
\pgfsys@useobject{currentmarker}{}%
\end{pgfscope}%
\begin{pgfscope}%
\pgfsys@transformshift{2.168505in}{2.111765in}%
\pgfsys@useobject{currentmarker}{}%
\end{pgfscope}%
\begin{pgfscope}%
\pgfsys@transformshift{2.147611in}{2.013300in}%
\pgfsys@useobject{currentmarker}{}%
\end{pgfscope}%
\begin{pgfscope}%
\pgfsys@transformshift{2.129769in}{1.975374in}%
\pgfsys@useobject{currentmarker}{}%
\end{pgfscope}%
\begin{pgfscope}%
\pgfsys@transformshift{2.111924in}{1.971839in}%
\pgfsys@useobject{currentmarker}{}%
\end{pgfscope}%
\begin{pgfscope}%
\pgfsys@transformshift{2.088683in}{2.007605in}%
\pgfsys@useobject{currentmarker}{}%
\end{pgfscope}%
\begin{pgfscope}%
\pgfsys@transformshift{2.074361in}{2.063641in}%
\pgfsys@useobject{currentmarker}{}%
\end{pgfscope}%
\begin{pgfscope}%
\pgfsys@transformshift{2.052762in}{2.225545in}%
\pgfsys@useobject{currentmarker}{}%
\end{pgfscope}%
\begin{pgfscope}%
\pgfsys@transformshift{2.033745in}{2.539610in}%
\pgfsys@useobject{currentmarker}{}%
\end{pgfscope}%
\begin{pgfscope}%
\pgfsys@transformshift{2.012147in}{2.784251in}%
\pgfsys@useobject{currentmarker}{}%
\end{pgfscope}%
\begin{pgfscope}%
\pgfsys@transformshift{1.996182in}{2.827341in}%
\pgfsys@useobject{currentmarker}{}%
\end{pgfscope}%
\begin{pgfscope}%
\pgfsys@transformshift{1.975288in}{2.671198in}%
\pgfsys@useobject{currentmarker}{}%
\end{pgfscope}%
\begin{pgfscope}%
\pgfsys@transformshift{1.955802in}{2.352245in}%
\pgfsys@useobject{currentmarker}{}%
\end{pgfscope}%
\begin{pgfscope}%
\pgfsys@transformshift{1.938898in}{2.159901in}%
\pgfsys@useobject{currentmarker}{}%
\end{pgfscope}%
\begin{pgfscope}%
\pgfsys@transformshift{1.919178in}{2.047908in}%
\pgfsys@useobject{currentmarker}{}%
\end{pgfscope}%
\begin{pgfscope}%
\pgfsys@transformshift{1.897578in}{1.985915in}%
\pgfsys@useobject{currentmarker}{}%
\end{pgfscope}%
\begin{pgfscope}%
\pgfsys@transformshift{1.875744in}{1.974008in}%
\pgfsys@useobject{currentmarker}{}%
\end{pgfscope}%
\begin{pgfscope}%
\pgfsys@transformshift{1.860015in}{2.014968in}%
\pgfsys@useobject{currentmarker}{}%
\end{pgfscope}%
\begin{pgfscope}%
\pgfsys@transformshift{1.843579in}{2.002286in}%
\pgfsys@useobject{currentmarker}{}%
\end{pgfscope}%
\begin{pgfscope}%
\pgfsys@transformshift{1.821747in}{1.974334in}%
\pgfsys@useobject{currentmarker}{}%
\end{pgfscope}%
\begin{pgfscope}%
\pgfsys@transformshift{1.803199in}{1.993912in}%
\pgfsys@useobject{currentmarker}{}%
\end{pgfscope}%
\begin{pgfscope}%
\pgfsys@transformshift{1.782539in}{2.067261in}%
\pgfsys@useobject{currentmarker}{}%
\end{pgfscope}%
\begin{pgfscope}%
\pgfsys@transformshift{1.764463in}{2.177183in}%
\pgfsys@useobject{currentmarker}{}%
\end{pgfscope}%
\begin{pgfscope}%
\pgfsys@transformshift{1.747323in}{2.405886in}%
\pgfsys@useobject{currentmarker}{}%
\end{pgfscope}%
\begin{pgfscope}%
\pgfsys@transformshift{1.726429in}{2.686533in}%
\pgfsys@useobject{currentmarker}{}%
\end{pgfscope}%
\begin{pgfscope}%
\pgfsys@transformshift{1.703186in}{2.846075in}%
\pgfsys@useobject{currentmarker}{}%
\end{pgfscope}%
\begin{pgfscope}%
\pgfsys@transformshift{1.686518in}{2.761260in}%
\pgfsys@useobject{currentmarker}{}%
\end{pgfscope}%
\begin{pgfscope}%
\pgfsys@transformshift{1.670084in}{2.518079in}%
\pgfsys@useobject{currentmarker}{}%
\end{pgfscope}%
\begin{pgfscope}%
\pgfsys@transformshift{1.650128in}{2.269045in}%
\pgfsys@useobject{currentmarker}{}%
\end{pgfscope}%
\begin{pgfscope}%
\pgfsys@transformshift{1.629937in}{2.090883in}%
\pgfsys@useobject{currentmarker}{}%
\end{pgfscope}%
\begin{pgfscope}%
\pgfsys@transformshift{1.609982in}{2.025035in}%
\pgfsys@useobject{currentmarker}{}%
\end{pgfscope}%
\begin{pgfscope}%
\pgfsys@transformshift{1.592608in}{1.987562in}%
\pgfsys@useobject{currentmarker}{}%
\end{pgfscope}%
\begin{pgfscope}%
\pgfsys@transformshift{1.573592in}{1.976129in}%
\pgfsys@useobject{currentmarker}{}%
\end{pgfscope}%
\begin{pgfscope}%
\pgfsys@transformshift{1.552463in}{2.017462in}%
\pgfsys@useobject{currentmarker}{}%
\end{pgfscope}%
\begin{pgfscope}%
\pgfsys@transformshift{1.534620in}{2.051119in}%
\pgfsys@useobject{currentmarker}{}%
\end{pgfscope}%
\begin{pgfscope}%
\pgfsys@transformshift{1.512317in}{2.211849in}%
\pgfsys@useobject{currentmarker}{}%
\end{pgfscope}%
\begin{pgfscope}%
\pgfsys@transformshift{1.495178in}{2.346129in}%
\pgfsys@useobject{currentmarker}{}%
\end{pgfscope}%
\begin{pgfscope}%
\pgfsys@transformshift{1.474518in}{2.669341in}%
\pgfsys@useobject{currentmarker}{}%
\end{pgfscope}%
\begin{pgfscope}%
\pgfsys@transformshift{1.456910in}{2.837059in}%
\pgfsys@useobject{currentmarker}{}%
\end{pgfscope}%
\begin{pgfscope}%
\pgfsys@transformshift{1.438599in}{2.851926in}%
\pgfsys@useobject{currentmarker}{}%
\end{pgfscope}%
\begin{pgfscope}%
\pgfsys@transformshift{1.420991in}{2.768299in}%
\pgfsys@useobject{currentmarker}{}%
\end{pgfscope}%
\begin{pgfscope}%
\pgfsys@transformshift{1.398686in}{2.606580in}%
\pgfsys@useobject{currentmarker}{}%
\end{pgfscope}%
\begin{pgfscope}%
\pgfsys@transformshift{1.379434in}{2.311192in}%
\pgfsys@useobject{currentmarker}{}%
\end{pgfscope}%
\begin{pgfscope}%
\pgfsys@transformshift{1.363472in}{2.136910in}%
\pgfsys@useobject{currentmarker}{}%
\end{pgfscope}%
\begin{pgfscope}%
\pgfsys@transformshift{1.342812in}{2.045229in}%
\pgfsys@useobject{currentmarker}{}%
\end{pgfscope}%
\begin{pgfscope}%
\pgfsys@transformshift{1.321212in}{1.998554in}%
\pgfsys@useobject{currentmarker}{}%
\end{pgfscope}%
\begin{pgfscope}%
\pgfsys@transformshift{1.303604in}{1.978047in}%
\pgfsys@useobject{currentmarker}{}%
\end{pgfscope}%
\begin{pgfscope}%
\pgfsys@transformshift{1.283647in}{2.004157in}%
\pgfsys@useobject{currentmarker}{}%
\end{pgfscope}%
\begin{pgfscope}%
\pgfsys@transformshift{1.266276in}{2.052704in}%
\pgfsys@useobject{currentmarker}{}%
\end{pgfscope}%
\begin{pgfscope}%
\pgfsys@transformshift{1.245145in}{2.159083in}%
\pgfsys@useobject{currentmarker}{}%
\end{pgfscope}%
\begin{pgfscope}%
\pgfsys@transformshift{1.226833in}{2.300576in}%
\pgfsys@useobject{currentmarker}{}%
\end{pgfscope}%
\begin{pgfscope}%
\pgfsys@transformshift{1.206877in}{2.551216in}%
\pgfsys@useobject{currentmarker}{}%
\end{pgfscope}%
\begin{pgfscope}%
\pgfsys@transformshift{1.188097in}{2.092962in}%
\pgfsys@useobject{currentmarker}{}%
\end{pgfscope}%
\begin{pgfscope}%
\pgfsys@transformshift{1.170252in}{2.013591in}%
\pgfsys@useobject{currentmarker}{}%
\end{pgfscope}%
\begin{pgfscope}%
\pgfsys@transformshift{1.148654in}{1.977565in}%
\pgfsys@useobject{currentmarker}{}%
\end{pgfscope}%
\begin{pgfscope}%
\pgfsys@transformshift{1.129872in}{2.002416in}%
\pgfsys@useobject{currentmarker}{}%
\end{pgfscope}%
\begin{pgfscope}%
\pgfsys@transformshift{1.110152in}{2.099709in}%
\pgfsys@useobject{currentmarker}{}%
\end{pgfscope}%
\begin{pgfscope}%
\pgfsys@transformshift{1.091839in}{2.225554in}%
\pgfsys@useobject{currentmarker}{}%
\end{pgfscope}%
\begin{pgfscope}%
\pgfsys@transformshift{1.071648in}{2.553462in}%
\pgfsys@useobject{currentmarker}{}%
\end{pgfscope}%
\begin{pgfscope}%
\pgfsys@transformshift{1.053571in}{2.820225in}%
\pgfsys@useobject{currentmarker}{}%
\end{pgfscope}%
\begin{pgfscope}%
\pgfsys@transformshift{1.031268in}{2.885302in}%
\pgfsys@useobject{currentmarker}{}%
\end{pgfscope}%
\begin{pgfscope}%
\pgfsys@transformshift{1.016242in}{2.740864in}%
\pgfsys@useobject{currentmarker}{}%
\end{pgfscope}%
\begin{pgfscope}%
\pgfsys@transformshift{0.994174in}{2.387709in}%
\pgfsys@useobject{currentmarker}{}%
\end{pgfscope}%
\begin{pgfscope}%
\pgfsys@transformshift{0.977269in}{2.224866in}%
\pgfsys@useobject{currentmarker}{}%
\end{pgfscope}%
\begin{pgfscope}%
\pgfsys@transformshift{0.955437in}{2.047576in}%
\pgfsys@useobject{currentmarker}{}%
\end{pgfscope}%
\begin{pgfscope}%
\pgfsys@transformshift{0.936889in}{2.001008in}%
\pgfsys@useobject{currentmarker}{}%
\end{pgfscope}%
\begin{pgfscope}%
\pgfsys@transformshift{0.919281in}{1.985302in}%
\pgfsys@useobject{currentmarker}{}%
\end{pgfscope}%
\begin{pgfscope}%
\pgfsys@transformshift{0.899090in}{2.024467in}%
\pgfsys@useobject{currentmarker}{}%
\end{pgfscope}%
\begin{pgfscope}%
\pgfsys@transformshift{0.880544in}{2.106330in}%
\pgfsys@useobject{currentmarker}{}%
\end{pgfscope}%
\begin{pgfscope}%
\pgfsys@transformshift{0.861293in}{2.306842in}%
\pgfsys@useobject{currentmarker}{}%
\end{pgfscope}%
\begin{pgfscope}%
\pgfsys@transformshift{0.842276in}{2.520701in}%
\pgfsys@useobject{currentmarker}{}%
\end{pgfscope}%
\begin{pgfscope}%
\pgfsys@transformshift{0.824199in}{2.803858in}%
\pgfsys@useobject{currentmarker}{}%
\end{pgfscope}%
\begin{pgfscope}%
\pgfsys@transformshift{0.803774in}{2.916361in}%
\pgfsys@useobject{currentmarker}{}%
\end{pgfscope}%
\begin{pgfscope}%
\pgfsys@transformshift{0.779826in}{2.737829in}%
\pgfsys@useobject{currentmarker}{}%
\end{pgfscope}%
\begin{pgfscope}%
\pgfsys@transformshift{0.763392in}{2.486183in}%
\pgfsys@useobject{currentmarker}{}%
\end{pgfscope}%
\begin{pgfscope}%
\pgfsys@transformshift{0.745315in}{2.246694in}%
\pgfsys@useobject{currentmarker}{}%
\end{pgfscope}%
\begin{pgfscope}%
\pgfsys@transformshift{0.728176in}{2.110381in}%
\pgfsys@useobject{currentmarker}{}%
\end{pgfscope}%
\begin{pgfscope}%
\pgfsys@transformshift{0.707516in}{2.024104in}%
\pgfsys@useobject{currentmarker}{}%
\end{pgfscope}%
\begin{pgfscope}%
\pgfsys@transformshift{0.685918in}{1.989775in}%
\pgfsys@useobject{currentmarker}{}%
\end{pgfscope}%
\begin{pgfscope}%
\pgfsys@transformshift{0.667839in}{2.000560in}%
\pgfsys@useobject{currentmarker}{}%
\end{pgfscope}%
\begin{pgfscope}%
\pgfsys@transformshift{0.650936in}{2.026900in}%
\pgfsys@useobject{currentmarker}{}%
\end{pgfscope}%
\begin{pgfscope}%
\pgfsys@transformshift{0.650231in}{2.027898in}%
\pgfsys@useobject{currentmarker}{}%
\end{pgfscope}%
\begin{pgfscope}%
\pgfsys@transformshift{0.656806in}{2.005192in}%
\pgfsys@useobject{currentmarker}{}%
\end{pgfscope}%
\begin{pgfscope}%
\pgfsys@transformshift{0.675117in}{1.996972in}%
\pgfsys@useobject{currentmarker}{}%
\end{pgfscope}%
\begin{pgfscope}%
\pgfsys@transformshift{0.693196in}{2.064038in}%
\pgfsys@useobject{currentmarker}{}%
\end{pgfscope}%
\begin{pgfscope}%
\pgfsys@transformshift{0.712916in}{2.215060in}%
\pgfsys@useobject{currentmarker}{}%
\end{pgfscope}%
\begin{pgfscope}%
\pgfsys@transformshift{0.733342in}{2.569054in}%
\pgfsys@useobject{currentmarker}{}%
\end{pgfscope}%
\begin{pgfscope}%
\pgfsys@transformshift{0.750950in}{2.883569in}%
\pgfsys@useobject{currentmarker}{}%
\end{pgfscope}%
\begin{pgfscope}%
\pgfsys@transformshift{0.771844in}{2.859083in}%
\pgfsys@useobject{currentmarker}{}%
\end{pgfscope}%
\begin{pgfscope}%
\pgfsys@transformshift{0.790626in}{2.543471in}%
\pgfsys@useobject{currentmarker}{}%
\end{pgfscope}%
\begin{pgfscope}%
\pgfsys@transformshift{0.808469in}{2.211846in}%
\pgfsys@useobject{currentmarker}{}%
\end{pgfscope}%
\begin{pgfscope}%
\pgfsys@transformshift{0.831007in}{2.041719in}%
\pgfsys@useobject{currentmarker}{}%
\end{pgfscope}%
\begin{pgfscope}%
\pgfsys@transformshift{0.847440in}{1.987445in}%
\pgfsys@useobject{currentmarker}{}%
\end{pgfscope}%
\begin{pgfscope}%
\pgfsys@transformshift{0.865519in}{2.009386in}%
\pgfsys@useobject{currentmarker}{}%
\end{pgfscope}%
\begin{pgfscope}%
\pgfsys@transformshift{0.885943in}{2.146250in}%
\pgfsys@useobject{currentmarker}{}%
\end{pgfscope}%
\begin{pgfscope}%
\pgfsys@transformshift{0.905430in}{2.327231in}%
\pgfsys@useobject{currentmarker}{}%
\end{pgfscope}%
\begin{pgfscope}%
\pgfsys@transformshift{0.922567in}{2.682411in}%
\pgfsys@useobject{currentmarker}{}%
\end{pgfscope}%
\begin{pgfscope}%
\pgfsys@transformshift{0.944871in}{2.893568in}%
\pgfsys@useobject{currentmarker}{}%
\end{pgfscope}%
\begin{pgfscope}%
\pgfsys@transformshift{0.961306in}{2.740527in}%
\pgfsys@useobject{currentmarker}{}%
\end{pgfscope}%
\begin{pgfscope}%
\pgfsys@transformshift{0.983375in}{2.316287in}%
\pgfsys@useobject{currentmarker}{}%
\end{pgfscope}%
\begin{pgfscope}%
\pgfsys@transformshift{1.001921in}{2.109004in}%
\pgfsys@useobject{currentmarker}{}%
\end{pgfscope}%
\begin{pgfscope}%
\pgfsys@transformshift{1.022112in}{2.000624in}%
\pgfsys@useobject{currentmarker}{}%
\end{pgfscope}%
\begin{pgfscope}%
\pgfsys@transformshift{1.039954in}{1.982302in}%
\pgfsys@useobject{currentmarker}{}%
\end{pgfscope}%
\begin{pgfscope}%
\pgfsys@transformshift{1.057328in}{2.034366in}%
\pgfsys@useobject{currentmarker}{}%
\end{pgfscope}%
\begin{pgfscope}%
\pgfsys@transformshift{1.079162in}{2.177152in}%
\pgfsys@useobject{currentmarker}{}%
\end{pgfscope}%
\begin{pgfscope}%
\pgfsys@transformshift{1.096534in}{2.477726in}%
\pgfsys@useobject{currentmarker}{}%
\end{pgfscope}%
\begin{pgfscope}%
\pgfsys@transformshift{1.114613in}{2.804440in}%
\pgfsys@useobject{currentmarker}{}%
\end{pgfscope}%
\begin{pgfscope}%
\pgfsys@transformshift{1.135273in}{2.839396in}%
\pgfsys@useobject{currentmarker}{}%
\end{pgfscope}%
\begin{pgfscope}%
\pgfsys@transformshift{1.155698in}{2.498706in}%
\pgfsys@useobject{currentmarker}{}%
\end{pgfscope}%
\begin{pgfscope}%
\pgfsys@transformshift{1.174478in}{2.207932in}%
\pgfsys@useobject{currentmarker}{}%
\end{pgfscope}%
\begin{pgfscope}%
\pgfsys@transformshift{1.195138in}{2.036906in}%
\pgfsys@useobject{currentmarker}{}%
\end{pgfscope}%
\begin{pgfscope}%
\pgfsys@transformshift{1.213921in}{1.981532in}%
\pgfsys@useobject{currentmarker}{}%
\end{pgfscope}%
\begin{pgfscope}%
\pgfsys@transformshift{1.231529in}{1.991719in}%
\pgfsys@useobject{currentmarker}{}%
\end{pgfscope}%
\begin{pgfscope}%
\pgfsys@transformshift{1.251954in}{2.063063in}%
\pgfsys@useobject{currentmarker}{}%
\end{pgfscope}%
\begin{pgfscope}%
\pgfsys@transformshift{1.271205in}{2.214274in}%
\pgfsys@useobject{currentmarker}{}%
\end{pgfscope}%
\begin{pgfscope}%
\pgfsys@transformshift{1.287873in}{2.531010in}%
\pgfsys@useobject{currentmarker}{}%
\end{pgfscope}%
\begin{pgfscope}%
\pgfsys@transformshift{1.309239in}{2.825401in}%
\pgfsys@useobject{currentmarker}{}%
\end{pgfscope}%
\begin{pgfscope}%
\pgfsys@transformshift{1.331542in}{2.780585in}%
\pgfsys@useobject{currentmarker}{}%
\end{pgfscope}%
\begin{pgfscope}%
\pgfsys@transformshift{1.348210in}{2.485380in}%
\pgfsys@useobject{currentmarker}{}%
\end{pgfscope}%
\begin{pgfscope}%
\pgfsys@transformshift{1.365818in}{2.215638in}%
\pgfsys@useobject{currentmarker}{}%
\end{pgfscope}%
\begin{pgfscope}%
\pgfsys@transformshift{1.386949in}{2.045814in}%
\pgfsys@useobject{currentmarker}{}%
\end{pgfscope}%
\begin{pgfscope}%
\pgfsys@transformshift{1.405729in}{1.987301in}%
\pgfsys@useobject{currentmarker}{}%
\end{pgfscope}%
\begin{pgfscope}%
\pgfsys@transformshift{1.425920in}{1.979948in}%
\pgfsys@useobject{currentmarker}{}%
\end{pgfscope}%
\begin{pgfscope}%
\pgfsys@transformshift{1.443528in}{2.028412in}%
\pgfsys@useobject{currentmarker}{}%
\end{pgfscope}%
\begin{pgfscope}%
\pgfsys@transformshift{1.462311in}{2.130751in}%
\pgfsys@useobject{currentmarker}{}%
\end{pgfscope}%
\begin{pgfscope}%
\pgfsys@transformshift{1.482031in}{2.408009in}%
\pgfsys@useobject{currentmarker}{}%
\end{pgfscope}%
\begin{pgfscope}%
\pgfsys@transformshift{1.500342in}{2.737310in}%
\pgfsys@useobject{currentmarker}{}%
\end{pgfscope}%
\begin{pgfscope}%
\pgfsys@transformshift{1.521707in}{2.843776in}%
\pgfsys@useobject{currentmarker}{}%
\end{pgfscope}%
\begin{pgfscope}%
\pgfsys@transformshift{1.539784in}{2.752252in}%
\pgfsys@useobject{currentmarker}{}%
\end{pgfscope}%
\begin{pgfscope}%
\pgfsys@transformshift{1.556689in}{2.447606in}%
\pgfsys@useobject{currentmarker}{}%
\end{pgfscope}%
\begin{pgfscope}%
\pgfsys@transformshift{1.579461in}{2.156154in}%
\pgfsys@useobject{currentmarker}{}%
\end{pgfscope}%
\begin{pgfscope}%
\pgfsys@transformshift{1.596600in}{2.036226in}%
\pgfsys@useobject{currentmarker}{}%
\end{pgfscope}%
\begin{pgfscope}%
\pgfsys@transformshift{1.617260in}{1.990837in}%
\pgfsys@useobject{currentmarker}{}%
\end{pgfscope}%
\begin{pgfscope}%
\pgfsys@transformshift{1.635337in}{1.973166in}%
\pgfsys@useobject{currentmarker}{}%
\end{pgfscope}%
\begin{pgfscope}%
\pgfsys@transformshift{1.658814in}{2.025734in}%
\pgfsys@useobject{currentmarker}{}%
\end{pgfscope}%
\begin{pgfscope}%
\pgfsys@transformshift{1.673605in}{2.095639in}%
\pgfsys@useobject{currentmarker}{}%
\end{pgfscope}%
\begin{pgfscope}%
\pgfsys@transformshift{1.693796in}{2.284436in}%
\pgfsys@useobject{currentmarker}{}%
\end{pgfscope}%
\begin{pgfscope}%
\pgfsys@transformshift{1.712342in}{2.597726in}%
\pgfsys@useobject{currentmarker}{}%
\end{pgfscope}%
\begin{pgfscope}%
\pgfsys@transformshift{1.732064in}{2.815189in}%
\pgfsys@useobject{currentmarker}{}%
\end{pgfscope}%
\begin{pgfscope}%
\pgfsys@transformshift{1.751081in}{2.773358in}%
\pgfsys@useobject{currentmarker}{}%
\end{pgfscope}%
\begin{pgfscope}%
\pgfsys@transformshift{1.769627in}{2.528585in}%
\pgfsys@useobject{currentmarker}{}%
\end{pgfscope}%
\begin{pgfscope}%
\pgfsys@transformshift{1.791695in}{2.284255in}%
\pgfsys@useobject{currentmarker}{}%
\end{pgfscope}%
\begin{pgfscope}%
\pgfsys@transformshift{1.809303in}{2.119837in}%
\pgfsys@useobject{currentmarker}{}%
\end{pgfscope}%
\begin{pgfscope}%
\pgfsys@transformshift{1.827146in}{2.039505in}%
\pgfsys@useobject{currentmarker}{}%
\end{pgfscope}%
\begin{pgfscope}%
\pgfsys@transformshift{1.849214in}{1.980828in}%
\pgfsys@useobject{currentmarker}{}%
\end{pgfscope}%
\begin{pgfscope}%
\pgfsys@transformshift{1.867528in}{1.974485in}%
\pgfsys@useobject{currentmarker}{}%
\end{pgfscope}%
\begin{pgfscope}%
\pgfsys@transformshift{1.887717in}{2.005081in}%
\pgfsys@useobject{currentmarker}{}%
\end{pgfscope}%
\begin{pgfscope}%
\pgfsys@transformshift{1.903213in}{2.068911in}%
\pgfsys@useobject{currentmarker}{}%
\end{pgfscope}%
\begin{pgfscope}%
\pgfsys@transformshift{1.923404in}{2.211346in}%
\pgfsys@useobject{currentmarker}{}%
\end{pgfscope}%
\begin{pgfscope}%
\pgfsys@transformshift{1.944298in}{2.486484in}%
\pgfsys@useobject{currentmarker}{}%
\end{pgfscope}%
\begin{pgfscope}%
\pgfsys@transformshift{1.965427in}{2.052486in}%
\pgfsys@useobject{currentmarker}{}%
\end{pgfscope}%
\begin{pgfscope}%
\pgfsys@transformshift{1.981626in}{2.182353in}%
\pgfsys@useobject{currentmarker}{}%
\end{pgfscope}%
\begin{pgfscope}%
\pgfsys@transformshift{2.001817in}{2.507287in}%
\pgfsys@useobject{currentmarker}{}%
\end{pgfscope}%
\begin{pgfscope}%
\pgfsys@transformshift{2.019894in}{2.768600in}%
\pgfsys@useobject{currentmarker}{}%
\end{pgfscope}%
\begin{pgfscope}%
\pgfsys@transformshift{2.036797in}{2.818290in}%
\pgfsys@useobject{currentmarker}{}%
\end{pgfscope}%
\begin{pgfscope}%
\pgfsys@transformshift{2.058631in}{2.557296in}%
\pgfsys@useobject{currentmarker}{}%
\end{pgfscope}%
\begin{pgfscope}%
\pgfsys@transformshift{2.076474in}{2.264013in}%
\pgfsys@useobject{currentmarker}{}%
\end{pgfscope}%
\begin{pgfscope}%
\pgfsys@transformshift{2.096899in}{2.063133in}%
\pgfsys@useobject{currentmarker}{}%
\end{pgfscope}%
\begin{pgfscope}%
\pgfsys@transformshift{2.115212in}{1.998297in}%
\pgfsys@useobject{currentmarker}{}%
\end{pgfscope}%
\begin{pgfscope}%
\pgfsys@transformshift{2.137047in}{1.970370in}%
\pgfsys@useobject{currentmarker}{}%
\end{pgfscope}%
\begin{pgfscope}%
\pgfsys@transformshift{2.153949in}{1.992676in}%
\pgfsys@useobject{currentmarker}{}%
\end{pgfscope}%
\begin{pgfscope}%
\pgfsys@transformshift{2.171557in}{2.055789in}%
\pgfsys@useobject{currentmarker}{}%
\end{pgfscope}%
\begin{pgfscope}%
\pgfsys@transformshift{2.193157in}{2.200735in}%
\pgfsys@useobject{currentmarker}{}%
\end{pgfscope}%
\begin{pgfscope}%
\pgfsys@transformshift{2.210529in}{2.455565in}%
\pgfsys@useobject{currentmarker}{}%
\end{pgfscope}%
\begin{pgfscope}%
\pgfsys@transformshift{2.229076in}{2.733373in}%
\pgfsys@useobject{currentmarker}{}%
\end{pgfscope}%
\begin{pgfscope}%
\pgfsys@transformshift{2.250911in}{2.814332in}%
\pgfsys@useobject{currentmarker}{}%
\end{pgfscope}%
\begin{pgfscope}%
\pgfsys@transformshift{2.271805in}{2.588768in}%
\pgfsys@useobject{currentmarker}{}%
\end{pgfscope}%
\begin{pgfscope}%
\pgfsys@transformshift{2.290587in}{2.266309in}%
\pgfsys@useobject{currentmarker}{}%
\end{pgfscope}%
\begin{pgfscope}%
\pgfsys@transformshift{2.307021in}{2.096069in}%
\pgfsys@useobject{currentmarker}{}%
\end{pgfscope}%
\begin{pgfscope}%
\pgfsys@transformshift{2.328384in}{2.007479in}%
\pgfsys@useobject{currentmarker}{}%
\end{pgfscope}%
\begin{pgfscope}%
\pgfsys@transformshift{2.346698in}{1.974718in}%
\pgfsys@useobject{currentmarker}{}%
\end{pgfscope}%
\begin{pgfscope}%
\pgfsys@transformshift{2.364306in}{1.976489in}%
\pgfsys@useobject{currentmarker}{}%
\end{pgfscope}%
\begin{pgfscope}%
\pgfsys@transformshift{2.386140in}{2.026095in}%
\pgfsys@useobject{currentmarker}{}%
\end{pgfscope}%
\begin{pgfscope}%
\pgfsys@transformshift{2.405391in}{2.079582in}%
\pgfsys@useobject{currentmarker}{}%
\end{pgfscope}%
\begin{pgfscope}%
\pgfsys@transformshift{2.423234in}{2.054730in}%
\pgfsys@useobject{currentmarker}{}%
\end{pgfscope}%
\begin{pgfscope}%
\pgfsys@transformshift{2.443659in}{2.157737in}%
\pgfsys@useobject{currentmarker}{}%
\end{pgfscope}%
\begin{pgfscope}%
\pgfsys@transformshift{2.462440in}{2.347383in}%
\pgfsys@useobject{currentmarker}{}%
\end{pgfscope}%
\begin{pgfscope}%
\pgfsys@transformshift{2.481691in}{2.693493in}%
\pgfsys@useobject{currentmarker}{}%
\end{pgfscope}%
\begin{pgfscope}%
\pgfsys@transformshift{2.502822in}{2.818396in}%
\pgfsys@useobject{currentmarker}{}%
\end{pgfscope}%
\begin{pgfscope}%
\pgfsys@transformshift{2.521133in}{2.664971in}%
\pgfsys@useobject{currentmarker}{}%
\end{pgfscope}%
\begin{pgfscope}%
\pgfsys@transformshift{2.538741in}{2.344366in}%
\pgfsys@useobject{currentmarker}{}%
\end{pgfscope}%
\begin{pgfscope}%
\pgfsys@transformshift{2.557054in}{2.134287in}%
\pgfsys@useobject{currentmarker}{}%
\end{pgfscope}%
\begin{pgfscope}%
\pgfsys@transformshift{2.577714in}{2.022581in}%
\pgfsys@useobject{currentmarker}{}%
\end{pgfscope}%
\begin{pgfscope}%
\pgfsys@transformshift{2.596260in}{1.977630in}%
\pgfsys@useobject{currentmarker}{}%
\end{pgfscope}%
\begin{pgfscope}%
\pgfsys@transformshift{2.615982in}{1.976019in}%
\pgfsys@useobject{currentmarker}{}%
\end{pgfscope}%
\begin{pgfscope}%
\pgfsys@transformshift{2.634997in}{2.020979in}%
\pgfsys@useobject{currentmarker}{}%
\end{pgfscope}%
\begin{pgfscope}%
\pgfsys@transformshift{2.653545in}{2.097618in}%
\pgfsys@useobject{currentmarker}{}%
\end{pgfscope}%
\begin{pgfscope}%
\pgfsys@transformshift{2.674205in}{2.271984in}%
\pgfsys@useobject{currentmarker}{}%
\end{pgfscope}%
\begin{pgfscope}%
\pgfsys@transformshift{2.695099in}{2.572380in}%
\pgfsys@useobject{currentmarker}{}%
\end{pgfscope}%
\begin{pgfscope}%
\pgfsys@transformshift{2.713178in}{2.708577in}%
\pgfsys@useobject{currentmarker}{}%
\end{pgfscope}%
\begin{pgfscope}%
\pgfsys@transformshift{2.730550in}{2.804563in}%
\pgfsys@useobject{currentmarker}{}%
\end{pgfscope}%
\begin{pgfscope}%
\pgfsys@transformshift{2.750975in}{2.759720in}%
\pgfsys@useobject{currentmarker}{}%
\end{pgfscope}%
\begin{pgfscope}%
\pgfsys@transformshift{2.769523in}{2.597007in}%
\pgfsys@useobject{currentmarker}{}%
\end{pgfscope}%
\begin{pgfscope}%
\pgfsys@transformshift{2.787834in}{2.275935in}%
\pgfsys@useobject{currentmarker}{}%
\end{pgfscope}%
\begin{pgfscope}%
\pgfsys@transformshift{2.810138in}{2.073952in}%
\pgfsys@useobject{currentmarker}{}%
\end{pgfscope}%
\begin{pgfscope}%
\pgfsys@transformshift{2.827746in}{2.024669in}%
\pgfsys@useobject{currentmarker}{}%
\end{pgfscope}%
\begin{pgfscope}%
\pgfsys@transformshift{2.850988in}{1.972785in}%
\pgfsys@useobject{currentmarker}{}%
\end{pgfscope}%
\begin{pgfscope}%
\pgfsys@transformshift{2.866248in}{1.980099in}%
\pgfsys@useobject{currentmarker}{}%
\end{pgfscope}%
\begin{pgfscope}%
\pgfsys@transformshift{2.884561in}{2.002194in}%
\pgfsys@useobject{currentmarker}{}%
\end{pgfscope}%
\begin{pgfscope}%
\pgfsys@transformshift{2.905456in}{2.087059in}%
\pgfsys@useobject{currentmarker}{}%
\end{pgfscope}%
\begin{pgfscope}%
\pgfsys@transformshift{2.924941in}{2.224091in}%
\pgfsys@useobject{currentmarker}{}%
\end{pgfscope}%
\begin{pgfscope}%
\pgfsys@transformshift{2.941846in}{2.493798in}%
\pgfsys@useobject{currentmarker}{}%
\end{pgfscope}%
\begin{pgfscope}%
\pgfsys@transformshift{2.962740in}{2.806868in}%
\pgfsys@useobject{currentmarker}{}%
\end{pgfscope}%
\begin{pgfscope}%
\pgfsys@transformshift{2.981992in}{2.819207in}%
\pgfsys@useobject{currentmarker}{}%
\end{pgfscope}%
\begin{pgfscope}%
\pgfsys@transformshift{2.998894in}{2.637481in}%
\pgfsys@useobject{currentmarker}{}%
\end{pgfscope}%
\begin{pgfscope}%
\pgfsys@transformshift{3.021434in}{2.289413in}%
\pgfsys@useobject{currentmarker}{}%
\end{pgfscope}%
\begin{pgfscope}%
\pgfsys@transformshift{3.038807in}{2.121512in}%
\pgfsys@useobject{currentmarker}{}%
\end{pgfscope}%
\begin{pgfscope}%
\pgfsys@transformshift{3.059467in}{2.028295in}%
\pgfsys@useobject{currentmarker}{}%
\end{pgfscope}%
\begin{pgfscope}%
\pgfsys@transformshift{3.077310in}{1.985493in}%
\pgfsys@useobject{currentmarker}{}%
\end{pgfscope}%
\begin{pgfscope}%
\pgfsys@transformshift{3.096561in}{1.974952in}%
\pgfsys@useobject{currentmarker}{}%
\end{pgfscope}%
\begin{pgfscope}%
\pgfsys@transformshift{3.118395in}{2.013893in}%
\pgfsys@useobject{currentmarker}{}%
\end{pgfscope}%
\begin{pgfscope}%
\pgfsys@transformshift{3.135298in}{2.077525in}%
\pgfsys@useobject{currentmarker}{}%
\end{pgfscope}%
\begin{pgfscope}%
\pgfsys@transformshift{3.154549in}{2.193801in}%
\pgfsys@useobject{currentmarker}{}%
\end{pgfscope}%
\begin{pgfscope}%
\pgfsys@transformshift{3.174269in}{2.410773in}%
\pgfsys@useobject{currentmarker}{}%
\end{pgfscope}%
\begin{pgfscope}%
\pgfsys@transformshift{3.191408in}{2.616901in}%
\pgfsys@useobject{currentmarker}{}%
\end{pgfscope}%
\begin{pgfscope}%
\pgfsys@transformshift{3.213243in}{2.804644in}%
\pgfsys@useobject{currentmarker}{}%
\end{pgfscope}%
\begin{pgfscope}%
\pgfsys@transformshift{3.230145in}{2.758818in}%
\pgfsys@useobject{currentmarker}{}%
\end{pgfscope}%
\begin{pgfscope}%
\pgfsys@transformshift{3.249162in}{2.841432in}%
\pgfsys@useobject{currentmarker}{}%
\end{pgfscope}%
\begin{pgfscope}%
\pgfsys@transformshift{3.266536in}{2.739494in}%
\pgfsys@useobject{currentmarker}{}%
\end{pgfscope}%
\begin{pgfscope}%
\pgfsys@transformshift{3.289073in}{2.392903in}%
\pgfsys@useobject{currentmarker}{}%
\end{pgfscope}%
\begin{pgfscope}%
\pgfsys@transformshift{3.309030in}{2.183944in}%
\pgfsys@useobject{currentmarker}{}%
\end{pgfscope}%
\begin{pgfscope}%
\pgfsys@transformshift{3.327107in}{2.081190in}%
\pgfsys@useobject{currentmarker}{}%
\end{pgfscope}%
\begin{pgfscope}%
\pgfsys@transformshift{3.345654in}{2.017536in}%
\pgfsys@useobject{currentmarker}{}%
\end{pgfscope}%
\begin{pgfscope}%
\pgfsys@transformshift{3.364437in}{1.975869in}%
\pgfsys@useobject{currentmarker}{}%
\end{pgfscope}%
\begin{pgfscope}%
\pgfsys@transformshift{3.382514in}{1.984599in}%
\pgfsys@useobject{currentmarker}{}%
\end{pgfscope}%
\begin{pgfscope}%
\pgfsys@transformshift{3.404582in}{2.013020in}%
\pgfsys@useobject{currentmarker}{}%
\end{pgfscope}%
\begin{pgfscope}%
\pgfsys@transformshift{3.423128in}{2.088307in}%
\pgfsys@useobject{currentmarker}{}%
\end{pgfscope}%
\begin{pgfscope}%
\pgfsys@transformshift{3.445197in}{2.251032in}%
\pgfsys@useobject{currentmarker}{}%
\end{pgfscope}%
\begin{pgfscope}%
\pgfsys@transformshift{3.460693in}{2.441261in}%
\pgfsys@useobject{currentmarker}{}%
\end{pgfscope}%
\begin{pgfscope}%
\pgfsys@transformshift{3.482527in}{2.728085in}%
\pgfsys@useobject{currentmarker}{}%
\end{pgfscope}%
\begin{pgfscope}%
\pgfsys@transformshift{3.498255in}{2.859756in}%
\pgfsys@useobject{currentmarker}{}%
\end{pgfscope}%
\begin{pgfscope}%
\pgfsys@transformshift{3.519152in}{2.832155in}%
\pgfsys@useobject{currentmarker}{}%
\end{pgfscope}%
\begin{pgfscope}%
\pgfsys@transformshift{3.538403in}{2.663760in}%
\pgfsys@useobject{currentmarker}{}%
\end{pgfscope}%
\begin{pgfscope}%
\pgfsys@transformshift{3.557183in}{2.406845in}%
\pgfsys@useobject{currentmarker}{}%
\end{pgfscope}%
\begin{pgfscope}%
\pgfsys@transformshift{3.576905in}{2.167362in}%
\pgfsys@useobject{currentmarker}{}%
\end{pgfscope}%
\begin{pgfscope}%
\pgfsys@transformshift{3.594748in}{2.675451in}%
\pgfsys@useobject{currentmarker}{}%
\end{pgfscope}%
\begin{pgfscope}%
\pgfsys@transformshift{3.615408in}{2.864836in}%
\pgfsys@useobject{currentmarker}{}%
\end{pgfscope}%
\begin{pgfscope}%
\pgfsys@transformshift{3.633719in}{2.819154in}%
\pgfsys@useobject{currentmarker}{}%
\end{pgfscope}%
\begin{pgfscope}%
\pgfsys@transformshift{3.653676in}{2.524175in}%
\pgfsys@useobject{currentmarker}{}%
\end{pgfscope}%
\begin{pgfscope}%
\pgfsys@transformshift{3.673161in}{2.235548in}%
\pgfsys@useobject{currentmarker}{}%
\end{pgfscope}%
\begin{pgfscope}%
\pgfsys@transformshift{3.691709in}{2.081443in}%
\pgfsys@useobject{currentmarker}{}%
\end{pgfscope}%
\begin{pgfscope}%
\pgfsys@transformshift{3.710960in}{2.006728in}%
\pgfsys@useobject{currentmarker}{}%
\end{pgfscope}%
\begin{pgfscope}%
\pgfsys@transformshift{3.733498in}{1.978371in}%
\pgfsys@useobject{currentmarker}{}%
\end{pgfscope}%
\begin{pgfscope}%
\pgfsys@transformshift{3.750401in}{2.003762in}%
\pgfsys@useobject{currentmarker}{}%
\end{pgfscope}%
\begin{pgfscope}%
\pgfsys@transformshift{3.768480in}{2.064581in}%
\pgfsys@useobject{currentmarker}{}%
\end{pgfscope}%
\begin{pgfscope}%
\pgfsys@transformshift{3.787731in}{2.180269in}%
\pgfsys@useobject{currentmarker}{}%
\end{pgfscope}%
\begin{pgfscope}%
\pgfsys@transformshift{3.809329in}{2.411959in}%
\pgfsys@useobject{currentmarker}{}%
\end{pgfscope}%
\begin{pgfscope}%
\pgfsys@transformshift{3.825999in}{2.674143in}%
\pgfsys@useobject{currentmarker}{}%
\end{pgfscope}%
\begin{pgfscope}%
\pgfsys@transformshift{3.851354in}{2.886767in}%
\pgfsys@useobject{currentmarker}{}%
\end{pgfscope}%
\begin{pgfscope}%
\pgfsys@transformshift{3.867319in}{2.826277in}%
\pgfsys@useobject{currentmarker}{}%
\end{pgfscope}%
\begin{pgfscope}%
\pgfsys@transformshift{3.886335in}{2.584361in}%
\pgfsys@useobject{currentmarker}{}%
\end{pgfscope}%
\begin{pgfscope}%
\pgfsys@transformshift{3.901829in}{2.351762in}%
\pgfsys@useobject{currentmarker}{}%
\end{pgfscope}%
\begin{pgfscope}%
\pgfsys@transformshift{3.922489in}{2.151666in}%
\pgfsys@useobject{currentmarker}{}%
\end{pgfscope}%
\begin{pgfscope}%
\pgfsys@transformshift{3.941272in}{2.051856in}%
\pgfsys@useobject{currentmarker}{}%
\end{pgfscope}%
\begin{pgfscope}%
\pgfsys@transformshift{3.962871in}{1.996697in}%
\pgfsys@useobject{currentmarker}{}%
\end{pgfscope}%
\begin{pgfscope}%
\pgfsys@transformshift{3.982357in}{1.981403in}%
\pgfsys@useobject{currentmarker}{}%
\end{pgfscope}%
\begin{pgfscope}%
\pgfsys@transformshift{4.000903in}{2.008721in}%
\pgfsys@useobject{currentmarker}{}%
\end{pgfscope}%
\begin{pgfscope}%
\pgfsys@transformshift{4.019451in}{2.071600in}%
\pgfsys@useobject{currentmarker}{}%
\end{pgfscope}%
\begin{pgfscope}%
\pgfsys@transformshift{4.038702in}{2.181954in}%
\pgfsys@useobject{currentmarker}{}%
\end{pgfscope}%
\begin{pgfscope}%
\pgfsys@transformshift{4.057250in}{2.350550in}%
\pgfsys@useobject{currentmarker}{}%
\end{pgfscope}%
\begin{pgfscope}%
\pgfsys@transformshift{4.076030in}{2.564405in}%
\pgfsys@useobject{currentmarker}{}%
\end{pgfscope}%
\begin{pgfscope}%
\pgfsys@transformshift{4.096456in}{2.838547in}%
\pgfsys@useobject{currentmarker}{}%
\end{pgfscope}%
\begin{pgfscope}%
\pgfsys@transformshift{4.115003in}{2.908719in}%
\pgfsys@useobject{currentmarker}{}%
\end{pgfscope}%
\begin{pgfscope}%
\pgfsys@transformshift{4.136367in}{2.737380in}%
\pgfsys@useobject{currentmarker}{}%
\end{pgfscope}%
\begin{pgfscope}%
\pgfsys@transformshift{4.155618in}{2.626944in}%
\pgfsys@useobject{currentmarker}{}%
\end{pgfscope}%
\begin{pgfscope}%
\pgfsys@transformshift{4.177218in}{2.328790in}%
\pgfsys@useobject{currentmarker}{}%
\end{pgfscope}%
\begin{pgfscope}%
\pgfsys@transformshift{4.193182in}{2.170010in}%
\pgfsys@useobject{currentmarker}{}%
\end{pgfscope}%
\begin{pgfscope}%
\pgfsys@transformshift{4.209147in}{2.070808in}%
\pgfsys@useobject{currentmarker}{}%
\end{pgfscope}%
\begin{pgfscope}%
\pgfsys@transformshift{4.230747in}{2.012861in}%
\pgfsys@useobject{currentmarker}{}%
\end{pgfscope}%
\begin{pgfscope}%
\pgfsys@transformshift{4.249762in}{1.987449in}%
\pgfsys@useobject{currentmarker}{}%
\end{pgfscope}%
\begin{pgfscope}%
\pgfsys@transformshift{4.269013in}{2.016473in}%
\pgfsys@useobject{currentmarker}{}%
\end{pgfscope}%
\begin{pgfscope}%
\pgfsys@transformshift{4.290144in}{2.076098in}%
\pgfsys@useobject{currentmarker}{}%
\end{pgfscope}%
\begin{pgfscope}%
\pgfsys@transformshift{4.306343in}{2.163247in}%
\pgfsys@useobject{currentmarker}{}%
\end{pgfscope}%
\begin{pgfscope}%
\pgfsys@transformshift{4.330289in}{2.417794in}%
\pgfsys@useobject{currentmarker}{}%
\end{pgfscope}%
\begin{pgfscope}%
\pgfsys@transformshift{4.346254in}{2.662377in}%
\pgfsys@useobject{currentmarker}{}%
\end{pgfscope}%
\begin{pgfscope}%
\pgfsys@transformshift{4.366680in}{2.892775in}%
\pgfsys@useobject{currentmarker}{}%
\end{pgfscope}%
\begin{pgfscope}%
\pgfsys@transformshift{4.385226in}{2.942701in}%
\pgfsys@useobject{currentmarker}{}%
\end{pgfscope}%
\begin{pgfscope}%
\pgfsys@transformshift{4.404008in}{2.902506in}%
\pgfsys@useobject{currentmarker}{}%
\end{pgfscope}%
\begin{pgfscope}%
\pgfsys@transformshift{4.422319in}{2.704652in}%
\pgfsys@useobject{currentmarker}{}%
\end{pgfscope}%
\begin{pgfscope}%
\pgfsys@transformshift{4.444624in}{2.414671in}%
\pgfsys@useobject{currentmarker}{}%
\end{pgfscope}%
\begin{pgfscope}%
\pgfsys@transformshift{4.461293in}{2.255625in}%
\pgfsys@useobject{currentmarker}{}%
\end{pgfscope}%
\begin{pgfscope}%
\pgfsys@transformshift{4.479370in}{2.115181in}%
\pgfsys@useobject{currentmarker}{}%
\end{pgfscope}%
\begin{pgfscope}%
\pgfsys@transformshift{4.479840in}{2.114526in}%
\pgfsys@useobject{currentmarker}{}%
\end{pgfscope}%
\begin{pgfscope}%
\pgfsys@transformshift{4.473971in}{2.164886in}%
\pgfsys@useobject{currentmarker}{}%
\end{pgfscope}%
\begin{pgfscope}%
\pgfsys@transformshift{4.454483in}{2.447379in}%
\pgfsys@useobject{currentmarker}{}%
\end{pgfscope}%
\begin{pgfscope}%
\pgfsys@transformshift{4.434529in}{2.824744in}%
\pgfsys@useobject{currentmarker}{}%
\end{pgfscope}%
\begin{pgfscope}%
\pgfsys@transformshift{4.417390in}{2.942807in}%
\pgfsys@useobject{currentmarker}{}%
\end{pgfscope}%
\begin{pgfscope}%
\pgfsys@transformshift{4.398844in}{2.701836in}%
\pgfsys@useobject{currentmarker}{}%
\end{pgfscope}%
\begin{pgfscope}%
\pgfsys@transformshift{4.377713in}{2.281115in}%
\pgfsys@useobject{currentmarker}{}%
\end{pgfscope}%
\begin{pgfscope}%
\pgfsys@transformshift{4.360105in}{2.100198in}%
\pgfsys@useobject{currentmarker}{}%
\end{pgfscope}%
\begin{pgfscope}%
\pgfsys@transformshift{4.338976in}{1.999276in}%
\pgfsys@useobject{currentmarker}{}%
\end{pgfscope}%
\begin{pgfscope}%
\pgfsys@transformshift{4.320899in}{1.999017in}%
\pgfsys@useobject{currentmarker}{}%
\end{pgfscope}%
\begin{pgfscope}%
\pgfsys@transformshift{4.302117in}{2.081213in}%
\pgfsys@useobject{currentmarker}{}%
\end{pgfscope}%
\begin{pgfscope}%
\pgfsys@transformshift{4.280988in}{2.317690in}%
\pgfsys@useobject{currentmarker}{}%
\end{pgfscope}%
\begin{pgfscope}%
\pgfsys@transformshift{4.261500in}{2.674859in}%
\pgfsys@useobject{currentmarker}{}%
\end{pgfscope}%
\begin{pgfscope}%
\pgfsys@transformshift{4.243189in}{2.907850in}%
\pgfsys@useobject{currentmarker}{}%
\end{pgfscope}%
\begin{pgfscope}%
\pgfsys@transformshift{4.225347in}{2.814338in}%
\pgfsys@useobject{currentmarker}{}%
\end{pgfscope}%
\begin{pgfscope}%
\pgfsys@transformshift{4.203043in}{2.370884in}%
\pgfsys@useobject{currentmarker}{}%
\end{pgfscope}%
\begin{pgfscope}%
\pgfsys@transformshift{4.186844in}{2.170831in}%
\pgfsys@useobject{currentmarker}{}%
\end{pgfscope}%
\begin{pgfscope}%
\pgfsys@transformshift{4.168296in}{2.041936in}%
\pgfsys@useobject{currentmarker}{}%
\end{pgfscope}%
\begin{pgfscope}%
\pgfsys@transformshift{4.147167in}{1.980881in}%
\pgfsys@useobject{currentmarker}{}%
\end{pgfscope}%
\begin{pgfscope}%
\pgfsys@transformshift{4.128854in}{2.020037in}%
\pgfsys@useobject{currentmarker}{}%
\end{pgfscope}%
\begin{pgfscope}%
\pgfsys@transformshift{4.109134in}{2.138065in}%
\pgfsys@useobject{currentmarker}{}%
\end{pgfscope}%
\begin{pgfscope}%
\pgfsys@transformshift{4.090117in}{2.390063in}%
\pgfsys@useobject{currentmarker}{}%
\end{pgfscope}%
\begin{pgfscope}%
\pgfsys@transformshift{4.068754in}{2.807960in}%
\pgfsys@useobject{currentmarker}{}%
\end{pgfscope}%
\begin{pgfscope}%
\pgfsys@transformshift{4.050441in}{2.882823in}%
\pgfsys@useobject{currentmarker}{}%
\end{pgfscope}%
\begin{pgfscope}%
\pgfsys@transformshift{4.031658in}{2.635713in}%
\pgfsys@useobject{currentmarker}{}%
\end{pgfscope}%
\begin{pgfscope}%
\pgfsys@transformshift{4.014756in}{2.284261in}%
\pgfsys@useobject{currentmarker}{}%
\end{pgfscope}%
\begin{pgfscope}%
\pgfsys@transformshift{3.992687in}{2.077867in}%
\pgfsys@useobject{currentmarker}{}%
\end{pgfscope}%
\begin{pgfscope}%
\pgfsys@transformshift{3.974610in}{1.992337in}%
\pgfsys@useobject{currentmarker}{}%
\end{pgfscope}%
\begin{pgfscope}%
\pgfsys@transformshift{3.954185in}{1.984121in}%
\pgfsys@useobject{currentmarker}{}%
\end{pgfscope}%
\begin{pgfscope}%
\pgfsys@transformshift{3.938220in}{2.031403in}%
\pgfsys@useobject{currentmarker}{}%
\end{pgfscope}%
\begin{pgfscope}%
\pgfsys@transformshift{3.916385in}{2.201877in}%
\pgfsys@useobject{currentmarker}{}%
\end{pgfscope}%
\begin{pgfscope}%
\pgfsys@transformshift{3.898074in}{2.486429in}%
\pgfsys@useobject{currentmarker}{}%
\end{pgfscope}%
\begin{pgfscope}%
\pgfsys@transformshift{3.878823in}{2.811413in}%
\pgfsys@useobject{currentmarker}{}%
\end{pgfscope}%
\begin{pgfscope}%
\pgfsys@transformshift{3.860041in}{2.844809in}%
\pgfsys@useobject{currentmarker}{}%
\end{pgfscope}%
\begin{pgfscope}%
\pgfsys@transformshift{3.838912in}{2.507825in}%
\pgfsys@useobject{currentmarker}{}%
\end{pgfscope}%
\begin{pgfscope}%
\pgfsys@transformshift{3.819660in}{2.194500in}%
\pgfsys@useobject{currentmarker}{}%
\end{pgfscope}%
\begin{pgfscope}%
\pgfsys@transformshift{3.802052in}{2.053196in}%
\pgfsys@useobject{currentmarker}{}%
\end{pgfscope}%
\begin{pgfscope}%
\pgfsys@transformshift{3.782565in}{1.988026in}%
\pgfsys@useobject{currentmarker}{}%
\end{pgfscope}%
\begin{pgfscope}%
\pgfsys@transformshift{3.761905in}{1.983412in}%
\pgfsys@useobject{currentmarker}{}%
\end{pgfscope}%
\begin{pgfscope}%
\pgfsys@transformshift{3.741716in}{2.054931in}%
\pgfsys@useobject{currentmarker}{}%
\end{pgfscope}%
\begin{pgfscope}%
\pgfsys@transformshift{3.724577in}{2.199704in}%
\pgfsys@useobject{currentmarker}{}%
\end{pgfscope}%
\begin{pgfscope}%
\pgfsys@transformshift{3.706029in}{2.510191in}%
\pgfsys@useobject{currentmarker}{}%
\end{pgfscope}%
\begin{pgfscope}%
\pgfsys@transformshift{3.686074in}{2.789599in}%
\pgfsys@useobject{currentmarker}{}%
\end{pgfscope}%
\begin{pgfscope}%
\pgfsys@transformshift{3.667058in}{2.829552in}%
\pgfsys@useobject{currentmarker}{}%
\end{pgfscope}%
\begin{pgfscope}%
\pgfsys@transformshift{3.647101in}{2.594954in}%
\pgfsys@useobject{currentmarker}{}%
\end{pgfscope}%
\begin{pgfscope}%
\pgfsys@transformshift{3.628321in}{2.291885in}%
\pgfsys@useobject{currentmarker}{}%
\end{pgfscope}%
\begin{pgfscope}%
\pgfsys@transformshift{3.608599in}{2.092083in}%
\pgfsys@useobject{currentmarker}{}%
\end{pgfscope}%
\begin{pgfscope}%
\pgfsys@transformshift{3.593339in}{2.025170in}%
\pgfsys@useobject{currentmarker}{}%
\end{pgfscope}%
\begin{pgfscope}%
\pgfsys@transformshift{3.571270in}{1.984677in}%
\pgfsys@useobject{currentmarker}{}%
\end{pgfscope}%
\begin{pgfscope}%
\pgfsys@transformshift{3.552019in}{1.981108in}%
\pgfsys@useobject{currentmarker}{}%
\end{pgfscope}%
\begin{pgfscope}%
\pgfsys@transformshift{3.533003in}{2.025216in}%
\pgfsys@useobject{currentmarker}{}%
\end{pgfscope}%
\begin{pgfscope}%
\pgfsys@transformshift{3.512343in}{2.181693in}%
\pgfsys@useobject{currentmarker}{}%
\end{pgfscope}%
\begin{pgfscope}%
\pgfsys@transformshift{3.496612in}{2.354463in}%
\pgfsys@useobject{currentmarker}{}%
\end{pgfscope}%
\begin{pgfscope}%
\pgfsys@transformshift{3.475718in}{2.734890in}%
\pgfsys@useobject{currentmarker}{}%
\end{pgfscope}%
\begin{pgfscope}%
\pgfsys@transformshift{3.458344in}{2.843208in}%
\pgfsys@useobject{currentmarker}{}%
\end{pgfscope}%
\begin{pgfscope}%
\pgfsys@transformshift{3.438155in}{2.767560in}%
\pgfsys@useobject{currentmarker}{}%
\end{pgfscope}%
\begin{pgfscope}%
\pgfsys@transformshift{3.415850in}{2.396336in}%
\pgfsys@useobject{currentmarker}{}%
\end{pgfscope}%
\begin{pgfscope}%
\pgfsys@transformshift{3.398713in}{2.155086in}%
\pgfsys@useobject{currentmarker}{}%
\end{pgfscope}%
\begin{pgfscope}%
\pgfsys@transformshift{3.378522in}{2.037143in}%
\pgfsys@useobject{currentmarker}{}%
\end{pgfscope}%
\begin{pgfscope}%
\pgfsys@transformshift{3.359505in}{1.983433in}%
\pgfsys@useobject{currentmarker}{}%
\end{pgfscope}%
\begin{pgfscope}%
\pgfsys@transformshift{3.337907in}{1.978423in}%
\pgfsys@useobject{currentmarker}{}%
\end{pgfscope}%
\begin{pgfscope}%
\pgfsys@transformshift{3.322646in}{2.014682in}%
\pgfsys@useobject{currentmarker}{}%
\end{pgfscope}%
\begin{pgfscope}%
\pgfsys@transformshift{3.300577in}{2.075803in}%
\pgfsys@useobject{currentmarker}{}%
\end{pgfscope}%
\begin{pgfscope}%
\pgfsys@transformshift{3.283204in}{2.245210in}%
\pgfsys@useobject{currentmarker}{}%
\end{pgfscope}%
\begin{pgfscope}%
\pgfsys@transformshift{3.263484in}{2.504697in}%
\pgfsys@useobject{currentmarker}{}%
\end{pgfscope}%
\begin{pgfscope}%
\pgfsys@transformshift{3.244232in}{2.794338in}%
\pgfsys@useobject{currentmarker}{}%
\end{pgfscope}%
\begin{pgfscope}%
\pgfsys@transformshift{3.226859in}{2.815982in}%
\pgfsys@useobject{currentmarker}{}%
\end{pgfscope}%
\begin{pgfscope}%
\pgfsys@transformshift{3.205025in}{2.521253in}%
\pgfsys@useobject{currentmarker}{}%
\end{pgfscope}%
\begin{pgfscope}%
\pgfsys@transformshift{3.186244in}{2.231368in}%
\pgfsys@useobject{currentmarker}{}%
\end{pgfscope}%
\begin{pgfscope}%
\pgfsys@transformshift{3.167228in}{2.113328in}%
\pgfsys@useobject{currentmarker}{}%
\end{pgfscope}%
\begin{pgfscope}%
\pgfsys@transformshift{3.145159in}{2.007425in}%
\pgfsys@useobject{currentmarker}{}%
\end{pgfscope}%
\begin{pgfscope}%
\pgfsys@transformshift{3.131072in}{1.976301in}%
\pgfsys@useobject{currentmarker}{}%
\end{pgfscope}%
\begin{pgfscope}%
\pgfsys@transformshift{3.109003in}{1.984108in}%
\pgfsys@useobject{currentmarker}{}%
\end{pgfscope}%
\begin{pgfscope}%
\pgfsys@transformshift{3.090457in}{2.038927in}%
\pgfsys@useobject{currentmarker}{}%
\end{pgfscope}%
\begin{pgfscope}%
\pgfsys@transformshift{3.071675in}{2.157502in}%
\pgfsys@useobject{currentmarker}{}%
\end{pgfscope}%
\begin{pgfscope}%
\pgfsys@transformshift{3.052893in}{2.401583in}%
\pgfsys@useobject{currentmarker}{}%
\end{pgfscope}%
\begin{pgfscope}%
\pgfsys@transformshift{3.034347in}{2.719984in}%
\pgfsys@useobject{currentmarker}{}%
\end{pgfscope}%
\begin{pgfscope}%
\pgfsys@transformshift{3.014859in}{2.144244in}%
\pgfsys@useobject{currentmarker}{}%
\end{pgfscope}%
\begin{pgfscope}%
\pgfsys@transformshift{2.994199in}{2.455638in}%
\pgfsys@useobject{currentmarker}{}%
\end{pgfscope}%
\begin{pgfscope}%
\pgfsys@transformshift{2.975182in}{2.711049in}%
\pgfsys@useobject{currentmarker}{}%
\end{pgfscope}%
\begin{pgfscope}%
\pgfsys@transformshift{2.956871in}{2.828816in}%
\pgfsys@useobject{currentmarker}{}%
\end{pgfscope}%
\begin{pgfscope}%
\pgfsys@transformshift{2.936446in}{2.646071in}%
\pgfsys@useobject{currentmarker}{}%
\end{pgfscope}%
\begin{pgfscope}%
\pgfsys@transformshift{2.916960in}{2.320087in}%
\pgfsys@useobject{currentmarker}{}%
\end{pgfscope}%
\begin{pgfscope}%
\pgfsys@transformshift{2.898647in}{2.122484in}%
\pgfsys@useobject{currentmarker}{}%
\end{pgfscope}%
\begin{pgfscope}%
\pgfsys@transformshift{2.876343in}{2.009911in}%
\pgfsys@useobject{currentmarker}{}%
\end{pgfscope}%
\begin{pgfscope}%
\pgfsys@transformshift{2.860615in}{1.975802in}%
\pgfsys@useobject{currentmarker}{}%
\end{pgfscope}%
\begin{pgfscope}%
\pgfsys@transformshift{2.839719in}{1.984173in}%
\pgfsys@useobject{currentmarker}{}%
\end{pgfscope}%
\begin{pgfscope}%
\pgfsys@transformshift{2.817181in}{2.044402in}%
\pgfsys@useobject{currentmarker}{}%
\end{pgfscope}%
\begin{pgfscope}%
\pgfsys@transformshift{2.802391in}{2.129521in}%
\pgfsys@useobject{currentmarker}{}%
\end{pgfscope}%
\begin{pgfscope}%
\pgfsys@transformshift{2.783374in}{2.319107in}%
\pgfsys@useobject{currentmarker}{}%
\end{pgfscope}%
\begin{pgfscope}%
\pgfsys@transformshift{2.761776in}{2.685474in}%
\pgfsys@useobject{currentmarker}{}%
\end{pgfscope}%
\begin{pgfscope}%
\pgfsys@transformshift{2.746515in}{2.817530in}%
\pgfsys@useobject{currentmarker}{}%
\end{pgfscope}%
\begin{pgfscope}%
\pgfsys@transformshift{2.725855in}{2.743189in}%
\pgfsys@useobject{currentmarker}{}%
\end{pgfscope}%
\begin{pgfscope}%
\pgfsys@transformshift{2.705900in}{2.427476in}%
\pgfsys@useobject{currentmarker}{}%
\end{pgfscope}%
\begin{pgfscope}%
\pgfsys@transformshift{2.688056in}{2.180378in}%
\pgfsys@useobject{currentmarker}{}%
\end{pgfscope}%
\begin{pgfscope}%
\pgfsys@transformshift{2.669041in}{2.059373in}%
\pgfsys@useobject{currentmarker}{}%
\end{pgfscope}%
\begin{pgfscope}%
\pgfsys@transformshift{2.647676in}{1.986927in}%
\pgfsys@useobject{currentmarker}{}%
\end{pgfscope}%
\begin{pgfscope}%
\pgfsys@transformshift{2.631711in}{1.970274in}%
\pgfsys@useobject{currentmarker}{}%
\end{pgfscope}%
\begin{pgfscope}%
\pgfsys@transformshift{2.610347in}{1.978439in}%
\pgfsys@useobject{currentmarker}{}%
\end{pgfscope}%
\begin{pgfscope}%
\pgfsys@transformshift{2.591565in}{2.023142in}%
\pgfsys@useobject{currentmarker}{}%
\end{pgfscope}%
\begin{pgfscope}%
\pgfsys@transformshift{2.569731in}{2.132774in}%
\pgfsys@useobject{currentmarker}{}%
\end{pgfscope}%
\begin{pgfscope}%
\pgfsys@transformshift{2.551888in}{2.260169in}%
\pgfsys@useobject{currentmarker}{}%
\end{pgfscope}%
\begin{pgfscope}%
\pgfsys@transformshift{2.532872in}{2.586624in}%
\pgfsys@useobject{currentmarker}{}%
\end{pgfscope}%
\begin{pgfscope}%
\pgfsys@transformshift{2.513855in}{2.800222in}%
\pgfsys@useobject{currentmarker}{}%
\end{pgfscope}%
\begin{pgfscope}%
\pgfsys@transformshift{2.495543in}{2.767075in}%
\pgfsys@useobject{currentmarker}{}%
\end{pgfscope}%
\begin{pgfscope}%
\pgfsys@transformshift{2.474178in}{2.504748in}%
\pgfsys@useobject{currentmarker}{}%
\end{pgfscope}%
\begin{pgfscope}%
\pgfsys@transformshift{2.456101in}{2.250579in}%
\pgfsys@useobject{currentmarker}{}%
\end{pgfscope}%
\begin{pgfscope}%
\pgfsys@transformshift{2.437319in}{2.091042in}%
\pgfsys@useobject{currentmarker}{}%
\end{pgfscope}%
\begin{pgfscope}%
\pgfsys@transformshift{2.419008in}{2.009098in}%
\pgfsys@useobject{currentmarker}{}%
\end{pgfscope}%
\begin{pgfscope}%
\pgfsys@transformshift{2.397173in}{1.971415in}%
\pgfsys@useobject{currentmarker}{}%
\end{pgfscope}%
\begin{pgfscope}%
\pgfsys@transformshift{2.379096in}{1.980091in}%
\pgfsys@useobject{currentmarker}{}%
\end{pgfscope}%
\begin{pgfscope}%
\pgfsys@transformshift{2.360783in}{2.020049in}%
\pgfsys@useobject{currentmarker}{}%
\end{pgfscope}%
\begin{pgfscope}%
\pgfsys@transformshift{2.339185in}{2.138056in}%
\pgfsys@useobject{currentmarker}{}%
\end{pgfscope}%
\begin{pgfscope}%
\pgfsys@transformshift{2.317586in}{2.367698in}%
\pgfsys@useobject{currentmarker}{}%
\end{pgfscope}%
\begin{pgfscope}%
\pgfsys@transformshift{2.301621in}{2.591443in}%
\pgfsys@useobject{currentmarker}{}%
\end{pgfscope}%
\begin{pgfscope}%
\pgfsys@transformshift{2.283778in}{2.775101in}%
\pgfsys@useobject{currentmarker}{}%
\end{pgfscope}%
\begin{pgfscope}%
\pgfsys@transformshift{2.264527in}{2.814906in}%
\pgfsys@useobject{currentmarker}{}%
\end{pgfscope}%
\begin{pgfscope}%
\pgfsys@transformshift{2.245745in}{2.645014in}%
\pgfsys@useobject{currentmarker}{}%
\end{pgfscope}%
\begin{pgfscope}%
\pgfsys@transformshift{2.225085in}{2.456907in}%
\pgfsys@useobject{currentmarker}{}%
\end{pgfscope}%
\begin{pgfscope}%
\pgfsys@transformshift{2.205130in}{2.248216in}%
\pgfsys@useobject{currentmarker}{}%
\end{pgfscope}%
\begin{pgfscope}%
\pgfsys@transformshift{2.187051in}{2.103088in}%
\pgfsys@useobject{currentmarker}{}%
\end{pgfscope}%
\begin{pgfscope}%
\pgfsys@transformshift{2.169209in}{2.020885in}%
\pgfsys@useobject{currentmarker}{}%
\end{pgfscope}%
\begin{pgfscope}%
\pgfsys@transformshift{2.147140in}{1.974024in}%
\pgfsys@useobject{currentmarker}{}%
\end{pgfscope}%
\begin{pgfscope}%
\pgfsys@transformshift{2.129532in}{1.985778in}%
\pgfsys@useobject{currentmarker}{}%
\end{pgfscope}%
\begin{pgfscope}%
\pgfsys@transformshift{2.110046in}{2.023406in}%
\pgfsys@useobject{currentmarker}{}%
\end{pgfscope}%
\begin{pgfscope}%
\pgfsys@transformshift{2.092204in}{2.094623in}%
\pgfsys@useobject{currentmarker}{}%
\end{pgfscope}%
\begin{pgfscope}%
\pgfsys@transformshift{2.070135in}{2.302287in}%
\pgfsys@useobject{currentmarker}{}%
\end{pgfscope}%
\begin{pgfscope}%
\pgfsys@transformshift{2.052293in}{2.595187in}%
\pgfsys@useobject{currentmarker}{}%
\end{pgfscope}%
\begin{pgfscope}%
\pgfsys@transformshift{2.033276in}{2.673128in}%
\pgfsys@useobject{currentmarker}{}%
\end{pgfscope}%
\begin{pgfscope}%
\pgfsys@transformshift{2.015434in}{2.006993in}%
\pgfsys@useobject{currentmarker}{}%
\end{pgfscope}%
\begin{pgfscope}%
\pgfsys@transformshift{1.996417in}{2.094331in}%
\pgfsys@useobject{currentmarker}{}%
\end{pgfscope}%
\begin{pgfscope}%
\pgfsys@transformshift{1.975757in}{2.286318in}%
\pgfsys@useobject{currentmarker}{}%
\end{pgfscope}%
\begin{pgfscope}%
\pgfsys@transformshift{1.954863in}{2.669069in}%
\pgfsys@useobject{currentmarker}{}%
\end{pgfscope}%
\begin{pgfscope}%
\pgfsys@transformshift{1.937020in}{2.823848in}%
\pgfsys@useobject{currentmarker}{}%
\end{pgfscope}%
\begin{pgfscope}%
\pgfsys@transformshift{1.916126in}{2.746438in}%
\pgfsys@useobject{currentmarker}{}%
\end{pgfscope}%
\begin{pgfscope}%
\pgfsys@transformshift{1.898518in}{2.541215in}%
\pgfsys@useobject{currentmarker}{}%
\end{pgfscope}%
\begin{pgfscope}%
\pgfsys@transformshift{1.875509in}{2.205758in}%
\pgfsys@useobject{currentmarker}{}%
\end{pgfscope}%
\begin{pgfscope}%
\pgfsys@transformshift{1.861188in}{2.081471in}%
\pgfsys@useobject{currentmarker}{}%
\end{pgfscope}%
\begin{pgfscope}%
\pgfsys@transformshift{1.841467in}{2.122953in}%
\pgfsys@useobject{currentmarker}{}%
\end{pgfscope}%
\begin{pgfscope}%
\pgfsys@transformshift{1.823625in}{2.025079in}%
\pgfsys@useobject{currentmarker}{}%
\end{pgfscope}%
\begin{pgfscope}%
\pgfsys@transformshift{1.802494in}{1.983298in}%
\pgfsys@useobject{currentmarker}{}%
\end{pgfscope}%
\begin{pgfscope}%
\pgfsys@transformshift{1.784183in}{1.977491in}%
\pgfsys@useobject{currentmarker}{}%
\end{pgfscope}%
\begin{pgfscope}%
\pgfsys@transformshift{1.764228in}{2.021373in}%
\pgfsys@useobject{currentmarker}{}%
\end{pgfscope}%
\begin{pgfscope}%
\pgfsys@transformshift{1.743803in}{2.123420in}%
\pgfsys@useobject{currentmarker}{}%
\end{pgfscope}%
\begin{pgfscope}%
\pgfsys@transformshift{1.725724in}{2.273273in}%
\pgfsys@useobject{currentmarker}{}%
\end{pgfscope}%
\begin{pgfscope}%
\pgfsys@transformshift{1.706707in}{2.586506in}%
\pgfsys@useobject{currentmarker}{}%
\end{pgfscope}%
\begin{pgfscope}%
\pgfsys@transformshift{1.688630in}{2.780691in}%
\pgfsys@useobject{currentmarker}{}%
\end{pgfscope}%
\begin{pgfscope}%
\pgfsys@transformshift{1.669848in}{2.847786in}%
\pgfsys@useobject{currentmarker}{}%
\end{pgfscope}%
\begin{pgfscope}%
\pgfsys@transformshift{1.648015in}{2.683946in}%
\pgfsys@useobject{currentmarker}{}%
\end{pgfscope}%
\begin{pgfscope}%
\pgfsys@transformshift{1.630407in}{2.378009in}%
\pgfsys@useobject{currentmarker}{}%
\end{pgfscope}%
\begin{pgfscope}%
\pgfsys@transformshift{1.609042in}{2.148337in}%
\pgfsys@useobject{currentmarker}{}%
\end{pgfscope}%
\begin{pgfscope}%
\pgfsys@transformshift{1.591905in}{2.054059in}%
\pgfsys@useobject{currentmarker}{}%
\end{pgfscope}%
\begin{pgfscope}%
\pgfsys@transformshift{1.571245in}{1.991772in}%
\pgfsys@useobject{currentmarker}{}%
\end{pgfscope}%
\begin{pgfscope}%
\pgfsys@transformshift{1.554106in}{1.975360in}%
\pgfsys@useobject{currentmarker}{}%
\end{pgfscope}%
\begin{pgfscope}%
\pgfsys@transformshift{1.531098in}{2.004633in}%
\pgfsys@useobject{currentmarker}{}%
\end{pgfscope}%
\begin{pgfscope}%
\pgfsys@transformshift{1.515133in}{2.053011in}%
\pgfsys@useobject{currentmarker}{}%
\end{pgfscope}%
\begin{pgfscope}%
\pgfsys@transformshift{1.494709in}{2.174896in}%
\pgfsys@useobject{currentmarker}{}%
\end{pgfscope}%
\begin{pgfscope}%
\pgfsys@transformshift{1.474284in}{2.444110in}%
\pgfsys@useobject{currentmarker}{}%
\end{pgfscope}%
\begin{pgfscope}%
\pgfsys@transformshift{1.456910in}{2.576245in}%
\pgfsys@useobject{currentmarker}{}%
\end{pgfscope}%
\begin{pgfscope}%
\pgfsys@transformshift{1.436954in}{2.815012in}%
\pgfsys@useobject{currentmarker}{}%
\end{pgfscope}%
\begin{pgfscope}%
\pgfsys@transformshift{1.418642in}{2.860957in}%
\pgfsys@useobject{currentmarker}{}%
\end{pgfscope}%
\begin{pgfscope}%
\pgfsys@transformshift{1.397982in}{2.735026in}%
\pgfsys@useobject{currentmarker}{}%
\end{pgfscope}%
\begin{pgfscope}%
\pgfsys@transformshift{1.380843in}{2.558005in}%
\pgfsys@useobject{currentmarker}{}%
\end{pgfscope}%
\begin{pgfscope}%
\pgfsys@transformshift{1.360654in}{2.275890in}%
\pgfsys@useobject{currentmarker}{}%
\end{pgfscope}%
\begin{pgfscope}%
\pgfsys@transformshift{1.336942in}{2.092482in}%
\pgfsys@useobject{currentmarker}{}%
\end{pgfscope}%
\begin{pgfscope}%
\pgfsys@transformshift{1.322621in}{2.031738in}%
\pgfsys@useobject{currentmarker}{}%
\end{pgfscope}%
\begin{pgfscope}%
\pgfsys@transformshift{1.301961in}{1.984764in}%
\pgfsys@useobject{currentmarker}{}%
\end{pgfscope}%
\begin{pgfscope}%
\pgfsys@transformshift{1.284822in}{1.980888in}%
\pgfsys@useobject{currentmarker}{}%
\end{pgfscope}%
\begin{pgfscope}%
\pgfsys@transformshift{1.264631in}{2.009729in}%
\pgfsys@useobject{currentmarker}{}%
\end{pgfscope}%
\begin{pgfscope}%
\pgfsys@transformshift{1.246554in}{2.050813in}%
\pgfsys@useobject{currentmarker}{}%
\end{pgfscope}%
\begin{pgfscope}%
\pgfsys@transformshift{1.225659in}{2.171386in}%
\pgfsys@useobject{currentmarker}{}%
\end{pgfscope}%
\begin{pgfscope}%
\pgfsys@transformshift{1.207348in}{2.439990in}%
\pgfsys@useobject{currentmarker}{}%
\end{pgfscope}%
\begin{pgfscope}%
\pgfsys@transformshift{1.187157in}{2.696912in}%
\pgfsys@useobject{currentmarker}{}%
\end{pgfscope}%
\begin{pgfscope}%
\pgfsys@transformshift{1.167435in}{2.868019in}%
\pgfsys@useobject{currentmarker}{}%
\end{pgfscope}%
\begin{pgfscope}%
\pgfsys@transformshift{1.149829in}{2.882425in}%
\pgfsys@useobject{currentmarker}{}%
\end{pgfscope}%
\begin{pgfscope}%
\pgfsys@transformshift{1.131515in}{2.744082in}%
\pgfsys@useobject{currentmarker}{}%
\end{pgfscope}%
\begin{pgfscope}%
\pgfsys@transformshift{1.111324in}{2.464563in}%
\pgfsys@useobject{currentmarker}{}%
\end{pgfscope}%
\begin{pgfscope}%
\pgfsys@transformshift{1.091370in}{2.212692in}%
\pgfsys@useobject{currentmarker}{}%
\end{pgfscope}%
\begin{pgfscope}%
\pgfsys@transformshift{1.073527in}{2.122672in}%
\pgfsys@useobject{currentmarker}{}%
\end{pgfscope}%
\begin{pgfscope}%
\pgfsys@transformshift{1.053336in}{2.048140in}%
\pgfsys@useobject{currentmarker}{}%
\end{pgfscope}%
\begin{pgfscope}%
\pgfsys@transformshift{1.032442in}{1.992204in}%
\pgfsys@useobject{currentmarker}{}%
\end{pgfscope}%
\begin{pgfscope}%
\pgfsys@transformshift{1.015068in}{1.983929in}%
\pgfsys@useobject{currentmarker}{}%
\end{pgfscope}%
\begin{pgfscope}%
\pgfsys@transformshift{0.996991in}{2.017685in}%
\pgfsys@useobject{currentmarker}{}%
\end{pgfscope}%
\begin{pgfscope}%
\pgfsys@transformshift{0.976566in}{2.037960in}%
\pgfsys@useobject{currentmarker}{}%
\end{pgfscope}%
\begin{pgfscope}%
\pgfsys@transformshift{0.955671in}{2.138939in}%
\pgfsys@useobject{currentmarker}{}%
\end{pgfscope}%
\begin{pgfscope}%
\pgfsys@transformshift{0.938298in}{2.307308in}%
\pgfsys@useobject{currentmarker}{}%
\end{pgfscope}%
\begin{pgfscope}%
\pgfsys@transformshift{0.916698in}{2.612916in}%
\pgfsys@useobject{currentmarker}{}%
\end{pgfscope}%
\begin{pgfscope}%
\pgfsys@transformshift{0.902142in}{2.834281in}%
\pgfsys@useobject{currentmarker}{}%
\end{pgfscope}%
\begin{pgfscope}%
\pgfsys@transformshift{0.881953in}{2.909879in}%
\pgfsys@useobject{currentmarker}{}%
\end{pgfscope}%
\begin{pgfscope}%
\pgfsys@transformshift{0.861059in}{2.844980in}%
\pgfsys@useobject{currentmarker}{}%
\end{pgfscope}%
\begin{pgfscope}%
\pgfsys@transformshift{0.839224in}{2.693728in}%
\pgfsys@useobject{currentmarker}{}%
\end{pgfscope}%
\begin{pgfscope}%
\pgfsys@transformshift{0.822791in}{2.414325in}%
\pgfsys@useobject{currentmarker}{}%
\end{pgfscope}%
\begin{pgfscope}%
\pgfsys@transformshift{0.801894in}{2.188173in}%
\pgfsys@useobject{currentmarker}{}%
\end{pgfscope}%
\begin{pgfscope}%
\pgfsys@transformshift{0.784286in}{2.087331in}%
\pgfsys@useobject{currentmarker}{}%
\end{pgfscope}%
\begin{pgfscope}%
\pgfsys@transformshift{0.762688in}{2.018805in}%
\pgfsys@useobject{currentmarker}{}%
\end{pgfscope}%
\begin{pgfscope}%
\pgfsys@transformshift{0.745549in}{1.988673in}%
\pgfsys@useobject{currentmarker}{}%
\end{pgfscope}%
\begin{pgfscope}%
\pgfsys@transformshift{0.727707in}{2.074173in}%
\pgfsys@useobject{currentmarker}{}%
\end{pgfscope}%
\begin{pgfscope}%
\pgfsys@transformshift{0.706812in}{2.015428in}%
\pgfsys@useobject{currentmarker}{}%
\end{pgfscope}%
\begin{pgfscope}%
\pgfsys@transformshift{0.687561in}{1.988018in}%
\pgfsys@useobject{currentmarker}{}%
\end{pgfscope}%
\begin{pgfscope}%
\pgfsys@transformshift{0.669719in}{2.020456in}%
\pgfsys@useobject{currentmarker}{}%
\end{pgfscope}%
\begin{pgfscope}%
\pgfsys@transformshift{0.648588in}{2.084186in}%
\pgfsys@useobject{currentmarker}{}%
\end{pgfscope}%
\begin{pgfscope}%
\pgfsys@transformshift{0.649528in}{2.088671in}%
\pgfsys@useobject{currentmarker}{}%
\end{pgfscope}%
\begin{pgfscope}%
\pgfsys@transformshift{0.657040in}{2.035542in}%
\pgfsys@useobject{currentmarker}{}%
\end{pgfscope}%
\begin{pgfscope}%
\pgfsys@transformshift{0.674414in}{1.989263in}%
\pgfsys@useobject{currentmarker}{}%
\end{pgfscope}%
\begin{pgfscope}%
\pgfsys@transformshift{0.696248in}{2.043706in}%
\pgfsys@useobject{currentmarker}{}%
\end{pgfscope}%
\begin{pgfscope}%
\pgfsys@transformshift{0.712213in}{2.144068in}%
\pgfsys@useobject{currentmarker}{}%
\end{pgfscope}%
\begin{pgfscope}%
\pgfsys@transformshift{0.733811in}{2.421919in}%
\pgfsys@useobject{currentmarker}{}%
\end{pgfscope}%
\begin{pgfscope}%
\pgfsys@transformshift{0.752124in}{2.798305in}%
\pgfsys@useobject{currentmarker}{}%
\end{pgfscope}%
\begin{pgfscope}%
\pgfsys@transformshift{0.771375in}{2.914428in}%
\pgfsys@useobject{currentmarker}{}%
\end{pgfscope}%
\begin{pgfscope}%
\pgfsys@transformshift{0.791564in}{2.671830in}%
\pgfsys@useobject{currentmarker}{}%
\end{pgfscope}%
\begin{pgfscope}%
\pgfsys@transformshift{0.808938in}{2.325422in}%
\pgfsys@useobject{currentmarker}{}%
\end{pgfscope}%
\begin{pgfscope}%
\pgfsys@transformshift{0.827486in}{2.107150in}%
\pgfsys@useobject{currentmarker}{}%
\end{pgfscope}%
\begin{pgfscope}%
\pgfsys@transformshift{0.849554in}{2.000504in}%
\pgfsys@useobject{currentmarker}{}%
\end{pgfscope}%
\begin{pgfscope}%
\pgfsys@transformshift{0.867397in}{1.989403in}%
\pgfsys@useobject{currentmarker}{}%
\end{pgfscope}%
\begin{pgfscope}%
\pgfsys@transformshift{0.885708in}{2.052444in}%
\pgfsys@useobject{currentmarker}{}%
\end{pgfscope}%
\begin{pgfscope}%
\pgfsys@transformshift{0.904491in}{2.195070in}%
\pgfsys@useobject{currentmarker}{}%
\end{pgfscope}%
\begin{pgfscope}%
\pgfsys@transformshift{0.925385in}{2.532556in}%
\pgfsys@useobject{currentmarker}{}%
\end{pgfscope}%
\begin{pgfscope}%
\pgfsys@transformshift{0.943698in}{2.853016in}%
\pgfsys@useobject{currentmarker}{}%
\end{pgfscope}%
\begin{pgfscope}%
\pgfsys@transformshift{0.961306in}{2.860760in}%
\pgfsys@useobject{currentmarker}{}%
\end{pgfscope}%
\begin{pgfscope}%
\pgfsys@transformshift{0.979383in}{2.550413in}%
\pgfsys@useobject{currentmarker}{}%
\end{pgfscope}%
\begin{pgfscope}%
\pgfsys@transformshift{1.000983in}{2.219161in}%
\pgfsys@useobject{currentmarker}{}%
\end{pgfscope}%
\begin{pgfscope}%
\pgfsys@transformshift{1.021407in}{2.042725in}%
\pgfsys@useobject{currentmarker}{}%
\end{pgfscope}%
\begin{pgfscope}%
\pgfsys@transformshift{1.039954in}{1.983970in}%
\pgfsys@useobject{currentmarker}{}%
\end{pgfscope}%
\begin{pgfscope}%
\pgfsys@transformshift{1.057797in}{1.997718in}%
\pgfsys@useobject{currentmarker}{}%
\end{pgfscope}%
\begin{pgfscope}%
\pgfsys@transformshift{1.079631in}{2.088347in}%
\pgfsys@useobject{currentmarker}{}%
\end{pgfscope}%
\begin{pgfscope}%
\pgfsys@transformshift{1.096534in}{2.254535in}%
\pgfsys@useobject{currentmarker}{}%
\end{pgfscope}%
\begin{pgfscope}%
\pgfsys@transformshift{1.117194in}{2.621046in}%
\pgfsys@useobject{currentmarker}{}%
\end{pgfscope}%
\begin{pgfscope}%
\pgfsys@transformshift{1.135273in}{2.861254in}%
\pgfsys@useobject{currentmarker}{}%
\end{pgfscope}%
\begin{pgfscope}%
\pgfsys@transformshift{1.155698in}{2.751343in}%
\pgfsys@useobject{currentmarker}{}%
\end{pgfscope}%
\begin{pgfscope}%
\pgfsys@transformshift{1.174949in}{2.385873in}%
\pgfsys@useobject{currentmarker}{}%
\end{pgfscope}%
\begin{pgfscope}%
\pgfsys@transformshift{1.194669in}{2.124360in}%
\pgfsys@useobject{currentmarker}{}%
\end{pgfscope}%
\begin{pgfscope}%
\pgfsys@transformshift{1.212981in}{2.019342in}%
\pgfsys@useobject{currentmarker}{}%
\end{pgfscope}%
\begin{pgfscope}%
\pgfsys@transformshift{1.231529in}{1.976580in}%
\pgfsys@useobject{currentmarker}{}%
\end{pgfscope}%
\begin{pgfscope}%
\pgfsys@transformshift{1.251014in}{2.009014in}%
\pgfsys@useobject{currentmarker}{}%
\end{pgfscope}%
\begin{pgfscope}%
\pgfsys@transformshift{1.270500in}{2.088784in}%
\pgfsys@useobject{currentmarker}{}%
\end{pgfscope}%
\begin{pgfscope}%
\pgfsys@transformshift{1.289753in}{2.260542in}%
\pgfsys@useobject{currentmarker}{}%
\end{pgfscope}%
\begin{pgfscope}%
\pgfsys@transformshift{1.308299in}{2.587399in}%
\pgfsys@useobject{currentmarker}{}%
\end{pgfscope}%
\begin{pgfscope}%
\pgfsys@transformshift{1.328724in}{2.843121in}%
\pgfsys@useobject{currentmarker}{}%
\end{pgfscope}%
\begin{pgfscope}%
\pgfsys@transformshift{1.346801in}{2.805648in}%
\pgfsys@useobject{currentmarker}{}%
\end{pgfscope}%
\begin{pgfscope}%
\pgfsys@transformshift{1.367932in}{2.437236in}%
\pgfsys@useobject{currentmarker}{}%
\end{pgfscope}%
\begin{pgfscope}%
\pgfsys@transformshift{1.385538in}{2.173490in}%
\pgfsys@useobject{currentmarker}{}%
\end{pgfscope}%
\begin{pgfscope}%
\pgfsys@transformshift{1.404555in}{2.044625in}%
\pgfsys@useobject{currentmarker}{}%
\end{pgfscope}%
\begin{pgfscope}%
\pgfsys@transformshift{1.424980in}{1.982124in}%
\pgfsys@useobject{currentmarker}{}%
\end{pgfscope}%
\begin{pgfscope}%
\pgfsys@transformshift{1.445875in}{1.985698in}%
\pgfsys@useobject{currentmarker}{}%
\end{pgfscope}%
\begin{pgfscope}%
\pgfsys@transformshift{1.464188in}{2.037272in}%
\pgfsys@useobject{currentmarker}{}%
\end{pgfscope}%
\begin{pgfscope}%
\pgfsys@transformshift{1.480856in}{2.086452in}%
\pgfsys@useobject{currentmarker}{}%
\end{pgfscope}%
\begin{pgfscope}%
\pgfsys@transformshift{1.501751in}{2.307272in}%
\pgfsys@useobject{currentmarker}{}%
\end{pgfscope}%
\begin{pgfscope}%
\pgfsys@transformshift{1.520064in}{2.650335in}%
\pgfsys@useobject{currentmarker}{}%
\end{pgfscope}%
\begin{pgfscope}%
\pgfsys@transformshift{1.542133in}{2.842349in}%
\pgfsys@useobject{currentmarker}{}%
\end{pgfscope}%
\begin{pgfscope}%
\pgfsys@transformshift{1.559741in}{2.737487in}%
\pgfsys@useobject{currentmarker}{}%
\end{pgfscope}%
\begin{pgfscope}%
\pgfsys@transformshift{1.578521in}{2.479836in}%
\pgfsys@useobject{currentmarker}{}%
\end{pgfscope}%
\begin{pgfscope}%
\pgfsys@transformshift{1.596835in}{2.217628in}%
\pgfsys@useobject{currentmarker}{}%
\end{pgfscope}%
\begin{pgfscope}%
\pgfsys@transformshift{1.615146in}{2.064622in}%
\pgfsys@useobject{currentmarker}{}%
\end{pgfscope}%
\begin{pgfscope}%
\pgfsys@transformshift{1.636511in}{2.791804in}%
\pgfsys@useobject{currentmarker}{}%
\end{pgfscope}%
\begin{pgfscope}%
\pgfsys@transformshift{1.654354in}{2.482725in}%
\pgfsys@useobject{currentmarker}{}%
\end{pgfscope}%
\begin{pgfscope}%
\pgfsys@transformshift{1.672902in}{2.177091in}%
\pgfsys@useobject{currentmarker}{}%
\end{pgfscope}%
\begin{pgfscope}%
\pgfsys@transformshift{1.690979in}{2.029625in}%
\pgfsys@useobject{currentmarker}{}%
\end{pgfscope}%
\begin{pgfscope}%
\pgfsys@transformshift{1.712576in}{1.980267in}%
\pgfsys@useobject{currentmarker}{}%
\end{pgfscope}%
\begin{pgfscope}%
\pgfsys@transformshift{1.732298in}{1.982163in}%
\pgfsys@useobject{currentmarker}{}%
\end{pgfscope}%
\begin{pgfscope}%
\pgfsys@transformshift{1.753193in}{2.047801in}%
\pgfsys@useobject{currentmarker}{}%
\end{pgfscope}%
\begin{pgfscope}%
\pgfsys@transformshift{1.771741in}{2.176794in}%
\pgfsys@useobject{currentmarker}{}%
\end{pgfscope}%
\begin{pgfscope}%
\pgfsys@transformshift{1.790286in}{2.442014in}%
\pgfsys@useobject{currentmarker}{}%
\end{pgfscope}%
\begin{pgfscope}%
\pgfsys@transformshift{1.810946in}{2.741944in}%
\pgfsys@useobject{currentmarker}{}%
\end{pgfscope}%
\begin{pgfscope}%
\pgfsys@transformshift{1.828789in}{2.824147in}%
\pgfsys@useobject{currentmarker}{}%
\end{pgfscope}%
\begin{pgfscope}%
\pgfsys@transformshift{1.846631in}{2.670439in}%
\pgfsys@useobject{currentmarker}{}%
\end{pgfscope}%
\begin{pgfscope}%
\pgfsys@transformshift{1.867528in}{2.324740in}%
\pgfsys@useobject{currentmarker}{}%
\end{pgfscope}%
\begin{pgfscope}%
\pgfsys@transformshift{1.885370in}{2.117700in}%
\pgfsys@useobject{currentmarker}{}%
\end{pgfscope}%
\begin{pgfscope}%
\pgfsys@transformshift{1.903447in}{2.020943in}%
\pgfsys@useobject{currentmarker}{}%
\end{pgfscope}%
\begin{pgfscope}%
\pgfsys@transformshift{1.925047in}{1.971756in}%
\pgfsys@useobject{currentmarker}{}%
\end{pgfscope}%
\begin{pgfscope}%
\pgfsys@transformshift{1.942655in}{1.980852in}%
\pgfsys@useobject{currentmarker}{}%
\end{pgfscope}%
\begin{pgfscope}%
\pgfsys@transformshift{1.963784in}{2.046040in}%
\pgfsys@useobject{currentmarker}{}%
\end{pgfscope}%
\begin{pgfscope}%
\pgfsys@transformshift{1.981861in}{2.155967in}%
\pgfsys@useobject{currentmarker}{}%
\end{pgfscope}%
\begin{pgfscope}%
\pgfsys@transformshift{2.002755in}{2.448865in}%
\pgfsys@useobject{currentmarker}{}%
\end{pgfscope}%
\begin{pgfscope}%
\pgfsys@transformshift{2.020129in}{2.745203in}%
\pgfsys@useobject{currentmarker}{}%
\end{pgfscope}%
\begin{pgfscope}%
\pgfsys@transformshift{2.041259in}{2.810874in}%
\pgfsys@useobject{currentmarker}{}%
\end{pgfscope}%
\begin{pgfscope}%
\pgfsys@transformshift{2.060274in}{2.597848in}%
\pgfsys@useobject{currentmarker}{}%
\end{pgfscope}%
\begin{pgfscope}%
\pgfsys@transformshift{2.078119in}{2.284571in}%
\pgfsys@useobject{currentmarker}{}%
\end{pgfscope}%
\begin{pgfscope}%
\pgfsys@transformshift{2.095021in}{2.100809in}%
\pgfsys@useobject{currentmarker}{}%
\end{pgfscope}%
\begin{pgfscope}%
\pgfsys@transformshift{2.117325in}{2.026113in}%
\pgfsys@useobject{currentmarker}{}%
\end{pgfscope}%
\begin{pgfscope}%
\pgfsys@transformshift{2.137281in}{1.978821in}%
\pgfsys@useobject{currentmarker}{}%
\end{pgfscope}%
\begin{pgfscope}%
\pgfsys@transformshift{2.156061in}{1.972970in}%
\pgfsys@useobject{currentmarker}{}%
\end{pgfscope}%
\begin{pgfscope}%
\pgfsys@transformshift{2.173669in}{2.010356in}%
\pgfsys@useobject{currentmarker}{}%
\end{pgfscope}%
\begin{pgfscope}%
\pgfsys@transformshift{2.194800in}{2.109954in}%
\pgfsys@useobject{currentmarker}{}%
\end{pgfscope}%
\begin{pgfscope}%
\pgfsys@transformshift{2.212877in}{2.287864in}%
\pgfsys@useobject{currentmarker}{}%
\end{pgfscope}%
\begin{pgfscope}%
\pgfsys@transformshift{2.230720in}{2.561800in}%
\pgfsys@useobject{currentmarker}{}%
\end{pgfscope}%
\begin{pgfscope}%
\pgfsys@transformshift{2.251614in}{2.809672in}%
\pgfsys@useobject{currentmarker}{}%
\end{pgfscope}%
\begin{pgfscope}%
\pgfsys@transformshift{2.268753in}{2.764431in}%
\pgfsys@useobject{currentmarker}{}%
\end{pgfscope}%
\begin{pgfscope}%
\pgfsys@transformshift{2.291056in}{2.429932in}%
\pgfsys@useobject{currentmarker}{}%
\end{pgfscope}%
\begin{pgfscope}%
\pgfsys@transformshift{2.309368in}{2.167799in}%
\pgfsys@useobject{currentmarker}{}%
\end{pgfscope}%
\begin{pgfscope}%
\pgfsys@transformshift{2.326741in}{2.047800in}%
\pgfsys@useobject{currentmarker}{}%
\end{pgfscope}%
\begin{pgfscope}%
\pgfsys@transformshift{2.347401in}{1.986980in}%
\pgfsys@useobject{currentmarker}{}%
\end{pgfscope}%
\begin{pgfscope}%
\pgfsys@transformshift{2.365480in}{1.966602in}%
\pgfsys@useobject{currentmarker}{}%
\end{pgfscope}%
\begin{pgfscope}%
\pgfsys@transformshift{2.387312in}{1.990493in}%
\pgfsys@useobject{currentmarker}{}%
\end{pgfscope}%
\begin{pgfscope}%
\pgfsys@transformshift{2.404686in}{2.047677in}%
\pgfsys@useobject{currentmarker}{}%
\end{pgfscope}%
\begin{pgfscope}%
\pgfsys@transformshift{2.422763in}{2.055356in}%
\pgfsys@useobject{currentmarker}{}%
\end{pgfscope}%
\begin{pgfscope}%
\pgfsys@transformshift{2.440136in}{2.131461in}%
\pgfsys@useobject{currentmarker}{}%
\end{pgfscope}%
\begin{pgfscope}%
\pgfsys@transformshift{2.461971in}{2.406733in}%
\pgfsys@useobject{currentmarker}{}%
\end{pgfscope}%
\begin{pgfscope}%
\pgfsys@transformshift{2.479110in}{2.707738in}%
\pgfsys@useobject{currentmarker}{}%
\end{pgfscope}%
\begin{pgfscope}%
\pgfsys@transformshift{2.503525in}{2.797143in}%
\pgfsys@useobject{currentmarker}{}%
\end{pgfscope}%
\begin{pgfscope}%
\pgfsys@transformshift{2.519255in}{2.754882in}%
\pgfsys@useobject{currentmarker}{}%
\end{pgfscope}%
\begin{pgfscope}%
\pgfsys@transformshift{2.543202in}{2.383102in}%
\pgfsys@useobject{currentmarker}{}%
\end{pgfscope}%
\begin{pgfscope}%
\pgfsys@transformshift{2.559401in}{2.167419in}%
\pgfsys@useobject{currentmarker}{}%
\end{pgfscope}%
\begin{pgfscope}%
\pgfsys@transformshift{2.576540in}{2.041407in}%
\pgfsys@useobject{currentmarker}{}%
\end{pgfscope}%
\begin{pgfscope}%
\pgfsys@transformshift{2.597669in}{1.981507in}%
\pgfsys@useobject{currentmarker}{}%
\end{pgfscope}%
\begin{pgfscope}%
\pgfsys@transformshift{2.615277in}{1.968915in}%
\pgfsys@useobject{currentmarker}{}%
\end{pgfscope}%
\begin{pgfscope}%
\pgfsys@transformshift{2.637814in}{2.007542in}%
\pgfsys@useobject{currentmarker}{}%
\end{pgfscope}%
\begin{pgfscope}%
\pgfsys@transformshift{2.652605in}{2.078845in}%
\pgfsys@useobject{currentmarker}{}%
\end{pgfscope}%
\begin{pgfscope}%
\pgfsys@transformshift{2.672562in}{2.213068in}%
\pgfsys@useobject{currentmarker}{}%
\end{pgfscope}%
\begin{pgfscope}%
\pgfsys@transformshift{2.693456in}{2.508442in}%
\pgfsys@useobject{currentmarker}{}%
\end{pgfscope}%
\begin{pgfscope}%
\pgfsys@transformshift{2.711533in}{2.763908in}%
\pgfsys@useobject{currentmarker}{}%
\end{pgfscope}%
\begin{pgfscope}%
\pgfsys@transformshift{2.732429in}{2.801330in}%
\pgfsys@useobject{currentmarker}{}%
\end{pgfscope}%
\begin{pgfscope}%
\pgfsys@transformshift{2.750506in}{2.583437in}%
\pgfsys@useobject{currentmarker}{}%
\end{pgfscope}%
\begin{pgfscope}%
\pgfsys@transformshift{2.771166in}{2.815529in}%
\pgfsys@useobject{currentmarker}{}%
\end{pgfscope}%
\begin{pgfscope}%
\pgfsys@transformshift{2.789712in}{2.756020in}%
\pgfsys@useobject{currentmarker}{}%
\end{pgfscope}%
\begin{pgfscope}%
\pgfsys@transformshift{2.807320in}{2.464057in}%
\pgfsys@useobject{currentmarker}{}%
\end{pgfscope}%
\begin{pgfscope}%
\pgfsys@transformshift{2.828920in}{2.478586in}%
\pgfsys@useobject{currentmarker}{}%
\end{pgfscope}%
\begin{pgfscope}%
\pgfsys@transformshift{2.846997in}{2.225158in}%
\pgfsys@useobject{currentmarker}{}%
\end{pgfscope}%
\begin{pgfscope}%
\pgfsys@transformshift{2.865310in}{2.065497in}%
\pgfsys@useobject{currentmarker}{}%
\end{pgfscope}%
\begin{pgfscope}%
\pgfsys@transformshift{2.882684in}{1.993579in}%
\pgfsys@useobject{currentmarker}{}%
\end{pgfscope}%
\begin{pgfscope}%
\pgfsys@transformshift{2.905221in}{1.973145in}%
\pgfsys@useobject{currentmarker}{}%
\end{pgfscope}%
\begin{pgfscope}%
\pgfsys@transformshift{2.923298in}{1.977202in}%
\pgfsys@useobject{currentmarker}{}%
\end{pgfscope}%
\begin{pgfscope}%
\pgfsys@transformshift{2.943489in}{2.032226in}%
\pgfsys@useobject{currentmarker}{}%
\end{pgfscope}%
\begin{pgfscope}%
\pgfsys@transformshift{2.961801in}{2.131230in}%
\pgfsys@useobject{currentmarker}{}%
\end{pgfscope}%
\begin{pgfscope}%
\pgfsys@transformshift{2.982695in}{2.215746in}%
\pgfsys@useobject{currentmarker}{}%
\end{pgfscope}%
\begin{pgfscope}%
\pgfsys@transformshift{3.001477in}{2.498545in}%
\pgfsys@useobject{currentmarker}{}%
\end{pgfscope}%
\begin{pgfscope}%
\pgfsys@transformshift{3.018851in}{2.751030in}%
\pgfsys@useobject{currentmarker}{}%
\end{pgfscope}%
\begin{pgfscope}%
\pgfsys@transformshift{3.040214in}{2.810932in}%
\pgfsys@useobject{currentmarker}{}%
\end{pgfscope}%
\begin{pgfscope}%
\pgfsys@transformshift{3.057353in}{2.586587in}%
\pgfsys@useobject{currentmarker}{}%
\end{pgfscope}%
\begin{pgfscope}%
\pgfsys@transformshift{3.075901in}{2.284188in}%
\pgfsys@useobject{currentmarker}{}%
\end{pgfscope}%
\begin{pgfscope}%
\pgfsys@transformshift{3.097030in}{2.078763in}%
\pgfsys@useobject{currentmarker}{}%
\end{pgfscope}%
\begin{pgfscope}%
\pgfsys@transformshift{3.115578in}{2.005452in}%
\pgfsys@useobject{currentmarker}{}%
\end{pgfscope}%
\begin{pgfscope}%
\pgfsys@transformshift{3.133889in}{1.981581in}%
\pgfsys@useobject{currentmarker}{}%
\end{pgfscope}%
\begin{pgfscope}%
\pgfsys@transformshift{3.154315in}{1.972786in}%
\pgfsys@useobject{currentmarker}{}%
\end{pgfscope}%
\begin{pgfscope}%
\pgfsys@transformshift{3.173097in}{2.010962in}%
\pgfsys@useobject{currentmarker}{}%
\end{pgfscope}%
\begin{pgfscope}%
\pgfsys@transformshift{3.194460in}{2.101149in}%
\pgfsys@useobject{currentmarker}{}%
\end{pgfscope}%
\begin{pgfscope}%
\pgfsys@transformshift{3.212303in}{2.248058in}%
\pgfsys@useobject{currentmarker}{}%
\end{pgfscope}%
\begin{pgfscope}%
\pgfsys@transformshift{3.232728in}{2.561242in}%
\pgfsys@useobject{currentmarker}{}%
\end{pgfscope}%
\begin{pgfscope}%
\pgfsys@transformshift{3.251276in}{2.792536in}%
\pgfsys@useobject{currentmarker}{}%
\end{pgfscope}%
\begin{pgfscope}%
\pgfsys@transformshift{3.269353in}{2.841531in}%
\pgfsys@useobject{currentmarker}{}%
\end{pgfscope}%
\begin{pgfscope}%
\pgfsys@transformshift{3.291187in}{2.670444in}%
\pgfsys@useobject{currentmarker}{}%
\end{pgfscope}%
\begin{pgfscope}%
\pgfsys@transformshift{3.308090in}{2.438834in}%
\pgfsys@useobject{currentmarker}{}%
\end{pgfscope}%
\begin{pgfscope}%
\pgfsys@transformshift{3.325698in}{2.214312in}%
\pgfsys@useobject{currentmarker}{}%
\end{pgfscope}%
\begin{pgfscope}%
\pgfsys@transformshift{3.347298in}{2.058903in}%
\pgfsys@useobject{currentmarker}{}%
\end{pgfscope}%
\begin{pgfscope}%
\pgfsys@transformshift{3.365844in}{1.998390in}%
\pgfsys@useobject{currentmarker}{}%
\end{pgfscope}%
\begin{pgfscope}%
\pgfsys@transformshift{3.386035in}{1.973498in}%
\pgfsys@useobject{currentmarker}{}%
\end{pgfscope}%
\begin{pgfscope}%
\pgfsys@transformshift{3.404817in}{1.987439in}%
\pgfsys@useobject{currentmarker}{}%
\end{pgfscope}%
\begin{pgfscope}%
\pgfsys@transformshift{3.422894in}{2.014764in}%
\pgfsys@useobject{currentmarker}{}%
\end{pgfscope}%
\begin{pgfscope}%
\pgfsys@transformshift{3.441442in}{2.091148in}%
\pgfsys@useobject{currentmarker}{}%
\end{pgfscope}%
\begin{pgfscope}%
\pgfsys@transformshift{3.461867in}{2.203190in}%
\pgfsys@useobject{currentmarker}{}%
\end{pgfscope}%
\begin{pgfscope}%
\pgfsys@transformshift{3.480413in}{2.427692in}%
\pgfsys@useobject{currentmarker}{}%
\end{pgfscope}%
\begin{pgfscope}%
\pgfsys@transformshift{3.498490in}{2.746311in}%
\pgfsys@useobject{currentmarker}{}%
\end{pgfscope}%
\begin{pgfscope}%
\pgfsys@transformshift{3.519855in}{2.858036in}%
\pgfsys@useobject{currentmarker}{}%
\end{pgfscope}%
\begin{pgfscope}%
\pgfsys@transformshift{3.538167in}{2.764817in}%
\pgfsys@useobject{currentmarker}{}%
\end{pgfscope}%
\begin{pgfscope}%
\pgfsys@transformshift{3.557183in}{2.510815in}%
\pgfsys@useobject{currentmarker}{}%
\end{pgfscope}%
\begin{pgfscope}%
\pgfsys@transformshift{3.579957in}{2.235621in}%
\pgfsys@useobject{currentmarker}{}%
\end{pgfscope}%
\begin{pgfscope}%
\pgfsys@transformshift{3.594513in}{2.108889in}%
\pgfsys@useobject{currentmarker}{}%
\end{pgfscope}%
\begin{pgfscope}%
\pgfsys@transformshift{3.615173in}{2.031030in}%
\pgfsys@useobject{currentmarker}{}%
\end{pgfscope}%
\begin{pgfscope}%
\pgfsys@transformshift{3.633485in}{1.991707in}%
\pgfsys@useobject{currentmarker}{}%
\end{pgfscope}%
\begin{pgfscope}%
\pgfsys@transformshift{3.653441in}{1.974772in}%
\pgfsys@useobject{currentmarker}{}%
\end{pgfscope}%
\begin{pgfscope}%
\pgfsys@transformshift{3.674336in}{2.000844in}%
\pgfsys@useobject{currentmarker}{}%
\end{pgfscope}%
\begin{pgfscope}%
\pgfsys@transformshift{3.693587in}{2.063402in}%
\pgfsys@useobject{currentmarker}{}%
\end{pgfscope}%
\begin{pgfscope}%
\pgfsys@transformshift{3.712369in}{2.186552in}%
\pgfsys@useobject{currentmarker}{}%
\end{pgfscope}%
\begin{pgfscope}%
\pgfsys@transformshift{3.729272in}{2.291425in}%
\pgfsys@useobject{currentmarker}{}%
\end{pgfscope}%
\begin{pgfscope}%
\pgfsys@transformshift{3.751341in}{2.565407in}%
\pgfsys@useobject{currentmarker}{}%
\end{pgfscope}%
\begin{pgfscope}%
\pgfsys@transformshift{3.769654in}{2.810081in}%
\pgfsys@useobject{currentmarker}{}%
\end{pgfscope}%
\begin{pgfscope}%
\pgfsys@transformshift{3.789140in}{2.877458in}%
\pgfsys@useobject{currentmarker}{}%
\end{pgfscope}%
\begin{pgfscope}%
\pgfsys@transformshift{3.807922in}{2.785497in}%
\pgfsys@useobject{currentmarker}{}%
\end{pgfscope}%
\begin{pgfscope}%
\pgfsys@transformshift{3.827407in}{2.528908in}%
\pgfsys@useobject{currentmarker}{}%
\end{pgfscope}%
\begin{pgfscope}%
\pgfsys@transformshift{3.846893in}{2.606098in}%
\pgfsys@useobject{currentmarker}{}%
\end{pgfscope}%
\begin{pgfscope}%
\pgfsys@transformshift{3.866144in}{2.884906in}%
\pgfsys@useobject{currentmarker}{}%
\end{pgfscope}%
\begin{pgfscope}%
\pgfsys@transformshift{3.884692in}{2.763293in}%
\pgfsys@useobject{currentmarker}{}%
\end{pgfscope}%
\begin{pgfscope}%
\pgfsys@transformshift{3.904412in}{2.461076in}%
\pgfsys@useobject{currentmarker}{}%
\end{pgfscope}%
\begin{pgfscope}%
\pgfsys@transformshift{3.923429in}{2.209965in}%
\pgfsys@useobject{currentmarker}{}%
\end{pgfscope}%
\begin{pgfscope}%
\pgfsys@transformshift{3.942915in}{2.097385in}%
\pgfsys@useobject{currentmarker}{}%
\end{pgfscope}%
\begin{pgfscope}%
\pgfsys@transformshift{3.961228in}{2.017602in}%
\pgfsys@useobject{currentmarker}{}%
\end{pgfscope}%
\begin{pgfscope}%
\pgfsys@transformshift{3.980714in}{1.982940in}%
\pgfsys@useobject{currentmarker}{}%
\end{pgfscope}%
\begin{pgfscope}%
\pgfsys@transformshift{3.999496in}{1.988788in}%
\pgfsys@useobject{currentmarker}{}%
\end{pgfscope}%
\begin{pgfscope}%
\pgfsys@transformshift{4.019920in}{2.036002in}%
\pgfsys@useobject{currentmarker}{}%
\end{pgfscope}%
\begin{pgfscope}%
\pgfsys@transformshift{4.037528in}{2.114393in}%
\pgfsys@useobject{currentmarker}{}%
\end{pgfscope}%
\begin{pgfscope}%
\pgfsys@transformshift{4.056779in}{2.265672in}%
\pgfsys@useobject{currentmarker}{}%
\end{pgfscope}%
\begin{pgfscope}%
\pgfsys@transformshift{4.076030in}{2.478805in}%
\pgfsys@useobject{currentmarker}{}%
\end{pgfscope}%
\begin{pgfscope}%
\pgfsys@transformshift{4.094343in}{2.789206in}%
\pgfsys@useobject{currentmarker}{}%
\end{pgfscope}%
\begin{pgfscope}%
\pgfsys@transformshift{4.114534in}{2.910499in}%
\pgfsys@useobject{currentmarker}{}%
\end{pgfscope}%
\begin{pgfscope}%
\pgfsys@transformshift{4.132611in}{2.851060in}%
\pgfsys@useobject{currentmarker}{}%
\end{pgfscope}%
\begin{pgfscope}%
\pgfsys@transformshift{4.154914in}{2.595684in}%
\pgfsys@useobject{currentmarker}{}%
\end{pgfscope}%
\begin{pgfscope}%
\pgfsys@transformshift{4.170645in}{2.370203in}%
\pgfsys@useobject{currentmarker}{}%
\end{pgfscope}%
\begin{pgfscope}%
\pgfsys@transformshift{4.192008in}{2.144727in}%
\pgfsys@useobject{currentmarker}{}%
\end{pgfscope}%
\begin{pgfscope}%
\pgfsys@transformshift{4.211730in}{2.059452in}%
\pgfsys@useobject{currentmarker}{}%
\end{pgfscope}%
\begin{pgfscope}%
\pgfsys@transformshift{4.230276in}{2.014245in}%
\pgfsys@useobject{currentmarker}{}%
\end{pgfscope}%
\begin{pgfscope}%
\pgfsys@transformshift{4.248119in}{1.984981in}%
\pgfsys@useobject{currentmarker}{}%
\end{pgfscope}%
\begin{pgfscope}%
\pgfsys@transformshift{4.265961in}{2.002096in}%
\pgfsys@useobject{currentmarker}{}%
\end{pgfscope}%
\begin{pgfscope}%
\pgfsys@transformshift{4.290378in}{2.048298in}%
\pgfsys@useobject{currentmarker}{}%
\end{pgfscope}%
\begin{pgfscope}%
\pgfsys@transformshift{4.308926in}{2.147895in}%
\pgfsys@useobject{currentmarker}{}%
\end{pgfscope}%
\begin{pgfscope}%
\pgfsys@transformshift{4.327237in}{2.281444in}%
\pgfsys@useobject{currentmarker}{}%
\end{pgfscope}%
\begin{pgfscope}%
\pgfsys@transformshift{4.346489in}{2.472941in}%
\pgfsys@useobject{currentmarker}{}%
\end{pgfscope}%
\begin{pgfscope}%
\pgfsys@transformshift{4.365271in}{2.791534in}%
\pgfsys@useobject{currentmarker}{}%
\end{pgfscope}%
\begin{pgfscope}%
\pgfsys@transformshift{4.384288in}{2.929482in}%
\pgfsys@useobject{currentmarker}{}%
\end{pgfscope}%
\begin{pgfscope}%
\pgfsys@transformshift{4.402834in}{2.912287in}%
\pgfsys@useobject{currentmarker}{}%
\end{pgfscope}%
\begin{pgfscope}%
\pgfsys@transformshift{4.421850in}{2.781190in}%
\pgfsys@useobject{currentmarker}{}%
\end{pgfscope}%
\begin{pgfscope}%
\pgfsys@transformshift{4.441102in}{2.504664in}%
\pgfsys@useobject{currentmarker}{}%
\end{pgfscope}%
\begin{pgfscope}%
\pgfsys@transformshift{4.462467in}{2.213896in}%
\pgfsys@useobject{currentmarker}{}%
\end{pgfscope}%
\begin{pgfscope}%
\pgfsys@transformshift{4.481718in}{2.107629in}%
\pgfsys@useobject{currentmarker}{}%
\end{pgfscope}%
\begin{pgfscope}%
\pgfsys@transformshift{4.482421in}{2.106069in}%
\pgfsys@useobject{currentmarker}{}%
\end{pgfscope}%
\begin{pgfscope}%
\pgfsys@transformshift{4.474674in}{2.166349in}%
\pgfsys@useobject{currentmarker}{}%
\end{pgfscope}%
\begin{pgfscope}%
\pgfsys@transformshift{4.455658in}{2.469153in}%
\pgfsys@useobject{currentmarker}{}%
\end{pgfscope}%
\begin{pgfscope}%
\pgfsys@transformshift{4.432180in}{2.890613in}%
\pgfsys@useobject{currentmarker}{}%
\end{pgfscope}%
\begin{pgfscope}%
\pgfsys@transformshift{4.416921in}{2.936309in}%
\pgfsys@useobject{currentmarker}{}%
\end{pgfscope}%
\begin{pgfscope}%
\pgfsys@transformshift{4.393678in}{2.576460in}%
\pgfsys@useobject{currentmarker}{}%
\end{pgfscope}%
\begin{pgfscope}%
\pgfsys@transformshift{4.377244in}{2.282903in}%
\pgfsys@useobject{currentmarker}{}%
\end{pgfscope}%
\begin{pgfscope}%
\pgfsys@transformshift{4.356350in}{2.078401in}%
\pgfsys@useobject{currentmarker}{}%
\end{pgfscope}%
\begin{pgfscope}%
\pgfsys@transformshift{4.338507in}{1.999800in}%
\pgfsys@useobject{currentmarker}{}%
\end{pgfscope}%
\begin{pgfscope}%
\pgfsys@transformshift{4.319959in}{1.993070in}%
\pgfsys@useobject{currentmarker}{}%
\end{pgfscope}%
\begin{pgfscope}%
\pgfsys@transformshift{4.302351in}{2.065375in}%
\pgfsys@useobject{currentmarker}{}%
\end{pgfscope}%
\begin{pgfscope}%
\pgfsys@transformshift{4.276291in}{2.360378in}%
\pgfsys@useobject{currentmarker}{}%
\end{pgfscope}%
\begin{pgfscope}%
\pgfsys@transformshift{4.261971in}{2.667585in}%
\pgfsys@useobject{currentmarker}{}%
\end{pgfscope}%
\begin{pgfscope}%
\pgfsys@transformshift{4.242484in}{2.905087in}%
\pgfsys@useobject{currentmarker}{}%
\end{pgfscope}%
\begin{pgfscope}%
\pgfsys@transformshift{4.224172in}{2.823330in}%
\pgfsys@useobject{currentmarker}{}%
\end{pgfscope}%
\begin{pgfscope}%
\pgfsys@transformshift{4.205156in}{2.445397in}%
\pgfsys@useobject{currentmarker}{}%
\end{pgfscope}%
\begin{pgfscope}%
\pgfsys@transformshift{4.184027in}{2.137838in}%
\pgfsys@useobject{currentmarker}{}%
\end{pgfscope}%
\begin{pgfscope}%
\pgfsys@transformshift{4.165244in}{2.020709in}%
\pgfsys@useobject{currentmarker}{}%
\end{pgfscope}%
\begin{pgfscope}%
\pgfsys@transformshift{4.146697in}{1.980344in}%
\pgfsys@useobject{currentmarker}{}%
\end{pgfscope}%
\begin{pgfscope}%
\pgfsys@transformshift{4.127916in}{2.018392in}%
\pgfsys@useobject{currentmarker}{}%
\end{pgfscope}%
\begin{pgfscope}%
\pgfsys@transformshift{4.108900in}{2.130032in}%
\pgfsys@useobject{currentmarker}{}%
\end{pgfscope}%
\begin{pgfscope}%
\pgfsys@transformshift{4.090821in}{2.380748in}%
\pgfsys@useobject{currentmarker}{}%
\end{pgfscope}%
\begin{pgfscope}%
\pgfsys@transformshift{4.069223in}{2.801391in}%
\pgfsys@useobject{currentmarker}{}%
\end{pgfscope}%
\begin{pgfscope}%
\pgfsys@transformshift{4.049972in}{2.883772in}%
\pgfsys@useobject{currentmarker}{}%
\end{pgfscope}%
\begin{pgfscope}%
\pgfsys@transformshift{4.034476in}{2.692796in}%
\pgfsys@useobject{currentmarker}{}%
\end{pgfscope}%
\begin{pgfscope}%
\pgfsys@transformshift{4.012878in}{2.276012in}%
\pgfsys@useobject{currentmarker}{}%
\end{pgfscope}%
\begin{pgfscope}%
\pgfsys@transformshift{3.991513in}{2.068770in}%
\pgfsys@useobject{currentmarker}{}%
\end{pgfscope}%
\begin{pgfscope}%
\pgfsys@transformshift{3.975548in}{2.001877in}%
\pgfsys@useobject{currentmarker}{}%
\end{pgfscope}%
\begin{pgfscope}%
\pgfsys@transformshift{3.957236in}{1.977739in}%
\pgfsys@useobject{currentmarker}{}%
\end{pgfscope}%
\begin{pgfscope}%
\pgfsys@transformshift{3.936342in}{2.037877in}%
\pgfsys@useobject{currentmarker}{}%
\end{pgfscope}%
\begin{pgfscope}%
\pgfsys@transformshift{3.916620in}{2.180386in}%
\pgfsys@useobject{currentmarker}{}%
\end{pgfscope}%
\begin{pgfscope}%
\pgfsys@transformshift{3.898074in}{2.484541in}%
\pgfsys@useobject{currentmarker}{}%
\end{pgfscope}%
\begin{pgfscope}%
\pgfsys@transformshift{3.875535in}{2.842538in}%
\pgfsys@useobject{currentmarker}{}%
\end{pgfscope}%
\begin{pgfscope}%
\pgfsys@transformshift{3.857927in}{2.819853in}%
\pgfsys@useobject{currentmarker}{}%
\end{pgfscope}%
\begin{pgfscope}%
\pgfsys@transformshift{3.842198in}{2.541051in}%
\pgfsys@useobject{currentmarker}{}%
\end{pgfscope}%
\begin{pgfscope}%
\pgfsys@transformshift{3.817546in}{2.147333in}%
\pgfsys@useobject{currentmarker}{}%
\end{pgfscope}%
\begin{pgfscope}%
\pgfsys@transformshift{3.801347in}{2.057177in}%
\pgfsys@useobject{currentmarker}{}%
\end{pgfscope}%
\begin{pgfscope}%
\pgfsys@transformshift{3.783505in}{1.987538in}%
\pgfsys@useobject{currentmarker}{}%
\end{pgfscope}%
\begin{pgfscope}%
\pgfsys@transformshift{3.765897in}{1.978054in}%
\pgfsys@useobject{currentmarker}{}%
\end{pgfscope}%
\begin{pgfscope}%
\pgfsys@transformshift{3.743123in}{2.043593in}%
\pgfsys@useobject{currentmarker}{}%
\end{pgfscope}%
\begin{pgfscope}%
\pgfsys@transformshift{3.724108in}{2.182260in}%
\pgfsys@useobject{currentmarker}{}%
\end{pgfscope}%
\begin{pgfscope}%
\pgfsys@transformshift{3.705794in}{2.482503in}%
\pgfsys@useobject{currentmarker}{}%
\end{pgfscope}%
\begin{pgfscope}%
\pgfsys@transformshift{3.687012in}{2.800152in}%
\pgfsys@useobject{currentmarker}{}%
\end{pgfscope}%
\begin{pgfscope}%
\pgfsys@transformshift{3.668701in}{2.850262in}%
\pgfsys@useobject{currentmarker}{}%
\end{pgfscope}%
\begin{pgfscope}%
\pgfsys@transformshift{3.646632in}{2.588113in}%
\pgfsys@useobject{currentmarker}{}%
\end{pgfscope}%
\begin{pgfscope}%
\pgfsys@transformshift{3.627615in}{2.269585in}%
\pgfsys@useobject{currentmarker}{}%
\end{pgfscope}%
\begin{pgfscope}%
\pgfsys@transformshift{3.610242in}{2.101143in}%
\pgfsys@useobject{currentmarker}{}%
\end{pgfscope}%
\begin{pgfscope}%
\pgfsys@transformshift{3.588410in}{2.013661in}%
\pgfsys@useobject{currentmarker}{}%
\end{pgfscope}%
\begin{pgfscope}%
\pgfsys@transformshift{3.572210in}{1.974027in}%
\pgfsys@useobject{currentmarker}{}%
\end{pgfscope}%
\begin{pgfscope}%
\pgfsys@transformshift{3.550845in}{1.981319in}%
\pgfsys@useobject{currentmarker}{}%
\end{pgfscope}%
\begin{pgfscope}%
\pgfsys@transformshift{3.531594in}{2.032930in}%
\pgfsys@useobject{currentmarker}{}%
\end{pgfscope}%
\begin{pgfscope}%
\pgfsys@transformshift{3.513046in}{2.147659in}%
\pgfsys@useobject{currentmarker}{}%
\end{pgfscope}%
\begin{pgfscope}%
\pgfsys@transformshift{3.494735in}{2.384712in}%
\pgfsys@useobject{currentmarker}{}%
\end{pgfscope}%
\begin{pgfscope}%
\pgfsys@transformshift{3.474309in}{2.770860in}%
\pgfsys@useobject{currentmarker}{}%
\end{pgfscope}%
\begin{pgfscope}%
\pgfsys@transformshift{3.455527in}{2.839872in}%
\pgfsys@useobject{currentmarker}{}%
\end{pgfscope}%
\begin{pgfscope}%
\pgfsys@transformshift{3.436510in}{2.662549in}%
\pgfsys@useobject{currentmarker}{}%
\end{pgfscope}%
\begin{pgfscope}%
\pgfsys@transformshift{3.417730in}{2.329913in}%
\pgfsys@useobject{currentmarker}{}%
\end{pgfscope}%
\begin{pgfscope}%
\pgfsys@transformshift{3.398713in}{2.134150in}%
\pgfsys@useobject{currentmarker}{}%
\end{pgfscope}%
\begin{pgfscope}%
\pgfsys@transformshift{3.379931in}{2.028097in}%
\pgfsys@useobject{currentmarker}{}%
\end{pgfscope}%
\begin{pgfscope}%
\pgfsys@transformshift{3.359271in}{1.972517in}%
\pgfsys@useobject{currentmarker}{}%
\end{pgfscope}%
\begin{pgfscope}%
\pgfsys@transformshift{3.339551in}{1.978233in}%
\pgfsys@useobject{currentmarker}{}%
\end{pgfscope}%
\begin{pgfscope}%
\pgfsys@transformshift{3.320299in}{2.029715in}%
\pgfsys@useobject{currentmarker}{}%
\end{pgfscope}%
\begin{pgfscope}%
\pgfsys@transformshift{3.302221in}{2.147548in}%
\pgfsys@useobject{currentmarker}{}%
\end{pgfscope}%
\begin{pgfscope}%
\pgfsys@transformshift{3.279917in}{2.442826in}%
\pgfsys@useobject{currentmarker}{}%
\end{pgfscope}%
\begin{pgfscope}%
\pgfsys@transformshift{3.261372in}{2.715644in}%
\pgfsys@useobject{currentmarker}{}%
\end{pgfscope}%
\begin{pgfscope}%
\pgfsys@transformshift{3.242589in}{2.827523in}%
\pgfsys@useobject{currentmarker}{}%
\end{pgfscope}%
\begin{pgfscope}%
\pgfsys@transformshift{3.224276in}{2.682577in}%
\pgfsys@useobject{currentmarker}{}%
\end{pgfscope}%
\begin{pgfscope}%
\pgfsys@transformshift{3.205496in}{2.393600in}%
\pgfsys@useobject{currentmarker}{}%
\end{pgfscope}%
\begin{pgfscope}%
\pgfsys@transformshift{3.186713in}{2.172414in}%
\pgfsys@useobject{currentmarker}{}%
\end{pgfscope}%
\begin{pgfscope}%
\pgfsys@transformshift{3.168636in}{2.070702in}%
\pgfsys@useobject{currentmarker}{}%
\end{pgfscope}%
\begin{pgfscope}%
\pgfsys@transformshift{3.146802in}{1.989842in}%
\pgfsys@useobject{currentmarker}{}%
\end{pgfscope}%
\begin{pgfscope}%
\pgfsys@transformshift{3.128489in}{1.968481in}%
\pgfsys@useobject{currentmarker}{}%
\end{pgfscope}%
\begin{pgfscope}%
\pgfsys@transformshift{3.108769in}{1.998210in}%
\pgfsys@useobject{currentmarker}{}%
\end{pgfscope}%
\begin{pgfscope}%
\pgfsys@transformshift{3.091161in}{2.066629in}%
\pgfsys@useobject{currentmarker}{}%
\end{pgfscope}%
\begin{pgfscope}%
\pgfsys@transformshift{3.072613in}{2.237218in}%
\pgfsys@useobject{currentmarker}{}%
\end{pgfscope}%
\begin{pgfscope}%
\pgfsys@transformshift{3.050781in}{2.573887in}%
\pgfsys@useobject{currentmarker}{}%
\end{pgfscope}%
\begin{pgfscope}%
\pgfsys@transformshift{3.035754in}{2.806111in}%
\pgfsys@useobject{currentmarker}{}%
\end{pgfscope}%
\begin{pgfscope}%
\pgfsys@transformshift{3.013450in}{2.806679in}%
\pgfsys@useobject{currentmarker}{}%
\end{pgfscope}%
\begin{pgfscope}%
\pgfsys@transformshift{2.995373in}{2.768453in}%
\pgfsys@useobject{currentmarker}{}%
\end{pgfscope}%
\begin{pgfscope}%
\pgfsys@transformshift{2.976357in}{2.479192in}%
\pgfsys@useobject{currentmarker}{}%
\end{pgfscope}%
\begin{pgfscope}%
\pgfsys@transformshift{2.956871in}{2.206441in}%
\pgfsys@useobject{currentmarker}{}%
\end{pgfscope}%
\begin{pgfscope}%
\pgfsys@transformshift{2.938794in}{2.069816in}%
\pgfsys@useobject{currentmarker}{}%
\end{pgfscope}%
\begin{pgfscope}%
\pgfsys@transformshift{2.916491in}{1.992513in}%
\pgfsys@useobject{currentmarker}{}%
\end{pgfscope}%
\begin{pgfscope}%
\pgfsys@transformshift{2.899352in}{1.970919in}%
\pgfsys@useobject{currentmarker}{}%
\end{pgfscope}%
\begin{pgfscope}%
\pgfsys@transformshift{2.877518in}{1.977642in}%
\pgfsys@useobject{currentmarker}{}%
\end{pgfscope}%
\begin{pgfscope}%
\pgfsys@transformshift{2.859206in}{2.011348in}%
\pgfsys@useobject{currentmarker}{}%
\end{pgfscope}%
\begin{pgfscope}%
\pgfsys@transformshift{2.839719in}{2.062987in}%
\pgfsys@useobject{currentmarker}{}%
\end{pgfscope}%
\begin{pgfscope}%
\pgfsys@transformshift{2.822816in}{2.194806in}%
\pgfsys@useobject{currentmarker}{}%
\end{pgfscope}%
\begin{pgfscope}%
\pgfsys@transformshift{2.802625in}{2.517219in}%
\pgfsys@useobject{currentmarker}{}%
\end{pgfscope}%
\begin{pgfscope}%
\pgfsys@transformshift{2.782200in}{2.779161in}%
\pgfsys@useobject{currentmarker}{}%
\end{pgfscope}%
\begin{pgfscope}%
\pgfsys@transformshift{2.763419in}{2.796048in}%
\pgfsys@useobject{currentmarker}{}%
\end{pgfscope}%
\begin{pgfscope}%
\pgfsys@transformshift{2.746749in}{2.628643in}%
\pgfsys@useobject{currentmarker}{}%
\end{pgfscope}%
\begin{pgfscope}%
\pgfsys@transformshift{2.725151in}{2.300569in}%
\pgfsys@useobject{currentmarker}{}%
\end{pgfscope}%
\begin{pgfscope}%
\pgfsys@transformshift{2.706134in}{2.114755in}%
\pgfsys@useobject{currentmarker}{}%
\end{pgfscope}%
\begin{pgfscope}%
\pgfsys@transformshift{2.687587in}{2.090725in}%
\pgfsys@useobject{currentmarker}{}%
\end{pgfscope}%
\begin{pgfscope}%
\pgfsys@transformshift{2.687821in}{2.034539in}%
\pgfsys@useobject{currentmarker}{}%
\end{pgfscope}%
\begin{pgfscope}%
\pgfsys@transformshift{2.669979in}{2.014720in}%
\pgfsys@useobject{currentmarker}{}%
\end{pgfscope}%
\begin{pgfscope}%
\pgfsys@transformshift{2.647207in}{1.971209in}%
\pgfsys@useobject{currentmarker}{}%
\end{pgfscope}%
\begin{pgfscope}%
\pgfsys@transformshift{2.628424in}{1.979925in}%
\pgfsys@useobject{currentmarker}{}%
\end{pgfscope}%
\begin{pgfscope}%
\pgfsys@transformshift{2.610582in}{2.025873in}%
\pgfsys@useobject{currentmarker}{}%
\end{pgfscope}%
\begin{pgfscope}%
\pgfsys@transformshift{2.591800in}{2.103393in}%
\pgfsys@useobject{currentmarker}{}%
\end{pgfscope}%
\begin{pgfscope}%
\pgfsys@transformshift{2.574897in}{2.296187in}%
\pgfsys@useobject{currentmarker}{}%
\end{pgfscope}%
\begin{pgfscope}%
\pgfsys@transformshift{2.551185in}{2.519233in}%
\pgfsys@useobject{currentmarker}{}%
\end{pgfscope}%
\begin{pgfscope}%
\pgfsys@transformshift{2.532637in}{2.772056in}%
\pgfsys@useobject{currentmarker}{}%
\end{pgfscope}%
\begin{pgfscope}%
\pgfsys@transformshift{2.513855in}{2.786284in}%
\pgfsys@useobject{currentmarker}{}%
\end{pgfscope}%
\begin{pgfscope}%
\pgfsys@transformshift{2.497187in}{2.578502in}%
\pgfsys@useobject{currentmarker}{}%
\end{pgfscope}%
\begin{pgfscope}%
\pgfsys@transformshift{2.473709in}{2.227167in}%
\pgfsys@useobject{currentmarker}{}%
\end{pgfscope}%
\begin{pgfscope}%
\pgfsys@transformshift{2.457276in}{2.081266in}%
\pgfsys@useobject{currentmarker}{}%
\end{pgfscope}%
\begin{pgfscope}%
\pgfsys@transformshift{2.437085in}{2.001664in}%
\pgfsys@useobject{currentmarker}{}%
\end{pgfscope}%
\begin{pgfscope}%
\pgfsys@transformshift{2.418773in}{2.001393in}%
\pgfsys@useobject{currentmarker}{}%
\end{pgfscope}%
\begin{pgfscope}%
\pgfsys@transformshift{2.396939in}{1.974453in}%
\pgfsys@useobject{currentmarker}{}%
\end{pgfscope}%
\begin{pgfscope}%
\pgfsys@transformshift{2.378391in}{1.971331in}%
\pgfsys@useobject{currentmarker}{}%
\end{pgfscope}%
\begin{pgfscope}%
\pgfsys@transformshift{2.360783in}{2.008471in}%
\pgfsys@useobject{currentmarker}{}%
\end{pgfscope}%
\begin{pgfscope}%
\pgfsys@transformshift{2.338480in}{2.074109in}%
\pgfsys@useobject{currentmarker}{}%
\end{pgfscope}%
\begin{pgfscope}%
\pgfsys@transformshift{2.322515in}{2.206983in}%
\pgfsys@useobject{currentmarker}{}%
\end{pgfscope}%
\begin{pgfscope}%
\pgfsys@transformshift{2.300683in}{2.578304in}%
\pgfsys@useobject{currentmarker}{}%
\end{pgfscope}%
\begin{pgfscope}%
\pgfsys@transformshift{2.278614in}{2.789302in}%
\pgfsys@useobject{currentmarker}{}%
\end{pgfscope}%
\begin{pgfscope}%
\pgfsys@transformshift{2.263353in}{2.804336in}%
\pgfsys@useobject{currentmarker}{}%
\end{pgfscope}%
\begin{pgfscope}%
\pgfsys@transformshift{2.245510in}{2.666236in}%
\pgfsys@useobject{currentmarker}{}%
\end{pgfscope}%
\begin{pgfscope}%
\pgfsys@transformshift{2.226728in}{2.432060in}%
\pgfsys@useobject{currentmarker}{}%
\end{pgfscope}%
\begin{pgfscope}%
\pgfsys@transformshift{2.207948in}{2.180793in}%
\pgfsys@useobject{currentmarker}{}%
\end{pgfscope}%
\begin{pgfscope}%
\pgfsys@transformshift{2.186582in}{2.051895in}%
\pgfsys@useobject{currentmarker}{}%
\end{pgfscope}%
\begin{pgfscope}%
\pgfsys@transformshift{2.168271in}{1.998660in}%
\pgfsys@useobject{currentmarker}{}%
\end{pgfscope}%
\begin{pgfscope}%
\pgfsys@transformshift{2.149489in}{1.976983in}%
\pgfsys@useobject{currentmarker}{}%
\end{pgfscope}%
\begin{pgfscope}%
\pgfsys@transformshift{2.131412in}{1.973106in}%
\pgfsys@useobject{currentmarker}{}%
\end{pgfscope}%
\begin{pgfscope}%
\pgfsys@transformshift{2.110046in}{2.017394in}%
\pgfsys@useobject{currentmarker}{}%
\end{pgfscope}%
\begin{pgfscope}%
\pgfsys@transformshift{2.091735in}{2.096457in}%
\pgfsys@useobject{currentmarker}{}%
\end{pgfscope}%
\begin{pgfscope}%
\pgfsys@transformshift{2.072718in}{2.277796in}%
\pgfsys@useobject{currentmarker}{}%
\end{pgfscope}%
\begin{pgfscope}%
\pgfsys@transformshift{2.050415in}{2.558643in}%
\pgfsys@useobject{currentmarker}{}%
\end{pgfscope}%
\begin{pgfscope}%
\pgfsys@transformshift{2.032573in}{2.759119in}%
\pgfsys@useobject{currentmarker}{}%
\end{pgfscope}%
\begin{pgfscope}%
\pgfsys@transformshift{2.014494in}{2.824389in}%
\pgfsys@useobject{currentmarker}{}%
\end{pgfscope}%
\begin{pgfscope}%
\pgfsys@transformshift{1.995243in}{2.779921in}%
\pgfsys@useobject{currentmarker}{}%
\end{pgfscope}%
\begin{pgfscope}%
\pgfsys@transformshift{1.975288in}{2.529955in}%
\pgfsys@useobject{currentmarker}{}%
\end{pgfscope}%
\begin{pgfscope}%
\pgfsys@transformshift{1.956506in}{2.232357in}%
\pgfsys@useobject{currentmarker}{}%
\end{pgfscope}%
\begin{pgfscope}%
\pgfsys@transformshift{1.937723in}{2.079949in}%
\pgfsys@useobject{currentmarker}{}%
\end{pgfscope}%
\begin{pgfscope}%
\pgfsys@transformshift{1.920586in}{2.047075in}%
\pgfsys@useobject{currentmarker}{}%
\end{pgfscope}%
\begin{pgfscope}%
\pgfsys@transformshift{1.896638in}{2.490384in}%
\pgfsys@useobject{currentmarker}{}%
\end{pgfscope}%
\begin{pgfscope}%
\pgfsys@transformshift{1.878327in}{2.207283in}%
\pgfsys@useobject{currentmarker}{}%
\end{pgfscope}%
\begin{pgfscope}%
\pgfsys@transformshift{1.860250in}{2.069707in}%
\pgfsys@useobject{currentmarker}{}%
\end{pgfscope}%
\begin{pgfscope}%
\pgfsys@transformshift{1.841233in}{1.993790in}%
\pgfsys@useobject{currentmarker}{}%
\end{pgfscope}%
\begin{pgfscope}%
\pgfsys@transformshift{1.820339in}{1.970246in}%
\pgfsys@useobject{currentmarker}{}%
\end{pgfscope}%
\begin{pgfscope}%
\pgfsys@transformshift{1.801322in}{1.988473in}%
\pgfsys@useobject{currentmarker}{}%
\end{pgfscope}%
\begin{pgfscope}%
\pgfsys@transformshift{1.783479in}{2.045259in}%
\pgfsys@useobject{currentmarker}{}%
\end{pgfscope}%
\begin{pgfscope}%
\pgfsys@transformshift{1.765166in}{2.174727in}%
\pgfsys@useobject{currentmarker}{}%
\end{pgfscope}%
\begin{pgfscope}%
\pgfsys@transformshift{1.744037in}{2.461929in}%
\pgfsys@useobject{currentmarker}{}%
\end{pgfscope}%
\begin{pgfscope}%
\pgfsys@transformshift{1.725255in}{2.756671in}%
\pgfsys@useobject{currentmarker}{}%
\end{pgfscope}%
\begin{pgfscope}%
\pgfsys@transformshift{1.706473in}{2.825455in}%
\pgfsys@useobject{currentmarker}{}%
\end{pgfscope}%
\begin{pgfscope}%
\pgfsys@transformshift{1.687927in}{2.834248in}%
\pgfsys@useobject{currentmarker}{}%
\end{pgfscope}%
\begin{pgfscope}%
\pgfsys@transformshift{1.669848in}{2.679606in}%
\pgfsys@useobject{currentmarker}{}%
\end{pgfscope}%
\begin{pgfscope}%
\pgfsys@transformshift{1.650362in}{2.375676in}%
\pgfsys@useobject{currentmarker}{}%
\end{pgfscope}%
\begin{pgfscope}%
\pgfsys@transformshift{1.626885in}{2.151384in}%
\pgfsys@useobject{currentmarker}{}%
\end{pgfscope}%
\begin{pgfscope}%
\pgfsys@transformshift{1.612329in}{2.065851in}%
\pgfsys@useobject{currentmarker}{}%
\end{pgfscope}%
\begin{pgfscope}%
\pgfsys@transformshift{1.591200in}{2.014037in}%
\pgfsys@useobject{currentmarker}{}%
\end{pgfscope}%
\begin{pgfscope}%
\pgfsys@transformshift{1.573826in}{1.979483in}%
\pgfsys@useobject{currentmarker}{}%
\end{pgfscope}%
\begin{pgfscope}%
\pgfsys@transformshift{1.553637in}{1.977007in}%
\pgfsys@useobject{currentmarker}{}%
\end{pgfscope}%
\begin{pgfscope}%
\pgfsys@transformshift{1.533212in}{2.019448in}%
\pgfsys@useobject{currentmarker}{}%
\end{pgfscope}%
\begin{pgfscope}%
\pgfsys@transformshift{1.515369in}{2.098137in}%
\pgfsys@useobject{currentmarker}{}%
\end{pgfscope}%
\begin{pgfscope}%
\pgfsys@transformshift{1.495647in}{2.278181in}%
\pgfsys@useobject{currentmarker}{}%
\end{pgfscope}%
\begin{pgfscope}%
\pgfsys@transformshift{1.474518in}{2.599355in}%
\pgfsys@useobject{currentmarker}{}%
\end{pgfscope}%
\begin{pgfscope}%
\pgfsys@transformshift{1.457145in}{2.800440in}%
\pgfsys@useobject{currentmarker}{}%
\end{pgfscope}%
\begin{pgfscope}%
\pgfsys@transformshift{1.437659in}{2.856282in}%
\pgfsys@useobject{currentmarker}{}%
\end{pgfscope}%
\begin{pgfscope}%
\pgfsys@transformshift{1.419582in}{2.777506in}%
\pgfsys@useobject{currentmarker}{}%
\end{pgfscope}%
\begin{pgfscope}%
\pgfsys@transformshift{1.419816in}{2.563497in}%
\pgfsys@useobject{currentmarker}{}%
\end{pgfscope}%
\begin{pgfscope}%
\pgfsys@transformshift{1.395868in}{2.406146in}%
\pgfsys@useobject{currentmarker}{}%
\end{pgfscope}%
\begin{pgfscope}%
\pgfsys@transformshift{1.380374in}{2.228365in}%
\pgfsys@useobject{currentmarker}{}%
\end{pgfscope}%
\begin{pgfscope}%
\pgfsys@transformshift{1.360183in}{2.084578in}%
\pgfsys@useobject{currentmarker}{}%
\end{pgfscope}%
\begin{pgfscope}%
\pgfsys@transformshift{1.342575in}{2.026722in}%
\pgfsys@useobject{currentmarker}{}%
\end{pgfscope}%
\begin{pgfscope}%
\pgfsys@transformshift{1.323795in}{1.984683in}%
\pgfsys@useobject{currentmarker}{}%
\end{pgfscope}%
\begin{pgfscope}%
\pgfsys@transformshift{1.305013in}{1.976485in}%
\pgfsys@useobject{currentmarker}{}%
\end{pgfscope}%
\begin{pgfscope}%
\pgfsys@transformshift{1.283178in}{2.010075in}%
\pgfsys@useobject{currentmarker}{}%
\end{pgfscope}%
\begin{pgfscope}%
\pgfsys@transformshift{1.265336in}{2.071088in}%
\pgfsys@useobject{currentmarker}{}%
\end{pgfscope}%
\begin{pgfscope}%
\pgfsys@transformshift{1.244676in}{2.213046in}%
\pgfsys@useobject{currentmarker}{}%
\end{pgfscope}%
\begin{pgfscope}%
\pgfsys@transformshift{1.227537in}{2.436806in}%
\pgfsys@useobject{currentmarker}{}%
\end{pgfscope}%
\begin{pgfscope}%
\pgfsys@transformshift{1.207817in}{2.682925in}%
\pgfsys@useobject{currentmarker}{}%
\end{pgfscope}%
\begin{pgfscope}%
\pgfsys@transformshift{1.187860in}{2.858775in}%
\pgfsys@useobject{currentmarker}{}%
\end{pgfscope}%
\begin{pgfscope}%
\pgfsys@transformshift{1.168140in}{2.876846in}%
\pgfsys@useobject{currentmarker}{}%
\end{pgfscope}%
\begin{pgfscope}%
\pgfsys@transformshift{1.146540in}{2.829174in}%
\pgfsys@useobject{currentmarker}{}%
\end{pgfscope}%
\begin{pgfscope}%
\pgfsys@transformshift{1.132455in}{2.637128in}%
\pgfsys@useobject{currentmarker}{}%
\end{pgfscope}%
\begin{pgfscope}%
\pgfsys@transformshift{1.109212in}{2.364217in}%
\pgfsys@useobject{currentmarker}{}%
\end{pgfscope}%
\begin{pgfscope}%
\pgfsys@transformshift{1.094187in}{2.196154in}%
\pgfsys@useobject{currentmarker}{}%
\end{pgfscope}%
\begin{pgfscope}%
\pgfsys@transformshift{1.073056in}{2.074494in}%
\pgfsys@useobject{currentmarker}{}%
\end{pgfscope}%
\begin{pgfscope}%
\pgfsys@transformshift{1.053102in}{2.010687in}%
\pgfsys@useobject{currentmarker}{}%
\end{pgfscope}%
\begin{pgfscope}%
\pgfsys@transformshift{1.031973in}{1.979091in}%
\pgfsys@useobject{currentmarker}{}%
\end{pgfscope}%
\begin{pgfscope}%
\pgfsys@transformshift{1.014365in}{1.991230in}%
\pgfsys@useobject{currentmarker}{}%
\end{pgfscope}%
\begin{pgfscope}%
\pgfsys@transformshift{0.996991in}{1.997881in}%
\pgfsys@useobject{currentmarker}{}%
\end{pgfscope}%
\begin{pgfscope}%
\pgfsys@transformshift{0.976097in}{2.054286in}%
\pgfsys@useobject{currentmarker}{}%
\end{pgfscope}%
\begin{pgfscope}%
\pgfsys@transformshift{0.958489in}{2.142527in}%
\pgfsys@useobject{currentmarker}{}%
\end{pgfscope}%
\begin{pgfscope}%
\pgfsys@transformshift{0.937829in}{2.363134in}%
\pgfsys@useobject{currentmarker}{}%
\end{pgfscope}%
\begin{pgfscope}%
\pgfsys@transformshift{0.917169in}{2.678328in}%
\pgfsys@useobject{currentmarker}{}%
\end{pgfscope}%
\begin{pgfscope}%
\pgfsys@transformshift{0.898856in}{2.869185in}%
\pgfsys@useobject{currentmarker}{}%
\end{pgfscope}%
\begin{pgfscope}%
\pgfsys@transformshift{0.881482in}{2.906519in}%
\pgfsys@useobject{currentmarker}{}%
\end{pgfscope}%
\begin{pgfscope}%
\pgfsys@transformshift{0.861059in}{2.771574in}%
\pgfsys@useobject{currentmarker}{}%
\end{pgfscope}%
\begin{pgfscope}%
\pgfsys@transformshift{0.841571in}{2.575916in}%
\pgfsys@useobject{currentmarker}{}%
\end{pgfscope}%
\begin{pgfscope}%
\pgfsys@transformshift{0.824199in}{2.302768in}%
\pgfsys@useobject{currentmarker}{}%
\end{pgfscope}%
\begin{pgfscope}%
\pgfsys@transformshift{0.804712in}{2.207426in}%
\pgfsys@useobject{currentmarker}{}%
\end{pgfscope}%
\begin{pgfscope}%
\pgfsys@transformshift{0.783348in}{2.077404in}%
\pgfsys@useobject{currentmarker}{}%
\end{pgfscope}%
\begin{pgfscope}%
\pgfsys@transformshift{0.764332in}{2.019693in}%
\pgfsys@useobject{currentmarker}{}%
\end{pgfscope}%
\begin{pgfscope}%
\pgfsys@transformshift{0.745549in}{2.191232in}%
\pgfsys@useobject{currentmarker}{}%
\end{pgfscope}%
\begin{pgfscope}%
\pgfsys@transformshift{0.725595in}{2.093567in}%
\pgfsys@useobject{currentmarker}{}%
\end{pgfscope}%
\begin{pgfscope}%
\pgfsys@transformshift{0.707516in}{2.015555in}%
\pgfsys@useobject{currentmarker}{}%
\end{pgfscope}%
\begin{pgfscope}%
\pgfsys@transformshift{0.689908in}{1.985118in}%
\pgfsys@useobject{currentmarker}{}%
\end{pgfscope}%
\begin{pgfscope}%
\pgfsys@transformshift{0.668310in}{2.013760in}%
\pgfsys@useobject{currentmarker}{}%
\end{pgfscope}%
\begin{pgfscope}%
\pgfsys@transformshift{0.647885in}{2.086733in}%
\pgfsys@useobject{currentmarker}{}%
\end{pgfscope}%
\begin{pgfscope}%
\pgfsys@transformshift{0.650702in}{2.076469in}%
\pgfsys@useobject{currentmarker}{}%
\end{pgfscope}%
\begin{pgfscope}%
\pgfsys@transformshift{0.656101in}{2.040514in}%
\pgfsys@useobject{currentmarker}{}%
\end{pgfscope}%
\begin{pgfscope}%
\pgfsys@transformshift{0.675117in}{1.984557in}%
\pgfsys@useobject{currentmarker}{}%
\end{pgfscope}%
\begin{pgfscope}%
\pgfsys@transformshift{0.694839in}{2.024970in}%
\pgfsys@useobject{currentmarker}{}%
\end{pgfscope}%
\begin{pgfscope}%
\pgfsys@transformshift{0.714325in}{2.136871in}%
\pgfsys@useobject{currentmarker}{}%
\end{pgfscope}%
\begin{pgfscope}%
\pgfsys@transformshift{0.732167in}{2.363827in}%
\pgfsys@useobject{currentmarker}{}%
\end{pgfscope}%
\begin{pgfscope}%
\pgfsys@transformshift{0.751184in}{2.752224in}%
\pgfsys@useobject{currentmarker}{}%
\end{pgfscope}%
\begin{pgfscope}%
\pgfsys@transformshift{0.770670in}{2.914078in}%
\pgfsys@useobject{currentmarker}{}%
\end{pgfscope}%
\begin{pgfscope}%
\pgfsys@transformshift{0.788747in}{2.760318in}%
\pgfsys@useobject{currentmarker}{}%
\end{pgfscope}%
\begin{pgfscope}%
\pgfsys@transformshift{0.808938in}{2.333683in}%
\pgfsys@useobject{currentmarker}{}%
\end{pgfscope}%
\begin{pgfscope}%
\pgfsys@transformshift{0.828424in}{2.113051in}%
\pgfsys@useobject{currentmarker}{}%
\end{pgfscope}%
\begin{pgfscope}%
\pgfsys@transformshift{0.849789in}{1.999266in}%
\pgfsys@useobject{currentmarker}{}%
\end{pgfscope}%
\begin{pgfscope}%
\pgfsys@transformshift{0.866457in}{1.983954in}%
\pgfsys@useobject{currentmarker}{}%
\end{pgfscope}%
\begin{pgfscope}%
\pgfsys@transformshift{0.887588in}{2.052080in}%
\pgfsys@useobject{currentmarker}{}%
\end{pgfscope}%
\begin{pgfscope}%
\pgfsys@transformshift{0.905899in}{2.184570in}%
\pgfsys@useobject{currentmarker}{}%
\end{pgfscope}%
\begin{pgfscope}%
\pgfsys@transformshift{0.924447in}{2.466347in}%
\pgfsys@useobject{currentmarker}{}%
\end{pgfscope}%
\begin{pgfscope}%
\pgfsys@transformshift{0.945341in}{2.839895in}%
\pgfsys@useobject{currentmarker}{}%
\end{pgfscope}%
\begin{pgfscope}%
\pgfsys@transformshift{0.963418in}{2.863309in}%
\pgfsys@useobject{currentmarker}{}%
\end{pgfscope}%
\begin{pgfscope}%
\pgfsys@transformshift{0.981026in}{2.568080in}%
\pgfsys@useobject{currentmarker}{}%
\end{pgfscope}%
\begin{pgfscope}%
\pgfsys@transformshift{1.001921in}{2.197240in}%
\pgfsys@useobject{currentmarker}{}%
\end{pgfscope}%
\begin{pgfscope}%
\pgfsys@transformshift{1.021643in}{2.035904in}%
\pgfsys@useobject{currentmarker}{}%
\end{pgfscope}%
\begin{pgfscope}%
\pgfsys@transformshift{1.040658in}{1.980716in}%
\pgfsys@useobject{currentmarker}{}%
\end{pgfscope}%
\begin{pgfscope}%
\pgfsys@transformshift{1.058737in}{1.994918in}%
\pgfsys@useobject{currentmarker}{}%
\end{pgfscope}%
\begin{pgfscope}%
\pgfsys@transformshift{1.078926in}{2.086551in}%
\pgfsys@useobject{currentmarker}{}%
\end{pgfscope}%
\begin{pgfscope}%
\pgfsys@transformshift{1.097005in}{2.253024in}%
\pgfsys@useobject{currentmarker}{}%
\end{pgfscope}%
\begin{pgfscope}%
\pgfsys@transformshift{1.117664in}{2.636083in}%
\pgfsys@useobject{currentmarker}{}%
\end{pgfscope}%
\begin{pgfscope}%
\pgfsys@transformshift{1.135976in}{2.861745in}%
\pgfsys@useobject{currentmarker}{}%
\end{pgfscope}%
\begin{pgfscope}%
\pgfsys@transformshift{1.154289in}{2.782493in}%
\pgfsys@useobject{currentmarker}{}%
\end{pgfscope}%
\begin{pgfscope}%
\pgfsys@transformshift{1.175418in}{2.387325in}%
\pgfsys@useobject{currentmarker}{}%
\end{pgfscope}%
\begin{pgfscope}%
\pgfsys@transformshift{1.195609in}{2.127822in}%
\pgfsys@useobject{currentmarker}{}%
\end{pgfscope}%
\begin{pgfscope}%
\pgfsys@transformshift{1.214155in}{2.011212in}%
\pgfsys@useobject{currentmarker}{}%
\end{pgfscope}%
\begin{pgfscope}%
\pgfsys@transformshift{1.231998in}{1.972633in}%
\pgfsys@useobject{currentmarker}{}%
\end{pgfscope}%
\begin{pgfscope}%
\pgfsys@transformshift{1.250545in}{1.997285in}%
\pgfsys@useobject{currentmarker}{}%
\end{pgfscope}%
\begin{pgfscope}%
\pgfsys@transformshift{1.271205in}{2.088372in}%
\pgfsys@useobject{currentmarker}{}%
\end{pgfscope}%
\begin{pgfscope}%
\pgfsys@transformshift{1.289282in}{2.260342in}%
\pgfsys@useobject{currentmarker}{}%
\end{pgfscope}%
\begin{pgfscope}%
\pgfsys@transformshift{1.307125in}{2.598355in}%
\pgfsys@useobject{currentmarker}{}%
\end{pgfscope}%
\begin{pgfscope}%
\pgfsys@transformshift{1.329193in}{2.841518in}%
\pgfsys@useobject{currentmarker}{}%
\end{pgfscope}%
\begin{pgfscope}%
\pgfsys@transformshift{1.347272in}{2.762452in}%
\pgfsys@useobject{currentmarker}{}%
\end{pgfscope}%
\begin{pgfscope}%
\pgfsys@transformshift{1.367227in}{2.363310in}%
\pgfsys@useobject{currentmarker}{}%
\end{pgfscope}%
\begin{pgfscope}%
\pgfsys@transformshift{1.384366in}{2.152700in}%
\pgfsys@useobject{currentmarker}{}%
\end{pgfscope}%
\begin{pgfscope}%
\pgfsys@transformshift{1.402208in}{2.031721in}%
\pgfsys@useobject{currentmarker}{}%
\end{pgfscope}%
\begin{pgfscope}%
\pgfsys@transformshift{1.423572in}{1.976818in}%
\pgfsys@useobject{currentmarker}{}%
\end{pgfscope}%
\begin{pgfscope}%
\pgfsys@transformshift{1.444937in}{1.990480in}%
\pgfsys@useobject{currentmarker}{}%
\end{pgfscope}%
\begin{pgfscope}%
\pgfsys@transformshift{1.462074in}{2.053275in}%
\pgfsys@useobject{currentmarker}{}%
\end{pgfscope}%
\begin{pgfscope}%
\pgfsys@transformshift{1.482971in}{2.169937in}%
\pgfsys@useobject{currentmarker}{}%
\end{pgfscope}%
\begin{pgfscope}%
\pgfsys@transformshift{1.501516in}{2.419914in}%
\pgfsys@useobject{currentmarker}{}%
\end{pgfscope}%
\begin{pgfscope}%
\pgfsys@transformshift{1.522176in}{2.671013in}%
\pgfsys@useobject{currentmarker}{}%
\end{pgfscope}%
\begin{pgfscope}%
\pgfsys@transformshift{1.540724in}{2.838333in}%
\pgfsys@useobject{currentmarker}{}%
\end{pgfscope}%
\begin{pgfscope}%
\pgfsys@transformshift{1.558801in}{2.707529in}%
\pgfsys@useobject{currentmarker}{}%
\end{pgfscope}%
\begin{pgfscope}%
\pgfsys@transformshift{1.576644in}{2.373943in}%
\pgfsys@useobject{currentmarker}{}%
\end{pgfscope}%
\begin{pgfscope}%
\pgfsys@transformshift{1.597538in}{2.147924in}%
\pgfsys@useobject{currentmarker}{}%
\end{pgfscope}%
\begin{pgfscope}%
\pgfsys@transformshift{1.615382in}{2.045158in}%
\pgfsys@useobject{currentmarker}{}%
\end{pgfscope}%
\begin{pgfscope}%
\pgfsys@transformshift{1.635806in}{1.979203in}%
\pgfsys@useobject{currentmarker}{}%
\end{pgfscope}%
\begin{pgfscope}%
\pgfsys@transformshift{1.653179in}{1.970676in}%
\pgfsys@useobject{currentmarker}{}%
\end{pgfscope}%
\begin{pgfscope}%
\pgfsys@transformshift{1.675014in}{2.011684in}%
\pgfsys@useobject{currentmarker}{}%
\end{pgfscope}%
\begin{pgfscope}%
\pgfsys@transformshift{1.696143in}{2.097372in}%
\pgfsys@useobject{currentmarker}{}%
\end{pgfscope}%
\begin{pgfscope}%
\pgfsys@transformshift{1.710933in}{2.189813in}%
\pgfsys@useobject{currentmarker}{}%
\end{pgfscope}%
\begin{pgfscope}%
\pgfsys@transformshift{1.731593in}{2.502373in}%
\pgfsys@useobject{currentmarker}{}%
\end{pgfscope}%
\begin{pgfscope}%
\pgfsys@transformshift{1.752489in}{2.796619in}%
\pgfsys@useobject{currentmarker}{}%
\end{pgfscope}%
\begin{pgfscope}%
\pgfsys@transformshift{1.770332in}{2.813792in}%
\pgfsys@useobject{currentmarker}{}%
\end{pgfscope}%
\begin{pgfscope}%
\pgfsys@transformshift{1.791461in}{2.577464in}%
\pgfsys@useobject{currentmarker}{}%
\end{pgfscope}%
\begin{pgfscope}%
\pgfsys@transformshift{1.809772in}{2.445710in}%
\pgfsys@useobject{currentmarker}{}%
\end{pgfscope}%
\begin{pgfscope}%
\pgfsys@transformshift{1.827617in}{2.175397in}%
\pgfsys@useobject{currentmarker}{}%
\end{pgfscope}%
\begin{pgfscope}%
\pgfsys@transformshift{1.844988in}{2.037753in}%
\pgfsys@useobject{currentmarker}{}%
\end{pgfscope}%
\begin{pgfscope}%
\pgfsys@transformshift{1.867291in}{1.977886in}%
\pgfsys@useobject{currentmarker}{}%
\end{pgfscope}%
\begin{pgfscope}%
\pgfsys@transformshift{1.884899in}{1.971701in}%
\pgfsys@useobject{currentmarker}{}%
\end{pgfscope}%
\begin{pgfscope}%
\pgfsys@transformshift{1.905796in}{2.014827in}%
\pgfsys@useobject{currentmarker}{}%
\end{pgfscope}%
\begin{pgfscope}%
\pgfsys@transformshift{1.923404in}{2.091998in}%
\pgfsys@useobject{currentmarker}{}%
\end{pgfscope}%
\begin{pgfscope}%
\pgfsys@transformshift{1.944064in}{2.226268in}%
\pgfsys@useobject{currentmarker}{}%
\end{pgfscope}%
\begin{pgfscope}%
\pgfsys@transformshift{1.961906in}{2.529651in}%
\pgfsys@useobject{currentmarker}{}%
\end{pgfscope}%
\begin{pgfscope}%
\pgfsys@transformshift{1.983738in}{2.811457in}%
\pgfsys@useobject{currentmarker}{}%
\end{pgfscope}%
\begin{pgfscope}%
\pgfsys@transformshift{2.000877in}{2.779465in}%
\pgfsys@useobject{currentmarker}{}%
\end{pgfscope}%
\begin{pgfscope}%
\pgfsys@transformshift{2.018954in}{2.575733in}%
\pgfsys@useobject{currentmarker}{}%
\end{pgfscope}%
\begin{pgfscope}%
\pgfsys@transformshift{2.041963in}{2.246303in}%
\pgfsys@useobject{currentmarker}{}%
\end{pgfscope}%
\begin{pgfscope}%
\pgfsys@transformshift{2.055579in}{2.066473in}%
\pgfsys@useobject{currentmarker}{}%
\end{pgfscope}%
\begin{pgfscope}%
\pgfsys@transformshift{2.077179in}{2.023250in}%
\pgfsys@useobject{currentmarker}{}%
\end{pgfscope}%
\begin{pgfscope}%
\pgfsys@transformshift{2.097134in}{1.974197in}%
\pgfsys@useobject{currentmarker}{}%
\end{pgfscope}%
\begin{pgfscope}%
\pgfsys@transformshift{2.097370in}{1.970294in}%
\pgfsys@useobject{currentmarker}{}%
\end{pgfscope}%
\begin{pgfscope}%
\pgfsys@transformshift{2.118264in}{1.977102in}%
\pgfsys@useobject{currentmarker}{}%
\end{pgfscope}%
\begin{pgfscope}%
\pgfsys@transformshift{2.137984in}{2.015918in}%
\pgfsys@useobject{currentmarker}{}%
\end{pgfscope}%
\begin{pgfscope}%
\pgfsys@transformshift{2.154418in}{2.088975in}%
\pgfsys@useobject{currentmarker}{}%
\end{pgfscope}%
\begin{pgfscope}%
\pgfsys@transformshift{2.175078in}{2.273445in}%
\pgfsys@useobject{currentmarker}{}%
\end{pgfscope}%
\begin{pgfscope}%
\pgfsys@transformshift{2.193392in}{2.538501in}%
\pgfsys@useobject{currentmarker}{}%
\end{pgfscope}%
\begin{pgfscope}%
\pgfsys@transformshift{2.211234in}{2.711025in}%
\pgfsys@useobject{currentmarker}{}%
\end{pgfscope}%
\begin{pgfscope}%
\pgfsys@transformshift{2.229780in}{2.812861in}%
\pgfsys@useobject{currentmarker}{}%
\end{pgfscope}%
\begin{pgfscope}%
\pgfsys@transformshift{2.250676in}{2.641130in}%
\pgfsys@useobject{currentmarker}{}%
\end{pgfscope}%
\begin{pgfscope}%
\pgfsys@transformshift{2.271571in}{2.316679in}%
\pgfsys@useobject{currentmarker}{}%
\end{pgfscope}%
\begin{pgfscope}%
\pgfsys@transformshift{2.290353in}{2.108029in}%
\pgfsys@useobject{currentmarker}{}%
\end{pgfscope}%
\begin{pgfscope}%
\pgfsys@transformshift{2.309133in}{2.018119in}%
\pgfsys@useobject{currentmarker}{}%
\end{pgfscope}%
\begin{pgfscope}%
\pgfsys@transformshift{2.326507in}{1.974026in}%
\pgfsys@useobject{currentmarker}{}%
\end{pgfscope}%
\begin{pgfscope}%
\pgfsys@transformshift{2.347636in}{1.972865in}%
\pgfsys@useobject{currentmarker}{}%
\end{pgfscope}%
\begin{pgfscope}%
\pgfsys@transformshift{2.369001in}{2.006196in}%
\pgfsys@useobject{currentmarker}{}%
\end{pgfscope}%
\begin{pgfscope}%
\pgfsys@transformshift{2.386609in}{2.042367in}%
\pgfsys@useobject{currentmarker}{}%
\end{pgfscope}%
\begin{pgfscope}%
\pgfsys@transformshift{2.404217in}{2.123471in}%
\pgfsys@useobject{currentmarker}{}%
\end{pgfscope}%
\begin{pgfscope}%
\pgfsys@transformshift{2.422528in}{2.305416in}%
\pgfsys@useobject{currentmarker}{}%
\end{pgfscope}%
\begin{pgfscope}%
\pgfsys@transformshift{2.442250in}{2.576490in}%
\pgfsys@useobject{currentmarker}{}%
\end{pgfscope}%
\begin{pgfscope}%
\pgfsys@transformshift{2.461031in}{2.799235in}%
\pgfsys@useobject{currentmarker}{}%
\end{pgfscope}%
\begin{pgfscope}%
\pgfsys@transformshift{2.482162in}{2.730055in}%
\pgfsys@useobject{currentmarker}{}%
\end{pgfscope}%
\begin{pgfscope}%
\pgfsys@transformshift{2.500473in}{2.437515in}%
\pgfsys@useobject{currentmarker}{}%
\end{pgfscope}%
\begin{pgfscope}%
\pgfsys@transformshift{2.517847in}{2.191904in}%
\pgfsys@useobject{currentmarker}{}%
\end{pgfscope}%
\begin{pgfscope}%
\pgfsys@transformshift{2.540150in}{2.048185in}%
\pgfsys@useobject{currentmarker}{}%
\end{pgfscope}%
\begin{pgfscope}%
\pgfsys@transformshift{2.559401in}{1.987981in}%
\pgfsys@useobject{currentmarker}{}%
\end{pgfscope}%
\begin{pgfscope}%
\pgfsys@transformshift{2.576774in}{1.975066in}%
\pgfsys@useobject{currentmarker}{}%
\end{pgfscope}%
\begin{pgfscope}%
\pgfsys@transformshift{2.598138in}{1.975487in}%
\pgfsys@useobject{currentmarker}{}%
\end{pgfscope}%
\begin{pgfscope}%
\pgfsys@transformshift{2.615982in}{2.006250in}%
\pgfsys@useobject{currentmarker}{}%
\end{pgfscope}%
\begin{pgfscope}%
\pgfsys@transformshift{2.634528in}{2.072095in}%
\pgfsys@useobject{currentmarker}{}%
\end{pgfscope}%
\begin{pgfscope}%
\pgfsys@transformshift{2.654014in}{2.213963in}%
\pgfsys@useobject{currentmarker}{}%
\end{pgfscope}%
\begin{pgfscope}%
\pgfsys@transformshift{2.674674in}{2.218785in}%
\pgfsys@useobject{currentmarker}{}%
\end{pgfscope}%
\begin{pgfscope}%
\pgfsys@transformshift{2.694396in}{2.512725in}%
\pgfsys@useobject{currentmarker}{}%
\end{pgfscope}%
\begin{pgfscope}%
\pgfsys@transformshift{2.713881in}{2.775349in}%
\pgfsys@useobject{currentmarker}{}%
\end{pgfscope}%
\begin{pgfscope}%
\pgfsys@transformshift{2.732664in}{2.812151in}%
\pgfsys@useobject{currentmarker}{}%
\end{pgfscope}%
\begin{pgfscope}%
\pgfsys@transformshift{2.750506in}{2.612814in}%
\pgfsys@useobject{currentmarker}{}%
\end{pgfscope}%
\begin{pgfscope}%
\pgfsys@transformshift{2.768583in}{2.319298in}%
\pgfsys@useobject{currentmarker}{}%
\end{pgfscope}%
\begin{pgfscope}%
\pgfsys@transformshift{2.789478in}{2.103192in}%
\pgfsys@useobject{currentmarker}{}%
\end{pgfscope}%
\begin{pgfscope}%
\pgfsys@transformshift{2.807320in}{2.017607in}%
\pgfsys@useobject{currentmarker}{}%
\end{pgfscope}%
\begin{pgfscope}%
\pgfsys@transformshift{2.826102in}{1.976127in}%
\pgfsys@useobject{currentmarker}{}%
\end{pgfscope}%
\begin{pgfscope}%
\pgfsys@transformshift{2.846997in}{1.976107in}%
\pgfsys@useobject{currentmarker}{}%
\end{pgfscope}%
\begin{pgfscope}%
\pgfsys@transformshift{2.866484in}{2.017116in}%
\pgfsys@useobject{currentmarker}{}%
\end{pgfscope}%
\begin{pgfscope}%
\pgfsys@transformshift{2.884327in}{2.091065in}%
\pgfsys@useobject{currentmarker}{}%
\end{pgfscope}%
\begin{pgfscope}%
\pgfsys@transformshift{2.908273in}{2.228465in}%
\pgfsys@useobject{currentmarker}{}%
\end{pgfscope}%
\begin{pgfscope}%
\pgfsys@transformshift{2.924004in}{2.451459in}%
\pgfsys@useobject{currentmarker}{}%
\end{pgfscope}%
\begin{pgfscope}%
\pgfsys@transformshift{2.944193in}{2.708897in}%
\pgfsys@useobject{currentmarker}{}%
\end{pgfscope}%
\begin{pgfscope}%
\pgfsys@transformshift{2.962271in}{2.827094in}%
\pgfsys@useobject{currentmarker}{}%
\end{pgfscope}%
\begin{pgfscope}%
\pgfsys@transformshift{2.981286in}{2.753137in}%
\pgfsys@useobject{currentmarker}{}%
\end{pgfscope}%
\begin{pgfscope}%
\pgfsys@transformshift{3.001948in}{2.441864in}%
\pgfsys@useobject{currentmarker}{}%
\end{pgfscope}%
\begin{pgfscope}%
\pgfsys@transformshift{3.019320in}{2.250648in}%
\pgfsys@useobject{currentmarker}{}%
\end{pgfscope}%
\begin{pgfscope}%
\pgfsys@transformshift{3.037399in}{2.100012in}%
\pgfsys@useobject{currentmarker}{}%
\end{pgfscope}%
\begin{pgfscope}%
\pgfsys@transformshift{3.037633in}{2.042056in}%
\pgfsys@useobject{currentmarker}{}%
\end{pgfscope}%
\begin{pgfscope}%
\pgfsys@transformshift{3.062519in}{2.024832in}%
\pgfsys@useobject{currentmarker}{}%
\end{pgfscope}%
\begin{pgfscope}%
\pgfsys@transformshift{3.077075in}{1.990780in}%
\pgfsys@useobject{currentmarker}{}%
\end{pgfscope}%
\begin{pgfscope}%
\pgfsys@transformshift{3.095152in}{2.623884in}%
\pgfsys@useobject{currentmarker}{}%
\end{pgfscope}%
\begin{pgfscope}%
\pgfsys@transformshift{3.116281in}{2.243495in}%
\pgfsys@useobject{currentmarker}{}%
\end{pgfscope}%
\begin{pgfscope}%
\pgfsys@transformshift{3.134358in}{2.085172in}%
\pgfsys@useobject{currentmarker}{}%
\end{pgfscope}%
\begin{pgfscope}%
\pgfsys@transformshift{3.152671in}{2.002815in}%
\pgfsys@useobject{currentmarker}{}%
\end{pgfscope}%
\begin{pgfscope}%
\pgfsys@transformshift{3.171217in}{1.972840in}%
\pgfsys@useobject{currentmarker}{}%
\end{pgfscope}%
\begin{pgfscope}%
\pgfsys@transformshift{3.191877in}{1.989981in}%
\pgfsys@useobject{currentmarker}{}%
\end{pgfscope}%
\begin{pgfscope}%
\pgfsys@transformshift{3.210191in}{2.040428in}%
\pgfsys@useobject{currentmarker}{}%
\end{pgfscope}%
\begin{pgfscope}%
\pgfsys@transformshift{3.232259in}{2.167889in}%
\pgfsys@useobject{currentmarker}{}%
\end{pgfscope}%
\begin{pgfscope}%
\pgfsys@transformshift{3.251979in}{2.416620in}%
\pgfsys@useobject{currentmarker}{}%
\end{pgfscope}%
\begin{pgfscope}%
\pgfsys@transformshift{3.271231in}{2.683738in}%
\pgfsys@useobject{currentmarker}{}%
\end{pgfscope}%
\begin{pgfscope}%
\pgfsys@transformshift{3.288839in}{2.839169in}%
\pgfsys@useobject{currentmarker}{}%
\end{pgfscope}%
\begin{pgfscope}%
\pgfsys@transformshift{3.306681in}{2.780310in}%
\pgfsys@useobject{currentmarker}{}%
\end{pgfscope}%
\begin{pgfscope}%
\pgfsys@transformshift{3.327812in}{2.556699in}%
\pgfsys@useobject{currentmarker}{}%
\end{pgfscope}%
\begin{pgfscope}%
\pgfsys@transformshift{3.348472in}{2.228552in}%
\pgfsys@useobject{currentmarker}{}%
\end{pgfscope}%
\begin{pgfscope}%
\pgfsys@transformshift{3.363966in}{2.102015in}%
\pgfsys@useobject{currentmarker}{}%
\end{pgfscope}%
\begin{pgfscope}%
\pgfsys@transformshift{3.384860in}{2.013721in}%
\pgfsys@useobject{currentmarker}{}%
\end{pgfscope}%
\begin{pgfscope}%
\pgfsys@transformshift{3.405286in}{1.977565in}%
\pgfsys@useobject{currentmarker}{}%
\end{pgfscope}%
\begin{pgfscope}%
\pgfsys@transformshift{3.424537in}{1.979704in}%
\pgfsys@useobject{currentmarker}{}%
\end{pgfscope}%
\begin{pgfscope}%
\pgfsys@transformshift{3.444259in}{2.027337in}%
\pgfsys@useobject{currentmarker}{}%
\end{pgfscope}%
\begin{pgfscope}%
\pgfsys@transformshift{3.459987in}{2.083900in}%
\pgfsys@useobject{currentmarker}{}%
\end{pgfscope}%
\begin{pgfscope}%
\pgfsys@transformshift{3.480647in}{2.256451in}%
\pgfsys@useobject{currentmarker}{}%
\end{pgfscope}%
\begin{pgfscope}%
\pgfsys@transformshift{3.500604in}{2.529846in}%
\pgfsys@useobject{currentmarker}{}%
\end{pgfscope}%
\begin{pgfscope}%
\pgfsys@transformshift{3.520795in}{2.824449in}%
\pgfsys@useobject{currentmarker}{}%
\end{pgfscope}%
\begin{pgfscope}%
\pgfsys@transformshift{3.540515in}{2.856996in}%
\pgfsys@useobject{currentmarker}{}%
\end{pgfscope}%
\begin{pgfscope}%
\pgfsys@transformshift{3.558358in}{2.727776in}%
\pgfsys@useobject{currentmarker}{}%
\end{pgfscope}%
\begin{pgfscope}%
\pgfsys@transformshift{3.577374in}{2.453201in}%
\pgfsys@useobject{currentmarker}{}%
\end{pgfscope}%
\begin{pgfscope}%
\pgfsys@transformshift{3.598269in}{2.181837in}%
\pgfsys@useobject{currentmarker}{}%
\end{pgfscope}%
\begin{pgfscope}%
\pgfsys@transformshift{3.616111in}{2.096488in}%
\pgfsys@useobject{currentmarker}{}%
\end{pgfscope}%
\begin{pgfscope}%
\pgfsys@transformshift{3.633954in}{2.040027in}%
\pgfsys@useobject{currentmarker}{}%
\end{pgfscope}%
\begin{pgfscope}%
\pgfsys@transformshift{3.652970in}{1.988931in}%
\pgfsys@useobject{currentmarker}{}%
\end{pgfscope}%
\begin{pgfscope}%
\pgfsys@transformshift{3.673630in}{1.981249in}%
\pgfsys@useobject{currentmarker}{}%
\end{pgfscope}%
\begin{pgfscope}%
\pgfsys@transformshift{3.693821in}{2.016784in}%
\pgfsys@useobject{currentmarker}{}%
\end{pgfscope}%
\begin{pgfscope}%
\pgfsys@transformshift{3.713073in}{2.086990in}%
\pgfsys@useobject{currentmarker}{}%
\end{pgfscope}%
\begin{pgfscope}%
\pgfsys@transformshift{3.733498in}{2.187836in}%
\pgfsys@useobject{currentmarker}{}%
\end{pgfscope}%
\begin{pgfscope}%
\pgfsys@transformshift{3.750637in}{2.381998in}%
\pgfsys@useobject{currentmarker}{}%
\end{pgfscope}%
\begin{pgfscope}%
\pgfsys@transformshift{3.769654in}{2.661559in}%
\pgfsys@useobject{currentmarker}{}%
\end{pgfscope}%
\begin{pgfscope}%
\pgfsys@transformshift{3.789140in}{2.854299in}%
\pgfsys@useobject{currentmarker}{}%
\end{pgfscope}%
\begin{pgfscope}%
\pgfsys@transformshift{3.808156in}{2.869246in}%
\pgfsys@useobject{currentmarker}{}%
\end{pgfscope}%
\begin{pgfscope}%
\pgfsys@transformshift{3.826702in}{2.712794in}%
\pgfsys@useobject{currentmarker}{}%
\end{pgfscope}%
\begin{pgfscope}%
\pgfsys@transformshift{3.846188in}{2.492686in}%
\pgfsys@useobject{currentmarker}{}%
\end{pgfscope}%
\begin{pgfscope}%
\pgfsys@transformshift{3.866850in}{2.236369in}%
\pgfsys@useobject{currentmarker}{}%
\end{pgfscope}%
\begin{pgfscope}%
\pgfsys@transformshift{3.885396in}{2.099013in}%
\pgfsys@useobject{currentmarker}{}%
\end{pgfscope}%
\begin{pgfscope}%
\pgfsys@transformshift{3.904647in}{2.024425in}%
\pgfsys@useobject{currentmarker}{}%
\end{pgfscope}%
\begin{pgfscope}%
\pgfsys@transformshift{3.922960in}{1.998602in}%
\pgfsys@useobject{currentmarker}{}%
\end{pgfscope}%
\begin{pgfscope}%
\pgfsys@transformshift{3.942680in}{1.979294in}%
\pgfsys@useobject{currentmarker}{}%
\end{pgfscope}%
\begin{pgfscope}%
\pgfsys@transformshift{3.960288in}{2.002863in}%
\pgfsys@useobject{currentmarker}{}%
\end{pgfscope}%
\begin{pgfscope}%
\pgfsys@transformshift{3.980479in}{2.033207in}%
\pgfsys@useobject{currentmarker}{}%
\end{pgfscope}%
\begin{pgfscope}%
\pgfsys@transformshift{3.999260in}{2.121491in}%
\pgfsys@useobject{currentmarker}{}%
\end{pgfscope}%
\begin{pgfscope}%
\pgfsys@transformshift{4.019216in}{2.270841in}%
\pgfsys@useobject{currentmarker}{}%
\end{pgfscope}%
\begin{pgfscope}%
\pgfsys@transformshift{4.037293in}{2.537020in}%
\pgfsys@useobject{currentmarker}{}%
\end{pgfscope}%
\begin{pgfscope}%
\pgfsys@transformshift{4.056544in}{2.803225in}%
\pgfsys@useobject{currentmarker}{}%
\end{pgfscope}%
\begin{pgfscope}%
\pgfsys@transformshift{4.075327in}{2.904597in}%
\pgfsys@useobject{currentmarker}{}%
\end{pgfscope}%
\begin{pgfscope}%
\pgfsys@transformshift{4.097630in}{2.880203in}%
\pgfsys@useobject{currentmarker}{}%
\end{pgfscope}%
\begin{pgfscope}%
\pgfsys@transformshift{4.116178in}{2.691189in}%
\pgfsys@useobject{currentmarker}{}%
\end{pgfscope}%
\begin{pgfscope}%
\pgfsys@transformshift{4.136367in}{2.379251in}%
\pgfsys@useobject{currentmarker}{}%
\end{pgfscope}%
\begin{pgfscope}%
\pgfsys@transformshift{4.154914in}{2.210299in}%
\pgfsys@useobject{currentmarker}{}%
\end{pgfscope}%
\begin{pgfscope}%
\pgfsys@transformshift{4.174400in}{2.084906in}%
\pgfsys@useobject{currentmarker}{}%
\end{pgfscope}%
\begin{pgfscope}%
\pgfsys@transformshift{4.192477in}{2.030860in}%
\pgfsys@useobject{currentmarker}{}%
\end{pgfscope}%
\begin{pgfscope}%
\pgfsys@transformshift{4.212199in}{1.985684in}%
\pgfsys@useobject{currentmarker}{}%
\end{pgfscope}%
\begin{pgfscope}%
\pgfsys@transformshift{4.229102in}{2.593701in}%
\pgfsys@useobject{currentmarker}{}%
\end{pgfscope}%
\begin{pgfscope}%
\pgfsys@transformshift{4.249998in}{2.882951in}%
\pgfsys@useobject{currentmarker}{}%
\end{pgfscope}%
\begin{pgfscope}%
\pgfsys@transformshift{4.268544in}{2.902532in}%
\pgfsys@useobject{currentmarker}{}%
\end{pgfscope}%
\begin{pgfscope}%
\pgfsys@transformshift{4.287561in}{2.696877in}%
\pgfsys@useobject{currentmarker}{}%
\end{pgfscope}%
\begin{pgfscope}%
\pgfsys@transformshift{4.308455in}{2.313021in}%
\pgfsys@useobject{currentmarker}{}%
\end{pgfscope}%
\begin{pgfscope}%
\pgfsys@transformshift{4.327237in}{2.137946in}%
\pgfsys@useobject{currentmarker}{}%
\end{pgfscope}%
\begin{pgfscope}%
\pgfsys@transformshift{4.346958in}{2.024202in}%
\pgfsys@useobject{currentmarker}{}%
\end{pgfscope}%
\begin{pgfscope}%
\pgfsys@transformshift{4.364331in}{1.985882in}%
\pgfsys@useobject{currentmarker}{}%
\end{pgfscope}%
\begin{pgfscope}%
\pgfsys@transformshift{4.388983in}{2.014195in}%
\pgfsys@useobject{currentmarker}{}%
\end{pgfscope}%
\begin{pgfscope}%
\pgfsys@transformshift{4.402365in}{2.079423in}%
\pgfsys@useobject{currentmarker}{}%
\end{pgfscope}%
\begin{pgfscope}%
\pgfsys@transformshift{4.421147in}{2.193590in}%
\pgfsys@useobject{currentmarker}{}%
\end{pgfscope}%
\begin{pgfscope}%
\pgfsys@transformshift{4.443919in}{2.506413in}%
\pgfsys@useobject{currentmarker}{}%
\end{pgfscope}%
\begin{pgfscope}%
\pgfsys@transformshift{4.462232in}{2.833791in}%
\pgfsys@useobject{currentmarker}{}%
\end{pgfscope}%
\begin{pgfscope}%
\pgfsys@transformshift{4.481249in}{2.942389in}%
\pgfsys@useobject{currentmarker}{}%
\end{pgfscope}%
\begin{pgfscope}%
\pgfsys@transformshift{4.481013in}{2.943912in}%
\pgfsys@useobject{currentmarker}{}%
\end{pgfscope}%
\begin{pgfscope}%
\pgfsys@transformshift{4.473266in}{2.910139in}%
\pgfsys@useobject{currentmarker}{}%
\end{pgfscope}%
\begin{pgfscope}%
\pgfsys@transformshift{4.454483in}{2.582159in}%
\pgfsys@useobject{currentmarker}{}%
\end{pgfscope}%
\begin{pgfscope}%
\pgfsys@transformshift{4.433824in}{2.209121in}%
\pgfsys@useobject{currentmarker}{}%
\end{pgfscope}%
\begin{pgfscope}%
\pgfsys@transformshift{4.415981in}{2.060825in}%
\pgfsys@useobject{currentmarker}{}%
\end{pgfscope}%
\begin{pgfscope}%
\pgfsys@transformshift{4.397904in}{1.992760in}%
\pgfsys@useobject{currentmarker}{}%
\end{pgfscope}%
\begin{pgfscope}%
\pgfsys@transformshift{4.377244in}{2.010457in}%
\pgfsys@useobject{currentmarker}{}%
\end{pgfscope}%
\begin{pgfscope}%
\pgfsys@transformshift{4.359402in}{2.104572in}%
\pgfsys@useobject{currentmarker}{}%
\end{pgfscope}%
\begin{pgfscope}%
\pgfsys@transformshift{4.342028in}{2.330684in}%
\pgfsys@useobject{currentmarker}{}%
\end{pgfscope}%
\begin{pgfscope}%
\pgfsys@transformshift{4.322072in}{2.727413in}%
\pgfsys@useobject{currentmarker}{}%
\end{pgfscope}%
\begin{pgfscope}%
\pgfsys@transformshift{4.300943in}{2.913953in}%
\pgfsys@useobject{currentmarker}{}%
\end{pgfscope}%
\begin{pgfscope}%
\pgfsys@transformshift{4.282866in}{2.690864in}%
\pgfsys@useobject{currentmarker}{}%
\end{pgfscope}%
\begin{pgfscope}%
\pgfsys@transformshift{4.260563in}{2.237691in}%
\pgfsys@useobject{currentmarker}{}%
\end{pgfscope}%
\begin{pgfscope}%
\pgfsys@transformshift{4.243892in}{2.078561in}%
\pgfsys@useobject{currentmarker}{}%
\end{pgfscope}%
\begin{pgfscope}%
\pgfsys@transformshift{4.224876in}{1.995341in}%
\pgfsys@useobject{currentmarker}{}%
\end{pgfscope}%
\begin{pgfscope}%
\pgfsys@transformshift{4.203043in}{1.989891in}%
\pgfsys@useobject{currentmarker}{}%
\end{pgfscope}%
\begin{pgfscope}%
\pgfsys@transformshift{4.185201in}{2.071119in}%
\pgfsys@useobject{currentmarker}{}%
\end{pgfscope}%
\begin{pgfscope}%
\pgfsys@transformshift{4.164541in}{2.276190in}%
\pgfsys@useobject{currentmarker}{}%
\end{pgfscope}%
\begin{pgfscope}%
\pgfsys@transformshift{4.146228in}{2.656041in}%
\pgfsys@useobject{currentmarker}{}%
\end{pgfscope}%
\begin{pgfscope}%
\pgfsys@transformshift{4.127445in}{2.886896in}%
\pgfsys@useobject{currentmarker}{}%
\end{pgfscope}%
\begin{pgfscope}%
\pgfsys@transformshift{4.105142in}{2.696345in}%
\pgfsys@useobject{currentmarker}{}%
\end{pgfscope}%
\begin{pgfscope}%
\pgfsys@transformshift{4.089883in}{2.359134in}%
\pgfsys@useobject{currentmarker}{}%
\end{pgfscope}%
\begin{pgfscope}%
\pgfsys@transformshift{4.071569in}{2.113965in}%
\pgfsys@useobject{currentmarker}{}%
\end{pgfscope}%
\begin{pgfscope}%
\pgfsys@transformshift{4.053024in}{2.014515in}%
\pgfsys@useobject{currentmarker}{}%
\end{pgfscope}%
\begin{pgfscope}%
\pgfsys@transformshift{4.031189in}{1.976406in}%
\pgfsys@useobject{currentmarker}{}%
\end{pgfscope}%
\begin{pgfscope}%
\pgfsys@transformshift{4.012173in}{2.010360in}%
\pgfsys@useobject{currentmarker}{}%
\end{pgfscope}%
\begin{pgfscope}%
\pgfsys@transformshift{3.993861in}{2.116406in}%
\pgfsys@useobject{currentmarker}{}%
\end{pgfscope}%
\begin{pgfscope}%
\pgfsys@transformshift{3.974844in}{2.372820in}%
\pgfsys@useobject{currentmarker}{}%
\end{pgfscope}%
\begin{pgfscope}%
\pgfsys@transformshift{3.953245in}{2.779074in}%
\pgfsys@useobject{currentmarker}{}%
\end{pgfscope}%
\begin{pgfscope}%
\pgfsys@transformshift{3.933994in}{2.863944in}%
\pgfsys@useobject{currentmarker}{}%
\end{pgfscope}%
\begin{pgfscope}%
\pgfsys@transformshift{3.915917in}{2.607188in}%
\pgfsys@useobject{currentmarker}{}%
\end{pgfscope}%
\begin{pgfscope}%
\pgfsys@transformshift{3.897134in}{2.250015in}%
\pgfsys@useobject{currentmarker}{}%
\end{pgfscope}%
\begin{pgfscope}%
\pgfsys@transformshift{3.878823in}{2.077000in}%
\pgfsys@useobject{currentmarker}{}%
\end{pgfscope}%
\begin{pgfscope}%
\pgfsys@transformshift{3.860041in}{2.000995in}%
\pgfsys@useobject{currentmarker}{}%
\end{pgfscope}%
\begin{pgfscope}%
\pgfsys@transformshift{3.840555in}{1.972444in}%
\pgfsys@useobject{currentmarker}{}%
\end{pgfscope}%
\begin{pgfscope}%
\pgfsys@transformshift{3.822476in}{2.009824in}%
\pgfsys@useobject{currentmarker}{}%
\end{pgfscope}%
\begin{pgfscope}%
\pgfsys@transformshift{3.800644in}{2.128556in}%
\pgfsys@useobject{currentmarker}{}%
\end{pgfscope}%
\begin{pgfscope}%
\pgfsys@transformshift{3.785148in}{2.282849in}%
\pgfsys@useobject{currentmarker}{}%
\end{pgfscope}%
\begin{pgfscope}%
\pgfsys@transformshift{3.765193in}{2.627410in}%
\pgfsys@useobject{currentmarker}{}%
\end{pgfscope}%
\begin{pgfscope}%
\pgfsys@transformshift{3.740307in}{2.853826in}%
\pgfsys@useobject{currentmarker}{}%
\end{pgfscope}%
\begin{pgfscope}%
\pgfsys@transformshift{3.723637in}{2.801485in}%
\pgfsys@useobject{currentmarker}{}%
\end{pgfscope}%
\begin{pgfscope}%
\pgfsys@transformshift{3.701099in}{2.399909in}%
\pgfsys@useobject{currentmarker}{}%
\end{pgfscope}%
\begin{pgfscope}%
\pgfsys@transformshift{3.686309in}{2.204350in}%
\pgfsys@useobject{currentmarker}{}%
\end{pgfscope}%
\begin{pgfscope}%
\pgfsys@transformshift{3.667527in}{2.064844in}%
\pgfsys@useobject{currentmarker}{}%
\end{pgfscope}%
\begin{pgfscope}%
\pgfsys@transformshift{3.649215in}{1.994193in}%
\pgfsys@useobject{currentmarker}{}%
\end{pgfscope}%
\begin{pgfscope}%
\pgfsys@transformshift{3.626912in}{2.830666in}%
\pgfsys@useobject{currentmarker}{}%
\end{pgfscope}%
\begin{pgfscope}%
\pgfsys@transformshift{3.608833in}{2.571949in}%
\pgfsys@useobject{currentmarker}{}%
\end{pgfscope}%
\begin{pgfscope}%
\pgfsys@transformshift{3.589818in}{2.242208in}%
\pgfsys@useobject{currentmarker}{}%
\end{pgfscope}%
\begin{pgfscope}%
\pgfsys@transformshift{3.568453in}{2.064753in}%
\pgfsys@useobject{currentmarker}{}%
\end{pgfscope}%
\begin{pgfscope}%
\pgfsys@transformshift{3.552957in}{2.003462in}%
\pgfsys@useobject{currentmarker}{}%
\end{pgfscope}%
\begin{pgfscope}%
\pgfsys@transformshift{3.531125in}{1.970055in}%
\pgfsys@useobject{currentmarker}{}%
\end{pgfscope}%
\begin{pgfscope}%
\pgfsys@transformshift{3.515160in}{1.986210in}%
\pgfsys@useobject{currentmarker}{}%
\end{pgfscope}%
\begin{pgfscope}%
\pgfsys@transformshift{3.493560in}{2.049273in}%
\pgfsys@useobject{currentmarker}{}%
\end{pgfscope}%
\begin{pgfscope}%
\pgfsys@transformshift{3.473840in}{2.190123in}%
\pgfsys@useobject{currentmarker}{}%
\end{pgfscope}%
\begin{pgfscope}%
\pgfsys@transformshift{3.452240in}{2.527995in}%
\pgfsys@useobject{currentmarker}{}%
\end{pgfscope}%
\begin{pgfscope}%
\pgfsys@transformshift{3.436746in}{2.789976in}%
\pgfsys@useobject{currentmarker}{}%
\end{pgfscope}%
\begin{pgfscope}%
\pgfsys@transformshift{3.417730in}{2.831449in}%
\pgfsys@useobject{currentmarker}{}%
\end{pgfscope}%
\begin{pgfscope}%
\pgfsys@transformshift{3.397539in}{2.620451in}%
\pgfsys@useobject{currentmarker}{}%
\end{pgfscope}%
\begin{pgfscope}%
\pgfsys@transformshift{3.379462in}{2.259378in}%
\pgfsys@useobject{currentmarker}{}%
\end{pgfscope}%
\begin{pgfscope}%
\pgfsys@transformshift{3.356922in}{2.073501in}%
\pgfsys@useobject{currentmarker}{}%
\end{pgfscope}%
\begin{pgfscope}%
\pgfsys@transformshift{3.340723in}{2.009523in}%
\pgfsys@useobject{currentmarker}{}%
\end{pgfscope}%
\begin{pgfscope}%
\pgfsys@transformshift{3.321708in}{1.969433in}%
\pgfsys@useobject{currentmarker}{}%
\end{pgfscope}%
\begin{pgfscope}%
\pgfsys@transformshift{3.301046in}{1.991500in}%
\pgfsys@useobject{currentmarker}{}%
\end{pgfscope}%
\begin{pgfscope}%
\pgfsys@transformshift{3.282031in}{2.056016in}%
\pgfsys@useobject{currentmarker}{}%
\end{pgfscope}%
\begin{pgfscope}%
\pgfsys@transformshift{3.264187in}{2.203332in}%
\pgfsys@useobject{currentmarker}{}%
\end{pgfscope}%
\begin{pgfscope}%
\pgfsys@transformshift{3.245407in}{2.498312in}%
\pgfsys@useobject{currentmarker}{}%
\end{pgfscope}%
\begin{pgfscope}%
\pgfsys@transformshift{3.223338in}{2.797823in}%
\pgfsys@useobject{currentmarker}{}%
\end{pgfscope}%
\begin{pgfscope}%
\pgfsys@transformshift{3.205730in}{2.807372in}%
\pgfsys@useobject{currentmarker}{}%
\end{pgfscope}%
\begin{pgfscope}%
\pgfsys@transformshift{3.186713in}{2.562230in}%
\pgfsys@useobject{currentmarker}{}%
\end{pgfscope}%
\begin{pgfscope}%
\pgfsys@transformshift{3.167462in}{2.258670in}%
\pgfsys@useobject{currentmarker}{}%
\end{pgfscope}%
\begin{pgfscope}%
\pgfsys@transformshift{3.148914in}{2.091964in}%
\pgfsys@useobject{currentmarker}{}%
\end{pgfscope}%
\begin{pgfscope}%
\pgfsys@transformshift{3.128960in}{2.003234in}%
\pgfsys@useobject{currentmarker}{}%
\end{pgfscope}%
\begin{pgfscope}%
\pgfsys@transformshift{3.110177in}{1.970327in}%
\pgfsys@useobject{currentmarker}{}%
\end{pgfscope}%
\begin{pgfscope}%
\pgfsys@transformshift{3.090457in}{1.983801in}%
\pgfsys@useobject{currentmarker}{}%
\end{pgfscope}%
\begin{pgfscope}%
\pgfsys@transformshift{3.071909in}{2.029797in}%
\pgfsys@useobject{currentmarker}{}%
\end{pgfscope}%
\begin{pgfscope}%
\pgfsys@transformshift{3.050075in}{2.113135in}%
\pgfsys@useobject{currentmarker}{}%
\end{pgfscope}%
\begin{pgfscope}%
\pgfsys@transformshift{3.031529in}{2.359379in}%
\pgfsys@useobject{currentmarker}{}%
\end{pgfscope}%
\begin{pgfscope}%
\pgfsys@transformshift{3.013685in}{2.682410in}%
\pgfsys@useobject{currentmarker}{}%
\end{pgfscope}%
\begin{pgfscope}%
\pgfsys@transformshift{2.993730in}{2.823597in}%
\pgfsys@useobject{currentmarker}{}%
\end{pgfscope}%
\begin{pgfscope}%
\pgfsys@transformshift{2.975419in}{2.708981in}%
\pgfsys@useobject{currentmarker}{}%
\end{pgfscope}%
\begin{pgfscope}%
\pgfsys@transformshift{2.957811in}{2.420710in}%
\pgfsys@useobject{currentmarker}{}%
\end{pgfscope}%
\begin{pgfscope}%
\pgfsys@transformshift{2.935271in}{2.161692in}%
\pgfsys@useobject{currentmarker}{}%
\end{pgfscope}%
\begin{pgfscope}%
\pgfsys@transformshift{2.916960in}{2.041235in}%
\pgfsys@useobject{currentmarker}{}%
\end{pgfscope}%
\begin{pgfscope}%
\pgfsys@transformshift{2.898647in}{1.985365in}%
\pgfsys@useobject{currentmarker}{}%
\end{pgfscope}%
\begin{pgfscope}%
\pgfsys@transformshift{2.879632in}{1.969949in}%
\pgfsys@useobject{currentmarker}{}%
\end{pgfscope}%
\begin{pgfscope}%
\pgfsys@transformshift{2.859441in}{1.990232in}%
\pgfsys@useobject{currentmarker}{}%
\end{pgfscope}%
\begin{pgfscope}%
\pgfsys@transformshift{2.839719in}{2.029189in}%
\pgfsys@useobject{currentmarker}{}%
\end{pgfscope}%
\begin{pgfscope}%
\pgfsys@transformshift{2.821407in}{2.139321in}%
\pgfsys@useobject{currentmarker}{}%
\end{pgfscope}%
\begin{pgfscope}%
\pgfsys@transformshift{2.802156in}{2.361400in}%
\pgfsys@useobject{currentmarker}{}%
\end{pgfscope}%
\begin{pgfscope}%
\pgfsys@transformshift{2.783845in}{2.655880in}%
\pgfsys@useobject{currentmarker}{}%
\end{pgfscope}%
\begin{pgfscope}%
\pgfsys@transformshift{2.765766in}{2.815734in}%
\pgfsys@useobject{currentmarker}{}%
\end{pgfscope}%
\begin{pgfscope}%
\pgfsys@transformshift{2.743463in}{2.673629in}%
\pgfsys@useobject{currentmarker}{}%
\end{pgfscope}%
\begin{pgfscope}%
\pgfsys@transformshift{2.724680in}{2.613114in}%
\pgfsys@useobject{currentmarker}{}%
\end{pgfscope}%
\begin{pgfscope}%
\pgfsys@transformshift{2.708012in}{2.312529in}%
\pgfsys@useobject{currentmarker}{}%
\end{pgfscope}%
\begin{pgfscope}%
\pgfsys@transformshift{2.686178in}{2.095619in}%
\pgfsys@useobject{currentmarker}{}%
\end{pgfscope}%
\begin{pgfscope}%
\pgfsys@transformshift{2.667632in}{2.022818in}%
\pgfsys@useobject{currentmarker}{}%
\end{pgfscope}%
\begin{pgfscope}%
\pgfsys@transformshift{2.648850in}{1.974708in}%
\pgfsys@useobject{currentmarker}{}%
\end{pgfscope}%
\begin{pgfscope}%
\pgfsys@transformshift{2.628893in}{1.971950in}%
\pgfsys@useobject{currentmarker}{}%
\end{pgfscope}%
\begin{pgfscope}%
\pgfsys@transformshift{2.609876in}{2.006732in}%
\pgfsys@useobject{currentmarker}{}%
\end{pgfscope}%
\begin{pgfscope}%
\pgfsys@transformshift{2.589217in}{2.099556in}%
\pgfsys@useobject{currentmarker}{}%
\end{pgfscope}%
\begin{pgfscope}%
\pgfsys@transformshift{2.573254in}{2.250341in}%
\pgfsys@useobject{currentmarker}{}%
\end{pgfscope}%
\begin{pgfscope}%
\pgfsys@transformshift{2.551419in}{2.609402in}%
\pgfsys@useobject{currentmarker}{}%
\end{pgfscope}%
\begin{pgfscope}%
\pgfsys@transformshift{2.532637in}{2.780200in}%
\pgfsys@useobject{currentmarker}{}%
\end{pgfscope}%
\begin{pgfscope}%
\pgfsys@transformshift{2.513855in}{2.785770in}%
\pgfsys@useobject{currentmarker}{}%
\end{pgfscope}%
\begin{pgfscope}%
\pgfsys@transformshift{2.496012in}{2.673783in}%
\pgfsys@useobject{currentmarker}{}%
\end{pgfscope}%
\begin{pgfscope}%
\pgfsys@transformshift{2.495778in}{2.786010in}%
\pgfsys@useobject{currentmarker}{}%
\end{pgfscope}%
\begin{pgfscope}%
\pgfsys@transformshift{2.474178in}{2.815841in}%
\pgfsys@useobject{currentmarker}{}%
\end{pgfscope}%
\begin{pgfscope}%
\pgfsys@transformshift{2.458684in}{2.722544in}%
\pgfsys@useobject{currentmarker}{}%
\end{pgfscope}%
\begin{pgfscope}%
\pgfsys@transformshift{2.432858in}{2.289383in}%
\pgfsys@useobject{currentmarker}{}%
\end{pgfscope}%
\begin{pgfscope}%
\pgfsys@transformshift{2.416894in}{2.111306in}%
\pgfsys@useobject{currentmarker}{}%
\end{pgfscope}%
\begin{pgfscope}%
\pgfsys@transformshift{2.399756in}{2.029313in}%
\pgfsys@useobject{currentmarker}{}%
\end{pgfscope}%
\begin{pgfscope}%
\pgfsys@transformshift{2.378626in}{1.976177in}%
\pgfsys@useobject{currentmarker}{}%
\end{pgfscope}%
\begin{pgfscope}%
\pgfsys@transformshift{2.360080in}{1.970514in}%
\pgfsys@useobject{currentmarker}{}%
\end{pgfscope}%
\begin{pgfscope}%
\pgfsys@transformshift{2.341063in}{2.003744in}%
\pgfsys@useobject{currentmarker}{}%
\end{pgfscope}%
\begin{pgfscope}%
\pgfsys@transformshift{2.322515in}{2.057416in}%
\pgfsys@useobject{currentmarker}{}%
\end{pgfscope}%
\begin{pgfscope}%
\pgfsys@transformshift{2.302560in}{2.194254in}%
\pgfsys@useobject{currentmarker}{}%
\end{pgfscope}%
\begin{pgfscope}%
\pgfsys@transformshift{2.282604in}{2.521339in}%
\pgfsys@useobject{currentmarker}{}%
\end{pgfscope}%
\begin{pgfscope}%
\pgfsys@transformshift{2.261944in}{2.806492in}%
\pgfsys@useobject{currentmarker}{}%
\end{pgfscope}%
\begin{pgfscope}%
\pgfsys@transformshift{2.245510in}{2.819504in}%
\pgfsys@useobject{currentmarker}{}%
\end{pgfscope}%
\begin{pgfscope}%
\pgfsys@transformshift{2.223676in}{2.626036in}%
\pgfsys@useobject{currentmarker}{}%
\end{pgfscope}%
\begin{pgfscope}%
\pgfsys@transformshift{2.207242in}{2.341624in}%
\pgfsys@useobject{currentmarker}{}%
\end{pgfscope}%
\begin{pgfscope}%
\pgfsys@transformshift{2.186817in}{2.135377in}%
\pgfsys@useobject{currentmarker}{}%
\end{pgfscope}%
\begin{pgfscope}%
\pgfsys@transformshift{2.168740in}{2.032537in}%
\pgfsys@useobject{currentmarker}{}%
\end{pgfscope}%
\begin{pgfscope}%
\pgfsys@transformshift{2.149723in}{1.983763in}%
\pgfsys@useobject{currentmarker}{}%
\end{pgfscope}%
\begin{pgfscope}%
\pgfsys@transformshift{2.126246in}{1.976394in}%
\pgfsys@useobject{currentmarker}{}%
\end{pgfscope}%
\begin{pgfscope}%
\pgfsys@transformshift{2.110281in}{1.989964in}%
\pgfsys@useobject{currentmarker}{}%
\end{pgfscope}%
\begin{pgfscope}%
\pgfsys@transformshift{2.088449in}{2.042450in}%
\pgfsys@useobject{currentmarker}{}%
\end{pgfscope}%
\begin{pgfscope}%
\pgfsys@transformshift{2.072718in}{2.141027in}%
\pgfsys@useobject{currentmarker}{}%
\end{pgfscope}%
\begin{pgfscope}%
\pgfsys@transformshift{2.050884in}{2.392809in}%
\pgfsys@useobject{currentmarker}{}%
\end{pgfscope}%
\begin{pgfscope}%
\pgfsys@transformshift{2.033511in}{2.613451in}%
\pgfsys@useobject{currentmarker}{}%
\end{pgfscope}%
\begin{pgfscope}%
\pgfsys@transformshift{2.014259in}{2.817319in}%
\pgfsys@useobject{currentmarker}{}%
\end{pgfscope}%
\begin{pgfscope}%
\pgfsys@transformshift{1.995948in}{2.767394in}%
\pgfsys@useobject{currentmarker}{}%
\end{pgfscope}%
\begin{pgfscope}%
\pgfsys@transformshift{1.974583in}{2.481447in}%
\pgfsys@useobject{currentmarker}{}%
\end{pgfscope}%
\begin{pgfscope}%
\pgfsys@transformshift{1.955802in}{2.272470in}%
\pgfsys@useobject{currentmarker}{}%
\end{pgfscope}%
\begin{pgfscope}%
\pgfsys@transformshift{1.937254in}{2.110181in}%
\pgfsys@useobject{currentmarker}{}%
\end{pgfscope}%
\begin{pgfscope}%
\pgfsys@transformshift{1.915420in}{2.011594in}%
\pgfsys@useobject{currentmarker}{}%
\end{pgfscope}%
\begin{pgfscope}%
\pgfsys@transformshift{1.901335in}{1.983238in}%
\pgfsys@useobject{currentmarker}{}%
\end{pgfscope}%
\begin{pgfscope}%
\pgfsys@transformshift{1.879030in}{1.973320in}%
\pgfsys@useobject{currentmarker}{}%
\end{pgfscope}%
\begin{pgfscope}%
\pgfsys@transformshift{1.860015in}{2.001076in}%
\pgfsys@useobject{currentmarker}{}%
\end{pgfscope}%
\begin{pgfscope}%
\pgfsys@transformshift{1.840293in}{2.061198in}%
\pgfsys@useobject{currentmarker}{}%
\end{pgfscope}%
\begin{pgfscope}%
\pgfsys@transformshift{1.821511in}{2.155535in}%
\pgfsys@useobject{currentmarker}{}%
\end{pgfscope}%
\begin{pgfscope}%
\pgfsys@transformshift{1.803199in}{2.319168in}%
\pgfsys@useobject{currentmarker}{}%
\end{pgfscope}%
\begin{pgfscope}%
\pgfsys@transformshift{1.782539in}{2.615100in}%
\pgfsys@useobject{currentmarker}{}%
\end{pgfscope}%
\begin{pgfscope}%
\pgfsys@transformshift{1.764697in}{2.803656in}%
\pgfsys@useobject{currentmarker}{}%
\end{pgfscope}%
\begin{pgfscope}%
\pgfsys@transformshift{1.742863in}{2.805908in}%
\pgfsys@useobject{currentmarker}{}%
\end{pgfscope}%
\begin{pgfscope}%
\pgfsys@transformshift{1.727369in}{2.680798in}%
\pgfsys@useobject{currentmarker}{}%
\end{pgfscope}%
\begin{pgfscope}%
\pgfsys@transformshift{1.705769in}{2.361976in}%
\pgfsys@useobject{currentmarker}{}%
\end{pgfscope}%
\begin{pgfscope}%
\pgfsys@transformshift{1.686987in}{2.156805in}%
\pgfsys@useobject{currentmarker}{}%
\end{pgfscope}%
\begin{pgfscope}%
\pgfsys@transformshift{1.668441in}{2.066643in}%
\pgfsys@useobject{currentmarker}{}%
\end{pgfscope}%
\begin{pgfscope}%
\pgfsys@transformshift{1.650128in}{2.004807in}%
\pgfsys@useobject{currentmarker}{}%
\end{pgfscope}%
\begin{pgfscope}%
\pgfsys@transformshift{1.630407in}{1.977721in}%
\pgfsys@useobject{currentmarker}{}%
\end{pgfscope}%
\begin{pgfscope}%
\pgfsys@transformshift{1.612799in}{1.979089in}%
\pgfsys@useobject{currentmarker}{}%
\end{pgfscope}%
\begin{pgfscope}%
\pgfsys@transformshift{1.592139in}{2.009612in}%
\pgfsys@useobject{currentmarker}{}%
\end{pgfscope}%
\begin{pgfscope}%
\pgfsys@transformshift{1.574061in}{2.081460in}%
\pgfsys@useobject{currentmarker}{}%
\end{pgfscope}%
\begin{pgfscope}%
\pgfsys@transformshift{1.550349in}{2.230842in}%
\pgfsys@useobject{currentmarker}{}%
\end{pgfscope}%
\begin{pgfscope}%
\pgfsys@transformshift{1.533212in}{2.414790in}%
\pgfsys@useobject{currentmarker}{}%
\end{pgfscope}%
\begin{pgfscope}%
\pgfsys@transformshift{1.512552in}{2.718568in}%
\pgfsys@useobject{currentmarker}{}%
\end{pgfscope}%
\begin{pgfscope}%
\pgfsys@transformshift{1.494709in}{2.842403in}%
\pgfsys@useobject{currentmarker}{}%
\end{pgfscope}%
\begin{pgfscope}%
\pgfsys@transformshift{1.476396in}{2.835991in}%
\pgfsys@useobject{currentmarker}{}%
\end{pgfscope}%
\begin{pgfscope}%
\pgfsys@transformshift{1.458553in}{2.674241in}%
\pgfsys@useobject{currentmarker}{}%
\end{pgfscope}%
\begin{pgfscope}%
\pgfsys@transformshift{1.438362in}{2.369011in}%
\pgfsys@useobject{currentmarker}{}%
\end{pgfscope}%
\begin{pgfscope}%
\pgfsys@transformshift{1.419111in}{2.202211in}%
\pgfsys@useobject{currentmarker}{}%
\end{pgfscope}%
\begin{pgfscope}%
\pgfsys@transformshift{1.396105in}{2.348555in}%
\pgfsys@useobject{currentmarker}{}%
\end{pgfscope}%
\begin{pgfscope}%
\pgfsys@transformshift{1.379200in}{2.172317in}%
\pgfsys@useobject{currentmarker}{}%
\end{pgfscope}%
\begin{pgfscope}%
\pgfsys@transformshift{1.358540in}{2.054511in}%
\pgfsys@useobject{currentmarker}{}%
\end{pgfscope}%
\begin{pgfscope}%
\pgfsys@transformshift{1.341872in}{2.000948in}%
\pgfsys@useobject{currentmarker}{}%
\end{pgfscope}%
\begin{pgfscope}%
\pgfsys@transformshift{1.323558in}{1.976164in}%
\pgfsys@useobject{currentmarker}{}%
\end{pgfscope}%
\begin{pgfscope}%
\pgfsys@transformshift{1.304073in}{1.987047in}%
\pgfsys@useobject{currentmarker}{}%
\end{pgfscope}%
\begin{pgfscope}%
\pgfsys@transformshift{1.283647in}{2.032249in}%
\pgfsys@useobject{currentmarker}{}%
\end{pgfscope}%
\begin{pgfscope}%
\pgfsys@transformshift{1.263458in}{2.118995in}%
\pgfsys@useobject{currentmarker}{}%
\end{pgfscope}%
\begin{pgfscope}%
\pgfsys@transformshift{1.246085in}{2.200294in}%
\pgfsys@useobject{currentmarker}{}%
\end{pgfscope}%
\begin{pgfscope}%
\pgfsys@transformshift{1.225190in}{2.482871in}%
\pgfsys@useobject{currentmarker}{}%
\end{pgfscope}%
\begin{pgfscope}%
\pgfsys@transformshift{1.206642in}{2.610037in}%
\pgfsys@useobject{currentmarker}{}%
\end{pgfscope}%
\begin{pgfscope}%
\pgfsys@transformshift{1.186922in}{2.836625in}%
\pgfsys@useobject{currentmarker}{}%
\end{pgfscope}%
\begin{pgfscope}%
\pgfsys@transformshift{1.169783in}{2.877010in}%
\pgfsys@useobject{currentmarker}{}%
\end{pgfscope}%
\begin{pgfscope}%
\pgfsys@transformshift{1.149123in}{2.872998in}%
\pgfsys@useobject{currentmarker}{}%
\end{pgfscope}%
\begin{pgfscope}%
\pgfsys@transformshift{1.131515in}{2.737042in}%
\pgfsys@useobject{currentmarker}{}%
\end{pgfscope}%
\begin{pgfscope}%
\pgfsys@transformshift{1.111561in}{2.409395in}%
\pgfsys@useobject{currentmarker}{}%
\end{pgfscope}%
\begin{pgfscope}%
\pgfsys@transformshift{1.091135in}{2.178199in}%
\pgfsys@useobject{currentmarker}{}%
\end{pgfscope}%
\begin{pgfscope}%
\pgfsys@transformshift{1.070475in}{2.070713in}%
\pgfsys@useobject{currentmarker}{}%
\end{pgfscope}%
\begin{pgfscope}%
\pgfsys@transformshift{1.052396in}{2.020366in}%
\pgfsys@useobject{currentmarker}{}%
\end{pgfscope}%
\begin{pgfscope}%
\pgfsys@transformshift{1.035963in}{1.985752in}%
\pgfsys@useobject{currentmarker}{}%
\end{pgfscope}%
\begin{pgfscope}%
\pgfsys@transformshift{1.015068in}{1.982739in}%
\pgfsys@useobject{currentmarker}{}%
\end{pgfscope}%
\begin{pgfscope}%
\pgfsys@transformshift{0.994877in}{2.026855in}%
\pgfsys@useobject{currentmarker}{}%
\end{pgfscope}%
\begin{pgfscope}%
\pgfsys@transformshift{0.978678in}{2.094696in}%
\pgfsys@useobject{currentmarker}{}%
\end{pgfscope}%
\begin{pgfscope}%
\pgfsys@transformshift{0.957549in}{2.239577in}%
\pgfsys@useobject{currentmarker}{}%
\end{pgfscope}%
\begin{pgfscope}%
\pgfsys@transformshift{0.939707in}{2.467360in}%
\pgfsys@useobject{currentmarker}{}%
\end{pgfscope}%
\begin{pgfscope}%
\pgfsys@transformshift{0.918107in}{2.769505in}%
\pgfsys@useobject{currentmarker}{}%
\end{pgfscope}%
\begin{pgfscope}%
\pgfsys@transformshift{0.897681in}{2.901833in}%
\pgfsys@useobject{currentmarker}{}%
\end{pgfscope}%
\begin{pgfscope}%
\pgfsys@transformshift{0.879604in}{2.885784in}%
\pgfsys@useobject{currentmarker}{}%
\end{pgfscope}%
\begin{pgfscope}%
\pgfsys@transformshift{0.861762in}{2.707330in}%
\pgfsys@useobject{currentmarker}{}%
\end{pgfscope}%
\begin{pgfscope}%
\pgfsys@transformshift{0.841571in}{2.475860in}%
\pgfsys@useobject{currentmarker}{}%
\end{pgfscope}%
\begin{pgfscope}%
\pgfsys@transformshift{0.824199in}{2.267735in}%
\pgfsys@useobject{currentmarker}{}%
\end{pgfscope}%
\begin{pgfscope}%
\pgfsys@transformshift{0.803069in}{2.159071in}%
\pgfsys@useobject{currentmarker}{}%
\end{pgfscope}%
\begin{pgfscope}%
\pgfsys@transformshift{0.785461in}{2.067562in}%
\pgfsys@useobject{currentmarker}{}%
\end{pgfscope}%
\begin{pgfscope}%
\pgfsys@transformshift{0.762454in}{2.004154in}%
\pgfsys@useobject{currentmarker}{}%
\end{pgfscope}%
\begin{pgfscope}%
\pgfsys@transformshift{0.744141in}{1.985315in}%
\pgfsys@useobject{currentmarker}{}%
\end{pgfscope}%
\begin{pgfscope}%
\pgfsys@transformshift{0.727003in}{1.998519in}%
\pgfsys@useobject{currentmarker}{}%
\end{pgfscope}%
\begin{pgfscope}%
\pgfsys@transformshift{0.706812in}{2.052236in}%
\pgfsys@useobject{currentmarker}{}%
\end{pgfscope}%
\begin{pgfscope}%
\pgfsys@transformshift{0.688499in}{2.122170in}%
\pgfsys@useobject{currentmarker}{}%
\end{pgfscope}%
\begin{pgfscope}%
\pgfsys@transformshift{0.667136in}{2.316550in}%
\pgfsys@useobject{currentmarker}{}%
\end{pgfscope}%
\begin{pgfscope}%
\pgfsys@transformshift{0.649293in}{2.530174in}%
\pgfsys@useobject{currentmarker}{}%
\end{pgfscope}%
\begin{pgfscope}%
\pgfsys@transformshift{0.650468in}{2.518926in}%
\pgfsys@useobject{currentmarker}{}%
\end{pgfscope}%
\begin{pgfscope}%
\pgfsys@transformshift{0.656337in}{2.393026in}%
\pgfsys@useobject{currentmarker}{}%
\end{pgfscope}%
\begin{pgfscope}%
\pgfsys@transformshift{0.676057in}{2.131602in}%
\pgfsys@useobject{currentmarker}{}%
\end{pgfscope}%
\begin{pgfscope}%
\pgfsys@transformshift{0.695777in}{2.015084in}%
\pgfsys@useobject{currentmarker}{}%
\end{pgfscope}%
\begin{pgfscope}%
\pgfsys@transformshift{0.713856in}{1.984300in}%
\pgfsys@useobject{currentmarker}{}%
\end{pgfscope}%
\begin{pgfscope}%
\pgfsys@transformshift{0.732167in}{2.033346in}%
\pgfsys@useobject{currentmarker}{}%
\end{pgfscope}%
\begin{pgfscope}%
\pgfsys@transformshift{0.754002in}{2.178000in}%
\pgfsys@useobject{currentmarker}{}%
\end{pgfscope}%
\begin{pgfscope}%
\pgfsys@transformshift{0.775836in}{2.521053in}%
\pgfsys@useobject{currentmarker}{}%
\end{pgfscope}%
\begin{pgfscope}%
\pgfsys@transformshift{0.791095in}{2.828361in}%
\pgfsys@useobject{currentmarker}{}%
\end{pgfscope}%
\begin{pgfscope}%
\pgfsys@transformshift{0.809643in}{2.899350in}%
\pgfsys@useobject{currentmarker}{}%
\end{pgfscope}%
\begin{pgfscope}%
\pgfsys@transformshift{0.828894in}{2.648521in}%
\pgfsys@useobject{currentmarker}{}%
\end{pgfscope}%
\begin{pgfscope}%
\pgfsys@transformshift{0.848146in}{2.280719in}%
\pgfsys@useobject{currentmarker}{}%
\end{pgfscope}%
\begin{pgfscope}%
\pgfsys@transformshift{0.865988in}{2.079233in}%
\pgfsys@useobject{currentmarker}{}%
\end{pgfscope}%
\begin{pgfscope}%
\pgfsys@transformshift{0.887588in}{1.990028in}%
\pgfsys@useobject{currentmarker}{}%
\end{pgfscope}%
\begin{pgfscope}%
\pgfsys@transformshift{0.905899in}{1.992021in}%
\pgfsys@useobject{currentmarker}{}%
\end{pgfscope}%
\begin{pgfscope}%
\pgfsys@transformshift{0.924682in}{2.056857in}%
\pgfsys@useobject{currentmarker}{}%
\end{pgfscope}%
\begin{pgfscope}%
\pgfsys@transformshift{0.943698in}{2.193635in}%
\pgfsys@useobject{currentmarker}{}%
\end{pgfscope}%
\begin{pgfscope}%
\pgfsys@transformshift{0.961070in}{2.491114in}%
\pgfsys@useobject{currentmarker}{}%
\end{pgfscope}%
\begin{pgfscope}%
\pgfsys@transformshift{0.982904in}{2.206688in}%
\pgfsys@useobject{currentmarker}{}%
\end{pgfscope}%
\begin{pgfscope}%
\pgfsys@transformshift{1.001217in}{2.050888in}%
\pgfsys@useobject{currentmarker}{}%
\end{pgfscope}%
\begin{pgfscope}%
\pgfsys@transformshift{1.022581in}{1.979130in}%
\pgfsys@useobject{currentmarker}{}%
\end{pgfscope}%
\begin{pgfscope}%
\pgfsys@transformshift{1.040658in}{1.993063in}%
\pgfsys@useobject{currentmarker}{}%
\end{pgfscope}%
\begin{pgfscope}%
\pgfsys@transformshift{1.058502in}{2.062380in}%
\pgfsys@useobject{currentmarker}{}%
\end{pgfscope}%
\begin{pgfscope}%
\pgfsys@transformshift{1.079631in}{2.263678in}%
\pgfsys@useobject{currentmarker}{}%
\end{pgfscope}%
\begin{pgfscope}%
\pgfsys@transformshift{1.100525in}{2.660036in}%
\pgfsys@useobject{currentmarker}{}%
\end{pgfscope}%
\begin{pgfscope}%
\pgfsys@transformshift{1.115550in}{2.854565in}%
\pgfsys@useobject{currentmarker}{}%
\end{pgfscope}%
\begin{pgfscope}%
\pgfsys@transformshift{1.136445in}{2.758194in}%
\pgfsys@useobject{currentmarker}{}%
\end{pgfscope}%
\begin{pgfscope}%
\pgfsys@transformshift{1.160393in}{2.309192in}%
\pgfsys@useobject{currentmarker}{}%
\end{pgfscope}%
\begin{pgfscope}%
\pgfsys@transformshift{1.171661in}{2.128115in}%
\pgfsys@useobject{currentmarker}{}%
\end{pgfscope}%
\begin{pgfscope}%
\pgfsys@transformshift{1.192792in}{2.010421in}%
\pgfsys@useobject{currentmarker}{}%
\end{pgfscope}%
\begin{pgfscope}%
\pgfsys@transformshift{1.216738in}{1.975101in}%
\pgfsys@useobject{currentmarker}{}%
\end{pgfscope}%
\begin{pgfscope}%
\pgfsys@transformshift{1.231529in}{2.012874in}%
\pgfsys@useobject{currentmarker}{}%
\end{pgfscope}%
\begin{pgfscope}%
\pgfsys@transformshift{1.249371in}{2.093932in}%
\pgfsys@useobject{currentmarker}{}%
\end{pgfscope}%
\begin{pgfscope}%
\pgfsys@transformshift{1.270265in}{2.321458in}%
\pgfsys@useobject{currentmarker}{}%
\end{pgfscope}%
\begin{pgfscope}%
\pgfsys@transformshift{1.288813in}{2.645695in}%
\pgfsys@useobject{currentmarker}{}%
\end{pgfscope}%
\begin{pgfscope}%
\pgfsys@transformshift{1.310177in}{2.850295in}%
\pgfsys@useobject{currentmarker}{}%
\end{pgfscope}%
\begin{pgfscope}%
\pgfsys@transformshift{1.330602in}{2.682788in}%
\pgfsys@useobject{currentmarker}{}%
\end{pgfscope}%
\begin{pgfscope}%
\pgfsys@transformshift{1.348210in}{2.330267in}%
\pgfsys@useobject{currentmarker}{}%
\end{pgfscope}%
\begin{pgfscope}%
\pgfsys@transformshift{1.366287in}{2.101907in}%
\pgfsys@useobject{currentmarker}{}%
\end{pgfscope}%
\begin{pgfscope}%
\pgfsys@transformshift{1.387183in}{2.000138in}%
\pgfsys@useobject{currentmarker}{}%
\end{pgfscope}%
\begin{pgfscope}%
\pgfsys@transformshift{1.405495in}{1.972096in}%
\pgfsys@useobject{currentmarker}{}%
\end{pgfscope}%
\begin{pgfscope}%
\pgfsys@transformshift{1.423806in}{1.985543in}%
\pgfsys@useobject{currentmarker}{}%
\end{pgfscope}%
\begin{pgfscope}%
\pgfsys@transformshift{1.443997in}{2.056585in}%
\pgfsys@useobject{currentmarker}{}%
\end{pgfscope}%
\begin{pgfscope}%
\pgfsys@transformshift{1.465362in}{2.223309in}%
\pgfsys@useobject{currentmarker}{}%
\end{pgfscope}%
\begin{pgfscope}%
\pgfsys@transformshift{1.483439in}{2.532117in}%
\pgfsys@useobject{currentmarker}{}%
\end{pgfscope}%
\begin{pgfscope}%
\pgfsys@transformshift{1.501282in}{2.764280in}%
\pgfsys@useobject{currentmarker}{}%
\end{pgfscope}%
\begin{pgfscope}%
\pgfsys@transformshift{1.521942in}{2.831984in}%
\pgfsys@useobject{currentmarker}{}%
\end{pgfscope}%
\begin{pgfscope}%
\pgfsys@transformshift{1.540019in}{2.625694in}%
\pgfsys@useobject{currentmarker}{}%
\end{pgfscope}%
\begin{pgfscope}%
\pgfsys@transformshift{1.558098in}{2.289387in}%
\pgfsys@useobject{currentmarker}{}%
\end{pgfscope}%
\begin{pgfscope}%
\pgfsys@transformshift{1.579227in}{2.065159in}%
\pgfsys@useobject{currentmarker}{}%
\end{pgfscope}%
\begin{pgfscope}%
\pgfsys@transformshift{1.597304in}{2.009537in}%
\pgfsys@useobject{currentmarker}{}%
\end{pgfscope}%
\begin{pgfscope}%
\pgfsys@transformshift{1.615382in}{1.972369in}%
\pgfsys@useobject{currentmarker}{}%
\end{pgfscope}%
\begin{pgfscope}%
\pgfsys@transformshift{1.636042in}{1.984273in}%
\pgfsys@useobject{currentmarker}{}%
\end{pgfscope}%
\begin{pgfscope}%
\pgfsys@transformshift{1.656702in}{2.038379in}%
\pgfsys@useobject{currentmarker}{}%
\end{pgfscope}%
\begin{pgfscope}%
\pgfsys@transformshift{1.674308in}{2.147602in}%
\pgfsys@useobject{currentmarker}{}%
\end{pgfscope}%
\begin{pgfscope}%
\pgfsys@transformshift{1.692387in}{2.374477in}%
\pgfsys@useobject{currentmarker}{}%
\end{pgfscope}%
\begin{pgfscope}%
\pgfsys@transformshift{1.713516in}{2.669815in}%
\pgfsys@useobject{currentmarker}{}%
\end{pgfscope}%
\begin{pgfscope}%
\pgfsys@transformshift{1.730890in}{2.828083in}%
\pgfsys@useobject{currentmarker}{}%
\end{pgfscope}%
\begin{pgfscope}%
\pgfsys@transformshift{1.750610in}{2.743660in}%
\pgfsys@useobject{currentmarker}{}%
\end{pgfscope}%
\begin{pgfscope}%
\pgfsys@transformshift{1.770566in}{2.510309in}%
\pgfsys@useobject{currentmarker}{}%
\end{pgfscope}%
\begin{pgfscope}%
\pgfsys@transformshift{1.790992in}{2.167563in}%
\pgfsys@useobject{currentmarker}{}%
\end{pgfscope}%
\begin{pgfscope}%
\pgfsys@transformshift{1.809772in}{2.035750in}%
\pgfsys@useobject{currentmarker}{}%
\end{pgfscope}%
\begin{pgfscope}%
\pgfsys@transformshift{1.827617in}{1.989661in}%
\pgfsys@useobject{currentmarker}{}%
\end{pgfscope}%
\begin{pgfscope}%
\pgfsys@transformshift{1.848040in}{1.970307in}%
\pgfsys@useobject{currentmarker}{}%
\end{pgfscope}%
\begin{pgfscope}%
\pgfsys@transformshift{1.866353in}{1.989534in}%
\pgfsys@useobject{currentmarker}{}%
\end{pgfscope}%
\begin{pgfscope}%
\pgfsys@transformshift{1.887482in}{2.040607in}%
\pgfsys@useobject{currentmarker}{}%
\end{pgfscope}%
\begin{pgfscope}%
\pgfsys@transformshift{1.903682in}{2.132962in}%
\pgfsys@useobject{currentmarker}{}%
\end{pgfscope}%
\begin{pgfscope}%
\pgfsys@transformshift{1.924342in}{2.363258in}%
\pgfsys@useobject{currentmarker}{}%
\end{pgfscope}%
\begin{pgfscope}%
\pgfsys@transformshift{1.942419in}{2.683335in}%
\pgfsys@useobject{currentmarker}{}%
\end{pgfscope}%
\begin{pgfscope}%
\pgfsys@transformshift{1.963315in}{2.818799in}%
\pgfsys@useobject{currentmarker}{}%
\end{pgfscope}%
\begin{pgfscope}%
\pgfsys@transformshift{1.982095in}{2.697033in}%
\pgfsys@useobject{currentmarker}{}%
\end{pgfscope}%
\begin{pgfscope}%
\pgfsys@transformshift{2.001817in}{2.471688in}%
\pgfsys@useobject{currentmarker}{}%
\end{pgfscope}%
\begin{pgfscope}%
\pgfsys@transformshift{2.019660in}{2.190087in}%
\pgfsys@useobject{currentmarker}{}%
\end{pgfscope}%
\begin{pgfscope}%
\pgfsys@transformshift{2.038676in}{2.063554in}%
\pgfsys@useobject{currentmarker}{}%
\end{pgfscope}%
\begin{pgfscope}%
\pgfsys@transformshift{2.058631in}{2.011548in}%
\pgfsys@useobject{currentmarker}{}%
\end{pgfscope}%
\begin{pgfscope}%
\pgfsys@transformshift{2.079996in}{1.977860in}%
\pgfsys@useobject{currentmarker}{}%
\end{pgfscope}%
\begin{pgfscope}%
\pgfsys@transformshift{2.096196in}{1.969724in}%
\pgfsys@useobject{currentmarker}{}%
\end{pgfscope}%
\begin{pgfscope}%
\pgfsys@transformshift{2.116621in}{2.006547in}%
\pgfsys@useobject{currentmarker}{}%
\end{pgfscope}%
\begin{pgfscope}%
\pgfsys@transformshift{2.133758in}{2.071795in}%
\pgfsys@useobject{currentmarker}{}%
\end{pgfscope}%
\begin{pgfscope}%
\pgfsys@transformshift{2.156532in}{2.255822in}%
\pgfsys@useobject{currentmarker}{}%
\end{pgfscope}%
\begin{pgfscope}%
\pgfsys@transformshift{2.172966in}{2.503795in}%
\pgfsys@useobject{currentmarker}{}%
\end{pgfscope}%
\begin{pgfscope}%
\pgfsys@transformshift{2.193860in}{2.781260in}%
\pgfsys@useobject{currentmarker}{}%
\end{pgfscope}%
\begin{pgfscope}%
\pgfsys@transformshift{2.211703in}{2.794586in}%
\pgfsys@useobject{currentmarker}{}%
\end{pgfscope}%
\begin{pgfscope}%
\pgfsys@transformshift{2.233068in}{2.529143in}%
\pgfsys@useobject{currentmarker}{}%
\end{pgfscope}%
\begin{pgfscope}%
\pgfsys@transformshift{2.251849in}{2.247322in}%
\pgfsys@useobject{currentmarker}{}%
\end{pgfscope}%
\begin{pgfscope}%
\pgfsys@transformshift{2.270396in}{2.085168in}%
\pgfsys@useobject{currentmarker}{}%
\end{pgfscope}%
\begin{pgfscope}%
\pgfsys@transformshift{2.291762in}{2.092541in}%
\pgfsys@useobject{currentmarker}{}%
\end{pgfscope}%
\begin{pgfscope}%
\pgfsys@transformshift{2.309839in}{2.010875in}%
\pgfsys@useobject{currentmarker}{}%
\end{pgfscope}%
\begin{pgfscope}%
\pgfsys@transformshift{2.330733in}{1.969922in}%
\pgfsys@useobject{currentmarker}{}%
\end{pgfscope}%
\begin{pgfscope}%
\pgfsys@transformshift{2.351158in}{1.981093in}%
\pgfsys@useobject{currentmarker}{}%
\end{pgfscope}%
\begin{pgfscope}%
\pgfsys@transformshift{2.366652in}{2.017960in}%
\pgfsys@useobject{currentmarker}{}%
\end{pgfscope}%
\begin{pgfscope}%
\pgfsys@transformshift{2.383323in}{2.104104in}%
\pgfsys@useobject{currentmarker}{}%
\end{pgfscope}%
\begin{pgfscope}%
\pgfsys@transformshift{2.404686in}{2.250215in}%
\pgfsys@useobject{currentmarker}{}%
\end{pgfscope}%
\begin{pgfscope}%
\pgfsys@transformshift{2.425580in}{2.494120in}%
\pgfsys@useobject{currentmarker}{}%
\end{pgfscope}%
\begin{pgfscope}%
\pgfsys@transformshift{2.443423in}{2.751561in}%
\pgfsys@useobject{currentmarker}{}%
\end{pgfscope}%
\begin{pgfscope}%
\pgfsys@transformshift{2.464788in}{2.811862in}%
\pgfsys@useobject{currentmarker}{}%
\end{pgfscope}%
\begin{pgfscope}%
\pgfsys@transformshift{2.482162in}{2.729269in}%
\pgfsys@useobject{currentmarker}{}%
\end{pgfscope}%
\begin{pgfscope}%
\pgfsys@transformshift{2.500239in}{2.534072in}%
\pgfsys@useobject{currentmarker}{}%
\end{pgfscope}%
\begin{pgfscope}%
\pgfsys@transformshift{2.518786in}{2.273538in}%
\pgfsys@useobject{currentmarker}{}%
\end{pgfscope}%
\begin{pgfscope}%
\pgfsys@transformshift{2.539446in}{2.122480in}%
\pgfsys@useobject{currentmarker}{}%
\end{pgfscope}%
\begin{pgfscope}%
\pgfsys@transformshift{2.557289in}{2.017865in}%
\pgfsys@useobject{currentmarker}{}%
\end{pgfscope}%
\begin{pgfscope}%
\pgfsys@transformshift{2.581938in}{1.971709in}%
\pgfsys@useobject{currentmarker}{}%
\end{pgfscope}%
\begin{pgfscope}%
\pgfsys@transformshift{2.597434in}{1.971921in}%
\pgfsys@useobject{currentmarker}{}%
\end{pgfscope}%
\begin{pgfscope}%
\pgfsys@transformshift{2.614573in}{2.009246in}%
\pgfsys@useobject{currentmarker}{}%
\end{pgfscope}%
\begin{pgfscope}%
\pgfsys@transformshift{2.636877in}{2.077683in}%
\pgfsys@useobject{currentmarker}{}%
\end{pgfscope}%
\begin{pgfscope}%
\pgfsys@transformshift{2.654250in}{2.237438in}%
\pgfsys@useobject{currentmarker}{}%
\end{pgfscope}%
\begin{pgfscope}%
\pgfsys@transformshift{2.675379in}{2.544383in}%
\pgfsys@useobject{currentmarker}{}%
\end{pgfscope}%
\begin{pgfscope}%
\pgfsys@transformshift{2.693690in}{2.779316in}%
\pgfsys@useobject{currentmarker}{}%
\end{pgfscope}%
\begin{pgfscope}%
\pgfsys@transformshift{2.711533in}{2.797954in}%
\pgfsys@useobject{currentmarker}{}%
\end{pgfscope}%
\begin{pgfscope}%
\pgfsys@transformshift{2.731958in}{2.534910in}%
\pgfsys@useobject{currentmarker}{}%
\end{pgfscope}%
\begin{pgfscope}%
\pgfsys@transformshift{2.749801in}{2.311460in}%
\pgfsys@useobject{currentmarker}{}%
\end{pgfscope}%
\begin{pgfscope}%
\pgfsys@transformshift{2.768583in}{2.143878in}%
\pgfsys@useobject{currentmarker}{}%
\end{pgfscope}%
\begin{pgfscope}%
\pgfsys@transformshift{2.789478in}{2.034394in}%
\pgfsys@useobject{currentmarker}{}%
\end{pgfscope}%
\begin{pgfscope}%
\pgfsys@transformshift{2.807791in}{1.989334in}%
\pgfsys@useobject{currentmarker}{}%
\end{pgfscope}%
\begin{pgfscope}%
\pgfsys@transformshift{2.828920in}{1.968291in}%
\pgfsys@useobject{currentmarker}{}%
\end{pgfscope}%
\begin{pgfscope}%
\pgfsys@transformshift{2.846293in}{1.987145in}%
\pgfsys@useobject{currentmarker}{}%
\end{pgfscope}%
\begin{pgfscope}%
\pgfsys@transformshift{2.864605in}{2.020496in}%
\pgfsys@useobject{currentmarker}{}%
\end{pgfscope}%
\begin{pgfscope}%
\pgfsys@transformshift{2.886204in}{2.110964in}%
\pgfsys@useobject{currentmarker}{}%
\end{pgfscope}%
\begin{pgfscope}%
\pgfsys@transformshift{2.903813in}{2.285783in}%
\pgfsys@useobject{currentmarker}{}%
\end{pgfscope}%
\begin{pgfscope}%
\pgfsys@transformshift{2.921655in}{2.448680in}%
\pgfsys@useobject{currentmarker}{}%
\end{pgfscope}%
\begin{pgfscope}%
\pgfsys@transformshift{2.943724in}{2.765236in}%
\pgfsys@useobject{currentmarker}{}%
\end{pgfscope}%
\begin{pgfscope}%
\pgfsys@transformshift{2.963915in}{2.808092in}%
\pgfsys@useobject{currentmarker}{}%
\end{pgfscope}%
\begin{pgfscope}%
\pgfsys@transformshift{2.981523in}{2.721858in}%
\pgfsys@useobject{currentmarker}{}%
\end{pgfscope}%
\begin{pgfscope}%
\pgfsys@transformshift{3.000303in}{2.552611in}%
\pgfsys@useobject{currentmarker}{}%
\end{pgfscope}%
\begin{pgfscope}%
\pgfsys@transformshift{3.017677in}{2.275712in}%
\pgfsys@useobject{currentmarker}{}%
\end{pgfscope}%
\begin{pgfscope}%
\pgfsys@transformshift{3.039745in}{2.082380in}%
\pgfsys@useobject{currentmarker}{}%
\end{pgfscope}%
\begin{pgfscope}%
\pgfsys@transformshift{3.056413in}{2.023344in}%
\pgfsys@useobject{currentmarker}{}%
\end{pgfscope}%
\begin{pgfscope}%
\pgfsys@transformshift{3.078719in}{1.978287in}%
\pgfsys@useobject{currentmarker}{}%
\end{pgfscope}%
\begin{pgfscope}%
\pgfsys@transformshift{3.096561in}{1.973278in}%
\pgfsys@useobject{currentmarker}{}%
\end{pgfscope}%
\begin{pgfscope}%
\pgfsys@transformshift{3.117455in}{2.002180in}%
\pgfsys@useobject{currentmarker}{}%
\end{pgfscope}%
\begin{pgfscope}%
\pgfsys@transformshift{3.136238in}{2.056936in}%
\pgfsys@useobject{currentmarker}{}%
\end{pgfscope}%
\begin{pgfscope}%
\pgfsys@transformshift{3.154549in}{2.122355in}%
\pgfsys@useobject{currentmarker}{}%
\end{pgfscope}%
\begin{pgfscope}%
\pgfsys@transformshift{3.173097in}{2.320467in}%
\pgfsys@useobject{currentmarker}{}%
\end{pgfscope}%
\begin{pgfscope}%
\pgfsys@transformshift{3.191174in}{2.595821in}%
\pgfsys@useobject{currentmarker}{}%
\end{pgfscope}%
\begin{pgfscope}%
\pgfsys@transformshift{3.212303in}{2.819661in}%
\pgfsys@useobject{currentmarker}{}%
\end{pgfscope}%
\begin{pgfscope}%
\pgfsys@transformshift{3.231319in}{2.819953in}%
\pgfsys@useobject{currentmarker}{}%
\end{pgfscope}%
\begin{pgfscope}%
\pgfsys@transformshift{3.247050in}{2.609956in}%
\pgfsys@useobject{currentmarker}{}%
\end{pgfscope}%
\begin{pgfscope}%
\pgfsys@transformshift{3.272170in}{2.306543in}%
\pgfsys@useobject{currentmarker}{}%
\end{pgfscope}%
\begin{pgfscope}%
\pgfsys@transformshift{3.288839in}{2.166677in}%
\pgfsys@useobject{currentmarker}{}%
\end{pgfscope}%
\begin{pgfscope}%
\pgfsys@transformshift{3.307621in}{2.059657in}%
\pgfsys@useobject{currentmarker}{}%
\end{pgfscope}%
\begin{pgfscope}%
\pgfsys@transformshift{3.328046in}{2.015370in}%
\pgfsys@useobject{currentmarker}{}%
\end{pgfscope}%
\begin{pgfscope}%
\pgfsys@transformshift{3.346358in}{2.009114in}%
\pgfsys@useobject{currentmarker}{}%
\end{pgfscope}%
\begin{pgfscope}%
\pgfsys@transformshift{3.363966in}{1.974961in}%
\pgfsys@useobject{currentmarker}{}%
\end{pgfscope}%
\begin{pgfscope}%
\pgfsys@transformshift{3.385566in}{1.987632in}%
\pgfsys@useobject{currentmarker}{}%
\end{pgfscope}%
\begin{pgfscope}%
\pgfsys@transformshift{3.403643in}{2.030907in}%
\pgfsys@useobject{currentmarker}{}%
\end{pgfscope}%
\begin{pgfscope}%
\pgfsys@transformshift{3.427120in}{2.111721in}%
\pgfsys@useobject{currentmarker}{}%
\end{pgfscope}%
\begin{pgfscope}%
\pgfsys@transformshift{3.443319in}{2.273188in}%
\pgfsys@useobject{currentmarker}{}%
\end{pgfscope}%
\begin{pgfscope}%
\pgfsys@transformshift{3.463979in}{2.519675in}%
\pgfsys@useobject{currentmarker}{}%
\end{pgfscope}%
\begin{pgfscope}%
\pgfsys@transformshift{3.482291in}{2.778583in}%
\pgfsys@useobject{currentmarker}{}%
\end{pgfscope}%
\begin{pgfscope}%
\pgfsys@transformshift{3.499899in}{2.858115in}%
\pgfsys@useobject{currentmarker}{}%
\end{pgfscope}%
\begin{pgfscope}%
\pgfsys@transformshift{3.518446in}{2.783410in}%
\pgfsys@useobject{currentmarker}{}%
\end{pgfscope}%
\begin{pgfscope}%
\pgfsys@transformshift{3.538637in}{2.479268in}%
\pgfsys@useobject{currentmarker}{}%
\end{pgfscope}%
\begin{pgfscope}%
\pgfsys@transformshift{3.557889in}{2.226507in}%
\pgfsys@useobject{currentmarker}{}%
\end{pgfscope}%
\begin{pgfscope}%
\pgfsys@transformshift{3.577843in}{2.093087in}%
\pgfsys@useobject{currentmarker}{}%
\end{pgfscope}%
\begin{pgfscope}%
\pgfsys@transformshift{3.600148in}{2.029065in}%
\pgfsys@useobject{currentmarker}{}%
\end{pgfscope}%
\begin{pgfscope}%
\pgfsys@transformshift{3.614939in}{1.989379in}%
\pgfsys@useobject{currentmarker}{}%
\end{pgfscope}%
\begin{pgfscope}%
\pgfsys@transformshift{3.635362in}{1.988951in}%
\pgfsys@useobject{currentmarker}{}%
\end{pgfscope}%
\begin{pgfscope}%
\pgfsys@transformshift{3.652970in}{1.980871in}%
\pgfsys@useobject{currentmarker}{}%
\end{pgfscope}%
\begin{pgfscope}%
\pgfsys@transformshift{3.674336in}{2.022530in}%
\pgfsys@useobject{currentmarker}{}%
\end{pgfscope}%
\begin{pgfscope}%
\pgfsys@transformshift{3.693118in}{2.095566in}%
\pgfsys@useobject{currentmarker}{}%
\end{pgfscope}%
\begin{pgfscope}%
\pgfsys@transformshift{3.711898in}{2.240403in}%
\pgfsys@useobject{currentmarker}{}%
\end{pgfscope}%
\begin{pgfscope}%
\pgfsys@transformshift{3.729272in}{2.458516in}%
\pgfsys@useobject{currentmarker}{}%
\end{pgfscope}%
\begin{pgfscope}%
\pgfsys@transformshift{3.748992in}{2.601563in}%
\pgfsys@useobject{currentmarker}{}%
\end{pgfscope}%
\begin{pgfscope}%
\pgfsys@transformshift{3.770123in}{2.854589in}%
\pgfsys@useobject{currentmarker}{}%
\end{pgfscope}%
\begin{pgfscope}%
\pgfsys@transformshift{3.789608in}{2.863432in}%
\pgfsys@useobject{currentmarker}{}%
\end{pgfscope}%
\begin{pgfscope}%
\pgfsys@transformshift{3.809565in}{2.701413in}%
\pgfsys@useobject{currentmarker}{}%
\end{pgfscope}%
\begin{pgfscope}%
\pgfsys@transformshift{3.828582in}{2.435400in}%
\pgfsys@useobject{currentmarker}{}%
\end{pgfscope}%
\begin{pgfscope}%
\pgfsys@transformshift{3.845016in}{2.231120in}%
\pgfsys@useobject{currentmarker}{}%
\end{pgfscope}%
\begin{pgfscope}%
\pgfsys@transformshift{3.864267in}{2.101563in}%
\pgfsys@useobject{currentmarker}{}%
\end{pgfscope}%
\begin{pgfscope}%
\pgfsys@transformshift{3.889387in}{2.005840in}%
\pgfsys@useobject{currentmarker}{}%
\end{pgfscope}%
\begin{pgfscope}%
\pgfsys@transformshift{3.905116in}{1.988898in}%
\pgfsys@useobject{currentmarker}{}%
\end{pgfscope}%
\begin{pgfscope}%
\pgfsys@transformshift{3.924838in}{1.977729in}%
\pgfsys@useobject{currentmarker}{}%
\end{pgfscope}%
\begin{pgfscope}%
\pgfsys@transformshift{3.939394in}{1.998874in}%
\pgfsys@useobject{currentmarker}{}%
\end{pgfscope}%
\begin{pgfscope}%
\pgfsys@transformshift{3.961931in}{2.056804in}%
\pgfsys@useobject{currentmarker}{}%
\end{pgfscope}%
\begin{pgfscope}%
\pgfsys@transformshift{3.980948in}{2.159039in}%
\pgfsys@useobject{currentmarker}{}%
\end{pgfscope}%
\begin{pgfscope}%
\pgfsys@transformshift{3.999965in}{2.283062in}%
\pgfsys@useobject{currentmarker}{}%
\end{pgfscope}%
\begin{pgfscope}%
\pgfsys@transformshift{4.019216in}{2.553934in}%
\pgfsys@useobject{currentmarker}{}%
\end{pgfscope}%
\begin{pgfscope}%
\pgfsys@transformshift{4.038233in}{2.797611in}%
\pgfsys@useobject{currentmarker}{}%
\end{pgfscope}%
\begin{pgfscope}%
\pgfsys@transformshift{4.057484in}{2.894975in}%
\pgfsys@useobject{currentmarker}{}%
\end{pgfscope}%
\begin{pgfscope}%
\pgfsys@transformshift{4.076030in}{2.895077in}%
\pgfsys@useobject{currentmarker}{}%
\end{pgfscope}%
\begin{pgfscope}%
\pgfsys@transformshift{4.098570in}{2.767263in}%
\pgfsys@useobject{currentmarker}{}%
\end{pgfscope}%
\begin{pgfscope}%
\pgfsys@transformshift{4.116412in}{2.568957in}%
\pgfsys@useobject{currentmarker}{}%
\end{pgfscope}%
\begin{pgfscope}%
\pgfsys@transformshift{4.134960in}{2.323196in}%
\pgfsys@useobject{currentmarker}{}%
\end{pgfscope}%
\begin{pgfscope}%
\pgfsys@transformshift{4.154446in}{2.161021in}%
\pgfsys@useobject{currentmarker}{}%
\end{pgfscope}%
\begin{pgfscope}%
\pgfsys@transformshift{4.173462in}{2.097365in}%
\pgfsys@useobject{currentmarker}{}%
\end{pgfscope}%
\begin{pgfscope}%
\pgfsys@transformshift{4.189896in}{2.027628in}%
\pgfsys@useobject{currentmarker}{}%
\end{pgfscope}%
\begin{pgfscope}%
\pgfsys@transformshift{4.211494in}{1.987096in}%
\pgfsys@useobject{currentmarker}{}%
\end{pgfscope}%
\begin{pgfscope}%
\pgfsys@transformshift{4.231919in}{1.994407in}%
\pgfsys@useobject{currentmarker}{}%
\end{pgfscope}%
\begin{pgfscope}%
\pgfsys@transformshift{4.250467in}{2.049812in}%
\pgfsys@useobject{currentmarker}{}%
\end{pgfscope}%
\begin{pgfscope}%
\pgfsys@transformshift{4.268075in}{2.901014in}%
\pgfsys@useobject{currentmarker}{}%
\end{pgfscope}%
\begin{pgfscope}%
\pgfsys@transformshift{4.290378in}{2.618440in}%
\pgfsys@useobject{currentmarker}{}%
\end{pgfscope}%
\begin{pgfscope}%
\pgfsys@transformshift{4.309629in}{2.305744in}%
\pgfsys@useobject{currentmarker}{}%
\end{pgfscope}%
\begin{pgfscope}%
\pgfsys@transformshift{4.324889in}{2.131432in}%
\pgfsys@useobject{currentmarker}{}%
\end{pgfscope}%
\begin{pgfscope}%
\pgfsys@transformshift{4.350010in}{2.023133in}%
\pgfsys@useobject{currentmarker}{}%
\end{pgfscope}%
\begin{pgfscope}%
\pgfsys@transformshift{4.365271in}{1.986391in}%
\pgfsys@useobject{currentmarker}{}%
\end{pgfscope}%
\begin{pgfscope}%
\pgfsys@transformshift{4.384991in}{2.004005in}%
\pgfsys@useobject{currentmarker}{}%
\end{pgfscope}%
\begin{pgfscope}%
\pgfsys@transformshift{4.404948in}{2.077282in}%
\pgfsys@useobject{currentmarker}{}%
\end{pgfscope}%
\begin{pgfscope}%
\pgfsys@transformshift{4.422790in}{2.212385in}%
\pgfsys@useobject{currentmarker}{}%
\end{pgfscope}%
\begin{pgfscope}%
\pgfsys@transformshift{4.441336in}{2.432721in}%
\pgfsys@useobject{currentmarker}{}%
\end{pgfscope}%
\begin{pgfscope}%
\pgfsys@transformshift{4.460353in}{2.764337in}%
\pgfsys@useobject{currentmarker}{}%
\end{pgfscope}%
\begin{pgfscope}%
\pgfsys@transformshift{4.479604in}{2.937732in}%
\pgfsys@useobject{currentmarker}{}%
\end{pgfscope}%
\begin{pgfscope}%
\pgfsys@transformshift{4.474206in}{2.905299in}%
\pgfsys@useobject{currentmarker}{}%
\end{pgfscope}%
\begin{pgfscope}%
\pgfsys@transformshift{4.456832in}{2.590321in}%
\pgfsys@useobject{currentmarker}{}%
\end{pgfscope}%
\begin{pgfscope}%
\pgfsys@transformshift{4.436875in}{2.220759in}%
\pgfsys@useobject{currentmarker}{}%
\end{pgfscope}%
\begin{pgfscope}%
\pgfsys@transformshift{4.416686in}{2.059718in}%
\pgfsys@useobject{currentmarker}{}%
\end{pgfscope}%
\begin{pgfscope}%
\pgfsys@transformshift{4.396495in}{1.987488in}%
\pgfsys@useobject{currentmarker}{}%
\end{pgfscope}%
\begin{pgfscope}%
\pgfsys@transformshift{4.378184in}{2.011150in}%
\pgfsys@useobject{currentmarker}{}%
\end{pgfscope}%
\begin{pgfscope}%
\pgfsys@transformshift{4.357993in}{2.135570in}%
\pgfsys@useobject{currentmarker}{}%
\end{pgfscope}%
\begin{pgfscope}%
\pgfsys@transformshift{4.339211in}{2.389293in}%
\pgfsys@useobject{currentmarker}{}%
\end{pgfscope}%
\begin{pgfscope}%
\pgfsys@transformshift{4.320428in}{2.790040in}%
\pgfsys@useobject{currentmarker}{}%
\end{pgfscope}%
\begin{pgfscope}%
\pgfsys@transformshift{4.302586in}{2.915908in}%
\pgfsys@useobject{currentmarker}{}%
\end{pgfscope}%
\begin{pgfscope}%
\pgfsys@transformshift{4.281926in}{2.638309in}%
\pgfsys@useobject{currentmarker}{}%
\end{pgfscope}%
\begin{pgfscope}%
\pgfsys@transformshift{4.260797in}{2.226218in}%
\pgfsys@useobject{currentmarker}{}%
\end{pgfscope}%
\begin{pgfscope}%
\pgfsys@transformshift{4.242484in}{2.062589in}%
\pgfsys@useobject{currentmarker}{}%
\end{pgfscope}%
\begin{pgfscope}%
\pgfsys@transformshift{4.224641in}{1.990542in}%
\pgfsys@useobject{currentmarker}{}%
\end{pgfscope}%
\begin{pgfscope}%
\pgfsys@transformshift{4.206330in}{1.996090in}%
\pgfsys@useobject{currentmarker}{}%
\end{pgfscope}%
\begin{pgfscope}%
\pgfsys@transformshift{4.186844in}{2.077363in}%
\pgfsys@useobject{currentmarker}{}%
\end{pgfscope}%
\begin{pgfscope}%
\pgfsys@transformshift{4.166419in}{2.315392in}%
\pgfsys@useobject{currentmarker}{}%
\end{pgfscope}%
\begin{pgfscope}%
\pgfsys@transformshift{4.146697in}{2.694176in}%
\pgfsys@useobject{currentmarker}{}%
\end{pgfscope}%
\begin{pgfscope}%
\pgfsys@transformshift{4.128151in}{2.891472in}%
\pgfsys@useobject{currentmarker}{}%
\end{pgfscope}%
\begin{pgfscope}%
\pgfsys@transformshift{4.109837in}{2.724548in}%
\pgfsys@useobject{currentmarker}{}%
\end{pgfscope}%
\begin{pgfscope}%
\pgfsys@transformshift{4.092935in}{2.338832in}%
\pgfsys@useobject{currentmarker}{}%
\end{pgfscope}%
\begin{pgfscope}%
\pgfsys@transformshift{4.067580in}{2.072121in}%
\pgfsys@useobject{currentmarker}{}%
\end{pgfscope}%
\begin{pgfscope}%
\pgfsys@transformshift{4.051380in}{2.005764in}%
\pgfsys@useobject{currentmarker}{}%
\end{pgfscope}%
\begin{pgfscope}%
\pgfsys@transformshift{4.033301in}{1.979041in}%
\pgfsys@useobject{currentmarker}{}%
\end{pgfscope}%
\begin{pgfscope}%
\pgfsys@transformshift{4.013112in}{2.025119in}%
\pgfsys@useobject{currentmarker}{}%
\end{pgfscope}%
\begin{pgfscope}%
\pgfsys@transformshift{3.994096in}{2.134489in}%
\pgfsys@useobject{currentmarker}{}%
\end{pgfscope}%
\begin{pgfscope}%
\pgfsys@transformshift{3.972730in}{2.462153in}%
\pgfsys@useobject{currentmarker}{}%
\end{pgfscope}%
\begin{pgfscope}%
\pgfsys@transformshift{3.954185in}{2.764741in}%
\pgfsys@useobject{currentmarker}{}%
\end{pgfscope}%
\begin{pgfscope}%
\pgfsys@transformshift{3.938454in}{2.869477in}%
\pgfsys@useobject{currentmarker}{}%
\end{pgfscope}%
\begin{pgfscope}%
\pgfsys@transformshift{3.917091in}{2.609194in}%
\pgfsys@useobject{currentmarker}{}%
\end{pgfscope}%
\begin{pgfscope}%
\pgfsys@transformshift{3.899483in}{2.246425in}%
\pgfsys@useobject{currentmarker}{}%
\end{pgfscope}%
\begin{pgfscope}%
\pgfsys@transformshift{3.875535in}{2.065279in}%
\pgfsys@useobject{currentmarker}{}%
\end{pgfscope}%
\begin{pgfscope}%
\pgfsys@transformshift{3.860275in}{2.001501in}%
\pgfsys@useobject{currentmarker}{}%
\end{pgfscope}%
\begin{pgfscope}%
\pgfsys@transformshift{3.841024in}{1.973756in}%
\pgfsys@useobject{currentmarker}{}%
\end{pgfscope}%
\begin{pgfscope}%
\pgfsys@transformshift{3.822712in}{2.004008in}%
\pgfsys@useobject{currentmarker}{}%
\end{pgfscope}%
\begin{pgfscope}%
\pgfsys@transformshift{3.801816in}{2.064382in}%
\pgfsys@useobject{currentmarker}{}%
\end{pgfscope}%
\begin{pgfscope}%
\pgfsys@transformshift{3.779044in}{2.266332in}%
\pgfsys@useobject{currentmarker}{}%
\end{pgfscope}%
\begin{pgfscope}%
\pgfsys@transformshift{3.764722in}{2.529992in}%
\pgfsys@useobject{currentmarker}{}%
\end{pgfscope}%
\begin{pgfscope}%
\pgfsys@transformshift{3.741011in}{2.819303in}%
\pgfsys@useobject{currentmarker}{}%
\end{pgfscope}%
\begin{pgfscope}%
\pgfsys@transformshift{3.721525in}{2.806413in}%
\pgfsys@useobject{currentmarker}{}%
\end{pgfscope}%
\begin{pgfscope}%
\pgfsys@transformshift{3.707438in}{2.587107in}%
\pgfsys@useobject{currentmarker}{}%
\end{pgfscope}%
\begin{pgfscope}%
\pgfsys@transformshift{3.684431in}{2.224935in}%
\pgfsys@useobject{currentmarker}{}%
\end{pgfscope}%
\begin{pgfscope}%
\pgfsys@transformshift{3.665883in}{2.093375in}%
\pgfsys@useobject{currentmarker}{}%
\end{pgfscope}%
\begin{pgfscope}%
\pgfsys@transformshift{3.647336in}{2.788207in}%
\pgfsys@useobject{currentmarker}{}%
\end{pgfscope}%
\begin{pgfscope}%
\pgfsys@transformshift{3.628321in}{2.818414in}%
\pgfsys@useobject{currentmarker}{}%
\end{pgfscope}%
\begin{pgfscope}%
\pgfsys@transformshift{3.610947in}{2.527283in}%
\pgfsys@useobject{currentmarker}{}%
\end{pgfscope}%
\begin{pgfscope}%
\pgfsys@transformshift{3.590522in}{2.206302in}%
\pgfsys@useobject{currentmarker}{}%
\end{pgfscope}%
\begin{pgfscope}%
\pgfsys@transformshift{3.571739in}{2.057030in}%
\pgfsys@useobject{currentmarker}{}%
\end{pgfscope}%
\begin{pgfscope}%
\pgfsys@transformshift{3.549436in}{1.988921in}%
\pgfsys@useobject{currentmarker}{}%
\end{pgfscope}%
\begin{pgfscope}%
\pgfsys@transformshift{3.532063in}{1.973822in}%
\pgfsys@useobject{currentmarker}{}%
\end{pgfscope}%
\begin{pgfscope}%
\pgfsys@transformshift{3.513751in}{2.006610in}%
\pgfsys@useobject{currentmarker}{}%
\end{pgfscope}%
\begin{pgfscope}%
\pgfsys@transformshift{3.494969in}{2.093686in}%
\pgfsys@useobject{currentmarker}{}%
\end{pgfscope}%
\begin{pgfscope}%
\pgfsys@transformshift{3.473369in}{2.355258in}%
\pgfsys@useobject{currentmarker}{}%
\end{pgfscope}%
\begin{pgfscope}%
\pgfsys@transformshift{3.454589in}{2.699562in}%
\pgfsys@useobject{currentmarker}{}%
\end{pgfscope}%
\begin{pgfscope}%
\pgfsys@transformshift{3.437450in}{2.836176in}%
\pgfsys@useobject{currentmarker}{}%
\end{pgfscope}%
\begin{pgfscope}%
\pgfsys@transformshift{3.417259in}{2.687491in}%
\pgfsys@useobject{currentmarker}{}%
\end{pgfscope}%
\begin{pgfscope}%
\pgfsys@transformshift{3.399887in}{2.375547in}%
\pgfsys@useobject{currentmarker}{}%
\end{pgfscope}%
\begin{pgfscope}%
\pgfsys@transformshift{3.374061in}{2.141980in}%
\pgfsys@useobject{currentmarker}{}%
\end{pgfscope}%
\begin{pgfscope}%
\pgfsys@transformshift{3.359740in}{2.052140in}%
\pgfsys@useobject{currentmarker}{}%
\end{pgfscope}%
\begin{pgfscope}%
\pgfsys@transformshift{3.338142in}{1.985466in}%
\pgfsys@useobject{currentmarker}{}%
\end{pgfscope}%
\begin{pgfscope}%
\pgfsys@transformshift{3.319829in}{1.972121in}%
\pgfsys@useobject{currentmarker}{}%
\end{pgfscope}%
\begin{pgfscope}%
\pgfsys@transformshift{3.300577in}{2.006743in}%
\pgfsys@useobject{currentmarker}{}%
\end{pgfscope}%
\begin{pgfscope}%
\pgfsys@transformshift{3.283204in}{2.063809in}%
\pgfsys@useobject{currentmarker}{}%
\end{pgfscope}%
\begin{pgfscope}%
\pgfsys@transformshift{3.263015in}{2.208635in}%
\pgfsys@useobject{currentmarker}{}%
\end{pgfscope}%
\begin{pgfscope}%
\pgfsys@transformshift{3.246110in}{2.419041in}%
\pgfsys@useobject{currentmarker}{}%
\end{pgfscope}%
\begin{pgfscope}%
\pgfsys@transformshift{3.224981in}{2.779190in}%
\pgfsys@useobject{currentmarker}{}%
\end{pgfscope}%
\begin{pgfscope}%
\pgfsys@transformshift{3.207842in}{2.817562in}%
\pgfsys@useobject{currentmarker}{}%
\end{pgfscope}%
\begin{pgfscope}%
\pgfsys@transformshift{3.185070in}{2.530275in}%
\pgfsys@useobject{currentmarker}{}%
\end{pgfscope}%
\begin{pgfscope}%
\pgfsys@transformshift{3.166053in}{2.243837in}%
\pgfsys@useobject{currentmarker}{}%
\end{pgfscope}%
\begin{pgfscope}%
\pgfsys@transformshift{3.147976in}{2.077844in}%
\pgfsys@useobject{currentmarker}{}%
\end{pgfscope}%
\begin{pgfscope}%
\pgfsys@transformshift{3.129663in}{2.006111in}%
\pgfsys@useobject{currentmarker}{}%
\end{pgfscope}%
\begin{pgfscope}%
\pgfsys@transformshift{3.107360in}{1.970287in}%
\pgfsys@useobject{currentmarker}{}%
\end{pgfscope}%
\begin{pgfscope}%
\pgfsys@transformshift{3.092569in}{1.982915in}%
\pgfsys@useobject{currentmarker}{}%
\end{pgfscope}%
\begin{pgfscope}%
\pgfsys@transformshift{3.069797in}{2.039760in}%
\pgfsys@useobject{currentmarker}{}%
\end{pgfscope}%
\begin{pgfscope}%
\pgfsys@transformshift{3.051015in}{2.144724in}%
\pgfsys@useobject{currentmarker}{}%
\end{pgfscope}%
\begin{pgfscope}%
\pgfsys@transformshift{3.032938in}{2.395679in}%
\pgfsys@useobject{currentmarker}{}%
\end{pgfscope}%
\begin{pgfscope}%
\pgfsys@transformshift{3.014156in}{2.590575in}%
\pgfsys@useobject{currentmarker}{}%
\end{pgfscope}%
\begin{pgfscope}%
\pgfsys@transformshift{2.995842in}{2.760584in}%
\pgfsys@useobject{currentmarker}{}%
\end{pgfscope}%
\begin{pgfscope}%
\pgfsys@transformshift{2.978471in}{2.800676in}%
\pgfsys@useobject{currentmarker}{}%
\end{pgfscope}%
\begin{pgfscope}%
\pgfsys@transformshift{2.955228in}{2.510632in}%
\pgfsys@useobject{currentmarker}{}%
\end{pgfscope}%
\begin{pgfscope}%
\pgfsys@transformshift{2.938089in}{2.239539in}%
\pgfsys@useobject{currentmarker}{}%
\end{pgfscope}%
\begin{pgfscope}%
\pgfsys@transformshift{2.915317in}{2.070266in}%
\pgfsys@useobject{currentmarker}{}%
\end{pgfscope}%
\begin{pgfscope}%
\pgfsys@transformshift{2.899586in}{2.004088in}%
\pgfsys@useobject{currentmarker}{}%
\end{pgfscope}%
\begin{pgfscope}%
\pgfsys@transformshift{2.881039in}{1.969233in}%
\pgfsys@useobject{currentmarker}{}%
\end{pgfscope}%
\begin{pgfscope}%
\pgfsys@transformshift{2.858970in}{1.974015in}%
\pgfsys@useobject{currentmarker}{}%
\end{pgfscope}%
\begin{pgfscope}%
\pgfsys@transformshift{2.840893in}{2.012124in}%
\pgfsys@useobject{currentmarker}{}%
\end{pgfscope}%
\begin{pgfscope}%
\pgfsys@transformshift{2.822111in}{2.101902in}%
\pgfsys@useobject{currentmarker}{}%
\end{pgfscope}%
\begin{pgfscope}%
\pgfsys@transformshift{2.800042in}{2.341519in}%
\pgfsys@useobject{currentmarker}{}%
\end{pgfscope}%
\begin{pgfscope}%
\pgfsys@transformshift{2.785017in}{2.596896in}%
\pgfsys@useobject{currentmarker}{}%
\end{pgfscope}%
\begin{pgfscope}%
\pgfsys@transformshift{2.759427in}{2.814698in}%
\pgfsys@useobject{currentmarker}{}%
\end{pgfscope}%
\begin{pgfscope}%
\pgfsys@transformshift{2.747454in}{2.769970in}%
\pgfsys@useobject{currentmarker}{}%
\end{pgfscope}%
\begin{pgfscope}%
\pgfsys@transformshift{2.725855in}{2.499751in}%
\pgfsys@useobject{currentmarker}{}%
\end{pgfscope}%
\begin{pgfscope}%
\pgfsys@transformshift{2.705900in}{2.220937in}%
\pgfsys@useobject{currentmarker}{}%
\end{pgfscope}%
\begin{pgfscope}%
\pgfsys@transformshift{2.685943in}{2.070775in}%
\pgfsys@useobject{currentmarker}{}%
\end{pgfscope}%
\begin{pgfscope}%
\pgfsys@transformshift{2.668101in}{1.998438in}%
\pgfsys@useobject{currentmarker}{}%
\end{pgfscope}%
\begin{pgfscope}%
\pgfsys@transformshift{2.645563in}{1.969951in}%
\pgfsys@useobject{currentmarker}{}%
\end{pgfscope}%
\begin{pgfscope}%
\pgfsys@transformshift{2.626547in}{1.969909in}%
\pgfsys@useobject{currentmarker}{}%
\end{pgfscope}%
\begin{pgfscope}%
\pgfsys@transformshift{2.608939in}{2.000923in}%
\pgfsys@useobject{currentmarker}{}%
\end{pgfscope}%
\begin{pgfscope}%
\pgfsys@transformshift{2.590391in}{2.055969in}%
\pgfsys@useobject{currentmarker}{}%
\end{pgfscope}%
\begin{pgfscope}%
\pgfsys@transformshift{2.571374in}{2.199888in}%
\pgfsys@useobject{currentmarker}{}%
\end{pgfscope}%
\begin{pgfscope}%
\pgfsys@transformshift{2.553297in}{2.483813in}%
\pgfsys@useobject{currentmarker}{}%
\end{pgfscope}%
\begin{pgfscope}%
\pgfsys@transformshift{2.533577in}{2.494535in}%
\pgfsys@useobject{currentmarker}{}%
\end{pgfscope}%
\begin{pgfscope}%
\pgfsys@transformshift{2.515734in}{2.771016in}%
\pgfsys@useobject{currentmarker}{}%
\end{pgfscope}%
\begin{pgfscope}%
\pgfsys@transformshift{2.494369in}{2.786041in}%
\pgfsys@useobject{currentmarker}{}%
\end{pgfscope}%
\begin{pgfscope}%
\pgfsys@transformshift{2.475118in}{2.570693in}%
\pgfsys@useobject{currentmarker}{}%
\end{pgfscope}%
\begin{pgfscope}%
\pgfsys@transformshift{2.458684in}{2.296148in}%
\pgfsys@useobject{currentmarker}{}%
\end{pgfscope}%
\begin{pgfscope}%
\pgfsys@transformshift{2.437085in}{2.095570in}%
\pgfsys@useobject{currentmarker}{}%
\end{pgfscope}%
\begin{pgfscope}%
\pgfsys@transformshift{2.417833in}{2.033647in}%
\pgfsys@useobject{currentmarker}{}%
\end{pgfscope}%
\begin{pgfscope}%
\pgfsys@transformshift{2.399051in}{1.982568in}%
\pgfsys@useobject{currentmarker}{}%
\end{pgfscope}%
\begin{pgfscope}%
\pgfsys@transformshift{2.379331in}{1.968425in}%
\pgfsys@useobject{currentmarker}{}%
\end{pgfscope}%
\begin{pgfscope}%
\pgfsys@transformshift{2.359845in}{2.004052in}%
\pgfsys@useobject{currentmarker}{}%
\end{pgfscope}%
\begin{pgfscope}%
\pgfsys@transformshift{2.338951in}{2.084674in}%
\pgfsys@useobject{currentmarker}{}%
\end{pgfscope}%
\begin{pgfscope}%
\pgfsys@transformshift{2.320637in}{2.262352in}%
\pgfsys@useobject{currentmarker}{}%
\end{pgfscope}%
\begin{pgfscope}%
\pgfsys@transformshift{2.303264in}{2.539077in}%
\pgfsys@useobject{currentmarker}{}%
\end{pgfscope}%
\begin{pgfscope}%
\pgfsys@transformshift{2.285421in}{2.791894in}%
\pgfsys@useobject{currentmarker}{}%
\end{pgfscope}%
\begin{pgfscope}%
\pgfsys@transformshift{2.265467in}{2.810792in}%
\pgfsys@useobject{currentmarker}{}%
\end{pgfscope}%
\begin{pgfscope}%
\pgfsys@transformshift{2.244572in}{2.593175in}%
\pgfsys@useobject{currentmarker}{}%
\end{pgfscope}%
\begin{pgfscope}%
\pgfsys@transformshift{2.222033in}{2.249364in}%
\pgfsys@useobject{currentmarker}{}%
\end{pgfscope}%
\begin{pgfscope}%
\pgfsys@transformshift{2.200199in}{2.131680in}%
\pgfsys@useobject{currentmarker}{}%
\end{pgfscope}%
\begin{pgfscope}%
\pgfsys@transformshift{2.189400in}{2.060512in}%
\pgfsys@useobject{currentmarker}{}%
\end{pgfscope}%
\begin{pgfscope}%
\pgfsys@transformshift{2.165922in}{1.999803in}%
\pgfsys@useobject{currentmarker}{}%
\end{pgfscope}%
\begin{pgfscope}%
\pgfsys@transformshift{2.147611in}{1.973097in}%
\pgfsys@useobject{currentmarker}{}%
\end{pgfscope}%
\begin{pgfscope}%
\pgfsys@transformshift{2.127654in}{1.992775in}%
\pgfsys@useobject{currentmarker}{}%
\end{pgfscope}%
\begin{pgfscope}%
\pgfsys@transformshift{2.111690in}{2.026875in}%
\pgfsys@useobject{currentmarker}{}%
\end{pgfscope}%
\begin{pgfscope}%
\pgfsys@transformshift{2.090326in}{2.132230in}%
\pgfsys@useobject{currentmarker}{}%
\end{pgfscope}%
\begin{pgfscope}%
\pgfsys@transformshift{2.070135in}{2.287466in}%
\pgfsys@useobject{currentmarker}{}%
\end{pgfscope}%
\begin{pgfscope}%
\pgfsys@transformshift{2.051824in}{2.585944in}%
\pgfsys@useobject{currentmarker}{}%
\end{pgfscope}%
\begin{pgfscope}%
\pgfsys@transformshift{2.033276in}{2.775121in}%
\pgfsys@useobject{currentmarker}{}%
\end{pgfscope}%
\begin{pgfscope}%
\pgfsys@transformshift{2.014730in}{2.818923in}%
\pgfsys@useobject{currentmarker}{}%
\end{pgfscope}%
\begin{pgfscope}%
\pgfsys@transformshift{1.996886in}{2.758027in}%
\pgfsys@useobject{currentmarker}{}%
\end{pgfscope}%
\begin{pgfscope}%
\pgfsys@transformshift{1.972471in}{2.361186in}%
\pgfsys@useobject{currentmarker}{}%
\end{pgfscope}%
\begin{pgfscope}%
\pgfsys@transformshift{1.956037in}{2.290228in}%
\pgfsys@useobject{currentmarker}{}%
\end{pgfscope}%
\begin{pgfscope}%
\pgfsys@transformshift{1.937020in}{2.130069in}%
\pgfsys@useobject{currentmarker}{}%
\end{pgfscope}%
\begin{pgfscope}%
\pgfsys@transformshift{1.918707in}{2.038020in}%
\pgfsys@useobject{currentmarker}{}%
\end{pgfscope}%
\begin{pgfscope}%
\pgfsys@transformshift{1.900864in}{1.985967in}%
\pgfsys@useobject{currentmarker}{}%
\end{pgfscope}%
\begin{pgfscope}%
\pgfsys@transformshift{1.878327in}{1.973963in}%
\pgfsys@useobject{currentmarker}{}%
\end{pgfscope}%
\begin{pgfscope}%
\pgfsys@transformshift{1.860250in}{2.000334in}%
\pgfsys@useobject{currentmarker}{}%
\end{pgfscope}%
\begin{pgfscope}%
\pgfsys@transformshift{1.844754in}{2.053577in}%
\pgfsys@useobject{currentmarker}{}%
\end{pgfscope}%
\begin{pgfscope}%
\pgfsys@transformshift{1.823625in}{2.176348in}%
\pgfsys@useobject{currentmarker}{}%
\end{pgfscope}%
\begin{pgfscope}%
\pgfsys@transformshift{1.801791in}{2.472650in}%
\pgfsys@useobject{currentmarker}{}%
\end{pgfscope}%
\begin{pgfscope}%
\pgfsys@transformshift{1.783243in}{2.681151in}%
\pgfsys@useobject{currentmarker}{}%
\end{pgfscope}%
\begin{pgfscope}%
\pgfsys@transformshift{1.765166in}{2.823250in}%
\pgfsys@useobject{currentmarker}{}%
\end{pgfscope}%
\begin{pgfscope}%
\pgfsys@transformshift{1.746620in}{2.795625in}%
\pgfsys@useobject{currentmarker}{}%
\end{pgfscope}%
\begin{pgfscope}%
\pgfsys@transformshift{1.726664in}{2.585812in}%
\pgfsys@useobject{currentmarker}{}%
\end{pgfscope}%
\begin{pgfscope}%
\pgfsys@transformshift{1.706943in}{2.289738in}%
\pgfsys@useobject{currentmarker}{}%
\end{pgfscope}%
\begin{pgfscope}%
\pgfsys@transformshift{1.688396in}{2.136496in}%
\pgfsys@useobject{currentmarker}{}%
\end{pgfscope}%
\begin{pgfscope}%
\pgfsys@transformshift{1.666327in}{2.067897in}%
\pgfsys@useobject{currentmarker}{}%
\end{pgfscope}%
\begin{pgfscope}%
\pgfsys@transformshift{1.651536in}{2.015007in}%
\pgfsys@useobject{currentmarker}{}%
\end{pgfscope}%
\begin{pgfscope}%
\pgfsys@transformshift{1.629233in}{1.976762in}%
\pgfsys@useobject{currentmarker}{}%
\end{pgfscope}%
\begin{pgfscope}%
\pgfsys@transformshift{1.607868in}{1.986094in}%
\pgfsys@useobject{currentmarker}{}%
\end{pgfscope}%
\begin{pgfscope}%
\pgfsys@transformshift{1.589791in}{2.009623in}%
\pgfsys@useobject{currentmarker}{}%
\end{pgfscope}%
\begin{pgfscope}%
\pgfsys@transformshift{1.571245in}{2.074246in}%
\pgfsys@useobject{currentmarker}{}%
\end{pgfscope}%
\begin{pgfscope}%
\pgfsys@transformshift{1.553166in}{2.203483in}%
\pgfsys@useobject{currentmarker}{}%
\end{pgfscope}%
\begin{pgfscope}%
\pgfsys@transformshift{1.534149in}{2.416903in}%
\pgfsys@useobject{currentmarker}{}%
\end{pgfscope}%
\begin{pgfscope}%
\pgfsys@transformshift{1.514664in}{2.720991in}%
\pgfsys@useobject{currentmarker}{}%
\end{pgfscope}%
\begin{pgfscope}%
\pgfsys@transformshift{1.497056in}{2.816159in}%
\pgfsys@useobject{currentmarker}{}%
\end{pgfscope}%
\begin{pgfscope}%
\pgfsys@transformshift{1.477101in}{2.851022in}%
\pgfsys@useobject{currentmarker}{}%
\end{pgfscope}%
\begin{pgfscope}%
\pgfsys@transformshift{1.457614in}{2.761425in}%
\pgfsys@useobject{currentmarker}{}%
\end{pgfscope}%
\begin{pgfscope}%
\pgfsys@transformshift{1.438128in}{2.467423in}%
\pgfsys@useobject{currentmarker}{}%
\end{pgfscope}%
\begin{pgfscope}%
\pgfsys@transformshift{1.417468in}{2.257471in}%
\pgfsys@useobject{currentmarker}{}%
\end{pgfscope}%
\begin{pgfscope}%
\pgfsys@transformshift{1.399391in}{2.116518in}%
\pgfsys@useobject{currentmarker}{}%
\end{pgfscope}%
\begin{pgfscope}%
\pgfsys@transformshift{1.382017in}{2.071462in}%
\pgfsys@useobject{currentmarker}{}%
\end{pgfscope}%
\begin{pgfscope}%
\pgfsys@transformshift{1.359009in}{2.010774in}%
\pgfsys@useobject{currentmarker}{}%
\end{pgfscope}%
\begin{pgfscope}%
\pgfsys@transformshift{1.341403in}{1.979597in}%
\pgfsys@useobject{currentmarker}{}%
\end{pgfscope}%
\begin{pgfscope}%
\pgfsys@transformshift{1.324264in}{1.981602in}%
\pgfsys@useobject{currentmarker}{}%
\end{pgfscope}%
\begin{pgfscope}%
\pgfsys@transformshift{1.303838in}{1.988034in}%
\pgfsys@useobject{currentmarker}{}%
\end{pgfscope}%
\begin{pgfscope}%
\pgfsys@transformshift{1.281066in}{2.027167in}%
\pgfsys@useobject{currentmarker}{}%
\end{pgfscope}%
\begin{pgfscope}%
\pgfsys@transformshift{1.265805in}{2.091790in}%
\pgfsys@useobject{currentmarker}{}%
\end{pgfscope}%
\begin{pgfscope}%
\pgfsys@transformshift{1.246085in}{2.260813in}%
\pgfsys@useobject{currentmarker}{}%
\end{pgfscope}%
\begin{pgfscope}%
\pgfsys@transformshift{1.224485in}{2.574484in}%
\pgfsys@useobject{currentmarker}{}%
\end{pgfscope}%
\begin{pgfscope}%
\pgfsys@transformshift{1.207817in}{2.715572in}%
\pgfsys@useobject{currentmarker}{}%
\end{pgfscope}%
\begin{pgfscope}%
\pgfsys@transformshift{1.187391in}{2.872321in}%
\pgfsys@useobject{currentmarker}{}%
\end{pgfscope}%
\begin{pgfscope}%
\pgfsys@transformshift{1.163445in}{2.820228in}%
\pgfsys@useobject{currentmarker}{}%
\end{pgfscope}%
\begin{pgfscope}%
\pgfsys@transformshift{1.148654in}{2.667395in}%
\pgfsys@useobject{currentmarker}{}%
\end{pgfscope}%
\begin{pgfscope}%
\pgfsys@transformshift{1.131046in}{2.410859in}%
\pgfsys@useobject{currentmarker}{}%
\end{pgfscope}%
\begin{pgfscope}%
\pgfsys@transformshift{1.110855in}{2.181940in}%
\pgfsys@useobject{currentmarker}{}%
\end{pgfscope}%
\begin{pgfscope}%
\pgfsys@transformshift{1.090195in}{2.077554in}%
\pgfsys@useobject{currentmarker}{}%
\end{pgfscope}%
\begin{pgfscope}%
\pgfsys@transformshift{1.073293in}{2.043118in}%
\pgfsys@useobject{currentmarker}{}%
\end{pgfscope}%
\begin{pgfscope}%
\pgfsys@transformshift{1.050519in}{1.995136in}%
\pgfsys@useobject{currentmarker}{}%
\end{pgfscope}%
\begin{pgfscope}%
\pgfsys@transformshift{1.029859in}{1.982069in}%
\pgfsys@useobject{currentmarker}{}%
\end{pgfscope}%
\begin{pgfscope}%
\pgfsys@transformshift{1.009433in}{2.009074in}%
\pgfsys@useobject{currentmarker}{}%
\end{pgfscope}%
\begin{pgfscope}%
\pgfsys@transformshift{0.994408in}{2.047964in}%
\pgfsys@useobject{currentmarker}{}%
\end{pgfscope}%
\begin{pgfscope}%
\pgfsys@transformshift{0.975861in}{2.114064in}%
\pgfsys@useobject{currentmarker}{}%
\end{pgfscope}%
\begin{pgfscope}%
\pgfsys@transformshift{0.954732in}{2.238083in}%
\pgfsys@useobject{currentmarker}{}%
\end{pgfscope}%
\begin{pgfscope}%
\pgfsys@transformshift{0.938063in}{2.434397in}%
\pgfsys@useobject{currentmarker}{}%
\end{pgfscope}%
\begin{pgfscope}%
\pgfsys@transformshift{0.920455in}{2.709580in}%
\pgfsys@useobject{currentmarker}{}%
\end{pgfscope}%
\begin{pgfscope}%
\pgfsys@transformshift{0.902378in}{2.867792in}%
\pgfsys@useobject{currentmarker}{}%
\end{pgfscope}%
\begin{pgfscope}%
\pgfsys@transformshift{0.881718in}{2.903400in}%
\pgfsys@useobject{currentmarker}{}%
\end{pgfscope}%
\begin{pgfscope}%
\pgfsys@transformshift{0.860588in}{2.729455in}%
\pgfsys@useobject{currentmarker}{}%
\end{pgfscope}%
\begin{pgfscope}%
\pgfsys@transformshift{0.842511in}{2.469451in}%
\pgfsys@useobject{currentmarker}{}%
\end{pgfscope}%
\begin{pgfscope}%
\pgfsys@transformshift{0.822554in}{2.230962in}%
\pgfsys@useobject{currentmarker}{}%
\end{pgfscope}%
\begin{pgfscope}%
\pgfsys@transformshift{0.802834in}{2.098009in}%
\pgfsys@useobject{currentmarker}{}%
\end{pgfscope}%
\begin{pgfscope}%
\pgfsys@transformshift{0.783583in}{2.044978in}%
\pgfsys@useobject{currentmarker}{}%
\end{pgfscope}%
\begin{pgfscope}%
\pgfsys@transformshift{0.765506in}{2.008445in}%
\pgfsys@useobject{currentmarker}{}%
\end{pgfscope}%
\begin{pgfscope}%
\pgfsys@transformshift{0.747427in}{1.986768in}%
\pgfsys@useobject{currentmarker}{}%
\end{pgfscope}%
\begin{pgfscope}%
\pgfsys@transformshift{0.725124in}{2.009994in}%
\pgfsys@useobject{currentmarker}{}%
\end{pgfscope}%
\begin{pgfscope}%
\pgfsys@transformshift{0.707047in}{2.057418in}%
\pgfsys@useobject{currentmarker}{}%
\end{pgfscope}%
\begin{pgfscope}%
\pgfsys@transformshift{0.690144in}{2.152731in}%
\pgfsys@useobject{currentmarker}{}%
\end{pgfscope}%
\begin{pgfscope}%
\pgfsys@transformshift{0.669013in}{2.302106in}%
\pgfsys@useobject{currentmarker}{}%
\end{pgfscope}%
\begin{pgfscope}%
\pgfsys@transformshift{0.651405in}{2.495201in}%
\pgfsys@useobject{currentmarker}{}%
\end{pgfscope}%
\begin{pgfscope}%
\pgfsys@transformshift{0.650702in}{2.483993in}%
\pgfsys@useobject{currentmarker}{}%
\end{pgfscope}%
\begin{pgfscope}%
\pgfsys@transformshift{0.653754in}{2.393474in}%
\pgfsys@useobject{currentmarker}{}%
\end{pgfscope}%
\begin{pgfscope}%
\pgfsys@transformshift{0.676057in}{2.108920in}%
\pgfsys@useobject{currentmarker}{}%
\end{pgfscope}%
\begin{pgfscope}%
\pgfsys@transformshift{0.695074in}{2.010136in}%
\pgfsys@useobject{currentmarker}{}%
\end{pgfscope}%
\begin{pgfscope}%
\pgfsys@transformshift{0.713620in}{1.988568in}%
\pgfsys@useobject{currentmarker}{}%
\end{pgfscope}%
\begin{pgfscope}%
\pgfsys@transformshift{0.733107in}{2.045973in}%
\pgfsys@useobject{currentmarker}{}%
\end{pgfscope}%
\begin{pgfscope}%
\pgfsys@transformshift{0.750244in}{2.176161in}%
\pgfsys@useobject{currentmarker}{}%
\end{pgfscope}%
\begin{pgfscope}%
\pgfsys@transformshift{0.769967in}{2.454364in}%
\pgfsys@useobject{currentmarker}{}%
\end{pgfscope}%
\begin{pgfscope}%
\pgfsys@transformshift{0.794616in}{2.887830in}%
\pgfsys@useobject{currentmarker}{}%
\end{pgfscope}%
\begin{pgfscope}%
\pgfsys@transformshift{0.810347in}{2.884590in}%
\pgfsys@useobject{currentmarker}{}%
\end{pgfscope}%
\begin{pgfscope}%
\pgfsys@transformshift{0.810816in}{2.753453in}%
\pgfsys@useobject{currentmarker}{}%
\end{pgfscope}%
\begin{pgfscope}%
\pgfsys@transformshift{0.830303in}{2.606321in}%
\pgfsys@useobject{currentmarker}{}%
\end{pgfscope}%
\begin{pgfscope}%
\pgfsys@transformshift{0.847440in}{2.254120in}%
\pgfsys@useobject{currentmarker}{}%
\end{pgfscope}%
\begin{pgfscope}%
\pgfsys@transformshift{0.866223in}{2.069539in}%
\pgfsys@useobject{currentmarker}{}%
\end{pgfscope}%
\begin{pgfscope}%
\pgfsys@transformshift{0.887588in}{1.987284in}%
\pgfsys@useobject{currentmarker}{}%
\end{pgfscope}%
\begin{pgfscope}%
\pgfsys@transformshift{0.906134in}{1.995042in}%
\pgfsys@useobject{currentmarker}{}%
\end{pgfscope}%
\begin{pgfscope}%
\pgfsys@transformshift{0.924447in}{2.061889in}%
\pgfsys@useobject{currentmarker}{}%
\end{pgfscope}%
\begin{pgfscope}%
\pgfsys@transformshift{0.943698in}{2.216405in}%
\pgfsys@useobject{currentmarker}{}%
\end{pgfscope}%
\begin{pgfscope}%
\pgfsys@transformshift{0.964593in}{2.579602in}%
\pgfsys@useobject{currentmarker}{}%
\end{pgfscope}%
\begin{pgfscope}%
\pgfsys@transformshift{0.982670in}{2.861885in}%
\pgfsys@useobject{currentmarker}{}%
\end{pgfscope}%
\begin{pgfscope}%
\pgfsys@transformshift{1.001217in}{2.829103in}%
\pgfsys@useobject{currentmarker}{}%
\end{pgfscope}%
\begin{pgfscope}%
\pgfsys@transformshift{1.021877in}{2.464559in}%
\pgfsys@useobject{currentmarker}{}%
\end{pgfscope}%
\begin{pgfscope}%
\pgfsys@transformshift{1.040189in}{2.179035in}%
\pgfsys@useobject{currentmarker}{}%
\end{pgfscope}%
\begin{pgfscope}%
\pgfsys@transformshift{1.057797in}{2.041615in}%
\pgfsys@useobject{currentmarker}{}%
\end{pgfscope}%
\begin{pgfscope}%
\pgfsys@transformshift{1.078691in}{1.979603in}%
\pgfsys@useobject{currentmarker}{}%
\end{pgfscope}%
\begin{pgfscope}%
\pgfsys@transformshift{1.097474in}{1.995529in}%
\pgfsys@useobject{currentmarker}{}%
\end{pgfscope}%
\begin{pgfscope}%
\pgfsys@transformshift{1.115082in}{2.060587in}%
\pgfsys@useobject{currentmarker}{}%
\end{pgfscope}%
\begin{pgfscope}%
\pgfsys@transformshift{1.134333in}{2.238770in}%
\pgfsys@useobject{currentmarker}{}%
\end{pgfscope}%
\begin{pgfscope}%
\pgfsys@transformshift{1.156636in}{2.640382in}%
\pgfsys@useobject{currentmarker}{}%
\end{pgfscope}%
\begin{pgfscope}%
\pgfsys@transformshift{1.174478in}{2.857347in}%
\pgfsys@useobject{currentmarker}{}%
\end{pgfscope}%
\begin{pgfscope}%
\pgfsys@transformshift{1.192321in}{2.780844in}%
\pgfsys@useobject{currentmarker}{}%
\end{pgfscope}%
\begin{pgfscope}%
\pgfsys@transformshift{1.212746in}{2.391751in}%
\pgfsys@useobject{currentmarker}{}%
\end{pgfscope}%
\begin{pgfscope}%
\pgfsys@transformshift{1.233172in}{2.130265in}%
\pgfsys@useobject{currentmarker}{}%
\end{pgfscope}%
\begin{pgfscope}%
\pgfsys@transformshift{1.249606in}{2.019358in}%
\pgfsys@useobject{currentmarker}{}%
\end{pgfscope}%
\begin{pgfscope}%
\pgfsys@transformshift{1.270971in}{1.974132in}%
\pgfsys@useobject{currentmarker}{}%
\end{pgfscope}%
\begin{pgfscope}%
\pgfsys@transformshift{1.291865in}{2.002111in}%
\pgfsys@useobject{currentmarker}{}%
\end{pgfscope}%
\begin{pgfscope}%
\pgfsys@transformshift{1.311351in}{2.090433in}%
\pgfsys@useobject{currentmarker}{}%
\end{pgfscope}%
\begin{pgfscope}%
\pgfsys@transformshift{1.327316in}{2.230472in}%
\pgfsys@useobject{currentmarker}{}%
\end{pgfscope}%
\begin{pgfscope}%
\pgfsys@transformshift{1.353140in}{2.683020in}%
\pgfsys@useobject{currentmarker}{}%
\end{pgfscope}%
\begin{pgfscope}%
\pgfsys@transformshift{1.365349in}{2.814212in}%
\pgfsys@useobject{currentmarker}{}%
\end{pgfscope}%
\begin{pgfscope}%
\pgfsys@transformshift{1.387887in}{2.778377in}%
\pgfsys@useobject{currentmarker}{}%
\end{pgfscope}%
\begin{pgfscope}%
\pgfsys@transformshift{1.404086in}{2.464966in}%
\pgfsys@useobject{currentmarker}{}%
\end{pgfscope}%
\begin{pgfscope}%
\pgfsys@transformshift{1.425451in}{2.146341in}%
\pgfsys@useobject{currentmarker}{}%
\end{pgfscope}%
\begin{pgfscope}%
\pgfsys@transformshift{1.444703in}{2.022418in}%
\pgfsys@useobject{currentmarker}{}%
\end{pgfscope}%
\begin{pgfscope}%
\pgfsys@transformshift{1.461605in}{1.981105in}%
\pgfsys@useobject{currentmarker}{}%
\end{pgfscope}%
\begin{pgfscope}%
\pgfsys@transformshift{1.481796in}{1.980458in}%
\pgfsys@useobject{currentmarker}{}%
\end{pgfscope}%
\begin{pgfscope}%
\pgfsys@transformshift{1.500579in}{2.029299in}%
\pgfsys@useobject{currentmarker}{}%
\end{pgfscope}%
\begin{pgfscope}%
\pgfsys@transformshift{1.518655in}{2.131601in}%
\pgfsys@useobject{currentmarker}{}%
\end{pgfscope}%
\begin{pgfscope}%
\pgfsys@transformshift{1.539550in}{2.388837in}%
\pgfsys@useobject{currentmarker}{}%
\end{pgfscope}%
\begin{pgfscope}%
\pgfsys@transformshift{1.557392in}{2.721147in}%
\pgfsys@useobject{currentmarker}{}%
\end{pgfscope}%
\begin{pgfscope}%
\pgfsys@transformshift{1.579930in}{2.822985in}%
\pgfsys@useobject{currentmarker}{}%
\end{pgfscope}%
\begin{pgfscope}%
\pgfsys@transformshift{1.598009in}{2.608405in}%
\pgfsys@useobject{currentmarker}{}%
\end{pgfscope}%
\begin{pgfscope}%
\pgfsys@transformshift{1.618903in}{2.315573in}%
\pgfsys@useobject{currentmarker}{}%
\end{pgfscope}%
\begin{pgfscope}%
\pgfsys@transformshift{1.636042in}{2.107566in}%
\pgfsys@useobject{currentmarker}{}%
\end{pgfscope}%
\begin{pgfscope}%
\pgfsys@transformshift{1.652476in}{2.017007in}%
\pgfsys@useobject{currentmarker}{}%
\end{pgfscope}%
\begin{pgfscope}%
\pgfsys@transformshift{1.673136in}{1.972699in}%
\pgfsys@useobject{currentmarker}{}%
\end{pgfscope}%
\begin{pgfscope}%
\pgfsys@transformshift{1.694265in}{1.988332in}%
\pgfsys@useobject{currentmarker}{}%
\end{pgfscope}%
\begin{pgfscope}%
\pgfsys@transformshift{1.712342in}{2.037028in}%
\pgfsys@useobject{currentmarker}{}%
\end{pgfscope}%
\begin{pgfscope}%
\pgfsys@transformshift{1.733473in}{2.130159in}%
\pgfsys@useobject{currentmarker}{}%
\end{pgfscope}%
\begin{pgfscope}%
\pgfsys@transformshift{1.752253in}{2.334537in}%
\pgfsys@useobject{currentmarker}{}%
\end{pgfscope}%
\begin{pgfscope}%
\pgfsys@transformshift{1.769392in}{2.577570in}%
\pgfsys@useobject{currentmarker}{}%
\end{pgfscope}%
\begin{pgfscope}%
\pgfsys@transformshift{1.790286in}{2.820342in}%
\pgfsys@useobject{currentmarker}{}%
\end{pgfscope}%
\begin{pgfscope}%
\pgfsys@transformshift{1.808600in}{2.791491in}%
\pgfsys@useobject{currentmarker}{}%
\end{pgfscope}%
\begin{pgfscope}%
\pgfsys@transformshift{1.826208in}{2.541905in}%
\pgfsys@useobject{currentmarker}{}%
\end{pgfscope}%
\begin{pgfscope}%
\pgfsys@transformshift{1.847337in}{2.195727in}%
\pgfsys@useobject{currentmarker}{}%
\end{pgfscope}%
\begin{pgfscope}%
\pgfsys@transformshift{1.865414in}{2.044709in}%
\pgfsys@useobject{currentmarker}{}%
\end{pgfscope}%
\begin{pgfscope}%
\pgfsys@transformshift{1.886074in}{1.992803in}%
\pgfsys@useobject{currentmarker}{}%
\end{pgfscope}%
\begin{pgfscope}%
\pgfsys@transformshift{1.905090in}{1.971195in}%
\pgfsys@useobject{currentmarker}{}%
\end{pgfscope}%
\begin{pgfscope}%
\pgfsys@transformshift{1.925047in}{1.989244in}%
\pgfsys@useobject{currentmarker}{}%
\end{pgfscope}%
\begin{pgfscope}%
\pgfsys@transformshift{1.944064in}{2.048315in}%
\pgfsys@useobject{currentmarker}{}%
\end{pgfscope}%
\begin{pgfscope}%
\pgfsys@transformshift{1.965192in}{2.199532in}%
\pgfsys@useobject{currentmarker}{}%
\end{pgfscope}%
\begin{pgfscope}%
\pgfsys@transformshift{1.979043in}{2.435794in}%
\pgfsys@useobject{currentmarker}{}%
\end{pgfscope}%
\begin{pgfscope}%
\pgfsys@transformshift{2.000643in}{2.663938in}%
\pgfsys@useobject{currentmarker}{}%
\end{pgfscope}%
\begin{pgfscope}%
\pgfsys@transformshift{2.021772in}{2.822087in}%
\pgfsys@useobject{currentmarker}{}%
\end{pgfscope}%
\begin{pgfscope}%
\pgfsys@transformshift{2.038911in}{2.778483in}%
\pgfsys@useobject{currentmarker}{}%
\end{pgfscope}%
\begin{pgfscope}%
\pgfsys@transformshift{2.056753in}{2.487223in}%
\pgfsys@useobject{currentmarker}{}%
\end{pgfscope}%
\begin{pgfscope}%
\pgfsys@transformshift{2.078822in}{2.183959in}%
\pgfsys@useobject{currentmarker}{}%
\end{pgfscope}%
\begin{pgfscope}%
\pgfsys@transformshift{2.099717in}{2.031919in}%
\pgfsys@useobject{currentmarker}{}%
\end{pgfscope}%
\begin{pgfscope}%
\pgfsys@transformshift{2.118030in}{1.982906in}%
\pgfsys@useobject{currentmarker}{}%
\end{pgfscope}%
\begin{pgfscope}%
\pgfsys@transformshift{2.135638in}{1.968467in}%
\pgfsys@useobject{currentmarker}{}%
\end{pgfscope}%
\begin{pgfscope}%
\pgfsys@transformshift{2.153715in}{1.999935in}%
\pgfsys@useobject{currentmarker}{}%
\end{pgfscope}%
\begin{pgfscope}%
\pgfsys@transformshift{2.175078in}{2.076602in}%
\pgfsys@useobject{currentmarker}{}%
\end{pgfscope}%
\begin{pgfscope}%
\pgfsys@transformshift{2.192452in}{2.226046in}%
\pgfsys@useobject{currentmarker}{}%
\end{pgfscope}%
\begin{pgfscope}%
\pgfsys@transformshift{2.213817in}{2.553589in}%
\pgfsys@useobject{currentmarker}{}%
\end{pgfscope}%
\begin{pgfscope}%
\pgfsys@transformshift{2.231189in}{2.627400in}%
\pgfsys@useobject{currentmarker}{}%
\end{pgfscope}%
\begin{pgfscope}%
\pgfsys@transformshift{2.255840in}{2.105618in}%
\pgfsys@useobject{currentmarker}{}%
\end{pgfscope}%
\begin{pgfscope}%
\pgfsys@transformshift{2.269693in}{2.265746in}%
\pgfsys@useobject{currentmarker}{}%
\end{pgfscope}%
\begin{pgfscope}%
\pgfsys@transformshift{2.288944in}{2.458728in}%
\pgfsys@useobject{currentmarker}{}%
\end{pgfscope}%
\begin{pgfscope}%
\pgfsys@transformshift{2.309133in}{2.789171in}%
\pgfsys@useobject{currentmarker}{}%
\end{pgfscope}%
\begin{pgfscope}%
\pgfsys@transformshift{2.326038in}{2.796245in}%
\pgfsys@useobject{currentmarker}{}%
\end{pgfscope}%
\begin{pgfscope}%
\pgfsys@transformshift{2.345992in}{2.496547in}%
\pgfsys@useobject{currentmarker}{}%
\end{pgfscope}%
\begin{pgfscope}%
\pgfsys@transformshift{2.367592in}{2.163946in}%
\pgfsys@useobject{currentmarker}{}%
\end{pgfscope}%
\begin{pgfscope}%
\pgfsys@transformshift{2.384731in}{2.032186in}%
\pgfsys@useobject{currentmarker}{}%
\end{pgfscope}%
\begin{pgfscope}%
\pgfsys@transformshift{2.405626in}{1.978218in}%
\pgfsys@useobject{currentmarker}{}%
\end{pgfscope}%
\begin{pgfscope}%
\pgfsys@transformshift{2.423937in}{1.970605in}%
\pgfsys@useobject{currentmarker}{}%
\end{pgfscope}%
\begin{pgfscope}%
\pgfsys@transformshift{2.444597in}{2.011456in}%
\pgfsys@useobject{currentmarker}{}%
\end{pgfscope}%
\begin{pgfscope}%
\pgfsys@transformshift{2.462205in}{2.091938in}%
\pgfsys@useobject{currentmarker}{}%
\end{pgfscope}%
\begin{pgfscope}%
\pgfsys@transformshift{2.480753in}{2.273571in}%
\pgfsys@useobject{currentmarker}{}%
\end{pgfscope}%
\begin{pgfscope}%
\pgfsys@transformshift{2.502822in}{2.567014in}%
\pgfsys@useobject{currentmarker}{}%
\end{pgfscope}%
\begin{pgfscope}%
\pgfsys@transformshift{2.518081in}{2.773615in}%
\pgfsys@useobject{currentmarker}{}%
\end{pgfscope}%
\begin{pgfscope}%
\pgfsys@transformshift{2.538272in}{2.790303in}%
\pgfsys@useobject{currentmarker}{}%
\end{pgfscope}%
\begin{pgfscope}%
\pgfsys@transformshift{2.559166in}{2.559836in}%
\pgfsys@useobject{currentmarker}{}%
\end{pgfscope}%
\begin{pgfscope}%
\pgfsys@transformshift{2.577243in}{2.257298in}%
\pgfsys@useobject{currentmarker}{}%
\end{pgfscope}%
\begin{pgfscope}%
\pgfsys@transformshift{2.595322in}{2.084901in}%
\pgfsys@useobject{currentmarker}{}%
\end{pgfscope}%
\begin{pgfscope}%
\pgfsys@transformshift{2.615982in}{1.999586in}%
\pgfsys@useobject{currentmarker}{}%
\end{pgfscope}%
\begin{pgfscope}%
\pgfsys@transformshift{2.633825in}{1.969879in}%
\pgfsys@useobject{currentmarker}{}%
\end{pgfscope}%
\begin{pgfscope}%
\pgfsys@transformshift{2.655659in}{1.984239in}%
\pgfsys@useobject{currentmarker}{}%
\end{pgfscope}%
\begin{pgfscope}%
\pgfsys@transformshift{2.673031in}{2.035474in}%
\pgfsys@useobject{currentmarker}{}%
\end{pgfscope}%
\begin{pgfscope}%
\pgfsys@transformshift{2.691109in}{2.111139in}%
\pgfsys@useobject{currentmarker}{}%
\end{pgfscope}%
\begin{pgfscope}%
\pgfsys@transformshift{2.712707in}{2.345418in}%
\pgfsys@useobject{currentmarker}{}%
\end{pgfscope}%
\begin{pgfscope}%
\pgfsys@transformshift{2.733838in}{2.664015in}%
\pgfsys@useobject{currentmarker}{}%
\end{pgfscope}%
\begin{pgfscope}%
\pgfsys@transformshift{2.751680in}{2.805955in}%
\pgfsys@useobject{currentmarker}{}%
\end{pgfscope}%
\begin{pgfscope}%
\pgfsys@transformshift{2.769992in}{2.764575in}%
\pgfsys@useobject{currentmarker}{}%
\end{pgfscope}%
\begin{pgfscope}%
\pgfsys@transformshift{2.792061in}{2.442040in}%
\pgfsys@useobject{currentmarker}{}%
\end{pgfscope}%
\begin{pgfscope}%
\pgfsys@transformshift{2.812486in}{2.151977in}%
\pgfsys@useobject{currentmarker}{}%
\end{pgfscope}%
\begin{pgfscope}%
\pgfsys@transformshift{2.827277in}{2.055972in}%
\pgfsys@useobject{currentmarker}{}%
\end{pgfscope}%
\begin{pgfscope}%
\pgfsys@transformshift{2.848640in}{1.986101in}%
\pgfsys@useobject{currentmarker}{}%
\end{pgfscope}%
\begin{pgfscope}%
\pgfsys@transformshift{2.866484in}{1.973313in}%
\pgfsys@useobject{currentmarker}{}%
\end{pgfscope}%
\begin{pgfscope}%
\pgfsys@transformshift{2.884092in}{1.975190in}%
\pgfsys@useobject{currentmarker}{}%
\end{pgfscope}%
\begin{pgfscope}%
\pgfsys@transformshift{2.906161in}{2.029543in}%
\pgfsys@useobject{currentmarker}{}%
\end{pgfscope}%
\begin{pgfscope}%
\pgfsys@transformshift{2.922829in}{2.124600in}%
\pgfsys@useobject{currentmarker}{}%
\end{pgfscope}%
\begin{pgfscope}%
\pgfsys@transformshift{2.946072in}{2.274468in}%
\pgfsys@useobject{currentmarker}{}%
\end{pgfscope}%
\begin{pgfscope}%
\pgfsys@transformshift{2.959454in}{2.482235in}%
\pgfsys@useobject{currentmarker}{}%
\end{pgfscope}%
\begin{pgfscope}%
\pgfsys@transformshift{2.980817in}{2.743088in}%
\pgfsys@useobject{currentmarker}{}%
\end{pgfscope}%
\begin{pgfscope}%
\pgfsys@transformshift{3.002183in}{2.812765in}%
\pgfsys@useobject{currentmarker}{}%
\end{pgfscope}%
\begin{pgfscope}%
\pgfsys@transformshift{3.019085in}{2.668823in}%
\pgfsys@useobject{currentmarker}{}%
\end{pgfscope}%
\begin{pgfscope}%
\pgfsys@transformshift{3.037399in}{2.348715in}%
\pgfsys@useobject{currentmarker}{}%
\end{pgfscope}%
\begin{pgfscope}%
\pgfsys@transformshift{3.056179in}{2.132630in}%
\pgfsys@useobject{currentmarker}{}%
\end{pgfscope}%
\begin{pgfscope}%
\pgfsys@transformshift{3.075901in}{2.029880in}%
\pgfsys@useobject{currentmarker}{}%
\end{pgfscope}%
\begin{pgfscope}%
\pgfsys@transformshift{3.098204in}{1.984399in}%
\pgfsys@useobject{currentmarker}{}%
\end{pgfscope}%
\begin{pgfscope}%
\pgfsys@transformshift{3.115107in}{1.970638in}%
\pgfsys@useobject{currentmarker}{}%
\end{pgfscope}%
\begin{pgfscope}%
\pgfsys@transformshift{3.132949in}{1.990179in}%
\pgfsys@useobject{currentmarker}{}%
\end{pgfscope}%
\begin{pgfscope}%
\pgfsys@transformshift{3.154549in}{2.044372in}%
\pgfsys@useobject{currentmarker}{}%
\end{pgfscope}%
\begin{pgfscope}%
\pgfsys@transformshift{3.174740in}{2.150507in}%
\pgfsys@useobject{currentmarker}{}%
\end{pgfscope}%
\begin{pgfscope}%
\pgfsys@transformshift{3.193757in}{2.346471in}%
\pgfsys@useobject{currentmarker}{}%
\end{pgfscope}%
\begin{pgfscope}%
\pgfsys@transformshift{3.212068in}{2.651113in}%
\pgfsys@useobject{currentmarker}{}%
\end{pgfscope}%
\begin{pgfscope}%
\pgfsys@transformshift{3.232259in}{2.826136in}%
\pgfsys@useobject{currentmarker}{}%
\end{pgfscope}%
\begin{pgfscope}%
\pgfsys@transformshift{3.250571in}{2.816527in}%
\pgfsys@useobject{currentmarker}{}%
\end{pgfscope}%
\begin{pgfscope}%
\pgfsys@transformshift{3.272170in}{2.616766in}%
\pgfsys@useobject{currentmarker}{}%
\end{pgfscope}%
\begin{pgfscope}%
\pgfsys@transformshift{3.286961in}{2.383401in}%
\pgfsys@useobject{currentmarker}{}%
\end{pgfscope}%
\begin{pgfscope}%
\pgfsys@transformshift{3.307621in}{2.171369in}%
\pgfsys@useobject{currentmarker}{}%
\end{pgfscope}%
\begin{pgfscope}%
\pgfsys@transformshift{3.326872in}{2.055079in}%
\pgfsys@useobject{currentmarker}{}%
\end{pgfscope}%
\begin{pgfscope}%
\pgfsys@transformshift{3.345184in}{1.999905in}%
\pgfsys@useobject{currentmarker}{}%
\end{pgfscope}%
\begin{pgfscope}%
\pgfsys@transformshift{3.367018in}{1.976201in}%
\pgfsys@useobject{currentmarker}{}%
\end{pgfscope}%
\begin{pgfscope}%
\pgfsys@transformshift{3.384157in}{1.979637in}%
\pgfsys@useobject{currentmarker}{}%
\end{pgfscope}%
\begin{pgfscope}%
\pgfsys@transformshift{3.401530in}{2.016560in}%
\pgfsys@useobject{currentmarker}{}%
\end{pgfscope}%
\begin{pgfscope}%
\pgfsys@transformshift{3.423128in}{1.999241in}%
\pgfsys@useobject{currentmarker}{}%
\end{pgfscope}%
\begin{pgfscope}%
\pgfsys@transformshift{3.443319in}{2.071549in}%
\pgfsys@useobject{currentmarker}{}%
\end{pgfscope}%
\begin{pgfscope}%
\pgfsys@transformshift{3.461867in}{2.182269in}%
\pgfsys@useobject{currentmarker}{}%
\end{pgfscope}%
\begin{pgfscope}%
\pgfsys@transformshift{3.481587in}{2.353900in}%
\pgfsys@useobject{currentmarker}{}%
\end{pgfscope}%
\begin{pgfscope}%
\pgfsys@transformshift{3.499195in}{2.622887in}%
\pgfsys@useobject{currentmarker}{}%
\end{pgfscope}%
\begin{pgfscope}%
\pgfsys@transformshift{3.520090in}{2.843844in}%
\pgfsys@useobject{currentmarker}{}%
\end{pgfscope}%
\begin{pgfscope}%
\pgfsys@transformshift{3.537463in}{2.817704in}%
\pgfsys@useobject{currentmarker}{}%
\end{pgfscope}%
\begin{pgfscope}%
\pgfsys@transformshift{3.556245in}{2.603371in}%
\pgfsys@useobject{currentmarker}{}%
\end{pgfscope}%
\begin{pgfscope}%
\pgfsys@transformshift{3.577374in}{2.295388in}%
\pgfsys@useobject{currentmarker}{}%
\end{pgfscope}%
\begin{pgfscope}%
\pgfsys@transformshift{3.599443in}{2.112955in}%
\pgfsys@useobject{currentmarker}{}%
\end{pgfscope}%
\begin{pgfscope}%
\pgfsys@transformshift{3.616816in}{2.038232in}%
\pgfsys@useobject{currentmarker}{}%
\end{pgfscope}%
\begin{pgfscope}%
\pgfsys@transformshift{3.634893in}{1.989304in}%
\pgfsys@useobject{currentmarker}{}%
\end{pgfscope}%
\begin{pgfscope}%
\pgfsys@transformshift{3.656259in}{1.999525in}%
\pgfsys@useobject{currentmarker}{}%
\end{pgfscope}%
\begin{pgfscope}%
\pgfsys@transformshift{3.672222in}{1.975558in}%
\pgfsys@useobject{currentmarker}{}%
\end{pgfscope}%
\begin{pgfscope}%
\pgfsys@transformshift{3.691004in}{1.988950in}%
\pgfsys@useobject{currentmarker}{}%
\end{pgfscope}%
\begin{pgfscope}%
\pgfsys@transformshift{3.713073in}{2.017861in}%
\pgfsys@useobject{currentmarker}{}%
\end{pgfscope}%
\begin{pgfscope}%
\pgfsys@transformshift{3.732558in}{2.110612in}%
\pgfsys@useobject{currentmarker}{}%
\end{pgfscope}%
\begin{pgfscope}%
\pgfsys@transformshift{3.748523in}{2.218386in}%
\pgfsys@useobject{currentmarker}{}%
\end{pgfscope}%
\begin{pgfscope}%
\pgfsys@transformshift{3.766600in}{2.453123in}%
\pgfsys@useobject{currentmarker}{}%
\end{pgfscope}%
\begin{pgfscope}%
\pgfsys@transformshift{3.788905in}{2.801299in}%
\pgfsys@useobject{currentmarker}{}%
\end{pgfscope}%
\begin{pgfscope}%
\pgfsys@transformshift{3.808156in}{2.883043in}%
\pgfsys@useobject{currentmarker}{}%
\end{pgfscope}%
\begin{pgfscope}%
\pgfsys@transformshift{3.827173in}{2.824310in}%
\pgfsys@useobject{currentmarker}{}%
\end{pgfscope}%
\begin{pgfscope}%
\pgfsys@transformshift{3.846424in}{2.570072in}%
\pgfsys@useobject{currentmarker}{}%
\end{pgfscope}%
\begin{pgfscope}%
\pgfsys@transformshift{3.865441in}{2.288137in}%
\pgfsys@useobject{currentmarker}{}%
\end{pgfscope}%
\begin{pgfscope}%
\pgfsys@transformshift{3.884221in}{2.129652in}%
\pgfsys@useobject{currentmarker}{}%
\end{pgfscope}%
\begin{pgfscope}%
\pgfsys@transformshift{3.903709in}{2.035524in}%
\pgfsys@useobject{currentmarker}{}%
\end{pgfscope}%
\begin{pgfscope}%
\pgfsys@transformshift{3.922255in}{1.992316in}%
\pgfsys@useobject{currentmarker}{}%
\end{pgfscope}%
\begin{pgfscope}%
\pgfsys@transformshift{3.943384in}{1.978726in}%
\pgfsys@useobject{currentmarker}{}%
\end{pgfscope}%
\begin{pgfscope}%
\pgfsys@transformshift{3.962871in}{2.002456in}%
\pgfsys@useobject{currentmarker}{}%
\end{pgfscope}%
\begin{pgfscope}%
\pgfsys@transformshift{3.980948in}{2.056333in}%
\pgfsys@useobject{currentmarker}{}%
\end{pgfscope}%
\begin{pgfscope}%
\pgfsys@transformshift{3.999260in}{2.147157in}%
\pgfsys@useobject{currentmarker}{}%
\end{pgfscope}%
\begin{pgfscope}%
\pgfsys@transformshift{4.018511in}{2.325883in}%
\pgfsys@useobject{currentmarker}{}%
\end{pgfscope}%
\begin{pgfscope}%
\pgfsys@transformshift{4.037998in}{2.620692in}%
\pgfsys@useobject{currentmarker}{}%
\end{pgfscope}%
\begin{pgfscope}%
\pgfsys@transformshift{4.056544in}{2.824823in}%
\pgfsys@useobject{currentmarker}{}%
\end{pgfscope}%
\begin{pgfscope}%
\pgfsys@transformshift{4.077439in}{2.905900in}%
\pgfsys@useobject{currentmarker}{}%
\end{pgfscope}%
\begin{pgfscope}%
\pgfsys@transformshift{4.096456in}{2.798355in}%
\pgfsys@useobject{currentmarker}{}%
\end{pgfscope}%
\begin{pgfscope}%
\pgfsys@transformshift{4.117821in}{2.912224in}%
\pgfsys@useobject{currentmarker}{}%
\end{pgfscope}%
\begin{pgfscope}%
\pgfsys@transformshift{4.133786in}{2.860124in}%
\pgfsys@useobject{currentmarker}{}%
\end{pgfscope}%
\begin{pgfscope}%
\pgfsys@transformshift{4.151628in}{2.630096in}%
\pgfsys@useobject{currentmarker}{}%
\end{pgfscope}%
\begin{pgfscope}%
\pgfsys@transformshift{4.174635in}{2.352382in}%
\pgfsys@useobject{currentmarker}{}%
\end{pgfscope}%
\begin{pgfscope}%
\pgfsys@transformshift{4.189425in}{2.201753in}%
\pgfsys@useobject{currentmarker}{}%
\end{pgfscope}%
\begin{pgfscope}%
\pgfsys@transformshift{4.209616in}{2.080525in}%
\pgfsys@useobject{currentmarker}{}%
\end{pgfscope}%
\begin{pgfscope}%
\pgfsys@transformshift{4.231216in}{2.001494in}%
\pgfsys@useobject{currentmarker}{}%
\end{pgfscope}%
\begin{pgfscope}%
\pgfsys@transformshift{4.249998in}{1.983722in}%
\pgfsys@useobject{currentmarker}{}%
\end{pgfscope}%
\begin{pgfscope}%
\pgfsys@transformshift{4.269484in}{2.006986in}%
\pgfsys@useobject{currentmarker}{}%
\end{pgfscope}%
\begin{pgfscope}%
\pgfsys@transformshift{4.288735in}{2.066625in}%
\pgfsys@useobject{currentmarker}{}%
\end{pgfscope}%
\begin{pgfscope}%
\pgfsys@transformshift{4.307281in}{2.178792in}%
\pgfsys@useobject{currentmarker}{}%
\end{pgfscope}%
\begin{pgfscope}%
\pgfsys@transformshift{4.325829in}{2.376832in}%
\pgfsys@useobject{currentmarker}{}%
\end{pgfscope}%
\begin{pgfscope}%
\pgfsys@transformshift{4.346020in}{2.601957in}%
\pgfsys@useobject{currentmarker}{}%
\end{pgfscope}%
\begin{pgfscope}%
\pgfsys@transformshift{4.346489in}{2.712394in}%
\pgfsys@useobject{currentmarker}{}%
\end{pgfscope}%
\begin{pgfscope}%
\pgfsys@transformshift{4.363862in}{2.793530in}%
\pgfsys@useobject{currentmarker}{}%
\end{pgfscope}%
\begin{pgfscope}%
\pgfsys@transformshift{4.385226in}{2.940548in}%
\pgfsys@useobject{currentmarker}{}%
\end{pgfscope}%
\begin{pgfscope}%
\pgfsys@transformshift{4.404242in}{2.858574in}%
\pgfsys@useobject{currentmarker}{}%
\end{pgfscope}%
\begin{pgfscope}%
\pgfsys@transformshift{4.423259in}{2.649402in}%
\pgfsys@useobject{currentmarker}{}%
\end{pgfscope}%
\begin{pgfscope}%
\pgfsys@transformshift{4.441807in}{2.388999in}%
\pgfsys@useobject{currentmarker}{}%
\end{pgfscope}%
\begin{pgfscope}%
\pgfsys@transformshift{4.460824in}{2.190943in}%
\pgfsys@useobject{currentmarker}{}%
\end{pgfscope}%
\begin{pgfscope}%
\pgfsys@transformshift{4.479840in}{2.082780in}%
\pgfsys@useobject{currentmarker}{}%
\end{pgfscope}%
\begin{pgfscope}%
\pgfsys@transformshift{4.480309in}{2.081800in}%
\pgfsys@useobject{currentmarker}{}%
\end{pgfscope}%
\begin{pgfscope}%
\pgfsys@transformshift{4.473735in}{2.122851in}%
\pgfsys@useobject{currentmarker}{}%
\end{pgfscope}%
\begin{pgfscope}%
\pgfsys@transformshift{4.456832in}{2.339296in}%
\pgfsys@useobject{currentmarker}{}%
\end{pgfscope}%
\begin{pgfscope}%
\pgfsys@transformshift{4.435703in}{2.766239in}%
\pgfsys@useobject{currentmarker}{}%
\end{pgfscope}%
\begin{pgfscope}%
\pgfsys@transformshift{4.417859in}{2.027261in}%
\pgfsys@useobject{currentmarker}{}%
\end{pgfscope}%
\begin{pgfscope}%
\pgfsys@transformshift{4.396964in}{1.981908in}%
\pgfsys@useobject{currentmarker}{}%
\end{pgfscope}%
\begin{pgfscope}%
\pgfsys@transformshift{4.373252in}{2.060594in}%
\pgfsys@useobject{currentmarker}{}%
\end{pgfscope}%
\begin{pgfscope}%
\pgfsys@transformshift{4.360810in}{2.138453in}%
\pgfsys@useobject{currentmarker}{}%
\end{pgfscope}%
\begin{pgfscope}%
\pgfsys@transformshift{4.340385in}{2.377650in}%
\pgfsys@useobject{currentmarker}{}%
\end{pgfscope}%
\begin{pgfscope}%
\pgfsys@transformshift{4.322072in}{2.754466in}%
\pgfsys@useobject{currentmarker}{}%
\end{pgfscope}%
\begin{pgfscope}%
\pgfsys@transformshift{4.301882in}{2.914988in}%
\pgfsys@useobject{currentmarker}{}%
\end{pgfscope}%
\begin{pgfscope}%
\pgfsys@transformshift{4.284509in}{2.657105in}%
\pgfsys@useobject{currentmarker}{}%
\end{pgfscope}%
\begin{pgfscope}%
\pgfsys@transformshift{4.260328in}{2.228785in}%
\pgfsys@useobject{currentmarker}{}%
\end{pgfscope}%
\begin{pgfscope}%
\pgfsys@transformshift{4.245772in}{2.084132in}%
\pgfsys@useobject{currentmarker}{}%
\end{pgfscope}%
\begin{pgfscope}%
\pgfsys@transformshift{4.224876in}{1.993494in}%
\pgfsys@useobject{currentmarker}{}%
\end{pgfscope}%
\begin{pgfscope}%
\pgfsys@transformshift{4.203278in}{1.993677in}%
\pgfsys@useobject{currentmarker}{}%
\end{pgfscope}%
\begin{pgfscope}%
\pgfsys@transformshift{4.184965in}{2.065719in}%
\pgfsys@useobject{currentmarker}{}%
\end{pgfscope}%
\begin{pgfscope}%
\pgfsys@transformshift{4.167122in}{2.238555in}%
\pgfsys@useobject{currentmarker}{}%
\end{pgfscope}%
\begin{pgfscope}%
\pgfsys@transformshift{4.149514in}{2.582901in}%
\pgfsys@useobject{currentmarker}{}%
\end{pgfscope}%
\begin{pgfscope}%
\pgfsys@transformshift{4.128151in}{2.882358in}%
\pgfsys@useobject{currentmarker}{}%
\end{pgfscope}%
\begin{pgfscope}%
\pgfsys@transformshift{4.111012in}{2.816413in}%
\pgfsys@useobject{currentmarker}{}%
\end{pgfscope}%
\begin{pgfscope}%
\pgfsys@transformshift{4.088709in}{2.351924in}%
\pgfsys@useobject{currentmarker}{}%
\end{pgfscope}%
\begin{pgfscope}%
\pgfsys@transformshift{4.070866in}{2.122999in}%
\pgfsys@useobject{currentmarker}{}%
\end{pgfscope}%
\begin{pgfscope}%
\pgfsys@transformshift{4.052318in}{2.017414in}%
\pgfsys@useobject{currentmarker}{}%
\end{pgfscope}%
\begin{pgfscope}%
\pgfsys@transformshift{4.031893in}{1.975956in}%
\pgfsys@useobject{currentmarker}{}%
\end{pgfscope}%
\begin{pgfscope}%
\pgfsys@transformshift{4.013112in}{2.014936in}%
\pgfsys@useobject{currentmarker}{}%
\end{pgfscope}%
\begin{pgfscope}%
\pgfsys@transformshift{3.994330in}{2.126059in}%
\pgfsys@useobject{currentmarker}{}%
\end{pgfscope}%
\begin{pgfscope}%
\pgfsys@transformshift{3.972965in}{2.437716in}%
\pgfsys@useobject{currentmarker}{}%
\end{pgfscope}%
\begin{pgfscope}%
\pgfsys@transformshift{3.957940in}{2.738769in}%
\pgfsys@useobject{currentmarker}{}%
\end{pgfscope}%
\begin{pgfscope}%
\pgfsys@transformshift{3.935168in}{2.858378in}%
\pgfsys@useobject{currentmarker}{}%
\end{pgfscope}%
\begin{pgfscope}%
\pgfsys@transformshift{3.917091in}{2.641408in}%
\pgfsys@useobject{currentmarker}{}%
\end{pgfscope}%
\begin{pgfscope}%
\pgfsys@transformshift{3.897840in}{2.265862in}%
\pgfsys@useobject{currentmarker}{}%
\end{pgfscope}%
\begin{pgfscope}%
\pgfsys@transformshift{3.879526in}{2.084797in}%
\pgfsys@useobject{currentmarker}{}%
\end{pgfscope}%
\begin{pgfscope}%
\pgfsys@transformshift{3.857458in}{2.037972in}%
\pgfsys@useobject{currentmarker}{}%
\end{pgfscope}%
\begin{pgfscope}%
\pgfsys@transformshift{3.839615in}{1.978257in}%
\pgfsys@useobject{currentmarker}{}%
\end{pgfscope}%
\begin{pgfscope}%
\pgfsys@transformshift{3.820364in}{1.984478in}%
\pgfsys@useobject{currentmarker}{}%
\end{pgfscope}%
\begin{pgfscope}%
\pgfsys@transformshift{3.801113in}{2.048850in}%
\pgfsys@useobject{currentmarker}{}%
\end{pgfscope}%
\begin{pgfscope}%
\pgfsys@transformshift{3.783270in}{2.197327in}%
\pgfsys@useobject{currentmarker}{}%
\end{pgfscope}%
\begin{pgfscope}%
\pgfsys@transformshift{3.761436in}{2.480708in}%
\pgfsys@useobject{currentmarker}{}%
\end{pgfscope}%
\begin{pgfscope}%
\pgfsys@transformshift{3.743594in}{2.794125in}%
\pgfsys@useobject{currentmarker}{}%
\end{pgfscope}%
\begin{pgfscope}%
\pgfsys@transformshift{3.724577in}{2.839493in}%
\pgfsys@useobject{currentmarker}{}%
\end{pgfscope}%
\begin{pgfscope}%
\pgfsys@transformshift{3.708143in}{2.566684in}%
\pgfsys@useobject{currentmarker}{}%
\end{pgfscope}%
\begin{pgfscope}%
\pgfsys@transformshift{3.684666in}{2.222633in}%
\pgfsys@useobject{currentmarker}{}%
\end{pgfscope}%
\begin{pgfscope}%
\pgfsys@transformshift{3.667997in}{2.083328in}%
\pgfsys@useobject{currentmarker}{}%
\end{pgfscope}%
\begin{pgfscope}%
\pgfsys@transformshift{3.648744in}{2.018388in}%
\pgfsys@useobject{currentmarker}{}%
\end{pgfscope}%
\begin{pgfscope}%
\pgfsys@transformshift{3.629024in}{1.974577in}%
\pgfsys@useobject{currentmarker}{}%
\end{pgfscope}%
\begin{pgfscope}%
\pgfsys@transformshift{3.606955in}{1.989006in}%
\pgfsys@useobject{currentmarker}{}%
\end{pgfscope}%
\begin{pgfscope}%
\pgfsys@transformshift{3.585356in}{2.068672in}%
\pgfsys@useobject{currentmarker}{}%
\end{pgfscope}%
\begin{pgfscope}%
\pgfsys@transformshift{3.570096in}{2.153623in}%
\pgfsys@useobject{currentmarker}{}%
\end{pgfscope}%
\begin{pgfscope}%
\pgfsys@transformshift{3.549436in}{2.446952in}%
\pgfsys@useobject{currentmarker}{}%
\end{pgfscope}%
\begin{pgfscope}%
\pgfsys@transformshift{3.533003in}{2.700948in}%
\pgfsys@useobject{currentmarker}{}%
\end{pgfscope}%
\begin{pgfscope}%
\pgfsys@transformshift{3.514689in}{2.838092in}%
\pgfsys@useobject{currentmarker}{}%
\end{pgfscope}%
\begin{pgfscope}%
\pgfsys@transformshift{3.494266in}{2.816018in}%
\pgfsys@useobject{currentmarker}{}%
\end{pgfscope}%
\begin{pgfscope}%
\pgfsys@transformshift{3.474075in}{2.577899in}%
\pgfsys@useobject{currentmarker}{}%
\end{pgfscope}%
\begin{pgfscope}%
\pgfsys@transformshift{3.457170in}{2.298420in}%
\pgfsys@useobject{currentmarker}{}%
\end{pgfscope}%
\begin{pgfscope}%
\pgfsys@transformshift{3.438155in}{2.106850in}%
\pgfsys@useobject{currentmarker}{}%
\end{pgfscope}%
\begin{pgfscope}%
\pgfsys@transformshift{3.412798in}{2.001193in}%
\pgfsys@useobject{currentmarker}{}%
\end{pgfscope}%
\begin{pgfscope}%
\pgfsys@transformshift{3.396364in}{1.985281in}%
\pgfsys@useobject{currentmarker}{}%
\end{pgfscope}%
\begin{pgfscope}%
\pgfsys@transformshift{3.377819in}{1.971208in}%
\pgfsys@useobject{currentmarker}{}%
\end{pgfscope}%
\begin{pgfscope}%
\pgfsys@transformshift{3.358802in}{2.002035in}%
\pgfsys@useobject{currentmarker}{}%
\end{pgfscope}%
\begin{pgfscope}%
\pgfsys@transformshift{3.339551in}{2.078088in}%
\pgfsys@useobject{currentmarker}{}%
\end{pgfscope}%
\begin{pgfscope}%
\pgfsys@transformshift{3.320768in}{2.256512in}%
\pgfsys@useobject{currentmarker}{}%
\end{pgfscope}%
\begin{pgfscope}%
\pgfsys@transformshift{3.301986in}{2.541807in}%
\pgfsys@useobject{currentmarker}{}%
\end{pgfscope}%
\begin{pgfscope}%
\pgfsys@transformshift{3.280623in}{2.780268in}%
\pgfsys@useobject{currentmarker}{}%
\end{pgfscope}%
\begin{pgfscope}%
\pgfsys@transformshift{3.263953in}{2.816115in}%
\pgfsys@useobject{currentmarker}{}%
\end{pgfscope}%
\begin{pgfscope}%
\pgfsys@transformshift{3.243293in}{2.676156in}%
\pgfsys@useobject{currentmarker}{}%
\end{pgfscope}%
\begin{pgfscope}%
\pgfsys@transformshift{3.223338in}{2.317275in}%
\pgfsys@useobject{currentmarker}{}%
\end{pgfscope}%
\begin{pgfscope}%
\pgfsys@transformshift{3.205025in}{2.125276in}%
\pgfsys@useobject{currentmarker}{}%
\end{pgfscope}%
\begin{pgfscope}%
\pgfsys@transformshift{3.184365in}{2.020489in}%
\pgfsys@useobject{currentmarker}{}%
\end{pgfscope}%
\begin{pgfscope}%
\pgfsys@transformshift{3.165584in}{1.975746in}%
\pgfsys@useobject{currentmarker}{}%
\end{pgfscope}%
\begin{pgfscope}%
\pgfsys@transformshift{3.147271in}{2.442819in}%
\pgfsys@useobject{currentmarker}{}%
\end{pgfscope}%
\begin{pgfscope}%
\pgfsys@transformshift{3.129429in}{2.773885in}%
\pgfsys@useobject{currentmarker}{}%
\end{pgfscope}%
\begin{pgfscope}%
\pgfsys@transformshift{3.109708in}{2.799490in}%
\pgfsys@useobject{currentmarker}{}%
\end{pgfscope}%
\begin{pgfscope}%
\pgfsys@transformshift{3.091161in}{2.523777in}%
\pgfsys@useobject{currentmarker}{}%
\end{pgfscope}%
\begin{pgfscope}%
\pgfsys@transformshift{3.066509in}{2.168285in}%
\pgfsys@useobject{currentmarker}{}%
\end{pgfscope}%
\begin{pgfscope}%
\pgfsys@transformshift{3.054301in}{2.070086in}%
\pgfsys@useobject{currentmarker}{}%
\end{pgfscope}%
\begin{pgfscope}%
\pgfsys@transformshift{3.035285in}{1.993179in}%
\pgfsys@useobject{currentmarker}{}%
\end{pgfscope}%
\begin{pgfscope}%
\pgfsys@transformshift{3.013450in}{1.969276in}%
\pgfsys@useobject{currentmarker}{}%
\end{pgfscope}%
\begin{pgfscope}%
\pgfsys@transformshift{2.995139in}{1.994975in}%
\pgfsys@useobject{currentmarker}{}%
\end{pgfscope}%
\begin{pgfscope}%
\pgfsys@transformshift{2.975419in}{2.086436in}%
\pgfsys@useobject{currentmarker}{}%
\end{pgfscope}%
\begin{pgfscope}%
\pgfsys@transformshift{2.955697in}{2.278229in}%
\pgfsys@useobject{currentmarker}{}%
\end{pgfscope}%
\begin{pgfscope}%
\pgfsys@transformshift{2.936211in}{2.624360in}%
\pgfsys@useobject{currentmarker}{}%
\end{pgfscope}%
\begin{pgfscope}%
\pgfsys@transformshift{2.918369in}{2.819553in}%
\pgfsys@useobject{currentmarker}{}%
\end{pgfscope}%
\begin{pgfscope}%
\pgfsys@transformshift{2.900055in}{2.733268in}%
\pgfsys@useobject{currentmarker}{}%
\end{pgfscope}%
\begin{pgfscope}%
\pgfsys@transformshift{2.879632in}{2.546170in}%
\pgfsys@useobject{currentmarker}{}%
\end{pgfscope}%
\begin{pgfscope}%
\pgfsys@transformshift{2.858501in}{2.215964in}%
\pgfsys@useobject{currentmarker}{}%
\end{pgfscope}%
\begin{pgfscope}%
\pgfsys@transformshift{2.840190in}{2.062911in}%
\pgfsys@useobject{currentmarker}{}%
\end{pgfscope}%
\begin{pgfscope}%
\pgfsys@transformshift{2.821407in}{1.993896in}%
\pgfsys@useobject{currentmarker}{}%
\end{pgfscope}%
\begin{pgfscope}%
\pgfsys@transformshift{2.803096in}{1.969280in}%
\pgfsys@useobject{currentmarker}{}%
\end{pgfscope}%
\begin{pgfscope}%
\pgfsys@transformshift{2.783374in}{1.990668in}%
\pgfsys@useobject{currentmarker}{}%
\end{pgfscope}%
\begin{pgfscope}%
\pgfsys@transformshift{2.762948in}{2.067132in}%
\pgfsys@useobject{currentmarker}{}%
\end{pgfscope}%
\begin{pgfscope}%
\pgfsys@transformshift{2.744402in}{2.235299in}%
\pgfsys@useobject{currentmarker}{}%
\end{pgfscope}%
\begin{pgfscope}%
\pgfsys@transformshift{2.743697in}{2.422114in}%
\pgfsys@useobject{currentmarker}{}%
\end{pgfscope}%
\begin{pgfscope}%
\pgfsys@transformshift{2.725386in}{2.555132in}%
\pgfsys@useobject{currentmarker}{}%
\end{pgfscope}%
\begin{pgfscope}%
\pgfsys@transformshift{2.707072in}{2.800475in}%
\pgfsys@useobject{currentmarker}{}%
\end{pgfscope}%
\begin{pgfscope}%
\pgfsys@transformshift{2.688761in}{2.807763in}%
\pgfsys@useobject{currentmarker}{}%
\end{pgfscope}%
\begin{pgfscope}%
\pgfsys@transformshift{2.666223in}{2.536938in}%
\pgfsys@useobject{currentmarker}{}%
\end{pgfscope}%
\begin{pgfscope}%
\pgfsys@transformshift{2.644624in}{2.198351in}%
\pgfsys@useobject{currentmarker}{}%
\end{pgfscope}%
\begin{pgfscope}%
\pgfsys@transformshift{2.629364in}{2.082409in}%
\pgfsys@useobject{currentmarker}{}%
\end{pgfscope}%
\begin{pgfscope}%
\pgfsys@transformshift{2.610113in}{2.004283in}%
\pgfsys@useobject{currentmarker}{}%
\end{pgfscope}%
\begin{pgfscope}%
\pgfsys@transformshift{2.589217in}{1.969476in}%
\pgfsys@useobject{currentmarker}{}%
\end{pgfscope}%
\begin{pgfscope}%
\pgfsys@transformshift{2.571140in}{1.984897in}%
\pgfsys@useobject{currentmarker}{}%
\end{pgfscope}%
\begin{pgfscope}%
\pgfsys@transformshift{2.551654in}{2.034867in}%
\pgfsys@useobject{currentmarker}{}%
\end{pgfscope}%
\begin{pgfscope}%
\pgfsys@transformshift{2.532403in}{2.150751in}%
\pgfsys@useobject{currentmarker}{}%
\end{pgfscope}%
\begin{pgfscope}%
\pgfsys@transformshift{2.515029in}{2.412500in}%
\pgfsys@useobject{currentmarker}{}%
\end{pgfscope}%
\begin{pgfscope}%
\pgfsys@transformshift{2.493195in}{2.732682in}%
\pgfsys@useobject{currentmarker}{}%
\end{pgfscope}%
\begin{pgfscope}%
\pgfsys@transformshift{2.475352in}{2.811916in}%
\pgfsys@useobject{currentmarker}{}%
\end{pgfscope}%
\begin{pgfscope}%
\pgfsys@transformshift{2.456101in}{2.744372in}%
\pgfsys@useobject{currentmarker}{}%
\end{pgfscope}%
\begin{pgfscope}%
\pgfsys@transformshift{2.437085in}{2.510887in}%
\pgfsys@useobject{currentmarker}{}%
\end{pgfscope}%
\begin{pgfscope}%
\pgfsys@transformshift{2.418302in}{2.228965in}%
\pgfsys@useobject{currentmarker}{}%
\end{pgfscope}%
\begin{pgfscope}%
\pgfsys@transformshift{2.397173in}{2.055613in}%
\pgfsys@useobject{currentmarker}{}%
\end{pgfscope}%
\begin{pgfscope}%
\pgfsys@transformshift{2.378391in}{2.020867in}%
\pgfsys@useobject{currentmarker}{}%
\end{pgfscope}%
\begin{pgfscope}%
\pgfsys@transformshift{2.360549in}{1.975865in}%
\pgfsys@useobject{currentmarker}{}%
\end{pgfscope}%
\begin{pgfscope}%
\pgfsys@transformshift{2.341063in}{1.973067in}%
\pgfsys@useobject{currentmarker}{}%
\end{pgfscope}%
\begin{pgfscope}%
\pgfsys@transformshift{2.323220in}{2.002858in}%
\pgfsys@useobject{currentmarker}{}%
\end{pgfscope}%
\begin{pgfscope}%
\pgfsys@transformshift{2.301621in}{2.099400in}%
\pgfsys@useobject{currentmarker}{}%
\end{pgfscope}%
\begin{pgfscope}%
\pgfsys@transformshift{2.286127in}{2.253236in}%
\pgfsys@useobject{currentmarker}{}%
\end{pgfscope}%
\begin{pgfscope}%
\pgfsys@transformshift{2.263587in}{2.615733in}%
\pgfsys@useobject{currentmarker}{}%
\end{pgfscope}%
\begin{pgfscope}%
\pgfsys@transformshift{2.246216in}{2.802539in}%
\pgfsys@useobject{currentmarker}{}%
\end{pgfscope}%
\begin{pgfscope}%
\pgfsys@transformshift{2.223911in}{2.813676in}%
\pgfsys@useobject{currentmarker}{}%
\end{pgfscope}%
\begin{pgfscope}%
\pgfsys@transformshift{2.205130in}{2.633693in}%
\pgfsys@useobject{currentmarker}{}%
\end{pgfscope}%
\begin{pgfscope}%
\pgfsys@transformshift{2.186582in}{2.331698in}%
\pgfsys@useobject{currentmarker}{}%
\end{pgfscope}%
\begin{pgfscope}%
\pgfsys@transformshift{2.168505in}{2.145001in}%
\pgfsys@useobject{currentmarker}{}%
\end{pgfscope}%
\begin{pgfscope}%
\pgfsys@transformshift{2.147140in}{2.025157in}%
\pgfsys@useobject{currentmarker}{}%
\end{pgfscope}%
\begin{pgfscope}%
\pgfsys@transformshift{2.127889in}{1.986975in}%
\pgfsys@useobject{currentmarker}{}%
\end{pgfscope}%
\begin{pgfscope}%
\pgfsys@transformshift{2.111690in}{1.971137in}%
\pgfsys@useobject{currentmarker}{}%
\end{pgfscope}%
\begin{pgfscope}%
\pgfsys@transformshift{2.091501in}{1.997158in}%
\pgfsys@useobject{currentmarker}{}%
\end{pgfscope}%
\begin{pgfscope}%
\pgfsys@transformshift{2.069901in}{2.070743in}%
\pgfsys@useobject{currentmarker}{}%
\end{pgfscope}%
\begin{pgfscope}%
\pgfsys@transformshift{2.052527in}{2.225210in}%
\pgfsys@useobject{currentmarker}{}%
\end{pgfscope}%
\begin{pgfscope}%
\pgfsys@transformshift{2.033042in}{2.535468in}%
\pgfsys@useobject{currentmarker}{}%
\end{pgfscope}%
\begin{pgfscope}%
\pgfsys@transformshift{2.013556in}{2.786774in}%
\pgfsys@useobject{currentmarker}{}%
\end{pgfscope}%
\begin{pgfscope}%
\pgfsys@transformshift{1.995477in}{2.821735in}%
\pgfsys@useobject{currentmarker}{}%
\end{pgfscope}%
\begin{pgfscope}%
\pgfsys@transformshift{1.973645in}{2.773645in}%
\pgfsys@useobject{currentmarker}{}%
\end{pgfscope}%
\begin{pgfscope}%
\pgfsys@transformshift{1.955331in}{2.514831in}%
\pgfsys@useobject{currentmarker}{}%
\end{pgfscope}%
\begin{pgfscope}%
\pgfsys@transformshift{1.937958in}{2.805213in}%
\pgfsys@useobject{currentmarker}{}%
\end{pgfscope}%
\begin{pgfscope}%
\pgfsys@transformshift{1.914951in}{2.752375in}%
\pgfsys@useobject{currentmarker}{}%
\end{pgfscope}%
\begin{pgfscope}%
\pgfsys@transformshift{1.900161in}{2.553864in}%
\pgfsys@useobject{currentmarker}{}%
\end{pgfscope}%
\begin{pgfscope}%
\pgfsys@transformshift{1.881379in}{2.234190in}%
\pgfsys@useobject{currentmarker}{}%
\end{pgfscope}%
\begin{pgfscope}%
\pgfsys@transformshift{1.860015in}{2.072130in}%
\pgfsys@useobject{currentmarker}{}%
\end{pgfscope}%
\begin{pgfscope}%
\pgfsys@transformshift{1.840528in}{2.011261in}%
\pgfsys@useobject{currentmarker}{}%
\end{pgfscope}%
\begin{pgfscope}%
\pgfsys@transformshift{1.822451in}{1.976376in}%
\pgfsys@useobject{currentmarker}{}%
\end{pgfscope}%
\begin{pgfscope}%
\pgfsys@transformshift{1.803903in}{1.984215in}%
\pgfsys@useobject{currentmarker}{}%
\end{pgfscope}%
\begin{pgfscope}%
\pgfsys@transformshift{1.781834in}{2.046160in}%
\pgfsys@useobject{currentmarker}{}%
\end{pgfscope}%
\begin{pgfscope}%
\pgfsys@transformshift{1.766809in}{2.128542in}%
\pgfsys@useobject{currentmarker}{}%
\end{pgfscope}%
\begin{pgfscope}%
\pgfsys@transformshift{1.745211in}{2.377772in}%
\pgfsys@useobject{currentmarker}{}%
\end{pgfscope}%
\begin{pgfscope}%
\pgfsys@transformshift{1.726429in}{2.672968in}%
\pgfsys@useobject{currentmarker}{}%
\end{pgfscope}%
\begin{pgfscope}%
\pgfsys@transformshift{1.708352in}{2.833929in}%
\pgfsys@useobject{currentmarker}{}%
\end{pgfscope}%
\begin{pgfscope}%
\pgfsys@transformshift{1.686752in}{2.819310in}%
\pgfsys@useobject{currentmarker}{}%
\end{pgfscope}%
\begin{pgfscope}%
\pgfsys@transformshift{1.668441in}{2.608628in}%
\pgfsys@useobject{currentmarker}{}%
\end{pgfscope}%
\begin{pgfscope}%
\pgfsys@transformshift{1.650128in}{2.334728in}%
\pgfsys@useobject{currentmarker}{}%
\end{pgfscope}%
\begin{pgfscope}%
\pgfsys@transformshift{1.628999in}{2.151079in}%
\pgfsys@useobject{currentmarker}{}%
\end{pgfscope}%
\begin{pgfscope}%
\pgfsys@transformshift{1.609982in}{2.050406in}%
\pgfsys@useobject{currentmarker}{}%
\end{pgfscope}%
\begin{pgfscope}%
\pgfsys@transformshift{1.590965in}{1.993371in}%
\pgfsys@useobject{currentmarker}{}%
\end{pgfscope}%
\begin{pgfscope}%
\pgfsys@transformshift{1.572652in}{1.973290in}%
\pgfsys@useobject{currentmarker}{}%
\end{pgfscope}%
\begin{pgfscope}%
\pgfsys@transformshift{1.554340in}{1.987307in}%
\pgfsys@useobject{currentmarker}{}%
\end{pgfscope}%
\begin{pgfscope}%
\pgfsys@transformshift{1.533681in}{2.052144in}%
\pgfsys@useobject{currentmarker}{}%
\end{pgfscope}%
\begin{pgfscope}%
\pgfsys@transformshift{1.514898in}{2.167313in}%
\pgfsys@useobject{currentmarker}{}%
\end{pgfscope}%
\begin{pgfscope}%
\pgfsys@transformshift{1.495647in}{2.393741in}%
\pgfsys@useobject{currentmarker}{}%
\end{pgfscope}%
\begin{pgfscope}%
\pgfsys@transformshift{1.475927in}{2.715713in}%
\pgfsys@useobject{currentmarker}{}%
\end{pgfscope}%
\begin{pgfscope}%
\pgfsys@transformshift{1.456205in}{2.854131in}%
\pgfsys@useobject{currentmarker}{}%
\end{pgfscope}%
\begin{pgfscope}%
\pgfsys@transformshift{1.438128in}{2.832514in}%
\pgfsys@useobject{currentmarker}{}%
\end{pgfscope}%
\begin{pgfscope}%
\pgfsys@transformshift{1.420285in}{2.652549in}%
\pgfsys@useobject{currentmarker}{}%
\end{pgfscope}%
\begin{pgfscope}%
\pgfsys@transformshift{1.397748in}{2.374693in}%
\pgfsys@useobject{currentmarker}{}%
\end{pgfscope}%
\begin{pgfscope}%
\pgfsys@transformshift{1.378731in}{2.209992in}%
\pgfsys@useobject{currentmarker}{}%
\end{pgfscope}%
\begin{pgfscope}%
\pgfsys@transformshift{1.360654in}{2.096168in}%
\pgfsys@useobject{currentmarker}{}%
\end{pgfscope}%
\begin{pgfscope}%
\pgfsys@transformshift{1.339758in}{2.015496in}%
\pgfsys@useobject{currentmarker}{}%
\end{pgfscope}%
\begin{pgfscope}%
\pgfsys@transformshift{1.322621in}{2.195596in}%
\pgfsys@useobject{currentmarker}{}%
\end{pgfscope}%
\begin{pgfscope}%
\pgfsys@transformshift{1.303604in}{2.075298in}%
\pgfsys@useobject{currentmarker}{}%
\end{pgfscope}%
\begin{pgfscope}%
\pgfsys@transformshift{1.283647in}{2.009475in}%
\pgfsys@useobject{currentmarker}{}%
\end{pgfscope}%
\begin{pgfscope}%
\pgfsys@transformshift{1.266510in}{1.979757in}%
\pgfsys@useobject{currentmarker}{}%
\end{pgfscope}%
\begin{pgfscope}%
\pgfsys@transformshift{1.245616in}{1.982053in}%
\pgfsys@useobject{currentmarker}{}%
\end{pgfscope}%
\begin{pgfscope}%
\pgfsys@transformshift{1.225659in}{2.034997in}%
\pgfsys@useobject{currentmarker}{}%
\end{pgfscope}%
\begin{pgfscope}%
\pgfsys@transformshift{1.208757in}{2.120077in}%
\pgfsys@useobject{currentmarker}{}%
\end{pgfscope}%
\begin{pgfscope}%
\pgfsys@transformshift{1.188097in}{2.298658in}%
\pgfsys@useobject{currentmarker}{}%
\end{pgfscope}%
\begin{pgfscope}%
\pgfsys@transformshift{1.167671in}{2.538549in}%
\pgfsys@useobject{currentmarker}{}%
\end{pgfscope}%
\begin{pgfscope}%
\pgfsys@transformshift{1.150532in}{2.773182in}%
\pgfsys@useobject{currentmarker}{}%
\end{pgfscope}%
\begin{pgfscope}%
\pgfsys@transformshift{1.128932in}{2.884683in}%
\pgfsys@useobject{currentmarker}{}%
\end{pgfscope}%
\begin{pgfscope}%
\pgfsys@transformshift{1.108978in}{2.807712in}%
\pgfsys@useobject{currentmarker}{}%
\end{pgfscope}%
\begin{pgfscope}%
\pgfsys@transformshift{1.091135in}{2.547051in}%
\pgfsys@useobject{currentmarker}{}%
\end{pgfscope}%
\begin{pgfscope}%
\pgfsys@transformshift{1.071413in}{2.270995in}%
\pgfsys@useobject{currentmarker}{}%
\end{pgfscope}%
\begin{pgfscope}%
\pgfsys@transformshift{1.053805in}{2.132796in}%
\pgfsys@useobject{currentmarker}{}%
\end{pgfscope}%
\begin{pgfscope}%
\pgfsys@transformshift{1.034319in}{2.056115in}%
\pgfsys@useobject{currentmarker}{}%
\end{pgfscope}%
\begin{pgfscope}%
\pgfsys@transformshift{1.016242in}{2.002482in}%
\pgfsys@useobject{currentmarker}{}%
\end{pgfscope}%
\begin{pgfscope}%
\pgfsys@transformshift{0.995348in}{1.981410in}%
\pgfsys@useobject{currentmarker}{}%
\end{pgfscope}%
\begin{pgfscope}%
\pgfsys@transformshift{0.977269in}{2.003269in}%
\pgfsys@useobject{currentmarker}{}%
\end{pgfscope}%
\begin{pgfscope}%
\pgfsys@transformshift{0.956140in}{2.053328in}%
\pgfsys@useobject{currentmarker}{}%
\end{pgfscope}%
\begin{pgfscope}%
\pgfsys@transformshift{0.939472in}{2.140438in}%
\pgfsys@useobject{currentmarker}{}%
\end{pgfscope}%
\begin{pgfscope}%
\pgfsys@transformshift{0.919047in}{2.303220in}%
\pgfsys@useobject{currentmarker}{}%
\end{pgfscope}%
\begin{pgfscope}%
\pgfsys@transformshift{0.899090in}{2.618444in}%
\pgfsys@useobject{currentmarker}{}%
\end{pgfscope}%
\begin{pgfscope}%
\pgfsys@transformshift{0.880779in}{2.842622in}%
\pgfsys@useobject{currentmarker}{}%
\end{pgfscope}%
\begin{pgfscope}%
\pgfsys@transformshift{0.860353in}{2.910902in}%
\pgfsys@useobject{currentmarker}{}%
\end{pgfscope}%
\begin{pgfscope}%
\pgfsys@transformshift{0.840868in}{2.828776in}%
\pgfsys@useobject{currentmarker}{}%
\end{pgfscope}%
\begin{pgfscope}%
\pgfsys@transformshift{0.823260in}{2.564559in}%
\pgfsys@useobject{currentmarker}{}%
\end{pgfscope}%
\begin{pgfscope}%
\pgfsys@transformshift{0.801894in}{2.283113in}%
\pgfsys@useobject{currentmarker}{}%
\end{pgfscope}%
\begin{pgfscope}%
\pgfsys@transformshift{0.783348in}{2.150053in}%
\pgfsys@useobject{currentmarker}{}%
\end{pgfscope}%
\begin{pgfscope}%
\pgfsys@transformshift{0.766444in}{2.069938in}%
\pgfsys@useobject{currentmarker}{}%
\end{pgfscope}%
\begin{pgfscope}%
\pgfsys@transformshift{0.746724in}{2.014753in}%
\pgfsys@useobject{currentmarker}{}%
\end{pgfscope}%
\begin{pgfscope}%
\pgfsys@transformshift{0.726767in}{1.985043in}%
\pgfsys@useobject{currentmarker}{}%
\end{pgfscope}%
\begin{pgfscope}%
\pgfsys@transformshift{0.707047in}{2.002811in}%
\pgfsys@useobject{currentmarker}{}%
\end{pgfscope}%
\begin{pgfscope}%
\pgfsys@transformshift{0.686621in}{2.056084in}%
\pgfsys@useobject{currentmarker}{}%
\end{pgfscope}%
\begin{pgfscope}%
\pgfsys@transformshift{0.668779in}{2.150454in}%
\pgfsys@useobject{currentmarker}{}%
\end{pgfscope}%
\begin{pgfscope}%
\pgfsys@transformshift{0.651640in}{2.295302in}%
\pgfsys@useobject{currentmarker}{}%
\end{pgfscope}%
\begin{pgfscope}%
\pgfsys@transformshift{0.651171in}{2.289913in}%
\pgfsys@useobject{currentmarker}{}%
\end{pgfscope}%
\begin{pgfscope}%
\pgfsys@transformshift{0.655163in}{2.239499in}%
\pgfsys@useobject{currentmarker}{}%
\end{pgfscope}%
\begin{pgfscope}%
\pgfsys@transformshift{0.673945in}{2.064223in}%
\pgfsys@useobject{currentmarker}{}%
\end{pgfscope}%
\begin{pgfscope}%
\pgfsys@transformshift{0.698360in}{1.986917in}%
\pgfsys@useobject{currentmarker}{}%
\end{pgfscope}%
\begin{pgfscope}%
\pgfsys@transformshift{0.711742in}{2.015616in}%
\pgfsys@useobject{currentmarker}{}%
\end{pgfscope}%
\begin{pgfscope}%
\pgfsys@transformshift{0.734516in}{2.142899in}%
\pgfsys@useobject{currentmarker}{}%
\end{pgfscope}%
\begin{pgfscope}%
\pgfsys@transformshift{0.754002in}{2.371397in}%
\pgfsys@useobject{currentmarker}{}%
\end{pgfscope}%
\begin{pgfscope}%
\pgfsys@transformshift{0.771610in}{2.761279in}%
\pgfsys@useobject{currentmarker}{}%
\end{pgfscope}%
\begin{pgfscope}%
\pgfsys@transformshift{0.790626in}{2.914565in}%
\pgfsys@useobject{currentmarker}{}%
\end{pgfscope}%
\begin{pgfscope}%
\pgfsys@transformshift{0.809172in}{2.736874in}%
\pgfsys@useobject{currentmarker}{}%
\end{pgfscope}%
\begin{pgfscope}%
\pgfsys@transformshift{0.827486in}{2.361101in}%
\pgfsys@useobject{currentmarker}{}%
\end{pgfscope}%
\begin{pgfscope}%
\pgfsys@transformshift{0.849554in}{2.093119in}%
\pgfsys@useobject{currentmarker}{}%
\end{pgfscope}%
\begin{pgfscope}%
\pgfsys@transformshift{0.867631in}{2.004332in}%
\pgfsys@useobject{currentmarker}{}%
\end{pgfscope}%
\begin{pgfscope}%
\pgfsys@transformshift{0.887117in}{1.986947in}%
\pgfsys@useobject{currentmarker}{}%
\end{pgfscope}%
\begin{pgfscope}%
\pgfsys@transformshift{0.905899in}{2.046148in}%
\pgfsys@useobject{currentmarker}{}%
\end{pgfscope}%
\begin{pgfscope}%
\pgfsys@transformshift{0.924447in}{2.181655in}%
\pgfsys@useobject{currentmarker}{}%
\end{pgfscope}%
\begin{pgfscope}%
\pgfsys@transformshift{0.943227in}{2.475650in}%
\pgfsys@useobject{currentmarker}{}%
\end{pgfscope}%
\begin{pgfscope}%
\pgfsys@transformshift{0.962010in}{2.819890in}%
\pgfsys@useobject{currentmarker}{}%
\end{pgfscope}%
\begin{pgfscope}%
\pgfsys@transformshift{0.983844in}{2.849901in}%
\pgfsys@useobject{currentmarker}{}%
\end{pgfscope}%
\begin{pgfscope}%
\pgfsys@transformshift{1.004035in}{2.545712in}%
\pgfsys@useobject{currentmarker}{}%
\end{pgfscope}%
\begin{pgfscope}%
\pgfsys@transformshift{1.021643in}{2.206663in}%
\pgfsys@useobject{currentmarker}{}%
\end{pgfscope}%
\begin{pgfscope}%
\pgfsys@transformshift{1.039954in}{2.051227in}%
\pgfsys@useobject{currentmarker}{}%
\end{pgfscope}%
\begin{pgfscope}%
\pgfsys@transformshift{1.057328in}{1.985225in}%
\pgfsys@useobject{currentmarker}{}%
\end{pgfscope}%
\begin{pgfscope}%
\pgfsys@transformshift{1.077048in}{1.991887in}%
\pgfsys@useobject{currentmarker}{}%
\end{pgfscope}%
\begin{pgfscope}%
\pgfsys@transformshift{1.095125in}{2.066790in}%
\pgfsys@useobject{currentmarker}{}%
\end{pgfscope}%
\begin{pgfscope}%
\pgfsys@transformshift{1.118837in}{2.248086in}%
\pgfsys@useobject{currentmarker}{}%
\end{pgfscope}%
\begin{pgfscope}%
\pgfsys@transformshift{1.135741in}{2.588251in}%
\pgfsys@useobject{currentmarker}{}%
\end{pgfscope}%
\begin{pgfscope}%
\pgfsys@transformshift{1.154524in}{2.850042in}%
\pgfsys@useobject{currentmarker}{}%
\end{pgfscope}%
\begin{pgfscope}%
\pgfsys@transformshift{1.173304in}{2.798802in}%
\pgfsys@useobject{currentmarker}{}%
\end{pgfscope}%
\begin{pgfscope}%
\pgfsys@transformshift{1.191383in}{2.459526in}%
\pgfsys@useobject{currentmarker}{}%
\end{pgfscope}%
\begin{pgfscope}%
\pgfsys@transformshift{1.213217in}{2.146331in}%
\pgfsys@useobject{currentmarker}{}%
\end{pgfscope}%
\begin{pgfscope}%
\pgfsys@transformshift{1.233172in}{2.019815in}%
\pgfsys@useobject{currentmarker}{}%
\end{pgfscope}%
\begin{pgfscope}%
\pgfsys@transformshift{1.251954in}{1.978162in}%
\pgfsys@useobject{currentmarker}{}%
\end{pgfscope}%
\begin{pgfscope}%
\pgfsys@transformshift{1.269562in}{1.988308in}%
\pgfsys@useobject{currentmarker}{}%
\end{pgfscope}%
\begin{pgfscope}%
\pgfsys@transformshift{1.289517in}{2.062248in}%
\pgfsys@useobject{currentmarker}{}%
\end{pgfscope}%
\begin{pgfscope}%
\pgfsys@transformshift{1.308064in}{2.210798in}%
\pgfsys@useobject{currentmarker}{}%
\end{pgfscope}%
\begin{pgfscope}%
\pgfsys@transformshift{1.325907in}{2.504845in}%
\pgfsys@useobject{currentmarker}{}%
\end{pgfscope}%
\begin{pgfscope}%
\pgfsys@transformshift{1.346801in}{2.830085in}%
\pgfsys@useobject{currentmarker}{}%
\end{pgfscope}%
\begin{pgfscope}%
\pgfsys@transformshift{1.366992in}{2.776420in}%
\pgfsys@useobject{currentmarker}{}%
\end{pgfscope}%
\begin{pgfscope}%
\pgfsys@transformshift{1.385538in}{2.439632in}%
\pgfsys@useobject{currentmarker}{}%
\end{pgfscope}%
\begin{pgfscope}%
\pgfsys@transformshift{1.403852in}{2.176773in}%
\pgfsys@useobject{currentmarker}{}%
\end{pgfscope}%
\begin{pgfscope}%
\pgfsys@transformshift{1.425686in}{2.026329in}%
\pgfsys@useobject{currentmarker}{}%
\end{pgfscope}%
\begin{pgfscope}%
\pgfsys@transformshift{1.442120in}{1.978555in}%
\pgfsys@useobject{currentmarker}{}%
\end{pgfscope}%
\begin{pgfscope}%
\pgfsys@transformshift{1.464188in}{1.991567in}%
\pgfsys@useobject{currentmarker}{}%
\end{pgfscope}%
\begin{pgfscope}%
\pgfsys@transformshift{1.480622in}{2.052238in}%
\pgfsys@useobject{currentmarker}{}%
\end{pgfscope}%
\begin{pgfscope}%
\pgfsys@transformshift{1.501751in}{2.202224in}%
\pgfsys@useobject{currentmarker}{}%
\end{pgfscope}%
\begin{pgfscope}%
\pgfsys@transformshift{1.519593in}{2.458294in}%
\pgfsys@useobject{currentmarker}{}%
\end{pgfscope}%
\begin{pgfscope}%
\pgfsys@transformshift{1.541662in}{2.789839in}%
\pgfsys@useobject{currentmarker}{}%
\end{pgfscope}%
\begin{pgfscope}%
\pgfsys@transformshift{1.557627in}{2.826366in}%
\pgfsys@useobject{currentmarker}{}%
\end{pgfscope}%
\begin{pgfscope}%
\pgfsys@transformshift{1.578287in}{2.515962in}%
\pgfsys@useobject{currentmarker}{}%
\end{pgfscope}%
\begin{pgfscope}%
\pgfsys@transformshift{1.596129in}{2.801612in}%
\pgfsys@useobject{currentmarker}{}%
\end{pgfscope}%
\begin{pgfscope}%
\pgfsys@transformshift{1.617963in}{2.810995in}%
\pgfsys@useobject{currentmarker}{}%
\end{pgfscope}%
\begin{pgfscope}%
\pgfsys@transformshift{1.636277in}{2.639206in}%
\pgfsys@useobject{currentmarker}{}%
\end{pgfscope}%
\begin{pgfscope}%
\pgfsys@transformshift{1.655762in}{2.277796in}%
\pgfsys@useobject{currentmarker}{}%
\end{pgfscope}%
\begin{pgfscope}%
\pgfsys@transformshift{1.674779in}{2.064250in}%
\pgfsys@useobject{currentmarker}{}%
\end{pgfscope}%
\begin{pgfscope}%
\pgfsys@transformshift{1.693325in}{1.994010in}%
\pgfsys@useobject{currentmarker}{}%
\end{pgfscope}%
\begin{pgfscope}%
\pgfsys@transformshift{1.713985in}{1.973845in}%
\pgfsys@useobject{currentmarker}{}%
\end{pgfscope}%
\begin{pgfscope}%
\pgfsys@transformshift{1.731593in}{2.007029in}%
\pgfsys@useobject{currentmarker}{}%
\end{pgfscope}%
\begin{pgfscope}%
\pgfsys@transformshift{1.752253in}{2.089221in}%
\pgfsys@useobject{currentmarker}{}%
\end{pgfscope}%
\begin{pgfscope}%
\pgfsys@transformshift{1.770566in}{2.271106in}%
\pgfsys@useobject{currentmarker}{}%
\end{pgfscope}%
\begin{pgfscope}%
\pgfsys@transformshift{1.789112in}{2.575804in}%
\pgfsys@useobject{currentmarker}{}%
\end{pgfscope}%
\begin{pgfscope}%
\pgfsys@transformshift{1.809303in}{2.817958in}%
\pgfsys@useobject{currentmarker}{}%
\end{pgfscope}%
\begin{pgfscope}%
\pgfsys@transformshift{1.828320in}{2.764732in}%
\pgfsys@useobject{currentmarker}{}%
\end{pgfscope}%
\begin{pgfscope}%
\pgfsys@transformshift{1.849449in}{2.457404in}%
\pgfsys@useobject{currentmarker}{}%
\end{pgfscope}%
\begin{pgfscope}%
\pgfsys@transformshift{1.867762in}{2.169734in}%
\pgfsys@useobject{currentmarker}{}%
\end{pgfscope}%
\begin{pgfscope}%
\pgfsys@transformshift{1.867997in}{2.078696in}%
\pgfsys@useobject{currentmarker}{}%
\end{pgfscope}%
\begin{pgfscope}%
\pgfsys@transformshift{1.887951in}{2.034975in}%
\pgfsys@useobject{currentmarker}{}%
\end{pgfscope}%
\begin{pgfscope}%
\pgfsys@transformshift{1.907204in}{1.981303in}%
\pgfsys@useobject{currentmarker}{}%
\end{pgfscope}%
\begin{pgfscope}%
\pgfsys@transformshift{1.924107in}{1.970241in}%
\pgfsys@useobject{currentmarker}{}%
\end{pgfscope}%
\begin{pgfscope}%
\pgfsys@transformshift{1.944767in}{2.005600in}%
\pgfsys@useobject{currentmarker}{}%
\end{pgfscope}%
\begin{pgfscope}%
\pgfsys@transformshift{1.962610in}{2.073965in}%
\pgfsys@useobject{currentmarker}{}%
\end{pgfscope}%
\begin{pgfscope}%
\pgfsys@transformshift{1.981157in}{2.223034in}%
\pgfsys@useobject{currentmarker}{}%
\end{pgfscope}%
\begin{pgfscope}%
\pgfsys@transformshift{2.002052in}{2.578485in}%
\pgfsys@useobject{currentmarker}{}%
\end{pgfscope}%
\begin{pgfscope}%
\pgfsys@transformshift{2.023415in}{2.819047in}%
\pgfsys@useobject{currentmarker}{}%
\end{pgfscope}%
\begin{pgfscope}%
\pgfsys@transformshift{2.037502in}{2.777672in}%
\pgfsys@useobject{currentmarker}{}%
\end{pgfscope}%
\begin{pgfscope}%
\pgfsys@transformshift{2.059102in}{2.559253in}%
\pgfsys@useobject{currentmarker}{}%
\end{pgfscope}%
\begin{pgfscope}%
\pgfsys@transformshift{2.076944in}{2.258430in}%
\pgfsys@useobject{currentmarker}{}%
\end{pgfscope}%
\begin{pgfscope}%
\pgfsys@transformshift{2.097839in}{2.063688in}%
\pgfsys@useobject{currentmarker}{}%
\end{pgfscope}%
\begin{pgfscope}%
\pgfsys@transformshift{2.117090in}{1.995451in}%
\pgfsys@useobject{currentmarker}{}%
\end{pgfscope}%
\begin{pgfscope}%
\pgfsys@transformshift{2.140333in}{1.971075in}%
\pgfsys@useobject{currentmarker}{}%
\end{pgfscope}%
\begin{pgfscope}%
\pgfsys@transformshift{2.155124in}{1.981693in}%
\pgfsys@useobject{currentmarker}{}%
\end{pgfscope}%
\begin{pgfscope}%
\pgfsys@transformshift{2.175078in}{2.029038in}%
\pgfsys@useobject{currentmarker}{}%
\end{pgfscope}%
\begin{pgfscope}%
\pgfsys@transformshift{2.193157in}{2.130807in}%
\pgfsys@useobject{currentmarker}{}%
\end{pgfscope}%
\begin{pgfscope}%
\pgfsys@transformshift{2.214520in}{2.345732in}%
\pgfsys@useobject{currentmarker}{}%
\end{pgfscope}%
\begin{pgfscope}%
\pgfsys@transformshift{2.232363in}{2.670956in}%
\pgfsys@useobject{currentmarker}{}%
\end{pgfscope}%
\begin{pgfscope}%
\pgfsys@transformshift{2.250676in}{2.715659in}%
\pgfsys@useobject{currentmarker}{}%
\end{pgfscope}%
\begin{pgfscope}%
\pgfsys@transformshift{2.274623in}{2.792397in}%
\pgfsys@useobject{currentmarker}{}%
\end{pgfscope}%
\begin{pgfscope}%
\pgfsys@transformshift{2.289648in}{2.589647in}%
\pgfsys@useobject{currentmarker}{}%
\end{pgfscope}%
\begin{pgfscope}%
\pgfsys@transformshift{2.307490in}{2.277382in}%
\pgfsys@useobject{currentmarker}{}%
\end{pgfscope}%
\begin{pgfscope}%
\pgfsys@transformshift{2.327916in}{2.077590in}%
\pgfsys@useobject{currentmarker}{}%
\end{pgfscope}%
\begin{pgfscope}%
\pgfsys@transformshift{2.345524in}{2.021211in}%
\pgfsys@useobject{currentmarker}{}%
\end{pgfscope}%
\begin{pgfscope}%
\pgfsys@transformshift{2.367123in}{1.975578in}%
\pgfsys@useobject{currentmarker}{}%
\end{pgfscope}%
\begin{pgfscope}%
\pgfsys@transformshift{2.384495in}{1.972344in}%
\pgfsys@useobject{currentmarker}{}%
\end{pgfscope}%
\begin{pgfscope}%
\pgfsys@transformshift{2.406564in}{2.011846in}%
\pgfsys@useobject{currentmarker}{}%
\end{pgfscope}%
\begin{pgfscope}%
\pgfsys@transformshift{2.423937in}{2.092153in}%
\pgfsys@useobject{currentmarker}{}%
\end{pgfscope}%
\begin{pgfscope}%
\pgfsys@transformshift{2.443659in}{2.271823in}%
\pgfsys@useobject{currentmarker}{}%
\end{pgfscope}%
\begin{pgfscope}%
\pgfsys@transformshift{2.461971in}{2.541003in}%
\pgfsys@useobject{currentmarker}{}%
\end{pgfscope}%
\begin{pgfscope}%
\pgfsys@transformshift{2.480987in}{2.793542in}%
\pgfsys@useobject{currentmarker}{}%
\end{pgfscope}%
\begin{pgfscope}%
\pgfsys@transformshift{2.502587in}{2.790147in}%
\pgfsys@useobject{currentmarker}{}%
\end{pgfscope}%
\begin{pgfscope}%
\pgfsys@transformshift{2.521133in}{2.543619in}%
\pgfsys@useobject{currentmarker}{}%
\end{pgfscope}%
\begin{pgfscope}%
\pgfsys@transformshift{2.541793in}{2.204400in}%
\pgfsys@useobject{currentmarker}{}%
\end{pgfscope}%
\begin{pgfscope}%
\pgfsys@transformshift{2.558227in}{2.074032in}%
\pgfsys@useobject{currentmarker}{}%
\end{pgfscope}%
\begin{pgfscope}%
\pgfsys@transformshift{2.578418in}{1.995012in}%
\pgfsys@useobject{currentmarker}{}%
\end{pgfscope}%
\begin{pgfscope}%
\pgfsys@transformshift{2.596026in}{2.013894in}%
\pgfsys@useobject{currentmarker}{}%
\end{pgfscope}%
\begin{pgfscope}%
\pgfsys@transformshift{2.613634in}{1.993700in}%
\pgfsys@useobject{currentmarker}{}%
\end{pgfscope}%
\begin{pgfscope}%
\pgfsys@transformshift{2.634997in}{1.968816in}%
\pgfsys@useobject{currentmarker}{}%
\end{pgfscope}%
\begin{pgfscope}%
\pgfsys@transformshift{2.653076in}{1.989389in}%
\pgfsys@useobject{currentmarker}{}%
\end{pgfscope}%
\begin{pgfscope}%
\pgfsys@transformshift{2.673736in}{2.059249in}%
\pgfsys@useobject{currentmarker}{}%
\end{pgfscope}%
\begin{pgfscope}%
\pgfsys@transformshift{2.692047in}{2.190820in}%
\pgfsys@useobject{currentmarker}{}%
\end{pgfscope}%
\begin{pgfscope}%
\pgfsys@transformshift{2.712473in}{2.482418in}%
\pgfsys@useobject{currentmarker}{}%
\end{pgfscope}%
\begin{pgfscope}%
\pgfsys@transformshift{2.731255in}{2.729354in}%
\pgfsys@useobject{currentmarker}{}%
\end{pgfscope}%
\begin{pgfscope}%
\pgfsys@transformshift{2.753324in}{2.811305in}%
\pgfsys@useobject{currentmarker}{}%
\end{pgfscope}%
\begin{pgfscope}%
\pgfsys@transformshift{2.773278in}{2.553753in}%
\pgfsys@useobject{currentmarker}{}%
\end{pgfscope}%
\begin{pgfscope}%
\pgfsys@transformshift{2.788540in}{2.537682in}%
\pgfsys@useobject{currentmarker}{}%
\end{pgfscope}%
\begin{pgfscope}%
\pgfsys@transformshift{2.810138in}{2.187921in}%
\pgfsys@useobject{currentmarker}{}%
\end{pgfscope}%
\begin{pgfscope}%
\pgfsys@transformshift{2.829154in}{2.047065in}%
\pgfsys@useobject{currentmarker}{}%
\end{pgfscope}%
\begin{pgfscope}%
\pgfsys@transformshift{2.844650in}{1.995958in}%
\pgfsys@useobject{currentmarker}{}%
\end{pgfscope}%
\begin{pgfscope}%
\pgfsys@transformshift{2.866248in}{1.971255in}%
\pgfsys@useobject{currentmarker}{}%
\end{pgfscope}%
\begin{pgfscope}%
\pgfsys@transformshift{2.888082in}{1.997306in}%
\pgfsys@useobject{currentmarker}{}%
\end{pgfscope}%
\begin{pgfscope}%
\pgfsys@transformshift{2.902873in}{2.044659in}%
\pgfsys@useobject{currentmarker}{}%
\end{pgfscope}%
\begin{pgfscope}%
\pgfsys@transformshift{2.924238in}{2.145512in}%
\pgfsys@useobject{currentmarker}{}%
\end{pgfscope}%
\begin{pgfscope}%
\pgfsys@transformshift{2.942080in}{2.339182in}%
\pgfsys@useobject{currentmarker}{}%
\end{pgfscope}%
\begin{pgfscope}%
\pgfsys@transformshift{2.963209in}{2.671295in}%
\pgfsys@useobject{currentmarker}{}%
\end{pgfscope}%
\begin{pgfscope}%
\pgfsys@transformshift{2.981286in}{2.784830in}%
\pgfsys@useobject{currentmarker}{}%
\end{pgfscope}%
\begin{pgfscope}%
\pgfsys@transformshift{2.998425in}{2.810881in}%
\pgfsys@useobject{currentmarker}{}%
\end{pgfscope}%
\begin{pgfscope}%
\pgfsys@transformshift{3.019320in}{2.552566in}%
\pgfsys@useobject{currentmarker}{}%
\end{pgfscope}%
\begin{pgfscope}%
\pgfsys@transformshift{3.037162in}{2.271180in}%
\pgfsys@useobject{currentmarker}{}%
\end{pgfscope}%
\begin{pgfscope}%
\pgfsys@transformshift{3.055945in}{2.107469in}%
\pgfsys@useobject{currentmarker}{}%
\end{pgfscope}%
\begin{pgfscope}%
\pgfsys@transformshift{3.076604in}{2.025950in}%
\pgfsys@useobject{currentmarker}{}%
\end{pgfscope}%
\begin{pgfscope}%
\pgfsys@transformshift{3.094681in}{1.981232in}%
\pgfsys@useobject{currentmarker}{}%
\end{pgfscope}%
\begin{pgfscope}%
\pgfsys@transformshift{3.116047in}{2.486862in}%
\pgfsys@useobject{currentmarker}{}%
\end{pgfscope}%
\begin{pgfscope}%
\pgfsys@transformshift{3.135063in}{2.622512in}%
\pgfsys@useobject{currentmarker}{}%
\end{pgfscope}%
\begin{pgfscope}%
\pgfsys@transformshift{3.155958in}{2.295020in}%
\pgfsys@useobject{currentmarker}{}%
\end{pgfscope}%
\begin{pgfscope}%
\pgfsys@transformshift{3.176149in}{2.080857in}%
\pgfsys@useobject{currentmarker}{}%
\end{pgfscope}%
\begin{pgfscope}%
\pgfsys@transformshift{3.191174in}{2.027702in}%
\pgfsys@useobject{currentmarker}{}%
\end{pgfscope}%
\begin{pgfscope}%
\pgfsys@transformshift{3.214651in}{1.972021in}%
\pgfsys@useobject{currentmarker}{}%
\end{pgfscope}%
\begin{pgfscope}%
\pgfsys@transformshift{3.229676in}{1.979140in}%
\pgfsys@useobject{currentmarker}{}%
\end{pgfscope}%
\begin{pgfscope}%
\pgfsys@transformshift{3.250805in}{2.028140in}%
\pgfsys@useobject{currentmarker}{}%
\end{pgfscope}%
\begin{pgfscope}%
\pgfsys@transformshift{3.269119in}{2.122907in}%
\pgfsys@useobject{currentmarker}{}%
\end{pgfscope}%
\begin{pgfscope}%
\pgfsys@transformshift{3.292830in}{2.457301in}%
\pgfsys@useobject{currentmarker}{}%
\end{pgfscope}%
\begin{pgfscope}%
\pgfsys@transformshift{3.307855in}{2.702809in}%
\pgfsys@useobject{currentmarker}{}%
\end{pgfscope}%
\begin{pgfscope}%
\pgfsys@transformshift{3.327107in}{2.844496in}%
\pgfsys@useobject{currentmarker}{}%
\end{pgfscope}%
\begin{pgfscope}%
\pgfsys@transformshift{3.345654in}{2.779494in}%
\pgfsys@useobject{currentmarker}{}%
\end{pgfscope}%
\begin{pgfscope}%
\pgfsys@transformshift{3.366080in}{2.531527in}%
\pgfsys@useobject{currentmarker}{}%
\end{pgfscope}%
\begin{pgfscope}%
\pgfsys@transformshift{3.383922in}{2.228888in}%
\pgfsys@useobject{currentmarker}{}%
\end{pgfscope}%
\begin{pgfscope}%
\pgfsys@transformshift{3.404817in}{2.063566in}%
\pgfsys@useobject{currentmarker}{}%
\end{pgfscope}%
\begin{pgfscope}%
\pgfsys@transformshift{3.423365in}{2.003960in}%
\pgfsys@useobject{currentmarker}{}%
\end{pgfscope}%
\begin{pgfscope}%
\pgfsys@transformshift{3.440971in}{1.974821in}%
\pgfsys@useobject{currentmarker}{}%
\end{pgfscope}%
\begin{pgfscope}%
\pgfsys@transformshift{3.462570in}{2.007478in}%
\pgfsys@useobject{currentmarker}{}%
\end{pgfscope}%
\begin{pgfscope}%
\pgfsys@transformshift{3.480178in}{2.075125in}%
\pgfsys@useobject{currentmarker}{}%
\end{pgfscope}%
\begin{pgfscope}%
\pgfsys@transformshift{3.500838in}{2.228136in}%
\pgfsys@useobject{currentmarker}{}%
\end{pgfscope}%
\begin{pgfscope}%
\pgfsys@transformshift{3.517272in}{2.329928in}%
\pgfsys@useobject{currentmarker}{}%
\end{pgfscope}%
\begin{pgfscope}%
\pgfsys@transformshift{3.536994in}{2.624672in}%
\pgfsys@useobject{currentmarker}{}%
\end{pgfscope}%
\begin{pgfscope}%
\pgfsys@transformshift{3.559063in}{2.854125in}%
\pgfsys@useobject{currentmarker}{}%
\end{pgfscope}%
\begin{pgfscope}%
\pgfsys@transformshift{3.579486in}{2.782457in}%
\pgfsys@useobject{currentmarker}{}%
\end{pgfscope}%
\begin{pgfscope}%
\pgfsys@transformshift{3.598034in}{2.548853in}%
\pgfsys@useobject{currentmarker}{}%
\end{pgfscope}%
\begin{pgfscope}%
\pgfsys@transformshift{3.615877in}{2.269835in}%
\pgfsys@useobject{currentmarker}{}%
\end{pgfscope}%
\begin{pgfscope}%
\pgfsys@transformshift{3.633485in}{2.125661in}%
\pgfsys@useobject{currentmarker}{}%
\end{pgfscope}%
\begin{pgfscope}%
\pgfsys@transformshift{3.654614in}{2.030312in}%
\pgfsys@useobject{currentmarker}{}%
\end{pgfscope}%
\begin{pgfscope}%
\pgfsys@transformshift{3.675744in}{1.983113in}%
\pgfsys@useobject{currentmarker}{}%
\end{pgfscope}%
\begin{pgfscope}%
\pgfsys@transformshift{3.690300in}{1.978450in}%
\pgfsys@useobject{currentmarker}{}%
\end{pgfscope}%
\begin{pgfscope}%
\pgfsys@transformshift{3.711898in}{2.028384in}%
\pgfsys@useobject{currentmarker}{}%
\end{pgfscope}%
\begin{pgfscope}%
\pgfsys@transformshift{3.731150in}{2.091615in}%
\pgfsys@useobject{currentmarker}{}%
\end{pgfscope}%
\begin{pgfscope}%
\pgfsys@transformshift{3.750872in}{2.236060in}%
\pgfsys@useobject{currentmarker}{}%
\end{pgfscope}%
\begin{pgfscope}%
\pgfsys@transformshift{3.769654in}{2.465120in}%
\pgfsys@useobject{currentmarker}{}%
\end{pgfscope}%
\begin{pgfscope}%
\pgfsys@transformshift{3.789843in}{2.764958in}%
\pgfsys@useobject{currentmarker}{}%
\end{pgfscope}%
\begin{pgfscope}%
\pgfsys@transformshift{3.808156in}{2.883296in}%
\pgfsys@useobject{currentmarker}{}%
\end{pgfscope}%
\begin{pgfscope}%
\pgfsys@transformshift{3.827407in}{2.832736in}%
\pgfsys@useobject{currentmarker}{}%
\end{pgfscope}%
\begin{pgfscope}%
\pgfsys@transformshift{3.845484in}{2.580508in}%
\pgfsys@useobject{currentmarker}{}%
\end{pgfscope}%
\begin{pgfscope}%
\pgfsys@transformshift{3.865675in}{2.259631in}%
\pgfsys@useobject{currentmarker}{}%
\end{pgfscope}%
\begin{pgfscope}%
\pgfsys@transformshift{3.884692in}{2.135986in}%
\pgfsys@useobject{currentmarker}{}%
\end{pgfscope}%
\begin{pgfscope}%
\pgfsys@transformshift{3.903004in}{2.046594in}%
\pgfsys@useobject{currentmarker}{}%
\end{pgfscope}%
\begin{pgfscope}%
\pgfsys@transformshift{3.922255in}{2.006209in}%
\pgfsys@useobject{currentmarker}{}%
\end{pgfscope}%
\begin{pgfscope}%
\pgfsys@transformshift{3.940332in}{1.981250in}%
\pgfsys@useobject{currentmarker}{}%
\end{pgfscope}%
\begin{pgfscope}%
\pgfsys@transformshift{3.961228in}{1.992374in}%
\pgfsys@useobject{currentmarker}{}%
\end{pgfscope}%
\begin{pgfscope}%
\pgfsys@transformshift{3.980479in}{2.041873in}%
\pgfsys@useobject{currentmarker}{}%
\end{pgfscope}%
\begin{pgfscope}%
\pgfsys@transformshift{4.002312in}{2.159895in}%
\pgfsys@useobject{currentmarker}{}%
\end{pgfscope}%
\begin{pgfscope}%
\pgfsys@transformshift{4.020156in}{2.323988in}%
\pgfsys@useobject{currentmarker}{}%
\end{pgfscope}%
\begin{pgfscope}%
\pgfsys@transformshift{4.038936in}{2.574999in}%
\pgfsys@useobject{currentmarker}{}%
\end{pgfscope}%
\begin{pgfscope}%
\pgfsys@transformshift{4.057719in}{2.809701in}%
\pgfsys@useobject{currentmarker}{}%
\end{pgfscope}%
\begin{pgfscope}%
\pgfsys@transformshift{4.076501in}{2.900362in}%
\pgfsys@useobject{currentmarker}{}%
\end{pgfscope}%
\begin{pgfscope}%
\pgfsys@transformshift{4.095518in}{2.901698in}%
\pgfsys@useobject{currentmarker}{}%
\end{pgfscope}%
\begin{pgfscope}%
\pgfsys@transformshift{4.117821in}{2.748215in}%
\pgfsys@useobject{currentmarker}{}%
\end{pgfscope}%
\begin{pgfscope}%
\pgfsys@transformshift{4.134960in}{2.500932in}%
\pgfsys@useobject{currentmarker}{}%
\end{pgfscope}%
\begin{pgfscope}%
\pgfsys@transformshift{4.153271in}{2.284355in}%
\pgfsys@useobject{currentmarker}{}%
\end{pgfscope}%
\begin{pgfscope}%
\pgfsys@transformshift{4.172288in}{2.123010in}%
\pgfsys@useobject{currentmarker}{}%
\end{pgfscope}%
\begin{pgfscope}%
\pgfsys@transformshift{4.190599in}{2.065343in}%
\pgfsys@useobject{currentmarker}{}%
\end{pgfscope}%
\begin{pgfscope}%
\pgfsys@transformshift{4.213139in}{1.998196in}%
\pgfsys@useobject{currentmarker}{}%
\end{pgfscope}%
\begin{pgfscope}%
\pgfsys@transformshift{4.232625in}{1.987543in}%
\pgfsys@useobject{currentmarker}{}%
\end{pgfscope}%
\begin{pgfscope}%
\pgfsys@transformshift{4.251171in}{2.005058in}%
\pgfsys@useobject{currentmarker}{}%
\end{pgfscope}%
\begin{pgfscope}%
\pgfsys@transformshift{4.269484in}{2.067171in}%
\pgfsys@useobject{currentmarker}{}%
\end{pgfscope}%
\begin{pgfscope}%
\pgfsys@transformshift{4.293196in}{2.199926in}%
\pgfsys@useobject{currentmarker}{}%
\end{pgfscope}%
\begin{pgfscope}%
\pgfsys@transformshift{4.308455in}{2.313277in}%
\pgfsys@useobject{currentmarker}{}%
\end{pgfscope}%
\begin{pgfscope}%
\pgfsys@transformshift{4.327237in}{2.559630in}%
\pgfsys@useobject{currentmarker}{}%
\end{pgfscope}%
\begin{pgfscope}%
\pgfsys@transformshift{4.346723in}{2.807219in}%
\pgfsys@useobject{currentmarker}{}%
\end{pgfscope}%
\begin{pgfscope}%
\pgfsys@transformshift{4.365271in}{2.927817in}%
\pgfsys@useobject{currentmarker}{}%
\end{pgfscope}%
\begin{pgfscope}%
\pgfsys@transformshift{4.383817in}{2.904494in}%
\pgfsys@useobject{currentmarker}{}%
\end{pgfscope}%
\begin{pgfscope}%
\pgfsys@transformshift{4.403539in}{2.012348in}%
\pgfsys@useobject{currentmarker}{}%
\end{pgfscope}%
\begin{pgfscope}%
\pgfsys@transformshift{4.422085in}{2.087217in}%
\pgfsys@useobject{currentmarker}{}%
\end{pgfscope}%
\begin{pgfscope}%
\pgfsys@transformshift{4.440867in}{2.249327in}%
\pgfsys@useobject{currentmarker}{}%
\end{pgfscope}%
\begin{pgfscope}%
\pgfsys@transformshift{4.460353in}{2.539700in}%
\pgfsys@useobject{currentmarker}{}%
\end{pgfscope}%
\begin{pgfscope}%
\pgfsys@transformshift{4.479135in}{2.865087in}%
\pgfsys@useobject{currentmarker}{}%
\end{pgfscope}%
\begin{pgfscope}%
\pgfsys@transformshift{4.481013in}{2.876595in}%
\pgfsys@useobject{currentmarker}{}%
\end{pgfscope}%
\begin{pgfscope}%
\pgfsys@transformshift{4.475143in}{2.796452in}%
\pgfsys@useobject{currentmarker}{}%
\end{pgfscope}%
\begin{pgfscope}%
\pgfsys@transformshift{4.452137in}{2.308735in}%
\pgfsys@useobject{currentmarker}{}%
\end{pgfscope}%
\begin{pgfscope}%
\pgfsys@transformshift{4.434998in}{2.107016in}%
\pgfsys@useobject{currentmarker}{}%
\end{pgfscope}%
\begin{pgfscope}%
\pgfsys@transformshift{4.416686in}{2.009315in}%
\pgfsys@useobject{currentmarker}{}%
\end{pgfscope}%
\begin{pgfscope}%
\pgfsys@transformshift{4.398139in}{1.989658in}%
\pgfsys@useobject{currentmarker}{}%
\end{pgfscope}%
\begin{pgfscope}%
\pgfsys@transformshift{4.376775in}{2.065538in}%
\pgfsys@useobject{currentmarker}{}%
\end{pgfscope}%
\begin{pgfscope}%
\pgfsys@transformshift{4.359402in}{2.245149in}%
\pgfsys@useobject{currentmarker}{}%
\end{pgfscope}%
\begin{pgfscope}%
\pgfsys@transformshift{4.339916in}{2.633455in}%
\pgfsys@useobject{currentmarker}{}%
\end{pgfscope}%
\begin{pgfscope}%
\pgfsys@transformshift{4.317611in}{2.912156in}%
\pgfsys@useobject{currentmarker}{}%
\end{pgfscope}%
\begin{pgfscope}%
\pgfsys@transformshift{4.303057in}{2.848395in}%
\pgfsys@useobject{currentmarker}{}%
\end{pgfscope}%
\begin{pgfscope}%
\pgfsys@transformshift{4.278171in}{2.349305in}%
\pgfsys@useobject{currentmarker}{}%
\end{pgfscope}%
\begin{pgfscope}%
\pgfsys@transformshift{4.263849in}{2.149409in}%
\pgfsys@useobject{currentmarker}{}%
\end{pgfscope}%
\begin{pgfscope}%
\pgfsys@transformshift{4.244363in}{2.026773in}%
\pgfsys@useobject{currentmarker}{}%
\end{pgfscope}%
\begin{pgfscope}%
\pgfsys@transformshift{4.226050in}{1.981024in}%
\pgfsys@useobject{currentmarker}{}%
\end{pgfscope}%
\begin{pgfscope}%
\pgfsys@transformshift{4.205624in}{2.024880in}%
\pgfsys@useobject{currentmarker}{}%
\end{pgfscope}%
\begin{pgfscope}%
\pgfsys@transformshift{4.188016in}{2.142664in}%
\pgfsys@useobject{currentmarker}{}%
\end{pgfscope}%
\begin{pgfscope}%
\pgfsys@transformshift{4.166419in}{2.468646in}%
\pgfsys@useobject{currentmarker}{}%
\end{pgfscope}%
\begin{pgfscope}%
\pgfsys@transformshift{4.149750in}{2.806053in}%
\pgfsys@useobject{currentmarker}{}%
\end{pgfscope}%
\begin{pgfscope}%
\pgfsys@transformshift{4.127916in}{2.873978in}%
\pgfsys@useobject{currentmarker}{}%
\end{pgfscope}%
\begin{pgfscope}%
\pgfsys@transformshift{4.109368in}{2.569603in}%
\pgfsys@useobject{currentmarker}{}%
\end{pgfscope}%
\begin{pgfscope}%
\pgfsys@transformshift{4.088709in}{2.202213in}%
\pgfsys@useobject{currentmarker}{}%
\end{pgfscope}%
\begin{pgfscope}%
\pgfsys@transformshift{4.067814in}{2.047971in}%
\pgfsys@useobject{currentmarker}{}%
\end{pgfscope}%
\begin{pgfscope}%
\pgfsys@transformshift{4.050206in}{1.996062in}%
\pgfsys@useobject{currentmarker}{}%
\end{pgfscope}%
\begin{pgfscope}%
\pgfsys@transformshift{4.033067in}{1.981682in}%
\pgfsys@useobject{currentmarker}{}%
\end{pgfscope}%
\begin{pgfscope}%
\pgfsys@transformshift{4.014287in}{2.039066in}%
\pgfsys@useobject{currentmarker}{}%
\end{pgfscope}%
\begin{pgfscope}%
\pgfsys@transformshift{3.993861in}{2.173124in}%
\pgfsys@useobject{currentmarker}{}%
\end{pgfscope}%
\begin{pgfscope}%
\pgfsys@transformshift{3.975782in}{2.457828in}%
\pgfsys@useobject{currentmarker}{}%
\end{pgfscope}%
\begin{pgfscope}%
\pgfsys@transformshift{3.954653in}{2.824363in}%
\pgfsys@useobject{currentmarker}{}%
\end{pgfscope}%
\begin{pgfscope}%
\pgfsys@transformshift{3.933525in}{2.827832in}%
\pgfsys@useobject{currentmarker}{}%
\end{pgfscope}%
\begin{pgfscope}%
\pgfsys@transformshift{3.916151in}{2.510422in}%
\pgfsys@useobject{currentmarker}{}%
\end{pgfscope}%
\begin{pgfscope}%
\pgfsys@transformshift{3.897840in}{2.198605in}%
\pgfsys@useobject{currentmarker}{}%
\end{pgfscope}%
\begin{pgfscope}%
\pgfsys@transformshift{3.878823in}{2.055180in}%
\pgfsys@useobject{currentmarker}{}%
\end{pgfscope}%
\begin{pgfscope}%
\pgfsys@transformshift{3.854875in}{1.980113in}%
\pgfsys@useobject{currentmarker}{}%
\end{pgfscope}%
\begin{pgfscope}%
\pgfsys@transformshift{3.838912in}{1.980894in}%
\pgfsys@useobject{currentmarker}{}%
\end{pgfscope}%
\begin{pgfscope}%
\pgfsys@transformshift{3.819660in}{2.443570in}%
\pgfsys@useobject{currentmarker}{}%
\end{pgfscope}%
\begin{pgfscope}%
\pgfsys@transformshift{3.801347in}{2.162045in}%
\pgfsys@useobject{currentmarker}{}%
\end{pgfscope}%
\begin{pgfscope}%
\pgfsys@transformshift{3.782565in}{2.035278in}%
\pgfsys@useobject{currentmarker}{}%
\end{pgfscope}%
\begin{pgfscope}%
\pgfsys@transformshift{3.765428in}{1.983740in}%
\pgfsys@useobject{currentmarker}{}%
\end{pgfscope}%
\begin{pgfscope}%
\pgfsys@transformshift{3.743359in}{1.993586in}%
\pgfsys@useobject{currentmarker}{}%
\end{pgfscope}%
\begin{pgfscope}%
\pgfsys@transformshift{3.725517in}{2.064112in}%
\pgfsys@useobject{currentmarker}{}%
\end{pgfscope}%
\begin{pgfscope}%
\pgfsys@transformshift{3.705091in}{2.266710in}%
\pgfsys@useobject{currentmarker}{}%
\end{pgfscope}%
\begin{pgfscope}%
\pgfsys@transformshift{3.685840in}{2.627871in}%
\pgfsys@useobject{currentmarker}{}%
\end{pgfscope}%
\begin{pgfscope}%
\pgfsys@transformshift{3.666589in}{2.830312in}%
\pgfsys@useobject{currentmarker}{}%
\end{pgfscope}%
\begin{pgfscope}%
\pgfsys@transformshift{3.648275in}{2.789944in}%
\pgfsys@useobject{currentmarker}{}%
\end{pgfscope}%
\begin{pgfscope}%
\pgfsys@transformshift{3.629259in}{2.481476in}%
\pgfsys@useobject{currentmarker}{}%
\end{pgfscope}%
\begin{pgfscope}%
\pgfsys@transformshift{3.607190in}{2.161751in}%
\pgfsys@useobject{currentmarker}{}%
\end{pgfscope}%
\begin{pgfscope}%
\pgfsys@transformshift{3.591696in}{2.051618in}%
\pgfsys@useobject{currentmarker}{}%
\end{pgfscope}%
\begin{pgfscope}%
\pgfsys@transformshift{3.569393in}{1.984062in}%
\pgfsys@useobject{currentmarker}{}%
\end{pgfscope}%
\begin{pgfscope}%
\pgfsys@transformshift{3.551079in}{1.982660in}%
\pgfsys@useobject{currentmarker}{}%
\end{pgfscope}%
\begin{pgfscope}%
\pgfsys@transformshift{3.529245in}{2.031684in}%
\pgfsys@useobject{currentmarker}{}%
\end{pgfscope}%
\begin{pgfscope}%
\pgfsys@transformshift{3.514220in}{2.125445in}%
\pgfsys@useobject{currentmarker}{}%
\end{pgfscope}%
\begin{pgfscope}%
\pgfsys@transformshift{3.496143in}{2.319920in}%
\pgfsys@useobject{currentmarker}{}%
\end{pgfscope}%
\begin{pgfscope}%
\pgfsys@transformshift{3.474309in}{2.617359in}%
\pgfsys@useobject{currentmarker}{}%
\end{pgfscope}%
\begin{pgfscope}%
\pgfsys@transformshift{3.455292in}{2.831630in}%
\pgfsys@useobject{currentmarker}{}%
\end{pgfscope}%
\begin{pgfscope}%
\pgfsys@transformshift{3.436510in}{2.751040in}%
\pgfsys@useobject{currentmarker}{}%
\end{pgfscope}%
\begin{pgfscope}%
\pgfsys@transformshift{3.417964in}{2.556225in}%
\pgfsys@useobject{currentmarker}{}%
\end{pgfscope}%
\begin{pgfscope}%
\pgfsys@transformshift{3.398713in}{2.245132in}%
\pgfsys@useobject{currentmarker}{}%
\end{pgfscope}%
\begin{pgfscope}%
\pgfsys@transformshift{3.376644in}{2.063117in}%
\pgfsys@useobject{currentmarker}{}%
\end{pgfscope}%
\begin{pgfscope}%
\pgfsys@transformshift{3.358802in}{1.992541in}%
\pgfsys@useobject{currentmarker}{}%
\end{pgfscope}%
\begin{pgfscope}%
\pgfsys@transformshift{3.340254in}{1.970462in}%
\pgfsys@useobject{currentmarker}{}%
\end{pgfscope}%
\begin{pgfscope}%
\pgfsys@transformshift{3.321472in}{2.000352in}%
\pgfsys@useobject{currentmarker}{}%
\end{pgfscope}%
\begin{pgfscope}%
\pgfsys@transformshift{3.302455in}{2.075039in}%
\pgfsys@useobject{currentmarker}{}%
\end{pgfscope}%
\begin{pgfscope}%
\pgfsys@transformshift{3.278040in}{2.285190in}%
\pgfsys@useobject{currentmarker}{}%
\end{pgfscope}%
\begin{pgfscope}%
\pgfsys@transformshift{3.265127in}{2.505892in}%
\pgfsys@useobject{currentmarker}{}%
\end{pgfscope}%
\begin{pgfscope}%
\pgfsys@transformshift{3.243763in}{2.270877in}%
\pgfsys@useobject{currentmarker}{}%
\end{pgfscope}%
\begin{pgfscope}%
\pgfsys@transformshift{3.226390in}{2.546817in}%
\pgfsys@useobject{currentmarker}{}%
\end{pgfscope}%
\begin{pgfscope}%
\pgfsys@transformshift{3.203616in}{2.821476in}%
\pgfsys@useobject{currentmarker}{}%
\end{pgfscope}%
\begin{pgfscope}%
\pgfsys@transformshift{3.184836in}{2.744370in}%
\pgfsys@useobject{currentmarker}{}%
\end{pgfscope}%
\begin{pgfscope}%
\pgfsys@transformshift{3.169105in}{2.493152in}%
\pgfsys@useobject{currentmarker}{}%
\end{pgfscope}%
\begin{pgfscope}%
\pgfsys@transformshift{3.147740in}{2.186596in}%
\pgfsys@useobject{currentmarker}{}%
\end{pgfscope}%
\begin{pgfscope}%
\pgfsys@transformshift{3.127316in}{2.045639in}%
\pgfsys@useobject{currentmarker}{}%
\end{pgfscope}%
\begin{pgfscope}%
\pgfsys@transformshift{3.109003in}{1.987203in}%
\pgfsys@useobject{currentmarker}{}%
\end{pgfscope}%
\begin{pgfscope}%
\pgfsys@transformshift{3.091161in}{1.970730in}%
\pgfsys@useobject{currentmarker}{}%
\end{pgfscope}%
\begin{pgfscope}%
\pgfsys@transformshift{3.071440in}{2.005394in}%
\pgfsys@useobject{currentmarker}{}%
\end{pgfscope}%
\begin{pgfscope}%
\pgfsys@transformshift{3.051249in}{2.073606in}%
\pgfsys@useobject{currentmarker}{}%
\end{pgfscope}%
\begin{pgfscope}%
\pgfsys@transformshift{3.032702in}{2.243484in}%
\pgfsys@useobject{currentmarker}{}%
\end{pgfscope}%
\begin{pgfscope}%
\pgfsys@transformshift{3.013921in}{2.556463in}%
\pgfsys@useobject{currentmarker}{}%
\end{pgfscope}%
\begin{pgfscope}%
\pgfsys@transformshift{2.996782in}{2.801882in}%
\pgfsys@useobject{currentmarker}{}%
\end{pgfscope}%
\begin{pgfscope}%
\pgfsys@transformshift{2.978000in}{2.795595in}%
\pgfsys@useobject{currentmarker}{}%
\end{pgfscope}%
\begin{pgfscope}%
\pgfsys@transformshift{2.956871in}{2.579310in}%
\pgfsys@useobject{currentmarker}{}%
\end{pgfscope}%
\begin{pgfscope}%
\pgfsys@transformshift{2.936446in}{2.243688in}%
\pgfsys@useobject{currentmarker}{}%
\end{pgfscope}%
\begin{pgfscope}%
\pgfsys@transformshift{2.917194in}{2.100158in}%
\pgfsys@useobject{currentmarker}{}%
\end{pgfscope}%
\begin{pgfscope}%
\pgfsys@transformshift{2.900761in}{2.024006in}%
\pgfsys@useobject{currentmarker}{}%
\end{pgfscope}%
\begin{pgfscope}%
\pgfsys@transformshift{2.878457in}{1.974832in}%
\pgfsys@useobject{currentmarker}{}%
\end{pgfscope}%
\begin{pgfscope}%
\pgfsys@transformshift{2.856858in}{1.976529in}%
\pgfsys@useobject{currentmarker}{}%
\end{pgfscope}%
\begin{pgfscope}%
\pgfsys@transformshift{2.838781in}{2.015470in}%
\pgfsys@useobject{currentmarker}{}%
\end{pgfscope}%
\begin{pgfscope}%
\pgfsys@transformshift{2.819999in}{2.109967in}%
\pgfsys@useobject{currentmarker}{}%
\end{pgfscope}%
\begin{pgfscope}%
\pgfsys@transformshift{2.800513in}{2.305193in}%
\pgfsys@useobject{currentmarker}{}%
\end{pgfscope}%
\begin{pgfscope}%
\pgfsys@transformshift{2.785722in}{2.581820in}%
\pgfsys@useobject{currentmarker}{}%
\end{pgfscope}%
\begin{pgfscope}%
\pgfsys@transformshift{2.765062in}{2.800955in}%
\pgfsys@useobject{currentmarker}{}%
\end{pgfscope}%
\begin{pgfscope}%
\pgfsys@transformshift{2.744871in}{2.802044in}%
\pgfsys@useobject{currentmarker}{}%
\end{pgfscope}%
\begin{pgfscope}%
\pgfsys@transformshift{2.723742in}{2.556812in}%
\pgfsys@useobject{currentmarker}{}%
\end{pgfscope}%
\begin{pgfscope}%
\pgfsys@transformshift{2.701674in}{2.251908in}%
\pgfsys@useobject{currentmarker}{}%
\end{pgfscope}%
\begin{pgfscope}%
\pgfsys@transformshift{2.686883in}{2.139102in}%
\pgfsys@useobject{currentmarker}{}%
\end{pgfscope}%
\begin{pgfscope}%
\pgfsys@transformshift{2.668101in}{2.039367in}%
\pgfsys@useobject{currentmarker}{}%
\end{pgfscope}%
\begin{pgfscope}%
\pgfsys@transformshift{2.649084in}{2.001237in}%
\pgfsys@useobject{currentmarker}{}%
\end{pgfscope}%
\begin{pgfscope}%
\pgfsys@transformshift{2.627721in}{1.969449in}%
\pgfsys@useobject{currentmarker}{}%
\end{pgfscope}%
\begin{pgfscope}%
\pgfsys@transformshift{2.610582in}{1.982741in}%
\pgfsys@useobject{currentmarker}{}%
\end{pgfscope}%
\begin{pgfscope}%
\pgfsys@transformshift{2.590391in}{2.033659in}%
\pgfsys@useobject{currentmarker}{}%
\end{pgfscope}%
\begin{pgfscope}%
\pgfsys@transformshift{2.569027in}{2.105460in}%
\pgfsys@useobject{currentmarker}{}%
\end{pgfscope}%
\begin{pgfscope}%
\pgfsys@transformshift{2.550480in}{2.321593in}%
\pgfsys@useobject{currentmarker}{}%
\end{pgfscope}%
\begin{pgfscope}%
\pgfsys@transformshift{2.532168in}{2.611958in}%
\pgfsys@useobject{currentmarker}{}%
\end{pgfscope}%
\begin{pgfscope}%
\pgfsys@transformshift{2.515734in}{2.808689in}%
\pgfsys@useobject{currentmarker}{}%
\end{pgfscope}%
\begin{pgfscope}%
\pgfsys@transformshift{2.494604in}{2.733188in}%
\pgfsys@useobject{currentmarker}{}%
\end{pgfscope}%
\begin{pgfscope}%
\pgfsys@transformshift{2.475587in}{2.457855in}%
\pgfsys@useobject{currentmarker}{}%
\end{pgfscope}%
\begin{pgfscope}%
\pgfsys@transformshift{2.455161in}{2.163590in}%
\pgfsys@useobject{currentmarker}{}%
\end{pgfscope}%
\begin{pgfscope}%
\pgfsys@transformshift{2.437319in}{2.060767in}%
\pgfsys@useobject{currentmarker}{}%
\end{pgfscope}%
\begin{pgfscope}%
\pgfsys@transformshift{2.420885in}{2.003552in}%
\pgfsys@useobject{currentmarker}{}%
\end{pgfscope}%
\begin{pgfscope}%
\pgfsys@transformshift{2.398582in}{1.969311in}%
\pgfsys@useobject{currentmarker}{}%
\end{pgfscope}%
\begin{pgfscope}%
\pgfsys@transformshift{2.376982in}{1.988073in}%
\pgfsys@useobject{currentmarker}{}%
\end{pgfscope}%
\begin{pgfscope}%
\pgfsys@transformshift{2.358436in}{2.042708in}%
\pgfsys@useobject{currentmarker}{}%
\end{pgfscope}%
\begin{pgfscope}%
\pgfsys@transformshift{2.342941in}{2.115908in}%
\pgfsys@useobject{currentmarker}{}%
\end{pgfscope}%
\begin{pgfscope}%
\pgfsys@transformshift{2.321343in}{2.366259in}%
\pgfsys@useobject{currentmarker}{}%
\end{pgfscope}%
\begin{pgfscope}%
\pgfsys@transformshift{2.303264in}{2.647824in}%
\pgfsys@useobject{currentmarker}{}%
\end{pgfscope}%
\begin{pgfscope}%
\pgfsys@transformshift{2.280961in}{2.816072in}%
\pgfsys@useobject{currentmarker}{}%
\end{pgfscope}%
\begin{pgfscope}%
\pgfsys@transformshift{2.263118in}{2.740929in}%
\pgfsys@useobject{currentmarker}{}%
\end{pgfscope}%
\begin{pgfscope}%
\pgfsys@transformshift{2.244102in}{2.501502in}%
\pgfsys@useobject{currentmarker}{}%
\end{pgfscope}%
\begin{pgfscope}%
\pgfsys@transformshift{2.225556in}{2.263090in}%
\pgfsys@useobject{currentmarker}{}%
\end{pgfscope}%
\begin{pgfscope}%
\pgfsys@transformshift{2.207477in}{2.109227in}%
\pgfsys@useobject{currentmarker}{}%
\end{pgfscope}%
\begin{pgfscope}%
\pgfsys@transformshift{2.187991in}{2.026897in}%
\pgfsys@useobject{currentmarker}{}%
\end{pgfscope}%
\begin{pgfscope}%
\pgfsys@transformshift{2.167566in}{1.986821in}%
\pgfsys@useobject{currentmarker}{}%
\end{pgfscope}%
\begin{pgfscope}%
\pgfsys@transformshift{2.148314in}{1.970873in}%
\pgfsys@useobject{currentmarker}{}%
\end{pgfscope}%
\begin{pgfscope}%
\pgfsys@transformshift{2.126480in}{2.004524in}%
\pgfsys@useobject{currentmarker}{}%
\end{pgfscope}%
\begin{pgfscope}%
\pgfsys@transformshift{2.110986in}{2.050319in}%
\pgfsys@useobject{currentmarker}{}%
\end{pgfscope}%
\begin{pgfscope}%
\pgfsys@transformshift{2.089387in}{2.102254in}%
\pgfsys@useobject{currentmarker}{}%
\end{pgfscope}%
\begin{pgfscope}%
\pgfsys@transformshift{2.070841in}{2.316495in}%
\pgfsys@useobject{currentmarker}{}%
\end{pgfscope}%
\begin{pgfscope}%
\pgfsys@transformshift{2.052762in}{2.574576in}%
\pgfsys@useobject{currentmarker}{}%
\end{pgfscope}%
\begin{pgfscope}%
\pgfsys@transformshift{2.034216in}{2.694536in}%
\pgfsys@useobject{currentmarker}{}%
\end{pgfscope}%
\begin{pgfscope}%
\pgfsys@transformshift{2.015199in}{2.826358in}%
\pgfsys@useobject{currentmarker}{}%
\end{pgfscope}%
\begin{pgfscope}%
\pgfsys@transformshift{1.993834in}{2.704117in}%
\pgfsys@useobject{currentmarker}{}%
\end{pgfscope}%
\begin{pgfscope}%
\pgfsys@transformshift{1.975288in}{2.446625in}%
\pgfsys@useobject{currentmarker}{}%
\end{pgfscope}%
\begin{pgfscope}%
\pgfsys@transformshift{1.957680in}{2.200917in}%
\pgfsys@useobject{currentmarker}{}%
\end{pgfscope}%
\begin{pgfscope}%
\pgfsys@transformshift{1.937489in}{2.074739in}%
\pgfsys@useobject{currentmarker}{}%
\end{pgfscope}%
\begin{pgfscope}%
\pgfsys@transformshift{1.920350in}{2.043625in}%
\pgfsys@useobject{currentmarker}{}%
\end{pgfscope}%
\begin{pgfscope}%
\pgfsys@transformshift{1.898518in}{1.985347in}%
\pgfsys@useobject{currentmarker}{}%
\end{pgfscope}%
\begin{pgfscope}%
\pgfsys@transformshift{1.879735in}{1.972332in}%
\pgfsys@useobject{currentmarker}{}%
\end{pgfscope}%
\begin{pgfscope}%
\pgfsys@transformshift{1.858136in}{2.004343in}%
\pgfsys@useobject{currentmarker}{}%
\end{pgfscope}%
\begin{pgfscope}%
\pgfsys@transformshift{1.839590in}{2.073092in}%
\pgfsys@useobject{currentmarker}{}%
\end{pgfscope}%
\begin{pgfscope}%
\pgfsys@transformshift{1.821042in}{2.210424in}%
\pgfsys@useobject{currentmarker}{}%
\end{pgfscope}%
\begin{pgfscope}%
\pgfsys@transformshift{1.801791in}{2.458381in}%
\pgfsys@useobject{currentmarker}{}%
\end{pgfscope}%
\begin{pgfscope}%
\pgfsys@transformshift{1.783008in}{2.687312in}%
\pgfsys@useobject{currentmarker}{}%
\end{pgfscope}%
\begin{pgfscope}%
\pgfsys@transformshift{1.764463in}{2.828475in}%
\pgfsys@useobject{currentmarker}{}%
\end{pgfscope}%
\begin{pgfscope}%
\pgfsys@transformshift{1.743568in}{2.788442in}%
\pgfsys@useobject{currentmarker}{}%
\end{pgfscope}%
\begin{pgfscope}%
\pgfsys@transformshift{1.727603in}{2.586373in}%
\pgfsys@useobject{currentmarker}{}%
\end{pgfscope}%
\begin{pgfscope}%
\pgfsys@transformshift{1.710230in}{2.321823in}%
\pgfsys@useobject{currentmarker}{}%
\end{pgfscope}%
\begin{pgfscope}%
\pgfsys@transformshift{1.687221in}{2.181589in}%
\pgfsys@useobject{currentmarker}{}%
\end{pgfscope}%
\begin{pgfscope}%
\pgfsys@transformshift{1.669848in}{2.089817in}%
\pgfsys@useobject{currentmarker}{}%
\end{pgfscope}%
\begin{pgfscope}%
\pgfsys@transformshift{1.648484in}{2.016546in}%
\pgfsys@useobject{currentmarker}{}%
\end{pgfscope}%
\begin{pgfscope}%
\pgfsys@transformshift{1.629233in}{1.986455in}%
\pgfsys@useobject{currentmarker}{}%
\end{pgfscope}%
\begin{pgfscope}%
\pgfsys@transformshift{1.610920in}{1.976938in}%
\pgfsys@useobject{currentmarker}{}%
\end{pgfscope}%
\begin{pgfscope}%
\pgfsys@transformshift{1.592374in}{2.000312in}%
\pgfsys@useobject{currentmarker}{}%
\end{pgfscope}%
\begin{pgfscope}%
\pgfsys@transformshift{1.571009in}{2.072652in}%
\pgfsys@useobject{currentmarker}{}%
\end{pgfscope}%
\begin{pgfscope}%
\pgfsys@transformshift{1.554106in}{2.190538in}%
\pgfsys@useobject{currentmarker}{}%
\end{pgfscope}%
\begin{pgfscope}%
\pgfsys@transformshift{1.530629in}{2.163084in}%
\pgfsys@useobject{currentmarker}{}%
\end{pgfscope}%
\begin{pgfscope}%
\pgfsys@transformshift{1.511612in}{2.374498in}%
\pgfsys@useobject{currentmarker}{}%
\end{pgfscope}%
\begin{pgfscope}%
\pgfsys@transformshift{1.494004in}{2.649821in}%
\pgfsys@useobject{currentmarker}{}%
\end{pgfscope}%
\begin{pgfscope}%
\pgfsys@transformshift{1.478510in}{2.806264in}%
\pgfsys@useobject{currentmarker}{}%
\end{pgfscope}%
\begin{pgfscope}%
\pgfsys@transformshift{1.459022in}{2.851778in}%
\pgfsys@useobject{currentmarker}{}%
\end{pgfscope}%
\begin{pgfscope}%
\pgfsys@transformshift{1.438128in}{2.675255in}%
\pgfsys@useobject{currentmarker}{}%
\end{pgfscope}%
\begin{pgfscope}%
\pgfsys@transformshift{1.417702in}{2.373180in}%
\pgfsys@useobject{currentmarker}{}%
\end{pgfscope}%
\begin{pgfscope}%
\pgfsys@transformshift{1.395165in}{2.162051in}%
\pgfsys@useobject{currentmarker}{}%
\end{pgfscope}%
\begin{pgfscope}%
\pgfsys@transformshift{1.378497in}{2.103200in}%
\pgfsys@useobject{currentmarker}{}%
\end{pgfscope}%
\begin{pgfscope}%
\pgfsys@transformshift{1.359949in}{2.045695in}%
\pgfsys@useobject{currentmarker}{}%
\end{pgfscope}%
\begin{pgfscope}%
\pgfsys@transformshift{1.339758in}{1.988781in}%
\pgfsys@useobject{currentmarker}{}%
\end{pgfscope}%
\begin{pgfscope}%
\pgfsys@transformshift{1.321681in}{1.977786in}%
\pgfsys@useobject{currentmarker}{}%
\end{pgfscope}%
\begin{pgfscope}%
\pgfsys@transformshift{1.302899in}{1.998672in}%
\pgfsys@useobject{currentmarker}{}%
\end{pgfscope}%
\begin{pgfscope}%
\pgfsys@transformshift{1.283884in}{2.055182in}%
\pgfsys@useobject{currentmarker}{}%
\end{pgfscope}%
\begin{pgfscope}%
\pgfsys@transformshift{1.262753in}{2.155325in}%
\pgfsys@useobject{currentmarker}{}%
\end{pgfscope}%
\begin{pgfscope}%
\pgfsys@transformshift{1.244442in}{2.320586in}%
\pgfsys@useobject{currentmarker}{}%
\end{pgfscope}%
\begin{pgfscope}%
\pgfsys@transformshift{1.226363in}{2.546722in}%
\pgfsys@useobject{currentmarker}{}%
\end{pgfscope}%
\begin{pgfscope}%
\pgfsys@transformshift{1.207582in}{2.694421in}%
\pgfsys@useobject{currentmarker}{}%
\end{pgfscope}%
\begin{pgfscope}%
\pgfsys@transformshift{1.187860in}{2.861189in}%
\pgfsys@useobject{currentmarker}{}%
\end{pgfscope}%
\begin{pgfscope}%
\pgfsys@transformshift{1.167435in}{2.866539in}%
\pgfsys@useobject{currentmarker}{}%
\end{pgfscope}%
\begin{pgfscope}%
\pgfsys@transformshift{1.148889in}{2.702284in}%
\pgfsys@useobject{currentmarker}{}%
\end{pgfscope}%
\begin{pgfscope}%
\pgfsys@transformshift{1.129403in}{2.412735in}%
\pgfsys@useobject{currentmarker}{}%
\end{pgfscope}%
\begin{pgfscope}%
\pgfsys@transformshift{1.110855in}{2.211894in}%
\pgfsys@useobject{currentmarker}{}%
\end{pgfscope}%
\begin{pgfscope}%
\pgfsys@transformshift{1.090901in}{2.099352in}%
\pgfsys@useobject{currentmarker}{}%
\end{pgfscope}%
\begin{pgfscope}%
\pgfsys@transformshift{1.073056in}{2.050660in}%
\pgfsys@useobject{currentmarker}{}%
\end{pgfscope}%
\begin{pgfscope}%
\pgfsys@transformshift{1.052867in}{1.994758in}%
\pgfsys@useobject{currentmarker}{}%
\end{pgfscope}%
\begin{pgfscope}%
\pgfsys@transformshift{1.035259in}{1.981884in}%
\pgfsys@useobject{currentmarker}{}%
\end{pgfscope}%
\begin{pgfscope}%
\pgfsys@transformshift{1.015537in}{2.018306in}%
\pgfsys@useobject{currentmarker}{}%
\end{pgfscope}%
\begin{pgfscope}%
\pgfsys@transformshift{0.994877in}{2.078711in}%
\pgfsys@useobject{currentmarker}{}%
\end{pgfscope}%
\begin{pgfscope}%
\pgfsys@transformshift{0.976097in}{2.265021in}%
\pgfsys@useobject{currentmarker}{}%
\end{pgfscope}%
\begin{pgfscope}%
\pgfsys@transformshift{0.956846in}{2.113302in}%
\pgfsys@useobject{currentmarker}{}%
\end{pgfscope}%
\begin{pgfscope}%
\pgfsys@transformshift{0.939707in}{2.038603in}%
\pgfsys@useobject{currentmarker}{}%
\end{pgfscope}%
\begin{pgfscope}%
\pgfsys@transformshift{0.917169in}{1.984115in}%
\pgfsys@useobject{currentmarker}{}%
\end{pgfscope}%
\begin{pgfscope}%
\pgfsys@transformshift{0.900499in}{1.991998in}%
\pgfsys@useobject{currentmarker}{}%
\end{pgfscope}%
\begin{pgfscope}%
\pgfsys@transformshift{0.879604in}{2.059333in}%
\pgfsys@useobject{currentmarker}{}%
\end{pgfscope}%
\begin{pgfscope}%
\pgfsys@transformshift{0.860119in}{2.189675in}%
\pgfsys@useobject{currentmarker}{}%
\end{pgfscope}%
\begin{pgfscope}%
\pgfsys@transformshift{0.843451in}{2.310101in}%
\pgfsys@useobject{currentmarker}{}%
\end{pgfscope}%
\begin{pgfscope}%
\pgfsys@transformshift{0.823025in}{2.595901in}%
\pgfsys@useobject{currentmarker}{}%
\end{pgfscope}%
\begin{pgfscope}%
\pgfsys@transformshift{0.802365in}{2.815133in}%
\pgfsys@useobject{currentmarker}{}%
\end{pgfscope}%
\begin{pgfscope}%
\pgfsys@transformshift{0.783348in}{2.916420in}%
\pgfsys@useobject{currentmarker}{}%
\end{pgfscope}%
\begin{pgfscope}%
\pgfsys@transformshift{0.762688in}{2.761078in}%
\pgfsys@useobject{currentmarker}{}%
\end{pgfscope}%
\begin{pgfscope}%
\pgfsys@transformshift{0.746489in}{2.495130in}%
\pgfsys@useobject{currentmarker}{}%
\end{pgfscope}%
\begin{pgfscope}%
\pgfsys@transformshift{0.726767in}{2.223490in}%
\pgfsys@useobject{currentmarker}{}%
\end{pgfscope}%
\begin{pgfscope}%
\pgfsys@transformshift{0.706344in}{2.083442in}%
\pgfsys@useobject{currentmarker}{}%
\end{pgfscope}%
\begin{pgfscope}%
\pgfsys@transformshift{0.688265in}{2.018157in}%
\pgfsys@useobject{currentmarker}{}%
\end{pgfscope}%
\begin{pgfscope}%
\pgfsys@transformshift{0.665258in}{1.987407in}%
\pgfsys@useobject{currentmarker}{}%
\end{pgfscope}%
\begin{pgfscope}%
\pgfsys@transformshift{0.649528in}{2.035791in}%
\pgfsys@useobject{currentmarker}{}%
\end{pgfscope}%
\begin{pgfscope}%
\pgfsys@transformshift{0.650231in}{2.038546in}%
\pgfsys@useobject{currentmarker}{}%
\end{pgfscope}%
\begin{pgfscope}%
\pgfsys@transformshift{0.655397in}{2.063862in}%
\pgfsys@useobject{currentmarker}{}%
\end{pgfscope}%
\begin{pgfscope}%
\pgfsys@transformshift{0.675823in}{2.247009in}%
\pgfsys@useobject{currentmarker}{}%
\end{pgfscope}%
\begin{pgfscope}%
\pgfsys@transformshift{0.695308in}{2.589218in}%
\pgfsys@useobject{currentmarker}{}%
\end{pgfscope}%
\begin{pgfscope}%
\pgfsys@transformshift{0.711508in}{2.872678in}%
\pgfsys@useobject{currentmarker}{}%
\end{pgfscope}%
\begin{pgfscope}%
\pgfsys@transformshift{0.733811in}{2.832193in}%
\pgfsys@useobject{currentmarker}{}%
\end{pgfscope}%
\begin{pgfscope}%
\pgfsys@transformshift{0.752124in}{2.484941in}%
\pgfsys@useobject{currentmarker}{}%
\end{pgfscope}%
\begin{pgfscope}%
\pgfsys@transformshift{0.770670in}{2.241265in}%
\pgfsys@useobject{currentmarker}{}%
\end{pgfscope}%
\begin{pgfscope}%
\pgfsys@transformshift{0.789452in}{2.060542in}%
\pgfsys@useobject{currentmarker}{}%
\end{pgfscope}%
\begin{pgfscope}%
\pgfsys@transformshift{0.808938in}{1.989034in}%
\pgfsys@useobject{currentmarker}{}%
\end{pgfscope}%
\begin{pgfscope}%
\pgfsys@transformshift{0.827720in}{1.999512in}%
\pgfsys@useobject{currentmarker}{}%
\end{pgfscope}%
\begin{pgfscope}%
\pgfsys@transformshift{0.849789in}{2.110131in}%
\pgfsys@useobject{currentmarker}{}%
\end{pgfscope}%
\begin{pgfscope}%
\pgfsys@transformshift{0.868335in}{2.321608in}%
\pgfsys@useobject{currentmarker}{}%
\end{pgfscope}%
\begin{pgfscope}%
\pgfsys@transformshift{0.887117in}{2.704670in}%
\pgfsys@useobject{currentmarker}{}%
\end{pgfscope}%
\begin{pgfscope}%
\pgfsys@transformshift{0.906134in}{2.893796in}%
\pgfsys@useobject{currentmarker}{}%
\end{pgfscope}%
\begin{pgfscope}%
\pgfsys@transformshift{0.925150in}{2.729133in}%
\pgfsys@useobject{currentmarker}{}%
\end{pgfscope}%
\begin{pgfscope}%
\pgfsys@transformshift{0.946279in}{2.290744in}%
\pgfsys@useobject{currentmarker}{}%
\end{pgfscope}%
\begin{pgfscope}%
\pgfsys@transformshift{0.962949in}{2.107550in}%
\pgfsys@useobject{currentmarker}{}%
\end{pgfscope}%
\begin{pgfscope}%
\pgfsys@transformshift{0.982904in}{2.009095in}%
\pgfsys@useobject{currentmarker}{}%
\end{pgfscope}%
\begin{pgfscope}%
\pgfsys@transformshift{1.000278in}{1.979865in}%
\pgfsys@useobject{currentmarker}{}%
\end{pgfscope}%
\begin{pgfscope}%
\pgfsys@transformshift{1.025867in}{2.061063in}%
\pgfsys@useobject{currentmarker}{}%
\end{pgfscope}%
\begin{pgfscope}%
\pgfsys@transformshift{1.042303in}{2.164098in}%
\pgfsys@useobject{currentmarker}{}%
\end{pgfscope}%
\begin{pgfscope}%
\pgfsys@transformshift{1.060380in}{2.437889in}%
\pgfsys@useobject{currentmarker}{}%
\end{pgfscope}%
\begin{pgfscope}%
\pgfsys@transformshift{1.079631in}{2.800132in}%
\pgfsys@useobject{currentmarker}{}%
\end{pgfscope}%
\begin{pgfscope}%
\pgfsys@transformshift{1.097942in}{2.851846in}%
\pgfsys@useobject{currentmarker}{}%
\end{pgfscope}%
\begin{pgfscope}%
\pgfsys@transformshift{1.116256in}{2.562058in}%
\pgfsys@useobject{currentmarker}{}%
\end{pgfscope}%
\begin{pgfscope}%
\pgfsys@transformshift{1.140437in}{2.173848in}%
\pgfsys@useobject{currentmarker}{}%
\end{pgfscope}%
\begin{pgfscope}%
\pgfsys@transformshift{1.154053in}{2.055184in}%
\pgfsys@useobject{currentmarker}{}%
\end{pgfscope}%
\begin{pgfscope}%
\pgfsys@transformshift{1.172835in}{1.984582in}%
\pgfsys@useobject{currentmarker}{}%
\end{pgfscope}%
\begin{pgfscope}%
\pgfsys@transformshift{1.194904in}{1.986647in}%
\pgfsys@useobject{currentmarker}{}%
\end{pgfscope}%
\begin{pgfscope}%
\pgfsys@transformshift{1.212512in}{2.049200in}%
\pgfsys@useobject{currentmarker}{}%
\end{pgfscope}%
\begin{pgfscope}%
\pgfsys@transformshift{1.232937in}{2.177518in}%
\pgfsys@useobject{currentmarker}{}%
\end{pgfscope}%
\begin{pgfscope}%
\pgfsys@transformshift{1.253363in}{2.432512in}%
\pgfsys@useobject{currentmarker}{}%
\end{pgfscope}%
\begin{pgfscope}%
\pgfsys@transformshift{1.271909in}{2.737024in}%
\pgfsys@useobject{currentmarker}{}%
\end{pgfscope}%
\begin{pgfscope}%
\pgfsys@transformshift{1.288108in}{2.853629in}%
\pgfsys@useobject{currentmarker}{}%
\end{pgfscope}%
\begin{pgfscope}%
\pgfsys@transformshift{1.309473in}{2.614610in}%
\pgfsys@useobject{currentmarker}{}%
\end{pgfscope}%
\begin{pgfscope}%
\pgfsys@transformshift{1.327550in}{2.277830in}%
\pgfsys@useobject{currentmarker}{}%
\end{pgfscope}%
\begin{pgfscope}%
\pgfsys@transformshift{1.349150in}{2.069731in}%
\pgfsys@useobject{currentmarker}{}%
\end{pgfscope}%
\begin{pgfscope}%
\pgfsys@transformshift{1.366523in}{1.996119in}%
\pgfsys@useobject{currentmarker}{}%
\end{pgfscope}%
\begin{pgfscope}%
\pgfsys@transformshift{1.387183in}{1.974257in}%
\pgfsys@useobject{currentmarker}{}%
\end{pgfscope}%
\begin{pgfscope}%
\pgfsys@transformshift{1.405964in}{2.012231in}%
\pgfsys@useobject{currentmarker}{}%
\end{pgfscope}%
\begin{pgfscope}%
\pgfsys@transformshift{1.423103in}{2.092855in}%
\pgfsys@useobject{currentmarker}{}%
\end{pgfscope}%
\begin{pgfscope}%
\pgfsys@transformshift{1.444232in}{2.312625in}%
\pgfsys@useobject{currentmarker}{}%
\end{pgfscope}%
\begin{pgfscope}%
\pgfsys@transformshift{1.462074in}{2.577581in}%
\pgfsys@useobject{currentmarker}{}%
\end{pgfscope}%
\begin{pgfscope}%
\pgfsys@transformshift{1.486491in}{2.840292in}%
\pgfsys@useobject{currentmarker}{}%
\end{pgfscope}%
\begin{pgfscope}%
\pgfsys@transformshift{1.501282in}{2.811321in}%
\pgfsys@useobject{currentmarker}{}%
\end{pgfscope}%
\begin{pgfscope}%
\pgfsys@transformshift{1.521473in}{2.784227in}%
\pgfsys@useobject{currentmarker}{}%
\end{pgfscope}%
\begin{pgfscope}%
\pgfsys@transformshift{1.539315in}{2.479755in}%
\pgfsys@useobject{currentmarker}{}%
\end{pgfscope}%
\begin{pgfscope}%
\pgfsys@transformshift{1.559270in}{2.166455in}%
\pgfsys@useobject{currentmarker}{}%
\end{pgfscope}%
\begin{pgfscope}%
\pgfsys@transformshift{1.578992in}{2.045618in}%
\pgfsys@useobject{currentmarker}{}%
\end{pgfscope}%
\begin{pgfscope}%
\pgfsys@transformshift{1.599418in}{1.982162in}%
\pgfsys@useobject{currentmarker}{}%
\end{pgfscope}%
\begin{pgfscope}%
\pgfsys@transformshift{1.617260in}{1.974434in}%
\pgfsys@useobject{currentmarker}{}%
\end{pgfscope}%
\begin{pgfscope}%
\pgfsys@transformshift{1.633928in}{2.005347in}%
\pgfsys@useobject{currentmarker}{}%
\end{pgfscope}%
\begin{pgfscope}%
\pgfsys@transformshift{1.655997in}{2.056631in}%
\pgfsys@useobject{currentmarker}{}%
\end{pgfscope}%
\begin{pgfscope}%
\pgfsys@transformshift{1.674074in}{2.151520in}%
\pgfsys@useobject{currentmarker}{}%
\end{pgfscope}%
\begin{pgfscope}%
\pgfsys@transformshift{1.693796in}{2.398793in}%
\pgfsys@useobject{currentmarker}{}%
\end{pgfscope}%
\begin{pgfscope}%
\pgfsys@transformshift{1.713751in}{2.387692in}%
\pgfsys@useobject{currentmarker}{}%
\end{pgfscope}%
\begin{pgfscope}%
\pgfsys@transformshift{1.729481in}{2.696380in}%
\pgfsys@useobject{currentmarker}{}%
\end{pgfscope}%
\begin{pgfscope}%
\pgfsys@transformshift{1.750141in}{2.828089in}%
\pgfsys@useobject{currentmarker}{}%
\end{pgfscope}%
\begin{pgfscope}%
\pgfsys@transformshift{1.770566in}{2.637079in}%
\pgfsys@useobject{currentmarker}{}%
\end{pgfscope}%
\begin{pgfscope}%
\pgfsys@transformshift{1.789112in}{2.300717in}%
\pgfsys@useobject{currentmarker}{}%
\end{pgfscope}%
\begin{pgfscope}%
\pgfsys@transformshift{1.809069in}{2.099114in}%
\pgfsys@useobject{currentmarker}{}%
\end{pgfscope}%
\begin{pgfscope}%
\pgfsys@transformshift{1.828789in}{2.086454in}%
\pgfsys@useobject{currentmarker}{}%
\end{pgfscope}%
\begin{pgfscope}%
\pgfsys@transformshift{1.845928in}{2.707679in}%
\pgfsys@useobject{currentmarker}{}%
\end{pgfscope}%
\begin{pgfscope}%
\pgfsys@transformshift{1.867057in}{2.815740in}%
\pgfsys@useobject{currentmarker}{}%
\end{pgfscope}%
\begin{pgfscope}%
\pgfsys@transformshift{1.888422in}{2.618060in}%
\pgfsys@useobject{currentmarker}{}%
\end{pgfscope}%
\begin{pgfscope}%
\pgfsys@transformshift{1.887717in}{2.399412in}%
\pgfsys@useobject{currentmarker}{}%
\end{pgfscope}%
\begin{pgfscope}%
\pgfsys@transformshift{1.906734in}{2.273315in}%
\pgfsys@useobject{currentmarker}{}%
\end{pgfscope}%
\begin{pgfscope}%
\pgfsys@transformshift{1.924107in}{2.080796in}%
\pgfsys@useobject{currentmarker}{}%
\end{pgfscope}%
\begin{pgfscope}%
\pgfsys@transformshift{1.942655in}{1.999890in}%
\pgfsys@useobject{currentmarker}{}%
\end{pgfscope}%
\begin{pgfscope}%
\pgfsys@transformshift{1.963078in}{1.969909in}%
\pgfsys@useobject{currentmarker}{}%
\end{pgfscope}%
\begin{pgfscope}%
\pgfsys@transformshift{1.982566in}{1.996097in}%
\pgfsys@useobject{currentmarker}{}%
\end{pgfscope}%
\begin{pgfscope}%
\pgfsys@transformshift{2.002286in}{2.073596in}%
\pgfsys@useobject{currentmarker}{}%
\end{pgfscope}%
\begin{pgfscope}%
\pgfsys@transformshift{2.019660in}{2.227794in}%
\pgfsys@useobject{currentmarker}{}%
\end{pgfscope}%
\begin{pgfscope}%
\pgfsys@transformshift{2.041023in}{2.596164in}%
\pgfsys@useobject{currentmarker}{}%
\end{pgfscope}%
\begin{pgfscope}%
\pgfsys@transformshift{2.058866in}{2.802483in}%
\pgfsys@useobject{currentmarker}{}%
\end{pgfscope}%
\begin{pgfscope}%
\pgfsys@transformshift{2.076944in}{2.774723in}%
\pgfsys@useobject{currentmarker}{}%
\end{pgfscope}%
\begin{pgfscope}%
\pgfsys@transformshift{2.097839in}{2.475990in}%
\pgfsys@useobject{currentmarker}{}%
\end{pgfscope}%
\begin{pgfscope}%
\pgfsys@transformshift{2.115681in}{2.186344in}%
\pgfsys@useobject{currentmarker}{}%
\end{pgfscope}%
\begin{pgfscope}%
\pgfsys@transformshift{2.139159in}{2.026044in}%
\pgfsys@useobject{currentmarker}{}%
\end{pgfscope}%
\begin{pgfscope}%
\pgfsys@transformshift{2.158175in}{1.977364in}%
\pgfsys@useobject{currentmarker}{}%
\end{pgfscope}%
\begin{pgfscope}%
\pgfsys@transformshift{2.175549in}{1.972430in}%
\pgfsys@useobject{currentmarker}{}%
\end{pgfscope}%
\begin{pgfscope}%
\pgfsys@transformshift{2.193157in}{1.994180in}%
\pgfsys@useobject{currentmarker}{}%
\end{pgfscope}%
\begin{pgfscope}%
\pgfsys@transformshift{2.211234in}{2.056100in}%
\pgfsys@useobject{currentmarker}{}%
\end{pgfscope}%
\begin{pgfscope}%
\pgfsys@transformshift{2.232597in}{2.223328in}%
\pgfsys@useobject{currentmarker}{}%
\end{pgfscope}%
\begin{pgfscope}%
\pgfsys@transformshift{2.250440in}{2.443118in}%
\pgfsys@useobject{currentmarker}{}%
\end{pgfscope}%
\begin{pgfscope}%
\pgfsys@transformshift{2.272274in}{2.771712in}%
\pgfsys@useobject{currentmarker}{}%
\end{pgfscope}%
\begin{pgfscope}%
\pgfsys@transformshift{2.289413in}{2.817617in}%
\pgfsys@useobject{currentmarker}{}%
\end{pgfscope}%
\begin{pgfscope}%
\pgfsys@transformshift{2.307961in}{2.672627in}%
\pgfsys@useobject{currentmarker}{}%
\end{pgfscope}%
\begin{pgfscope}%
\pgfsys@transformshift{2.329090in}{2.281682in}%
\pgfsys@useobject{currentmarker}{}%
\end{pgfscope}%
\begin{pgfscope}%
\pgfsys@transformshift{2.347636in}{2.098864in}%
\pgfsys@useobject{currentmarker}{}%
\end{pgfscope}%
\begin{pgfscope}%
\pgfsys@transformshift{2.365009in}{2.015434in}%
\pgfsys@useobject{currentmarker}{}%
\end{pgfscope}%
\begin{pgfscope}%
\pgfsys@transformshift{2.386843in}{1.971066in}%
\pgfsys@useobject{currentmarker}{}%
\end{pgfscope}%
\begin{pgfscope}%
\pgfsys@transformshift{2.403748in}{1.978167in}%
\pgfsys@useobject{currentmarker}{}%
\end{pgfscope}%
\begin{pgfscope}%
\pgfsys@transformshift{2.424408in}{2.033067in}%
\pgfsys@useobject{currentmarker}{}%
\end{pgfscope}%
\begin{pgfscope}%
\pgfsys@transformshift{2.442954in}{2.138132in}%
\pgfsys@useobject{currentmarker}{}%
\end{pgfscope}%
\begin{pgfscope}%
\pgfsys@transformshift{2.462440in}{2.359820in}%
\pgfsys@useobject{currentmarker}{}%
\end{pgfscope}%
\begin{pgfscope}%
\pgfsys@transformshift{2.481456in}{2.563823in}%
\pgfsys@useobject{currentmarker}{}%
\end{pgfscope}%
\begin{pgfscope}%
\pgfsys@transformshift{2.499299in}{2.722888in}%
\pgfsys@useobject{currentmarker}{}%
\end{pgfscope}%
\begin{pgfscope}%
\pgfsys@transformshift{2.520430in}{2.811813in}%
\pgfsys@useobject{currentmarker}{}%
\end{pgfscope}%
\begin{pgfscope}%
\pgfsys@transformshift{2.539210in}{2.620168in}%
\pgfsys@useobject{currentmarker}{}%
\end{pgfscope}%
\begin{pgfscope}%
\pgfsys@transformshift{2.557289in}{2.323854in}%
\pgfsys@useobject{currentmarker}{}%
\end{pgfscope}%
\begin{pgfscope}%
\pgfsys@transformshift{2.577714in}{2.107917in}%
\pgfsys@useobject{currentmarker}{}%
\end{pgfscope}%
\begin{pgfscope}%
\pgfsys@transformshift{2.595791in}{2.015636in}%
\pgfsys@useobject{currentmarker}{}%
\end{pgfscope}%
\begin{pgfscope}%
\pgfsys@transformshift{2.617391in}{1.974643in}%
\pgfsys@useobject{currentmarker}{}%
\end{pgfscope}%
\begin{pgfscope}%
\pgfsys@transformshift{2.635233in}{1.975726in}%
\pgfsys@useobject{currentmarker}{}%
\end{pgfscope}%
\begin{pgfscope}%
\pgfsys@transformshift{2.653076in}{2.015123in}%
\pgfsys@useobject{currentmarker}{}%
\end{pgfscope}%
\begin{pgfscope}%
\pgfsys@transformshift{2.673736in}{2.102012in}%
\pgfsys@useobject{currentmarker}{}%
\end{pgfscope}%
\begin{pgfscope}%
\pgfsys@transformshift{2.691578in}{2.286152in}%
\pgfsys@useobject{currentmarker}{}%
\end{pgfscope}%
\begin{pgfscope}%
\pgfsys@transformshift{2.713647in}{2.607485in}%
\pgfsys@useobject{currentmarker}{}%
\end{pgfscope}%
\begin{pgfscope}%
\pgfsys@transformshift{2.731021in}{2.801580in}%
\pgfsys@useobject{currentmarker}{}%
\end{pgfscope}%
\begin{pgfscope}%
\pgfsys@transformshift{2.752149in}{2.773940in}%
\pgfsys@useobject{currentmarker}{}%
\end{pgfscope}%
\begin{pgfscope}%
\pgfsys@transformshift{2.770461in}{2.502122in}%
\pgfsys@useobject{currentmarker}{}%
\end{pgfscope}%
\begin{pgfscope}%
\pgfsys@transformshift{2.788305in}{2.232256in}%
\pgfsys@useobject{currentmarker}{}%
\end{pgfscope}%
\begin{pgfscope}%
\pgfsys@transformshift{2.807556in}{2.078357in}%
\pgfsys@useobject{currentmarker}{}%
\end{pgfscope}%
\begin{pgfscope}%
\pgfsys@transformshift{2.828685in}{2.001370in}%
\pgfsys@useobject{currentmarker}{}%
\end{pgfscope}%
\begin{pgfscope}%
\pgfsys@transformshift{2.846059in}{1.972052in}%
\pgfsys@useobject{currentmarker}{}%
\end{pgfscope}%
\begin{pgfscope}%
\pgfsys@transformshift{2.866484in}{1.979343in}%
\pgfsys@useobject{currentmarker}{}%
\end{pgfscope}%
\begin{pgfscope}%
\pgfsys@transformshift{2.886673in}{2.010456in}%
\pgfsys@useobject{currentmarker}{}%
\end{pgfscope}%
\begin{pgfscope}%
\pgfsys@transformshift{2.904047in}{2.074976in}%
\pgfsys@useobject{currentmarker}{}%
\end{pgfscope}%
\begin{pgfscope}%
\pgfsys@transformshift{2.924707in}{2.253100in}%
\pgfsys@useobject{currentmarker}{}%
\end{pgfscope}%
\begin{pgfscope}%
\pgfsys@transformshift{2.942784in}{2.498748in}%
\pgfsys@useobject{currentmarker}{}%
\end{pgfscope}%
\begin{pgfscope}%
\pgfsys@transformshift{2.963209in}{2.793265in}%
\pgfsys@useobject{currentmarker}{}%
\end{pgfscope}%
\begin{pgfscope}%
\pgfsys@transformshift{2.979643in}{2.825691in}%
\pgfsys@useobject{currentmarker}{}%
\end{pgfscope}%
\begin{pgfscope}%
\pgfsys@transformshift{3.000774in}{2.758848in}%
\pgfsys@useobject{currentmarker}{}%
\end{pgfscope}%
\begin{pgfscope}%
\pgfsys@transformshift{3.022372in}{2.442936in}%
\pgfsys@useobject{currentmarker}{}%
\end{pgfscope}%
\begin{pgfscope}%
\pgfsys@transformshift{3.038807in}{2.558834in}%
\pgfsys@useobject{currentmarker}{}%
\end{pgfscope}%
\begin{pgfscope}%
\pgfsys@transformshift{3.057822in}{2.268756in}%
\pgfsys@useobject{currentmarker}{}%
\end{pgfscope}%
\begin{pgfscope}%
\pgfsys@transformshift{3.078013in}{2.084790in}%
\pgfsys@useobject{currentmarker}{}%
\end{pgfscope}%
\begin{pgfscope}%
\pgfsys@transformshift{3.097030in}{2.013239in}%
\pgfsys@useobject{currentmarker}{}%
\end{pgfscope}%
\begin{pgfscope}%
\pgfsys@transformshift{3.113464in}{1.977676in}%
\pgfsys@useobject{currentmarker}{}%
\end{pgfscope}%
\begin{pgfscope}%
\pgfsys@transformshift{3.134595in}{1.985102in}%
\pgfsys@useobject{currentmarker}{}%
\end{pgfscope}%
\begin{pgfscope}%
\pgfsys@transformshift{3.155958in}{2.016661in}%
\pgfsys@useobject{currentmarker}{}%
\end{pgfscope}%
\begin{pgfscope}%
\pgfsys@transformshift{3.174269in}{2.097821in}%
\pgfsys@useobject{currentmarker}{}%
\end{pgfscope}%
\begin{pgfscope}%
\pgfsys@transformshift{3.191643in}{2.218750in}%
\pgfsys@useobject{currentmarker}{}%
\end{pgfscope}%
\begin{pgfscope}%
\pgfsys@transformshift{3.212068in}{2.449843in}%
\pgfsys@useobject{currentmarker}{}%
\end{pgfscope}%
\begin{pgfscope}%
\pgfsys@transformshift{3.232025in}{2.698350in}%
\pgfsys@useobject{currentmarker}{}%
\end{pgfscope}%
\begin{pgfscope}%
\pgfsys@transformshift{3.250336in}{2.840665in}%
\pgfsys@useobject{currentmarker}{}%
\end{pgfscope}%
\begin{pgfscope}%
\pgfsys@transformshift{3.267475in}{2.758492in}%
\pgfsys@useobject{currentmarker}{}%
\end{pgfscope}%
\begin{pgfscope}%
\pgfsys@transformshift{3.287664in}{2.503496in}%
\pgfsys@useobject{currentmarker}{}%
\end{pgfscope}%
\begin{pgfscope}%
\pgfsys@transformshift{3.309499in}{2.333129in}%
\pgfsys@useobject{currentmarker}{}%
\end{pgfscope}%
\begin{pgfscope}%
\pgfsys@transformshift{3.324524in}{2.147873in}%
\pgfsys@useobject{currentmarker}{}%
\end{pgfscope}%
\begin{pgfscope}%
\pgfsys@transformshift{3.348001in}{2.024221in}%
\pgfsys@useobject{currentmarker}{}%
\end{pgfscope}%
\begin{pgfscope}%
\pgfsys@transformshift{3.364671in}{1.989834in}%
\pgfsys@useobject{currentmarker}{}%
\end{pgfscope}%
\begin{pgfscope}%
\pgfsys@transformshift{3.384626in}{1.977469in}%
\pgfsys@useobject{currentmarker}{}%
\end{pgfscope}%
\begin{pgfscope}%
\pgfsys@transformshift{3.403408in}{2.009734in}%
\pgfsys@useobject{currentmarker}{}%
\end{pgfscope}%
\begin{pgfscope}%
\pgfsys@transformshift{3.423365in}{2.089481in}%
\pgfsys@useobject{currentmarker}{}%
\end{pgfscope}%
\begin{pgfscope}%
\pgfsys@transformshift{3.442850in}{2.193893in}%
\pgfsys@useobject{currentmarker}{}%
\end{pgfscope}%
\begin{pgfscope}%
\pgfsys@transformshift{3.463276in}{2.336624in}%
\pgfsys@useobject{currentmarker}{}%
\end{pgfscope}%
\begin{pgfscope}%
\pgfsys@transformshift{3.481118in}{2.619353in}%
\pgfsys@useobject{currentmarker}{}%
\end{pgfscope}%
\begin{pgfscope}%
\pgfsys@transformshift{3.498961in}{2.832578in}%
\pgfsys@useobject{currentmarker}{}%
\end{pgfscope}%
\begin{pgfscope}%
\pgfsys@transformshift{3.520324in}{2.836975in}%
\pgfsys@useobject{currentmarker}{}%
\end{pgfscope}%
\begin{pgfscope}%
\pgfsys@transformshift{3.537698in}{2.656671in}%
\pgfsys@useobject{currentmarker}{}%
\end{pgfscope}%
\begin{pgfscope}%
\pgfsys@transformshift{3.556480in}{2.438508in}%
\pgfsys@useobject{currentmarker}{}%
\end{pgfscope}%
\begin{pgfscope}%
\pgfsys@transformshift{3.577140in}{2.199234in}%
\pgfsys@useobject{currentmarker}{}%
\end{pgfscope}%
\begin{pgfscope}%
\pgfsys@transformshift{3.594982in}{2.089103in}%
\pgfsys@useobject{currentmarker}{}%
\end{pgfscope}%
\begin{pgfscope}%
\pgfsys@transformshift{3.616346in}{2.014445in}%
\pgfsys@useobject{currentmarker}{}%
\end{pgfscope}%
\begin{pgfscope}%
\pgfsys@transformshift{3.634190in}{1.983345in}%
\pgfsys@useobject{currentmarker}{}%
\end{pgfscope}%
\begin{pgfscope}%
\pgfsys@transformshift{3.653910in}{1.980838in}%
\pgfsys@useobject{currentmarker}{}%
\end{pgfscope}%
\begin{pgfscope}%
\pgfsys@transformshift{3.673630in}{2.023737in}%
\pgfsys@useobject{currentmarker}{}%
\end{pgfscope}%
\begin{pgfscope}%
\pgfsys@transformshift{3.692413in}{2.073496in}%
\pgfsys@useobject{currentmarker}{}%
\end{pgfscope}%
\begin{pgfscope}%
\pgfsys@transformshift{3.713778in}{2.172205in}%
\pgfsys@useobject{currentmarker}{}%
\end{pgfscope}%
\begin{pgfscope}%
\pgfsys@transformshift{3.728332in}{2.320249in}%
\pgfsys@useobject{currentmarker}{}%
\end{pgfscope}%
\begin{pgfscope}%
\pgfsys@transformshift{3.748992in}{2.598639in}%
\pgfsys@useobject{currentmarker}{}%
\end{pgfscope}%
\begin{pgfscope}%
\pgfsys@transformshift{3.769888in}{2.013315in}%
\pgfsys@useobject{currentmarker}{}%
\end{pgfscope}%
\begin{pgfscope}%
\pgfsys@transformshift{3.791486in}{2.095301in}%
\pgfsys@useobject{currentmarker}{}%
\end{pgfscope}%
\begin{pgfscope}%
\pgfsys@transformshift{3.804399in}{2.242767in}%
\pgfsys@useobject{currentmarker}{}%
\end{pgfscope}%
\begin{pgfscope}%
\pgfsys@transformshift{3.826937in}{2.476491in}%
\pgfsys@useobject{currentmarker}{}%
\end{pgfscope}%
\begin{pgfscope}%
\pgfsys@transformshift{3.846659in}{2.774062in}%
\pgfsys@useobject{currentmarker}{}%
\end{pgfscope}%
\begin{pgfscope}%
\pgfsys@transformshift{3.866379in}{2.885934in}%
\pgfsys@useobject{currentmarker}{}%
\end{pgfscope}%
\begin{pgfscope}%
\pgfsys@transformshift{3.885396in}{2.808468in}%
\pgfsys@useobject{currentmarker}{}%
\end{pgfscope}%
\begin{pgfscope}%
\pgfsys@transformshift{3.902300in}{2.600272in}%
\pgfsys@useobject{currentmarker}{}%
\end{pgfscope}%
\begin{pgfscope}%
\pgfsys@transformshift{3.924132in}{2.280679in}%
\pgfsys@useobject{currentmarker}{}%
\end{pgfscope}%
\begin{pgfscope}%
\pgfsys@transformshift{3.941975in}{2.128900in}%
\pgfsys@useobject{currentmarker}{}%
\end{pgfscope}%
\begin{pgfscope}%
\pgfsys@transformshift{3.960054in}{2.030699in}%
\pgfsys@useobject{currentmarker}{}%
\end{pgfscope}%
\begin{pgfscope}%
\pgfsys@transformshift{3.979540in}{1.988468in}%
\pgfsys@useobject{currentmarker}{}%
\end{pgfscope}%
\begin{pgfscope}%
\pgfsys@transformshift{3.997382in}{1.982899in}%
\pgfsys@useobject{currentmarker}{}%
\end{pgfscope}%
\begin{pgfscope}%
\pgfsys@transformshift{4.018276in}{2.027223in}%
\pgfsys@useobject{currentmarker}{}%
\end{pgfscope}%
\begin{pgfscope}%
\pgfsys@transformshift{4.037293in}{2.114971in}%
\pgfsys@useobject{currentmarker}{}%
\end{pgfscope}%
\begin{pgfscope}%
\pgfsys@transformshift{4.057719in}{2.251653in}%
\pgfsys@useobject{currentmarker}{}%
\end{pgfscope}%
\begin{pgfscope}%
\pgfsys@transformshift{4.079318in}{2.527041in}%
\pgfsys@useobject{currentmarker}{}%
\end{pgfscope}%
\begin{pgfscope}%
\pgfsys@transformshift{4.095047in}{2.775809in}%
\pgfsys@useobject{currentmarker}{}%
\end{pgfscope}%
\begin{pgfscope}%
\pgfsys@transformshift{4.113126in}{2.910659in}%
\pgfsys@useobject{currentmarker}{}%
\end{pgfscope}%
\begin{pgfscope}%
\pgfsys@transformshift{4.133315in}{2.831028in}%
\pgfsys@useobject{currentmarker}{}%
\end{pgfscope}%
\begin{pgfscope}%
\pgfsys@transformshift{4.152097in}{2.614778in}%
\pgfsys@useobject{currentmarker}{}%
\end{pgfscope}%
\begin{pgfscope}%
\pgfsys@transformshift{4.174635in}{2.291560in}%
\pgfsys@useobject{currentmarker}{}%
\end{pgfscope}%
\begin{pgfscope}%
\pgfsys@transformshift{4.192008in}{2.162138in}%
\pgfsys@useobject{currentmarker}{}%
\end{pgfscope}%
\begin{pgfscope}%
\pgfsys@transformshift{4.210556in}{2.056030in}%
\pgfsys@useobject{currentmarker}{}%
\end{pgfscope}%
\begin{pgfscope}%
\pgfsys@transformshift{4.229338in}{2.004086in}%
\pgfsys@useobject{currentmarker}{}%
\end{pgfscope}%
\begin{pgfscope}%
\pgfsys@transformshift{4.248824in}{1.987442in}%
\pgfsys@useobject{currentmarker}{}%
\end{pgfscope}%
\begin{pgfscope}%
\pgfsys@transformshift{4.267604in}{2.006755in}%
\pgfsys@useobject{currentmarker}{}%
\end{pgfscope}%
\begin{pgfscope}%
\pgfsys@transformshift{4.287795in}{2.062504in}%
\pgfsys@useobject{currentmarker}{}%
\end{pgfscope}%
\begin{pgfscope}%
\pgfsys@transformshift{4.308455in}{2.157932in}%
\pgfsys@useobject{currentmarker}{}%
\end{pgfscope}%
\begin{pgfscope}%
\pgfsys@transformshift{4.330993in}{2.407239in}%
\pgfsys@useobject{currentmarker}{}%
\end{pgfscope}%
\begin{pgfscope}%
\pgfsys@transformshift{4.344846in}{2.603080in}%
\pgfsys@useobject{currentmarker}{}%
\end{pgfscope}%
\begin{pgfscope}%
\pgfsys@transformshift{4.362688in}{2.826767in}%
\pgfsys@useobject{currentmarker}{}%
\end{pgfscope}%
\begin{pgfscope}%
\pgfsys@transformshift{4.385226in}{2.939003in}%
\pgfsys@useobject{currentmarker}{}%
\end{pgfscope}%
\begin{pgfscope}%
\pgfsys@transformshift{4.405886in}{2.860670in}%
\pgfsys@useobject{currentmarker}{}%
\end{pgfscope}%
\begin{pgfscope}%
\pgfsys@transformshift{4.423494in}{2.476261in}%
\pgfsys@useobject{currentmarker}{}%
\end{pgfscope}%
\begin{pgfscope}%
\pgfsys@transformshift{4.442745in}{2.773837in}%
\pgfsys@useobject{currentmarker}{}%
\end{pgfscope}%
\begin{pgfscope}%
\pgfsys@transformshift{4.461762in}{2.939419in}%
\pgfsys@useobject{currentmarker}{}%
\end{pgfscope}%
\begin{pgfscope}%
\pgfsys@transformshift{4.481484in}{2.894238in}%
\pgfsys@useobject{currentmarker}{}%
\end{pgfscope}%
\begin{pgfscope}%
\pgfsys@transformshift{4.481953in}{2.887247in}%
\pgfsys@useobject{currentmarker}{}%
\end{pgfscope}%
\begin{pgfscope}%
\pgfsys@transformshift{4.473735in}{2.941555in}%
\pgfsys@useobject{currentmarker}{}%
\end{pgfscope}%
\begin{pgfscope}%
\pgfsys@transformshift{4.455892in}{2.856702in}%
\pgfsys@useobject{currentmarker}{}%
\end{pgfscope}%
\begin{pgfscope}%
\pgfsys@transformshift{4.435232in}{2.425192in}%
\pgfsys@useobject{currentmarker}{}%
\end{pgfscope}%
\begin{pgfscope}%
\pgfsys@transformshift{4.417859in}{2.168514in}%
\pgfsys@useobject{currentmarker}{}%
\end{pgfscope}%
\begin{pgfscope}%
\pgfsys@transformshift{4.398373in}{2.035880in}%
\pgfsys@useobject{currentmarker}{}%
\end{pgfscope}%
\begin{pgfscope}%
\pgfsys@transformshift{4.377244in}{1.985019in}%
\pgfsys@useobject{currentmarker}{}%
\end{pgfscope}%
\begin{pgfscope}%
\pgfsys@transformshift{4.358931in}{2.033467in}%
\pgfsys@useobject{currentmarker}{}%
\end{pgfscope}%
\begin{pgfscope}%
\pgfsys@transformshift{4.338742in}{2.180092in}%
\pgfsys@useobject{currentmarker}{}%
\end{pgfscope}%
\begin{pgfscope}%
\pgfsys@transformshift{4.322072in}{2.472035in}%
\pgfsys@useobject{currentmarker}{}%
\end{pgfscope}%
\begin{pgfscope}%
\pgfsys@transformshift{4.300943in}{2.861304in}%
\pgfsys@useobject{currentmarker}{}%
\end{pgfscope}%
\begin{pgfscope}%
\pgfsys@transformshift{4.281926in}{2.895143in}%
\pgfsys@useobject{currentmarker}{}%
\end{pgfscope}%
\begin{pgfscope}%
\pgfsys@transformshift{4.264318in}{2.613185in}%
\pgfsys@useobject{currentmarker}{}%
\end{pgfscope}%
\begin{pgfscope}%
\pgfsys@transformshift{4.243892in}{2.224019in}%
\pgfsys@useobject{currentmarker}{}%
\end{pgfscope}%
\begin{pgfscope}%
\pgfsys@transformshift{4.222998in}{2.051785in}%
\pgfsys@useobject{currentmarker}{}%
\end{pgfscope}%
\begin{pgfscope}%
\pgfsys@transformshift{4.205156in}{1.987127in}%
\pgfsys@useobject{currentmarker}{}%
\end{pgfscope}%
\begin{pgfscope}%
\pgfsys@transformshift{4.186610in}{1.999141in}%
\pgfsys@useobject{currentmarker}{}%
\end{pgfscope}%
\begin{pgfscope}%
\pgfsys@transformshift{4.166184in}{2.092264in}%
\pgfsys@useobject{currentmarker}{}%
\end{pgfscope}%
\begin{pgfscope}%
\pgfsys@transformshift{4.148105in}{2.290817in}%
\pgfsys@useobject{currentmarker}{}%
\end{pgfscope}%
\begin{pgfscope}%
\pgfsys@transformshift{4.127916in}{2.717775in}%
\pgfsys@useobject{currentmarker}{}%
\end{pgfscope}%
\begin{pgfscope}%
\pgfsys@transformshift{4.109368in}{2.893767in}%
\pgfsys@useobject{currentmarker}{}%
\end{pgfscope}%
\begin{pgfscope}%
\pgfsys@transformshift{4.089177in}{2.715118in}%
\pgfsys@useobject{currentmarker}{}%
\end{pgfscope}%
\begin{pgfscope}%
\pgfsys@transformshift{4.070866in}{2.356767in}%
\pgfsys@useobject{currentmarker}{}%
\end{pgfscope}%
\begin{pgfscope}%
\pgfsys@transformshift{4.052789in}{2.120927in}%
\pgfsys@useobject{currentmarker}{}%
\end{pgfscope}%
\begin{pgfscope}%
\pgfsys@transformshift{4.031658in}{2.002828in}%
\pgfsys@useobject{currentmarker}{}%
\end{pgfscope}%
\begin{pgfscope}%
\pgfsys@transformshift{4.012407in}{1.978569in}%
\pgfsys@useobject{currentmarker}{}%
\end{pgfscope}%
\begin{pgfscope}%
\pgfsys@transformshift{3.993390in}{2.028389in}%
\pgfsys@useobject{currentmarker}{}%
\end{pgfscope}%
\begin{pgfscope}%
\pgfsys@transformshift{3.971087in}{2.182216in}%
\pgfsys@useobject{currentmarker}{}%
\end{pgfscope}%
\begin{pgfscope}%
\pgfsys@transformshift{3.956062in}{2.413143in}%
\pgfsys@useobject{currentmarker}{}%
\end{pgfscope}%
\begin{pgfscope}%
\pgfsys@transformshift{3.935637in}{2.810116in}%
\pgfsys@useobject{currentmarker}{}%
\end{pgfscope}%
\begin{pgfscope}%
\pgfsys@transformshift{3.919672in}{2.861025in}%
\pgfsys@useobject{currentmarker}{}%
\end{pgfscope}%
\begin{pgfscope}%
\pgfsys@transformshift{3.897369in}{2.586537in}%
\pgfsys@useobject{currentmarker}{}%
\end{pgfscope}%
\begin{pgfscope}%
\pgfsys@transformshift{3.878586in}{2.252589in}%
\pgfsys@useobject{currentmarker}{}%
\end{pgfscope}%
\begin{pgfscope}%
\pgfsys@transformshift{3.856989in}{2.056884in}%
\pgfsys@useobject{currentmarker}{}%
\end{pgfscope}%
\begin{pgfscope}%
\pgfsys@transformshift{3.841258in}{2.000072in}%
\pgfsys@useobject{currentmarker}{}%
\end{pgfscope}%
\begin{pgfscope}%
\pgfsys@transformshift{3.819190in}{1.974495in}%
\pgfsys@useobject{currentmarker}{}%
\end{pgfscope}%
\begin{pgfscope}%
\pgfsys@transformshift{3.801113in}{2.000924in}%
\pgfsys@useobject{currentmarker}{}%
\end{pgfscope}%
\begin{pgfscope}%
\pgfsys@transformshift{3.783270in}{2.070516in}%
\pgfsys@useobject{currentmarker}{}%
\end{pgfscope}%
\begin{pgfscope}%
\pgfsys@transformshift{3.764019in}{2.271973in}%
\pgfsys@useobject{currentmarker}{}%
\end{pgfscope}%
\begin{pgfscope}%
\pgfsys@transformshift{3.744297in}{2.623593in}%
\pgfsys@useobject{currentmarker}{}%
\end{pgfscope}%
\begin{pgfscope}%
\pgfsys@transformshift{3.726689in}{2.846093in}%
\pgfsys@useobject{currentmarker}{}%
\end{pgfscope}%
\begin{pgfscope}%
\pgfsys@transformshift{3.706734in}{2.772154in}%
\pgfsys@useobject{currentmarker}{}%
\end{pgfscope}%
\begin{pgfscope}%
\pgfsys@transformshift{3.688186in}{2.502656in}%
\pgfsys@useobject{currentmarker}{}%
\end{pgfscope}%
\begin{pgfscope}%
\pgfsys@transformshift{3.666589in}{2.267958in}%
\pgfsys@useobject{currentmarker}{}%
\end{pgfscope}%
\begin{pgfscope}%
\pgfsys@transformshift{3.647806in}{2.238131in}%
\pgfsys@useobject{currentmarker}{}%
\end{pgfscope}%
\begin{pgfscope}%
\pgfsys@transformshift{3.628555in}{2.067728in}%
\pgfsys@useobject{currentmarker}{}%
\end{pgfscope}%
\begin{pgfscope}%
\pgfsys@transformshift{3.610713in}{1.996522in}%
\pgfsys@useobject{currentmarker}{}%
\end{pgfscope}%
\begin{pgfscope}%
\pgfsys@transformshift{3.588644in}{1.972896in}%
\pgfsys@useobject{currentmarker}{}%
\end{pgfscope}%
\begin{pgfscope}%
\pgfsys@transformshift{3.570331in}{2.013512in}%
\pgfsys@useobject{currentmarker}{}%
\end{pgfscope}%
\begin{pgfscope}%
\pgfsys@transformshift{3.554366in}{2.107578in}%
\pgfsys@useobject{currentmarker}{}%
\end{pgfscope}%
\begin{pgfscope}%
\pgfsys@transformshift{3.533471in}{2.349466in}%
\pgfsys@useobject{currentmarker}{}%
\end{pgfscope}%
\begin{pgfscope}%
\pgfsys@transformshift{3.514455in}{2.443467in}%
\pgfsys@useobject{currentmarker}{}%
\end{pgfscope}%
\begin{pgfscope}%
\pgfsys@transformshift{3.495674in}{2.753158in}%
\pgfsys@useobject{currentmarker}{}%
\end{pgfscope}%
\begin{pgfscope}%
\pgfsys@transformshift{3.477127in}{2.045169in}%
\pgfsys@useobject{currentmarker}{}%
\end{pgfscope}%
\begin{pgfscope}%
\pgfsys@transformshift{3.454589in}{2.211273in}%
\pgfsys@useobject{currentmarker}{}%
\end{pgfscope}%
\begin{pgfscope}%
\pgfsys@transformshift{3.436746in}{2.526113in}%
\pgfsys@useobject{currentmarker}{}%
\end{pgfscope}%
\begin{pgfscope}%
\pgfsys@transformshift{3.417964in}{2.811097in}%
\pgfsys@useobject{currentmarker}{}%
\end{pgfscope}%
\begin{pgfscope}%
\pgfsys@transformshift{3.397773in}{2.763116in}%
\pgfsys@useobject{currentmarker}{}%
\end{pgfscope}%
\begin{pgfscope}%
\pgfsys@transformshift{3.378053in}{2.517360in}%
\pgfsys@useobject{currentmarker}{}%
\end{pgfscope}%
\begin{pgfscope}%
\pgfsys@transformshift{3.359271in}{2.215958in}%
\pgfsys@useobject{currentmarker}{}%
\end{pgfscope}%
\begin{pgfscope}%
\pgfsys@transformshift{3.340959in}{2.068930in}%
\pgfsys@useobject{currentmarker}{}%
\end{pgfscope}%
\begin{pgfscope}%
\pgfsys@transformshift{3.321708in}{1.994177in}%
\pgfsys@useobject{currentmarker}{}%
\end{pgfscope}%
\begin{pgfscope}%
\pgfsys@transformshift{3.299874in}{1.971043in}%
\pgfsys@useobject{currentmarker}{}%
\end{pgfscope}%
\begin{pgfscope}%
\pgfsys@transformshift{3.280857in}{1.995102in}%
\pgfsys@useobject{currentmarker}{}%
\end{pgfscope}%
\begin{pgfscope}%
\pgfsys@transformshift{3.262544in}{2.062155in}%
\pgfsys@useobject{currentmarker}{}%
\end{pgfscope}%
\begin{pgfscope}%
\pgfsys@transformshift{3.243998in}{2.221405in}%
\pgfsys@useobject{currentmarker}{}%
\end{pgfscope}%
\begin{pgfscope}%
\pgfsys@transformshift{3.227564in}{2.510277in}%
\pgfsys@useobject{currentmarker}{}%
\end{pgfscope}%
\begin{pgfscope}%
\pgfsys@transformshift{3.205025in}{2.803587in}%
\pgfsys@useobject{currentmarker}{}%
\end{pgfscope}%
\begin{pgfscope}%
\pgfsys@transformshift{3.189296in}{2.809289in}%
\pgfsys@useobject{currentmarker}{}%
\end{pgfscope}%
\begin{pgfscope}%
\pgfsys@transformshift{3.167931in}{2.537967in}%
\pgfsys@useobject{currentmarker}{}%
\end{pgfscope}%
\begin{pgfscope}%
\pgfsys@transformshift{3.148445in}{2.229407in}%
\pgfsys@useobject{currentmarker}{}%
\end{pgfscope}%
\begin{pgfscope}%
\pgfsys@transformshift{3.126377in}{2.055346in}%
\pgfsys@useobject{currentmarker}{}%
\end{pgfscope}%
\begin{pgfscope}%
\pgfsys@transformshift{3.111821in}{2.003332in}%
\pgfsys@useobject{currentmarker}{}%
\end{pgfscope}%
\begin{pgfscope}%
\pgfsys@transformshift{3.088812in}{1.968873in}%
\pgfsys@useobject{currentmarker}{}%
\end{pgfscope}%
\begin{pgfscope}%
\pgfsys@transformshift{3.073553in}{1.986318in}%
\pgfsys@useobject{currentmarker}{}%
\end{pgfscope}%
\begin{pgfscope}%
\pgfsys@transformshift{3.052893in}{2.052411in}%
\pgfsys@useobject{currentmarker}{}%
\end{pgfscope}%
\begin{pgfscope}%
\pgfsys@transformshift{3.034347in}{2.148921in}%
\pgfsys@useobject{currentmarker}{}%
\end{pgfscope}%
\begin{pgfscope}%
\pgfsys@transformshift{3.012747in}{2.386655in}%
\pgfsys@useobject{currentmarker}{}%
\end{pgfscope}%
\begin{pgfscope}%
\pgfsys@transformshift{2.993025in}{2.736946in}%
\pgfsys@useobject{currentmarker}{}%
\end{pgfscope}%
\begin{pgfscope}%
\pgfsys@transformshift{2.974479in}{2.824579in}%
\pgfsys@useobject{currentmarker}{}%
\end{pgfscope}%
\begin{pgfscope}%
\pgfsys@transformshift{2.954288in}{2.717519in}%
\pgfsys@useobject{currentmarker}{}%
\end{pgfscope}%
\begin{pgfscope}%
\pgfsys@transformshift{2.938089in}{2.486367in}%
\pgfsys@useobject{currentmarker}{}%
\end{pgfscope}%
\begin{pgfscope}%
\pgfsys@transformshift{2.918603in}{2.184845in}%
\pgfsys@useobject{currentmarker}{}%
\end{pgfscope}%
\begin{pgfscope}%
\pgfsys@transformshift{2.899821in}{2.202071in}%
\pgfsys@useobject{currentmarker}{}%
\end{pgfscope}%
\begin{pgfscope}%
\pgfsys@transformshift{2.879161in}{2.038983in}%
\pgfsys@useobject{currentmarker}{}%
\end{pgfscope}%
\begin{pgfscope}%
\pgfsys@transformshift{2.860615in}{1.985525in}%
\pgfsys@useobject{currentmarker}{}%
\end{pgfscope}%
\begin{pgfscope}%
\pgfsys@transformshift{2.838781in}{1.972414in}%
\pgfsys@useobject{currentmarker}{}%
\end{pgfscope}%
\begin{pgfscope}%
\pgfsys@transformshift{2.819059in}{2.013348in}%
\pgfsys@useobject{currentmarker}{}%
\end{pgfscope}%
\begin{pgfscope}%
\pgfsys@transformshift{2.804739in}{2.086202in}%
\pgfsys@useobject{currentmarker}{}%
\end{pgfscope}%
\begin{pgfscope}%
\pgfsys@transformshift{2.804268in}{2.189129in}%
\pgfsys@useobject{currentmarker}{}%
\end{pgfscope}%
\begin{pgfscope}%
\pgfsys@transformshift{2.783139in}{2.243069in}%
\pgfsys@useobject{currentmarker}{}%
\end{pgfscope}%
\begin{pgfscope}%
\pgfsys@transformshift{2.764123in}{2.540034in}%
\pgfsys@useobject{currentmarker}{}%
\end{pgfscope}%
\begin{pgfscope}%
\pgfsys@transformshift{2.746280in}{2.778584in}%
\pgfsys@useobject{currentmarker}{}%
\end{pgfscope}%
\begin{pgfscope}%
\pgfsys@transformshift{2.724211in}{2.782520in}%
\pgfsys@useobject{currentmarker}{}%
\end{pgfscope}%
\begin{pgfscope}%
\pgfsys@transformshift{2.705664in}{2.546539in}%
\pgfsys@useobject{currentmarker}{}%
\end{pgfscope}%
\begin{pgfscope}%
\pgfsys@transformshift{2.683831in}{2.202771in}%
\pgfsys@useobject{currentmarker}{}%
\end{pgfscope}%
\begin{pgfscope}%
\pgfsys@transformshift{2.668335in}{2.087276in}%
\pgfsys@useobject{currentmarker}{}%
\end{pgfscope}%
\begin{pgfscope}%
\pgfsys@transformshift{2.648381in}{1.997753in}%
\pgfsys@useobject{currentmarker}{}%
\end{pgfscope}%
\begin{pgfscope}%
\pgfsys@transformshift{2.627250in}{1.969311in}%
\pgfsys@useobject{currentmarker}{}%
\end{pgfscope}%
\begin{pgfscope}%
\pgfsys@transformshift{2.611756in}{1.980771in}%
\pgfsys@useobject{currentmarker}{}%
\end{pgfscope}%
\begin{pgfscope}%
\pgfsys@transformshift{2.593208in}{2.037675in}%
\pgfsys@useobject{currentmarker}{}%
\end{pgfscope}%
\begin{pgfscope}%
\pgfsys@transformshift{2.566445in}{2.207038in}%
\pgfsys@useobject{currentmarker}{}%
\end{pgfscope}%
\begin{pgfscope}%
\pgfsys@transformshift{2.549540in}{2.439928in}%
\pgfsys@useobject{currentmarker}{}%
\end{pgfscope}%
\begin{pgfscope}%
\pgfsys@transformshift{2.531697in}{2.720301in}%
\pgfsys@useobject{currentmarker}{}%
\end{pgfscope}%
\begin{pgfscope}%
\pgfsys@transformshift{2.513151in}{2.814503in}%
\pgfsys@useobject{currentmarker}{}%
\end{pgfscope}%
\begin{pgfscope}%
\pgfsys@transformshift{2.494369in}{2.673701in}%
\pgfsys@useobject{currentmarker}{}%
\end{pgfscope}%
\begin{pgfscope}%
\pgfsys@transformshift{2.476058in}{2.458772in}%
\pgfsys@useobject{currentmarker}{}%
\end{pgfscope}%
\begin{pgfscope}%
\pgfsys@transformshift{2.455161in}{2.182170in}%
\pgfsys@useobject{currentmarker}{}%
\end{pgfscope}%
\begin{pgfscope}%
\pgfsys@transformshift{2.438259in}{2.067199in}%
\pgfsys@useobject{currentmarker}{}%
\end{pgfscope}%
\begin{pgfscope}%
\pgfsys@transformshift{2.416894in}{2.004753in}%
\pgfsys@useobject{currentmarker}{}%
\end{pgfscope}%
\begin{pgfscope}%
\pgfsys@transformshift{2.398817in}{1.982698in}%
\pgfsys@useobject{currentmarker}{}%
\end{pgfscope}%
\begin{pgfscope}%
\pgfsys@transformshift{2.380271in}{1.969951in}%
\pgfsys@useobject{currentmarker}{}%
\end{pgfscope}%
\begin{pgfscope}%
\pgfsys@transformshift{2.359140in}{2.002265in}%
\pgfsys@useobject{currentmarker}{}%
\end{pgfscope}%
\begin{pgfscope}%
\pgfsys@transformshift{2.339889in}{2.073114in}%
\pgfsys@useobject{currentmarker}{}%
\end{pgfscope}%
\begin{pgfscope}%
\pgfsys@transformshift{2.321577in}{2.208504in}%
\pgfsys@useobject{currentmarker}{}%
\end{pgfscope}%
\begin{pgfscope}%
\pgfsys@transformshift{2.303029in}{2.485488in}%
\pgfsys@useobject{currentmarker}{}%
\end{pgfscope}%
\begin{pgfscope}%
\pgfsys@transformshift{2.281901in}{2.746115in}%
\pgfsys@useobject{currentmarker}{}%
\end{pgfscope}%
\begin{pgfscope}%
\pgfsys@transformshift{2.265467in}{2.820446in}%
\pgfsys@useobject{currentmarker}{}%
\end{pgfscope}%
\begin{pgfscope}%
\pgfsys@transformshift{2.245745in}{2.647889in}%
\pgfsys@useobject{currentmarker}{}%
\end{pgfscope}%
\begin{pgfscope}%
\pgfsys@transformshift{2.225556in}{2.386322in}%
\pgfsys@useobject{currentmarker}{}%
\end{pgfscope}%
\begin{pgfscope}%
\pgfsys@transformshift{2.204659in}{2.210965in}%
\pgfsys@useobject{currentmarker}{}%
\end{pgfscope}%
\begin{pgfscope}%
\pgfsys@transformshift{2.185879in}{2.075106in}%
\pgfsys@useobject{currentmarker}{}%
\end{pgfscope}%
\begin{pgfscope}%
\pgfsys@transformshift{2.167566in}{2.037302in}%
\pgfsys@useobject{currentmarker}{}%
\end{pgfscope}%
\begin{pgfscope}%
\pgfsys@transformshift{2.147845in}{2.001771in}%
\pgfsys@useobject{currentmarker}{}%
\end{pgfscope}%
\begin{pgfscope}%
\pgfsys@transformshift{2.129769in}{1.984793in}%
\pgfsys@useobject{currentmarker}{}%
\end{pgfscope}%
\begin{pgfscope}%
\pgfsys@transformshift{2.111924in}{1.973487in}%
\pgfsys@useobject{currentmarker}{}%
\end{pgfscope}%
\begin{pgfscope}%
\pgfsys@transformshift{2.089621in}{2.006247in}%
\pgfsys@useobject{currentmarker}{}%
\end{pgfscope}%
\begin{pgfscope}%
\pgfsys@transformshift{2.070841in}{2.087306in}%
\pgfsys@useobject{currentmarker}{}%
\end{pgfscope}%
\begin{pgfscope}%
\pgfsys@transformshift{2.053467in}{2.269938in}%
\pgfsys@useobject{currentmarker}{}%
\end{pgfscope}%
\begin{pgfscope}%
\pgfsys@transformshift{2.036094in}{2.624421in}%
\pgfsys@useobject{currentmarker}{}%
\end{pgfscope}%
\begin{pgfscope}%
\pgfsys@transformshift{2.016137in}{2.808828in}%
\pgfsys@useobject{currentmarker}{}%
\end{pgfscope}%
\begin{pgfscope}%
\pgfsys@transformshift{1.993599in}{2.804660in}%
\pgfsys@useobject{currentmarker}{}%
\end{pgfscope}%
\begin{pgfscope}%
\pgfsys@transformshift{1.975522in}{2.597319in}%
\pgfsys@useobject{currentmarker}{}%
\end{pgfscope}%
\begin{pgfscope}%
\pgfsys@transformshift{1.956975in}{2.349356in}%
\pgfsys@useobject{currentmarker}{}%
\end{pgfscope}%
\begin{pgfscope}%
\pgfsys@transformshift{1.934906in}{2.126721in}%
\pgfsys@useobject{currentmarker}{}%
\end{pgfscope}%
\begin{pgfscope}%
\pgfsys@transformshift{1.917298in}{2.040262in}%
\pgfsys@useobject{currentmarker}{}%
\end{pgfscope}%
\begin{pgfscope}%
\pgfsys@transformshift{1.899690in}{2.000145in}%
\pgfsys@useobject{currentmarker}{}%
\end{pgfscope}%
\begin{pgfscope}%
\pgfsys@transformshift{1.875978in}{1.976563in}%
\pgfsys@useobject{currentmarker}{}%
\end{pgfscope}%
\begin{pgfscope}%
\pgfsys@transformshift{1.858606in}{1.994420in}%
\pgfsys@useobject{currentmarker}{}%
\end{pgfscope}%
\begin{pgfscope}%
\pgfsys@transformshift{1.840293in}{2.051277in}%
\pgfsys@useobject{currentmarker}{}%
\end{pgfscope}%
\begin{pgfscope}%
\pgfsys@transformshift{1.821747in}{2.147685in}%
\pgfsys@useobject{currentmarker}{}%
\end{pgfscope}%
\begin{pgfscope}%
\pgfsys@transformshift{1.802965in}{2.277686in}%
\pgfsys@useobject{currentmarker}{}%
\end{pgfscope}%
\begin{pgfscope}%
\pgfsys@transformshift{1.781834in}{2.586343in}%
\pgfsys@useobject{currentmarker}{}%
\end{pgfscope}%
\begin{pgfscope}%
\pgfsys@transformshift{1.763288in}{2.487237in}%
\pgfsys@useobject{currentmarker}{}%
\end{pgfscope}%
\begin{pgfscope}%
\pgfsys@transformshift{1.746620in}{2.759034in}%
\pgfsys@useobject{currentmarker}{}%
\end{pgfscope}%
\begin{pgfscope}%
\pgfsys@transformshift{1.726195in}{2.848582in}%
\pgfsys@useobject{currentmarker}{}%
\end{pgfscope}%
\begin{pgfscope}%
\pgfsys@transformshift{1.707178in}{2.741552in}%
\pgfsys@useobject{currentmarker}{}%
\end{pgfscope}%
\begin{pgfscope}%
\pgfsys@transformshift{1.686047in}{2.539256in}%
\pgfsys@useobject{currentmarker}{}%
\end{pgfscope}%
\begin{pgfscope}%
\pgfsys@transformshift{1.670319in}{2.350815in}%
\pgfsys@useobject{currentmarker}{}%
\end{pgfscope}%
\begin{pgfscope}%
\pgfsys@transformshift{1.648015in}{2.816623in}%
\pgfsys@useobject{currentmarker}{}%
\end{pgfscope}%
\begin{pgfscope}%
\pgfsys@transformshift{1.629937in}{2.823682in}%
\pgfsys@useobject{currentmarker}{}%
\end{pgfscope}%
\begin{pgfscope}%
\pgfsys@transformshift{1.611625in}{2.621634in}%
\pgfsys@useobject{currentmarker}{}%
\end{pgfscope}%
\end{pgfscope}%
\begin{pgfscope}%
\pgfsetrectcap%
\pgfsetmiterjoin%
\pgfsetlinewidth{0.501875pt}%
\definecolor{currentstroke}{rgb}{0.000000,0.000000,0.000000}%
\pgfsetstrokecolor{currentstroke}%
\pgfsetdash{}{0pt}%
\pgfpathmoveto{\pgfqpoint{0.444748in}{1.917688in}}%
\pgfpathlineto{\pgfqpoint{0.444748in}{2.993810in}}%
\pgfusepath{stroke}%
\end{pgfscope}%
\begin{pgfscope}%
\pgfsetrectcap%
\pgfsetmiterjoin%
\pgfsetlinewidth{0.501875pt}%
\definecolor{currentstroke}{rgb}{0.000000,0.000000,0.000000}%
\pgfsetstrokecolor{currentstroke}%
\pgfsetdash{}{0pt}%
\pgfpathmoveto{\pgfqpoint{4.676167in}{1.917688in}}%
\pgfpathlineto{\pgfqpoint{4.676167in}{2.993810in}}%
\pgfusepath{stroke}%
\end{pgfscope}%
\begin{pgfscope}%
\pgfsetrectcap%
\pgfsetmiterjoin%
\pgfsetlinewidth{0.501875pt}%
\definecolor{currentstroke}{rgb}{0.000000,0.000000,0.000000}%
\pgfsetstrokecolor{currentstroke}%
\pgfsetdash{}{0pt}%
\pgfpathmoveto{\pgfqpoint{0.444748in}{1.917688in}}%
\pgfpathlineto{\pgfqpoint{4.676167in}{1.917688in}}%
\pgfusepath{stroke}%
\end{pgfscope}%
\begin{pgfscope}%
\pgfsetrectcap%
\pgfsetmiterjoin%
\pgfsetlinewidth{0.501875pt}%
\definecolor{currentstroke}{rgb}{0.000000,0.000000,0.000000}%
\pgfsetstrokecolor{currentstroke}%
\pgfsetdash{}{0pt}%
\pgfpathmoveto{\pgfqpoint{0.444748in}{2.993810in}}%
\pgfpathlineto{\pgfqpoint{4.676167in}{2.993810in}}%
\pgfusepath{stroke}%
\end{pgfscope}%
\begin{pgfscope}%
\definecolor{textcolor}{rgb}{0.000000,0.000000,0.000000}%
\pgfsetstrokecolor{textcolor}%
\pgfsetfillcolor{textcolor}%
\pgftext[x=2.560458in,y=3.077144in,,base]{\color{textcolor}\rmfamily\fontsize{12.000000}{14.400000}\selectfont T = \qty{3.2}{\kelvin}}%
\end{pgfscope}%
\begin{pgfscope}%
\pgfsetbuttcap%
\pgfsetmiterjoin%
\definecolor{currentfill}{rgb}{1.000000,1.000000,1.000000}%
\pgfsetfillcolor{currentfill}%
\pgfsetlinewidth{0.000000pt}%
\definecolor{currentstroke}{rgb}{0.000000,0.000000,0.000000}%
\pgfsetstrokecolor{currentstroke}%
\pgfsetstrokeopacity{0.000000}%
\pgfsetdash{}{0pt}%
\pgfpathmoveto{\pgfqpoint{0.444748in}{0.431673in}}%
\pgfpathlineto{\pgfqpoint{4.676167in}{0.431673in}}%
\pgfpathlineto{\pgfqpoint{4.676167in}{1.507795in}}%
\pgfpathlineto{\pgfqpoint{0.444748in}{1.507795in}}%
\pgfpathlineto{\pgfqpoint{0.444748in}{0.431673in}}%
\pgfpathclose%
\pgfusepath{fill}%
\end{pgfscope}%
\begin{pgfscope}%
\pgfsetbuttcap%
\pgfsetroundjoin%
\definecolor{currentfill}{rgb}{0.000000,0.000000,0.000000}%
\pgfsetfillcolor{currentfill}%
\pgfsetlinewidth{0.501875pt}%
\definecolor{currentstroke}{rgb}{0.000000,0.000000,0.000000}%
\pgfsetstrokecolor{currentstroke}%
\pgfsetdash{}{0pt}%
\pgfsys@defobject{currentmarker}{\pgfqpoint{0.000000in}{0.000000in}}{\pgfqpoint{0.000000in}{0.041667in}}{%
\pgfpathmoveto{\pgfqpoint{0.000000in}{0.000000in}}%
\pgfpathlineto{\pgfqpoint{0.000000in}{0.041667in}}%
\pgfusepath{stroke,fill}%
}%
\begin{pgfscope}%
\pgfsys@transformshift{0.643182in}{0.431673in}%
\pgfsys@useobject{currentmarker}{}%
\end{pgfscope}%
\end{pgfscope}%
\begin{pgfscope}%
\pgfsetbuttcap%
\pgfsetroundjoin%
\definecolor{currentfill}{rgb}{0.000000,0.000000,0.000000}%
\pgfsetfillcolor{currentfill}%
\pgfsetlinewidth{0.501875pt}%
\definecolor{currentstroke}{rgb}{0.000000,0.000000,0.000000}%
\pgfsetstrokecolor{currentstroke}%
\pgfsetdash{}{0pt}%
\pgfsys@defobject{currentmarker}{\pgfqpoint{0.000000in}{-0.041667in}}{\pgfqpoint{0.000000in}{0.000000in}}{%
\pgfpathmoveto{\pgfqpoint{0.000000in}{0.000000in}}%
\pgfpathlineto{\pgfqpoint{0.000000in}{-0.041667in}}%
\pgfusepath{stroke,fill}%
}%
\begin{pgfscope}%
\pgfsys@transformshift{0.643182in}{1.507795in}%
\pgfsys@useobject{currentmarker}{}%
\end{pgfscope}%
\end{pgfscope}%
\begin{pgfscope}%
\definecolor{textcolor}{rgb}{0.000000,0.000000,0.000000}%
\pgfsetstrokecolor{textcolor}%
\pgfsetfillcolor{textcolor}%
\pgftext[x=0.643182in,y=0.383062in,,top]{\color{textcolor}\rmfamily\fontsize{10.000000}{12.000000}\selectfont \(\displaystyle {\ensuremath{-}10.0}\)}%
\end{pgfscope}%
\begin{pgfscope}%
\pgfsetbuttcap%
\pgfsetroundjoin%
\definecolor{currentfill}{rgb}{0.000000,0.000000,0.000000}%
\pgfsetfillcolor{currentfill}%
\pgfsetlinewidth{0.501875pt}%
\definecolor{currentstroke}{rgb}{0.000000,0.000000,0.000000}%
\pgfsetstrokecolor{currentstroke}%
\pgfsetdash{}{0pt}%
\pgfsys@defobject{currentmarker}{\pgfqpoint{0.000000in}{0.000000in}}{\pgfqpoint{0.000000in}{0.041667in}}{%
\pgfpathmoveto{\pgfqpoint{0.000000in}{0.000000in}}%
\pgfpathlineto{\pgfqpoint{0.000000in}{0.041667in}}%
\pgfusepath{stroke,fill}%
}%
\begin{pgfscope}%
\pgfsys@transformshift{1.123645in}{0.431673in}%
\pgfsys@useobject{currentmarker}{}%
\end{pgfscope}%
\end{pgfscope}%
\begin{pgfscope}%
\pgfsetbuttcap%
\pgfsetroundjoin%
\definecolor{currentfill}{rgb}{0.000000,0.000000,0.000000}%
\pgfsetfillcolor{currentfill}%
\pgfsetlinewidth{0.501875pt}%
\definecolor{currentstroke}{rgb}{0.000000,0.000000,0.000000}%
\pgfsetstrokecolor{currentstroke}%
\pgfsetdash{}{0pt}%
\pgfsys@defobject{currentmarker}{\pgfqpoint{0.000000in}{-0.041667in}}{\pgfqpoint{0.000000in}{0.000000in}}{%
\pgfpathmoveto{\pgfqpoint{0.000000in}{0.000000in}}%
\pgfpathlineto{\pgfqpoint{0.000000in}{-0.041667in}}%
\pgfusepath{stroke,fill}%
}%
\begin{pgfscope}%
\pgfsys@transformshift{1.123645in}{1.507795in}%
\pgfsys@useobject{currentmarker}{}%
\end{pgfscope}%
\end{pgfscope}%
\begin{pgfscope}%
\definecolor{textcolor}{rgb}{0.000000,0.000000,0.000000}%
\pgfsetstrokecolor{textcolor}%
\pgfsetfillcolor{textcolor}%
\pgftext[x=1.123645in,y=0.383062in,,top]{\color{textcolor}\rmfamily\fontsize{10.000000}{12.000000}\selectfont \(\displaystyle {\ensuremath{-}7.5}\)}%
\end{pgfscope}%
\begin{pgfscope}%
\pgfsetbuttcap%
\pgfsetroundjoin%
\definecolor{currentfill}{rgb}{0.000000,0.000000,0.000000}%
\pgfsetfillcolor{currentfill}%
\pgfsetlinewidth{0.501875pt}%
\definecolor{currentstroke}{rgb}{0.000000,0.000000,0.000000}%
\pgfsetstrokecolor{currentstroke}%
\pgfsetdash{}{0pt}%
\pgfsys@defobject{currentmarker}{\pgfqpoint{0.000000in}{0.000000in}}{\pgfqpoint{0.000000in}{0.041667in}}{%
\pgfpathmoveto{\pgfqpoint{0.000000in}{0.000000in}}%
\pgfpathlineto{\pgfqpoint{0.000000in}{0.041667in}}%
\pgfusepath{stroke,fill}%
}%
\begin{pgfscope}%
\pgfsys@transformshift{1.604109in}{0.431673in}%
\pgfsys@useobject{currentmarker}{}%
\end{pgfscope}%
\end{pgfscope}%
\begin{pgfscope}%
\pgfsetbuttcap%
\pgfsetroundjoin%
\definecolor{currentfill}{rgb}{0.000000,0.000000,0.000000}%
\pgfsetfillcolor{currentfill}%
\pgfsetlinewidth{0.501875pt}%
\definecolor{currentstroke}{rgb}{0.000000,0.000000,0.000000}%
\pgfsetstrokecolor{currentstroke}%
\pgfsetdash{}{0pt}%
\pgfsys@defobject{currentmarker}{\pgfqpoint{0.000000in}{-0.041667in}}{\pgfqpoint{0.000000in}{0.000000in}}{%
\pgfpathmoveto{\pgfqpoint{0.000000in}{0.000000in}}%
\pgfpathlineto{\pgfqpoint{0.000000in}{-0.041667in}}%
\pgfusepath{stroke,fill}%
}%
\begin{pgfscope}%
\pgfsys@transformshift{1.604109in}{1.507795in}%
\pgfsys@useobject{currentmarker}{}%
\end{pgfscope}%
\end{pgfscope}%
\begin{pgfscope}%
\definecolor{textcolor}{rgb}{0.000000,0.000000,0.000000}%
\pgfsetstrokecolor{textcolor}%
\pgfsetfillcolor{textcolor}%
\pgftext[x=1.604109in,y=0.383062in,,top]{\color{textcolor}\rmfamily\fontsize{10.000000}{12.000000}\selectfont \(\displaystyle {\ensuremath{-}5.0}\)}%
\end{pgfscope}%
\begin{pgfscope}%
\pgfsetbuttcap%
\pgfsetroundjoin%
\definecolor{currentfill}{rgb}{0.000000,0.000000,0.000000}%
\pgfsetfillcolor{currentfill}%
\pgfsetlinewidth{0.501875pt}%
\definecolor{currentstroke}{rgb}{0.000000,0.000000,0.000000}%
\pgfsetstrokecolor{currentstroke}%
\pgfsetdash{}{0pt}%
\pgfsys@defobject{currentmarker}{\pgfqpoint{0.000000in}{0.000000in}}{\pgfqpoint{0.000000in}{0.041667in}}{%
\pgfpathmoveto{\pgfqpoint{0.000000in}{0.000000in}}%
\pgfpathlineto{\pgfqpoint{0.000000in}{0.041667in}}%
\pgfusepath{stroke,fill}%
}%
\begin{pgfscope}%
\pgfsys@transformshift{2.084572in}{0.431673in}%
\pgfsys@useobject{currentmarker}{}%
\end{pgfscope}%
\end{pgfscope}%
\begin{pgfscope}%
\pgfsetbuttcap%
\pgfsetroundjoin%
\definecolor{currentfill}{rgb}{0.000000,0.000000,0.000000}%
\pgfsetfillcolor{currentfill}%
\pgfsetlinewidth{0.501875pt}%
\definecolor{currentstroke}{rgb}{0.000000,0.000000,0.000000}%
\pgfsetstrokecolor{currentstroke}%
\pgfsetdash{}{0pt}%
\pgfsys@defobject{currentmarker}{\pgfqpoint{0.000000in}{-0.041667in}}{\pgfqpoint{0.000000in}{0.000000in}}{%
\pgfpathmoveto{\pgfqpoint{0.000000in}{0.000000in}}%
\pgfpathlineto{\pgfqpoint{0.000000in}{-0.041667in}}%
\pgfusepath{stroke,fill}%
}%
\begin{pgfscope}%
\pgfsys@transformshift{2.084572in}{1.507795in}%
\pgfsys@useobject{currentmarker}{}%
\end{pgfscope}%
\end{pgfscope}%
\begin{pgfscope}%
\definecolor{textcolor}{rgb}{0.000000,0.000000,0.000000}%
\pgfsetstrokecolor{textcolor}%
\pgfsetfillcolor{textcolor}%
\pgftext[x=2.084572in,y=0.383062in,,top]{\color{textcolor}\rmfamily\fontsize{10.000000}{12.000000}\selectfont \(\displaystyle {\ensuremath{-}2.5}\)}%
\end{pgfscope}%
\begin{pgfscope}%
\pgfsetbuttcap%
\pgfsetroundjoin%
\definecolor{currentfill}{rgb}{0.000000,0.000000,0.000000}%
\pgfsetfillcolor{currentfill}%
\pgfsetlinewidth{0.501875pt}%
\definecolor{currentstroke}{rgb}{0.000000,0.000000,0.000000}%
\pgfsetstrokecolor{currentstroke}%
\pgfsetdash{}{0pt}%
\pgfsys@defobject{currentmarker}{\pgfqpoint{0.000000in}{0.000000in}}{\pgfqpoint{0.000000in}{0.041667in}}{%
\pgfpathmoveto{\pgfqpoint{0.000000in}{0.000000in}}%
\pgfpathlineto{\pgfqpoint{0.000000in}{0.041667in}}%
\pgfusepath{stroke,fill}%
}%
\begin{pgfscope}%
\pgfsys@transformshift{2.565036in}{0.431673in}%
\pgfsys@useobject{currentmarker}{}%
\end{pgfscope}%
\end{pgfscope}%
\begin{pgfscope}%
\pgfsetbuttcap%
\pgfsetroundjoin%
\definecolor{currentfill}{rgb}{0.000000,0.000000,0.000000}%
\pgfsetfillcolor{currentfill}%
\pgfsetlinewidth{0.501875pt}%
\definecolor{currentstroke}{rgb}{0.000000,0.000000,0.000000}%
\pgfsetstrokecolor{currentstroke}%
\pgfsetdash{}{0pt}%
\pgfsys@defobject{currentmarker}{\pgfqpoint{0.000000in}{-0.041667in}}{\pgfqpoint{0.000000in}{0.000000in}}{%
\pgfpathmoveto{\pgfqpoint{0.000000in}{0.000000in}}%
\pgfpathlineto{\pgfqpoint{0.000000in}{-0.041667in}}%
\pgfusepath{stroke,fill}%
}%
\begin{pgfscope}%
\pgfsys@transformshift{2.565036in}{1.507795in}%
\pgfsys@useobject{currentmarker}{}%
\end{pgfscope}%
\end{pgfscope}%
\begin{pgfscope}%
\definecolor{textcolor}{rgb}{0.000000,0.000000,0.000000}%
\pgfsetstrokecolor{textcolor}%
\pgfsetfillcolor{textcolor}%
\pgftext[x=2.565036in,y=0.383062in,,top]{\color{textcolor}\rmfamily\fontsize{10.000000}{12.000000}\selectfont \(\displaystyle {0.0}\)}%
\end{pgfscope}%
\begin{pgfscope}%
\pgfsetbuttcap%
\pgfsetroundjoin%
\definecolor{currentfill}{rgb}{0.000000,0.000000,0.000000}%
\pgfsetfillcolor{currentfill}%
\pgfsetlinewidth{0.501875pt}%
\definecolor{currentstroke}{rgb}{0.000000,0.000000,0.000000}%
\pgfsetstrokecolor{currentstroke}%
\pgfsetdash{}{0pt}%
\pgfsys@defobject{currentmarker}{\pgfqpoint{0.000000in}{0.000000in}}{\pgfqpoint{0.000000in}{0.041667in}}{%
\pgfpathmoveto{\pgfqpoint{0.000000in}{0.000000in}}%
\pgfpathlineto{\pgfqpoint{0.000000in}{0.041667in}}%
\pgfusepath{stroke,fill}%
}%
\begin{pgfscope}%
\pgfsys@transformshift{3.045499in}{0.431673in}%
\pgfsys@useobject{currentmarker}{}%
\end{pgfscope}%
\end{pgfscope}%
\begin{pgfscope}%
\pgfsetbuttcap%
\pgfsetroundjoin%
\definecolor{currentfill}{rgb}{0.000000,0.000000,0.000000}%
\pgfsetfillcolor{currentfill}%
\pgfsetlinewidth{0.501875pt}%
\definecolor{currentstroke}{rgb}{0.000000,0.000000,0.000000}%
\pgfsetstrokecolor{currentstroke}%
\pgfsetdash{}{0pt}%
\pgfsys@defobject{currentmarker}{\pgfqpoint{0.000000in}{-0.041667in}}{\pgfqpoint{0.000000in}{0.000000in}}{%
\pgfpathmoveto{\pgfqpoint{0.000000in}{0.000000in}}%
\pgfpathlineto{\pgfqpoint{0.000000in}{-0.041667in}}%
\pgfusepath{stroke,fill}%
}%
\begin{pgfscope}%
\pgfsys@transformshift{3.045499in}{1.507795in}%
\pgfsys@useobject{currentmarker}{}%
\end{pgfscope}%
\end{pgfscope}%
\begin{pgfscope}%
\definecolor{textcolor}{rgb}{0.000000,0.000000,0.000000}%
\pgfsetstrokecolor{textcolor}%
\pgfsetfillcolor{textcolor}%
\pgftext[x=3.045499in,y=0.383062in,,top]{\color{textcolor}\rmfamily\fontsize{10.000000}{12.000000}\selectfont \(\displaystyle {2.5}\)}%
\end{pgfscope}%
\begin{pgfscope}%
\pgfsetbuttcap%
\pgfsetroundjoin%
\definecolor{currentfill}{rgb}{0.000000,0.000000,0.000000}%
\pgfsetfillcolor{currentfill}%
\pgfsetlinewidth{0.501875pt}%
\definecolor{currentstroke}{rgb}{0.000000,0.000000,0.000000}%
\pgfsetstrokecolor{currentstroke}%
\pgfsetdash{}{0pt}%
\pgfsys@defobject{currentmarker}{\pgfqpoint{0.000000in}{0.000000in}}{\pgfqpoint{0.000000in}{0.041667in}}{%
\pgfpathmoveto{\pgfqpoint{0.000000in}{0.000000in}}%
\pgfpathlineto{\pgfqpoint{0.000000in}{0.041667in}}%
\pgfusepath{stroke,fill}%
}%
\begin{pgfscope}%
\pgfsys@transformshift{3.525963in}{0.431673in}%
\pgfsys@useobject{currentmarker}{}%
\end{pgfscope}%
\end{pgfscope}%
\begin{pgfscope}%
\pgfsetbuttcap%
\pgfsetroundjoin%
\definecolor{currentfill}{rgb}{0.000000,0.000000,0.000000}%
\pgfsetfillcolor{currentfill}%
\pgfsetlinewidth{0.501875pt}%
\definecolor{currentstroke}{rgb}{0.000000,0.000000,0.000000}%
\pgfsetstrokecolor{currentstroke}%
\pgfsetdash{}{0pt}%
\pgfsys@defobject{currentmarker}{\pgfqpoint{0.000000in}{-0.041667in}}{\pgfqpoint{0.000000in}{0.000000in}}{%
\pgfpathmoveto{\pgfqpoint{0.000000in}{0.000000in}}%
\pgfpathlineto{\pgfqpoint{0.000000in}{-0.041667in}}%
\pgfusepath{stroke,fill}%
}%
\begin{pgfscope}%
\pgfsys@transformshift{3.525963in}{1.507795in}%
\pgfsys@useobject{currentmarker}{}%
\end{pgfscope}%
\end{pgfscope}%
\begin{pgfscope}%
\definecolor{textcolor}{rgb}{0.000000,0.000000,0.000000}%
\pgfsetstrokecolor{textcolor}%
\pgfsetfillcolor{textcolor}%
\pgftext[x=3.525963in,y=0.383062in,,top]{\color{textcolor}\rmfamily\fontsize{10.000000}{12.000000}\selectfont \(\displaystyle {5.0}\)}%
\end{pgfscope}%
\begin{pgfscope}%
\pgfsetbuttcap%
\pgfsetroundjoin%
\definecolor{currentfill}{rgb}{0.000000,0.000000,0.000000}%
\pgfsetfillcolor{currentfill}%
\pgfsetlinewidth{0.501875pt}%
\definecolor{currentstroke}{rgb}{0.000000,0.000000,0.000000}%
\pgfsetstrokecolor{currentstroke}%
\pgfsetdash{}{0pt}%
\pgfsys@defobject{currentmarker}{\pgfqpoint{0.000000in}{0.000000in}}{\pgfqpoint{0.000000in}{0.041667in}}{%
\pgfpathmoveto{\pgfqpoint{0.000000in}{0.000000in}}%
\pgfpathlineto{\pgfqpoint{0.000000in}{0.041667in}}%
\pgfusepath{stroke,fill}%
}%
\begin{pgfscope}%
\pgfsys@transformshift{4.006426in}{0.431673in}%
\pgfsys@useobject{currentmarker}{}%
\end{pgfscope}%
\end{pgfscope}%
\begin{pgfscope}%
\pgfsetbuttcap%
\pgfsetroundjoin%
\definecolor{currentfill}{rgb}{0.000000,0.000000,0.000000}%
\pgfsetfillcolor{currentfill}%
\pgfsetlinewidth{0.501875pt}%
\definecolor{currentstroke}{rgb}{0.000000,0.000000,0.000000}%
\pgfsetstrokecolor{currentstroke}%
\pgfsetdash{}{0pt}%
\pgfsys@defobject{currentmarker}{\pgfqpoint{0.000000in}{-0.041667in}}{\pgfqpoint{0.000000in}{0.000000in}}{%
\pgfpathmoveto{\pgfqpoint{0.000000in}{0.000000in}}%
\pgfpathlineto{\pgfqpoint{0.000000in}{-0.041667in}}%
\pgfusepath{stroke,fill}%
}%
\begin{pgfscope}%
\pgfsys@transformshift{4.006426in}{1.507795in}%
\pgfsys@useobject{currentmarker}{}%
\end{pgfscope}%
\end{pgfscope}%
\begin{pgfscope}%
\definecolor{textcolor}{rgb}{0.000000,0.000000,0.000000}%
\pgfsetstrokecolor{textcolor}%
\pgfsetfillcolor{textcolor}%
\pgftext[x=4.006426in,y=0.383062in,,top]{\color{textcolor}\rmfamily\fontsize{10.000000}{12.000000}\selectfont \(\displaystyle {7.5}\)}%
\end{pgfscope}%
\begin{pgfscope}%
\pgfsetbuttcap%
\pgfsetroundjoin%
\definecolor{currentfill}{rgb}{0.000000,0.000000,0.000000}%
\pgfsetfillcolor{currentfill}%
\pgfsetlinewidth{0.501875pt}%
\definecolor{currentstroke}{rgb}{0.000000,0.000000,0.000000}%
\pgfsetstrokecolor{currentstroke}%
\pgfsetdash{}{0pt}%
\pgfsys@defobject{currentmarker}{\pgfqpoint{0.000000in}{0.000000in}}{\pgfqpoint{0.000000in}{0.041667in}}{%
\pgfpathmoveto{\pgfqpoint{0.000000in}{0.000000in}}%
\pgfpathlineto{\pgfqpoint{0.000000in}{0.041667in}}%
\pgfusepath{stroke,fill}%
}%
\begin{pgfscope}%
\pgfsys@transformshift{4.486890in}{0.431673in}%
\pgfsys@useobject{currentmarker}{}%
\end{pgfscope}%
\end{pgfscope}%
\begin{pgfscope}%
\pgfsetbuttcap%
\pgfsetroundjoin%
\definecolor{currentfill}{rgb}{0.000000,0.000000,0.000000}%
\pgfsetfillcolor{currentfill}%
\pgfsetlinewidth{0.501875pt}%
\definecolor{currentstroke}{rgb}{0.000000,0.000000,0.000000}%
\pgfsetstrokecolor{currentstroke}%
\pgfsetdash{}{0pt}%
\pgfsys@defobject{currentmarker}{\pgfqpoint{0.000000in}{-0.041667in}}{\pgfqpoint{0.000000in}{0.000000in}}{%
\pgfpathmoveto{\pgfqpoint{0.000000in}{0.000000in}}%
\pgfpathlineto{\pgfqpoint{0.000000in}{-0.041667in}}%
\pgfusepath{stroke,fill}%
}%
\begin{pgfscope}%
\pgfsys@transformshift{4.486890in}{1.507795in}%
\pgfsys@useobject{currentmarker}{}%
\end{pgfscope}%
\end{pgfscope}%
\begin{pgfscope}%
\definecolor{textcolor}{rgb}{0.000000,0.000000,0.000000}%
\pgfsetstrokecolor{textcolor}%
\pgfsetfillcolor{textcolor}%
\pgftext[x=4.486890in,y=0.383062in,,top]{\color{textcolor}\rmfamily\fontsize{10.000000}{12.000000}\selectfont \(\displaystyle {10.0}\)}%
\end{pgfscope}%
\begin{pgfscope}%
\pgfsetbuttcap%
\pgfsetroundjoin%
\definecolor{currentfill}{rgb}{0.000000,0.000000,0.000000}%
\pgfsetfillcolor{currentfill}%
\pgfsetlinewidth{0.501875pt}%
\definecolor{currentstroke}{rgb}{0.000000,0.000000,0.000000}%
\pgfsetstrokecolor{currentstroke}%
\pgfsetdash{}{0pt}%
\pgfsys@defobject{currentmarker}{\pgfqpoint{0.000000in}{0.000000in}}{\pgfqpoint{0.000000in}{0.020833in}}{%
\pgfpathmoveto{\pgfqpoint{0.000000in}{0.000000in}}%
\pgfpathlineto{\pgfqpoint{0.000000in}{0.020833in}}%
\pgfusepath{stroke,fill}%
}%
\begin{pgfscope}%
\pgfsys@transformshift{0.450996in}{0.431673in}%
\pgfsys@useobject{currentmarker}{}%
\end{pgfscope}%
\end{pgfscope}%
\begin{pgfscope}%
\pgfsetbuttcap%
\pgfsetroundjoin%
\definecolor{currentfill}{rgb}{0.000000,0.000000,0.000000}%
\pgfsetfillcolor{currentfill}%
\pgfsetlinewidth{0.501875pt}%
\definecolor{currentstroke}{rgb}{0.000000,0.000000,0.000000}%
\pgfsetstrokecolor{currentstroke}%
\pgfsetdash{}{0pt}%
\pgfsys@defobject{currentmarker}{\pgfqpoint{0.000000in}{-0.020833in}}{\pgfqpoint{0.000000in}{0.000000in}}{%
\pgfpathmoveto{\pgfqpoint{0.000000in}{0.000000in}}%
\pgfpathlineto{\pgfqpoint{0.000000in}{-0.020833in}}%
\pgfusepath{stroke,fill}%
}%
\begin{pgfscope}%
\pgfsys@transformshift{0.450996in}{1.507795in}%
\pgfsys@useobject{currentmarker}{}%
\end{pgfscope}%
\end{pgfscope}%
\begin{pgfscope}%
\pgfsetbuttcap%
\pgfsetroundjoin%
\definecolor{currentfill}{rgb}{0.000000,0.000000,0.000000}%
\pgfsetfillcolor{currentfill}%
\pgfsetlinewidth{0.501875pt}%
\definecolor{currentstroke}{rgb}{0.000000,0.000000,0.000000}%
\pgfsetstrokecolor{currentstroke}%
\pgfsetdash{}{0pt}%
\pgfsys@defobject{currentmarker}{\pgfqpoint{0.000000in}{0.000000in}}{\pgfqpoint{0.000000in}{0.020833in}}{%
\pgfpathmoveto{\pgfqpoint{0.000000in}{0.000000in}}%
\pgfpathlineto{\pgfqpoint{0.000000in}{0.020833in}}%
\pgfusepath{stroke,fill}%
}%
\begin{pgfscope}%
\pgfsys@transformshift{0.547089in}{0.431673in}%
\pgfsys@useobject{currentmarker}{}%
\end{pgfscope}%
\end{pgfscope}%
\begin{pgfscope}%
\pgfsetbuttcap%
\pgfsetroundjoin%
\definecolor{currentfill}{rgb}{0.000000,0.000000,0.000000}%
\pgfsetfillcolor{currentfill}%
\pgfsetlinewidth{0.501875pt}%
\definecolor{currentstroke}{rgb}{0.000000,0.000000,0.000000}%
\pgfsetstrokecolor{currentstroke}%
\pgfsetdash{}{0pt}%
\pgfsys@defobject{currentmarker}{\pgfqpoint{0.000000in}{-0.020833in}}{\pgfqpoint{0.000000in}{0.000000in}}{%
\pgfpathmoveto{\pgfqpoint{0.000000in}{0.000000in}}%
\pgfpathlineto{\pgfqpoint{0.000000in}{-0.020833in}}%
\pgfusepath{stroke,fill}%
}%
\begin{pgfscope}%
\pgfsys@transformshift{0.547089in}{1.507795in}%
\pgfsys@useobject{currentmarker}{}%
\end{pgfscope}%
\end{pgfscope}%
\begin{pgfscope}%
\pgfsetbuttcap%
\pgfsetroundjoin%
\definecolor{currentfill}{rgb}{0.000000,0.000000,0.000000}%
\pgfsetfillcolor{currentfill}%
\pgfsetlinewidth{0.501875pt}%
\definecolor{currentstroke}{rgb}{0.000000,0.000000,0.000000}%
\pgfsetstrokecolor{currentstroke}%
\pgfsetdash{}{0pt}%
\pgfsys@defobject{currentmarker}{\pgfqpoint{0.000000in}{0.000000in}}{\pgfqpoint{0.000000in}{0.020833in}}{%
\pgfpathmoveto{\pgfqpoint{0.000000in}{0.000000in}}%
\pgfpathlineto{\pgfqpoint{0.000000in}{0.020833in}}%
\pgfusepath{stroke,fill}%
}%
\begin{pgfscope}%
\pgfsys@transformshift{0.739275in}{0.431673in}%
\pgfsys@useobject{currentmarker}{}%
\end{pgfscope}%
\end{pgfscope}%
\begin{pgfscope}%
\pgfsetbuttcap%
\pgfsetroundjoin%
\definecolor{currentfill}{rgb}{0.000000,0.000000,0.000000}%
\pgfsetfillcolor{currentfill}%
\pgfsetlinewidth{0.501875pt}%
\definecolor{currentstroke}{rgb}{0.000000,0.000000,0.000000}%
\pgfsetstrokecolor{currentstroke}%
\pgfsetdash{}{0pt}%
\pgfsys@defobject{currentmarker}{\pgfqpoint{0.000000in}{-0.020833in}}{\pgfqpoint{0.000000in}{0.000000in}}{%
\pgfpathmoveto{\pgfqpoint{0.000000in}{0.000000in}}%
\pgfpathlineto{\pgfqpoint{0.000000in}{-0.020833in}}%
\pgfusepath{stroke,fill}%
}%
\begin{pgfscope}%
\pgfsys@transformshift{0.739275in}{1.507795in}%
\pgfsys@useobject{currentmarker}{}%
\end{pgfscope}%
\end{pgfscope}%
\begin{pgfscope}%
\pgfsetbuttcap%
\pgfsetroundjoin%
\definecolor{currentfill}{rgb}{0.000000,0.000000,0.000000}%
\pgfsetfillcolor{currentfill}%
\pgfsetlinewidth{0.501875pt}%
\definecolor{currentstroke}{rgb}{0.000000,0.000000,0.000000}%
\pgfsetstrokecolor{currentstroke}%
\pgfsetdash{}{0pt}%
\pgfsys@defobject{currentmarker}{\pgfqpoint{0.000000in}{0.000000in}}{\pgfqpoint{0.000000in}{0.020833in}}{%
\pgfpathmoveto{\pgfqpoint{0.000000in}{0.000000in}}%
\pgfpathlineto{\pgfqpoint{0.000000in}{0.020833in}}%
\pgfusepath{stroke,fill}%
}%
\begin{pgfscope}%
\pgfsys@transformshift{0.835367in}{0.431673in}%
\pgfsys@useobject{currentmarker}{}%
\end{pgfscope}%
\end{pgfscope}%
\begin{pgfscope}%
\pgfsetbuttcap%
\pgfsetroundjoin%
\definecolor{currentfill}{rgb}{0.000000,0.000000,0.000000}%
\pgfsetfillcolor{currentfill}%
\pgfsetlinewidth{0.501875pt}%
\definecolor{currentstroke}{rgb}{0.000000,0.000000,0.000000}%
\pgfsetstrokecolor{currentstroke}%
\pgfsetdash{}{0pt}%
\pgfsys@defobject{currentmarker}{\pgfqpoint{0.000000in}{-0.020833in}}{\pgfqpoint{0.000000in}{0.000000in}}{%
\pgfpathmoveto{\pgfqpoint{0.000000in}{0.000000in}}%
\pgfpathlineto{\pgfqpoint{0.000000in}{-0.020833in}}%
\pgfusepath{stroke,fill}%
}%
\begin{pgfscope}%
\pgfsys@transformshift{0.835367in}{1.507795in}%
\pgfsys@useobject{currentmarker}{}%
\end{pgfscope}%
\end{pgfscope}%
\begin{pgfscope}%
\pgfsetbuttcap%
\pgfsetroundjoin%
\definecolor{currentfill}{rgb}{0.000000,0.000000,0.000000}%
\pgfsetfillcolor{currentfill}%
\pgfsetlinewidth{0.501875pt}%
\definecolor{currentstroke}{rgb}{0.000000,0.000000,0.000000}%
\pgfsetstrokecolor{currentstroke}%
\pgfsetdash{}{0pt}%
\pgfsys@defobject{currentmarker}{\pgfqpoint{0.000000in}{0.000000in}}{\pgfqpoint{0.000000in}{0.020833in}}{%
\pgfpathmoveto{\pgfqpoint{0.000000in}{0.000000in}}%
\pgfpathlineto{\pgfqpoint{0.000000in}{0.020833in}}%
\pgfusepath{stroke,fill}%
}%
\begin{pgfscope}%
\pgfsys@transformshift{0.931460in}{0.431673in}%
\pgfsys@useobject{currentmarker}{}%
\end{pgfscope}%
\end{pgfscope}%
\begin{pgfscope}%
\pgfsetbuttcap%
\pgfsetroundjoin%
\definecolor{currentfill}{rgb}{0.000000,0.000000,0.000000}%
\pgfsetfillcolor{currentfill}%
\pgfsetlinewidth{0.501875pt}%
\definecolor{currentstroke}{rgb}{0.000000,0.000000,0.000000}%
\pgfsetstrokecolor{currentstroke}%
\pgfsetdash{}{0pt}%
\pgfsys@defobject{currentmarker}{\pgfqpoint{0.000000in}{-0.020833in}}{\pgfqpoint{0.000000in}{0.000000in}}{%
\pgfpathmoveto{\pgfqpoint{0.000000in}{0.000000in}}%
\pgfpathlineto{\pgfqpoint{0.000000in}{-0.020833in}}%
\pgfusepath{stroke,fill}%
}%
\begin{pgfscope}%
\pgfsys@transformshift{0.931460in}{1.507795in}%
\pgfsys@useobject{currentmarker}{}%
\end{pgfscope}%
\end{pgfscope}%
\begin{pgfscope}%
\pgfsetbuttcap%
\pgfsetroundjoin%
\definecolor{currentfill}{rgb}{0.000000,0.000000,0.000000}%
\pgfsetfillcolor{currentfill}%
\pgfsetlinewidth{0.501875pt}%
\definecolor{currentstroke}{rgb}{0.000000,0.000000,0.000000}%
\pgfsetstrokecolor{currentstroke}%
\pgfsetdash{}{0pt}%
\pgfsys@defobject{currentmarker}{\pgfqpoint{0.000000in}{0.000000in}}{\pgfqpoint{0.000000in}{0.020833in}}{%
\pgfpathmoveto{\pgfqpoint{0.000000in}{0.000000in}}%
\pgfpathlineto{\pgfqpoint{0.000000in}{0.020833in}}%
\pgfusepath{stroke,fill}%
}%
\begin{pgfscope}%
\pgfsys@transformshift{1.027553in}{0.431673in}%
\pgfsys@useobject{currentmarker}{}%
\end{pgfscope}%
\end{pgfscope}%
\begin{pgfscope}%
\pgfsetbuttcap%
\pgfsetroundjoin%
\definecolor{currentfill}{rgb}{0.000000,0.000000,0.000000}%
\pgfsetfillcolor{currentfill}%
\pgfsetlinewidth{0.501875pt}%
\definecolor{currentstroke}{rgb}{0.000000,0.000000,0.000000}%
\pgfsetstrokecolor{currentstroke}%
\pgfsetdash{}{0pt}%
\pgfsys@defobject{currentmarker}{\pgfqpoint{0.000000in}{-0.020833in}}{\pgfqpoint{0.000000in}{0.000000in}}{%
\pgfpathmoveto{\pgfqpoint{0.000000in}{0.000000in}}%
\pgfpathlineto{\pgfqpoint{0.000000in}{-0.020833in}}%
\pgfusepath{stroke,fill}%
}%
\begin{pgfscope}%
\pgfsys@transformshift{1.027553in}{1.507795in}%
\pgfsys@useobject{currentmarker}{}%
\end{pgfscope}%
\end{pgfscope}%
\begin{pgfscope}%
\pgfsetbuttcap%
\pgfsetroundjoin%
\definecolor{currentfill}{rgb}{0.000000,0.000000,0.000000}%
\pgfsetfillcolor{currentfill}%
\pgfsetlinewidth{0.501875pt}%
\definecolor{currentstroke}{rgb}{0.000000,0.000000,0.000000}%
\pgfsetstrokecolor{currentstroke}%
\pgfsetdash{}{0pt}%
\pgfsys@defobject{currentmarker}{\pgfqpoint{0.000000in}{0.000000in}}{\pgfqpoint{0.000000in}{0.020833in}}{%
\pgfpathmoveto{\pgfqpoint{0.000000in}{0.000000in}}%
\pgfpathlineto{\pgfqpoint{0.000000in}{0.020833in}}%
\pgfusepath{stroke,fill}%
}%
\begin{pgfscope}%
\pgfsys@transformshift{1.219738in}{0.431673in}%
\pgfsys@useobject{currentmarker}{}%
\end{pgfscope}%
\end{pgfscope}%
\begin{pgfscope}%
\pgfsetbuttcap%
\pgfsetroundjoin%
\definecolor{currentfill}{rgb}{0.000000,0.000000,0.000000}%
\pgfsetfillcolor{currentfill}%
\pgfsetlinewidth{0.501875pt}%
\definecolor{currentstroke}{rgb}{0.000000,0.000000,0.000000}%
\pgfsetstrokecolor{currentstroke}%
\pgfsetdash{}{0pt}%
\pgfsys@defobject{currentmarker}{\pgfqpoint{0.000000in}{-0.020833in}}{\pgfqpoint{0.000000in}{0.000000in}}{%
\pgfpathmoveto{\pgfqpoint{0.000000in}{0.000000in}}%
\pgfpathlineto{\pgfqpoint{0.000000in}{-0.020833in}}%
\pgfusepath{stroke,fill}%
}%
\begin{pgfscope}%
\pgfsys@transformshift{1.219738in}{1.507795in}%
\pgfsys@useobject{currentmarker}{}%
\end{pgfscope}%
\end{pgfscope}%
\begin{pgfscope}%
\pgfsetbuttcap%
\pgfsetroundjoin%
\definecolor{currentfill}{rgb}{0.000000,0.000000,0.000000}%
\pgfsetfillcolor{currentfill}%
\pgfsetlinewidth{0.501875pt}%
\definecolor{currentstroke}{rgb}{0.000000,0.000000,0.000000}%
\pgfsetstrokecolor{currentstroke}%
\pgfsetdash{}{0pt}%
\pgfsys@defobject{currentmarker}{\pgfqpoint{0.000000in}{0.000000in}}{\pgfqpoint{0.000000in}{0.020833in}}{%
\pgfpathmoveto{\pgfqpoint{0.000000in}{0.000000in}}%
\pgfpathlineto{\pgfqpoint{0.000000in}{0.020833in}}%
\pgfusepath{stroke,fill}%
}%
\begin{pgfscope}%
\pgfsys@transformshift{1.315831in}{0.431673in}%
\pgfsys@useobject{currentmarker}{}%
\end{pgfscope}%
\end{pgfscope}%
\begin{pgfscope}%
\pgfsetbuttcap%
\pgfsetroundjoin%
\definecolor{currentfill}{rgb}{0.000000,0.000000,0.000000}%
\pgfsetfillcolor{currentfill}%
\pgfsetlinewidth{0.501875pt}%
\definecolor{currentstroke}{rgb}{0.000000,0.000000,0.000000}%
\pgfsetstrokecolor{currentstroke}%
\pgfsetdash{}{0pt}%
\pgfsys@defobject{currentmarker}{\pgfqpoint{0.000000in}{-0.020833in}}{\pgfqpoint{0.000000in}{0.000000in}}{%
\pgfpathmoveto{\pgfqpoint{0.000000in}{0.000000in}}%
\pgfpathlineto{\pgfqpoint{0.000000in}{-0.020833in}}%
\pgfusepath{stroke,fill}%
}%
\begin{pgfscope}%
\pgfsys@transformshift{1.315831in}{1.507795in}%
\pgfsys@useobject{currentmarker}{}%
\end{pgfscope}%
\end{pgfscope}%
\begin{pgfscope}%
\pgfsetbuttcap%
\pgfsetroundjoin%
\definecolor{currentfill}{rgb}{0.000000,0.000000,0.000000}%
\pgfsetfillcolor{currentfill}%
\pgfsetlinewidth{0.501875pt}%
\definecolor{currentstroke}{rgb}{0.000000,0.000000,0.000000}%
\pgfsetstrokecolor{currentstroke}%
\pgfsetdash{}{0pt}%
\pgfsys@defobject{currentmarker}{\pgfqpoint{0.000000in}{0.000000in}}{\pgfqpoint{0.000000in}{0.020833in}}{%
\pgfpathmoveto{\pgfqpoint{0.000000in}{0.000000in}}%
\pgfpathlineto{\pgfqpoint{0.000000in}{0.020833in}}%
\pgfusepath{stroke,fill}%
}%
\begin{pgfscope}%
\pgfsys@transformshift{1.411923in}{0.431673in}%
\pgfsys@useobject{currentmarker}{}%
\end{pgfscope}%
\end{pgfscope}%
\begin{pgfscope}%
\pgfsetbuttcap%
\pgfsetroundjoin%
\definecolor{currentfill}{rgb}{0.000000,0.000000,0.000000}%
\pgfsetfillcolor{currentfill}%
\pgfsetlinewidth{0.501875pt}%
\definecolor{currentstroke}{rgb}{0.000000,0.000000,0.000000}%
\pgfsetstrokecolor{currentstroke}%
\pgfsetdash{}{0pt}%
\pgfsys@defobject{currentmarker}{\pgfqpoint{0.000000in}{-0.020833in}}{\pgfqpoint{0.000000in}{0.000000in}}{%
\pgfpathmoveto{\pgfqpoint{0.000000in}{0.000000in}}%
\pgfpathlineto{\pgfqpoint{0.000000in}{-0.020833in}}%
\pgfusepath{stroke,fill}%
}%
\begin{pgfscope}%
\pgfsys@transformshift{1.411923in}{1.507795in}%
\pgfsys@useobject{currentmarker}{}%
\end{pgfscope}%
\end{pgfscope}%
\begin{pgfscope}%
\pgfsetbuttcap%
\pgfsetroundjoin%
\definecolor{currentfill}{rgb}{0.000000,0.000000,0.000000}%
\pgfsetfillcolor{currentfill}%
\pgfsetlinewidth{0.501875pt}%
\definecolor{currentstroke}{rgb}{0.000000,0.000000,0.000000}%
\pgfsetstrokecolor{currentstroke}%
\pgfsetdash{}{0pt}%
\pgfsys@defobject{currentmarker}{\pgfqpoint{0.000000in}{0.000000in}}{\pgfqpoint{0.000000in}{0.020833in}}{%
\pgfpathmoveto{\pgfqpoint{0.000000in}{0.000000in}}%
\pgfpathlineto{\pgfqpoint{0.000000in}{0.020833in}}%
\pgfusepath{stroke,fill}%
}%
\begin{pgfscope}%
\pgfsys@transformshift{1.508016in}{0.431673in}%
\pgfsys@useobject{currentmarker}{}%
\end{pgfscope}%
\end{pgfscope}%
\begin{pgfscope}%
\pgfsetbuttcap%
\pgfsetroundjoin%
\definecolor{currentfill}{rgb}{0.000000,0.000000,0.000000}%
\pgfsetfillcolor{currentfill}%
\pgfsetlinewidth{0.501875pt}%
\definecolor{currentstroke}{rgb}{0.000000,0.000000,0.000000}%
\pgfsetstrokecolor{currentstroke}%
\pgfsetdash{}{0pt}%
\pgfsys@defobject{currentmarker}{\pgfqpoint{0.000000in}{-0.020833in}}{\pgfqpoint{0.000000in}{0.000000in}}{%
\pgfpathmoveto{\pgfqpoint{0.000000in}{0.000000in}}%
\pgfpathlineto{\pgfqpoint{0.000000in}{-0.020833in}}%
\pgfusepath{stroke,fill}%
}%
\begin{pgfscope}%
\pgfsys@transformshift{1.508016in}{1.507795in}%
\pgfsys@useobject{currentmarker}{}%
\end{pgfscope}%
\end{pgfscope}%
\begin{pgfscope}%
\pgfsetbuttcap%
\pgfsetroundjoin%
\definecolor{currentfill}{rgb}{0.000000,0.000000,0.000000}%
\pgfsetfillcolor{currentfill}%
\pgfsetlinewidth{0.501875pt}%
\definecolor{currentstroke}{rgb}{0.000000,0.000000,0.000000}%
\pgfsetstrokecolor{currentstroke}%
\pgfsetdash{}{0pt}%
\pgfsys@defobject{currentmarker}{\pgfqpoint{0.000000in}{0.000000in}}{\pgfqpoint{0.000000in}{0.020833in}}{%
\pgfpathmoveto{\pgfqpoint{0.000000in}{0.000000in}}%
\pgfpathlineto{\pgfqpoint{0.000000in}{0.020833in}}%
\pgfusepath{stroke,fill}%
}%
\begin{pgfscope}%
\pgfsys@transformshift{1.700201in}{0.431673in}%
\pgfsys@useobject{currentmarker}{}%
\end{pgfscope}%
\end{pgfscope}%
\begin{pgfscope}%
\pgfsetbuttcap%
\pgfsetroundjoin%
\definecolor{currentfill}{rgb}{0.000000,0.000000,0.000000}%
\pgfsetfillcolor{currentfill}%
\pgfsetlinewidth{0.501875pt}%
\definecolor{currentstroke}{rgb}{0.000000,0.000000,0.000000}%
\pgfsetstrokecolor{currentstroke}%
\pgfsetdash{}{0pt}%
\pgfsys@defobject{currentmarker}{\pgfqpoint{0.000000in}{-0.020833in}}{\pgfqpoint{0.000000in}{0.000000in}}{%
\pgfpathmoveto{\pgfqpoint{0.000000in}{0.000000in}}%
\pgfpathlineto{\pgfqpoint{0.000000in}{-0.020833in}}%
\pgfusepath{stroke,fill}%
}%
\begin{pgfscope}%
\pgfsys@transformshift{1.700201in}{1.507795in}%
\pgfsys@useobject{currentmarker}{}%
\end{pgfscope}%
\end{pgfscope}%
\begin{pgfscope}%
\pgfsetbuttcap%
\pgfsetroundjoin%
\definecolor{currentfill}{rgb}{0.000000,0.000000,0.000000}%
\pgfsetfillcolor{currentfill}%
\pgfsetlinewidth{0.501875pt}%
\definecolor{currentstroke}{rgb}{0.000000,0.000000,0.000000}%
\pgfsetstrokecolor{currentstroke}%
\pgfsetdash{}{0pt}%
\pgfsys@defobject{currentmarker}{\pgfqpoint{0.000000in}{0.000000in}}{\pgfqpoint{0.000000in}{0.020833in}}{%
\pgfpathmoveto{\pgfqpoint{0.000000in}{0.000000in}}%
\pgfpathlineto{\pgfqpoint{0.000000in}{0.020833in}}%
\pgfusepath{stroke,fill}%
}%
\begin{pgfscope}%
\pgfsys@transformshift{1.796294in}{0.431673in}%
\pgfsys@useobject{currentmarker}{}%
\end{pgfscope}%
\end{pgfscope}%
\begin{pgfscope}%
\pgfsetbuttcap%
\pgfsetroundjoin%
\definecolor{currentfill}{rgb}{0.000000,0.000000,0.000000}%
\pgfsetfillcolor{currentfill}%
\pgfsetlinewidth{0.501875pt}%
\definecolor{currentstroke}{rgb}{0.000000,0.000000,0.000000}%
\pgfsetstrokecolor{currentstroke}%
\pgfsetdash{}{0pt}%
\pgfsys@defobject{currentmarker}{\pgfqpoint{0.000000in}{-0.020833in}}{\pgfqpoint{0.000000in}{0.000000in}}{%
\pgfpathmoveto{\pgfqpoint{0.000000in}{0.000000in}}%
\pgfpathlineto{\pgfqpoint{0.000000in}{-0.020833in}}%
\pgfusepath{stroke,fill}%
}%
\begin{pgfscope}%
\pgfsys@transformshift{1.796294in}{1.507795in}%
\pgfsys@useobject{currentmarker}{}%
\end{pgfscope}%
\end{pgfscope}%
\begin{pgfscope}%
\pgfsetbuttcap%
\pgfsetroundjoin%
\definecolor{currentfill}{rgb}{0.000000,0.000000,0.000000}%
\pgfsetfillcolor{currentfill}%
\pgfsetlinewidth{0.501875pt}%
\definecolor{currentstroke}{rgb}{0.000000,0.000000,0.000000}%
\pgfsetstrokecolor{currentstroke}%
\pgfsetdash{}{0pt}%
\pgfsys@defobject{currentmarker}{\pgfqpoint{0.000000in}{0.000000in}}{\pgfqpoint{0.000000in}{0.020833in}}{%
\pgfpathmoveto{\pgfqpoint{0.000000in}{0.000000in}}%
\pgfpathlineto{\pgfqpoint{0.000000in}{0.020833in}}%
\pgfusepath{stroke,fill}%
}%
\begin{pgfscope}%
\pgfsys@transformshift{1.892387in}{0.431673in}%
\pgfsys@useobject{currentmarker}{}%
\end{pgfscope}%
\end{pgfscope}%
\begin{pgfscope}%
\pgfsetbuttcap%
\pgfsetroundjoin%
\definecolor{currentfill}{rgb}{0.000000,0.000000,0.000000}%
\pgfsetfillcolor{currentfill}%
\pgfsetlinewidth{0.501875pt}%
\definecolor{currentstroke}{rgb}{0.000000,0.000000,0.000000}%
\pgfsetstrokecolor{currentstroke}%
\pgfsetdash{}{0pt}%
\pgfsys@defobject{currentmarker}{\pgfqpoint{0.000000in}{-0.020833in}}{\pgfqpoint{0.000000in}{0.000000in}}{%
\pgfpathmoveto{\pgfqpoint{0.000000in}{0.000000in}}%
\pgfpathlineto{\pgfqpoint{0.000000in}{-0.020833in}}%
\pgfusepath{stroke,fill}%
}%
\begin{pgfscope}%
\pgfsys@transformshift{1.892387in}{1.507795in}%
\pgfsys@useobject{currentmarker}{}%
\end{pgfscope}%
\end{pgfscope}%
\begin{pgfscope}%
\pgfsetbuttcap%
\pgfsetroundjoin%
\definecolor{currentfill}{rgb}{0.000000,0.000000,0.000000}%
\pgfsetfillcolor{currentfill}%
\pgfsetlinewidth{0.501875pt}%
\definecolor{currentstroke}{rgb}{0.000000,0.000000,0.000000}%
\pgfsetstrokecolor{currentstroke}%
\pgfsetdash{}{0pt}%
\pgfsys@defobject{currentmarker}{\pgfqpoint{0.000000in}{0.000000in}}{\pgfqpoint{0.000000in}{0.020833in}}{%
\pgfpathmoveto{\pgfqpoint{0.000000in}{0.000000in}}%
\pgfpathlineto{\pgfqpoint{0.000000in}{0.020833in}}%
\pgfusepath{stroke,fill}%
}%
\begin{pgfscope}%
\pgfsys@transformshift{1.988480in}{0.431673in}%
\pgfsys@useobject{currentmarker}{}%
\end{pgfscope}%
\end{pgfscope}%
\begin{pgfscope}%
\pgfsetbuttcap%
\pgfsetroundjoin%
\definecolor{currentfill}{rgb}{0.000000,0.000000,0.000000}%
\pgfsetfillcolor{currentfill}%
\pgfsetlinewidth{0.501875pt}%
\definecolor{currentstroke}{rgb}{0.000000,0.000000,0.000000}%
\pgfsetstrokecolor{currentstroke}%
\pgfsetdash{}{0pt}%
\pgfsys@defobject{currentmarker}{\pgfqpoint{0.000000in}{-0.020833in}}{\pgfqpoint{0.000000in}{0.000000in}}{%
\pgfpathmoveto{\pgfqpoint{0.000000in}{0.000000in}}%
\pgfpathlineto{\pgfqpoint{0.000000in}{-0.020833in}}%
\pgfusepath{stroke,fill}%
}%
\begin{pgfscope}%
\pgfsys@transformshift{1.988480in}{1.507795in}%
\pgfsys@useobject{currentmarker}{}%
\end{pgfscope}%
\end{pgfscope}%
\begin{pgfscope}%
\pgfsetbuttcap%
\pgfsetroundjoin%
\definecolor{currentfill}{rgb}{0.000000,0.000000,0.000000}%
\pgfsetfillcolor{currentfill}%
\pgfsetlinewidth{0.501875pt}%
\definecolor{currentstroke}{rgb}{0.000000,0.000000,0.000000}%
\pgfsetstrokecolor{currentstroke}%
\pgfsetdash{}{0pt}%
\pgfsys@defobject{currentmarker}{\pgfqpoint{0.000000in}{0.000000in}}{\pgfqpoint{0.000000in}{0.020833in}}{%
\pgfpathmoveto{\pgfqpoint{0.000000in}{0.000000in}}%
\pgfpathlineto{\pgfqpoint{0.000000in}{0.020833in}}%
\pgfusepath{stroke,fill}%
}%
\begin{pgfscope}%
\pgfsys@transformshift{2.180665in}{0.431673in}%
\pgfsys@useobject{currentmarker}{}%
\end{pgfscope}%
\end{pgfscope}%
\begin{pgfscope}%
\pgfsetbuttcap%
\pgfsetroundjoin%
\definecolor{currentfill}{rgb}{0.000000,0.000000,0.000000}%
\pgfsetfillcolor{currentfill}%
\pgfsetlinewidth{0.501875pt}%
\definecolor{currentstroke}{rgb}{0.000000,0.000000,0.000000}%
\pgfsetstrokecolor{currentstroke}%
\pgfsetdash{}{0pt}%
\pgfsys@defobject{currentmarker}{\pgfqpoint{0.000000in}{-0.020833in}}{\pgfqpoint{0.000000in}{0.000000in}}{%
\pgfpathmoveto{\pgfqpoint{0.000000in}{0.000000in}}%
\pgfpathlineto{\pgfqpoint{0.000000in}{-0.020833in}}%
\pgfusepath{stroke,fill}%
}%
\begin{pgfscope}%
\pgfsys@transformshift{2.180665in}{1.507795in}%
\pgfsys@useobject{currentmarker}{}%
\end{pgfscope}%
\end{pgfscope}%
\begin{pgfscope}%
\pgfsetbuttcap%
\pgfsetroundjoin%
\definecolor{currentfill}{rgb}{0.000000,0.000000,0.000000}%
\pgfsetfillcolor{currentfill}%
\pgfsetlinewidth{0.501875pt}%
\definecolor{currentstroke}{rgb}{0.000000,0.000000,0.000000}%
\pgfsetstrokecolor{currentstroke}%
\pgfsetdash{}{0pt}%
\pgfsys@defobject{currentmarker}{\pgfqpoint{0.000000in}{0.000000in}}{\pgfqpoint{0.000000in}{0.020833in}}{%
\pgfpathmoveto{\pgfqpoint{0.000000in}{0.000000in}}%
\pgfpathlineto{\pgfqpoint{0.000000in}{0.020833in}}%
\pgfusepath{stroke,fill}%
}%
\begin{pgfscope}%
\pgfsys@transformshift{2.276758in}{0.431673in}%
\pgfsys@useobject{currentmarker}{}%
\end{pgfscope}%
\end{pgfscope}%
\begin{pgfscope}%
\pgfsetbuttcap%
\pgfsetroundjoin%
\definecolor{currentfill}{rgb}{0.000000,0.000000,0.000000}%
\pgfsetfillcolor{currentfill}%
\pgfsetlinewidth{0.501875pt}%
\definecolor{currentstroke}{rgb}{0.000000,0.000000,0.000000}%
\pgfsetstrokecolor{currentstroke}%
\pgfsetdash{}{0pt}%
\pgfsys@defobject{currentmarker}{\pgfqpoint{0.000000in}{-0.020833in}}{\pgfqpoint{0.000000in}{0.000000in}}{%
\pgfpathmoveto{\pgfqpoint{0.000000in}{0.000000in}}%
\pgfpathlineto{\pgfqpoint{0.000000in}{-0.020833in}}%
\pgfusepath{stroke,fill}%
}%
\begin{pgfscope}%
\pgfsys@transformshift{2.276758in}{1.507795in}%
\pgfsys@useobject{currentmarker}{}%
\end{pgfscope}%
\end{pgfscope}%
\begin{pgfscope}%
\pgfsetbuttcap%
\pgfsetroundjoin%
\definecolor{currentfill}{rgb}{0.000000,0.000000,0.000000}%
\pgfsetfillcolor{currentfill}%
\pgfsetlinewidth{0.501875pt}%
\definecolor{currentstroke}{rgb}{0.000000,0.000000,0.000000}%
\pgfsetstrokecolor{currentstroke}%
\pgfsetdash{}{0pt}%
\pgfsys@defobject{currentmarker}{\pgfqpoint{0.000000in}{0.000000in}}{\pgfqpoint{0.000000in}{0.020833in}}{%
\pgfpathmoveto{\pgfqpoint{0.000000in}{0.000000in}}%
\pgfpathlineto{\pgfqpoint{0.000000in}{0.020833in}}%
\pgfusepath{stroke,fill}%
}%
\begin{pgfscope}%
\pgfsys@transformshift{2.372850in}{0.431673in}%
\pgfsys@useobject{currentmarker}{}%
\end{pgfscope}%
\end{pgfscope}%
\begin{pgfscope}%
\pgfsetbuttcap%
\pgfsetroundjoin%
\definecolor{currentfill}{rgb}{0.000000,0.000000,0.000000}%
\pgfsetfillcolor{currentfill}%
\pgfsetlinewidth{0.501875pt}%
\definecolor{currentstroke}{rgb}{0.000000,0.000000,0.000000}%
\pgfsetstrokecolor{currentstroke}%
\pgfsetdash{}{0pt}%
\pgfsys@defobject{currentmarker}{\pgfqpoint{0.000000in}{-0.020833in}}{\pgfqpoint{0.000000in}{0.000000in}}{%
\pgfpathmoveto{\pgfqpoint{0.000000in}{0.000000in}}%
\pgfpathlineto{\pgfqpoint{0.000000in}{-0.020833in}}%
\pgfusepath{stroke,fill}%
}%
\begin{pgfscope}%
\pgfsys@transformshift{2.372850in}{1.507795in}%
\pgfsys@useobject{currentmarker}{}%
\end{pgfscope}%
\end{pgfscope}%
\begin{pgfscope}%
\pgfsetbuttcap%
\pgfsetroundjoin%
\definecolor{currentfill}{rgb}{0.000000,0.000000,0.000000}%
\pgfsetfillcolor{currentfill}%
\pgfsetlinewidth{0.501875pt}%
\definecolor{currentstroke}{rgb}{0.000000,0.000000,0.000000}%
\pgfsetstrokecolor{currentstroke}%
\pgfsetdash{}{0pt}%
\pgfsys@defobject{currentmarker}{\pgfqpoint{0.000000in}{0.000000in}}{\pgfqpoint{0.000000in}{0.020833in}}{%
\pgfpathmoveto{\pgfqpoint{0.000000in}{0.000000in}}%
\pgfpathlineto{\pgfqpoint{0.000000in}{0.020833in}}%
\pgfusepath{stroke,fill}%
}%
\begin{pgfscope}%
\pgfsys@transformshift{2.468943in}{0.431673in}%
\pgfsys@useobject{currentmarker}{}%
\end{pgfscope}%
\end{pgfscope}%
\begin{pgfscope}%
\pgfsetbuttcap%
\pgfsetroundjoin%
\definecolor{currentfill}{rgb}{0.000000,0.000000,0.000000}%
\pgfsetfillcolor{currentfill}%
\pgfsetlinewidth{0.501875pt}%
\definecolor{currentstroke}{rgb}{0.000000,0.000000,0.000000}%
\pgfsetstrokecolor{currentstroke}%
\pgfsetdash{}{0pt}%
\pgfsys@defobject{currentmarker}{\pgfqpoint{0.000000in}{-0.020833in}}{\pgfqpoint{0.000000in}{0.000000in}}{%
\pgfpathmoveto{\pgfqpoint{0.000000in}{0.000000in}}%
\pgfpathlineto{\pgfqpoint{0.000000in}{-0.020833in}}%
\pgfusepath{stroke,fill}%
}%
\begin{pgfscope}%
\pgfsys@transformshift{2.468943in}{1.507795in}%
\pgfsys@useobject{currentmarker}{}%
\end{pgfscope}%
\end{pgfscope}%
\begin{pgfscope}%
\pgfsetbuttcap%
\pgfsetroundjoin%
\definecolor{currentfill}{rgb}{0.000000,0.000000,0.000000}%
\pgfsetfillcolor{currentfill}%
\pgfsetlinewidth{0.501875pt}%
\definecolor{currentstroke}{rgb}{0.000000,0.000000,0.000000}%
\pgfsetstrokecolor{currentstroke}%
\pgfsetdash{}{0pt}%
\pgfsys@defobject{currentmarker}{\pgfqpoint{0.000000in}{0.000000in}}{\pgfqpoint{0.000000in}{0.020833in}}{%
\pgfpathmoveto{\pgfqpoint{0.000000in}{0.000000in}}%
\pgfpathlineto{\pgfqpoint{0.000000in}{0.020833in}}%
\pgfusepath{stroke,fill}%
}%
\begin{pgfscope}%
\pgfsys@transformshift{2.661128in}{0.431673in}%
\pgfsys@useobject{currentmarker}{}%
\end{pgfscope}%
\end{pgfscope}%
\begin{pgfscope}%
\pgfsetbuttcap%
\pgfsetroundjoin%
\definecolor{currentfill}{rgb}{0.000000,0.000000,0.000000}%
\pgfsetfillcolor{currentfill}%
\pgfsetlinewidth{0.501875pt}%
\definecolor{currentstroke}{rgb}{0.000000,0.000000,0.000000}%
\pgfsetstrokecolor{currentstroke}%
\pgfsetdash{}{0pt}%
\pgfsys@defobject{currentmarker}{\pgfqpoint{0.000000in}{-0.020833in}}{\pgfqpoint{0.000000in}{0.000000in}}{%
\pgfpathmoveto{\pgfqpoint{0.000000in}{0.000000in}}%
\pgfpathlineto{\pgfqpoint{0.000000in}{-0.020833in}}%
\pgfusepath{stroke,fill}%
}%
\begin{pgfscope}%
\pgfsys@transformshift{2.661128in}{1.507795in}%
\pgfsys@useobject{currentmarker}{}%
\end{pgfscope}%
\end{pgfscope}%
\begin{pgfscope}%
\pgfsetbuttcap%
\pgfsetroundjoin%
\definecolor{currentfill}{rgb}{0.000000,0.000000,0.000000}%
\pgfsetfillcolor{currentfill}%
\pgfsetlinewidth{0.501875pt}%
\definecolor{currentstroke}{rgb}{0.000000,0.000000,0.000000}%
\pgfsetstrokecolor{currentstroke}%
\pgfsetdash{}{0pt}%
\pgfsys@defobject{currentmarker}{\pgfqpoint{0.000000in}{0.000000in}}{\pgfqpoint{0.000000in}{0.020833in}}{%
\pgfpathmoveto{\pgfqpoint{0.000000in}{0.000000in}}%
\pgfpathlineto{\pgfqpoint{0.000000in}{0.020833in}}%
\pgfusepath{stroke,fill}%
}%
\begin{pgfscope}%
\pgfsys@transformshift{2.757221in}{0.431673in}%
\pgfsys@useobject{currentmarker}{}%
\end{pgfscope}%
\end{pgfscope}%
\begin{pgfscope}%
\pgfsetbuttcap%
\pgfsetroundjoin%
\definecolor{currentfill}{rgb}{0.000000,0.000000,0.000000}%
\pgfsetfillcolor{currentfill}%
\pgfsetlinewidth{0.501875pt}%
\definecolor{currentstroke}{rgb}{0.000000,0.000000,0.000000}%
\pgfsetstrokecolor{currentstroke}%
\pgfsetdash{}{0pt}%
\pgfsys@defobject{currentmarker}{\pgfqpoint{0.000000in}{-0.020833in}}{\pgfqpoint{0.000000in}{0.000000in}}{%
\pgfpathmoveto{\pgfqpoint{0.000000in}{0.000000in}}%
\pgfpathlineto{\pgfqpoint{0.000000in}{-0.020833in}}%
\pgfusepath{stroke,fill}%
}%
\begin{pgfscope}%
\pgfsys@transformshift{2.757221in}{1.507795in}%
\pgfsys@useobject{currentmarker}{}%
\end{pgfscope}%
\end{pgfscope}%
\begin{pgfscope}%
\pgfsetbuttcap%
\pgfsetroundjoin%
\definecolor{currentfill}{rgb}{0.000000,0.000000,0.000000}%
\pgfsetfillcolor{currentfill}%
\pgfsetlinewidth{0.501875pt}%
\definecolor{currentstroke}{rgb}{0.000000,0.000000,0.000000}%
\pgfsetstrokecolor{currentstroke}%
\pgfsetdash{}{0pt}%
\pgfsys@defobject{currentmarker}{\pgfqpoint{0.000000in}{0.000000in}}{\pgfqpoint{0.000000in}{0.020833in}}{%
\pgfpathmoveto{\pgfqpoint{0.000000in}{0.000000in}}%
\pgfpathlineto{\pgfqpoint{0.000000in}{0.020833in}}%
\pgfusepath{stroke,fill}%
}%
\begin{pgfscope}%
\pgfsys@transformshift{2.853314in}{0.431673in}%
\pgfsys@useobject{currentmarker}{}%
\end{pgfscope}%
\end{pgfscope}%
\begin{pgfscope}%
\pgfsetbuttcap%
\pgfsetroundjoin%
\definecolor{currentfill}{rgb}{0.000000,0.000000,0.000000}%
\pgfsetfillcolor{currentfill}%
\pgfsetlinewidth{0.501875pt}%
\definecolor{currentstroke}{rgb}{0.000000,0.000000,0.000000}%
\pgfsetstrokecolor{currentstroke}%
\pgfsetdash{}{0pt}%
\pgfsys@defobject{currentmarker}{\pgfqpoint{0.000000in}{-0.020833in}}{\pgfqpoint{0.000000in}{0.000000in}}{%
\pgfpathmoveto{\pgfqpoint{0.000000in}{0.000000in}}%
\pgfpathlineto{\pgfqpoint{0.000000in}{-0.020833in}}%
\pgfusepath{stroke,fill}%
}%
\begin{pgfscope}%
\pgfsys@transformshift{2.853314in}{1.507795in}%
\pgfsys@useobject{currentmarker}{}%
\end{pgfscope}%
\end{pgfscope}%
\begin{pgfscope}%
\pgfsetbuttcap%
\pgfsetroundjoin%
\definecolor{currentfill}{rgb}{0.000000,0.000000,0.000000}%
\pgfsetfillcolor{currentfill}%
\pgfsetlinewidth{0.501875pt}%
\definecolor{currentstroke}{rgb}{0.000000,0.000000,0.000000}%
\pgfsetstrokecolor{currentstroke}%
\pgfsetdash{}{0pt}%
\pgfsys@defobject{currentmarker}{\pgfqpoint{0.000000in}{0.000000in}}{\pgfqpoint{0.000000in}{0.020833in}}{%
\pgfpathmoveto{\pgfqpoint{0.000000in}{0.000000in}}%
\pgfpathlineto{\pgfqpoint{0.000000in}{0.020833in}}%
\pgfusepath{stroke,fill}%
}%
\begin{pgfscope}%
\pgfsys@transformshift{2.949407in}{0.431673in}%
\pgfsys@useobject{currentmarker}{}%
\end{pgfscope}%
\end{pgfscope}%
\begin{pgfscope}%
\pgfsetbuttcap%
\pgfsetroundjoin%
\definecolor{currentfill}{rgb}{0.000000,0.000000,0.000000}%
\pgfsetfillcolor{currentfill}%
\pgfsetlinewidth{0.501875pt}%
\definecolor{currentstroke}{rgb}{0.000000,0.000000,0.000000}%
\pgfsetstrokecolor{currentstroke}%
\pgfsetdash{}{0pt}%
\pgfsys@defobject{currentmarker}{\pgfqpoint{0.000000in}{-0.020833in}}{\pgfqpoint{0.000000in}{0.000000in}}{%
\pgfpathmoveto{\pgfqpoint{0.000000in}{0.000000in}}%
\pgfpathlineto{\pgfqpoint{0.000000in}{-0.020833in}}%
\pgfusepath{stroke,fill}%
}%
\begin{pgfscope}%
\pgfsys@transformshift{2.949407in}{1.507795in}%
\pgfsys@useobject{currentmarker}{}%
\end{pgfscope}%
\end{pgfscope}%
\begin{pgfscope}%
\pgfsetbuttcap%
\pgfsetroundjoin%
\definecolor{currentfill}{rgb}{0.000000,0.000000,0.000000}%
\pgfsetfillcolor{currentfill}%
\pgfsetlinewidth{0.501875pt}%
\definecolor{currentstroke}{rgb}{0.000000,0.000000,0.000000}%
\pgfsetstrokecolor{currentstroke}%
\pgfsetdash{}{0pt}%
\pgfsys@defobject{currentmarker}{\pgfqpoint{0.000000in}{0.000000in}}{\pgfqpoint{0.000000in}{0.020833in}}{%
\pgfpathmoveto{\pgfqpoint{0.000000in}{0.000000in}}%
\pgfpathlineto{\pgfqpoint{0.000000in}{0.020833in}}%
\pgfusepath{stroke,fill}%
}%
\begin{pgfscope}%
\pgfsys@transformshift{3.141592in}{0.431673in}%
\pgfsys@useobject{currentmarker}{}%
\end{pgfscope}%
\end{pgfscope}%
\begin{pgfscope}%
\pgfsetbuttcap%
\pgfsetroundjoin%
\definecolor{currentfill}{rgb}{0.000000,0.000000,0.000000}%
\pgfsetfillcolor{currentfill}%
\pgfsetlinewidth{0.501875pt}%
\definecolor{currentstroke}{rgb}{0.000000,0.000000,0.000000}%
\pgfsetstrokecolor{currentstroke}%
\pgfsetdash{}{0pt}%
\pgfsys@defobject{currentmarker}{\pgfqpoint{0.000000in}{-0.020833in}}{\pgfqpoint{0.000000in}{0.000000in}}{%
\pgfpathmoveto{\pgfqpoint{0.000000in}{0.000000in}}%
\pgfpathlineto{\pgfqpoint{0.000000in}{-0.020833in}}%
\pgfusepath{stroke,fill}%
}%
\begin{pgfscope}%
\pgfsys@transformshift{3.141592in}{1.507795in}%
\pgfsys@useobject{currentmarker}{}%
\end{pgfscope}%
\end{pgfscope}%
\begin{pgfscope}%
\pgfsetbuttcap%
\pgfsetroundjoin%
\definecolor{currentfill}{rgb}{0.000000,0.000000,0.000000}%
\pgfsetfillcolor{currentfill}%
\pgfsetlinewidth{0.501875pt}%
\definecolor{currentstroke}{rgb}{0.000000,0.000000,0.000000}%
\pgfsetstrokecolor{currentstroke}%
\pgfsetdash{}{0pt}%
\pgfsys@defobject{currentmarker}{\pgfqpoint{0.000000in}{0.000000in}}{\pgfqpoint{0.000000in}{0.020833in}}{%
\pgfpathmoveto{\pgfqpoint{0.000000in}{0.000000in}}%
\pgfpathlineto{\pgfqpoint{0.000000in}{0.020833in}}%
\pgfusepath{stroke,fill}%
}%
\begin{pgfscope}%
\pgfsys@transformshift{3.237685in}{0.431673in}%
\pgfsys@useobject{currentmarker}{}%
\end{pgfscope}%
\end{pgfscope}%
\begin{pgfscope}%
\pgfsetbuttcap%
\pgfsetroundjoin%
\definecolor{currentfill}{rgb}{0.000000,0.000000,0.000000}%
\pgfsetfillcolor{currentfill}%
\pgfsetlinewidth{0.501875pt}%
\definecolor{currentstroke}{rgb}{0.000000,0.000000,0.000000}%
\pgfsetstrokecolor{currentstroke}%
\pgfsetdash{}{0pt}%
\pgfsys@defobject{currentmarker}{\pgfqpoint{0.000000in}{-0.020833in}}{\pgfqpoint{0.000000in}{0.000000in}}{%
\pgfpathmoveto{\pgfqpoint{0.000000in}{0.000000in}}%
\pgfpathlineto{\pgfqpoint{0.000000in}{-0.020833in}}%
\pgfusepath{stroke,fill}%
}%
\begin{pgfscope}%
\pgfsys@transformshift{3.237685in}{1.507795in}%
\pgfsys@useobject{currentmarker}{}%
\end{pgfscope}%
\end{pgfscope}%
\begin{pgfscope}%
\pgfsetbuttcap%
\pgfsetroundjoin%
\definecolor{currentfill}{rgb}{0.000000,0.000000,0.000000}%
\pgfsetfillcolor{currentfill}%
\pgfsetlinewidth{0.501875pt}%
\definecolor{currentstroke}{rgb}{0.000000,0.000000,0.000000}%
\pgfsetstrokecolor{currentstroke}%
\pgfsetdash{}{0pt}%
\pgfsys@defobject{currentmarker}{\pgfqpoint{0.000000in}{0.000000in}}{\pgfqpoint{0.000000in}{0.020833in}}{%
\pgfpathmoveto{\pgfqpoint{0.000000in}{0.000000in}}%
\pgfpathlineto{\pgfqpoint{0.000000in}{0.020833in}}%
\pgfusepath{stroke,fill}%
}%
\begin{pgfscope}%
\pgfsys@transformshift{3.333777in}{0.431673in}%
\pgfsys@useobject{currentmarker}{}%
\end{pgfscope}%
\end{pgfscope}%
\begin{pgfscope}%
\pgfsetbuttcap%
\pgfsetroundjoin%
\definecolor{currentfill}{rgb}{0.000000,0.000000,0.000000}%
\pgfsetfillcolor{currentfill}%
\pgfsetlinewidth{0.501875pt}%
\definecolor{currentstroke}{rgb}{0.000000,0.000000,0.000000}%
\pgfsetstrokecolor{currentstroke}%
\pgfsetdash{}{0pt}%
\pgfsys@defobject{currentmarker}{\pgfqpoint{0.000000in}{-0.020833in}}{\pgfqpoint{0.000000in}{0.000000in}}{%
\pgfpathmoveto{\pgfqpoint{0.000000in}{0.000000in}}%
\pgfpathlineto{\pgfqpoint{0.000000in}{-0.020833in}}%
\pgfusepath{stroke,fill}%
}%
\begin{pgfscope}%
\pgfsys@transformshift{3.333777in}{1.507795in}%
\pgfsys@useobject{currentmarker}{}%
\end{pgfscope}%
\end{pgfscope}%
\begin{pgfscope}%
\pgfsetbuttcap%
\pgfsetroundjoin%
\definecolor{currentfill}{rgb}{0.000000,0.000000,0.000000}%
\pgfsetfillcolor{currentfill}%
\pgfsetlinewidth{0.501875pt}%
\definecolor{currentstroke}{rgb}{0.000000,0.000000,0.000000}%
\pgfsetstrokecolor{currentstroke}%
\pgfsetdash{}{0pt}%
\pgfsys@defobject{currentmarker}{\pgfqpoint{0.000000in}{0.000000in}}{\pgfqpoint{0.000000in}{0.020833in}}{%
\pgfpathmoveto{\pgfqpoint{0.000000in}{0.000000in}}%
\pgfpathlineto{\pgfqpoint{0.000000in}{0.020833in}}%
\pgfusepath{stroke,fill}%
}%
\begin{pgfscope}%
\pgfsys@transformshift{3.429870in}{0.431673in}%
\pgfsys@useobject{currentmarker}{}%
\end{pgfscope}%
\end{pgfscope}%
\begin{pgfscope}%
\pgfsetbuttcap%
\pgfsetroundjoin%
\definecolor{currentfill}{rgb}{0.000000,0.000000,0.000000}%
\pgfsetfillcolor{currentfill}%
\pgfsetlinewidth{0.501875pt}%
\definecolor{currentstroke}{rgb}{0.000000,0.000000,0.000000}%
\pgfsetstrokecolor{currentstroke}%
\pgfsetdash{}{0pt}%
\pgfsys@defobject{currentmarker}{\pgfqpoint{0.000000in}{-0.020833in}}{\pgfqpoint{0.000000in}{0.000000in}}{%
\pgfpathmoveto{\pgfqpoint{0.000000in}{0.000000in}}%
\pgfpathlineto{\pgfqpoint{0.000000in}{-0.020833in}}%
\pgfusepath{stroke,fill}%
}%
\begin{pgfscope}%
\pgfsys@transformshift{3.429870in}{1.507795in}%
\pgfsys@useobject{currentmarker}{}%
\end{pgfscope}%
\end{pgfscope}%
\begin{pgfscope}%
\pgfsetbuttcap%
\pgfsetroundjoin%
\definecolor{currentfill}{rgb}{0.000000,0.000000,0.000000}%
\pgfsetfillcolor{currentfill}%
\pgfsetlinewidth{0.501875pt}%
\definecolor{currentstroke}{rgb}{0.000000,0.000000,0.000000}%
\pgfsetstrokecolor{currentstroke}%
\pgfsetdash{}{0pt}%
\pgfsys@defobject{currentmarker}{\pgfqpoint{0.000000in}{0.000000in}}{\pgfqpoint{0.000000in}{0.020833in}}{%
\pgfpathmoveto{\pgfqpoint{0.000000in}{0.000000in}}%
\pgfpathlineto{\pgfqpoint{0.000000in}{0.020833in}}%
\pgfusepath{stroke,fill}%
}%
\begin{pgfscope}%
\pgfsys@transformshift{3.622055in}{0.431673in}%
\pgfsys@useobject{currentmarker}{}%
\end{pgfscope}%
\end{pgfscope}%
\begin{pgfscope}%
\pgfsetbuttcap%
\pgfsetroundjoin%
\definecolor{currentfill}{rgb}{0.000000,0.000000,0.000000}%
\pgfsetfillcolor{currentfill}%
\pgfsetlinewidth{0.501875pt}%
\definecolor{currentstroke}{rgb}{0.000000,0.000000,0.000000}%
\pgfsetstrokecolor{currentstroke}%
\pgfsetdash{}{0pt}%
\pgfsys@defobject{currentmarker}{\pgfqpoint{0.000000in}{-0.020833in}}{\pgfqpoint{0.000000in}{0.000000in}}{%
\pgfpathmoveto{\pgfqpoint{0.000000in}{0.000000in}}%
\pgfpathlineto{\pgfqpoint{0.000000in}{-0.020833in}}%
\pgfusepath{stroke,fill}%
}%
\begin{pgfscope}%
\pgfsys@transformshift{3.622055in}{1.507795in}%
\pgfsys@useobject{currentmarker}{}%
\end{pgfscope}%
\end{pgfscope}%
\begin{pgfscope}%
\pgfsetbuttcap%
\pgfsetroundjoin%
\definecolor{currentfill}{rgb}{0.000000,0.000000,0.000000}%
\pgfsetfillcolor{currentfill}%
\pgfsetlinewidth{0.501875pt}%
\definecolor{currentstroke}{rgb}{0.000000,0.000000,0.000000}%
\pgfsetstrokecolor{currentstroke}%
\pgfsetdash{}{0pt}%
\pgfsys@defobject{currentmarker}{\pgfqpoint{0.000000in}{0.000000in}}{\pgfqpoint{0.000000in}{0.020833in}}{%
\pgfpathmoveto{\pgfqpoint{0.000000in}{0.000000in}}%
\pgfpathlineto{\pgfqpoint{0.000000in}{0.020833in}}%
\pgfusepath{stroke,fill}%
}%
\begin{pgfscope}%
\pgfsys@transformshift{3.718148in}{0.431673in}%
\pgfsys@useobject{currentmarker}{}%
\end{pgfscope}%
\end{pgfscope}%
\begin{pgfscope}%
\pgfsetbuttcap%
\pgfsetroundjoin%
\definecolor{currentfill}{rgb}{0.000000,0.000000,0.000000}%
\pgfsetfillcolor{currentfill}%
\pgfsetlinewidth{0.501875pt}%
\definecolor{currentstroke}{rgb}{0.000000,0.000000,0.000000}%
\pgfsetstrokecolor{currentstroke}%
\pgfsetdash{}{0pt}%
\pgfsys@defobject{currentmarker}{\pgfqpoint{0.000000in}{-0.020833in}}{\pgfqpoint{0.000000in}{0.000000in}}{%
\pgfpathmoveto{\pgfqpoint{0.000000in}{0.000000in}}%
\pgfpathlineto{\pgfqpoint{0.000000in}{-0.020833in}}%
\pgfusepath{stroke,fill}%
}%
\begin{pgfscope}%
\pgfsys@transformshift{3.718148in}{1.507795in}%
\pgfsys@useobject{currentmarker}{}%
\end{pgfscope}%
\end{pgfscope}%
\begin{pgfscope}%
\pgfsetbuttcap%
\pgfsetroundjoin%
\definecolor{currentfill}{rgb}{0.000000,0.000000,0.000000}%
\pgfsetfillcolor{currentfill}%
\pgfsetlinewidth{0.501875pt}%
\definecolor{currentstroke}{rgb}{0.000000,0.000000,0.000000}%
\pgfsetstrokecolor{currentstroke}%
\pgfsetdash{}{0pt}%
\pgfsys@defobject{currentmarker}{\pgfqpoint{0.000000in}{0.000000in}}{\pgfqpoint{0.000000in}{0.020833in}}{%
\pgfpathmoveto{\pgfqpoint{0.000000in}{0.000000in}}%
\pgfpathlineto{\pgfqpoint{0.000000in}{0.020833in}}%
\pgfusepath{stroke,fill}%
}%
\begin{pgfscope}%
\pgfsys@transformshift{3.814241in}{0.431673in}%
\pgfsys@useobject{currentmarker}{}%
\end{pgfscope}%
\end{pgfscope}%
\begin{pgfscope}%
\pgfsetbuttcap%
\pgfsetroundjoin%
\definecolor{currentfill}{rgb}{0.000000,0.000000,0.000000}%
\pgfsetfillcolor{currentfill}%
\pgfsetlinewidth{0.501875pt}%
\definecolor{currentstroke}{rgb}{0.000000,0.000000,0.000000}%
\pgfsetstrokecolor{currentstroke}%
\pgfsetdash{}{0pt}%
\pgfsys@defobject{currentmarker}{\pgfqpoint{0.000000in}{-0.020833in}}{\pgfqpoint{0.000000in}{0.000000in}}{%
\pgfpathmoveto{\pgfqpoint{0.000000in}{0.000000in}}%
\pgfpathlineto{\pgfqpoint{0.000000in}{-0.020833in}}%
\pgfusepath{stroke,fill}%
}%
\begin{pgfscope}%
\pgfsys@transformshift{3.814241in}{1.507795in}%
\pgfsys@useobject{currentmarker}{}%
\end{pgfscope}%
\end{pgfscope}%
\begin{pgfscope}%
\pgfsetbuttcap%
\pgfsetroundjoin%
\definecolor{currentfill}{rgb}{0.000000,0.000000,0.000000}%
\pgfsetfillcolor{currentfill}%
\pgfsetlinewidth{0.501875pt}%
\definecolor{currentstroke}{rgb}{0.000000,0.000000,0.000000}%
\pgfsetstrokecolor{currentstroke}%
\pgfsetdash{}{0pt}%
\pgfsys@defobject{currentmarker}{\pgfqpoint{0.000000in}{0.000000in}}{\pgfqpoint{0.000000in}{0.020833in}}{%
\pgfpathmoveto{\pgfqpoint{0.000000in}{0.000000in}}%
\pgfpathlineto{\pgfqpoint{0.000000in}{0.020833in}}%
\pgfusepath{stroke,fill}%
}%
\begin{pgfscope}%
\pgfsys@transformshift{3.910334in}{0.431673in}%
\pgfsys@useobject{currentmarker}{}%
\end{pgfscope}%
\end{pgfscope}%
\begin{pgfscope}%
\pgfsetbuttcap%
\pgfsetroundjoin%
\definecolor{currentfill}{rgb}{0.000000,0.000000,0.000000}%
\pgfsetfillcolor{currentfill}%
\pgfsetlinewidth{0.501875pt}%
\definecolor{currentstroke}{rgb}{0.000000,0.000000,0.000000}%
\pgfsetstrokecolor{currentstroke}%
\pgfsetdash{}{0pt}%
\pgfsys@defobject{currentmarker}{\pgfqpoint{0.000000in}{-0.020833in}}{\pgfqpoint{0.000000in}{0.000000in}}{%
\pgfpathmoveto{\pgfqpoint{0.000000in}{0.000000in}}%
\pgfpathlineto{\pgfqpoint{0.000000in}{-0.020833in}}%
\pgfusepath{stroke,fill}%
}%
\begin{pgfscope}%
\pgfsys@transformshift{3.910334in}{1.507795in}%
\pgfsys@useobject{currentmarker}{}%
\end{pgfscope}%
\end{pgfscope}%
\begin{pgfscope}%
\pgfsetbuttcap%
\pgfsetroundjoin%
\definecolor{currentfill}{rgb}{0.000000,0.000000,0.000000}%
\pgfsetfillcolor{currentfill}%
\pgfsetlinewidth{0.501875pt}%
\definecolor{currentstroke}{rgb}{0.000000,0.000000,0.000000}%
\pgfsetstrokecolor{currentstroke}%
\pgfsetdash{}{0pt}%
\pgfsys@defobject{currentmarker}{\pgfqpoint{0.000000in}{0.000000in}}{\pgfqpoint{0.000000in}{0.020833in}}{%
\pgfpathmoveto{\pgfqpoint{0.000000in}{0.000000in}}%
\pgfpathlineto{\pgfqpoint{0.000000in}{0.020833in}}%
\pgfusepath{stroke,fill}%
}%
\begin{pgfscope}%
\pgfsys@transformshift{4.102519in}{0.431673in}%
\pgfsys@useobject{currentmarker}{}%
\end{pgfscope}%
\end{pgfscope}%
\begin{pgfscope}%
\pgfsetbuttcap%
\pgfsetroundjoin%
\definecolor{currentfill}{rgb}{0.000000,0.000000,0.000000}%
\pgfsetfillcolor{currentfill}%
\pgfsetlinewidth{0.501875pt}%
\definecolor{currentstroke}{rgb}{0.000000,0.000000,0.000000}%
\pgfsetstrokecolor{currentstroke}%
\pgfsetdash{}{0pt}%
\pgfsys@defobject{currentmarker}{\pgfqpoint{0.000000in}{-0.020833in}}{\pgfqpoint{0.000000in}{0.000000in}}{%
\pgfpathmoveto{\pgfqpoint{0.000000in}{0.000000in}}%
\pgfpathlineto{\pgfqpoint{0.000000in}{-0.020833in}}%
\pgfusepath{stroke,fill}%
}%
\begin{pgfscope}%
\pgfsys@transformshift{4.102519in}{1.507795in}%
\pgfsys@useobject{currentmarker}{}%
\end{pgfscope}%
\end{pgfscope}%
\begin{pgfscope}%
\pgfsetbuttcap%
\pgfsetroundjoin%
\definecolor{currentfill}{rgb}{0.000000,0.000000,0.000000}%
\pgfsetfillcolor{currentfill}%
\pgfsetlinewidth{0.501875pt}%
\definecolor{currentstroke}{rgb}{0.000000,0.000000,0.000000}%
\pgfsetstrokecolor{currentstroke}%
\pgfsetdash{}{0pt}%
\pgfsys@defobject{currentmarker}{\pgfqpoint{0.000000in}{0.000000in}}{\pgfqpoint{0.000000in}{0.020833in}}{%
\pgfpathmoveto{\pgfqpoint{0.000000in}{0.000000in}}%
\pgfpathlineto{\pgfqpoint{0.000000in}{0.020833in}}%
\pgfusepath{stroke,fill}%
}%
\begin{pgfscope}%
\pgfsys@transformshift{4.198612in}{0.431673in}%
\pgfsys@useobject{currentmarker}{}%
\end{pgfscope}%
\end{pgfscope}%
\begin{pgfscope}%
\pgfsetbuttcap%
\pgfsetroundjoin%
\definecolor{currentfill}{rgb}{0.000000,0.000000,0.000000}%
\pgfsetfillcolor{currentfill}%
\pgfsetlinewidth{0.501875pt}%
\definecolor{currentstroke}{rgb}{0.000000,0.000000,0.000000}%
\pgfsetstrokecolor{currentstroke}%
\pgfsetdash{}{0pt}%
\pgfsys@defobject{currentmarker}{\pgfqpoint{0.000000in}{-0.020833in}}{\pgfqpoint{0.000000in}{0.000000in}}{%
\pgfpathmoveto{\pgfqpoint{0.000000in}{0.000000in}}%
\pgfpathlineto{\pgfqpoint{0.000000in}{-0.020833in}}%
\pgfusepath{stroke,fill}%
}%
\begin{pgfscope}%
\pgfsys@transformshift{4.198612in}{1.507795in}%
\pgfsys@useobject{currentmarker}{}%
\end{pgfscope}%
\end{pgfscope}%
\begin{pgfscope}%
\pgfsetbuttcap%
\pgfsetroundjoin%
\definecolor{currentfill}{rgb}{0.000000,0.000000,0.000000}%
\pgfsetfillcolor{currentfill}%
\pgfsetlinewidth{0.501875pt}%
\definecolor{currentstroke}{rgb}{0.000000,0.000000,0.000000}%
\pgfsetstrokecolor{currentstroke}%
\pgfsetdash{}{0pt}%
\pgfsys@defobject{currentmarker}{\pgfqpoint{0.000000in}{0.000000in}}{\pgfqpoint{0.000000in}{0.020833in}}{%
\pgfpathmoveto{\pgfqpoint{0.000000in}{0.000000in}}%
\pgfpathlineto{\pgfqpoint{0.000000in}{0.020833in}}%
\pgfusepath{stroke,fill}%
}%
\begin{pgfscope}%
\pgfsys@transformshift{4.294704in}{0.431673in}%
\pgfsys@useobject{currentmarker}{}%
\end{pgfscope}%
\end{pgfscope}%
\begin{pgfscope}%
\pgfsetbuttcap%
\pgfsetroundjoin%
\definecolor{currentfill}{rgb}{0.000000,0.000000,0.000000}%
\pgfsetfillcolor{currentfill}%
\pgfsetlinewidth{0.501875pt}%
\definecolor{currentstroke}{rgb}{0.000000,0.000000,0.000000}%
\pgfsetstrokecolor{currentstroke}%
\pgfsetdash{}{0pt}%
\pgfsys@defobject{currentmarker}{\pgfqpoint{0.000000in}{-0.020833in}}{\pgfqpoint{0.000000in}{0.000000in}}{%
\pgfpathmoveto{\pgfqpoint{0.000000in}{0.000000in}}%
\pgfpathlineto{\pgfqpoint{0.000000in}{-0.020833in}}%
\pgfusepath{stroke,fill}%
}%
\begin{pgfscope}%
\pgfsys@transformshift{4.294704in}{1.507795in}%
\pgfsys@useobject{currentmarker}{}%
\end{pgfscope}%
\end{pgfscope}%
\begin{pgfscope}%
\pgfsetbuttcap%
\pgfsetroundjoin%
\definecolor{currentfill}{rgb}{0.000000,0.000000,0.000000}%
\pgfsetfillcolor{currentfill}%
\pgfsetlinewidth{0.501875pt}%
\definecolor{currentstroke}{rgb}{0.000000,0.000000,0.000000}%
\pgfsetstrokecolor{currentstroke}%
\pgfsetdash{}{0pt}%
\pgfsys@defobject{currentmarker}{\pgfqpoint{0.000000in}{0.000000in}}{\pgfqpoint{0.000000in}{0.020833in}}{%
\pgfpathmoveto{\pgfqpoint{0.000000in}{0.000000in}}%
\pgfpathlineto{\pgfqpoint{0.000000in}{0.020833in}}%
\pgfusepath{stroke,fill}%
}%
\begin{pgfscope}%
\pgfsys@transformshift{4.390797in}{0.431673in}%
\pgfsys@useobject{currentmarker}{}%
\end{pgfscope}%
\end{pgfscope}%
\begin{pgfscope}%
\pgfsetbuttcap%
\pgfsetroundjoin%
\definecolor{currentfill}{rgb}{0.000000,0.000000,0.000000}%
\pgfsetfillcolor{currentfill}%
\pgfsetlinewidth{0.501875pt}%
\definecolor{currentstroke}{rgb}{0.000000,0.000000,0.000000}%
\pgfsetstrokecolor{currentstroke}%
\pgfsetdash{}{0pt}%
\pgfsys@defobject{currentmarker}{\pgfqpoint{0.000000in}{-0.020833in}}{\pgfqpoint{0.000000in}{0.000000in}}{%
\pgfpathmoveto{\pgfqpoint{0.000000in}{0.000000in}}%
\pgfpathlineto{\pgfqpoint{0.000000in}{-0.020833in}}%
\pgfusepath{stroke,fill}%
}%
\begin{pgfscope}%
\pgfsys@transformshift{4.390797in}{1.507795in}%
\pgfsys@useobject{currentmarker}{}%
\end{pgfscope}%
\end{pgfscope}%
\begin{pgfscope}%
\pgfsetbuttcap%
\pgfsetroundjoin%
\definecolor{currentfill}{rgb}{0.000000,0.000000,0.000000}%
\pgfsetfillcolor{currentfill}%
\pgfsetlinewidth{0.501875pt}%
\definecolor{currentstroke}{rgb}{0.000000,0.000000,0.000000}%
\pgfsetstrokecolor{currentstroke}%
\pgfsetdash{}{0pt}%
\pgfsys@defobject{currentmarker}{\pgfqpoint{0.000000in}{0.000000in}}{\pgfqpoint{0.000000in}{0.020833in}}{%
\pgfpathmoveto{\pgfqpoint{0.000000in}{0.000000in}}%
\pgfpathlineto{\pgfqpoint{0.000000in}{0.020833in}}%
\pgfusepath{stroke,fill}%
}%
\begin{pgfscope}%
\pgfsys@transformshift{4.582982in}{0.431673in}%
\pgfsys@useobject{currentmarker}{}%
\end{pgfscope}%
\end{pgfscope}%
\begin{pgfscope}%
\pgfsetbuttcap%
\pgfsetroundjoin%
\definecolor{currentfill}{rgb}{0.000000,0.000000,0.000000}%
\pgfsetfillcolor{currentfill}%
\pgfsetlinewidth{0.501875pt}%
\definecolor{currentstroke}{rgb}{0.000000,0.000000,0.000000}%
\pgfsetstrokecolor{currentstroke}%
\pgfsetdash{}{0pt}%
\pgfsys@defobject{currentmarker}{\pgfqpoint{0.000000in}{-0.020833in}}{\pgfqpoint{0.000000in}{0.000000in}}{%
\pgfpathmoveto{\pgfqpoint{0.000000in}{0.000000in}}%
\pgfpathlineto{\pgfqpoint{0.000000in}{-0.020833in}}%
\pgfusepath{stroke,fill}%
}%
\begin{pgfscope}%
\pgfsys@transformshift{4.582982in}{1.507795in}%
\pgfsys@useobject{currentmarker}{}%
\end{pgfscope}%
\end{pgfscope}%
\begin{pgfscope}%
\definecolor{textcolor}{rgb}{0.000000,0.000000,0.000000}%
\pgfsetstrokecolor{textcolor}%
\pgfsetfillcolor{textcolor}%
\pgftext[x=2.560458in,y=0.201367in,,top]{\color{textcolor}\rmfamily\fontsize{12.000000}{14.400000}\selectfont \(\displaystyle \mu_0 H\) (\unit{mT})}%
\end{pgfscope}%
\begin{pgfscope}%
\pgfsetbuttcap%
\pgfsetroundjoin%
\definecolor{currentfill}{rgb}{0.000000,0.000000,0.000000}%
\pgfsetfillcolor{currentfill}%
\pgfsetlinewidth{0.501875pt}%
\definecolor{currentstroke}{rgb}{0.000000,0.000000,0.000000}%
\pgfsetstrokecolor{currentstroke}%
\pgfsetdash{}{0pt}%
\pgfsys@defobject{currentmarker}{\pgfqpoint{0.000000in}{0.000000in}}{\pgfqpoint{0.041667in}{0.000000in}}{%
\pgfpathmoveto{\pgfqpoint{0.000000in}{0.000000in}}%
\pgfpathlineto{\pgfqpoint{0.041667in}{0.000000in}}%
\pgfusepath{stroke,fill}%
}%
\begin{pgfscope}%
\pgfsys@transformshift{0.444748in}{0.744789in}%
\pgfsys@useobject{currentmarker}{}%
\end{pgfscope}%
\end{pgfscope}%
\begin{pgfscope}%
\pgfsetbuttcap%
\pgfsetroundjoin%
\definecolor{currentfill}{rgb}{0.000000,0.000000,0.000000}%
\pgfsetfillcolor{currentfill}%
\pgfsetlinewidth{0.501875pt}%
\definecolor{currentstroke}{rgb}{0.000000,0.000000,0.000000}%
\pgfsetstrokecolor{currentstroke}%
\pgfsetdash{}{0pt}%
\pgfsys@defobject{currentmarker}{\pgfqpoint{-0.041667in}{0.000000in}}{\pgfqpoint{-0.000000in}{0.000000in}}{%
\pgfpathmoveto{\pgfqpoint{-0.000000in}{0.000000in}}%
\pgfpathlineto{\pgfqpoint{-0.041667in}{0.000000in}}%
\pgfusepath{stroke,fill}%
}%
\begin{pgfscope}%
\pgfsys@transformshift{4.676167in}{0.744789in}%
\pgfsys@useobject{currentmarker}{}%
\end{pgfscope}%
\end{pgfscope}%
\begin{pgfscope}%
\definecolor{textcolor}{rgb}{0.000000,0.000000,0.000000}%
\pgfsetstrokecolor{textcolor}%
\pgfsetfillcolor{textcolor}%
\pgftext[x=0.257248in, y=0.696572in, left, base]{\color{textcolor}\rmfamily\fontsize{10.000000}{12.000000}\selectfont \(\displaystyle {25}\)}%
\end{pgfscope}%
\begin{pgfscope}%
\pgfsetbuttcap%
\pgfsetroundjoin%
\definecolor{currentfill}{rgb}{0.000000,0.000000,0.000000}%
\pgfsetfillcolor{currentfill}%
\pgfsetlinewidth{0.501875pt}%
\definecolor{currentstroke}{rgb}{0.000000,0.000000,0.000000}%
\pgfsetstrokecolor{currentstroke}%
\pgfsetdash{}{0pt}%
\pgfsys@defobject{currentmarker}{\pgfqpoint{0.000000in}{0.000000in}}{\pgfqpoint{0.041667in}{0.000000in}}{%
\pgfpathmoveto{\pgfqpoint{0.000000in}{0.000000in}}%
\pgfpathlineto{\pgfqpoint{0.041667in}{0.000000in}}%
\pgfusepath{stroke,fill}%
}%
\begin{pgfscope}%
\pgfsys@transformshift{0.444748in}{1.082706in}%
\pgfsys@useobject{currentmarker}{}%
\end{pgfscope}%
\end{pgfscope}%
\begin{pgfscope}%
\pgfsetbuttcap%
\pgfsetroundjoin%
\definecolor{currentfill}{rgb}{0.000000,0.000000,0.000000}%
\pgfsetfillcolor{currentfill}%
\pgfsetlinewidth{0.501875pt}%
\definecolor{currentstroke}{rgb}{0.000000,0.000000,0.000000}%
\pgfsetstrokecolor{currentstroke}%
\pgfsetdash{}{0pt}%
\pgfsys@defobject{currentmarker}{\pgfqpoint{-0.041667in}{0.000000in}}{\pgfqpoint{-0.000000in}{0.000000in}}{%
\pgfpathmoveto{\pgfqpoint{-0.000000in}{0.000000in}}%
\pgfpathlineto{\pgfqpoint{-0.041667in}{0.000000in}}%
\pgfusepath{stroke,fill}%
}%
\begin{pgfscope}%
\pgfsys@transformshift{4.676167in}{1.082706in}%
\pgfsys@useobject{currentmarker}{}%
\end{pgfscope}%
\end{pgfscope}%
\begin{pgfscope}%
\definecolor{textcolor}{rgb}{0.000000,0.000000,0.000000}%
\pgfsetstrokecolor{textcolor}%
\pgfsetfillcolor{textcolor}%
\pgftext[x=0.257248in, y=1.034489in, left, base]{\color{textcolor}\rmfamily\fontsize{10.000000}{12.000000}\selectfont \(\displaystyle {50}\)}%
\end{pgfscope}%
\begin{pgfscope}%
\pgfsetbuttcap%
\pgfsetroundjoin%
\definecolor{currentfill}{rgb}{0.000000,0.000000,0.000000}%
\pgfsetfillcolor{currentfill}%
\pgfsetlinewidth{0.501875pt}%
\definecolor{currentstroke}{rgb}{0.000000,0.000000,0.000000}%
\pgfsetstrokecolor{currentstroke}%
\pgfsetdash{}{0pt}%
\pgfsys@defobject{currentmarker}{\pgfqpoint{0.000000in}{0.000000in}}{\pgfqpoint{0.041667in}{0.000000in}}{%
\pgfpathmoveto{\pgfqpoint{0.000000in}{0.000000in}}%
\pgfpathlineto{\pgfqpoint{0.041667in}{0.000000in}}%
\pgfusepath{stroke,fill}%
}%
\begin{pgfscope}%
\pgfsys@transformshift{0.444748in}{1.420623in}%
\pgfsys@useobject{currentmarker}{}%
\end{pgfscope}%
\end{pgfscope}%
\begin{pgfscope}%
\pgfsetbuttcap%
\pgfsetroundjoin%
\definecolor{currentfill}{rgb}{0.000000,0.000000,0.000000}%
\pgfsetfillcolor{currentfill}%
\pgfsetlinewidth{0.501875pt}%
\definecolor{currentstroke}{rgb}{0.000000,0.000000,0.000000}%
\pgfsetstrokecolor{currentstroke}%
\pgfsetdash{}{0pt}%
\pgfsys@defobject{currentmarker}{\pgfqpoint{-0.041667in}{0.000000in}}{\pgfqpoint{-0.000000in}{0.000000in}}{%
\pgfpathmoveto{\pgfqpoint{-0.000000in}{0.000000in}}%
\pgfpathlineto{\pgfqpoint{-0.041667in}{0.000000in}}%
\pgfusepath{stroke,fill}%
}%
\begin{pgfscope}%
\pgfsys@transformshift{4.676167in}{1.420623in}%
\pgfsys@useobject{currentmarker}{}%
\end{pgfscope}%
\end{pgfscope}%
\begin{pgfscope}%
\definecolor{textcolor}{rgb}{0.000000,0.000000,0.000000}%
\pgfsetstrokecolor{textcolor}%
\pgfsetfillcolor{textcolor}%
\pgftext[x=0.257248in, y=1.372406in, left, base]{\color{textcolor}\rmfamily\fontsize{10.000000}{12.000000}\selectfont \(\displaystyle {75}\)}%
\end{pgfscope}%
\begin{pgfscope}%
\pgfsetbuttcap%
\pgfsetroundjoin%
\definecolor{currentfill}{rgb}{0.000000,0.000000,0.000000}%
\pgfsetfillcolor{currentfill}%
\pgfsetlinewidth{0.501875pt}%
\definecolor{currentstroke}{rgb}{0.000000,0.000000,0.000000}%
\pgfsetstrokecolor{currentstroke}%
\pgfsetdash{}{0pt}%
\pgfsys@defobject{currentmarker}{\pgfqpoint{0.000000in}{0.000000in}}{\pgfqpoint{0.020833in}{0.000000in}}{%
\pgfpathmoveto{\pgfqpoint{0.000000in}{0.000000in}}%
\pgfpathlineto{\pgfqpoint{0.020833in}{0.000000in}}%
\pgfusepath{stroke,fill}%
}%
\begin{pgfscope}%
\pgfsys@transformshift{0.444748in}{0.474456in}%
\pgfsys@useobject{currentmarker}{}%
\end{pgfscope}%
\end{pgfscope}%
\begin{pgfscope}%
\pgfsetbuttcap%
\pgfsetroundjoin%
\definecolor{currentfill}{rgb}{0.000000,0.000000,0.000000}%
\pgfsetfillcolor{currentfill}%
\pgfsetlinewidth{0.501875pt}%
\definecolor{currentstroke}{rgb}{0.000000,0.000000,0.000000}%
\pgfsetstrokecolor{currentstroke}%
\pgfsetdash{}{0pt}%
\pgfsys@defobject{currentmarker}{\pgfqpoint{-0.020833in}{0.000000in}}{\pgfqpoint{-0.000000in}{0.000000in}}{%
\pgfpathmoveto{\pgfqpoint{-0.000000in}{0.000000in}}%
\pgfpathlineto{\pgfqpoint{-0.020833in}{0.000000in}}%
\pgfusepath{stroke,fill}%
}%
\begin{pgfscope}%
\pgfsys@transformshift{4.676167in}{0.474456in}%
\pgfsys@useobject{currentmarker}{}%
\end{pgfscope}%
\end{pgfscope}%
\begin{pgfscope}%
\pgfsetbuttcap%
\pgfsetroundjoin%
\definecolor{currentfill}{rgb}{0.000000,0.000000,0.000000}%
\pgfsetfillcolor{currentfill}%
\pgfsetlinewidth{0.501875pt}%
\definecolor{currentstroke}{rgb}{0.000000,0.000000,0.000000}%
\pgfsetstrokecolor{currentstroke}%
\pgfsetdash{}{0pt}%
\pgfsys@defobject{currentmarker}{\pgfqpoint{0.000000in}{0.000000in}}{\pgfqpoint{0.020833in}{0.000000in}}{%
\pgfpathmoveto{\pgfqpoint{0.000000in}{0.000000in}}%
\pgfpathlineto{\pgfqpoint{0.020833in}{0.000000in}}%
\pgfusepath{stroke,fill}%
}%
\begin{pgfscope}%
\pgfsys@transformshift{0.444748in}{0.542039in}%
\pgfsys@useobject{currentmarker}{}%
\end{pgfscope}%
\end{pgfscope}%
\begin{pgfscope}%
\pgfsetbuttcap%
\pgfsetroundjoin%
\definecolor{currentfill}{rgb}{0.000000,0.000000,0.000000}%
\pgfsetfillcolor{currentfill}%
\pgfsetlinewidth{0.501875pt}%
\definecolor{currentstroke}{rgb}{0.000000,0.000000,0.000000}%
\pgfsetstrokecolor{currentstroke}%
\pgfsetdash{}{0pt}%
\pgfsys@defobject{currentmarker}{\pgfqpoint{-0.020833in}{0.000000in}}{\pgfqpoint{-0.000000in}{0.000000in}}{%
\pgfpathmoveto{\pgfqpoint{-0.000000in}{0.000000in}}%
\pgfpathlineto{\pgfqpoint{-0.020833in}{0.000000in}}%
\pgfusepath{stroke,fill}%
}%
\begin{pgfscope}%
\pgfsys@transformshift{4.676167in}{0.542039in}%
\pgfsys@useobject{currentmarker}{}%
\end{pgfscope}%
\end{pgfscope}%
\begin{pgfscope}%
\pgfsetbuttcap%
\pgfsetroundjoin%
\definecolor{currentfill}{rgb}{0.000000,0.000000,0.000000}%
\pgfsetfillcolor{currentfill}%
\pgfsetlinewidth{0.501875pt}%
\definecolor{currentstroke}{rgb}{0.000000,0.000000,0.000000}%
\pgfsetstrokecolor{currentstroke}%
\pgfsetdash{}{0pt}%
\pgfsys@defobject{currentmarker}{\pgfqpoint{0.000000in}{0.000000in}}{\pgfqpoint{0.020833in}{0.000000in}}{%
\pgfpathmoveto{\pgfqpoint{0.000000in}{0.000000in}}%
\pgfpathlineto{\pgfqpoint{0.020833in}{0.000000in}}%
\pgfusepath{stroke,fill}%
}%
\begin{pgfscope}%
\pgfsys@transformshift{0.444748in}{0.609623in}%
\pgfsys@useobject{currentmarker}{}%
\end{pgfscope}%
\end{pgfscope}%
\begin{pgfscope}%
\pgfsetbuttcap%
\pgfsetroundjoin%
\definecolor{currentfill}{rgb}{0.000000,0.000000,0.000000}%
\pgfsetfillcolor{currentfill}%
\pgfsetlinewidth{0.501875pt}%
\definecolor{currentstroke}{rgb}{0.000000,0.000000,0.000000}%
\pgfsetstrokecolor{currentstroke}%
\pgfsetdash{}{0pt}%
\pgfsys@defobject{currentmarker}{\pgfqpoint{-0.020833in}{0.000000in}}{\pgfqpoint{-0.000000in}{0.000000in}}{%
\pgfpathmoveto{\pgfqpoint{-0.000000in}{0.000000in}}%
\pgfpathlineto{\pgfqpoint{-0.020833in}{0.000000in}}%
\pgfusepath{stroke,fill}%
}%
\begin{pgfscope}%
\pgfsys@transformshift{4.676167in}{0.609623in}%
\pgfsys@useobject{currentmarker}{}%
\end{pgfscope}%
\end{pgfscope}%
\begin{pgfscope}%
\pgfsetbuttcap%
\pgfsetroundjoin%
\definecolor{currentfill}{rgb}{0.000000,0.000000,0.000000}%
\pgfsetfillcolor{currentfill}%
\pgfsetlinewidth{0.501875pt}%
\definecolor{currentstroke}{rgb}{0.000000,0.000000,0.000000}%
\pgfsetstrokecolor{currentstroke}%
\pgfsetdash{}{0pt}%
\pgfsys@defobject{currentmarker}{\pgfqpoint{0.000000in}{0.000000in}}{\pgfqpoint{0.020833in}{0.000000in}}{%
\pgfpathmoveto{\pgfqpoint{0.000000in}{0.000000in}}%
\pgfpathlineto{\pgfqpoint{0.020833in}{0.000000in}}%
\pgfusepath{stroke,fill}%
}%
\begin{pgfscope}%
\pgfsys@transformshift{0.444748in}{0.677206in}%
\pgfsys@useobject{currentmarker}{}%
\end{pgfscope}%
\end{pgfscope}%
\begin{pgfscope}%
\pgfsetbuttcap%
\pgfsetroundjoin%
\definecolor{currentfill}{rgb}{0.000000,0.000000,0.000000}%
\pgfsetfillcolor{currentfill}%
\pgfsetlinewidth{0.501875pt}%
\definecolor{currentstroke}{rgb}{0.000000,0.000000,0.000000}%
\pgfsetstrokecolor{currentstroke}%
\pgfsetdash{}{0pt}%
\pgfsys@defobject{currentmarker}{\pgfqpoint{-0.020833in}{0.000000in}}{\pgfqpoint{-0.000000in}{0.000000in}}{%
\pgfpathmoveto{\pgfqpoint{-0.000000in}{0.000000in}}%
\pgfpathlineto{\pgfqpoint{-0.020833in}{0.000000in}}%
\pgfusepath{stroke,fill}%
}%
\begin{pgfscope}%
\pgfsys@transformshift{4.676167in}{0.677206in}%
\pgfsys@useobject{currentmarker}{}%
\end{pgfscope}%
\end{pgfscope}%
\begin{pgfscope}%
\pgfsetbuttcap%
\pgfsetroundjoin%
\definecolor{currentfill}{rgb}{0.000000,0.000000,0.000000}%
\pgfsetfillcolor{currentfill}%
\pgfsetlinewidth{0.501875pt}%
\definecolor{currentstroke}{rgb}{0.000000,0.000000,0.000000}%
\pgfsetstrokecolor{currentstroke}%
\pgfsetdash{}{0pt}%
\pgfsys@defobject{currentmarker}{\pgfqpoint{0.000000in}{0.000000in}}{\pgfqpoint{0.020833in}{0.000000in}}{%
\pgfpathmoveto{\pgfqpoint{0.000000in}{0.000000in}}%
\pgfpathlineto{\pgfqpoint{0.020833in}{0.000000in}}%
\pgfusepath{stroke,fill}%
}%
\begin{pgfscope}%
\pgfsys@transformshift{0.444748in}{0.812373in}%
\pgfsys@useobject{currentmarker}{}%
\end{pgfscope}%
\end{pgfscope}%
\begin{pgfscope}%
\pgfsetbuttcap%
\pgfsetroundjoin%
\definecolor{currentfill}{rgb}{0.000000,0.000000,0.000000}%
\pgfsetfillcolor{currentfill}%
\pgfsetlinewidth{0.501875pt}%
\definecolor{currentstroke}{rgb}{0.000000,0.000000,0.000000}%
\pgfsetstrokecolor{currentstroke}%
\pgfsetdash{}{0pt}%
\pgfsys@defobject{currentmarker}{\pgfqpoint{-0.020833in}{0.000000in}}{\pgfqpoint{-0.000000in}{0.000000in}}{%
\pgfpathmoveto{\pgfqpoint{-0.000000in}{0.000000in}}%
\pgfpathlineto{\pgfqpoint{-0.020833in}{0.000000in}}%
\pgfusepath{stroke,fill}%
}%
\begin{pgfscope}%
\pgfsys@transformshift{4.676167in}{0.812373in}%
\pgfsys@useobject{currentmarker}{}%
\end{pgfscope}%
\end{pgfscope}%
\begin{pgfscope}%
\pgfsetbuttcap%
\pgfsetroundjoin%
\definecolor{currentfill}{rgb}{0.000000,0.000000,0.000000}%
\pgfsetfillcolor{currentfill}%
\pgfsetlinewidth{0.501875pt}%
\definecolor{currentstroke}{rgb}{0.000000,0.000000,0.000000}%
\pgfsetstrokecolor{currentstroke}%
\pgfsetdash{}{0pt}%
\pgfsys@defobject{currentmarker}{\pgfqpoint{0.000000in}{0.000000in}}{\pgfqpoint{0.020833in}{0.000000in}}{%
\pgfpathmoveto{\pgfqpoint{0.000000in}{0.000000in}}%
\pgfpathlineto{\pgfqpoint{0.020833in}{0.000000in}}%
\pgfusepath{stroke,fill}%
}%
\begin{pgfscope}%
\pgfsys@transformshift{0.444748in}{0.879956in}%
\pgfsys@useobject{currentmarker}{}%
\end{pgfscope}%
\end{pgfscope}%
\begin{pgfscope}%
\pgfsetbuttcap%
\pgfsetroundjoin%
\definecolor{currentfill}{rgb}{0.000000,0.000000,0.000000}%
\pgfsetfillcolor{currentfill}%
\pgfsetlinewidth{0.501875pt}%
\definecolor{currentstroke}{rgb}{0.000000,0.000000,0.000000}%
\pgfsetstrokecolor{currentstroke}%
\pgfsetdash{}{0pt}%
\pgfsys@defobject{currentmarker}{\pgfqpoint{-0.020833in}{0.000000in}}{\pgfqpoint{-0.000000in}{0.000000in}}{%
\pgfpathmoveto{\pgfqpoint{-0.000000in}{0.000000in}}%
\pgfpathlineto{\pgfqpoint{-0.020833in}{0.000000in}}%
\pgfusepath{stroke,fill}%
}%
\begin{pgfscope}%
\pgfsys@transformshift{4.676167in}{0.879956in}%
\pgfsys@useobject{currentmarker}{}%
\end{pgfscope}%
\end{pgfscope}%
\begin{pgfscope}%
\pgfsetbuttcap%
\pgfsetroundjoin%
\definecolor{currentfill}{rgb}{0.000000,0.000000,0.000000}%
\pgfsetfillcolor{currentfill}%
\pgfsetlinewidth{0.501875pt}%
\definecolor{currentstroke}{rgb}{0.000000,0.000000,0.000000}%
\pgfsetstrokecolor{currentstroke}%
\pgfsetdash{}{0pt}%
\pgfsys@defobject{currentmarker}{\pgfqpoint{0.000000in}{0.000000in}}{\pgfqpoint{0.020833in}{0.000000in}}{%
\pgfpathmoveto{\pgfqpoint{0.000000in}{0.000000in}}%
\pgfpathlineto{\pgfqpoint{0.020833in}{0.000000in}}%
\pgfusepath{stroke,fill}%
}%
\begin{pgfscope}%
\pgfsys@transformshift{0.444748in}{0.947540in}%
\pgfsys@useobject{currentmarker}{}%
\end{pgfscope}%
\end{pgfscope}%
\begin{pgfscope}%
\pgfsetbuttcap%
\pgfsetroundjoin%
\definecolor{currentfill}{rgb}{0.000000,0.000000,0.000000}%
\pgfsetfillcolor{currentfill}%
\pgfsetlinewidth{0.501875pt}%
\definecolor{currentstroke}{rgb}{0.000000,0.000000,0.000000}%
\pgfsetstrokecolor{currentstroke}%
\pgfsetdash{}{0pt}%
\pgfsys@defobject{currentmarker}{\pgfqpoint{-0.020833in}{0.000000in}}{\pgfqpoint{-0.000000in}{0.000000in}}{%
\pgfpathmoveto{\pgfqpoint{-0.000000in}{0.000000in}}%
\pgfpathlineto{\pgfqpoint{-0.020833in}{0.000000in}}%
\pgfusepath{stroke,fill}%
}%
\begin{pgfscope}%
\pgfsys@transformshift{4.676167in}{0.947540in}%
\pgfsys@useobject{currentmarker}{}%
\end{pgfscope}%
\end{pgfscope}%
\begin{pgfscope}%
\pgfsetbuttcap%
\pgfsetroundjoin%
\definecolor{currentfill}{rgb}{0.000000,0.000000,0.000000}%
\pgfsetfillcolor{currentfill}%
\pgfsetlinewidth{0.501875pt}%
\definecolor{currentstroke}{rgb}{0.000000,0.000000,0.000000}%
\pgfsetstrokecolor{currentstroke}%
\pgfsetdash{}{0pt}%
\pgfsys@defobject{currentmarker}{\pgfqpoint{0.000000in}{0.000000in}}{\pgfqpoint{0.020833in}{0.000000in}}{%
\pgfpathmoveto{\pgfqpoint{0.000000in}{0.000000in}}%
\pgfpathlineto{\pgfqpoint{0.020833in}{0.000000in}}%
\pgfusepath{stroke,fill}%
}%
\begin{pgfscope}%
\pgfsys@transformshift{0.444748in}{1.015123in}%
\pgfsys@useobject{currentmarker}{}%
\end{pgfscope}%
\end{pgfscope}%
\begin{pgfscope}%
\pgfsetbuttcap%
\pgfsetroundjoin%
\definecolor{currentfill}{rgb}{0.000000,0.000000,0.000000}%
\pgfsetfillcolor{currentfill}%
\pgfsetlinewidth{0.501875pt}%
\definecolor{currentstroke}{rgb}{0.000000,0.000000,0.000000}%
\pgfsetstrokecolor{currentstroke}%
\pgfsetdash{}{0pt}%
\pgfsys@defobject{currentmarker}{\pgfqpoint{-0.020833in}{0.000000in}}{\pgfqpoint{-0.000000in}{0.000000in}}{%
\pgfpathmoveto{\pgfqpoint{-0.000000in}{0.000000in}}%
\pgfpathlineto{\pgfqpoint{-0.020833in}{0.000000in}}%
\pgfusepath{stroke,fill}%
}%
\begin{pgfscope}%
\pgfsys@transformshift{4.676167in}{1.015123in}%
\pgfsys@useobject{currentmarker}{}%
\end{pgfscope}%
\end{pgfscope}%
\begin{pgfscope}%
\pgfsetbuttcap%
\pgfsetroundjoin%
\definecolor{currentfill}{rgb}{0.000000,0.000000,0.000000}%
\pgfsetfillcolor{currentfill}%
\pgfsetlinewidth{0.501875pt}%
\definecolor{currentstroke}{rgb}{0.000000,0.000000,0.000000}%
\pgfsetstrokecolor{currentstroke}%
\pgfsetdash{}{0pt}%
\pgfsys@defobject{currentmarker}{\pgfqpoint{0.000000in}{0.000000in}}{\pgfqpoint{0.020833in}{0.000000in}}{%
\pgfpathmoveto{\pgfqpoint{0.000000in}{0.000000in}}%
\pgfpathlineto{\pgfqpoint{0.020833in}{0.000000in}}%
\pgfusepath{stroke,fill}%
}%
\begin{pgfscope}%
\pgfsys@transformshift{0.444748in}{1.150290in}%
\pgfsys@useobject{currentmarker}{}%
\end{pgfscope}%
\end{pgfscope}%
\begin{pgfscope}%
\pgfsetbuttcap%
\pgfsetroundjoin%
\definecolor{currentfill}{rgb}{0.000000,0.000000,0.000000}%
\pgfsetfillcolor{currentfill}%
\pgfsetlinewidth{0.501875pt}%
\definecolor{currentstroke}{rgb}{0.000000,0.000000,0.000000}%
\pgfsetstrokecolor{currentstroke}%
\pgfsetdash{}{0pt}%
\pgfsys@defobject{currentmarker}{\pgfqpoint{-0.020833in}{0.000000in}}{\pgfqpoint{-0.000000in}{0.000000in}}{%
\pgfpathmoveto{\pgfqpoint{-0.000000in}{0.000000in}}%
\pgfpathlineto{\pgfqpoint{-0.020833in}{0.000000in}}%
\pgfusepath{stroke,fill}%
}%
\begin{pgfscope}%
\pgfsys@transformshift{4.676167in}{1.150290in}%
\pgfsys@useobject{currentmarker}{}%
\end{pgfscope}%
\end{pgfscope}%
\begin{pgfscope}%
\pgfsetbuttcap%
\pgfsetroundjoin%
\definecolor{currentfill}{rgb}{0.000000,0.000000,0.000000}%
\pgfsetfillcolor{currentfill}%
\pgfsetlinewidth{0.501875pt}%
\definecolor{currentstroke}{rgb}{0.000000,0.000000,0.000000}%
\pgfsetstrokecolor{currentstroke}%
\pgfsetdash{}{0pt}%
\pgfsys@defobject{currentmarker}{\pgfqpoint{0.000000in}{0.000000in}}{\pgfqpoint{0.020833in}{0.000000in}}{%
\pgfpathmoveto{\pgfqpoint{0.000000in}{0.000000in}}%
\pgfpathlineto{\pgfqpoint{0.020833in}{0.000000in}}%
\pgfusepath{stroke,fill}%
}%
\begin{pgfscope}%
\pgfsys@transformshift{0.444748in}{1.217873in}%
\pgfsys@useobject{currentmarker}{}%
\end{pgfscope}%
\end{pgfscope}%
\begin{pgfscope}%
\pgfsetbuttcap%
\pgfsetroundjoin%
\definecolor{currentfill}{rgb}{0.000000,0.000000,0.000000}%
\pgfsetfillcolor{currentfill}%
\pgfsetlinewidth{0.501875pt}%
\definecolor{currentstroke}{rgb}{0.000000,0.000000,0.000000}%
\pgfsetstrokecolor{currentstroke}%
\pgfsetdash{}{0pt}%
\pgfsys@defobject{currentmarker}{\pgfqpoint{-0.020833in}{0.000000in}}{\pgfqpoint{-0.000000in}{0.000000in}}{%
\pgfpathmoveto{\pgfqpoint{-0.000000in}{0.000000in}}%
\pgfpathlineto{\pgfqpoint{-0.020833in}{0.000000in}}%
\pgfusepath{stroke,fill}%
}%
\begin{pgfscope}%
\pgfsys@transformshift{4.676167in}{1.217873in}%
\pgfsys@useobject{currentmarker}{}%
\end{pgfscope}%
\end{pgfscope}%
\begin{pgfscope}%
\pgfsetbuttcap%
\pgfsetroundjoin%
\definecolor{currentfill}{rgb}{0.000000,0.000000,0.000000}%
\pgfsetfillcolor{currentfill}%
\pgfsetlinewidth{0.501875pt}%
\definecolor{currentstroke}{rgb}{0.000000,0.000000,0.000000}%
\pgfsetstrokecolor{currentstroke}%
\pgfsetdash{}{0pt}%
\pgfsys@defobject{currentmarker}{\pgfqpoint{0.000000in}{0.000000in}}{\pgfqpoint{0.020833in}{0.000000in}}{%
\pgfpathmoveto{\pgfqpoint{0.000000in}{0.000000in}}%
\pgfpathlineto{\pgfqpoint{0.020833in}{0.000000in}}%
\pgfusepath{stroke,fill}%
}%
\begin{pgfscope}%
\pgfsys@transformshift{0.444748in}{1.285457in}%
\pgfsys@useobject{currentmarker}{}%
\end{pgfscope}%
\end{pgfscope}%
\begin{pgfscope}%
\pgfsetbuttcap%
\pgfsetroundjoin%
\definecolor{currentfill}{rgb}{0.000000,0.000000,0.000000}%
\pgfsetfillcolor{currentfill}%
\pgfsetlinewidth{0.501875pt}%
\definecolor{currentstroke}{rgb}{0.000000,0.000000,0.000000}%
\pgfsetstrokecolor{currentstroke}%
\pgfsetdash{}{0pt}%
\pgfsys@defobject{currentmarker}{\pgfqpoint{-0.020833in}{0.000000in}}{\pgfqpoint{-0.000000in}{0.000000in}}{%
\pgfpathmoveto{\pgfqpoint{-0.000000in}{0.000000in}}%
\pgfpathlineto{\pgfqpoint{-0.020833in}{0.000000in}}%
\pgfusepath{stroke,fill}%
}%
\begin{pgfscope}%
\pgfsys@transformshift{4.676167in}{1.285457in}%
\pgfsys@useobject{currentmarker}{}%
\end{pgfscope}%
\end{pgfscope}%
\begin{pgfscope}%
\pgfsetbuttcap%
\pgfsetroundjoin%
\definecolor{currentfill}{rgb}{0.000000,0.000000,0.000000}%
\pgfsetfillcolor{currentfill}%
\pgfsetlinewidth{0.501875pt}%
\definecolor{currentstroke}{rgb}{0.000000,0.000000,0.000000}%
\pgfsetstrokecolor{currentstroke}%
\pgfsetdash{}{0pt}%
\pgfsys@defobject{currentmarker}{\pgfqpoint{0.000000in}{0.000000in}}{\pgfqpoint{0.020833in}{0.000000in}}{%
\pgfpathmoveto{\pgfqpoint{0.000000in}{0.000000in}}%
\pgfpathlineto{\pgfqpoint{0.020833in}{0.000000in}}%
\pgfusepath{stroke,fill}%
}%
\begin{pgfscope}%
\pgfsys@transformshift{0.444748in}{1.353040in}%
\pgfsys@useobject{currentmarker}{}%
\end{pgfscope}%
\end{pgfscope}%
\begin{pgfscope}%
\pgfsetbuttcap%
\pgfsetroundjoin%
\definecolor{currentfill}{rgb}{0.000000,0.000000,0.000000}%
\pgfsetfillcolor{currentfill}%
\pgfsetlinewidth{0.501875pt}%
\definecolor{currentstroke}{rgb}{0.000000,0.000000,0.000000}%
\pgfsetstrokecolor{currentstroke}%
\pgfsetdash{}{0pt}%
\pgfsys@defobject{currentmarker}{\pgfqpoint{-0.020833in}{0.000000in}}{\pgfqpoint{-0.000000in}{0.000000in}}{%
\pgfpathmoveto{\pgfqpoint{-0.000000in}{0.000000in}}%
\pgfpathlineto{\pgfqpoint{-0.020833in}{0.000000in}}%
\pgfusepath{stroke,fill}%
}%
\begin{pgfscope}%
\pgfsys@transformshift{4.676167in}{1.353040in}%
\pgfsys@useobject{currentmarker}{}%
\end{pgfscope}%
\end{pgfscope}%
\begin{pgfscope}%
\pgfsetbuttcap%
\pgfsetroundjoin%
\definecolor{currentfill}{rgb}{0.000000,0.000000,0.000000}%
\pgfsetfillcolor{currentfill}%
\pgfsetlinewidth{0.501875pt}%
\definecolor{currentstroke}{rgb}{0.000000,0.000000,0.000000}%
\pgfsetstrokecolor{currentstroke}%
\pgfsetdash{}{0pt}%
\pgfsys@defobject{currentmarker}{\pgfqpoint{0.000000in}{0.000000in}}{\pgfqpoint{0.020833in}{0.000000in}}{%
\pgfpathmoveto{\pgfqpoint{0.000000in}{0.000000in}}%
\pgfpathlineto{\pgfqpoint{0.020833in}{0.000000in}}%
\pgfusepath{stroke,fill}%
}%
\begin{pgfscope}%
\pgfsys@transformshift{0.444748in}{1.488207in}%
\pgfsys@useobject{currentmarker}{}%
\end{pgfscope}%
\end{pgfscope}%
\begin{pgfscope}%
\pgfsetbuttcap%
\pgfsetroundjoin%
\definecolor{currentfill}{rgb}{0.000000,0.000000,0.000000}%
\pgfsetfillcolor{currentfill}%
\pgfsetlinewidth{0.501875pt}%
\definecolor{currentstroke}{rgb}{0.000000,0.000000,0.000000}%
\pgfsetstrokecolor{currentstroke}%
\pgfsetdash{}{0pt}%
\pgfsys@defobject{currentmarker}{\pgfqpoint{-0.020833in}{0.000000in}}{\pgfqpoint{-0.000000in}{0.000000in}}{%
\pgfpathmoveto{\pgfqpoint{-0.000000in}{0.000000in}}%
\pgfpathlineto{\pgfqpoint{-0.020833in}{0.000000in}}%
\pgfusepath{stroke,fill}%
}%
\begin{pgfscope}%
\pgfsys@transformshift{4.676167in}{1.488207in}%
\pgfsys@useobject{currentmarker}{}%
\end{pgfscope}%
\end{pgfscope}%
\begin{pgfscope}%
\definecolor{textcolor}{rgb}{0.000000,0.000000,0.000000}%
\pgfsetstrokecolor{textcolor}%
\pgfsetfillcolor{textcolor}%
\pgftext[x=0.201692in,y=0.969734in,,bottom,rotate=90.000000]{\color{textcolor}\rmfamily\fontsize{12.000000}{14.400000}\selectfont \(\displaystyle V_s\) (\unit{\micro\volt})}%
\end{pgfscope}%
\begin{pgfscope}%
\pgfpathrectangle{\pgfqpoint{0.444748in}{0.431673in}}{\pgfqpoint{4.231419in}{1.076123in}}%
\pgfusepath{clip}%
\pgfsetbuttcap%
\pgfsetroundjoin%
\pgfsetlinewidth{1.003750pt}%
\definecolor{currentstroke}{rgb}{0.047059,0.364706,0.647059}%
\pgfsetstrokecolor{currentstroke}%
\pgfsetdash{{3.700000pt}{1.600000pt}}{0.000000pt}%
\pgfpathmoveto{\pgfqpoint{0.651405in}{0.827459in}}%
\pgfpathlineto{\pgfqpoint{0.656571in}{0.716994in}}%
\pgfpathlineto{\pgfqpoint{0.677231in}{0.549367in}}%
\pgfpathlineto{\pgfqpoint{0.693900in}{0.547809in}}%
\pgfpathlineto{\pgfqpoint{0.711977in}{0.687463in}}%
\pgfpathlineto{\pgfqpoint{0.730993in}{0.887701in}}%
\pgfpathlineto{\pgfqpoint{0.753296in}{1.258731in}}%
\pgfpathlineto{\pgfqpoint{0.772548in}{1.433710in}}%
\pgfpathlineto{\pgfqpoint{0.790861in}{1.411637in}}%
\pgfpathlineto{\pgfqpoint{0.809409in}{1.198081in}}%
\pgfpathlineto{\pgfqpoint{0.829363in}{0.857472in}}%
\pgfpathlineto{\pgfqpoint{0.847440in}{0.652079in}}%
\pgfpathlineto{\pgfqpoint{0.866223in}{0.524337in}}%
\pgfpathlineto{\pgfqpoint{0.885005in}{0.565578in}}%
\pgfpathlineto{\pgfqpoint{0.906368in}{0.724102in}}%
\pgfpathlineto{\pgfqpoint{0.923976in}{0.959619in}}%
\pgfpathlineto{\pgfqpoint{0.943227in}{1.269522in}}%
\pgfpathlineto{\pgfqpoint{0.964122in}{1.424922in}}%
\pgfpathlineto{\pgfqpoint{0.982435in}{1.347542in}}%
\pgfpathlineto{\pgfqpoint{1.019294in}{0.774601in}}%
\pgfpathlineto{\pgfqpoint{1.042537in}{0.566204in}}%
\pgfpathlineto{\pgfqpoint{1.060145in}{0.510209in}}%
\pgfpathlineto{\pgfqpoint{1.079162in}{0.603305in}}%
\pgfpathlineto{\pgfqpoint{1.101934in}{0.831438in}}%
\pgfpathlineto{\pgfqpoint{1.115082in}{1.017417in}}%
\pgfpathlineto{\pgfqpoint{1.132690in}{1.364888in}}%
\pgfpathlineto{\pgfqpoint{1.154993in}{1.412231in}}%
\pgfpathlineto{\pgfqpoint{1.175653in}{1.264651in}}%
\pgfpathlineto{\pgfqpoint{1.194669in}{0.970458in}}%
\pgfpathlineto{\pgfqpoint{1.213452in}{0.701576in}}%
\pgfpathlineto{\pgfqpoint{1.230120in}{0.555614in}}%
\pgfpathlineto{\pgfqpoint{1.251014in}{0.501347in}}%
\pgfpathlineto{\pgfqpoint{1.271674in}{0.594090in}}%
\pgfpathlineto{\pgfqpoint{1.291865in}{0.771975in}}%
\pgfpathlineto{\pgfqpoint{1.309708in}{1.010767in}}%
\pgfpathlineto{\pgfqpoint{1.328255in}{1.284171in}}%
\pgfpathlineto{\pgfqpoint{1.348445in}{1.400691in}}%
\pgfpathlineto{\pgfqpoint{1.365818in}{1.332313in}}%
\pgfpathlineto{\pgfqpoint{1.386713in}{1.059492in}}%
\pgfpathlineto{\pgfqpoint{1.405260in}{0.773507in}}%
\pgfpathlineto{\pgfqpoint{1.423806in}{0.592560in}}%
\pgfpathlineto{\pgfqpoint{1.445875in}{0.497147in}}%
\pgfpathlineto{\pgfqpoint{1.464423in}{0.542085in}}%
\pgfpathlineto{\pgfqpoint{1.481562in}{0.643917in}}%
\pgfpathlineto{\pgfqpoint{1.503865in}{0.663195in}}%
\pgfpathlineto{\pgfqpoint{1.520299in}{0.830665in}}%
\pgfpathlineto{\pgfqpoint{1.539784in}{1.122299in}}%
\pgfpathlineto{\pgfqpoint{1.561150in}{1.363789in}}%
\pgfpathlineto{\pgfqpoint{1.578992in}{1.381900in}}%
\pgfpathlineto{\pgfqpoint{1.596835in}{1.273866in}}%
\pgfpathlineto{\pgfqpoint{1.620077in}{0.874415in}}%
\pgfpathlineto{\pgfqpoint{1.634163in}{1.076515in}}%
\pgfpathlineto{\pgfqpoint{1.662101in}{0.711893in}}%
\pgfpathlineto{\pgfqpoint{1.676422in}{0.585807in}}%
\pgfpathlineto{\pgfqpoint{1.694030in}{0.501210in}}%
\pgfpathlineto{\pgfqpoint{1.711873in}{0.502547in}}%
\pgfpathlineto{\pgfqpoint{1.732064in}{0.621674in}}%
\pgfpathlineto{\pgfqpoint{1.751784in}{0.821213in}}%
\pgfpathlineto{\pgfqpoint{1.771270in}{1.088643in}}%
\pgfpathlineto{\pgfqpoint{1.789818in}{1.318142in}}%
\pgfpathlineto{\pgfqpoint{1.809538in}{1.384268in}}%
\pgfpathlineto{\pgfqpoint{1.827617in}{1.329828in}}%
\pgfpathlineto{\pgfqpoint{1.845459in}{1.094138in}}%
\pgfpathlineto{\pgfqpoint{1.865885in}{0.781697in}}%
\pgfpathlineto{\pgfqpoint{1.886779in}{0.625747in}}%
\pgfpathlineto{\pgfqpoint{1.904151in}{0.522586in}}%
\pgfpathlineto{\pgfqpoint{1.925516in}{0.493638in}}%
\pgfpathlineto{\pgfqpoint{1.943124in}{0.574142in}}%
\pgfpathlineto{\pgfqpoint{1.960497in}{0.722375in}}%
\pgfpathlineto{\pgfqpoint{1.982332in}{0.923401in}}%
\pgfpathlineto{\pgfqpoint{2.000174in}{1.205632in}}%
\pgfpathlineto{\pgfqpoint{2.021303in}{1.376522in}}%
\pgfpathlineto{\pgfqpoint{2.038911in}{1.366969in}}%
\pgfpathlineto{\pgfqpoint{2.056519in}{1.283590in}}%
\pgfpathlineto{\pgfqpoint{2.079762in}{0.967981in}}%
\pgfpathlineto{\pgfqpoint{2.095725in}{0.727587in}}%
\pgfpathlineto{\pgfqpoint{2.116856in}{0.575517in}}%
\pgfpathlineto{\pgfqpoint{2.135167in}{0.502396in}}%
\pgfpathlineto{\pgfqpoint{2.156298in}{0.508727in}}%
\pgfpathlineto{\pgfqpoint{2.173201in}{0.599219in}}%
\pgfpathlineto{\pgfqpoint{2.193860in}{0.746125in}}%
\pgfpathlineto{\pgfqpoint{2.210529in}{0.979138in}}%
\pgfpathlineto{\pgfqpoint{2.230016in}{1.222040in}}%
\pgfpathlineto{\pgfqpoint{2.251849in}{1.366509in}}%
\pgfpathlineto{\pgfqpoint{2.269457in}{1.355809in}}%
\pgfpathlineto{\pgfqpoint{2.290116in}{1.147495in}}%
\pgfpathlineto{\pgfqpoint{2.308195in}{0.873210in}}%
\pgfpathlineto{\pgfqpoint{2.326272in}{0.677807in}}%
\pgfpathlineto{\pgfqpoint{2.347401in}{0.544975in}}%
\pgfpathlineto{\pgfqpoint{2.365244in}{0.486243in}}%
\pgfpathlineto{\pgfqpoint{2.385435in}{0.526128in}}%
\pgfpathlineto{\pgfqpoint{2.404217in}{0.621748in}}%
\pgfpathlineto{\pgfqpoint{2.426286in}{0.757438in}}%
\pgfpathlineto{\pgfqpoint{2.442250in}{0.968610in}}%
\pgfpathlineto{\pgfqpoint{2.465728in}{1.260924in}}%
\pgfpathlineto{\pgfqpoint{2.481691in}{1.370044in}}%
\pgfpathlineto{\pgfqpoint{2.500239in}{1.354621in}}%
\pgfpathlineto{\pgfqpoint{2.518081in}{1.177500in}}%
\pgfpathlineto{\pgfqpoint{2.538975in}{0.862166in}}%
\pgfpathlineto{\pgfqpoint{2.560575in}{0.664652in}}%
\pgfpathlineto{\pgfqpoint{2.578418in}{0.545524in}}%
\pgfpathlineto{\pgfqpoint{2.596260in}{0.491997in}}%
\pgfpathlineto{\pgfqpoint{2.620206in}{0.535026in}}%
\pgfpathlineto{\pgfqpoint{2.634997in}{0.619086in}}%
\pgfpathlineto{\pgfqpoint{2.654250in}{0.761150in}}%
\pgfpathlineto{\pgfqpoint{2.673736in}{1.039714in}}%
\pgfpathlineto{\pgfqpoint{2.691344in}{1.236492in}}%
\pgfpathlineto{\pgfqpoint{2.712004in}{1.316294in}}%
\pgfpathlineto{\pgfqpoint{2.731255in}{1.378676in}}%
\pgfpathlineto{\pgfqpoint{2.749098in}{1.353488in}}%
\pgfpathlineto{\pgfqpoint{2.770226in}{1.241821in}}%
\pgfpathlineto{\pgfqpoint{2.788069in}{1.066938in}}%
\pgfpathlineto{\pgfqpoint{2.810138in}{0.828347in}}%
\pgfpathlineto{\pgfqpoint{2.828685in}{0.677055in}}%
\pgfpathlineto{\pgfqpoint{2.845119in}{0.555424in}}%
\pgfpathlineto{\pgfqpoint{2.865779in}{0.492642in}}%
\pgfpathlineto{\pgfqpoint{2.884327in}{0.540331in}}%
\pgfpathlineto{\pgfqpoint{2.904752in}{0.677282in}}%
\pgfpathlineto{\pgfqpoint{2.924941in}{0.865493in}}%
\pgfpathlineto{\pgfqpoint{2.942784in}{1.119906in}}%
\pgfpathlineto{\pgfqpoint{2.961801in}{1.318563in}}%
\pgfpathlineto{\pgfqpoint{2.978940in}{1.385793in}}%
\pgfpathlineto{\pgfqpoint{3.002417in}{1.338988in}}%
\pgfpathlineto{\pgfqpoint{3.021903in}{1.129226in}}%
\pgfpathlineto{\pgfqpoint{3.040451in}{0.891552in}}%
\pgfpathlineto{\pgfqpoint{3.057822in}{0.959461in}}%
\pgfpathlineto{\pgfqpoint{3.078719in}{0.694749in}}%
\pgfpathlineto{\pgfqpoint{3.097499in}{0.578449in}}%
\pgfpathlineto{\pgfqpoint{3.114872in}{0.503666in}}%
\pgfpathlineto{\pgfqpoint{3.136003in}{0.510866in}}%
\pgfpathlineto{\pgfqpoint{3.153609in}{0.555213in}}%
\pgfpathlineto{\pgfqpoint{3.172392in}{0.672216in}}%
\pgfpathlineto{\pgfqpoint{3.193052in}{0.857682in}}%
\pgfpathlineto{\pgfqpoint{3.211128in}{1.105789in}}%
\pgfpathlineto{\pgfqpoint{3.232025in}{1.284018in}}%
\pgfpathlineto{\pgfqpoint{3.250102in}{1.387724in}}%
\pgfpathlineto{\pgfqpoint{3.267241in}{1.374080in}}%
\pgfpathlineto{\pgfqpoint{3.288604in}{1.233174in}}%
\pgfpathlineto{\pgfqpoint{3.306681in}{0.990425in}}%
\pgfpathlineto{\pgfqpoint{3.325463in}{0.775548in}}%
\pgfpathlineto{\pgfqpoint{3.347767in}{0.632350in}}%
\pgfpathlineto{\pgfqpoint{3.367489in}{0.529151in}}%
\pgfpathlineto{\pgfqpoint{3.382748in}{0.501349in}}%
\pgfpathlineto{\pgfqpoint{3.403408in}{0.536708in}}%
\pgfpathlineto{\pgfqpoint{3.422894in}{0.622425in}}%
\pgfpathlineto{\pgfqpoint{3.442850in}{0.790267in}}%
\pgfpathlineto{\pgfqpoint{3.481587in}{1.244704in}}%
\pgfpathlineto{\pgfqpoint{3.500604in}{1.374924in}}%
\pgfpathlineto{\pgfqpoint{3.517272in}{1.399878in}}%
\pgfpathlineto{\pgfqpoint{3.535585in}{1.401172in}}%
\pgfpathlineto{\pgfqpoint{3.556949in}{1.285980in}}%
\pgfpathlineto{\pgfqpoint{3.596157in}{0.787040in}}%
\pgfpathlineto{\pgfqpoint{3.615642in}{0.663438in}}%
\pgfpathlineto{\pgfqpoint{3.635833in}{0.566809in}}%
\pgfpathlineto{\pgfqpoint{3.653676in}{0.505799in}}%
\pgfpathlineto{\pgfqpoint{3.674570in}{0.543005in}}%
\pgfpathlineto{\pgfqpoint{3.692647in}{0.634681in}}%
\pgfpathlineto{\pgfqpoint{3.710960in}{0.766855in}}%
\pgfpathlineto{\pgfqpoint{3.770826in}{1.345647in}}%
\pgfpathlineto{\pgfqpoint{3.788200in}{1.409443in}}%
\pgfpathlineto{\pgfqpoint{3.809094in}{1.416417in}}%
\pgfpathlineto{\pgfqpoint{3.828816in}{1.326168in}}%
\pgfpathlineto{\pgfqpoint{3.866850in}{0.879788in}}%
\pgfpathlineto{\pgfqpoint{3.884456in}{0.722787in}}%
\pgfpathlineto{\pgfqpoint{3.903473in}{0.602966in}}%
\pgfpathlineto{\pgfqpoint{3.922020in}{0.641329in}}%
\pgfpathlineto{\pgfqpoint{3.939159in}{0.543749in}}%
\pgfpathlineto{\pgfqpoint{3.960757in}{0.514620in}}%
\pgfpathlineto{\pgfqpoint{3.980714in}{0.571049in}}%
\pgfpathlineto{\pgfqpoint{3.999731in}{0.664568in}}%
\pgfpathlineto{\pgfqpoint{4.038702in}{0.998962in}}%
\pgfpathlineto{\pgfqpoint{4.056544in}{1.231245in}}%
\pgfpathlineto{\pgfqpoint{4.077439in}{1.391888in}}%
\pgfpathlineto{\pgfqpoint{4.094109in}{1.438505in}}%
\pgfpathlineto{\pgfqpoint{4.115943in}{1.400726in}}%
\pgfpathlineto{\pgfqpoint{4.134960in}{1.281647in}}%
\pgfpathlineto{\pgfqpoint{4.154680in}{1.090561in}}%
\pgfpathlineto{\pgfqpoint{4.177218in}{0.824246in}}%
\pgfpathlineto{\pgfqpoint{4.193417in}{0.692740in}}%
\pgfpathlineto{\pgfqpoint{4.212199in}{0.621438in}}%
\pgfpathlineto{\pgfqpoint{4.228164in}{0.543992in}}%
\pgfpathlineto{\pgfqpoint{4.249998in}{0.553150in}}%
\pgfpathlineto{\pgfqpoint{4.268544in}{0.653730in}}%
\pgfpathlineto{\pgfqpoint{4.288501in}{0.768223in}}%
\pgfpathlineto{\pgfqpoint{4.306578in}{0.945631in}}%
\pgfpathlineto{\pgfqpoint{4.325829in}{1.112598in}}%
\pgfpathlineto{\pgfqpoint{4.345314in}{1.249371in}}%
\pgfpathlineto{\pgfqpoint{4.364566in}{1.416971in}}%
\pgfpathlineto{\pgfqpoint{4.383348in}{1.456355in}}%
\pgfpathlineto{\pgfqpoint{4.402130in}{1.435375in}}%
\pgfpathlineto{\pgfqpoint{4.421616in}{1.303697in}}%
\pgfpathlineto{\pgfqpoint{4.441102in}{1.241355in}}%
\pgfpathlineto{\pgfqpoint{4.459415in}{0.994908in}}%
\pgfpathlineto{\pgfqpoint{4.479840in}{0.763916in}}%
\pgfpathlineto{\pgfqpoint{4.473971in}{0.880316in}}%
\pgfpathlineto{\pgfqpoint{4.438989in}{1.458072in}}%
\pgfpathlineto{\pgfqpoint{4.416215in}{1.049879in}}%
\pgfpathlineto{\pgfqpoint{4.397435in}{0.631414in}}%
\pgfpathlineto{\pgfqpoint{4.376539in}{0.526410in}}%
\pgfpathlineto{\pgfqpoint{4.358227in}{0.609637in}}%
\pgfpathlineto{\pgfqpoint{4.340619in}{0.790739in}}%
\pgfpathlineto{\pgfqpoint{4.316908in}{1.192304in}}%
\pgfpathlineto{\pgfqpoint{4.303057in}{1.384170in}}%
\pgfpathlineto{\pgfqpoint{4.283335in}{1.436605in}}%
\pgfpathlineto{\pgfqpoint{4.265258in}{1.306771in}}%
\pgfpathlineto{\pgfqpoint{4.242720in}{0.949168in}}%
\pgfpathlineto{\pgfqpoint{4.224876in}{0.686560in}}%
\pgfpathlineto{\pgfqpoint{4.204216in}{0.552239in}}%
\pgfpathlineto{\pgfqpoint{4.184965in}{0.543523in}}%
\pgfpathlineto{\pgfqpoint{4.166653in}{0.671408in}}%
\pgfpathlineto{\pgfqpoint{4.149280in}{0.895143in}}%
\pgfpathlineto{\pgfqpoint{4.129089in}{1.257923in}}%
\pgfpathlineto{\pgfqpoint{4.109837in}{1.414769in}}%
\pgfpathlineto{\pgfqpoint{4.088005in}{1.368478in}}%
\pgfpathlineto{\pgfqpoint{4.052318in}{0.845371in}}%
\pgfpathlineto{\pgfqpoint{4.034710in}{0.644764in}}%
\pgfpathlineto{\pgfqpoint{4.012407in}{0.531043in}}%
\pgfpathlineto{\pgfqpoint{3.991044in}{0.534209in}}%
\pgfpathlineto{\pgfqpoint{3.973436in}{0.664426in}}%
\pgfpathlineto{\pgfqpoint{3.952776in}{0.913314in}}%
\pgfpathlineto{\pgfqpoint{3.938454in}{1.189982in}}%
\pgfpathlineto{\pgfqpoint{3.914039in}{1.403401in}}%
\pgfpathlineto{\pgfqpoint{3.897840in}{1.389228in}}%
\pgfpathlineto{\pgfqpoint{3.880232in}{1.202815in}}%
\pgfpathlineto{\pgfqpoint{3.858866in}{0.844137in}}%
\pgfpathlineto{\pgfqpoint{3.836329in}{0.609755in}}%
\pgfpathlineto{\pgfqpoint{3.822242in}{0.526696in}}%
\pgfpathlineto{\pgfqpoint{3.800407in}{0.510817in}}%
\pgfpathlineto{\pgfqpoint{3.783036in}{0.563963in}}%
\pgfpathlineto{\pgfqpoint{3.764488in}{0.723816in}}%
\pgfpathlineto{\pgfqpoint{3.723871in}{1.293546in}}%
\pgfpathlineto{\pgfqpoint{3.704857in}{1.399444in}}%
\pgfpathlineto{\pgfqpoint{3.686543in}{1.347046in}}%
\pgfpathlineto{\pgfqpoint{3.665649in}{1.139367in}}%
\pgfpathlineto{\pgfqpoint{3.648275in}{0.890375in}}%
\pgfpathlineto{\pgfqpoint{3.608130in}{0.574011in}}%
\pgfpathlineto{\pgfqpoint{3.587939in}{0.495982in}}%
\pgfpathlineto{\pgfqpoint{3.571270in}{0.546564in}}%
\pgfpathlineto{\pgfqpoint{3.551550in}{0.677049in}}%
\pgfpathlineto{\pgfqpoint{3.532534in}{0.832466in}}%
\pgfpathlineto{\pgfqpoint{3.512812in}{1.138528in}}%
\pgfpathlineto{\pgfqpoint{3.493091in}{1.328386in}}%
\pgfpathlineto{\pgfqpoint{3.474778in}{1.376493in}}%
\pgfpathlineto{\pgfqpoint{3.456232in}{1.368241in}}%
\pgfpathlineto{\pgfqpoint{3.436981in}{1.166402in}}%
\pgfpathlineto{\pgfqpoint{3.417495in}{0.853829in}}%
\pgfpathlineto{\pgfqpoint{3.398008in}{0.660703in}}%
\pgfpathlineto{\pgfqpoint{3.375939in}{0.545218in}}%
\pgfpathlineto{\pgfqpoint{3.357628in}{0.491135in}}%
\pgfpathlineto{\pgfqpoint{3.338376in}{0.542061in}}%
\pgfpathlineto{\pgfqpoint{3.321943in}{0.664294in}}%
\pgfpathlineto{\pgfqpoint{3.299639in}{0.849518in}}%
\pgfpathlineto{\pgfqpoint{3.282500in}{0.678948in}}%
\pgfpathlineto{\pgfqpoint{3.262780in}{0.939778in}}%
\pgfpathlineto{\pgfqpoint{3.241884in}{1.245493in}}%
\pgfpathlineto{\pgfqpoint{3.224747in}{1.377827in}}%
\pgfpathlineto{\pgfqpoint{3.206199in}{1.376771in}}%
\pgfpathlineto{\pgfqpoint{3.183427in}{1.205942in}}%
\pgfpathlineto{\pgfqpoint{3.165114in}{0.916662in}}%
\pgfpathlineto{\pgfqpoint{3.149854in}{0.735339in}}%
\pgfpathlineto{\pgfqpoint{3.128725in}{0.586334in}}%
\pgfpathlineto{\pgfqpoint{3.109003in}{0.496535in}}%
\pgfpathlineto{\pgfqpoint{3.089986in}{0.517580in}}%
\pgfpathlineto{\pgfqpoint{3.070735in}{0.620072in}}%
\pgfpathlineto{\pgfqpoint{3.053362in}{0.773176in}}%
\pgfpathlineto{\pgfqpoint{3.033876in}{1.015309in}}%
\pgfpathlineto{\pgfqpoint{3.013921in}{0.517500in}}%
\pgfpathlineto{\pgfqpoint{2.993496in}{0.621445in}}%
\pgfpathlineto{\pgfqpoint{2.975419in}{0.736992in}}%
\pgfpathlineto{\pgfqpoint{2.956637in}{0.993191in}}%
\pgfpathlineto{\pgfqpoint{2.934568in}{1.311727in}}%
\pgfpathlineto{\pgfqpoint{2.916725in}{1.381521in}}%
\pgfpathlineto{\pgfqpoint{2.897943in}{1.284646in}}%
\pgfpathlineto{\pgfqpoint{2.879632in}{1.023190in}}%
\pgfpathlineto{\pgfqpoint{2.860144in}{0.768568in}}%
\pgfpathlineto{\pgfqpoint{2.840658in}{0.600968in}}%
\pgfpathlineto{\pgfqpoint{2.819999in}{0.497918in}}%
\pgfpathlineto{\pgfqpoint{2.801451in}{0.506964in}}%
\pgfpathlineto{\pgfqpoint{2.785957in}{0.556405in}}%
\pgfpathlineto{\pgfqpoint{2.763888in}{0.720268in}}%
\pgfpathlineto{\pgfqpoint{2.723977in}{1.272618in}}%
\pgfpathlineto{\pgfqpoint{2.706134in}{1.378500in}}%
\pgfpathlineto{\pgfqpoint{2.687821in}{1.349271in}}%
\pgfpathlineto{\pgfqpoint{2.666458in}{1.136017in}}%
\pgfpathlineto{\pgfqpoint{2.645093in}{0.806604in}}%
\pgfpathlineto{\pgfqpoint{2.625372in}{0.633282in}}%
\pgfpathlineto{\pgfqpoint{2.607061in}{0.521688in}}%
\pgfpathlineto{\pgfqpoint{2.591565in}{0.486114in}}%
\pgfpathlineto{\pgfqpoint{2.571374in}{0.503131in}}%
\pgfpathlineto{\pgfqpoint{2.553063in}{0.595591in}}%
\pgfpathlineto{\pgfqpoint{2.532872in}{0.750611in}}%
\pgfpathlineto{\pgfqpoint{2.512681in}{0.963804in}}%
\pgfpathlineto{\pgfqpoint{2.494135in}{1.257237in}}%
\pgfpathlineto{\pgfqpoint{2.475587in}{1.376505in}}%
\pgfpathlineto{\pgfqpoint{2.456807in}{1.341630in}}%
\pgfpathlineto{\pgfqpoint{2.434738in}{1.099482in}}%
\pgfpathlineto{\pgfqpoint{2.420182in}{0.986205in}}%
\pgfpathlineto{\pgfqpoint{2.398348in}{0.723886in}}%
\pgfpathlineto{\pgfqpoint{2.379331in}{0.587870in}}%
\pgfpathlineto{\pgfqpoint{2.360314in}{0.504039in}}%
\pgfpathlineto{\pgfqpoint{2.342706in}{0.501232in}}%
\pgfpathlineto{\pgfqpoint{2.320403in}{0.596452in}}%
\pgfpathlineto{\pgfqpoint{2.301152in}{0.735426in}}%
\pgfpathlineto{\pgfqpoint{2.264527in}{1.267038in}}%
\pgfpathlineto{\pgfqpoint{2.245510in}{1.373588in}}%
\pgfpathlineto{\pgfqpoint{2.227433in}{1.363464in}}%
\pgfpathlineto{\pgfqpoint{2.209120in}{1.210193in}}%
\pgfpathlineto{\pgfqpoint{2.187522in}{1.017280in}}%
\pgfpathlineto{\pgfqpoint{2.165688in}{0.743290in}}%
\pgfpathlineto{\pgfqpoint{2.149020in}{0.620986in}}%
\pgfpathlineto{\pgfqpoint{2.128594in}{0.521441in}}%
\pgfpathlineto{\pgfqpoint{2.109578in}{0.489202in}}%
\pgfpathlineto{\pgfqpoint{2.091030in}{0.539859in}}%
\pgfpathlineto{\pgfqpoint{2.072249in}{0.643244in}}%
\pgfpathlineto{\pgfqpoint{2.053702in}{0.827485in}}%
\pgfpathlineto{\pgfqpoint{2.032573in}{1.125371in}}%
\pgfpathlineto{\pgfqpoint{2.014259in}{1.332578in}}%
\pgfpathlineto{\pgfqpoint{1.996417in}{1.387186in}}%
\pgfpathlineto{\pgfqpoint{1.975522in}{1.316886in}}%
\pgfpathlineto{\pgfqpoint{1.956506in}{1.190287in}}%
\pgfpathlineto{\pgfqpoint{1.937020in}{0.920081in}}%
\pgfpathlineto{\pgfqpoint{1.919178in}{0.766273in}}%
\pgfpathlineto{\pgfqpoint{1.899455in}{0.617259in}}%
\pgfpathlineto{\pgfqpoint{1.877152in}{0.524508in}}%
\pgfpathlineto{\pgfqpoint{1.862831in}{0.500543in}}%
\pgfpathlineto{\pgfqpoint{1.840528in}{0.551123in}}%
\pgfpathlineto{\pgfqpoint{1.823625in}{0.636815in}}%
\pgfpathlineto{\pgfqpoint{1.802260in}{0.825519in}}%
\pgfpathlineto{\pgfqpoint{1.781600in}{1.104609in}}%
\pgfpathlineto{\pgfqpoint{1.763992in}{1.298273in}}%
\pgfpathlineto{\pgfqpoint{1.745446in}{1.385760in}}%
\pgfpathlineto{\pgfqpoint{1.727603in}{1.386380in}}%
\pgfpathlineto{\pgfqpoint{1.704595in}{1.277721in}}%
\pgfpathlineto{\pgfqpoint{1.689335in}{1.160532in}}%
\pgfpathlineto{\pgfqpoint{1.665624in}{0.852795in}}%
\pgfpathlineto{\pgfqpoint{1.649659in}{0.744774in}}%
\pgfpathlineto{\pgfqpoint{1.630407in}{0.631228in}}%
\pgfpathlineto{\pgfqpoint{1.608808in}{0.524662in}}%
\pgfpathlineto{\pgfqpoint{1.591200in}{0.505386in}}%
\pgfpathlineto{\pgfqpoint{1.572417in}{0.549960in}}%
\pgfpathlineto{\pgfqpoint{1.554575in}{0.627393in}}%
\pgfpathlineto{\pgfqpoint{1.534149in}{0.798174in}}%
\pgfpathlineto{\pgfqpoint{1.515369in}{0.925110in}}%
\pgfpathlineto{\pgfqpoint{1.495647in}{1.019799in}}%
\pgfpathlineto{\pgfqpoint{1.474753in}{1.288496in}}%
\pgfpathlineto{\pgfqpoint{1.454798in}{1.400751in}}%
\pgfpathlineto{\pgfqpoint{1.436485in}{1.400011in}}%
\pgfpathlineto{\pgfqpoint{1.416294in}{1.315918in}}%
\pgfpathlineto{\pgfqpoint{1.400331in}{1.129688in}}%
\pgfpathlineto{\pgfqpoint{1.382017in}{0.967516in}}%
\pgfpathlineto{\pgfqpoint{1.359480in}{0.752984in}}%
\pgfpathlineto{\pgfqpoint{1.341637in}{0.620455in}}%
\pgfpathlineto{\pgfqpoint{1.322855in}{0.543535in}}%
\pgfpathlineto{\pgfqpoint{1.303838in}{0.509089in}}%
\pgfpathlineto{\pgfqpoint{1.282944in}{0.577654in}}%
\pgfpathlineto{\pgfqpoint{1.266979in}{0.570602in}}%
\pgfpathlineto{\pgfqpoint{1.245379in}{0.578465in}}%
\pgfpathlineto{\pgfqpoint{1.223547in}{0.742367in}}%
\pgfpathlineto{\pgfqpoint{1.205234in}{0.943179in}}%
\pgfpathlineto{\pgfqpoint{1.187157in}{0.687668in}}%
\pgfpathlineto{\pgfqpoint{1.166731in}{0.859190in}}%
\pgfpathlineto{\pgfqpoint{1.148889in}{1.053079in}}%
\pgfpathlineto{\pgfqpoint{1.130341in}{1.308123in}}%
\pgfpathlineto{\pgfqpoint{1.110386in}{1.419895in}}%
\pgfpathlineto{\pgfqpoint{1.091370in}{1.413397in}}%
\pgfpathlineto{\pgfqpoint{1.072353in}{1.310356in}}%
\pgfpathlineto{\pgfqpoint{1.051927in}{1.059672in}}%
\pgfpathlineto{\pgfqpoint{1.033616in}{0.851138in}}%
\pgfpathlineto{\pgfqpoint{1.012251in}{0.678533in}}%
\pgfpathlineto{\pgfqpoint{0.996286in}{0.592597in}}%
\pgfpathlineto{\pgfqpoint{0.973983in}{0.527035in}}%
\pgfpathlineto{\pgfqpoint{0.955671in}{0.544335in}}%
\pgfpathlineto{\pgfqpoint{0.937829in}{0.636678in}}%
\pgfpathlineto{\pgfqpoint{0.919750in}{0.783798in}}%
\pgfpathlineto{\pgfqpoint{0.898856in}{0.995509in}}%
\pgfpathlineto{\pgfqpoint{0.881013in}{1.209260in}}%
\pgfpathlineto{\pgfqpoint{0.859179in}{1.400818in}}%
\pgfpathlineto{\pgfqpoint{0.842511in}{1.443732in}}%
\pgfpathlineto{\pgfqpoint{0.821382in}{1.394214in}}%
\pgfpathlineto{\pgfqpoint{0.803303in}{1.289916in}}%
\pgfpathlineto{\pgfqpoint{0.786400in}{1.141965in}}%
\pgfpathlineto{\pgfqpoint{0.765975in}{0.939162in}}%
\pgfpathlineto{\pgfqpoint{0.745784in}{0.772344in}}%
\pgfpathlineto{\pgfqpoint{0.728881in}{0.647862in}}%
\pgfpathlineto{\pgfqpoint{0.707987in}{0.546436in}}%
\pgfpathlineto{\pgfqpoint{0.690379in}{0.534274in}}%
\pgfpathlineto{\pgfqpoint{0.668779in}{0.605294in}}%
\pgfpathlineto{\pgfqpoint{0.651876in}{0.710348in}}%
\pgfpathlineto{\pgfqpoint{0.655163in}{0.653238in}}%
\pgfpathlineto{\pgfqpoint{0.674179in}{1.342367in}}%
\pgfpathlineto{\pgfqpoint{0.695308in}{1.445693in}}%
\pgfpathlineto{\pgfqpoint{0.714091in}{1.336915in}}%
\pgfpathlineto{\pgfqpoint{0.732402in}{1.013531in}}%
\pgfpathlineto{\pgfqpoint{0.751888in}{0.730151in}}%
\pgfpathlineto{\pgfqpoint{0.771375in}{0.560403in}}%
\pgfpathlineto{\pgfqpoint{0.789452in}{0.537750in}}%
\pgfpathlineto{\pgfqpoint{0.810347in}{0.670934in}}%
\pgfpathlineto{\pgfqpoint{0.830303in}{0.935707in}}%
\pgfpathlineto{\pgfqpoint{0.849083in}{1.261025in}}%
\pgfpathlineto{\pgfqpoint{0.868571in}{1.423178in}}%
\pgfpathlineto{\pgfqpoint{0.887822in}{1.396769in}}%
\pgfpathlineto{\pgfqpoint{0.905196in}{1.162980in}}%
\pgfpathlineto{\pgfqpoint{0.925619in}{0.828999in}}%
\pgfpathlineto{\pgfqpoint{0.944871in}{0.621601in}}%
\pgfpathlineto{\pgfqpoint{0.963887in}{0.514944in}}%
\pgfpathlineto{\pgfqpoint{0.982904in}{0.573552in}}%
\pgfpathlineto{\pgfqpoint{1.002392in}{0.729880in}}%
\pgfpathlineto{\pgfqpoint{1.021172in}{0.977885in}}%
\pgfpathlineto{\pgfqpoint{1.039251in}{1.283897in}}%
\pgfpathlineto{\pgfqpoint{1.058737in}{1.413665in}}%
\pgfpathlineto{\pgfqpoint{1.080805in}{1.324544in}}%
\pgfpathlineto{\pgfqpoint{1.099822in}{1.412822in}}%
\pgfpathlineto{\pgfqpoint{1.117430in}{1.318424in}}%
\pgfpathlineto{\pgfqpoint{1.154758in}{0.728294in}}%
\pgfpathlineto{\pgfqpoint{1.176592in}{0.548778in}}%
\pgfpathlineto{\pgfqpoint{1.192321in}{0.501541in}}%
\pgfpathlineto{\pgfqpoint{1.213921in}{0.591647in}}%
\pgfpathlineto{\pgfqpoint{1.233172in}{0.750420in}}%
\pgfpathlineto{\pgfqpoint{1.252189in}{0.988629in}}%
\pgfpathlineto{\pgfqpoint{1.270971in}{1.281039in}}%
\pgfpathlineto{\pgfqpoint{1.289048in}{1.399539in}}%
\pgfpathlineto{\pgfqpoint{1.311116in}{1.330571in}}%
\pgfpathlineto{\pgfqpoint{1.329899in}{1.061309in}}%
\pgfpathlineto{\pgfqpoint{1.348915in}{0.771549in}}%
\pgfpathlineto{\pgfqpoint{1.368167in}{0.594613in}}%
\pgfpathlineto{\pgfqpoint{1.387887in}{0.513479in}}%
\pgfpathlineto{\pgfqpoint{1.405495in}{0.514744in}}%
\pgfpathlineto{\pgfqpoint{1.424980in}{0.614759in}}%
\pgfpathlineto{\pgfqpoint{1.442354in}{0.746258in}}%
\pgfpathlineto{\pgfqpoint{1.465128in}{1.028052in}}%
\pgfpathlineto{\pgfqpoint{1.482736in}{1.284831in}}%
\pgfpathlineto{\pgfqpoint{1.501282in}{1.391272in}}%
\pgfpathlineto{\pgfqpoint{1.523116in}{1.356925in}}%
\pgfpathlineto{\pgfqpoint{1.542602in}{1.146748in}}%
\pgfpathlineto{\pgfqpoint{1.560210in}{0.835521in}}%
\pgfpathlineto{\pgfqpoint{1.578521in}{0.638814in}}%
\pgfpathlineto{\pgfqpoint{1.600355in}{0.518760in}}%
\pgfpathlineto{\pgfqpoint{1.618434in}{0.498928in}}%
\pgfpathlineto{\pgfqpoint{1.634163in}{0.534201in}}%
\pgfpathlineto{\pgfqpoint{1.657406in}{0.681269in}}%
\pgfpathlineto{\pgfqpoint{1.673370in}{0.847281in}}%
\pgfpathlineto{\pgfqpoint{1.694499in}{1.161855in}}%
\pgfpathlineto{\pgfqpoint{1.713282in}{1.354125in}}%
\pgfpathlineto{\pgfqpoint{1.736525in}{1.381553in}}%
\pgfpathlineto{\pgfqpoint{1.753427in}{1.295229in}}%
\pgfpathlineto{\pgfqpoint{1.770566in}{1.049402in}}%
\pgfpathlineto{\pgfqpoint{1.788409in}{0.935304in}}%
\pgfpathlineto{\pgfqpoint{1.809303in}{0.732018in}}%
\pgfpathlineto{\pgfqpoint{1.827380in}{0.615406in}}%
\pgfpathlineto{\pgfqpoint{1.845694in}{0.512073in}}%
\pgfpathlineto{\pgfqpoint{1.867762in}{0.498444in}}%
\pgfpathlineto{\pgfqpoint{1.884430in}{0.561051in}}%
\pgfpathlineto{\pgfqpoint{1.904621in}{0.681650in}}%
\pgfpathlineto{\pgfqpoint{1.923167in}{0.809657in}}%
\pgfpathlineto{\pgfqpoint{1.944533in}{1.118957in}}%
\pgfpathlineto{\pgfqpoint{1.961435in}{1.211505in}}%
\pgfpathlineto{\pgfqpoint{1.982566in}{1.375166in}}%
\pgfpathlineto{\pgfqpoint{2.000643in}{1.351492in}}%
\pgfpathlineto{\pgfqpoint{2.021068in}{1.133206in}}%
\pgfpathlineto{\pgfqpoint{2.038911in}{0.888490in}}%
\pgfpathlineto{\pgfqpoint{2.057693in}{0.677508in}}%
\pgfpathlineto{\pgfqpoint{2.078353in}{0.541653in}}%
\pgfpathlineto{\pgfqpoint{2.097134in}{0.488099in}}%
\pgfpathlineto{\pgfqpoint{2.117559in}{1.328354in}}%
\pgfpathlineto{\pgfqpoint{2.136810in}{1.145283in}}%
\pgfpathlineto{\pgfqpoint{2.156298in}{0.826025in}}%
\pgfpathlineto{\pgfqpoint{2.174140in}{0.633165in}}%
\pgfpathlineto{\pgfqpoint{2.195269in}{0.515668in}}%
\pgfpathlineto{\pgfqpoint{2.213817in}{0.496334in}}%
\pgfpathlineto{\pgfqpoint{2.230485in}{0.579213in}}%
\pgfpathlineto{\pgfqpoint{2.251614in}{0.747805in}}%
\pgfpathlineto{\pgfqpoint{2.293874in}{1.321288in}}%
\pgfpathlineto{\pgfqpoint{2.309368in}{1.380577in}}%
\pgfpathlineto{\pgfqpoint{2.328384in}{1.313099in}}%
\pgfpathlineto{\pgfqpoint{2.349281in}{1.164838in}}%
\pgfpathlineto{\pgfqpoint{2.367827in}{0.882567in}}%
\pgfpathlineto{\pgfqpoint{2.388487in}{0.679786in}}%
\pgfpathlineto{\pgfqpoint{2.407034in}{0.542561in}}%
\pgfpathlineto{\pgfqpoint{2.422999in}{0.488408in}}%
\pgfpathlineto{\pgfqpoint{2.444128in}{0.528715in}}%
\pgfpathlineto{\pgfqpoint{2.461736in}{0.647576in}}%
\pgfpathlineto{\pgfqpoint{2.480753in}{0.817108in}}%
\pgfpathlineto{\pgfqpoint{2.503996in}{1.164108in}}%
\pgfpathlineto{\pgfqpoint{2.522542in}{1.345895in}}%
\pgfpathlineto{\pgfqpoint{2.521604in}{1.374143in}}%
\pgfpathlineto{\pgfqpoint{2.539210in}{1.378729in}}%
\pgfpathlineto{\pgfqpoint{2.560106in}{1.242815in}}%
\pgfpathlineto{\pgfqpoint{2.578183in}{0.961691in}}%
\pgfpathlineto{\pgfqpoint{2.595791in}{0.857016in}}%
\pgfpathlineto{\pgfqpoint{2.616920in}{0.634028in}}%
\pgfpathlineto{\pgfqpoint{2.634294in}{0.532867in}}%
\pgfpathlineto{\pgfqpoint{2.652605in}{0.486770in}}%
\pgfpathlineto{\pgfqpoint{2.673501in}{0.560526in}}%
\pgfpathlineto{\pgfqpoint{2.692282in}{0.681783in}}%
\pgfpathlineto{\pgfqpoint{2.712707in}{0.896847in}}%
\pgfpathlineto{\pgfqpoint{2.731255in}{1.147694in}}%
\pgfpathlineto{\pgfqpoint{2.749098in}{1.328583in}}%
\pgfpathlineto{\pgfqpoint{2.769289in}{1.375384in}}%
\pgfpathlineto{\pgfqpoint{2.791121in}{1.223613in}}%
\pgfpathlineto{\pgfqpoint{2.808729in}{0.996477in}}%
\pgfpathlineto{\pgfqpoint{2.826808in}{0.735558in}}%
\pgfpathlineto{\pgfqpoint{2.847233in}{0.588527in}}%
\pgfpathlineto{\pgfqpoint{2.865310in}{0.520431in}}%
\pgfpathlineto{\pgfqpoint{2.884561in}{0.493999in}}%
\pgfpathlineto{\pgfqpoint{2.904516in}{0.575353in}}%
\pgfpathlineto{\pgfqpoint{2.922829in}{0.689842in}}%
\pgfpathlineto{\pgfqpoint{2.943255in}{0.877598in}}%
\pgfpathlineto{\pgfqpoint{2.964149in}{1.083212in}}%
\pgfpathlineto{\pgfqpoint{2.979409in}{1.287092in}}%
\pgfpathlineto{\pgfqpoint{3.000774in}{1.385144in}}%
\pgfpathlineto{\pgfqpoint{3.021434in}{1.339523in}}%
\pgfpathlineto{\pgfqpoint{3.039511in}{1.145237in}}%
\pgfpathlineto{\pgfqpoint{3.057353in}{0.890601in}}%
\pgfpathlineto{\pgfqpoint{3.078013in}{0.774744in}}%
\pgfpathlineto{\pgfqpoint{3.097264in}{0.616156in}}%
\pgfpathlineto{\pgfqpoint{3.116986in}{0.519873in}}%
\pgfpathlineto{\pgfqpoint{3.135767in}{0.501251in}}%
\pgfpathlineto{\pgfqpoint{3.153609in}{0.576560in}}%
\pgfpathlineto{\pgfqpoint{3.174740in}{0.721302in}}%
\pgfpathlineto{\pgfqpoint{3.193522in}{0.810784in}}%
\pgfpathlineto{\pgfqpoint{3.211128in}{1.047504in}}%
\pgfpathlineto{\pgfqpoint{3.231788in}{1.302435in}}%
\pgfpathlineto{\pgfqpoint{3.250102in}{1.385148in}}%
\pgfpathlineto{\pgfqpoint{3.270056in}{1.360539in}}%
\pgfpathlineto{\pgfqpoint{3.286492in}{1.188214in}}%
\pgfpathlineto{\pgfqpoint{3.308561in}{0.988849in}}%
\pgfpathlineto{\pgfqpoint{3.324760in}{0.812801in}}%
\pgfpathlineto{\pgfqpoint{3.345420in}{0.628641in}}%
\pgfpathlineto{\pgfqpoint{3.366783in}{0.523012in}}%
\pgfpathlineto{\pgfqpoint{3.386503in}{0.501100in}}%
\pgfpathlineto{\pgfqpoint{3.402939in}{0.568523in}}%
\pgfpathlineto{\pgfqpoint{3.424068in}{0.671337in}}%
\pgfpathlineto{\pgfqpoint{3.440971in}{0.791043in}}%
\pgfpathlineto{\pgfqpoint{3.461631in}{0.989081in}}%
\pgfpathlineto{\pgfqpoint{3.480884in}{1.232697in}}%
\pgfpathlineto{\pgfqpoint{3.498726in}{1.369386in}}%
\pgfpathlineto{\pgfqpoint{3.519152in}{1.405496in}}%
\pgfpathlineto{\pgfqpoint{3.538403in}{1.351851in}}%
\pgfpathlineto{\pgfqpoint{3.596860in}{0.751021in}}%
\pgfpathlineto{\pgfqpoint{3.615408in}{1.399332in}}%
\pgfpathlineto{\pgfqpoint{3.634190in}{1.395179in}}%
\pgfpathlineto{\pgfqpoint{3.652736in}{1.267330in}}%
\pgfpathlineto{\pgfqpoint{3.674101in}{0.952363in}}%
\pgfpathlineto{\pgfqpoint{3.694761in}{0.745282in}}%
\pgfpathlineto{\pgfqpoint{3.711898in}{0.614447in}}%
\pgfpathlineto{\pgfqpoint{3.729741in}{0.522135in}}%
\pgfpathlineto{\pgfqpoint{3.749697in}{0.530375in}}%
\pgfpathlineto{\pgfqpoint{3.768009in}{0.632389in}}%
\pgfpathlineto{\pgfqpoint{3.787025in}{0.754459in}}%
\pgfpathlineto{\pgfqpoint{3.807451in}{0.980005in}}%
\pgfpathlineto{\pgfqpoint{3.825059in}{1.218607in}}%
\pgfpathlineto{\pgfqpoint{3.846659in}{1.379476in}}%
\pgfpathlineto{\pgfqpoint{3.864736in}{1.422091in}}%
\pgfpathlineto{\pgfqpoint{3.885161in}{1.344936in}}%
\pgfpathlineto{\pgfqpoint{3.904178in}{1.163386in}}%
\pgfpathlineto{\pgfqpoint{3.942680in}{0.743394in}}%
\pgfpathlineto{\pgfqpoint{3.962871in}{0.601864in}}%
\pgfpathlineto{\pgfqpoint{3.980714in}{0.533773in}}%
\pgfpathlineto{\pgfqpoint{3.999965in}{0.520803in}}%
\pgfpathlineto{\pgfqpoint{4.016868in}{0.582500in}}%
\pgfpathlineto{\pgfqpoint{4.058188in}{0.862445in}}%
\pgfpathlineto{\pgfqpoint{4.081196in}{1.157188in}}%
\pgfpathlineto{\pgfqpoint{4.098570in}{1.334832in}}%
\pgfpathlineto{\pgfqpoint{4.114534in}{1.425973in}}%
\pgfpathlineto{\pgfqpoint{4.135429in}{1.315365in}}%
\pgfpathlineto{\pgfqpoint{4.154446in}{1.364931in}}%
\pgfpathlineto{\pgfqpoint{4.174635in}{1.440342in}}%
\pgfpathlineto{\pgfqpoint{4.195295in}{1.379994in}}%
\pgfpathlineto{\pgfqpoint{4.211025in}{1.254603in}}%
\pgfpathlineto{\pgfqpoint{4.228867in}{1.005091in}}%
\pgfpathlineto{\pgfqpoint{4.249762in}{0.793726in}}%
\pgfpathlineto{\pgfqpoint{4.267604in}{0.654560in}}%
\pgfpathlineto{\pgfqpoint{4.292021in}{0.540117in}}%
\pgfpathlineto{\pgfqpoint{4.307281in}{0.532228in}}%
\pgfpathlineto{\pgfqpoint{4.326769in}{0.629305in}}%
\pgfpathlineto{\pgfqpoint{4.343202in}{0.754533in}}%
\pgfpathlineto{\pgfqpoint{4.383817in}{1.156353in}}%
\pgfpathlineto{\pgfqpoint{4.401425in}{1.368953in}}%
\pgfpathlineto{\pgfqpoint{4.421381in}{1.450075in}}%
\pgfpathlineto{\pgfqpoint{4.444624in}{1.431291in}}%
\pgfpathlineto{\pgfqpoint{4.462232in}{1.295183in}}%
\pgfpathlineto{\pgfqpoint{4.481013in}{1.129165in}}%
\pgfpathlineto{\pgfqpoint{4.482421in}{1.142825in}}%
\pgfpathlineto{\pgfqpoint{4.473266in}{1.278363in}}%
\pgfpathlineto{\pgfqpoint{4.455892in}{1.437995in}}%
\pgfpathlineto{\pgfqpoint{4.436875in}{1.428658in}}%
\pgfpathlineto{\pgfqpoint{4.419267in}{1.155250in}}%
\pgfpathlineto{\pgfqpoint{4.396495in}{0.862359in}}%
\pgfpathlineto{\pgfqpoint{4.376539in}{0.640585in}}%
\pgfpathlineto{\pgfqpoint{4.360340in}{0.543316in}}%
\pgfpathlineto{\pgfqpoint{4.340150in}{0.584706in}}%
\pgfpathlineto{\pgfqpoint{4.321837in}{0.733718in}}%
\pgfpathlineto{\pgfqpoint{4.282160in}{1.351961in}}%
\pgfpathlineto{\pgfqpoint{4.261266in}{1.440033in}}%
\pgfpathlineto{\pgfqpoint{4.242955in}{1.340463in}}%
\pgfpathlineto{\pgfqpoint{4.223233in}{0.996599in}}%
\pgfpathlineto{\pgfqpoint{4.204921in}{0.754373in}}%
\pgfpathlineto{\pgfqpoint{4.188253in}{0.597487in}}%
\pgfpathlineto{\pgfqpoint{4.167122in}{0.515026in}}%
\pgfpathlineto{\pgfqpoint{4.148811in}{0.623908in}}%
\pgfpathlineto{\pgfqpoint{4.126976in}{0.839584in}}%
\pgfpathlineto{\pgfqpoint{4.109837in}{1.135686in}}%
\pgfpathlineto{\pgfqpoint{4.088474in}{1.389811in}}%
\pgfpathlineto{\pgfqpoint{4.070397in}{1.416910in}}%
\pgfpathlineto{\pgfqpoint{4.052084in}{1.271894in}}%
\pgfpathlineto{\pgfqpoint{4.030250in}{0.878259in}}%
\pgfpathlineto{\pgfqpoint{4.014521in}{0.671748in}}%
\pgfpathlineto{\pgfqpoint{3.992218in}{0.530168in}}%
\pgfpathlineto{\pgfqpoint{3.973670in}{0.529278in}}%
\pgfpathlineto{\pgfqpoint{3.955828in}{0.654141in}}%
\pgfpathlineto{\pgfqpoint{3.933759in}{0.893635in}}%
\pgfpathlineto{\pgfqpoint{3.915917in}{1.208304in}}%
\pgfpathlineto{\pgfqpoint{3.895726in}{1.399763in}}%
\pgfpathlineto{\pgfqpoint{3.882109in}{1.390370in}}%
\pgfpathlineto{\pgfqpoint{3.860041in}{1.208980in}}%
\pgfpathlineto{\pgfqpoint{3.839615in}{0.861707in}}%
\pgfpathlineto{\pgfqpoint{3.821538in}{0.655848in}}%
\pgfpathlineto{\pgfqpoint{3.801816in}{0.528467in}}%
\pgfpathlineto{\pgfqpoint{3.782799in}{0.507334in}}%
\pgfpathlineto{\pgfqpoint{3.762845in}{0.628465in}}%
\pgfpathlineto{\pgfqpoint{3.745471in}{0.815919in}}%
\pgfpathlineto{\pgfqpoint{3.725046in}{1.135099in}}%
\pgfpathlineto{\pgfqpoint{3.703448in}{1.311001in}}%
\pgfpathlineto{\pgfqpoint{3.686074in}{1.400660in}}%
\pgfpathlineto{\pgfqpoint{3.667527in}{1.322898in}}%
\pgfpathlineto{\pgfqpoint{3.646632in}{1.077300in}}%
\pgfpathlineto{\pgfqpoint{3.629493in}{0.809468in}}%
\pgfpathlineto{\pgfqpoint{3.612825in}{0.637036in}}%
\pgfpathlineto{\pgfqpoint{3.588644in}{0.514689in}}%
\pgfpathlineto{\pgfqpoint{3.570565in}{0.507796in}}%
\pgfpathlineto{\pgfqpoint{3.551550in}{0.605591in}}%
\pgfpathlineto{\pgfqpoint{3.533237in}{0.762820in}}%
\pgfpathlineto{\pgfqpoint{3.512577in}{1.054045in}}%
\pgfpathlineto{\pgfqpoint{3.494969in}{1.228290in}}%
\pgfpathlineto{\pgfqpoint{3.473606in}{1.387794in}}%
\pgfpathlineto{\pgfqpoint{3.457406in}{1.380395in}}%
\pgfpathlineto{\pgfqpoint{3.436041in}{1.200905in}}%
\pgfpathlineto{\pgfqpoint{3.417495in}{0.930891in}}%
\pgfpathlineto{\pgfqpoint{3.398713in}{0.713908in}}%
\pgfpathlineto{\pgfqpoint{3.379462in}{0.583441in}}%
\pgfpathlineto{\pgfqpoint{3.358097in}{0.493006in}}%
\pgfpathlineto{\pgfqpoint{3.340723in}{1.240339in}}%
\pgfpathlineto{\pgfqpoint{3.322880in}{0.952215in}}%
\pgfpathlineto{\pgfqpoint{3.301046in}{0.683258in}}%
\pgfpathlineto{\pgfqpoint{3.281326in}{0.548460in}}%
\pgfpathlineto{\pgfqpoint{3.262780in}{0.494750in}}%
\pgfpathlineto{\pgfqpoint{3.244467in}{0.532658in}}%
\pgfpathlineto{\pgfqpoint{3.224512in}{0.680639in}}%
\pgfpathlineto{\pgfqpoint{3.207373in}{0.868598in}}%
\pgfpathlineto{\pgfqpoint{3.188356in}{1.189834in}}%
\pgfpathlineto{\pgfqpoint{3.169809in}{1.369433in}}%
\pgfpathlineto{\pgfqpoint{3.147740in}{1.343654in}}%
\pgfpathlineto{\pgfqpoint{3.128725in}{1.155362in}}%
\pgfpathlineto{\pgfqpoint{3.110646in}{0.876218in}}%
\pgfpathlineto{\pgfqpoint{3.091395in}{0.694600in}}%
\pgfpathlineto{\pgfqpoint{3.070501in}{0.538898in}}%
\pgfpathlineto{\pgfqpoint{3.051953in}{0.491009in}}%
\pgfpathlineto{\pgfqpoint{3.030355in}{0.562094in}}%
\pgfpathlineto{\pgfqpoint{3.015330in}{0.661811in}}%
\pgfpathlineto{\pgfqpoint{2.992791in}{0.921781in}}%
\pgfpathlineto{\pgfqpoint{2.976357in}{1.197275in}}%
\pgfpathlineto{\pgfqpoint{2.957106in}{1.361740in}}%
\pgfpathlineto{\pgfqpoint{2.935506in}{1.353490in}}%
\pgfpathlineto{\pgfqpoint{2.917429in}{1.184580in}}%
\pgfpathlineto{\pgfqpoint{2.896534in}{0.875634in}}%
\pgfpathlineto{\pgfqpoint{2.859675in}{0.664661in}}%
\pgfpathlineto{\pgfqpoint{2.840893in}{0.542317in}}%
\pgfpathlineto{\pgfqpoint{2.822816in}{0.485161in}}%
\pgfpathlineto{\pgfqpoint{2.803330in}{0.547505in}}%
\pgfpathlineto{\pgfqpoint{2.785488in}{0.662043in}}%
\pgfpathlineto{\pgfqpoint{2.763888in}{0.924744in}}%
\pgfpathlineto{\pgfqpoint{2.744871in}{1.228470in}}%
\pgfpathlineto{\pgfqpoint{2.726089in}{1.370731in}}%
\pgfpathlineto{\pgfqpoint{2.706603in}{1.075833in}}%
\pgfpathlineto{\pgfqpoint{2.688761in}{1.329429in}}%
\pgfpathlineto{\pgfqpoint{2.667867in}{1.380027in}}%
\pgfpathlineto{\pgfqpoint{2.648850in}{1.313043in}}%
\pgfpathlineto{\pgfqpoint{2.610347in}{0.765765in}}%
\pgfpathlineto{\pgfqpoint{2.591096in}{0.599853in}}%
\pgfpathlineto{\pgfqpoint{2.570436in}{0.498056in}}%
\pgfpathlineto{\pgfqpoint{2.550480in}{0.506232in}}%
\pgfpathlineto{\pgfqpoint{2.531697in}{0.605349in}}%
\pgfpathlineto{\pgfqpoint{2.513386in}{0.769893in}}%
\pgfpathlineto{\pgfqpoint{2.494838in}{1.042991in}}%
\pgfpathlineto{\pgfqpoint{2.475587in}{1.302751in}}%
\pgfpathlineto{\pgfqpoint{2.455398in}{1.378311in}}%
\pgfpathlineto{\pgfqpoint{2.436147in}{1.292935in}}%
\pgfpathlineto{\pgfqpoint{2.418773in}{1.058167in}}%
\pgfpathlineto{\pgfqpoint{2.399051in}{0.822843in}}%
\pgfpathlineto{\pgfqpoint{2.376748in}{0.626036in}}%
\pgfpathlineto{\pgfqpoint{2.358671in}{0.523711in}}%
\pgfpathlineto{\pgfqpoint{2.341063in}{0.486120in}}%
\pgfpathlineto{\pgfqpoint{2.322046in}{0.538773in}}%
\pgfpathlineto{\pgfqpoint{2.303029in}{0.648693in}}%
\pgfpathlineto{\pgfqpoint{2.278378in}{0.925603in}}%
\pgfpathlineto{\pgfqpoint{2.262649in}{1.156007in}}%
\pgfpathlineto{\pgfqpoint{2.247388in}{1.326696in}}%
\pgfpathlineto{\pgfqpoint{2.225556in}{1.377505in}}%
\pgfpathlineto{\pgfqpoint{2.206539in}{1.287829in}}%
\pgfpathlineto{\pgfqpoint{2.185408in}{1.072801in}}%
\pgfpathlineto{\pgfqpoint{2.169680in}{0.851182in}}%
\pgfpathlineto{\pgfqpoint{2.151366in}{0.683331in}}%
\pgfpathlineto{\pgfqpoint{2.129769in}{0.580869in}}%
\pgfpathlineto{\pgfqpoint{2.111455in}{0.501476in}}%
\pgfpathlineto{\pgfqpoint{2.090092in}{0.526122in}}%
\pgfpathlineto{\pgfqpoint{2.071310in}{0.609458in}}%
\pgfpathlineto{\pgfqpoint{2.052762in}{0.770591in}}%
\pgfpathlineto{\pgfqpoint{2.033981in}{1.023911in}}%
\pgfpathlineto{\pgfqpoint{2.016608in}{1.273535in}}%
\pgfpathlineto{\pgfqpoint{1.992425in}{1.386097in}}%
\pgfpathlineto{\pgfqpoint{1.977166in}{1.324959in}}%
\pgfpathlineto{\pgfqpoint{1.956975in}{1.361401in}}%
\pgfpathlineto{\pgfqpoint{1.935846in}{1.373627in}}%
\pgfpathlineto{\pgfqpoint{1.919646in}{1.274061in}}%
\pgfpathlineto{\pgfqpoint{1.898987in}{0.992050in}}%
\pgfpathlineto{\pgfqpoint{1.880910in}{0.763722in}}%
\pgfpathlineto{\pgfqpoint{1.859779in}{0.595442in}}%
\pgfpathlineto{\pgfqpoint{1.839824in}{0.517256in}}%
\pgfpathlineto{\pgfqpoint{1.821042in}{0.500880in}}%
\pgfpathlineto{\pgfqpoint{1.802025in}{0.566887in}}%
\pgfpathlineto{\pgfqpoint{1.783714in}{0.672936in}}%
\pgfpathlineto{\pgfqpoint{1.765166in}{0.834119in}}%
\pgfpathlineto{\pgfqpoint{1.743097in}{1.140334in}}%
\pgfpathlineto{\pgfqpoint{1.724786in}{1.283723in}}%
\pgfpathlineto{\pgfqpoint{1.705769in}{1.388960in}}%
\pgfpathlineto{\pgfqpoint{1.688630in}{1.391965in}}%
\pgfpathlineto{\pgfqpoint{1.668205in}{1.285253in}}%
\pgfpathlineto{\pgfqpoint{1.628293in}{0.799897in}}%
\pgfpathlineto{\pgfqpoint{1.610685in}{0.656839in}}%
\pgfpathlineto{\pgfqpoint{1.592374in}{0.563908in}}%
\pgfpathlineto{\pgfqpoint{1.570540in}{0.503360in}}%
\pgfpathlineto{\pgfqpoint{1.555749in}{0.520444in}}%
\pgfpathlineto{\pgfqpoint{1.531332in}{0.630346in}}%
\pgfpathlineto{\pgfqpoint{1.515369in}{0.748225in}}%
\pgfpathlineto{\pgfqpoint{1.496821in}{0.955459in}}%
\pgfpathlineto{\pgfqpoint{1.478039in}{1.212346in}}%
\pgfpathlineto{\pgfqpoint{1.458084in}{1.380860in}}%
\pgfpathlineto{\pgfqpoint{1.437659in}{1.408217in}}%
\pgfpathlineto{\pgfqpoint{1.416294in}{1.356308in}}%
\pgfpathlineto{\pgfqpoint{1.397748in}{1.237796in}}%
\pgfpathlineto{\pgfqpoint{1.378497in}{1.064696in}}%
\pgfpathlineto{\pgfqpoint{1.361123in}{0.882717in}}%
\pgfpathlineto{\pgfqpoint{1.336002in}{0.649100in}}%
\pgfpathlineto{\pgfqpoint{1.323558in}{0.580634in}}%
\pgfpathlineto{\pgfqpoint{1.301961in}{0.514712in}}%
\pgfpathlineto{\pgfqpoint{1.284118in}{0.532851in}}%
\pgfpathlineto{\pgfqpoint{1.264867in}{0.590922in}}%
\pgfpathlineto{\pgfqpoint{1.224719in}{0.892520in}}%
\pgfpathlineto{\pgfqpoint{1.207348in}{1.148587in}}%
\pgfpathlineto{\pgfqpoint{1.188097in}{1.343366in}}%
\pgfpathlineto{\pgfqpoint{1.169314in}{1.413562in}}%
\pgfpathlineto{\pgfqpoint{1.147949in}{1.421136in}}%
\pgfpathlineto{\pgfqpoint{1.132455in}{1.375970in}}%
\pgfpathlineto{\pgfqpoint{1.111324in}{1.277639in}}%
\pgfpathlineto{\pgfqpoint{1.092544in}{1.059754in}}%
\pgfpathlineto{\pgfqpoint{1.070241in}{0.830270in}}%
\pgfpathlineto{\pgfqpoint{1.051693in}{0.680616in}}%
\pgfpathlineto{\pgfqpoint{1.033382in}{0.591219in}}%
\pgfpathlineto{\pgfqpoint{1.014834in}{0.541002in}}%
\pgfpathlineto{\pgfqpoint{0.993939in}{0.525714in}}%
\pgfpathlineto{\pgfqpoint{0.974688in}{0.607979in}}%
\pgfpathlineto{\pgfqpoint{0.956609in}{0.726588in}}%
\pgfpathlineto{\pgfqpoint{0.938063in}{0.763944in}}%
\pgfpathlineto{\pgfqpoint{0.900733in}{1.228744in}}%
\pgfpathlineto{\pgfqpoint{0.881953in}{1.381568in}}%
\pgfpathlineto{\pgfqpoint{0.859884in}{1.441986in}}%
\pgfpathlineto{\pgfqpoint{0.842042in}{1.406303in}}%
\pgfpathlineto{\pgfqpoint{0.820677in}{1.259878in}}%
\pgfpathlineto{\pgfqpoint{0.802131in}{1.044428in}}%
\pgfpathlineto{\pgfqpoint{0.784052in}{0.857067in}}%
\pgfpathlineto{\pgfqpoint{0.767384in}{0.816847in}}%
\pgfpathlineto{\pgfqpoint{0.747193in}{0.655784in}}%
\pgfpathlineto{\pgfqpoint{0.726767in}{0.578828in}}%
\pgfpathlineto{\pgfqpoint{0.706344in}{0.531987in}}%
\pgfpathlineto{\pgfqpoint{0.685447in}{0.590318in}}%
\pgfpathlineto{\pgfqpoint{0.650468in}{0.815066in}}%
\pgfpathlineto{\pgfqpoint{0.651876in}{0.796028in}}%
\pgfpathlineto{\pgfqpoint{0.657040in}{0.741917in}}%
\pgfpathlineto{\pgfqpoint{0.674648in}{0.611435in}}%
\pgfpathlineto{\pgfqpoint{0.696951in}{0.836344in}}%
\pgfpathlineto{\pgfqpoint{0.714559in}{1.128149in}}%
\pgfpathlineto{\pgfqpoint{0.731933in}{1.362384in}}%
\pgfpathlineto{\pgfqpoint{0.750715in}{1.445179in}}%
\pgfpathlineto{\pgfqpoint{0.773018in}{1.295605in}}%
\pgfpathlineto{\pgfqpoint{0.790861in}{0.961277in}}%
\pgfpathlineto{\pgfqpoint{0.810347in}{0.709987in}}%
\pgfpathlineto{\pgfqpoint{0.829129in}{0.552729in}}%
\pgfpathlineto{\pgfqpoint{0.846502in}{0.532530in}}%
\pgfpathlineto{\pgfqpoint{0.869743in}{0.693651in}}%
\pgfpathlineto{\pgfqpoint{0.886648in}{1.264537in}}%
\pgfpathlineto{\pgfqpoint{0.905665in}{1.424904in}}%
\pgfpathlineto{\pgfqpoint{0.925619in}{1.373359in}}%
\pgfpathlineto{\pgfqpoint{0.944636in}{1.112476in}}%
\pgfpathlineto{\pgfqpoint{0.962949in}{0.786487in}}%
\pgfpathlineto{\pgfqpoint{0.986661in}{0.561312in}}%
\pgfpathlineto{\pgfqpoint{1.003095in}{0.510460in}}%
\pgfpathlineto{\pgfqpoint{1.020469in}{0.599592in}}%
\pgfpathlineto{\pgfqpoint{1.039251in}{0.772002in}}%
\pgfpathlineto{\pgfqpoint{1.061318in}{1.094606in}}%
\pgfpathlineto{\pgfqpoint{1.081040in}{1.350989in}}%
\pgfpathlineto{\pgfqpoint{1.098179in}{1.412509in}}%
\pgfpathlineto{\pgfqpoint{1.114847in}{1.315366in}}%
\pgfpathlineto{\pgfqpoint{1.138559in}{0.927358in}}%
\pgfpathlineto{\pgfqpoint{1.158279in}{0.678706in}}%
\pgfpathlineto{\pgfqpoint{1.176358in}{0.534046in}}%
\pgfpathlineto{\pgfqpoint{1.195844in}{0.516339in}}%
\pgfpathlineto{\pgfqpoint{1.214624in}{0.620710in}}%
\pgfpathlineto{\pgfqpoint{1.233172in}{0.796124in}}%
\pgfpathlineto{\pgfqpoint{1.250076in}{1.000679in}}%
\pgfpathlineto{\pgfqpoint{1.271205in}{1.329982in}}%
\pgfpathlineto{\pgfqpoint{1.290222in}{1.403178in}}%
\pgfpathlineto{\pgfqpoint{1.310177in}{1.340270in}}%
\pgfpathlineto{\pgfqpoint{1.330602in}{1.019023in}}%
\pgfpathlineto{\pgfqpoint{1.346332in}{0.803367in}}%
\pgfpathlineto{\pgfqpoint{1.368401in}{0.597128in}}%
\pgfpathlineto{\pgfqpoint{1.389764in}{0.499643in}}%
\pgfpathlineto{\pgfqpoint{1.404555in}{0.513626in}}%
\pgfpathlineto{\pgfqpoint{1.427329in}{0.663394in}}%
\pgfpathlineto{\pgfqpoint{1.443997in}{0.836194in}}%
\pgfpathlineto{\pgfqpoint{1.461605in}{1.104362in}}%
\pgfpathlineto{\pgfqpoint{1.481562in}{0.513012in}}%
\pgfpathlineto{\pgfqpoint{1.502456in}{0.518592in}}%
\pgfpathlineto{\pgfqpoint{1.521473in}{0.634174in}}%
\pgfpathlineto{\pgfqpoint{1.540490in}{0.824173in}}%
\pgfpathlineto{\pgfqpoint{1.559506in}{1.146584in}}%
\pgfpathlineto{\pgfqpoint{1.578052in}{1.350173in}}%
\pgfpathlineto{\pgfqpoint{1.595660in}{1.386836in}}%
\pgfpathlineto{\pgfqpoint{1.617260in}{1.212989in}}%
\pgfpathlineto{\pgfqpoint{1.636277in}{0.982194in}}%
\pgfpathlineto{\pgfqpoint{1.654588in}{0.712761in}}%
\pgfpathlineto{\pgfqpoint{1.673605in}{0.557087in}}%
\pgfpathlineto{\pgfqpoint{1.694970in}{0.491685in}}%
\pgfpathlineto{\pgfqpoint{1.712576in}{0.541733in}}%
\pgfpathlineto{\pgfqpoint{1.733473in}{0.715059in}}%
\pgfpathlineto{\pgfqpoint{1.752489in}{0.954367in}}%
\pgfpathlineto{\pgfqpoint{1.770566in}{1.242151in}}%
\pgfpathlineto{\pgfqpoint{1.790992in}{1.381388in}}%
\pgfpathlineto{\pgfqpoint{1.808834in}{1.358575in}}%
\pgfpathlineto{\pgfqpoint{1.829963in}{1.112334in}}%
\pgfpathlineto{\pgfqpoint{1.848980in}{0.850149in}}%
\pgfpathlineto{\pgfqpoint{1.864710in}{0.648698in}}%
\pgfpathlineto{\pgfqpoint{1.887248in}{0.531397in}}%
\pgfpathlineto{\pgfqpoint{1.905325in}{0.490761in}}%
\pgfpathlineto{\pgfqpoint{1.925047in}{0.534411in}}%
\pgfpathlineto{\pgfqpoint{1.944767in}{0.667965in}}%
\pgfpathlineto{\pgfqpoint{1.963315in}{0.841090in}}%
\pgfpathlineto{\pgfqpoint{1.980686in}{1.128494in}}%
\pgfpathlineto{\pgfqpoint{1.999469in}{1.300464in}}%
\pgfpathlineto{\pgfqpoint{2.019660in}{1.381746in}}%
\pgfpathlineto{\pgfqpoint{2.041728in}{1.291139in}}%
\pgfpathlineto{\pgfqpoint{2.059336in}{1.381711in}}%
\pgfpathlineto{\pgfqpoint{2.077179in}{1.317837in}}%
\pgfpathlineto{\pgfqpoint{2.101125in}{0.978549in}}%
\pgfpathlineto{\pgfqpoint{2.112864in}{0.759392in}}%
\pgfpathlineto{\pgfqpoint{2.137281in}{0.585383in}}%
\pgfpathlineto{\pgfqpoint{2.154889in}{0.515344in}}%
\pgfpathlineto{\pgfqpoint{2.175549in}{0.503136in}}%
\pgfpathlineto{\pgfqpoint{2.192921in}{0.587266in}}%
\pgfpathlineto{\pgfqpoint{2.211468in}{0.730369in}}%
\pgfpathlineto{\pgfqpoint{2.231894in}{0.986716in}}%
\pgfpathlineto{\pgfqpoint{2.253728in}{1.275522in}}%
\pgfpathlineto{\pgfqpoint{2.268988in}{1.366296in}}%
\pgfpathlineto{\pgfqpoint{2.288944in}{1.344736in}}%
\pgfpathlineto{\pgfqpoint{2.308664in}{1.152901in}}%
\pgfpathlineto{\pgfqpoint{2.326741in}{0.863681in}}%
\pgfpathlineto{\pgfqpoint{2.346227in}{0.658074in}}%
\pgfpathlineto{\pgfqpoint{2.366889in}{0.530191in}}%
\pgfpathlineto{\pgfqpoint{2.383792in}{0.485517in}}%
\pgfpathlineto{\pgfqpoint{2.405860in}{0.544493in}}%
\pgfpathlineto{\pgfqpoint{2.426051in}{0.661860in}}%
\pgfpathlineto{\pgfqpoint{2.443188in}{0.829772in}}%
\pgfpathlineto{\pgfqpoint{2.462910in}{0.996541in}}%
\pgfpathlineto{\pgfqpoint{2.481691in}{1.138755in}}%
\pgfpathlineto{\pgfqpoint{2.499299in}{1.341272in}}%
\pgfpathlineto{\pgfqpoint{2.520430in}{1.376773in}}%
\pgfpathlineto{\pgfqpoint{2.538975in}{1.282416in}}%
\pgfpathlineto{\pgfqpoint{2.557054in}{1.038478in}}%
\pgfpathlineto{\pgfqpoint{2.577478in}{0.726857in}}%
\pgfpathlineto{\pgfqpoint{2.598609in}{0.570281in}}%
\pgfpathlineto{\pgfqpoint{2.616217in}{0.522940in}}%
\pgfpathlineto{\pgfqpoint{2.634528in}{0.489436in}}%
\pgfpathlineto{\pgfqpoint{2.655188in}{0.563219in}}%
\pgfpathlineto{\pgfqpoint{2.673265in}{0.657899in}}%
\pgfpathlineto{\pgfqpoint{2.691813in}{0.847833in}}%
\pgfpathlineto{\pgfqpoint{2.714821in}{1.136466in}}%
\pgfpathlineto{\pgfqpoint{2.732193in}{1.251011in}}%
\pgfpathlineto{\pgfqpoint{2.749801in}{1.344779in}}%
\pgfpathlineto{\pgfqpoint{2.769757in}{1.370880in}}%
\pgfpathlineto{\pgfqpoint{2.790417in}{1.191996in}}%
\pgfpathlineto{\pgfqpoint{2.808025in}{0.903704in}}%
\pgfpathlineto{\pgfqpoint{2.826102in}{0.705900in}}%
\pgfpathlineto{\pgfqpoint{2.847233in}{0.552550in}}%
\pgfpathlineto{\pgfqpoint{2.864136in}{0.498662in}}%
\pgfpathlineto{\pgfqpoint{2.883387in}{0.517741in}}%
\pgfpathlineto{\pgfqpoint{2.901464in}{0.627217in}}%
\pgfpathlineto{\pgfqpoint{2.922829in}{0.795591in}}%
\pgfpathlineto{\pgfqpoint{2.944898in}{1.066200in}}%
\pgfpathlineto{\pgfqpoint{2.964384in}{1.224768in}}%
\pgfpathlineto{\pgfqpoint{2.980114in}{1.342661in}}%
\pgfpathlineto{\pgfqpoint{3.001948in}{1.378203in}}%
\pgfpathlineto{\pgfqpoint{3.019554in}{1.248951in}}%
\pgfpathlineto{\pgfqpoint{3.043502in}{0.932710in}}%
\pgfpathlineto{\pgfqpoint{3.057119in}{0.770880in}}%
\pgfpathlineto{\pgfqpoint{3.078013in}{0.598118in}}%
\pgfpathlineto{\pgfqpoint{3.097264in}{0.526443in}}%
\pgfpathlineto{\pgfqpoint{3.114872in}{0.495761in}}%
\pgfpathlineto{\pgfqpoint{3.137176in}{0.578483in}}%
\pgfpathlineto{\pgfqpoint{3.152437in}{0.666131in}}%
\pgfpathlineto{\pgfqpoint{3.172157in}{0.798683in}}%
\pgfpathlineto{\pgfqpoint{3.191877in}{1.037040in}}%
\pgfpathlineto{\pgfqpoint{3.214886in}{1.298444in}}%
\pgfpathlineto{\pgfqpoint{3.230145in}{1.354740in}}%
\pgfpathlineto{\pgfqpoint{3.249396in}{1.394107in}}%
\pgfpathlineto{\pgfqpoint{3.270056in}{1.310172in}}%
\pgfpathlineto{\pgfqpoint{3.288135in}{1.119604in}}%
\pgfpathlineto{\pgfqpoint{3.309499in}{0.852144in}}%
\pgfpathlineto{\pgfqpoint{3.326638in}{0.693629in}}%
\pgfpathlineto{\pgfqpoint{3.347063in}{0.580167in}}%
\pgfpathlineto{\pgfqpoint{3.367018in}{0.503966in}}%
\pgfpathlineto{\pgfqpoint{3.385800in}{0.509882in}}%
\pgfpathlineto{\pgfqpoint{3.403877in}{0.564951in}}%
\pgfpathlineto{\pgfqpoint{3.423834in}{0.712675in}}%
\pgfpathlineto{\pgfqpoint{3.441910in}{0.876232in}}%
\pgfpathlineto{\pgfqpoint{3.460224in}{1.026728in}}%
\pgfpathlineto{\pgfqpoint{3.482996in}{1.281067in}}%
\pgfpathlineto{\pgfqpoint{3.500369in}{0.713218in}}%
\pgfpathlineto{\pgfqpoint{3.523612in}{1.019889in}}%
\pgfpathlineto{\pgfqpoint{3.558826in}{1.376645in}}%
\pgfpathlineto{\pgfqpoint{3.576905in}{1.406371in}}%
\pgfpathlineto{\pgfqpoint{3.594513in}{1.337139in}}%
\pgfpathlineto{\pgfqpoint{3.615642in}{1.114973in}}%
\pgfpathlineto{\pgfqpoint{3.632781in}{0.902108in}}%
\pgfpathlineto{\pgfqpoint{3.654850in}{0.704923in}}%
\pgfpathlineto{\pgfqpoint{3.672222in}{0.577892in}}%
\pgfpathlineto{\pgfqpoint{3.690535in}{0.510586in}}%
\pgfpathlineto{\pgfqpoint{3.711898in}{0.544865in}}%
\pgfpathlineto{\pgfqpoint{3.733029in}{0.631305in}}%
\pgfpathlineto{\pgfqpoint{3.751341in}{0.697711in}}%
\pgfpathlineto{\pgfqpoint{3.769183in}{0.848680in}}%
\pgfpathlineto{\pgfqpoint{3.789140in}{1.123376in}}%
\pgfpathlineto{\pgfqpoint{3.808391in}{1.330573in}}%
\pgfpathlineto{\pgfqpoint{3.825764in}{1.416159in}}%
\pgfpathlineto{\pgfqpoint{3.846659in}{1.405303in}}%
\pgfpathlineto{\pgfqpoint{3.864970in}{1.313850in}}%
\pgfpathlineto{\pgfqpoint{3.886335in}{1.046207in}}%
\pgfpathlineto{\pgfqpoint{3.920612in}{0.703609in}}%
\pgfpathlineto{\pgfqpoint{3.943149in}{0.568548in}}%
\pgfpathlineto{\pgfqpoint{3.961697in}{0.521977in}}%
\pgfpathlineto{\pgfqpoint{3.979305in}{0.524299in}}%
\pgfpathlineto{\pgfqpoint{4.000199in}{0.610827in}}%
\pgfpathlineto{\pgfqpoint{4.018276in}{0.741777in}}%
\pgfpathlineto{\pgfqpoint{4.039642in}{0.878877in}}%
\pgfpathlineto{\pgfqpoint{4.057015in}{1.107000in}}%
\pgfpathlineto{\pgfqpoint{4.077910in}{1.294859in}}%
\pgfpathlineto{\pgfqpoint{4.096690in}{1.418168in}}%
\pgfpathlineto{\pgfqpoint{4.114769in}{1.436462in}}%
\pgfpathlineto{\pgfqpoint{4.134020in}{1.377032in}}%
\pgfpathlineto{\pgfqpoint{4.154211in}{1.239146in}}%
\pgfpathlineto{\pgfqpoint{4.171348in}{0.993501in}}%
\pgfpathlineto{\pgfqpoint{4.192948in}{0.765759in}}%
\pgfpathlineto{\pgfqpoint{4.211494in}{0.715635in}}%
\pgfpathlineto{\pgfqpoint{4.228633in}{0.582603in}}%
\pgfpathlineto{\pgfqpoint{4.249998in}{0.526834in}}%
\pgfpathlineto{\pgfqpoint{4.267604in}{0.584449in}}%
\pgfpathlineto{\pgfqpoint{4.288735in}{0.715882in}}%
\pgfpathlineto{\pgfqpoint{4.307517in}{0.876583in}}%
\pgfpathlineto{\pgfqpoint{4.327003in}{1.090056in}}%
\pgfpathlineto{\pgfqpoint{4.345549in}{1.275892in}}%
\pgfpathlineto{\pgfqpoint{4.366914in}{1.427435in}}%
\pgfpathlineto{\pgfqpoint{4.388277in}{1.457294in}}%
\pgfpathlineto{\pgfqpoint{4.406825in}{1.423981in}}%
\pgfpathlineto{\pgfqpoint{4.421381in}{1.322675in}}%
\pgfpathlineto{\pgfqpoint{4.460587in}{0.886592in}}%
\pgfpathlineto{\pgfqpoint{4.478195in}{0.733075in}}%
\pgfpathlineto{\pgfqpoint{4.474440in}{0.775050in}}%
\pgfpathlineto{\pgfqpoint{4.456832in}{1.053846in}}%
\pgfpathlineto{\pgfqpoint{4.435467in}{1.355331in}}%
\pgfpathlineto{\pgfqpoint{4.417155in}{1.455934in}}%
\pgfpathlineto{\pgfqpoint{4.400253in}{1.348469in}}%
\pgfpathlineto{\pgfqpoint{4.378887in}{1.122722in}}%
\pgfpathlineto{\pgfqpoint{4.357758in}{0.805129in}}%
\pgfpathlineto{\pgfqpoint{4.340854in}{0.634329in}}%
\pgfpathlineto{\pgfqpoint{4.319959in}{0.526156in}}%
\pgfpathlineto{\pgfqpoint{4.300474in}{0.608964in}}%
\pgfpathlineto{\pgfqpoint{4.282866in}{0.785574in}}%
\pgfpathlineto{\pgfqpoint{4.261500in}{1.157498in}}%
\pgfpathlineto{\pgfqpoint{4.243658in}{1.382345in}}%
\pgfpathlineto{\pgfqpoint{4.225112in}{1.437226in}}%
\pgfpathlineto{\pgfqpoint{4.205861in}{1.300430in}}%
\pgfpathlineto{\pgfqpoint{4.184496in}{0.953186in}}%
\pgfpathlineto{\pgfqpoint{4.168062in}{0.724901in}}%
\pgfpathlineto{\pgfqpoint{4.146462in}{0.562335in}}%
\pgfpathlineto{\pgfqpoint{4.126273in}{0.532151in}}%
\pgfpathlineto{\pgfqpoint{4.109134in}{0.646355in}}%
\pgfpathlineto{\pgfqpoint{4.090117in}{0.867444in}}%
\pgfpathlineto{\pgfqpoint{4.071100in}{1.203489in}}%
\pgfpathlineto{\pgfqpoint{4.050206in}{1.409032in}}%
\pgfpathlineto{\pgfqpoint{4.033301in}{1.402727in}}%
\pgfpathlineto{\pgfqpoint{4.012407in}{1.187818in}}%
\pgfpathlineto{\pgfqpoint{3.992218in}{0.853301in}}%
\pgfpathlineto{\pgfqpoint{3.974374in}{0.656220in}}%
\pgfpathlineto{\pgfqpoint{3.953950in}{0.521727in}}%
\pgfpathlineto{\pgfqpoint{3.936811in}{0.527455in}}%
\pgfpathlineto{\pgfqpoint{3.916151in}{0.678505in}}%
\pgfpathlineto{\pgfqpoint{3.898074in}{0.886669in}}%
\pgfpathlineto{\pgfqpoint{3.873891in}{1.279622in}}%
\pgfpathlineto{\pgfqpoint{3.859335in}{1.396472in}}%
\pgfpathlineto{\pgfqpoint{3.837737in}{1.361403in}}%
\pgfpathlineto{\pgfqpoint{3.823181in}{1.160801in}}%
\pgfpathlineto{\pgfqpoint{3.800644in}{0.822384in}}%
\pgfpathlineto{\pgfqpoint{3.782565in}{0.641147in}}%
\pgfpathlineto{\pgfqpoint{3.761202in}{0.509886in}}%
\pgfpathlineto{\pgfqpoint{3.745471in}{0.516811in}}%
\pgfpathlineto{\pgfqpoint{3.724108in}{0.642497in}}%
\pgfpathlineto{\pgfqpoint{3.700630in}{0.918278in}}%
\pgfpathlineto{\pgfqpoint{3.686309in}{1.168926in}}%
\pgfpathlineto{\pgfqpoint{3.663066in}{1.374344in}}%
\pgfpathlineto{\pgfqpoint{3.647806in}{1.395471in}}%
\pgfpathlineto{\pgfqpoint{3.626912in}{1.337651in}}%
\pgfpathlineto{\pgfqpoint{3.610007in}{1.099313in}}%
\pgfpathlineto{\pgfqpoint{3.591696in}{0.823003in}}%
\pgfpathlineto{\pgfqpoint{3.566810in}{0.576509in}}%
\pgfpathlineto{\pgfqpoint{3.552723in}{0.526367in}}%
\pgfpathlineto{\pgfqpoint{3.531828in}{0.507698in}}%
\pgfpathlineto{\pgfqpoint{3.514926in}{0.575475in}}%
\pgfpathlineto{\pgfqpoint{3.493560in}{0.757485in}}%
\pgfpathlineto{\pgfqpoint{3.476423in}{0.940213in}}%
\pgfpathlineto{\pgfqpoint{3.454823in}{1.265971in}}%
\pgfpathlineto{\pgfqpoint{3.436981in}{1.380585in}}%
\pgfpathlineto{\pgfqpoint{3.415850in}{1.336875in}}%
\pgfpathlineto{\pgfqpoint{3.378522in}{0.891159in}}%
\pgfpathlineto{\pgfqpoint{3.359271in}{0.697321in}}%
\pgfpathlineto{\pgfqpoint{3.338611in}{0.546167in}}%
\pgfpathlineto{\pgfqpoint{3.321237in}{0.491847in}}%
\pgfpathlineto{\pgfqpoint{3.301986in}{0.528116in}}%
\pgfpathlineto{\pgfqpoint{3.281561in}{0.633023in}}%
\pgfpathlineto{\pgfqpoint{3.263249in}{0.790544in}}%
\pgfpathlineto{\pgfqpoint{3.245407in}{1.055139in}}%
\pgfpathlineto{\pgfqpoint{3.225216in}{1.321766in}}%
\pgfpathlineto{\pgfqpoint{3.203852in}{1.382169in}}%
\pgfpathlineto{\pgfqpoint{3.186244in}{1.280077in}}%
\pgfpathlineto{\pgfqpoint{3.168871in}{1.048897in}}%
\pgfpathlineto{\pgfqpoint{3.147037in}{0.840844in}}%
\pgfpathlineto{\pgfqpoint{3.129194in}{0.648139in}}%
\pgfpathlineto{\pgfqpoint{3.110646in}{0.531643in}}%
\pgfpathlineto{\pgfqpoint{3.092804in}{0.491090in}}%
\pgfpathlineto{\pgfqpoint{3.070032in}{0.549216in}}%
\pgfpathlineto{\pgfqpoint{3.052424in}{0.682219in}}%
\pgfpathlineto{\pgfqpoint{3.034347in}{0.892268in}}%
\pgfpathlineto{\pgfqpoint{3.012513in}{1.170645in}}%
\pgfpathlineto{\pgfqpoint{2.993730in}{1.353777in}}%
\pgfpathlineto{\pgfqpoint{2.974948in}{1.375120in}}%
\pgfpathlineto{\pgfqpoint{2.956402in}{1.302658in}}%
\pgfpathlineto{\pgfqpoint{2.935977in}{1.025955in}}%
\pgfpathlineto{\pgfqpoint{2.919072in}{0.881924in}}%
\pgfpathlineto{\pgfqpoint{2.897943in}{0.650371in}}%
\pgfpathlineto{\pgfqpoint{2.878692in}{0.556316in}}%
\pgfpathlineto{\pgfqpoint{2.859441in}{0.507757in}}%
\pgfpathlineto{\pgfqpoint{2.841364in}{0.503541in}}%
\pgfpathlineto{\pgfqpoint{2.822111in}{0.667485in}}%
\pgfpathlineto{\pgfqpoint{2.800982in}{0.770929in}}%
\pgfpathlineto{\pgfqpoint{2.763888in}{1.286236in}}%
\pgfpathlineto{\pgfqpoint{2.745811in}{1.377644in}}%
\pgfpathlineto{\pgfqpoint{2.726089in}{1.321695in}}%
\pgfpathlineto{\pgfqpoint{2.704491in}{1.179914in}}%
\pgfpathlineto{\pgfqpoint{2.686649in}{0.944263in}}%
\pgfpathlineto{\pgfqpoint{2.627955in}{0.492850in}}%
\pgfpathlineto{\pgfqpoint{2.611756in}{0.501349in}}%
\pgfpathlineto{\pgfqpoint{2.592739in}{0.579518in}}%
\pgfpathlineto{\pgfqpoint{2.573723in}{0.717641in}}%
\pgfpathlineto{\pgfqpoint{2.551419in}{0.930797in}}%
\pgfpathlineto{\pgfqpoint{2.531934in}{1.201006in}}%
\pgfpathlineto{\pgfqpoint{2.513151in}{1.357948in}}%
\pgfpathlineto{\pgfqpoint{2.495778in}{1.369714in}}%
\pgfpathlineto{\pgfqpoint{2.473944in}{1.204821in}}%
\pgfpathlineto{\pgfqpoint{2.455632in}{0.950265in}}%
\pgfpathlineto{\pgfqpoint{2.435910in}{0.725760in}}%
\pgfpathlineto{\pgfqpoint{2.414781in}{1.379472in}}%
\pgfpathlineto{\pgfqpoint{2.399756in}{1.342790in}}%
\pgfpathlineto{\pgfqpoint{2.376513in}{1.066155in}}%
\pgfpathlineto{\pgfqpoint{2.359374in}{0.808725in}}%
\pgfpathlineto{\pgfqpoint{2.340828in}{0.642345in}}%
\pgfpathlineto{\pgfqpoint{2.321343in}{0.527735in}}%
\pgfpathlineto{\pgfqpoint{2.300212in}{0.493518in}}%
\pgfpathlineto{\pgfqpoint{2.285892in}{0.551694in}}%
\pgfpathlineto{\pgfqpoint{2.263824in}{0.707588in}}%
\pgfpathlineto{\pgfqpoint{2.244336in}{0.916237in}}%
\pgfpathlineto{\pgfqpoint{2.226025in}{1.144291in}}%
\pgfpathlineto{\pgfqpoint{2.206304in}{1.349933in}}%
\pgfpathlineto{\pgfqpoint{2.187288in}{1.379891in}}%
\pgfpathlineto{\pgfqpoint{2.168271in}{1.289719in}}%
\pgfpathlineto{\pgfqpoint{2.146437in}{1.003098in}}%
\pgfpathlineto{\pgfqpoint{2.129298in}{0.760768in}}%
\pgfpathlineto{\pgfqpoint{2.109578in}{0.615239in}}%
\pgfpathlineto{\pgfqpoint{2.087978in}{0.516114in}}%
\pgfpathlineto{\pgfqpoint{2.072718in}{0.598253in}}%
\pgfpathlineto{\pgfqpoint{2.050650in}{0.534157in}}%
\pgfpathlineto{\pgfqpoint{2.033042in}{0.491703in}}%
\pgfpathlineto{\pgfqpoint{2.014025in}{0.544808in}}%
\pgfpathlineto{\pgfqpoint{1.995008in}{0.613048in}}%
\pgfpathlineto{\pgfqpoint{1.974348in}{0.806180in}}%
\pgfpathlineto{\pgfqpoint{1.954863in}{1.112252in}}%
\pgfpathlineto{\pgfqpoint{1.937489in}{1.301302in}}%
\pgfpathlineto{\pgfqpoint{1.919412in}{1.388560in}}%
\pgfpathlineto{\pgfqpoint{1.898518in}{1.330303in}}%
\pgfpathlineto{\pgfqpoint{1.880439in}{1.112180in}}%
\pgfpathlineto{\pgfqpoint{1.862831in}{0.839826in}}%
\pgfpathlineto{\pgfqpoint{1.843111in}{0.673812in}}%
\pgfpathlineto{\pgfqpoint{1.823859in}{0.570495in}}%
\pgfpathlineto{\pgfqpoint{1.800382in}{0.496049in}}%
\pgfpathlineto{\pgfqpoint{1.784417in}{0.528746in}}%
\pgfpathlineto{\pgfqpoint{1.762348in}{0.649854in}}%
\pgfpathlineto{\pgfqpoint{1.744272in}{0.848907in}}%
\pgfpathlineto{\pgfqpoint{1.724786in}{1.129146in}}%
\pgfpathlineto{\pgfqpoint{1.706238in}{1.316728in}}%
\pgfpathlineto{\pgfqpoint{1.688161in}{1.397404in}}%
\pgfpathlineto{\pgfqpoint{1.666561in}{1.322597in}}%
\pgfpathlineto{\pgfqpoint{1.647545in}{1.129141in}}%
\pgfpathlineto{\pgfqpoint{1.629233in}{0.959772in}}%
\pgfpathlineto{\pgfqpoint{1.611391in}{0.771363in}}%
\pgfpathlineto{\pgfqpoint{1.589322in}{0.604891in}}%
\pgfpathlineto{\pgfqpoint{1.571948in}{0.521653in}}%
\pgfpathlineto{\pgfqpoint{1.552463in}{0.508321in}}%
\pgfpathlineto{\pgfqpoint{1.534855in}{0.578089in}}%
\pgfpathlineto{\pgfqpoint{1.514898in}{0.684279in}}%
\pgfpathlineto{\pgfqpoint{1.496352in}{0.869484in}}%
\pgfpathlineto{\pgfqpoint{1.473578in}{1.162127in}}%
\pgfpathlineto{\pgfqpoint{1.456441in}{0.681845in}}%
\pgfpathlineto{\pgfqpoint{1.436250in}{0.642777in}}%
\pgfpathlineto{\pgfqpoint{1.416999in}{0.845142in}}%
\pgfpathlineto{\pgfqpoint{1.400565in}{1.075016in}}%
\pgfpathlineto{\pgfqpoint{1.383661in}{1.278495in}}%
\pgfpathlineto{\pgfqpoint{1.358540in}{1.409606in}}%
\pgfpathlineto{\pgfqpoint{1.343281in}{1.385910in}}%
\pgfpathlineto{\pgfqpoint{1.323558in}{1.290970in}}%
\pgfpathlineto{\pgfqpoint{1.305013in}{1.036705in}}%
\pgfpathlineto{\pgfqpoint{1.283884in}{0.781312in}}%
\pgfpathlineto{\pgfqpoint{1.261344in}{0.614964in}}%
\pgfpathlineto{\pgfqpoint{1.246554in}{0.534016in}}%
\pgfpathlineto{\pgfqpoint{1.227771in}{0.504658in}}%
\pgfpathlineto{\pgfqpoint{1.206877in}{0.574888in}}%
\pgfpathlineto{\pgfqpoint{1.187157in}{0.703815in}}%
\pgfpathlineto{\pgfqpoint{1.168609in}{0.902955in}}%
\pgfpathlineto{\pgfqpoint{1.150063in}{1.168542in}}%
\pgfpathlineto{\pgfqpoint{1.128463in}{1.357965in}}%
\pgfpathlineto{\pgfqpoint{1.112264in}{1.420675in}}%
\pgfpathlineto{\pgfqpoint{1.091604in}{1.390853in}}%
\pgfpathlineto{\pgfqpoint{1.073527in}{1.233685in}}%
\pgfpathlineto{\pgfqpoint{1.054042in}{1.003401in}}%
\pgfpathlineto{\pgfqpoint{1.035494in}{0.806409in}}%
\pgfpathlineto{\pgfqpoint{1.010842in}{0.628188in}}%
\pgfpathlineto{\pgfqpoint{0.995348in}{1.034644in}}%
\pgfpathlineto{\pgfqpoint{0.976331in}{0.810153in}}%
\pgfpathlineto{\pgfqpoint{0.957549in}{0.661883in}}%
\pgfpathlineto{\pgfqpoint{0.936420in}{0.541881in}}%
\pgfpathlineto{\pgfqpoint{0.920924in}{0.521226in}}%
\pgfpathlineto{\pgfqpoint{0.900970in}{0.605029in}}%
\pgfpathlineto{\pgfqpoint{0.882422in}{0.695083in}}%
\pgfpathlineto{\pgfqpoint{0.862702in}{0.893778in}}%
\pgfpathlineto{\pgfqpoint{0.840162in}{1.135936in}}%
\pgfpathlineto{\pgfqpoint{0.823728in}{1.347063in}}%
\pgfpathlineto{\pgfqpoint{0.804008in}{1.438517in}}%
\pgfpathlineto{\pgfqpoint{0.785931in}{1.426506in}}%
\pgfpathlineto{\pgfqpoint{0.761983in}{1.235138in}}%
\pgfpathlineto{\pgfqpoint{0.745784in}{1.062589in}}%
\pgfpathlineto{\pgfqpoint{0.727472in}{0.840460in}}%
\pgfpathlineto{\pgfqpoint{0.708456in}{0.692642in}}%
\pgfpathlineto{\pgfqpoint{0.685918in}{0.568224in}}%
\pgfpathlineto{\pgfqpoint{0.668545in}{0.530441in}}%
\pgfpathlineto{\pgfqpoint{0.650468in}{0.582347in}}%
\pgfpathlineto{\pgfqpoint{0.647416in}{0.583101in}}%
\pgfpathlineto{\pgfqpoint{0.654223in}{0.547149in}}%
\pgfpathlineto{\pgfqpoint{0.675588in}{0.586242in}}%
\pgfpathlineto{\pgfqpoint{0.694368in}{0.744207in}}%
\pgfpathlineto{\pgfqpoint{0.712447in}{0.982058in}}%
\pgfpathlineto{\pgfqpoint{0.734282in}{1.327334in}}%
\pgfpathlineto{\pgfqpoint{0.750481in}{1.438849in}}%
\pgfpathlineto{\pgfqpoint{0.771610in}{1.370684in}}%
\pgfpathlineto{\pgfqpoint{0.790626in}{1.101014in}}%
\pgfpathlineto{\pgfqpoint{0.810112in}{0.769965in}}%
\pgfpathlineto{\pgfqpoint{0.828424in}{0.584181in}}%
\pgfpathlineto{\pgfqpoint{0.846971in}{0.519039in}}%
\pgfpathlineto{\pgfqpoint{0.866457in}{0.632778in}}%
\pgfpathlineto{\pgfqpoint{0.885943in}{0.823580in}}%
\pgfpathlineto{\pgfqpoint{0.905196in}{1.160047in}}%
\pgfpathlineto{\pgfqpoint{0.924211in}{1.390826in}}%
\pgfpathlineto{\pgfqpoint{0.944636in}{1.413417in}}%
\pgfpathlineto{\pgfqpoint{0.962715in}{1.216562in}}%
\pgfpathlineto{\pgfqpoint{0.982435in}{0.896489in}}%
\pgfpathlineto{\pgfqpoint{1.002392in}{0.659109in}}%
\pgfpathlineto{\pgfqpoint{1.020938in}{0.525913in}}%
\pgfpathlineto{\pgfqpoint{1.039015in}{0.540815in}}%
\pgfpathlineto{\pgfqpoint{1.058737in}{0.682831in}}%
\pgfpathlineto{\pgfqpoint{1.079865in}{0.946811in}}%
\pgfpathlineto{\pgfqpoint{1.100291in}{1.273643in}}%
\pgfpathlineto{\pgfqpoint{1.115550in}{1.206497in}}%
\pgfpathlineto{\pgfqpoint{1.137619in}{1.406008in}}%
\pgfpathlineto{\pgfqpoint{1.154289in}{1.382788in}}%
\pgfpathlineto{\pgfqpoint{1.173540in}{1.157231in}}%
\pgfpathlineto{\pgfqpoint{1.194904in}{0.792986in}}%
\pgfpathlineto{\pgfqpoint{1.214860in}{0.595338in}}%
\pgfpathlineto{\pgfqpoint{1.231294in}{0.507149in}}%
\pgfpathlineto{\pgfqpoint{1.250076in}{0.541157in}}%
\pgfpathlineto{\pgfqpoint{1.274726in}{0.729332in}}%
\pgfpathlineto{\pgfqpoint{1.293977in}{1.001124in}}%
\pgfpathlineto{\pgfqpoint{1.310411in}{1.259624in}}%
\pgfpathlineto{\pgfqpoint{1.325907in}{1.389846in}}%
\pgfpathlineto{\pgfqpoint{1.347741in}{1.357218in}}%
\pgfpathlineto{\pgfqpoint{1.366992in}{1.114418in}}%
\pgfpathlineto{\pgfqpoint{1.385538in}{0.807365in}}%
\pgfpathlineto{\pgfqpoint{1.405260in}{0.604387in}}%
\pgfpathlineto{\pgfqpoint{1.425215in}{0.502291in}}%
\pgfpathlineto{\pgfqpoint{1.447049in}{0.553896in}}%
\pgfpathlineto{\pgfqpoint{1.463719in}{0.649248in}}%
\pgfpathlineto{\pgfqpoint{1.480856in}{0.831690in}}%
\pgfpathlineto{\pgfqpoint{1.500342in}{1.106045in}}%
\pgfpathlineto{\pgfqpoint{1.522647in}{1.357133in}}%
\pgfpathlineto{\pgfqpoint{1.537907in}{1.389398in}}%
\pgfpathlineto{\pgfqpoint{1.562793in}{1.199446in}}%
\pgfpathlineto{\pgfqpoint{1.578521in}{0.932304in}}%
\pgfpathlineto{\pgfqpoint{1.598009in}{0.690808in}}%
\pgfpathlineto{\pgfqpoint{1.617026in}{0.584061in}}%
\pgfpathlineto{\pgfqpoint{1.636277in}{0.883183in}}%
\pgfpathlineto{\pgfqpoint{1.654588in}{0.685086in}}%
\pgfpathlineto{\pgfqpoint{1.672665in}{0.549931in}}%
\pgfpathlineto{\pgfqpoint{1.694970in}{0.493281in}}%
\pgfpathlineto{\pgfqpoint{1.715394in}{0.546903in}}%
\pgfpathlineto{\pgfqpoint{1.733236in}{0.674933in}}%
\pgfpathlineto{\pgfqpoint{1.751315in}{0.885235in}}%
\pgfpathlineto{\pgfqpoint{1.770566in}{1.151726in}}%
\pgfpathlineto{\pgfqpoint{1.790286in}{1.355293in}}%
\pgfpathlineto{\pgfqpoint{1.808129in}{1.384375in}}%
\pgfpathlineto{\pgfqpoint{1.829494in}{1.247099in}}%
\pgfpathlineto{\pgfqpoint{1.848277in}{0.943608in}}%
\pgfpathlineto{\pgfqpoint{1.867291in}{0.703853in}}%
\pgfpathlineto{\pgfqpoint{1.885839in}{0.565621in}}%
\pgfpathlineto{\pgfqpoint{1.904621in}{0.497887in}}%
\pgfpathlineto{\pgfqpoint{1.922933in}{0.514619in}}%
\pgfpathlineto{\pgfqpoint{1.945001in}{0.634799in}}%
\pgfpathlineto{\pgfqpoint{1.963549in}{0.810815in}}%
\pgfpathlineto{\pgfqpoint{1.980452in}{1.028821in}}%
\pgfpathlineto{\pgfqpoint{2.000877in}{1.312991in}}%
\pgfpathlineto{\pgfqpoint{2.021303in}{1.382292in}}%
\pgfpathlineto{\pgfqpoint{2.038442in}{1.311474in}}%
\pgfpathlineto{\pgfqpoint{2.056753in}{1.060491in}}%
\pgfpathlineto{\pgfqpoint{2.079291in}{0.772814in}}%
\pgfpathlineto{\pgfqpoint{2.096196in}{0.628653in}}%
\pgfpathlineto{\pgfqpoint{2.115916in}{0.517629in}}%
\pgfpathlineto{\pgfqpoint{2.136341in}{0.492332in}}%
\pgfpathlineto{\pgfqpoint{2.154889in}{0.569521in}}%
\pgfpathlineto{\pgfqpoint{2.172261in}{0.654554in}}%
\pgfpathlineto{\pgfqpoint{2.193392in}{0.853675in}}%
\pgfpathlineto{\pgfqpoint{2.211234in}{1.132278in}}%
\pgfpathlineto{\pgfqpoint{2.233537in}{1.347047in}}%
\pgfpathlineto{\pgfqpoint{2.252319in}{1.376190in}}%
\pgfpathlineto{\pgfqpoint{2.271571in}{1.259078in}}%
\pgfpathlineto{\pgfqpoint{2.291762in}{0.951104in}}%
\pgfpathlineto{\pgfqpoint{2.307725in}{0.762628in}}%
\pgfpathlineto{\pgfqpoint{2.328384in}{0.592403in}}%
\pgfpathlineto{\pgfqpoint{2.349984in}{0.495487in}}%
\pgfpathlineto{\pgfqpoint{2.367358in}{0.499079in}}%
\pgfpathlineto{\pgfqpoint{2.384966in}{0.578726in}}%
\pgfpathlineto{\pgfqpoint{2.403748in}{0.693011in}}%
\pgfpathlineto{\pgfqpoint{2.424408in}{0.930057in}}%
\pgfpathlineto{\pgfqpoint{2.442016in}{1.186435in}}%
\pgfpathlineto{\pgfqpoint{2.459622in}{1.349328in}}%
\pgfpathlineto{\pgfqpoint{2.485448in}{1.353398in}}%
\pgfpathlineto{\pgfqpoint{2.501882in}{1.281745in}}%
\pgfpathlineto{\pgfqpoint{2.541089in}{0.771134in}}%
\pgfpathlineto{\pgfqpoint{2.555646in}{0.631232in}}%
\pgfpathlineto{\pgfqpoint{2.576306in}{0.520202in}}%
\pgfpathlineto{\pgfqpoint{2.597434in}{0.498834in}}%
\pgfpathlineto{\pgfqpoint{2.614808in}{0.489066in}}%
\pgfpathlineto{\pgfqpoint{2.633354in}{0.513334in}}%
\pgfpathlineto{\pgfqpoint{2.654485in}{0.633990in}}%
\pgfpathlineto{\pgfqpoint{2.672327in}{0.789559in}}%
\pgfpathlineto{\pgfqpoint{2.712707in}{1.290560in}}%
\pgfpathlineto{\pgfqpoint{2.730081in}{1.376732in}}%
\pgfpathlineto{\pgfqpoint{2.750741in}{1.324359in}}%
\pgfpathlineto{\pgfqpoint{2.769289in}{1.226106in}}%
\pgfpathlineto{\pgfqpoint{2.789712in}{0.911870in}}%
\pgfpathlineto{\pgfqpoint{2.808494in}{0.717886in}}%
\pgfpathlineto{\pgfqpoint{2.827511in}{0.571949in}}%
\pgfpathlineto{\pgfqpoint{2.846997in}{0.503639in}}%
\pgfpathlineto{\pgfqpoint{2.864370in}{0.507784in}}%
\pgfpathlineto{\pgfqpoint{2.888787in}{0.597170in}}%
\pgfpathlineto{\pgfqpoint{2.904047in}{0.706175in}}%
\pgfpathlineto{\pgfqpoint{2.926585in}{0.920642in}}%
\pgfpathlineto{\pgfqpoint{2.944663in}{1.160471in}}%
\pgfpathlineto{\pgfqpoint{2.962271in}{1.332058in}}%
\pgfpathlineto{\pgfqpoint{2.978234in}{0.661988in}}%
\pgfpathlineto{\pgfqpoint{3.020728in}{1.055832in}}%
\pgfpathlineto{\pgfqpoint{3.038102in}{1.301587in}}%
\pgfpathlineto{\pgfqpoint{3.058293in}{1.387198in}}%
\pgfpathlineto{\pgfqpoint{3.074727in}{1.277358in}}%
\pgfpathlineto{\pgfqpoint{3.097030in}{1.107391in}}%
\pgfpathlineto{\pgfqpoint{3.113698in}{0.841441in}}%
\pgfpathlineto{\pgfqpoint{3.134358in}{0.636276in}}%
\pgfpathlineto{\pgfqpoint{3.155018in}{0.532738in}}%
\pgfpathlineto{\pgfqpoint{3.173800in}{0.490777in}}%
\pgfpathlineto{\pgfqpoint{3.194695in}{0.566372in}}%
\pgfpathlineto{\pgfqpoint{3.213008in}{0.658431in}}%
\pgfpathlineto{\pgfqpoint{3.230382in}{0.793479in}}%
\pgfpathlineto{\pgfqpoint{3.253857in}{1.125990in}}%
\pgfpathlineto{\pgfqpoint{3.268884in}{1.296817in}}%
\pgfpathlineto{\pgfqpoint{3.290247in}{1.394400in}}%
\pgfpathlineto{\pgfqpoint{3.307621in}{1.345058in}}%
\pgfpathlineto{\pgfqpoint{3.325463in}{1.224104in}}%
\pgfpathlineto{\pgfqpoint{3.347767in}{0.902653in}}%
\pgfpathlineto{\pgfqpoint{3.365609in}{0.698351in}}%
\pgfpathlineto{\pgfqpoint{3.383452in}{0.594650in}}%
\pgfpathlineto{\pgfqpoint{3.403643in}{0.504258in}}%
\pgfpathlineto{\pgfqpoint{3.421719in}{0.511648in}}%
\pgfpathlineto{\pgfqpoint{3.443554in}{0.610651in}}%
\pgfpathlineto{\pgfqpoint{3.461162in}{0.744755in}}%
\pgfpathlineto{\pgfqpoint{3.479944in}{0.939049in}}%
\pgfpathlineto{\pgfqpoint{3.499899in}{1.201948in}}%
\pgfpathlineto{\pgfqpoint{3.519152in}{1.340016in}}%
\pgfpathlineto{\pgfqpoint{3.539575in}{1.406676in}}%
\pgfpathlineto{\pgfqpoint{3.558123in}{1.351587in}}%
\pgfpathlineto{\pgfqpoint{3.578314in}{1.143453in}}%
\pgfpathlineto{\pgfqpoint{3.596157in}{0.921997in}}%
\pgfpathlineto{\pgfqpoint{3.613999in}{0.758986in}}%
\pgfpathlineto{\pgfqpoint{3.635599in}{0.612113in}}%
\pgfpathlineto{\pgfqpoint{3.653441in}{0.526967in}}%
\pgfpathlineto{\pgfqpoint{3.671284in}{0.504328in}}%
\pgfpathlineto{\pgfqpoint{3.693118in}{0.571366in}}%
\pgfpathlineto{\pgfqpoint{3.710726in}{0.651352in}}%
\pgfpathlineto{\pgfqpoint{3.731855in}{0.829534in}}%
\pgfpathlineto{\pgfqpoint{3.753455in}{1.057749in}}%
\pgfpathlineto{\pgfqpoint{3.771297in}{1.285017in}}%
\pgfpathlineto{\pgfqpoint{3.788434in}{1.357740in}}%
\pgfpathlineto{\pgfqpoint{3.806513in}{1.413669in}}%
\pgfpathlineto{\pgfqpoint{3.828582in}{1.396399in}}%
\pgfpathlineto{\pgfqpoint{3.845719in}{1.292610in}}%
\pgfpathlineto{\pgfqpoint{3.884927in}{0.807595in}}%
\pgfpathlineto{\pgfqpoint{3.903238in}{0.662947in}}%
\pgfpathlineto{\pgfqpoint{3.920846in}{0.574436in}}%
\pgfpathlineto{\pgfqpoint{3.945029in}{0.512072in}}%
\pgfpathlineto{\pgfqpoint{3.960054in}{0.536111in}}%
\pgfpathlineto{\pgfqpoint{3.999496in}{0.722550in}}%
\pgfpathlineto{\pgfqpoint{4.016868in}{0.867895in}}%
\pgfpathlineto{\pgfqpoint{4.038467in}{1.117981in}}%
\pgfpathlineto{\pgfqpoint{4.059831in}{1.342578in}}%
\pgfpathlineto{\pgfqpoint{4.077204in}{1.422897in}}%
\pgfpathlineto{\pgfqpoint{4.096456in}{1.437911in}}%
\pgfpathlineto{\pgfqpoint{4.119698in}{1.332808in}}%
\pgfpathlineto{\pgfqpoint{4.134723in}{1.243912in}}%
\pgfpathlineto{\pgfqpoint{4.153037in}{1.000286in}}%
\pgfpathlineto{\pgfqpoint{4.174166in}{0.789204in}}%
\pgfpathlineto{\pgfqpoint{4.191539in}{0.704134in}}%
\pgfpathlineto{\pgfqpoint{4.209616in}{0.582344in}}%
\pgfpathlineto{\pgfqpoint{4.230981in}{0.529872in}}%
\pgfpathlineto{\pgfqpoint{4.249058in}{0.527147in}}%
\pgfpathlineto{\pgfqpoint{4.269484in}{0.620881in}}%
\pgfpathlineto{\pgfqpoint{4.287326in}{0.736828in}}%
\pgfpathlineto{\pgfqpoint{4.305872in}{1.208858in}}%
\pgfpathlineto{\pgfqpoint{4.325360in}{0.917833in}}%
\pgfpathlineto{\pgfqpoint{4.349072in}{0.653951in}}%
\pgfpathlineto{\pgfqpoint{4.366209in}{0.547664in}}%
\pgfpathlineto{\pgfqpoint{4.384053in}{0.535654in}}%
\pgfpathlineto{\pgfqpoint{4.401425in}{0.636506in}}%
\pgfpathlineto{\pgfqpoint{4.420442in}{0.727370in}}%
\pgfpathlineto{\pgfqpoint{4.442276in}{0.980626in}}%
\pgfpathlineto{\pgfqpoint{4.460587in}{1.249321in}}%
\pgfpathlineto{\pgfqpoint{4.479604in}{1.417801in}}%
\pgfpathlineto{\pgfqpoint{4.474909in}{1.374273in}}%
\pgfpathlineto{\pgfqpoint{4.434294in}{0.762545in}}%
\pgfpathlineto{\pgfqpoint{4.412460in}{0.567095in}}%
\pgfpathlineto{\pgfqpoint{4.396964in}{0.525988in}}%
\pgfpathlineto{\pgfqpoint{4.378418in}{0.643895in}}%
\pgfpathlineto{\pgfqpoint{4.357053in}{0.906293in}}%
\pgfpathlineto{\pgfqpoint{4.338976in}{1.235771in}}%
\pgfpathlineto{\pgfqpoint{4.321134in}{1.416588in}}%
\pgfpathlineto{\pgfqpoint{4.298594in}{1.396273in}}%
\pgfpathlineto{\pgfqpoint{4.282866in}{1.197566in}}%
\pgfpathlineto{\pgfqpoint{4.264318in}{0.891944in}}%
\pgfpathlineto{\pgfqpoint{4.243658in}{0.650638in}}%
\pgfpathlineto{\pgfqpoint{4.221824in}{0.517078in}}%
\pgfpathlineto{\pgfqpoint{4.204216in}{0.568597in}}%
\pgfpathlineto{\pgfqpoint{4.187079in}{0.722847in}}%
\pgfpathlineto{\pgfqpoint{4.165713in}{1.020805in}}%
\pgfpathlineto{\pgfqpoint{4.148576in}{1.315217in}}%
\pgfpathlineto{\pgfqpoint{4.127445in}{1.426413in}}%
\pgfpathlineto{\pgfqpoint{4.107960in}{1.306220in}}%
\pgfpathlineto{\pgfqpoint{4.072275in}{0.718551in}}%
\pgfpathlineto{\pgfqpoint{4.049972in}{0.569712in}}%
\pgfpathlineto{\pgfqpoint{4.033538in}{0.506127in}}%
\pgfpathlineto{\pgfqpoint{4.011469in}{0.605582in}}%
\pgfpathlineto{\pgfqpoint{3.994330in}{0.776640in}}%
\pgfpathlineto{\pgfqpoint{3.973670in}{1.110849in}}%
\pgfpathlineto{\pgfqpoint{3.956062in}{1.342830in}}%
\pgfpathlineto{\pgfqpoint{3.935402in}{1.408692in}}%
\pgfpathlineto{\pgfqpoint{3.915682in}{1.330362in}}%
\pgfpathlineto{\pgfqpoint{3.897840in}{1.411203in}}%
\pgfpathlineto{\pgfqpoint{3.877414in}{1.296558in}}%
\pgfpathlineto{\pgfqpoint{3.860980in}{1.026621in}}%
\pgfpathlineto{\pgfqpoint{3.838912in}{0.739603in}}%
\pgfpathlineto{\pgfqpoint{3.821067in}{0.584819in}}%
\pgfpathlineto{\pgfqpoint{3.797121in}{0.500770in}}%
\pgfpathlineto{\pgfqpoint{3.783036in}{0.555593in}}%
\pgfpathlineto{\pgfqpoint{3.763785in}{0.713099in}}%
\pgfpathlineto{\pgfqpoint{3.722228in}{1.296621in}}%
\pgfpathlineto{\pgfqpoint{3.705560in}{1.394452in}}%
\pgfpathlineto{\pgfqpoint{3.686778in}{1.362101in}}%
\pgfpathlineto{\pgfqpoint{3.667292in}{1.118641in}}%
\pgfpathlineto{\pgfqpoint{3.648981in}{0.853432in}}%
\pgfpathlineto{\pgfqpoint{3.627615in}{0.663139in}}%
\pgfpathlineto{\pgfqpoint{3.611182in}{0.548764in}}%
\pgfpathlineto{\pgfqpoint{3.592634in}{0.494836in}}%
\pgfpathlineto{\pgfqpoint{3.568688in}{0.566348in}}%
\pgfpathlineto{\pgfqpoint{3.552019in}{0.648871in}}%
\pgfpathlineto{\pgfqpoint{3.530420in}{0.859072in}}%
\pgfpathlineto{\pgfqpoint{3.513517in}{1.089590in}}%
\pgfpathlineto{\pgfqpoint{3.497318in}{1.275090in}}%
\pgfpathlineto{\pgfqpoint{3.474778in}{1.392501in}}%
\pgfpathlineto{\pgfqpoint{3.451537in}{1.319439in}}%
\pgfpathlineto{\pgfqpoint{3.415616in}{0.863605in}}%
\pgfpathlineto{\pgfqpoint{3.398242in}{0.671343in}}%
\pgfpathlineto{\pgfqpoint{3.377348in}{0.533625in}}%
\pgfpathlineto{\pgfqpoint{3.355984in}{0.587054in}}%
\pgfpathlineto{\pgfqpoint{3.341194in}{0.510330in}}%
\pgfpathlineto{\pgfqpoint{3.320768in}{0.511277in}}%
\pgfpathlineto{\pgfqpoint{3.303864in}{0.594435in}}%
\pgfpathlineto{\pgfqpoint{3.286021in}{0.814645in}}%
\pgfpathlineto{\pgfqpoint{3.263953in}{1.068276in}}%
\pgfpathlineto{\pgfqpoint{3.244936in}{1.324057in}}%
\pgfpathlineto{\pgfqpoint{3.226624in}{1.382168in}}%
\pgfpathlineto{\pgfqpoint{3.207608in}{1.279356in}}%
\pgfpathlineto{\pgfqpoint{3.187888in}{0.999065in}}%
\pgfpathlineto{\pgfqpoint{3.168636in}{0.743006in}}%
\pgfpathlineto{\pgfqpoint{3.147271in}{0.584720in}}%
\pgfpathlineto{\pgfqpoint{3.130603in}{0.502240in}}%
\pgfpathlineto{\pgfqpoint{3.108534in}{0.494865in}}%
\pgfpathlineto{\pgfqpoint{3.090926in}{0.567055in}}%
\pgfpathlineto{\pgfqpoint{3.069561in}{0.729257in}}%
\pgfpathlineto{\pgfqpoint{3.051718in}{0.953766in}}%
\pgfpathlineto{\pgfqpoint{3.030355in}{1.251512in}}%
\pgfpathlineto{\pgfqpoint{3.014859in}{1.363668in}}%
\pgfpathlineto{\pgfqpoint{2.996548in}{1.365620in}}%
\pgfpathlineto{\pgfqpoint{2.974010in}{1.168963in}}%
\pgfpathlineto{\pgfqpoint{2.955462in}{0.931499in}}%
\pgfpathlineto{\pgfqpoint{2.936680in}{0.721984in}}%
\pgfpathlineto{\pgfqpoint{2.918134in}{0.583802in}}%
\pgfpathlineto{\pgfqpoint{2.896769in}{0.489600in}}%
\pgfpathlineto{\pgfqpoint{2.878692in}{0.510694in}}%
\pgfpathlineto{\pgfqpoint{2.859441in}{0.586541in}}%
\pgfpathlineto{\pgfqpoint{2.841364in}{0.729138in}}%
\pgfpathlineto{\pgfqpoint{2.821642in}{0.992059in}}%
\pgfpathlineto{\pgfqpoint{2.803096in}{1.131293in}}%
\pgfpathlineto{\pgfqpoint{2.777975in}{1.366052in}}%
\pgfpathlineto{\pgfqpoint{2.763183in}{1.370780in}}%
\pgfpathlineto{\pgfqpoint{2.743463in}{1.378402in}}%
\pgfpathlineto{\pgfqpoint{2.726560in}{1.294557in}}%
\pgfpathlineto{\pgfqpoint{2.689935in}{0.839581in}}%
\pgfpathlineto{\pgfqpoint{2.669744in}{0.634885in}}%
\pgfpathlineto{\pgfqpoint{2.646736in}{0.513722in}}%
\pgfpathlineto{\pgfqpoint{2.628893in}{0.483686in}}%
\pgfpathlineto{\pgfqpoint{2.610816in}{0.545648in}}%
\pgfpathlineto{\pgfqpoint{2.592974in}{0.645766in}}%
\pgfpathlineto{\pgfqpoint{2.570905in}{0.877386in}}%
\pgfpathlineto{\pgfqpoint{2.554940in}{1.104603in}}%
\pgfpathlineto{\pgfqpoint{2.534515in}{1.335279in}}%
\pgfpathlineto{\pgfqpoint{2.514089in}{1.375824in}}%
\pgfpathlineto{\pgfqpoint{2.495543in}{1.285967in}}%
\pgfpathlineto{\pgfqpoint{2.476996in}{1.057579in}}%
\pgfpathlineto{\pgfqpoint{2.454693in}{0.818342in}}%
\pgfpathlineto{\pgfqpoint{2.436850in}{0.640030in}}%
\pgfpathlineto{\pgfqpoint{2.419711in}{0.535790in}}%
\pgfpathlineto{\pgfqpoint{2.396470in}{0.484901in}}%
\pgfpathlineto{\pgfqpoint{2.380034in}{0.523034in}}%
\pgfpathlineto{\pgfqpoint{2.360314in}{0.604485in}}%
\pgfpathlineto{\pgfqpoint{2.342472in}{0.723909in}}%
\pgfpathlineto{\pgfqpoint{2.301855in}{1.256687in}}%
\pgfpathlineto{\pgfqpoint{2.282604in}{1.343072in}}%
\pgfpathlineto{\pgfqpoint{2.264058in}{1.375061in}}%
\pgfpathlineto{\pgfqpoint{2.245041in}{1.338112in}}%
\pgfpathlineto{\pgfqpoint{2.226494in}{1.160631in}}%
\pgfpathlineto{\pgfqpoint{2.205599in}{0.917599in}}%
\pgfpathlineto{\pgfqpoint{2.186113in}{0.727389in}}%
\pgfpathlineto{\pgfqpoint{2.149020in}{0.532134in}}%
\pgfpathlineto{\pgfqpoint{2.129298in}{0.489360in}}%
\pgfpathlineto{\pgfqpoint{2.109109in}{0.546822in}}%
\pgfpathlineto{\pgfqpoint{2.089857in}{0.669049in}}%
\pgfpathlineto{\pgfqpoint{2.071075in}{0.840017in}}%
\pgfpathlineto{\pgfqpoint{2.051824in}{0.895748in}}%
\pgfpathlineto{\pgfqpoint{2.032807in}{1.012019in}}%
\pgfpathlineto{\pgfqpoint{2.013556in}{1.149406in}}%
\pgfpathlineto{\pgfqpoint{1.995477in}{1.349098in}}%
\pgfpathlineto{\pgfqpoint{1.976931in}{1.383740in}}%
\pgfpathlineto{\pgfqpoint{1.958618in}{1.289039in}}%
\pgfpathlineto{\pgfqpoint{1.935846in}{1.112995in}}%
\pgfpathlineto{\pgfqpoint{1.917534in}{0.850834in}}%
\pgfpathlineto{\pgfqpoint{1.899690in}{0.709644in}}%
\pgfpathlineto{\pgfqpoint{1.877621in}{0.566063in}}%
\pgfpathlineto{\pgfqpoint{1.859779in}{0.497994in}}%
\pgfpathlineto{\pgfqpoint{1.840998in}{0.513366in}}%
\pgfpathlineto{\pgfqpoint{1.822451in}{0.590452in}}%
\pgfpathlineto{\pgfqpoint{1.802965in}{0.722188in}}%
\pgfpathlineto{\pgfqpoint{1.783479in}{0.969085in}}%
\pgfpathlineto{\pgfqpoint{1.765400in}{1.143124in}}%
\pgfpathlineto{\pgfqpoint{1.744037in}{1.332813in}}%
\pgfpathlineto{\pgfqpoint{1.725724in}{1.397066in}}%
\pgfpathlineto{\pgfqpoint{1.707178in}{1.340998in}}%
\pgfpathlineto{\pgfqpoint{1.688630in}{1.203969in}}%
\pgfpathlineto{\pgfqpoint{1.667736in}{0.914363in}}%
\pgfpathlineto{\pgfqpoint{1.648719in}{0.742239in}}%
\pgfpathlineto{\pgfqpoint{1.629702in}{0.631282in}}%
\pgfpathlineto{\pgfqpoint{1.611391in}{0.539622in}}%
\pgfpathlineto{\pgfqpoint{1.590025in}{0.499314in}}%
\pgfpathlineto{\pgfqpoint{1.571009in}{0.518035in}}%
\pgfpathlineto{\pgfqpoint{1.553166in}{0.606101in}}%
\pgfpathlineto{\pgfqpoint{1.532037in}{0.749673in}}%
\pgfpathlineto{\pgfqpoint{1.515838in}{0.729755in}}%
\pgfpathlineto{\pgfqpoint{1.493769in}{0.996507in}}%
\pgfpathlineto{\pgfqpoint{1.471935in}{1.285500in}}%
\pgfpathlineto{\pgfqpoint{1.456205in}{1.309415in}}%
\pgfpathlineto{\pgfqpoint{1.437893in}{1.308811in}}%
\pgfpathlineto{\pgfqpoint{1.419582in}{1.407042in}}%
\pgfpathlineto{\pgfqpoint{1.397748in}{1.370989in}}%
\pgfpathlineto{\pgfqpoint{1.378731in}{1.219881in}}%
\pgfpathlineto{\pgfqpoint{1.360889in}{0.959318in}}%
\pgfpathlineto{\pgfqpoint{1.342341in}{0.773298in}}%
\pgfpathlineto{\pgfqpoint{1.320977in}{0.627990in}}%
\pgfpathlineto{\pgfqpoint{1.302195in}{0.549408in}}%
\pgfpathlineto{\pgfqpoint{1.283884in}{0.504211in}}%
\pgfpathlineto{\pgfqpoint{1.264631in}{0.545989in}}%
\pgfpathlineto{\pgfqpoint{1.246554in}{0.644067in}}%
\pgfpathlineto{\pgfqpoint{1.224956in}{0.772958in}}%
\pgfpathlineto{\pgfqpoint{1.187860in}{1.229519in}}%
\pgfpathlineto{\pgfqpoint{1.169549in}{1.379601in}}%
\pgfpathlineto{\pgfqpoint{1.151001in}{1.425685in}}%
\pgfpathlineto{\pgfqpoint{1.128698in}{1.393641in}}%
\pgfpathlineto{\pgfqpoint{1.111324in}{1.259646in}}%
\pgfpathlineto{\pgfqpoint{1.091839in}{1.012964in}}%
\pgfpathlineto{\pgfqpoint{1.071884in}{0.831071in}}%
\pgfpathlineto{\pgfqpoint{1.052633in}{0.683223in}}%
\pgfpathlineto{\pgfqpoint{1.032911in}{0.606505in}}%
\pgfpathlineto{\pgfqpoint{1.014834in}{0.556919in}}%
\pgfpathlineto{\pgfqpoint{0.995817in}{0.518851in}}%
\pgfpathlineto{\pgfqpoint{0.978443in}{0.584241in}}%
\pgfpathlineto{\pgfqpoint{0.957315in}{0.700444in}}%
\pgfpathlineto{\pgfqpoint{0.937594in}{0.842829in}}%
\pgfpathlineto{\pgfqpoint{0.919047in}{0.953170in}}%
\pgfpathlineto{\pgfqpoint{0.899795in}{1.203197in}}%
\pgfpathlineto{\pgfqpoint{0.881482in}{1.377529in}}%
\pgfpathlineto{\pgfqpoint{0.860588in}{1.440868in}}%
\pgfpathlineto{\pgfqpoint{0.842276in}{1.413230in}}%
\pgfpathlineto{\pgfqpoint{0.823260in}{1.278502in}}%
\pgfpathlineto{\pgfqpoint{0.802365in}{0.998950in}}%
\pgfpathlineto{\pgfqpoint{0.766915in}{0.703187in}}%
\pgfpathlineto{\pgfqpoint{0.747663in}{0.590390in}}%
\pgfpathlineto{\pgfqpoint{0.725829in}{0.525605in}}%
\pgfpathlineto{\pgfqpoint{0.707281in}{0.531937in}}%
\pgfpathlineto{\pgfqpoint{0.687796in}{0.545619in}}%
\pgfpathlineto{\pgfqpoint{0.668779in}{0.782260in}}%
\pgfpathlineto{\pgfqpoint{0.647650in}{0.638943in}}%
\pgfpathlineto{\pgfqpoint{0.650936in}{0.650057in}}%
\pgfpathlineto{\pgfqpoint{0.658449in}{0.735568in}}%
\pgfpathlineto{\pgfqpoint{0.676292in}{0.974777in}}%
\pgfpathlineto{\pgfqpoint{0.694368in}{1.293065in}}%
\pgfpathlineto{\pgfqpoint{0.712682in}{1.437467in}}%
\pgfpathlineto{\pgfqpoint{0.733811in}{1.386487in}}%
\pgfpathlineto{\pgfqpoint{0.752358in}{1.128477in}}%
\pgfpathlineto{\pgfqpoint{0.769496in}{0.821447in}}%
\pgfpathlineto{\pgfqpoint{0.794382in}{0.573049in}}%
\pgfpathlineto{\pgfqpoint{0.809643in}{0.519378in}}%
\pgfpathlineto{\pgfqpoint{0.827251in}{0.619186in}}%
\pgfpathlineto{\pgfqpoint{0.848380in}{0.839070in}}%
\pgfpathlineto{\pgfqpoint{0.866928in}{1.149741in}}%
\pgfpathlineto{\pgfqpoint{0.885943in}{1.392934in}}%
\pgfpathlineto{\pgfqpoint{0.907777in}{1.404545in}}%
\pgfpathlineto{\pgfqpoint{0.923742in}{1.235526in}}%
\pgfpathlineto{\pgfqpoint{0.944167in}{0.855245in}}%
\pgfpathlineto{\pgfqpoint{0.963653in}{0.636719in}}%
\pgfpathlineto{\pgfqpoint{0.980792in}{0.522653in}}%
\pgfpathlineto{\pgfqpoint{1.000512in}{0.533437in}}%
\pgfpathlineto{\pgfqpoint{1.022112in}{0.697922in}}%
\pgfpathlineto{\pgfqpoint{1.041363in}{0.929905in}}%
\pgfpathlineto{\pgfqpoint{1.061083in}{1.245559in}}%
\pgfpathlineto{\pgfqpoint{1.079631in}{1.406386in}}%
\pgfpathlineto{\pgfqpoint{1.097005in}{1.393809in}}%
\pgfpathlineto{\pgfqpoint{1.117899in}{1.123778in}}%
\pgfpathlineto{\pgfqpoint{1.136916in}{0.798423in}}%
\pgfpathlineto{\pgfqpoint{1.156636in}{0.602591in}}%
\pgfpathlineto{\pgfqpoint{1.176358in}{0.506067in}}%
\pgfpathlineto{\pgfqpoint{1.195138in}{0.548726in}}%
\pgfpathlineto{\pgfqpoint{1.213921in}{0.689219in}}%
\pgfpathlineto{\pgfqpoint{1.234112in}{0.920557in}}%
\pgfpathlineto{\pgfqpoint{1.253832in}{1.228627in}}%
\pgfpathlineto{\pgfqpoint{1.270971in}{1.393031in}}%
\pgfpathlineto{\pgfqpoint{1.291160in}{1.390356in}}%
\pgfpathlineto{\pgfqpoint{1.309473in}{1.211735in}}%
\pgfpathlineto{\pgfqpoint{1.348681in}{0.678412in}}%
\pgfpathlineto{\pgfqpoint{1.365818in}{0.541156in}}%
\pgfpathlineto{\pgfqpoint{1.386009in}{0.499779in}}%
\pgfpathlineto{\pgfqpoint{1.407609in}{0.606297in}}%
\pgfpathlineto{\pgfqpoint{1.425215in}{0.749017in}}%
\pgfpathlineto{\pgfqpoint{1.442823in}{0.926556in}}%
\pgfpathlineto{\pgfqpoint{1.465128in}{1.190069in}}%
\pgfpathlineto{\pgfqpoint{1.481562in}{1.366563in}}%
\pgfpathlineto{\pgfqpoint{1.501751in}{1.379894in}}%
\pgfpathlineto{\pgfqpoint{1.519830in}{1.263150in}}%
\pgfpathlineto{\pgfqpoint{1.542367in}{0.940081in}}%
\pgfpathlineto{\pgfqpoint{1.561150in}{0.824778in}}%
\pgfpathlineto{\pgfqpoint{1.578287in}{0.620413in}}%
\pgfpathlineto{\pgfqpoint{1.598009in}{0.520485in}}%
\pgfpathlineto{\pgfqpoint{1.616555in}{0.494195in}}%
\pgfpathlineto{\pgfqpoint{1.636277in}{0.568226in}}%
\pgfpathlineto{\pgfqpoint{1.653648in}{0.670872in}}%
\pgfpathlineto{\pgfqpoint{1.676188in}{0.832896in}}%
\pgfpathlineto{\pgfqpoint{1.695205in}{1.005137in}}%
\pgfpathlineto{\pgfqpoint{1.713282in}{1.245171in}}%
\pgfpathlineto{\pgfqpoint{1.728776in}{1.382488in}}%
\pgfpathlineto{\pgfqpoint{1.749906in}{1.362324in}}%
\pgfpathlineto{\pgfqpoint{1.769861in}{1.155295in}}%
\pgfpathlineto{\pgfqpoint{1.788409in}{0.861099in}}%
\pgfpathlineto{\pgfqpoint{1.807660in}{0.663831in}}%
\pgfpathlineto{\pgfqpoint{1.829729in}{0.549850in}}%
\pgfpathlineto{\pgfqpoint{1.848745in}{0.496525in}}%
\pgfpathlineto{\pgfqpoint{1.867291in}{0.516179in}}%
\pgfpathlineto{\pgfqpoint{1.885136in}{0.612214in}}%
\pgfpathlineto{\pgfqpoint{1.904387in}{0.732585in}}%
\pgfpathlineto{\pgfqpoint{1.922229in}{0.635957in}}%
\pgfpathlineto{\pgfqpoint{1.941246in}{0.521237in}}%
\pgfpathlineto{\pgfqpoint{1.964018in}{0.509458in}}%
\pgfpathlineto{\pgfqpoint{1.982095in}{0.613379in}}%
\pgfpathlineto{\pgfqpoint{2.001583in}{0.792151in}}%
\pgfpathlineto{\pgfqpoint{2.020129in}{1.047380in}}%
\pgfpathlineto{\pgfqpoint{2.038442in}{1.257670in}}%
\pgfpathlineto{\pgfqpoint{2.061683in}{1.382107in}}%
\pgfpathlineto{\pgfqpoint{2.078119in}{1.319914in}}%
\pgfpathlineto{\pgfqpoint{2.103239in}{0.988151in}}%
\pgfpathlineto{\pgfqpoint{2.116856in}{0.771038in}}%
\pgfpathlineto{\pgfqpoint{2.134464in}{0.604975in}}%
\pgfpathlineto{\pgfqpoint{2.151837in}{0.521025in}}%
\pgfpathlineto{\pgfqpoint{2.172966in}{0.495733in}}%
\pgfpathlineto{\pgfqpoint{2.193860in}{0.592264in}}%
\pgfpathlineto{\pgfqpoint{2.211937in}{0.729651in}}%
\pgfpathlineto{\pgfqpoint{2.236120in}{0.934580in}}%
\pgfpathlineto{\pgfqpoint{2.251614in}{1.235492in}}%
\pgfpathlineto{\pgfqpoint{2.270865in}{1.374729in}}%
\pgfpathlineto{\pgfqpoint{2.290587in}{1.343290in}}%
\pgfpathlineto{\pgfqpoint{2.309604in}{1.185908in}}%
\pgfpathlineto{\pgfqpoint{2.327212in}{0.875543in}}%
\pgfpathlineto{\pgfqpoint{2.346463in}{0.680371in}}%
\pgfpathlineto{\pgfqpoint{2.365009in}{0.546946in}}%
\pgfpathlineto{\pgfqpoint{2.386374in}{0.485497in}}%
\pgfpathlineto{\pgfqpoint{2.405860in}{0.532878in}}%
\pgfpathlineto{\pgfqpoint{2.421825in}{0.628993in}}%
\pgfpathlineto{\pgfqpoint{2.442250in}{0.721780in}}%
\pgfpathlineto{\pgfqpoint{2.463145in}{0.974383in}}%
\pgfpathlineto{\pgfqpoint{2.482162in}{1.227993in}}%
\pgfpathlineto{\pgfqpoint{2.502351in}{1.374638in}}%
\pgfpathlineto{\pgfqpoint{2.519255in}{1.358132in}}%
\pgfpathlineto{\pgfqpoint{2.539210in}{1.194772in}}%
\pgfpathlineto{\pgfqpoint{2.560106in}{0.884398in}}%
\pgfpathlineto{\pgfqpoint{2.578418in}{0.769846in}}%
\pgfpathlineto{\pgfqpoint{2.596731in}{0.617977in}}%
\pgfpathlineto{\pgfqpoint{2.614337in}{0.514740in}}%
\pgfpathlineto{\pgfqpoint{2.636877in}{0.501575in}}%
\pgfpathlineto{\pgfqpoint{2.656128in}{0.601867in}}%
\pgfpathlineto{\pgfqpoint{2.673265in}{0.732974in}}%
\pgfpathlineto{\pgfqpoint{2.692518in}{0.937614in}}%
\pgfpathlineto{\pgfqpoint{2.713178in}{1.237067in}}%
\pgfpathlineto{\pgfqpoint{2.730550in}{1.356368in}}%
\pgfpathlineto{\pgfqpoint{2.751680in}{1.365307in}}%
\pgfpathlineto{\pgfqpoint{2.770226in}{1.262105in}}%
\pgfpathlineto{\pgfqpoint{2.788069in}{0.994132in}}%
\pgfpathlineto{\pgfqpoint{2.810608in}{0.721169in}}%
\pgfpathlineto{\pgfqpoint{2.827746in}{0.604745in}}%
\pgfpathlineto{\pgfqpoint{2.845824in}{0.543900in}}%
\pgfpathlineto{\pgfqpoint{2.865779in}{0.489556in}}%
\pgfpathlineto{\pgfqpoint{2.884327in}{0.534124in}}%
\pgfpathlineto{\pgfqpoint{2.905690in}{0.652311in}}%
\pgfpathlineto{\pgfqpoint{2.922829in}{0.833787in}}%
\pgfpathlineto{\pgfqpoint{2.962035in}{1.301923in}}%
\pgfpathlineto{\pgfqpoint{2.980348in}{1.378730in}}%
\pgfpathlineto{\pgfqpoint{3.000539in}{1.353125in}}%
\pgfpathlineto{\pgfqpoint{3.018147in}{1.232639in}}%
\pgfpathlineto{\pgfqpoint{3.057588in}{0.736642in}}%
\pgfpathlineto{\pgfqpoint{3.076604in}{0.607607in}}%
\pgfpathlineto{\pgfqpoint{3.096327in}{0.518532in}}%
\pgfpathlineto{\pgfqpoint{3.115341in}{0.504362in}}%
\pgfpathlineto{\pgfqpoint{3.132012in}{0.554443in}}%
\pgfpathlineto{\pgfqpoint{3.153846in}{0.689660in}}%
\pgfpathlineto{\pgfqpoint{3.174975in}{0.912664in}}%
\pgfpathlineto{\pgfqpoint{3.189296in}{1.125415in}}%
\pgfpathlineto{\pgfqpoint{3.210660in}{0.664454in}}%
\pgfpathlineto{\pgfqpoint{3.250805in}{1.045156in}}%
\pgfpathlineto{\pgfqpoint{3.270293in}{1.316692in}}%
\pgfpathlineto{\pgfqpoint{3.287430in}{1.394442in}}%
\pgfpathlineto{\pgfqpoint{3.306212in}{1.357403in}}%
\pgfpathlineto{\pgfqpoint{3.327812in}{1.210637in}}%
\pgfpathlineto{\pgfqpoint{3.345420in}{0.962029in}}%
\pgfpathlineto{\pgfqpoint{3.366080in}{0.751348in}}%
\pgfpathlineto{\pgfqpoint{3.384391in}{0.602312in}}%
\pgfpathlineto{\pgfqpoint{3.404582in}{0.513724in}}%
\pgfpathlineto{\pgfqpoint{3.425477in}{0.499175in}}%
\pgfpathlineto{\pgfqpoint{3.443788in}{0.568417in}}%
\pgfpathlineto{\pgfqpoint{3.478535in}{0.841447in}}%
\pgfpathlineto{\pgfqpoint{3.521264in}{1.343581in}}%
\pgfpathlineto{\pgfqpoint{3.537698in}{1.405486in}}%
\pgfpathlineto{\pgfqpoint{3.559063in}{1.362818in}}%
\pgfpathlineto{\pgfqpoint{3.576671in}{1.221663in}}%
\pgfpathlineto{\pgfqpoint{3.597565in}{0.925408in}}%
\pgfpathlineto{\pgfqpoint{3.614468in}{0.798360in}}%
\pgfpathlineto{\pgfqpoint{3.639354in}{0.634130in}}%
\pgfpathlineto{\pgfqpoint{3.654379in}{0.555720in}}%
\pgfpathlineto{\pgfqpoint{3.674336in}{0.504672in}}%
\pgfpathlineto{\pgfqpoint{3.693821in}{0.537196in}}%
\pgfpathlineto{\pgfqpoint{3.711429in}{0.634918in}}%
\pgfpathlineto{\pgfqpoint{3.729741in}{0.768310in}}%
\pgfpathlineto{\pgfqpoint{3.749228in}{0.951369in}}%
\pgfpathlineto{\pgfqpoint{3.770123in}{1.181948in}}%
\pgfpathlineto{\pgfqpoint{3.789374in}{1.333777in}}%
\pgfpathlineto{\pgfqpoint{3.805808in}{1.417332in}}%
\pgfpathlineto{\pgfqpoint{3.827407in}{1.143057in}}%
\pgfpathlineto{\pgfqpoint{3.845016in}{1.323077in}}%
\pgfpathlineto{\pgfqpoint{3.866850in}{1.417739in}}%
\pgfpathlineto{\pgfqpoint{3.884221in}{1.404492in}}%
\pgfpathlineto{\pgfqpoint{3.902300in}{1.259225in}}%
\pgfpathlineto{\pgfqpoint{3.923195in}{0.982665in}}%
\pgfpathlineto{\pgfqpoint{3.941037in}{0.786994in}}%
\pgfpathlineto{\pgfqpoint{3.960288in}{0.652605in}}%
\pgfpathlineto{\pgfqpoint{3.979305in}{0.546708in}}%
\pgfpathlineto{\pgfqpoint{3.999496in}{0.516305in}}%
\pgfpathlineto{\pgfqpoint{4.018042in}{0.577167in}}%
\pgfpathlineto{\pgfqpoint{4.039642in}{0.700228in}}%
\pgfpathlineto{\pgfqpoint{4.058188in}{0.903590in}}%
\pgfpathlineto{\pgfqpoint{4.094578in}{1.322999in}}%
\pgfpathlineto{\pgfqpoint{4.117352in}{1.428745in}}%
\pgfpathlineto{\pgfqpoint{4.134020in}{1.427704in}}%
\pgfpathlineto{\pgfqpoint{4.151863in}{1.363871in}}%
\pgfpathlineto{\pgfqpoint{4.175574in}{1.134031in}}%
\pgfpathlineto{\pgfqpoint{4.190131in}{0.904161in}}%
\pgfpathlineto{\pgfqpoint{4.230276in}{0.614801in}}%
\pgfpathlineto{\pgfqpoint{4.250233in}{0.554705in}}%
\pgfpathlineto{\pgfqpoint{4.268075in}{0.524129in}}%
\pgfpathlineto{\pgfqpoint{4.307986in}{0.675119in}}%
\pgfpathlineto{\pgfqpoint{4.327003in}{0.815774in}}%
\pgfpathlineto{\pgfqpoint{4.345314in}{1.008470in}}%
\pgfpathlineto{\pgfqpoint{4.366680in}{1.259102in}}%
\pgfpathlineto{\pgfqpoint{4.384288in}{1.409409in}}%
\pgfpathlineto{\pgfqpoint{4.402365in}{1.458389in}}%
\pgfpathlineto{\pgfqpoint{4.423494in}{1.416947in}}%
\pgfpathlineto{\pgfqpoint{4.441572in}{1.276617in}}%
\pgfpathlineto{\pgfqpoint{4.459649in}{1.061061in}}%
\pgfpathlineto{\pgfqpoint{4.478901in}{1.254496in}}%
\pgfpathlineto{\pgfqpoint{4.480309in}{1.249757in}}%
\pgfpathlineto{\pgfqpoint{4.474206in}{1.340836in}}%
\pgfpathlineto{\pgfqpoint{4.456127in}{1.457255in}}%
\pgfpathlineto{\pgfqpoint{4.438050in}{1.406568in}}%
\pgfpathlineto{\pgfqpoint{4.415278in}{1.096003in}}%
\pgfpathlineto{\pgfqpoint{4.397670in}{0.812173in}}%
\pgfpathlineto{\pgfqpoint{4.378418in}{0.615780in}}%
\pgfpathlineto{\pgfqpoint{4.354236in}{0.531878in}}%
\pgfpathlineto{\pgfqpoint{4.339916in}{0.629111in}}%
\pgfpathlineto{\pgfqpoint{4.323246in}{0.814847in}}%
\pgfpathlineto{\pgfqpoint{4.300943in}{1.142265in}}%
\pgfpathlineto{\pgfqpoint{4.279343in}{1.411004in}}%
\pgfpathlineto{\pgfqpoint{4.261971in}{1.430896in}}%
\pgfpathlineto{\pgfqpoint{4.243189in}{1.280103in}}%
\pgfpathlineto{\pgfqpoint{4.225112in}{0.968102in}}%
\pgfpathlineto{\pgfqpoint{4.203512in}{0.682928in}}%
\pgfpathlineto{\pgfqpoint{4.185670in}{0.551944in}}%
\pgfpathlineto{\pgfqpoint{4.165950in}{0.536210in}}%
\pgfpathlineto{\pgfqpoint{4.147636in}{0.578450in}}%
\pgfpathlineto{\pgfqpoint{4.127680in}{0.773365in}}%
\pgfpathlineto{\pgfqpoint{4.110308in}{1.055578in}}%
\pgfpathlineto{\pgfqpoint{4.086831in}{1.382672in}}%
\pgfpathlineto{\pgfqpoint{4.072744in}{1.426073in}}%
\pgfpathlineto{\pgfqpoint{4.053024in}{1.298283in}}%
\pgfpathlineto{\pgfqpoint{4.028137in}{0.898379in}}%
\pgfpathlineto{\pgfqpoint{4.010764in}{0.699068in}}%
\pgfpathlineto{\pgfqpoint{3.994096in}{0.562579in}}%
\pgfpathlineto{\pgfqpoint{3.974844in}{0.508279in}}%
\pgfpathlineto{\pgfqpoint{3.954419in}{0.627936in}}%
\pgfpathlineto{\pgfqpoint{3.940097in}{0.785782in}}%
\pgfpathlineto{\pgfqpoint{3.918029in}{1.136148in}}%
\pgfpathlineto{\pgfqpoint{3.896194in}{1.389277in}}%
\pgfpathlineto{\pgfqpoint{3.876709in}{1.396620in}}%
\pgfpathlineto{\pgfqpoint{3.856754in}{1.203912in}}%
\pgfpathlineto{\pgfqpoint{3.839146in}{0.897634in}}%
\pgfpathlineto{\pgfqpoint{3.821773in}{0.702419in}}%
\pgfpathlineto{\pgfqpoint{3.802756in}{0.545356in}}%
\pgfpathlineto{\pgfqpoint{3.783270in}{0.498208in}}%
\pgfpathlineto{\pgfqpoint{3.764488in}{0.586479in}}%
\pgfpathlineto{\pgfqpoint{3.742419in}{0.789049in}}%
\pgfpathlineto{\pgfqpoint{3.724811in}{1.071597in}}%
\pgfpathlineto{\pgfqpoint{3.704857in}{1.346691in}}%
\pgfpathlineto{\pgfqpoint{3.687012in}{1.400199in}}%
\pgfpathlineto{\pgfqpoint{3.663537in}{1.242556in}}%
\pgfpathlineto{\pgfqpoint{3.648510in}{1.008965in}}%
\pgfpathlineto{\pgfqpoint{3.627146in}{0.743924in}}%
\pgfpathlineto{\pgfqpoint{3.611182in}{0.601310in}}%
\pgfpathlineto{\pgfqpoint{3.590287in}{0.504194in}}%
\pgfpathlineto{\pgfqpoint{3.569156in}{0.523250in}}%
\pgfpathlineto{\pgfqpoint{3.551550in}{0.629943in}}%
\pgfpathlineto{\pgfqpoint{3.534177in}{0.795366in}}%
\pgfpathlineto{\pgfqpoint{3.513517in}{0.962358in}}%
\pgfpathlineto{\pgfqpoint{3.492857in}{1.257095in}}%
\pgfpathlineto{\pgfqpoint{3.475483in}{1.386969in}}%
\pgfpathlineto{\pgfqpoint{3.453884in}{1.374775in}}%
\pgfpathlineto{\pgfqpoint{3.436510in}{1.198331in}}%
\pgfpathlineto{\pgfqpoint{3.415616in}{0.903064in}}%
\pgfpathlineto{\pgfqpoint{3.397539in}{0.760691in}}%
\pgfpathlineto{\pgfqpoint{3.360679in}{0.518312in}}%
\pgfpathlineto{\pgfqpoint{3.338376in}{0.499866in}}%
\pgfpathlineto{\pgfqpoint{3.321237in}{0.579482in}}%
\pgfpathlineto{\pgfqpoint{3.299639in}{0.738749in}}%
\pgfpathlineto{\pgfqpoint{3.278980in}{1.031484in}}%
\pgfpathlineto{\pgfqpoint{3.264423in}{0.636356in}}%
\pgfpathlineto{\pgfqpoint{3.243293in}{0.509518in}}%
\pgfpathlineto{\pgfqpoint{3.222867in}{0.514525in}}%
\pgfpathlineto{\pgfqpoint{3.208077in}{0.603007in}}%
\pgfpathlineto{\pgfqpoint{3.187888in}{0.789711in}}%
\pgfpathlineto{\pgfqpoint{3.166053in}{1.115166in}}%
\pgfpathlineto{\pgfqpoint{3.148445in}{1.281701in}}%
\pgfpathlineto{\pgfqpoint{3.128489in}{1.383278in}}%
\pgfpathlineto{\pgfqpoint{3.110412in}{1.297616in}}%
\pgfpathlineto{\pgfqpoint{3.091395in}{1.046228in}}%
\pgfpathlineto{\pgfqpoint{3.070266in}{0.752098in}}%
\pgfpathlineto{\pgfqpoint{3.052658in}{0.595614in}}%
\pgfpathlineto{\pgfqpoint{3.032938in}{0.494365in}}%
\pgfpathlineto{\pgfqpoint{3.011807in}{0.519686in}}%
\pgfpathlineto{\pgfqpoint{2.974714in}{0.750082in}}%
\pgfpathlineto{\pgfqpoint{2.957811in}{1.038468in}}%
\pgfpathlineto{\pgfqpoint{2.935037in}{1.316348in}}%
\pgfpathlineto{\pgfqpoint{2.914142in}{1.379102in}}%
\pgfpathlineto{\pgfqpoint{2.898883in}{1.310900in}}%
\pgfpathlineto{\pgfqpoint{2.878926in}{1.030970in}}%
\pgfpathlineto{\pgfqpoint{2.854509in}{0.714099in}}%
\pgfpathlineto{\pgfqpoint{2.841364in}{0.609484in}}%
\pgfpathlineto{\pgfqpoint{2.819999in}{0.505687in}}%
\pgfpathlineto{\pgfqpoint{2.804739in}{0.492869in}}%
\pgfpathlineto{\pgfqpoint{2.782905in}{0.584485in}}%
\pgfpathlineto{\pgfqpoint{2.762714in}{0.737700in}}%
\pgfpathlineto{\pgfqpoint{2.744637in}{1.006464in}}%
\pgfpathlineto{\pgfqpoint{2.722099in}{1.311725in}}%
\pgfpathlineto{\pgfqpoint{2.705664in}{1.381449in}}%
\pgfpathlineto{\pgfqpoint{2.688056in}{1.318996in}}%
\pgfpathlineto{\pgfqpoint{2.666692in}{1.224512in}}%
\pgfpathlineto{\pgfqpoint{2.647676in}{0.924341in}}%
\pgfpathlineto{\pgfqpoint{2.627485in}{0.692707in}}%
\pgfpathlineto{\pgfqpoint{2.610113in}{0.567330in}}%
\pgfpathlineto{\pgfqpoint{2.590625in}{0.490304in}}%
\pgfpathlineto{\pgfqpoint{2.571140in}{0.507001in}}%
\pgfpathlineto{\pgfqpoint{2.550714in}{0.624152in}}%
\pgfpathlineto{\pgfqpoint{2.533811in}{0.758020in}}%
\pgfpathlineto{\pgfqpoint{2.515498in}{1.024478in}}%
\pgfpathlineto{\pgfqpoint{2.493666in}{1.315278in}}%
\pgfpathlineto{\pgfqpoint{2.474884in}{1.369935in}}%
\pgfpathlineto{\pgfqpoint{2.458215in}{1.358489in}}%
\pgfpathlineto{\pgfqpoint{2.438024in}{1.187588in}}%
\pgfpathlineto{\pgfqpoint{2.438259in}{0.979663in}}%
\pgfpathlineto{\pgfqpoint{2.417833in}{0.870626in}}%
\pgfpathlineto{\pgfqpoint{2.400460in}{0.701396in}}%
\pgfpathlineto{\pgfqpoint{2.381914in}{0.574653in}}%
\pgfpathlineto{\pgfqpoint{2.359611in}{0.491390in}}%
\pgfpathlineto{\pgfqpoint{2.341063in}{0.482035in}}%
\pgfpathlineto{\pgfqpoint{2.319463in}{0.537591in}}%
\pgfpathlineto{\pgfqpoint{2.301855in}{0.657989in}}%
\pgfpathlineto{\pgfqpoint{2.263353in}{1.098035in}}%
\pgfpathlineto{\pgfqpoint{2.243867in}{1.346899in}}%
\pgfpathlineto{\pgfqpoint{2.225790in}{1.381037in}}%
\pgfpathlineto{\pgfqpoint{2.207477in}{1.295655in}}%
\pgfpathlineto{\pgfqpoint{2.186348in}{1.026107in}}%
\pgfpathlineto{\pgfqpoint{2.167800in}{0.773385in}}%
\pgfpathlineto{\pgfqpoint{2.149020in}{0.629992in}}%
\pgfpathlineto{\pgfqpoint{2.130003in}{0.529653in}}%
\pgfpathlineto{\pgfqpoint{2.111924in}{0.491304in}}%
\pgfpathlineto{\pgfqpoint{2.090326in}{0.536978in}}%
\pgfpathlineto{\pgfqpoint{2.071075in}{0.612512in}}%
\pgfpathlineto{\pgfqpoint{2.052527in}{0.732296in}}%
\pgfpathlineto{\pgfqpoint{2.013321in}{1.276617in}}%
\pgfpathlineto{\pgfqpoint{1.993834in}{1.385351in}}%
\pgfpathlineto{\pgfqpoint{1.975757in}{1.342122in}}%
\pgfpathlineto{\pgfqpoint{1.957680in}{1.199864in}}%
\pgfpathlineto{\pgfqpoint{1.937723in}{0.968649in}}%
\pgfpathlineto{\pgfqpoint{1.919881in}{0.788909in}}%
\pgfpathlineto{\pgfqpoint{1.898283in}{0.609286in}}%
\pgfpathlineto{\pgfqpoint{1.879501in}{0.538592in}}%
\pgfpathlineto{\pgfqpoint{1.861893in}{0.505890in}}%
\pgfpathlineto{\pgfqpoint{1.842407in}{0.495362in}}%
\pgfpathlineto{\pgfqpoint{1.820573in}{0.572875in}}%
\pgfpathlineto{\pgfqpoint{1.804374in}{0.687131in}}%
\pgfpathlineto{\pgfqpoint{1.785357in}{0.892116in}}%
\pgfpathlineto{\pgfqpoint{1.764931in}{1.196182in}}%
\pgfpathlineto{\pgfqpoint{1.746384in}{1.367136in}}%
\pgfpathlineto{\pgfqpoint{1.724315in}{1.392967in}}%
\pgfpathlineto{\pgfqpoint{1.686752in}{1.147719in}}%
\pgfpathlineto{\pgfqpoint{1.668675in}{0.924351in}}%
\pgfpathlineto{\pgfqpoint{1.646841in}{0.733579in}}%
\pgfpathlineto{\pgfqpoint{1.631111in}{0.629886in}}%
\pgfpathlineto{\pgfqpoint{1.606696in}{0.524493in}}%
\pgfpathlineto{\pgfqpoint{1.590496in}{0.498234in}}%
\pgfpathlineto{\pgfqpoint{1.574061in}{0.532831in}}%
\pgfpathlineto{\pgfqpoint{1.554106in}{0.617407in}}%
\pgfpathlineto{\pgfqpoint{1.534855in}{0.791061in}}%
\pgfpathlineto{\pgfqpoint{1.511377in}{0.957710in}}%
\pgfpathlineto{\pgfqpoint{1.495178in}{1.168736in}}%
\pgfpathlineto{\pgfqpoint{1.476865in}{1.356547in}}%
\pgfpathlineto{\pgfqpoint{1.454327in}{1.408926in}}%
\pgfpathlineto{\pgfqpoint{1.436485in}{1.382594in}}%
\pgfpathlineto{\pgfqpoint{1.421225in}{1.255066in}}%
\pgfpathlineto{\pgfqpoint{1.400094in}{1.100432in}}%
\pgfpathlineto{\pgfqpoint{1.380374in}{1.244184in}}%
\pgfpathlineto{\pgfqpoint{1.362297in}{0.999313in}}%
\pgfpathlineto{\pgfqpoint{1.341637in}{0.908990in}}%
\pgfpathlineto{\pgfqpoint{1.319334in}{0.733607in}}%
\pgfpathlineto{\pgfqpoint{1.300786in}{0.600993in}}%
\pgfpathlineto{\pgfqpoint{1.285996in}{0.530788in}}%
\pgfpathlineto{\pgfqpoint{1.263927in}{0.509461in}}%
\pgfpathlineto{\pgfqpoint{1.245145in}{0.553735in}}%
\pgfpathlineto{\pgfqpoint{1.226363in}{0.640137in}}%
\pgfpathlineto{\pgfqpoint{1.207817in}{0.756410in}}%
\pgfpathlineto{\pgfqpoint{1.189269in}{0.984596in}}%
\pgfpathlineto{\pgfqpoint{1.166497in}{1.290492in}}%
\pgfpathlineto{\pgfqpoint{1.151706in}{1.393438in}}%
\pgfpathlineto{\pgfqpoint{1.129169in}{1.424844in}}%
\pgfpathlineto{\pgfqpoint{1.111795in}{1.400665in}}%
\pgfpathlineto{\pgfqpoint{1.089961in}{1.222180in}}%
\pgfpathlineto{\pgfqpoint{1.071413in}{0.965322in}}%
\pgfpathlineto{\pgfqpoint{1.053336in}{0.803030in}}%
\pgfpathlineto{\pgfqpoint{1.034319in}{0.669341in}}%
\pgfpathlineto{\pgfqpoint{1.016242in}{0.567631in}}%
\pgfpathlineto{\pgfqpoint{0.994643in}{0.518116in}}%
\pgfpathlineto{\pgfqpoint{0.977975in}{0.564712in}}%
\pgfpathlineto{\pgfqpoint{0.957080in}{0.681978in}}%
\pgfpathlineto{\pgfqpoint{0.938767in}{0.832096in}}%
\pgfpathlineto{\pgfqpoint{0.917403in}{1.051385in}}%
\pgfpathlineto{\pgfqpoint{0.898856in}{1.284594in}}%
\pgfpathlineto{\pgfqpoint{0.880073in}{1.415329in}}%
\pgfpathlineto{\pgfqpoint{0.862467in}{1.443121in}}%
\pgfpathlineto{\pgfqpoint{0.843451in}{1.405237in}}%
\pgfpathlineto{\pgfqpoint{0.821851in}{0.652925in}}%
\pgfpathlineto{\pgfqpoint{0.803303in}{0.840023in}}%
\pgfpathlineto{\pgfqpoint{0.784523in}{1.118190in}}%
\pgfpathlineto{\pgfqpoint{0.763157in}{1.371231in}}%
\pgfpathlineto{\pgfqpoint{0.747427in}{1.442042in}}%
\pgfpathlineto{\pgfqpoint{0.725595in}{1.406102in}}%
\pgfpathlineto{\pgfqpoint{0.707516in}{1.261367in}}%
\pgfpathlineto{\pgfqpoint{0.688736in}{1.011205in}}%
\pgfpathlineto{\pgfqpoint{0.666667in}{0.759187in}}%
\pgfpathlineto{\pgfqpoint{0.648822in}{0.625161in}}%
\pgfpathlineto{\pgfqpoint{0.650231in}{0.631428in}}%
\pgfpathlineto{\pgfqpoint{0.676526in}{0.933369in}}%
\pgfpathlineto{\pgfqpoint{0.695543in}{1.273894in}}%
\pgfpathlineto{\pgfqpoint{0.713151in}{1.433076in}}%
\pgfpathlineto{\pgfqpoint{0.731228in}{1.410598in}}%
\pgfpathlineto{\pgfqpoint{0.750950in}{1.167743in}}%
\pgfpathlineto{\pgfqpoint{0.772079in}{0.806387in}}%
\pgfpathlineto{\pgfqpoint{0.790156in}{0.607843in}}%
\pgfpathlineto{\pgfqpoint{0.807529in}{0.663866in}}%
\pgfpathlineto{\pgfqpoint{0.827486in}{0.927731in}}%
\pgfpathlineto{\pgfqpoint{0.847440in}{1.251173in}}%
\pgfpathlineto{\pgfqpoint{0.868571in}{1.425760in}}%
\pgfpathlineto{\pgfqpoint{0.886648in}{1.369900in}}%
\pgfpathlineto{\pgfqpoint{0.905430in}{1.093811in}}%
\pgfpathlineto{\pgfqpoint{0.925385in}{0.721880in}}%
\pgfpathlineto{\pgfqpoint{0.944167in}{0.556671in}}%
\pgfpathlineto{\pgfqpoint{0.962479in}{0.510954in}}%
\pgfpathlineto{\pgfqpoint{0.983844in}{0.641007in}}%
\pgfpathlineto{\pgfqpoint{1.003330in}{0.849025in}}%
\pgfpathlineto{\pgfqpoint{1.022581in}{1.175362in}}%
\pgfpathlineto{\pgfqpoint{1.038077in}{1.370891in}}%
\pgfpathlineto{\pgfqpoint{1.059909in}{1.393990in}}%
\pgfpathlineto{\pgfqpoint{1.079397in}{1.195987in}}%
\pgfpathlineto{\pgfqpoint{1.097239in}{0.875967in}}%
\pgfpathlineto{\pgfqpoint{1.117664in}{0.628675in}}%
\pgfpathlineto{\pgfqpoint{1.136445in}{0.515152in}}%
\pgfpathlineto{\pgfqpoint{1.156167in}{0.521014in}}%
\pgfpathlineto{\pgfqpoint{1.175184in}{0.633899in}}%
\pgfpathlineto{\pgfqpoint{1.195609in}{0.803275in}}%
\pgfpathlineto{\pgfqpoint{1.213921in}{1.086807in}}%
\pgfpathlineto{\pgfqpoint{1.230120in}{1.316664in}}%
\pgfpathlineto{\pgfqpoint{1.251720in}{1.403264in}}%
\pgfpathlineto{\pgfqpoint{1.270971in}{1.356261in}}%
\pgfpathlineto{\pgfqpoint{1.290456in}{1.122335in}}%
\pgfpathlineto{\pgfqpoint{1.309473in}{0.797804in}}%
\pgfpathlineto{\pgfqpoint{1.328255in}{0.620686in}}%
\pgfpathlineto{\pgfqpoint{1.347272in}{0.509965in}}%
\pgfpathlineto{\pgfqpoint{1.365818in}{0.496390in}}%
\pgfpathlineto{\pgfqpoint{1.384366in}{0.582173in}}%
\pgfpathlineto{\pgfqpoint{1.404791in}{0.727363in}}%
\pgfpathlineto{\pgfqpoint{1.426860in}{0.976195in}}%
\pgfpathlineto{\pgfqpoint{1.445406in}{1.268210in}}%
\pgfpathlineto{\pgfqpoint{1.462311in}{1.377471in}}%
\pgfpathlineto{\pgfqpoint{1.480856in}{1.374341in}}%
\pgfpathlineto{\pgfqpoint{1.501516in}{1.186063in}}%
\pgfpathlineto{\pgfqpoint{1.520064in}{0.925220in}}%
\pgfpathlineto{\pgfqpoint{1.540724in}{0.679253in}}%
\pgfpathlineto{\pgfqpoint{1.559506in}{0.544008in}}%
\pgfpathlineto{\pgfqpoint{1.578758in}{0.498728in}}%
\pgfpathlineto{\pgfqpoint{1.597538in}{0.506971in}}%
\pgfpathlineto{\pgfqpoint{1.617026in}{0.592599in}}%
\pgfpathlineto{\pgfqpoint{1.635103in}{0.739747in}}%
\pgfpathlineto{\pgfqpoint{1.658814in}{0.957398in}}%
\pgfpathlineto{\pgfqpoint{1.674545in}{1.048352in}}%
\pgfpathlineto{\pgfqpoint{1.696143in}{0.711823in}}%
\pgfpathlineto{\pgfqpoint{1.711873in}{0.579721in}}%
\pgfpathlineto{\pgfqpoint{1.732533in}{0.489116in}}%
\pgfpathlineto{\pgfqpoint{1.751081in}{0.532970in}}%
\pgfpathlineto{\pgfqpoint{1.769861in}{0.667467in}}%
\pgfpathlineto{\pgfqpoint{1.789349in}{0.852964in}}%
\pgfpathlineto{\pgfqpoint{1.807660in}{1.143508in}}%
\pgfpathlineto{\pgfqpoint{1.830668in}{1.365658in}}%
\pgfpathlineto{\pgfqpoint{1.848511in}{1.367477in}}%
\pgfpathlineto{\pgfqpoint{1.867997in}{1.177289in}}%
\pgfpathlineto{\pgfqpoint{1.886544in}{0.896429in}}%
\pgfpathlineto{\pgfqpoint{1.908142in}{0.632488in}}%
\pgfpathlineto{\pgfqpoint{1.924810in}{0.524389in}}%
\pgfpathlineto{\pgfqpoint{1.943827in}{0.484006in}}%
\pgfpathlineto{\pgfqpoint{1.961906in}{0.509792in}}%
\pgfpathlineto{\pgfqpoint{1.980686in}{0.618143in}}%
\pgfpathlineto{\pgfqpoint{2.003226in}{0.842604in}}%
\pgfpathlineto{\pgfqpoint{2.019894in}{1.118544in}}%
\pgfpathlineto{\pgfqpoint{2.038911in}{1.303432in}}%
\pgfpathlineto{\pgfqpoint{2.061214in}{1.379215in}}%
\pgfpathlineto{\pgfqpoint{2.077179in}{1.291925in}}%
\pgfpathlineto{\pgfqpoint{2.098308in}{0.971349in}}%
\pgfpathlineto{\pgfqpoint{2.113098in}{0.722803in}}%
\pgfpathlineto{\pgfqpoint{2.135872in}{0.573215in}}%
\pgfpathlineto{\pgfqpoint{2.157236in}{0.486716in}}%
\pgfpathlineto{\pgfqpoint{2.175313in}{0.514289in}}%
\pgfpathlineto{\pgfqpoint{2.193626in}{0.616906in}}%
\pgfpathlineto{\pgfqpoint{2.211937in}{0.774322in}}%
\pgfpathlineto{\pgfqpoint{2.232597in}{1.062059in}}%
\pgfpathlineto{\pgfqpoint{2.250676in}{1.297491in}}%
\pgfpathlineto{\pgfqpoint{2.268284in}{1.380327in}}%
\pgfpathlineto{\pgfqpoint{2.286127in}{1.347617in}}%
\pgfpathlineto{\pgfqpoint{2.307961in}{1.165529in}}%
\pgfpathlineto{\pgfqpoint{2.329090in}{0.827242in}}%
\pgfpathlineto{\pgfqpoint{2.346932in}{0.657575in}}%
\pgfpathlineto{\pgfqpoint{2.364540in}{0.539717in}}%
\pgfpathlineto{\pgfqpoint{2.385200in}{0.480587in}}%
\pgfpathlineto{\pgfqpoint{2.403982in}{0.523291in}}%
\pgfpathlineto{\pgfqpoint{2.442485in}{0.752407in}}%
\pgfpathlineto{\pgfqpoint{2.463145in}{1.064320in}}%
\pgfpathlineto{\pgfqpoint{2.481222in}{1.303823in}}%
\pgfpathlineto{\pgfqpoint{2.501413in}{1.374892in}}%
\pgfpathlineto{\pgfqpoint{2.521133in}{1.344924in}}%
\pgfpathlineto{\pgfqpoint{2.538272in}{1.153977in}}%
\pgfpathlineto{\pgfqpoint{2.559401in}{0.793687in}}%
\pgfpathlineto{\pgfqpoint{2.577949in}{0.640772in}}%
\pgfpathlineto{\pgfqpoint{2.598843in}{0.518002in}}%
\pgfpathlineto{\pgfqpoint{2.634997in}{0.490799in}}%
\pgfpathlineto{\pgfqpoint{2.655188in}{0.576109in}}%
\pgfpathlineto{\pgfqpoint{2.673031in}{0.682830in}}%
\pgfpathlineto{\pgfqpoint{2.694161in}{0.918262in}}%
\pgfpathlineto{\pgfqpoint{2.715525in}{1.220713in}}%
\pgfpathlineto{\pgfqpoint{2.731021in}{1.355452in}}%
\pgfpathlineto{\pgfqpoint{2.748863in}{1.374770in}}%
\pgfpathlineto{\pgfqpoint{2.768818in}{1.274566in}}%
\pgfpathlineto{\pgfqpoint{2.790652in}{0.960640in}}%
\pgfpathlineto{\pgfqpoint{2.808260in}{0.808046in}}%
\pgfpathlineto{\pgfqpoint{2.825399in}{0.631344in}}%
\pgfpathlineto{\pgfqpoint{2.847233in}{0.519168in}}%
\pgfpathlineto{\pgfqpoint{2.865310in}{0.485630in}}%
\pgfpathlineto{\pgfqpoint{2.886204in}{0.558537in}}%
\pgfpathlineto{\pgfqpoint{2.903107in}{0.610551in}}%
\pgfpathlineto{\pgfqpoint{2.924941in}{0.761490in}}%
\pgfpathlineto{\pgfqpoint{2.942784in}{0.989641in}}%
\pgfpathlineto{\pgfqpoint{2.960157in}{1.241739in}}%
\pgfpathlineto{\pgfqpoint{2.981286in}{1.379149in}}%
\pgfpathlineto{\pgfqpoint{2.999600in}{1.378339in}}%
\pgfpathlineto{\pgfqpoint{3.020494in}{1.242163in}}%
\pgfpathlineto{\pgfqpoint{3.039042in}{0.965196in}}%
\pgfpathlineto{\pgfqpoint{3.059702in}{0.703921in}}%
\pgfpathlineto{\pgfqpoint{3.078013in}{0.586854in}}%
\pgfpathlineto{\pgfqpoint{3.096090in}{0.515858in}}%
\pgfpathlineto{\pgfqpoint{3.116516in}{0.496085in}}%
\pgfpathlineto{\pgfqpoint{3.134829in}{0.561892in}}%
\pgfpathlineto{\pgfqpoint{3.152671in}{0.658652in}}%
\pgfpathlineto{\pgfqpoint{3.173331in}{0.867810in}}%
\pgfpathlineto{\pgfqpoint{3.212537in}{1.300335in}}%
\pgfpathlineto{\pgfqpoint{3.229442in}{1.383717in}}%
\pgfpathlineto{\pgfqpoint{3.251979in}{1.377003in}}%
\pgfpathlineto{\pgfqpoint{3.267944in}{1.303979in}}%
\pgfpathlineto{\pgfqpoint{3.287430in}{1.046698in}}%
\pgfpathlineto{\pgfqpoint{3.308090in}{0.832417in}}%
\pgfpathlineto{\pgfqpoint{3.326169in}{0.676373in}}%
\pgfpathlineto{\pgfqpoint{3.347063in}{0.548977in}}%
\pgfpathlineto{\pgfqpoint{3.364906in}{0.498554in}}%
\pgfpathlineto{\pgfqpoint{3.384157in}{0.516718in}}%
\pgfpathlineto{\pgfqpoint{3.402703in}{0.603254in}}%
\pgfpathlineto{\pgfqpoint{3.425242in}{0.757935in}}%
\pgfpathlineto{\pgfqpoint{3.442145in}{0.934590in}}%
\pgfpathlineto{\pgfqpoint{3.463510in}{1.187746in}}%
\pgfpathlineto{\pgfqpoint{3.481587in}{1.333622in}}%
\pgfpathlineto{\pgfqpoint{3.500135in}{1.404643in}}%
\pgfpathlineto{\pgfqpoint{3.520795in}{1.375415in}}%
\pgfpathlineto{\pgfqpoint{3.538167in}{1.239238in}}%
\pgfpathlineto{\pgfqpoint{3.556245in}{1.058762in}}%
\pgfpathlineto{\pgfqpoint{3.575731in}{0.834831in}}%
\pgfpathlineto{\pgfqpoint{3.595451in}{0.697669in}}%
\pgfpathlineto{\pgfqpoint{3.615877in}{0.571126in}}%
\pgfpathlineto{\pgfqpoint{3.634425in}{0.506708in}}%
\pgfpathlineto{\pgfqpoint{3.656022in}{0.523727in}}%
\pgfpathlineto{\pgfqpoint{3.673396in}{0.599779in}}%
\pgfpathlineto{\pgfqpoint{3.692882in}{0.639963in}}%
\pgfpathlineto{\pgfqpoint{3.712604in}{0.761327in}}%
\pgfpathlineto{\pgfqpoint{3.731150in}{0.899028in}}%
\pgfpathlineto{\pgfqpoint{3.768714in}{1.332115in}}%
\pgfpathlineto{\pgfqpoint{3.790314in}{1.412082in}}%
\pgfpathlineto{\pgfqpoint{3.808860in}{1.420787in}}%
\pgfpathlineto{\pgfqpoint{3.826233in}{1.411850in}}%
\pgfpathlineto{\pgfqpoint{3.845016in}{1.366909in}}%
\pgfpathlineto{\pgfqpoint{3.865205in}{1.219053in}}%
\pgfpathlineto{\pgfqpoint{3.886335in}{0.903933in}}%
\pgfpathlineto{\pgfqpoint{3.903943in}{0.718804in}}%
\pgfpathlineto{\pgfqpoint{3.922960in}{0.595242in}}%
\pgfpathlineto{\pgfqpoint{3.944323in}{0.525459in}}%
\pgfpathlineto{\pgfqpoint{3.960288in}{0.513323in}}%
\pgfpathlineto{\pgfqpoint{3.981888in}{0.586930in}}%
\pgfpathlineto{\pgfqpoint{3.998791in}{0.671829in}}%
\pgfpathlineto{\pgfqpoint{4.022034in}{0.849701in}}%
\pgfpathlineto{\pgfqpoint{4.039407in}{1.079562in}}%
\pgfpathlineto{\pgfqpoint{4.058188in}{1.281157in}}%
\pgfpathlineto{\pgfqpoint{4.075092in}{1.401411in}}%
\pgfpathlineto{\pgfqpoint{4.096926in}{1.438349in}}%
\pgfpathlineto{\pgfqpoint{4.115003in}{1.422689in}}%
\pgfpathlineto{\pgfqpoint{4.134960in}{1.321267in}}%
\pgfpathlineto{\pgfqpoint{4.154680in}{1.009043in}}%
\pgfpathlineto{\pgfqpoint{4.172522in}{1.295102in}}%
\pgfpathlineto{\pgfqpoint{4.193651in}{1.435650in}}%
\pgfpathlineto{\pgfqpoint{4.211730in}{1.419161in}}%
\pgfpathlineto{\pgfqpoint{4.229807in}{1.288330in}}%
\pgfpathlineto{\pgfqpoint{4.249762in}{0.998836in}}%
\pgfpathlineto{\pgfqpoint{4.267841in}{0.799684in}}%
\pgfpathlineto{\pgfqpoint{4.288970in}{0.624557in}}%
\pgfpathlineto{\pgfqpoint{4.307046in}{0.543725in}}%
\pgfpathlineto{\pgfqpoint{4.328412in}{0.548304in}}%
\pgfpathlineto{\pgfqpoint{4.344846in}{0.619919in}}%
\pgfpathlineto{\pgfqpoint{4.365740in}{0.748188in}}%
\pgfpathlineto{\pgfqpoint{4.384522in}{0.928130in}}%
\pgfpathlineto{\pgfqpoint{4.404008in}{1.182988in}}%
\pgfpathlineto{\pgfqpoint{4.423494in}{1.387687in}}%
\pgfpathlineto{\pgfqpoint{4.443216in}{1.458881in}}%
\pgfpathlineto{\pgfqpoint{4.462701in}{1.430785in}}%
\pgfpathlineto{\pgfqpoint{4.481718in}{1.282729in}}%
\pgfpathlineto{\pgfqpoint{4.482187in}{1.289494in}}%
\pgfpathlineto{\pgfqpoint{4.472562in}{1.393421in}}%
\pgfpathlineto{\pgfqpoint{4.453546in}{1.458848in}}%
\pgfpathlineto{\pgfqpoint{4.435467in}{1.354954in}}%
\pgfpathlineto{\pgfqpoint{4.416921in}{1.048459in}}%
\pgfpathlineto{\pgfqpoint{4.399078in}{0.772423in}}%
\pgfpathlineto{\pgfqpoint{4.377010in}{0.577365in}}%
\pgfpathlineto{\pgfqpoint{4.358462in}{0.529317in}}%
\pgfpathlineto{\pgfqpoint{4.338271in}{0.721470in}}%
\pgfpathlineto{\pgfqpoint{4.318316in}{1.019056in}}%
\pgfpathlineto{\pgfqpoint{4.302117in}{1.294636in}}%
\pgfpathlineto{\pgfqpoint{4.281223in}{1.441015in}}%
\pgfpathlineto{\pgfqpoint{4.262675in}{1.373312in}}%
\pgfpathlineto{\pgfqpoint{4.226050in}{0.802340in}}%
\pgfpathlineto{\pgfqpoint{4.204687in}{0.591417in}}%
\pgfpathlineto{\pgfqpoint{4.186139in}{0.516521in}}%
\pgfpathlineto{\pgfqpoint{4.167357in}{0.579141in}}%
\pgfpathlineto{\pgfqpoint{4.146228in}{0.789725in}}%
\pgfpathlineto{\pgfqpoint{4.128151in}{1.094400in}}%
\pgfpathlineto{\pgfqpoint{4.110543in}{1.354000in}}%
\pgfpathlineto{\pgfqpoint{4.089883in}{1.365996in}}%
\pgfpathlineto{\pgfqpoint{4.072509in}{1.423079in}}%
\pgfpathlineto{\pgfqpoint{4.049972in}{1.244020in}}%
\pgfpathlineto{\pgfqpoint{4.032833in}{0.940287in}}%
\pgfpathlineto{\pgfqpoint{4.013581in}{0.698572in}}%
\pgfpathlineto{\pgfqpoint{3.992921in}{0.542117in}}%
\pgfpathlineto{\pgfqpoint{3.973670in}{0.516240in}}%
\pgfpathlineto{\pgfqpoint{3.955122in}{0.623197in}}%
\pgfpathlineto{\pgfqpoint{3.938454in}{0.809385in}}%
\pgfpathlineto{\pgfqpoint{3.917325in}{1.153822in}}%
\pgfpathlineto{\pgfqpoint{3.896900in}{1.382896in}}%
\pgfpathlineto{\pgfqpoint{3.879057in}{1.397696in}}%
\pgfpathlineto{\pgfqpoint{3.859101in}{1.202259in}}%
\pgfpathlineto{\pgfqpoint{3.838675in}{0.859706in}}%
\pgfpathlineto{\pgfqpoint{3.822242in}{0.675671in}}%
\pgfpathlineto{\pgfqpoint{3.801582in}{0.521590in}}%
\pgfpathlineto{\pgfqpoint{3.783270in}{0.497262in}}%
\pgfpathlineto{\pgfqpoint{3.763785in}{0.588309in}}%
\pgfpathlineto{\pgfqpoint{3.742654in}{0.792530in}}%
\pgfpathlineto{\pgfqpoint{3.725046in}{1.077109in}}%
\pgfpathlineto{\pgfqpoint{3.705091in}{1.351198in}}%
\pgfpathlineto{\pgfqpoint{3.686543in}{1.401449in}}%
\pgfpathlineto{\pgfqpoint{3.666352in}{1.288862in}}%
\pgfpathlineto{\pgfqpoint{3.648981in}{1.021342in}}%
\pgfpathlineto{\pgfqpoint{3.628084in}{0.722166in}}%
\pgfpathlineto{\pgfqpoint{3.607190in}{0.573657in}}%
\pgfpathlineto{\pgfqpoint{3.590053in}{1.316005in}}%
\pgfpathlineto{\pgfqpoint{3.551079in}{0.760520in}}%
\pgfpathlineto{\pgfqpoint{3.534411in}{0.604472in}}%
\pgfpathlineto{\pgfqpoint{3.514220in}{0.498234in}}%
\pgfpathlineto{\pgfqpoint{3.492857in}{0.526663in}}%
\pgfpathlineto{\pgfqpoint{3.475952in}{0.590444in}}%
\pgfpathlineto{\pgfqpoint{3.457641in}{0.764417in}}%
\pgfpathlineto{\pgfqpoint{3.436981in}{1.089225in}}%
\pgfpathlineto{\pgfqpoint{3.417495in}{1.343634in}}%
\pgfpathlineto{\pgfqpoint{3.398479in}{1.383228in}}%
\pgfpathlineto{\pgfqpoint{3.378991in}{1.224553in}}%
\pgfpathlineto{\pgfqpoint{3.360211in}{0.958930in}}%
\pgfpathlineto{\pgfqpoint{3.340254in}{0.729902in}}%
\pgfpathlineto{\pgfqpoint{3.317716in}{0.566792in}}%
\pgfpathlineto{\pgfqpoint{3.301752in}{0.489105in}}%
\pgfpathlineto{\pgfqpoint{3.281795in}{0.520341in}}%
\pgfpathlineto{\pgfqpoint{3.264423in}{0.616275in}}%
\pgfpathlineto{\pgfqpoint{3.246110in}{0.726595in}}%
\pgfpathlineto{\pgfqpoint{3.205496in}{1.314717in}}%
\pgfpathlineto{\pgfqpoint{3.188356in}{1.385631in}}%
\pgfpathlineto{\pgfqpoint{3.165819in}{1.290905in}}%
\pgfpathlineto{\pgfqpoint{3.148445in}{1.054410in}}%
\pgfpathlineto{\pgfqpoint{3.130837in}{0.793805in}}%
\pgfpathlineto{\pgfqpoint{3.110646in}{0.613749in}}%
\pgfpathlineto{\pgfqpoint{3.090457in}{0.498688in}}%
\pgfpathlineto{\pgfqpoint{3.070735in}{0.502792in}}%
\pgfpathlineto{\pgfqpoint{3.052893in}{0.588222in}}%
\pgfpathlineto{\pgfqpoint{3.035050in}{0.746950in}}%
\pgfpathlineto{\pgfqpoint{2.993730in}{1.274367in}}%
\pgfpathlineto{\pgfqpoint{2.976122in}{1.377790in}}%
\pgfpathlineto{\pgfqpoint{2.954523in}{1.329871in}}%
\pgfpathlineto{\pgfqpoint{2.937385in}{1.211763in}}%
\pgfpathlineto{\pgfqpoint{2.918134in}{0.932664in}}%
\pgfpathlineto{\pgfqpoint{2.900526in}{0.733322in}}%
\pgfpathlineto{\pgfqpoint{2.878692in}{0.567570in}}%
\pgfpathlineto{\pgfqpoint{2.859910in}{0.494447in}}%
\pgfpathlineto{\pgfqpoint{2.840893in}{0.513596in}}%
\pgfpathlineto{\pgfqpoint{2.820704in}{0.624651in}}%
\pgfpathlineto{\pgfqpoint{2.781496in}{1.001905in}}%
\pgfpathlineto{\pgfqpoint{2.763183in}{1.273987in}}%
\pgfpathlineto{\pgfqpoint{2.742994in}{1.378134in}}%
\pgfpathlineto{\pgfqpoint{2.726324in}{1.353349in}}%
\pgfpathlineto{\pgfqpoint{2.707543in}{1.212234in}}%
\pgfpathlineto{\pgfqpoint{2.685240in}{0.910251in}}%
\pgfpathlineto{\pgfqpoint{2.667161in}{0.706559in}}%
\pgfpathlineto{\pgfqpoint{2.651433in}{0.596797in}}%
\pgfpathlineto{\pgfqpoint{2.629833in}{0.494713in}}%
\pgfpathlineto{\pgfqpoint{2.610816in}{0.500107in}}%
\pgfpathlineto{\pgfqpoint{2.570202in}{0.664578in}}%
\pgfpathlineto{\pgfqpoint{2.551888in}{0.874222in}}%
\pgfpathlineto{\pgfqpoint{2.532872in}{1.155069in}}%
\pgfpathlineto{\pgfqpoint{2.514089in}{1.343749in}}%
\pgfpathlineto{\pgfqpoint{2.495778in}{1.372126in}}%
\pgfpathlineto{\pgfqpoint{2.473944in}{1.230092in}}%
\pgfpathlineto{\pgfqpoint{2.454927in}{1.031294in}}%
\pgfpathlineto{\pgfqpoint{2.436381in}{0.799299in}}%
\pgfpathlineto{\pgfqpoint{2.399522in}{0.568572in}}%
\pgfpathlineto{\pgfqpoint{2.374870in}{1.236558in}}%
\pgfpathlineto{\pgfqpoint{2.358436in}{0.988231in}}%
\pgfpathlineto{\pgfqpoint{2.340594in}{0.736489in}}%
\pgfpathlineto{\pgfqpoint{2.319229in}{0.571728in}}%
\pgfpathlineto{\pgfqpoint{2.303264in}{0.496112in}}%
\pgfpathlineto{\pgfqpoint{2.284718in}{0.498293in}}%
\pgfpathlineto{\pgfqpoint{2.263824in}{0.586194in}}%
\pgfpathlineto{\pgfqpoint{2.244102in}{0.750633in}}%
\pgfpathlineto{\pgfqpoint{2.203251in}{1.322979in}}%
\pgfpathlineto{\pgfqpoint{2.188226in}{1.382676in}}%
\pgfpathlineto{\pgfqpoint{2.169680in}{1.325829in}}%
\pgfpathlineto{\pgfqpoint{2.148314in}{1.082581in}}%
\pgfpathlineto{\pgfqpoint{2.132350in}{0.883361in}}%
\pgfpathlineto{\pgfqpoint{2.110046in}{0.669930in}}%
\pgfpathlineto{\pgfqpoint{2.087978in}{0.532597in}}%
\pgfpathlineto{\pgfqpoint{2.070370in}{0.492242in}}%
\pgfpathlineto{\pgfqpoint{2.051589in}{0.524237in}}%
\pgfpathlineto{\pgfqpoint{2.033042in}{0.635762in}}%
\pgfpathlineto{\pgfqpoint{2.014494in}{0.727016in}}%
\pgfpathlineto{\pgfqpoint{1.976931in}{1.262387in}}%
\pgfpathlineto{\pgfqpoint{1.955566in}{1.386512in}}%
\pgfpathlineto{\pgfqpoint{1.938898in}{1.367742in}}%
\pgfpathlineto{\pgfqpoint{1.917063in}{1.158525in}}%
\pgfpathlineto{\pgfqpoint{1.899221in}{0.927870in}}%
\pgfpathlineto{\pgfqpoint{1.878561in}{0.700018in}}%
\pgfpathlineto{\pgfqpoint{1.860015in}{0.568894in}}%
\pgfpathlineto{\pgfqpoint{1.841233in}{0.494748in}}%
\pgfpathlineto{\pgfqpoint{1.822216in}{0.517862in}}%
\pgfpathlineto{\pgfqpoint{1.803434in}{0.602778in}}%
\pgfpathlineto{\pgfqpoint{1.784417in}{0.734998in}}%
\pgfpathlineto{\pgfqpoint{1.763288in}{0.935470in}}%
\pgfpathlineto{\pgfqpoint{1.745211in}{1.200177in}}%
\pgfpathlineto{\pgfqpoint{1.726429in}{1.356762in}}%
\pgfpathlineto{\pgfqpoint{1.707881in}{1.393628in}}%
\pgfpathlineto{\pgfqpoint{1.689101in}{1.353597in}}%
\pgfpathlineto{\pgfqpoint{1.667267in}{1.145331in}}%
\pgfpathlineto{\pgfqpoint{1.648719in}{0.926405in}}%
\pgfpathlineto{\pgfqpoint{1.630642in}{0.738030in}}%
\pgfpathlineto{\pgfqpoint{1.609748in}{0.596479in}}%
\pgfpathlineto{\pgfqpoint{1.592139in}{0.514388in}}%
\pgfpathlineto{\pgfqpoint{1.571948in}{0.511569in}}%
\pgfpathlineto{\pgfqpoint{1.553401in}{0.581411in}}%
\pgfpathlineto{\pgfqpoint{1.534620in}{0.685141in}}%
\pgfpathlineto{\pgfqpoint{1.514664in}{0.809731in}}%
\pgfpathlineto{\pgfqpoint{1.494473in}{1.013355in}}%
\pgfpathlineto{\pgfqpoint{1.475927in}{1.276994in}}%
\pgfpathlineto{\pgfqpoint{1.457379in}{1.390296in}}%
\pgfpathlineto{\pgfqpoint{1.436485in}{1.400602in}}%
\pgfpathlineto{\pgfqpoint{1.415121in}{1.279685in}}%
\pgfpathlineto{\pgfqpoint{1.398686in}{1.073930in}}%
\pgfpathlineto{\pgfqpoint{1.378497in}{0.836464in}}%
\pgfpathlineto{\pgfqpoint{1.360183in}{0.756606in}}%
\pgfpathlineto{\pgfqpoint{1.342812in}{0.658633in}}%
\pgfpathlineto{\pgfqpoint{1.321212in}{0.549602in}}%
\pgfpathlineto{\pgfqpoint{1.302899in}{0.503197in}}%
\pgfpathlineto{\pgfqpoint{1.284353in}{0.524600in}}%
\pgfpathlineto{\pgfqpoint{1.263458in}{0.625308in}}%
\pgfpathlineto{\pgfqpoint{1.245145in}{0.759907in}}%
\pgfpathlineto{\pgfqpoint{1.227302in}{0.904694in}}%
\pgfpathlineto{\pgfqpoint{1.208757in}{1.158960in}}%
\pgfpathlineto{\pgfqpoint{1.186451in}{1.367834in}}%
\pgfpathlineto{\pgfqpoint{1.167671in}{1.419039in}}%
\pgfpathlineto{\pgfqpoint{1.148184in}{1.411320in}}%
\pgfpathlineto{\pgfqpoint{1.128698in}{1.315024in}}%
\pgfpathlineto{\pgfqpoint{1.109916in}{1.416780in}}%
\pgfpathlineto{\pgfqpoint{1.088318in}{1.284505in}}%
\pgfpathlineto{\pgfqpoint{1.054042in}{0.882428in}}%
\pgfpathlineto{\pgfqpoint{1.032207in}{0.688972in}}%
\pgfpathlineto{\pgfqpoint{1.013894in}{0.615129in}}%
\pgfpathlineto{\pgfqpoint{0.995348in}{0.528751in}}%
\pgfpathlineto{\pgfqpoint{0.976800in}{0.526563in}}%
\pgfpathlineto{\pgfqpoint{0.954497in}{0.621800in}}%
\pgfpathlineto{\pgfqpoint{0.935715in}{0.772592in}}%
\pgfpathlineto{\pgfqpoint{0.917169in}{0.985316in}}%
\pgfpathlineto{\pgfqpoint{0.899090in}{1.211008in}}%
\pgfpathlineto{\pgfqpoint{0.881013in}{1.378343in}}%
\pgfpathlineto{\pgfqpoint{0.861527in}{1.437482in}}%
\pgfpathlineto{\pgfqpoint{0.839928in}{1.410806in}}%
\pgfpathlineto{\pgfqpoint{0.822085in}{1.327897in}}%
\pgfpathlineto{\pgfqpoint{0.803537in}{1.091951in}}%
\pgfpathlineto{\pgfqpoint{0.785695in}{0.900825in}}%
\pgfpathlineto{\pgfqpoint{0.763626in}{0.723205in}}%
\pgfpathlineto{\pgfqpoint{0.745080in}{0.623240in}}%
\pgfpathlineto{\pgfqpoint{0.726767in}{0.582461in}}%
\pgfpathlineto{\pgfqpoint{0.709395in}{0.533517in}}%
\pgfpathlineto{\pgfqpoint{0.688030in}{0.546232in}}%
\pgfpathlineto{\pgfqpoint{0.664553in}{0.677066in}}%
\pgfpathlineto{\pgfqpoint{0.649059in}{0.819301in}}%
\pgfpathlineto{\pgfqpoint{0.653754in}{0.740687in}}%
\pgfpathlineto{\pgfqpoint{0.676292in}{0.560222in}}%
\pgfpathlineto{\pgfqpoint{0.694368in}{0.543136in}}%
\pgfpathlineto{\pgfqpoint{0.712213in}{0.660911in}}%
\pgfpathlineto{\pgfqpoint{0.733342in}{0.895922in}}%
\pgfpathlineto{\pgfqpoint{0.751888in}{1.198284in}}%
\pgfpathlineto{\pgfqpoint{0.772548in}{1.427036in}}%
\pgfpathlineto{\pgfqpoint{0.791330in}{1.419582in}}%
\pgfpathlineto{\pgfqpoint{0.809172in}{1.231821in}}%
\pgfpathlineto{\pgfqpoint{0.830069in}{0.871814in}}%
\pgfpathlineto{\pgfqpoint{0.830303in}{0.726585in}}%
\pgfpathlineto{\pgfqpoint{0.846502in}{0.668329in}}%
\pgfpathlineto{\pgfqpoint{0.866457in}{0.531062in}}%
\pgfpathlineto{\pgfqpoint{0.887822in}{0.565337in}}%
\pgfpathlineto{\pgfqpoint{0.906134in}{0.710802in}}%
\pgfpathlineto{\pgfqpoint{0.923976in}{0.929802in}}%
\pgfpathlineto{\pgfqpoint{0.944636in}{1.287160in}}%
\pgfpathlineto{\pgfqpoint{0.962949in}{1.422448in}}%
\pgfpathlineto{\pgfqpoint{0.984313in}{1.348420in}}%
\pgfpathlineto{\pgfqpoint{1.004504in}{1.004104in}}%
\pgfpathlineto{\pgfqpoint{1.019998in}{0.761994in}}%
\pgfpathlineto{\pgfqpoint{1.043006in}{0.554993in}}%
\pgfpathlineto{\pgfqpoint{1.059440in}{0.507977in}}%
\pgfpathlineto{\pgfqpoint{1.079397in}{0.606935in}}%
\pgfpathlineto{\pgfqpoint{1.095361in}{0.752829in}}%
\pgfpathlineto{\pgfqpoint{1.117430in}{1.071898in}}%
\pgfpathlineto{\pgfqpoint{1.136916in}{1.313548in}}%
\pgfpathlineto{\pgfqpoint{1.155698in}{1.413003in}}%
\pgfpathlineto{\pgfqpoint{1.172366in}{1.338529in}}%
\pgfpathlineto{\pgfqpoint{1.194669in}{0.986750in}}%
\pgfpathlineto{\pgfqpoint{1.214389in}{0.709168in}}%
\pgfpathlineto{\pgfqpoint{1.233641in}{0.557932in}}%
\pgfpathlineto{\pgfqpoint{1.253832in}{0.494558in}}%
\pgfpathlineto{\pgfqpoint{1.270031in}{0.554552in}}%
\pgfpathlineto{\pgfqpoint{1.290691in}{0.694541in}}%
\pgfpathlineto{\pgfqpoint{1.311585in}{0.940967in}}%
\pgfpathlineto{\pgfqpoint{1.328724in}{1.259907in}}%
\pgfpathlineto{\pgfqpoint{1.347272in}{1.397701in}}%
\pgfpathlineto{\pgfqpoint{1.368167in}{1.330744in}}%
\pgfpathlineto{\pgfqpoint{1.385538in}{1.366601in}}%
\pgfpathlineto{\pgfqpoint{1.405260in}{1.279011in}}%
\pgfpathlineto{\pgfqpoint{1.424043in}{1.039383in}}%
\pgfpathlineto{\pgfqpoint{1.443059in}{0.738232in}}%
\pgfpathlineto{\pgfqpoint{1.461840in}{0.567652in}}%
\pgfpathlineto{\pgfqpoint{1.482031in}{0.490294in}}%
\pgfpathlineto{\pgfqpoint{1.501987in}{0.551134in}}%
\pgfpathlineto{\pgfqpoint{1.522647in}{0.701305in}}%
\pgfpathlineto{\pgfqpoint{1.541898in}{0.946330in}}%
\pgfpathlineto{\pgfqpoint{1.564436in}{1.261271in}}%
\pgfpathlineto{\pgfqpoint{1.579695in}{1.382344in}}%
\pgfpathlineto{\pgfqpoint{1.598712in}{1.353357in}}%
\pgfpathlineto{\pgfqpoint{1.620546in}{1.063196in}}%
\pgfpathlineto{\pgfqpoint{1.635806in}{0.813286in}}%
\pgfpathlineto{\pgfqpoint{1.658109in}{0.587776in}}%
\pgfpathlineto{\pgfqpoint{1.674308in}{0.505262in}}%
\pgfpathlineto{\pgfqpoint{1.693091in}{0.485940in}}%
\pgfpathlineto{\pgfqpoint{1.711873in}{0.559550in}}%
\pgfpathlineto{\pgfqpoint{1.730655in}{0.647915in}}%
\pgfpathlineto{\pgfqpoint{1.749436in}{0.867861in}}%
\pgfpathlineto{\pgfqpoint{1.772210in}{1.159173in}}%
\pgfpathlineto{\pgfqpoint{1.790755in}{1.362490in}}%
\pgfpathlineto{\pgfqpoint{1.810243in}{1.371423in}}%
\pgfpathlineto{\pgfqpoint{1.828320in}{1.222533in}}%
\pgfpathlineto{\pgfqpoint{1.848040in}{0.894585in}}%
\pgfpathlineto{\pgfqpoint{1.867291in}{0.659647in}}%
\pgfpathlineto{\pgfqpoint{1.885136in}{0.522541in}}%
\pgfpathlineto{\pgfqpoint{1.905090in}{0.485458in}}%
\pgfpathlineto{\pgfqpoint{1.923873in}{0.531576in}}%
\pgfpathlineto{\pgfqpoint{1.944298in}{0.639225in}}%
\pgfpathlineto{\pgfqpoint{1.963315in}{0.832307in}}%
\pgfpathlineto{\pgfqpoint{1.981861in}{1.113858in}}%
\pgfpathlineto{\pgfqpoint{2.001112in}{1.335158in}}%
\pgfpathlineto{\pgfqpoint{2.019660in}{1.379521in}}%
\pgfpathlineto{\pgfqpoint{2.041728in}{1.243832in}}%
\pgfpathlineto{\pgfqpoint{2.078119in}{0.707021in}}%
\pgfpathlineto{\pgfqpoint{2.097839in}{0.569059in}}%
\pgfpathlineto{\pgfqpoint{2.116387in}{0.490843in}}%
\pgfpathlineto{\pgfqpoint{2.136107in}{0.496181in}}%
\pgfpathlineto{\pgfqpoint{2.157001in}{0.595926in}}%
\pgfpathlineto{\pgfqpoint{2.175549in}{0.714851in}}%
\pgfpathlineto{\pgfqpoint{2.193860in}{0.927153in}}%
\pgfpathlineto{\pgfqpoint{2.212408in}{1.208555in}}%
\pgfpathlineto{\pgfqpoint{2.230954in}{1.359049in}}%
\pgfpathlineto{\pgfqpoint{2.251614in}{1.363113in}}%
\pgfpathlineto{\pgfqpoint{2.271571in}{1.177983in}}%
\pgfpathlineto{\pgfqpoint{2.290116in}{0.881565in}}%
\pgfpathlineto{\pgfqpoint{2.311013in}{0.660246in}}%
\pgfpathlineto{\pgfqpoint{2.329793in}{0.552341in}}%
\pgfpathlineto{\pgfqpoint{2.346227in}{0.488070in}}%
\pgfpathlineto{\pgfqpoint{2.367827in}{0.664211in}}%
\pgfpathlineto{\pgfqpoint{2.385435in}{0.545772in}}%
\pgfpathlineto{\pgfqpoint{2.385200in}{0.499143in}}%
\pgfpathlineto{\pgfqpoint{2.407034in}{0.481991in}}%
\pgfpathlineto{\pgfqpoint{2.425111in}{0.530060in}}%
\pgfpathlineto{\pgfqpoint{2.442016in}{0.636450in}}%
\pgfpathlineto{\pgfqpoint{2.460327in}{0.789171in}}%
\pgfpathlineto{\pgfqpoint{2.502116in}{1.323151in}}%
\pgfpathlineto{\pgfqpoint{2.520195in}{1.380530in}}%
\pgfpathlineto{\pgfqpoint{2.538507in}{1.296227in}}%
\pgfpathlineto{\pgfqpoint{2.558463in}{0.997528in}}%
\pgfpathlineto{\pgfqpoint{2.577009in}{0.749269in}}%
\pgfpathlineto{\pgfqpoint{2.598374in}{0.581708in}}%
\pgfpathlineto{\pgfqpoint{2.615982in}{0.502260in}}%
\pgfpathlineto{\pgfqpoint{2.633825in}{0.488992in}}%
\pgfpathlineto{\pgfqpoint{2.654719in}{0.570455in}}%
\pgfpathlineto{\pgfqpoint{2.672093in}{0.685940in}}%
\pgfpathlineto{\pgfqpoint{2.692282in}{0.801687in}}%
\pgfpathlineto{\pgfqpoint{2.711298in}{1.012530in}}%
\pgfpathlineto{\pgfqpoint{2.732193in}{1.264030in}}%
\pgfpathlineto{\pgfqpoint{2.750037in}{1.371442in}}%
\pgfpathlineto{\pgfqpoint{2.771166in}{1.339874in}}%
\pgfpathlineto{\pgfqpoint{2.788305in}{1.142746in}}%
\pgfpathlineto{\pgfqpoint{2.805911in}{0.860725in}}%
\pgfpathlineto{\pgfqpoint{2.849580in}{0.561923in}}%
\pgfpathlineto{\pgfqpoint{2.866484in}{0.491080in}}%
\pgfpathlineto{\pgfqpoint{2.884796in}{0.506733in}}%
\pgfpathlineto{\pgfqpoint{2.905925in}{0.598524in}}%
\pgfpathlineto{\pgfqpoint{2.923533in}{0.669305in}}%
\pgfpathlineto{\pgfqpoint{2.941612in}{0.860555in}}%
\pgfpathlineto{\pgfqpoint{2.961566in}{1.173332in}}%
\pgfpathlineto{\pgfqpoint{2.980114in}{1.349108in}}%
\pgfpathlineto{\pgfqpoint{3.000774in}{1.385771in}}%
\pgfpathlineto{\pgfqpoint{3.019085in}{1.320232in}}%
\pgfpathlineto{\pgfqpoint{3.055945in}{0.797656in}}%
\pgfpathlineto{\pgfqpoint{3.076604in}{0.664950in}}%
\pgfpathlineto{\pgfqpoint{3.098908in}{0.539330in}}%
\pgfpathlineto{\pgfqpoint{3.115341in}{0.488810in}}%
\pgfpathlineto{\pgfqpoint{3.136707in}{0.537990in}}%
\pgfpathlineto{\pgfqpoint{3.153846in}{0.624192in}}%
\pgfpathlineto{\pgfqpoint{3.173097in}{0.755502in}}%
\pgfpathlineto{\pgfqpoint{3.193052in}{0.954623in}}%
\pgfpathlineto{\pgfqpoint{3.211599in}{1.184364in}}%
\pgfpathlineto{\pgfqpoint{3.231319in}{1.371292in}}%
\pgfpathlineto{\pgfqpoint{3.250336in}{1.394155in}}%
\pgfpathlineto{\pgfqpoint{3.267710in}{1.326573in}}%
\pgfpathlineto{\pgfqpoint{3.288370in}{1.090313in}}%
\pgfpathlineto{\pgfqpoint{3.309030in}{0.798761in}}%
\pgfpathlineto{\pgfqpoint{3.327577in}{0.673420in}}%
\pgfpathlineto{\pgfqpoint{3.345889in}{0.563009in}}%
\pgfpathlineto{\pgfqpoint{3.363731in}{0.502619in}}%
\pgfpathlineto{\pgfqpoint{3.385097in}{0.513236in}}%
\pgfpathlineto{\pgfqpoint{3.408572in}{0.590026in}}%
\pgfpathlineto{\pgfqpoint{3.422659in}{0.674913in}}%
\pgfpathlineto{\pgfqpoint{3.442850in}{0.821763in}}%
\pgfpathlineto{\pgfqpoint{3.462336in}{1.020555in}}%
\pgfpathlineto{\pgfqpoint{3.481353in}{1.258756in}}%
\pgfpathlineto{\pgfqpoint{3.499430in}{1.386015in}}%
\pgfpathlineto{\pgfqpoint{3.519386in}{1.396280in}}%
\pgfpathlineto{\pgfqpoint{3.539341in}{1.333017in}}%
\pgfpathlineto{\pgfqpoint{3.555306in}{1.162221in}}%
\pgfpathlineto{\pgfqpoint{3.579252in}{0.888378in}}%
\pgfpathlineto{\pgfqpoint{3.597800in}{0.744205in}}%
\pgfpathlineto{\pgfqpoint{3.616346in}{0.636711in}}%
\pgfpathlineto{\pgfqpoint{3.633954in}{0.566859in}}%
\pgfpathlineto{\pgfqpoint{3.656022in}{0.504558in}}%
\pgfpathlineto{\pgfqpoint{3.673161in}{0.554304in}}%
\pgfpathlineto{\pgfqpoint{3.690535in}{0.642445in}}%
\pgfpathlineto{\pgfqpoint{3.712369in}{0.800629in}}%
\pgfpathlineto{\pgfqpoint{3.729741in}{0.949449in}}%
\pgfpathlineto{\pgfqpoint{3.754392in}{1.245752in}}%
\pgfpathlineto{\pgfqpoint{3.771063in}{1.374395in}}%
\pgfpathlineto{\pgfqpoint{3.790077in}{1.394819in}}%
\pgfpathlineto{\pgfqpoint{3.807216in}{1.417455in}}%
\pgfpathlineto{\pgfqpoint{3.826233in}{1.341841in}}%
\pgfpathlineto{\pgfqpoint{3.864501in}{0.894977in}}%
\pgfpathlineto{\pgfqpoint{3.883987in}{0.796517in}}%
\pgfpathlineto{\pgfqpoint{3.901126in}{0.635406in}}%
\pgfpathlineto{\pgfqpoint{3.926247in}{0.705699in}}%
\pgfpathlineto{\pgfqpoint{3.942446in}{0.579439in}}%
\pgfpathlineto{\pgfqpoint{3.961228in}{0.512277in}}%
\pgfpathlineto{\pgfqpoint{3.983060in}{0.518014in}}%
\pgfpathlineto{\pgfqpoint{3.999496in}{0.600029in}}%
\pgfpathlineto{\pgfqpoint{4.018511in}{0.721554in}}%
\pgfpathlineto{\pgfqpoint{4.036590in}{0.864503in}}%
\pgfpathlineto{\pgfqpoint{4.058424in}{1.141904in}}%
\pgfpathlineto{\pgfqpoint{4.076501in}{1.284755in}}%
\pgfpathlineto{\pgfqpoint{4.096456in}{1.419720in}}%
\pgfpathlineto{\pgfqpoint{4.114298in}{1.433232in}}%
\pgfpathlineto{\pgfqpoint{4.135429in}{1.328352in}}%
\pgfpathlineto{\pgfqpoint{4.154446in}{1.113324in}}%
\pgfpathlineto{\pgfqpoint{4.192713in}{0.761688in}}%
\pgfpathlineto{\pgfqpoint{4.209851in}{0.618801in}}%
\pgfpathlineto{\pgfqpoint{4.231685in}{0.539515in}}%
\pgfpathlineto{\pgfqpoint{4.250467in}{0.527203in}}%
\pgfpathlineto{\pgfqpoint{4.267841in}{0.577005in}}%
\pgfpathlineto{\pgfqpoint{4.284978in}{0.677198in}}%
\pgfpathlineto{\pgfqpoint{4.306109in}{0.836136in}}%
\pgfpathlineto{\pgfqpoint{4.324889in}{1.014246in}}%
\pgfpathlineto{\pgfqpoint{4.346254in}{1.283364in}}%
\pgfpathlineto{\pgfqpoint{4.364097in}{1.387295in}}%
\pgfpathlineto{\pgfqpoint{4.384757in}{1.456069in}}%
\pgfpathlineto{\pgfqpoint{4.402599in}{1.426140in}}%
\pgfpathlineto{\pgfqpoint{4.421381in}{1.309164in}}%
\pgfpathlineto{\pgfqpoint{4.441572in}{1.123927in}}%
\pgfpathlineto{\pgfqpoint{4.460824in}{0.911677in}}%
\pgfpathlineto{\pgfqpoint{4.481953in}{0.710852in}}%
\pgfpathlineto{\pgfqpoint{4.475143in}{0.776270in}}%
\pgfpathlineto{\pgfqpoint{4.453780in}{1.144712in}}%
\pgfpathlineto{\pgfqpoint{4.434998in}{1.389163in}}%
\pgfpathlineto{\pgfqpoint{4.416452in}{1.457723in}}%
\pgfpathlineto{\pgfqpoint{4.397199in}{1.349966in}}%
\pgfpathlineto{\pgfqpoint{4.377010in}{1.008676in}}%
\pgfpathlineto{\pgfqpoint{4.357524in}{0.744670in}}%
\pgfpathlineto{\pgfqpoint{4.338271in}{0.585302in}}%
\pgfpathlineto{\pgfqpoint{4.317142in}{0.531477in}}%
\pgfpathlineto{\pgfqpoint{4.302117in}{0.631711in}}%
\pgfpathlineto{\pgfqpoint{4.280988in}{0.874649in}}%
\pgfpathlineto{\pgfqpoint{4.262675in}{1.203448in}}%
\pgfpathlineto{\pgfqpoint{4.243424in}{1.410110in}}%
\pgfpathlineto{\pgfqpoint{4.224407in}{1.425814in}}%
\pgfpathlineto{\pgfqpoint{4.206799in}{1.255729in}}%
\pgfpathlineto{\pgfqpoint{4.185435in}{0.905806in}}%
\pgfpathlineto{\pgfqpoint{4.166653in}{0.664308in}}%
\pgfpathlineto{\pgfqpoint{4.149514in}{0.550003in}}%
\pgfpathlineto{\pgfqpoint{4.128151in}{0.544158in}}%
\pgfpathlineto{\pgfqpoint{4.110777in}{0.665197in}}%
\pgfpathlineto{\pgfqpoint{4.088709in}{0.906863in}}%
\pgfpathlineto{\pgfqpoint{4.071806in}{1.224859in}}%
\pgfpathlineto{\pgfqpoint{4.070866in}{1.368935in}}%
\pgfpathlineto{\pgfqpoint{4.050206in}{1.413912in}}%
\pgfpathlineto{\pgfqpoint{4.033067in}{1.396352in}}%
\pgfpathlineto{\pgfqpoint{4.010998in}{1.111587in}}%
\pgfpathlineto{\pgfqpoint{3.994799in}{0.849169in}}%
\pgfpathlineto{\pgfqpoint{3.974374in}{0.629801in}}%
\pgfpathlineto{\pgfqpoint{3.954185in}{0.511850in}}%
\pgfpathlineto{\pgfqpoint{3.936576in}{0.523709in}}%
\pgfpathlineto{\pgfqpoint{3.912394in}{0.696605in}}%
\pgfpathlineto{\pgfqpoint{3.899246in}{0.890096in}}%
\pgfpathlineto{\pgfqpoint{3.878823in}{1.232179in}}%
\pgfpathlineto{\pgfqpoint{3.857927in}{1.405040in}}%
\pgfpathlineto{\pgfqpoint{3.840555in}{1.410345in}}%
\pgfpathlineto{\pgfqpoint{3.816607in}{1.244918in}}%
\pgfpathlineto{\pgfqpoint{3.802287in}{0.992558in}}%
\pgfpathlineto{\pgfqpoint{3.783036in}{0.730224in}}%
\pgfpathlineto{\pgfqpoint{3.765662in}{0.566689in}}%
\pgfpathlineto{\pgfqpoint{3.744768in}{0.496334in}}%
\pgfpathlineto{\pgfqpoint{3.725517in}{0.571855in}}%
\pgfpathlineto{\pgfqpoint{3.704151in}{0.761255in}}%
\pgfpathlineto{\pgfqpoint{3.687249in}{0.974842in}}%
\pgfpathlineto{\pgfqpoint{3.667527in}{1.249554in}}%
\pgfpathlineto{\pgfqpoint{3.647806in}{1.394546in}}%
\pgfpathlineto{\pgfqpoint{3.630667in}{1.359873in}}%
\pgfpathlineto{\pgfqpoint{3.609069in}{1.101668in}}%
\pgfpathlineto{\pgfqpoint{3.589347in}{0.783352in}}%
\pgfpathlineto{\pgfqpoint{3.568688in}{0.600755in}}%
\pgfpathlineto{\pgfqpoint{3.550376in}{0.506890in}}%
\pgfpathlineto{\pgfqpoint{3.532768in}{0.500819in}}%
\pgfpathlineto{\pgfqpoint{3.512108in}{0.609014in}}%
\pgfpathlineto{\pgfqpoint{3.495438in}{0.779514in}}%
\pgfpathlineto{\pgfqpoint{3.475249in}{1.108543in}}%
\pgfpathlineto{\pgfqpoint{3.457170in}{1.337536in}}%
\pgfpathlineto{\pgfqpoint{3.436510in}{1.392004in}}%
\pgfpathlineto{\pgfqpoint{3.418902in}{1.309922in}}%
\pgfpathlineto{\pgfqpoint{3.380636in}{0.793570in}}%
\pgfpathlineto{\pgfqpoint{3.359036in}{0.621130in}}%
\pgfpathlineto{\pgfqpoint{3.338142in}{0.518752in}}%
\pgfpathlineto{\pgfqpoint{3.321472in}{0.494395in}}%
\pgfpathlineto{\pgfqpoint{3.300812in}{0.587786in}}%
\pgfpathlineto{\pgfqpoint{3.282266in}{0.711500in}}%
\pgfpathlineto{\pgfqpoint{3.263953in}{0.953723in}}%
\pgfpathlineto{\pgfqpoint{3.243527in}{1.284247in}}%
\pgfpathlineto{\pgfqpoint{3.224276in}{1.385299in}}%
\pgfpathlineto{\pgfqpoint{3.205025in}{1.338776in}}%
\pgfpathlineto{\pgfqpoint{3.188356in}{1.219200in}}%
\pgfpathlineto{\pgfqpoint{3.166522in}{0.914894in}}%
\pgfpathlineto{\pgfqpoint{3.148914in}{0.966095in}}%
\pgfpathlineto{\pgfqpoint{3.128020in}{0.690705in}}%
\pgfpathlineto{\pgfqpoint{3.110177in}{0.604121in}}%
\pgfpathlineto{\pgfqpoint{3.110177in}{0.604121in}}%
\pgfusepath{stroke}%
\end{pgfscope}%
\begin{pgfscope}%
\pgfpathrectangle{\pgfqpoint{0.444748in}{0.431673in}}{\pgfqpoint{4.231419in}{1.076123in}}%
\pgfusepath{clip}%
\pgfsetbuttcap%
\pgfsetroundjoin%
\definecolor{currentfill}{rgb}{0.047059,0.364706,0.647059}%
\pgfsetfillcolor{currentfill}%
\pgfsetlinewidth{1.003750pt}%
\definecolor{currentstroke}{rgb}{0.047059,0.364706,0.647059}%
\pgfsetstrokecolor{currentstroke}%
\pgfsetdash{}{0pt}%
\pgfsys@defobject{currentmarker}{\pgfqpoint{-0.010417in}{-0.010417in}}{\pgfqpoint{0.010417in}{0.010417in}}{%
\pgfpathmoveto{\pgfqpoint{0.000000in}{-0.010417in}}%
\pgfpathcurveto{\pgfqpoint{0.002763in}{-0.010417in}}{\pgfqpoint{0.005412in}{-0.009319in}}{\pgfqpoint{0.007366in}{-0.007366in}}%
\pgfpathcurveto{\pgfqpoint{0.009319in}{-0.005412in}}{\pgfqpoint{0.010417in}{-0.002763in}}{\pgfqpoint{0.010417in}{0.000000in}}%
\pgfpathcurveto{\pgfqpoint{0.010417in}{0.002763in}}{\pgfqpoint{0.009319in}{0.005412in}}{\pgfqpoint{0.007366in}{0.007366in}}%
\pgfpathcurveto{\pgfqpoint{0.005412in}{0.009319in}}{\pgfqpoint{0.002763in}{0.010417in}}{\pgfqpoint{0.000000in}{0.010417in}}%
\pgfpathcurveto{\pgfqpoint{-0.002763in}{0.010417in}}{\pgfqpoint{-0.005412in}{0.009319in}}{\pgfqpoint{-0.007366in}{0.007366in}}%
\pgfpathcurveto{\pgfqpoint{-0.009319in}{0.005412in}}{\pgfqpoint{-0.010417in}{0.002763in}}{\pgfqpoint{-0.010417in}{0.000000in}}%
\pgfpathcurveto{\pgfqpoint{-0.010417in}{-0.002763in}}{\pgfqpoint{-0.009319in}{-0.005412in}}{\pgfqpoint{-0.007366in}{-0.007366in}}%
\pgfpathcurveto{\pgfqpoint{-0.005412in}{-0.009319in}}{\pgfqpoint{-0.002763in}{-0.010417in}}{\pgfqpoint{0.000000in}{-0.010417in}}%
\pgfpathlineto{\pgfqpoint{0.000000in}{-0.010417in}}%
\pgfpathclose%
\pgfusepath{stroke,fill}%
}%
\begin{pgfscope}%
\pgfsys@transformshift{0.651405in}{0.827459in}%
\pgfsys@useobject{currentmarker}{}%
\end{pgfscope}%
\begin{pgfscope}%
\pgfsys@transformshift{0.656571in}{0.716994in}%
\pgfsys@useobject{currentmarker}{}%
\end{pgfscope}%
\begin{pgfscope}%
\pgfsys@transformshift{0.677231in}{0.549367in}%
\pgfsys@useobject{currentmarker}{}%
\end{pgfscope}%
\begin{pgfscope}%
\pgfsys@transformshift{0.693900in}{0.547809in}%
\pgfsys@useobject{currentmarker}{}%
\end{pgfscope}%
\begin{pgfscope}%
\pgfsys@transformshift{0.711977in}{0.687463in}%
\pgfsys@useobject{currentmarker}{}%
\end{pgfscope}%
\begin{pgfscope}%
\pgfsys@transformshift{0.730993in}{0.887701in}%
\pgfsys@useobject{currentmarker}{}%
\end{pgfscope}%
\begin{pgfscope}%
\pgfsys@transformshift{0.753296in}{1.258731in}%
\pgfsys@useobject{currentmarker}{}%
\end{pgfscope}%
\begin{pgfscope}%
\pgfsys@transformshift{0.772548in}{1.433710in}%
\pgfsys@useobject{currentmarker}{}%
\end{pgfscope}%
\begin{pgfscope}%
\pgfsys@transformshift{0.790861in}{1.411637in}%
\pgfsys@useobject{currentmarker}{}%
\end{pgfscope}%
\begin{pgfscope}%
\pgfsys@transformshift{0.809409in}{1.198081in}%
\pgfsys@useobject{currentmarker}{}%
\end{pgfscope}%
\begin{pgfscope}%
\pgfsys@transformshift{0.829363in}{0.857472in}%
\pgfsys@useobject{currentmarker}{}%
\end{pgfscope}%
\begin{pgfscope}%
\pgfsys@transformshift{0.847440in}{0.652079in}%
\pgfsys@useobject{currentmarker}{}%
\end{pgfscope}%
\begin{pgfscope}%
\pgfsys@transformshift{0.866223in}{0.524337in}%
\pgfsys@useobject{currentmarker}{}%
\end{pgfscope}%
\begin{pgfscope}%
\pgfsys@transformshift{0.885005in}{0.565578in}%
\pgfsys@useobject{currentmarker}{}%
\end{pgfscope}%
\begin{pgfscope}%
\pgfsys@transformshift{0.906368in}{0.724102in}%
\pgfsys@useobject{currentmarker}{}%
\end{pgfscope}%
\begin{pgfscope}%
\pgfsys@transformshift{0.923976in}{0.959619in}%
\pgfsys@useobject{currentmarker}{}%
\end{pgfscope}%
\begin{pgfscope}%
\pgfsys@transformshift{0.943227in}{1.269522in}%
\pgfsys@useobject{currentmarker}{}%
\end{pgfscope}%
\begin{pgfscope}%
\pgfsys@transformshift{0.964122in}{1.424922in}%
\pgfsys@useobject{currentmarker}{}%
\end{pgfscope}%
\begin{pgfscope}%
\pgfsys@transformshift{0.982435in}{1.347542in}%
\pgfsys@useobject{currentmarker}{}%
\end{pgfscope}%
\begin{pgfscope}%
\pgfsys@transformshift{1.003798in}{1.016900in}%
\pgfsys@useobject{currentmarker}{}%
\end{pgfscope}%
\begin{pgfscope}%
\pgfsys@transformshift{1.019294in}{0.774601in}%
\pgfsys@useobject{currentmarker}{}%
\end{pgfscope}%
\begin{pgfscope}%
\pgfsys@transformshift{1.042537in}{0.566204in}%
\pgfsys@useobject{currentmarker}{}%
\end{pgfscope}%
\begin{pgfscope}%
\pgfsys@transformshift{1.060145in}{0.510209in}%
\pgfsys@useobject{currentmarker}{}%
\end{pgfscope}%
\begin{pgfscope}%
\pgfsys@transformshift{1.079162in}{0.603305in}%
\pgfsys@useobject{currentmarker}{}%
\end{pgfscope}%
\begin{pgfscope}%
\pgfsys@transformshift{1.101934in}{0.831438in}%
\pgfsys@useobject{currentmarker}{}%
\end{pgfscope}%
\begin{pgfscope}%
\pgfsys@transformshift{1.115082in}{1.017417in}%
\pgfsys@useobject{currentmarker}{}%
\end{pgfscope}%
\begin{pgfscope}%
\pgfsys@transformshift{1.132690in}{1.364888in}%
\pgfsys@useobject{currentmarker}{}%
\end{pgfscope}%
\begin{pgfscope}%
\pgfsys@transformshift{1.154993in}{1.412231in}%
\pgfsys@useobject{currentmarker}{}%
\end{pgfscope}%
\begin{pgfscope}%
\pgfsys@transformshift{1.175653in}{1.264651in}%
\pgfsys@useobject{currentmarker}{}%
\end{pgfscope}%
\begin{pgfscope}%
\pgfsys@transformshift{1.194669in}{0.970458in}%
\pgfsys@useobject{currentmarker}{}%
\end{pgfscope}%
\begin{pgfscope}%
\pgfsys@transformshift{1.213452in}{0.701576in}%
\pgfsys@useobject{currentmarker}{}%
\end{pgfscope}%
\begin{pgfscope}%
\pgfsys@transformshift{1.230120in}{0.555614in}%
\pgfsys@useobject{currentmarker}{}%
\end{pgfscope}%
\begin{pgfscope}%
\pgfsys@transformshift{1.251014in}{0.501347in}%
\pgfsys@useobject{currentmarker}{}%
\end{pgfscope}%
\begin{pgfscope}%
\pgfsys@transformshift{1.271674in}{0.594090in}%
\pgfsys@useobject{currentmarker}{}%
\end{pgfscope}%
\begin{pgfscope}%
\pgfsys@transformshift{1.291865in}{0.771975in}%
\pgfsys@useobject{currentmarker}{}%
\end{pgfscope}%
\begin{pgfscope}%
\pgfsys@transformshift{1.309708in}{1.010767in}%
\pgfsys@useobject{currentmarker}{}%
\end{pgfscope}%
\begin{pgfscope}%
\pgfsys@transformshift{1.328255in}{1.284171in}%
\pgfsys@useobject{currentmarker}{}%
\end{pgfscope}%
\begin{pgfscope}%
\pgfsys@transformshift{1.348445in}{1.400691in}%
\pgfsys@useobject{currentmarker}{}%
\end{pgfscope}%
\begin{pgfscope}%
\pgfsys@transformshift{1.365818in}{1.332313in}%
\pgfsys@useobject{currentmarker}{}%
\end{pgfscope}%
\begin{pgfscope}%
\pgfsys@transformshift{1.386713in}{1.059492in}%
\pgfsys@useobject{currentmarker}{}%
\end{pgfscope}%
\begin{pgfscope}%
\pgfsys@transformshift{1.405260in}{0.773507in}%
\pgfsys@useobject{currentmarker}{}%
\end{pgfscope}%
\begin{pgfscope}%
\pgfsys@transformshift{1.423806in}{0.592560in}%
\pgfsys@useobject{currentmarker}{}%
\end{pgfscope}%
\begin{pgfscope}%
\pgfsys@transformshift{1.445875in}{0.497147in}%
\pgfsys@useobject{currentmarker}{}%
\end{pgfscope}%
\begin{pgfscope}%
\pgfsys@transformshift{1.464423in}{0.542085in}%
\pgfsys@useobject{currentmarker}{}%
\end{pgfscope}%
\begin{pgfscope}%
\pgfsys@transformshift{1.481562in}{0.643917in}%
\pgfsys@useobject{currentmarker}{}%
\end{pgfscope}%
\begin{pgfscope}%
\pgfsys@transformshift{1.503865in}{0.663195in}%
\pgfsys@useobject{currentmarker}{}%
\end{pgfscope}%
\begin{pgfscope}%
\pgfsys@transformshift{1.520299in}{0.830665in}%
\pgfsys@useobject{currentmarker}{}%
\end{pgfscope}%
\begin{pgfscope}%
\pgfsys@transformshift{1.539784in}{1.122299in}%
\pgfsys@useobject{currentmarker}{}%
\end{pgfscope}%
\begin{pgfscope}%
\pgfsys@transformshift{1.561150in}{1.363789in}%
\pgfsys@useobject{currentmarker}{}%
\end{pgfscope}%
\begin{pgfscope}%
\pgfsys@transformshift{1.578992in}{1.381900in}%
\pgfsys@useobject{currentmarker}{}%
\end{pgfscope}%
\begin{pgfscope}%
\pgfsys@transformshift{1.596835in}{1.273866in}%
\pgfsys@useobject{currentmarker}{}%
\end{pgfscope}%
\begin{pgfscope}%
\pgfsys@transformshift{1.620077in}{0.874415in}%
\pgfsys@useobject{currentmarker}{}%
\end{pgfscope}%
\begin{pgfscope}%
\pgfsys@transformshift{1.634163in}{1.076515in}%
\pgfsys@useobject{currentmarker}{}%
\end{pgfscope}%
\begin{pgfscope}%
\pgfsys@transformshift{1.662101in}{0.711893in}%
\pgfsys@useobject{currentmarker}{}%
\end{pgfscope}%
\begin{pgfscope}%
\pgfsys@transformshift{1.676422in}{0.585807in}%
\pgfsys@useobject{currentmarker}{}%
\end{pgfscope}%
\begin{pgfscope}%
\pgfsys@transformshift{1.694030in}{0.501210in}%
\pgfsys@useobject{currentmarker}{}%
\end{pgfscope}%
\begin{pgfscope}%
\pgfsys@transformshift{1.711873in}{0.502547in}%
\pgfsys@useobject{currentmarker}{}%
\end{pgfscope}%
\begin{pgfscope}%
\pgfsys@transformshift{1.732064in}{0.621674in}%
\pgfsys@useobject{currentmarker}{}%
\end{pgfscope}%
\begin{pgfscope}%
\pgfsys@transformshift{1.751784in}{0.821213in}%
\pgfsys@useobject{currentmarker}{}%
\end{pgfscope}%
\begin{pgfscope}%
\pgfsys@transformshift{1.771270in}{1.088643in}%
\pgfsys@useobject{currentmarker}{}%
\end{pgfscope}%
\begin{pgfscope}%
\pgfsys@transformshift{1.789818in}{1.318142in}%
\pgfsys@useobject{currentmarker}{}%
\end{pgfscope}%
\begin{pgfscope}%
\pgfsys@transformshift{1.809538in}{1.384268in}%
\pgfsys@useobject{currentmarker}{}%
\end{pgfscope}%
\begin{pgfscope}%
\pgfsys@transformshift{1.827617in}{1.329828in}%
\pgfsys@useobject{currentmarker}{}%
\end{pgfscope}%
\begin{pgfscope}%
\pgfsys@transformshift{1.845459in}{1.094138in}%
\pgfsys@useobject{currentmarker}{}%
\end{pgfscope}%
\begin{pgfscope}%
\pgfsys@transformshift{1.865885in}{0.781697in}%
\pgfsys@useobject{currentmarker}{}%
\end{pgfscope}%
\begin{pgfscope}%
\pgfsys@transformshift{1.886779in}{0.625747in}%
\pgfsys@useobject{currentmarker}{}%
\end{pgfscope}%
\begin{pgfscope}%
\pgfsys@transformshift{1.904151in}{0.522586in}%
\pgfsys@useobject{currentmarker}{}%
\end{pgfscope}%
\begin{pgfscope}%
\pgfsys@transformshift{1.925516in}{0.493638in}%
\pgfsys@useobject{currentmarker}{}%
\end{pgfscope}%
\begin{pgfscope}%
\pgfsys@transformshift{1.943124in}{0.574142in}%
\pgfsys@useobject{currentmarker}{}%
\end{pgfscope}%
\begin{pgfscope}%
\pgfsys@transformshift{1.960497in}{0.722375in}%
\pgfsys@useobject{currentmarker}{}%
\end{pgfscope}%
\begin{pgfscope}%
\pgfsys@transformshift{1.982332in}{0.923401in}%
\pgfsys@useobject{currentmarker}{}%
\end{pgfscope}%
\begin{pgfscope}%
\pgfsys@transformshift{2.000174in}{1.205632in}%
\pgfsys@useobject{currentmarker}{}%
\end{pgfscope}%
\begin{pgfscope}%
\pgfsys@transformshift{2.021303in}{1.376522in}%
\pgfsys@useobject{currentmarker}{}%
\end{pgfscope}%
\begin{pgfscope}%
\pgfsys@transformshift{2.038911in}{1.366969in}%
\pgfsys@useobject{currentmarker}{}%
\end{pgfscope}%
\begin{pgfscope}%
\pgfsys@transformshift{2.056519in}{1.283590in}%
\pgfsys@useobject{currentmarker}{}%
\end{pgfscope}%
\begin{pgfscope}%
\pgfsys@transformshift{2.079762in}{0.967981in}%
\pgfsys@useobject{currentmarker}{}%
\end{pgfscope}%
\begin{pgfscope}%
\pgfsys@transformshift{2.095725in}{0.727587in}%
\pgfsys@useobject{currentmarker}{}%
\end{pgfscope}%
\begin{pgfscope}%
\pgfsys@transformshift{2.116856in}{0.575517in}%
\pgfsys@useobject{currentmarker}{}%
\end{pgfscope}%
\begin{pgfscope}%
\pgfsys@transformshift{2.135167in}{0.502396in}%
\pgfsys@useobject{currentmarker}{}%
\end{pgfscope}%
\begin{pgfscope}%
\pgfsys@transformshift{2.156298in}{0.508727in}%
\pgfsys@useobject{currentmarker}{}%
\end{pgfscope}%
\begin{pgfscope}%
\pgfsys@transformshift{2.173201in}{0.599219in}%
\pgfsys@useobject{currentmarker}{}%
\end{pgfscope}%
\begin{pgfscope}%
\pgfsys@transformshift{2.193860in}{0.746125in}%
\pgfsys@useobject{currentmarker}{}%
\end{pgfscope}%
\begin{pgfscope}%
\pgfsys@transformshift{2.210529in}{0.979138in}%
\pgfsys@useobject{currentmarker}{}%
\end{pgfscope}%
\begin{pgfscope}%
\pgfsys@transformshift{2.230016in}{1.222040in}%
\pgfsys@useobject{currentmarker}{}%
\end{pgfscope}%
\begin{pgfscope}%
\pgfsys@transformshift{2.251849in}{1.366509in}%
\pgfsys@useobject{currentmarker}{}%
\end{pgfscope}%
\begin{pgfscope}%
\pgfsys@transformshift{2.269457in}{1.355809in}%
\pgfsys@useobject{currentmarker}{}%
\end{pgfscope}%
\begin{pgfscope}%
\pgfsys@transformshift{2.290116in}{1.147495in}%
\pgfsys@useobject{currentmarker}{}%
\end{pgfscope}%
\begin{pgfscope}%
\pgfsys@transformshift{2.308195in}{0.873210in}%
\pgfsys@useobject{currentmarker}{}%
\end{pgfscope}%
\begin{pgfscope}%
\pgfsys@transformshift{2.326272in}{0.677807in}%
\pgfsys@useobject{currentmarker}{}%
\end{pgfscope}%
\begin{pgfscope}%
\pgfsys@transformshift{2.347401in}{0.544975in}%
\pgfsys@useobject{currentmarker}{}%
\end{pgfscope}%
\begin{pgfscope}%
\pgfsys@transformshift{2.365244in}{0.486243in}%
\pgfsys@useobject{currentmarker}{}%
\end{pgfscope}%
\begin{pgfscope}%
\pgfsys@transformshift{2.385435in}{0.526128in}%
\pgfsys@useobject{currentmarker}{}%
\end{pgfscope}%
\begin{pgfscope}%
\pgfsys@transformshift{2.404217in}{0.621748in}%
\pgfsys@useobject{currentmarker}{}%
\end{pgfscope}%
\begin{pgfscope}%
\pgfsys@transformshift{2.426286in}{0.757438in}%
\pgfsys@useobject{currentmarker}{}%
\end{pgfscope}%
\begin{pgfscope}%
\pgfsys@transformshift{2.442250in}{0.968610in}%
\pgfsys@useobject{currentmarker}{}%
\end{pgfscope}%
\begin{pgfscope}%
\pgfsys@transformshift{2.465728in}{1.260924in}%
\pgfsys@useobject{currentmarker}{}%
\end{pgfscope}%
\begin{pgfscope}%
\pgfsys@transformshift{2.481691in}{1.370044in}%
\pgfsys@useobject{currentmarker}{}%
\end{pgfscope}%
\begin{pgfscope}%
\pgfsys@transformshift{2.500239in}{1.354621in}%
\pgfsys@useobject{currentmarker}{}%
\end{pgfscope}%
\begin{pgfscope}%
\pgfsys@transformshift{2.518081in}{1.177500in}%
\pgfsys@useobject{currentmarker}{}%
\end{pgfscope}%
\begin{pgfscope}%
\pgfsys@transformshift{2.538975in}{0.862166in}%
\pgfsys@useobject{currentmarker}{}%
\end{pgfscope}%
\begin{pgfscope}%
\pgfsys@transformshift{2.560575in}{0.664652in}%
\pgfsys@useobject{currentmarker}{}%
\end{pgfscope}%
\begin{pgfscope}%
\pgfsys@transformshift{2.578418in}{0.545524in}%
\pgfsys@useobject{currentmarker}{}%
\end{pgfscope}%
\begin{pgfscope}%
\pgfsys@transformshift{2.596260in}{0.491997in}%
\pgfsys@useobject{currentmarker}{}%
\end{pgfscope}%
\begin{pgfscope}%
\pgfsys@transformshift{2.620206in}{0.535026in}%
\pgfsys@useobject{currentmarker}{}%
\end{pgfscope}%
\begin{pgfscope}%
\pgfsys@transformshift{2.634997in}{0.619086in}%
\pgfsys@useobject{currentmarker}{}%
\end{pgfscope}%
\begin{pgfscope}%
\pgfsys@transformshift{2.654250in}{0.761150in}%
\pgfsys@useobject{currentmarker}{}%
\end{pgfscope}%
\begin{pgfscope}%
\pgfsys@transformshift{2.673736in}{1.039714in}%
\pgfsys@useobject{currentmarker}{}%
\end{pgfscope}%
\begin{pgfscope}%
\pgfsys@transformshift{2.691344in}{1.236492in}%
\pgfsys@useobject{currentmarker}{}%
\end{pgfscope}%
\begin{pgfscope}%
\pgfsys@transformshift{2.712004in}{1.316294in}%
\pgfsys@useobject{currentmarker}{}%
\end{pgfscope}%
\begin{pgfscope}%
\pgfsys@transformshift{2.731255in}{1.378676in}%
\pgfsys@useobject{currentmarker}{}%
\end{pgfscope}%
\begin{pgfscope}%
\pgfsys@transformshift{2.749098in}{1.353488in}%
\pgfsys@useobject{currentmarker}{}%
\end{pgfscope}%
\begin{pgfscope}%
\pgfsys@transformshift{2.770226in}{1.241821in}%
\pgfsys@useobject{currentmarker}{}%
\end{pgfscope}%
\begin{pgfscope}%
\pgfsys@transformshift{2.788069in}{1.066938in}%
\pgfsys@useobject{currentmarker}{}%
\end{pgfscope}%
\begin{pgfscope}%
\pgfsys@transformshift{2.810138in}{0.828347in}%
\pgfsys@useobject{currentmarker}{}%
\end{pgfscope}%
\begin{pgfscope}%
\pgfsys@transformshift{2.828685in}{0.677055in}%
\pgfsys@useobject{currentmarker}{}%
\end{pgfscope}%
\begin{pgfscope}%
\pgfsys@transformshift{2.845119in}{0.555424in}%
\pgfsys@useobject{currentmarker}{}%
\end{pgfscope}%
\begin{pgfscope}%
\pgfsys@transformshift{2.865779in}{0.492642in}%
\pgfsys@useobject{currentmarker}{}%
\end{pgfscope}%
\begin{pgfscope}%
\pgfsys@transformshift{2.884327in}{0.540331in}%
\pgfsys@useobject{currentmarker}{}%
\end{pgfscope}%
\begin{pgfscope}%
\pgfsys@transformshift{2.904752in}{0.677282in}%
\pgfsys@useobject{currentmarker}{}%
\end{pgfscope}%
\begin{pgfscope}%
\pgfsys@transformshift{2.924941in}{0.865493in}%
\pgfsys@useobject{currentmarker}{}%
\end{pgfscope}%
\begin{pgfscope}%
\pgfsys@transformshift{2.942784in}{1.119906in}%
\pgfsys@useobject{currentmarker}{}%
\end{pgfscope}%
\begin{pgfscope}%
\pgfsys@transformshift{2.961801in}{1.318563in}%
\pgfsys@useobject{currentmarker}{}%
\end{pgfscope}%
\begin{pgfscope}%
\pgfsys@transformshift{2.978940in}{1.385793in}%
\pgfsys@useobject{currentmarker}{}%
\end{pgfscope}%
\begin{pgfscope}%
\pgfsys@transformshift{3.002417in}{1.338988in}%
\pgfsys@useobject{currentmarker}{}%
\end{pgfscope}%
\begin{pgfscope}%
\pgfsys@transformshift{3.021903in}{1.129226in}%
\pgfsys@useobject{currentmarker}{}%
\end{pgfscope}%
\begin{pgfscope}%
\pgfsys@transformshift{3.040451in}{0.891552in}%
\pgfsys@useobject{currentmarker}{}%
\end{pgfscope}%
\begin{pgfscope}%
\pgfsys@transformshift{3.057822in}{0.959461in}%
\pgfsys@useobject{currentmarker}{}%
\end{pgfscope}%
\begin{pgfscope}%
\pgfsys@transformshift{3.078719in}{0.694749in}%
\pgfsys@useobject{currentmarker}{}%
\end{pgfscope}%
\begin{pgfscope}%
\pgfsys@transformshift{3.097499in}{0.578449in}%
\pgfsys@useobject{currentmarker}{}%
\end{pgfscope}%
\begin{pgfscope}%
\pgfsys@transformshift{3.114872in}{0.503666in}%
\pgfsys@useobject{currentmarker}{}%
\end{pgfscope}%
\begin{pgfscope}%
\pgfsys@transformshift{3.136003in}{0.510866in}%
\pgfsys@useobject{currentmarker}{}%
\end{pgfscope}%
\begin{pgfscope}%
\pgfsys@transformshift{3.153609in}{0.555213in}%
\pgfsys@useobject{currentmarker}{}%
\end{pgfscope}%
\begin{pgfscope}%
\pgfsys@transformshift{3.172392in}{0.672216in}%
\pgfsys@useobject{currentmarker}{}%
\end{pgfscope}%
\begin{pgfscope}%
\pgfsys@transformshift{3.193052in}{0.857682in}%
\pgfsys@useobject{currentmarker}{}%
\end{pgfscope}%
\begin{pgfscope}%
\pgfsys@transformshift{3.211128in}{1.105789in}%
\pgfsys@useobject{currentmarker}{}%
\end{pgfscope}%
\begin{pgfscope}%
\pgfsys@transformshift{3.232025in}{1.284018in}%
\pgfsys@useobject{currentmarker}{}%
\end{pgfscope}%
\begin{pgfscope}%
\pgfsys@transformshift{3.250102in}{1.387724in}%
\pgfsys@useobject{currentmarker}{}%
\end{pgfscope}%
\begin{pgfscope}%
\pgfsys@transformshift{3.267241in}{1.374080in}%
\pgfsys@useobject{currentmarker}{}%
\end{pgfscope}%
\begin{pgfscope}%
\pgfsys@transformshift{3.288604in}{1.233174in}%
\pgfsys@useobject{currentmarker}{}%
\end{pgfscope}%
\begin{pgfscope}%
\pgfsys@transformshift{3.306681in}{0.990425in}%
\pgfsys@useobject{currentmarker}{}%
\end{pgfscope}%
\begin{pgfscope}%
\pgfsys@transformshift{3.325463in}{0.775548in}%
\pgfsys@useobject{currentmarker}{}%
\end{pgfscope}%
\begin{pgfscope}%
\pgfsys@transformshift{3.347767in}{0.632350in}%
\pgfsys@useobject{currentmarker}{}%
\end{pgfscope}%
\begin{pgfscope}%
\pgfsys@transformshift{3.367489in}{0.529151in}%
\pgfsys@useobject{currentmarker}{}%
\end{pgfscope}%
\begin{pgfscope}%
\pgfsys@transformshift{3.382748in}{0.501349in}%
\pgfsys@useobject{currentmarker}{}%
\end{pgfscope}%
\begin{pgfscope}%
\pgfsys@transformshift{3.403408in}{0.536708in}%
\pgfsys@useobject{currentmarker}{}%
\end{pgfscope}%
\begin{pgfscope}%
\pgfsys@transformshift{3.422894in}{0.622425in}%
\pgfsys@useobject{currentmarker}{}%
\end{pgfscope}%
\begin{pgfscope}%
\pgfsys@transformshift{3.442850in}{0.790267in}%
\pgfsys@useobject{currentmarker}{}%
\end{pgfscope}%
\begin{pgfscope}%
\pgfsys@transformshift{3.461162in}{1.002525in}%
\pgfsys@useobject{currentmarker}{}%
\end{pgfscope}%
\begin{pgfscope}%
\pgfsys@transformshift{3.481587in}{1.244704in}%
\pgfsys@useobject{currentmarker}{}%
\end{pgfscope}%
\begin{pgfscope}%
\pgfsys@transformshift{3.500604in}{1.374924in}%
\pgfsys@useobject{currentmarker}{}%
\end{pgfscope}%
\begin{pgfscope}%
\pgfsys@transformshift{3.517272in}{1.399878in}%
\pgfsys@useobject{currentmarker}{}%
\end{pgfscope}%
\begin{pgfscope}%
\pgfsys@transformshift{3.535585in}{1.401172in}%
\pgfsys@useobject{currentmarker}{}%
\end{pgfscope}%
\begin{pgfscope}%
\pgfsys@transformshift{3.556949in}{1.285980in}%
\pgfsys@useobject{currentmarker}{}%
\end{pgfscope}%
\begin{pgfscope}%
\pgfsys@transformshift{3.578314in}{1.012854in}%
\pgfsys@useobject{currentmarker}{}%
\end{pgfscope}%
\begin{pgfscope}%
\pgfsys@transformshift{3.596157in}{0.787040in}%
\pgfsys@useobject{currentmarker}{}%
\end{pgfscope}%
\begin{pgfscope}%
\pgfsys@transformshift{3.615642in}{0.663438in}%
\pgfsys@useobject{currentmarker}{}%
\end{pgfscope}%
\begin{pgfscope}%
\pgfsys@transformshift{3.635833in}{0.566809in}%
\pgfsys@useobject{currentmarker}{}%
\end{pgfscope}%
\begin{pgfscope}%
\pgfsys@transformshift{3.653676in}{0.505799in}%
\pgfsys@useobject{currentmarker}{}%
\end{pgfscope}%
\begin{pgfscope}%
\pgfsys@transformshift{3.674570in}{0.543005in}%
\pgfsys@useobject{currentmarker}{}%
\end{pgfscope}%
\begin{pgfscope}%
\pgfsys@transformshift{3.692647in}{0.634681in}%
\pgfsys@useobject{currentmarker}{}%
\end{pgfscope}%
\begin{pgfscope}%
\pgfsys@transformshift{3.710960in}{0.766855in}%
\pgfsys@useobject{currentmarker}{}%
\end{pgfscope}%
\begin{pgfscope}%
\pgfsys@transformshift{3.732558in}{0.978515in}%
\pgfsys@useobject{currentmarker}{}%
\end{pgfscope}%
\begin{pgfscope}%
\pgfsys@transformshift{3.748758in}{1.128828in}%
\pgfsys@useobject{currentmarker}{}%
\end{pgfscope}%
\begin{pgfscope}%
\pgfsys@transformshift{3.770826in}{1.345647in}%
\pgfsys@useobject{currentmarker}{}%
\end{pgfscope}%
\begin{pgfscope}%
\pgfsys@transformshift{3.788200in}{1.409443in}%
\pgfsys@useobject{currentmarker}{}%
\end{pgfscope}%
\begin{pgfscope}%
\pgfsys@transformshift{3.809094in}{1.416417in}%
\pgfsys@useobject{currentmarker}{}%
\end{pgfscope}%
\begin{pgfscope}%
\pgfsys@transformshift{3.828816in}{1.326168in}%
\pgfsys@useobject{currentmarker}{}%
\end{pgfscope}%
\begin{pgfscope}%
\pgfsys@transformshift{3.850180in}{1.074330in}%
\pgfsys@useobject{currentmarker}{}%
\end{pgfscope}%
\begin{pgfscope}%
\pgfsys@transformshift{3.866850in}{0.879788in}%
\pgfsys@useobject{currentmarker}{}%
\end{pgfscope}%
\begin{pgfscope}%
\pgfsys@transformshift{3.884456in}{0.722787in}%
\pgfsys@useobject{currentmarker}{}%
\end{pgfscope}%
\begin{pgfscope}%
\pgfsys@transformshift{3.903473in}{0.602966in}%
\pgfsys@useobject{currentmarker}{}%
\end{pgfscope}%
\begin{pgfscope}%
\pgfsys@transformshift{3.922020in}{0.641329in}%
\pgfsys@useobject{currentmarker}{}%
\end{pgfscope}%
\begin{pgfscope}%
\pgfsys@transformshift{3.939159in}{0.543749in}%
\pgfsys@useobject{currentmarker}{}%
\end{pgfscope}%
\begin{pgfscope}%
\pgfsys@transformshift{3.960757in}{0.514620in}%
\pgfsys@useobject{currentmarker}{}%
\end{pgfscope}%
\begin{pgfscope}%
\pgfsys@transformshift{3.980714in}{0.571049in}%
\pgfsys@useobject{currentmarker}{}%
\end{pgfscope}%
\begin{pgfscope}%
\pgfsys@transformshift{3.999731in}{0.664568in}%
\pgfsys@useobject{currentmarker}{}%
\end{pgfscope}%
\begin{pgfscope}%
\pgfsys@transformshift{4.021094in}{0.851134in}%
\pgfsys@useobject{currentmarker}{}%
\end{pgfscope}%
\begin{pgfscope}%
\pgfsys@transformshift{4.038702in}{0.998962in}%
\pgfsys@useobject{currentmarker}{}%
\end{pgfscope}%
\begin{pgfscope}%
\pgfsys@transformshift{4.056544in}{1.231245in}%
\pgfsys@useobject{currentmarker}{}%
\end{pgfscope}%
\begin{pgfscope}%
\pgfsys@transformshift{4.077439in}{1.391888in}%
\pgfsys@useobject{currentmarker}{}%
\end{pgfscope}%
\begin{pgfscope}%
\pgfsys@transformshift{4.094109in}{1.438505in}%
\pgfsys@useobject{currentmarker}{}%
\end{pgfscope}%
\begin{pgfscope}%
\pgfsys@transformshift{4.115943in}{1.400726in}%
\pgfsys@useobject{currentmarker}{}%
\end{pgfscope}%
\begin{pgfscope}%
\pgfsys@transformshift{4.134960in}{1.281647in}%
\pgfsys@useobject{currentmarker}{}%
\end{pgfscope}%
\begin{pgfscope}%
\pgfsys@transformshift{4.154680in}{1.090561in}%
\pgfsys@useobject{currentmarker}{}%
\end{pgfscope}%
\begin{pgfscope}%
\pgfsys@transformshift{4.177218in}{0.824246in}%
\pgfsys@useobject{currentmarker}{}%
\end{pgfscope}%
\begin{pgfscope}%
\pgfsys@transformshift{4.193417in}{0.692740in}%
\pgfsys@useobject{currentmarker}{}%
\end{pgfscope}%
\begin{pgfscope}%
\pgfsys@transformshift{4.212199in}{0.621438in}%
\pgfsys@useobject{currentmarker}{}%
\end{pgfscope}%
\begin{pgfscope}%
\pgfsys@transformshift{4.228164in}{0.543992in}%
\pgfsys@useobject{currentmarker}{}%
\end{pgfscope}%
\begin{pgfscope}%
\pgfsys@transformshift{4.249998in}{0.553150in}%
\pgfsys@useobject{currentmarker}{}%
\end{pgfscope}%
\begin{pgfscope}%
\pgfsys@transformshift{4.268544in}{0.653730in}%
\pgfsys@useobject{currentmarker}{}%
\end{pgfscope}%
\begin{pgfscope}%
\pgfsys@transformshift{4.288501in}{0.768223in}%
\pgfsys@useobject{currentmarker}{}%
\end{pgfscope}%
\begin{pgfscope}%
\pgfsys@transformshift{4.306578in}{0.945631in}%
\pgfsys@useobject{currentmarker}{}%
\end{pgfscope}%
\begin{pgfscope}%
\pgfsys@transformshift{4.325829in}{1.112598in}%
\pgfsys@useobject{currentmarker}{}%
\end{pgfscope}%
\begin{pgfscope}%
\pgfsys@transformshift{4.345314in}{1.249371in}%
\pgfsys@useobject{currentmarker}{}%
\end{pgfscope}%
\begin{pgfscope}%
\pgfsys@transformshift{4.364566in}{1.416971in}%
\pgfsys@useobject{currentmarker}{}%
\end{pgfscope}%
\begin{pgfscope}%
\pgfsys@transformshift{4.383348in}{1.456355in}%
\pgfsys@useobject{currentmarker}{}%
\end{pgfscope}%
\begin{pgfscope}%
\pgfsys@transformshift{4.402130in}{1.435375in}%
\pgfsys@useobject{currentmarker}{}%
\end{pgfscope}%
\begin{pgfscope}%
\pgfsys@transformshift{4.421616in}{1.303697in}%
\pgfsys@useobject{currentmarker}{}%
\end{pgfscope}%
\begin{pgfscope}%
\pgfsys@transformshift{4.441102in}{1.241355in}%
\pgfsys@useobject{currentmarker}{}%
\end{pgfscope}%
\begin{pgfscope}%
\pgfsys@transformshift{4.459415in}{0.994908in}%
\pgfsys@useobject{currentmarker}{}%
\end{pgfscope}%
\begin{pgfscope}%
\pgfsys@transformshift{4.478432in}{0.778469in}%
\pgfsys@useobject{currentmarker}{}%
\end{pgfscope}%
\begin{pgfscope}%
\pgfsys@transformshift{4.479840in}{0.763916in}%
\pgfsys@useobject{currentmarker}{}%
\end{pgfscope}%
\begin{pgfscope}%
\pgfsys@transformshift{4.473971in}{0.880316in}%
\pgfsys@useobject{currentmarker}{}%
\end{pgfscope}%
\begin{pgfscope}%
\pgfsys@transformshift{4.456127in}{1.176547in}%
\pgfsys@useobject{currentmarker}{}%
\end{pgfscope}%
\begin{pgfscope}%
\pgfsys@transformshift{4.438989in}{1.458072in}%
\pgfsys@useobject{currentmarker}{}%
\end{pgfscope}%
\begin{pgfscope}%
\pgfsys@transformshift{4.416215in}{1.049879in}%
\pgfsys@useobject{currentmarker}{}%
\end{pgfscope}%
\begin{pgfscope}%
\pgfsys@transformshift{4.397435in}{0.631414in}%
\pgfsys@useobject{currentmarker}{}%
\end{pgfscope}%
\begin{pgfscope}%
\pgfsys@transformshift{4.376539in}{0.526410in}%
\pgfsys@useobject{currentmarker}{}%
\end{pgfscope}%
\begin{pgfscope}%
\pgfsys@transformshift{4.358227in}{0.609637in}%
\pgfsys@useobject{currentmarker}{}%
\end{pgfscope}%
\begin{pgfscope}%
\pgfsys@transformshift{4.340619in}{0.790739in}%
\pgfsys@useobject{currentmarker}{}%
\end{pgfscope}%
\begin{pgfscope}%
\pgfsys@transformshift{4.316908in}{1.192304in}%
\pgfsys@useobject{currentmarker}{}%
\end{pgfscope}%
\begin{pgfscope}%
\pgfsys@transformshift{4.303057in}{1.384170in}%
\pgfsys@useobject{currentmarker}{}%
\end{pgfscope}%
\begin{pgfscope}%
\pgfsys@transformshift{4.283335in}{1.436605in}%
\pgfsys@useobject{currentmarker}{}%
\end{pgfscope}%
\begin{pgfscope}%
\pgfsys@transformshift{4.265258in}{1.306771in}%
\pgfsys@useobject{currentmarker}{}%
\end{pgfscope}%
\begin{pgfscope}%
\pgfsys@transformshift{4.242720in}{0.949168in}%
\pgfsys@useobject{currentmarker}{}%
\end{pgfscope}%
\begin{pgfscope}%
\pgfsys@transformshift{4.224876in}{0.686560in}%
\pgfsys@useobject{currentmarker}{}%
\end{pgfscope}%
\begin{pgfscope}%
\pgfsys@transformshift{4.204216in}{0.552239in}%
\pgfsys@useobject{currentmarker}{}%
\end{pgfscope}%
\begin{pgfscope}%
\pgfsys@transformshift{4.184965in}{0.543523in}%
\pgfsys@useobject{currentmarker}{}%
\end{pgfscope}%
\begin{pgfscope}%
\pgfsys@transformshift{4.166653in}{0.671408in}%
\pgfsys@useobject{currentmarker}{}%
\end{pgfscope}%
\begin{pgfscope}%
\pgfsys@transformshift{4.149280in}{0.895143in}%
\pgfsys@useobject{currentmarker}{}%
\end{pgfscope}%
\begin{pgfscope}%
\pgfsys@transformshift{4.129089in}{1.257923in}%
\pgfsys@useobject{currentmarker}{}%
\end{pgfscope}%
\begin{pgfscope}%
\pgfsys@transformshift{4.109837in}{1.414769in}%
\pgfsys@useobject{currentmarker}{}%
\end{pgfscope}%
\begin{pgfscope}%
\pgfsys@transformshift{4.088005in}{1.368478in}%
\pgfsys@useobject{currentmarker}{}%
\end{pgfscope}%
\begin{pgfscope}%
\pgfsys@transformshift{4.069926in}{1.107672in}%
\pgfsys@useobject{currentmarker}{}%
\end{pgfscope}%
\begin{pgfscope}%
\pgfsys@transformshift{4.052318in}{0.845371in}%
\pgfsys@useobject{currentmarker}{}%
\end{pgfscope}%
\begin{pgfscope}%
\pgfsys@transformshift{4.034710in}{0.644764in}%
\pgfsys@useobject{currentmarker}{}%
\end{pgfscope}%
\begin{pgfscope}%
\pgfsys@transformshift{4.012407in}{0.531043in}%
\pgfsys@useobject{currentmarker}{}%
\end{pgfscope}%
\begin{pgfscope}%
\pgfsys@transformshift{3.991044in}{0.534209in}%
\pgfsys@useobject{currentmarker}{}%
\end{pgfscope}%
\begin{pgfscope}%
\pgfsys@transformshift{3.973436in}{0.664426in}%
\pgfsys@useobject{currentmarker}{}%
\end{pgfscope}%
\begin{pgfscope}%
\pgfsys@transformshift{3.952776in}{0.913314in}%
\pgfsys@useobject{currentmarker}{}%
\end{pgfscope}%
\begin{pgfscope}%
\pgfsys@transformshift{3.938454in}{1.189982in}%
\pgfsys@useobject{currentmarker}{}%
\end{pgfscope}%
\begin{pgfscope}%
\pgfsys@transformshift{3.914039in}{1.403401in}%
\pgfsys@useobject{currentmarker}{}%
\end{pgfscope}%
\begin{pgfscope}%
\pgfsys@transformshift{3.897840in}{1.389228in}%
\pgfsys@useobject{currentmarker}{}%
\end{pgfscope}%
\begin{pgfscope}%
\pgfsys@transformshift{3.880232in}{1.202815in}%
\pgfsys@useobject{currentmarker}{}%
\end{pgfscope}%
\begin{pgfscope}%
\pgfsys@transformshift{3.858866in}{0.844137in}%
\pgfsys@useobject{currentmarker}{}%
\end{pgfscope}%
\begin{pgfscope}%
\pgfsys@transformshift{3.836329in}{0.609755in}%
\pgfsys@useobject{currentmarker}{}%
\end{pgfscope}%
\begin{pgfscope}%
\pgfsys@transformshift{3.822242in}{0.526696in}%
\pgfsys@useobject{currentmarker}{}%
\end{pgfscope}%
\begin{pgfscope}%
\pgfsys@transformshift{3.800407in}{0.510817in}%
\pgfsys@useobject{currentmarker}{}%
\end{pgfscope}%
\begin{pgfscope}%
\pgfsys@transformshift{3.783036in}{0.563963in}%
\pgfsys@useobject{currentmarker}{}%
\end{pgfscope}%
\begin{pgfscope}%
\pgfsys@transformshift{3.764488in}{0.723816in}%
\pgfsys@useobject{currentmarker}{}%
\end{pgfscope}%
\begin{pgfscope}%
\pgfsys@transformshift{3.742654in}{1.038375in}%
\pgfsys@useobject{currentmarker}{}%
\end{pgfscope}%
\begin{pgfscope}%
\pgfsys@transformshift{3.723871in}{1.293546in}%
\pgfsys@useobject{currentmarker}{}%
\end{pgfscope}%
\begin{pgfscope}%
\pgfsys@transformshift{3.704857in}{1.399444in}%
\pgfsys@useobject{currentmarker}{}%
\end{pgfscope}%
\begin{pgfscope}%
\pgfsys@transformshift{3.686543in}{1.347046in}%
\pgfsys@useobject{currentmarker}{}%
\end{pgfscope}%
\begin{pgfscope}%
\pgfsys@transformshift{3.665649in}{1.139367in}%
\pgfsys@useobject{currentmarker}{}%
\end{pgfscope}%
\begin{pgfscope}%
\pgfsys@transformshift{3.648275in}{0.890375in}%
\pgfsys@useobject{currentmarker}{}%
\end{pgfscope}%
\begin{pgfscope}%
\pgfsys@transformshift{3.626912in}{0.723123in}%
\pgfsys@useobject{currentmarker}{}%
\end{pgfscope}%
\begin{pgfscope}%
\pgfsys@transformshift{3.608130in}{0.574011in}%
\pgfsys@useobject{currentmarker}{}%
\end{pgfscope}%
\begin{pgfscope}%
\pgfsys@transformshift{3.587939in}{0.495982in}%
\pgfsys@useobject{currentmarker}{}%
\end{pgfscope}%
\begin{pgfscope}%
\pgfsys@transformshift{3.571270in}{0.546564in}%
\pgfsys@useobject{currentmarker}{}%
\end{pgfscope}%
\begin{pgfscope}%
\pgfsys@transformshift{3.551550in}{0.677049in}%
\pgfsys@useobject{currentmarker}{}%
\end{pgfscope}%
\begin{pgfscope}%
\pgfsys@transformshift{3.532534in}{0.832466in}%
\pgfsys@useobject{currentmarker}{}%
\end{pgfscope}%
\begin{pgfscope}%
\pgfsys@transformshift{3.512812in}{1.138528in}%
\pgfsys@useobject{currentmarker}{}%
\end{pgfscope}%
\begin{pgfscope}%
\pgfsys@transformshift{3.493091in}{1.328386in}%
\pgfsys@useobject{currentmarker}{}%
\end{pgfscope}%
\begin{pgfscope}%
\pgfsys@transformshift{3.474778in}{1.376493in}%
\pgfsys@useobject{currentmarker}{}%
\end{pgfscope}%
\begin{pgfscope}%
\pgfsys@transformshift{3.456232in}{1.368241in}%
\pgfsys@useobject{currentmarker}{}%
\end{pgfscope}%
\begin{pgfscope}%
\pgfsys@transformshift{3.436981in}{1.166402in}%
\pgfsys@useobject{currentmarker}{}%
\end{pgfscope}%
\begin{pgfscope}%
\pgfsys@transformshift{3.417495in}{0.853829in}%
\pgfsys@useobject{currentmarker}{}%
\end{pgfscope}%
\begin{pgfscope}%
\pgfsys@transformshift{3.398008in}{0.660703in}%
\pgfsys@useobject{currentmarker}{}%
\end{pgfscope}%
\begin{pgfscope}%
\pgfsys@transformshift{3.375939in}{0.545218in}%
\pgfsys@useobject{currentmarker}{}%
\end{pgfscope}%
\begin{pgfscope}%
\pgfsys@transformshift{3.357628in}{0.491135in}%
\pgfsys@useobject{currentmarker}{}%
\end{pgfscope}%
\begin{pgfscope}%
\pgfsys@transformshift{3.338376in}{0.542061in}%
\pgfsys@useobject{currentmarker}{}%
\end{pgfscope}%
\begin{pgfscope}%
\pgfsys@transformshift{3.321943in}{0.664294in}%
\pgfsys@useobject{currentmarker}{}%
\end{pgfscope}%
\begin{pgfscope}%
\pgfsys@transformshift{3.299639in}{0.849518in}%
\pgfsys@useobject{currentmarker}{}%
\end{pgfscope}%
\begin{pgfscope}%
\pgfsys@transformshift{3.282500in}{0.678948in}%
\pgfsys@useobject{currentmarker}{}%
\end{pgfscope}%
\begin{pgfscope}%
\pgfsys@transformshift{3.262780in}{0.939778in}%
\pgfsys@useobject{currentmarker}{}%
\end{pgfscope}%
\begin{pgfscope}%
\pgfsys@transformshift{3.241884in}{1.245493in}%
\pgfsys@useobject{currentmarker}{}%
\end{pgfscope}%
\begin{pgfscope}%
\pgfsys@transformshift{3.224747in}{1.377827in}%
\pgfsys@useobject{currentmarker}{}%
\end{pgfscope}%
\begin{pgfscope}%
\pgfsys@transformshift{3.206199in}{1.376771in}%
\pgfsys@useobject{currentmarker}{}%
\end{pgfscope}%
\begin{pgfscope}%
\pgfsys@transformshift{3.183427in}{1.205942in}%
\pgfsys@useobject{currentmarker}{}%
\end{pgfscope}%
\begin{pgfscope}%
\pgfsys@transformshift{3.165114in}{0.916662in}%
\pgfsys@useobject{currentmarker}{}%
\end{pgfscope}%
\begin{pgfscope}%
\pgfsys@transformshift{3.149854in}{0.735339in}%
\pgfsys@useobject{currentmarker}{}%
\end{pgfscope}%
\begin{pgfscope}%
\pgfsys@transformshift{3.128725in}{0.586334in}%
\pgfsys@useobject{currentmarker}{}%
\end{pgfscope}%
\begin{pgfscope}%
\pgfsys@transformshift{3.109003in}{0.496535in}%
\pgfsys@useobject{currentmarker}{}%
\end{pgfscope}%
\begin{pgfscope}%
\pgfsys@transformshift{3.089986in}{0.517580in}%
\pgfsys@useobject{currentmarker}{}%
\end{pgfscope}%
\begin{pgfscope}%
\pgfsys@transformshift{3.070735in}{0.620072in}%
\pgfsys@useobject{currentmarker}{}%
\end{pgfscope}%
\begin{pgfscope}%
\pgfsys@transformshift{3.053362in}{0.773176in}%
\pgfsys@useobject{currentmarker}{}%
\end{pgfscope}%
\begin{pgfscope}%
\pgfsys@transformshift{3.033876in}{1.015309in}%
\pgfsys@useobject{currentmarker}{}%
\end{pgfscope}%
\begin{pgfscope}%
\pgfsys@transformshift{3.013921in}{0.517500in}%
\pgfsys@useobject{currentmarker}{}%
\end{pgfscope}%
\begin{pgfscope}%
\pgfsys@transformshift{2.993496in}{0.621445in}%
\pgfsys@useobject{currentmarker}{}%
\end{pgfscope}%
\begin{pgfscope}%
\pgfsys@transformshift{2.975419in}{0.736992in}%
\pgfsys@useobject{currentmarker}{}%
\end{pgfscope}%
\begin{pgfscope}%
\pgfsys@transformshift{2.956637in}{0.993191in}%
\pgfsys@useobject{currentmarker}{}%
\end{pgfscope}%
\begin{pgfscope}%
\pgfsys@transformshift{2.934568in}{1.311727in}%
\pgfsys@useobject{currentmarker}{}%
\end{pgfscope}%
\begin{pgfscope}%
\pgfsys@transformshift{2.916725in}{1.381521in}%
\pgfsys@useobject{currentmarker}{}%
\end{pgfscope}%
\begin{pgfscope}%
\pgfsys@transformshift{2.897943in}{1.284646in}%
\pgfsys@useobject{currentmarker}{}%
\end{pgfscope}%
\begin{pgfscope}%
\pgfsys@transformshift{2.879632in}{1.023190in}%
\pgfsys@useobject{currentmarker}{}%
\end{pgfscope}%
\begin{pgfscope}%
\pgfsys@transformshift{2.860144in}{0.768568in}%
\pgfsys@useobject{currentmarker}{}%
\end{pgfscope}%
\begin{pgfscope}%
\pgfsys@transformshift{2.840658in}{0.600968in}%
\pgfsys@useobject{currentmarker}{}%
\end{pgfscope}%
\begin{pgfscope}%
\pgfsys@transformshift{2.819999in}{0.497918in}%
\pgfsys@useobject{currentmarker}{}%
\end{pgfscope}%
\begin{pgfscope}%
\pgfsys@transformshift{2.801451in}{0.506964in}%
\pgfsys@useobject{currentmarker}{}%
\end{pgfscope}%
\begin{pgfscope}%
\pgfsys@transformshift{2.785957in}{0.556405in}%
\pgfsys@useobject{currentmarker}{}%
\end{pgfscope}%
\begin{pgfscope}%
\pgfsys@transformshift{2.763888in}{0.720268in}%
\pgfsys@useobject{currentmarker}{}%
\end{pgfscope}%
\begin{pgfscope}%
\pgfsys@transformshift{2.746046in}{0.971951in}%
\pgfsys@useobject{currentmarker}{}%
\end{pgfscope}%
\begin{pgfscope}%
\pgfsys@transformshift{2.723977in}{1.272618in}%
\pgfsys@useobject{currentmarker}{}%
\end{pgfscope}%
\begin{pgfscope}%
\pgfsys@transformshift{2.706134in}{1.378500in}%
\pgfsys@useobject{currentmarker}{}%
\end{pgfscope}%
\begin{pgfscope}%
\pgfsys@transformshift{2.687821in}{1.349271in}%
\pgfsys@useobject{currentmarker}{}%
\end{pgfscope}%
\begin{pgfscope}%
\pgfsys@transformshift{2.666458in}{1.136017in}%
\pgfsys@useobject{currentmarker}{}%
\end{pgfscope}%
\begin{pgfscope}%
\pgfsys@transformshift{2.645093in}{0.806604in}%
\pgfsys@useobject{currentmarker}{}%
\end{pgfscope}%
\begin{pgfscope}%
\pgfsys@transformshift{2.625372in}{0.633282in}%
\pgfsys@useobject{currentmarker}{}%
\end{pgfscope}%
\begin{pgfscope}%
\pgfsys@transformshift{2.607061in}{0.521688in}%
\pgfsys@useobject{currentmarker}{}%
\end{pgfscope}%
\begin{pgfscope}%
\pgfsys@transformshift{2.591565in}{0.486114in}%
\pgfsys@useobject{currentmarker}{}%
\end{pgfscope}%
\begin{pgfscope}%
\pgfsys@transformshift{2.571374in}{0.503131in}%
\pgfsys@useobject{currentmarker}{}%
\end{pgfscope}%
\begin{pgfscope}%
\pgfsys@transformshift{2.553063in}{0.595591in}%
\pgfsys@useobject{currentmarker}{}%
\end{pgfscope}%
\begin{pgfscope}%
\pgfsys@transformshift{2.532872in}{0.750611in}%
\pgfsys@useobject{currentmarker}{}%
\end{pgfscope}%
\begin{pgfscope}%
\pgfsys@transformshift{2.512681in}{0.963804in}%
\pgfsys@useobject{currentmarker}{}%
\end{pgfscope}%
\begin{pgfscope}%
\pgfsys@transformshift{2.494135in}{1.257237in}%
\pgfsys@useobject{currentmarker}{}%
\end{pgfscope}%
\begin{pgfscope}%
\pgfsys@transformshift{2.475587in}{1.376505in}%
\pgfsys@useobject{currentmarker}{}%
\end{pgfscope}%
\begin{pgfscope}%
\pgfsys@transformshift{2.456807in}{1.341630in}%
\pgfsys@useobject{currentmarker}{}%
\end{pgfscope}%
\begin{pgfscope}%
\pgfsys@transformshift{2.434738in}{1.099482in}%
\pgfsys@useobject{currentmarker}{}%
\end{pgfscope}%
\begin{pgfscope}%
\pgfsys@transformshift{2.420182in}{0.986205in}%
\pgfsys@useobject{currentmarker}{}%
\end{pgfscope}%
\begin{pgfscope}%
\pgfsys@transformshift{2.398348in}{0.723886in}%
\pgfsys@useobject{currentmarker}{}%
\end{pgfscope}%
\begin{pgfscope}%
\pgfsys@transformshift{2.379331in}{0.587870in}%
\pgfsys@useobject{currentmarker}{}%
\end{pgfscope}%
\begin{pgfscope}%
\pgfsys@transformshift{2.360314in}{0.504039in}%
\pgfsys@useobject{currentmarker}{}%
\end{pgfscope}%
\begin{pgfscope}%
\pgfsys@transformshift{2.342706in}{0.501232in}%
\pgfsys@useobject{currentmarker}{}%
\end{pgfscope}%
\begin{pgfscope}%
\pgfsys@transformshift{2.320403in}{0.596452in}%
\pgfsys@useobject{currentmarker}{}%
\end{pgfscope}%
\begin{pgfscope}%
\pgfsys@transformshift{2.301152in}{0.735426in}%
\pgfsys@useobject{currentmarker}{}%
\end{pgfscope}%
\begin{pgfscope}%
\pgfsys@transformshift{2.282604in}{1.000740in}%
\pgfsys@useobject{currentmarker}{}%
\end{pgfscope}%
\begin{pgfscope}%
\pgfsys@transformshift{2.264527in}{1.267038in}%
\pgfsys@useobject{currentmarker}{}%
\end{pgfscope}%
\begin{pgfscope}%
\pgfsys@transformshift{2.245510in}{1.373588in}%
\pgfsys@useobject{currentmarker}{}%
\end{pgfscope}%
\begin{pgfscope}%
\pgfsys@transformshift{2.227433in}{1.363464in}%
\pgfsys@useobject{currentmarker}{}%
\end{pgfscope}%
\begin{pgfscope}%
\pgfsys@transformshift{2.209120in}{1.210193in}%
\pgfsys@useobject{currentmarker}{}%
\end{pgfscope}%
\begin{pgfscope}%
\pgfsys@transformshift{2.187522in}{1.017280in}%
\pgfsys@useobject{currentmarker}{}%
\end{pgfscope}%
\begin{pgfscope}%
\pgfsys@transformshift{2.165688in}{0.743290in}%
\pgfsys@useobject{currentmarker}{}%
\end{pgfscope}%
\begin{pgfscope}%
\pgfsys@transformshift{2.149020in}{0.620986in}%
\pgfsys@useobject{currentmarker}{}%
\end{pgfscope}%
\begin{pgfscope}%
\pgfsys@transformshift{2.128594in}{0.521441in}%
\pgfsys@useobject{currentmarker}{}%
\end{pgfscope}%
\begin{pgfscope}%
\pgfsys@transformshift{2.109578in}{0.489202in}%
\pgfsys@useobject{currentmarker}{}%
\end{pgfscope}%
\begin{pgfscope}%
\pgfsys@transformshift{2.091030in}{0.539859in}%
\pgfsys@useobject{currentmarker}{}%
\end{pgfscope}%
\begin{pgfscope}%
\pgfsys@transformshift{2.072249in}{0.643244in}%
\pgfsys@useobject{currentmarker}{}%
\end{pgfscope}%
\begin{pgfscope}%
\pgfsys@transformshift{2.053702in}{0.827485in}%
\pgfsys@useobject{currentmarker}{}%
\end{pgfscope}%
\begin{pgfscope}%
\pgfsys@transformshift{2.032573in}{1.125371in}%
\pgfsys@useobject{currentmarker}{}%
\end{pgfscope}%
\begin{pgfscope}%
\pgfsys@transformshift{2.014259in}{1.332578in}%
\pgfsys@useobject{currentmarker}{}%
\end{pgfscope}%
\begin{pgfscope}%
\pgfsys@transformshift{1.996417in}{1.387186in}%
\pgfsys@useobject{currentmarker}{}%
\end{pgfscope}%
\begin{pgfscope}%
\pgfsys@transformshift{1.975522in}{1.316886in}%
\pgfsys@useobject{currentmarker}{}%
\end{pgfscope}%
\begin{pgfscope}%
\pgfsys@transformshift{1.956506in}{1.190287in}%
\pgfsys@useobject{currentmarker}{}%
\end{pgfscope}%
\begin{pgfscope}%
\pgfsys@transformshift{1.937020in}{0.920081in}%
\pgfsys@useobject{currentmarker}{}%
\end{pgfscope}%
\begin{pgfscope}%
\pgfsys@transformshift{1.919178in}{0.766273in}%
\pgfsys@useobject{currentmarker}{}%
\end{pgfscope}%
\begin{pgfscope}%
\pgfsys@transformshift{1.899455in}{0.617259in}%
\pgfsys@useobject{currentmarker}{}%
\end{pgfscope}%
\begin{pgfscope}%
\pgfsys@transformshift{1.877152in}{0.524508in}%
\pgfsys@useobject{currentmarker}{}%
\end{pgfscope}%
\begin{pgfscope}%
\pgfsys@transformshift{1.862831in}{0.500543in}%
\pgfsys@useobject{currentmarker}{}%
\end{pgfscope}%
\begin{pgfscope}%
\pgfsys@transformshift{1.840528in}{0.551123in}%
\pgfsys@useobject{currentmarker}{}%
\end{pgfscope}%
\begin{pgfscope}%
\pgfsys@transformshift{1.823625in}{0.636815in}%
\pgfsys@useobject{currentmarker}{}%
\end{pgfscope}%
\begin{pgfscope}%
\pgfsys@transformshift{1.802260in}{0.825519in}%
\pgfsys@useobject{currentmarker}{}%
\end{pgfscope}%
\begin{pgfscope}%
\pgfsys@transformshift{1.781600in}{1.104609in}%
\pgfsys@useobject{currentmarker}{}%
\end{pgfscope}%
\begin{pgfscope}%
\pgfsys@transformshift{1.763992in}{1.298273in}%
\pgfsys@useobject{currentmarker}{}%
\end{pgfscope}%
\begin{pgfscope}%
\pgfsys@transformshift{1.745446in}{1.385760in}%
\pgfsys@useobject{currentmarker}{}%
\end{pgfscope}%
\begin{pgfscope}%
\pgfsys@transformshift{1.727603in}{1.386380in}%
\pgfsys@useobject{currentmarker}{}%
\end{pgfscope}%
\begin{pgfscope}%
\pgfsys@transformshift{1.704595in}{1.277721in}%
\pgfsys@useobject{currentmarker}{}%
\end{pgfscope}%
\begin{pgfscope}%
\pgfsys@transformshift{1.689335in}{1.160532in}%
\pgfsys@useobject{currentmarker}{}%
\end{pgfscope}%
\begin{pgfscope}%
\pgfsys@transformshift{1.665624in}{0.852795in}%
\pgfsys@useobject{currentmarker}{}%
\end{pgfscope}%
\begin{pgfscope}%
\pgfsys@transformshift{1.649659in}{0.744774in}%
\pgfsys@useobject{currentmarker}{}%
\end{pgfscope}%
\begin{pgfscope}%
\pgfsys@transformshift{1.630407in}{0.631228in}%
\pgfsys@useobject{currentmarker}{}%
\end{pgfscope}%
\begin{pgfscope}%
\pgfsys@transformshift{1.608808in}{0.524662in}%
\pgfsys@useobject{currentmarker}{}%
\end{pgfscope}%
\begin{pgfscope}%
\pgfsys@transformshift{1.591200in}{0.505386in}%
\pgfsys@useobject{currentmarker}{}%
\end{pgfscope}%
\begin{pgfscope}%
\pgfsys@transformshift{1.572417in}{0.549960in}%
\pgfsys@useobject{currentmarker}{}%
\end{pgfscope}%
\begin{pgfscope}%
\pgfsys@transformshift{1.554575in}{0.627393in}%
\pgfsys@useobject{currentmarker}{}%
\end{pgfscope}%
\begin{pgfscope}%
\pgfsys@transformshift{1.534149in}{0.798174in}%
\pgfsys@useobject{currentmarker}{}%
\end{pgfscope}%
\begin{pgfscope}%
\pgfsys@transformshift{1.515369in}{0.925110in}%
\pgfsys@useobject{currentmarker}{}%
\end{pgfscope}%
\begin{pgfscope}%
\pgfsys@transformshift{1.495647in}{1.019799in}%
\pgfsys@useobject{currentmarker}{}%
\end{pgfscope}%
\begin{pgfscope}%
\pgfsys@transformshift{1.474753in}{1.288496in}%
\pgfsys@useobject{currentmarker}{}%
\end{pgfscope}%
\begin{pgfscope}%
\pgfsys@transformshift{1.454798in}{1.400751in}%
\pgfsys@useobject{currentmarker}{}%
\end{pgfscope}%
\begin{pgfscope}%
\pgfsys@transformshift{1.436485in}{1.400011in}%
\pgfsys@useobject{currentmarker}{}%
\end{pgfscope}%
\begin{pgfscope}%
\pgfsys@transformshift{1.416294in}{1.315918in}%
\pgfsys@useobject{currentmarker}{}%
\end{pgfscope}%
\begin{pgfscope}%
\pgfsys@transformshift{1.400331in}{1.129688in}%
\pgfsys@useobject{currentmarker}{}%
\end{pgfscope}%
\begin{pgfscope}%
\pgfsys@transformshift{1.382017in}{0.967516in}%
\pgfsys@useobject{currentmarker}{}%
\end{pgfscope}%
\begin{pgfscope}%
\pgfsys@transformshift{1.359480in}{0.752984in}%
\pgfsys@useobject{currentmarker}{}%
\end{pgfscope}%
\begin{pgfscope}%
\pgfsys@transformshift{1.341637in}{0.620455in}%
\pgfsys@useobject{currentmarker}{}%
\end{pgfscope}%
\begin{pgfscope}%
\pgfsys@transformshift{1.322855in}{0.543535in}%
\pgfsys@useobject{currentmarker}{}%
\end{pgfscope}%
\begin{pgfscope}%
\pgfsys@transformshift{1.303838in}{0.509089in}%
\pgfsys@useobject{currentmarker}{}%
\end{pgfscope}%
\begin{pgfscope}%
\pgfsys@transformshift{1.282944in}{0.577654in}%
\pgfsys@useobject{currentmarker}{}%
\end{pgfscope}%
\begin{pgfscope}%
\pgfsys@transformshift{1.266979in}{0.570602in}%
\pgfsys@useobject{currentmarker}{}%
\end{pgfscope}%
\begin{pgfscope}%
\pgfsys@transformshift{1.245379in}{0.578465in}%
\pgfsys@useobject{currentmarker}{}%
\end{pgfscope}%
\begin{pgfscope}%
\pgfsys@transformshift{1.223547in}{0.742367in}%
\pgfsys@useobject{currentmarker}{}%
\end{pgfscope}%
\begin{pgfscope}%
\pgfsys@transformshift{1.205234in}{0.943179in}%
\pgfsys@useobject{currentmarker}{}%
\end{pgfscope}%
\begin{pgfscope}%
\pgfsys@transformshift{1.187157in}{0.687668in}%
\pgfsys@useobject{currentmarker}{}%
\end{pgfscope}%
\begin{pgfscope}%
\pgfsys@transformshift{1.166731in}{0.859190in}%
\pgfsys@useobject{currentmarker}{}%
\end{pgfscope}%
\begin{pgfscope}%
\pgfsys@transformshift{1.148889in}{1.053079in}%
\pgfsys@useobject{currentmarker}{}%
\end{pgfscope}%
\begin{pgfscope}%
\pgfsys@transformshift{1.130341in}{1.308123in}%
\pgfsys@useobject{currentmarker}{}%
\end{pgfscope}%
\begin{pgfscope}%
\pgfsys@transformshift{1.110386in}{1.419895in}%
\pgfsys@useobject{currentmarker}{}%
\end{pgfscope}%
\begin{pgfscope}%
\pgfsys@transformshift{1.091370in}{1.413397in}%
\pgfsys@useobject{currentmarker}{}%
\end{pgfscope}%
\begin{pgfscope}%
\pgfsys@transformshift{1.072353in}{1.310356in}%
\pgfsys@useobject{currentmarker}{}%
\end{pgfscope}%
\begin{pgfscope}%
\pgfsys@transformshift{1.051927in}{1.059672in}%
\pgfsys@useobject{currentmarker}{}%
\end{pgfscope}%
\begin{pgfscope}%
\pgfsys@transformshift{1.033616in}{0.851138in}%
\pgfsys@useobject{currentmarker}{}%
\end{pgfscope}%
\begin{pgfscope}%
\pgfsys@transformshift{1.012251in}{0.678533in}%
\pgfsys@useobject{currentmarker}{}%
\end{pgfscope}%
\begin{pgfscope}%
\pgfsys@transformshift{0.996286in}{0.592597in}%
\pgfsys@useobject{currentmarker}{}%
\end{pgfscope}%
\begin{pgfscope}%
\pgfsys@transformshift{0.973983in}{0.527035in}%
\pgfsys@useobject{currentmarker}{}%
\end{pgfscope}%
\begin{pgfscope}%
\pgfsys@transformshift{0.955671in}{0.544335in}%
\pgfsys@useobject{currentmarker}{}%
\end{pgfscope}%
\begin{pgfscope}%
\pgfsys@transformshift{0.937829in}{0.636678in}%
\pgfsys@useobject{currentmarker}{}%
\end{pgfscope}%
\begin{pgfscope}%
\pgfsys@transformshift{0.919750in}{0.783798in}%
\pgfsys@useobject{currentmarker}{}%
\end{pgfscope}%
\begin{pgfscope}%
\pgfsys@transformshift{0.898856in}{0.995509in}%
\pgfsys@useobject{currentmarker}{}%
\end{pgfscope}%
\begin{pgfscope}%
\pgfsys@transformshift{0.881013in}{1.209260in}%
\pgfsys@useobject{currentmarker}{}%
\end{pgfscope}%
\begin{pgfscope}%
\pgfsys@transformshift{0.859179in}{1.400818in}%
\pgfsys@useobject{currentmarker}{}%
\end{pgfscope}%
\begin{pgfscope}%
\pgfsys@transformshift{0.842511in}{1.443732in}%
\pgfsys@useobject{currentmarker}{}%
\end{pgfscope}%
\begin{pgfscope}%
\pgfsys@transformshift{0.821382in}{1.394214in}%
\pgfsys@useobject{currentmarker}{}%
\end{pgfscope}%
\begin{pgfscope}%
\pgfsys@transformshift{0.803303in}{1.289916in}%
\pgfsys@useobject{currentmarker}{}%
\end{pgfscope}%
\begin{pgfscope}%
\pgfsys@transformshift{0.786400in}{1.141965in}%
\pgfsys@useobject{currentmarker}{}%
\end{pgfscope}%
\begin{pgfscope}%
\pgfsys@transformshift{0.765975in}{0.939162in}%
\pgfsys@useobject{currentmarker}{}%
\end{pgfscope}%
\begin{pgfscope}%
\pgfsys@transformshift{0.745784in}{0.772344in}%
\pgfsys@useobject{currentmarker}{}%
\end{pgfscope}%
\begin{pgfscope}%
\pgfsys@transformshift{0.728881in}{0.647862in}%
\pgfsys@useobject{currentmarker}{}%
\end{pgfscope}%
\begin{pgfscope}%
\pgfsys@transformshift{0.707987in}{0.546436in}%
\pgfsys@useobject{currentmarker}{}%
\end{pgfscope}%
\begin{pgfscope}%
\pgfsys@transformshift{0.690379in}{0.534274in}%
\pgfsys@useobject{currentmarker}{}%
\end{pgfscope}%
\begin{pgfscope}%
\pgfsys@transformshift{0.668779in}{0.605294in}%
\pgfsys@useobject{currentmarker}{}%
\end{pgfscope}%
\begin{pgfscope}%
\pgfsys@transformshift{0.651876in}{0.710348in}%
\pgfsys@useobject{currentmarker}{}%
\end{pgfscope}%
\begin{pgfscope}%
\pgfsys@transformshift{0.655163in}{0.653238in}%
\pgfsys@useobject{currentmarker}{}%
\end{pgfscope}%
\begin{pgfscope}%
\pgfsys@transformshift{0.674179in}{1.342367in}%
\pgfsys@useobject{currentmarker}{}%
\end{pgfscope}%
\begin{pgfscope}%
\pgfsys@transformshift{0.695308in}{1.445693in}%
\pgfsys@useobject{currentmarker}{}%
\end{pgfscope}%
\begin{pgfscope}%
\pgfsys@transformshift{0.714091in}{1.336915in}%
\pgfsys@useobject{currentmarker}{}%
\end{pgfscope}%
\begin{pgfscope}%
\pgfsys@transformshift{0.732402in}{1.013531in}%
\pgfsys@useobject{currentmarker}{}%
\end{pgfscope}%
\begin{pgfscope}%
\pgfsys@transformshift{0.751888in}{0.730151in}%
\pgfsys@useobject{currentmarker}{}%
\end{pgfscope}%
\begin{pgfscope}%
\pgfsys@transformshift{0.771375in}{0.560403in}%
\pgfsys@useobject{currentmarker}{}%
\end{pgfscope}%
\begin{pgfscope}%
\pgfsys@transformshift{0.789452in}{0.537750in}%
\pgfsys@useobject{currentmarker}{}%
\end{pgfscope}%
\begin{pgfscope}%
\pgfsys@transformshift{0.810347in}{0.670934in}%
\pgfsys@useobject{currentmarker}{}%
\end{pgfscope}%
\begin{pgfscope}%
\pgfsys@transformshift{0.830303in}{0.935707in}%
\pgfsys@useobject{currentmarker}{}%
\end{pgfscope}%
\begin{pgfscope}%
\pgfsys@transformshift{0.849083in}{1.261025in}%
\pgfsys@useobject{currentmarker}{}%
\end{pgfscope}%
\begin{pgfscope}%
\pgfsys@transformshift{0.868571in}{1.423178in}%
\pgfsys@useobject{currentmarker}{}%
\end{pgfscope}%
\begin{pgfscope}%
\pgfsys@transformshift{0.887822in}{1.396769in}%
\pgfsys@useobject{currentmarker}{}%
\end{pgfscope}%
\begin{pgfscope}%
\pgfsys@transformshift{0.905196in}{1.162980in}%
\pgfsys@useobject{currentmarker}{}%
\end{pgfscope}%
\begin{pgfscope}%
\pgfsys@transformshift{0.925619in}{0.828999in}%
\pgfsys@useobject{currentmarker}{}%
\end{pgfscope}%
\begin{pgfscope}%
\pgfsys@transformshift{0.944871in}{0.621601in}%
\pgfsys@useobject{currentmarker}{}%
\end{pgfscope}%
\begin{pgfscope}%
\pgfsys@transformshift{0.963887in}{0.514944in}%
\pgfsys@useobject{currentmarker}{}%
\end{pgfscope}%
\begin{pgfscope}%
\pgfsys@transformshift{0.982904in}{0.573552in}%
\pgfsys@useobject{currentmarker}{}%
\end{pgfscope}%
\begin{pgfscope}%
\pgfsys@transformshift{1.002392in}{0.729880in}%
\pgfsys@useobject{currentmarker}{}%
\end{pgfscope}%
\begin{pgfscope}%
\pgfsys@transformshift{1.021172in}{0.977885in}%
\pgfsys@useobject{currentmarker}{}%
\end{pgfscope}%
\begin{pgfscope}%
\pgfsys@transformshift{1.039251in}{1.283897in}%
\pgfsys@useobject{currentmarker}{}%
\end{pgfscope}%
\begin{pgfscope}%
\pgfsys@transformshift{1.058737in}{1.413665in}%
\pgfsys@useobject{currentmarker}{}%
\end{pgfscope}%
\begin{pgfscope}%
\pgfsys@transformshift{1.080805in}{1.324544in}%
\pgfsys@useobject{currentmarker}{}%
\end{pgfscope}%
\begin{pgfscope}%
\pgfsys@transformshift{1.099822in}{1.412822in}%
\pgfsys@useobject{currentmarker}{}%
\end{pgfscope}%
\begin{pgfscope}%
\pgfsys@transformshift{1.117430in}{1.318424in}%
\pgfsys@useobject{currentmarker}{}%
\end{pgfscope}%
\begin{pgfscope}%
\pgfsys@transformshift{1.136681in}{1.017866in}%
\pgfsys@useobject{currentmarker}{}%
\end{pgfscope}%
\begin{pgfscope}%
\pgfsys@transformshift{1.154758in}{0.728294in}%
\pgfsys@useobject{currentmarker}{}%
\end{pgfscope}%
\begin{pgfscope}%
\pgfsys@transformshift{1.176592in}{0.548778in}%
\pgfsys@useobject{currentmarker}{}%
\end{pgfscope}%
\begin{pgfscope}%
\pgfsys@transformshift{1.192321in}{0.501541in}%
\pgfsys@useobject{currentmarker}{}%
\end{pgfscope}%
\begin{pgfscope}%
\pgfsys@transformshift{1.213921in}{0.591647in}%
\pgfsys@useobject{currentmarker}{}%
\end{pgfscope}%
\begin{pgfscope}%
\pgfsys@transformshift{1.233172in}{0.750420in}%
\pgfsys@useobject{currentmarker}{}%
\end{pgfscope}%
\begin{pgfscope}%
\pgfsys@transformshift{1.252189in}{0.988629in}%
\pgfsys@useobject{currentmarker}{}%
\end{pgfscope}%
\begin{pgfscope}%
\pgfsys@transformshift{1.270971in}{1.281039in}%
\pgfsys@useobject{currentmarker}{}%
\end{pgfscope}%
\begin{pgfscope}%
\pgfsys@transformshift{1.289048in}{1.399539in}%
\pgfsys@useobject{currentmarker}{}%
\end{pgfscope}%
\begin{pgfscope}%
\pgfsys@transformshift{1.311116in}{1.330571in}%
\pgfsys@useobject{currentmarker}{}%
\end{pgfscope}%
\begin{pgfscope}%
\pgfsys@transformshift{1.329899in}{1.061309in}%
\pgfsys@useobject{currentmarker}{}%
\end{pgfscope}%
\begin{pgfscope}%
\pgfsys@transformshift{1.348915in}{0.771549in}%
\pgfsys@useobject{currentmarker}{}%
\end{pgfscope}%
\begin{pgfscope}%
\pgfsys@transformshift{1.368167in}{0.594613in}%
\pgfsys@useobject{currentmarker}{}%
\end{pgfscope}%
\begin{pgfscope}%
\pgfsys@transformshift{1.387887in}{0.513479in}%
\pgfsys@useobject{currentmarker}{}%
\end{pgfscope}%
\begin{pgfscope}%
\pgfsys@transformshift{1.405495in}{0.514744in}%
\pgfsys@useobject{currentmarker}{}%
\end{pgfscope}%
\begin{pgfscope}%
\pgfsys@transformshift{1.424980in}{0.614759in}%
\pgfsys@useobject{currentmarker}{}%
\end{pgfscope}%
\begin{pgfscope}%
\pgfsys@transformshift{1.442354in}{0.746258in}%
\pgfsys@useobject{currentmarker}{}%
\end{pgfscope}%
\begin{pgfscope}%
\pgfsys@transformshift{1.465128in}{1.028052in}%
\pgfsys@useobject{currentmarker}{}%
\end{pgfscope}%
\begin{pgfscope}%
\pgfsys@transformshift{1.482736in}{1.284831in}%
\pgfsys@useobject{currentmarker}{}%
\end{pgfscope}%
\begin{pgfscope}%
\pgfsys@transformshift{1.501282in}{1.391272in}%
\pgfsys@useobject{currentmarker}{}%
\end{pgfscope}%
\begin{pgfscope}%
\pgfsys@transformshift{1.523116in}{1.356925in}%
\pgfsys@useobject{currentmarker}{}%
\end{pgfscope}%
\begin{pgfscope}%
\pgfsys@transformshift{1.542602in}{1.146748in}%
\pgfsys@useobject{currentmarker}{}%
\end{pgfscope}%
\begin{pgfscope}%
\pgfsys@transformshift{1.560210in}{0.835521in}%
\pgfsys@useobject{currentmarker}{}%
\end{pgfscope}%
\begin{pgfscope}%
\pgfsys@transformshift{1.578521in}{0.638814in}%
\pgfsys@useobject{currentmarker}{}%
\end{pgfscope}%
\begin{pgfscope}%
\pgfsys@transformshift{1.600355in}{0.518760in}%
\pgfsys@useobject{currentmarker}{}%
\end{pgfscope}%
\begin{pgfscope}%
\pgfsys@transformshift{1.618434in}{0.498928in}%
\pgfsys@useobject{currentmarker}{}%
\end{pgfscope}%
\begin{pgfscope}%
\pgfsys@transformshift{1.634163in}{0.534201in}%
\pgfsys@useobject{currentmarker}{}%
\end{pgfscope}%
\begin{pgfscope}%
\pgfsys@transformshift{1.657406in}{0.681269in}%
\pgfsys@useobject{currentmarker}{}%
\end{pgfscope}%
\begin{pgfscope}%
\pgfsys@transformshift{1.673370in}{0.847281in}%
\pgfsys@useobject{currentmarker}{}%
\end{pgfscope}%
\begin{pgfscope}%
\pgfsys@transformshift{1.694499in}{1.161855in}%
\pgfsys@useobject{currentmarker}{}%
\end{pgfscope}%
\begin{pgfscope}%
\pgfsys@transformshift{1.713282in}{1.354125in}%
\pgfsys@useobject{currentmarker}{}%
\end{pgfscope}%
\begin{pgfscope}%
\pgfsys@transformshift{1.736525in}{1.381553in}%
\pgfsys@useobject{currentmarker}{}%
\end{pgfscope}%
\begin{pgfscope}%
\pgfsys@transformshift{1.753427in}{1.295229in}%
\pgfsys@useobject{currentmarker}{}%
\end{pgfscope}%
\begin{pgfscope}%
\pgfsys@transformshift{1.770566in}{1.049402in}%
\pgfsys@useobject{currentmarker}{}%
\end{pgfscope}%
\begin{pgfscope}%
\pgfsys@transformshift{1.788409in}{0.935304in}%
\pgfsys@useobject{currentmarker}{}%
\end{pgfscope}%
\begin{pgfscope}%
\pgfsys@transformshift{1.809303in}{0.732018in}%
\pgfsys@useobject{currentmarker}{}%
\end{pgfscope}%
\begin{pgfscope}%
\pgfsys@transformshift{1.827380in}{0.615406in}%
\pgfsys@useobject{currentmarker}{}%
\end{pgfscope}%
\begin{pgfscope}%
\pgfsys@transformshift{1.845694in}{0.512073in}%
\pgfsys@useobject{currentmarker}{}%
\end{pgfscope}%
\begin{pgfscope}%
\pgfsys@transformshift{1.867762in}{0.498444in}%
\pgfsys@useobject{currentmarker}{}%
\end{pgfscope}%
\begin{pgfscope}%
\pgfsys@transformshift{1.884430in}{0.561051in}%
\pgfsys@useobject{currentmarker}{}%
\end{pgfscope}%
\begin{pgfscope}%
\pgfsys@transformshift{1.904621in}{0.681650in}%
\pgfsys@useobject{currentmarker}{}%
\end{pgfscope}%
\begin{pgfscope}%
\pgfsys@transformshift{1.923167in}{0.809657in}%
\pgfsys@useobject{currentmarker}{}%
\end{pgfscope}%
\begin{pgfscope}%
\pgfsys@transformshift{1.944533in}{1.118957in}%
\pgfsys@useobject{currentmarker}{}%
\end{pgfscope}%
\begin{pgfscope}%
\pgfsys@transformshift{1.961435in}{1.211505in}%
\pgfsys@useobject{currentmarker}{}%
\end{pgfscope}%
\begin{pgfscope}%
\pgfsys@transformshift{1.982566in}{1.375166in}%
\pgfsys@useobject{currentmarker}{}%
\end{pgfscope}%
\begin{pgfscope}%
\pgfsys@transformshift{2.000643in}{1.351492in}%
\pgfsys@useobject{currentmarker}{}%
\end{pgfscope}%
\begin{pgfscope}%
\pgfsys@transformshift{2.021068in}{1.133206in}%
\pgfsys@useobject{currentmarker}{}%
\end{pgfscope}%
\begin{pgfscope}%
\pgfsys@transformshift{2.038911in}{0.888490in}%
\pgfsys@useobject{currentmarker}{}%
\end{pgfscope}%
\begin{pgfscope}%
\pgfsys@transformshift{2.057693in}{0.677508in}%
\pgfsys@useobject{currentmarker}{}%
\end{pgfscope}%
\begin{pgfscope}%
\pgfsys@transformshift{2.078353in}{0.541653in}%
\pgfsys@useobject{currentmarker}{}%
\end{pgfscope}%
\begin{pgfscope}%
\pgfsys@transformshift{2.097134in}{0.488099in}%
\pgfsys@useobject{currentmarker}{}%
\end{pgfscope}%
\begin{pgfscope}%
\pgfsys@transformshift{2.117559in}{1.328354in}%
\pgfsys@useobject{currentmarker}{}%
\end{pgfscope}%
\begin{pgfscope}%
\pgfsys@transformshift{2.136810in}{1.145283in}%
\pgfsys@useobject{currentmarker}{}%
\end{pgfscope}%
\begin{pgfscope}%
\pgfsys@transformshift{2.156298in}{0.826025in}%
\pgfsys@useobject{currentmarker}{}%
\end{pgfscope}%
\begin{pgfscope}%
\pgfsys@transformshift{2.174140in}{0.633165in}%
\pgfsys@useobject{currentmarker}{}%
\end{pgfscope}%
\begin{pgfscope}%
\pgfsys@transformshift{2.195269in}{0.515668in}%
\pgfsys@useobject{currentmarker}{}%
\end{pgfscope}%
\begin{pgfscope}%
\pgfsys@transformshift{2.213817in}{0.496334in}%
\pgfsys@useobject{currentmarker}{}%
\end{pgfscope}%
\begin{pgfscope}%
\pgfsys@transformshift{2.230485in}{0.579213in}%
\pgfsys@useobject{currentmarker}{}%
\end{pgfscope}%
\begin{pgfscope}%
\pgfsys@transformshift{2.251614in}{0.747805in}%
\pgfsys@useobject{currentmarker}{}%
\end{pgfscope}%
\begin{pgfscope}%
\pgfsys@transformshift{2.269222in}{0.983740in}%
\pgfsys@useobject{currentmarker}{}%
\end{pgfscope}%
\begin{pgfscope}%
\pgfsys@transformshift{2.293874in}{1.321288in}%
\pgfsys@useobject{currentmarker}{}%
\end{pgfscope}%
\begin{pgfscope}%
\pgfsys@transformshift{2.309368in}{1.380577in}%
\pgfsys@useobject{currentmarker}{}%
\end{pgfscope}%
\begin{pgfscope}%
\pgfsys@transformshift{2.328384in}{1.313099in}%
\pgfsys@useobject{currentmarker}{}%
\end{pgfscope}%
\begin{pgfscope}%
\pgfsys@transformshift{2.349281in}{1.164838in}%
\pgfsys@useobject{currentmarker}{}%
\end{pgfscope}%
\begin{pgfscope}%
\pgfsys@transformshift{2.367827in}{0.882567in}%
\pgfsys@useobject{currentmarker}{}%
\end{pgfscope}%
\begin{pgfscope}%
\pgfsys@transformshift{2.388487in}{0.679786in}%
\pgfsys@useobject{currentmarker}{}%
\end{pgfscope}%
\begin{pgfscope}%
\pgfsys@transformshift{2.407034in}{0.542561in}%
\pgfsys@useobject{currentmarker}{}%
\end{pgfscope}%
\begin{pgfscope}%
\pgfsys@transformshift{2.422999in}{0.488408in}%
\pgfsys@useobject{currentmarker}{}%
\end{pgfscope}%
\begin{pgfscope}%
\pgfsys@transformshift{2.444128in}{0.528715in}%
\pgfsys@useobject{currentmarker}{}%
\end{pgfscope}%
\begin{pgfscope}%
\pgfsys@transformshift{2.461736in}{0.647576in}%
\pgfsys@useobject{currentmarker}{}%
\end{pgfscope}%
\begin{pgfscope}%
\pgfsys@transformshift{2.480753in}{0.817108in}%
\pgfsys@useobject{currentmarker}{}%
\end{pgfscope}%
\begin{pgfscope}%
\pgfsys@transformshift{2.503996in}{1.164108in}%
\pgfsys@useobject{currentmarker}{}%
\end{pgfscope}%
\begin{pgfscope}%
\pgfsys@transformshift{2.522542in}{1.345895in}%
\pgfsys@useobject{currentmarker}{}%
\end{pgfscope}%
\begin{pgfscope}%
\pgfsys@transformshift{2.521604in}{1.374143in}%
\pgfsys@useobject{currentmarker}{}%
\end{pgfscope}%
\begin{pgfscope}%
\pgfsys@transformshift{2.539210in}{1.378729in}%
\pgfsys@useobject{currentmarker}{}%
\end{pgfscope}%
\begin{pgfscope}%
\pgfsys@transformshift{2.560106in}{1.242815in}%
\pgfsys@useobject{currentmarker}{}%
\end{pgfscope}%
\begin{pgfscope}%
\pgfsys@transformshift{2.578183in}{0.961691in}%
\pgfsys@useobject{currentmarker}{}%
\end{pgfscope}%
\begin{pgfscope}%
\pgfsys@transformshift{2.595791in}{0.857016in}%
\pgfsys@useobject{currentmarker}{}%
\end{pgfscope}%
\begin{pgfscope}%
\pgfsys@transformshift{2.616920in}{0.634028in}%
\pgfsys@useobject{currentmarker}{}%
\end{pgfscope}%
\begin{pgfscope}%
\pgfsys@transformshift{2.634294in}{0.532867in}%
\pgfsys@useobject{currentmarker}{}%
\end{pgfscope}%
\begin{pgfscope}%
\pgfsys@transformshift{2.652605in}{0.486770in}%
\pgfsys@useobject{currentmarker}{}%
\end{pgfscope}%
\begin{pgfscope}%
\pgfsys@transformshift{2.673501in}{0.560526in}%
\pgfsys@useobject{currentmarker}{}%
\end{pgfscope}%
\begin{pgfscope}%
\pgfsys@transformshift{2.692282in}{0.681783in}%
\pgfsys@useobject{currentmarker}{}%
\end{pgfscope}%
\begin{pgfscope}%
\pgfsys@transformshift{2.712707in}{0.896847in}%
\pgfsys@useobject{currentmarker}{}%
\end{pgfscope}%
\begin{pgfscope}%
\pgfsys@transformshift{2.731255in}{1.147694in}%
\pgfsys@useobject{currentmarker}{}%
\end{pgfscope}%
\begin{pgfscope}%
\pgfsys@transformshift{2.749098in}{1.328583in}%
\pgfsys@useobject{currentmarker}{}%
\end{pgfscope}%
\begin{pgfscope}%
\pgfsys@transformshift{2.769289in}{1.375384in}%
\pgfsys@useobject{currentmarker}{}%
\end{pgfscope}%
\begin{pgfscope}%
\pgfsys@transformshift{2.791121in}{1.223613in}%
\pgfsys@useobject{currentmarker}{}%
\end{pgfscope}%
\begin{pgfscope}%
\pgfsys@transformshift{2.808729in}{0.996477in}%
\pgfsys@useobject{currentmarker}{}%
\end{pgfscope}%
\begin{pgfscope}%
\pgfsys@transformshift{2.826808in}{0.735558in}%
\pgfsys@useobject{currentmarker}{}%
\end{pgfscope}%
\begin{pgfscope}%
\pgfsys@transformshift{2.847233in}{0.588527in}%
\pgfsys@useobject{currentmarker}{}%
\end{pgfscope}%
\begin{pgfscope}%
\pgfsys@transformshift{2.865310in}{0.520431in}%
\pgfsys@useobject{currentmarker}{}%
\end{pgfscope}%
\begin{pgfscope}%
\pgfsys@transformshift{2.884561in}{0.493999in}%
\pgfsys@useobject{currentmarker}{}%
\end{pgfscope}%
\begin{pgfscope}%
\pgfsys@transformshift{2.904516in}{0.575353in}%
\pgfsys@useobject{currentmarker}{}%
\end{pgfscope}%
\begin{pgfscope}%
\pgfsys@transformshift{2.922829in}{0.689842in}%
\pgfsys@useobject{currentmarker}{}%
\end{pgfscope}%
\begin{pgfscope}%
\pgfsys@transformshift{2.943255in}{0.877598in}%
\pgfsys@useobject{currentmarker}{}%
\end{pgfscope}%
\begin{pgfscope}%
\pgfsys@transformshift{2.964149in}{1.083212in}%
\pgfsys@useobject{currentmarker}{}%
\end{pgfscope}%
\begin{pgfscope}%
\pgfsys@transformshift{2.979409in}{1.287092in}%
\pgfsys@useobject{currentmarker}{}%
\end{pgfscope}%
\begin{pgfscope}%
\pgfsys@transformshift{3.000774in}{1.385144in}%
\pgfsys@useobject{currentmarker}{}%
\end{pgfscope}%
\begin{pgfscope}%
\pgfsys@transformshift{3.021434in}{1.339523in}%
\pgfsys@useobject{currentmarker}{}%
\end{pgfscope}%
\begin{pgfscope}%
\pgfsys@transformshift{3.039511in}{1.145237in}%
\pgfsys@useobject{currentmarker}{}%
\end{pgfscope}%
\begin{pgfscope}%
\pgfsys@transformshift{3.057353in}{0.890601in}%
\pgfsys@useobject{currentmarker}{}%
\end{pgfscope}%
\begin{pgfscope}%
\pgfsys@transformshift{3.078013in}{0.774744in}%
\pgfsys@useobject{currentmarker}{}%
\end{pgfscope}%
\begin{pgfscope}%
\pgfsys@transformshift{3.097264in}{0.616156in}%
\pgfsys@useobject{currentmarker}{}%
\end{pgfscope}%
\begin{pgfscope}%
\pgfsys@transformshift{3.116986in}{0.519873in}%
\pgfsys@useobject{currentmarker}{}%
\end{pgfscope}%
\begin{pgfscope}%
\pgfsys@transformshift{3.135767in}{0.501251in}%
\pgfsys@useobject{currentmarker}{}%
\end{pgfscope}%
\begin{pgfscope}%
\pgfsys@transformshift{3.153609in}{0.576560in}%
\pgfsys@useobject{currentmarker}{}%
\end{pgfscope}%
\begin{pgfscope}%
\pgfsys@transformshift{3.174740in}{0.721302in}%
\pgfsys@useobject{currentmarker}{}%
\end{pgfscope}%
\begin{pgfscope}%
\pgfsys@transformshift{3.193522in}{0.810784in}%
\pgfsys@useobject{currentmarker}{}%
\end{pgfscope}%
\begin{pgfscope}%
\pgfsys@transformshift{3.211128in}{1.047504in}%
\pgfsys@useobject{currentmarker}{}%
\end{pgfscope}%
\begin{pgfscope}%
\pgfsys@transformshift{3.231788in}{1.302435in}%
\pgfsys@useobject{currentmarker}{}%
\end{pgfscope}%
\begin{pgfscope}%
\pgfsys@transformshift{3.250102in}{1.385148in}%
\pgfsys@useobject{currentmarker}{}%
\end{pgfscope}%
\begin{pgfscope}%
\pgfsys@transformshift{3.270056in}{1.360539in}%
\pgfsys@useobject{currentmarker}{}%
\end{pgfscope}%
\begin{pgfscope}%
\pgfsys@transformshift{3.286492in}{1.188214in}%
\pgfsys@useobject{currentmarker}{}%
\end{pgfscope}%
\begin{pgfscope}%
\pgfsys@transformshift{3.308561in}{0.988849in}%
\pgfsys@useobject{currentmarker}{}%
\end{pgfscope}%
\begin{pgfscope}%
\pgfsys@transformshift{3.324760in}{0.812801in}%
\pgfsys@useobject{currentmarker}{}%
\end{pgfscope}%
\begin{pgfscope}%
\pgfsys@transformshift{3.345420in}{0.628641in}%
\pgfsys@useobject{currentmarker}{}%
\end{pgfscope}%
\begin{pgfscope}%
\pgfsys@transformshift{3.366783in}{0.523012in}%
\pgfsys@useobject{currentmarker}{}%
\end{pgfscope}%
\begin{pgfscope}%
\pgfsys@transformshift{3.386503in}{0.501100in}%
\pgfsys@useobject{currentmarker}{}%
\end{pgfscope}%
\begin{pgfscope}%
\pgfsys@transformshift{3.402939in}{0.568523in}%
\pgfsys@useobject{currentmarker}{}%
\end{pgfscope}%
\begin{pgfscope}%
\pgfsys@transformshift{3.424068in}{0.671337in}%
\pgfsys@useobject{currentmarker}{}%
\end{pgfscope}%
\begin{pgfscope}%
\pgfsys@transformshift{3.440971in}{0.791043in}%
\pgfsys@useobject{currentmarker}{}%
\end{pgfscope}%
\begin{pgfscope}%
\pgfsys@transformshift{3.461631in}{0.989081in}%
\pgfsys@useobject{currentmarker}{}%
\end{pgfscope}%
\begin{pgfscope}%
\pgfsys@transformshift{3.480884in}{1.232697in}%
\pgfsys@useobject{currentmarker}{}%
\end{pgfscope}%
\begin{pgfscope}%
\pgfsys@transformshift{3.498726in}{1.369386in}%
\pgfsys@useobject{currentmarker}{}%
\end{pgfscope}%
\begin{pgfscope}%
\pgfsys@transformshift{3.519152in}{1.405496in}%
\pgfsys@useobject{currentmarker}{}%
\end{pgfscope}%
\begin{pgfscope}%
\pgfsys@transformshift{3.538403in}{1.351851in}%
\pgfsys@useobject{currentmarker}{}%
\end{pgfscope}%
\begin{pgfscope}%
\pgfsys@transformshift{3.560235in}{1.131121in}%
\pgfsys@useobject{currentmarker}{}%
\end{pgfscope}%
\begin{pgfscope}%
\pgfsys@transformshift{3.575966in}{0.967524in}%
\pgfsys@useobject{currentmarker}{}%
\end{pgfscope}%
\begin{pgfscope}%
\pgfsys@transformshift{3.596860in}{0.751021in}%
\pgfsys@useobject{currentmarker}{}%
\end{pgfscope}%
\begin{pgfscope}%
\pgfsys@transformshift{3.615408in}{1.399332in}%
\pgfsys@useobject{currentmarker}{}%
\end{pgfscope}%
\begin{pgfscope}%
\pgfsys@transformshift{3.634190in}{1.395179in}%
\pgfsys@useobject{currentmarker}{}%
\end{pgfscope}%
\begin{pgfscope}%
\pgfsys@transformshift{3.652736in}{1.267330in}%
\pgfsys@useobject{currentmarker}{}%
\end{pgfscope}%
\begin{pgfscope}%
\pgfsys@transformshift{3.674101in}{0.952363in}%
\pgfsys@useobject{currentmarker}{}%
\end{pgfscope}%
\begin{pgfscope}%
\pgfsys@transformshift{3.694761in}{0.745282in}%
\pgfsys@useobject{currentmarker}{}%
\end{pgfscope}%
\begin{pgfscope}%
\pgfsys@transformshift{3.711898in}{0.614447in}%
\pgfsys@useobject{currentmarker}{}%
\end{pgfscope}%
\begin{pgfscope}%
\pgfsys@transformshift{3.729741in}{0.522135in}%
\pgfsys@useobject{currentmarker}{}%
\end{pgfscope}%
\begin{pgfscope}%
\pgfsys@transformshift{3.749697in}{0.530375in}%
\pgfsys@useobject{currentmarker}{}%
\end{pgfscope}%
\begin{pgfscope}%
\pgfsys@transformshift{3.768009in}{0.632389in}%
\pgfsys@useobject{currentmarker}{}%
\end{pgfscope}%
\begin{pgfscope}%
\pgfsys@transformshift{3.787025in}{0.754459in}%
\pgfsys@useobject{currentmarker}{}%
\end{pgfscope}%
\begin{pgfscope}%
\pgfsys@transformshift{3.807451in}{0.980005in}%
\pgfsys@useobject{currentmarker}{}%
\end{pgfscope}%
\begin{pgfscope}%
\pgfsys@transformshift{3.825059in}{1.218607in}%
\pgfsys@useobject{currentmarker}{}%
\end{pgfscope}%
\begin{pgfscope}%
\pgfsys@transformshift{3.846659in}{1.379476in}%
\pgfsys@useobject{currentmarker}{}%
\end{pgfscope}%
\begin{pgfscope}%
\pgfsys@transformshift{3.864736in}{1.422091in}%
\pgfsys@useobject{currentmarker}{}%
\end{pgfscope}%
\begin{pgfscope}%
\pgfsys@transformshift{3.885161in}{1.344936in}%
\pgfsys@useobject{currentmarker}{}%
\end{pgfscope}%
\begin{pgfscope}%
\pgfsys@transformshift{3.904178in}{1.163386in}%
\pgfsys@useobject{currentmarker}{}%
\end{pgfscope}%
\begin{pgfscope}%
\pgfsys@transformshift{3.925072in}{0.940774in}%
\pgfsys@useobject{currentmarker}{}%
\end{pgfscope}%
\begin{pgfscope}%
\pgfsys@transformshift{3.942680in}{0.743394in}%
\pgfsys@useobject{currentmarker}{}%
\end{pgfscope}%
\begin{pgfscope}%
\pgfsys@transformshift{3.962871in}{0.601864in}%
\pgfsys@useobject{currentmarker}{}%
\end{pgfscope}%
\begin{pgfscope}%
\pgfsys@transformshift{3.980714in}{0.533773in}%
\pgfsys@useobject{currentmarker}{}%
\end{pgfscope}%
\begin{pgfscope}%
\pgfsys@transformshift{3.999965in}{0.520803in}%
\pgfsys@useobject{currentmarker}{}%
\end{pgfscope}%
\begin{pgfscope}%
\pgfsys@transformshift{4.016868in}{0.582500in}%
\pgfsys@useobject{currentmarker}{}%
\end{pgfscope}%
\begin{pgfscope}%
\pgfsys@transformshift{4.040345in}{0.737829in}%
\pgfsys@useobject{currentmarker}{}%
\end{pgfscope}%
\begin{pgfscope}%
\pgfsys@transformshift{4.058188in}{0.862445in}%
\pgfsys@useobject{currentmarker}{}%
\end{pgfscope}%
\begin{pgfscope}%
\pgfsys@transformshift{4.081196in}{1.157188in}%
\pgfsys@useobject{currentmarker}{}%
\end{pgfscope}%
\begin{pgfscope}%
\pgfsys@transformshift{4.098570in}{1.334832in}%
\pgfsys@useobject{currentmarker}{}%
\end{pgfscope}%
\begin{pgfscope}%
\pgfsys@transformshift{4.114534in}{1.425973in}%
\pgfsys@useobject{currentmarker}{}%
\end{pgfscope}%
\begin{pgfscope}%
\pgfsys@transformshift{4.135429in}{1.315365in}%
\pgfsys@useobject{currentmarker}{}%
\end{pgfscope}%
\begin{pgfscope}%
\pgfsys@transformshift{4.154446in}{1.364931in}%
\pgfsys@useobject{currentmarker}{}%
\end{pgfscope}%
\begin{pgfscope}%
\pgfsys@transformshift{4.174635in}{1.440342in}%
\pgfsys@useobject{currentmarker}{}%
\end{pgfscope}%
\begin{pgfscope}%
\pgfsys@transformshift{4.195295in}{1.379994in}%
\pgfsys@useobject{currentmarker}{}%
\end{pgfscope}%
\begin{pgfscope}%
\pgfsys@transformshift{4.211025in}{1.254603in}%
\pgfsys@useobject{currentmarker}{}%
\end{pgfscope}%
\begin{pgfscope}%
\pgfsys@transformshift{4.228867in}{1.005091in}%
\pgfsys@useobject{currentmarker}{}%
\end{pgfscope}%
\begin{pgfscope}%
\pgfsys@transformshift{4.249762in}{0.793726in}%
\pgfsys@useobject{currentmarker}{}%
\end{pgfscope}%
\begin{pgfscope}%
\pgfsys@transformshift{4.267604in}{0.654560in}%
\pgfsys@useobject{currentmarker}{}%
\end{pgfscope}%
\begin{pgfscope}%
\pgfsys@transformshift{4.292021in}{0.540117in}%
\pgfsys@useobject{currentmarker}{}%
\end{pgfscope}%
\begin{pgfscope}%
\pgfsys@transformshift{4.307281in}{0.532228in}%
\pgfsys@useobject{currentmarker}{}%
\end{pgfscope}%
\begin{pgfscope}%
\pgfsys@transformshift{4.326769in}{0.629305in}%
\pgfsys@useobject{currentmarker}{}%
\end{pgfscope}%
\begin{pgfscope}%
\pgfsys@transformshift{4.343202in}{0.754533in}%
\pgfsys@useobject{currentmarker}{}%
\end{pgfscope}%
\begin{pgfscope}%
\pgfsys@transformshift{4.366680in}{0.983465in}%
\pgfsys@useobject{currentmarker}{}%
\end{pgfscope}%
\begin{pgfscope}%
\pgfsys@transformshift{4.383817in}{1.156353in}%
\pgfsys@useobject{currentmarker}{}%
\end{pgfscope}%
\begin{pgfscope}%
\pgfsys@transformshift{4.401425in}{1.368953in}%
\pgfsys@useobject{currentmarker}{}%
\end{pgfscope}%
\begin{pgfscope}%
\pgfsys@transformshift{4.421381in}{1.450075in}%
\pgfsys@useobject{currentmarker}{}%
\end{pgfscope}%
\begin{pgfscope}%
\pgfsys@transformshift{4.444624in}{1.431291in}%
\pgfsys@useobject{currentmarker}{}%
\end{pgfscope}%
\begin{pgfscope}%
\pgfsys@transformshift{4.462232in}{1.295183in}%
\pgfsys@useobject{currentmarker}{}%
\end{pgfscope}%
\begin{pgfscope}%
\pgfsys@transformshift{4.481013in}{1.129165in}%
\pgfsys@useobject{currentmarker}{}%
\end{pgfscope}%
\begin{pgfscope}%
\pgfsys@transformshift{4.482421in}{1.142825in}%
\pgfsys@useobject{currentmarker}{}%
\end{pgfscope}%
\begin{pgfscope}%
\pgfsys@transformshift{4.473266in}{1.278363in}%
\pgfsys@useobject{currentmarker}{}%
\end{pgfscope}%
\begin{pgfscope}%
\pgfsys@transformshift{4.455892in}{1.437995in}%
\pgfsys@useobject{currentmarker}{}%
\end{pgfscope}%
\begin{pgfscope}%
\pgfsys@transformshift{4.436875in}{1.428658in}%
\pgfsys@useobject{currentmarker}{}%
\end{pgfscope}%
\begin{pgfscope}%
\pgfsys@transformshift{4.419267in}{1.155250in}%
\pgfsys@useobject{currentmarker}{}%
\end{pgfscope}%
\begin{pgfscope}%
\pgfsys@transformshift{4.396495in}{0.862359in}%
\pgfsys@useobject{currentmarker}{}%
\end{pgfscope}%
\begin{pgfscope}%
\pgfsys@transformshift{4.376539in}{0.640585in}%
\pgfsys@useobject{currentmarker}{}%
\end{pgfscope}%
\begin{pgfscope}%
\pgfsys@transformshift{4.360340in}{0.543316in}%
\pgfsys@useobject{currentmarker}{}%
\end{pgfscope}%
\begin{pgfscope}%
\pgfsys@transformshift{4.340150in}{0.584706in}%
\pgfsys@useobject{currentmarker}{}%
\end{pgfscope}%
\begin{pgfscope}%
\pgfsys@transformshift{4.321837in}{0.733718in}%
\pgfsys@useobject{currentmarker}{}%
\end{pgfscope}%
\begin{pgfscope}%
\pgfsys@transformshift{4.301882in}{1.044067in}%
\pgfsys@useobject{currentmarker}{}%
\end{pgfscope}%
\begin{pgfscope}%
\pgfsys@transformshift{4.282160in}{1.351961in}%
\pgfsys@useobject{currentmarker}{}%
\end{pgfscope}%
\begin{pgfscope}%
\pgfsys@transformshift{4.261266in}{1.440033in}%
\pgfsys@useobject{currentmarker}{}%
\end{pgfscope}%
\begin{pgfscope}%
\pgfsys@transformshift{4.242955in}{1.340463in}%
\pgfsys@useobject{currentmarker}{}%
\end{pgfscope}%
\begin{pgfscope}%
\pgfsys@transformshift{4.223233in}{0.996599in}%
\pgfsys@useobject{currentmarker}{}%
\end{pgfscope}%
\begin{pgfscope}%
\pgfsys@transformshift{4.204921in}{0.754373in}%
\pgfsys@useobject{currentmarker}{}%
\end{pgfscope}%
\begin{pgfscope}%
\pgfsys@transformshift{4.188253in}{0.597487in}%
\pgfsys@useobject{currentmarker}{}%
\end{pgfscope}%
\begin{pgfscope}%
\pgfsys@transformshift{4.167122in}{0.515026in}%
\pgfsys@useobject{currentmarker}{}%
\end{pgfscope}%
\begin{pgfscope}%
\pgfsys@transformshift{4.148811in}{0.623908in}%
\pgfsys@useobject{currentmarker}{}%
\end{pgfscope}%
\begin{pgfscope}%
\pgfsys@transformshift{4.126976in}{0.839584in}%
\pgfsys@useobject{currentmarker}{}%
\end{pgfscope}%
\begin{pgfscope}%
\pgfsys@transformshift{4.109837in}{1.135686in}%
\pgfsys@useobject{currentmarker}{}%
\end{pgfscope}%
\begin{pgfscope}%
\pgfsys@transformshift{4.088474in}{1.389811in}%
\pgfsys@useobject{currentmarker}{}%
\end{pgfscope}%
\begin{pgfscope}%
\pgfsys@transformshift{4.070397in}{1.416910in}%
\pgfsys@useobject{currentmarker}{}%
\end{pgfscope}%
\begin{pgfscope}%
\pgfsys@transformshift{4.052084in}{1.271894in}%
\pgfsys@useobject{currentmarker}{}%
\end{pgfscope}%
\begin{pgfscope}%
\pgfsys@transformshift{4.030250in}{0.878259in}%
\pgfsys@useobject{currentmarker}{}%
\end{pgfscope}%
\begin{pgfscope}%
\pgfsys@transformshift{4.014521in}{0.671748in}%
\pgfsys@useobject{currentmarker}{}%
\end{pgfscope}%
\begin{pgfscope}%
\pgfsys@transformshift{3.992218in}{0.530168in}%
\pgfsys@useobject{currentmarker}{}%
\end{pgfscope}%
\begin{pgfscope}%
\pgfsys@transformshift{3.973670in}{0.529278in}%
\pgfsys@useobject{currentmarker}{}%
\end{pgfscope}%
\begin{pgfscope}%
\pgfsys@transformshift{3.955828in}{0.654141in}%
\pgfsys@useobject{currentmarker}{}%
\end{pgfscope}%
\begin{pgfscope}%
\pgfsys@transformshift{3.933759in}{0.893635in}%
\pgfsys@useobject{currentmarker}{}%
\end{pgfscope}%
\begin{pgfscope}%
\pgfsys@transformshift{3.915917in}{1.208304in}%
\pgfsys@useobject{currentmarker}{}%
\end{pgfscope}%
\begin{pgfscope}%
\pgfsys@transformshift{3.895726in}{1.399763in}%
\pgfsys@useobject{currentmarker}{}%
\end{pgfscope}%
\begin{pgfscope}%
\pgfsys@transformshift{3.882109in}{1.390370in}%
\pgfsys@useobject{currentmarker}{}%
\end{pgfscope}%
\begin{pgfscope}%
\pgfsys@transformshift{3.860041in}{1.208980in}%
\pgfsys@useobject{currentmarker}{}%
\end{pgfscope}%
\begin{pgfscope}%
\pgfsys@transformshift{3.839615in}{0.861707in}%
\pgfsys@useobject{currentmarker}{}%
\end{pgfscope}%
\begin{pgfscope}%
\pgfsys@transformshift{3.821538in}{0.655848in}%
\pgfsys@useobject{currentmarker}{}%
\end{pgfscope}%
\begin{pgfscope}%
\pgfsys@transformshift{3.801816in}{0.528467in}%
\pgfsys@useobject{currentmarker}{}%
\end{pgfscope}%
\begin{pgfscope}%
\pgfsys@transformshift{3.782799in}{0.507334in}%
\pgfsys@useobject{currentmarker}{}%
\end{pgfscope}%
\begin{pgfscope}%
\pgfsys@transformshift{3.762845in}{0.628465in}%
\pgfsys@useobject{currentmarker}{}%
\end{pgfscope}%
\begin{pgfscope}%
\pgfsys@transformshift{3.745471in}{0.815919in}%
\pgfsys@useobject{currentmarker}{}%
\end{pgfscope}%
\begin{pgfscope}%
\pgfsys@transformshift{3.725046in}{1.135099in}%
\pgfsys@useobject{currentmarker}{}%
\end{pgfscope}%
\begin{pgfscope}%
\pgfsys@transformshift{3.703448in}{1.311001in}%
\pgfsys@useobject{currentmarker}{}%
\end{pgfscope}%
\begin{pgfscope}%
\pgfsys@transformshift{3.686074in}{1.400660in}%
\pgfsys@useobject{currentmarker}{}%
\end{pgfscope}%
\begin{pgfscope}%
\pgfsys@transformshift{3.667527in}{1.322898in}%
\pgfsys@useobject{currentmarker}{}%
\end{pgfscope}%
\begin{pgfscope}%
\pgfsys@transformshift{3.646632in}{1.077300in}%
\pgfsys@useobject{currentmarker}{}%
\end{pgfscope}%
\begin{pgfscope}%
\pgfsys@transformshift{3.629493in}{0.809468in}%
\pgfsys@useobject{currentmarker}{}%
\end{pgfscope}%
\begin{pgfscope}%
\pgfsys@transformshift{3.612825in}{0.637036in}%
\pgfsys@useobject{currentmarker}{}%
\end{pgfscope}%
\begin{pgfscope}%
\pgfsys@transformshift{3.588644in}{0.514689in}%
\pgfsys@useobject{currentmarker}{}%
\end{pgfscope}%
\begin{pgfscope}%
\pgfsys@transformshift{3.570565in}{0.507796in}%
\pgfsys@useobject{currentmarker}{}%
\end{pgfscope}%
\begin{pgfscope}%
\pgfsys@transformshift{3.551550in}{0.605591in}%
\pgfsys@useobject{currentmarker}{}%
\end{pgfscope}%
\begin{pgfscope}%
\pgfsys@transformshift{3.533237in}{0.762820in}%
\pgfsys@useobject{currentmarker}{}%
\end{pgfscope}%
\begin{pgfscope}%
\pgfsys@transformshift{3.512577in}{1.054045in}%
\pgfsys@useobject{currentmarker}{}%
\end{pgfscope}%
\begin{pgfscope}%
\pgfsys@transformshift{3.494969in}{1.228290in}%
\pgfsys@useobject{currentmarker}{}%
\end{pgfscope}%
\begin{pgfscope}%
\pgfsys@transformshift{3.473606in}{1.387794in}%
\pgfsys@useobject{currentmarker}{}%
\end{pgfscope}%
\begin{pgfscope}%
\pgfsys@transformshift{3.457406in}{1.380395in}%
\pgfsys@useobject{currentmarker}{}%
\end{pgfscope}%
\begin{pgfscope}%
\pgfsys@transformshift{3.436041in}{1.200905in}%
\pgfsys@useobject{currentmarker}{}%
\end{pgfscope}%
\begin{pgfscope}%
\pgfsys@transformshift{3.417495in}{0.930891in}%
\pgfsys@useobject{currentmarker}{}%
\end{pgfscope}%
\begin{pgfscope}%
\pgfsys@transformshift{3.398713in}{0.713908in}%
\pgfsys@useobject{currentmarker}{}%
\end{pgfscope}%
\begin{pgfscope}%
\pgfsys@transformshift{3.379462in}{0.583441in}%
\pgfsys@useobject{currentmarker}{}%
\end{pgfscope}%
\begin{pgfscope}%
\pgfsys@transformshift{3.358097in}{0.493006in}%
\pgfsys@useobject{currentmarker}{}%
\end{pgfscope}%
\begin{pgfscope}%
\pgfsys@transformshift{3.340723in}{1.240339in}%
\pgfsys@useobject{currentmarker}{}%
\end{pgfscope}%
\begin{pgfscope}%
\pgfsys@transformshift{3.322880in}{0.952215in}%
\pgfsys@useobject{currentmarker}{}%
\end{pgfscope}%
\begin{pgfscope}%
\pgfsys@transformshift{3.301046in}{0.683258in}%
\pgfsys@useobject{currentmarker}{}%
\end{pgfscope}%
\begin{pgfscope}%
\pgfsys@transformshift{3.281326in}{0.548460in}%
\pgfsys@useobject{currentmarker}{}%
\end{pgfscope}%
\begin{pgfscope}%
\pgfsys@transformshift{3.262780in}{0.494750in}%
\pgfsys@useobject{currentmarker}{}%
\end{pgfscope}%
\begin{pgfscope}%
\pgfsys@transformshift{3.244467in}{0.532658in}%
\pgfsys@useobject{currentmarker}{}%
\end{pgfscope}%
\begin{pgfscope}%
\pgfsys@transformshift{3.224512in}{0.680639in}%
\pgfsys@useobject{currentmarker}{}%
\end{pgfscope}%
\begin{pgfscope}%
\pgfsys@transformshift{3.207373in}{0.868598in}%
\pgfsys@useobject{currentmarker}{}%
\end{pgfscope}%
\begin{pgfscope}%
\pgfsys@transformshift{3.188356in}{1.189834in}%
\pgfsys@useobject{currentmarker}{}%
\end{pgfscope}%
\begin{pgfscope}%
\pgfsys@transformshift{3.169809in}{1.369433in}%
\pgfsys@useobject{currentmarker}{}%
\end{pgfscope}%
\begin{pgfscope}%
\pgfsys@transformshift{3.147740in}{1.343654in}%
\pgfsys@useobject{currentmarker}{}%
\end{pgfscope}%
\begin{pgfscope}%
\pgfsys@transformshift{3.128725in}{1.155362in}%
\pgfsys@useobject{currentmarker}{}%
\end{pgfscope}%
\begin{pgfscope}%
\pgfsys@transformshift{3.110646in}{0.876218in}%
\pgfsys@useobject{currentmarker}{}%
\end{pgfscope}%
\begin{pgfscope}%
\pgfsys@transformshift{3.091395in}{0.694600in}%
\pgfsys@useobject{currentmarker}{}%
\end{pgfscope}%
\begin{pgfscope}%
\pgfsys@transformshift{3.070501in}{0.538898in}%
\pgfsys@useobject{currentmarker}{}%
\end{pgfscope}%
\begin{pgfscope}%
\pgfsys@transformshift{3.051953in}{0.491009in}%
\pgfsys@useobject{currentmarker}{}%
\end{pgfscope}%
\begin{pgfscope}%
\pgfsys@transformshift{3.030355in}{0.562094in}%
\pgfsys@useobject{currentmarker}{}%
\end{pgfscope}%
\begin{pgfscope}%
\pgfsys@transformshift{3.015330in}{0.661811in}%
\pgfsys@useobject{currentmarker}{}%
\end{pgfscope}%
\begin{pgfscope}%
\pgfsys@transformshift{2.992791in}{0.921781in}%
\pgfsys@useobject{currentmarker}{}%
\end{pgfscope}%
\begin{pgfscope}%
\pgfsys@transformshift{2.976357in}{1.197275in}%
\pgfsys@useobject{currentmarker}{}%
\end{pgfscope}%
\begin{pgfscope}%
\pgfsys@transformshift{2.957106in}{1.361740in}%
\pgfsys@useobject{currentmarker}{}%
\end{pgfscope}%
\begin{pgfscope}%
\pgfsys@transformshift{2.935506in}{1.353490in}%
\pgfsys@useobject{currentmarker}{}%
\end{pgfscope}%
\begin{pgfscope}%
\pgfsys@transformshift{2.917429in}{1.184580in}%
\pgfsys@useobject{currentmarker}{}%
\end{pgfscope}%
\begin{pgfscope}%
\pgfsys@transformshift{2.896534in}{0.875634in}%
\pgfsys@useobject{currentmarker}{}%
\end{pgfscope}%
\begin{pgfscope}%
\pgfsys@transformshift{2.878223in}{0.771759in}%
\pgfsys@useobject{currentmarker}{}%
\end{pgfscope}%
\begin{pgfscope}%
\pgfsys@transformshift{2.859675in}{0.664661in}%
\pgfsys@useobject{currentmarker}{}%
\end{pgfscope}%
\begin{pgfscope}%
\pgfsys@transformshift{2.840893in}{0.542317in}%
\pgfsys@useobject{currentmarker}{}%
\end{pgfscope}%
\begin{pgfscope}%
\pgfsys@transformshift{2.822816in}{0.485161in}%
\pgfsys@useobject{currentmarker}{}%
\end{pgfscope}%
\begin{pgfscope}%
\pgfsys@transformshift{2.803330in}{0.547505in}%
\pgfsys@useobject{currentmarker}{}%
\end{pgfscope}%
\begin{pgfscope}%
\pgfsys@transformshift{2.785488in}{0.662043in}%
\pgfsys@useobject{currentmarker}{}%
\end{pgfscope}%
\begin{pgfscope}%
\pgfsys@transformshift{2.763888in}{0.924744in}%
\pgfsys@useobject{currentmarker}{}%
\end{pgfscope}%
\begin{pgfscope}%
\pgfsys@transformshift{2.744871in}{1.228470in}%
\pgfsys@useobject{currentmarker}{}%
\end{pgfscope}%
\begin{pgfscope}%
\pgfsys@transformshift{2.726089in}{1.370731in}%
\pgfsys@useobject{currentmarker}{}%
\end{pgfscope}%
\begin{pgfscope}%
\pgfsys@transformshift{2.706603in}{1.075833in}%
\pgfsys@useobject{currentmarker}{}%
\end{pgfscope}%
\begin{pgfscope}%
\pgfsys@transformshift{2.688761in}{1.329429in}%
\pgfsys@useobject{currentmarker}{}%
\end{pgfscope}%
\begin{pgfscope}%
\pgfsys@transformshift{2.667867in}{1.380027in}%
\pgfsys@useobject{currentmarker}{}%
\end{pgfscope}%
\begin{pgfscope}%
\pgfsys@transformshift{2.648850in}{1.313043in}%
\pgfsys@useobject{currentmarker}{}%
\end{pgfscope}%
\begin{pgfscope}%
\pgfsys@transformshift{2.629364in}{1.033098in}%
\pgfsys@useobject{currentmarker}{}%
\end{pgfscope}%
\begin{pgfscope}%
\pgfsys@transformshift{2.610347in}{0.765765in}%
\pgfsys@useobject{currentmarker}{}%
\end{pgfscope}%
\begin{pgfscope}%
\pgfsys@transformshift{2.591096in}{0.599853in}%
\pgfsys@useobject{currentmarker}{}%
\end{pgfscope}%
\begin{pgfscope}%
\pgfsys@transformshift{2.570436in}{0.498056in}%
\pgfsys@useobject{currentmarker}{}%
\end{pgfscope}%
\begin{pgfscope}%
\pgfsys@transformshift{2.550480in}{0.506232in}%
\pgfsys@useobject{currentmarker}{}%
\end{pgfscope}%
\begin{pgfscope}%
\pgfsys@transformshift{2.531697in}{0.605349in}%
\pgfsys@useobject{currentmarker}{}%
\end{pgfscope}%
\begin{pgfscope}%
\pgfsys@transformshift{2.513386in}{0.769893in}%
\pgfsys@useobject{currentmarker}{}%
\end{pgfscope}%
\begin{pgfscope}%
\pgfsys@transformshift{2.494838in}{1.042991in}%
\pgfsys@useobject{currentmarker}{}%
\end{pgfscope}%
\begin{pgfscope}%
\pgfsys@transformshift{2.475587in}{1.302751in}%
\pgfsys@useobject{currentmarker}{}%
\end{pgfscope}%
\begin{pgfscope}%
\pgfsys@transformshift{2.455398in}{1.378311in}%
\pgfsys@useobject{currentmarker}{}%
\end{pgfscope}%
\begin{pgfscope}%
\pgfsys@transformshift{2.436147in}{1.292935in}%
\pgfsys@useobject{currentmarker}{}%
\end{pgfscope}%
\begin{pgfscope}%
\pgfsys@transformshift{2.418773in}{1.058167in}%
\pgfsys@useobject{currentmarker}{}%
\end{pgfscope}%
\begin{pgfscope}%
\pgfsys@transformshift{2.399051in}{0.822843in}%
\pgfsys@useobject{currentmarker}{}%
\end{pgfscope}%
\begin{pgfscope}%
\pgfsys@transformshift{2.376748in}{0.626036in}%
\pgfsys@useobject{currentmarker}{}%
\end{pgfscope}%
\begin{pgfscope}%
\pgfsys@transformshift{2.358671in}{0.523711in}%
\pgfsys@useobject{currentmarker}{}%
\end{pgfscope}%
\begin{pgfscope}%
\pgfsys@transformshift{2.341063in}{0.486120in}%
\pgfsys@useobject{currentmarker}{}%
\end{pgfscope}%
\begin{pgfscope}%
\pgfsys@transformshift{2.322046in}{0.538773in}%
\pgfsys@useobject{currentmarker}{}%
\end{pgfscope}%
\begin{pgfscope}%
\pgfsys@transformshift{2.303029in}{0.648693in}%
\pgfsys@useobject{currentmarker}{}%
\end{pgfscope}%
\begin{pgfscope}%
\pgfsys@transformshift{2.278378in}{0.925603in}%
\pgfsys@useobject{currentmarker}{}%
\end{pgfscope}%
\begin{pgfscope}%
\pgfsys@transformshift{2.262649in}{1.156007in}%
\pgfsys@useobject{currentmarker}{}%
\end{pgfscope}%
\begin{pgfscope}%
\pgfsys@transformshift{2.247388in}{1.326696in}%
\pgfsys@useobject{currentmarker}{}%
\end{pgfscope}%
\begin{pgfscope}%
\pgfsys@transformshift{2.225556in}{1.377505in}%
\pgfsys@useobject{currentmarker}{}%
\end{pgfscope}%
\begin{pgfscope}%
\pgfsys@transformshift{2.206539in}{1.287829in}%
\pgfsys@useobject{currentmarker}{}%
\end{pgfscope}%
\begin{pgfscope}%
\pgfsys@transformshift{2.185408in}{1.072801in}%
\pgfsys@useobject{currentmarker}{}%
\end{pgfscope}%
\begin{pgfscope}%
\pgfsys@transformshift{2.169680in}{0.851182in}%
\pgfsys@useobject{currentmarker}{}%
\end{pgfscope}%
\begin{pgfscope}%
\pgfsys@transformshift{2.151366in}{0.683331in}%
\pgfsys@useobject{currentmarker}{}%
\end{pgfscope}%
\begin{pgfscope}%
\pgfsys@transformshift{2.129769in}{0.580869in}%
\pgfsys@useobject{currentmarker}{}%
\end{pgfscope}%
\begin{pgfscope}%
\pgfsys@transformshift{2.111455in}{0.501476in}%
\pgfsys@useobject{currentmarker}{}%
\end{pgfscope}%
\begin{pgfscope}%
\pgfsys@transformshift{2.090092in}{0.526122in}%
\pgfsys@useobject{currentmarker}{}%
\end{pgfscope}%
\begin{pgfscope}%
\pgfsys@transformshift{2.071310in}{0.609458in}%
\pgfsys@useobject{currentmarker}{}%
\end{pgfscope}%
\begin{pgfscope}%
\pgfsys@transformshift{2.052762in}{0.770591in}%
\pgfsys@useobject{currentmarker}{}%
\end{pgfscope}%
\begin{pgfscope}%
\pgfsys@transformshift{2.033981in}{1.023911in}%
\pgfsys@useobject{currentmarker}{}%
\end{pgfscope}%
\begin{pgfscope}%
\pgfsys@transformshift{2.016608in}{1.273535in}%
\pgfsys@useobject{currentmarker}{}%
\end{pgfscope}%
\begin{pgfscope}%
\pgfsys@transformshift{1.992425in}{1.386097in}%
\pgfsys@useobject{currentmarker}{}%
\end{pgfscope}%
\begin{pgfscope}%
\pgfsys@transformshift{1.977166in}{1.324959in}%
\pgfsys@useobject{currentmarker}{}%
\end{pgfscope}%
\begin{pgfscope}%
\pgfsys@transformshift{1.956975in}{1.361401in}%
\pgfsys@useobject{currentmarker}{}%
\end{pgfscope}%
\begin{pgfscope}%
\pgfsys@transformshift{1.935846in}{1.373627in}%
\pgfsys@useobject{currentmarker}{}%
\end{pgfscope}%
\begin{pgfscope}%
\pgfsys@transformshift{1.919646in}{1.274061in}%
\pgfsys@useobject{currentmarker}{}%
\end{pgfscope}%
\begin{pgfscope}%
\pgfsys@transformshift{1.898987in}{0.992050in}%
\pgfsys@useobject{currentmarker}{}%
\end{pgfscope}%
\begin{pgfscope}%
\pgfsys@transformshift{1.880910in}{0.763722in}%
\pgfsys@useobject{currentmarker}{}%
\end{pgfscope}%
\begin{pgfscope}%
\pgfsys@transformshift{1.859779in}{0.595442in}%
\pgfsys@useobject{currentmarker}{}%
\end{pgfscope}%
\begin{pgfscope}%
\pgfsys@transformshift{1.839824in}{0.517256in}%
\pgfsys@useobject{currentmarker}{}%
\end{pgfscope}%
\begin{pgfscope}%
\pgfsys@transformshift{1.821042in}{0.500880in}%
\pgfsys@useobject{currentmarker}{}%
\end{pgfscope}%
\begin{pgfscope}%
\pgfsys@transformshift{1.802025in}{0.566887in}%
\pgfsys@useobject{currentmarker}{}%
\end{pgfscope}%
\begin{pgfscope}%
\pgfsys@transformshift{1.783714in}{0.672936in}%
\pgfsys@useobject{currentmarker}{}%
\end{pgfscope}%
\begin{pgfscope}%
\pgfsys@transformshift{1.765166in}{0.834119in}%
\pgfsys@useobject{currentmarker}{}%
\end{pgfscope}%
\begin{pgfscope}%
\pgfsys@transformshift{1.743097in}{1.140334in}%
\pgfsys@useobject{currentmarker}{}%
\end{pgfscope}%
\begin{pgfscope}%
\pgfsys@transformshift{1.724786in}{1.283723in}%
\pgfsys@useobject{currentmarker}{}%
\end{pgfscope}%
\begin{pgfscope}%
\pgfsys@transformshift{1.705769in}{1.388960in}%
\pgfsys@useobject{currentmarker}{}%
\end{pgfscope}%
\begin{pgfscope}%
\pgfsys@transformshift{1.688630in}{1.391965in}%
\pgfsys@useobject{currentmarker}{}%
\end{pgfscope}%
\begin{pgfscope}%
\pgfsys@transformshift{1.668205in}{1.285253in}%
\pgfsys@useobject{currentmarker}{}%
\end{pgfscope}%
\begin{pgfscope}%
\pgfsys@transformshift{1.647076in}{1.035235in}%
\pgfsys@useobject{currentmarker}{}%
\end{pgfscope}%
\begin{pgfscope}%
\pgfsys@transformshift{1.628293in}{0.799897in}%
\pgfsys@useobject{currentmarker}{}%
\end{pgfscope}%
\begin{pgfscope}%
\pgfsys@transformshift{1.610685in}{0.656839in}%
\pgfsys@useobject{currentmarker}{}%
\end{pgfscope}%
\begin{pgfscope}%
\pgfsys@transformshift{1.592374in}{0.563908in}%
\pgfsys@useobject{currentmarker}{}%
\end{pgfscope}%
\begin{pgfscope}%
\pgfsys@transformshift{1.570540in}{0.503360in}%
\pgfsys@useobject{currentmarker}{}%
\end{pgfscope}%
\begin{pgfscope}%
\pgfsys@transformshift{1.555749in}{0.520444in}%
\pgfsys@useobject{currentmarker}{}%
\end{pgfscope}%
\begin{pgfscope}%
\pgfsys@transformshift{1.531332in}{0.630346in}%
\pgfsys@useobject{currentmarker}{}%
\end{pgfscope}%
\begin{pgfscope}%
\pgfsys@transformshift{1.515369in}{0.748225in}%
\pgfsys@useobject{currentmarker}{}%
\end{pgfscope}%
\begin{pgfscope}%
\pgfsys@transformshift{1.496821in}{0.955459in}%
\pgfsys@useobject{currentmarker}{}%
\end{pgfscope}%
\begin{pgfscope}%
\pgfsys@transformshift{1.478039in}{1.212346in}%
\pgfsys@useobject{currentmarker}{}%
\end{pgfscope}%
\begin{pgfscope}%
\pgfsys@transformshift{1.458084in}{1.380860in}%
\pgfsys@useobject{currentmarker}{}%
\end{pgfscope}%
\begin{pgfscope}%
\pgfsys@transformshift{1.437659in}{1.408217in}%
\pgfsys@useobject{currentmarker}{}%
\end{pgfscope}%
\begin{pgfscope}%
\pgfsys@transformshift{1.416294in}{1.356308in}%
\pgfsys@useobject{currentmarker}{}%
\end{pgfscope}%
\begin{pgfscope}%
\pgfsys@transformshift{1.397748in}{1.237796in}%
\pgfsys@useobject{currentmarker}{}%
\end{pgfscope}%
\begin{pgfscope}%
\pgfsys@transformshift{1.378497in}{1.064696in}%
\pgfsys@useobject{currentmarker}{}%
\end{pgfscope}%
\begin{pgfscope}%
\pgfsys@transformshift{1.361123in}{0.882717in}%
\pgfsys@useobject{currentmarker}{}%
\end{pgfscope}%
\begin{pgfscope}%
\pgfsys@transformshift{1.336002in}{0.649100in}%
\pgfsys@useobject{currentmarker}{}%
\end{pgfscope}%
\begin{pgfscope}%
\pgfsys@transformshift{1.323558in}{0.580634in}%
\pgfsys@useobject{currentmarker}{}%
\end{pgfscope}%
\begin{pgfscope}%
\pgfsys@transformshift{1.301961in}{0.514712in}%
\pgfsys@useobject{currentmarker}{}%
\end{pgfscope}%
\begin{pgfscope}%
\pgfsys@transformshift{1.284118in}{0.532851in}%
\pgfsys@useobject{currentmarker}{}%
\end{pgfscope}%
\begin{pgfscope}%
\pgfsys@transformshift{1.264867in}{0.590922in}%
\pgfsys@useobject{currentmarker}{}%
\end{pgfscope}%
\begin{pgfscope}%
\pgfsys@transformshift{1.245379in}{0.738459in}%
\pgfsys@useobject{currentmarker}{}%
\end{pgfscope}%
\begin{pgfscope}%
\pgfsys@transformshift{1.224719in}{0.892520in}%
\pgfsys@useobject{currentmarker}{}%
\end{pgfscope}%
\begin{pgfscope}%
\pgfsys@transformshift{1.207348in}{1.148587in}%
\pgfsys@useobject{currentmarker}{}%
\end{pgfscope}%
\begin{pgfscope}%
\pgfsys@transformshift{1.188097in}{1.343366in}%
\pgfsys@useobject{currentmarker}{}%
\end{pgfscope}%
\begin{pgfscope}%
\pgfsys@transformshift{1.169314in}{1.413562in}%
\pgfsys@useobject{currentmarker}{}%
\end{pgfscope}%
\begin{pgfscope}%
\pgfsys@transformshift{1.147949in}{1.421136in}%
\pgfsys@useobject{currentmarker}{}%
\end{pgfscope}%
\begin{pgfscope}%
\pgfsys@transformshift{1.132455in}{1.375970in}%
\pgfsys@useobject{currentmarker}{}%
\end{pgfscope}%
\begin{pgfscope}%
\pgfsys@transformshift{1.111324in}{1.277639in}%
\pgfsys@useobject{currentmarker}{}%
\end{pgfscope}%
\begin{pgfscope}%
\pgfsys@transformshift{1.092544in}{1.059754in}%
\pgfsys@useobject{currentmarker}{}%
\end{pgfscope}%
\begin{pgfscope}%
\pgfsys@transformshift{1.070241in}{0.830270in}%
\pgfsys@useobject{currentmarker}{}%
\end{pgfscope}%
\begin{pgfscope}%
\pgfsys@transformshift{1.051693in}{0.680616in}%
\pgfsys@useobject{currentmarker}{}%
\end{pgfscope}%
\begin{pgfscope}%
\pgfsys@transformshift{1.033382in}{0.591219in}%
\pgfsys@useobject{currentmarker}{}%
\end{pgfscope}%
\begin{pgfscope}%
\pgfsys@transformshift{1.014834in}{0.541002in}%
\pgfsys@useobject{currentmarker}{}%
\end{pgfscope}%
\begin{pgfscope}%
\pgfsys@transformshift{0.993939in}{0.525714in}%
\pgfsys@useobject{currentmarker}{}%
\end{pgfscope}%
\begin{pgfscope}%
\pgfsys@transformshift{0.974688in}{0.607979in}%
\pgfsys@useobject{currentmarker}{}%
\end{pgfscope}%
\begin{pgfscope}%
\pgfsys@transformshift{0.956609in}{0.726588in}%
\pgfsys@useobject{currentmarker}{}%
\end{pgfscope}%
\begin{pgfscope}%
\pgfsys@transformshift{0.938063in}{0.763944in}%
\pgfsys@useobject{currentmarker}{}%
\end{pgfscope}%
\begin{pgfscope}%
\pgfsys@transformshift{0.919281in}{0.996494in}%
\pgfsys@useobject{currentmarker}{}%
\end{pgfscope}%
\begin{pgfscope}%
\pgfsys@transformshift{0.900733in}{1.228744in}%
\pgfsys@useobject{currentmarker}{}%
\end{pgfscope}%
\begin{pgfscope}%
\pgfsys@transformshift{0.881953in}{1.381568in}%
\pgfsys@useobject{currentmarker}{}%
\end{pgfscope}%
\begin{pgfscope}%
\pgfsys@transformshift{0.859884in}{1.441986in}%
\pgfsys@useobject{currentmarker}{}%
\end{pgfscope}%
\begin{pgfscope}%
\pgfsys@transformshift{0.842042in}{1.406303in}%
\pgfsys@useobject{currentmarker}{}%
\end{pgfscope}%
\begin{pgfscope}%
\pgfsys@transformshift{0.820677in}{1.259878in}%
\pgfsys@useobject{currentmarker}{}%
\end{pgfscope}%
\begin{pgfscope}%
\pgfsys@transformshift{0.802131in}{1.044428in}%
\pgfsys@useobject{currentmarker}{}%
\end{pgfscope}%
\begin{pgfscope}%
\pgfsys@transformshift{0.784052in}{0.857067in}%
\pgfsys@useobject{currentmarker}{}%
\end{pgfscope}%
\begin{pgfscope}%
\pgfsys@transformshift{0.767384in}{0.816847in}%
\pgfsys@useobject{currentmarker}{}%
\end{pgfscope}%
\begin{pgfscope}%
\pgfsys@transformshift{0.747193in}{0.655784in}%
\pgfsys@useobject{currentmarker}{}%
\end{pgfscope}%
\begin{pgfscope}%
\pgfsys@transformshift{0.726767in}{0.578828in}%
\pgfsys@useobject{currentmarker}{}%
\end{pgfscope}%
\begin{pgfscope}%
\pgfsys@transformshift{0.706344in}{0.531987in}%
\pgfsys@useobject{currentmarker}{}%
\end{pgfscope}%
\begin{pgfscope}%
\pgfsys@transformshift{0.685447in}{0.590318in}%
\pgfsys@useobject{currentmarker}{}%
\end{pgfscope}%
\begin{pgfscope}%
\pgfsys@transformshift{0.668545in}{0.697140in}%
\pgfsys@useobject{currentmarker}{}%
\end{pgfscope}%
\begin{pgfscope}%
\pgfsys@transformshift{0.650468in}{0.815066in}%
\pgfsys@useobject{currentmarker}{}%
\end{pgfscope}%
\begin{pgfscope}%
\pgfsys@transformshift{0.651876in}{0.796028in}%
\pgfsys@useobject{currentmarker}{}%
\end{pgfscope}%
\begin{pgfscope}%
\pgfsys@transformshift{0.657040in}{0.741917in}%
\pgfsys@useobject{currentmarker}{}%
\end{pgfscope}%
\begin{pgfscope}%
\pgfsys@transformshift{0.674648in}{0.611435in}%
\pgfsys@useobject{currentmarker}{}%
\end{pgfscope}%
\begin{pgfscope}%
\pgfsys@transformshift{0.696951in}{0.836344in}%
\pgfsys@useobject{currentmarker}{}%
\end{pgfscope}%
\begin{pgfscope}%
\pgfsys@transformshift{0.714559in}{1.128149in}%
\pgfsys@useobject{currentmarker}{}%
\end{pgfscope}%
\begin{pgfscope}%
\pgfsys@transformshift{0.731933in}{1.362384in}%
\pgfsys@useobject{currentmarker}{}%
\end{pgfscope}%
\begin{pgfscope}%
\pgfsys@transformshift{0.750715in}{1.445179in}%
\pgfsys@useobject{currentmarker}{}%
\end{pgfscope}%
\begin{pgfscope}%
\pgfsys@transformshift{0.773018in}{1.295605in}%
\pgfsys@useobject{currentmarker}{}%
\end{pgfscope}%
\begin{pgfscope}%
\pgfsys@transformshift{0.790861in}{0.961277in}%
\pgfsys@useobject{currentmarker}{}%
\end{pgfscope}%
\begin{pgfscope}%
\pgfsys@transformshift{0.810347in}{0.709987in}%
\pgfsys@useobject{currentmarker}{}%
\end{pgfscope}%
\begin{pgfscope}%
\pgfsys@transformshift{0.829129in}{0.552729in}%
\pgfsys@useobject{currentmarker}{}%
\end{pgfscope}%
\begin{pgfscope}%
\pgfsys@transformshift{0.846502in}{0.532530in}%
\pgfsys@useobject{currentmarker}{}%
\end{pgfscope}%
\begin{pgfscope}%
\pgfsys@transformshift{0.869743in}{0.693651in}%
\pgfsys@useobject{currentmarker}{}%
\end{pgfscope}%
\begin{pgfscope}%
\pgfsys@transformshift{0.886648in}{1.264537in}%
\pgfsys@useobject{currentmarker}{}%
\end{pgfscope}%
\begin{pgfscope}%
\pgfsys@transformshift{0.905665in}{1.424904in}%
\pgfsys@useobject{currentmarker}{}%
\end{pgfscope}%
\begin{pgfscope}%
\pgfsys@transformshift{0.925619in}{1.373359in}%
\pgfsys@useobject{currentmarker}{}%
\end{pgfscope}%
\begin{pgfscope}%
\pgfsys@transformshift{0.944636in}{1.112476in}%
\pgfsys@useobject{currentmarker}{}%
\end{pgfscope}%
\begin{pgfscope}%
\pgfsys@transformshift{0.962949in}{0.786487in}%
\pgfsys@useobject{currentmarker}{}%
\end{pgfscope}%
\begin{pgfscope}%
\pgfsys@transformshift{0.986661in}{0.561312in}%
\pgfsys@useobject{currentmarker}{}%
\end{pgfscope}%
\begin{pgfscope}%
\pgfsys@transformshift{1.003095in}{0.510460in}%
\pgfsys@useobject{currentmarker}{}%
\end{pgfscope}%
\begin{pgfscope}%
\pgfsys@transformshift{1.020469in}{0.599592in}%
\pgfsys@useobject{currentmarker}{}%
\end{pgfscope}%
\begin{pgfscope}%
\pgfsys@transformshift{1.039251in}{0.772002in}%
\pgfsys@useobject{currentmarker}{}%
\end{pgfscope}%
\begin{pgfscope}%
\pgfsys@transformshift{1.061318in}{1.094606in}%
\pgfsys@useobject{currentmarker}{}%
\end{pgfscope}%
\begin{pgfscope}%
\pgfsys@transformshift{1.081040in}{1.350989in}%
\pgfsys@useobject{currentmarker}{}%
\end{pgfscope}%
\begin{pgfscope}%
\pgfsys@transformshift{1.098179in}{1.412509in}%
\pgfsys@useobject{currentmarker}{}%
\end{pgfscope}%
\begin{pgfscope}%
\pgfsys@transformshift{1.114847in}{1.315366in}%
\pgfsys@useobject{currentmarker}{}%
\end{pgfscope}%
\begin{pgfscope}%
\pgfsys@transformshift{1.138559in}{0.927358in}%
\pgfsys@useobject{currentmarker}{}%
\end{pgfscope}%
\begin{pgfscope}%
\pgfsys@transformshift{1.158279in}{0.678706in}%
\pgfsys@useobject{currentmarker}{}%
\end{pgfscope}%
\begin{pgfscope}%
\pgfsys@transformshift{1.176358in}{0.534046in}%
\pgfsys@useobject{currentmarker}{}%
\end{pgfscope}%
\begin{pgfscope}%
\pgfsys@transformshift{1.195844in}{0.516339in}%
\pgfsys@useobject{currentmarker}{}%
\end{pgfscope}%
\begin{pgfscope}%
\pgfsys@transformshift{1.214624in}{0.620710in}%
\pgfsys@useobject{currentmarker}{}%
\end{pgfscope}%
\begin{pgfscope}%
\pgfsys@transformshift{1.233172in}{0.796124in}%
\pgfsys@useobject{currentmarker}{}%
\end{pgfscope}%
\begin{pgfscope}%
\pgfsys@transformshift{1.250076in}{1.000679in}%
\pgfsys@useobject{currentmarker}{}%
\end{pgfscope}%
\begin{pgfscope}%
\pgfsys@transformshift{1.271205in}{1.329982in}%
\pgfsys@useobject{currentmarker}{}%
\end{pgfscope}%
\begin{pgfscope}%
\pgfsys@transformshift{1.290222in}{1.403178in}%
\pgfsys@useobject{currentmarker}{}%
\end{pgfscope}%
\begin{pgfscope}%
\pgfsys@transformshift{1.310177in}{1.340270in}%
\pgfsys@useobject{currentmarker}{}%
\end{pgfscope}%
\begin{pgfscope}%
\pgfsys@transformshift{1.330602in}{1.019023in}%
\pgfsys@useobject{currentmarker}{}%
\end{pgfscope}%
\begin{pgfscope}%
\pgfsys@transformshift{1.346332in}{0.803367in}%
\pgfsys@useobject{currentmarker}{}%
\end{pgfscope}%
\begin{pgfscope}%
\pgfsys@transformshift{1.368401in}{0.597128in}%
\pgfsys@useobject{currentmarker}{}%
\end{pgfscope}%
\begin{pgfscope}%
\pgfsys@transformshift{1.389764in}{0.499643in}%
\pgfsys@useobject{currentmarker}{}%
\end{pgfscope}%
\begin{pgfscope}%
\pgfsys@transformshift{1.404555in}{0.513626in}%
\pgfsys@useobject{currentmarker}{}%
\end{pgfscope}%
\begin{pgfscope}%
\pgfsys@transformshift{1.427329in}{0.663394in}%
\pgfsys@useobject{currentmarker}{}%
\end{pgfscope}%
\begin{pgfscope}%
\pgfsys@transformshift{1.443997in}{0.836194in}%
\pgfsys@useobject{currentmarker}{}%
\end{pgfscope}%
\begin{pgfscope}%
\pgfsys@transformshift{1.461605in}{1.104362in}%
\pgfsys@useobject{currentmarker}{}%
\end{pgfscope}%
\begin{pgfscope}%
\pgfsys@transformshift{1.481562in}{0.513012in}%
\pgfsys@useobject{currentmarker}{}%
\end{pgfscope}%
\begin{pgfscope}%
\pgfsys@transformshift{1.502456in}{0.518592in}%
\pgfsys@useobject{currentmarker}{}%
\end{pgfscope}%
\begin{pgfscope}%
\pgfsys@transformshift{1.521473in}{0.634174in}%
\pgfsys@useobject{currentmarker}{}%
\end{pgfscope}%
\begin{pgfscope}%
\pgfsys@transformshift{1.540490in}{0.824173in}%
\pgfsys@useobject{currentmarker}{}%
\end{pgfscope}%
\begin{pgfscope}%
\pgfsys@transformshift{1.559506in}{1.146584in}%
\pgfsys@useobject{currentmarker}{}%
\end{pgfscope}%
\begin{pgfscope}%
\pgfsys@transformshift{1.578052in}{1.350173in}%
\pgfsys@useobject{currentmarker}{}%
\end{pgfscope}%
\begin{pgfscope}%
\pgfsys@transformshift{1.595660in}{1.386836in}%
\pgfsys@useobject{currentmarker}{}%
\end{pgfscope}%
\begin{pgfscope}%
\pgfsys@transformshift{1.617260in}{1.212989in}%
\pgfsys@useobject{currentmarker}{}%
\end{pgfscope}%
\begin{pgfscope}%
\pgfsys@transformshift{1.636277in}{0.982194in}%
\pgfsys@useobject{currentmarker}{}%
\end{pgfscope}%
\begin{pgfscope}%
\pgfsys@transformshift{1.654588in}{0.712761in}%
\pgfsys@useobject{currentmarker}{}%
\end{pgfscope}%
\begin{pgfscope}%
\pgfsys@transformshift{1.673605in}{0.557087in}%
\pgfsys@useobject{currentmarker}{}%
\end{pgfscope}%
\begin{pgfscope}%
\pgfsys@transformshift{1.694970in}{0.491685in}%
\pgfsys@useobject{currentmarker}{}%
\end{pgfscope}%
\begin{pgfscope}%
\pgfsys@transformshift{1.712576in}{0.541733in}%
\pgfsys@useobject{currentmarker}{}%
\end{pgfscope}%
\begin{pgfscope}%
\pgfsys@transformshift{1.733473in}{0.715059in}%
\pgfsys@useobject{currentmarker}{}%
\end{pgfscope}%
\begin{pgfscope}%
\pgfsys@transformshift{1.752489in}{0.954367in}%
\pgfsys@useobject{currentmarker}{}%
\end{pgfscope}%
\begin{pgfscope}%
\pgfsys@transformshift{1.770566in}{1.242151in}%
\pgfsys@useobject{currentmarker}{}%
\end{pgfscope}%
\begin{pgfscope}%
\pgfsys@transformshift{1.790992in}{1.381388in}%
\pgfsys@useobject{currentmarker}{}%
\end{pgfscope}%
\begin{pgfscope}%
\pgfsys@transformshift{1.808834in}{1.358575in}%
\pgfsys@useobject{currentmarker}{}%
\end{pgfscope}%
\begin{pgfscope}%
\pgfsys@transformshift{1.829963in}{1.112334in}%
\pgfsys@useobject{currentmarker}{}%
\end{pgfscope}%
\begin{pgfscope}%
\pgfsys@transformshift{1.848980in}{0.850149in}%
\pgfsys@useobject{currentmarker}{}%
\end{pgfscope}%
\begin{pgfscope}%
\pgfsys@transformshift{1.864710in}{0.648698in}%
\pgfsys@useobject{currentmarker}{}%
\end{pgfscope}%
\begin{pgfscope}%
\pgfsys@transformshift{1.887248in}{0.531397in}%
\pgfsys@useobject{currentmarker}{}%
\end{pgfscope}%
\begin{pgfscope}%
\pgfsys@transformshift{1.905325in}{0.490761in}%
\pgfsys@useobject{currentmarker}{}%
\end{pgfscope}%
\begin{pgfscope}%
\pgfsys@transformshift{1.925047in}{0.534411in}%
\pgfsys@useobject{currentmarker}{}%
\end{pgfscope}%
\begin{pgfscope}%
\pgfsys@transformshift{1.944767in}{0.667965in}%
\pgfsys@useobject{currentmarker}{}%
\end{pgfscope}%
\begin{pgfscope}%
\pgfsys@transformshift{1.963315in}{0.841090in}%
\pgfsys@useobject{currentmarker}{}%
\end{pgfscope}%
\begin{pgfscope}%
\pgfsys@transformshift{1.980686in}{1.128494in}%
\pgfsys@useobject{currentmarker}{}%
\end{pgfscope}%
\begin{pgfscope}%
\pgfsys@transformshift{1.999469in}{1.300464in}%
\pgfsys@useobject{currentmarker}{}%
\end{pgfscope}%
\begin{pgfscope}%
\pgfsys@transformshift{2.019660in}{1.381746in}%
\pgfsys@useobject{currentmarker}{}%
\end{pgfscope}%
\begin{pgfscope}%
\pgfsys@transformshift{2.041728in}{1.291139in}%
\pgfsys@useobject{currentmarker}{}%
\end{pgfscope}%
\begin{pgfscope}%
\pgfsys@transformshift{2.059336in}{1.381711in}%
\pgfsys@useobject{currentmarker}{}%
\end{pgfscope}%
\begin{pgfscope}%
\pgfsys@transformshift{2.077179in}{1.317837in}%
\pgfsys@useobject{currentmarker}{}%
\end{pgfscope}%
\begin{pgfscope}%
\pgfsys@transformshift{2.101125in}{0.978549in}%
\pgfsys@useobject{currentmarker}{}%
\end{pgfscope}%
\begin{pgfscope}%
\pgfsys@transformshift{2.112864in}{0.759392in}%
\pgfsys@useobject{currentmarker}{}%
\end{pgfscope}%
\begin{pgfscope}%
\pgfsys@transformshift{2.137281in}{0.585383in}%
\pgfsys@useobject{currentmarker}{}%
\end{pgfscope}%
\begin{pgfscope}%
\pgfsys@transformshift{2.154889in}{0.515344in}%
\pgfsys@useobject{currentmarker}{}%
\end{pgfscope}%
\begin{pgfscope}%
\pgfsys@transformshift{2.175549in}{0.503136in}%
\pgfsys@useobject{currentmarker}{}%
\end{pgfscope}%
\begin{pgfscope}%
\pgfsys@transformshift{2.192921in}{0.587266in}%
\pgfsys@useobject{currentmarker}{}%
\end{pgfscope}%
\begin{pgfscope}%
\pgfsys@transformshift{2.211468in}{0.730369in}%
\pgfsys@useobject{currentmarker}{}%
\end{pgfscope}%
\begin{pgfscope}%
\pgfsys@transformshift{2.231894in}{0.986716in}%
\pgfsys@useobject{currentmarker}{}%
\end{pgfscope}%
\begin{pgfscope}%
\pgfsys@transformshift{2.253728in}{1.275522in}%
\pgfsys@useobject{currentmarker}{}%
\end{pgfscope}%
\begin{pgfscope}%
\pgfsys@transformshift{2.268988in}{1.366296in}%
\pgfsys@useobject{currentmarker}{}%
\end{pgfscope}%
\begin{pgfscope}%
\pgfsys@transformshift{2.288944in}{1.344736in}%
\pgfsys@useobject{currentmarker}{}%
\end{pgfscope}%
\begin{pgfscope}%
\pgfsys@transformshift{2.308664in}{1.152901in}%
\pgfsys@useobject{currentmarker}{}%
\end{pgfscope}%
\begin{pgfscope}%
\pgfsys@transformshift{2.326741in}{0.863681in}%
\pgfsys@useobject{currentmarker}{}%
\end{pgfscope}%
\begin{pgfscope}%
\pgfsys@transformshift{2.346227in}{0.658074in}%
\pgfsys@useobject{currentmarker}{}%
\end{pgfscope}%
\begin{pgfscope}%
\pgfsys@transformshift{2.366889in}{0.530191in}%
\pgfsys@useobject{currentmarker}{}%
\end{pgfscope}%
\begin{pgfscope}%
\pgfsys@transformshift{2.383792in}{0.485517in}%
\pgfsys@useobject{currentmarker}{}%
\end{pgfscope}%
\begin{pgfscope}%
\pgfsys@transformshift{2.405860in}{0.544493in}%
\pgfsys@useobject{currentmarker}{}%
\end{pgfscope}%
\begin{pgfscope}%
\pgfsys@transformshift{2.426051in}{0.661860in}%
\pgfsys@useobject{currentmarker}{}%
\end{pgfscope}%
\begin{pgfscope}%
\pgfsys@transformshift{2.443188in}{0.829772in}%
\pgfsys@useobject{currentmarker}{}%
\end{pgfscope}%
\begin{pgfscope}%
\pgfsys@transformshift{2.462910in}{0.996541in}%
\pgfsys@useobject{currentmarker}{}%
\end{pgfscope}%
\begin{pgfscope}%
\pgfsys@transformshift{2.481691in}{1.138755in}%
\pgfsys@useobject{currentmarker}{}%
\end{pgfscope}%
\begin{pgfscope}%
\pgfsys@transformshift{2.499299in}{1.341272in}%
\pgfsys@useobject{currentmarker}{}%
\end{pgfscope}%
\begin{pgfscope}%
\pgfsys@transformshift{2.520430in}{1.376773in}%
\pgfsys@useobject{currentmarker}{}%
\end{pgfscope}%
\begin{pgfscope}%
\pgfsys@transformshift{2.538975in}{1.282416in}%
\pgfsys@useobject{currentmarker}{}%
\end{pgfscope}%
\begin{pgfscope}%
\pgfsys@transformshift{2.557054in}{1.038478in}%
\pgfsys@useobject{currentmarker}{}%
\end{pgfscope}%
\begin{pgfscope}%
\pgfsys@transformshift{2.577478in}{0.726857in}%
\pgfsys@useobject{currentmarker}{}%
\end{pgfscope}%
\begin{pgfscope}%
\pgfsys@transformshift{2.598609in}{0.570281in}%
\pgfsys@useobject{currentmarker}{}%
\end{pgfscope}%
\begin{pgfscope}%
\pgfsys@transformshift{2.616217in}{0.522940in}%
\pgfsys@useobject{currentmarker}{}%
\end{pgfscope}%
\begin{pgfscope}%
\pgfsys@transformshift{2.634528in}{0.489436in}%
\pgfsys@useobject{currentmarker}{}%
\end{pgfscope}%
\begin{pgfscope}%
\pgfsys@transformshift{2.655188in}{0.563219in}%
\pgfsys@useobject{currentmarker}{}%
\end{pgfscope}%
\begin{pgfscope}%
\pgfsys@transformshift{2.673265in}{0.657899in}%
\pgfsys@useobject{currentmarker}{}%
\end{pgfscope}%
\begin{pgfscope}%
\pgfsys@transformshift{2.691813in}{0.847833in}%
\pgfsys@useobject{currentmarker}{}%
\end{pgfscope}%
\begin{pgfscope}%
\pgfsys@transformshift{2.714821in}{1.136466in}%
\pgfsys@useobject{currentmarker}{}%
\end{pgfscope}%
\begin{pgfscope}%
\pgfsys@transformshift{2.732193in}{1.251011in}%
\pgfsys@useobject{currentmarker}{}%
\end{pgfscope}%
\begin{pgfscope}%
\pgfsys@transformshift{2.749801in}{1.344779in}%
\pgfsys@useobject{currentmarker}{}%
\end{pgfscope}%
\begin{pgfscope}%
\pgfsys@transformshift{2.769757in}{1.370880in}%
\pgfsys@useobject{currentmarker}{}%
\end{pgfscope}%
\begin{pgfscope}%
\pgfsys@transformshift{2.790417in}{1.191996in}%
\pgfsys@useobject{currentmarker}{}%
\end{pgfscope}%
\begin{pgfscope}%
\pgfsys@transformshift{2.808025in}{0.903704in}%
\pgfsys@useobject{currentmarker}{}%
\end{pgfscope}%
\begin{pgfscope}%
\pgfsys@transformshift{2.826102in}{0.705900in}%
\pgfsys@useobject{currentmarker}{}%
\end{pgfscope}%
\begin{pgfscope}%
\pgfsys@transformshift{2.847233in}{0.552550in}%
\pgfsys@useobject{currentmarker}{}%
\end{pgfscope}%
\begin{pgfscope}%
\pgfsys@transformshift{2.864136in}{0.498662in}%
\pgfsys@useobject{currentmarker}{}%
\end{pgfscope}%
\begin{pgfscope}%
\pgfsys@transformshift{2.883387in}{0.517741in}%
\pgfsys@useobject{currentmarker}{}%
\end{pgfscope}%
\begin{pgfscope}%
\pgfsys@transformshift{2.901464in}{0.627217in}%
\pgfsys@useobject{currentmarker}{}%
\end{pgfscope}%
\begin{pgfscope}%
\pgfsys@transformshift{2.922829in}{0.795591in}%
\pgfsys@useobject{currentmarker}{}%
\end{pgfscope}%
\begin{pgfscope}%
\pgfsys@transformshift{2.944898in}{1.066200in}%
\pgfsys@useobject{currentmarker}{}%
\end{pgfscope}%
\begin{pgfscope}%
\pgfsys@transformshift{2.964384in}{1.224768in}%
\pgfsys@useobject{currentmarker}{}%
\end{pgfscope}%
\begin{pgfscope}%
\pgfsys@transformshift{2.980114in}{1.342661in}%
\pgfsys@useobject{currentmarker}{}%
\end{pgfscope}%
\begin{pgfscope}%
\pgfsys@transformshift{3.001948in}{1.378203in}%
\pgfsys@useobject{currentmarker}{}%
\end{pgfscope}%
\begin{pgfscope}%
\pgfsys@transformshift{3.019554in}{1.248951in}%
\pgfsys@useobject{currentmarker}{}%
\end{pgfscope}%
\begin{pgfscope}%
\pgfsys@transformshift{3.043502in}{0.932710in}%
\pgfsys@useobject{currentmarker}{}%
\end{pgfscope}%
\begin{pgfscope}%
\pgfsys@transformshift{3.057119in}{0.770880in}%
\pgfsys@useobject{currentmarker}{}%
\end{pgfscope}%
\begin{pgfscope}%
\pgfsys@transformshift{3.078013in}{0.598118in}%
\pgfsys@useobject{currentmarker}{}%
\end{pgfscope}%
\begin{pgfscope}%
\pgfsys@transformshift{3.097264in}{0.526443in}%
\pgfsys@useobject{currentmarker}{}%
\end{pgfscope}%
\begin{pgfscope}%
\pgfsys@transformshift{3.114872in}{0.495761in}%
\pgfsys@useobject{currentmarker}{}%
\end{pgfscope}%
\begin{pgfscope}%
\pgfsys@transformshift{3.137176in}{0.578483in}%
\pgfsys@useobject{currentmarker}{}%
\end{pgfscope}%
\begin{pgfscope}%
\pgfsys@transformshift{3.152437in}{0.666131in}%
\pgfsys@useobject{currentmarker}{}%
\end{pgfscope}%
\begin{pgfscope}%
\pgfsys@transformshift{3.172157in}{0.798683in}%
\pgfsys@useobject{currentmarker}{}%
\end{pgfscope}%
\begin{pgfscope}%
\pgfsys@transformshift{3.191877in}{1.037040in}%
\pgfsys@useobject{currentmarker}{}%
\end{pgfscope}%
\begin{pgfscope}%
\pgfsys@transformshift{3.214886in}{1.298444in}%
\pgfsys@useobject{currentmarker}{}%
\end{pgfscope}%
\begin{pgfscope}%
\pgfsys@transformshift{3.230145in}{1.354740in}%
\pgfsys@useobject{currentmarker}{}%
\end{pgfscope}%
\begin{pgfscope}%
\pgfsys@transformshift{3.249396in}{1.394107in}%
\pgfsys@useobject{currentmarker}{}%
\end{pgfscope}%
\begin{pgfscope}%
\pgfsys@transformshift{3.270056in}{1.310172in}%
\pgfsys@useobject{currentmarker}{}%
\end{pgfscope}%
\begin{pgfscope}%
\pgfsys@transformshift{3.288135in}{1.119604in}%
\pgfsys@useobject{currentmarker}{}%
\end{pgfscope}%
\begin{pgfscope}%
\pgfsys@transformshift{3.309499in}{0.852144in}%
\pgfsys@useobject{currentmarker}{}%
\end{pgfscope}%
\begin{pgfscope}%
\pgfsys@transformshift{3.326638in}{0.693629in}%
\pgfsys@useobject{currentmarker}{}%
\end{pgfscope}%
\begin{pgfscope}%
\pgfsys@transformshift{3.347063in}{0.580167in}%
\pgfsys@useobject{currentmarker}{}%
\end{pgfscope}%
\begin{pgfscope}%
\pgfsys@transformshift{3.367018in}{0.503966in}%
\pgfsys@useobject{currentmarker}{}%
\end{pgfscope}%
\begin{pgfscope}%
\pgfsys@transformshift{3.385800in}{0.509882in}%
\pgfsys@useobject{currentmarker}{}%
\end{pgfscope}%
\begin{pgfscope}%
\pgfsys@transformshift{3.403877in}{0.564951in}%
\pgfsys@useobject{currentmarker}{}%
\end{pgfscope}%
\begin{pgfscope}%
\pgfsys@transformshift{3.423834in}{0.712675in}%
\pgfsys@useobject{currentmarker}{}%
\end{pgfscope}%
\begin{pgfscope}%
\pgfsys@transformshift{3.441910in}{0.876232in}%
\pgfsys@useobject{currentmarker}{}%
\end{pgfscope}%
\begin{pgfscope}%
\pgfsys@transformshift{3.460224in}{1.026728in}%
\pgfsys@useobject{currentmarker}{}%
\end{pgfscope}%
\begin{pgfscope}%
\pgfsys@transformshift{3.482996in}{1.281067in}%
\pgfsys@useobject{currentmarker}{}%
\end{pgfscope}%
\begin{pgfscope}%
\pgfsys@transformshift{3.500369in}{0.713218in}%
\pgfsys@useobject{currentmarker}{}%
\end{pgfscope}%
\begin{pgfscope}%
\pgfsys@transformshift{3.523612in}{1.019889in}%
\pgfsys@useobject{currentmarker}{}%
\end{pgfscope}%
\begin{pgfscope}%
\pgfsys@transformshift{3.537229in}{1.154863in}%
\pgfsys@useobject{currentmarker}{}%
\end{pgfscope}%
\begin{pgfscope}%
\pgfsys@transformshift{3.558826in}{1.376645in}%
\pgfsys@useobject{currentmarker}{}%
\end{pgfscope}%
\begin{pgfscope}%
\pgfsys@transformshift{3.576905in}{1.406371in}%
\pgfsys@useobject{currentmarker}{}%
\end{pgfscope}%
\begin{pgfscope}%
\pgfsys@transformshift{3.594513in}{1.337139in}%
\pgfsys@useobject{currentmarker}{}%
\end{pgfscope}%
\begin{pgfscope}%
\pgfsys@transformshift{3.615642in}{1.114973in}%
\pgfsys@useobject{currentmarker}{}%
\end{pgfscope}%
\begin{pgfscope}%
\pgfsys@transformshift{3.632781in}{0.902108in}%
\pgfsys@useobject{currentmarker}{}%
\end{pgfscope}%
\begin{pgfscope}%
\pgfsys@transformshift{3.654850in}{0.704923in}%
\pgfsys@useobject{currentmarker}{}%
\end{pgfscope}%
\begin{pgfscope}%
\pgfsys@transformshift{3.672222in}{0.577892in}%
\pgfsys@useobject{currentmarker}{}%
\end{pgfscope}%
\begin{pgfscope}%
\pgfsys@transformshift{3.690535in}{0.510586in}%
\pgfsys@useobject{currentmarker}{}%
\end{pgfscope}%
\begin{pgfscope}%
\pgfsys@transformshift{3.711898in}{0.544865in}%
\pgfsys@useobject{currentmarker}{}%
\end{pgfscope}%
\begin{pgfscope}%
\pgfsys@transformshift{3.733029in}{0.631305in}%
\pgfsys@useobject{currentmarker}{}%
\end{pgfscope}%
\begin{pgfscope}%
\pgfsys@transformshift{3.751341in}{0.697711in}%
\pgfsys@useobject{currentmarker}{}%
\end{pgfscope}%
\begin{pgfscope}%
\pgfsys@transformshift{3.769183in}{0.848680in}%
\pgfsys@useobject{currentmarker}{}%
\end{pgfscope}%
\begin{pgfscope}%
\pgfsys@transformshift{3.789140in}{1.123376in}%
\pgfsys@useobject{currentmarker}{}%
\end{pgfscope}%
\begin{pgfscope}%
\pgfsys@transformshift{3.808391in}{1.330573in}%
\pgfsys@useobject{currentmarker}{}%
\end{pgfscope}%
\begin{pgfscope}%
\pgfsys@transformshift{3.825764in}{1.416159in}%
\pgfsys@useobject{currentmarker}{}%
\end{pgfscope}%
\begin{pgfscope}%
\pgfsys@transformshift{3.846659in}{1.405303in}%
\pgfsys@useobject{currentmarker}{}%
\end{pgfscope}%
\begin{pgfscope}%
\pgfsys@transformshift{3.864970in}{1.313850in}%
\pgfsys@useobject{currentmarker}{}%
\end{pgfscope}%
\begin{pgfscope}%
\pgfsys@transformshift{3.886335in}{1.046207in}%
\pgfsys@useobject{currentmarker}{}%
\end{pgfscope}%
\begin{pgfscope}%
\pgfsys@transformshift{3.903709in}{0.877520in}%
\pgfsys@useobject{currentmarker}{}%
\end{pgfscope}%
\begin{pgfscope}%
\pgfsys@transformshift{3.920612in}{0.703609in}%
\pgfsys@useobject{currentmarker}{}%
\end{pgfscope}%
\begin{pgfscope}%
\pgfsys@transformshift{3.943149in}{0.568548in}%
\pgfsys@useobject{currentmarker}{}%
\end{pgfscope}%
\begin{pgfscope}%
\pgfsys@transformshift{3.961697in}{0.521977in}%
\pgfsys@useobject{currentmarker}{}%
\end{pgfscope}%
\begin{pgfscope}%
\pgfsys@transformshift{3.979305in}{0.524299in}%
\pgfsys@useobject{currentmarker}{}%
\end{pgfscope}%
\begin{pgfscope}%
\pgfsys@transformshift{4.000199in}{0.610827in}%
\pgfsys@useobject{currentmarker}{}%
\end{pgfscope}%
\begin{pgfscope}%
\pgfsys@transformshift{4.018276in}{0.741777in}%
\pgfsys@useobject{currentmarker}{}%
\end{pgfscope}%
\begin{pgfscope}%
\pgfsys@transformshift{4.039642in}{0.878877in}%
\pgfsys@useobject{currentmarker}{}%
\end{pgfscope}%
\begin{pgfscope}%
\pgfsys@transformshift{4.057015in}{1.107000in}%
\pgfsys@useobject{currentmarker}{}%
\end{pgfscope}%
\begin{pgfscope}%
\pgfsys@transformshift{4.077910in}{1.294859in}%
\pgfsys@useobject{currentmarker}{}%
\end{pgfscope}%
\begin{pgfscope}%
\pgfsys@transformshift{4.096690in}{1.418168in}%
\pgfsys@useobject{currentmarker}{}%
\end{pgfscope}%
\begin{pgfscope}%
\pgfsys@transformshift{4.114769in}{1.436462in}%
\pgfsys@useobject{currentmarker}{}%
\end{pgfscope}%
\begin{pgfscope}%
\pgfsys@transformshift{4.134020in}{1.377032in}%
\pgfsys@useobject{currentmarker}{}%
\end{pgfscope}%
\begin{pgfscope}%
\pgfsys@transformshift{4.154211in}{1.239146in}%
\pgfsys@useobject{currentmarker}{}%
\end{pgfscope}%
\begin{pgfscope}%
\pgfsys@transformshift{4.171348in}{0.993501in}%
\pgfsys@useobject{currentmarker}{}%
\end{pgfscope}%
\begin{pgfscope}%
\pgfsys@transformshift{4.192948in}{0.765759in}%
\pgfsys@useobject{currentmarker}{}%
\end{pgfscope}%
\begin{pgfscope}%
\pgfsys@transformshift{4.211494in}{0.715635in}%
\pgfsys@useobject{currentmarker}{}%
\end{pgfscope}%
\begin{pgfscope}%
\pgfsys@transformshift{4.228633in}{0.582603in}%
\pgfsys@useobject{currentmarker}{}%
\end{pgfscope}%
\begin{pgfscope}%
\pgfsys@transformshift{4.249998in}{0.526834in}%
\pgfsys@useobject{currentmarker}{}%
\end{pgfscope}%
\begin{pgfscope}%
\pgfsys@transformshift{4.267604in}{0.584449in}%
\pgfsys@useobject{currentmarker}{}%
\end{pgfscope}%
\begin{pgfscope}%
\pgfsys@transformshift{4.288735in}{0.715882in}%
\pgfsys@useobject{currentmarker}{}%
\end{pgfscope}%
\begin{pgfscope}%
\pgfsys@transformshift{4.307517in}{0.876583in}%
\pgfsys@useobject{currentmarker}{}%
\end{pgfscope}%
\begin{pgfscope}%
\pgfsys@transformshift{4.327003in}{1.090056in}%
\pgfsys@useobject{currentmarker}{}%
\end{pgfscope}%
\begin{pgfscope}%
\pgfsys@transformshift{4.345549in}{1.275892in}%
\pgfsys@useobject{currentmarker}{}%
\end{pgfscope}%
\begin{pgfscope}%
\pgfsys@transformshift{4.366914in}{1.427435in}%
\pgfsys@useobject{currentmarker}{}%
\end{pgfscope}%
\begin{pgfscope}%
\pgfsys@transformshift{4.388277in}{1.457294in}%
\pgfsys@useobject{currentmarker}{}%
\end{pgfscope}%
\begin{pgfscope}%
\pgfsys@transformshift{4.406825in}{1.423981in}%
\pgfsys@useobject{currentmarker}{}%
\end{pgfscope}%
\begin{pgfscope}%
\pgfsys@transformshift{4.421381in}{1.322675in}%
\pgfsys@useobject{currentmarker}{}%
\end{pgfscope}%
\begin{pgfscope}%
\pgfsys@transformshift{4.445562in}{1.051520in}%
\pgfsys@useobject{currentmarker}{}%
\end{pgfscope}%
\begin{pgfscope}%
\pgfsys@transformshift{4.460587in}{0.886592in}%
\pgfsys@useobject{currentmarker}{}%
\end{pgfscope}%
\begin{pgfscope}%
\pgfsys@transformshift{4.478195in}{0.733075in}%
\pgfsys@useobject{currentmarker}{}%
\end{pgfscope}%
\begin{pgfscope}%
\pgfsys@transformshift{4.478195in}{0.737928in}%
\pgfsys@useobject{currentmarker}{}%
\end{pgfscope}%
\begin{pgfscope}%
\pgfsys@transformshift{4.474440in}{0.775050in}%
\pgfsys@useobject{currentmarker}{}%
\end{pgfscope}%
\begin{pgfscope}%
\pgfsys@transformshift{4.456832in}{1.053846in}%
\pgfsys@useobject{currentmarker}{}%
\end{pgfscope}%
\begin{pgfscope}%
\pgfsys@transformshift{4.435467in}{1.355331in}%
\pgfsys@useobject{currentmarker}{}%
\end{pgfscope}%
\begin{pgfscope}%
\pgfsys@transformshift{4.417155in}{1.455934in}%
\pgfsys@useobject{currentmarker}{}%
\end{pgfscope}%
\begin{pgfscope}%
\pgfsys@transformshift{4.400253in}{1.348469in}%
\pgfsys@useobject{currentmarker}{}%
\end{pgfscope}%
\begin{pgfscope}%
\pgfsys@transformshift{4.378887in}{1.122722in}%
\pgfsys@useobject{currentmarker}{}%
\end{pgfscope}%
\begin{pgfscope}%
\pgfsys@transformshift{4.357758in}{0.805129in}%
\pgfsys@useobject{currentmarker}{}%
\end{pgfscope}%
\begin{pgfscope}%
\pgfsys@transformshift{4.340854in}{0.634329in}%
\pgfsys@useobject{currentmarker}{}%
\end{pgfscope}%
\begin{pgfscope}%
\pgfsys@transformshift{4.319959in}{0.526156in}%
\pgfsys@useobject{currentmarker}{}%
\end{pgfscope}%
\begin{pgfscope}%
\pgfsys@transformshift{4.300474in}{0.608964in}%
\pgfsys@useobject{currentmarker}{}%
\end{pgfscope}%
\begin{pgfscope}%
\pgfsys@transformshift{4.282866in}{0.785574in}%
\pgfsys@useobject{currentmarker}{}%
\end{pgfscope}%
\begin{pgfscope}%
\pgfsys@transformshift{4.261500in}{1.157498in}%
\pgfsys@useobject{currentmarker}{}%
\end{pgfscope}%
\begin{pgfscope}%
\pgfsys@transformshift{4.243658in}{1.382345in}%
\pgfsys@useobject{currentmarker}{}%
\end{pgfscope}%
\begin{pgfscope}%
\pgfsys@transformshift{4.225112in}{1.437226in}%
\pgfsys@useobject{currentmarker}{}%
\end{pgfscope}%
\begin{pgfscope}%
\pgfsys@transformshift{4.205861in}{1.300430in}%
\pgfsys@useobject{currentmarker}{}%
\end{pgfscope}%
\begin{pgfscope}%
\pgfsys@transformshift{4.184496in}{0.953186in}%
\pgfsys@useobject{currentmarker}{}%
\end{pgfscope}%
\begin{pgfscope}%
\pgfsys@transformshift{4.168062in}{0.724901in}%
\pgfsys@useobject{currentmarker}{}%
\end{pgfscope}%
\begin{pgfscope}%
\pgfsys@transformshift{4.146462in}{0.562335in}%
\pgfsys@useobject{currentmarker}{}%
\end{pgfscope}%
\begin{pgfscope}%
\pgfsys@transformshift{4.126273in}{0.532151in}%
\pgfsys@useobject{currentmarker}{}%
\end{pgfscope}%
\begin{pgfscope}%
\pgfsys@transformshift{4.109134in}{0.646355in}%
\pgfsys@useobject{currentmarker}{}%
\end{pgfscope}%
\begin{pgfscope}%
\pgfsys@transformshift{4.090117in}{0.867444in}%
\pgfsys@useobject{currentmarker}{}%
\end{pgfscope}%
\begin{pgfscope}%
\pgfsys@transformshift{4.071100in}{1.203489in}%
\pgfsys@useobject{currentmarker}{}%
\end{pgfscope}%
\begin{pgfscope}%
\pgfsys@transformshift{4.050206in}{1.409032in}%
\pgfsys@useobject{currentmarker}{}%
\end{pgfscope}%
\begin{pgfscope}%
\pgfsys@transformshift{4.033301in}{1.402727in}%
\pgfsys@useobject{currentmarker}{}%
\end{pgfscope}%
\begin{pgfscope}%
\pgfsys@transformshift{4.012407in}{1.187818in}%
\pgfsys@useobject{currentmarker}{}%
\end{pgfscope}%
\begin{pgfscope}%
\pgfsys@transformshift{3.992218in}{0.853301in}%
\pgfsys@useobject{currentmarker}{}%
\end{pgfscope}%
\begin{pgfscope}%
\pgfsys@transformshift{3.974374in}{0.656220in}%
\pgfsys@useobject{currentmarker}{}%
\end{pgfscope}%
\begin{pgfscope}%
\pgfsys@transformshift{3.953950in}{0.521727in}%
\pgfsys@useobject{currentmarker}{}%
\end{pgfscope}%
\begin{pgfscope}%
\pgfsys@transformshift{3.936811in}{0.527455in}%
\pgfsys@useobject{currentmarker}{}%
\end{pgfscope}%
\begin{pgfscope}%
\pgfsys@transformshift{3.916151in}{0.678505in}%
\pgfsys@useobject{currentmarker}{}%
\end{pgfscope}%
\begin{pgfscope}%
\pgfsys@transformshift{3.898074in}{0.886669in}%
\pgfsys@useobject{currentmarker}{}%
\end{pgfscope}%
\begin{pgfscope}%
\pgfsys@transformshift{3.873891in}{1.279622in}%
\pgfsys@useobject{currentmarker}{}%
\end{pgfscope}%
\begin{pgfscope}%
\pgfsys@transformshift{3.859335in}{1.396472in}%
\pgfsys@useobject{currentmarker}{}%
\end{pgfscope}%
\begin{pgfscope}%
\pgfsys@transformshift{3.837737in}{1.361403in}%
\pgfsys@useobject{currentmarker}{}%
\end{pgfscope}%
\begin{pgfscope}%
\pgfsys@transformshift{3.823181in}{1.160801in}%
\pgfsys@useobject{currentmarker}{}%
\end{pgfscope}%
\begin{pgfscope}%
\pgfsys@transformshift{3.800644in}{0.822384in}%
\pgfsys@useobject{currentmarker}{}%
\end{pgfscope}%
\begin{pgfscope}%
\pgfsys@transformshift{3.782565in}{0.641147in}%
\pgfsys@useobject{currentmarker}{}%
\end{pgfscope}%
\begin{pgfscope}%
\pgfsys@transformshift{3.761202in}{0.509886in}%
\pgfsys@useobject{currentmarker}{}%
\end{pgfscope}%
\begin{pgfscope}%
\pgfsys@transformshift{3.745471in}{0.516811in}%
\pgfsys@useobject{currentmarker}{}%
\end{pgfscope}%
\begin{pgfscope}%
\pgfsys@transformshift{3.724108in}{0.642497in}%
\pgfsys@useobject{currentmarker}{}%
\end{pgfscope}%
\begin{pgfscope}%
\pgfsys@transformshift{3.700630in}{0.918278in}%
\pgfsys@useobject{currentmarker}{}%
\end{pgfscope}%
\begin{pgfscope}%
\pgfsys@transformshift{3.686309in}{1.168926in}%
\pgfsys@useobject{currentmarker}{}%
\end{pgfscope}%
\begin{pgfscope}%
\pgfsys@transformshift{3.663066in}{1.374344in}%
\pgfsys@useobject{currentmarker}{}%
\end{pgfscope}%
\begin{pgfscope}%
\pgfsys@transformshift{3.647806in}{1.395471in}%
\pgfsys@useobject{currentmarker}{}%
\end{pgfscope}%
\begin{pgfscope}%
\pgfsys@transformshift{3.626912in}{1.337651in}%
\pgfsys@useobject{currentmarker}{}%
\end{pgfscope}%
\begin{pgfscope}%
\pgfsys@transformshift{3.610007in}{1.099313in}%
\pgfsys@useobject{currentmarker}{}%
\end{pgfscope}%
\begin{pgfscope}%
\pgfsys@transformshift{3.591696in}{0.823003in}%
\pgfsys@useobject{currentmarker}{}%
\end{pgfscope}%
\begin{pgfscope}%
\pgfsys@transformshift{3.566810in}{0.576509in}%
\pgfsys@useobject{currentmarker}{}%
\end{pgfscope}%
\begin{pgfscope}%
\pgfsys@transformshift{3.552723in}{0.526367in}%
\pgfsys@useobject{currentmarker}{}%
\end{pgfscope}%
\begin{pgfscope}%
\pgfsys@transformshift{3.531828in}{0.507698in}%
\pgfsys@useobject{currentmarker}{}%
\end{pgfscope}%
\begin{pgfscope}%
\pgfsys@transformshift{3.514926in}{0.575475in}%
\pgfsys@useobject{currentmarker}{}%
\end{pgfscope}%
\begin{pgfscope}%
\pgfsys@transformshift{3.493560in}{0.757485in}%
\pgfsys@useobject{currentmarker}{}%
\end{pgfscope}%
\begin{pgfscope}%
\pgfsys@transformshift{3.476423in}{0.940213in}%
\pgfsys@useobject{currentmarker}{}%
\end{pgfscope}%
\begin{pgfscope}%
\pgfsys@transformshift{3.454823in}{1.265971in}%
\pgfsys@useobject{currentmarker}{}%
\end{pgfscope}%
\begin{pgfscope}%
\pgfsys@transformshift{3.436981in}{1.380585in}%
\pgfsys@useobject{currentmarker}{}%
\end{pgfscope}%
\begin{pgfscope}%
\pgfsys@transformshift{3.415850in}{1.336875in}%
\pgfsys@useobject{currentmarker}{}%
\end{pgfscope}%
\begin{pgfscope}%
\pgfsys@transformshift{3.398947in}{1.140054in}%
\pgfsys@useobject{currentmarker}{}%
\end{pgfscope}%
\begin{pgfscope}%
\pgfsys@transformshift{3.378522in}{0.891159in}%
\pgfsys@useobject{currentmarker}{}%
\end{pgfscope}%
\begin{pgfscope}%
\pgfsys@transformshift{3.359271in}{0.697321in}%
\pgfsys@useobject{currentmarker}{}%
\end{pgfscope}%
\begin{pgfscope}%
\pgfsys@transformshift{3.338611in}{0.546167in}%
\pgfsys@useobject{currentmarker}{}%
\end{pgfscope}%
\begin{pgfscope}%
\pgfsys@transformshift{3.321237in}{0.491847in}%
\pgfsys@useobject{currentmarker}{}%
\end{pgfscope}%
\begin{pgfscope}%
\pgfsys@transformshift{3.301986in}{0.528116in}%
\pgfsys@useobject{currentmarker}{}%
\end{pgfscope}%
\begin{pgfscope}%
\pgfsys@transformshift{3.281561in}{0.633023in}%
\pgfsys@useobject{currentmarker}{}%
\end{pgfscope}%
\begin{pgfscope}%
\pgfsys@transformshift{3.263249in}{0.790544in}%
\pgfsys@useobject{currentmarker}{}%
\end{pgfscope}%
\begin{pgfscope}%
\pgfsys@transformshift{3.245407in}{1.055139in}%
\pgfsys@useobject{currentmarker}{}%
\end{pgfscope}%
\begin{pgfscope}%
\pgfsys@transformshift{3.225216in}{1.321766in}%
\pgfsys@useobject{currentmarker}{}%
\end{pgfscope}%
\begin{pgfscope}%
\pgfsys@transformshift{3.203852in}{1.382169in}%
\pgfsys@useobject{currentmarker}{}%
\end{pgfscope}%
\begin{pgfscope}%
\pgfsys@transformshift{3.186244in}{1.280077in}%
\pgfsys@useobject{currentmarker}{}%
\end{pgfscope}%
\begin{pgfscope}%
\pgfsys@transformshift{3.168871in}{1.048897in}%
\pgfsys@useobject{currentmarker}{}%
\end{pgfscope}%
\begin{pgfscope}%
\pgfsys@transformshift{3.147037in}{0.840844in}%
\pgfsys@useobject{currentmarker}{}%
\end{pgfscope}%
\begin{pgfscope}%
\pgfsys@transformshift{3.129194in}{0.648139in}%
\pgfsys@useobject{currentmarker}{}%
\end{pgfscope}%
\begin{pgfscope}%
\pgfsys@transformshift{3.110646in}{0.531643in}%
\pgfsys@useobject{currentmarker}{}%
\end{pgfscope}%
\begin{pgfscope}%
\pgfsys@transformshift{3.092804in}{0.491090in}%
\pgfsys@useobject{currentmarker}{}%
\end{pgfscope}%
\begin{pgfscope}%
\pgfsys@transformshift{3.070032in}{0.549216in}%
\pgfsys@useobject{currentmarker}{}%
\end{pgfscope}%
\begin{pgfscope}%
\pgfsys@transformshift{3.052424in}{0.682219in}%
\pgfsys@useobject{currentmarker}{}%
\end{pgfscope}%
\begin{pgfscope}%
\pgfsys@transformshift{3.034347in}{0.892268in}%
\pgfsys@useobject{currentmarker}{}%
\end{pgfscope}%
\begin{pgfscope}%
\pgfsys@transformshift{3.012513in}{1.170645in}%
\pgfsys@useobject{currentmarker}{}%
\end{pgfscope}%
\begin{pgfscope}%
\pgfsys@transformshift{2.993730in}{1.353777in}%
\pgfsys@useobject{currentmarker}{}%
\end{pgfscope}%
\begin{pgfscope}%
\pgfsys@transformshift{2.974948in}{1.375120in}%
\pgfsys@useobject{currentmarker}{}%
\end{pgfscope}%
\begin{pgfscope}%
\pgfsys@transformshift{2.956402in}{1.302658in}%
\pgfsys@useobject{currentmarker}{}%
\end{pgfscope}%
\begin{pgfscope}%
\pgfsys@transformshift{2.935977in}{1.025955in}%
\pgfsys@useobject{currentmarker}{}%
\end{pgfscope}%
\begin{pgfscope}%
\pgfsys@transformshift{2.919072in}{0.881924in}%
\pgfsys@useobject{currentmarker}{}%
\end{pgfscope}%
\begin{pgfscope}%
\pgfsys@transformshift{2.897943in}{0.650371in}%
\pgfsys@useobject{currentmarker}{}%
\end{pgfscope}%
\begin{pgfscope}%
\pgfsys@transformshift{2.878692in}{0.556316in}%
\pgfsys@useobject{currentmarker}{}%
\end{pgfscope}%
\begin{pgfscope}%
\pgfsys@transformshift{2.859441in}{0.507757in}%
\pgfsys@useobject{currentmarker}{}%
\end{pgfscope}%
\begin{pgfscope}%
\pgfsys@transformshift{2.841364in}{0.503541in}%
\pgfsys@useobject{currentmarker}{}%
\end{pgfscope}%
\begin{pgfscope}%
\pgfsys@transformshift{2.822111in}{0.667485in}%
\pgfsys@useobject{currentmarker}{}%
\end{pgfscope}%
\begin{pgfscope}%
\pgfsys@transformshift{2.800982in}{0.770929in}%
\pgfsys@useobject{currentmarker}{}%
\end{pgfscope}%
\begin{pgfscope}%
\pgfsys@transformshift{2.763888in}{1.286236in}%
\pgfsys@useobject{currentmarker}{}%
\end{pgfscope}%
\begin{pgfscope}%
\pgfsys@transformshift{2.745811in}{1.377644in}%
\pgfsys@useobject{currentmarker}{}%
\end{pgfscope}%
\begin{pgfscope}%
\pgfsys@transformshift{2.726089in}{1.321695in}%
\pgfsys@useobject{currentmarker}{}%
\end{pgfscope}%
\begin{pgfscope}%
\pgfsys@transformshift{2.704491in}{1.179914in}%
\pgfsys@useobject{currentmarker}{}%
\end{pgfscope}%
\begin{pgfscope}%
\pgfsys@transformshift{2.686649in}{0.944263in}%
\pgfsys@useobject{currentmarker}{}%
\end{pgfscope}%
\begin{pgfscope}%
\pgfsys@transformshift{2.627955in}{0.492850in}%
\pgfsys@useobject{currentmarker}{}%
\end{pgfscope}%
\begin{pgfscope}%
\pgfsys@transformshift{2.611756in}{0.501349in}%
\pgfsys@useobject{currentmarker}{}%
\end{pgfscope}%
\begin{pgfscope}%
\pgfsys@transformshift{2.592739in}{0.579518in}%
\pgfsys@useobject{currentmarker}{}%
\end{pgfscope}%
\begin{pgfscope}%
\pgfsys@transformshift{2.573723in}{0.717641in}%
\pgfsys@useobject{currentmarker}{}%
\end{pgfscope}%
\begin{pgfscope}%
\pgfsys@transformshift{2.551419in}{0.930797in}%
\pgfsys@useobject{currentmarker}{}%
\end{pgfscope}%
\begin{pgfscope}%
\pgfsys@transformshift{2.531934in}{1.201006in}%
\pgfsys@useobject{currentmarker}{}%
\end{pgfscope}%
\begin{pgfscope}%
\pgfsys@transformshift{2.513151in}{1.357948in}%
\pgfsys@useobject{currentmarker}{}%
\end{pgfscope}%
\begin{pgfscope}%
\pgfsys@transformshift{2.495778in}{1.369714in}%
\pgfsys@useobject{currentmarker}{}%
\end{pgfscope}%
\begin{pgfscope}%
\pgfsys@transformshift{2.473944in}{1.204821in}%
\pgfsys@useobject{currentmarker}{}%
\end{pgfscope}%
\begin{pgfscope}%
\pgfsys@transformshift{2.455632in}{0.950265in}%
\pgfsys@useobject{currentmarker}{}%
\end{pgfscope}%
\begin{pgfscope}%
\pgfsys@transformshift{2.435910in}{0.725760in}%
\pgfsys@useobject{currentmarker}{}%
\end{pgfscope}%
\begin{pgfscope}%
\pgfsys@transformshift{2.414781in}{1.379472in}%
\pgfsys@useobject{currentmarker}{}%
\end{pgfscope}%
\begin{pgfscope}%
\pgfsys@transformshift{2.399756in}{1.342790in}%
\pgfsys@useobject{currentmarker}{}%
\end{pgfscope}%
\begin{pgfscope}%
\pgfsys@transformshift{2.376513in}{1.066155in}%
\pgfsys@useobject{currentmarker}{}%
\end{pgfscope}%
\begin{pgfscope}%
\pgfsys@transformshift{2.359374in}{0.808725in}%
\pgfsys@useobject{currentmarker}{}%
\end{pgfscope}%
\begin{pgfscope}%
\pgfsys@transformshift{2.340828in}{0.642345in}%
\pgfsys@useobject{currentmarker}{}%
\end{pgfscope}%
\begin{pgfscope}%
\pgfsys@transformshift{2.321343in}{0.527735in}%
\pgfsys@useobject{currentmarker}{}%
\end{pgfscope}%
\begin{pgfscope}%
\pgfsys@transformshift{2.300212in}{0.493518in}%
\pgfsys@useobject{currentmarker}{}%
\end{pgfscope}%
\begin{pgfscope}%
\pgfsys@transformshift{2.285892in}{0.551694in}%
\pgfsys@useobject{currentmarker}{}%
\end{pgfscope}%
\begin{pgfscope}%
\pgfsys@transformshift{2.263824in}{0.707588in}%
\pgfsys@useobject{currentmarker}{}%
\end{pgfscope}%
\begin{pgfscope}%
\pgfsys@transformshift{2.244336in}{0.916237in}%
\pgfsys@useobject{currentmarker}{}%
\end{pgfscope}%
\begin{pgfscope}%
\pgfsys@transformshift{2.226025in}{1.144291in}%
\pgfsys@useobject{currentmarker}{}%
\end{pgfscope}%
\begin{pgfscope}%
\pgfsys@transformshift{2.206304in}{1.349933in}%
\pgfsys@useobject{currentmarker}{}%
\end{pgfscope}%
\begin{pgfscope}%
\pgfsys@transformshift{2.187288in}{1.379891in}%
\pgfsys@useobject{currentmarker}{}%
\end{pgfscope}%
\begin{pgfscope}%
\pgfsys@transformshift{2.168271in}{1.289719in}%
\pgfsys@useobject{currentmarker}{}%
\end{pgfscope}%
\begin{pgfscope}%
\pgfsys@transformshift{2.146437in}{1.003098in}%
\pgfsys@useobject{currentmarker}{}%
\end{pgfscope}%
\begin{pgfscope}%
\pgfsys@transformshift{2.129298in}{0.760768in}%
\pgfsys@useobject{currentmarker}{}%
\end{pgfscope}%
\begin{pgfscope}%
\pgfsys@transformshift{2.109578in}{0.615239in}%
\pgfsys@useobject{currentmarker}{}%
\end{pgfscope}%
\begin{pgfscope}%
\pgfsys@transformshift{2.087978in}{0.516114in}%
\pgfsys@useobject{currentmarker}{}%
\end{pgfscope}%
\begin{pgfscope}%
\pgfsys@transformshift{2.072718in}{0.598253in}%
\pgfsys@useobject{currentmarker}{}%
\end{pgfscope}%
\begin{pgfscope}%
\pgfsys@transformshift{2.050650in}{0.534157in}%
\pgfsys@useobject{currentmarker}{}%
\end{pgfscope}%
\begin{pgfscope}%
\pgfsys@transformshift{2.033042in}{0.491703in}%
\pgfsys@useobject{currentmarker}{}%
\end{pgfscope}%
\begin{pgfscope}%
\pgfsys@transformshift{2.014025in}{0.544808in}%
\pgfsys@useobject{currentmarker}{}%
\end{pgfscope}%
\begin{pgfscope}%
\pgfsys@transformshift{1.995008in}{0.613048in}%
\pgfsys@useobject{currentmarker}{}%
\end{pgfscope}%
\begin{pgfscope}%
\pgfsys@transformshift{1.974348in}{0.806180in}%
\pgfsys@useobject{currentmarker}{}%
\end{pgfscope}%
\begin{pgfscope}%
\pgfsys@transformshift{1.954863in}{1.112252in}%
\pgfsys@useobject{currentmarker}{}%
\end{pgfscope}%
\begin{pgfscope}%
\pgfsys@transformshift{1.937489in}{1.301302in}%
\pgfsys@useobject{currentmarker}{}%
\end{pgfscope}%
\begin{pgfscope}%
\pgfsys@transformshift{1.919412in}{1.388560in}%
\pgfsys@useobject{currentmarker}{}%
\end{pgfscope}%
\begin{pgfscope}%
\pgfsys@transformshift{1.898518in}{1.330303in}%
\pgfsys@useobject{currentmarker}{}%
\end{pgfscope}%
\begin{pgfscope}%
\pgfsys@transformshift{1.880439in}{1.112180in}%
\pgfsys@useobject{currentmarker}{}%
\end{pgfscope}%
\begin{pgfscope}%
\pgfsys@transformshift{1.862831in}{0.839826in}%
\pgfsys@useobject{currentmarker}{}%
\end{pgfscope}%
\begin{pgfscope}%
\pgfsys@transformshift{1.843111in}{0.673812in}%
\pgfsys@useobject{currentmarker}{}%
\end{pgfscope}%
\begin{pgfscope}%
\pgfsys@transformshift{1.823859in}{0.570495in}%
\pgfsys@useobject{currentmarker}{}%
\end{pgfscope}%
\begin{pgfscope}%
\pgfsys@transformshift{1.800382in}{0.496049in}%
\pgfsys@useobject{currentmarker}{}%
\end{pgfscope}%
\begin{pgfscope}%
\pgfsys@transformshift{1.784417in}{0.528746in}%
\pgfsys@useobject{currentmarker}{}%
\end{pgfscope}%
\begin{pgfscope}%
\pgfsys@transformshift{1.762348in}{0.649854in}%
\pgfsys@useobject{currentmarker}{}%
\end{pgfscope}%
\begin{pgfscope}%
\pgfsys@transformshift{1.744272in}{0.848907in}%
\pgfsys@useobject{currentmarker}{}%
\end{pgfscope}%
\begin{pgfscope}%
\pgfsys@transformshift{1.724786in}{1.129146in}%
\pgfsys@useobject{currentmarker}{}%
\end{pgfscope}%
\begin{pgfscope}%
\pgfsys@transformshift{1.706238in}{1.316728in}%
\pgfsys@useobject{currentmarker}{}%
\end{pgfscope}%
\begin{pgfscope}%
\pgfsys@transformshift{1.688161in}{1.397404in}%
\pgfsys@useobject{currentmarker}{}%
\end{pgfscope}%
\begin{pgfscope}%
\pgfsys@transformshift{1.666561in}{1.322597in}%
\pgfsys@useobject{currentmarker}{}%
\end{pgfscope}%
\begin{pgfscope}%
\pgfsys@transformshift{1.647545in}{1.129141in}%
\pgfsys@useobject{currentmarker}{}%
\end{pgfscope}%
\begin{pgfscope}%
\pgfsys@transformshift{1.629233in}{0.959772in}%
\pgfsys@useobject{currentmarker}{}%
\end{pgfscope}%
\begin{pgfscope}%
\pgfsys@transformshift{1.611391in}{0.771363in}%
\pgfsys@useobject{currentmarker}{}%
\end{pgfscope}%
\begin{pgfscope}%
\pgfsys@transformshift{1.589322in}{0.604891in}%
\pgfsys@useobject{currentmarker}{}%
\end{pgfscope}%
\begin{pgfscope}%
\pgfsys@transformshift{1.571948in}{0.521653in}%
\pgfsys@useobject{currentmarker}{}%
\end{pgfscope}%
\begin{pgfscope}%
\pgfsys@transformshift{1.552463in}{0.508321in}%
\pgfsys@useobject{currentmarker}{}%
\end{pgfscope}%
\begin{pgfscope}%
\pgfsys@transformshift{1.534855in}{0.578089in}%
\pgfsys@useobject{currentmarker}{}%
\end{pgfscope}%
\begin{pgfscope}%
\pgfsys@transformshift{1.514898in}{0.684279in}%
\pgfsys@useobject{currentmarker}{}%
\end{pgfscope}%
\begin{pgfscope}%
\pgfsys@transformshift{1.496352in}{0.869484in}%
\pgfsys@useobject{currentmarker}{}%
\end{pgfscope}%
\begin{pgfscope}%
\pgfsys@transformshift{1.473578in}{1.162127in}%
\pgfsys@useobject{currentmarker}{}%
\end{pgfscope}%
\begin{pgfscope}%
\pgfsys@transformshift{1.456441in}{0.681845in}%
\pgfsys@useobject{currentmarker}{}%
\end{pgfscope}%
\begin{pgfscope}%
\pgfsys@transformshift{1.436250in}{0.642777in}%
\pgfsys@useobject{currentmarker}{}%
\end{pgfscope}%
\begin{pgfscope}%
\pgfsys@transformshift{1.416999in}{0.845142in}%
\pgfsys@useobject{currentmarker}{}%
\end{pgfscope}%
\begin{pgfscope}%
\pgfsys@transformshift{1.400565in}{1.075016in}%
\pgfsys@useobject{currentmarker}{}%
\end{pgfscope}%
\begin{pgfscope}%
\pgfsys@transformshift{1.383661in}{1.278495in}%
\pgfsys@useobject{currentmarker}{}%
\end{pgfscope}%
\begin{pgfscope}%
\pgfsys@transformshift{1.358540in}{1.409606in}%
\pgfsys@useobject{currentmarker}{}%
\end{pgfscope}%
\begin{pgfscope}%
\pgfsys@transformshift{1.343281in}{1.385910in}%
\pgfsys@useobject{currentmarker}{}%
\end{pgfscope}%
\begin{pgfscope}%
\pgfsys@transformshift{1.323558in}{1.290970in}%
\pgfsys@useobject{currentmarker}{}%
\end{pgfscope}%
\begin{pgfscope}%
\pgfsys@transformshift{1.305013in}{1.036705in}%
\pgfsys@useobject{currentmarker}{}%
\end{pgfscope}%
\begin{pgfscope}%
\pgfsys@transformshift{1.283884in}{0.781312in}%
\pgfsys@useobject{currentmarker}{}%
\end{pgfscope}%
\begin{pgfscope}%
\pgfsys@transformshift{1.261344in}{0.614964in}%
\pgfsys@useobject{currentmarker}{}%
\end{pgfscope}%
\begin{pgfscope}%
\pgfsys@transformshift{1.246554in}{0.534016in}%
\pgfsys@useobject{currentmarker}{}%
\end{pgfscope}%
\begin{pgfscope}%
\pgfsys@transformshift{1.227771in}{0.504658in}%
\pgfsys@useobject{currentmarker}{}%
\end{pgfscope}%
\begin{pgfscope}%
\pgfsys@transformshift{1.206877in}{0.574888in}%
\pgfsys@useobject{currentmarker}{}%
\end{pgfscope}%
\begin{pgfscope}%
\pgfsys@transformshift{1.187157in}{0.703815in}%
\pgfsys@useobject{currentmarker}{}%
\end{pgfscope}%
\begin{pgfscope}%
\pgfsys@transformshift{1.168609in}{0.902955in}%
\pgfsys@useobject{currentmarker}{}%
\end{pgfscope}%
\begin{pgfscope}%
\pgfsys@transformshift{1.150063in}{1.168542in}%
\pgfsys@useobject{currentmarker}{}%
\end{pgfscope}%
\begin{pgfscope}%
\pgfsys@transformshift{1.128463in}{1.357965in}%
\pgfsys@useobject{currentmarker}{}%
\end{pgfscope}%
\begin{pgfscope}%
\pgfsys@transformshift{1.112264in}{1.420675in}%
\pgfsys@useobject{currentmarker}{}%
\end{pgfscope}%
\begin{pgfscope}%
\pgfsys@transformshift{1.091604in}{1.390853in}%
\pgfsys@useobject{currentmarker}{}%
\end{pgfscope}%
\begin{pgfscope}%
\pgfsys@transformshift{1.073527in}{1.233685in}%
\pgfsys@useobject{currentmarker}{}%
\end{pgfscope}%
\begin{pgfscope}%
\pgfsys@transformshift{1.054042in}{1.003401in}%
\pgfsys@useobject{currentmarker}{}%
\end{pgfscope}%
\begin{pgfscope}%
\pgfsys@transformshift{1.035494in}{0.806409in}%
\pgfsys@useobject{currentmarker}{}%
\end{pgfscope}%
\begin{pgfscope}%
\pgfsys@transformshift{1.010842in}{0.628188in}%
\pgfsys@useobject{currentmarker}{}%
\end{pgfscope}%
\begin{pgfscope}%
\pgfsys@transformshift{0.995348in}{1.034644in}%
\pgfsys@useobject{currentmarker}{}%
\end{pgfscope}%
\begin{pgfscope}%
\pgfsys@transformshift{0.976331in}{0.810153in}%
\pgfsys@useobject{currentmarker}{}%
\end{pgfscope}%
\begin{pgfscope}%
\pgfsys@transformshift{0.957549in}{0.661883in}%
\pgfsys@useobject{currentmarker}{}%
\end{pgfscope}%
\begin{pgfscope}%
\pgfsys@transformshift{0.936420in}{0.541881in}%
\pgfsys@useobject{currentmarker}{}%
\end{pgfscope}%
\begin{pgfscope}%
\pgfsys@transformshift{0.920924in}{0.521226in}%
\pgfsys@useobject{currentmarker}{}%
\end{pgfscope}%
\begin{pgfscope}%
\pgfsys@transformshift{0.900970in}{0.605029in}%
\pgfsys@useobject{currentmarker}{}%
\end{pgfscope}%
\begin{pgfscope}%
\pgfsys@transformshift{0.882422in}{0.695083in}%
\pgfsys@useobject{currentmarker}{}%
\end{pgfscope}%
\begin{pgfscope}%
\pgfsys@transformshift{0.862702in}{0.893778in}%
\pgfsys@useobject{currentmarker}{}%
\end{pgfscope}%
\begin{pgfscope}%
\pgfsys@transformshift{0.840162in}{1.135936in}%
\pgfsys@useobject{currentmarker}{}%
\end{pgfscope}%
\begin{pgfscope}%
\pgfsys@transformshift{0.823728in}{1.347063in}%
\pgfsys@useobject{currentmarker}{}%
\end{pgfscope}%
\begin{pgfscope}%
\pgfsys@transformshift{0.804008in}{1.438517in}%
\pgfsys@useobject{currentmarker}{}%
\end{pgfscope}%
\begin{pgfscope}%
\pgfsys@transformshift{0.785931in}{1.426506in}%
\pgfsys@useobject{currentmarker}{}%
\end{pgfscope}%
\begin{pgfscope}%
\pgfsys@transformshift{0.761983in}{1.235138in}%
\pgfsys@useobject{currentmarker}{}%
\end{pgfscope}%
\begin{pgfscope}%
\pgfsys@transformshift{0.745784in}{1.062589in}%
\pgfsys@useobject{currentmarker}{}%
\end{pgfscope}%
\begin{pgfscope}%
\pgfsys@transformshift{0.727472in}{0.840460in}%
\pgfsys@useobject{currentmarker}{}%
\end{pgfscope}%
\begin{pgfscope}%
\pgfsys@transformshift{0.708456in}{0.692642in}%
\pgfsys@useobject{currentmarker}{}%
\end{pgfscope}%
\begin{pgfscope}%
\pgfsys@transformshift{0.685918in}{0.568224in}%
\pgfsys@useobject{currentmarker}{}%
\end{pgfscope}%
\begin{pgfscope}%
\pgfsys@transformshift{0.668545in}{0.530441in}%
\pgfsys@useobject{currentmarker}{}%
\end{pgfscope}%
\begin{pgfscope}%
\pgfsys@transformshift{0.650468in}{0.582347in}%
\pgfsys@useobject{currentmarker}{}%
\end{pgfscope}%
\begin{pgfscope}%
\pgfsys@transformshift{0.647416in}{0.583101in}%
\pgfsys@useobject{currentmarker}{}%
\end{pgfscope}%
\begin{pgfscope}%
\pgfsys@transformshift{0.654223in}{0.547149in}%
\pgfsys@useobject{currentmarker}{}%
\end{pgfscope}%
\begin{pgfscope}%
\pgfsys@transformshift{0.675588in}{0.586242in}%
\pgfsys@useobject{currentmarker}{}%
\end{pgfscope}%
\begin{pgfscope}%
\pgfsys@transformshift{0.694368in}{0.744207in}%
\pgfsys@useobject{currentmarker}{}%
\end{pgfscope}%
\begin{pgfscope}%
\pgfsys@transformshift{0.712447in}{0.982058in}%
\pgfsys@useobject{currentmarker}{}%
\end{pgfscope}%
\begin{pgfscope}%
\pgfsys@transformshift{0.734282in}{1.327334in}%
\pgfsys@useobject{currentmarker}{}%
\end{pgfscope}%
\begin{pgfscope}%
\pgfsys@transformshift{0.750481in}{1.438849in}%
\pgfsys@useobject{currentmarker}{}%
\end{pgfscope}%
\begin{pgfscope}%
\pgfsys@transformshift{0.771610in}{1.370684in}%
\pgfsys@useobject{currentmarker}{}%
\end{pgfscope}%
\begin{pgfscope}%
\pgfsys@transformshift{0.790626in}{1.101014in}%
\pgfsys@useobject{currentmarker}{}%
\end{pgfscope}%
\begin{pgfscope}%
\pgfsys@transformshift{0.810112in}{0.769965in}%
\pgfsys@useobject{currentmarker}{}%
\end{pgfscope}%
\begin{pgfscope}%
\pgfsys@transformshift{0.828424in}{0.584181in}%
\pgfsys@useobject{currentmarker}{}%
\end{pgfscope}%
\begin{pgfscope}%
\pgfsys@transformshift{0.846971in}{0.519039in}%
\pgfsys@useobject{currentmarker}{}%
\end{pgfscope}%
\begin{pgfscope}%
\pgfsys@transformshift{0.866457in}{0.632778in}%
\pgfsys@useobject{currentmarker}{}%
\end{pgfscope}%
\begin{pgfscope}%
\pgfsys@transformshift{0.885943in}{0.823580in}%
\pgfsys@useobject{currentmarker}{}%
\end{pgfscope}%
\begin{pgfscope}%
\pgfsys@transformshift{0.905196in}{1.160047in}%
\pgfsys@useobject{currentmarker}{}%
\end{pgfscope}%
\begin{pgfscope}%
\pgfsys@transformshift{0.924211in}{1.390826in}%
\pgfsys@useobject{currentmarker}{}%
\end{pgfscope}%
\begin{pgfscope}%
\pgfsys@transformshift{0.944636in}{1.413417in}%
\pgfsys@useobject{currentmarker}{}%
\end{pgfscope}%
\begin{pgfscope}%
\pgfsys@transformshift{0.962715in}{1.216562in}%
\pgfsys@useobject{currentmarker}{}%
\end{pgfscope}%
\begin{pgfscope}%
\pgfsys@transformshift{0.982435in}{0.896489in}%
\pgfsys@useobject{currentmarker}{}%
\end{pgfscope}%
\begin{pgfscope}%
\pgfsys@transformshift{1.002392in}{0.659109in}%
\pgfsys@useobject{currentmarker}{}%
\end{pgfscope}%
\begin{pgfscope}%
\pgfsys@transformshift{1.020938in}{0.525913in}%
\pgfsys@useobject{currentmarker}{}%
\end{pgfscope}%
\begin{pgfscope}%
\pgfsys@transformshift{1.039015in}{0.540815in}%
\pgfsys@useobject{currentmarker}{}%
\end{pgfscope}%
\begin{pgfscope}%
\pgfsys@transformshift{1.058737in}{0.682831in}%
\pgfsys@useobject{currentmarker}{}%
\end{pgfscope}%
\begin{pgfscope}%
\pgfsys@transformshift{1.079865in}{0.946811in}%
\pgfsys@useobject{currentmarker}{}%
\end{pgfscope}%
\begin{pgfscope}%
\pgfsys@transformshift{1.100291in}{1.273643in}%
\pgfsys@useobject{currentmarker}{}%
\end{pgfscope}%
\begin{pgfscope}%
\pgfsys@transformshift{1.115550in}{1.206497in}%
\pgfsys@useobject{currentmarker}{}%
\end{pgfscope}%
\begin{pgfscope}%
\pgfsys@transformshift{1.137619in}{1.406008in}%
\pgfsys@useobject{currentmarker}{}%
\end{pgfscope}%
\begin{pgfscope}%
\pgfsys@transformshift{1.154289in}{1.382788in}%
\pgfsys@useobject{currentmarker}{}%
\end{pgfscope}%
\begin{pgfscope}%
\pgfsys@transformshift{1.173540in}{1.157231in}%
\pgfsys@useobject{currentmarker}{}%
\end{pgfscope}%
\begin{pgfscope}%
\pgfsys@transformshift{1.194904in}{0.792986in}%
\pgfsys@useobject{currentmarker}{}%
\end{pgfscope}%
\begin{pgfscope}%
\pgfsys@transformshift{1.214860in}{0.595338in}%
\pgfsys@useobject{currentmarker}{}%
\end{pgfscope}%
\begin{pgfscope}%
\pgfsys@transformshift{1.231294in}{0.507149in}%
\pgfsys@useobject{currentmarker}{}%
\end{pgfscope}%
\begin{pgfscope}%
\pgfsys@transformshift{1.250076in}{0.541157in}%
\pgfsys@useobject{currentmarker}{}%
\end{pgfscope}%
\begin{pgfscope}%
\pgfsys@transformshift{1.274726in}{0.729332in}%
\pgfsys@useobject{currentmarker}{}%
\end{pgfscope}%
\begin{pgfscope}%
\pgfsys@transformshift{1.293977in}{1.001124in}%
\pgfsys@useobject{currentmarker}{}%
\end{pgfscope}%
\begin{pgfscope}%
\pgfsys@transformshift{1.310411in}{1.259624in}%
\pgfsys@useobject{currentmarker}{}%
\end{pgfscope}%
\begin{pgfscope}%
\pgfsys@transformshift{1.325907in}{1.389846in}%
\pgfsys@useobject{currentmarker}{}%
\end{pgfscope}%
\begin{pgfscope}%
\pgfsys@transformshift{1.347741in}{1.357218in}%
\pgfsys@useobject{currentmarker}{}%
\end{pgfscope}%
\begin{pgfscope}%
\pgfsys@transformshift{1.366992in}{1.114418in}%
\pgfsys@useobject{currentmarker}{}%
\end{pgfscope}%
\begin{pgfscope}%
\pgfsys@transformshift{1.385538in}{0.807365in}%
\pgfsys@useobject{currentmarker}{}%
\end{pgfscope}%
\begin{pgfscope}%
\pgfsys@transformshift{1.405260in}{0.604387in}%
\pgfsys@useobject{currentmarker}{}%
\end{pgfscope}%
\begin{pgfscope}%
\pgfsys@transformshift{1.425215in}{0.502291in}%
\pgfsys@useobject{currentmarker}{}%
\end{pgfscope}%
\begin{pgfscope}%
\pgfsys@transformshift{1.447049in}{0.553896in}%
\pgfsys@useobject{currentmarker}{}%
\end{pgfscope}%
\begin{pgfscope}%
\pgfsys@transformshift{1.463719in}{0.649248in}%
\pgfsys@useobject{currentmarker}{}%
\end{pgfscope}%
\begin{pgfscope}%
\pgfsys@transformshift{1.480856in}{0.831690in}%
\pgfsys@useobject{currentmarker}{}%
\end{pgfscope}%
\begin{pgfscope}%
\pgfsys@transformshift{1.500342in}{1.106045in}%
\pgfsys@useobject{currentmarker}{}%
\end{pgfscope}%
\begin{pgfscope}%
\pgfsys@transformshift{1.522647in}{1.357133in}%
\pgfsys@useobject{currentmarker}{}%
\end{pgfscope}%
\begin{pgfscope}%
\pgfsys@transformshift{1.537907in}{1.389398in}%
\pgfsys@useobject{currentmarker}{}%
\end{pgfscope}%
\begin{pgfscope}%
\pgfsys@transformshift{1.562793in}{1.199446in}%
\pgfsys@useobject{currentmarker}{}%
\end{pgfscope}%
\begin{pgfscope}%
\pgfsys@transformshift{1.578521in}{0.932304in}%
\pgfsys@useobject{currentmarker}{}%
\end{pgfscope}%
\begin{pgfscope}%
\pgfsys@transformshift{1.598009in}{0.690808in}%
\pgfsys@useobject{currentmarker}{}%
\end{pgfscope}%
\begin{pgfscope}%
\pgfsys@transformshift{1.617026in}{0.584061in}%
\pgfsys@useobject{currentmarker}{}%
\end{pgfscope}%
\begin{pgfscope}%
\pgfsys@transformshift{1.636277in}{0.883183in}%
\pgfsys@useobject{currentmarker}{}%
\end{pgfscope}%
\begin{pgfscope}%
\pgfsys@transformshift{1.654588in}{0.685086in}%
\pgfsys@useobject{currentmarker}{}%
\end{pgfscope}%
\begin{pgfscope}%
\pgfsys@transformshift{1.672665in}{0.549931in}%
\pgfsys@useobject{currentmarker}{}%
\end{pgfscope}%
\begin{pgfscope}%
\pgfsys@transformshift{1.694970in}{0.493281in}%
\pgfsys@useobject{currentmarker}{}%
\end{pgfscope}%
\begin{pgfscope}%
\pgfsys@transformshift{1.715394in}{0.546903in}%
\pgfsys@useobject{currentmarker}{}%
\end{pgfscope}%
\begin{pgfscope}%
\pgfsys@transformshift{1.733236in}{0.674933in}%
\pgfsys@useobject{currentmarker}{}%
\end{pgfscope}%
\begin{pgfscope}%
\pgfsys@transformshift{1.751315in}{0.885235in}%
\pgfsys@useobject{currentmarker}{}%
\end{pgfscope}%
\begin{pgfscope}%
\pgfsys@transformshift{1.770566in}{1.151726in}%
\pgfsys@useobject{currentmarker}{}%
\end{pgfscope}%
\begin{pgfscope}%
\pgfsys@transformshift{1.790286in}{1.355293in}%
\pgfsys@useobject{currentmarker}{}%
\end{pgfscope}%
\begin{pgfscope}%
\pgfsys@transformshift{1.808129in}{1.384375in}%
\pgfsys@useobject{currentmarker}{}%
\end{pgfscope}%
\begin{pgfscope}%
\pgfsys@transformshift{1.829494in}{1.247099in}%
\pgfsys@useobject{currentmarker}{}%
\end{pgfscope}%
\begin{pgfscope}%
\pgfsys@transformshift{1.848277in}{0.943608in}%
\pgfsys@useobject{currentmarker}{}%
\end{pgfscope}%
\begin{pgfscope}%
\pgfsys@transformshift{1.867291in}{0.703853in}%
\pgfsys@useobject{currentmarker}{}%
\end{pgfscope}%
\begin{pgfscope}%
\pgfsys@transformshift{1.885839in}{0.565621in}%
\pgfsys@useobject{currentmarker}{}%
\end{pgfscope}%
\begin{pgfscope}%
\pgfsys@transformshift{1.904621in}{0.497887in}%
\pgfsys@useobject{currentmarker}{}%
\end{pgfscope}%
\begin{pgfscope}%
\pgfsys@transformshift{1.922933in}{0.514619in}%
\pgfsys@useobject{currentmarker}{}%
\end{pgfscope}%
\begin{pgfscope}%
\pgfsys@transformshift{1.945001in}{0.634799in}%
\pgfsys@useobject{currentmarker}{}%
\end{pgfscope}%
\begin{pgfscope}%
\pgfsys@transformshift{1.963549in}{0.810815in}%
\pgfsys@useobject{currentmarker}{}%
\end{pgfscope}%
\begin{pgfscope}%
\pgfsys@transformshift{1.980452in}{1.028821in}%
\pgfsys@useobject{currentmarker}{}%
\end{pgfscope}%
\begin{pgfscope}%
\pgfsys@transformshift{2.000877in}{1.312991in}%
\pgfsys@useobject{currentmarker}{}%
\end{pgfscope}%
\begin{pgfscope}%
\pgfsys@transformshift{2.021303in}{1.382292in}%
\pgfsys@useobject{currentmarker}{}%
\end{pgfscope}%
\begin{pgfscope}%
\pgfsys@transformshift{2.038442in}{1.311474in}%
\pgfsys@useobject{currentmarker}{}%
\end{pgfscope}%
\begin{pgfscope}%
\pgfsys@transformshift{2.056753in}{1.060491in}%
\pgfsys@useobject{currentmarker}{}%
\end{pgfscope}%
\begin{pgfscope}%
\pgfsys@transformshift{2.079291in}{0.772814in}%
\pgfsys@useobject{currentmarker}{}%
\end{pgfscope}%
\begin{pgfscope}%
\pgfsys@transformshift{2.096196in}{0.628653in}%
\pgfsys@useobject{currentmarker}{}%
\end{pgfscope}%
\begin{pgfscope}%
\pgfsys@transformshift{2.115916in}{0.517629in}%
\pgfsys@useobject{currentmarker}{}%
\end{pgfscope}%
\begin{pgfscope}%
\pgfsys@transformshift{2.136341in}{0.492332in}%
\pgfsys@useobject{currentmarker}{}%
\end{pgfscope}%
\begin{pgfscope}%
\pgfsys@transformshift{2.154889in}{0.569521in}%
\pgfsys@useobject{currentmarker}{}%
\end{pgfscope}%
\begin{pgfscope}%
\pgfsys@transformshift{2.172261in}{0.654554in}%
\pgfsys@useobject{currentmarker}{}%
\end{pgfscope}%
\begin{pgfscope}%
\pgfsys@transformshift{2.193392in}{0.853675in}%
\pgfsys@useobject{currentmarker}{}%
\end{pgfscope}%
\begin{pgfscope}%
\pgfsys@transformshift{2.211234in}{1.132278in}%
\pgfsys@useobject{currentmarker}{}%
\end{pgfscope}%
\begin{pgfscope}%
\pgfsys@transformshift{2.233537in}{1.347047in}%
\pgfsys@useobject{currentmarker}{}%
\end{pgfscope}%
\begin{pgfscope}%
\pgfsys@transformshift{2.252319in}{1.376190in}%
\pgfsys@useobject{currentmarker}{}%
\end{pgfscope}%
\begin{pgfscope}%
\pgfsys@transformshift{2.271571in}{1.259078in}%
\pgfsys@useobject{currentmarker}{}%
\end{pgfscope}%
\begin{pgfscope}%
\pgfsys@transformshift{2.291762in}{0.951104in}%
\pgfsys@useobject{currentmarker}{}%
\end{pgfscope}%
\begin{pgfscope}%
\pgfsys@transformshift{2.307725in}{0.762628in}%
\pgfsys@useobject{currentmarker}{}%
\end{pgfscope}%
\begin{pgfscope}%
\pgfsys@transformshift{2.328384in}{0.592403in}%
\pgfsys@useobject{currentmarker}{}%
\end{pgfscope}%
\begin{pgfscope}%
\pgfsys@transformshift{2.349984in}{0.495487in}%
\pgfsys@useobject{currentmarker}{}%
\end{pgfscope}%
\begin{pgfscope}%
\pgfsys@transformshift{2.367358in}{0.499079in}%
\pgfsys@useobject{currentmarker}{}%
\end{pgfscope}%
\begin{pgfscope}%
\pgfsys@transformshift{2.384966in}{0.578726in}%
\pgfsys@useobject{currentmarker}{}%
\end{pgfscope}%
\begin{pgfscope}%
\pgfsys@transformshift{2.403748in}{0.693011in}%
\pgfsys@useobject{currentmarker}{}%
\end{pgfscope}%
\begin{pgfscope}%
\pgfsys@transformshift{2.424408in}{0.930057in}%
\pgfsys@useobject{currentmarker}{}%
\end{pgfscope}%
\begin{pgfscope}%
\pgfsys@transformshift{2.442016in}{1.186435in}%
\pgfsys@useobject{currentmarker}{}%
\end{pgfscope}%
\begin{pgfscope}%
\pgfsys@transformshift{2.459622in}{1.349328in}%
\pgfsys@useobject{currentmarker}{}%
\end{pgfscope}%
\begin{pgfscope}%
\pgfsys@transformshift{2.485448in}{1.353398in}%
\pgfsys@useobject{currentmarker}{}%
\end{pgfscope}%
\begin{pgfscope}%
\pgfsys@transformshift{2.501882in}{1.281745in}%
\pgfsys@useobject{currentmarker}{}%
\end{pgfscope}%
\begin{pgfscope}%
\pgfsys@transformshift{2.519959in}{1.048171in}%
\pgfsys@useobject{currentmarker}{}%
\end{pgfscope}%
\begin{pgfscope}%
\pgfsys@transformshift{2.541089in}{0.771134in}%
\pgfsys@useobject{currentmarker}{}%
\end{pgfscope}%
\begin{pgfscope}%
\pgfsys@transformshift{2.555646in}{0.631232in}%
\pgfsys@useobject{currentmarker}{}%
\end{pgfscope}%
\begin{pgfscope}%
\pgfsys@transformshift{2.576306in}{0.520202in}%
\pgfsys@useobject{currentmarker}{}%
\end{pgfscope}%
\begin{pgfscope}%
\pgfsys@transformshift{2.597434in}{0.498834in}%
\pgfsys@useobject{currentmarker}{}%
\end{pgfscope}%
\begin{pgfscope}%
\pgfsys@transformshift{2.614808in}{0.489066in}%
\pgfsys@useobject{currentmarker}{}%
\end{pgfscope}%
\begin{pgfscope}%
\pgfsys@transformshift{2.633354in}{0.513334in}%
\pgfsys@useobject{currentmarker}{}%
\end{pgfscope}%
\begin{pgfscope}%
\pgfsys@transformshift{2.654485in}{0.633990in}%
\pgfsys@useobject{currentmarker}{}%
\end{pgfscope}%
\begin{pgfscope}%
\pgfsys@transformshift{2.672327in}{0.789559in}%
\pgfsys@useobject{currentmarker}{}%
\end{pgfscope}%
\begin{pgfscope}%
\pgfsys@transformshift{2.694161in}{1.054829in}%
\pgfsys@useobject{currentmarker}{}%
\end{pgfscope}%
\begin{pgfscope}%
\pgfsys@transformshift{2.712707in}{1.290560in}%
\pgfsys@useobject{currentmarker}{}%
\end{pgfscope}%
\begin{pgfscope}%
\pgfsys@transformshift{2.730081in}{1.376732in}%
\pgfsys@useobject{currentmarker}{}%
\end{pgfscope}%
\begin{pgfscope}%
\pgfsys@transformshift{2.750741in}{1.324359in}%
\pgfsys@useobject{currentmarker}{}%
\end{pgfscope}%
\begin{pgfscope}%
\pgfsys@transformshift{2.769289in}{1.226106in}%
\pgfsys@useobject{currentmarker}{}%
\end{pgfscope}%
\begin{pgfscope}%
\pgfsys@transformshift{2.789712in}{0.911870in}%
\pgfsys@useobject{currentmarker}{}%
\end{pgfscope}%
\begin{pgfscope}%
\pgfsys@transformshift{2.808494in}{0.717886in}%
\pgfsys@useobject{currentmarker}{}%
\end{pgfscope}%
\begin{pgfscope}%
\pgfsys@transformshift{2.827511in}{0.571949in}%
\pgfsys@useobject{currentmarker}{}%
\end{pgfscope}%
\begin{pgfscope}%
\pgfsys@transformshift{2.846997in}{0.503639in}%
\pgfsys@useobject{currentmarker}{}%
\end{pgfscope}%
\begin{pgfscope}%
\pgfsys@transformshift{2.864370in}{0.507784in}%
\pgfsys@useobject{currentmarker}{}%
\end{pgfscope}%
\begin{pgfscope}%
\pgfsys@transformshift{2.888787in}{0.597170in}%
\pgfsys@useobject{currentmarker}{}%
\end{pgfscope}%
\begin{pgfscope}%
\pgfsys@transformshift{2.904047in}{0.706175in}%
\pgfsys@useobject{currentmarker}{}%
\end{pgfscope}%
\begin{pgfscope}%
\pgfsys@transformshift{2.926585in}{0.920642in}%
\pgfsys@useobject{currentmarker}{}%
\end{pgfscope}%
\begin{pgfscope}%
\pgfsys@transformshift{2.944663in}{1.160471in}%
\pgfsys@useobject{currentmarker}{}%
\end{pgfscope}%
\begin{pgfscope}%
\pgfsys@transformshift{2.962271in}{1.332058in}%
\pgfsys@useobject{currentmarker}{}%
\end{pgfscope}%
\begin{pgfscope}%
\pgfsys@transformshift{2.978234in}{0.661988in}%
\pgfsys@useobject{currentmarker}{}%
\end{pgfscope}%
\begin{pgfscope}%
\pgfsys@transformshift{2.999365in}{0.859934in}%
\pgfsys@useobject{currentmarker}{}%
\end{pgfscope}%
\begin{pgfscope}%
\pgfsys@transformshift{3.020728in}{1.055832in}%
\pgfsys@useobject{currentmarker}{}%
\end{pgfscope}%
\begin{pgfscope}%
\pgfsys@transformshift{3.038102in}{1.301587in}%
\pgfsys@useobject{currentmarker}{}%
\end{pgfscope}%
\begin{pgfscope}%
\pgfsys@transformshift{3.058293in}{1.387198in}%
\pgfsys@useobject{currentmarker}{}%
\end{pgfscope}%
\begin{pgfscope}%
\pgfsys@transformshift{3.074727in}{1.277358in}%
\pgfsys@useobject{currentmarker}{}%
\end{pgfscope}%
\begin{pgfscope}%
\pgfsys@transformshift{3.097030in}{1.107391in}%
\pgfsys@useobject{currentmarker}{}%
\end{pgfscope}%
\begin{pgfscope}%
\pgfsys@transformshift{3.113698in}{0.841441in}%
\pgfsys@useobject{currentmarker}{}%
\end{pgfscope}%
\begin{pgfscope}%
\pgfsys@transformshift{3.134358in}{0.636276in}%
\pgfsys@useobject{currentmarker}{}%
\end{pgfscope}%
\begin{pgfscope}%
\pgfsys@transformshift{3.155018in}{0.532738in}%
\pgfsys@useobject{currentmarker}{}%
\end{pgfscope}%
\begin{pgfscope}%
\pgfsys@transformshift{3.173800in}{0.490777in}%
\pgfsys@useobject{currentmarker}{}%
\end{pgfscope}%
\begin{pgfscope}%
\pgfsys@transformshift{3.194695in}{0.566372in}%
\pgfsys@useobject{currentmarker}{}%
\end{pgfscope}%
\begin{pgfscope}%
\pgfsys@transformshift{3.213008in}{0.658431in}%
\pgfsys@useobject{currentmarker}{}%
\end{pgfscope}%
\begin{pgfscope}%
\pgfsys@transformshift{3.230382in}{0.793479in}%
\pgfsys@useobject{currentmarker}{}%
\end{pgfscope}%
\begin{pgfscope}%
\pgfsys@transformshift{3.253857in}{1.125990in}%
\pgfsys@useobject{currentmarker}{}%
\end{pgfscope}%
\begin{pgfscope}%
\pgfsys@transformshift{3.268884in}{1.296817in}%
\pgfsys@useobject{currentmarker}{}%
\end{pgfscope}%
\begin{pgfscope}%
\pgfsys@transformshift{3.290247in}{1.394400in}%
\pgfsys@useobject{currentmarker}{}%
\end{pgfscope}%
\begin{pgfscope}%
\pgfsys@transformshift{3.307621in}{1.345058in}%
\pgfsys@useobject{currentmarker}{}%
\end{pgfscope}%
\begin{pgfscope}%
\pgfsys@transformshift{3.325463in}{1.224104in}%
\pgfsys@useobject{currentmarker}{}%
\end{pgfscope}%
\begin{pgfscope}%
\pgfsys@transformshift{3.347767in}{0.902653in}%
\pgfsys@useobject{currentmarker}{}%
\end{pgfscope}%
\begin{pgfscope}%
\pgfsys@transformshift{3.365609in}{0.698351in}%
\pgfsys@useobject{currentmarker}{}%
\end{pgfscope}%
\begin{pgfscope}%
\pgfsys@transformshift{3.383452in}{0.594650in}%
\pgfsys@useobject{currentmarker}{}%
\end{pgfscope}%
\begin{pgfscope}%
\pgfsys@transformshift{3.403643in}{0.504258in}%
\pgfsys@useobject{currentmarker}{}%
\end{pgfscope}%
\begin{pgfscope}%
\pgfsys@transformshift{3.421719in}{0.511648in}%
\pgfsys@useobject{currentmarker}{}%
\end{pgfscope}%
\begin{pgfscope}%
\pgfsys@transformshift{3.443554in}{0.610651in}%
\pgfsys@useobject{currentmarker}{}%
\end{pgfscope}%
\begin{pgfscope}%
\pgfsys@transformshift{3.461162in}{0.744755in}%
\pgfsys@useobject{currentmarker}{}%
\end{pgfscope}%
\begin{pgfscope}%
\pgfsys@transformshift{3.479944in}{0.939049in}%
\pgfsys@useobject{currentmarker}{}%
\end{pgfscope}%
\begin{pgfscope}%
\pgfsys@transformshift{3.499899in}{1.201948in}%
\pgfsys@useobject{currentmarker}{}%
\end{pgfscope}%
\begin{pgfscope}%
\pgfsys@transformshift{3.519152in}{1.340016in}%
\pgfsys@useobject{currentmarker}{}%
\end{pgfscope}%
\begin{pgfscope}%
\pgfsys@transformshift{3.539575in}{1.406676in}%
\pgfsys@useobject{currentmarker}{}%
\end{pgfscope}%
\begin{pgfscope}%
\pgfsys@transformshift{3.558123in}{1.351587in}%
\pgfsys@useobject{currentmarker}{}%
\end{pgfscope}%
\begin{pgfscope}%
\pgfsys@transformshift{3.578314in}{1.143453in}%
\pgfsys@useobject{currentmarker}{}%
\end{pgfscope}%
\begin{pgfscope}%
\pgfsys@transformshift{3.596157in}{0.921997in}%
\pgfsys@useobject{currentmarker}{}%
\end{pgfscope}%
\begin{pgfscope}%
\pgfsys@transformshift{3.613999in}{0.758986in}%
\pgfsys@useobject{currentmarker}{}%
\end{pgfscope}%
\begin{pgfscope}%
\pgfsys@transformshift{3.635599in}{0.612113in}%
\pgfsys@useobject{currentmarker}{}%
\end{pgfscope}%
\begin{pgfscope}%
\pgfsys@transformshift{3.653441in}{0.526967in}%
\pgfsys@useobject{currentmarker}{}%
\end{pgfscope}%
\begin{pgfscope}%
\pgfsys@transformshift{3.671284in}{0.504328in}%
\pgfsys@useobject{currentmarker}{}%
\end{pgfscope}%
\begin{pgfscope}%
\pgfsys@transformshift{3.693118in}{0.571366in}%
\pgfsys@useobject{currentmarker}{}%
\end{pgfscope}%
\begin{pgfscope}%
\pgfsys@transformshift{3.710726in}{0.651352in}%
\pgfsys@useobject{currentmarker}{}%
\end{pgfscope}%
\begin{pgfscope}%
\pgfsys@transformshift{3.731855in}{0.829534in}%
\pgfsys@useobject{currentmarker}{}%
\end{pgfscope}%
\begin{pgfscope}%
\pgfsys@transformshift{3.753455in}{1.057749in}%
\pgfsys@useobject{currentmarker}{}%
\end{pgfscope}%
\begin{pgfscope}%
\pgfsys@transformshift{3.771297in}{1.285017in}%
\pgfsys@useobject{currentmarker}{}%
\end{pgfscope}%
\begin{pgfscope}%
\pgfsys@transformshift{3.788434in}{1.357740in}%
\pgfsys@useobject{currentmarker}{}%
\end{pgfscope}%
\begin{pgfscope}%
\pgfsys@transformshift{3.806513in}{1.413669in}%
\pgfsys@useobject{currentmarker}{}%
\end{pgfscope}%
\begin{pgfscope}%
\pgfsys@transformshift{3.828582in}{1.396399in}%
\pgfsys@useobject{currentmarker}{}%
\end{pgfscope}%
\begin{pgfscope}%
\pgfsys@transformshift{3.845719in}{1.292610in}%
\pgfsys@useobject{currentmarker}{}%
\end{pgfscope}%
\begin{pgfscope}%
\pgfsys@transformshift{3.866613in}{1.027248in}%
\pgfsys@useobject{currentmarker}{}%
\end{pgfscope}%
\begin{pgfscope}%
\pgfsys@transformshift{3.884927in}{0.807595in}%
\pgfsys@useobject{currentmarker}{}%
\end{pgfscope}%
\begin{pgfscope}%
\pgfsys@transformshift{3.903238in}{0.662947in}%
\pgfsys@useobject{currentmarker}{}%
\end{pgfscope}%
\begin{pgfscope}%
\pgfsys@transformshift{3.920846in}{0.574436in}%
\pgfsys@useobject{currentmarker}{}%
\end{pgfscope}%
\begin{pgfscope}%
\pgfsys@transformshift{3.945029in}{0.512072in}%
\pgfsys@useobject{currentmarker}{}%
\end{pgfscope}%
\begin{pgfscope}%
\pgfsys@transformshift{3.960054in}{0.536111in}%
\pgfsys@useobject{currentmarker}{}%
\end{pgfscope}%
\begin{pgfscope}%
\pgfsys@transformshift{3.982357in}{0.639137in}%
\pgfsys@useobject{currentmarker}{}%
\end{pgfscope}%
\begin{pgfscope}%
\pgfsys@transformshift{3.999496in}{0.722550in}%
\pgfsys@useobject{currentmarker}{}%
\end{pgfscope}%
\begin{pgfscope}%
\pgfsys@transformshift{4.016868in}{0.867895in}%
\pgfsys@useobject{currentmarker}{}%
\end{pgfscope}%
\begin{pgfscope}%
\pgfsys@transformshift{4.038467in}{1.117981in}%
\pgfsys@useobject{currentmarker}{}%
\end{pgfscope}%
\begin{pgfscope}%
\pgfsys@transformshift{4.059831in}{1.342578in}%
\pgfsys@useobject{currentmarker}{}%
\end{pgfscope}%
\begin{pgfscope}%
\pgfsys@transformshift{4.077204in}{1.422897in}%
\pgfsys@useobject{currentmarker}{}%
\end{pgfscope}%
\begin{pgfscope}%
\pgfsys@transformshift{4.096456in}{1.437911in}%
\pgfsys@useobject{currentmarker}{}%
\end{pgfscope}%
\begin{pgfscope}%
\pgfsys@transformshift{4.119698in}{1.332808in}%
\pgfsys@useobject{currentmarker}{}%
\end{pgfscope}%
\begin{pgfscope}%
\pgfsys@transformshift{4.134723in}{1.243912in}%
\pgfsys@useobject{currentmarker}{}%
\end{pgfscope}%
\begin{pgfscope}%
\pgfsys@transformshift{4.153037in}{1.000286in}%
\pgfsys@useobject{currentmarker}{}%
\end{pgfscope}%
\begin{pgfscope}%
\pgfsys@transformshift{4.174166in}{0.789204in}%
\pgfsys@useobject{currentmarker}{}%
\end{pgfscope}%
\begin{pgfscope}%
\pgfsys@transformshift{4.191539in}{0.704134in}%
\pgfsys@useobject{currentmarker}{}%
\end{pgfscope}%
\begin{pgfscope}%
\pgfsys@transformshift{4.209616in}{0.582344in}%
\pgfsys@useobject{currentmarker}{}%
\end{pgfscope}%
\begin{pgfscope}%
\pgfsys@transformshift{4.230981in}{0.529872in}%
\pgfsys@useobject{currentmarker}{}%
\end{pgfscope}%
\begin{pgfscope}%
\pgfsys@transformshift{4.249058in}{0.527147in}%
\pgfsys@useobject{currentmarker}{}%
\end{pgfscope}%
\begin{pgfscope}%
\pgfsys@transformshift{4.269484in}{0.620881in}%
\pgfsys@useobject{currentmarker}{}%
\end{pgfscope}%
\begin{pgfscope}%
\pgfsys@transformshift{4.287326in}{0.736828in}%
\pgfsys@useobject{currentmarker}{}%
\end{pgfscope}%
\begin{pgfscope}%
\pgfsys@transformshift{4.305872in}{1.208858in}%
\pgfsys@useobject{currentmarker}{}%
\end{pgfscope}%
\begin{pgfscope}%
\pgfsys@transformshift{4.325360in}{0.917833in}%
\pgfsys@useobject{currentmarker}{}%
\end{pgfscope}%
\begin{pgfscope}%
\pgfsys@transformshift{4.349072in}{0.653951in}%
\pgfsys@useobject{currentmarker}{}%
\end{pgfscope}%
\begin{pgfscope}%
\pgfsys@transformshift{4.366209in}{0.547664in}%
\pgfsys@useobject{currentmarker}{}%
\end{pgfscope}%
\begin{pgfscope}%
\pgfsys@transformshift{4.384053in}{0.535654in}%
\pgfsys@useobject{currentmarker}{}%
\end{pgfscope}%
\begin{pgfscope}%
\pgfsys@transformshift{4.401425in}{0.636506in}%
\pgfsys@useobject{currentmarker}{}%
\end{pgfscope}%
\begin{pgfscope}%
\pgfsys@transformshift{4.420442in}{0.727370in}%
\pgfsys@useobject{currentmarker}{}%
\end{pgfscope}%
\begin{pgfscope}%
\pgfsys@transformshift{4.442276in}{0.980626in}%
\pgfsys@useobject{currentmarker}{}%
\end{pgfscope}%
\begin{pgfscope}%
\pgfsys@transformshift{4.460587in}{1.249321in}%
\pgfsys@useobject{currentmarker}{}%
\end{pgfscope}%
\begin{pgfscope}%
\pgfsys@transformshift{4.479604in}{1.417801in}%
\pgfsys@useobject{currentmarker}{}%
\end{pgfscope}%
\begin{pgfscope}%
\pgfsys@transformshift{4.477726in}{1.411023in}%
\pgfsys@useobject{currentmarker}{}%
\end{pgfscope}%
\begin{pgfscope}%
\pgfsys@transformshift{4.474909in}{1.374273in}%
\pgfsys@useobject{currentmarker}{}%
\end{pgfscope}%
\begin{pgfscope}%
\pgfsys@transformshift{4.455423in}{1.088442in}%
\pgfsys@useobject{currentmarker}{}%
\end{pgfscope}%
\begin{pgfscope}%
\pgfsys@transformshift{4.434294in}{0.762545in}%
\pgfsys@useobject{currentmarker}{}%
\end{pgfscope}%
\begin{pgfscope}%
\pgfsys@transformshift{4.412460in}{0.567095in}%
\pgfsys@useobject{currentmarker}{}%
\end{pgfscope}%
\begin{pgfscope}%
\pgfsys@transformshift{4.396964in}{0.525988in}%
\pgfsys@useobject{currentmarker}{}%
\end{pgfscope}%
\begin{pgfscope}%
\pgfsys@transformshift{4.378418in}{0.643895in}%
\pgfsys@useobject{currentmarker}{}%
\end{pgfscope}%
\begin{pgfscope}%
\pgfsys@transformshift{4.357053in}{0.906293in}%
\pgfsys@useobject{currentmarker}{}%
\end{pgfscope}%
\begin{pgfscope}%
\pgfsys@transformshift{4.338976in}{1.235771in}%
\pgfsys@useobject{currentmarker}{}%
\end{pgfscope}%
\begin{pgfscope}%
\pgfsys@transformshift{4.321134in}{1.416588in}%
\pgfsys@useobject{currentmarker}{}%
\end{pgfscope}%
\begin{pgfscope}%
\pgfsys@transformshift{4.298594in}{1.396273in}%
\pgfsys@useobject{currentmarker}{}%
\end{pgfscope}%
\begin{pgfscope}%
\pgfsys@transformshift{4.282866in}{1.197566in}%
\pgfsys@useobject{currentmarker}{}%
\end{pgfscope}%
\begin{pgfscope}%
\pgfsys@transformshift{4.264318in}{0.891944in}%
\pgfsys@useobject{currentmarker}{}%
\end{pgfscope}%
\begin{pgfscope}%
\pgfsys@transformshift{4.243658in}{0.650638in}%
\pgfsys@useobject{currentmarker}{}%
\end{pgfscope}%
\begin{pgfscope}%
\pgfsys@transformshift{4.221824in}{0.517078in}%
\pgfsys@useobject{currentmarker}{}%
\end{pgfscope}%
\begin{pgfscope}%
\pgfsys@transformshift{4.204216in}{0.568597in}%
\pgfsys@useobject{currentmarker}{}%
\end{pgfscope}%
\begin{pgfscope}%
\pgfsys@transformshift{4.187079in}{0.722847in}%
\pgfsys@useobject{currentmarker}{}%
\end{pgfscope}%
\begin{pgfscope}%
\pgfsys@transformshift{4.165713in}{1.020805in}%
\pgfsys@useobject{currentmarker}{}%
\end{pgfscope}%
\begin{pgfscope}%
\pgfsys@transformshift{4.148576in}{1.315217in}%
\pgfsys@useobject{currentmarker}{}%
\end{pgfscope}%
\begin{pgfscope}%
\pgfsys@transformshift{4.127445in}{1.426413in}%
\pgfsys@useobject{currentmarker}{}%
\end{pgfscope}%
\begin{pgfscope}%
\pgfsys@transformshift{4.107960in}{1.306220in}%
\pgfsys@useobject{currentmarker}{}%
\end{pgfscope}%
\begin{pgfscope}%
\pgfsys@transformshift{4.088943in}{0.990412in}%
\pgfsys@useobject{currentmarker}{}%
\end{pgfscope}%
\begin{pgfscope}%
\pgfsys@transformshift{4.072275in}{0.718551in}%
\pgfsys@useobject{currentmarker}{}%
\end{pgfscope}%
\begin{pgfscope}%
\pgfsys@transformshift{4.049972in}{0.569712in}%
\pgfsys@useobject{currentmarker}{}%
\end{pgfscope}%
\begin{pgfscope}%
\pgfsys@transformshift{4.033538in}{0.506127in}%
\pgfsys@useobject{currentmarker}{}%
\end{pgfscope}%
\begin{pgfscope}%
\pgfsys@transformshift{4.011469in}{0.605582in}%
\pgfsys@useobject{currentmarker}{}%
\end{pgfscope}%
\begin{pgfscope}%
\pgfsys@transformshift{3.994330in}{0.776640in}%
\pgfsys@useobject{currentmarker}{}%
\end{pgfscope}%
\begin{pgfscope}%
\pgfsys@transformshift{3.973670in}{1.110849in}%
\pgfsys@useobject{currentmarker}{}%
\end{pgfscope}%
\begin{pgfscope}%
\pgfsys@transformshift{3.956062in}{1.342830in}%
\pgfsys@useobject{currentmarker}{}%
\end{pgfscope}%
\begin{pgfscope}%
\pgfsys@transformshift{3.935402in}{1.408692in}%
\pgfsys@useobject{currentmarker}{}%
\end{pgfscope}%
\begin{pgfscope}%
\pgfsys@transformshift{3.915682in}{1.330362in}%
\pgfsys@useobject{currentmarker}{}%
\end{pgfscope}%
\begin{pgfscope}%
\pgfsys@transformshift{3.897840in}{1.411203in}%
\pgfsys@useobject{currentmarker}{}%
\end{pgfscope}%
\begin{pgfscope}%
\pgfsys@transformshift{3.877414in}{1.296558in}%
\pgfsys@useobject{currentmarker}{}%
\end{pgfscope}%
\begin{pgfscope}%
\pgfsys@transformshift{3.860980in}{1.026621in}%
\pgfsys@useobject{currentmarker}{}%
\end{pgfscope}%
\begin{pgfscope}%
\pgfsys@transformshift{3.838912in}{0.739603in}%
\pgfsys@useobject{currentmarker}{}%
\end{pgfscope}%
\begin{pgfscope}%
\pgfsys@transformshift{3.821067in}{0.584819in}%
\pgfsys@useobject{currentmarker}{}%
\end{pgfscope}%
\begin{pgfscope}%
\pgfsys@transformshift{3.797121in}{0.500770in}%
\pgfsys@useobject{currentmarker}{}%
\end{pgfscope}%
\begin{pgfscope}%
\pgfsys@transformshift{3.783036in}{0.555593in}%
\pgfsys@useobject{currentmarker}{}%
\end{pgfscope}%
\begin{pgfscope}%
\pgfsys@transformshift{3.763785in}{0.713099in}%
\pgfsys@useobject{currentmarker}{}%
\end{pgfscope}%
\begin{pgfscope}%
\pgfsys@transformshift{3.744768in}{0.977489in}%
\pgfsys@useobject{currentmarker}{}%
\end{pgfscope}%
\begin{pgfscope}%
\pgfsys@transformshift{3.722228in}{1.296621in}%
\pgfsys@useobject{currentmarker}{}%
\end{pgfscope}%
\begin{pgfscope}%
\pgfsys@transformshift{3.705560in}{1.394452in}%
\pgfsys@useobject{currentmarker}{}%
\end{pgfscope}%
\begin{pgfscope}%
\pgfsys@transformshift{3.686778in}{1.362101in}%
\pgfsys@useobject{currentmarker}{}%
\end{pgfscope}%
\begin{pgfscope}%
\pgfsys@transformshift{3.667292in}{1.118641in}%
\pgfsys@useobject{currentmarker}{}%
\end{pgfscope}%
\begin{pgfscope}%
\pgfsys@transformshift{3.648981in}{0.853432in}%
\pgfsys@useobject{currentmarker}{}%
\end{pgfscope}%
\begin{pgfscope}%
\pgfsys@transformshift{3.627615in}{0.663139in}%
\pgfsys@useobject{currentmarker}{}%
\end{pgfscope}%
\begin{pgfscope}%
\pgfsys@transformshift{3.611182in}{0.548764in}%
\pgfsys@useobject{currentmarker}{}%
\end{pgfscope}%
\begin{pgfscope}%
\pgfsys@transformshift{3.592634in}{0.494836in}%
\pgfsys@useobject{currentmarker}{}%
\end{pgfscope}%
\begin{pgfscope}%
\pgfsys@transformshift{3.568688in}{0.566348in}%
\pgfsys@useobject{currentmarker}{}%
\end{pgfscope}%
\begin{pgfscope}%
\pgfsys@transformshift{3.552019in}{0.648871in}%
\pgfsys@useobject{currentmarker}{}%
\end{pgfscope}%
\begin{pgfscope}%
\pgfsys@transformshift{3.530420in}{0.859072in}%
\pgfsys@useobject{currentmarker}{}%
\end{pgfscope}%
\begin{pgfscope}%
\pgfsys@transformshift{3.513517in}{1.089590in}%
\pgfsys@useobject{currentmarker}{}%
\end{pgfscope}%
\begin{pgfscope}%
\pgfsys@transformshift{3.497318in}{1.275090in}%
\pgfsys@useobject{currentmarker}{}%
\end{pgfscope}%
\begin{pgfscope}%
\pgfsys@transformshift{3.474778in}{1.392501in}%
\pgfsys@useobject{currentmarker}{}%
\end{pgfscope}%
\begin{pgfscope}%
\pgfsys@transformshift{3.451537in}{1.319439in}%
\pgfsys@useobject{currentmarker}{}%
\end{pgfscope}%
\begin{pgfscope}%
\pgfsys@transformshift{3.436981in}{1.134573in}%
\pgfsys@useobject{currentmarker}{}%
\end{pgfscope}%
\begin{pgfscope}%
\pgfsys@transformshift{3.415616in}{0.863605in}%
\pgfsys@useobject{currentmarker}{}%
\end{pgfscope}%
\begin{pgfscope}%
\pgfsys@transformshift{3.398242in}{0.671343in}%
\pgfsys@useobject{currentmarker}{}%
\end{pgfscope}%
\begin{pgfscope}%
\pgfsys@transformshift{3.377348in}{0.533625in}%
\pgfsys@useobject{currentmarker}{}%
\end{pgfscope}%
\begin{pgfscope}%
\pgfsys@transformshift{3.355984in}{0.587054in}%
\pgfsys@useobject{currentmarker}{}%
\end{pgfscope}%
\begin{pgfscope}%
\pgfsys@transformshift{3.341194in}{0.510330in}%
\pgfsys@useobject{currentmarker}{}%
\end{pgfscope}%
\begin{pgfscope}%
\pgfsys@transformshift{3.320768in}{0.511277in}%
\pgfsys@useobject{currentmarker}{}%
\end{pgfscope}%
\begin{pgfscope}%
\pgfsys@transformshift{3.303864in}{0.594435in}%
\pgfsys@useobject{currentmarker}{}%
\end{pgfscope}%
\begin{pgfscope}%
\pgfsys@transformshift{3.286021in}{0.814645in}%
\pgfsys@useobject{currentmarker}{}%
\end{pgfscope}%
\begin{pgfscope}%
\pgfsys@transformshift{3.263953in}{1.068276in}%
\pgfsys@useobject{currentmarker}{}%
\end{pgfscope}%
\begin{pgfscope}%
\pgfsys@transformshift{3.244936in}{1.324057in}%
\pgfsys@useobject{currentmarker}{}%
\end{pgfscope}%
\begin{pgfscope}%
\pgfsys@transformshift{3.226624in}{1.382168in}%
\pgfsys@useobject{currentmarker}{}%
\end{pgfscope}%
\begin{pgfscope}%
\pgfsys@transformshift{3.207608in}{1.279356in}%
\pgfsys@useobject{currentmarker}{}%
\end{pgfscope}%
\begin{pgfscope}%
\pgfsys@transformshift{3.187888in}{0.999065in}%
\pgfsys@useobject{currentmarker}{}%
\end{pgfscope}%
\begin{pgfscope}%
\pgfsys@transformshift{3.168636in}{0.743006in}%
\pgfsys@useobject{currentmarker}{}%
\end{pgfscope}%
\begin{pgfscope}%
\pgfsys@transformshift{3.147271in}{0.584720in}%
\pgfsys@useobject{currentmarker}{}%
\end{pgfscope}%
\begin{pgfscope}%
\pgfsys@transformshift{3.130603in}{0.502240in}%
\pgfsys@useobject{currentmarker}{}%
\end{pgfscope}%
\begin{pgfscope}%
\pgfsys@transformshift{3.108534in}{0.494865in}%
\pgfsys@useobject{currentmarker}{}%
\end{pgfscope}%
\begin{pgfscope}%
\pgfsys@transformshift{3.090926in}{0.567055in}%
\pgfsys@useobject{currentmarker}{}%
\end{pgfscope}%
\begin{pgfscope}%
\pgfsys@transformshift{3.069561in}{0.729257in}%
\pgfsys@useobject{currentmarker}{}%
\end{pgfscope}%
\begin{pgfscope}%
\pgfsys@transformshift{3.051718in}{0.953766in}%
\pgfsys@useobject{currentmarker}{}%
\end{pgfscope}%
\begin{pgfscope}%
\pgfsys@transformshift{3.030355in}{1.251512in}%
\pgfsys@useobject{currentmarker}{}%
\end{pgfscope}%
\begin{pgfscope}%
\pgfsys@transformshift{3.014859in}{1.363668in}%
\pgfsys@useobject{currentmarker}{}%
\end{pgfscope}%
\begin{pgfscope}%
\pgfsys@transformshift{2.996548in}{1.365620in}%
\pgfsys@useobject{currentmarker}{}%
\end{pgfscope}%
\begin{pgfscope}%
\pgfsys@transformshift{2.974010in}{1.168963in}%
\pgfsys@useobject{currentmarker}{}%
\end{pgfscope}%
\begin{pgfscope}%
\pgfsys@transformshift{2.955462in}{0.931499in}%
\pgfsys@useobject{currentmarker}{}%
\end{pgfscope}%
\begin{pgfscope}%
\pgfsys@transformshift{2.936680in}{0.721984in}%
\pgfsys@useobject{currentmarker}{}%
\end{pgfscope}%
\begin{pgfscope}%
\pgfsys@transformshift{2.918134in}{0.583802in}%
\pgfsys@useobject{currentmarker}{}%
\end{pgfscope}%
\begin{pgfscope}%
\pgfsys@transformshift{2.896769in}{0.489600in}%
\pgfsys@useobject{currentmarker}{}%
\end{pgfscope}%
\begin{pgfscope}%
\pgfsys@transformshift{2.878692in}{0.510694in}%
\pgfsys@useobject{currentmarker}{}%
\end{pgfscope}%
\begin{pgfscope}%
\pgfsys@transformshift{2.859441in}{0.586541in}%
\pgfsys@useobject{currentmarker}{}%
\end{pgfscope}%
\begin{pgfscope}%
\pgfsys@transformshift{2.841364in}{0.729138in}%
\pgfsys@useobject{currentmarker}{}%
\end{pgfscope}%
\begin{pgfscope}%
\pgfsys@transformshift{2.821642in}{0.992059in}%
\pgfsys@useobject{currentmarker}{}%
\end{pgfscope}%
\begin{pgfscope}%
\pgfsys@transformshift{2.803096in}{1.131293in}%
\pgfsys@useobject{currentmarker}{}%
\end{pgfscope}%
\begin{pgfscope}%
\pgfsys@transformshift{2.777975in}{1.366052in}%
\pgfsys@useobject{currentmarker}{}%
\end{pgfscope}%
\begin{pgfscope}%
\pgfsys@transformshift{2.763183in}{1.370780in}%
\pgfsys@useobject{currentmarker}{}%
\end{pgfscope}%
\begin{pgfscope}%
\pgfsys@transformshift{2.743463in}{1.378402in}%
\pgfsys@useobject{currentmarker}{}%
\end{pgfscope}%
\begin{pgfscope}%
\pgfsys@transformshift{2.726560in}{1.294557in}%
\pgfsys@useobject{currentmarker}{}%
\end{pgfscope}%
\begin{pgfscope}%
\pgfsys@transformshift{2.703317in}{1.007455in}%
\pgfsys@useobject{currentmarker}{}%
\end{pgfscope}%
\begin{pgfscope}%
\pgfsys@transformshift{2.689935in}{0.839581in}%
\pgfsys@useobject{currentmarker}{}%
\end{pgfscope}%
\begin{pgfscope}%
\pgfsys@transformshift{2.669744in}{0.634885in}%
\pgfsys@useobject{currentmarker}{}%
\end{pgfscope}%
\begin{pgfscope}%
\pgfsys@transformshift{2.646736in}{0.513722in}%
\pgfsys@useobject{currentmarker}{}%
\end{pgfscope}%
\begin{pgfscope}%
\pgfsys@transformshift{2.628893in}{0.483686in}%
\pgfsys@useobject{currentmarker}{}%
\end{pgfscope}%
\begin{pgfscope}%
\pgfsys@transformshift{2.610816in}{0.545648in}%
\pgfsys@useobject{currentmarker}{}%
\end{pgfscope}%
\begin{pgfscope}%
\pgfsys@transformshift{2.592974in}{0.645766in}%
\pgfsys@useobject{currentmarker}{}%
\end{pgfscope}%
\begin{pgfscope}%
\pgfsys@transformshift{2.570905in}{0.877386in}%
\pgfsys@useobject{currentmarker}{}%
\end{pgfscope}%
\begin{pgfscope}%
\pgfsys@transformshift{2.554940in}{1.104603in}%
\pgfsys@useobject{currentmarker}{}%
\end{pgfscope}%
\begin{pgfscope}%
\pgfsys@transformshift{2.534515in}{1.335279in}%
\pgfsys@useobject{currentmarker}{}%
\end{pgfscope}%
\begin{pgfscope}%
\pgfsys@transformshift{2.514089in}{1.375824in}%
\pgfsys@useobject{currentmarker}{}%
\end{pgfscope}%
\begin{pgfscope}%
\pgfsys@transformshift{2.495543in}{1.285967in}%
\pgfsys@useobject{currentmarker}{}%
\end{pgfscope}%
\begin{pgfscope}%
\pgfsys@transformshift{2.476996in}{1.057579in}%
\pgfsys@useobject{currentmarker}{}%
\end{pgfscope}%
\begin{pgfscope}%
\pgfsys@transformshift{2.454693in}{0.818342in}%
\pgfsys@useobject{currentmarker}{}%
\end{pgfscope}%
\begin{pgfscope}%
\pgfsys@transformshift{2.436850in}{0.640030in}%
\pgfsys@useobject{currentmarker}{}%
\end{pgfscope}%
\begin{pgfscope}%
\pgfsys@transformshift{2.419711in}{0.535790in}%
\pgfsys@useobject{currentmarker}{}%
\end{pgfscope}%
\begin{pgfscope}%
\pgfsys@transformshift{2.396470in}{0.484901in}%
\pgfsys@useobject{currentmarker}{}%
\end{pgfscope}%
\begin{pgfscope}%
\pgfsys@transformshift{2.380034in}{0.523034in}%
\pgfsys@useobject{currentmarker}{}%
\end{pgfscope}%
\begin{pgfscope}%
\pgfsys@transformshift{2.360314in}{0.604485in}%
\pgfsys@useobject{currentmarker}{}%
\end{pgfscope}%
\begin{pgfscope}%
\pgfsys@transformshift{2.342472in}{0.723909in}%
\pgfsys@useobject{currentmarker}{}%
\end{pgfscope}%
\begin{pgfscope}%
\pgfsys@transformshift{2.320637in}{1.005212in}%
\pgfsys@useobject{currentmarker}{}%
\end{pgfscope}%
\begin{pgfscope}%
\pgfsys@transformshift{2.301855in}{1.256687in}%
\pgfsys@useobject{currentmarker}{}%
\end{pgfscope}%
\begin{pgfscope}%
\pgfsys@transformshift{2.282604in}{1.343072in}%
\pgfsys@useobject{currentmarker}{}%
\end{pgfscope}%
\begin{pgfscope}%
\pgfsys@transformshift{2.264058in}{1.375061in}%
\pgfsys@useobject{currentmarker}{}%
\end{pgfscope}%
\begin{pgfscope}%
\pgfsys@transformshift{2.245041in}{1.338112in}%
\pgfsys@useobject{currentmarker}{}%
\end{pgfscope}%
\begin{pgfscope}%
\pgfsys@transformshift{2.226494in}{1.160631in}%
\pgfsys@useobject{currentmarker}{}%
\end{pgfscope}%
\begin{pgfscope}%
\pgfsys@transformshift{2.205599in}{0.917599in}%
\pgfsys@useobject{currentmarker}{}%
\end{pgfscope}%
\begin{pgfscope}%
\pgfsys@transformshift{2.186113in}{0.727389in}%
\pgfsys@useobject{currentmarker}{}%
\end{pgfscope}%
\begin{pgfscope}%
\pgfsys@transformshift{2.167800in}{0.629730in}%
\pgfsys@useobject{currentmarker}{}%
\end{pgfscope}%
\begin{pgfscope}%
\pgfsys@transformshift{2.149020in}{0.532134in}%
\pgfsys@useobject{currentmarker}{}%
\end{pgfscope}%
\begin{pgfscope}%
\pgfsys@transformshift{2.129298in}{0.489360in}%
\pgfsys@useobject{currentmarker}{}%
\end{pgfscope}%
\begin{pgfscope}%
\pgfsys@transformshift{2.109109in}{0.546822in}%
\pgfsys@useobject{currentmarker}{}%
\end{pgfscope}%
\begin{pgfscope}%
\pgfsys@transformshift{2.089857in}{0.669049in}%
\pgfsys@useobject{currentmarker}{}%
\end{pgfscope}%
\begin{pgfscope}%
\pgfsys@transformshift{2.071075in}{0.840017in}%
\pgfsys@useobject{currentmarker}{}%
\end{pgfscope}%
\begin{pgfscope}%
\pgfsys@transformshift{2.051824in}{0.895748in}%
\pgfsys@useobject{currentmarker}{}%
\end{pgfscope}%
\begin{pgfscope}%
\pgfsys@transformshift{2.032807in}{1.012019in}%
\pgfsys@useobject{currentmarker}{}%
\end{pgfscope}%
\begin{pgfscope}%
\pgfsys@transformshift{2.013556in}{1.149406in}%
\pgfsys@useobject{currentmarker}{}%
\end{pgfscope}%
\begin{pgfscope}%
\pgfsys@transformshift{1.995477in}{1.349098in}%
\pgfsys@useobject{currentmarker}{}%
\end{pgfscope}%
\begin{pgfscope}%
\pgfsys@transformshift{1.976931in}{1.383740in}%
\pgfsys@useobject{currentmarker}{}%
\end{pgfscope}%
\begin{pgfscope}%
\pgfsys@transformshift{1.958618in}{1.289039in}%
\pgfsys@useobject{currentmarker}{}%
\end{pgfscope}%
\begin{pgfscope}%
\pgfsys@transformshift{1.935846in}{1.112995in}%
\pgfsys@useobject{currentmarker}{}%
\end{pgfscope}%
\begin{pgfscope}%
\pgfsys@transformshift{1.917534in}{0.850834in}%
\pgfsys@useobject{currentmarker}{}%
\end{pgfscope}%
\begin{pgfscope}%
\pgfsys@transformshift{1.899690in}{0.709644in}%
\pgfsys@useobject{currentmarker}{}%
\end{pgfscope}%
\begin{pgfscope}%
\pgfsys@transformshift{1.877621in}{0.566063in}%
\pgfsys@useobject{currentmarker}{}%
\end{pgfscope}%
\begin{pgfscope}%
\pgfsys@transformshift{1.859779in}{0.497994in}%
\pgfsys@useobject{currentmarker}{}%
\end{pgfscope}%
\begin{pgfscope}%
\pgfsys@transformshift{1.840998in}{0.513366in}%
\pgfsys@useobject{currentmarker}{}%
\end{pgfscope}%
\begin{pgfscope}%
\pgfsys@transformshift{1.822451in}{0.590452in}%
\pgfsys@useobject{currentmarker}{}%
\end{pgfscope}%
\begin{pgfscope}%
\pgfsys@transformshift{1.802965in}{0.722188in}%
\pgfsys@useobject{currentmarker}{}%
\end{pgfscope}%
\begin{pgfscope}%
\pgfsys@transformshift{1.783479in}{0.969085in}%
\pgfsys@useobject{currentmarker}{}%
\end{pgfscope}%
\begin{pgfscope}%
\pgfsys@transformshift{1.765400in}{1.143124in}%
\pgfsys@useobject{currentmarker}{}%
\end{pgfscope}%
\begin{pgfscope}%
\pgfsys@transformshift{1.744037in}{1.332813in}%
\pgfsys@useobject{currentmarker}{}%
\end{pgfscope}%
\begin{pgfscope}%
\pgfsys@transformshift{1.725724in}{1.397066in}%
\pgfsys@useobject{currentmarker}{}%
\end{pgfscope}%
\begin{pgfscope}%
\pgfsys@transformshift{1.707178in}{1.340998in}%
\pgfsys@useobject{currentmarker}{}%
\end{pgfscope}%
\begin{pgfscope}%
\pgfsys@transformshift{1.688630in}{1.203969in}%
\pgfsys@useobject{currentmarker}{}%
\end{pgfscope}%
\begin{pgfscope}%
\pgfsys@transformshift{1.667736in}{0.914363in}%
\pgfsys@useobject{currentmarker}{}%
\end{pgfscope}%
\begin{pgfscope}%
\pgfsys@transformshift{1.648719in}{0.742239in}%
\pgfsys@useobject{currentmarker}{}%
\end{pgfscope}%
\begin{pgfscope}%
\pgfsys@transformshift{1.629702in}{0.631282in}%
\pgfsys@useobject{currentmarker}{}%
\end{pgfscope}%
\begin{pgfscope}%
\pgfsys@transformshift{1.611391in}{0.539622in}%
\pgfsys@useobject{currentmarker}{}%
\end{pgfscope}%
\begin{pgfscope}%
\pgfsys@transformshift{1.590025in}{0.499314in}%
\pgfsys@useobject{currentmarker}{}%
\end{pgfscope}%
\begin{pgfscope}%
\pgfsys@transformshift{1.571009in}{0.518035in}%
\pgfsys@useobject{currentmarker}{}%
\end{pgfscope}%
\begin{pgfscope}%
\pgfsys@transformshift{1.553166in}{0.606101in}%
\pgfsys@useobject{currentmarker}{}%
\end{pgfscope}%
\begin{pgfscope}%
\pgfsys@transformshift{1.532037in}{0.749673in}%
\pgfsys@useobject{currentmarker}{}%
\end{pgfscope}%
\begin{pgfscope}%
\pgfsys@transformshift{1.515838in}{0.729755in}%
\pgfsys@useobject{currentmarker}{}%
\end{pgfscope}%
\begin{pgfscope}%
\pgfsys@transformshift{1.493769in}{0.996507in}%
\pgfsys@useobject{currentmarker}{}%
\end{pgfscope}%
\begin{pgfscope}%
\pgfsys@transformshift{1.471935in}{1.285500in}%
\pgfsys@useobject{currentmarker}{}%
\end{pgfscope}%
\begin{pgfscope}%
\pgfsys@transformshift{1.456205in}{1.309415in}%
\pgfsys@useobject{currentmarker}{}%
\end{pgfscope}%
\begin{pgfscope}%
\pgfsys@transformshift{1.437893in}{1.308811in}%
\pgfsys@useobject{currentmarker}{}%
\end{pgfscope}%
\begin{pgfscope}%
\pgfsys@transformshift{1.419582in}{1.407042in}%
\pgfsys@useobject{currentmarker}{}%
\end{pgfscope}%
\begin{pgfscope}%
\pgfsys@transformshift{1.397748in}{1.370989in}%
\pgfsys@useobject{currentmarker}{}%
\end{pgfscope}%
\begin{pgfscope}%
\pgfsys@transformshift{1.378731in}{1.219881in}%
\pgfsys@useobject{currentmarker}{}%
\end{pgfscope}%
\begin{pgfscope}%
\pgfsys@transformshift{1.360889in}{0.959318in}%
\pgfsys@useobject{currentmarker}{}%
\end{pgfscope}%
\begin{pgfscope}%
\pgfsys@transformshift{1.342341in}{0.773298in}%
\pgfsys@useobject{currentmarker}{}%
\end{pgfscope}%
\begin{pgfscope}%
\pgfsys@transformshift{1.320977in}{0.627990in}%
\pgfsys@useobject{currentmarker}{}%
\end{pgfscope}%
\begin{pgfscope}%
\pgfsys@transformshift{1.302195in}{0.549408in}%
\pgfsys@useobject{currentmarker}{}%
\end{pgfscope}%
\begin{pgfscope}%
\pgfsys@transformshift{1.283884in}{0.504211in}%
\pgfsys@useobject{currentmarker}{}%
\end{pgfscope}%
\begin{pgfscope}%
\pgfsys@transformshift{1.264631in}{0.545989in}%
\pgfsys@useobject{currentmarker}{}%
\end{pgfscope}%
\begin{pgfscope}%
\pgfsys@transformshift{1.246554in}{0.644067in}%
\pgfsys@useobject{currentmarker}{}%
\end{pgfscope}%
\begin{pgfscope}%
\pgfsys@transformshift{1.224956in}{0.772958in}%
\pgfsys@useobject{currentmarker}{}%
\end{pgfscope}%
\begin{pgfscope}%
\pgfsys@transformshift{1.206408in}{0.998625in}%
\pgfsys@useobject{currentmarker}{}%
\end{pgfscope}%
\begin{pgfscope}%
\pgfsys@transformshift{1.187860in}{1.229519in}%
\pgfsys@useobject{currentmarker}{}%
\end{pgfscope}%
\begin{pgfscope}%
\pgfsys@transformshift{1.169549in}{1.379601in}%
\pgfsys@useobject{currentmarker}{}%
\end{pgfscope}%
\begin{pgfscope}%
\pgfsys@transformshift{1.151001in}{1.425685in}%
\pgfsys@useobject{currentmarker}{}%
\end{pgfscope}%
\begin{pgfscope}%
\pgfsys@transformshift{1.128698in}{1.393641in}%
\pgfsys@useobject{currentmarker}{}%
\end{pgfscope}%
\begin{pgfscope}%
\pgfsys@transformshift{1.111324in}{1.259646in}%
\pgfsys@useobject{currentmarker}{}%
\end{pgfscope}%
\begin{pgfscope}%
\pgfsys@transformshift{1.091839in}{1.012964in}%
\pgfsys@useobject{currentmarker}{}%
\end{pgfscope}%
\begin{pgfscope}%
\pgfsys@transformshift{1.071884in}{0.831071in}%
\pgfsys@useobject{currentmarker}{}%
\end{pgfscope}%
\begin{pgfscope}%
\pgfsys@transformshift{1.052633in}{0.683223in}%
\pgfsys@useobject{currentmarker}{}%
\end{pgfscope}%
\begin{pgfscope}%
\pgfsys@transformshift{1.032911in}{0.606505in}%
\pgfsys@useobject{currentmarker}{}%
\end{pgfscope}%
\begin{pgfscope}%
\pgfsys@transformshift{1.014834in}{0.556919in}%
\pgfsys@useobject{currentmarker}{}%
\end{pgfscope}%
\begin{pgfscope}%
\pgfsys@transformshift{0.995817in}{0.518851in}%
\pgfsys@useobject{currentmarker}{}%
\end{pgfscope}%
\begin{pgfscope}%
\pgfsys@transformshift{0.978443in}{0.584241in}%
\pgfsys@useobject{currentmarker}{}%
\end{pgfscope}%
\begin{pgfscope}%
\pgfsys@transformshift{0.957315in}{0.700444in}%
\pgfsys@useobject{currentmarker}{}%
\end{pgfscope}%
\begin{pgfscope}%
\pgfsys@transformshift{0.937594in}{0.842829in}%
\pgfsys@useobject{currentmarker}{}%
\end{pgfscope}%
\begin{pgfscope}%
\pgfsys@transformshift{0.919047in}{0.953170in}%
\pgfsys@useobject{currentmarker}{}%
\end{pgfscope}%
\begin{pgfscope}%
\pgfsys@transformshift{0.899795in}{1.203197in}%
\pgfsys@useobject{currentmarker}{}%
\end{pgfscope}%
\begin{pgfscope}%
\pgfsys@transformshift{0.881482in}{1.377529in}%
\pgfsys@useobject{currentmarker}{}%
\end{pgfscope}%
\begin{pgfscope}%
\pgfsys@transformshift{0.860588in}{1.440868in}%
\pgfsys@useobject{currentmarker}{}%
\end{pgfscope}%
\begin{pgfscope}%
\pgfsys@transformshift{0.842276in}{1.413230in}%
\pgfsys@useobject{currentmarker}{}%
\end{pgfscope}%
\begin{pgfscope}%
\pgfsys@transformshift{0.823260in}{1.278502in}%
\pgfsys@useobject{currentmarker}{}%
\end{pgfscope}%
\begin{pgfscope}%
\pgfsys@transformshift{0.802365in}{0.998950in}%
\pgfsys@useobject{currentmarker}{}%
\end{pgfscope}%
\begin{pgfscope}%
\pgfsys@transformshift{0.783348in}{0.838760in}%
\pgfsys@useobject{currentmarker}{}%
\end{pgfscope}%
\begin{pgfscope}%
\pgfsys@transformshift{0.766915in}{0.703187in}%
\pgfsys@useobject{currentmarker}{}%
\end{pgfscope}%
\begin{pgfscope}%
\pgfsys@transformshift{0.747663in}{0.590390in}%
\pgfsys@useobject{currentmarker}{}%
\end{pgfscope}%
\begin{pgfscope}%
\pgfsys@transformshift{0.725829in}{0.525605in}%
\pgfsys@useobject{currentmarker}{}%
\end{pgfscope}%
\begin{pgfscope}%
\pgfsys@transformshift{0.707281in}{0.531937in}%
\pgfsys@useobject{currentmarker}{}%
\end{pgfscope}%
\begin{pgfscope}%
\pgfsys@transformshift{0.687796in}{0.545619in}%
\pgfsys@useobject{currentmarker}{}%
\end{pgfscope}%
\begin{pgfscope}%
\pgfsys@transformshift{0.668779in}{0.782260in}%
\pgfsys@useobject{currentmarker}{}%
\end{pgfscope}%
\begin{pgfscope}%
\pgfsys@transformshift{0.647650in}{0.638943in}%
\pgfsys@useobject{currentmarker}{}%
\end{pgfscope}%
\begin{pgfscope}%
\pgfsys@transformshift{0.650936in}{0.650057in}%
\pgfsys@useobject{currentmarker}{}%
\end{pgfscope}%
\begin{pgfscope}%
\pgfsys@transformshift{0.658449in}{0.735568in}%
\pgfsys@useobject{currentmarker}{}%
\end{pgfscope}%
\begin{pgfscope}%
\pgfsys@transformshift{0.676292in}{0.974777in}%
\pgfsys@useobject{currentmarker}{}%
\end{pgfscope}%
\begin{pgfscope}%
\pgfsys@transformshift{0.694368in}{1.293065in}%
\pgfsys@useobject{currentmarker}{}%
\end{pgfscope}%
\begin{pgfscope}%
\pgfsys@transformshift{0.712682in}{1.437467in}%
\pgfsys@useobject{currentmarker}{}%
\end{pgfscope}%
\begin{pgfscope}%
\pgfsys@transformshift{0.733811in}{1.386487in}%
\pgfsys@useobject{currentmarker}{}%
\end{pgfscope}%
\begin{pgfscope}%
\pgfsys@transformshift{0.752358in}{1.128477in}%
\pgfsys@useobject{currentmarker}{}%
\end{pgfscope}%
\begin{pgfscope}%
\pgfsys@transformshift{0.769496in}{0.821447in}%
\pgfsys@useobject{currentmarker}{}%
\end{pgfscope}%
\begin{pgfscope}%
\pgfsys@transformshift{0.794382in}{0.573049in}%
\pgfsys@useobject{currentmarker}{}%
\end{pgfscope}%
\begin{pgfscope}%
\pgfsys@transformshift{0.809643in}{0.519378in}%
\pgfsys@useobject{currentmarker}{}%
\end{pgfscope}%
\begin{pgfscope}%
\pgfsys@transformshift{0.827251in}{0.619186in}%
\pgfsys@useobject{currentmarker}{}%
\end{pgfscope}%
\begin{pgfscope}%
\pgfsys@transformshift{0.848380in}{0.839070in}%
\pgfsys@useobject{currentmarker}{}%
\end{pgfscope}%
\begin{pgfscope}%
\pgfsys@transformshift{0.866928in}{1.149741in}%
\pgfsys@useobject{currentmarker}{}%
\end{pgfscope}%
\begin{pgfscope}%
\pgfsys@transformshift{0.885943in}{1.392934in}%
\pgfsys@useobject{currentmarker}{}%
\end{pgfscope}%
\begin{pgfscope}%
\pgfsys@transformshift{0.907777in}{1.404545in}%
\pgfsys@useobject{currentmarker}{}%
\end{pgfscope}%
\begin{pgfscope}%
\pgfsys@transformshift{0.923742in}{1.235526in}%
\pgfsys@useobject{currentmarker}{}%
\end{pgfscope}%
\begin{pgfscope}%
\pgfsys@transformshift{0.944167in}{0.855245in}%
\pgfsys@useobject{currentmarker}{}%
\end{pgfscope}%
\begin{pgfscope}%
\pgfsys@transformshift{0.963653in}{0.636719in}%
\pgfsys@useobject{currentmarker}{}%
\end{pgfscope}%
\begin{pgfscope}%
\pgfsys@transformshift{0.980792in}{0.522653in}%
\pgfsys@useobject{currentmarker}{}%
\end{pgfscope}%
\begin{pgfscope}%
\pgfsys@transformshift{1.000512in}{0.533437in}%
\pgfsys@useobject{currentmarker}{}%
\end{pgfscope}%
\begin{pgfscope}%
\pgfsys@transformshift{1.022112in}{0.697922in}%
\pgfsys@useobject{currentmarker}{}%
\end{pgfscope}%
\begin{pgfscope}%
\pgfsys@transformshift{1.041363in}{0.929905in}%
\pgfsys@useobject{currentmarker}{}%
\end{pgfscope}%
\begin{pgfscope}%
\pgfsys@transformshift{1.061083in}{1.245559in}%
\pgfsys@useobject{currentmarker}{}%
\end{pgfscope}%
\begin{pgfscope}%
\pgfsys@transformshift{1.079631in}{1.406386in}%
\pgfsys@useobject{currentmarker}{}%
\end{pgfscope}%
\begin{pgfscope}%
\pgfsys@transformshift{1.097005in}{1.393809in}%
\pgfsys@useobject{currentmarker}{}%
\end{pgfscope}%
\begin{pgfscope}%
\pgfsys@transformshift{1.117899in}{1.123778in}%
\pgfsys@useobject{currentmarker}{}%
\end{pgfscope}%
\begin{pgfscope}%
\pgfsys@transformshift{1.136916in}{0.798423in}%
\pgfsys@useobject{currentmarker}{}%
\end{pgfscope}%
\begin{pgfscope}%
\pgfsys@transformshift{1.156636in}{0.602591in}%
\pgfsys@useobject{currentmarker}{}%
\end{pgfscope}%
\begin{pgfscope}%
\pgfsys@transformshift{1.176358in}{0.506067in}%
\pgfsys@useobject{currentmarker}{}%
\end{pgfscope}%
\begin{pgfscope}%
\pgfsys@transformshift{1.195138in}{0.548726in}%
\pgfsys@useobject{currentmarker}{}%
\end{pgfscope}%
\begin{pgfscope}%
\pgfsys@transformshift{1.213921in}{0.689219in}%
\pgfsys@useobject{currentmarker}{}%
\end{pgfscope}%
\begin{pgfscope}%
\pgfsys@transformshift{1.234112in}{0.920557in}%
\pgfsys@useobject{currentmarker}{}%
\end{pgfscope}%
\begin{pgfscope}%
\pgfsys@transformshift{1.253832in}{1.228627in}%
\pgfsys@useobject{currentmarker}{}%
\end{pgfscope}%
\begin{pgfscope}%
\pgfsys@transformshift{1.270971in}{1.393031in}%
\pgfsys@useobject{currentmarker}{}%
\end{pgfscope}%
\begin{pgfscope}%
\pgfsys@transformshift{1.291160in}{1.390356in}%
\pgfsys@useobject{currentmarker}{}%
\end{pgfscope}%
\begin{pgfscope}%
\pgfsys@transformshift{1.309473in}{1.211735in}%
\pgfsys@useobject{currentmarker}{}%
\end{pgfscope}%
\begin{pgfscope}%
\pgfsys@transformshift{1.330368in}{0.919354in}%
\pgfsys@useobject{currentmarker}{}%
\end{pgfscope}%
\begin{pgfscope}%
\pgfsys@transformshift{1.348681in}{0.678412in}%
\pgfsys@useobject{currentmarker}{}%
\end{pgfscope}%
\begin{pgfscope}%
\pgfsys@transformshift{1.365818in}{0.541156in}%
\pgfsys@useobject{currentmarker}{}%
\end{pgfscope}%
\begin{pgfscope}%
\pgfsys@transformshift{1.386009in}{0.499779in}%
\pgfsys@useobject{currentmarker}{}%
\end{pgfscope}%
\begin{pgfscope}%
\pgfsys@transformshift{1.407609in}{0.606297in}%
\pgfsys@useobject{currentmarker}{}%
\end{pgfscope}%
\begin{pgfscope}%
\pgfsys@transformshift{1.425215in}{0.749017in}%
\pgfsys@useobject{currentmarker}{}%
\end{pgfscope}%
\begin{pgfscope}%
\pgfsys@transformshift{1.442823in}{0.926556in}%
\pgfsys@useobject{currentmarker}{}%
\end{pgfscope}%
\begin{pgfscope}%
\pgfsys@transformshift{1.465128in}{1.190069in}%
\pgfsys@useobject{currentmarker}{}%
\end{pgfscope}%
\begin{pgfscope}%
\pgfsys@transformshift{1.481562in}{1.366563in}%
\pgfsys@useobject{currentmarker}{}%
\end{pgfscope}%
\begin{pgfscope}%
\pgfsys@transformshift{1.501751in}{1.379894in}%
\pgfsys@useobject{currentmarker}{}%
\end{pgfscope}%
\begin{pgfscope}%
\pgfsys@transformshift{1.519830in}{1.263150in}%
\pgfsys@useobject{currentmarker}{}%
\end{pgfscope}%
\begin{pgfscope}%
\pgfsys@transformshift{1.542367in}{0.940081in}%
\pgfsys@useobject{currentmarker}{}%
\end{pgfscope}%
\begin{pgfscope}%
\pgfsys@transformshift{1.561150in}{0.824778in}%
\pgfsys@useobject{currentmarker}{}%
\end{pgfscope}%
\begin{pgfscope}%
\pgfsys@transformshift{1.578287in}{0.620413in}%
\pgfsys@useobject{currentmarker}{}%
\end{pgfscope}%
\begin{pgfscope}%
\pgfsys@transformshift{1.598009in}{0.520485in}%
\pgfsys@useobject{currentmarker}{}%
\end{pgfscope}%
\begin{pgfscope}%
\pgfsys@transformshift{1.616555in}{0.494195in}%
\pgfsys@useobject{currentmarker}{}%
\end{pgfscope}%
\begin{pgfscope}%
\pgfsys@transformshift{1.636277in}{0.568226in}%
\pgfsys@useobject{currentmarker}{}%
\end{pgfscope}%
\begin{pgfscope}%
\pgfsys@transformshift{1.653648in}{0.670872in}%
\pgfsys@useobject{currentmarker}{}%
\end{pgfscope}%
\begin{pgfscope}%
\pgfsys@transformshift{1.676188in}{0.832896in}%
\pgfsys@useobject{currentmarker}{}%
\end{pgfscope}%
\begin{pgfscope}%
\pgfsys@transformshift{1.695205in}{1.005137in}%
\pgfsys@useobject{currentmarker}{}%
\end{pgfscope}%
\begin{pgfscope}%
\pgfsys@transformshift{1.713282in}{1.245171in}%
\pgfsys@useobject{currentmarker}{}%
\end{pgfscope}%
\begin{pgfscope}%
\pgfsys@transformshift{1.728776in}{1.382488in}%
\pgfsys@useobject{currentmarker}{}%
\end{pgfscope}%
\begin{pgfscope}%
\pgfsys@transformshift{1.749906in}{1.362324in}%
\pgfsys@useobject{currentmarker}{}%
\end{pgfscope}%
\begin{pgfscope}%
\pgfsys@transformshift{1.769861in}{1.155295in}%
\pgfsys@useobject{currentmarker}{}%
\end{pgfscope}%
\begin{pgfscope}%
\pgfsys@transformshift{1.788409in}{0.861099in}%
\pgfsys@useobject{currentmarker}{}%
\end{pgfscope}%
\begin{pgfscope}%
\pgfsys@transformshift{1.807660in}{0.663831in}%
\pgfsys@useobject{currentmarker}{}%
\end{pgfscope}%
\begin{pgfscope}%
\pgfsys@transformshift{1.829729in}{0.549850in}%
\pgfsys@useobject{currentmarker}{}%
\end{pgfscope}%
\begin{pgfscope}%
\pgfsys@transformshift{1.848745in}{0.496525in}%
\pgfsys@useobject{currentmarker}{}%
\end{pgfscope}%
\begin{pgfscope}%
\pgfsys@transformshift{1.867291in}{0.516179in}%
\pgfsys@useobject{currentmarker}{}%
\end{pgfscope}%
\begin{pgfscope}%
\pgfsys@transformshift{1.885136in}{0.612214in}%
\pgfsys@useobject{currentmarker}{}%
\end{pgfscope}%
\begin{pgfscope}%
\pgfsys@transformshift{1.904387in}{0.732585in}%
\pgfsys@useobject{currentmarker}{}%
\end{pgfscope}%
\begin{pgfscope}%
\pgfsys@transformshift{1.922229in}{0.635957in}%
\pgfsys@useobject{currentmarker}{}%
\end{pgfscope}%
\begin{pgfscope}%
\pgfsys@transformshift{1.941246in}{0.521237in}%
\pgfsys@useobject{currentmarker}{}%
\end{pgfscope}%
\begin{pgfscope}%
\pgfsys@transformshift{1.964018in}{0.509458in}%
\pgfsys@useobject{currentmarker}{}%
\end{pgfscope}%
\begin{pgfscope}%
\pgfsys@transformshift{1.982095in}{0.613379in}%
\pgfsys@useobject{currentmarker}{}%
\end{pgfscope}%
\begin{pgfscope}%
\pgfsys@transformshift{2.001583in}{0.792151in}%
\pgfsys@useobject{currentmarker}{}%
\end{pgfscope}%
\begin{pgfscope}%
\pgfsys@transformshift{2.020129in}{1.047380in}%
\pgfsys@useobject{currentmarker}{}%
\end{pgfscope}%
\begin{pgfscope}%
\pgfsys@transformshift{2.038442in}{1.257670in}%
\pgfsys@useobject{currentmarker}{}%
\end{pgfscope}%
\begin{pgfscope}%
\pgfsys@transformshift{2.061683in}{1.382107in}%
\pgfsys@useobject{currentmarker}{}%
\end{pgfscope}%
\begin{pgfscope}%
\pgfsys@transformshift{2.078119in}{1.319914in}%
\pgfsys@useobject{currentmarker}{}%
\end{pgfscope}%
\begin{pgfscope}%
\pgfsys@transformshift{2.103239in}{0.988151in}%
\pgfsys@useobject{currentmarker}{}%
\end{pgfscope}%
\begin{pgfscope}%
\pgfsys@transformshift{2.116856in}{0.771038in}%
\pgfsys@useobject{currentmarker}{}%
\end{pgfscope}%
\begin{pgfscope}%
\pgfsys@transformshift{2.134464in}{0.604975in}%
\pgfsys@useobject{currentmarker}{}%
\end{pgfscope}%
\begin{pgfscope}%
\pgfsys@transformshift{2.151837in}{0.521025in}%
\pgfsys@useobject{currentmarker}{}%
\end{pgfscope}%
\begin{pgfscope}%
\pgfsys@transformshift{2.172966in}{0.495733in}%
\pgfsys@useobject{currentmarker}{}%
\end{pgfscope}%
\begin{pgfscope}%
\pgfsys@transformshift{2.193860in}{0.592264in}%
\pgfsys@useobject{currentmarker}{}%
\end{pgfscope}%
\begin{pgfscope}%
\pgfsys@transformshift{2.211937in}{0.729651in}%
\pgfsys@useobject{currentmarker}{}%
\end{pgfscope}%
\begin{pgfscope}%
\pgfsys@transformshift{2.236120in}{0.934580in}%
\pgfsys@useobject{currentmarker}{}%
\end{pgfscope}%
\begin{pgfscope}%
\pgfsys@transformshift{2.251614in}{1.235492in}%
\pgfsys@useobject{currentmarker}{}%
\end{pgfscope}%
\begin{pgfscope}%
\pgfsys@transformshift{2.270865in}{1.374729in}%
\pgfsys@useobject{currentmarker}{}%
\end{pgfscope}%
\begin{pgfscope}%
\pgfsys@transformshift{2.290587in}{1.343290in}%
\pgfsys@useobject{currentmarker}{}%
\end{pgfscope}%
\begin{pgfscope}%
\pgfsys@transformshift{2.309604in}{1.185908in}%
\pgfsys@useobject{currentmarker}{}%
\end{pgfscope}%
\begin{pgfscope}%
\pgfsys@transformshift{2.327212in}{0.875543in}%
\pgfsys@useobject{currentmarker}{}%
\end{pgfscope}%
\begin{pgfscope}%
\pgfsys@transformshift{2.346463in}{0.680371in}%
\pgfsys@useobject{currentmarker}{}%
\end{pgfscope}%
\begin{pgfscope}%
\pgfsys@transformshift{2.365009in}{0.546946in}%
\pgfsys@useobject{currentmarker}{}%
\end{pgfscope}%
\begin{pgfscope}%
\pgfsys@transformshift{2.386374in}{0.485497in}%
\pgfsys@useobject{currentmarker}{}%
\end{pgfscope}%
\begin{pgfscope}%
\pgfsys@transformshift{2.405860in}{0.532878in}%
\pgfsys@useobject{currentmarker}{}%
\end{pgfscope}%
\begin{pgfscope}%
\pgfsys@transformshift{2.421825in}{0.628993in}%
\pgfsys@useobject{currentmarker}{}%
\end{pgfscope}%
\begin{pgfscope}%
\pgfsys@transformshift{2.442250in}{0.721780in}%
\pgfsys@useobject{currentmarker}{}%
\end{pgfscope}%
\begin{pgfscope}%
\pgfsys@transformshift{2.463145in}{0.974383in}%
\pgfsys@useobject{currentmarker}{}%
\end{pgfscope}%
\begin{pgfscope}%
\pgfsys@transformshift{2.482162in}{1.227993in}%
\pgfsys@useobject{currentmarker}{}%
\end{pgfscope}%
\begin{pgfscope}%
\pgfsys@transformshift{2.502351in}{1.374638in}%
\pgfsys@useobject{currentmarker}{}%
\end{pgfscope}%
\begin{pgfscope}%
\pgfsys@transformshift{2.519255in}{1.358132in}%
\pgfsys@useobject{currentmarker}{}%
\end{pgfscope}%
\begin{pgfscope}%
\pgfsys@transformshift{2.539210in}{1.194772in}%
\pgfsys@useobject{currentmarker}{}%
\end{pgfscope}%
\begin{pgfscope}%
\pgfsys@transformshift{2.560106in}{0.884398in}%
\pgfsys@useobject{currentmarker}{}%
\end{pgfscope}%
\begin{pgfscope}%
\pgfsys@transformshift{2.578418in}{0.769846in}%
\pgfsys@useobject{currentmarker}{}%
\end{pgfscope}%
\begin{pgfscope}%
\pgfsys@transformshift{2.596731in}{0.617977in}%
\pgfsys@useobject{currentmarker}{}%
\end{pgfscope}%
\begin{pgfscope}%
\pgfsys@transformshift{2.614337in}{0.514740in}%
\pgfsys@useobject{currentmarker}{}%
\end{pgfscope}%
\begin{pgfscope}%
\pgfsys@transformshift{2.636877in}{0.501575in}%
\pgfsys@useobject{currentmarker}{}%
\end{pgfscope}%
\begin{pgfscope}%
\pgfsys@transformshift{2.656128in}{0.601867in}%
\pgfsys@useobject{currentmarker}{}%
\end{pgfscope}%
\begin{pgfscope}%
\pgfsys@transformshift{2.673265in}{0.732974in}%
\pgfsys@useobject{currentmarker}{}%
\end{pgfscope}%
\begin{pgfscope}%
\pgfsys@transformshift{2.692518in}{0.937614in}%
\pgfsys@useobject{currentmarker}{}%
\end{pgfscope}%
\begin{pgfscope}%
\pgfsys@transformshift{2.713178in}{1.237067in}%
\pgfsys@useobject{currentmarker}{}%
\end{pgfscope}%
\begin{pgfscope}%
\pgfsys@transformshift{2.730550in}{1.356368in}%
\pgfsys@useobject{currentmarker}{}%
\end{pgfscope}%
\begin{pgfscope}%
\pgfsys@transformshift{2.751680in}{1.365307in}%
\pgfsys@useobject{currentmarker}{}%
\end{pgfscope}%
\begin{pgfscope}%
\pgfsys@transformshift{2.770226in}{1.262105in}%
\pgfsys@useobject{currentmarker}{}%
\end{pgfscope}%
\begin{pgfscope}%
\pgfsys@transformshift{2.788069in}{0.994132in}%
\pgfsys@useobject{currentmarker}{}%
\end{pgfscope}%
\begin{pgfscope}%
\pgfsys@transformshift{2.810608in}{0.721169in}%
\pgfsys@useobject{currentmarker}{}%
\end{pgfscope}%
\begin{pgfscope}%
\pgfsys@transformshift{2.827746in}{0.604745in}%
\pgfsys@useobject{currentmarker}{}%
\end{pgfscope}%
\begin{pgfscope}%
\pgfsys@transformshift{2.845824in}{0.543900in}%
\pgfsys@useobject{currentmarker}{}%
\end{pgfscope}%
\begin{pgfscope}%
\pgfsys@transformshift{2.865779in}{0.489556in}%
\pgfsys@useobject{currentmarker}{}%
\end{pgfscope}%
\begin{pgfscope}%
\pgfsys@transformshift{2.884327in}{0.534124in}%
\pgfsys@useobject{currentmarker}{}%
\end{pgfscope}%
\begin{pgfscope}%
\pgfsys@transformshift{2.905690in}{0.652311in}%
\pgfsys@useobject{currentmarker}{}%
\end{pgfscope}%
\begin{pgfscope}%
\pgfsys@transformshift{2.922829in}{0.833787in}%
\pgfsys@useobject{currentmarker}{}%
\end{pgfscope}%
\begin{pgfscope}%
\pgfsys@transformshift{2.944898in}{1.095416in}%
\pgfsys@useobject{currentmarker}{}%
\end{pgfscope}%
\begin{pgfscope}%
\pgfsys@transformshift{2.962035in}{1.301923in}%
\pgfsys@useobject{currentmarker}{}%
\end{pgfscope}%
\begin{pgfscope}%
\pgfsys@transformshift{2.980348in}{1.378730in}%
\pgfsys@useobject{currentmarker}{}%
\end{pgfscope}%
\begin{pgfscope}%
\pgfsys@transformshift{3.000539in}{1.353125in}%
\pgfsys@useobject{currentmarker}{}%
\end{pgfscope}%
\begin{pgfscope}%
\pgfsys@transformshift{3.018147in}{1.232639in}%
\pgfsys@useobject{currentmarker}{}%
\end{pgfscope}%
\begin{pgfscope}%
\pgfsys@transformshift{3.038102in}{0.986173in}%
\pgfsys@useobject{currentmarker}{}%
\end{pgfscope}%
\begin{pgfscope}%
\pgfsys@transformshift{3.057588in}{0.736642in}%
\pgfsys@useobject{currentmarker}{}%
\end{pgfscope}%
\begin{pgfscope}%
\pgfsys@transformshift{3.076604in}{0.607607in}%
\pgfsys@useobject{currentmarker}{}%
\end{pgfscope}%
\begin{pgfscope}%
\pgfsys@transformshift{3.096327in}{0.518532in}%
\pgfsys@useobject{currentmarker}{}%
\end{pgfscope}%
\begin{pgfscope}%
\pgfsys@transformshift{3.115341in}{0.504362in}%
\pgfsys@useobject{currentmarker}{}%
\end{pgfscope}%
\begin{pgfscope}%
\pgfsys@transformshift{3.132012in}{0.554443in}%
\pgfsys@useobject{currentmarker}{}%
\end{pgfscope}%
\begin{pgfscope}%
\pgfsys@transformshift{3.153846in}{0.689660in}%
\pgfsys@useobject{currentmarker}{}%
\end{pgfscope}%
\begin{pgfscope}%
\pgfsys@transformshift{3.174975in}{0.912664in}%
\pgfsys@useobject{currentmarker}{}%
\end{pgfscope}%
\begin{pgfscope}%
\pgfsys@transformshift{3.189296in}{1.125415in}%
\pgfsys@useobject{currentmarker}{}%
\end{pgfscope}%
\begin{pgfscope}%
\pgfsys@transformshift{3.210660in}{0.664454in}%
\pgfsys@useobject{currentmarker}{}%
\end{pgfscope}%
\begin{pgfscope}%
\pgfsys@transformshift{3.231085in}{0.853786in}%
\pgfsys@useobject{currentmarker}{}%
\end{pgfscope}%
\begin{pgfscope}%
\pgfsys@transformshift{3.250805in}{1.045156in}%
\pgfsys@useobject{currentmarker}{}%
\end{pgfscope}%
\begin{pgfscope}%
\pgfsys@transformshift{3.270293in}{1.316692in}%
\pgfsys@useobject{currentmarker}{}%
\end{pgfscope}%
\begin{pgfscope}%
\pgfsys@transformshift{3.287430in}{1.394442in}%
\pgfsys@useobject{currentmarker}{}%
\end{pgfscope}%
\begin{pgfscope}%
\pgfsys@transformshift{3.306212in}{1.357403in}%
\pgfsys@useobject{currentmarker}{}%
\end{pgfscope}%
\begin{pgfscope}%
\pgfsys@transformshift{3.327812in}{1.210637in}%
\pgfsys@useobject{currentmarker}{}%
\end{pgfscope}%
\begin{pgfscope}%
\pgfsys@transformshift{3.345420in}{0.962029in}%
\pgfsys@useobject{currentmarker}{}%
\end{pgfscope}%
\begin{pgfscope}%
\pgfsys@transformshift{3.366080in}{0.751348in}%
\pgfsys@useobject{currentmarker}{}%
\end{pgfscope}%
\begin{pgfscope}%
\pgfsys@transformshift{3.384391in}{0.602312in}%
\pgfsys@useobject{currentmarker}{}%
\end{pgfscope}%
\begin{pgfscope}%
\pgfsys@transformshift{3.404582in}{0.513724in}%
\pgfsys@useobject{currentmarker}{}%
\end{pgfscope}%
\begin{pgfscope}%
\pgfsys@transformshift{3.425477in}{0.499175in}%
\pgfsys@useobject{currentmarker}{}%
\end{pgfscope}%
\begin{pgfscope}%
\pgfsys@transformshift{3.443788in}{0.568417in}%
\pgfsys@useobject{currentmarker}{}%
\end{pgfscope}%
\begin{pgfscope}%
\pgfsys@transformshift{3.461162in}{0.703341in}%
\pgfsys@useobject{currentmarker}{}%
\end{pgfscope}%
\begin{pgfscope}%
\pgfsys@transformshift{3.478535in}{0.841447in}%
\pgfsys@useobject{currentmarker}{}%
\end{pgfscope}%
\begin{pgfscope}%
\pgfsys@transformshift{3.500604in}{1.101312in}%
\pgfsys@useobject{currentmarker}{}%
\end{pgfscope}%
\begin{pgfscope}%
\pgfsys@transformshift{3.521264in}{1.343581in}%
\pgfsys@useobject{currentmarker}{}%
\end{pgfscope}%
\begin{pgfscope}%
\pgfsys@transformshift{3.537698in}{1.405486in}%
\pgfsys@useobject{currentmarker}{}%
\end{pgfscope}%
\begin{pgfscope}%
\pgfsys@transformshift{3.559063in}{1.362818in}%
\pgfsys@useobject{currentmarker}{}%
\end{pgfscope}%
\begin{pgfscope}%
\pgfsys@transformshift{3.576671in}{1.221663in}%
\pgfsys@useobject{currentmarker}{}%
\end{pgfscope}%
\begin{pgfscope}%
\pgfsys@transformshift{3.597565in}{0.925408in}%
\pgfsys@useobject{currentmarker}{}%
\end{pgfscope}%
\begin{pgfscope}%
\pgfsys@transformshift{3.614468in}{0.798360in}%
\pgfsys@useobject{currentmarker}{}%
\end{pgfscope}%
\begin{pgfscope}%
\pgfsys@transformshift{3.639354in}{0.634130in}%
\pgfsys@useobject{currentmarker}{}%
\end{pgfscope}%
\begin{pgfscope}%
\pgfsys@transformshift{3.654379in}{0.555720in}%
\pgfsys@useobject{currentmarker}{}%
\end{pgfscope}%
\begin{pgfscope}%
\pgfsys@transformshift{3.674336in}{0.504672in}%
\pgfsys@useobject{currentmarker}{}%
\end{pgfscope}%
\begin{pgfscope}%
\pgfsys@transformshift{3.693821in}{0.537196in}%
\pgfsys@useobject{currentmarker}{}%
\end{pgfscope}%
\begin{pgfscope}%
\pgfsys@transformshift{3.711429in}{0.634918in}%
\pgfsys@useobject{currentmarker}{}%
\end{pgfscope}%
\begin{pgfscope}%
\pgfsys@transformshift{3.729741in}{0.768310in}%
\pgfsys@useobject{currentmarker}{}%
\end{pgfscope}%
\begin{pgfscope}%
\pgfsys@transformshift{3.749228in}{0.951369in}%
\pgfsys@useobject{currentmarker}{}%
\end{pgfscope}%
\begin{pgfscope}%
\pgfsys@transformshift{3.770123in}{1.181948in}%
\pgfsys@useobject{currentmarker}{}%
\end{pgfscope}%
\begin{pgfscope}%
\pgfsys@transformshift{3.789374in}{1.333777in}%
\pgfsys@useobject{currentmarker}{}%
\end{pgfscope}%
\begin{pgfscope}%
\pgfsys@transformshift{3.805808in}{1.417332in}%
\pgfsys@useobject{currentmarker}{}%
\end{pgfscope}%
\begin{pgfscope}%
\pgfsys@transformshift{3.827407in}{1.143057in}%
\pgfsys@useobject{currentmarker}{}%
\end{pgfscope}%
\begin{pgfscope}%
\pgfsys@transformshift{3.845016in}{1.323077in}%
\pgfsys@useobject{currentmarker}{}%
\end{pgfscope}%
\begin{pgfscope}%
\pgfsys@transformshift{3.866850in}{1.417739in}%
\pgfsys@useobject{currentmarker}{}%
\end{pgfscope}%
\begin{pgfscope}%
\pgfsys@transformshift{3.884221in}{1.404492in}%
\pgfsys@useobject{currentmarker}{}%
\end{pgfscope}%
\begin{pgfscope}%
\pgfsys@transformshift{3.902300in}{1.259225in}%
\pgfsys@useobject{currentmarker}{}%
\end{pgfscope}%
\begin{pgfscope}%
\pgfsys@transformshift{3.923195in}{0.982665in}%
\pgfsys@useobject{currentmarker}{}%
\end{pgfscope}%
\begin{pgfscope}%
\pgfsys@transformshift{3.941037in}{0.786994in}%
\pgfsys@useobject{currentmarker}{}%
\end{pgfscope}%
\begin{pgfscope}%
\pgfsys@transformshift{3.960288in}{0.652605in}%
\pgfsys@useobject{currentmarker}{}%
\end{pgfscope}%
\begin{pgfscope}%
\pgfsys@transformshift{3.979305in}{0.546708in}%
\pgfsys@useobject{currentmarker}{}%
\end{pgfscope}%
\begin{pgfscope}%
\pgfsys@transformshift{3.999496in}{0.516305in}%
\pgfsys@useobject{currentmarker}{}%
\end{pgfscope}%
\begin{pgfscope}%
\pgfsys@transformshift{4.018042in}{0.577167in}%
\pgfsys@useobject{currentmarker}{}%
\end{pgfscope}%
\begin{pgfscope}%
\pgfsys@transformshift{4.039642in}{0.700228in}%
\pgfsys@useobject{currentmarker}{}%
\end{pgfscope}%
\begin{pgfscope}%
\pgfsys@transformshift{4.058188in}{0.903590in}%
\pgfsys@useobject{currentmarker}{}%
\end{pgfscope}%
\begin{pgfscope}%
\pgfsys@transformshift{4.076501in}{1.119281in}%
\pgfsys@useobject{currentmarker}{}%
\end{pgfscope}%
\begin{pgfscope}%
\pgfsys@transformshift{4.094578in}{1.322999in}%
\pgfsys@useobject{currentmarker}{}%
\end{pgfscope}%
\begin{pgfscope}%
\pgfsys@transformshift{4.117352in}{1.428745in}%
\pgfsys@useobject{currentmarker}{}%
\end{pgfscope}%
\begin{pgfscope}%
\pgfsys@transformshift{4.134020in}{1.427704in}%
\pgfsys@useobject{currentmarker}{}%
\end{pgfscope}%
\begin{pgfscope}%
\pgfsys@transformshift{4.151863in}{1.363871in}%
\pgfsys@useobject{currentmarker}{}%
\end{pgfscope}%
\begin{pgfscope}%
\pgfsys@transformshift{4.175574in}{1.134031in}%
\pgfsys@useobject{currentmarker}{}%
\end{pgfscope}%
\begin{pgfscope}%
\pgfsys@transformshift{4.190131in}{0.904161in}%
\pgfsys@useobject{currentmarker}{}%
\end{pgfscope}%
\begin{pgfscope}%
\pgfsys@transformshift{4.212668in}{0.745830in}%
\pgfsys@useobject{currentmarker}{}%
\end{pgfscope}%
\begin{pgfscope}%
\pgfsys@transformshift{4.230276in}{0.614801in}%
\pgfsys@useobject{currentmarker}{}%
\end{pgfscope}%
\begin{pgfscope}%
\pgfsys@transformshift{4.250233in}{0.554705in}%
\pgfsys@useobject{currentmarker}{}%
\end{pgfscope}%
\begin{pgfscope}%
\pgfsys@transformshift{4.268075in}{0.524129in}%
\pgfsys@useobject{currentmarker}{}%
\end{pgfscope}%
\begin{pgfscope}%
\pgfsys@transformshift{4.290144in}{0.606736in}%
\pgfsys@useobject{currentmarker}{}%
\end{pgfscope}%
\begin{pgfscope}%
\pgfsys@transformshift{4.307986in}{0.675119in}%
\pgfsys@useobject{currentmarker}{}%
\end{pgfscope}%
\begin{pgfscope}%
\pgfsys@transformshift{4.327003in}{0.815774in}%
\pgfsys@useobject{currentmarker}{}%
\end{pgfscope}%
\begin{pgfscope}%
\pgfsys@transformshift{4.345314in}{1.008470in}%
\pgfsys@useobject{currentmarker}{}%
\end{pgfscope}%
\begin{pgfscope}%
\pgfsys@transformshift{4.366680in}{1.259102in}%
\pgfsys@useobject{currentmarker}{}%
\end{pgfscope}%
\begin{pgfscope}%
\pgfsys@transformshift{4.384288in}{1.409409in}%
\pgfsys@useobject{currentmarker}{}%
\end{pgfscope}%
\begin{pgfscope}%
\pgfsys@transformshift{4.402365in}{1.458389in}%
\pgfsys@useobject{currentmarker}{}%
\end{pgfscope}%
\begin{pgfscope}%
\pgfsys@transformshift{4.423494in}{1.416947in}%
\pgfsys@useobject{currentmarker}{}%
\end{pgfscope}%
\begin{pgfscope}%
\pgfsys@transformshift{4.441572in}{1.276617in}%
\pgfsys@useobject{currentmarker}{}%
\end{pgfscope}%
\begin{pgfscope}%
\pgfsys@transformshift{4.459649in}{1.061061in}%
\pgfsys@useobject{currentmarker}{}%
\end{pgfscope}%
\begin{pgfscope}%
\pgfsys@transformshift{4.478901in}{1.254496in}%
\pgfsys@useobject{currentmarker}{}%
\end{pgfscope}%
\begin{pgfscope}%
\pgfsys@transformshift{4.480309in}{1.249757in}%
\pgfsys@useobject{currentmarker}{}%
\end{pgfscope}%
\begin{pgfscope}%
\pgfsys@transformshift{4.474206in}{1.340836in}%
\pgfsys@useobject{currentmarker}{}%
\end{pgfscope}%
\begin{pgfscope}%
\pgfsys@transformshift{4.456127in}{1.457255in}%
\pgfsys@useobject{currentmarker}{}%
\end{pgfscope}%
\begin{pgfscope}%
\pgfsys@transformshift{4.438050in}{1.406568in}%
\pgfsys@useobject{currentmarker}{}%
\end{pgfscope}%
\begin{pgfscope}%
\pgfsys@transformshift{4.415278in}{1.096003in}%
\pgfsys@useobject{currentmarker}{}%
\end{pgfscope}%
\begin{pgfscope}%
\pgfsys@transformshift{4.397670in}{0.812173in}%
\pgfsys@useobject{currentmarker}{}%
\end{pgfscope}%
\begin{pgfscope}%
\pgfsys@transformshift{4.378418in}{0.615780in}%
\pgfsys@useobject{currentmarker}{}%
\end{pgfscope}%
\begin{pgfscope}%
\pgfsys@transformshift{4.354236in}{0.531878in}%
\pgfsys@useobject{currentmarker}{}%
\end{pgfscope}%
\begin{pgfscope}%
\pgfsys@transformshift{4.339916in}{0.629111in}%
\pgfsys@useobject{currentmarker}{}%
\end{pgfscope}%
\begin{pgfscope}%
\pgfsys@transformshift{4.323246in}{0.814847in}%
\pgfsys@useobject{currentmarker}{}%
\end{pgfscope}%
\begin{pgfscope}%
\pgfsys@transformshift{4.300943in}{1.142265in}%
\pgfsys@useobject{currentmarker}{}%
\end{pgfscope}%
\begin{pgfscope}%
\pgfsys@transformshift{4.279343in}{1.411004in}%
\pgfsys@useobject{currentmarker}{}%
\end{pgfscope}%
\begin{pgfscope}%
\pgfsys@transformshift{4.261971in}{1.430896in}%
\pgfsys@useobject{currentmarker}{}%
\end{pgfscope}%
\begin{pgfscope}%
\pgfsys@transformshift{4.243189in}{1.280103in}%
\pgfsys@useobject{currentmarker}{}%
\end{pgfscope}%
\begin{pgfscope}%
\pgfsys@transformshift{4.225112in}{0.968102in}%
\pgfsys@useobject{currentmarker}{}%
\end{pgfscope}%
\begin{pgfscope}%
\pgfsys@transformshift{4.203512in}{0.682928in}%
\pgfsys@useobject{currentmarker}{}%
\end{pgfscope}%
\begin{pgfscope}%
\pgfsys@transformshift{4.185670in}{0.551944in}%
\pgfsys@useobject{currentmarker}{}%
\end{pgfscope}%
\begin{pgfscope}%
\pgfsys@transformshift{4.165950in}{0.536210in}%
\pgfsys@useobject{currentmarker}{}%
\end{pgfscope}%
\begin{pgfscope}%
\pgfsys@transformshift{4.147636in}{0.578450in}%
\pgfsys@useobject{currentmarker}{}%
\end{pgfscope}%
\begin{pgfscope}%
\pgfsys@transformshift{4.127680in}{0.773365in}%
\pgfsys@useobject{currentmarker}{}%
\end{pgfscope}%
\begin{pgfscope}%
\pgfsys@transformshift{4.110308in}{1.055578in}%
\pgfsys@useobject{currentmarker}{}%
\end{pgfscope}%
\begin{pgfscope}%
\pgfsys@transformshift{4.086831in}{1.382672in}%
\pgfsys@useobject{currentmarker}{}%
\end{pgfscope}%
\begin{pgfscope}%
\pgfsys@transformshift{4.072744in}{1.426073in}%
\pgfsys@useobject{currentmarker}{}%
\end{pgfscope}%
\begin{pgfscope}%
\pgfsys@transformshift{4.053024in}{1.298283in}%
\pgfsys@useobject{currentmarker}{}%
\end{pgfscope}%
\begin{pgfscope}%
\pgfsys@transformshift{4.028137in}{0.898379in}%
\pgfsys@useobject{currentmarker}{}%
\end{pgfscope}%
\begin{pgfscope}%
\pgfsys@transformshift{4.010764in}{0.699068in}%
\pgfsys@useobject{currentmarker}{}%
\end{pgfscope}%
\begin{pgfscope}%
\pgfsys@transformshift{3.994096in}{0.562579in}%
\pgfsys@useobject{currentmarker}{}%
\end{pgfscope}%
\begin{pgfscope}%
\pgfsys@transformshift{3.974844in}{0.508279in}%
\pgfsys@useobject{currentmarker}{}%
\end{pgfscope}%
\begin{pgfscope}%
\pgfsys@transformshift{3.954419in}{0.627936in}%
\pgfsys@useobject{currentmarker}{}%
\end{pgfscope}%
\begin{pgfscope}%
\pgfsys@transformshift{3.940097in}{0.785782in}%
\pgfsys@useobject{currentmarker}{}%
\end{pgfscope}%
\begin{pgfscope}%
\pgfsys@transformshift{3.918029in}{1.136148in}%
\pgfsys@useobject{currentmarker}{}%
\end{pgfscope}%
\begin{pgfscope}%
\pgfsys@transformshift{3.896194in}{1.389277in}%
\pgfsys@useobject{currentmarker}{}%
\end{pgfscope}%
\begin{pgfscope}%
\pgfsys@transformshift{3.876709in}{1.396620in}%
\pgfsys@useobject{currentmarker}{}%
\end{pgfscope}%
\begin{pgfscope}%
\pgfsys@transformshift{3.856754in}{1.203912in}%
\pgfsys@useobject{currentmarker}{}%
\end{pgfscope}%
\begin{pgfscope}%
\pgfsys@transformshift{3.839146in}{0.897634in}%
\pgfsys@useobject{currentmarker}{}%
\end{pgfscope}%
\begin{pgfscope}%
\pgfsys@transformshift{3.821773in}{0.702419in}%
\pgfsys@useobject{currentmarker}{}%
\end{pgfscope}%
\begin{pgfscope}%
\pgfsys@transformshift{3.802756in}{0.545356in}%
\pgfsys@useobject{currentmarker}{}%
\end{pgfscope}%
\begin{pgfscope}%
\pgfsys@transformshift{3.783270in}{0.498208in}%
\pgfsys@useobject{currentmarker}{}%
\end{pgfscope}%
\begin{pgfscope}%
\pgfsys@transformshift{3.764488in}{0.586479in}%
\pgfsys@useobject{currentmarker}{}%
\end{pgfscope}%
\begin{pgfscope}%
\pgfsys@transformshift{3.742419in}{0.789049in}%
\pgfsys@useobject{currentmarker}{}%
\end{pgfscope}%
\begin{pgfscope}%
\pgfsys@transformshift{3.724811in}{1.071597in}%
\pgfsys@useobject{currentmarker}{}%
\end{pgfscope}%
\begin{pgfscope}%
\pgfsys@transformshift{3.704857in}{1.346691in}%
\pgfsys@useobject{currentmarker}{}%
\end{pgfscope}%
\begin{pgfscope}%
\pgfsys@transformshift{3.687012in}{1.400199in}%
\pgfsys@useobject{currentmarker}{}%
\end{pgfscope}%
\begin{pgfscope}%
\pgfsys@transformshift{3.663537in}{1.242556in}%
\pgfsys@useobject{currentmarker}{}%
\end{pgfscope}%
\begin{pgfscope}%
\pgfsys@transformshift{3.648510in}{1.008965in}%
\pgfsys@useobject{currentmarker}{}%
\end{pgfscope}%
\begin{pgfscope}%
\pgfsys@transformshift{3.627146in}{0.743924in}%
\pgfsys@useobject{currentmarker}{}%
\end{pgfscope}%
\begin{pgfscope}%
\pgfsys@transformshift{3.611182in}{0.601310in}%
\pgfsys@useobject{currentmarker}{}%
\end{pgfscope}%
\begin{pgfscope}%
\pgfsys@transformshift{3.590287in}{0.504194in}%
\pgfsys@useobject{currentmarker}{}%
\end{pgfscope}%
\begin{pgfscope}%
\pgfsys@transformshift{3.569156in}{0.523250in}%
\pgfsys@useobject{currentmarker}{}%
\end{pgfscope}%
\begin{pgfscope}%
\pgfsys@transformshift{3.551550in}{0.629943in}%
\pgfsys@useobject{currentmarker}{}%
\end{pgfscope}%
\begin{pgfscope}%
\pgfsys@transformshift{3.534177in}{0.795366in}%
\pgfsys@useobject{currentmarker}{}%
\end{pgfscope}%
\begin{pgfscope}%
\pgfsys@transformshift{3.513517in}{0.962358in}%
\pgfsys@useobject{currentmarker}{}%
\end{pgfscope}%
\begin{pgfscope}%
\pgfsys@transformshift{3.492857in}{1.257095in}%
\pgfsys@useobject{currentmarker}{}%
\end{pgfscope}%
\begin{pgfscope}%
\pgfsys@transformshift{3.475483in}{1.386969in}%
\pgfsys@useobject{currentmarker}{}%
\end{pgfscope}%
\begin{pgfscope}%
\pgfsys@transformshift{3.453884in}{1.374775in}%
\pgfsys@useobject{currentmarker}{}%
\end{pgfscope}%
\begin{pgfscope}%
\pgfsys@transformshift{3.436510in}{1.198331in}%
\pgfsys@useobject{currentmarker}{}%
\end{pgfscope}%
\begin{pgfscope}%
\pgfsys@transformshift{3.415616in}{0.903064in}%
\pgfsys@useobject{currentmarker}{}%
\end{pgfscope}%
\begin{pgfscope}%
\pgfsys@transformshift{3.397539in}{0.760691in}%
\pgfsys@useobject{currentmarker}{}%
\end{pgfscope}%
\begin{pgfscope}%
\pgfsys@transformshift{3.376879in}{0.627970in}%
\pgfsys@useobject{currentmarker}{}%
\end{pgfscope}%
\begin{pgfscope}%
\pgfsys@transformshift{3.360679in}{0.518312in}%
\pgfsys@useobject{currentmarker}{}%
\end{pgfscope}%
\begin{pgfscope}%
\pgfsys@transformshift{3.338376in}{0.499866in}%
\pgfsys@useobject{currentmarker}{}%
\end{pgfscope}%
\begin{pgfscope}%
\pgfsys@transformshift{3.321237in}{0.579482in}%
\pgfsys@useobject{currentmarker}{}%
\end{pgfscope}%
\begin{pgfscope}%
\pgfsys@transformshift{3.299639in}{0.738749in}%
\pgfsys@useobject{currentmarker}{}%
\end{pgfscope}%
\begin{pgfscope}%
\pgfsys@transformshift{3.278980in}{1.031484in}%
\pgfsys@useobject{currentmarker}{}%
\end{pgfscope}%
\begin{pgfscope}%
\pgfsys@transformshift{3.264423in}{0.636356in}%
\pgfsys@useobject{currentmarker}{}%
\end{pgfscope}%
\begin{pgfscope}%
\pgfsys@transformshift{3.243293in}{0.509518in}%
\pgfsys@useobject{currentmarker}{}%
\end{pgfscope}%
\begin{pgfscope}%
\pgfsys@transformshift{3.222867in}{0.514525in}%
\pgfsys@useobject{currentmarker}{}%
\end{pgfscope}%
\begin{pgfscope}%
\pgfsys@transformshift{3.208077in}{0.603007in}%
\pgfsys@useobject{currentmarker}{}%
\end{pgfscope}%
\begin{pgfscope}%
\pgfsys@transformshift{3.187888in}{0.789711in}%
\pgfsys@useobject{currentmarker}{}%
\end{pgfscope}%
\begin{pgfscope}%
\pgfsys@transformshift{3.166053in}{1.115166in}%
\pgfsys@useobject{currentmarker}{}%
\end{pgfscope}%
\begin{pgfscope}%
\pgfsys@transformshift{3.148445in}{1.281701in}%
\pgfsys@useobject{currentmarker}{}%
\end{pgfscope}%
\begin{pgfscope}%
\pgfsys@transformshift{3.128489in}{1.383278in}%
\pgfsys@useobject{currentmarker}{}%
\end{pgfscope}%
\begin{pgfscope}%
\pgfsys@transformshift{3.110412in}{1.297616in}%
\pgfsys@useobject{currentmarker}{}%
\end{pgfscope}%
\begin{pgfscope}%
\pgfsys@transformshift{3.091395in}{1.046228in}%
\pgfsys@useobject{currentmarker}{}%
\end{pgfscope}%
\begin{pgfscope}%
\pgfsys@transformshift{3.070266in}{0.752098in}%
\pgfsys@useobject{currentmarker}{}%
\end{pgfscope}%
\begin{pgfscope}%
\pgfsys@transformshift{3.052658in}{0.595614in}%
\pgfsys@useobject{currentmarker}{}%
\end{pgfscope}%
\begin{pgfscope}%
\pgfsys@transformshift{3.032938in}{0.494365in}%
\pgfsys@useobject{currentmarker}{}%
\end{pgfscope}%
\begin{pgfscope}%
\pgfsys@transformshift{3.011807in}{0.519686in}%
\pgfsys@useobject{currentmarker}{}%
\end{pgfscope}%
\begin{pgfscope}%
\pgfsys@transformshift{2.994199in}{0.630804in}%
\pgfsys@useobject{currentmarker}{}%
\end{pgfscope}%
\begin{pgfscope}%
\pgfsys@transformshift{2.974714in}{0.750082in}%
\pgfsys@useobject{currentmarker}{}%
\end{pgfscope}%
\begin{pgfscope}%
\pgfsys@transformshift{2.957811in}{1.038468in}%
\pgfsys@useobject{currentmarker}{}%
\end{pgfscope}%
\begin{pgfscope}%
\pgfsys@transformshift{2.935037in}{1.316348in}%
\pgfsys@useobject{currentmarker}{}%
\end{pgfscope}%
\begin{pgfscope}%
\pgfsys@transformshift{2.914142in}{1.379102in}%
\pgfsys@useobject{currentmarker}{}%
\end{pgfscope}%
\begin{pgfscope}%
\pgfsys@transformshift{2.898883in}{1.310900in}%
\pgfsys@useobject{currentmarker}{}%
\end{pgfscope}%
\begin{pgfscope}%
\pgfsys@transformshift{2.878926in}{1.030970in}%
\pgfsys@useobject{currentmarker}{}%
\end{pgfscope}%
\begin{pgfscope}%
\pgfsys@transformshift{2.854509in}{0.714099in}%
\pgfsys@useobject{currentmarker}{}%
\end{pgfscope}%
\begin{pgfscope}%
\pgfsys@transformshift{2.841364in}{0.609484in}%
\pgfsys@useobject{currentmarker}{}%
\end{pgfscope}%
\begin{pgfscope}%
\pgfsys@transformshift{2.819999in}{0.505687in}%
\pgfsys@useobject{currentmarker}{}%
\end{pgfscope}%
\begin{pgfscope}%
\pgfsys@transformshift{2.804739in}{0.492869in}%
\pgfsys@useobject{currentmarker}{}%
\end{pgfscope}%
\begin{pgfscope}%
\pgfsys@transformshift{2.782905in}{0.584485in}%
\pgfsys@useobject{currentmarker}{}%
\end{pgfscope}%
\begin{pgfscope}%
\pgfsys@transformshift{2.762714in}{0.737700in}%
\pgfsys@useobject{currentmarker}{}%
\end{pgfscope}%
\begin{pgfscope}%
\pgfsys@transformshift{2.744637in}{1.006464in}%
\pgfsys@useobject{currentmarker}{}%
\end{pgfscope}%
\begin{pgfscope}%
\pgfsys@transformshift{2.722099in}{1.311725in}%
\pgfsys@useobject{currentmarker}{}%
\end{pgfscope}%
\begin{pgfscope}%
\pgfsys@transformshift{2.705664in}{1.381449in}%
\pgfsys@useobject{currentmarker}{}%
\end{pgfscope}%
\begin{pgfscope}%
\pgfsys@transformshift{2.688056in}{1.318996in}%
\pgfsys@useobject{currentmarker}{}%
\end{pgfscope}%
\begin{pgfscope}%
\pgfsys@transformshift{2.666692in}{1.224512in}%
\pgfsys@useobject{currentmarker}{}%
\end{pgfscope}%
\begin{pgfscope}%
\pgfsys@transformshift{2.647676in}{0.924341in}%
\pgfsys@useobject{currentmarker}{}%
\end{pgfscope}%
\begin{pgfscope}%
\pgfsys@transformshift{2.627485in}{0.692707in}%
\pgfsys@useobject{currentmarker}{}%
\end{pgfscope}%
\begin{pgfscope}%
\pgfsys@transformshift{2.610113in}{0.567330in}%
\pgfsys@useobject{currentmarker}{}%
\end{pgfscope}%
\begin{pgfscope}%
\pgfsys@transformshift{2.590625in}{0.490304in}%
\pgfsys@useobject{currentmarker}{}%
\end{pgfscope}%
\begin{pgfscope}%
\pgfsys@transformshift{2.571140in}{0.507001in}%
\pgfsys@useobject{currentmarker}{}%
\end{pgfscope}%
\begin{pgfscope}%
\pgfsys@transformshift{2.550714in}{0.624152in}%
\pgfsys@useobject{currentmarker}{}%
\end{pgfscope}%
\begin{pgfscope}%
\pgfsys@transformshift{2.533811in}{0.758020in}%
\pgfsys@useobject{currentmarker}{}%
\end{pgfscope}%
\begin{pgfscope}%
\pgfsys@transformshift{2.515498in}{1.024478in}%
\pgfsys@useobject{currentmarker}{}%
\end{pgfscope}%
\begin{pgfscope}%
\pgfsys@transformshift{2.493666in}{1.315278in}%
\pgfsys@useobject{currentmarker}{}%
\end{pgfscope}%
\begin{pgfscope}%
\pgfsys@transformshift{2.474884in}{1.369935in}%
\pgfsys@useobject{currentmarker}{}%
\end{pgfscope}%
\begin{pgfscope}%
\pgfsys@transformshift{2.458215in}{1.358489in}%
\pgfsys@useobject{currentmarker}{}%
\end{pgfscope}%
\begin{pgfscope}%
\pgfsys@transformshift{2.438024in}{1.187588in}%
\pgfsys@useobject{currentmarker}{}%
\end{pgfscope}%
\begin{pgfscope}%
\pgfsys@transformshift{2.438259in}{0.979663in}%
\pgfsys@useobject{currentmarker}{}%
\end{pgfscope}%
\begin{pgfscope}%
\pgfsys@transformshift{2.417833in}{0.870626in}%
\pgfsys@useobject{currentmarker}{}%
\end{pgfscope}%
\begin{pgfscope}%
\pgfsys@transformshift{2.400460in}{0.701396in}%
\pgfsys@useobject{currentmarker}{}%
\end{pgfscope}%
\begin{pgfscope}%
\pgfsys@transformshift{2.381914in}{0.574653in}%
\pgfsys@useobject{currentmarker}{}%
\end{pgfscope}%
\begin{pgfscope}%
\pgfsys@transformshift{2.359611in}{0.491390in}%
\pgfsys@useobject{currentmarker}{}%
\end{pgfscope}%
\begin{pgfscope}%
\pgfsys@transformshift{2.341063in}{0.482035in}%
\pgfsys@useobject{currentmarker}{}%
\end{pgfscope}%
\begin{pgfscope}%
\pgfsys@transformshift{2.319463in}{0.537591in}%
\pgfsys@useobject{currentmarker}{}%
\end{pgfscope}%
\begin{pgfscope}%
\pgfsys@transformshift{2.301855in}{0.657989in}%
\pgfsys@useobject{currentmarker}{}%
\end{pgfscope}%
\begin{pgfscope}%
\pgfsys@transformshift{2.282370in}{0.876663in}%
\pgfsys@useobject{currentmarker}{}%
\end{pgfscope}%
\begin{pgfscope}%
\pgfsys@transformshift{2.263353in}{1.098035in}%
\pgfsys@useobject{currentmarker}{}%
\end{pgfscope}%
\begin{pgfscope}%
\pgfsys@transformshift{2.243867in}{1.346899in}%
\pgfsys@useobject{currentmarker}{}%
\end{pgfscope}%
\begin{pgfscope}%
\pgfsys@transformshift{2.225790in}{1.381037in}%
\pgfsys@useobject{currentmarker}{}%
\end{pgfscope}%
\begin{pgfscope}%
\pgfsys@transformshift{2.207477in}{1.295655in}%
\pgfsys@useobject{currentmarker}{}%
\end{pgfscope}%
\begin{pgfscope}%
\pgfsys@transformshift{2.186348in}{1.026107in}%
\pgfsys@useobject{currentmarker}{}%
\end{pgfscope}%
\begin{pgfscope}%
\pgfsys@transformshift{2.167800in}{0.773385in}%
\pgfsys@useobject{currentmarker}{}%
\end{pgfscope}%
\begin{pgfscope}%
\pgfsys@transformshift{2.149020in}{0.629992in}%
\pgfsys@useobject{currentmarker}{}%
\end{pgfscope}%
\begin{pgfscope}%
\pgfsys@transformshift{2.130003in}{0.529653in}%
\pgfsys@useobject{currentmarker}{}%
\end{pgfscope}%
\begin{pgfscope}%
\pgfsys@transformshift{2.111924in}{0.491304in}%
\pgfsys@useobject{currentmarker}{}%
\end{pgfscope}%
\begin{pgfscope}%
\pgfsys@transformshift{2.090326in}{0.536978in}%
\pgfsys@useobject{currentmarker}{}%
\end{pgfscope}%
\begin{pgfscope}%
\pgfsys@transformshift{2.071075in}{0.612512in}%
\pgfsys@useobject{currentmarker}{}%
\end{pgfscope}%
\begin{pgfscope}%
\pgfsys@transformshift{2.052527in}{0.732296in}%
\pgfsys@useobject{currentmarker}{}%
\end{pgfscope}%
\begin{pgfscope}%
\pgfsys@transformshift{2.034450in}{0.985848in}%
\pgfsys@useobject{currentmarker}{}%
\end{pgfscope}%
\begin{pgfscope}%
\pgfsys@transformshift{2.013321in}{1.276617in}%
\pgfsys@useobject{currentmarker}{}%
\end{pgfscope}%
\begin{pgfscope}%
\pgfsys@transformshift{1.993834in}{1.385351in}%
\pgfsys@useobject{currentmarker}{}%
\end{pgfscope}%
\begin{pgfscope}%
\pgfsys@transformshift{1.975757in}{1.342122in}%
\pgfsys@useobject{currentmarker}{}%
\end{pgfscope}%
\begin{pgfscope}%
\pgfsys@transformshift{1.957680in}{1.199864in}%
\pgfsys@useobject{currentmarker}{}%
\end{pgfscope}%
\begin{pgfscope}%
\pgfsys@transformshift{1.937723in}{0.968649in}%
\pgfsys@useobject{currentmarker}{}%
\end{pgfscope}%
\begin{pgfscope}%
\pgfsys@transformshift{1.919881in}{0.788909in}%
\pgfsys@useobject{currentmarker}{}%
\end{pgfscope}%
\begin{pgfscope}%
\pgfsys@transformshift{1.898283in}{0.609286in}%
\pgfsys@useobject{currentmarker}{}%
\end{pgfscope}%
\begin{pgfscope}%
\pgfsys@transformshift{1.879501in}{0.538592in}%
\pgfsys@useobject{currentmarker}{}%
\end{pgfscope}%
\begin{pgfscope}%
\pgfsys@transformshift{1.861893in}{0.505890in}%
\pgfsys@useobject{currentmarker}{}%
\end{pgfscope}%
\begin{pgfscope}%
\pgfsys@transformshift{1.842407in}{0.495362in}%
\pgfsys@useobject{currentmarker}{}%
\end{pgfscope}%
\begin{pgfscope}%
\pgfsys@transformshift{1.820573in}{0.572875in}%
\pgfsys@useobject{currentmarker}{}%
\end{pgfscope}%
\begin{pgfscope}%
\pgfsys@transformshift{1.804374in}{0.687131in}%
\pgfsys@useobject{currentmarker}{}%
\end{pgfscope}%
\begin{pgfscope}%
\pgfsys@transformshift{1.785357in}{0.892116in}%
\pgfsys@useobject{currentmarker}{}%
\end{pgfscope}%
\begin{pgfscope}%
\pgfsys@transformshift{1.764931in}{1.196182in}%
\pgfsys@useobject{currentmarker}{}%
\end{pgfscope}%
\begin{pgfscope}%
\pgfsys@transformshift{1.746384in}{1.367136in}%
\pgfsys@useobject{currentmarker}{}%
\end{pgfscope}%
\begin{pgfscope}%
\pgfsys@transformshift{1.724315in}{1.392967in}%
\pgfsys@useobject{currentmarker}{}%
\end{pgfscope}%
\begin{pgfscope}%
\pgfsys@transformshift{1.710230in}{1.303243in}%
\pgfsys@useobject{currentmarker}{}%
\end{pgfscope}%
\begin{pgfscope}%
\pgfsys@transformshift{1.686752in}{1.147719in}%
\pgfsys@useobject{currentmarker}{}%
\end{pgfscope}%
\begin{pgfscope}%
\pgfsys@transformshift{1.668675in}{0.924351in}%
\pgfsys@useobject{currentmarker}{}%
\end{pgfscope}%
\begin{pgfscope}%
\pgfsys@transformshift{1.646841in}{0.733579in}%
\pgfsys@useobject{currentmarker}{}%
\end{pgfscope}%
\begin{pgfscope}%
\pgfsys@transformshift{1.631111in}{0.629886in}%
\pgfsys@useobject{currentmarker}{}%
\end{pgfscope}%
\begin{pgfscope}%
\pgfsys@transformshift{1.606696in}{0.524493in}%
\pgfsys@useobject{currentmarker}{}%
\end{pgfscope}%
\begin{pgfscope}%
\pgfsys@transformshift{1.590496in}{0.498234in}%
\pgfsys@useobject{currentmarker}{}%
\end{pgfscope}%
\begin{pgfscope}%
\pgfsys@transformshift{1.574061in}{0.532831in}%
\pgfsys@useobject{currentmarker}{}%
\end{pgfscope}%
\begin{pgfscope}%
\pgfsys@transformshift{1.554106in}{0.617407in}%
\pgfsys@useobject{currentmarker}{}%
\end{pgfscope}%
\begin{pgfscope}%
\pgfsys@transformshift{1.534855in}{0.791061in}%
\pgfsys@useobject{currentmarker}{}%
\end{pgfscope}%
\begin{pgfscope}%
\pgfsys@transformshift{1.511377in}{0.957710in}%
\pgfsys@useobject{currentmarker}{}%
\end{pgfscope}%
\begin{pgfscope}%
\pgfsys@transformshift{1.495178in}{1.168736in}%
\pgfsys@useobject{currentmarker}{}%
\end{pgfscope}%
\begin{pgfscope}%
\pgfsys@transformshift{1.476865in}{1.356547in}%
\pgfsys@useobject{currentmarker}{}%
\end{pgfscope}%
\begin{pgfscope}%
\pgfsys@transformshift{1.454327in}{1.408926in}%
\pgfsys@useobject{currentmarker}{}%
\end{pgfscope}%
\begin{pgfscope}%
\pgfsys@transformshift{1.436485in}{1.382594in}%
\pgfsys@useobject{currentmarker}{}%
\end{pgfscope}%
\begin{pgfscope}%
\pgfsys@transformshift{1.421225in}{1.255066in}%
\pgfsys@useobject{currentmarker}{}%
\end{pgfscope}%
\begin{pgfscope}%
\pgfsys@transformshift{1.400094in}{1.100432in}%
\pgfsys@useobject{currentmarker}{}%
\end{pgfscope}%
\begin{pgfscope}%
\pgfsys@transformshift{1.380374in}{1.244184in}%
\pgfsys@useobject{currentmarker}{}%
\end{pgfscope}%
\begin{pgfscope}%
\pgfsys@transformshift{1.362297in}{0.999313in}%
\pgfsys@useobject{currentmarker}{}%
\end{pgfscope}%
\begin{pgfscope}%
\pgfsys@transformshift{1.341637in}{0.908990in}%
\pgfsys@useobject{currentmarker}{}%
\end{pgfscope}%
\begin{pgfscope}%
\pgfsys@transformshift{1.319334in}{0.733607in}%
\pgfsys@useobject{currentmarker}{}%
\end{pgfscope}%
\begin{pgfscope}%
\pgfsys@transformshift{1.300786in}{0.600993in}%
\pgfsys@useobject{currentmarker}{}%
\end{pgfscope}%
\begin{pgfscope}%
\pgfsys@transformshift{1.285996in}{0.530788in}%
\pgfsys@useobject{currentmarker}{}%
\end{pgfscope}%
\begin{pgfscope}%
\pgfsys@transformshift{1.263927in}{0.509461in}%
\pgfsys@useobject{currentmarker}{}%
\end{pgfscope}%
\begin{pgfscope}%
\pgfsys@transformshift{1.245145in}{0.553735in}%
\pgfsys@useobject{currentmarker}{}%
\end{pgfscope}%
\begin{pgfscope}%
\pgfsys@transformshift{1.226363in}{0.640137in}%
\pgfsys@useobject{currentmarker}{}%
\end{pgfscope}%
\begin{pgfscope}%
\pgfsys@transformshift{1.207817in}{0.756410in}%
\pgfsys@useobject{currentmarker}{}%
\end{pgfscope}%
\begin{pgfscope}%
\pgfsys@transformshift{1.189269in}{0.984596in}%
\pgfsys@useobject{currentmarker}{}%
\end{pgfscope}%
\begin{pgfscope}%
\pgfsys@transformshift{1.166497in}{1.290492in}%
\pgfsys@useobject{currentmarker}{}%
\end{pgfscope}%
\begin{pgfscope}%
\pgfsys@transformshift{1.151706in}{1.393438in}%
\pgfsys@useobject{currentmarker}{}%
\end{pgfscope}%
\begin{pgfscope}%
\pgfsys@transformshift{1.129169in}{1.424844in}%
\pgfsys@useobject{currentmarker}{}%
\end{pgfscope}%
\begin{pgfscope}%
\pgfsys@transformshift{1.111795in}{1.400665in}%
\pgfsys@useobject{currentmarker}{}%
\end{pgfscope}%
\begin{pgfscope}%
\pgfsys@transformshift{1.089961in}{1.222180in}%
\pgfsys@useobject{currentmarker}{}%
\end{pgfscope}%
\begin{pgfscope}%
\pgfsys@transformshift{1.071413in}{0.965322in}%
\pgfsys@useobject{currentmarker}{}%
\end{pgfscope}%
\begin{pgfscope}%
\pgfsys@transformshift{1.053336in}{0.803030in}%
\pgfsys@useobject{currentmarker}{}%
\end{pgfscope}%
\begin{pgfscope}%
\pgfsys@transformshift{1.034319in}{0.669341in}%
\pgfsys@useobject{currentmarker}{}%
\end{pgfscope}%
\begin{pgfscope}%
\pgfsys@transformshift{1.016242in}{0.567631in}%
\pgfsys@useobject{currentmarker}{}%
\end{pgfscope}%
\begin{pgfscope}%
\pgfsys@transformshift{0.994643in}{0.518116in}%
\pgfsys@useobject{currentmarker}{}%
\end{pgfscope}%
\begin{pgfscope}%
\pgfsys@transformshift{0.977975in}{0.564712in}%
\pgfsys@useobject{currentmarker}{}%
\end{pgfscope}%
\begin{pgfscope}%
\pgfsys@transformshift{0.957080in}{0.681978in}%
\pgfsys@useobject{currentmarker}{}%
\end{pgfscope}%
\begin{pgfscope}%
\pgfsys@transformshift{0.938767in}{0.832096in}%
\pgfsys@useobject{currentmarker}{}%
\end{pgfscope}%
\begin{pgfscope}%
\pgfsys@transformshift{0.917403in}{1.051385in}%
\pgfsys@useobject{currentmarker}{}%
\end{pgfscope}%
\begin{pgfscope}%
\pgfsys@transformshift{0.898856in}{1.284594in}%
\pgfsys@useobject{currentmarker}{}%
\end{pgfscope}%
\begin{pgfscope}%
\pgfsys@transformshift{0.880073in}{1.415329in}%
\pgfsys@useobject{currentmarker}{}%
\end{pgfscope}%
\begin{pgfscope}%
\pgfsys@transformshift{0.862467in}{1.443121in}%
\pgfsys@useobject{currentmarker}{}%
\end{pgfscope}%
\begin{pgfscope}%
\pgfsys@transformshift{0.843451in}{1.405237in}%
\pgfsys@useobject{currentmarker}{}%
\end{pgfscope}%
\begin{pgfscope}%
\pgfsys@transformshift{0.821851in}{0.652925in}%
\pgfsys@useobject{currentmarker}{}%
\end{pgfscope}%
\begin{pgfscope}%
\pgfsys@transformshift{0.803303in}{0.840023in}%
\pgfsys@useobject{currentmarker}{}%
\end{pgfscope}%
\begin{pgfscope}%
\pgfsys@transformshift{0.784523in}{1.118190in}%
\pgfsys@useobject{currentmarker}{}%
\end{pgfscope}%
\begin{pgfscope}%
\pgfsys@transformshift{0.763157in}{1.371231in}%
\pgfsys@useobject{currentmarker}{}%
\end{pgfscope}%
\begin{pgfscope}%
\pgfsys@transformshift{0.747427in}{1.442042in}%
\pgfsys@useobject{currentmarker}{}%
\end{pgfscope}%
\begin{pgfscope}%
\pgfsys@transformshift{0.725595in}{1.406102in}%
\pgfsys@useobject{currentmarker}{}%
\end{pgfscope}%
\begin{pgfscope}%
\pgfsys@transformshift{0.707516in}{1.261367in}%
\pgfsys@useobject{currentmarker}{}%
\end{pgfscope}%
\begin{pgfscope}%
\pgfsys@transformshift{0.688736in}{1.011205in}%
\pgfsys@useobject{currentmarker}{}%
\end{pgfscope}%
\begin{pgfscope}%
\pgfsys@transformshift{0.666667in}{0.759187in}%
\pgfsys@useobject{currentmarker}{}%
\end{pgfscope}%
\begin{pgfscope}%
\pgfsys@transformshift{0.648822in}{0.625161in}%
\pgfsys@useobject{currentmarker}{}%
\end{pgfscope}%
\begin{pgfscope}%
\pgfsys@transformshift{0.650231in}{0.631428in}%
\pgfsys@useobject{currentmarker}{}%
\end{pgfscope}%
\begin{pgfscope}%
\pgfsys@transformshift{0.657275in}{0.713457in}%
\pgfsys@useobject{currentmarker}{}%
\end{pgfscope}%
\begin{pgfscope}%
\pgfsys@transformshift{0.676526in}{0.933369in}%
\pgfsys@useobject{currentmarker}{}%
\end{pgfscope}%
\begin{pgfscope}%
\pgfsys@transformshift{0.695543in}{1.273894in}%
\pgfsys@useobject{currentmarker}{}%
\end{pgfscope}%
\begin{pgfscope}%
\pgfsys@transformshift{0.713151in}{1.433076in}%
\pgfsys@useobject{currentmarker}{}%
\end{pgfscope}%
\begin{pgfscope}%
\pgfsys@transformshift{0.731228in}{1.410598in}%
\pgfsys@useobject{currentmarker}{}%
\end{pgfscope}%
\begin{pgfscope}%
\pgfsys@transformshift{0.750950in}{1.167743in}%
\pgfsys@useobject{currentmarker}{}%
\end{pgfscope}%
\begin{pgfscope}%
\pgfsys@transformshift{0.772079in}{0.806387in}%
\pgfsys@useobject{currentmarker}{}%
\end{pgfscope}%
\begin{pgfscope}%
\pgfsys@transformshift{0.790156in}{0.607843in}%
\pgfsys@useobject{currentmarker}{}%
\end{pgfscope}%
\begin{pgfscope}%
\pgfsys@transformshift{0.807529in}{0.663866in}%
\pgfsys@useobject{currentmarker}{}%
\end{pgfscope}%
\begin{pgfscope}%
\pgfsys@transformshift{0.827486in}{0.927731in}%
\pgfsys@useobject{currentmarker}{}%
\end{pgfscope}%
\begin{pgfscope}%
\pgfsys@transformshift{0.847440in}{1.251173in}%
\pgfsys@useobject{currentmarker}{}%
\end{pgfscope}%
\begin{pgfscope}%
\pgfsys@transformshift{0.868571in}{1.425760in}%
\pgfsys@useobject{currentmarker}{}%
\end{pgfscope}%
\begin{pgfscope}%
\pgfsys@transformshift{0.886648in}{1.369900in}%
\pgfsys@useobject{currentmarker}{}%
\end{pgfscope}%
\begin{pgfscope}%
\pgfsys@transformshift{0.905430in}{1.093811in}%
\pgfsys@useobject{currentmarker}{}%
\end{pgfscope}%
\begin{pgfscope}%
\pgfsys@transformshift{0.925385in}{0.721880in}%
\pgfsys@useobject{currentmarker}{}%
\end{pgfscope}%
\begin{pgfscope}%
\pgfsys@transformshift{0.944167in}{0.556671in}%
\pgfsys@useobject{currentmarker}{}%
\end{pgfscope}%
\begin{pgfscope}%
\pgfsys@transformshift{0.962479in}{0.510954in}%
\pgfsys@useobject{currentmarker}{}%
\end{pgfscope}%
\begin{pgfscope}%
\pgfsys@transformshift{0.983844in}{0.641007in}%
\pgfsys@useobject{currentmarker}{}%
\end{pgfscope}%
\begin{pgfscope}%
\pgfsys@transformshift{1.003330in}{0.849025in}%
\pgfsys@useobject{currentmarker}{}%
\end{pgfscope}%
\begin{pgfscope}%
\pgfsys@transformshift{1.022581in}{1.175362in}%
\pgfsys@useobject{currentmarker}{}%
\end{pgfscope}%
\begin{pgfscope}%
\pgfsys@transformshift{1.038077in}{1.370891in}%
\pgfsys@useobject{currentmarker}{}%
\end{pgfscope}%
\begin{pgfscope}%
\pgfsys@transformshift{1.059909in}{1.393990in}%
\pgfsys@useobject{currentmarker}{}%
\end{pgfscope}%
\begin{pgfscope}%
\pgfsys@transformshift{1.079397in}{1.195987in}%
\pgfsys@useobject{currentmarker}{}%
\end{pgfscope}%
\begin{pgfscope}%
\pgfsys@transformshift{1.097239in}{0.875967in}%
\pgfsys@useobject{currentmarker}{}%
\end{pgfscope}%
\begin{pgfscope}%
\pgfsys@transformshift{1.117664in}{0.628675in}%
\pgfsys@useobject{currentmarker}{}%
\end{pgfscope}%
\begin{pgfscope}%
\pgfsys@transformshift{1.136445in}{0.515152in}%
\pgfsys@useobject{currentmarker}{}%
\end{pgfscope}%
\begin{pgfscope}%
\pgfsys@transformshift{1.156167in}{0.521014in}%
\pgfsys@useobject{currentmarker}{}%
\end{pgfscope}%
\begin{pgfscope}%
\pgfsys@transformshift{1.175184in}{0.633899in}%
\pgfsys@useobject{currentmarker}{}%
\end{pgfscope}%
\begin{pgfscope}%
\pgfsys@transformshift{1.195609in}{0.803275in}%
\pgfsys@useobject{currentmarker}{}%
\end{pgfscope}%
\begin{pgfscope}%
\pgfsys@transformshift{1.213921in}{1.086807in}%
\pgfsys@useobject{currentmarker}{}%
\end{pgfscope}%
\begin{pgfscope}%
\pgfsys@transformshift{1.230120in}{1.316664in}%
\pgfsys@useobject{currentmarker}{}%
\end{pgfscope}%
\begin{pgfscope}%
\pgfsys@transformshift{1.251720in}{1.403264in}%
\pgfsys@useobject{currentmarker}{}%
\end{pgfscope}%
\begin{pgfscope}%
\pgfsys@transformshift{1.270971in}{1.356261in}%
\pgfsys@useobject{currentmarker}{}%
\end{pgfscope}%
\begin{pgfscope}%
\pgfsys@transformshift{1.290456in}{1.122335in}%
\pgfsys@useobject{currentmarker}{}%
\end{pgfscope}%
\begin{pgfscope}%
\pgfsys@transformshift{1.309473in}{0.797804in}%
\pgfsys@useobject{currentmarker}{}%
\end{pgfscope}%
\begin{pgfscope}%
\pgfsys@transformshift{1.328255in}{0.620686in}%
\pgfsys@useobject{currentmarker}{}%
\end{pgfscope}%
\begin{pgfscope}%
\pgfsys@transformshift{1.347272in}{0.509965in}%
\pgfsys@useobject{currentmarker}{}%
\end{pgfscope}%
\begin{pgfscope}%
\pgfsys@transformshift{1.365818in}{0.496390in}%
\pgfsys@useobject{currentmarker}{}%
\end{pgfscope}%
\begin{pgfscope}%
\pgfsys@transformshift{1.384366in}{0.582173in}%
\pgfsys@useobject{currentmarker}{}%
\end{pgfscope}%
\begin{pgfscope}%
\pgfsys@transformshift{1.404791in}{0.727363in}%
\pgfsys@useobject{currentmarker}{}%
\end{pgfscope}%
\begin{pgfscope}%
\pgfsys@transformshift{1.426860in}{0.976195in}%
\pgfsys@useobject{currentmarker}{}%
\end{pgfscope}%
\begin{pgfscope}%
\pgfsys@transformshift{1.445406in}{1.268210in}%
\pgfsys@useobject{currentmarker}{}%
\end{pgfscope}%
\begin{pgfscope}%
\pgfsys@transformshift{1.462311in}{1.377471in}%
\pgfsys@useobject{currentmarker}{}%
\end{pgfscope}%
\begin{pgfscope}%
\pgfsys@transformshift{1.480856in}{1.374341in}%
\pgfsys@useobject{currentmarker}{}%
\end{pgfscope}%
\begin{pgfscope}%
\pgfsys@transformshift{1.501516in}{1.186063in}%
\pgfsys@useobject{currentmarker}{}%
\end{pgfscope}%
\begin{pgfscope}%
\pgfsys@transformshift{1.520064in}{0.925220in}%
\pgfsys@useobject{currentmarker}{}%
\end{pgfscope}%
\begin{pgfscope}%
\pgfsys@transformshift{1.540724in}{0.679253in}%
\pgfsys@useobject{currentmarker}{}%
\end{pgfscope}%
\begin{pgfscope}%
\pgfsys@transformshift{1.559506in}{0.544008in}%
\pgfsys@useobject{currentmarker}{}%
\end{pgfscope}%
\begin{pgfscope}%
\pgfsys@transformshift{1.578758in}{0.498728in}%
\pgfsys@useobject{currentmarker}{}%
\end{pgfscope}%
\begin{pgfscope}%
\pgfsys@transformshift{1.597538in}{0.506971in}%
\pgfsys@useobject{currentmarker}{}%
\end{pgfscope}%
\begin{pgfscope}%
\pgfsys@transformshift{1.617026in}{0.592599in}%
\pgfsys@useobject{currentmarker}{}%
\end{pgfscope}%
\begin{pgfscope}%
\pgfsys@transformshift{1.635103in}{0.739747in}%
\pgfsys@useobject{currentmarker}{}%
\end{pgfscope}%
\begin{pgfscope}%
\pgfsys@transformshift{1.658814in}{0.957398in}%
\pgfsys@useobject{currentmarker}{}%
\end{pgfscope}%
\begin{pgfscope}%
\pgfsys@transformshift{1.674545in}{1.048352in}%
\pgfsys@useobject{currentmarker}{}%
\end{pgfscope}%
\begin{pgfscope}%
\pgfsys@transformshift{1.696143in}{0.711823in}%
\pgfsys@useobject{currentmarker}{}%
\end{pgfscope}%
\begin{pgfscope}%
\pgfsys@transformshift{1.711873in}{0.579721in}%
\pgfsys@useobject{currentmarker}{}%
\end{pgfscope}%
\begin{pgfscope}%
\pgfsys@transformshift{1.732533in}{0.489116in}%
\pgfsys@useobject{currentmarker}{}%
\end{pgfscope}%
\begin{pgfscope}%
\pgfsys@transformshift{1.751081in}{0.532970in}%
\pgfsys@useobject{currentmarker}{}%
\end{pgfscope}%
\begin{pgfscope}%
\pgfsys@transformshift{1.769861in}{0.667467in}%
\pgfsys@useobject{currentmarker}{}%
\end{pgfscope}%
\begin{pgfscope}%
\pgfsys@transformshift{1.789349in}{0.852964in}%
\pgfsys@useobject{currentmarker}{}%
\end{pgfscope}%
\begin{pgfscope}%
\pgfsys@transformshift{1.807660in}{1.143508in}%
\pgfsys@useobject{currentmarker}{}%
\end{pgfscope}%
\begin{pgfscope}%
\pgfsys@transformshift{1.830668in}{1.365658in}%
\pgfsys@useobject{currentmarker}{}%
\end{pgfscope}%
\begin{pgfscope}%
\pgfsys@transformshift{1.848511in}{1.367477in}%
\pgfsys@useobject{currentmarker}{}%
\end{pgfscope}%
\begin{pgfscope}%
\pgfsys@transformshift{1.867997in}{1.177289in}%
\pgfsys@useobject{currentmarker}{}%
\end{pgfscope}%
\begin{pgfscope}%
\pgfsys@transformshift{1.886544in}{0.896429in}%
\pgfsys@useobject{currentmarker}{}%
\end{pgfscope}%
\begin{pgfscope}%
\pgfsys@transformshift{1.908142in}{0.632488in}%
\pgfsys@useobject{currentmarker}{}%
\end{pgfscope}%
\begin{pgfscope}%
\pgfsys@transformshift{1.924810in}{0.524389in}%
\pgfsys@useobject{currentmarker}{}%
\end{pgfscope}%
\begin{pgfscope}%
\pgfsys@transformshift{1.943827in}{0.484006in}%
\pgfsys@useobject{currentmarker}{}%
\end{pgfscope}%
\begin{pgfscope}%
\pgfsys@transformshift{1.961906in}{0.509792in}%
\pgfsys@useobject{currentmarker}{}%
\end{pgfscope}%
\begin{pgfscope}%
\pgfsys@transformshift{1.980686in}{0.618143in}%
\pgfsys@useobject{currentmarker}{}%
\end{pgfscope}%
\begin{pgfscope}%
\pgfsys@transformshift{2.003226in}{0.842604in}%
\pgfsys@useobject{currentmarker}{}%
\end{pgfscope}%
\begin{pgfscope}%
\pgfsys@transformshift{2.019894in}{1.118544in}%
\pgfsys@useobject{currentmarker}{}%
\end{pgfscope}%
\begin{pgfscope}%
\pgfsys@transformshift{2.038911in}{1.303432in}%
\pgfsys@useobject{currentmarker}{}%
\end{pgfscope}%
\begin{pgfscope}%
\pgfsys@transformshift{2.061214in}{1.379215in}%
\pgfsys@useobject{currentmarker}{}%
\end{pgfscope}%
\begin{pgfscope}%
\pgfsys@transformshift{2.077179in}{1.291925in}%
\pgfsys@useobject{currentmarker}{}%
\end{pgfscope}%
\begin{pgfscope}%
\pgfsys@transformshift{2.098308in}{0.971349in}%
\pgfsys@useobject{currentmarker}{}%
\end{pgfscope}%
\begin{pgfscope}%
\pgfsys@transformshift{2.113098in}{0.722803in}%
\pgfsys@useobject{currentmarker}{}%
\end{pgfscope}%
\begin{pgfscope}%
\pgfsys@transformshift{2.135872in}{0.573215in}%
\pgfsys@useobject{currentmarker}{}%
\end{pgfscope}%
\begin{pgfscope}%
\pgfsys@transformshift{2.157236in}{0.486716in}%
\pgfsys@useobject{currentmarker}{}%
\end{pgfscope}%
\begin{pgfscope}%
\pgfsys@transformshift{2.175313in}{0.514289in}%
\pgfsys@useobject{currentmarker}{}%
\end{pgfscope}%
\begin{pgfscope}%
\pgfsys@transformshift{2.193626in}{0.616906in}%
\pgfsys@useobject{currentmarker}{}%
\end{pgfscope}%
\begin{pgfscope}%
\pgfsys@transformshift{2.211937in}{0.774322in}%
\pgfsys@useobject{currentmarker}{}%
\end{pgfscope}%
\begin{pgfscope}%
\pgfsys@transformshift{2.232597in}{1.062059in}%
\pgfsys@useobject{currentmarker}{}%
\end{pgfscope}%
\begin{pgfscope}%
\pgfsys@transformshift{2.250676in}{1.297491in}%
\pgfsys@useobject{currentmarker}{}%
\end{pgfscope}%
\begin{pgfscope}%
\pgfsys@transformshift{2.268284in}{1.380327in}%
\pgfsys@useobject{currentmarker}{}%
\end{pgfscope}%
\begin{pgfscope}%
\pgfsys@transformshift{2.286127in}{1.347617in}%
\pgfsys@useobject{currentmarker}{}%
\end{pgfscope}%
\begin{pgfscope}%
\pgfsys@transformshift{2.307961in}{1.165529in}%
\pgfsys@useobject{currentmarker}{}%
\end{pgfscope}%
\begin{pgfscope}%
\pgfsys@transformshift{2.329090in}{0.827242in}%
\pgfsys@useobject{currentmarker}{}%
\end{pgfscope}%
\begin{pgfscope}%
\pgfsys@transformshift{2.346932in}{0.657575in}%
\pgfsys@useobject{currentmarker}{}%
\end{pgfscope}%
\begin{pgfscope}%
\pgfsys@transformshift{2.364540in}{0.539717in}%
\pgfsys@useobject{currentmarker}{}%
\end{pgfscope}%
\begin{pgfscope}%
\pgfsys@transformshift{2.385200in}{0.480587in}%
\pgfsys@useobject{currentmarker}{}%
\end{pgfscope}%
\begin{pgfscope}%
\pgfsys@transformshift{2.403982in}{0.523291in}%
\pgfsys@useobject{currentmarker}{}%
\end{pgfscope}%
\begin{pgfscope}%
\pgfsys@transformshift{2.424408in}{0.642884in}%
\pgfsys@useobject{currentmarker}{}%
\end{pgfscope}%
\begin{pgfscope}%
\pgfsys@transformshift{2.442485in}{0.752407in}%
\pgfsys@useobject{currentmarker}{}%
\end{pgfscope}%
\begin{pgfscope}%
\pgfsys@transformshift{2.463145in}{1.064320in}%
\pgfsys@useobject{currentmarker}{}%
\end{pgfscope}%
\begin{pgfscope}%
\pgfsys@transformshift{2.481222in}{1.303823in}%
\pgfsys@useobject{currentmarker}{}%
\end{pgfscope}%
\begin{pgfscope}%
\pgfsys@transformshift{2.501413in}{1.374892in}%
\pgfsys@useobject{currentmarker}{}%
\end{pgfscope}%
\begin{pgfscope}%
\pgfsys@transformshift{2.521133in}{1.344924in}%
\pgfsys@useobject{currentmarker}{}%
\end{pgfscope}%
\begin{pgfscope}%
\pgfsys@transformshift{2.538272in}{1.153977in}%
\pgfsys@useobject{currentmarker}{}%
\end{pgfscope}%
\begin{pgfscope}%
\pgfsys@transformshift{2.559401in}{0.793687in}%
\pgfsys@useobject{currentmarker}{}%
\end{pgfscope}%
\begin{pgfscope}%
\pgfsys@transformshift{2.577949in}{0.640772in}%
\pgfsys@useobject{currentmarker}{}%
\end{pgfscope}%
\begin{pgfscope}%
\pgfsys@transformshift{2.598843in}{0.518002in}%
\pgfsys@useobject{currentmarker}{}%
\end{pgfscope}%
\begin{pgfscope}%
\pgfsys@transformshift{2.616686in}{0.504283in}%
\pgfsys@useobject{currentmarker}{}%
\end{pgfscope}%
\begin{pgfscope}%
\pgfsys@transformshift{2.634997in}{0.490799in}%
\pgfsys@useobject{currentmarker}{}%
\end{pgfscope}%
\begin{pgfscope}%
\pgfsys@transformshift{2.655188in}{0.576109in}%
\pgfsys@useobject{currentmarker}{}%
\end{pgfscope}%
\begin{pgfscope}%
\pgfsys@transformshift{2.673031in}{0.682830in}%
\pgfsys@useobject{currentmarker}{}%
\end{pgfscope}%
\begin{pgfscope}%
\pgfsys@transformshift{2.694161in}{0.918262in}%
\pgfsys@useobject{currentmarker}{}%
\end{pgfscope}%
\begin{pgfscope}%
\pgfsys@transformshift{2.715525in}{1.220713in}%
\pgfsys@useobject{currentmarker}{}%
\end{pgfscope}%
\begin{pgfscope}%
\pgfsys@transformshift{2.731021in}{1.355452in}%
\pgfsys@useobject{currentmarker}{}%
\end{pgfscope}%
\begin{pgfscope}%
\pgfsys@transformshift{2.748863in}{1.374770in}%
\pgfsys@useobject{currentmarker}{}%
\end{pgfscope}%
\begin{pgfscope}%
\pgfsys@transformshift{2.768818in}{1.274566in}%
\pgfsys@useobject{currentmarker}{}%
\end{pgfscope}%
\begin{pgfscope}%
\pgfsys@transformshift{2.790652in}{0.960640in}%
\pgfsys@useobject{currentmarker}{}%
\end{pgfscope}%
\begin{pgfscope}%
\pgfsys@transformshift{2.808260in}{0.808046in}%
\pgfsys@useobject{currentmarker}{}%
\end{pgfscope}%
\begin{pgfscope}%
\pgfsys@transformshift{2.825399in}{0.631344in}%
\pgfsys@useobject{currentmarker}{}%
\end{pgfscope}%
\begin{pgfscope}%
\pgfsys@transformshift{2.847233in}{0.519168in}%
\pgfsys@useobject{currentmarker}{}%
\end{pgfscope}%
\begin{pgfscope}%
\pgfsys@transformshift{2.865310in}{0.485630in}%
\pgfsys@useobject{currentmarker}{}%
\end{pgfscope}%
\begin{pgfscope}%
\pgfsys@transformshift{2.886204in}{0.558537in}%
\pgfsys@useobject{currentmarker}{}%
\end{pgfscope}%
\begin{pgfscope}%
\pgfsys@transformshift{2.903107in}{0.610551in}%
\pgfsys@useobject{currentmarker}{}%
\end{pgfscope}%
\begin{pgfscope}%
\pgfsys@transformshift{2.924941in}{0.761490in}%
\pgfsys@useobject{currentmarker}{}%
\end{pgfscope}%
\begin{pgfscope}%
\pgfsys@transformshift{2.942784in}{0.989641in}%
\pgfsys@useobject{currentmarker}{}%
\end{pgfscope}%
\begin{pgfscope}%
\pgfsys@transformshift{2.960157in}{1.241739in}%
\pgfsys@useobject{currentmarker}{}%
\end{pgfscope}%
\begin{pgfscope}%
\pgfsys@transformshift{2.981286in}{1.379149in}%
\pgfsys@useobject{currentmarker}{}%
\end{pgfscope}%
\begin{pgfscope}%
\pgfsys@transformshift{2.999600in}{1.378339in}%
\pgfsys@useobject{currentmarker}{}%
\end{pgfscope}%
\begin{pgfscope}%
\pgfsys@transformshift{3.020494in}{1.242163in}%
\pgfsys@useobject{currentmarker}{}%
\end{pgfscope}%
\begin{pgfscope}%
\pgfsys@transformshift{3.039042in}{0.965196in}%
\pgfsys@useobject{currentmarker}{}%
\end{pgfscope}%
\begin{pgfscope}%
\pgfsys@transformshift{3.059702in}{0.703921in}%
\pgfsys@useobject{currentmarker}{}%
\end{pgfscope}%
\begin{pgfscope}%
\pgfsys@transformshift{3.078013in}{0.586854in}%
\pgfsys@useobject{currentmarker}{}%
\end{pgfscope}%
\begin{pgfscope}%
\pgfsys@transformshift{3.096090in}{0.515858in}%
\pgfsys@useobject{currentmarker}{}%
\end{pgfscope}%
\begin{pgfscope}%
\pgfsys@transformshift{3.116516in}{0.496085in}%
\pgfsys@useobject{currentmarker}{}%
\end{pgfscope}%
\begin{pgfscope}%
\pgfsys@transformshift{3.134829in}{0.561892in}%
\pgfsys@useobject{currentmarker}{}%
\end{pgfscope}%
\begin{pgfscope}%
\pgfsys@transformshift{3.152671in}{0.658652in}%
\pgfsys@useobject{currentmarker}{}%
\end{pgfscope}%
\begin{pgfscope}%
\pgfsys@transformshift{3.173331in}{0.867810in}%
\pgfsys@useobject{currentmarker}{}%
\end{pgfscope}%
\begin{pgfscope}%
\pgfsys@transformshift{3.191877in}{1.074503in}%
\pgfsys@useobject{currentmarker}{}%
\end{pgfscope}%
\begin{pgfscope}%
\pgfsys@transformshift{3.212537in}{1.300335in}%
\pgfsys@useobject{currentmarker}{}%
\end{pgfscope}%
\begin{pgfscope}%
\pgfsys@transformshift{3.229442in}{1.383717in}%
\pgfsys@useobject{currentmarker}{}%
\end{pgfscope}%
\begin{pgfscope}%
\pgfsys@transformshift{3.251979in}{1.377003in}%
\pgfsys@useobject{currentmarker}{}%
\end{pgfscope}%
\begin{pgfscope}%
\pgfsys@transformshift{3.267944in}{1.303979in}%
\pgfsys@useobject{currentmarker}{}%
\end{pgfscope}%
\begin{pgfscope}%
\pgfsys@transformshift{3.287430in}{1.046698in}%
\pgfsys@useobject{currentmarker}{}%
\end{pgfscope}%
\begin{pgfscope}%
\pgfsys@transformshift{3.308090in}{0.832417in}%
\pgfsys@useobject{currentmarker}{}%
\end{pgfscope}%
\begin{pgfscope}%
\pgfsys@transformshift{3.326169in}{0.676373in}%
\pgfsys@useobject{currentmarker}{}%
\end{pgfscope}%
\begin{pgfscope}%
\pgfsys@transformshift{3.347063in}{0.548977in}%
\pgfsys@useobject{currentmarker}{}%
\end{pgfscope}%
\begin{pgfscope}%
\pgfsys@transformshift{3.364906in}{0.498554in}%
\pgfsys@useobject{currentmarker}{}%
\end{pgfscope}%
\begin{pgfscope}%
\pgfsys@transformshift{3.384157in}{0.516718in}%
\pgfsys@useobject{currentmarker}{}%
\end{pgfscope}%
\begin{pgfscope}%
\pgfsys@transformshift{3.402703in}{0.603254in}%
\pgfsys@useobject{currentmarker}{}%
\end{pgfscope}%
\begin{pgfscope}%
\pgfsys@transformshift{3.425242in}{0.757935in}%
\pgfsys@useobject{currentmarker}{}%
\end{pgfscope}%
\begin{pgfscope}%
\pgfsys@transformshift{3.442145in}{0.934590in}%
\pgfsys@useobject{currentmarker}{}%
\end{pgfscope}%
\begin{pgfscope}%
\pgfsys@transformshift{3.463510in}{1.187746in}%
\pgfsys@useobject{currentmarker}{}%
\end{pgfscope}%
\begin{pgfscope}%
\pgfsys@transformshift{3.481587in}{1.333622in}%
\pgfsys@useobject{currentmarker}{}%
\end{pgfscope}%
\begin{pgfscope}%
\pgfsys@transformshift{3.500135in}{1.404643in}%
\pgfsys@useobject{currentmarker}{}%
\end{pgfscope}%
\begin{pgfscope}%
\pgfsys@transformshift{3.520795in}{1.375415in}%
\pgfsys@useobject{currentmarker}{}%
\end{pgfscope}%
\begin{pgfscope}%
\pgfsys@transformshift{3.538167in}{1.239238in}%
\pgfsys@useobject{currentmarker}{}%
\end{pgfscope}%
\begin{pgfscope}%
\pgfsys@transformshift{3.556245in}{1.058762in}%
\pgfsys@useobject{currentmarker}{}%
\end{pgfscope}%
\begin{pgfscope}%
\pgfsys@transformshift{3.575731in}{0.834831in}%
\pgfsys@useobject{currentmarker}{}%
\end{pgfscope}%
\begin{pgfscope}%
\pgfsys@transformshift{3.595451in}{0.697669in}%
\pgfsys@useobject{currentmarker}{}%
\end{pgfscope}%
\begin{pgfscope}%
\pgfsys@transformshift{3.615877in}{0.571126in}%
\pgfsys@useobject{currentmarker}{}%
\end{pgfscope}%
\begin{pgfscope}%
\pgfsys@transformshift{3.634425in}{0.506708in}%
\pgfsys@useobject{currentmarker}{}%
\end{pgfscope}%
\begin{pgfscope}%
\pgfsys@transformshift{3.656022in}{0.523727in}%
\pgfsys@useobject{currentmarker}{}%
\end{pgfscope}%
\begin{pgfscope}%
\pgfsys@transformshift{3.673396in}{0.599779in}%
\pgfsys@useobject{currentmarker}{}%
\end{pgfscope}%
\begin{pgfscope}%
\pgfsys@transformshift{3.692882in}{0.639963in}%
\pgfsys@useobject{currentmarker}{}%
\end{pgfscope}%
\begin{pgfscope}%
\pgfsys@transformshift{3.712604in}{0.761327in}%
\pgfsys@useobject{currentmarker}{}%
\end{pgfscope}%
\begin{pgfscope}%
\pgfsys@transformshift{3.731150in}{0.899028in}%
\pgfsys@useobject{currentmarker}{}%
\end{pgfscope}%
\begin{pgfscope}%
\pgfsys@transformshift{3.748758in}{1.104343in}%
\pgfsys@useobject{currentmarker}{}%
\end{pgfscope}%
\begin{pgfscope}%
\pgfsys@transformshift{3.768714in}{1.332115in}%
\pgfsys@useobject{currentmarker}{}%
\end{pgfscope}%
\begin{pgfscope}%
\pgfsys@transformshift{3.790314in}{1.412082in}%
\pgfsys@useobject{currentmarker}{}%
\end{pgfscope}%
\begin{pgfscope}%
\pgfsys@transformshift{3.808860in}{1.420787in}%
\pgfsys@useobject{currentmarker}{}%
\end{pgfscope}%
\begin{pgfscope}%
\pgfsys@transformshift{3.826233in}{1.411850in}%
\pgfsys@useobject{currentmarker}{}%
\end{pgfscope}%
\begin{pgfscope}%
\pgfsys@transformshift{3.845016in}{1.366909in}%
\pgfsys@useobject{currentmarker}{}%
\end{pgfscope}%
\begin{pgfscope}%
\pgfsys@transformshift{3.865205in}{1.219053in}%
\pgfsys@useobject{currentmarker}{}%
\end{pgfscope}%
\begin{pgfscope}%
\pgfsys@transformshift{3.886335in}{0.903933in}%
\pgfsys@useobject{currentmarker}{}%
\end{pgfscope}%
\begin{pgfscope}%
\pgfsys@transformshift{3.903943in}{0.718804in}%
\pgfsys@useobject{currentmarker}{}%
\end{pgfscope}%
\begin{pgfscope}%
\pgfsys@transformshift{3.922960in}{0.595242in}%
\pgfsys@useobject{currentmarker}{}%
\end{pgfscope}%
\begin{pgfscope}%
\pgfsys@transformshift{3.944323in}{0.525459in}%
\pgfsys@useobject{currentmarker}{}%
\end{pgfscope}%
\begin{pgfscope}%
\pgfsys@transformshift{3.960288in}{0.513323in}%
\pgfsys@useobject{currentmarker}{}%
\end{pgfscope}%
\begin{pgfscope}%
\pgfsys@transformshift{3.981888in}{0.586930in}%
\pgfsys@useobject{currentmarker}{}%
\end{pgfscope}%
\begin{pgfscope}%
\pgfsys@transformshift{3.998791in}{0.671829in}%
\pgfsys@useobject{currentmarker}{}%
\end{pgfscope}%
\begin{pgfscope}%
\pgfsys@transformshift{4.022034in}{0.849701in}%
\pgfsys@useobject{currentmarker}{}%
\end{pgfscope}%
\begin{pgfscope}%
\pgfsys@transformshift{4.039407in}{1.079562in}%
\pgfsys@useobject{currentmarker}{}%
\end{pgfscope}%
\begin{pgfscope}%
\pgfsys@transformshift{4.058188in}{1.281157in}%
\pgfsys@useobject{currentmarker}{}%
\end{pgfscope}%
\begin{pgfscope}%
\pgfsys@transformshift{4.075092in}{1.401411in}%
\pgfsys@useobject{currentmarker}{}%
\end{pgfscope}%
\begin{pgfscope}%
\pgfsys@transformshift{4.096926in}{1.438349in}%
\pgfsys@useobject{currentmarker}{}%
\end{pgfscope}%
\begin{pgfscope}%
\pgfsys@transformshift{4.115003in}{1.422689in}%
\pgfsys@useobject{currentmarker}{}%
\end{pgfscope}%
\begin{pgfscope}%
\pgfsys@transformshift{4.134960in}{1.321267in}%
\pgfsys@useobject{currentmarker}{}%
\end{pgfscope}%
\begin{pgfscope}%
\pgfsys@transformshift{4.154680in}{1.009043in}%
\pgfsys@useobject{currentmarker}{}%
\end{pgfscope}%
\begin{pgfscope}%
\pgfsys@transformshift{4.172522in}{1.295102in}%
\pgfsys@useobject{currentmarker}{}%
\end{pgfscope}%
\begin{pgfscope}%
\pgfsys@transformshift{4.193651in}{1.435650in}%
\pgfsys@useobject{currentmarker}{}%
\end{pgfscope}%
\begin{pgfscope}%
\pgfsys@transformshift{4.211730in}{1.419161in}%
\pgfsys@useobject{currentmarker}{}%
\end{pgfscope}%
\begin{pgfscope}%
\pgfsys@transformshift{4.229807in}{1.288330in}%
\pgfsys@useobject{currentmarker}{}%
\end{pgfscope}%
\begin{pgfscope}%
\pgfsys@transformshift{4.249762in}{0.998836in}%
\pgfsys@useobject{currentmarker}{}%
\end{pgfscope}%
\begin{pgfscope}%
\pgfsys@transformshift{4.267841in}{0.799684in}%
\pgfsys@useobject{currentmarker}{}%
\end{pgfscope}%
\begin{pgfscope}%
\pgfsys@transformshift{4.288970in}{0.624557in}%
\pgfsys@useobject{currentmarker}{}%
\end{pgfscope}%
\begin{pgfscope}%
\pgfsys@transformshift{4.307046in}{0.543725in}%
\pgfsys@useobject{currentmarker}{}%
\end{pgfscope}%
\begin{pgfscope}%
\pgfsys@transformshift{4.328412in}{0.548304in}%
\pgfsys@useobject{currentmarker}{}%
\end{pgfscope}%
\begin{pgfscope}%
\pgfsys@transformshift{4.344846in}{0.619919in}%
\pgfsys@useobject{currentmarker}{}%
\end{pgfscope}%
\begin{pgfscope}%
\pgfsys@transformshift{4.365740in}{0.748188in}%
\pgfsys@useobject{currentmarker}{}%
\end{pgfscope}%
\begin{pgfscope}%
\pgfsys@transformshift{4.384522in}{0.928130in}%
\pgfsys@useobject{currentmarker}{}%
\end{pgfscope}%
\begin{pgfscope}%
\pgfsys@transformshift{4.404008in}{1.182988in}%
\pgfsys@useobject{currentmarker}{}%
\end{pgfscope}%
\begin{pgfscope}%
\pgfsys@transformshift{4.423494in}{1.387687in}%
\pgfsys@useobject{currentmarker}{}%
\end{pgfscope}%
\begin{pgfscope}%
\pgfsys@transformshift{4.443216in}{1.458881in}%
\pgfsys@useobject{currentmarker}{}%
\end{pgfscope}%
\begin{pgfscope}%
\pgfsys@transformshift{4.462701in}{1.430785in}%
\pgfsys@useobject{currentmarker}{}%
\end{pgfscope}%
\begin{pgfscope}%
\pgfsys@transformshift{4.481718in}{1.282729in}%
\pgfsys@useobject{currentmarker}{}%
\end{pgfscope}%
\begin{pgfscope}%
\pgfsys@transformshift{4.482187in}{1.289494in}%
\pgfsys@useobject{currentmarker}{}%
\end{pgfscope}%
\begin{pgfscope}%
\pgfsys@transformshift{4.472562in}{1.393421in}%
\pgfsys@useobject{currentmarker}{}%
\end{pgfscope}%
\begin{pgfscope}%
\pgfsys@transformshift{4.453546in}{1.458848in}%
\pgfsys@useobject{currentmarker}{}%
\end{pgfscope}%
\begin{pgfscope}%
\pgfsys@transformshift{4.435467in}{1.354954in}%
\pgfsys@useobject{currentmarker}{}%
\end{pgfscope}%
\begin{pgfscope}%
\pgfsys@transformshift{4.416921in}{1.048459in}%
\pgfsys@useobject{currentmarker}{}%
\end{pgfscope}%
\begin{pgfscope}%
\pgfsys@transformshift{4.399078in}{0.772423in}%
\pgfsys@useobject{currentmarker}{}%
\end{pgfscope}%
\begin{pgfscope}%
\pgfsys@transformshift{4.377010in}{0.577365in}%
\pgfsys@useobject{currentmarker}{}%
\end{pgfscope}%
\begin{pgfscope}%
\pgfsys@transformshift{4.358462in}{0.529317in}%
\pgfsys@useobject{currentmarker}{}%
\end{pgfscope}%
\begin{pgfscope}%
\pgfsys@transformshift{4.338271in}{0.721470in}%
\pgfsys@useobject{currentmarker}{}%
\end{pgfscope}%
\begin{pgfscope}%
\pgfsys@transformshift{4.318316in}{1.019056in}%
\pgfsys@useobject{currentmarker}{}%
\end{pgfscope}%
\begin{pgfscope}%
\pgfsys@transformshift{4.302117in}{1.294636in}%
\pgfsys@useobject{currentmarker}{}%
\end{pgfscope}%
\begin{pgfscope}%
\pgfsys@transformshift{4.281223in}{1.441015in}%
\pgfsys@useobject{currentmarker}{}%
\end{pgfscope}%
\begin{pgfscope}%
\pgfsys@transformshift{4.262675in}{1.373312in}%
\pgfsys@useobject{currentmarker}{}%
\end{pgfscope}%
\begin{pgfscope}%
\pgfsys@transformshift{4.242015in}{1.043282in}%
\pgfsys@useobject{currentmarker}{}%
\end{pgfscope}%
\begin{pgfscope}%
\pgfsys@transformshift{4.226050in}{0.802340in}%
\pgfsys@useobject{currentmarker}{}%
\end{pgfscope}%
\begin{pgfscope}%
\pgfsys@transformshift{4.204687in}{0.591417in}%
\pgfsys@useobject{currentmarker}{}%
\end{pgfscope}%
\begin{pgfscope}%
\pgfsys@transformshift{4.186139in}{0.516521in}%
\pgfsys@useobject{currentmarker}{}%
\end{pgfscope}%
\begin{pgfscope}%
\pgfsys@transformshift{4.167357in}{0.579141in}%
\pgfsys@useobject{currentmarker}{}%
\end{pgfscope}%
\begin{pgfscope}%
\pgfsys@transformshift{4.146228in}{0.789725in}%
\pgfsys@useobject{currentmarker}{}%
\end{pgfscope}%
\begin{pgfscope}%
\pgfsys@transformshift{4.128151in}{1.094400in}%
\pgfsys@useobject{currentmarker}{}%
\end{pgfscope}%
\begin{pgfscope}%
\pgfsys@transformshift{4.110543in}{1.354000in}%
\pgfsys@useobject{currentmarker}{}%
\end{pgfscope}%
\begin{pgfscope}%
\pgfsys@transformshift{4.089883in}{1.365996in}%
\pgfsys@useobject{currentmarker}{}%
\end{pgfscope}%
\begin{pgfscope}%
\pgfsys@transformshift{4.072509in}{1.423079in}%
\pgfsys@useobject{currentmarker}{}%
\end{pgfscope}%
\begin{pgfscope}%
\pgfsys@transformshift{4.049972in}{1.244020in}%
\pgfsys@useobject{currentmarker}{}%
\end{pgfscope}%
\begin{pgfscope}%
\pgfsys@transformshift{4.032833in}{0.940287in}%
\pgfsys@useobject{currentmarker}{}%
\end{pgfscope}%
\begin{pgfscope}%
\pgfsys@transformshift{4.013581in}{0.698572in}%
\pgfsys@useobject{currentmarker}{}%
\end{pgfscope}%
\begin{pgfscope}%
\pgfsys@transformshift{3.992921in}{0.542117in}%
\pgfsys@useobject{currentmarker}{}%
\end{pgfscope}%
\begin{pgfscope}%
\pgfsys@transformshift{3.973670in}{0.516240in}%
\pgfsys@useobject{currentmarker}{}%
\end{pgfscope}%
\begin{pgfscope}%
\pgfsys@transformshift{3.955122in}{0.623197in}%
\pgfsys@useobject{currentmarker}{}%
\end{pgfscope}%
\begin{pgfscope}%
\pgfsys@transformshift{3.938454in}{0.809385in}%
\pgfsys@useobject{currentmarker}{}%
\end{pgfscope}%
\begin{pgfscope}%
\pgfsys@transformshift{3.917325in}{1.153822in}%
\pgfsys@useobject{currentmarker}{}%
\end{pgfscope}%
\begin{pgfscope}%
\pgfsys@transformshift{3.896900in}{1.382896in}%
\pgfsys@useobject{currentmarker}{}%
\end{pgfscope}%
\begin{pgfscope}%
\pgfsys@transformshift{3.879057in}{1.397696in}%
\pgfsys@useobject{currentmarker}{}%
\end{pgfscope}%
\begin{pgfscope}%
\pgfsys@transformshift{3.859101in}{1.202259in}%
\pgfsys@useobject{currentmarker}{}%
\end{pgfscope}%
\begin{pgfscope}%
\pgfsys@transformshift{3.838675in}{0.859706in}%
\pgfsys@useobject{currentmarker}{}%
\end{pgfscope}%
\begin{pgfscope}%
\pgfsys@transformshift{3.822242in}{0.675671in}%
\pgfsys@useobject{currentmarker}{}%
\end{pgfscope}%
\begin{pgfscope}%
\pgfsys@transformshift{3.801582in}{0.521590in}%
\pgfsys@useobject{currentmarker}{}%
\end{pgfscope}%
\begin{pgfscope}%
\pgfsys@transformshift{3.783270in}{0.497262in}%
\pgfsys@useobject{currentmarker}{}%
\end{pgfscope}%
\begin{pgfscope}%
\pgfsys@transformshift{3.763785in}{0.588309in}%
\pgfsys@useobject{currentmarker}{}%
\end{pgfscope}%
\begin{pgfscope}%
\pgfsys@transformshift{3.742654in}{0.792530in}%
\pgfsys@useobject{currentmarker}{}%
\end{pgfscope}%
\begin{pgfscope}%
\pgfsys@transformshift{3.725046in}{1.077109in}%
\pgfsys@useobject{currentmarker}{}%
\end{pgfscope}%
\begin{pgfscope}%
\pgfsys@transformshift{3.705091in}{1.351198in}%
\pgfsys@useobject{currentmarker}{}%
\end{pgfscope}%
\begin{pgfscope}%
\pgfsys@transformshift{3.686543in}{1.401449in}%
\pgfsys@useobject{currentmarker}{}%
\end{pgfscope}%
\begin{pgfscope}%
\pgfsys@transformshift{3.666352in}{1.288862in}%
\pgfsys@useobject{currentmarker}{}%
\end{pgfscope}%
\begin{pgfscope}%
\pgfsys@transformshift{3.648981in}{1.021342in}%
\pgfsys@useobject{currentmarker}{}%
\end{pgfscope}%
\begin{pgfscope}%
\pgfsys@transformshift{3.628084in}{0.722166in}%
\pgfsys@useobject{currentmarker}{}%
\end{pgfscope}%
\begin{pgfscope}%
\pgfsys@transformshift{3.607190in}{0.573657in}%
\pgfsys@useobject{currentmarker}{}%
\end{pgfscope}%
\begin{pgfscope}%
\pgfsys@transformshift{3.590053in}{1.316005in}%
\pgfsys@useobject{currentmarker}{}%
\end{pgfscope}%
\begin{pgfscope}%
\pgfsys@transformshift{3.572445in}{1.070648in}%
\pgfsys@useobject{currentmarker}{}%
\end{pgfscope}%
\begin{pgfscope}%
\pgfsys@transformshift{3.551079in}{0.760520in}%
\pgfsys@useobject{currentmarker}{}%
\end{pgfscope}%
\begin{pgfscope}%
\pgfsys@transformshift{3.534411in}{0.604472in}%
\pgfsys@useobject{currentmarker}{}%
\end{pgfscope}%
\begin{pgfscope}%
\pgfsys@transformshift{3.514220in}{0.498234in}%
\pgfsys@useobject{currentmarker}{}%
\end{pgfscope}%
\begin{pgfscope}%
\pgfsys@transformshift{3.492857in}{0.526663in}%
\pgfsys@useobject{currentmarker}{}%
\end{pgfscope}%
\begin{pgfscope}%
\pgfsys@transformshift{3.475952in}{0.590444in}%
\pgfsys@useobject{currentmarker}{}%
\end{pgfscope}%
\begin{pgfscope}%
\pgfsys@transformshift{3.457641in}{0.764417in}%
\pgfsys@useobject{currentmarker}{}%
\end{pgfscope}%
\begin{pgfscope}%
\pgfsys@transformshift{3.436981in}{1.089225in}%
\pgfsys@useobject{currentmarker}{}%
\end{pgfscope}%
\begin{pgfscope}%
\pgfsys@transformshift{3.417495in}{1.343634in}%
\pgfsys@useobject{currentmarker}{}%
\end{pgfscope}%
\begin{pgfscope}%
\pgfsys@transformshift{3.398479in}{1.383228in}%
\pgfsys@useobject{currentmarker}{}%
\end{pgfscope}%
\begin{pgfscope}%
\pgfsys@transformshift{3.378991in}{1.224553in}%
\pgfsys@useobject{currentmarker}{}%
\end{pgfscope}%
\begin{pgfscope}%
\pgfsys@transformshift{3.360211in}{0.958930in}%
\pgfsys@useobject{currentmarker}{}%
\end{pgfscope}%
\begin{pgfscope}%
\pgfsys@transformshift{3.340254in}{0.729902in}%
\pgfsys@useobject{currentmarker}{}%
\end{pgfscope}%
\begin{pgfscope}%
\pgfsys@transformshift{3.317716in}{0.566792in}%
\pgfsys@useobject{currentmarker}{}%
\end{pgfscope}%
\begin{pgfscope}%
\pgfsys@transformshift{3.301752in}{0.489105in}%
\pgfsys@useobject{currentmarker}{}%
\end{pgfscope}%
\begin{pgfscope}%
\pgfsys@transformshift{3.281795in}{0.520341in}%
\pgfsys@useobject{currentmarker}{}%
\end{pgfscope}%
\begin{pgfscope}%
\pgfsys@transformshift{3.264423in}{0.616275in}%
\pgfsys@useobject{currentmarker}{}%
\end{pgfscope}%
\begin{pgfscope}%
\pgfsys@transformshift{3.246110in}{0.726595in}%
\pgfsys@useobject{currentmarker}{}%
\end{pgfscope}%
\begin{pgfscope}%
\pgfsys@transformshift{3.226859in}{1.008955in}%
\pgfsys@useobject{currentmarker}{}%
\end{pgfscope}%
\begin{pgfscope}%
\pgfsys@transformshift{3.205496in}{1.314717in}%
\pgfsys@useobject{currentmarker}{}%
\end{pgfscope}%
\begin{pgfscope}%
\pgfsys@transformshift{3.188356in}{1.385631in}%
\pgfsys@useobject{currentmarker}{}%
\end{pgfscope}%
\begin{pgfscope}%
\pgfsys@transformshift{3.165819in}{1.290905in}%
\pgfsys@useobject{currentmarker}{}%
\end{pgfscope}%
\begin{pgfscope}%
\pgfsys@transformshift{3.148445in}{1.054410in}%
\pgfsys@useobject{currentmarker}{}%
\end{pgfscope}%
\begin{pgfscope}%
\pgfsys@transformshift{3.130837in}{0.793805in}%
\pgfsys@useobject{currentmarker}{}%
\end{pgfscope}%
\begin{pgfscope}%
\pgfsys@transformshift{3.110646in}{0.613749in}%
\pgfsys@useobject{currentmarker}{}%
\end{pgfscope}%
\begin{pgfscope}%
\pgfsys@transformshift{3.090457in}{0.498688in}%
\pgfsys@useobject{currentmarker}{}%
\end{pgfscope}%
\begin{pgfscope}%
\pgfsys@transformshift{3.070735in}{0.502792in}%
\pgfsys@useobject{currentmarker}{}%
\end{pgfscope}%
\begin{pgfscope}%
\pgfsys@transformshift{3.052893in}{0.588222in}%
\pgfsys@useobject{currentmarker}{}%
\end{pgfscope}%
\begin{pgfscope}%
\pgfsys@transformshift{3.035050in}{0.746950in}%
\pgfsys@useobject{currentmarker}{}%
\end{pgfscope}%
\begin{pgfscope}%
\pgfsys@transformshift{3.014390in}{1.010807in}%
\pgfsys@useobject{currentmarker}{}%
\end{pgfscope}%
\begin{pgfscope}%
\pgfsys@transformshift{2.993730in}{1.274367in}%
\pgfsys@useobject{currentmarker}{}%
\end{pgfscope}%
\begin{pgfscope}%
\pgfsys@transformshift{2.976122in}{1.377790in}%
\pgfsys@useobject{currentmarker}{}%
\end{pgfscope}%
\begin{pgfscope}%
\pgfsys@transformshift{2.954523in}{1.329871in}%
\pgfsys@useobject{currentmarker}{}%
\end{pgfscope}%
\begin{pgfscope}%
\pgfsys@transformshift{2.937385in}{1.211763in}%
\pgfsys@useobject{currentmarker}{}%
\end{pgfscope}%
\begin{pgfscope}%
\pgfsys@transformshift{2.918134in}{0.932664in}%
\pgfsys@useobject{currentmarker}{}%
\end{pgfscope}%
\begin{pgfscope}%
\pgfsys@transformshift{2.900526in}{0.733322in}%
\pgfsys@useobject{currentmarker}{}%
\end{pgfscope}%
\begin{pgfscope}%
\pgfsys@transformshift{2.878692in}{0.567570in}%
\pgfsys@useobject{currentmarker}{}%
\end{pgfscope}%
\begin{pgfscope}%
\pgfsys@transformshift{2.859910in}{0.494447in}%
\pgfsys@useobject{currentmarker}{}%
\end{pgfscope}%
\begin{pgfscope}%
\pgfsys@transformshift{2.840893in}{0.513596in}%
\pgfsys@useobject{currentmarker}{}%
\end{pgfscope}%
\begin{pgfscope}%
\pgfsys@transformshift{2.820704in}{0.624651in}%
\pgfsys@useobject{currentmarker}{}%
\end{pgfscope}%
\begin{pgfscope}%
\pgfsys@transformshift{2.801216in}{0.815291in}%
\pgfsys@useobject{currentmarker}{}%
\end{pgfscope}%
\begin{pgfscope}%
\pgfsys@transformshift{2.781496in}{1.001905in}%
\pgfsys@useobject{currentmarker}{}%
\end{pgfscope}%
\begin{pgfscope}%
\pgfsys@transformshift{2.763183in}{1.273987in}%
\pgfsys@useobject{currentmarker}{}%
\end{pgfscope}%
\begin{pgfscope}%
\pgfsys@transformshift{2.742994in}{1.378134in}%
\pgfsys@useobject{currentmarker}{}%
\end{pgfscope}%
\begin{pgfscope}%
\pgfsys@transformshift{2.726324in}{1.353349in}%
\pgfsys@useobject{currentmarker}{}%
\end{pgfscope}%
\begin{pgfscope}%
\pgfsys@transformshift{2.707543in}{1.212234in}%
\pgfsys@useobject{currentmarker}{}%
\end{pgfscope}%
\begin{pgfscope}%
\pgfsys@transformshift{2.685240in}{0.910251in}%
\pgfsys@useobject{currentmarker}{}%
\end{pgfscope}%
\begin{pgfscope}%
\pgfsys@transformshift{2.667161in}{0.706559in}%
\pgfsys@useobject{currentmarker}{}%
\end{pgfscope}%
\begin{pgfscope}%
\pgfsys@transformshift{2.651433in}{0.596797in}%
\pgfsys@useobject{currentmarker}{}%
\end{pgfscope}%
\begin{pgfscope}%
\pgfsys@transformshift{2.629833in}{0.494713in}%
\pgfsys@useobject{currentmarker}{}%
\end{pgfscope}%
\begin{pgfscope}%
\pgfsys@transformshift{2.610816in}{0.500107in}%
\pgfsys@useobject{currentmarker}{}%
\end{pgfscope}%
\begin{pgfscope}%
\pgfsys@transformshift{2.592268in}{0.576475in}%
\pgfsys@useobject{currentmarker}{}%
\end{pgfscope}%
\begin{pgfscope}%
\pgfsys@transformshift{2.570202in}{0.664578in}%
\pgfsys@useobject{currentmarker}{}%
\end{pgfscope}%
\begin{pgfscope}%
\pgfsys@transformshift{2.551888in}{0.874222in}%
\pgfsys@useobject{currentmarker}{}%
\end{pgfscope}%
\begin{pgfscope}%
\pgfsys@transformshift{2.532872in}{1.155069in}%
\pgfsys@useobject{currentmarker}{}%
\end{pgfscope}%
\begin{pgfscope}%
\pgfsys@transformshift{2.514089in}{1.343749in}%
\pgfsys@useobject{currentmarker}{}%
\end{pgfscope}%
\begin{pgfscope}%
\pgfsys@transformshift{2.495778in}{1.372126in}%
\pgfsys@useobject{currentmarker}{}%
\end{pgfscope}%
\begin{pgfscope}%
\pgfsys@transformshift{2.473944in}{1.230092in}%
\pgfsys@useobject{currentmarker}{}%
\end{pgfscope}%
\begin{pgfscope}%
\pgfsys@transformshift{2.454927in}{1.031294in}%
\pgfsys@useobject{currentmarker}{}%
\end{pgfscope}%
\begin{pgfscope}%
\pgfsys@transformshift{2.436381in}{0.799299in}%
\pgfsys@useobject{currentmarker}{}%
\end{pgfscope}%
\begin{pgfscope}%
\pgfsys@transformshift{2.415250in}{0.669861in}%
\pgfsys@useobject{currentmarker}{}%
\end{pgfscope}%
\begin{pgfscope}%
\pgfsys@transformshift{2.399522in}{0.568572in}%
\pgfsys@useobject{currentmarker}{}%
\end{pgfscope}%
\begin{pgfscope}%
\pgfsys@transformshift{2.374870in}{1.236558in}%
\pgfsys@useobject{currentmarker}{}%
\end{pgfscope}%
\begin{pgfscope}%
\pgfsys@transformshift{2.358436in}{0.988231in}%
\pgfsys@useobject{currentmarker}{}%
\end{pgfscope}%
\begin{pgfscope}%
\pgfsys@transformshift{2.340594in}{0.736489in}%
\pgfsys@useobject{currentmarker}{}%
\end{pgfscope}%
\begin{pgfscope}%
\pgfsys@transformshift{2.319229in}{0.571728in}%
\pgfsys@useobject{currentmarker}{}%
\end{pgfscope}%
\begin{pgfscope}%
\pgfsys@transformshift{2.303264in}{0.496112in}%
\pgfsys@useobject{currentmarker}{}%
\end{pgfscope}%
\begin{pgfscope}%
\pgfsys@transformshift{2.284718in}{0.498293in}%
\pgfsys@useobject{currentmarker}{}%
\end{pgfscope}%
\begin{pgfscope}%
\pgfsys@transformshift{2.263824in}{0.586194in}%
\pgfsys@useobject{currentmarker}{}%
\end{pgfscope}%
\begin{pgfscope}%
\pgfsys@transformshift{2.244102in}{0.750633in}%
\pgfsys@useobject{currentmarker}{}%
\end{pgfscope}%
\begin{pgfscope}%
\pgfsys@transformshift{2.226259in}{1.001011in}%
\pgfsys@useobject{currentmarker}{}%
\end{pgfscope}%
\begin{pgfscope}%
\pgfsys@transformshift{2.203251in}{1.322979in}%
\pgfsys@useobject{currentmarker}{}%
\end{pgfscope}%
\begin{pgfscope}%
\pgfsys@transformshift{2.188226in}{1.382676in}%
\pgfsys@useobject{currentmarker}{}%
\end{pgfscope}%
\begin{pgfscope}%
\pgfsys@transformshift{2.169680in}{1.325829in}%
\pgfsys@useobject{currentmarker}{}%
\end{pgfscope}%
\begin{pgfscope}%
\pgfsys@transformshift{2.148314in}{1.082581in}%
\pgfsys@useobject{currentmarker}{}%
\end{pgfscope}%
\begin{pgfscope}%
\pgfsys@transformshift{2.132350in}{0.883361in}%
\pgfsys@useobject{currentmarker}{}%
\end{pgfscope}%
\begin{pgfscope}%
\pgfsys@transformshift{2.110046in}{0.669930in}%
\pgfsys@useobject{currentmarker}{}%
\end{pgfscope}%
\begin{pgfscope}%
\pgfsys@transformshift{2.087978in}{0.532597in}%
\pgfsys@useobject{currentmarker}{}%
\end{pgfscope}%
\begin{pgfscope}%
\pgfsys@transformshift{2.070370in}{0.492242in}%
\pgfsys@useobject{currentmarker}{}%
\end{pgfscope}%
\begin{pgfscope}%
\pgfsys@transformshift{2.051589in}{0.524237in}%
\pgfsys@useobject{currentmarker}{}%
\end{pgfscope}%
\begin{pgfscope}%
\pgfsys@transformshift{2.033042in}{0.635762in}%
\pgfsys@useobject{currentmarker}{}%
\end{pgfscope}%
\begin{pgfscope}%
\pgfsys@transformshift{2.014494in}{0.727016in}%
\pgfsys@useobject{currentmarker}{}%
\end{pgfscope}%
\begin{pgfscope}%
\pgfsys@transformshift{1.995713in}{0.994053in}%
\pgfsys@useobject{currentmarker}{}%
\end{pgfscope}%
\begin{pgfscope}%
\pgfsys@transformshift{1.976931in}{1.262387in}%
\pgfsys@useobject{currentmarker}{}%
\end{pgfscope}%
\begin{pgfscope}%
\pgfsys@transformshift{1.955566in}{1.386512in}%
\pgfsys@useobject{currentmarker}{}%
\end{pgfscope}%
\begin{pgfscope}%
\pgfsys@transformshift{1.938898in}{1.367742in}%
\pgfsys@useobject{currentmarker}{}%
\end{pgfscope}%
\begin{pgfscope}%
\pgfsys@transformshift{1.917063in}{1.158525in}%
\pgfsys@useobject{currentmarker}{}%
\end{pgfscope}%
\begin{pgfscope}%
\pgfsys@transformshift{1.899221in}{0.927870in}%
\pgfsys@useobject{currentmarker}{}%
\end{pgfscope}%
\begin{pgfscope}%
\pgfsys@transformshift{1.878561in}{0.700018in}%
\pgfsys@useobject{currentmarker}{}%
\end{pgfscope}%
\begin{pgfscope}%
\pgfsys@transformshift{1.860015in}{0.568894in}%
\pgfsys@useobject{currentmarker}{}%
\end{pgfscope}%
\begin{pgfscope}%
\pgfsys@transformshift{1.841233in}{0.494748in}%
\pgfsys@useobject{currentmarker}{}%
\end{pgfscope}%
\begin{pgfscope}%
\pgfsys@transformshift{1.822216in}{0.517862in}%
\pgfsys@useobject{currentmarker}{}%
\end{pgfscope}%
\begin{pgfscope}%
\pgfsys@transformshift{1.803434in}{0.602778in}%
\pgfsys@useobject{currentmarker}{}%
\end{pgfscope}%
\begin{pgfscope}%
\pgfsys@transformshift{1.784417in}{0.734998in}%
\pgfsys@useobject{currentmarker}{}%
\end{pgfscope}%
\begin{pgfscope}%
\pgfsys@transformshift{1.763288in}{0.935470in}%
\pgfsys@useobject{currentmarker}{}%
\end{pgfscope}%
\begin{pgfscope}%
\pgfsys@transformshift{1.745211in}{1.200177in}%
\pgfsys@useobject{currentmarker}{}%
\end{pgfscope}%
\begin{pgfscope}%
\pgfsys@transformshift{1.726429in}{1.356762in}%
\pgfsys@useobject{currentmarker}{}%
\end{pgfscope}%
\begin{pgfscope}%
\pgfsys@transformshift{1.707881in}{1.393628in}%
\pgfsys@useobject{currentmarker}{}%
\end{pgfscope}%
\begin{pgfscope}%
\pgfsys@transformshift{1.689101in}{1.353597in}%
\pgfsys@useobject{currentmarker}{}%
\end{pgfscope}%
\begin{pgfscope}%
\pgfsys@transformshift{1.667267in}{1.145331in}%
\pgfsys@useobject{currentmarker}{}%
\end{pgfscope}%
\begin{pgfscope}%
\pgfsys@transformshift{1.648719in}{0.926405in}%
\pgfsys@useobject{currentmarker}{}%
\end{pgfscope}%
\begin{pgfscope}%
\pgfsys@transformshift{1.630642in}{0.738030in}%
\pgfsys@useobject{currentmarker}{}%
\end{pgfscope}%
\begin{pgfscope}%
\pgfsys@transformshift{1.609748in}{0.596479in}%
\pgfsys@useobject{currentmarker}{}%
\end{pgfscope}%
\begin{pgfscope}%
\pgfsys@transformshift{1.592139in}{0.514388in}%
\pgfsys@useobject{currentmarker}{}%
\end{pgfscope}%
\begin{pgfscope}%
\pgfsys@transformshift{1.571948in}{0.511569in}%
\pgfsys@useobject{currentmarker}{}%
\end{pgfscope}%
\begin{pgfscope}%
\pgfsys@transformshift{1.553401in}{0.581411in}%
\pgfsys@useobject{currentmarker}{}%
\end{pgfscope}%
\begin{pgfscope}%
\pgfsys@transformshift{1.534620in}{0.685141in}%
\pgfsys@useobject{currentmarker}{}%
\end{pgfscope}%
\begin{pgfscope}%
\pgfsys@transformshift{1.514664in}{0.809731in}%
\pgfsys@useobject{currentmarker}{}%
\end{pgfscope}%
\begin{pgfscope}%
\pgfsys@transformshift{1.494473in}{1.013355in}%
\pgfsys@useobject{currentmarker}{}%
\end{pgfscope}%
\begin{pgfscope}%
\pgfsys@transformshift{1.475927in}{1.276994in}%
\pgfsys@useobject{currentmarker}{}%
\end{pgfscope}%
\begin{pgfscope}%
\pgfsys@transformshift{1.457379in}{1.390296in}%
\pgfsys@useobject{currentmarker}{}%
\end{pgfscope}%
\begin{pgfscope}%
\pgfsys@transformshift{1.436485in}{1.400602in}%
\pgfsys@useobject{currentmarker}{}%
\end{pgfscope}%
\begin{pgfscope}%
\pgfsys@transformshift{1.415121in}{1.279685in}%
\pgfsys@useobject{currentmarker}{}%
\end{pgfscope}%
\begin{pgfscope}%
\pgfsys@transformshift{1.398686in}{1.073930in}%
\pgfsys@useobject{currentmarker}{}%
\end{pgfscope}%
\begin{pgfscope}%
\pgfsys@transformshift{1.378497in}{0.836464in}%
\pgfsys@useobject{currentmarker}{}%
\end{pgfscope}%
\begin{pgfscope}%
\pgfsys@transformshift{1.360183in}{0.756606in}%
\pgfsys@useobject{currentmarker}{}%
\end{pgfscope}%
\begin{pgfscope}%
\pgfsys@transformshift{1.342812in}{0.658633in}%
\pgfsys@useobject{currentmarker}{}%
\end{pgfscope}%
\begin{pgfscope}%
\pgfsys@transformshift{1.321212in}{0.549602in}%
\pgfsys@useobject{currentmarker}{}%
\end{pgfscope}%
\begin{pgfscope}%
\pgfsys@transformshift{1.302899in}{0.503197in}%
\pgfsys@useobject{currentmarker}{}%
\end{pgfscope}%
\begin{pgfscope}%
\pgfsys@transformshift{1.284353in}{0.524600in}%
\pgfsys@useobject{currentmarker}{}%
\end{pgfscope}%
\begin{pgfscope}%
\pgfsys@transformshift{1.263458in}{0.625308in}%
\pgfsys@useobject{currentmarker}{}%
\end{pgfscope}%
\begin{pgfscope}%
\pgfsys@transformshift{1.245145in}{0.759907in}%
\pgfsys@useobject{currentmarker}{}%
\end{pgfscope}%
\begin{pgfscope}%
\pgfsys@transformshift{1.227302in}{0.904694in}%
\pgfsys@useobject{currentmarker}{}%
\end{pgfscope}%
\begin{pgfscope}%
\pgfsys@transformshift{1.208757in}{1.158960in}%
\pgfsys@useobject{currentmarker}{}%
\end{pgfscope}%
\begin{pgfscope}%
\pgfsys@transformshift{1.186451in}{1.367834in}%
\pgfsys@useobject{currentmarker}{}%
\end{pgfscope}%
\begin{pgfscope}%
\pgfsys@transformshift{1.167671in}{1.419039in}%
\pgfsys@useobject{currentmarker}{}%
\end{pgfscope}%
\begin{pgfscope}%
\pgfsys@transformshift{1.148184in}{1.411320in}%
\pgfsys@useobject{currentmarker}{}%
\end{pgfscope}%
\begin{pgfscope}%
\pgfsys@transformshift{1.128698in}{1.315024in}%
\pgfsys@useobject{currentmarker}{}%
\end{pgfscope}%
\begin{pgfscope}%
\pgfsys@transformshift{1.109916in}{1.416780in}%
\pgfsys@useobject{currentmarker}{}%
\end{pgfscope}%
\begin{pgfscope}%
\pgfsys@transformshift{1.088318in}{1.284505in}%
\pgfsys@useobject{currentmarker}{}%
\end{pgfscope}%
\begin{pgfscope}%
\pgfsys@transformshift{1.072118in}{1.098368in}%
\pgfsys@useobject{currentmarker}{}%
\end{pgfscope}%
\begin{pgfscope}%
\pgfsys@transformshift{1.054042in}{0.882428in}%
\pgfsys@useobject{currentmarker}{}%
\end{pgfscope}%
\begin{pgfscope}%
\pgfsys@transformshift{1.032207in}{0.688972in}%
\pgfsys@useobject{currentmarker}{}%
\end{pgfscope}%
\begin{pgfscope}%
\pgfsys@transformshift{1.013894in}{0.615129in}%
\pgfsys@useobject{currentmarker}{}%
\end{pgfscope}%
\begin{pgfscope}%
\pgfsys@transformshift{0.995348in}{0.528751in}%
\pgfsys@useobject{currentmarker}{}%
\end{pgfscope}%
\begin{pgfscope}%
\pgfsys@transformshift{0.976800in}{0.526563in}%
\pgfsys@useobject{currentmarker}{}%
\end{pgfscope}%
\begin{pgfscope}%
\pgfsys@transformshift{0.954497in}{0.621800in}%
\pgfsys@useobject{currentmarker}{}%
\end{pgfscope}%
\begin{pgfscope}%
\pgfsys@transformshift{0.935715in}{0.772592in}%
\pgfsys@useobject{currentmarker}{}%
\end{pgfscope}%
\begin{pgfscope}%
\pgfsys@transformshift{0.917169in}{0.985316in}%
\pgfsys@useobject{currentmarker}{}%
\end{pgfscope}%
\begin{pgfscope}%
\pgfsys@transformshift{0.899090in}{1.211008in}%
\pgfsys@useobject{currentmarker}{}%
\end{pgfscope}%
\begin{pgfscope}%
\pgfsys@transformshift{0.881013in}{1.378343in}%
\pgfsys@useobject{currentmarker}{}%
\end{pgfscope}%
\begin{pgfscope}%
\pgfsys@transformshift{0.861527in}{1.437482in}%
\pgfsys@useobject{currentmarker}{}%
\end{pgfscope}%
\begin{pgfscope}%
\pgfsys@transformshift{0.839928in}{1.410806in}%
\pgfsys@useobject{currentmarker}{}%
\end{pgfscope}%
\begin{pgfscope}%
\pgfsys@transformshift{0.822085in}{1.327897in}%
\pgfsys@useobject{currentmarker}{}%
\end{pgfscope}%
\begin{pgfscope}%
\pgfsys@transformshift{0.803537in}{1.091951in}%
\pgfsys@useobject{currentmarker}{}%
\end{pgfscope}%
\begin{pgfscope}%
\pgfsys@transformshift{0.785695in}{0.900825in}%
\pgfsys@useobject{currentmarker}{}%
\end{pgfscope}%
\begin{pgfscope}%
\pgfsys@transformshift{0.763626in}{0.723205in}%
\pgfsys@useobject{currentmarker}{}%
\end{pgfscope}%
\begin{pgfscope}%
\pgfsys@transformshift{0.745080in}{0.623240in}%
\pgfsys@useobject{currentmarker}{}%
\end{pgfscope}%
\begin{pgfscope}%
\pgfsys@transformshift{0.726767in}{0.582461in}%
\pgfsys@useobject{currentmarker}{}%
\end{pgfscope}%
\begin{pgfscope}%
\pgfsys@transformshift{0.709395in}{0.533517in}%
\pgfsys@useobject{currentmarker}{}%
\end{pgfscope}%
\begin{pgfscope}%
\pgfsys@transformshift{0.688030in}{0.546232in}%
\pgfsys@useobject{currentmarker}{}%
\end{pgfscope}%
\begin{pgfscope}%
\pgfsys@transformshift{0.664553in}{0.677066in}%
\pgfsys@useobject{currentmarker}{}%
\end{pgfscope}%
\begin{pgfscope}%
\pgfsys@transformshift{0.650231in}{0.801643in}%
\pgfsys@useobject{currentmarker}{}%
\end{pgfscope}%
\begin{pgfscope}%
\pgfsys@transformshift{0.649059in}{0.819301in}%
\pgfsys@useobject{currentmarker}{}%
\end{pgfscope}%
\begin{pgfscope}%
\pgfsys@transformshift{0.653754in}{0.740687in}%
\pgfsys@useobject{currentmarker}{}%
\end{pgfscope}%
\begin{pgfscope}%
\pgfsys@transformshift{0.676292in}{0.560222in}%
\pgfsys@useobject{currentmarker}{}%
\end{pgfscope}%
\begin{pgfscope}%
\pgfsys@transformshift{0.694368in}{0.543136in}%
\pgfsys@useobject{currentmarker}{}%
\end{pgfscope}%
\begin{pgfscope}%
\pgfsys@transformshift{0.712213in}{0.660911in}%
\pgfsys@useobject{currentmarker}{}%
\end{pgfscope}%
\begin{pgfscope}%
\pgfsys@transformshift{0.733342in}{0.895922in}%
\pgfsys@useobject{currentmarker}{}%
\end{pgfscope}%
\begin{pgfscope}%
\pgfsys@transformshift{0.751888in}{1.198284in}%
\pgfsys@useobject{currentmarker}{}%
\end{pgfscope}%
\begin{pgfscope}%
\pgfsys@transformshift{0.772548in}{1.427036in}%
\pgfsys@useobject{currentmarker}{}%
\end{pgfscope}%
\begin{pgfscope}%
\pgfsys@transformshift{0.791330in}{1.419582in}%
\pgfsys@useobject{currentmarker}{}%
\end{pgfscope}%
\begin{pgfscope}%
\pgfsys@transformshift{0.809172in}{1.231821in}%
\pgfsys@useobject{currentmarker}{}%
\end{pgfscope}%
\begin{pgfscope}%
\pgfsys@transformshift{0.830069in}{0.871814in}%
\pgfsys@useobject{currentmarker}{}%
\end{pgfscope}%
\begin{pgfscope}%
\pgfsys@transformshift{0.830303in}{0.726585in}%
\pgfsys@useobject{currentmarker}{}%
\end{pgfscope}%
\begin{pgfscope}%
\pgfsys@transformshift{0.846502in}{0.668329in}%
\pgfsys@useobject{currentmarker}{}%
\end{pgfscope}%
\begin{pgfscope}%
\pgfsys@transformshift{0.866457in}{0.531062in}%
\pgfsys@useobject{currentmarker}{}%
\end{pgfscope}%
\begin{pgfscope}%
\pgfsys@transformshift{0.887822in}{0.565337in}%
\pgfsys@useobject{currentmarker}{}%
\end{pgfscope}%
\begin{pgfscope}%
\pgfsys@transformshift{0.906134in}{0.710802in}%
\pgfsys@useobject{currentmarker}{}%
\end{pgfscope}%
\begin{pgfscope}%
\pgfsys@transformshift{0.923976in}{0.929802in}%
\pgfsys@useobject{currentmarker}{}%
\end{pgfscope}%
\begin{pgfscope}%
\pgfsys@transformshift{0.944636in}{1.287160in}%
\pgfsys@useobject{currentmarker}{}%
\end{pgfscope}%
\begin{pgfscope}%
\pgfsys@transformshift{0.962949in}{1.422448in}%
\pgfsys@useobject{currentmarker}{}%
\end{pgfscope}%
\begin{pgfscope}%
\pgfsys@transformshift{0.984313in}{1.348420in}%
\pgfsys@useobject{currentmarker}{}%
\end{pgfscope}%
\begin{pgfscope}%
\pgfsys@transformshift{1.004504in}{1.004104in}%
\pgfsys@useobject{currentmarker}{}%
\end{pgfscope}%
\begin{pgfscope}%
\pgfsys@transformshift{1.019998in}{0.761994in}%
\pgfsys@useobject{currentmarker}{}%
\end{pgfscope}%
\begin{pgfscope}%
\pgfsys@transformshift{1.043006in}{0.554993in}%
\pgfsys@useobject{currentmarker}{}%
\end{pgfscope}%
\begin{pgfscope}%
\pgfsys@transformshift{1.059440in}{0.507977in}%
\pgfsys@useobject{currentmarker}{}%
\end{pgfscope}%
\begin{pgfscope}%
\pgfsys@transformshift{1.079397in}{0.606935in}%
\pgfsys@useobject{currentmarker}{}%
\end{pgfscope}%
\begin{pgfscope}%
\pgfsys@transformshift{1.095361in}{0.752829in}%
\pgfsys@useobject{currentmarker}{}%
\end{pgfscope}%
\begin{pgfscope}%
\pgfsys@transformshift{1.117430in}{1.071898in}%
\pgfsys@useobject{currentmarker}{}%
\end{pgfscope}%
\begin{pgfscope}%
\pgfsys@transformshift{1.136916in}{1.313548in}%
\pgfsys@useobject{currentmarker}{}%
\end{pgfscope}%
\begin{pgfscope}%
\pgfsys@transformshift{1.155698in}{1.413003in}%
\pgfsys@useobject{currentmarker}{}%
\end{pgfscope}%
\begin{pgfscope}%
\pgfsys@transformshift{1.172366in}{1.338529in}%
\pgfsys@useobject{currentmarker}{}%
\end{pgfscope}%
\begin{pgfscope}%
\pgfsys@transformshift{1.194669in}{0.986750in}%
\pgfsys@useobject{currentmarker}{}%
\end{pgfscope}%
\begin{pgfscope}%
\pgfsys@transformshift{1.214389in}{0.709168in}%
\pgfsys@useobject{currentmarker}{}%
\end{pgfscope}%
\begin{pgfscope}%
\pgfsys@transformshift{1.233641in}{0.557932in}%
\pgfsys@useobject{currentmarker}{}%
\end{pgfscope}%
\begin{pgfscope}%
\pgfsys@transformshift{1.253832in}{0.494558in}%
\pgfsys@useobject{currentmarker}{}%
\end{pgfscope}%
\begin{pgfscope}%
\pgfsys@transformshift{1.270031in}{0.554552in}%
\pgfsys@useobject{currentmarker}{}%
\end{pgfscope}%
\begin{pgfscope}%
\pgfsys@transformshift{1.290691in}{0.694541in}%
\pgfsys@useobject{currentmarker}{}%
\end{pgfscope}%
\begin{pgfscope}%
\pgfsys@transformshift{1.311585in}{0.940967in}%
\pgfsys@useobject{currentmarker}{}%
\end{pgfscope}%
\begin{pgfscope}%
\pgfsys@transformshift{1.328724in}{1.259907in}%
\pgfsys@useobject{currentmarker}{}%
\end{pgfscope}%
\begin{pgfscope}%
\pgfsys@transformshift{1.347272in}{1.397701in}%
\pgfsys@useobject{currentmarker}{}%
\end{pgfscope}%
\begin{pgfscope}%
\pgfsys@transformshift{1.368167in}{1.330744in}%
\pgfsys@useobject{currentmarker}{}%
\end{pgfscope}%
\begin{pgfscope}%
\pgfsys@transformshift{1.385538in}{1.366601in}%
\pgfsys@useobject{currentmarker}{}%
\end{pgfscope}%
\begin{pgfscope}%
\pgfsys@transformshift{1.405260in}{1.279011in}%
\pgfsys@useobject{currentmarker}{}%
\end{pgfscope}%
\begin{pgfscope}%
\pgfsys@transformshift{1.424043in}{1.039383in}%
\pgfsys@useobject{currentmarker}{}%
\end{pgfscope}%
\begin{pgfscope}%
\pgfsys@transformshift{1.443059in}{0.738232in}%
\pgfsys@useobject{currentmarker}{}%
\end{pgfscope}%
\begin{pgfscope}%
\pgfsys@transformshift{1.461840in}{0.567652in}%
\pgfsys@useobject{currentmarker}{}%
\end{pgfscope}%
\begin{pgfscope}%
\pgfsys@transformshift{1.482031in}{0.490294in}%
\pgfsys@useobject{currentmarker}{}%
\end{pgfscope}%
\begin{pgfscope}%
\pgfsys@transformshift{1.501987in}{0.551134in}%
\pgfsys@useobject{currentmarker}{}%
\end{pgfscope}%
\begin{pgfscope}%
\pgfsys@transformshift{1.522647in}{0.701305in}%
\pgfsys@useobject{currentmarker}{}%
\end{pgfscope}%
\begin{pgfscope}%
\pgfsys@transformshift{1.541898in}{0.946330in}%
\pgfsys@useobject{currentmarker}{}%
\end{pgfscope}%
\begin{pgfscope}%
\pgfsys@transformshift{1.564436in}{1.261271in}%
\pgfsys@useobject{currentmarker}{}%
\end{pgfscope}%
\begin{pgfscope}%
\pgfsys@transformshift{1.579695in}{1.382344in}%
\pgfsys@useobject{currentmarker}{}%
\end{pgfscope}%
\begin{pgfscope}%
\pgfsys@transformshift{1.598712in}{1.353357in}%
\pgfsys@useobject{currentmarker}{}%
\end{pgfscope}%
\begin{pgfscope}%
\pgfsys@transformshift{1.620546in}{1.063196in}%
\pgfsys@useobject{currentmarker}{}%
\end{pgfscope}%
\begin{pgfscope}%
\pgfsys@transformshift{1.635806in}{0.813286in}%
\pgfsys@useobject{currentmarker}{}%
\end{pgfscope}%
\begin{pgfscope}%
\pgfsys@transformshift{1.658109in}{0.587776in}%
\pgfsys@useobject{currentmarker}{}%
\end{pgfscope}%
\begin{pgfscope}%
\pgfsys@transformshift{1.674308in}{0.505262in}%
\pgfsys@useobject{currentmarker}{}%
\end{pgfscope}%
\begin{pgfscope}%
\pgfsys@transformshift{1.693091in}{0.485940in}%
\pgfsys@useobject{currentmarker}{}%
\end{pgfscope}%
\begin{pgfscope}%
\pgfsys@transformshift{1.711873in}{0.559550in}%
\pgfsys@useobject{currentmarker}{}%
\end{pgfscope}%
\begin{pgfscope}%
\pgfsys@transformshift{1.730655in}{0.647915in}%
\pgfsys@useobject{currentmarker}{}%
\end{pgfscope}%
\begin{pgfscope}%
\pgfsys@transformshift{1.749436in}{0.867861in}%
\pgfsys@useobject{currentmarker}{}%
\end{pgfscope}%
\begin{pgfscope}%
\pgfsys@transformshift{1.772210in}{1.159173in}%
\pgfsys@useobject{currentmarker}{}%
\end{pgfscope}%
\begin{pgfscope}%
\pgfsys@transformshift{1.790755in}{1.362490in}%
\pgfsys@useobject{currentmarker}{}%
\end{pgfscope}%
\begin{pgfscope}%
\pgfsys@transformshift{1.810243in}{1.371423in}%
\pgfsys@useobject{currentmarker}{}%
\end{pgfscope}%
\begin{pgfscope}%
\pgfsys@transformshift{1.828320in}{1.222533in}%
\pgfsys@useobject{currentmarker}{}%
\end{pgfscope}%
\begin{pgfscope}%
\pgfsys@transformshift{1.848040in}{0.894585in}%
\pgfsys@useobject{currentmarker}{}%
\end{pgfscope}%
\begin{pgfscope}%
\pgfsys@transformshift{1.867291in}{0.659647in}%
\pgfsys@useobject{currentmarker}{}%
\end{pgfscope}%
\begin{pgfscope}%
\pgfsys@transformshift{1.885136in}{0.522541in}%
\pgfsys@useobject{currentmarker}{}%
\end{pgfscope}%
\begin{pgfscope}%
\pgfsys@transformshift{1.905090in}{0.485458in}%
\pgfsys@useobject{currentmarker}{}%
\end{pgfscope}%
\begin{pgfscope}%
\pgfsys@transformshift{1.923873in}{0.531576in}%
\pgfsys@useobject{currentmarker}{}%
\end{pgfscope}%
\begin{pgfscope}%
\pgfsys@transformshift{1.944298in}{0.639225in}%
\pgfsys@useobject{currentmarker}{}%
\end{pgfscope}%
\begin{pgfscope}%
\pgfsys@transformshift{1.963315in}{0.832307in}%
\pgfsys@useobject{currentmarker}{}%
\end{pgfscope}%
\begin{pgfscope}%
\pgfsys@transformshift{1.981861in}{1.113858in}%
\pgfsys@useobject{currentmarker}{}%
\end{pgfscope}%
\begin{pgfscope}%
\pgfsys@transformshift{2.001112in}{1.335158in}%
\pgfsys@useobject{currentmarker}{}%
\end{pgfscope}%
\begin{pgfscope}%
\pgfsys@transformshift{2.019660in}{1.379521in}%
\pgfsys@useobject{currentmarker}{}%
\end{pgfscope}%
\begin{pgfscope}%
\pgfsys@transformshift{2.041728in}{1.243832in}%
\pgfsys@useobject{currentmarker}{}%
\end{pgfscope}%
\begin{pgfscope}%
\pgfsys@transformshift{2.057693in}{1.010872in}%
\pgfsys@useobject{currentmarker}{}%
\end{pgfscope}%
\begin{pgfscope}%
\pgfsys@transformshift{2.078119in}{0.707021in}%
\pgfsys@useobject{currentmarker}{}%
\end{pgfscope}%
\begin{pgfscope}%
\pgfsys@transformshift{2.097839in}{0.569059in}%
\pgfsys@useobject{currentmarker}{}%
\end{pgfscope}%
\begin{pgfscope}%
\pgfsys@transformshift{2.116387in}{0.490843in}%
\pgfsys@useobject{currentmarker}{}%
\end{pgfscope}%
\begin{pgfscope}%
\pgfsys@transformshift{2.136107in}{0.496181in}%
\pgfsys@useobject{currentmarker}{}%
\end{pgfscope}%
\begin{pgfscope}%
\pgfsys@transformshift{2.157001in}{0.595926in}%
\pgfsys@useobject{currentmarker}{}%
\end{pgfscope}%
\begin{pgfscope}%
\pgfsys@transformshift{2.175549in}{0.714851in}%
\pgfsys@useobject{currentmarker}{}%
\end{pgfscope}%
\begin{pgfscope}%
\pgfsys@transformshift{2.193860in}{0.927153in}%
\pgfsys@useobject{currentmarker}{}%
\end{pgfscope}%
\begin{pgfscope}%
\pgfsys@transformshift{2.212408in}{1.208555in}%
\pgfsys@useobject{currentmarker}{}%
\end{pgfscope}%
\begin{pgfscope}%
\pgfsys@transformshift{2.230954in}{1.359049in}%
\pgfsys@useobject{currentmarker}{}%
\end{pgfscope}%
\begin{pgfscope}%
\pgfsys@transformshift{2.251614in}{1.363113in}%
\pgfsys@useobject{currentmarker}{}%
\end{pgfscope}%
\begin{pgfscope}%
\pgfsys@transformshift{2.271571in}{1.177983in}%
\pgfsys@useobject{currentmarker}{}%
\end{pgfscope}%
\begin{pgfscope}%
\pgfsys@transformshift{2.290116in}{0.881565in}%
\pgfsys@useobject{currentmarker}{}%
\end{pgfscope}%
\begin{pgfscope}%
\pgfsys@transformshift{2.311013in}{0.660246in}%
\pgfsys@useobject{currentmarker}{}%
\end{pgfscope}%
\begin{pgfscope}%
\pgfsys@transformshift{2.329793in}{0.552341in}%
\pgfsys@useobject{currentmarker}{}%
\end{pgfscope}%
\begin{pgfscope}%
\pgfsys@transformshift{2.346227in}{0.488070in}%
\pgfsys@useobject{currentmarker}{}%
\end{pgfscope}%
\begin{pgfscope}%
\pgfsys@transformshift{2.367827in}{0.664211in}%
\pgfsys@useobject{currentmarker}{}%
\end{pgfscope}%
\begin{pgfscope}%
\pgfsys@transformshift{2.385435in}{0.545772in}%
\pgfsys@useobject{currentmarker}{}%
\end{pgfscope}%
\begin{pgfscope}%
\pgfsys@transformshift{2.385200in}{0.499143in}%
\pgfsys@useobject{currentmarker}{}%
\end{pgfscope}%
\begin{pgfscope}%
\pgfsys@transformshift{2.407034in}{0.481991in}%
\pgfsys@useobject{currentmarker}{}%
\end{pgfscope}%
\begin{pgfscope}%
\pgfsys@transformshift{2.425111in}{0.530060in}%
\pgfsys@useobject{currentmarker}{}%
\end{pgfscope}%
\begin{pgfscope}%
\pgfsys@transformshift{2.442016in}{0.636450in}%
\pgfsys@useobject{currentmarker}{}%
\end{pgfscope}%
\begin{pgfscope}%
\pgfsys@transformshift{2.460327in}{0.789171in}%
\pgfsys@useobject{currentmarker}{}%
\end{pgfscope}%
\begin{pgfscope}%
\pgfsys@transformshift{2.482162in}{1.071291in}%
\pgfsys@useobject{currentmarker}{}%
\end{pgfscope}%
\begin{pgfscope}%
\pgfsys@transformshift{2.502116in}{1.323151in}%
\pgfsys@useobject{currentmarker}{}%
\end{pgfscope}%
\begin{pgfscope}%
\pgfsys@transformshift{2.520195in}{1.380530in}%
\pgfsys@useobject{currentmarker}{}%
\end{pgfscope}%
\begin{pgfscope}%
\pgfsys@transformshift{2.538507in}{1.296227in}%
\pgfsys@useobject{currentmarker}{}%
\end{pgfscope}%
\begin{pgfscope}%
\pgfsys@transformshift{2.558463in}{0.997528in}%
\pgfsys@useobject{currentmarker}{}%
\end{pgfscope}%
\begin{pgfscope}%
\pgfsys@transformshift{2.577009in}{0.749269in}%
\pgfsys@useobject{currentmarker}{}%
\end{pgfscope}%
\begin{pgfscope}%
\pgfsys@transformshift{2.598374in}{0.581708in}%
\pgfsys@useobject{currentmarker}{}%
\end{pgfscope}%
\begin{pgfscope}%
\pgfsys@transformshift{2.615982in}{0.502260in}%
\pgfsys@useobject{currentmarker}{}%
\end{pgfscope}%
\begin{pgfscope}%
\pgfsys@transformshift{2.633825in}{0.488992in}%
\pgfsys@useobject{currentmarker}{}%
\end{pgfscope}%
\begin{pgfscope}%
\pgfsys@transformshift{2.654719in}{0.570455in}%
\pgfsys@useobject{currentmarker}{}%
\end{pgfscope}%
\begin{pgfscope}%
\pgfsys@transformshift{2.672093in}{0.685940in}%
\pgfsys@useobject{currentmarker}{}%
\end{pgfscope}%
\begin{pgfscope}%
\pgfsys@transformshift{2.692282in}{0.801687in}%
\pgfsys@useobject{currentmarker}{}%
\end{pgfscope}%
\begin{pgfscope}%
\pgfsys@transformshift{2.711298in}{1.012530in}%
\pgfsys@useobject{currentmarker}{}%
\end{pgfscope}%
\begin{pgfscope}%
\pgfsys@transformshift{2.732193in}{1.264030in}%
\pgfsys@useobject{currentmarker}{}%
\end{pgfscope}%
\begin{pgfscope}%
\pgfsys@transformshift{2.750037in}{1.371442in}%
\pgfsys@useobject{currentmarker}{}%
\end{pgfscope}%
\begin{pgfscope}%
\pgfsys@transformshift{2.771166in}{1.339874in}%
\pgfsys@useobject{currentmarker}{}%
\end{pgfscope}%
\begin{pgfscope}%
\pgfsys@transformshift{2.788305in}{1.142746in}%
\pgfsys@useobject{currentmarker}{}%
\end{pgfscope}%
\begin{pgfscope}%
\pgfsys@transformshift{2.805911in}{0.860725in}%
\pgfsys@useobject{currentmarker}{}%
\end{pgfscope}%
\begin{pgfscope}%
\pgfsys@transformshift{2.827746in}{0.708297in}%
\pgfsys@useobject{currentmarker}{}%
\end{pgfscope}%
\begin{pgfscope}%
\pgfsys@transformshift{2.849580in}{0.561923in}%
\pgfsys@useobject{currentmarker}{}%
\end{pgfscope}%
\begin{pgfscope}%
\pgfsys@transformshift{2.866484in}{0.491080in}%
\pgfsys@useobject{currentmarker}{}%
\end{pgfscope}%
\begin{pgfscope}%
\pgfsys@transformshift{2.884796in}{0.506733in}%
\pgfsys@useobject{currentmarker}{}%
\end{pgfscope}%
\begin{pgfscope}%
\pgfsys@transformshift{2.905925in}{0.598524in}%
\pgfsys@useobject{currentmarker}{}%
\end{pgfscope}%
\begin{pgfscope}%
\pgfsys@transformshift{2.923533in}{0.669305in}%
\pgfsys@useobject{currentmarker}{}%
\end{pgfscope}%
\begin{pgfscope}%
\pgfsys@transformshift{2.941612in}{0.860555in}%
\pgfsys@useobject{currentmarker}{}%
\end{pgfscope}%
\begin{pgfscope}%
\pgfsys@transformshift{2.961566in}{1.173332in}%
\pgfsys@useobject{currentmarker}{}%
\end{pgfscope}%
\begin{pgfscope}%
\pgfsys@transformshift{2.980114in}{1.349108in}%
\pgfsys@useobject{currentmarker}{}%
\end{pgfscope}%
\begin{pgfscope}%
\pgfsys@transformshift{3.000774in}{1.385771in}%
\pgfsys@useobject{currentmarker}{}%
\end{pgfscope}%
\begin{pgfscope}%
\pgfsys@transformshift{3.019085in}{1.320232in}%
\pgfsys@useobject{currentmarker}{}%
\end{pgfscope}%
\begin{pgfscope}%
\pgfsys@transformshift{3.040919in}{1.009644in}%
\pgfsys@useobject{currentmarker}{}%
\end{pgfscope}%
\begin{pgfscope}%
\pgfsys@transformshift{3.055945in}{0.797656in}%
\pgfsys@useobject{currentmarker}{}%
\end{pgfscope}%
\begin{pgfscope}%
\pgfsys@transformshift{3.076604in}{0.664950in}%
\pgfsys@useobject{currentmarker}{}%
\end{pgfscope}%
\begin{pgfscope}%
\pgfsys@transformshift{3.098908in}{0.539330in}%
\pgfsys@useobject{currentmarker}{}%
\end{pgfscope}%
\begin{pgfscope}%
\pgfsys@transformshift{3.115341in}{0.488810in}%
\pgfsys@useobject{currentmarker}{}%
\end{pgfscope}%
\begin{pgfscope}%
\pgfsys@transformshift{3.136707in}{0.537990in}%
\pgfsys@useobject{currentmarker}{}%
\end{pgfscope}%
\begin{pgfscope}%
\pgfsys@transformshift{3.153846in}{0.624192in}%
\pgfsys@useobject{currentmarker}{}%
\end{pgfscope}%
\begin{pgfscope}%
\pgfsys@transformshift{3.173097in}{0.755502in}%
\pgfsys@useobject{currentmarker}{}%
\end{pgfscope}%
\begin{pgfscope}%
\pgfsys@transformshift{3.193052in}{0.954623in}%
\pgfsys@useobject{currentmarker}{}%
\end{pgfscope}%
\begin{pgfscope}%
\pgfsys@transformshift{3.211599in}{1.184364in}%
\pgfsys@useobject{currentmarker}{}%
\end{pgfscope}%
\begin{pgfscope}%
\pgfsys@transformshift{3.231319in}{1.371292in}%
\pgfsys@useobject{currentmarker}{}%
\end{pgfscope}%
\begin{pgfscope}%
\pgfsys@transformshift{3.250336in}{1.394155in}%
\pgfsys@useobject{currentmarker}{}%
\end{pgfscope}%
\begin{pgfscope}%
\pgfsys@transformshift{3.267710in}{1.326573in}%
\pgfsys@useobject{currentmarker}{}%
\end{pgfscope}%
\begin{pgfscope}%
\pgfsys@transformshift{3.288370in}{1.090313in}%
\pgfsys@useobject{currentmarker}{}%
\end{pgfscope}%
\begin{pgfscope}%
\pgfsys@transformshift{3.309030in}{0.798761in}%
\pgfsys@useobject{currentmarker}{}%
\end{pgfscope}%
\begin{pgfscope}%
\pgfsys@transformshift{3.327577in}{0.673420in}%
\pgfsys@useobject{currentmarker}{}%
\end{pgfscope}%
\begin{pgfscope}%
\pgfsys@transformshift{3.345889in}{0.563009in}%
\pgfsys@useobject{currentmarker}{}%
\end{pgfscope}%
\begin{pgfscope}%
\pgfsys@transformshift{3.363731in}{0.502619in}%
\pgfsys@useobject{currentmarker}{}%
\end{pgfscope}%
\begin{pgfscope}%
\pgfsys@transformshift{3.385097in}{0.513236in}%
\pgfsys@useobject{currentmarker}{}%
\end{pgfscope}%
\begin{pgfscope}%
\pgfsys@transformshift{3.408572in}{0.590026in}%
\pgfsys@useobject{currentmarker}{}%
\end{pgfscope}%
\begin{pgfscope}%
\pgfsys@transformshift{3.422659in}{0.674913in}%
\pgfsys@useobject{currentmarker}{}%
\end{pgfscope}%
\begin{pgfscope}%
\pgfsys@transformshift{3.442850in}{0.821763in}%
\pgfsys@useobject{currentmarker}{}%
\end{pgfscope}%
\begin{pgfscope}%
\pgfsys@transformshift{3.462336in}{1.020555in}%
\pgfsys@useobject{currentmarker}{}%
\end{pgfscope}%
\begin{pgfscope}%
\pgfsys@transformshift{3.481353in}{1.258756in}%
\pgfsys@useobject{currentmarker}{}%
\end{pgfscope}%
\begin{pgfscope}%
\pgfsys@transformshift{3.499430in}{1.386015in}%
\pgfsys@useobject{currentmarker}{}%
\end{pgfscope}%
\begin{pgfscope}%
\pgfsys@transformshift{3.519386in}{1.396280in}%
\pgfsys@useobject{currentmarker}{}%
\end{pgfscope}%
\begin{pgfscope}%
\pgfsys@transformshift{3.539341in}{1.333017in}%
\pgfsys@useobject{currentmarker}{}%
\end{pgfscope}%
\begin{pgfscope}%
\pgfsys@transformshift{3.555306in}{1.162221in}%
\pgfsys@useobject{currentmarker}{}%
\end{pgfscope}%
\begin{pgfscope}%
\pgfsys@transformshift{3.579252in}{0.888378in}%
\pgfsys@useobject{currentmarker}{}%
\end{pgfscope}%
\begin{pgfscope}%
\pgfsys@transformshift{3.597800in}{0.744205in}%
\pgfsys@useobject{currentmarker}{}%
\end{pgfscope}%
\begin{pgfscope}%
\pgfsys@transformshift{3.616346in}{0.636711in}%
\pgfsys@useobject{currentmarker}{}%
\end{pgfscope}%
\begin{pgfscope}%
\pgfsys@transformshift{3.633954in}{0.566859in}%
\pgfsys@useobject{currentmarker}{}%
\end{pgfscope}%
\begin{pgfscope}%
\pgfsys@transformshift{3.656022in}{0.504558in}%
\pgfsys@useobject{currentmarker}{}%
\end{pgfscope}%
\begin{pgfscope}%
\pgfsys@transformshift{3.673161in}{0.554304in}%
\pgfsys@useobject{currentmarker}{}%
\end{pgfscope}%
\begin{pgfscope}%
\pgfsys@transformshift{3.690535in}{0.642445in}%
\pgfsys@useobject{currentmarker}{}%
\end{pgfscope}%
\begin{pgfscope}%
\pgfsys@transformshift{3.712369in}{0.800629in}%
\pgfsys@useobject{currentmarker}{}%
\end{pgfscope}%
\begin{pgfscope}%
\pgfsys@transformshift{3.729741in}{0.949449in}%
\pgfsys@useobject{currentmarker}{}%
\end{pgfscope}%
\begin{pgfscope}%
\pgfsys@transformshift{3.754392in}{1.245752in}%
\pgfsys@useobject{currentmarker}{}%
\end{pgfscope}%
\begin{pgfscope}%
\pgfsys@transformshift{3.771063in}{1.374395in}%
\pgfsys@useobject{currentmarker}{}%
\end{pgfscope}%
\begin{pgfscope}%
\pgfsys@transformshift{3.790077in}{1.394819in}%
\pgfsys@useobject{currentmarker}{}%
\end{pgfscope}%
\begin{pgfscope}%
\pgfsys@transformshift{3.807216in}{1.417455in}%
\pgfsys@useobject{currentmarker}{}%
\end{pgfscope}%
\begin{pgfscope}%
\pgfsys@transformshift{3.826233in}{1.341841in}%
\pgfsys@useobject{currentmarker}{}%
\end{pgfscope}%
\begin{pgfscope}%
\pgfsys@transformshift{3.846659in}{1.104975in}%
\pgfsys@useobject{currentmarker}{}%
\end{pgfscope}%
\begin{pgfscope}%
\pgfsys@transformshift{3.864501in}{0.894977in}%
\pgfsys@useobject{currentmarker}{}%
\end{pgfscope}%
\begin{pgfscope}%
\pgfsys@transformshift{3.883987in}{0.796517in}%
\pgfsys@useobject{currentmarker}{}%
\end{pgfscope}%
\begin{pgfscope}%
\pgfsys@transformshift{3.901126in}{0.635406in}%
\pgfsys@useobject{currentmarker}{}%
\end{pgfscope}%
\begin{pgfscope}%
\pgfsys@transformshift{3.926247in}{0.705699in}%
\pgfsys@useobject{currentmarker}{}%
\end{pgfscope}%
\begin{pgfscope}%
\pgfsys@transformshift{3.942446in}{0.579439in}%
\pgfsys@useobject{currentmarker}{}%
\end{pgfscope}%
\begin{pgfscope}%
\pgfsys@transformshift{3.961228in}{0.512277in}%
\pgfsys@useobject{currentmarker}{}%
\end{pgfscope}%
\begin{pgfscope}%
\pgfsys@transformshift{3.983060in}{0.518014in}%
\pgfsys@useobject{currentmarker}{}%
\end{pgfscope}%
\begin{pgfscope}%
\pgfsys@transformshift{3.999496in}{0.600029in}%
\pgfsys@useobject{currentmarker}{}%
\end{pgfscope}%
\begin{pgfscope}%
\pgfsys@transformshift{4.018511in}{0.721554in}%
\pgfsys@useobject{currentmarker}{}%
\end{pgfscope}%
\begin{pgfscope}%
\pgfsys@transformshift{4.036590in}{0.864503in}%
\pgfsys@useobject{currentmarker}{}%
\end{pgfscope}%
\begin{pgfscope}%
\pgfsys@transformshift{4.058424in}{1.141904in}%
\pgfsys@useobject{currentmarker}{}%
\end{pgfscope}%
\begin{pgfscope}%
\pgfsys@transformshift{4.076501in}{1.284755in}%
\pgfsys@useobject{currentmarker}{}%
\end{pgfscope}%
\begin{pgfscope}%
\pgfsys@transformshift{4.096456in}{1.419720in}%
\pgfsys@useobject{currentmarker}{}%
\end{pgfscope}%
\begin{pgfscope}%
\pgfsys@transformshift{4.114298in}{1.433232in}%
\pgfsys@useobject{currentmarker}{}%
\end{pgfscope}%
\begin{pgfscope}%
\pgfsys@transformshift{4.135429in}{1.328352in}%
\pgfsys@useobject{currentmarker}{}%
\end{pgfscope}%
\begin{pgfscope}%
\pgfsys@transformshift{4.154446in}{1.113324in}%
\pgfsys@useobject{currentmarker}{}%
\end{pgfscope}%
\begin{pgfscope}%
\pgfsys@transformshift{4.171114in}{0.957690in}%
\pgfsys@useobject{currentmarker}{}%
\end{pgfscope}%
\begin{pgfscope}%
\pgfsys@transformshift{4.192713in}{0.761688in}%
\pgfsys@useobject{currentmarker}{}%
\end{pgfscope}%
\begin{pgfscope}%
\pgfsys@transformshift{4.209851in}{0.618801in}%
\pgfsys@useobject{currentmarker}{}%
\end{pgfscope}%
\begin{pgfscope}%
\pgfsys@transformshift{4.231685in}{0.539515in}%
\pgfsys@useobject{currentmarker}{}%
\end{pgfscope}%
\begin{pgfscope}%
\pgfsys@transformshift{4.250467in}{0.527203in}%
\pgfsys@useobject{currentmarker}{}%
\end{pgfscope}%
\begin{pgfscope}%
\pgfsys@transformshift{4.267841in}{0.577005in}%
\pgfsys@useobject{currentmarker}{}%
\end{pgfscope}%
\begin{pgfscope}%
\pgfsys@transformshift{4.284978in}{0.677198in}%
\pgfsys@useobject{currentmarker}{}%
\end{pgfscope}%
\begin{pgfscope}%
\pgfsys@transformshift{4.306109in}{0.836136in}%
\pgfsys@useobject{currentmarker}{}%
\end{pgfscope}%
\begin{pgfscope}%
\pgfsys@transformshift{4.324889in}{1.014246in}%
\pgfsys@useobject{currentmarker}{}%
\end{pgfscope}%
\begin{pgfscope}%
\pgfsys@transformshift{4.346254in}{1.283364in}%
\pgfsys@useobject{currentmarker}{}%
\end{pgfscope}%
\begin{pgfscope}%
\pgfsys@transformshift{4.364097in}{1.387295in}%
\pgfsys@useobject{currentmarker}{}%
\end{pgfscope}%
\begin{pgfscope}%
\pgfsys@transformshift{4.384757in}{1.456069in}%
\pgfsys@useobject{currentmarker}{}%
\end{pgfscope}%
\begin{pgfscope}%
\pgfsys@transformshift{4.402599in}{1.426140in}%
\pgfsys@useobject{currentmarker}{}%
\end{pgfscope}%
\begin{pgfscope}%
\pgfsys@transformshift{4.421381in}{1.309164in}%
\pgfsys@useobject{currentmarker}{}%
\end{pgfscope}%
\begin{pgfscope}%
\pgfsys@transformshift{4.441572in}{1.123927in}%
\pgfsys@useobject{currentmarker}{}%
\end{pgfscope}%
\begin{pgfscope}%
\pgfsys@transformshift{4.460824in}{0.911677in}%
\pgfsys@useobject{currentmarker}{}%
\end{pgfscope}%
\begin{pgfscope}%
\pgfsys@transformshift{4.481953in}{0.710852in}%
\pgfsys@useobject{currentmarker}{}%
\end{pgfscope}%
\begin{pgfscope}%
\pgfsys@transformshift{4.482421in}{0.716808in}%
\pgfsys@useobject{currentmarker}{}%
\end{pgfscope}%
\begin{pgfscope}%
\pgfsys@transformshift{4.475143in}{0.776270in}%
\pgfsys@useobject{currentmarker}{}%
\end{pgfscope}%
\begin{pgfscope}%
\pgfsys@transformshift{4.453780in}{1.144712in}%
\pgfsys@useobject{currentmarker}{}%
\end{pgfscope}%
\begin{pgfscope}%
\pgfsys@transformshift{4.434998in}{1.389163in}%
\pgfsys@useobject{currentmarker}{}%
\end{pgfscope}%
\begin{pgfscope}%
\pgfsys@transformshift{4.416452in}{1.457723in}%
\pgfsys@useobject{currentmarker}{}%
\end{pgfscope}%
\begin{pgfscope}%
\pgfsys@transformshift{4.397199in}{1.349966in}%
\pgfsys@useobject{currentmarker}{}%
\end{pgfscope}%
\begin{pgfscope}%
\pgfsys@transformshift{4.377010in}{1.008676in}%
\pgfsys@useobject{currentmarker}{}%
\end{pgfscope}%
\begin{pgfscope}%
\pgfsys@transformshift{4.357524in}{0.744670in}%
\pgfsys@useobject{currentmarker}{}%
\end{pgfscope}%
\begin{pgfscope}%
\pgfsys@transformshift{4.338271in}{0.585302in}%
\pgfsys@useobject{currentmarker}{}%
\end{pgfscope}%
\begin{pgfscope}%
\pgfsys@transformshift{4.317142in}{0.531477in}%
\pgfsys@useobject{currentmarker}{}%
\end{pgfscope}%
\begin{pgfscope}%
\pgfsys@transformshift{4.302117in}{0.631711in}%
\pgfsys@useobject{currentmarker}{}%
\end{pgfscope}%
\begin{pgfscope}%
\pgfsys@transformshift{4.280988in}{0.874649in}%
\pgfsys@useobject{currentmarker}{}%
\end{pgfscope}%
\begin{pgfscope}%
\pgfsys@transformshift{4.262675in}{1.203448in}%
\pgfsys@useobject{currentmarker}{}%
\end{pgfscope}%
\begin{pgfscope}%
\pgfsys@transformshift{4.243424in}{1.410110in}%
\pgfsys@useobject{currentmarker}{}%
\end{pgfscope}%
\begin{pgfscope}%
\pgfsys@transformshift{4.224407in}{1.425814in}%
\pgfsys@useobject{currentmarker}{}%
\end{pgfscope}%
\begin{pgfscope}%
\pgfsys@transformshift{4.206799in}{1.255729in}%
\pgfsys@useobject{currentmarker}{}%
\end{pgfscope}%
\begin{pgfscope}%
\pgfsys@transformshift{4.185435in}{0.905806in}%
\pgfsys@useobject{currentmarker}{}%
\end{pgfscope}%
\begin{pgfscope}%
\pgfsys@transformshift{4.166653in}{0.664308in}%
\pgfsys@useobject{currentmarker}{}%
\end{pgfscope}%
\begin{pgfscope}%
\pgfsys@transformshift{4.149514in}{0.550003in}%
\pgfsys@useobject{currentmarker}{}%
\end{pgfscope}%
\begin{pgfscope}%
\pgfsys@transformshift{4.128151in}{0.544158in}%
\pgfsys@useobject{currentmarker}{}%
\end{pgfscope}%
\begin{pgfscope}%
\pgfsys@transformshift{4.110777in}{0.665197in}%
\pgfsys@useobject{currentmarker}{}%
\end{pgfscope}%
\begin{pgfscope}%
\pgfsys@transformshift{4.088709in}{0.906863in}%
\pgfsys@useobject{currentmarker}{}%
\end{pgfscope}%
\begin{pgfscope}%
\pgfsys@transformshift{4.071806in}{1.224859in}%
\pgfsys@useobject{currentmarker}{}%
\end{pgfscope}%
\begin{pgfscope}%
\pgfsys@transformshift{4.070866in}{1.368935in}%
\pgfsys@useobject{currentmarker}{}%
\end{pgfscope}%
\begin{pgfscope}%
\pgfsys@transformshift{4.050206in}{1.413912in}%
\pgfsys@useobject{currentmarker}{}%
\end{pgfscope}%
\begin{pgfscope}%
\pgfsys@transformshift{4.033067in}{1.396352in}%
\pgfsys@useobject{currentmarker}{}%
\end{pgfscope}%
\begin{pgfscope}%
\pgfsys@transformshift{4.010998in}{1.111587in}%
\pgfsys@useobject{currentmarker}{}%
\end{pgfscope}%
\begin{pgfscope}%
\pgfsys@transformshift{3.994799in}{0.849169in}%
\pgfsys@useobject{currentmarker}{}%
\end{pgfscope}%
\begin{pgfscope}%
\pgfsys@transformshift{3.974374in}{0.629801in}%
\pgfsys@useobject{currentmarker}{}%
\end{pgfscope}%
\begin{pgfscope}%
\pgfsys@transformshift{3.954185in}{0.511850in}%
\pgfsys@useobject{currentmarker}{}%
\end{pgfscope}%
\begin{pgfscope}%
\pgfsys@transformshift{3.936576in}{0.523709in}%
\pgfsys@useobject{currentmarker}{}%
\end{pgfscope}%
\begin{pgfscope}%
\pgfsys@transformshift{3.912394in}{0.696605in}%
\pgfsys@useobject{currentmarker}{}%
\end{pgfscope}%
\begin{pgfscope}%
\pgfsys@transformshift{3.899246in}{0.890096in}%
\pgfsys@useobject{currentmarker}{}%
\end{pgfscope}%
\begin{pgfscope}%
\pgfsys@transformshift{3.878823in}{1.232179in}%
\pgfsys@useobject{currentmarker}{}%
\end{pgfscope}%
\begin{pgfscope}%
\pgfsys@transformshift{3.857927in}{1.405040in}%
\pgfsys@useobject{currentmarker}{}%
\end{pgfscope}%
\begin{pgfscope}%
\pgfsys@transformshift{3.840555in}{1.410345in}%
\pgfsys@useobject{currentmarker}{}%
\end{pgfscope}%
\begin{pgfscope}%
\pgfsys@transformshift{3.816607in}{1.244918in}%
\pgfsys@useobject{currentmarker}{}%
\end{pgfscope}%
\begin{pgfscope}%
\pgfsys@transformshift{3.802287in}{0.992558in}%
\pgfsys@useobject{currentmarker}{}%
\end{pgfscope}%
\begin{pgfscope}%
\pgfsys@transformshift{3.783036in}{0.730224in}%
\pgfsys@useobject{currentmarker}{}%
\end{pgfscope}%
\begin{pgfscope}%
\pgfsys@transformshift{3.765662in}{0.566689in}%
\pgfsys@useobject{currentmarker}{}%
\end{pgfscope}%
\begin{pgfscope}%
\pgfsys@transformshift{3.744768in}{0.496334in}%
\pgfsys@useobject{currentmarker}{}%
\end{pgfscope}%
\begin{pgfscope}%
\pgfsys@transformshift{3.725517in}{0.571855in}%
\pgfsys@useobject{currentmarker}{}%
\end{pgfscope}%
\begin{pgfscope}%
\pgfsys@transformshift{3.704151in}{0.761255in}%
\pgfsys@useobject{currentmarker}{}%
\end{pgfscope}%
\begin{pgfscope}%
\pgfsys@transformshift{3.687249in}{0.974842in}%
\pgfsys@useobject{currentmarker}{}%
\end{pgfscope}%
\begin{pgfscope}%
\pgfsys@transformshift{3.667527in}{1.249554in}%
\pgfsys@useobject{currentmarker}{}%
\end{pgfscope}%
\begin{pgfscope}%
\pgfsys@transformshift{3.647806in}{1.394546in}%
\pgfsys@useobject{currentmarker}{}%
\end{pgfscope}%
\begin{pgfscope}%
\pgfsys@transformshift{3.630667in}{1.359873in}%
\pgfsys@useobject{currentmarker}{}%
\end{pgfscope}%
\begin{pgfscope}%
\pgfsys@transformshift{3.609069in}{1.101668in}%
\pgfsys@useobject{currentmarker}{}%
\end{pgfscope}%
\begin{pgfscope}%
\pgfsys@transformshift{3.589347in}{0.783352in}%
\pgfsys@useobject{currentmarker}{}%
\end{pgfscope}%
\begin{pgfscope}%
\pgfsys@transformshift{3.568688in}{0.600755in}%
\pgfsys@useobject{currentmarker}{}%
\end{pgfscope}%
\begin{pgfscope}%
\pgfsys@transformshift{3.550376in}{0.506890in}%
\pgfsys@useobject{currentmarker}{}%
\end{pgfscope}%
\begin{pgfscope}%
\pgfsys@transformshift{3.532768in}{0.500819in}%
\pgfsys@useobject{currentmarker}{}%
\end{pgfscope}%
\begin{pgfscope}%
\pgfsys@transformshift{3.512108in}{0.609014in}%
\pgfsys@useobject{currentmarker}{}%
\end{pgfscope}%
\begin{pgfscope}%
\pgfsys@transformshift{3.495438in}{0.779514in}%
\pgfsys@useobject{currentmarker}{}%
\end{pgfscope}%
\begin{pgfscope}%
\pgfsys@transformshift{3.475249in}{1.108543in}%
\pgfsys@useobject{currentmarker}{}%
\end{pgfscope}%
\begin{pgfscope}%
\pgfsys@transformshift{3.457170in}{1.337536in}%
\pgfsys@useobject{currentmarker}{}%
\end{pgfscope}%
\begin{pgfscope}%
\pgfsys@transformshift{3.436510in}{1.392004in}%
\pgfsys@useobject{currentmarker}{}%
\end{pgfscope}%
\begin{pgfscope}%
\pgfsys@transformshift{3.418902in}{1.309922in}%
\pgfsys@useobject{currentmarker}{}%
\end{pgfscope}%
\begin{pgfscope}%
\pgfsys@transformshift{3.397304in}{1.025541in}%
\pgfsys@useobject{currentmarker}{}%
\end{pgfscope}%
\begin{pgfscope}%
\pgfsys@transformshift{3.380636in}{0.793570in}%
\pgfsys@useobject{currentmarker}{}%
\end{pgfscope}%
\begin{pgfscope}%
\pgfsys@transformshift{3.359036in}{0.621130in}%
\pgfsys@useobject{currentmarker}{}%
\end{pgfscope}%
\begin{pgfscope}%
\pgfsys@transformshift{3.338142in}{0.518752in}%
\pgfsys@useobject{currentmarker}{}%
\end{pgfscope}%
\begin{pgfscope}%
\pgfsys@transformshift{3.321472in}{0.494395in}%
\pgfsys@useobject{currentmarker}{}%
\end{pgfscope}%
\begin{pgfscope}%
\pgfsys@transformshift{3.300812in}{0.587786in}%
\pgfsys@useobject{currentmarker}{}%
\end{pgfscope}%
\begin{pgfscope}%
\pgfsys@transformshift{3.282266in}{0.711500in}%
\pgfsys@useobject{currentmarker}{}%
\end{pgfscope}%
\begin{pgfscope}%
\pgfsys@transformshift{3.263953in}{0.953723in}%
\pgfsys@useobject{currentmarker}{}%
\end{pgfscope}%
\begin{pgfscope}%
\pgfsys@transformshift{3.243527in}{1.284247in}%
\pgfsys@useobject{currentmarker}{}%
\end{pgfscope}%
\begin{pgfscope}%
\pgfsys@transformshift{3.224276in}{1.385299in}%
\pgfsys@useobject{currentmarker}{}%
\end{pgfscope}%
\begin{pgfscope}%
\pgfsys@transformshift{3.205025in}{1.338776in}%
\pgfsys@useobject{currentmarker}{}%
\end{pgfscope}%
\begin{pgfscope}%
\pgfsys@transformshift{3.188356in}{1.219200in}%
\pgfsys@useobject{currentmarker}{}%
\end{pgfscope}%
\begin{pgfscope}%
\pgfsys@transformshift{3.166522in}{0.914894in}%
\pgfsys@useobject{currentmarker}{}%
\end{pgfscope}%
\begin{pgfscope}%
\pgfsys@transformshift{3.148914in}{0.966095in}%
\pgfsys@useobject{currentmarker}{}%
\end{pgfscope}%
\begin{pgfscope}%
\pgfsys@transformshift{3.128020in}{0.690705in}%
\pgfsys@useobject{currentmarker}{}%
\end{pgfscope}%
\begin{pgfscope}%
\pgfsys@transformshift{3.110177in}{0.604121in}%
\pgfsys@useobject{currentmarker}{}%
\end{pgfscope}%
\end{pgfscope}%
\begin{pgfscope}%
\pgfsetrectcap%
\pgfsetmiterjoin%
\pgfsetlinewidth{0.501875pt}%
\definecolor{currentstroke}{rgb}{0.000000,0.000000,0.000000}%
\pgfsetstrokecolor{currentstroke}%
\pgfsetdash{}{0pt}%
\pgfpathmoveto{\pgfqpoint{0.444748in}{0.431673in}}%
\pgfpathlineto{\pgfqpoint{0.444748in}{1.507795in}}%
\pgfusepath{stroke}%
\end{pgfscope}%
\begin{pgfscope}%
\pgfsetrectcap%
\pgfsetmiterjoin%
\pgfsetlinewidth{0.501875pt}%
\definecolor{currentstroke}{rgb}{0.000000,0.000000,0.000000}%
\pgfsetstrokecolor{currentstroke}%
\pgfsetdash{}{0pt}%
\pgfpathmoveto{\pgfqpoint{4.676167in}{0.431673in}}%
\pgfpathlineto{\pgfqpoint{4.676167in}{1.507795in}}%
\pgfusepath{stroke}%
\end{pgfscope}%
\begin{pgfscope}%
\pgfsetrectcap%
\pgfsetmiterjoin%
\pgfsetlinewidth{0.501875pt}%
\definecolor{currentstroke}{rgb}{0.000000,0.000000,0.000000}%
\pgfsetstrokecolor{currentstroke}%
\pgfsetdash{}{0pt}%
\pgfpathmoveto{\pgfqpoint{0.444748in}{0.431673in}}%
\pgfpathlineto{\pgfqpoint{4.676167in}{0.431673in}}%
\pgfusepath{stroke}%
\end{pgfscope}%
\begin{pgfscope}%
\pgfsetrectcap%
\pgfsetmiterjoin%
\pgfsetlinewidth{0.501875pt}%
\definecolor{currentstroke}{rgb}{0.000000,0.000000,0.000000}%
\pgfsetstrokecolor{currentstroke}%
\pgfsetdash{}{0pt}%
\pgfpathmoveto{\pgfqpoint{0.444748in}{1.507795in}}%
\pgfpathlineto{\pgfqpoint{4.676167in}{1.507795in}}%
\pgfusepath{stroke}%
\end{pgfscope}%
\begin{pgfscope}%
\definecolor{textcolor}{rgb}{0.000000,0.000000,0.000000}%
\pgfsetstrokecolor{textcolor}%
\pgfsetfillcolor{textcolor}%
\pgftext[x=2.560458in,y=1.591129in,,base]{\color{textcolor}\rmfamily\fontsize{12.000000}{14.400000}\selectfont T = \qty{3.4}{\kelvin}}%
\end{pgfscope}%
\end{pgfpicture}%
\makeatother%
\endgroup%

	\caption{SQIs at \qtyrange{2.8}{3.6}{\kelvin} of sample CP2. We note that they appear to have a slightly curved background. We used a bias current of \qty{300}{\micro\ampere} except for \qty{2.8}{\kelvin} where \qty{350}{\micro\ampere} was used. Near zero field the patterns seem quite stable but further out become more noisy.}
	\label{fig:CP2.6B_revisited_SQIs}
\end{figure}

After the calibrations the current-phase relations were measured. They are shown in Figure~\ref{fig:CP2.6B_revisited_CPRs}. From the slope in our data we determined the mutual inductance to be 0.4 times the simulated value in Table~\ref{tab:CP2.6B-geometries}. For the loop inductance the value is 3.5 times the simulated value. These deviations might be explained by the fact that the simulation does not take the layer of \ce{Cu} into account. Additionally these values do not match what we previously found for this sample. Possibly this is because the calibration was not optimal. The amplitude of the oscillations appears to decrease as the temperature increases. Except for the measurement at \qty{3.6}{\kelvin} where it increases again. We suspect that this is due to a human error. A likely explanation is that a different bias current was used for the dc-SQUID compared to the bias current used for the calibration.

\begin{figure}[ht!]
	\centering
	%% Creator: Matplotlib, PGF backend
%%
%% To include the figure in your LaTeX document, write
%%   \input{<filename>.pgf}
%%
%% Make sure the required packages are loaded in your preamble
%%   \usepackage{pgf}
%%
%% Also ensure that all the required font packages are loaded; for instance,
%% the lmodern package is sometimes necessary when using math font.
%%   \usepackage{lmodern}
%%
%% Figures using additional raster images can only be included by \input if
%% they are in the same directory as the main LaTeX file. For loading figures
%% from other directories you can use the `import` package
%%   \usepackage{import}
%%
%% and then include the figures with
%%   \import{<path to file>}{<filename>.pgf}
%%
%% Matplotlib used the following preamble
%%   \usepackage{siunitx}
%%   \usepackage{fontspec}
%%   \setmainfont{Times New Roman.ttf}[Path=\detokenize{/System/Library/Fonts/Supplemental/}]
%%   \setsansfont{DejaVuSans.ttf}[Path=\detokenize{/Users/julian/UL-BRP-analysis/venv/lib/python3.10/site-packages/matplotlib/mpl-data/fonts/ttf/}]
%%   \setmonofont{DejaVuSansMono.ttf}[Path=\detokenize{/Users/julian/UL-BRP-analysis/venv/lib/python3.10/site-packages/matplotlib/mpl-data/fonts/ttf/}]
%%   \makeatletter\@ifpackageloaded{underscore}{}{\usepackage[strings]{underscore}}\makeatother
%%
\begingroup%
\makeatletter%
\begin{pgfpicture}%
\pgfpathrectangle{\pgfpointorigin}{\pgfqpoint{4.717276in}{2.246916in}}%
\pgfusepath{use as bounding box, clip}%
\begin{pgfscope}%
\pgfsetbuttcap%
\pgfsetmiterjoin%
\definecolor{currentfill}{rgb}{1.000000,1.000000,1.000000}%
\pgfsetfillcolor{currentfill}%
\pgfsetlinewidth{0.000000pt}%
\definecolor{currentstroke}{rgb}{1.000000,1.000000,1.000000}%
\pgfsetstrokecolor{currentstroke}%
\pgfsetdash{}{0pt}%
\pgfpathmoveto{\pgfqpoint{0.000000in}{0.000000in}}%
\pgfpathlineto{\pgfqpoint{4.717276in}{0.000000in}}%
\pgfpathlineto{\pgfqpoint{4.717276in}{2.246916in}}%
\pgfpathlineto{\pgfqpoint{0.000000in}{2.246916in}}%
\pgfpathlineto{\pgfqpoint{0.000000in}{0.000000in}}%
\pgfpathclose%
\pgfusepath{fill}%
\end{pgfscope}%
\begin{pgfscope}%
\pgfsetbuttcap%
\pgfsetmiterjoin%
\definecolor{currentfill}{rgb}{1.000000,1.000000,1.000000}%
\pgfsetfillcolor{currentfill}%
\pgfsetlinewidth{0.000000pt}%
\definecolor{currentstroke}{rgb}{0.000000,0.000000,0.000000}%
\pgfsetstrokecolor{currentstroke}%
\pgfsetstrokeopacity{0.000000}%
\pgfsetdash{}{0pt}%
\pgfpathmoveto{\pgfqpoint{0.552773in}{0.431673in}}%
\pgfpathlineto{\pgfqpoint{4.291580in}{0.431673in}}%
\pgfpathlineto{\pgfqpoint{4.291580in}{2.196916in}}%
\pgfpathlineto{\pgfqpoint{0.552773in}{2.196916in}}%
\pgfpathlineto{\pgfqpoint{0.552773in}{0.431673in}}%
\pgfpathclose%
\pgfusepath{fill}%
\end{pgfscope}%
\begin{pgfscope}%
\pgfsetbuttcap%
\pgfsetroundjoin%
\definecolor{currentfill}{rgb}{0.000000,0.000000,0.000000}%
\pgfsetfillcolor{currentfill}%
\pgfsetlinewidth{0.501875pt}%
\definecolor{currentstroke}{rgb}{0.000000,0.000000,0.000000}%
\pgfsetstrokecolor{currentstroke}%
\pgfsetdash{}{0pt}%
\pgfsys@defobject{currentmarker}{\pgfqpoint{0.000000in}{0.000000in}}{\pgfqpoint{0.000000in}{0.041667in}}{%
\pgfpathmoveto{\pgfqpoint{0.000000in}{0.000000in}}%
\pgfpathlineto{\pgfqpoint{0.000000in}{0.041667in}}%
\pgfusepath{stroke,fill}%
}%
\begin{pgfscope}%
\pgfsys@transformshift{0.599752in}{0.431673in}%
\pgfsys@useobject{currentmarker}{}%
\end{pgfscope}%
\end{pgfscope}%
\begin{pgfscope}%
\pgfsetbuttcap%
\pgfsetroundjoin%
\definecolor{currentfill}{rgb}{0.000000,0.000000,0.000000}%
\pgfsetfillcolor{currentfill}%
\pgfsetlinewidth{0.501875pt}%
\definecolor{currentstroke}{rgb}{0.000000,0.000000,0.000000}%
\pgfsetstrokecolor{currentstroke}%
\pgfsetdash{}{0pt}%
\pgfsys@defobject{currentmarker}{\pgfqpoint{0.000000in}{-0.041667in}}{\pgfqpoint{0.000000in}{0.000000in}}{%
\pgfpathmoveto{\pgfqpoint{0.000000in}{0.000000in}}%
\pgfpathlineto{\pgfqpoint{0.000000in}{-0.041667in}}%
\pgfusepath{stroke,fill}%
}%
\begin{pgfscope}%
\pgfsys@transformshift{0.599752in}{2.196916in}%
\pgfsys@useobject{currentmarker}{}%
\end{pgfscope}%
\end{pgfscope}%
\begin{pgfscope}%
\definecolor{textcolor}{rgb}{0.000000,0.000000,0.000000}%
\pgfsetstrokecolor{textcolor}%
\pgfsetfillcolor{textcolor}%
\pgftext[x=0.599752in,y=0.383062in,,top]{\color{textcolor}\rmfamily\fontsize{10.000000}{12.000000}\selectfont \(\displaystyle -6\pi\)}%
\end{pgfscope}%
\begin{pgfscope}%
\pgfsetbuttcap%
\pgfsetroundjoin%
\definecolor{currentfill}{rgb}{0.000000,0.000000,0.000000}%
\pgfsetfillcolor{currentfill}%
\pgfsetlinewidth{0.501875pt}%
\definecolor{currentstroke}{rgb}{0.000000,0.000000,0.000000}%
\pgfsetstrokecolor{currentstroke}%
\pgfsetdash{}{0pt}%
\pgfsys@defobject{currentmarker}{\pgfqpoint{0.000000in}{0.000000in}}{\pgfqpoint{0.000000in}{0.041667in}}{%
\pgfpathmoveto{\pgfqpoint{0.000000in}{0.000000in}}%
\pgfpathlineto{\pgfqpoint{0.000000in}{0.041667in}}%
\pgfusepath{stroke,fill}%
}%
\begin{pgfscope}%
\pgfsys@transformshift{1.212818in}{0.431673in}%
\pgfsys@useobject{currentmarker}{}%
\end{pgfscope}%
\end{pgfscope}%
\begin{pgfscope}%
\pgfsetbuttcap%
\pgfsetroundjoin%
\definecolor{currentfill}{rgb}{0.000000,0.000000,0.000000}%
\pgfsetfillcolor{currentfill}%
\pgfsetlinewidth{0.501875pt}%
\definecolor{currentstroke}{rgb}{0.000000,0.000000,0.000000}%
\pgfsetstrokecolor{currentstroke}%
\pgfsetdash{}{0pt}%
\pgfsys@defobject{currentmarker}{\pgfqpoint{0.000000in}{-0.041667in}}{\pgfqpoint{0.000000in}{0.000000in}}{%
\pgfpathmoveto{\pgfqpoint{0.000000in}{0.000000in}}%
\pgfpathlineto{\pgfqpoint{0.000000in}{-0.041667in}}%
\pgfusepath{stroke,fill}%
}%
\begin{pgfscope}%
\pgfsys@transformshift{1.212818in}{2.196916in}%
\pgfsys@useobject{currentmarker}{}%
\end{pgfscope}%
\end{pgfscope}%
\begin{pgfscope}%
\definecolor{textcolor}{rgb}{0.000000,0.000000,0.000000}%
\pgfsetstrokecolor{textcolor}%
\pgfsetfillcolor{textcolor}%
\pgftext[x=1.212818in,y=0.383062in,,top]{\color{textcolor}\rmfamily\fontsize{10.000000}{12.000000}\selectfont \(\displaystyle -4\pi\)}%
\end{pgfscope}%
\begin{pgfscope}%
\pgfsetbuttcap%
\pgfsetroundjoin%
\definecolor{currentfill}{rgb}{0.000000,0.000000,0.000000}%
\pgfsetfillcolor{currentfill}%
\pgfsetlinewidth{0.501875pt}%
\definecolor{currentstroke}{rgb}{0.000000,0.000000,0.000000}%
\pgfsetstrokecolor{currentstroke}%
\pgfsetdash{}{0pt}%
\pgfsys@defobject{currentmarker}{\pgfqpoint{0.000000in}{0.000000in}}{\pgfqpoint{0.000000in}{0.041667in}}{%
\pgfpathmoveto{\pgfqpoint{0.000000in}{0.000000in}}%
\pgfpathlineto{\pgfqpoint{0.000000in}{0.041667in}}%
\pgfusepath{stroke,fill}%
}%
\begin{pgfscope}%
\pgfsys@transformshift{1.825885in}{0.431673in}%
\pgfsys@useobject{currentmarker}{}%
\end{pgfscope}%
\end{pgfscope}%
\begin{pgfscope}%
\pgfsetbuttcap%
\pgfsetroundjoin%
\definecolor{currentfill}{rgb}{0.000000,0.000000,0.000000}%
\pgfsetfillcolor{currentfill}%
\pgfsetlinewidth{0.501875pt}%
\definecolor{currentstroke}{rgb}{0.000000,0.000000,0.000000}%
\pgfsetstrokecolor{currentstroke}%
\pgfsetdash{}{0pt}%
\pgfsys@defobject{currentmarker}{\pgfqpoint{0.000000in}{-0.041667in}}{\pgfqpoint{0.000000in}{0.000000in}}{%
\pgfpathmoveto{\pgfqpoint{0.000000in}{0.000000in}}%
\pgfpathlineto{\pgfqpoint{0.000000in}{-0.041667in}}%
\pgfusepath{stroke,fill}%
}%
\begin{pgfscope}%
\pgfsys@transformshift{1.825885in}{2.196916in}%
\pgfsys@useobject{currentmarker}{}%
\end{pgfscope}%
\end{pgfscope}%
\begin{pgfscope}%
\definecolor{textcolor}{rgb}{0.000000,0.000000,0.000000}%
\pgfsetstrokecolor{textcolor}%
\pgfsetfillcolor{textcolor}%
\pgftext[x=1.825885in,y=0.383062in,,top]{\color{textcolor}\rmfamily\fontsize{10.000000}{12.000000}\selectfont \(\displaystyle -2\pi\)}%
\end{pgfscope}%
\begin{pgfscope}%
\pgfsetbuttcap%
\pgfsetroundjoin%
\definecolor{currentfill}{rgb}{0.000000,0.000000,0.000000}%
\pgfsetfillcolor{currentfill}%
\pgfsetlinewidth{0.501875pt}%
\definecolor{currentstroke}{rgb}{0.000000,0.000000,0.000000}%
\pgfsetstrokecolor{currentstroke}%
\pgfsetdash{}{0pt}%
\pgfsys@defobject{currentmarker}{\pgfqpoint{0.000000in}{0.000000in}}{\pgfqpoint{0.000000in}{0.041667in}}{%
\pgfpathmoveto{\pgfqpoint{0.000000in}{0.000000in}}%
\pgfpathlineto{\pgfqpoint{0.000000in}{0.041667in}}%
\pgfusepath{stroke,fill}%
}%
\begin{pgfscope}%
\pgfsys@transformshift{2.438952in}{0.431673in}%
\pgfsys@useobject{currentmarker}{}%
\end{pgfscope}%
\end{pgfscope}%
\begin{pgfscope}%
\pgfsetbuttcap%
\pgfsetroundjoin%
\definecolor{currentfill}{rgb}{0.000000,0.000000,0.000000}%
\pgfsetfillcolor{currentfill}%
\pgfsetlinewidth{0.501875pt}%
\definecolor{currentstroke}{rgb}{0.000000,0.000000,0.000000}%
\pgfsetstrokecolor{currentstroke}%
\pgfsetdash{}{0pt}%
\pgfsys@defobject{currentmarker}{\pgfqpoint{0.000000in}{-0.041667in}}{\pgfqpoint{0.000000in}{0.000000in}}{%
\pgfpathmoveto{\pgfqpoint{0.000000in}{0.000000in}}%
\pgfpathlineto{\pgfqpoint{0.000000in}{-0.041667in}}%
\pgfusepath{stroke,fill}%
}%
\begin{pgfscope}%
\pgfsys@transformshift{2.438952in}{2.196916in}%
\pgfsys@useobject{currentmarker}{}%
\end{pgfscope}%
\end{pgfscope}%
\begin{pgfscope}%
\definecolor{textcolor}{rgb}{0.000000,0.000000,0.000000}%
\pgfsetstrokecolor{textcolor}%
\pgfsetfillcolor{textcolor}%
\pgftext[x=2.438952in,y=0.383062in,,top]{\color{textcolor}\rmfamily\fontsize{10.000000}{12.000000}\selectfont 0}%
\end{pgfscope}%
\begin{pgfscope}%
\pgfsetbuttcap%
\pgfsetroundjoin%
\definecolor{currentfill}{rgb}{0.000000,0.000000,0.000000}%
\pgfsetfillcolor{currentfill}%
\pgfsetlinewidth{0.501875pt}%
\definecolor{currentstroke}{rgb}{0.000000,0.000000,0.000000}%
\pgfsetstrokecolor{currentstroke}%
\pgfsetdash{}{0pt}%
\pgfsys@defobject{currentmarker}{\pgfqpoint{0.000000in}{0.000000in}}{\pgfqpoint{0.000000in}{0.041667in}}{%
\pgfpathmoveto{\pgfqpoint{0.000000in}{0.000000in}}%
\pgfpathlineto{\pgfqpoint{0.000000in}{0.041667in}}%
\pgfusepath{stroke,fill}%
}%
\begin{pgfscope}%
\pgfsys@transformshift{3.052018in}{0.431673in}%
\pgfsys@useobject{currentmarker}{}%
\end{pgfscope}%
\end{pgfscope}%
\begin{pgfscope}%
\pgfsetbuttcap%
\pgfsetroundjoin%
\definecolor{currentfill}{rgb}{0.000000,0.000000,0.000000}%
\pgfsetfillcolor{currentfill}%
\pgfsetlinewidth{0.501875pt}%
\definecolor{currentstroke}{rgb}{0.000000,0.000000,0.000000}%
\pgfsetstrokecolor{currentstroke}%
\pgfsetdash{}{0pt}%
\pgfsys@defobject{currentmarker}{\pgfqpoint{0.000000in}{-0.041667in}}{\pgfqpoint{0.000000in}{0.000000in}}{%
\pgfpathmoveto{\pgfqpoint{0.000000in}{0.000000in}}%
\pgfpathlineto{\pgfqpoint{0.000000in}{-0.041667in}}%
\pgfusepath{stroke,fill}%
}%
\begin{pgfscope}%
\pgfsys@transformshift{3.052018in}{2.196916in}%
\pgfsys@useobject{currentmarker}{}%
\end{pgfscope}%
\end{pgfscope}%
\begin{pgfscope}%
\definecolor{textcolor}{rgb}{0.000000,0.000000,0.000000}%
\pgfsetstrokecolor{textcolor}%
\pgfsetfillcolor{textcolor}%
\pgftext[x=3.052018in,y=0.383062in,,top]{\color{textcolor}\rmfamily\fontsize{10.000000}{12.000000}\selectfont \(\displaystyle 2\pi\)}%
\end{pgfscope}%
\begin{pgfscope}%
\pgfsetbuttcap%
\pgfsetroundjoin%
\definecolor{currentfill}{rgb}{0.000000,0.000000,0.000000}%
\pgfsetfillcolor{currentfill}%
\pgfsetlinewidth{0.501875pt}%
\definecolor{currentstroke}{rgb}{0.000000,0.000000,0.000000}%
\pgfsetstrokecolor{currentstroke}%
\pgfsetdash{}{0pt}%
\pgfsys@defobject{currentmarker}{\pgfqpoint{0.000000in}{0.000000in}}{\pgfqpoint{0.000000in}{0.041667in}}{%
\pgfpathmoveto{\pgfqpoint{0.000000in}{0.000000in}}%
\pgfpathlineto{\pgfqpoint{0.000000in}{0.041667in}}%
\pgfusepath{stroke,fill}%
}%
\begin{pgfscope}%
\pgfsys@transformshift{3.665085in}{0.431673in}%
\pgfsys@useobject{currentmarker}{}%
\end{pgfscope}%
\end{pgfscope}%
\begin{pgfscope}%
\pgfsetbuttcap%
\pgfsetroundjoin%
\definecolor{currentfill}{rgb}{0.000000,0.000000,0.000000}%
\pgfsetfillcolor{currentfill}%
\pgfsetlinewidth{0.501875pt}%
\definecolor{currentstroke}{rgb}{0.000000,0.000000,0.000000}%
\pgfsetstrokecolor{currentstroke}%
\pgfsetdash{}{0pt}%
\pgfsys@defobject{currentmarker}{\pgfqpoint{0.000000in}{-0.041667in}}{\pgfqpoint{0.000000in}{0.000000in}}{%
\pgfpathmoveto{\pgfqpoint{0.000000in}{0.000000in}}%
\pgfpathlineto{\pgfqpoint{0.000000in}{-0.041667in}}%
\pgfusepath{stroke,fill}%
}%
\begin{pgfscope}%
\pgfsys@transformshift{3.665085in}{2.196916in}%
\pgfsys@useobject{currentmarker}{}%
\end{pgfscope}%
\end{pgfscope}%
\begin{pgfscope}%
\definecolor{textcolor}{rgb}{0.000000,0.000000,0.000000}%
\pgfsetstrokecolor{textcolor}%
\pgfsetfillcolor{textcolor}%
\pgftext[x=3.665085in,y=0.383062in,,top]{\color{textcolor}\rmfamily\fontsize{10.000000}{12.000000}\selectfont \(\displaystyle 4\pi\)}%
\end{pgfscope}%
\begin{pgfscope}%
\pgfsetbuttcap%
\pgfsetroundjoin%
\definecolor{currentfill}{rgb}{0.000000,0.000000,0.000000}%
\pgfsetfillcolor{currentfill}%
\pgfsetlinewidth{0.501875pt}%
\definecolor{currentstroke}{rgb}{0.000000,0.000000,0.000000}%
\pgfsetstrokecolor{currentstroke}%
\pgfsetdash{}{0pt}%
\pgfsys@defobject{currentmarker}{\pgfqpoint{0.000000in}{0.000000in}}{\pgfqpoint{0.000000in}{0.041667in}}{%
\pgfpathmoveto{\pgfqpoint{0.000000in}{0.000000in}}%
\pgfpathlineto{\pgfqpoint{0.000000in}{0.041667in}}%
\pgfusepath{stroke,fill}%
}%
\begin{pgfscope}%
\pgfsys@transformshift{4.278152in}{0.431673in}%
\pgfsys@useobject{currentmarker}{}%
\end{pgfscope}%
\end{pgfscope}%
\begin{pgfscope}%
\pgfsetbuttcap%
\pgfsetroundjoin%
\definecolor{currentfill}{rgb}{0.000000,0.000000,0.000000}%
\pgfsetfillcolor{currentfill}%
\pgfsetlinewidth{0.501875pt}%
\definecolor{currentstroke}{rgb}{0.000000,0.000000,0.000000}%
\pgfsetstrokecolor{currentstroke}%
\pgfsetdash{}{0pt}%
\pgfsys@defobject{currentmarker}{\pgfqpoint{0.000000in}{-0.041667in}}{\pgfqpoint{0.000000in}{0.000000in}}{%
\pgfpathmoveto{\pgfqpoint{0.000000in}{0.000000in}}%
\pgfpathlineto{\pgfqpoint{0.000000in}{-0.041667in}}%
\pgfusepath{stroke,fill}%
}%
\begin{pgfscope}%
\pgfsys@transformshift{4.278152in}{2.196916in}%
\pgfsys@useobject{currentmarker}{}%
\end{pgfscope}%
\end{pgfscope}%
\begin{pgfscope}%
\definecolor{textcolor}{rgb}{0.000000,0.000000,0.000000}%
\pgfsetstrokecolor{textcolor}%
\pgfsetfillcolor{textcolor}%
\pgftext[x=4.278152in,y=0.383062in,,top]{\color{textcolor}\rmfamily\fontsize{10.000000}{12.000000}\selectfont \(\displaystyle 6\pi\)}%
\end{pgfscope}%
\begin{pgfscope}%
\pgfsetbuttcap%
\pgfsetroundjoin%
\definecolor{currentfill}{rgb}{0.000000,0.000000,0.000000}%
\pgfsetfillcolor{currentfill}%
\pgfsetlinewidth{0.501875pt}%
\definecolor{currentstroke}{rgb}{0.000000,0.000000,0.000000}%
\pgfsetstrokecolor{currentstroke}%
\pgfsetdash{}{0pt}%
\pgfsys@defobject{currentmarker}{\pgfqpoint{0.000000in}{0.000000in}}{\pgfqpoint{0.000000in}{0.020833in}}{%
\pgfpathmoveto{\pgfqpoint{0.000000in}{0.000000in}}%
\pgfpathlineto{\pgfqpoint{0.000000in}{0.020833in}}%
\pgfusepath{stroke,fill}%
}%
\begin{pgfscope}%
\pgfsys@transformshift{0.753018in}{0.431673in}%
\pgfsys@useobject{currentmarker}{}%
\end{pgfscope}%
\end{pgfscope}%
\begin{pgfscope}%
\pgfsetbuttcap%
\pgfsetroundjoin%
\definecolor{currentfill}{rgb}{0.000000,0.000000,0.000000}%
\pgfsetfillcolor{currentfill}%
\pgfsetlinewidth{0.501875pt}%
\definecolor{currentstroke}{rgb}{0.000000,0.000000,0.000000}%
\pgfsetstrokecolor{currentstroke}%
\pgfsetdash{}{0pt}%
\pgfsys@defobject{currentmarker}{\pgfqpoint{0.000000in}{-0.020833in}}{\pgfqpoint{0.000000in}{0.000000in}}{%
\pgfpathmoveto{\pgfqpoint{0.000000in}{0.000000in}}%
\pgfpathlineto{\pgfqpoint{0.000000in}{-0.020833in}}%
\pgfusepath{stroke,fill}%
}%
\begin{pgfscope}%
\pgfsys@transformshift{0.753018in}{2.196916in}%
\pgfsys@useobject{currentmarker}{}%
\end{pgfscope}%
\end{pgfscope}%
\begin{pgfscope}%
\pgfsetbuttcap%
\pgfsetroundjoin%
\definecolor{currentfill}{rgb}{0.000000,0.000000,0.000000}%
\pgfsetfillcolor{currentfill}%
\pgfsetlinewidth{0.501875pt}%
\definecolor{currentstroke}{rgb}{0.000000,0.000000,0.000000}%
\pgfsetstrokecolor{currentstroke}%
\pgfsetdash{}{0pt}%
\pgfsys@defobject{currentmarker}{\pgfqpoint{0.000000in}{0.000000in}}{\pgfqpoint{0.000000in}{0.020833in}}{%
\pgfpathmoveto{\pgfqpoint{0.000000in}{0.000000in}}%
\pgfpathlineto{\pgfqpoint{0.000000in}{0.020833in}}%
\pgfusepath{stroke,fill}%
}%
\begin{pgfscope}%
\pgfsys@transformshift{0.906285in}{0.431673in}%
\pgfsys@useobject{currentmarker}{}%
\end{pgfscope}%
\end{pgfscope}%
\begin{pgfscope}%
\pgfsetbuttcap%
\pgfsetroundjoin%
\definecolor{currentfill}{rgb}{0.000000,0.000000,0.000000}%
\pgfsetfillcolor{currentfill}%
\pgfsetlinewidth{0.501875pt}%
\definecolor{currentstroke}{rgb}{0.000000,0.000000,0.000000}%
\pgfsetstrokecolor{currentstroke}%
\pgfsetdash{}{0pt}%
\pgfsys@defobject{currentmarker}{\pgfqpoint{0.000000in}{-0.020833in}}{\pgfqpoint{0.000000in}{0.000000in}}{%
\pgfpathmoveto{\pgfqpoint{0.000000in}{0.000000in}}%
\pgfpathlineto{\pgfqpoint{0.000000in}{-0.020833in}}%
\pgfusepath{stroke,fill}%
}%
\begin{pgfscope}%
\pgfsys@transformshift{0.906285in}{2.196916in}%
\pgfsys@useobject{currentmarker}{}%
\end{pgfscope}%
\end{pgfscope}%
\begin{pgfscope}%
\pgfsetbuttcap%
\pgfsetroundjoin%
\definecolor{currentfill}{rgb}{0.000000,0.000000,0.000000}%
\pgfsetfillcolor{currentfill}%
\pgfsetlinewidth{0.501875pt}%
\definecolor{currentstroke}{rgb}{0.000000,0.000000,0.000000}%
\pgfsetstrokecolor{currentstroke}%
\pgfsetdash{}{0pt}%
\pgfsys@defobject{currentmarker}{\pgfqpoint{0.000000in}{0.000000in}}{\pgfqpoint{0.000000in}{0.020833in}}{%
\pgfpathmoveto{\pgfqpoint{0.000000in}{0.000000in}}%
\pgfpathlineto{\pgfqpoint{0.000000in}{0.020833in}}%
\pgfusepath{stroke,fill}%
}%
\begin{pgfscope}%
\pgfsys@transformshift{1.059552in}{0.431673in}%
\pgfsys@useobject{currentmarker}{}%
\end{pgfscope}%
\end{pgfscope}%
\begin{pgfscope}%
\pgfsetbuttcap%
\pgfsetroundjoin%
\definecolor{currentfill}{rgb}{0.000000,0.000000,0.000000}%
\pgfsetfillcolor{currentfill}%
\pgfsetlinewidth{0.501875pt}%
\definecolor{currentstroke}{rgb}{0.000000,0.000000,0.000000}%
\pgfsetstrokecolor{currentstroke}%
\pgfsetdash{}{0pt}%
\pgfsys@defobject{currentmarker}{\pgfqpoint{0.000000in}{-0.020833in}}{\pgfqpoint{0.000000in}{0.000000in}}{%
\pgfpathmoveto{\pgfqpoint{0.000000in}{0.000000in}}%
\pgfpathlineto{\pgfqpoint{0.000000in}{-0.020833in}}%
\pgfusepath{stroke,fill}%
}%
\begin{pgfscope}%
\pgfsys@transformshift{1.059552in}{2.196916in}%
\pgfsys@useobject{currentmarker}{}%
\end{pgfscope}%
\end{pgfscope}%
\begin{pgfscope}%
\pgfsetbuttcap%
\pgfsetroundjoin%
\definecolor{currentfill}{rgb}{0.000000,0.000000,0.000000}%
\pgfsetfillcolor{currentfill}%
\pgfsetlinewidth{0.501875pt}%
\definecolor{currentstroke}{rgb}{0.000000,0.000000,0.000000}%
\pgfsetstrokecolor{currentstroke}%
\pgfsetdash{}{0pt}%
\pgfsys@defobject{currentmarker}{\pgfqpoint{0.000000in}{0.000000in}}{\pgfqpoint{0.000000in}{0.020833in}}{%
\pgfpathmoveto{\pgfqpoint{0.000000in}{0.000000in}}%
\pgfpathlineto{\pgfqpoint{0.000000in}{0.020833in}}%
\pgfusepath{stroke,fill}%
}%
\begin{pgfscope}%
\pgfsys@transformshift{1.366085in}{0.431673in}%
\pgfsys@useobject{currentmarker}{}%
\end{pgfscope}%
\end{pgfscope}%
\begin{pgfscope}%
\pgfsetbuttcap%
\pgfsetroundjoin%
\definecolor{currentfill}{rgb}{0.000000,0.000000,0.000000}%
\pgfsetfillcolor{currentfill}%
\pgfsetlinewidth{0.501875pt}%
\definecolor{currentstroke}{rgb}{0.000000,0.000000,0.000000}%
\pgfsetstrokecolor{currentstroke}%
\pgfsetdash{}{0pt}%
\pgfsys@defobject{currentmarker}{\pgfqpoint{0.000000in}{-0.020833in}}{\pgfqpoint{0.000000in}{0.000000in}}{%
\pgfpathmoveto{\pgfqpoint{0.000000in}{0.000000in}}%
\pgfpathlineto{\pgfqpoint{0.000000in}{-0.020833in}}%
\pgfusepath{stroke,fill}%
}%
\begin{pgfscope}%
\pgfsys@transformshift{1.366085in}{2.196916in}%
\pgfsys@useobject{currentmarker}{}%
\end{pgfscope}%
\end{pgfscope}%
\begin{pgfscope}%
\pgfsetbuttcap%
\pgfsetroundjoin%
\definecolor{currentfill}{rgb}{0.000000,0.000000,0.000000}%
\pgfsetfillcolor{currentfill}%
\pgfsetlinewidth{0.501875pt}%
\definecolor{currentstroke}{rgb}{0.000000,0.000000,0.000000}%
\pgfsetstrokecolor{currentstroke}%
\pgfsetdash{}{0pt}%
\pgfsys@defobject{currentmarker}{\pgfqpoint{0.000000in}{0.000000in}}{\pgfqpoint{0.000000in}{0.020833in}}{%
\pgfpathmoveto{\pgfqpoint{0.000000in}{0.000000in}}%
\pgfpathlineto{\pgfqpoint{0.000000in}{0.020833in}}%
\pgfusepath{stroke,fill}%
}%
\begin{pgfscope}%
\pgfsys@transformshift{1.519352in}{0.431673in}%
\pgfsys@useobject{currentmarker}{}%
\end{pgfscope}%
\end{pgfscope}%
\begin{pgfscope}%
\pgfsetbuttcap%
\pgfsetroundjoin%
\definecolor{currentfill}{rgb}{0.000000,0.000000,0.000000}%
\pgfsetfillcolor{currentfill}%
\pgfsetlinewidth{0.501875pt}%
\definecolor{currentstroke}{rgb}{0.000000,0.000000,0.000000}%
\pgfsetstrokecolor{currentstroke}%
\pgfsetdash{}{0pt}%
\pgfsys@defobject{currentmarker}{\pgfqpoint{0.000000in}{-0.020833in}}{\pgfqpoint{0.000000in}{0.000000in}}{%
\pgfpathmoveto{\pgfqpoint{0.000000in}{0.000000in}}%
\pgfpathlineto{\pgfqpoint{0.000000in}{-0.020833in}}%
\pgfusepath{stroke,fill}%
}%
\begin{pgfscope}%
\pgfsys@transformshift{1.519352in}{2.196916in}%
\pgfsys@useobject{currentmarker}{}%
\end{pgfscope}%
\end{pgfscope}%
\begin{pgfscope}%
\pgfsetbuttcap%
\pgfsetroundjoin%
\definecolor{currentfill}{rgb}{0.000000,0.000000,0.000000}%
\pgfsetfillcolor{currentfill}%
\pgfsetlinewidth{0.501875pt}%
\definecolor{currentstroke}{rgb}{0.000000,0.000000,0.000000}%
\pgfsetstrokecolor{currentstroke}%
\pgfsetdash{}{0pt}%
\pgfsys@defobject{currentmarker}{\pgfqpoint{0.000000in}{0.000000in}}{\pgfqpoint{0.000000in}{0.020833in}}{%
\pgfpathmoveto{\pgfqpoint{0.000000in}{0.000000in}}%
\pgfpathlineto{\pgfqpoint{0.000000in}{0.020833in}}%
\pgfusepath{stroke,fill}%
}%
\begin{pgfscope}%
\pgfsys@transformshift{1.672618in}{0.431673in}%
\pgfsys@useobject{currentmarker}{}%
\end{pgfscope}%
\end{pgfscope}%
\begin{pgfscope}%
\pgfsetbuttcap%
\pgfsetroundjoin%
\definecolor{currentfill}{rgb}{0.000000,0.000000,0.000000}%
\pgfsetfillcolor{currentfill}%
\pgfsetlinewidth{0.501875pt}%
\definecolor{currentstroke}{rgb}{0.000000,0.000000,0.000000}%
\pgfsetstrokecolor{currentstroke}%
\pgfsetdash{}{0pt}%
\pgfsys@defobject{currentmarker}{\pgfqpoint{0.000000in}{-0.020833in}}{\pgfqpoint{0.000000in}{0.000000in}}{%
\pgfpathmoveto{\pgfqpoint{0.000000in}{0.000000in}}%
\pgfpathlineto{\pgfqpoint{0.000000in}{-0.020833in}}%
\pgfusepath{stroke,fill}%
}%
\begin{pgfscope}%
\pgfsys@transformshift{1.672618in}{2.196916in}%
\pgfsys@useobject{currentmarker}{}%
\end{pgfscope}%
\end{pgfscope}%
\begin{pgfscope}%
\pgfsetbuttcap%
\pgfsetroundjoin%
\definecolor{currentfill}{rgb}{0.000000,0.000000,0.000000}%
\pgfsetfillcolor{currentfill}%
\pgfsetlinewidth{0.501875pt}%
\definecolor{currentstroke}{rgb}{0.000000,0.000000,0.000000}%
\pgfsetstrokecolor{currentstroke}%
\pgfsetdash{}{0pt}%
\pgfsys@defobject{currentmarker}{\pgfqpoint{0.000000in}{0.000000in}}{\pgfqpoint{0.000000in}{0.020833in}}{%
\pgfpathmoveto{\pgfqpoint{0.000000in}{0.000000in}}%
\pgfpathlineto{\pgfqpoint{0.000000in}{0.020833in}}%
\pgfusepath{stroke,fill}%
}%
\begin{pgfscope}%
\pgfsys@transformshift{1.979152in}{0.431673in}%
\pgfsys@useobject{currentmarker}{}%
\end{pgfscope}%
\end{pgfscope}%
\begin{pgfscope}%
\pgfsetbuttcap%
\pgfsetroundjoin%
\definecolor{currentfill}{rgb}{0.000000,0.000000,0.000000}%
\pgfsetfillcolor{currentfill}%
\pgfsetlinewidth{0.501875pt}%
\definecolor{currentstroke}{rgb}{0.000000,0.000000,0.000000}%
\pgfsetstrokecolor{currentstroke}%
\pgfsetdash{}{0pt}%
\pgfsys@defobject{currentmarker}{\pgfqpoint{0.000000in}{-0.020833in}}{\pgfqpoint{0.000000in}{0.000000in}}{%
\pgfpathmoveto{\pgfqpoint{0.000000in}{0.000000in}}%
\pgfpathlineto{\pgfqpoint{0.000000in}{-0.020833in}}%
\pgfusepath{stroke,fill}%
}%
\begin{pgfscope}%
\pgfsys@transformshift{1.979152in}{2.196916in}%
\pgfsys@useobject{currentmarker}{}%
\end{pgfscope}%
\end{pgfscope}%
\begin{pgfscope}%
\pgfsetbuttcap%
\pgfsetroundjoin%
\definecolor{currentfill}{rgb}{0.000000,0.000000,0.000000}%
\pgfsetfillcolor{currentfill}%
\pgfsetlinewidth{0.501875pt}%
\definecolor{currentstroke}{rgb}{0.000000,0.000000,0.000000}%
\pgfsetstrokecolor{currentstroke}%
\pgfsetdash{}{0pt}%
\pgfsys@defobject{currentmarker}{\pgfqpoint{0.000000in}{0.000000in}}{\pgfqpoint{0.000000in}{0.020833in}}{%
\pgfpathmoveto{\pgfqpoint{0.000000in}{0.000000in}}%
\pgfpathlineto{\pgfqpoint{0.000000in}{0.020833in}}%
\pgfusepath{stroke,fill}%
}%
\begin{pgfscope}%
\pgfsys@transformshift{2.132418in}{0.431673in}%
\pgfsys@useobject{currentmarker}{}%
\end{pgfscope}%
\end{pgfscope}%
\begin{pgfscope}%
\pgfsetbuttcap%
\pgfsetroundjoin%
\definecolor{currentfill}{rgb}{0.000000,0.000000,0.000000}%
\pgfsetfillcolor{currentfill}%
\pgfsetlinewidth{0.501875pt}%
\definecolor{currentstroke}{rgb}{0.000000,0.000000,0.000000}%
\pgfsetstrokecolor{currentstroke}%
\pgfsetdash{}{0pt}%
\pgfsys@defobject{currentmarker}{\pgfqpoint{0.000000in}{-0.020833in}}{\pgfqpoint{0.000000in}{0.000000in}}{%
\pgfpathmoveto{\pgfqpoint{0.000000in}{0.000000in}}%
\pgfpathlineto{\pgfqpoint{0.000000in}{-0.020833in}}%
\pgfusepath{stroke,fill}%
}%
\begin{pgfscope}%
\pgfsys@transformshift{2.132418in}{2.196916in}%
\pgfsys@useobject{currentmarker}{}%
\end{pgfscope}%
\end{pgfscope}%
\begin{pgfscope}%
\pgfsetbuttcap%
\pgfsetroundjoin%
\definecolor{currentfill}{rgb}{0.000000,0.000000,0.000000}%
\pgfsetfillcolor{currentfill}%
\pgfsetlinewidth{0.501875pt}%
\definecolor{currentstroke}{rgb}{0.000000,0.000000,0.000000}%
\pgfsetstrokecolor{currentstroke}%
\pgfsetdash{}{0pt}%
\pgfsys@defobject{currentmarker}{\pgfqpoint{0.000000in}{0.000000in}}{\pgfqpoint{0.000000in}{0.020833in}}{%
\pgfpathmoveto{\pgfqpoint{0.000000in}{0.000000in}}%
\pgfpathlineto{\pgfqpoint{0.000000in}{0.020833in}}%
\pgfusepath{stroke,fill}%
}%
\begin{pgfscope}%
\pgfsys@transformshift{2.285685in}{0.431673in}%
\pgfsys@useobject{currentmarker}{}%
\end{pgfscope}%
\end{pgfscope}%
\begin{pgfscope}%
\pgfsetbuttcap%
\pgfsetroundjoin%
\definecolor{currentfill}{rgb}{0.000000,0.000000,0.000000}%
\pgfsetfillcolor{currentfill}%
\pgfsetlinewidth{0.501875pt}%
\definecolor{currentstroke}{rgb}{0.000000,0.000000,0.000000}%
\pgfsetstrokecolor{currentstroke}%
\pgfsetdash{}{0pt}%
\pgfsys@defobject{currentmarker}{\pgfqpoint{0.000000in}{-0.020833in}}{\pgfqpoint{0.000000in}{0.000000in}}{%
\pgfpathmoveto{\pgfqpoint{0.000000in}{0.000000in}}%
\pgfpathlineto{\pgfqpoint{0.000000in}{-0.020833in}}%
\pgfusepath{stroke,fill}%
}%
\begin{pgfscope}%
\pgfsys@transformshift{2.285685in}{2.196916in}%
\pgfsys@useobject{currentmarker}{}%
\end{pgfscope}%
\end{pgfscope}%
\begin{pgfscope}%
\pgfsetbuttcap%
\pgfsetroundjoin%
\definecolor{currentfill}{rgb}{0.000000,0.000000,0.000000}%
\pgfsetfillcolor{currentfill}%
\pgfsetlinewidth{0.501875pt}%
\definecolor{currentstroke}{rgb}{0.000000,0.000000,0.000000}%
\pgfsetstrokecolor{currentstroke}%
\pgfsetdash{}{0pt}%
\pgfsys@defobject{currentmarker}{\pgfqpoint{0.000000in}{0.000000in}}{\pgfqpoint{0.000000in}{0.020833in}}{%
\pgfpathmoveto{\pgfqpoint{0.000000in}{0.000000in}}%
\pgfpathlineto{\pgfqpoint{0.000000in}{0.020833in}}%
\pgfusepath{stroke,fill}%
}%
\begin{pgfscope}%
\pgfsys@transformshift{2.592218in}{0.431673in}%
\pgfsys@useobject{currentmarker}{}%
\end{pgfscope}%
\end{pgfscope}%
\begin{pgfscope}%
\pgfsetbuttcap%
\pgfsetroundjoin%
\definecolor{currentfill}{rgb}{0.000000,0.000000,0.000000}%
\pgfsetfillcolor{currentfill}%
\pgfsetlinewidth{0.501875pt}%
\definecolor{currentstroke}{rgb}{0.000000,0.000000,0.000000}%
\pgfsetstrokecolor{currentstroke}%
\pgfsetdash{}{0pt}%
\pgfsys@defobject{currentmarker}{\pgfqpoint{0.000000in}{-0.020833in}}{\pgfqpoint{0.000000in}{0.000000in}}{%
\pgfpathmoveto{\pgfqpoint{0.000000in}{0.000000in}}%
\pgfpathlineto{\pgfqpoint{0.000000in}{-0.020833in}}%
\pgfusepath{stroke,fill}%
}%
\begin{pgfscope}%
\pgfsys@transformshift{2.592218in}{2.196916in}%
\pgfsys@useobject{currentmarker}{}%
\end{pgfscope}%
\end{pgfscope}%
\begin{pgfscope}%
\pgfsetbuttcap%
\pgfsetroundjoin%
\definecolor{currentfill}{rgb}{0.000000,0.000000,0.000000}%
\pgfsetfillcolor{currentfill}%
\pgfsetlinewidth{0.501875pt}%
\definecolor{currentstroke}{rgb}{0.000000,0.000000,0.000000}%
\pgfsetstrokecolor{currentstroke}%
\pgfsetdash{}{0pt}%
\pgfsys@defobject{currentmarker}{\pgfqpoint{0.000000in}{0.000000in}}{\pgfqpoint{0.000000in}{0.020833in}}{%
\pgfpathmoveto{\pgfqpoint{0.000000in}{0.000000in}}%
\pgfpathlineto{\pgfqpoint{0.000000in}{0.020833in}}%
\pgfusepath{stroke,fill}%
}%
\begin{pgfscope}%
\pgfsys@transformshift{2.745485in}{0.431673in}%
\pgfsys@useobject{currentmarker}{}%
\end{pgfscope}%
\end{pgfscope}%
\begin{pgfscope}%
\pgfsetbuttcap%
\pgfsetroundjoin%
\definecolor{currentfill}{rgb}{0.000000,0.000000,0.000000}%
\pgfsetfillcolor{currentfill}%
\pgfsetlinewidth{0.501875pt}%
\definecolor{currentstroke}{rgb}{0.000000,0.000000,0.000000}%
\pgfsetstrokecolor{currentstroke}%
\pgfsetdash{}{0pt}%
\pgfsys@defobject{currentmarker}{\pgfqpoint{0.000000in}{-0.020833in}}{\pgfqpoint{0.000000in}{0.000000in}}{%
\pgfpathmoveto{\pgfqpoint{0.000000in}{0.000000in}}%
\pgfpathlineto{\pgfqpoint{0.000000in}{-0.020833in}}%
\pgfusepath{stroke,fill}%
}%
\begin{pgfscope}%
\pgfsys@transformshift{2.745485in}{2.196916in}%
\pgfsys@useobject{currentmarker}{}%
\end{pgfscope}%
\end{pgfscope}%
\begin{pgfscope}%
\pgfsetbuttcap%
\pgfsetroundjoin%
\definecolor{currentfill}{rgb}{0.000000,0.000000,0.000000}%
\pgfsetfillcolor{currentfill}%
\pgfsetlinewidth{0.501875pt}%
\definecolor{currentstroke}{rgb}{0.000000,0.000000,0.000000}%
\pgfsetstrokecolor{currentstroke}%
\pgfsetdash{}{0pt}%
\pgfsys@defobject{currentmarker}{\pgfqpoint{0.000000in}{0.000000in}}{\pgfqpoint{0.000000in}{0.020833in}}{%
\pgfpathmoveto{\pgfqpoint{0.000000in}{0.000000in}}%
\pgfpathlineto{\pgfqpoint{0.000000in}{0.020833in}}%
\pgfusepath{stroke,fill}%
}%
\begin{pgfscope}%
\pgfsys@transformshift{2.898752in}{0.431673in}%
\pgfsys@useobject{currentmarker}{}%
\end{pgfscope}%
\end{pgfscope}%
\begin{pgfscope}%
\pgfsetbuttcap%
\pgfsetroundjoin%
\definecolor{currentfill}{rgb}{0.000000,0.000000,0.000000}%
\pgfsetfillcolor{currentfill}%
\pgfsetlinewidth{0.501875pt}%
\definecolor{currentstroke}{rgb}{0.000000,0.000000,0.000000}%
\pgfsetstrokecolor{currentstroke}%
\pgfsetdash{}{0pt}%
\pgfsys@defobject{currentmarker}{\pgfqpoint{0.000000in}{-0.020833in}}{\pgfqpoint{0.000000in}{0.000000in}}{%
\pgfpathmoveto{\pgfqpoint{0.000000in}{0.000000in}}%
\pgfpathlineto{\pgfqpoint{0.000000in}{-0.020833in}}%
\pgfusepath{stroke,fill}%
}%
\begin{pgfscope}%
\pgfsys@transformshift{2.898752in}{2.196916in}%
\pgfsys@useobject{currentmarker}{}%
\end{pgfscope}%
\end{pgfscope}%
\begin{pgfscope}%
\pgfsetbuttcap%
\pgfsetroundjoin%
\definecolor{currentfill}{rgb}{0.000000,0.000000,0.000000}%
\pgfsetfillcolor{currentfill}%
\pgfsetlinewidth{0.501875pt}%
\definecolor{currentstroke}{rgb}{0.000000,0.000000,0.000000}%
\pgfsetstrokecolor{currentstroke}%
\pgfsetdash{}{0pt}%
\pgfsys@defobject{currentmarker}{\pgfqpoint{0.000000in}{0.000000in}}{\pgfqpoint{0.000000in}{0.020833in}}{%
\pgfpathmoveto{\pgfqpoint{0.000000in}{0.000000in}}%
\pgfpathlineto{\pgfqpoint{0.000000in}{0.020833in}}%
\pgfusepath{stroke,fill}%
}%
\begin{pgfscope}%
\pgfsys@transformshift{3.205285in}{0.431673in}%
\pgfsys@useobject{currentmarker}{}%
\end{pgfscope}%
\end{pgfscope}%
\begin{pgfscope}%
\pgfsetbuttcap%
\pgfsetroundjoin%
\definecolor{currentfill}{rgb}{0.000000,0.000000,0.000000}%
\pgfsetfillcolor{currentfill}%
\pgfsetlinewidth{0.501875pt}%
\definecolor{currentstroke}{rgb}{0.000000,0.000000,0.000000}%
\pgfsetstrokecolor{currentstroke}%
\pgfsetdash{}{0pt}%
\pgfsys@defobject{currentmarker}{\pgfqpoint{0.000000in}{-0.020833in}}{\pgfqpoint{0.000000in}{0.000000in}}{%
\pgfpathmoveto{\pgfqpoint{0.000000in}{0.000000in}}%
\pgfpathlineto{\pgfqpoint{0.000000in}{-0.020833in}}%
\pgfusepath{stroke,fill}%
}%
\begin{pgfscope}%
\pgfsys@transformshift{3.205285in}{2.196916in}%
\pgfsys@useobject{currentmarker}{}%
\end{pgfscope}%
\end{pgfscope}%
\begin{pgfscope}%
\pgfsetbuttcap%
\pgfsetroundjoin%
\definecolor{currentfill}{rgb}{0.000000,0.000000,0.000000}%
\pgfsetfillcolor{currentfill}%
\pgfsetlinewidth{0.501875pt}%
\definecolor{currentstroke}{rgb}{0.000000,0.000000,0.000000}%
\pgfsetstrokecolor{currentstroke}%
\pgfsetdash{}{0pt}%
\pgfsys@defobject{currentmarker}{\pgfqpoint{0.000000in}{0.000000in}}{\pgfqpoint{0.000000in}{0.020833in}}{%
\pgfpathmoveto{\pgfqpoint{0.000000in}{0.000000in}}%
\pgfpathlineto{\pgfqpoint{0.000000in}{0.020833in}}%
\pgfusepath{stroke,fill}%
}%
\begin{pgfscope}%
\pgfsys@transformshift{3.358552in}{0.431673in}%
\pgfsys@useobject{currentmarker}{}%
\end{pgfscope}%
\end{pgfscope}%
\begin{pgfscope}%
\pgfsetbuttcap%
\pgfsetroundjoin%
\definecolor{currentfill}{rgb}{0.000000,0.000000,0.000000}%
\pgfsetfillcolor{currentfill}%
\pgfsetlinewidth{0.501875pt}%
\definecolor{currentstroke}{rgb}{0.000000,0.000000,0.000000}%
\pgfsetstrokecolor{currentstroke}%
\pgfsetdash{}{0pt}%
\pgfsys@defobject{currentmarker}{\pgfqpoint{0.000000in}{-0.020833in}}{\pgfqpoint{0.000000in}{0.000000in}}{%
\pgfpathmoveto{\pgfqpoint{0.000000in}{0.000000in}}%
\pgfpathlineto{\pgfqpoint{0.000000in}{-0.020833in}}%
\pgfusepath{stroke,fill}%
}%
\begin{pgfscope}%
\pgfsys@transformshift{3.358552in}{2.196916in}%
\pgfsys@useobject{currentmarker}{}%
\end{pgfscope}%
\end{pgfscope}%
\begin{pgfscope}%
\pgfsetbuttcap%
\pgfsetroundjoin%
\definecolor{currentfill}{rgb}{0.000000,0.000000,0.000000}%
\pgfsetfillcolor{currentfill}%
\pgfsetlinewidth{0.501875pt}%
\definecolor{currentstroke}{rgb}{0.000000,0.000000,0.000000}%
\pgfsetstrokecolor{currentstroke}%
\pgfsetdash{}{0pt}%
\pgfsys@defobject{currentmarker}{\pgfqpoint{0.000000in}{0.000000in}}{\pgfqpoint{0.000000in}{0.020833in}}{%
\pgfpathmoveto{\pgfqpoint{0.000000in}{0.000000in}}%
\pgfpathlineto{\pgfqpoint{0.000000in}{0.020833in}}%
\pgfusepath{stroke,fill}%
}%
\begin{pgfscope}%
\pgfsys@transformshift{3.511818in}{0.431673in}%
\pgfsys@useobject{currentmarker}{}%
\end{pgfscope}%
\end{pgfscope}%
\begin{pgfscope}%
\pgfsetbuttcap%
\pgfsetroundjoin%
\definecolor{currentfill}{rgb}{0.000000,0.000000,0.000000}%
\pgfsetfillcolor{currentfill}%
\pgfsetlinewidth{0.501875pt}%
\definecolor{currentstroke}{rgb}{0.000000,0.000000,0.000000}%
\pgfsetstrokecolor{currentstroke}%
\pgfsetdash{}{0pt}%
\pgfsys@defobject{currentmarker}{\pgfqpoint{0.000000in}{-0.020833in}}{\pgfqpoint{0.000000in}{0.000000in}}{%
\pgfpathmoveto{\pgfqpoint{0.000000in}{0.000000in}}%
\pgfpathlineto{\pgfqpoint{0.000000in}{-0.020833in}}%
\pgfusepath{stroke,fill}%
}%
\begin{pgfscope}%
\pgfsys@transformshift{3.511818in}{2.196916in}%
\pgfsys@useobject{currentmarker}{}%
\end{pgfscope}%
\end{pgfscope}%
\begin{pgfscope}%
\pgfsetbuttcap%
\pgfsetroundjoin%
\definecolor{currentfill}{rgb}{0.000000,0.000000,0.000000}%
\pgfsetfillcolor{currentfill}%
\pgfsetlinewidth{0.501875pt}%
\definecolor{currentstroke}{rgb}{0.000000,0.000000,0.000000}%
\pgfsetstrokecolor{currentstroke}%
\pgfsetdash{}{0pt}%
\pgfsys@defobject{currentmarker}{\pgfqpoint{0.000000in}{0.000000in}}{\pgfqpoint{0.000000in}{0.020833in}}{%
\pgfpathmoveto{\pgfqpoint{0.000000in}{0.000000in}}%
\pgfpathlineto{\pgfqpoint{0.000000in}{0.020833in}}%
\pgfusepath{stroke,fill}%
}%
\begin{pgfscope}%
\pgfsys@transformshift{3.818352in}{0.431673in}%
\pgfsys@useobject{currentmarker}{}%
\end{pgfscope}%
\end{pgfscope}%
\begin{pgfscope}%
\pgfsetbuttcap%
\pgfsetroundjoin%
\definecolor{currentfill}{rgb}{0.000000,0.000000,0.000000}%
\pgfsetfillcolor{currentfill}%
\pgfsetlinewidth{0.501875pt}%
\definecolor{currentstroke}{rgb}{0.000000,0.000000,0.000000}%
\pgfsetstrokecolor{currentstroke}%
\pgfsetdash{}{0pt}%
\pgfsys@defobject{currentmarker}{\pgfqpoint{0.000000in}{-0.020833in}}{\pgfqpoint{0.000000in}{0.000000in}}{%
\pgfpathmoveto{\pgfqpoint{0.000000in}{0.000000in}}%
\pgfpathlineto{\pgfqpoint{0.000000in}{-0.020833in}}%
\pgfusepath{stroke,fill}%
}%
\begin{pgfscope}%
\pgfsys@transformshift{3.818352in}{2.196916in}%
\pgfsys@useobject{currentmarker}{}%
\end{pgfscope}%
\end{pgfscope}%
\begin{pgfscope}%
\pgfsetbuttcap%
\pgfsetroundjoin%
\definecolor{currentfill}{rgb}{0.000000,0.000000,0.000000}%
\pgfsetfillcolor{currentfill}%
\pgfsetlinewidth{0.501875pt}%
\definecolor{currentstroke}{rgb}{0.000000,0.000000,0.000000}%
\pgfsetstrokecolor{currentstroke}%
\pgfsetdash{}{0pt}%
\pgfsys@defobject{currentmarker}{\pgfqpoint{0.000000in}{0.000000in}}{\pgfqpoint{0.000000in}{0.020833in}}{%
\pgfpathmoveto{\pgfqpoint{0.000000in}{0.000000in}}%
\pgfpathlineto{\pgfqpoint{0.000000in}{0.020833in}}%
\pgfusepath{stroke,fill}%
}%
\begin{pgfscope}%
\pgfsys@transformshift{3.971619in}{0.431673in}%
\pgfsys@useobject{currentmarker}{}%
\end{pgfscope}%
\end{pgfscope}%
\begin{pgfscope}%
\pgfsetbuttcap%
\pgfsetroundjoin%
\definecolor{currentfill}{rgb}{0.000000,0.000000,0.000000}%
\pgfsetfillcolor{currentfill}%
\pgfsetlinewidth{0.501875pt}%
\definecolor{currentstroke}{rgb}{0.000000,0.000000,0.000000}%
\pgfsetstrokecolor{currentstroke}%
\pgfsetdash{}{0pt}%
\pgfsys@defobject{currentmarker}{\pgfqpoint{0.000000in}{-0.020833in}}{\pgfqpoint{0.000000in}{0.000000in}}{%
\pgfpathmoveto{\pgfqpoint{0.000000in}{0.000000in}}%
\pgfpathlineto{\pgfqpoint{0.000000in}{-0.020833in}}%
\pgfusepath{stroke,fill}%
}%
\begin{pgfscope}%
\pgfsys@transformshift{3.971619in}{2.196916in}%
\pgfsys@useobject{currentmarker}{}%
\end{pgfscope}%
\end{pgfscope}%
\begin{pgfscope}%
\pgfsetbuttcap%
\pgfsetroundjoin%
\definecolor{currentfill}{rgb}{0.000000,0.000000,0.000000}%
\pgfsetfillcolor{currentfill}%
\pgfsetlinewidth{0.501875pt}%
\definecolor{currentstroke}{rgb}{0.000000,0.000000,0.000000}%
\pgfsetstrokecolor{currentstroke}%
\pgfsetdash{}{0pt}%
\pgfsys@defobject{currentmarker}{\pgfqpoint{0.000000in}{0.000000in}}{\pgfqpoint{0.000000in}{0.020833in}}{%
\pgfpathmoveto{\pgfqpoint{0.000000in}{0.000000in}}%
\pgfpathlineto{\pgfqpoint{0.000000in}{0.020833in}}%
\pgfusepath{stroke,fill}%
}%
\begin{pgfscope}%
\pgfsys@transformshift{4.124885in}{0.431673in}%
\pgfsys@useobject{currentmarker}{}%
\end{pgfscope}%
\end{pgfscope}%
\begin{pgfscope}%
\pgfsetbuttcap%
\pgfsetroundjoin%
\definecolor{currentfill}{rgb}{0.000000,0.000000,0.000000}%
\pgfsetfillcolor{currentfill}%
\pgfsetlinewidth{0.501875pt}%
\definecolor{currentstroke}{rgb}{0.000000,0.000000,0.000000}%
\pgfsetstrokecolor{currentstroke}%
\pgfsetdash{}{0pt}%
\pgfsys@defobject{currentmarker}{\pgfqpoint{0.000000in}{-0.020833in}}{\pgfqpoint{0.000000in}{0.000000in}}{%
\pgfpathmoveto{\pgfqpoint{0.000000in}{0.000000in}}%
\pgfpathlineto{\pgfqpoint{0.000000in}{-0.020833in}}%
\pgfusepath{stroke,fill}%
}%
\begin{pgfscope}%
\pgfsys@transformshift{4.124885in}{2.196916in}%
\pgfsys@useobject{currentmarker}{}%
\end{pgfscope}%
\end{pgfscope}%
\begin{pgfscope}%
\definecolor{textcolor}{rgb}{0.000000,0.000000,0.000000}%
\pgfsetstrokecolor{textcolor}%
\pgfsetfillcolor{textcolor}%
\pgftext[x=2.422177in,y=0.201367in,,top]{\color{textcolor}\rmfamily\fontsize{12.000000}{14.400000}\selectfont \(\displaystyle \gamma\)}%
\end{pgfscope}%
\begin{pgfscope}%
\pgfsetbuttcap%
\pgfsetroundjoin%
\definecolor{currentfill}{rgb}{0.000000,0.000000,0.000000}%
\pgfsetfillcolor{currentfill}%
\pgfsetlinewidth{0.501875pt}%
\definecolor{currentstroke}{rgb}{0.000000,0.000000,0.000000}%
\pgfsetstrokecolor{currentstroke}%
\pgfsetdash{}{0pt}%
\pgfsys@defobject{currentmarker}{\pgfqpoint{0.000000in}{0.000000in}}{\pgfqpoint{0.041667in}{0.000000in}}{%
\pgfpathmoveto{\pgfqpoint{0.000000in}{0.000000in}}%
\pgfpathlineto{\pgfqpoint{0.041667in}{0.000000in}}%
\pgfusepath{stroke,fill}%
}%
\begin{pgfscope}%
\pgfsys@transformshift{0.552773in}{0.441321in}%
\pgfsys@useobject{currentmarker}{}%
\end{pgfscope}%
\end{pgfscope}%
\begin{pgfscope}%
\pgfsetbuttcap%
\pgfsetroundjoin%
\definecolor{currentfill}{rgb}{0.000000,0.000000,0.000000}%
\pgfsetfillcolor{currentfill}%
\pgfsetlinewidth{0.501875pt}%
\definecolor{currentstroke}{rgb}{0.000000,0.000000,0.000000}%
\pgfsetstrokecolor{currentstroke}%
\pgfsetdash{}{0pt}%
\pgfsys@defobject{currentmarker}{\pgfqpoint{-0.041667in}{0.000000in}}{\pgfqpoint{-0.000000in}{0.000000in}}{%
\pgfpathmoveto{\pgfqpoint{-0.000000in}{0.000000in}}%
\pgfpathlineto{\pgfqpoint{-0.041667in}{0.000000in}}%
\pgfusepath{stroke,fill}%
}%
\begin{pgfscope}%
\pgfsys@transformshift{4.291580in}{0.441321in}%
\pgfsys@useobject{currentmarker}{}%
\end{pgfscope}%
\end{pgfscope}%
\begin{pgfscope}%
\definecolor{textcolor}{rgb}{0.000000,0.000000,0.000000}%
\pgfsetstrokecolor{textcolor}%
\pgfsetfillcolor{textcolor}%
\pgftext[x=0.257248in, y=0.393103in, left, base]{\color{textcolor}\rmfamily\fontsize{10.000000}{12.000000}\selectfont \(\displaystyle {\ensuremath{-}50}\)}%
\end{pgfscope}%
\begin{pgfscope}%
\pgfsetbuttcap%
\pgfsetroundjoin%
\definecolor{currentfill}{rgb}{0.000000,0.000000,0.000000}%
\pgfsetfillcolor{currentfill}%
\pgfsetlinewidth{0.501875pt}%
\definecolor{currentstroke}{rgb}{0.000000,0.000000,0.000000}%
\pgfsetstrokecolor{currentstroke}%
\pgfsetdash{}{0pt}%
\pgfsys@defobject{currentmarker}{\pgfqpoint{0.000000in}{0.000000in}}{\pgfqpoint{0.041667in}{0.000000in}}{%
\pgfpathmoveto{\pgfqpoint{0.000000in}{0.000000in}}%
\pgfpathlineto{\pgfqpoint{0.041667in}{0.000000in}}%
\pgfusepath{stroke,fill}%
}%
\begin{pgfscope}%
\pgfsys@transformshift{0.552773in}{0.740787in}%
\pgfsys@useobject{currentmarker}{}%
\end{pgfscope}%
\end{pgfscope}%
\begin{pgfscope}%
\pgfsetbuttcap%
\pgfsetroundjoin%
\definecolor{currentfill}{rgb}{0.000000,0.000000,0.000000}%
\pgfsetfillcolor{currentfill}%
\pgfsetlinewidth{0.501875pt}%
\definecolor{currentstroke}{rgb}{0.000000,0.000000,0.000000}%
\pgfsetstrokecolor{currentstroke}%
\pgfsetdash{}{0pt}%
\pgfsys@defobject{currentmarker}{\pgfqpoint{-0.041667in}{0.000000in}}{\pgfqpoint{-0.000000in}{0.000000in}}{%
\pgfpathmoveto{\pgfqpoint{-0.000000in}{0.000000in}}%
\pgfpathlineto{\pgfqpoint{-0.041667in}{0.000000in}}%
\pgfusepath{stroke,fill}%
}%
\begin{pgfscope}%
\pgfsys@transformshift{4.291580in}{0.740787in}%
\pgfsys@useobject{currentmarker}{}%
\end{pgfscope}%
\end{pgfscope}%
\begin{pgfscope}%
\definecolor{textcolor}{rgb}{0.000000,0.000000,0.000000}%
\pgfsetstrokecolor{textcolor}%
\pgfsetfillcolor{textcolor}%
\pgftext[x=0.434718in, y=0.692570in, left, base]{\color{textcolor}\rmfamily\fontsize{10.000000}{12.000000}\selectfont \(\displaystyle {0}\)}%
\end{pgfscope}%
\begin{pgfscope}%
\pgfsetbuttcap%
\pgfsetroundjoin%
\definecolor{currentfill}{rgb}{0.000000,0.000000,0.000000}%
\pgfsetfillcolor{currentfill}%
\pgfsetlinewidth{0.501875pt}%
\definecolor{currentstroke}{rgb}{0.000000,0.000000,0.000000}%
\pgfsetstrokecolor{currentstroke}%
\pgfsetdash{}{0pt}%
\pgfsys@defobject{currentmarker}{\pgfqpoint{0.000000in}{0.000000in}}{\pgfqpoint{0.041667in}{0.000000in}}{%
\pgfpathmoveto{\pgfqpoint{0.000000in}{0.000000in}}%
\pgfpathlineto{\pgfqpoint{0.041667in}{0.000000in}}%
\pgfusepath{stroke,fill}%
}%
\begin{pgfscope}%
\pgfsys@transformshift{0.552773in}{1.040254in}%
\pgfsys@useobject{currentmarker}{}%
\end{pgfscope}%
\end{pgfscope}%
\begin{pgfscope}%
\pgfsetbuttcap%
\pgfsetroundjoin%
\definecolor{currentfill}{rgb}{0.000000,0.000000,0.000000}%
\pgfsetfillcolor{currentfill}%
\pgfsetlinewidth{0.501875pt}%
\definecolor{currentstroke}{rgb}{0.000000,0.000000,0.000000}%
\pgfsetstrokecolor{currentstroke}%
\pgfsetdash{}{0pt}%
\pgfsys@defobject{currentmarker}{\pgfqpoint{-0.041667in}{0.000000in}}{\pgfqpoint{-0.000000in}{0.000000in}}{%
\pgfpathmoveto{\pgfqpoint{-0.000000in}{0.000000in}}%
\pgfpathlineto{\pgfqpoint{-0.041667in}{0.000000in}}%
\pgfusepath{stroke,fill}%
}%
\begin{pgfscope}%
\pgfsys@transformshift{4.291580in}{1.040254in}%
\pgfsys@useobject{currentmarker}{}%
\end{pgfscope}%
\end{pgfscope}%
\begin{pgfscope}%
\definecolor{textcolor}{rgb}{0.000000,0.000000,0.000000}%
\pgfsetstrokecolor{textcolor}%
\pgfsetfillcolor{textcolor}%
\pgftext[x=0.365273in, y=0.992036in, left, base]{\color{textcolor}\rmfamily\fontsize{10.000000}{12.000000}\selectfont \(\displaystyle {50}\)}%
\end{pgfscope}%
\begin{pgfscope}%
\pgfsetbuttcap%
\pgfsetroundjoin%
\definecolor{currentfill}{rgb}{0.000000,0.000000,0.000000}%
\pgfsetfillcolor{currentfill}%
\pgfsetlinewidth{0.501875pt}%
\definecolor{currentstroke}{rgb}{0.000000,0.000000,0.000000}%
\pgfsetstrokecolor{currentstroke}%
\pgfsetdash{}{0pt}%
\pgfsys@defobject{currentmarker}{\pgfqpoint{0.000000in}{0.000000in}}{\pgfqpoint{0.041667in}{0.000000in}}{%
\pgfpathmoveto{\pgfqpoint{0.000000in}{0.000000in}}%
\pgfpathlineto{\pgfqpoint{0.041667in}{0.000000in}}%
\pgfusepath{stroke,fill}%
}%
\begin{pgfscope}%
\pgfsys@transformshift{0.552773in}{1.339721in}%
\pgfsys@useobject{currentmarker}{}%
\end{pgfscope}%
\end{pgfscope}%
\begin{pgfscope}%
\pgfsetbuttcap%
\pgfsetroundjoin%
\definecolor{currentfill}{rgb}{0.000000,0.000000,0.000000}%
\pgfsetfillcolor{currentfill}%
\pgfsetlinewidth{0.501875pt}%
\definecolor{currentstroke}{rgb}{0.000000,0.000000,0.000000}%
\pgfsetstrokecolor{currentstroke}%
\pgfsetdash{}{0pt}%
\pgfsys@defobject{currentmarker}{\pgfqpoint{-0.041667in}{0.000000in}}{\pgfqpoint{-0.000000in}{0.000000in}}{%
\pgfpathmoveto{\pgfqpoint{-0.000000in}{0.000000in}}%
\pgfpathlineto{\pgfqpoint{-0.041667in}{0.000000in}}%
\pgfusepath{stroke,fill}%
}%
\begin{pgfscope}%
\pgfsys@transformshift{4.291580in}{1.339721in}%
\pgfsys@useobject{currentmarker}{}%
\end{pgfscope}%
\end{pgfscope}%
\begin{pgfscope}%
\definecolor{textcolor}{rgb}{0.000000,0.000000,0.000000}%
\pgfsetstrokecolor{textcolor}%
\pgfsetfillcolor{textcolor}%
\pgftext[x=0.295828in, y=1.291503in, left, base]{\color{textcolor}\rmfamily\fontsize{10.000000}{12.000000}\selectfont \(\displaystyle {100}\)}%
\end{pgfscope}%
\begin{pgfscope}%
\pgfsetbuttcap%
\pgfsetroundjoin%
\definecolor{currentfill}{rgb}{0.000000,0.000000,0.000000}%
\pgfsetfillcolor{currentfill}%
\pgfsetlinewidth{0.501875pt}%
\definecolor{currentstroke}{rgb}{0.000000,0.000000,0.000000}%
\pgfsetstrokecolor{currentstroke}%
\pgfsetdash{}{0pt}%
\pgfsys@defobject{currentmarker}{\pgfqpoint{0.000000in}{0.000000in}}{\pgfqpoint{0.041667in}{0.000000in}}{%
\pgfpathmoveto{\pgfqpoint{0.000000in}{0.000000in}}%
\pgfpathlineto{\pgfqpoint{0.041667in}{0.000000in}}%
\pgfusepath{stroke,fill}%
}%
\begin{pgfscope}%
\pgfsys@transformshift{0.552773in}{1.639188in}%
\pgfsys@useobject{currentmarker}{}%
\end{pgfscope}%
\end{pgfscope}%
\begin{pgfscope}%
\pgfsetbuttcap%
\pgfsetroundjoin%
\definecolor{currentfill}{rgb}{0.000000,0.000000,0.000000}%
\pgfsetfillcolor{currentfill}%
\pgfsetlinewidth{0.501875pt}%
\definecolor{currentstroke}{rgb}{0.000000,0.000000,0.000000}%
\pgfsetstrokecolor{currentstroke}%
\pgfsetdash{}{0pt}%
\pgfsys@defobject{currentmarker}{\pgfqpoint{-0.041667in}{0.000000in}}{\pgfqpoint{-0.000000in}{0.000000in}}{%
\pgfpathmoveto{\pgfqpoint{-0.000000in}{0.000000in}}%
\pgfpathlineto{\pgfqpoint{-0.041667in}{0.000000in}}%
\pgfusepath{stroke,fill}%
}%
\begin{pgfscope}%
\pgfsys@transformshift{4.291580in}{1.639188in}%
\pgfsys@useobject{currentmarker}{}%
\end{pgfscope}%
\end{pgfscope}%
\begin{pgfscope}%
\definecolor{textcolor}{rgb}{0.000000,0.000000,0.000000}%
\pgfsetstrokecolor{textcolor}%
\pgfsetfillcolor{textcolor}%
\pgftext[x=0.295828in, y=1.590970in, left, base]{\color{textcolor}\rmfamily\fontsize{10.000000}{12.000000}\selectfont \(\displaystyle {150}\)}%
\end{pgfscope}%
\begin{pgfscope}%
\pgfsetbuttcap%
\pgfsetroundjoin%
\definecolor{currentfill}{rgb}{0.000000,0.000000,0.000000}%
\pgfsetfillcolor{currentfill}%
\pgfsetlinewidth{0.501875pt}%
\definecolor{currentstroke}{rgb}{0.000000,0.000000,0.000000}%
\pgfsetstrokecolor{currentstroke}%
\pgfsetdash{}{0pt}%
\pgfsys@defobject{currentmarker}{\pgfqpoint{0.000000in}{0.000000in}}{\pgfqpoint{0.041667in}{0.000000in}}{%
\pgfpathmoveto{\pgfqpoint{0.000000in}{0.000000in}}%
\pgfpathlineto{\pgfqpoint{0.041667in}{0.000000in}}%
\pgfusepath{stroke,fill}%
}%
\begin{pgfscope}%
\pgfsys@transformshift{0.552773in}{1.938654in}%
\pgfsys@useobject{currentmarker}{}%
\end{pgfscope}%
\end{pgfscope}%
\begin{pgfscope}%
\pgfsetbuttcap%
\pgfsetroundjoin%
\definecolor{currentfill}{rgb}{0.000000,0.000000,0.000000}%
\pgfsetfillcolor{currentfill}%
\pgfsetlinewidth{0.501875pt}%
\definecolor{currentstroke}{rgb}{0.000000,0.000000,0.000000}%
\pgfsetstrokecolor{currentstroke}%
\pgfsetdash{}{0pt}%
\pgfsys@defobject{currentmarker}{\pgfqpoint{-0.041667in}{0.000000in}}{\pgfqpoint{-0.000000in}{0.000000in}}{%
\pgfpathmoveto{\pgfqpoint{-0.000000in}{0.000000in}}%
\pgfpathlineto{\pgfqpoint{-0.041667in}{0.000000in}}%
\pgfusepath{stroke,fill}%
}%
\begin{pgfscope}%
\pgfsys@transformshift{4.291580in}{1.938654in}%
\pgfsys@useobject{currentmarker}{}%
\end{pgfscope}%
\end{pgfscope}%
\begin{pgfscope}%
\definecolor{textcolor}{rgb}{0.000000,0.000000,0.000000}%
\pgfsetstrokecolor{textcolor}%
\pgfsetfillcolor{textcolor}%
\pgftext[x=0.295828in, y=1.890437in, left, base]{\color{textcolor}\rmfamily\fontsize{10.000000}{12.000000}\selectfont \(\displaystyle {200}\)}%
\end{pgfscope}%
\begin{pgfscope}%
\pgfsetbuttcap%
\pgfsetroundjoin%
\definecolor{currentfill}{rgb}{0.000000,0.000000,0.000000}%
\pgfsetfillcolor{currentfill}%
\pgfsetlinewidth{0.501875pt}%
\definecolor{currentstroke}{rgb}{0.000000,0.000000,0.000000}%
\pgfsetstrokecolor{currentstroke}%
\pgfsetdash{}{0pt}%
\pgfsys@defobject{currentmarker}{\pgfqpoint{0.000000in}{0.000000in}}{\pgfqpoint{0.020833in}{0.000000in}}{%
\pgfpathmoveto{\pgfqpoint{0.000000in}{0.000000in}}%
\pgfpathlineto{\pgfqpoint{0.020833in}{0.000000in}}%
\pgfusepath{stroke,fill}%
}%
\begin{pgfscope}%
\pgfsys@transformshift{0.552773in}{0.501214in}%
\pgfsys@useobject{currentmarker}{}%
\end{pgfscope}%
\end{pgfscope}%
\begin{pgfscope}%
\pgfsetbuttcap%
\pgfsetroundjoin%
\definecolor{currentfill}{rgb}{0.000000,0.000000,0.000000}%
\pgfsetfillcolor{currentfill}%
\pgfsetlinewidth{0.501875pt}%
\definecolor{currentstroke}{rgb}{0.000000,0.000000,0.000000}%
\pgfsetstrokecolor{currentstroke}%
\pgfsetdash{}{0pt}%
\pgfsys@defobject{currentmarker}{\pgfqpoint{-0.020833in}{0.000000in}}{\pgfqpoint{-0.000000in}{0.000000in}}{%
\pgfpathmoveto{\pgfqpoint{-0.000000in}{0.000000in}}%
\pgfpathlineto{\pgfqpoint{-0.020833in}{0.000000in}}%
\pgfusepath{stroke,fill}%
}%
\begin{pgfscope}%
\pgfsys@transformshift{4.291580in}{0.501214in}%
\pgfsys@useobject{currentmarker}{}%
\end{pgfscope}%
\end{pgfscope}%
\begin{pgfscope}%
\pgfsetbuttcap%
\pgfsetroundjoin%
\definecolor{currentfill}{rgb}{0.000000,0.000000,0.000000}%
\pgfsetfillcolor{currentfill}%
\pgfsetlinewidth{0.501875pt}%
\definecolor{currentstroke}{rgb}{0.000000,0.000000,0.000000}%
\pgfsetstrokecolor{currentstroke}%
\pgfsetdash{}{0pt}%
\pgfsys@defobject{currentmarker}{\pgfqpoint{0.000000in}{0.000000in}}{\pgfqpoint{0.020833in}{0.000000in}}{%
\pgfpathmoveto{\pgfqpoint{0.000000in}{0.000000in}}%
\pgfpathlineto{\pgfqpoint{0.020833in}{0.000000in}}%
\pgfusepath{stroke,fill}%
}%
\begin{pgfscope}%
\pgfsys@transformshift{0.552773in}{0.561107in}%
\pgfsys@useobject{currentmarker}{}%
\end{pgfscope}%
\end{pgfscope}%
\begin{pgfscope}%
\pgfsetbuttcap%
\pgfsetroundjoin%
\definecolor{currentfill}{rgb}{0.000000,0.000000,0.000000}%
\pgfsetfillcolor{currentfill}%
\pgfsetlinewidth{0.501875pt}%
\definecolor{currentstroke}{rgb}{0.000000,0.000000,0.000000}%
\pgfsetstrokecolor{currentstroke}%
\pgfsetdash{}{0pt}%
\pgfsys@defobject{currentmarker}{\pgfqpoint{-0.020833in}{0.000000in}}{\pgfqpoint{-0.000000in}{0.000000in}}{%
\pgfpathmoveto{\pgfqpoint{-0.000000in}{0.000000in}}%
\pgfpathlineto{\pgfqpoint{-0.020833in}{0.000000in}}%
\pgfusepath{stroke,fill}%
}%
\begin{pgfscope}%
\pgfsys@transformshift{4.291580in}{0.561107in}%
\pgfsys@useobject{currentmarker}{}%
\end{pgfscope}%
\end{pgfscope}%
\begin{pgfscope}%
\pgfsetbuttcap%
\pgfsetroundjoin%
\definecolor{currentfill}{rgb}{0.000000,0.000000,0.000000}%
\pgfsetfillcolor{currentfill}%
\pgfsetlinewidth{0.501875pt}%
\definecolor{currentstroke}{rgb}{0.000000,0.000000,0.000000}%
\pgfsetstrokecolor{currentstroke}%
\pgfsetdash{}{0pt}%
\pgfsys@defobject{currentmarker}{\pgfqpoint{0.000000in}{0.000000in}}{\pgfqpoint{0.020833in}{0.000000in}}{%
\pgfpathmoveto{\pgfqpoint{0.000000in}{0.000000in}}%
\pgfpathlineto{\pgfqpoint{0.020833in}{0.000000in}}%
\pgfusepath{stroke,fill}%
}%
\begin{pgfscope}%
\pgfsys@transformshift{0.552773in}{0.621001in}%
\pgfsys@useobject{currentmarker}{}%
\end{pgfscope}%
\end{pgfscope}%
\begin{pgfscope}%
\pgfsetbuttcap%
\pgfsetroundjoin%
\definecolor{currentfill}{rgb}{0.000000,0.000000,0.000000}%
\pgfsetfillcolor{currentfill}%
\pgfsetlinewidth{0.501875pt}%
\definecolor{currentstroke}{rgb}{0.000000,0.000000,0.000000}%
\pgfsetstrokecolor{currentstroke}%
\pgfsetdash{}{0pt}%
\pgfsys@defobject{currentmarker}{\pgfqpoint{-0.020833in}{0.000000in}}{\pgfqpoint{-0.000000in}{0.000000in}}{%
\pgfpathmoveto{\pgfqpoint{-0.000000in}{0.000000in}}%
\pgfpathlineto{\pgfqpoint{-0.020833in}{0.000000in}}%
\pgfusepath{stroke,fill}%
}%
\begin{pgfscope}%
\pgfsys@transformshift{4.291580in}{0.621001in}%
\pgfsys@useobject{currentmarker}{}%
\end{pgfscope}%
\end{pgfscope}%
\begin{pgfscope}%
\pgfsetbuttcap%
\pgfsetroundjoin%
\definecolor{currentfill}{rgb}{0.000000,0.000000,0.000000}%
\pgfsetfillcolor{currentfill}%
\pgfsetlinewidth{0.501875pt}%
\definecolor{currentstroke}{rgb}{0.000000,0.000000,0.000000}%
\pgfsetstrokecolor{currentstroke}%
\pgfsetdash{}{0pt}%
\pgfsys@defobject{currentmarker}{\pgfqpoint{0.000000in}{0.000000in}}{\pgfqpoint{0.020833in}{0.000000in}}{%
\pgfpathmoveto{\pgfqpoint{0.000000in}{0.000000in}}%
\pgfpathlineto{\pgfqpoint{0.020833in}{0.000000in}}%
\pgfusepath{stroke,fill}%
}%
\begin{pgfscope}%
\pgfsys@transformshift{0.552773in}{0.680894in}%
\pgfsys@useobject{currentmarker}{}%
\end{pgfscope}%
\end{pgfscope}%
\begin{pgfscope}%
\pgfsetbuttcap%
\pgfsetroundjoin%
\definecolor{currentfill}{rgb}{0.000000,0.000000,0.000000}%
\pgfsetfillcolor{currentfill}%
\pgfsetlinewidth{0.501875pt}%
\definecolor{currentstroke}{rgb}{0.000000,0.000000,0.000000}%
\pgfsetstrokecolor{currentstroke}%
\pgfsetdash{}{0pt}%
\pgfsys@defobject{currentmarker}{\pgfqpoint{-0.020833in}{0.000000in}}{\pgfqpoint{-0.000000in}{0.000000in}}{%
\pgfpathmoveto{\pgfqpoint{-0.000000in}{0.000000in}}%
\pgfpathlineto{\pgfqpoint{-0.020833in}{0.000000in}}%
\pgfusepath{stroke,fill}%
}%
\begin{pgfscope}%
\pgfsys@transformshift{4.291580in}{0.680894in}%
\pgfsys@useobject{currentmarker}{}%
\end{pgfscope}%
\end{pgfscope}%
\begin{pgfscope}%
\pgfsetbuttcap%
\pgfsetroundjoin%
\definecolor{currentfill}{rgb}{0.000000,0.000000,0.000000}%
\pgfsetfillcolor{currentfill}%
\pgfsetlinewidth{0.501875pt}%
\definecolor{currentstroke}{rgb}{0.000000,0.000000,0.000000}%
\pgfsetstrokecolor{currentstroke}%
\pgfsetdash{}{0pt}%
\pgfsys@defobject{currentmarker}{\pgfqpoint{0.000000in}{0.000000in}}{\pgfqpoint{0.020833in}{0.000000in}}{%
\pgfpathmoveto{\pgfqpoint{0.000000in}{0.000000in}}%
\pgfpathlineto{\pgfqpoint{0.020833in}{0.000000in}}%
\pgfusepath{stroke,fill}%
}%
\begin{pgfscope}%
\pgfsys@transformshift{0.552773in}{0.800681in}%
\pgfsys@useobject{currentmarker}{}%
\end{pgfscope}%
\end{pgfscope}%
\begin{pgfscope}%
\pgfsetbuttcap%
\pgfsetroundjoin%
\definecolor{currentfill}{rgb}{0.000000,0.000000,0.000000}%
\pgfsetfillcolor{currentfill}%
\pgfsetlinewidth{0.501875pt}%
\definecolor{currentstroke}{rgb}{0.000000,0.000000,0.000000}%
\pgfsetstrokecolor{currentstroke}%
\pgfsetdash{}{0pt}%
\pgfsys@defobject{currentmarker}{\pgfqpoint{-0.020833in}{0.000000in}}{\pgfqpoint{-0.000000in}{0.000000in}}{%
\pgfpathmoveto{\pgfqpoint{-0.000000in}{0.000000in}}%
\pgfpathlineto{\pgfqpoint{-0.020833in}{0.000000in}}%
\pgfusepath{stroke,fill}%
}%
\begin{pgfscope}%
\pgfsys@transformshift{4.291580in}{0.800681in}%
\pgfsys@useobject{currentmarker}{}%
\end{pgfscope}%
\end{pgfscope}%
\begin{pgfscope}%
\pgfsetbuttcap%
\pgfsetroundjoin%
\definecolor{currentfill}{rgb}{0.000000,0.000000,0.000000}%
\pgfsetfillcolor{currentfill}%
\pgfsetlinewidth{0.501875pt}%
\definecolor{currentstroke}{rgb}{0.000000,0.000000,0.000000}%
\pgfsetstrokecolor{currentstroke}%
\pgfsetdash{}{0pt}%
\pgfsys@defobject{currentmarker}{\pgfqpoint{0.000000in}{0.000000in}}{\pgfqpoint{0.020833in}{0.000000in}}{%
\pgfpathmoveto{\pgfqpoint{0.000000in}{0.000000in}}%
\pgfpathlineto{\pgfqpoint{0.020833in}{0.000000in}}%
\pgfusepath{stroke,fill}%
}%
\begin{pgfscope}%
\pgfsys@transformshift{0.552773in}{0.860574in}%
\pgfsys@useobject{currentmarker}{}%
\end{pgfscope}%
\end{pgfscope}%
\begin{pgfscope}%
\pgfsetbuttcap%
\pgfsetroundjoin%
\definecolor{currentfill}{rgb}{0.000000,0.000000,0.000000}%
\pgfsetfillcolor{currentfill}%
\pgfsetlinewidth{0.501875pt}%
\definecolor{currentstroke}{rgb}{0.000000,0.000000,0.000000}%
\pgfsetstrokecolor{currentstroke}%
\pgfsetdash{}{0pt}%
\pgfsys@defobject{currentmarker}{\pgfqpoint{-0.020833in}{0.000000in}}{\pgfqpoint{-0.000000in}{0.000000in}}{%
\pgfpathmoveto{\pgfqpoint{-0.000000in}{0.000000in}}%
\pgfpathlineto{\pgfqpoint{-0.020833in}{0.000000in}}%
\pgfusepath{stroke,fill}%
}%
\begin{pgfscope}%
\pgfsys@transformshift{4.291580in}{0.860574in}%
\pgfsys@useobject{currentmarker}{}%
\end{pgfscope}%
\end{pgfscope}%
\begin{pgfscope}%
\pgfsetbuttcap%
\pgfsetroundjoin%
\definecolor{currentfill}{rgb}{0.000000,0.000000,0.000000}%
\pgfsetfillcolor{currentfill}%
\pgfsetlinewidth{0.501875pt}%
\definecolor{currentstroke}{rgb}{0.000000,0.000000,0.000000}%
\pgfsetstrokecolor{currentstroke}%
\pgfsetdash{}{0pt}%
\pgfsys@defobject{currentmarker}{\pgfqpoint{0.000000in}{0.000000in}}{\pgfqpoint{0.020833in}{0.000000in}}{%
\pgfpathmoveto{\pgfqpoint{0.000000in}{0.000000in}}%
\pgfpathlineto{\pgfqpoint{0.020833in}{0.000000in}}%
\pgfusepath{stroke,fill}%
}%
\begin{pgfscope}%
\pgfsys@transformshift{0.552773in}{0.920467in}%
\pgfsys@useobject{currentmarker}{}%
\end{pgfscope}%
\end{pgfscope}%
\begin{pgfscope}%
\pgfsetbuttcap%
\pgfsetroundjoin%
\definecolor{currentfill}{rgb}{0.000000,0.000000,0.000000}%
\pgfsetfillcolor{currentfill}%
\pgfsetlinewidth{0.501875pt}%
\definecolor{currentstroke}{rgb}{0.000000,0.000000,0.000000}%
\pgfsetstrokecolor{currentstroke}%
\pgfsetdash{}{0pt}%
\pgfsys@defobject{currentmarker}{\pgfqpoint{-0.020833in}{0.000000in}}{\pgfqpoint{-0.000000in}{0.000000in}}{%
\pgfpathmoveto{\pgfqpoint{-0.000000in}{0.000000in}}%
\pgfpathlineto{\pgfqpoint{-0.020833in}{0.000000in}}%
\pgfusepath{stroke,fill}%
}%
\begin{pgfscope}%
\pgfsys@transformshift{4.291580in}{0.920467in}%
\pgfsys@useobject{currentmarker}{}%
\end{pgfscope}%
\end{pgfscope}%
\begin{pgfscope}%
\pgfsetbuttcap%
\pgfsetroundjoin%
\definecolor{currentfill}{rgb}{0.000000,0.000000,0.000000}%
\pgfsetfillcolor{currentfill}%
\pgfsetlinewidth{0.501875pt}%
\definecolor{currentstroke}{rgb}{0.000000,0.000000,0.000000}%
\pgfsetstrokecolor{currentstroke}%
\pgfsetdash{}{0pt}%
\pgfsys@defobject{currentmarker}{\pgfqpoint{0.000000in}{0.000000in}}{\pgfqpoint{0.020833in}{0.000000in}}{%
\pgfpathmoveto{\pgfqpoint{0.000000in}{0.000000in}}%
\pgfpathlineto{\pgfqpoint{0.020833in}{0.000000in}}%
\pgfusepath{stroke,fill}%
}%
\begin{pgfscope}%
\pgfsys@transformshift{0.552773in}{0.980361in}%
\pgfsys@useobject{currentmarker}{}%
\end{pgfscope}%
\end{pgfscope}%
\begin{pgfscope}%
\pgfsetbuttcap%
\pgfsetroundjoin%
\definecolor{currentfill}{rgb}{0.000000,0.000000,0.000000}%
\pgfsetfillcolor{currentfill}%
\pgfsetlinewidth{0.501875pt}%
\definecolor{currentstroke}{rgb}{0.000000,0.000000,0.000000}%
\pgfsetstrokecolor{currentstroke}%
\pgfsetdash{}{0pt}%
\pgfsys@defobject{currentmarker}{\pgfqpoint{-0.020833in}{0.000000in}}{\pgfqpoint{-0.000000in}{0.000000in}}{%
\pgfpathmoveto{\pgfqpoint{-0.000000in}{0.000000in}}%
\pgfpathlineto{\pgfqpoint{-0.020833in}{0.000000in}}%
\pgfusepath{stroke,fill}%
}%
\begin{pgfscope}%
\pgfsys@transformshift{4.291580in}{0.980361in}%
\pgfsys@useobject{currentmarker}{}%
\end{pgfscope}%
\end{pgfscope}%
\begin{pgfscope}%
\pgfsetbuttcap%
\pgfsetroundjoin%
\definecolor{currentfill}{rgb}{0.000000,0.000000,0.000000}%
\pgfsetfillcolor{currentfill}%
\pgfsetlinewidth{0.501875pt}%
\definecolor{currentstroke}{rgb}{0.000000,0.000000,0.000000}%
\pgfsetstrokecolor{currentstroke}%
\pgfsetdash{}{0pt}%
\pgfsys@defobject{currentmarker}{\pgfqpoint{0.000000in}{0.000000in}}{\pgfqpoint{0.020833in}{0.000000in}}{%
\pgfpathmoveto{\pgfqpoint{0.000000in}{0.000000in}}%
\pgfpathlineto{\pgfqpoint{0.020833in}{0.000000in}}%
\pgfusepath{stroke,fill}%
}%
\begin{pgfscope}%
\pgfsys@transformshift{0.552773in}{1.100147in}%
\pgfsys@useobject{currentmarker}{}%
\end{pgfscope}%
\end{pgfscope}%
\begin{pgfscope}%
\pgfsetbuttcap%
\pgfsetroundjoin%
\definecolor{currentfill}{rgb}{0.000000,0.000000,0.000000}%
\pgfsetfillcolor{currentfill}%
\pgfsetlinewidth{0.501875pt}%
\definecolor{currentstroke}{rgb}{0.000000,0.000000,0.000000}%
\pgfsetstrokecolor{currentstroke}%
\pgfsetdash{}{0pt}%
\pgfsys@defobject{currentmarker}{\pgfqpoint{-0.020833in}{0.000000in}}{\pgfqpoint{-0.000000in}{0.000000in}}{%
\pgfpathmoveto{\pgfqpoint{-0.000000in}{0.000000in}}%
\pgfpathlineto{\pgfqpoint{-0.020833in}{0.000000in}}%
\pgfusepath{stroke,fill}%
}%
\begin{pgfscope}%
\pgfsys@transformshift{4.291580in}{1.100147in}%
\pgfsys@useobject{currentmarker}{}%
\end{pgfscope}%
\end{pgfscope}%
\begin{pgfscope}%
\pgfsetbuttcap%
\pgfsetroundjoin%
\definecolor{currentfill}{rgb}{0.000000,0.000000,0.000000}%
\pgfsetfillcolor{currentfill}%
\pgfsetlinewidth{0.501875pt}%
\definecolor{currentstroke}{rgb}{0.000000,0.000000,0.000000}%
\pgfsetstrokecolor{currentstroke}%
\pgfsetdash{}{0pt}%
\pgfsys@defobject{currentmarker}{\pgfqpoint{0.000000in}{0.000000in}}{\pgfqpoint{0.020833in}{0.000000in}}{%
\pgfpathmoveto{\pgfqpoint{0.000000in}{0.000000in}}%
\pgfpathlineto{\pgfqpoint{0.020833in}{0.000000in}}%
\pgfusepath{stroke,fill}%
}%
\begin{pgfscope}%
\pgfsys@transformshift{0.552773in}{1.160041in}%
\pgfsys@useobject{currentmarker}{}%
\end{pgfscope}%
\end{pgfscope}%
\begin{pgfscope}%
\pgfsetbuttcap%
\pgfsetroundjoin%
\definecolor{currentfill}{rgb}{0.000000,0.000000,0.000000}%
\pgfsetfillcolor{currentfill}%
\pgfsetlinewidth{0.501875pt}%
\definecolor{currentstroke}{rgb}{0.000000,0.000000,0.000000}%
\pgfsetstrokecolor{currentstroke}%
\pgfsetdash{}{0pt}%
\pgfsys@defobject{currentmarker}{\pgfqpoint{-0.020833in}{0.000000in}}{\pgfqpoint{-0.000000in}{0.000000in}}{%
\pgfpathmoveto{\pgfqpoint{-0.000000in}{0.000000in}}%
\pgfpathlineto{\pgfqpoint{-0.020833in}{0.000000in}}%
\pgfusepath{stroke,fill}%
}%
\begin{pgfscope}%
\pgfsys@transformshift{4.291580in}{1.160041in}%
\pgfsys@useobject{currentmarker}{}%
\end{pgfscope}%
\end{pgfscope}%
\begin{pgfscope}%
\pgfsetbuttcap%
\pgfsetroundjoin%
\definecolor{currentfill}{rgb}{0.000000,0.000000,0.000000}%
\pgfsetfillcolor{currentfill}%
\pgfsetlinewidth{0.501875pt}%
\definecolor{currentstroke}{rgb}{0.000000,0.000000,0.000000}%
\pgfsetstrokecolor{currentstroke}%
\pgfsetdash{}{0pt}%
\pgfsys@defobject{currentmarker}{\pgfqpoint{0.000000in}{0.000000in}}{\pgfqpoint{0.020833in}{0.000000in}}{%
\pgfpathmoveto{\pgfqpoint{0.000000in}{0.000000in}}%
\pgfpathlineto{\pgfqpoint{0.020833in}{0.000000in}}%
\pgfusepath{stroke,fill}%
}%
\begin{pgfscope}%
\pgfsys@transformshift{0.552773in}{1.219934in}%
\pgfsys@useobject{currentmarker}{}%
\end{pgfscope}%
\end{pgfscope}%
\begin{pgfscope}%
\pgfsetbuttcap%
\pgfsetroundjoin%
\definecolor{currentfill}{rgb}{0.000000,0.000000,0.000000}%
\pgfsetfillcolor{currentfill}%
\pgfsetlinewidth{0.501875pt}%
\definecolor{currentstroke}{rgb}{0.000000,0.000000,0.000000}%
\pgfsetstrokecolor{currentstroke}%
\pgfsetdash{}{0pt}%
\pgfsys@defobject{currentmarker}{\pgfqpoint{-0.020833in}{0.000000in}}{\pgfqpoint{-0.000000in}{0.000000in}}{%
\pgfpathmoveto{\pgfqpoint{-0.000000in}{0.000000in}}%
\pgfpathlineto{\pgfqpoint{-0.020833in}{0.000000in}}%
\pgfusepath{stroke,fill}%
}%
\begin{pgfscope}%
\pgfsys@transformshift{4.291580in}{1.219934in}%
\pgfsys@useobject{currentmarker}{}%
\end{pgfscope}%
\end{pgfscope}%
\begin{pgfscope}%
\pgfsetbuttcap%
\pgfsetroundjoin%
\definecolor{currentfill}{rgb}{0.000000,0.000000,0.000000}%
\pgfsetfillcolor{currentfill}%
\pgfsetlinewidth{0.501875pt}%
\definecolor{currentstroke}{rgb}{0.000000,0.000000,0.000000}%
\pgfsetstrokecolor{currentstroke}%
\pgfsetdash{}{0pt}%
\pgfsys@defobject{currentmarker}{\pgfqpoint{0.000000in}{0.000000in}}{\pgfqpoint{0.020833in}{0.000000in}}{%
\pgfpathmoveto{\pgfqpoint{0.000000in}{0.000000in}}%
\pgfpathlineto{\pgfqpoint{0.020833in}{0.000000in}}%
\pgfusepath{stroke,fill}%
}%
\begin{pgfscope}%
\pgfsys@transformshift{0.552773in}{1.279828in}%
\pgfsys@useobject{currentmarker}{}%
\end{pgfscope}%
\end{pgfscope}%
\begin{pgfscope}%
\pgfsetbuttcap%
\pgfsetroundjoin%
\definecolor{currentfill}{rgb}{0.000000,0.000000,0.000000}%
\pgfsetfillcolor{currentfill}%
\pgfsetlinewidth{0.501875pt}%
\definecolor{currentstroke}{rgb}{0.000000,0.000000,0.000000}%
\pgfsetstrokecolor{currentstroke}%
\pgfsetdash{}{0pt}%
\pgfsys@defobject{currentmarker}{\pgfqpoint{-0.020833in}{0.000000in}}{\pgfqpoint{-0.000000in}{0.000000in}}{%
\pgfpathmoveto{\pgfqpoint{-0.000000in}{0.000000in}}%
\pgfpathlineto{\pgfqpoint{-0.020833in}{0.000000in}}%
\pgfusepath{stroke,fill}%
}%
\begin{pgfscope}%
\pgfsys@transformshift{4.291580in}{1.279828in}%
\pgfsys@useobject{currentmarker}{}%
\end{pgfscope}%
\end{pgfscope}%
\begin{pgfscope}%
\pgfsetbuttcap%
\pgfsetroundjoin%
\definecolor{currentfill}{rgb}{0.000000,0.000000,0.000000}%
\pgfsetfillcolor{currentfill}%
\pgfsetlinewidth{0.501875pt}%
\definecolor{currentstroke}{rgb}{0.000000,0.000000,0.000000}%
\pgfsetstrokecolor{currentstroke}%
\pgfsetdash{}{0pt}%
\pgfsys@defobject{currentmarker}{\pgfqpoint{0.000000in}{0.000000in}}{\pgfqpoint{0.020833in}{0.000000in}}{%
\pgfpathmoveto{\pgfqpoint{0.000000in}{0.000000in}}%
\pgfpathlineto{\pgfqpoint{0.020833in}{0.000000in}}%
\pgfusepath{stroke,fill}%
}%
\begin{pgfscope}%
\pgfsys@transformshift{0.552773in}{1.399614in}%
\pgfsys@useobject{currentmarker}{}%
\end{pgfscope}%
\end{pgfscope}%
\begin{pgfscope}%
\pgfsetbuttcap%
\pgfsetroundjoin%
\definecolor{currentfill}{rgb}{0.000000,0.000000,0.000000}%
\pgfsetfillcolor{currentfill}%
\pgfsetlinewidth{0.501875pt}%
\definecolor{currentstroke}{rgb}{0.000000,0.000000,0.000000}%
\pgfsetstrokecolor{currentstroke}%
\pgfsetdash{}{0pt}%
\pgfsys@defobject{currentmarker}{\pgfqpoint{-0.020833in}{0.000000in}}{\pgfqpoint{-0.000000in}{0.000000in}}{%
\pgfpathmoveto{\pgfqpoint{-0.000000in}{0.000000in}}%
\pgfpathlineto{\pgfqpoint{-0.020833in}{0.000000in}}%
\pgfusepath{stroke,fill}%
}%
\begin{pgfscope}%
\pgfsys@transformshift{4.291580in}{1.399614in}%
\pgfsys@useobject{currentmarker}{}%
\end{pgfscope}%
\end{pgfscope}%
\begin{pgfscope}%
\pgfsetbuttcap%
\pgfsetroundjoin%
\definecolor{currentfill}{rgb}{0.000000,0.000000,0.000000}%
\pgfsetfillcolor{currentfill}%
\pgfsetlinewidth{0.501875pt}%
\definecolor{currentstroke}{rgb}{0.000000,0.000000,0.000000}%
\pgfsetstrokecolor{currentstroke}%
\pgfsetdash{}{0pt}%
\pgfsys@defobject{currentmarker}{\pgfqpoint{0.000000in}{0.000000in}}{\pgfqpoint{0.020833in}{0.000000in}}{%
\pgfpathmoveto{\pgfqpoint{0.000000in}{0.000000in}}%
\pgfpathlineto{\pgfqpoint{0.020833in}{0.000000in}}%
\pgfusepath{stroke,fill}%
}%
\begin{pgfscope}%
\pgfsys@transformshift{0.552773in}{1.459508in}%
\pgfsys@useobject{currentmarker}{}%
\end{pgfscope}%
\end{pgfscope}%
\begin{pgfscope}%
\pgfsetbuttcap%
\pgfsetroundjoin%
\definecolor{currentfill}{rgb}{0.000000,0.000000,0.000000}%
\pgfsetfillcolor{currentfill}%
\pgfsetlinewidth{0.501875pt}%
\definecolor{currentstroke}{rgb}{0.000000,0.000000,0.000000}%
\pgfsetstrokecolor{currentstroke}%
\pgfsetdash{}{0pt}%
\pgfsys@defobject{currentmarker}{\pgfqpoint{-0.020833in}{0.000000in}}{\pgfqpoint{-0.000000in}{0.000000in}}{%
\pgfpathmoveto{\pgfqpoint{-0.000000in}{0.000000in}}%
\pgfpathlineto{\pgfqpoint{-0.020833in}{0.000000in}}%
\pgfusepath{stroke,fill}%
}%
\begin{pgfscope}%
\pgfsys@transformshift{4.291580in}{1.459508in}%
\pgfsys@useobject{currentmarker}{}%
\end{pgfscope}%
\end{pgfscope}%
\begin{pgfscope}%
\pgfsetbuttcap%
\pgfsetroundjoin%
\definecolor{currentfill}{rgb}{0.000000,0.000000,0.000000}%
\pgfsetfillcolor{currentfill}%
\pgfsetlinewidth{0.501875pt}%
\definecolor{currentstroke}{rgb}{0.000000,0.000000,0.000000}%
\pgfsetstrokecolor{currentstroke}%
\pgfsetdash{}{0pt}%
\pgfsys@defobject{currentmarker}{\pgfqpoint{0.000000in}{0.000000in}}{\pgfqpoint{0.020833in}{0.000000in}}{%
\pgfpathmoveto{\pgfqpoint{0.000000in}{0.000000in}}%
\pgfpathlineto{\pgfqpoint{0.020833in}{0.000000in}}%
\pgfusepath{stroke,fill}%
}%
\begin{pgfscope}%
\pgfsys@transformshift{0.552773in}{1.519401in}%
\pgfsys@useobject{currentmarker}{}%
\end{pgfscope}%
\end{pgfscope}%
\begin{pgfscope}%
\pgfsetbuttcap%
\pgfsetroundjoin%
\definecolor{currentfill}{rgb}{0.000000,0.000000,0.000000}%
\pgfsetfillcolor{currentfill}%
\pgfsetlinewidth{0.501875pt}%
\definecolor{currentstroke}{rgb}{0.000000,0.000000,0.000000}%
\pgfsetstrokecolor{currentstroke}%
\pgfsetdash{}{0pt}%
\pgfsys@defobject{currentmarker}{\pgfqpoint{-0.020833in}{0.000000in}}{\pgfqpoint{-0.000000in}{0.000000in}}{%
\pgfpathmoveto{\pgfqpoint{-0.000000in}{0.000000in}}%
\pgfpathlineto{\pgfqpoint{-0.020833in}{0.000000in}}%
\pgfusepath{stroke,fill}%
}%
\begin{pgfscope}%
\pgfsys@transformshift{4.291580in}{1.519401in}%
\pgfsys@useobject{currentmarker}{}%
\end{pgfscope}%
\end{pgfscope}%
\begin{pgfscope}%
\pgfsetbuttcap%
\pgfsetroundjoin%
\definecolor{currentfill}{rgb}{0.000000,0.000000,0.000000}%
\pgfsetfillcolor{currentfill}%
\pgfsetlinewidth{0.501875pt}%
\definecolor{currentstroke}{rgb}{0.000000,0.000000,0.000000}%
\pgfsetstrokecolor{currentstroke}%
\pgfsetdash{}{0pt}%
\pgfsys@defobject{currentmarker}{\pgfqpoint{0.000000in}{0.000000in}}{\pgfqpoint{0.020833in}{0.000000in}}{%
\pgfpathmoveto{\pgfqpoint{0.000000in}{0.000000in}}%
\pgfpathlineto{\pgfqpoint{0.020833in}{0.000000in}}%
\pgfusepath{stroke,fill}%
}%
\begin{pgfscope}%
\pgfsys@transformshift{0.552773in}{1.579294in}%
\pgfsys@useobject{currentmarker}{}%
\end{pgfscope}%
\end{pgfscope}%
\begin{pgfscope}%
\pgfsetbuttcap%
\pgfsetroundjoin%
\definecolor{currentfill}{rgb}{0.000000,0.000000,0.000000}%
\pgfsetfillcolor{currentfill}%
\pgfsetlinewidth{0.501875pt}%
\definecolor{currentstroke}{rgb}{0.000000,0.000000,0.000000}%
\pgfsetstrokecolor{currentstroke}%
\pgfsetdash{}{0pt}%
\pgfsys@defobject{currentmarker}{\pgfqpoint{-0.020833in}{0.000000in}}{\pgfqpoint{-0.000000in}{0.000000in}}{%
\pgfpathmoveto{\pgfqpoint{-0.000000in}{0.000000in}}%
\pgfpathlineto{\pgfqpoint{-0.020833in}{0.000000in}}%
\pgfusepath{stroke,fill}%
}%
\begin{pgfscope}%
\pgfsys@transformshift{4.291580in}{1.579294in}%
\pgfsys@useobject{currentmarker}{}%
\end{pgfscope}%
\end{pgfscope}%
\begin{pgfscope}%
\pgfsetbuttcap%
\pgfsetroundjoin%
\definecolor{currentfill}{rgb}{0.000000,0.000000,0.000000}%
\pgfsetfillcolor{currentfill}%
\pgfsetlinewidth{0.501875pt}%
\definecolor{currentstroke}{rgb}{0.000000,0.000000,0.000000}%
\pgfsetstrokecolor{currentstroke}%
\pgfsetdash{}{0pt}%
\pgfsys@defobject{currentmarker}{\pgfqpoint{0.000000in}{0.000000in}}{\pgfqpoint{0.020833in}{0.000000in}}{%
\pgfpathmoveto{\pgfqpoint{0.000000in}{0.000000in}}%
\pgfpathlineto{\pgfqpoint{0.020833in}{0.000000in}}%
\pgfusepath{stroke,fill}%
}%
\begin{pgfscope}%
\pgfsys@transformshift{0.552773in}{1.699081in}%
\pgfsys@useobject{currentmarker}{}%
\end{pgfscope}%
\end{pgfscope}%
\begin{pgfscope}%
\pgfsetbuttcap%
\pgfsetroundjoin%
\definecolor{currentfill}{rgb}{0.000000,0.000000,0.000000}%
\pgfsetfillcolor{currentfill}%
\pgfsetlinewidth{0.501875pt}%
\definecolor{currentstroke}{rgb}{0.000000,0.000000,0.000000}%
\pgfsetstrokecolor{currentstroke}%
\pgfsetdash{}{0pt}%
\pgfsys@defobject{currentmarker}{\pgfqpoint{-0.020833in}{0.000000in}}{\pgfqpoint{-0.000000in}{0.000000in}}{%
\pgfpathmoveto{\pgfqpoint{-0.000000in}{0.000000in}}%
\pgfpathlineto{\pgfqpoint{-0.020833in}{0.000000in}}%
\pgfusepath{stroke,fill}%
}%
\begin{pgfscope}%
\pgfsys@transformshift{4.291580in}{1.699081in}%
\pgfsys@useobject{currentmarker}{}%
\end{pgfscope}%
\end{pgfscope}%
\begin{pgfscope}%
\pgfsetbuttcap%
\pgfsetroundjoin%
\definecolor{currentfill}{rgb}{0.000000,0.000000,0.000000}%
\pgfsetfillcolor{currentfill}%
\pgfsetlinewidth{0.501875pt}%
\definecolor{currentstroke}{rgb}{0.000000,0.000000,0.000000}%
\pgfsetstrokecolor{currentstroke}%
\pgfsetdash{}{0pt}%
\pgfsys@defobject{currentmarker}{\pgfqpoint{0.000000in}{0.000000in}}{\pgfqpoint{0.020833in}{0.000000in}}{%
\pgfpathmoveto{\pgfqpoint{0.000000in}{0.000000in}}%
\pgfpathlineto{\pgfqpoint{0.020833in}{0.000000in}}%
\pgfusepath{stroke,fill}%
}%
\begin{pgfscope}%
\pgfsys@transformshift{0.552773in}{1.758974in}%
\pgfsys@useobject{currentmarker}{}%
\end{pgfscope}%
\end{pgfscope}%
\begin{pgfscope}%
\pgfsetbuttcap%
\pgfsetroundjoin%
\definecolor{currentfill}{rgb}{0.000000,0.000000,0.000000}%
\pgfsetfillcolor{currentfill}%
\pgfsetlinewidth{0.501875pt}%
\definecolor{currentstroke}{rgb}{0.000000,0.000000,0.000000}%
\pgfsetstrokecolor{currentstroke}%
\pgfsetdash{}{0pt}%
\pgfsys@defobject{currentmarker}{\pgfqpoint{-0.020833in}{0.000000in}}{\pgfqpoint{-0.000000in}{0.000000in}}{%
\pgfpathmoveto{\pgfqpoint{-0.000000in}{0.000000in}}%
\pgfpathlineto{\pgfqpoint{-0.020833in}{0.000000in}}%
\pgfusepath{stroke,fill}%
}%
\begin{pgfscope}%
\pgfsys@transformshift{4.291580in}{1.758974in}%
\pgfsys@useobject{currentmarker}{}%
\end{pgfscope}%
\end{pgfscope}%
\begin{pgfscope}%
\pgfsetbuttcap%
\pgfsetroundjoin%
\definecolor{currentfill}{rgb}{0.000000,0.000000,0.000000}%
\pgfsetfillcolor{currentfill}%
\pgfsetlinewidth{0.501875pt}%
\definecolor{currentstroke}{rgb}{0.000000,0.000000,0.000000}%
\pgfsetstrokecolor{currentstroke}%
\pgfsetdash{}{0pt}%
\pgfsys@defobject{currentmarker}{\pgfqpoint{0.000000in}{0.000000in}}{\pgfqpoint{0.020833in}{0.000000in}}{%
\pgfpathmoveto{\pgfqpoint{0.000000in}{0.000000in}}%
\pgfpathlineto{\pgfqpoint{0.020833in}{0.000000in}}%
\pgfusepath{stroke,fill}%
}%
\begin{pgfscope}%
\pgfsys@transformshift{0.552773in}{1.818868in}%
\pgfsys@useobject{currentmarker}{}%
\end{pgfscope}%
\end{pgfscope}%
\begin{pgfscope}%
\pgfsetbuttcap%
\pgfsetroundjoin%
\definecolor{currentfill}{rgb}{0.000000,0.000000,0.000000}%
\pgfsetfillcolor{currentfill}%
\pgfsetlinewidth{0.501875pt}%
\definecolor{currentstroke}{rgb}{0.000000,0.000000,0.000000}%
\pgfsetstrokecolor{currentstroke}%
\pgfsetdash{}{0pt}%
\pgfsys@defobject{currentmarker}{\pgfqpoint{-0.020833in}{0.000000in}}{\pgfqpoint{-0.000000in}{0.000000in}}{%
\pgfpathmoveto{\pgfqpoint{-0.000000in}{0.000000in}}%
\pgfpathlineto{\pgfqpoint{-0.020833in}{0.000000in}}%
\pgfusepath{stroke,fill}%
}%
\begin{pgfscope}%
\pgfsys@transformshift{4.291580in}{1.818868in}%
\pgfsys@useobject{currentmarker}{}%
\end{pgfscope}%
\end{pgfscope}%
\begin{pgfscope}%
\pgfsetbuttcap%
\pgfsetroundjoin%
\definecolor{currentfill}{rgb}{0.000000,0.000000,0.000000}%
\pgfsetfillcolor{currentfill}%
\pgfsetlinewidth{0.501875pt}%
\definecolor{currentstroke}{rgb}{0.000000,0.000000,0.000000}%
\pgfsetstrokecolor{currentstroke}%
\pgfsetdash{}{0pt}%
\pgfsys@defobject{currentmarker}{\pgfqpoint{0.000000in}{0.000000in}}{\pgfqpoint{0.020833in}{0.000000in}}{%
\pgfpathmoveto{\pgfqpoint{0.000000in}{0.000000in}}%
\pgfpathlineto{\pgfqpoint{0.020833in}{0.000000in}}%
\pgfusepath{stroke,fill}%
}%
\begin{pgfscope}%
\pgfsys@transformshift{0.552773in}{1.878761in}%
\pgfsys@useobject{currentmarker}{}%
\end{pgfscope}%
\end{pgfscope}%
\begin{pgfscope}%
\pgfsetbuttcap%
\pgfsetroundjoin%
\definecolor{currentfill}{rgb}{0.000000,0.000000,0.000000}%
\pgfsetfillcolor{currentfill}%
\pgfsetlinewidth{0.501875pt}%
\definecolor{currentstroke}{rgb}{0.000000,0.000000,0.000000}%
\pgfsetstrokecolor{currentstroke}%
\pgfsetdash{}{0pt}%
\pgfsys@defobject{currentmarker}{\pgfqpoint{-0.020833in}{0.000000in}}{\pgfqpoint{-0.000000in}{0.000000in}}{%
\pgfpathmoveto{\pgfqpoint{-0.000000in}{0.000000in}}%
\pgfpathlineto{\pgfqpoint{-0.020833in}{0.000000in}}%
\pgfusepath{stroke,fill}%
}%
\begin{pgfscope}%
\pgfsys@transformshift{4.291580in}{1.878761in}%
\pgfsys@useobject{currentmarker}{}%
\end{pgfscope}%
\end{pgfscope}%
\begin{pgfscope}%
\pgfsetbuttcap%
\pgfsetroundjoin%
\definecolor{currentfill}{rgb}{0.000000,0.000000,0.000000}%
\pgfsetfillcolor{currentfill}%
\pgfsetlinewidth{0.501875pt}%
\definecolor{currentstroke}{rgb}{0.000000,0.000000,0.000000}%
\pgfsetstrokecolor{currentstroke}%
\pgfsetdash{}{0pt}%
\pgfsys@defobject{currentmarker}{\pgfqpoint{0.000000in}{0.000000in}}{\pgfqpoint{0.020833in}{0.000000in}}{%
\pgfpathmoveto{\pgfqpoint{0.000000in}{0.000000in}}%
\pgfpathlineto{\pgfqpoint{0.020833in}{0.000000in}}%
\pgfusepath{stroke,fill}%
}%
\begin{pgfscope}%
\pgfsys@transformshift{0.552773in}{1.998548in}%
\pgfsys@useobject{currentmarker}{}%
\end{pgfscope}%
\end{pgfscope}%
\begin{pgfscope}%
\pgfsetbuttcap%
\pgfsetroundjoin%
\definecolor{currentfill}{rgb}{0.000000,0.000000,0.000000}%
\pgfsetfillcolor{currentfill}%
\pgfsetlinewidth{0.501875pt}%
\definecolor{currentstroke}{rgb}{0.000000,0.000000,0.000000}%
\pgfsetstrokecolor{currentstroke}%
\pgfsetdash{}{0pt}%
\pgfsys@defobject{currentmarker}{\pgfqpoint{-0.020833in}{0.000000in}}{\pgfqpoint{-0.000000in}{0.000000in}}{%
\pgfpathmoveto{\pgfqpoint{-0.000000in}{0.000000in}}%
\pgfpathlineto{\pgfqpoint{-0.020833in}{0.000000in}}%
\pgfusepath{stroke,fill}%
}%
\begin{pgfscope}%
\pgfsys@transformshift{4.291580in}{1.998548in}%
\pgfsys@useobject{currentmarker}{}%
\end{pgfscope}%
\end{pgfscope}%
\begin{pgfscope}%
\pgfsetbuttcap%
\pgfsetroundjoin%
\definecolor{currentfill}{rgb}{0.000000,0.000000,0.000000}%
\pgfsetfillcolor{currentfill}%
\pgfsetlinewidth{0.501875pt}%
\definecolor{currentstroke}{rgb}{0.000000,0.000000,0.000000}%
\pgfsetstrokecolor{currentstroke}%
\pgfsetdash{}{0pt}%
\pgfsys@defobject{currentmarker}{\pgfqpoint{0.000000in}{0.000000in}}{\pgfqpoint{0.020833in}{0.000000in}}{%
\pgfpathmoveto{\pgfqpoint{0.000000in}{0.000000in}}%
\pgfpathlineto{\pgfqpoint{0.020833in}{0.000000in}}%
\pgfusepath{stroke,fill}%
}%
\begin{pgfscope}%
\pgfsys@transformshift{0.552773in}{2.058441in}%
\pgfsys@useobject{currentmarker}{}%
\end{pgfscope}%
\end{pgfscope}%
\begin{pgfscope}%
\pgfsetbuttcap%
\pgfsetroundjoin%
\definecolor{currentfill}{rgb}{0.000000,0.000000,0.000000}%
\pgfsetfillcolor{currentfill}%
\pgfsetlinewidth{0.501875pt}%
\definecolor{currentstroke}{rgb}{0.000000,0.000000,0.000000}%
\pgfsetstrokecolor{currentstroke}%
\pgfsetdash{}{0pt}%
\pgfsys@defobject{currentmarker}{\pgfqpoint{-0.020833in}{0.000000in}}{\pgfqpoint{-0.000000in}{0.000000in}}{%
\pgfpathmoveto{\pgfqpoint{-0.000000in}{0.000000in}}%
\pgfpathlineto{\pgfqpoint{-0.020833in}{0.000000in}}%
\pgfusepath{stroke,fill}%
}%
\begin{pgfscope}%
\pgfsys@transformshift{4.291580in}{2.058441in}%
\pgfsys@useobject{currentmarker}{}%
\end{pgfscope}%
\end{pgfscope}%
\begin{pgfscope}%
\pgfsetbuttcap%
\pgfsetroundjoin%
\definecolor{currentfill}{rgb}{0.000000,0.000000,0.000000}%
\pgfsetfillcolor{currentfill}%
\pgfsetlinewidth{0.501875pt}%
\definecolor{currentstroke}{rgb}{0.000000,0.000000,0.000000}%
\pgfsetstrokecolor{currentstroke}%
\pgfsetdash{}{0pt}%
\pgfsys@defobject{currentmarker}{\pgfqpoint{0.000000in}{0.000000in}}{\pgfqpoint{0.020833in}{0.000000in}}{%
\pgfpathmoveto{\pgfqpoint{0.000000in}{0.000000in}}%
\pgfpathlineto{\pgfqpoint{0.020833in}{0.000000in}}%
\pgfusepath{stroke,fill}%
}%
\begin{pgfscope}%
\pgfsys@transformshift{0.552773in}{2.118335in}%
\pgfsys@useobject{currentmarker}{}%
\end{pgfscope}%
\end{pgfscope}%
\begin{pgfscope}%
\pgfsetbuttcap%
\pgfsetroundjoin%
\definecolor{currentfill}{rgb}{0.000000,0.000000,0.000000}%
\pgfsetfillcolor{currentfill}%
\pgfsetlinewidth{0.501875pt}%
\definecolor{currentstroke}{rgb}{0.000000,0.000000,0.000000}%
\pgfsetstrokecolor{currentstroke}%
\pgfsetdash{}{0pt}%
\pgfsys@defobject{currentmarker}{\pgfqpoint{-0.020833in}{0.000000in}}{\pgfqpoint{-0.000000in}{0.000000in}}{%
\pgfpathmoveto{\pgfqpoint{-0.000000in}{0.000000in}}%
\pgfpathlineto{\pgfqpoint{-0.020833in}{0.000000in}}%
\pgfusepath{stroke,fill}%
}%
\begin{pgfscope}%
\pgfsys@transformshift{4.291580in}{2.118335in}%
\pgfsys@useobject{currentmarker}{}%
\end{pgfscope}%
\end{pgfscope}%
\begin{pgfscope}%
\pgfsetbuttcap%
\pgfsetroundjoin%
\definecolor{currentfill}{rgb}{0.000000,0.000000,0.000000}%
\pgfsetfillcolor{currentfill}%
\pgfsetlinewidth{0.501875pt}%
\definecolor{currentstroke}{rgb}{0.000000,0.000000,0.000000}%
\pgfsetstrokecolor{currentstroke}%
\pgfsetdash{}{0pt}%
\pgfsys@defobject{currentmarker}{\pgfqpoint{0.000000in}{0.000000in}}{\pgfqpoint{0.020833in}{0.000000in}}{%
\pgfpathmoveto{\pgfqpoint{0.000000in}{0.000000in}}%
\pgfpathlineto{\pgfqpoint{0.020833in}{0.000000in}}%
\pgfusepath{stroke,fill}%
}%
\begin{pgfscope}%
\pgfsys@transformshift{0.552773in}{2.178228in}%
\pgfsys@useobject{currentmarker}{}%
\end{pgfscope}%
\end{pgfscope}%
\begin{pgfscope}%
\pgfsetbuttcap%
\pgfsetroundjoin%
\definecolor{currentfill}{rgb}{0.000000,0.000000,0.000000}%
\pgfsetfillcolor{currentfill}%
\pgfsetlinewidth{0.501875pt}%
\definecolor{currentstroke}{rgb}{0.000000,0.000000,0.000000}%
\pgfsetstrokecolor{currentstroke}%
\pgfsetdash{}{0pt}%
\pgfsys@defobject{currentmarker}{\pgfqpoint{-0.020833in}{0.000000in}}{\pgfqpoint{-0.000000in}{0.000000in}}{%
\pgfpathmoveto{\pgfqpoint{-0.000000in}{0.000000in}}%
\pgfpathlineto{\pgfqpoint{-0.020833in}{0.000000in}}%
\pgfusepath{stroke,fill}%
}%
\begin{pgfscope}%
\pgfsys@transformshift{4.291580in}{2.178228in}%
\pgfsys@useobject{currentmarker}{}%
\end{pgfscope}%
\end{pgfscope}%
\begin{pgfscope}%
\definecolor{textcolor}{rgb}{0.000000,0.000000,0.000000}%
\pgfsetstrokecolor{textcolor}%
\pgfsetfillcolor{textcolor}%
\pgftext[x=0.201692in,y=1.314295in,,bottom,rotate=90.000000]{\color{textcolor}\rmfamily\fontsize{12.000000}{14.400000}\selectfont \(\displaystyle I_s\) (\unit{\micro\ampere})}%
\end{pgfscope}%
\begin{pgfscope}%
\pgfpathrectangle{\pgfqpoint{0.552773in}{0.431673in}}{\pgfqpoint{3.738807in}{1.765244in}}%
\pgfusepath{clip}%
\pgfsetbuttcap%
\pgfsetroundjoin%
\pgfsetlinewidth{1.003750pt}%
\definecolor{currentstroke}{rgb}{0.667253,0.779176,0.992959}%
\pgfsetstrokecolor{currentstroke}%
\pgfsetdash{}{0pt}%
\pgfpathmoveto{\pgfqpoint{0.946281in}{0.917169in}}%
\pgfpathlineto{\pgfqpoint{0.946281in}{0.917169in}}%
\pgfusepath{stroke}%
\end{pgfscope}%
\begin{pgfscope}%
\pgfpathrectangle{\pgfqpoint{0.552773in}{0.431673in}}{\pgfqpoint{3.738807in}{1.765244in}}%
\pgfusepath{clip}%
\pgfsetbuttcap%
\pgfsetroundjoin%
\pgfsetlinewidth{1.003750pt}%
\definecolor{currentstroke}{rgb}{0.667253,0.779176,0.992959}%
\pgfsetstrokecolor{currentstroke}%
\pgfsetdash{}{0pt}%
\pgfpathmoveto{\pgfqpoint{0.968004in}{0.896826in}}%
\pgfpathlineto{\pgfqpoint{0.968004in}{0.896826in}}%
\pgfusepath{stroke}%
\end{pgfscope}%
\begin{pgfscope}%
\pgfpathrectangle{\pgfqpoint{0.552773in}{0.431673in}}{\pgfqpoint{3.738807in}{1.765244in}}%
\pgfusepath{clip}%
\pgfsetbuttcap%
\pgfsetroundjoin%
\pgfsetlinewidth{1.003750pt}%
\definecolor{currentstroke}{rgb}{0.667253,0.779176,0.992959}%
\pgfsetstrokecolor{currentstroke}%
\pgfsetdash{}{0pt}%
\pgfpathmoveto{\pgfqpoint{0.983367in}{0.864904in}}%
\pgfpathlineto{\pgfqpoint{0.983367in}{0.864904in}}%
\pgfusepath{stroke}%
\end{pgfscope}%
\begin{pgfscope}%
\pgfpathrectangle{\pgfqpoint{0.552773in}{0.431673in}}{\pgfqpoint{3.738807in}{1.765244in}}%
\pgfusepath{clip}%
\pgfsetbuttcap%
\pgfsetroundjoin%
\pgfsetlinewidth{1.003750pt}%
\definecolor{currentstroke}{rgb}{0.667253,0.779176,0.992959}%
\pgfsetstrokecolor{currentstroke}%
\pgfsetdash{}{0pt}%
\pgfpathmoveto{\pgfqpoint{0.997288in}{0.830357in}}%
\pgfpathlineto{\pgfqpoint{0.997288in}{0.830357in}}%
\pgfusepath{stroke}%
\end{pgfscope}%
\begin{pgfscope}%
\pgfpathrectangle{\pgfqpoint{0.552773in}{0.431673in}}{\pgfqpoint{3.738807in}{1.765244in}}%
\pgfusepath{clip}%
\pgfsetbuttcap%
\pgfsetroundjoin%
\pgfsetlinewidth{1.003750pt}%
\definecolor{currentstroke}{rgb}{0.667253,0.779176,0.992959}%
\pgfsetstrokecolor{currentstroke}%
\pgfsetdash{}{0pt}%
\pgfpathmoveto{\pgfqpoint{1.003998in}{0.782680in}}%
\pgfpathlineto{\pgfqpoint{1.003998in}{0.782680in}}%
\pgfusepath{stroke}%
\end{pgfscope}%
\begin{pgfscope}%
\pgfpathrectangle{\pgfqpoint{0.552773in}{0.431673in}}{\pgfqpoint{3.738807in}{1.765244in}}%
\pgfusepath{clip}%
\pgfsetbuttcap%
\pgfsetroundjoin%
\pgfsetlinewidth{1.003750pt}%
\definecolor{currentstroke}{rgb}{0.667253,0.779176,0.992959}%
\pgfsetstrokecolor{currentstroke}%
\pgfsetdash{}{0pt}%
\pgfpathmoveto{\pgfqpoint{1.002620in}{0.720278in}}%
\pgfpathlineto{\pgfqpoint{1.002620in}{0.720278in}}%
\pgfusepath{stroke}%
\end{pgfscope}%
\begin{pgfscope}%
\pgfpathrectangle{\pgfqpoint{0.552773in}{0.431673in}}{\pgfqpoint{3.738807in}{1.765244in}}%
\pgfusepath{clip}%
\pgfsetbuttcap%
\pgfsetroundjoin%
\pgfsetlinewidth{1.003750pt}%
\definecolor{currentstroke}{rgb}{0.667253,0.779176,0.992959}%
\pgfsetstrokecolor{currentstroke}%
\pgfsetdash{}{0pt}%
\pgfpathmoveto{\pgfqpoint{1.016428in}{0.685524in}}%
\pgfpathlineto{\pgfqpoint{1.016428in}{0.685524in}}%
\pgfusepath{stroke}%
\end{pgfscope}%
\begin{pgfscope}%
\pgfpathrectangle{\pgfqpoint{0.552773in}{0.431673in}}{\pgfqpoint{3.738807in}{1.765244in}}%
\pgfusepath{clip}%
\pgfsetbuttcap%
\pgfsetroundjoin%
\pgfsetlinewidth{1.003750pt}%
\definecolor{currentstroke}{rgb}{0.667253,0.779176,0.992959}%
\pgfsetstrokecolor{currentstroke}%
\pgfsetdash{}{0pt}%
\pgfpathmoveto{\pgfqpoint{1.016366in}{0.625519in}}%
\pgfpathlineto{\pgfqpoint{1.016366in}{0.625519in}}%
\pgfusepath{stroke}%
\end{pgfscope}%
\begin{pgfscope}%
\pgfpathrectangle{\pgfqpoint{0.552773in}{0.431673in}}{\pgfqpoint{3.738807in}{1.765244in}}%
\pgfusepath{clip}%
\pgfsetbuttcap%
\pgfsetroundjoin%
\pgfsetlinewidth{1.003750pt}%
\definecolor{currentstroke}{rgb}{0.667253,0.779176,0.992959}%
\pgfsetstrokecolor{currentstroke}%
\pgfsetdash{}{0pt}%
\pgfpathmoveto{\pgfqpoint{1.044197in}{0.616297in}}%
\pgfpathlineto{\pgfqpoint{1.044197in}{0.616297in}}%
\pgfusepath{stroke}%
\end{pgfscope}%
\begin{pgfscope}%
\pgfpathrectangle{\pgfqpoint{0.552773in}{0.431673in}}{\pgfqpoint{3.738807in}{1.765244in}}%
\pgfusepath{clip}%
\pgfsetbuttcap%
\pgfsetroundjoin%
\pgfsetlinewidth{1.003750pt}%
\definecolor{currentstroke}{rgb}{0.667253,0.779176,0.992959}%
\pgfsetstrokecolor{currentstroke}%
\pgfsetdash{}{0pt}%
\pgfpathmoveto{\pgfqpoint{1.061881in}{0.588600in}}%
\pgfpathlineto{\pgfqpoint{1.061881in}{0.588600in}}%
\pgfusepath{stroke}%
\end{pgfscope}%
\begin{pgfscope}%
\pgfpathrectangle{\pgfqpoint{0.552773in}{0.431673in}}{\pgfqpoint{3.738807in}{1.765244in}}%
\pgfusepath{clip}%
\pgfsetbuttcap%
\pgfsetroundjoin%
\pgfsetlinewidth{1.003750pt}%
\definecolor{currentstroke}{rgb}{0.667253,0.779176,0.992959}%
\pgfsetstrokecolor{currentstroke}%
\pgfsetdash{}{0pt}%
\pgfpathmoveto{\pgfqpoint{1.100011in}{0.598129in}}%
\pgfpathlineto{\pgfqpoint{1.100011in}{0.598129in}}%
\pgfusepath{stroke}%
\end{pgfscope}%
\begin{pgfscope}%
\pgfpathrectangle{\pgfqpoint{0.552773in}{0.431673in}}{\pgfqpoint{3.738807in}{1.765244in}}%
\pgfusepath{clip}%
\pgfsetbuttcap%
\pgfsetroundjoin%
\pgfsetlinewidth{1.003750pt}%
\definecolor{currentstroke}{rgb}{0.667253,0.779176,0.992959}%
\pgfsetstrokecolor{currentstroke}%
\pgfsetdash{}{0pt}%
\pgfpathmoveto{\pgfqpoint{1.133830in}{0.599811in}}%
\pgfpathlineto{\pgfqpoint{1.133830in}{0.599811in}}%
\pgfusepath{stroke}%
\end{pgfscope}%
\begin{pgfscope}%
\pgfpathrectangle{\pgfqpoint{0.552773in}{0.431673in}}{\pgfqpoint{3.738807in}{1.765244in}}%
\pgfusepath{clip}%
\pgfsetbuttcap%
\pgfsetroundjoin%
\pgfsetlinewidth{1.003750pt}%
\definecolor{currentstroke}{rgb}{0.667253,0.779176,0.992959}%
\pgfsetstrokecolor{currentstroke}%
\pgfsetdash{}{0pt}%
\pgfpathmoveto{\pgfqpoint{1.178595in}{0.621420in}}%
\pgfpathlineto{\pgfqpoint{1.178595in}{0.621420in}}%
\pgfusepath{stroke}%
\end{pgfscope}%
\begin{pgfscope}%
\pgfpathrectangle{\pgfqpoint{0.552773in}{0.431673in}}{\pgfqpoint{3.738807in}{1.765244in}}%
\pgfusepath{clip}%
\pgfsetbuttcap%
\pgfsetroundjoin%
\pgfsetlinewidth{1.003750pt}%
\definecolor{currentstroke}{rgb}{0.667253,0.779176,0.992959}%
\pgfsetstrokecolor{currentstroke}%
\pgfsetdash{}{0pt}%
\pgfpathmoveto{\pgfqpoint{1.241991in}{0.676951in}}%
\pgfpathlineto{\pgfqpoint{1.241991in}{0.676951in}}%
\pgfusepath{stroke}%
\end{pgfscope}%
\begin{pgfscope}%
\pgfpathrectangle{\pgfqpoint{0.552773in}{0.431673in}}{\pgfqpoint{3.738807in}{1.765244in}}%
\pgfusepath{clip}%
\pgfsetbuttcap%
\pgfsetroundjoin%
\pgfsetlinewidth{1.003750pt}%
\definecolor{currentstroke}{rgb}{0.667253,0.779176,0.992959}%
\pgfsetstrokecolor{currentstroke}%
\pgfsetdash{}{0pt}%
\pgfpathmoveto{\pgfqpoint{1.316482in}{0.752683in}}%
\pgfpathlineto{\pgfqpoint{1.316482in}{0.752683in}}%
\pgfusepath{stroke}%
\end{pgfscope}%
\begin{pgfscope}%
\pgfpathrectangle{\pgfqpoint{0.552773in}{0.431673in}}{\pgfqpoint{3.738807in}{1.765244in}}%
\pgfusepath{clip}%
\pgfsetbuttcap%
\pgfsetroundjoin%
\pgfsetlinewidth{1.003750pt}%
\definecolor{currentstroke}{rgb}{0.667253,0.779176,0.992959}%
\pgfsetstrokecolor{currentstroke}%
\pgfsetdash{}{0pt}%
\pgfpathmoveto{\pgfqpoint{1.386975in}{0.821134in}}%
\pgfpathlineto{\pgfqpoint{1.386975in}{0.821134in}}%
\pgfusepath{stroke}%
\end{pgfscope}%
\begin{pgfscope}%
\pgfpathrectangle{\pgfqpoint{0.552773in}{0.431673in}}{\pgfqpoint{3.738807in}{1.765244in}}%
\pgfusepath{clip}%
\pgfsetbuttcap%
\pgfsetroundjoin%
\pgfsetlinewidth{1.003750pt}%
\definecolor{currentstroke}{rgb}{0.667253,0.779176,0.992959}%
\pgfsetstrokecolor{currentstroke}%
\pgfsetdash{}{0pt}%
\pgfpathmoveto{\pgfqpoint{1.458963in}{0.892310in}}%
\pgfpathlineto{\pgfqpoint{1.458963in}{0.892310in}}%
\pgfusepath{stroke}%
\end{pgfscope}%
\begin{pgfscope}%
\pgfpathrectangle{\pgfqpoint{0.552773in}{0.431673in}}{\pgfqpoint{3.738807in}{1.765244in}}%
\pgfusepath{clip}%
\pgfsetbuttcap%
\pgfsetroundjoin%
\pgfsetlinewidth{1.003750pt}%
\definecolor{currentstroke}{rgb}{0.667253,0.779176,0.992959}%
\pgfsetstrokecolor{currentstroke}%
\pgfsetdash{}{0pt}%
\pgfpathmoveto{\pgfqpoint{1.515751in}{0.935810in}}%
\pgfpathlineto{\pgfqpoint{1.515751in}{0.935810in}}%
\pgfusepath{stroke}%
\end{pgfscope}%
\begin{pgfscope}%
\pgfpathrectangle{\pgfqpoint{0.552773in}{0.431673in}}{\pgfqpoint{3.738807in}{1.765244in}}%
\pgfusepath{clip}%
\pgfsetbuttcap%
\pgfsetroundjoin%
\pgfsetlinewidth{1.003750pt}%
\definecolor{currentstroke}{rgb}{0.667253,0.779176,0.992959}%
\pgfsetstrokecolor{currentstroke}%
\pgfsetdash{}{0pt}%
\pgfpathmoveto{\pgfqpoint{1.552230in}{0.942332in}}%
\pgfpathlineto{\pgfqpoint{1.552230in}{0.942332in}}%
\pgfusepath{stroke}%
\end{pgfscope}%
\begin{pgfscope}%
\pgfpathrectangle{\pgfqpoint{0.552773in}{0.431673in}}{\pgfqpoint{3.738807in}{1.765244in}}%
\pgfusepath{clip}%
\pgfsetbuttcap%
\pgfsetroundjoin%
\pgfsetlinewidth{1.003750pt}%
\definecolor{currentstroke}{rgb}{0.667253,0.779176,0.992959}%
\pgfsetstrokecolor{currentstroke}%
\pgfsetdash{}{0pt}%
\pgfpathmoveto{\pgfqpoint{1.571226in}{0.917025in}}%
\pgfpathlineto{\pgfqpoint{1.571226in}{0.917025in}}%
\pgfusepath{stroke}%
\end{pgfscope}%
\begin{pgfscope}%
\pgfpathrectangle{\pgfqpoint{0.552773in}{0.431673in}}{\pgfqpoint{3.738807in}{1.765244in}}%
\pgfusepath{clip}%
\pgfsetbuttcap%
\pgfsetroundjoin%
\pgfsetlinewidth{1.003750pt}%
\definecolor{currentstroke}{rgb}{0.667253,0.779176,0.992959}%
\pgfsetstrokecolor{currentstroke}%
\pgfsetdash{}{0pt}%
\pgfpathmoveto{\pgfqpoint{1.578686in}{0.870715in}}%
\pgfpathlineto{\pgfqpoint{1.578686in}{0.870715in}}%
\pgfusepath{stroke}%
\end{pgfscope}%
\begin{pgfscope}%
\pgfpathrectangle{\pgfqpoint{0.552773in}{0.431673in}}{\pgfqpoint{3.738807in}{1.765244in}}%
\pgfusepath{clip}%
\pgfsetbuttcap%
\pgfsetroundjoin%
\pgfsetlinewidth{1.003750pt}%
\definecolor{currentstroke}{rgb}{0.667253,0.779176,0.992959}%
\pgfsetstrokecolor{currentstroke}%
\pgfsetdash{}{0pt}%
\pgfpathmoveto{\pgfqpoint{1.598839in}{0.847513in}}%
\pgfpathlineto{\pgfqpoint{1.598839in}{0.847513in}}%
\pgfusepath{stroke}%
\end{pgfscope}%
\begin{pgfscope}%
\pgfpathrectangle{\pgfqpoint{0.552773in}{0.431673in}}{\pgfqpoint{3.738807in}{1.765244in}}%
\pgfusepath{clip}%
\pgfsetbuttcap%
\pgfsetroundjoin%
\pgfsetlinewidth{1.003750pt}%
\definecolor{currentstroke}{rgb}{0.667253,0.779176,0.992959}%
\pgfsetstrokecolor{currentstroke}%
\pgfsetdash{}{0pt}%
\pgfpathmoveto{\pgfqpoint{1.615264in}{0.817524in}}%
\pgfpathlineto{\pgfqpoint{1.615264in}{0.817524in}}%
\pgfusepath{stroke}%
\end{pgfscope}%
\begin{pgfscope}%
\pgfpathrectangle{\pgfqpoint{0.552773in}{0.431673in}}{\pgfqpoint{3.738807in}{1.765244in}}%
\pgfusepath{clip}%
\pgfsetbuttcap%
\pgfsetroundjoin%
\pgfsetlinewidth{1.003750pt}%
\definecolor{currentstroke}{rgb}{0.667253,0.779176,0.992959}%
\pgfsetstrokecolor{currentstroke}%
\pgfsetdash{}{0pt}%
\pgfpathmoveto{\pgfqpoint{1.629986in}{0.784436in}}%
\pgfpathlineto{\pgfqpoint{1.629986in}{0.784436in}}%
\pgfusepath{stroke}%
\end{pgfscope}%
\begin{pgfscope}%
\pgfpathrectangle{\pgfqpoint{0.552773in}{0.431673in}}{\pgfqpoint{3.738807in}{1.765244in}}%
\pgfusepath{clip}%
\pgfsetbuttcap%
\pgfsetroundjoin%
\pgfsetlinewidth{1.003750pt}%
\definecolor{currentstroke}{rgb}{0.667253,0.779176,0.992959}%
\pgfsetstrokecolor{currentstroke}%
\pgfsetdash{}{0pt}%
\pgfpathmoveto{\pgfqpoint{1.633661in}{0.731234in}}%
\pgfpathlineto{\pgfqpoint{1.633661in}{0.731234in}}%
\pgfusepath{stroke}%
\end{pgfscope}%
\begin{pgfscope}%
\pgfpathrectangle{\pgfqpoint{0.552773in}{0.431673in}}{\pgfqpoint{3.738807in}{1.765244in}}%
\pgfusepath{clip}%
\pgfsetbuttcap%
\pgfsetroundjoin%
\pgfsetlinewidth{1.003750pt}%
\definecolor{currentstroke}{rgb}{0.667253,0.779176,0.992959}%
\pgfsetstrokecolor{currentstroke}%
\pgfsetdash{}{0pt}%
\pgfpathmoveto{\pgfqpoint{1.644849in}{0.691711in}}%
\pgfpathlineto{\pgfqpoint{1.644849in}{0.691711in}}%
\pgfusepath{stroke}%
\end{pgfscope}%
\begin{pgfscope}%
\pgfpathrectangle{\pgfqpoint{0.552773in}{0.431673in}}{\pgfqpoint{3.738807in}{1.765244in}}%
\pgfusepath{clip}%
\pgfsetbuttcap%
\pgfsetroundjoin%
\pgfsetlinewidth{1.003750pt}%
\definecolor{currentstroke}{rgb}{0.667253,0.779176,0.992959}%
\pgfsetstrokecolor{currentstroke}%
\pgfsetdash{}{0pt}%
\pgfpathmoveto{\pgfqpoint{1.654597in}{0.649565in}}%
\pgfpathlineto{\pgfqpoint{1.654597in}{0.649565in}}%
\pgfusepath{stroke}%
\end{pgfscope}%
\begin{pgfscope}%
\pgfpathrectangle{\pgfqpoint{0.552773in}{0.431673in}}{\pgfqpoint{3.738807in}{1.765244in}}%
\pgfusepath{clip}%
\pgfsetbuttcap%
\pgfsetroundjoin%
\pgfsetlinewidth{1.003750pt}%
\definecolor{currentstroke}{rgb}{0.667253,0.779176,0.992959}%
\pgfsetstrokecolor{currentstroke}%
\pgfsetdash{}{0pt}%
\pgfpathmoveto{\pgfqpoint{1.665591in}{0.609687in}}%
\pgfpathlineto{\pgfqpoint{1.665591in}{0.609687in}}%
\pgfusepath{stroke}%
\end{pgfscope}%
\begin{pgfscope}%
\pgfpathrectangle{\pgfqpoint{0.552773in}{0.431673in}}{\pgfqpoint{3.738807in}{1.765244in}}%
\pgfusepath{clip}%
\pgfsetbuttcap%
\pgfsetroundjoin%
\pgfsetlinewidth{1.003750pt}%
\definecolor{currentstroke}{rgb}{0.667253,0.779176,0.992959}%
\pgfsetstrokecolor{currentstroke}%
\pgfsetdash{}{0pt}%
\pgfpathmoveto{\pgfqpoint{1.686427in}{0.587731in}}%
\pgfpathlineto{\pgfqpoint{1.686427in}{0.587731in}}%
\pgfusepath{stroke}%
\end{pgfscope}%
\begin{pgfscope}%
\pgfpathrectangle{\pgfqpoint{0.552773in}{0.431673in}}{\pgfqpoint{3.738807in}{1.765244in}}%
\pgfusepath{clip}%
\pgfsetbuttcap%
\pgfsetroundjoin%
\pgfsetlinewidth{1.003750pt}%
\definecolor{currentstroke}{rgb}{0.667253,0.779176,0.992959}%
\pgfsetstrokecolor{currentstroke}%
\pgfsetdash{}{0pt}%
\pgfpathmoveto{\pgfqpoint{1.711171in}{0.572888in}}%
\pgfpathlineto{\pgfqpoint{1.711171in}{0.572888in}}%
\pgfusepath{stroke}%
\end{pgfscope}%
\begin{pgfscope}%
\pgfpathrectangle{\pgfqpoint{0.552773in}{0.431673in}}{\pgfqpoint{3.738807in}{1.765244in}}%
\pgfusepath{clip}%
\pgfsetbuttcap%
\pgfsetroundjoin%
\pgfsetlinewidth{1.003750pt}%
\definecolor{currentstroke}{rgb}{0.667253,0.779176,0.992959}%
\pgfsetstrokecolor{currentstroke}%
\pgfsetdash{}{0pt}%
\pgfpathmoveto{\pgfqpoint{1.758001in}{0.598259in}}%
\pgfpathlineto{\pgfqpoint{1.758001in}{0.598259in}}%
\pgfusepath{stroke}%
\end{pgfscope}%
\begin{pgfscope}%
\pgfpathrectangle{\pgfqpoint{0.552773in}{0.431673in}}{\pgfqpoint{3.738807in}{1.765244in}}%
\pgfusepath{clip}%
\pgfsetbuttcap%
\pgfsetroundjoin%
\pgfsetlinewidth{1.003750pt}%
\definecolor{currentstroke}{rgb}{0.667253,0.779176,0.992959}%
\pgfsetstrokecolor{currentstroke}%
\pgfsetdash{}{0pt}%
\pgfpathmoveto{\pgfqpoint{1.816028in}{0.644014in}}%
\pgfpathlineto{\pgfqpoint{1.816028in}{0.644014in}}%
\pgfusepath{stroke}%
\end{pgfscope}%
\begin{pgfscope}%
\pgfpathrectangle{\pgfqpoint{0.552773in}{0.431673in}}{\pgfqpoint{3.738807in}{1.765244in}}%
\pgfusepath{clip}%
\pgfsetbuttcap%
\pgfsetroundjoin%
\pgfsetlinewidth{1.003750pt}%
\definecolor{currentstroke}{rgb}{0.667253,0.779176,0.992959}%
\pgfsetstrokecolor{currentstroke}%
\pgfsetdash{}{0pt}%
\pgfpathmoveto{\pgfqpoint{1.879012in}{0.698794in}}%
\pgfpathlineto{\pgfqpoint{1.879012in}{0.698794in}}%
\pgfusepath{stroke}%
\end{pgfscope}%
\begin{pgfscope}%
\pgfpathrectangle{\pgfqpoint{0.552773in}{0.431673in}}{\pgfqpoint{3.738807in}{1.765244in}}%
\pgfusepath{clip}%
\pgfsetbuttcap%
\pgfsetroundjoin%
\pgfsetlinewidth{1.003750pt}%
\definecolor{currentstroke}{rgb}{0.667253,0.779176,0.992959}%
\pgfsetstrokecolor{currentstroke}%
\pgfsetdash{}{0pt}%
\pgfpathmoveto{\pgfqpoint{1.944926in}{0.758909in}}%
\pgfpathlineto{\pgfqpoint{1.944926in}{0.758909in}}%
\pgfusepath{stroke}%
\end{pgfscope}%
\begin{pgfscope}%
\pgfpathrectangle{\pgfqpoint{0.552773in}{0.431673in}}{\pgfqpoint{3.738807in}{1.765244in}}%
\pgfusepath{clip}%
\pgfsetbuttcap%
\pgfsetroundjoin%
\pgfsetlinewidth{1.003750pt}%
\definecolor{currentstroke}{rgb}{0.667253,0.779176,0.992959}%
\pgfsetstrokecolor{currentstroke}%
\pgfsetdash{}{0pt}%
\pgfpathmoveto{\pgfqpoint{2.017564in}{0.831267in}}%
\pgfpathlineto{\pgfqpoint{2.017564in}{0.831267in}}%
\pgfusepath{stroke}%
\end{pgfscope}%
\begin{pgfscope}%
\pgfpathrectangle{\pgfqpoint{0.552773in}{0.431673in}}{\pgfqpoint{3.738807in}{1.765244in}}%
\pgfusepath{clip}%
\pgfsetbuttcap%
\pgfsetroundjoin%
\pgfsetlinewidth{1.003750pt}%
\definecolor{currentstroke}{rgb}{0.667253,0.779176,0.992959}%
\pgfsetstrokecolor{currentstroke}%
\pgfsetdash{}{0pt}%
\pgfpathmoveto{\pgfqpoint{2.077394in}{0.880307in}}%
\pgfpathlineto{\pgfqpoint{2.077394in}{0.880307in}}%
\pgfusepath{stroke}%
\end{pgfscope}%
\begin{pgfscope}%
\pgfpathrectangle{\pgfqpoint{0.552773in}{0.431673in}}{\pgfqpoint{3.738807in}{1.765244in}}%
\pgfusepath{clip}%
\pgfsetbuttcap%
\pgfsetroundjoin%
\pgfsetlinewidth{1.003750pt}%
\definecolor{currentstroke}{rgb}{0.667253,0.779176,0.992959}%
\pgfsetstrokecolor{currentstroke}%
\pgfsetdash{}{0pt}%
\pgfpathmoveto{\pgfqpoint{2.120452in}{0.898808in}}%
\pgfpathlineto{\pgfqpoint{2.120452in}{0.898808in}}%
\pgfusepath{stroke}%
\end{pgfscope}%
\begin{pgfscope}%
\pgfpathrectangle{\pgfqpoint{0.552773in}{0.431673in}}{\pgfqpoint{3.738807in}{1.765244in}}%
\pgfusepath{clip}%
\pgfsetbuttcap%
\pgfsetroundjoin%
\pgfsetlinewidth{1.003750pt}%
\definecolor{currentstroke}{rgb}{0.667253,0.779176,0.992959}%
\pgfsetstrokecolor{currentstroke}%
\pgfsetdash{}{0pt}%
\pgfpathmoveto{\pgfqpoint{2.148843in}{0.890606in}}%
\pgfpathlineto{\pgfqpoint{2.148843in}{0.890606in}}%
\pgfusepath{stroke}%
\end{pgfscope}%
\begin{pgfscope}%
\pgfpathrectangle{\pgfqpoint{0.552773in}{0.431673in}}{\pgfqpoint{3.738807in}{1.765244in}}%
\pgfusepath{clip}%
\pgfsetbuttcap%
\pgfsetroundjoin%
\pgfsetlinewidth{1.003750pt}%
\definecolor{currentstroke}{rgb}{0.667253,0.779176,0.992959}%
\pgfsetstrokecolor{currentstroke}%
\pgfsetdash{}{0pt}%
\pgfpathmoveto{\pgfqpoint{2.169787in}{0.868845in}}%
\pgfpathlineto{\pgfqpoint{2.169787in}{0.868845in}}%
\pgfusepath{stroke}%
\end{pgfscope}%
\begin{pgfscope}%
\pgfpathrectangle{\pgfqpoint{0.552773in}{0.431673in}}{\pgfqpoint{3.738807in}{1.765244in}}%
\pgfusepath{clip}%
\pgfsetbuttcap%
\pgfsetroundjoin%
\pgfsetlinewidth{1.003750pt}%
\definecolor{currentstroke}{rgb}{0.667253,0.779176,0.992959}%
\pgfsetstrokecolor{currentstroke}%
\pgfsetdash{}{0pt}%
\pgfpathmoveto{\pgfqpoint{2.191999in}{0.849394in}}%
\pgfpathlineto{\pgfqpoint{2.191999in}{0.849394in}}%
\pgfusepath{stroke}%
\end{pgfscope}%
\begin{pgfscope}%
\pgfpathrectangle{\pgfqpoint{0.552773in}{0.431673in}}{\pgfqpoint{3.738807in}{1.765244in}}%
\pgfusepath{clip}%
\pgfsetbuttcap%
\pgfsetroundjoin%
\pgfsetlinewidth{1.003750pt}%
\definecolor{currentstroke}{rgb}{0.667253,0.779176,0.992959}%
\pgfsetstrokecolor{currentstroke}%
\pgfsetdash{}{0pt}%
\pgfpathmoveto{\pgfqpoint{2.201330in}{0.806489in}}%
\pgfpathlineto{\pgfqpoint{2.201330in}{0.806489in}}%
\pgfusepath{stroke}%
\end{pgfscope}%
\begin{pgfscope}%
\pgfpathrectangle{\pgfqpoint{0.552773in}{0.431673in}}{\pgfqpoint{3.738807in}{1.765244in}}%
\pgfusepath{clip}%
\pgfsetbuttcap%
\pgfsetroundjoin%
\pgfsetlinewidth{1.003750pt}%
\definecolor{currentstroke}{rgb}{0.667253,0.779176,0.992959}%
\pgfsetstrokecolor{currentstroke}%
\pgfsetdash{}{0pt}%
\pgfpathmoveto{\pgfqpoint{2.202600in}{0.748907in}}%
\pgfpathlineto{\pgfqpoint{2.202600in}{0.748907in}}%
\pgfusepath{stroke}%
\end{pgfscope}%
\begin{pgfscope}%
\pgfpathrectangle{\pgfqpoint{0.552773in}{0.431673in}}{\pgfqpoint{3.738807in}{1.765244in}}%
\pgfusepath{clip}%
\pgfsetbuttcap%
\pgfsetroundjoin%
\pgfsetlinewidth{1.003750pt}%
\definecolor{currentstroke}{rgb}{0.667253,0.779176,0.992959}%
\pgfsetstrokecolor{currentstroke}%
\pgfsetdash{}{0pt}%
\pgfpathmoveto{\pgfqpoint{2.202588in}{0.688992in}}%
\pgfpathlineto{\pgfqpoint{2.202588in}{0.688992in}}%
\pgfusepath{stroke}%
\end{pgfscope}%
\begin{pgfscope}%
\pgfpathrectangle{\pgfqpoint{0.552773in}{0.431673in}}{\pgfqpoint{3.738807in}{1.765244in}}%
\pgfusepath{clip}%
\pgfsetbuttcap%
\pgfsetroundjoin%
\pgfsetlinewidth{1.003750pt}%
\definecolor{currentstroke}{rgb}{0.667253,0.779176,0.992959}%
\pgfsetstrokecolor{currentstroke}%
\pgfsetdash{}{0pt}%
\pgfpathmoveto{\pgfqpoint{2.206526in}{0.636268in}}%
\pgfpathlineto{\pgfqpoint{2.206526in}{0.636268in}}%
\pgfusepath{stroke}%
\end{pgfscope}%
\begin{pgfscope}%
\pgfpathrectangle{\pgfqpoint{0.552773in}{0.431673in}}{\pgfqpoint{3.738807in}{1.765244in}}%
\pgfusepath{clip}%
\pgfsetbuttcap%
\pgfsetroundjoin%
\pgfsetlinewidth{1.003750pt}%
\definecolor{currentstroke}{rgb}{0.667253,0.779176,0.992959}%
\pgfsetstrokecolor{currentstroke}%
\pgfsetdash{}{0pt}%
\pgfpathmoveto{\pgfqpoint{2.205818in}{0.575086in}}%
\pgfpathlineto{\pgfqpoint{2.205818in}{0.575086in}}%
\pgfusepath{stroke}%
\end{pgfscope}%
\begin{pgfscope}%
\pgfpathrectangle{\pgfqpoint{0.552773in}{0.431673in}}{\pgfqpoint{3.738807in}{1.765244in}}%
\pgfusepath{clip}%
\pgfsetbuttcap%
\pgfsetroundjoin%
\pgfsetlinewidth{1.003750pt}%
\definecolor{currentstroke}{rgb}{0.667253,0.779176,0.992959}%
\pgfsetstrokecolor{currentstroke}%
\pgfsetdash{}{0pt}%
\pgfpathmoveto{\pgfqpoint{2.225473in}{0.550978in}}%
\pgfpathlineto{\pgfqpoint{2.225473in}{0.550978in}}%
\pgfusepath{stroke}%
\end{pgfscope}%
\begin{pgfscope}%
\pgfpathrectangle{\pgfqpoint{0.552773in}{0.431673in}}{\pgfqpoint{3.738807in}{1.765244in}}%
\pgfusepath{clip}%
\pgfsetbuttcap%
\pgfsetroundjoin%
\pgfsetlinewidth{1.003750pt}%
\definecolor{currentstroke}{rgb}{0.667253,0.779176,0.992959}%
\pgfsetstrokecolor{currentstroke}%
\pgfsetdash{}{0pt}%
\pgfpathmoveto{\pgfqpoint{2.247923in}{0.531960in}}%
\pgfpathlineto{\pgfqpoint{2.247923in}{0.531960in}}%
\pgfusepath{stroke}%
\end{pgfscope}%
\begin{pgfscope}%
\pgfpathrectangle{\pgfqpoint{0.552773in}{0.431673in}}{\pgfqpoint{3.738807in}{1.765244in}}%
\pgfusepath{clip}%
\pgfsetbuttcap%
\pgfsetroundjoin%
\pgfsetlinewidth{1.003750pt}%
\definecolor{currentstroke}{rgb}{0.667253,0.779176,0.992959}%
\pgfsetstrokecolor{currentstroke}%
\pgfsetdash{}{0pt}%
\pgfpathmoveto{\pgfqpoint{2.269807in}{0.511911in}}%
\pgfpathlineto{\pgfqpoint{2.269807in}{0.511911in}}%
\pgfusepath{stroke}%
\end{pgfscope}%
\begin{pgfscope}%
\pgfpathrectangle{\pgfqpoint{0.552773in}{0.431673in}}{\pgfqpoint{3.738807in}{1.765244in}}%
\pgfusepath{clip}%
\pgfsetbuttcap%
\pgfsetroundjoin%
\pgfsetlinewidth{1.003750pt}%
\definecolor{currentstroke}{rgb}{0.667253,0.779176,0.992959}%
\pgfsetstrokecolor{currentstroke}%
\pgfsetdash{}{0pt}%
\pgfpathmoveto{\pgfqpoint{2.310165in}{0.525497in}}%
\pgfpathlineto{\pgfqpoint{2.310165in}{0.525497in}}%
\pgfusepath{stroke}%
\end{pgfscope}%
\begin{pgfscope}%
\pgfpathrectangle{\pgfqpoint{0.552773in}{0.431673in}}{\pgfqpoint{3.738807in}{1.765244in}}%
\pgfusepath{clip}%
\pgfsetbuttcap%
\pgfsetroundjoin%
\pgfsetlinewidth{1.003750pt}%
\definecolor{currentstroke}{rgb}{0.667253,0.779176,0.992959}%
\pgfsetstrokecolor{currentstroke}%
\pgfsetdash{}{0pt}%
\pgfpathmoveto{\pgfqpoint{2.369172in}{0.573036in}}%
\pgfpathlineto{\pgfqpoint{2.369172in}{0.573036in}}%
\pgfusepath{stroke}%
\end{pgfscope}%
\begin{pgfscope}%
\pgfpathrectangle{\pgfqpoint{0.552773in}{0.431673in}}{\pgfqpoint{3.738807in}{1.765244in}}%
\pgfusepath{clip}%
\pgfsetbuttcap%
\pgfsetroundjoin%
\pgfsetlinewidth{1.003750pt}%
\definecolor{currentstroke}{rgb}{0.667253,0.779176,0.992959}%
\pgfsetstrokecolor{currentstroke}%
\pgfsetdash{}{0pt}%
\pgfpathmoveto{\pgfqpoint{2.433038in}{0.629423in}}%
\pgfpathlineto{\pgfqpoint{2.433038in}{0.629423in}}%
\pgfusepath{stroke}%
\end{pgfscope}%
\begin{pgfscope}%
\pgfpathrectangle{\pgfqpoint{0.552773in}{0.431673in}}{\pgfqpoint{3.738807in}{1.765244in}}%
\pgfusepath{clip}%
\pgfsetbuttcap%
\pgfsetroundjoin%
\pgfsetlinewidth{1.003750pt}%
\definecolor{currentstroke}{rgb}{0.667253,0.779176,0.992959}%
\pgfsetstrokecolor{currentstroke}%
\pgfsetdash{}{0pt}%
\pgfpathmoveto{\pgfqpoint{2.512551in}{0.714298in}}%
\pgfpathlineto{\pgfqpoint{2.512551in}{0.714298in}}%
\pgfusepath{stroke}%
\end{pgfscope}%
\begin{pgfscope}%
\pgfpathrectangle{\pgfqpoint{0.552773in}{0.431673in}}{\pgfqpoint{3.738807in}{1.765244in}}%
\pgfusepath{clip}%
\pgfsetbuttcap%
\pgfsetroundjoin%
\pgfsetlinewidth{1.003750pt}%
\definecolor{currentstroke}{rgb}{0.667253,0.779176,0.992959}%
\pgfsetstrokecolor{currentstroke}%
\pgfsetdash{}{0pt}%
\pgfpathmoveto{\pgfqpoint{2.584448in}{0.785307in}}%
\pgfpathlineto{\pgfqpoint{2.584448in}{0.785307in}}%
\pgfusepath{stroke}%
\end{pgfscope}%
\begin{pgfscope}%
\pgfpathrectangle{\pgfqpoint{0.552773in}{0.431673in}}{\pgfqpoint{3.738807in}{1.765244in}}%
\pgfusepath{clip}%
\pgfsetbuttcap%
\pgfsetroundjoin%
\pgfsetlinewidth{1.003750pt}%
\definecolor{currentstroke}{rgb}{0.667253,0.779176,0.992959}%
\pgfsetstrokecolor{currentstroke}%
\pgfsetdash{}{0pt}%
\pgfpathmoveto{\pgfqpoint{2.654396in}{0.852767in}}%
\pgfpathlineto{\pgfqpoint{2.654396in}{0.852767in}}%
\pgfusepath{stroke}%
\end{pgfscope}%
\begin{pgfscope}%
\pgfpathrectangle{\pgfqpoint{0.552773in}{0.431673in}}{\pgfqpoint{3.738807in}{1.765244in}}%
\pgfusepath{clip}%
\pgfsetbuttcap%
\pgfsetroundjoin%
\pgfsetlinewidth{1.003750pt}%
\definecolor{currentstroke}{rgb}{0.667253,0.779176,0.992959}%
\pgfsetstrokecolor{currentstroke}%
\pgfsetdash{}{0pt}%
\pgfpathmoveto{\pgfqpoint{2.709975in}{0.894066in}}%
\pgfpathlineto{\pgfqpoint{2.709975in}{0.894066in}}%
\pgfusepath{stroke}%
\end{pgfscope}%
\begin{pgfscope}%
\pgfpathrectangle{\pgfqpoint{0.552773in}{0.431673in}}{\pgfqpoint{3.738807in}{1.765244in}}%
\pgfusepath{clip}%
\pgfsetbuttcap%
\pgfsetroundjoin%
\pgfsetlinewidth{1.003750pt}%
\definecolor{currentstroke}{rgb}{0.667253,0.779176,0.992959}%
\pgfsetstrokecolor{currentstroke}%
\pgfsetdash{}{0pt}%
\pgfpathmoveto{\pgfqpoint{2.745485in}{0.898825in}}%
\pgfpathlineto{\pgfqpoint{2.745485in}{0.898825in}}%
\pgfusepath{stroke}%
\end{pgfscope}%
\begin{pgfscope}%
\pgfpathrectangle{\pgfqpoint{0.552773in}{0.431673in}}{\pgfqpoint{3.738807in}{1.765244in}}%
\pgfusepath{clip}%
\pgfsetbuttcap%
\pgfsetroundjoin%
\pgfsetlinewidth{1.003750pt}%
\definecolor{currentstroke}{rgb}{0.667253,0.779176,0.992959}%
\pgfsetstrokecolor{currentstroke}%
\pgfsetdash{}{0pt}%
\pgfpathmoveto{\pgfqpoint{2.765939in}{0.876173in}}%
\pgfpathlineto{\pgfqpoint{2.765939in}{0.876173in}}%
\pgfusepath{stroke}%
\end{pgfscope}%
\begin{pgfscope}%
\pgfpathrectangle{\pgfqpoint{0.552773in}{0.431673in}}{\pgfqpoint{3.738807in}{1.765244in}}%
\pgfusepath{clip}%
\pgfsetbuttcap%
\pgfsetroundjoin%
\pgfsetlinewidth{1.003750pt}%
\definecolor{currentstroke}{rgb}{0.667253,0.779176,0.992959}%
\pgfsetstrokecolor{currentstroke}%
\pgfsetdash{}{0pt}%
\pgfpathmoveto{\pgfqpoint{2.777619in}{0.837544in}}%
\pgfpathlineto{\pgfqpoint{2.777619in}{0.837544in}}%
\pgfusepath{stroke}%
\end{pgfscope}%
\begin{pgfscope}%
\pgfpathrectangle{\pgfqpoint{0.552773in}{0.431673in}}{\pgfqpoint{3.738807in}{1.765244in}}%
\pgfusepath{clip}%
\pgfsetbuttcap%
\pgfsetroundjoin%
\pgfsetlinewidth{1.003750pt}%
\definecolor{currentstroke}{rgb}{0.667253,0.779176,0.992959}%
\pgfsetstrokecolor{currentstroke}%
\pgfsetdash{}{0pt}%
\pgfpathmoveto{\pgfqpoint{2.773219in}{0.769640in}}%
\pgfpathlineto{\pgfqpoint{2.773219in}{0.769640in}}%
\pgfusepath{stroke}%
\end{pgfscope}%
\begin{pgfscope}%
\pgfpathrectangle{\pgfqpoint{0.552773in}{0.431673in}}{\pgfqpoint{3.738807in}{1.765244in}}%
\pgfusepath{clip}%
\pgfsetbuttcap%
\pgfsetroundjoin%
\pgfsetlinewidth{1.003750pt}%
\definecolor{currentstroke}{rgb}{0.667253,0.779176,0.992959}%
\pgfsetstrokecolor{currentstroke}%
\pgfsetdash{}{0pt}%
\pgfpathmoveto{\pgfqpoint{2.773093in}{0.709516in}}%
\pgfpathlineto{\pgfqpoint{2.773093in}{0.709516in}}%
\pgfusepath{stroke}%
\end{pgfscope}%
\begin{pgfscope}%
\pgfpathrectangle{\pgfqpoint{0.552773in}{0.431673in}}{\pgfqpoint{3.738807in}{1.765244in}}%
\pgfusepath{clip}%
\pgfsetbuttcap%
\pgfsetroundjoin%
\pgfsetlinewidth{1.003750pt}%
\definecolor{currentstroke}{rgb}{0.667253,0.779176,0.992959}%
\pgfsetstrokecolor{currentstroke}%
\pgfsetdash{}{0pt}%
\pgfpathmoveto{\pgfqpoint{2.775290in}{0.653624in}}%
\pgfpathlineto{\pgfqpoint{2.775290in}{0.653624in}}%
\pgfusepath{stroke}%
\end{pgfscope}%
\begin{pgfscope}%
\pgfpathrectangle{\pgfqpoint{0.552773in}{0.431673in}}{\pgfqpoint{3.738807in}{1.765244in}}%
\pgfusepath{clip}%
\pgfsetbuttcap%
\pgfsetroundjoin%
\pgfsetlinewidth{1.003750pt}%
\definecolor{currentstroke}{rgb}{0.667253,0.779176,0.992959}%
\pgfsetstrokecolor{currentstroke}%
\pgfsetdash{}{0pt}%
\pgfpathmoveto{\pgfqpoint{2.792438in}{0.624951in}}%
\pgfpathlineto{\pgfqpoint{2.792438in}{0.624951in}}%
\pgfusepath{stroke}%
\end{pgfscope}%
\begin{pgfscope}%
\pgfpathrectangle{\pgfqpoint{0.552773in}{0.431673in}}{\pgfqpoint{3.738807in}{1.765244in}}%
\pgfusepath{clip}%
\pgfsetbuttcap%
\pgfsetroundjoin%
\pgfsetlinewidth{1.003750pt}%
\definecolor{currentstroke}{rgb}{0.667253,0.779176,0.992959}%
\pgfsetstrokecolor{currentstroke}%
\pgfsetdash{}{0pt}%
\pgfpathmoveto{\pgfqpoint{2.801363in}{0.581308in}}%
\pgfpathlineto{\pgfqpoint{2.801363in}{0.581308in}}%
\pgfusepath{stroke}%
\end{pgfscope}%
\begin{pgfscope}%
\pgfpathrectangle{\pgfqpoint{0.552773in}{0.431673in}}{\pgfqpoint{3.738807in}{1.765244in}}%
\pgfusepath{clip}%
\pgfsetbuttcap%
\pgfsetroundjoin%
\pgfsetlinewidth{1.003750pt}%
\definecolor{currentstroke}{rgb}{0.667253,0.779176,0.992959}%
\pgfsetstrokecolor{currentstroke}%
\pgfsetdash{}{0pt}%
\pgfpathmoveto{\pgfqpoint{2.816359in}{0.548717in}}%
\pgfpathlineto{\pgfqpoint{2.816359in}{0.548717in}}%
\pgfusepath{stroke}%
\end{pgfscope}%
\begin{pgfscope}%
\pgfpathrectangle{\pgfqpoint{0.552773in}{0.431673in}}{\pgfqpoint{3.738807in}{1.765244in}}%
\pgfusepath{clip}%
\pgfsetbuttcap%
\pgfsetroundjoin%
\pgfsetlinewidth{1.003750pt}%
\definecolor{currentstroke}{rgb}{0.667253,0.779176,0.992959}%
\pgfsetstrokecolor{currentstroke}%
\pgfsetdash{}{0pt}%
\pgfpathmoveto{\pgfqpoint{2.831119in}{0.515697in}}%
\pgfpathlineto{\pgfqpoint{2.831119in}{0.515697in}}%
\pgfusepath{stroke}%
\end{pgfscope}%
\begin{pgfscope}%
\pgfpathrectangle{\pgfqpoint{0.552773in}{0.431673in}}{\pgfqpoint{3.738807in}{1.765244in}}%
\pgfusepath{clip}%
\pgfsetbuttcap%
\pgfsetroundjoin%
\pgfsetlinewidth{1.003750pt}%
\definecolor{currentstroke}{rgb}{0.667253,0.779176,0.992959}%
\pgfsetstrokecolor{currentstroke}%
\pgfsetdash{}{0pt}%
\pgfpathmoveto{\pgfqpoint{2.870334in}{0.527203in}}%
\pgfpathlineto{\pgfqpoint{2.870334in}{0.527203in}}%
\pgfusepath{stroke}%
\end{pgfscope}%
\begin{pgfscope}%
\pgfpathrectangle{\pgfqpoint{0.552773in}{0.431673in}}{\pgfqpoint{3.738807in}{1.765244in}}%
\pgfusepath{clip}%
\pgfsetbuttcap%
\pgfsetroundjoin%
\pgfsetlinewidth{1.003750pt}%
\definecolor{currentstroke}{rgb}{0.667253,0.779176,0.992959}%
\pgfsetstrokecolor{currentstroke}%
\pgfsetdash{}{0pt}%
\pgfpathmoveto{\pgfqpoint{2.914301in}{0.547359in}}%
\pgfpathlineto{\pgfqpoint{2.914301in}{0.547359in}}%
\pgfusepath{stroke}%
\end{pgfscope}%
\begin{pgfscope}%
\pgfpathrectangle{\pgfqpoint{0.552773in}{0.431673in}}{\pgfqpoint{3.738807in}{1.765244in}}%
\pgfusepath{clip}%
\pgfsetbuttcap%
\pgfsetroundjoin%
\pgfsetlinewidth{1.003750pt}%
\definecolor{currentstroke}{rgb}{0.667253,0.779176,0.992959}%
\pgfsetstrokecolor{currentstroke}%
\pgfsetdash{}{0pt}%
\pgfpathmoveto{\pgfqpoint{2.968424in}{0.586007in}}%
\pgfpathlineto{\pgfqpoint{2.968424in}{0.586007in}}%
\pgfusepath{stroke}%
\end{pgfscope}%
\begin{pgfscope}%
\pgfpathrectangle{\pgfqpoint{0.552773in}{0.431673in}}{\pgfqpoint{3.738807in}{1.765244in}}%
\pgfusepath{clip}%
\pgfsetbuttcap%
\pgfsetroundjoin%
\pgfsetlinewidth{1.003750pt}%
\definecolor{currentstroke}{rgb}{0.667253,0.779176,0.992959}%
\pgfsetstrokecolor{currentstroke}%
\pgfsetdash{}{0pt}%
\pgfpathmoveto{\pgfqpoint{3.031090in}{0.640210in}}%
\pgfpathlineto{\pgfqpoint{3.031090in}{0.640210in}}%
\pgfusepath{stroke}%
\end{pgfscope}%
\begin{pgfscope}%
\pgfpathrectangle{\pgfqpoint{0.552773in}{0.431673in}}{\pgfqpoint{3.738807in}{1.765244in}}%
\pgfusepath{clip}%
\pgfsetbuttcap%
\pgfsetroundjoin%
\pgfsetlinewidth{1.003750pt}%
\definecolor{currentstroke}{rgb}{0.667253,0.779176,0.992959}%
\pgfsetstrokecolor{currentstroke}%
\pgfsetdash{}{0pt}%
\pgfpathmoveto{\pgfqpoint{3.108456in}{0.721175in}}%
\pgfpathlineto{\pgfqpoint{3.108456in}{0.721175in}}%
\pgfusepath{stroke}%
\end{pgfscope}%
\begin{pgfscope}%
\pgfpathrectangle{\pgfqpoint{0.552773in}{0.431673in}}{\pgfqpoint{3.738807in}{1.765244in}}%
\pgfusepath{clip}%
\pgfsetbuttcap%
\pgfsetroundjoin%
\pgfsetlinewidth{1.003750pt}%
\definecolor{currentstroke}{rgb}{0.667253,0.779176,0.992959}%
\pgfsetstrokecolor{currentstroke}%
\pgfsetdash{}{0pt}%
\pgfpathmoveto{\pgfqpoint{3.183879in}{0.798603in}}%
\pgfpathlineto{\pgfqpoint{3.183879in}{0.798603in}}%
\pgfusepath{stroke}%
\end{pgfscope}%
\begin{pgfscope}%
\pgfpathrectangle{\pgfqpoint{0.552773in}{0.431673in}}{\pgfqpoint{3.738807in}{1.765244in}}%
\pgfusepath{clip}%
\pgfsetbuttcap%
\pgfsetroundjoin%
\pgfsetlinewidth{1.003750pt}%
\definecolor{currentstroke}{rgb}{0.667253,0.779176,0.992959}%
\pgfsetstrokecolor{currentstroke}%
\pgfsetdash{}{0pt}%
\pgfpathmoveto{\pgfqpoint{3.274791in}{0.904233in}}%
\pgfpathlineto{\pgfqpoint{3.274791in}{0.904233in}}%
\pgfusepath{stroke}%
\end{pgfscope}%
\begin{pgfscope}%
\pgfpathrectangle{\pgfqpoint{0.552773in}{0.431673in}}{\pgfqpoint{3.738807in}{1.765244in}}%
\pgfusepath{clip}%
\pgfsetbuttcap%
\pgfsetroundjoin%
\pgfsetlinewidth{1.003750pt}%
\definecolor{currentstroke}{rgb}{0.667253,0.779176,0.992959}%
\pgfsetstrokecolor{currentstroke}%
\pgfsetdash{}{0pt}%
\pgfpathmoveto{\pgfqpoint{3.351275in}{0.983593in}}%
\pgfpathlineto{\pgfqpoint{3.351275in}{0.983593in}}%
\pgfusepath{stroke}%
\end{pgfscope}%
\begin{pgfscope}%
\pgfpathrectangle{\pgfqpoint{0.552773in}{0.431673in}}{\pgfqpoint{3.738807in}{1.765244in}}%
\pgfusepath{clip}%
\pgfsetbuttcap%
\pgfsetroundjoin%
\pgfsetlinewidth{1.003750pt}%
\definecolor{currentstroke}{rgb}{0.667253,0.779176,0.992959}%
\pgfsetstrokecolor{currentstroke}%
\pgfsetdash{}{0pt}%
\pgfpathmoveto{\pgfqpoint{3.402422in}{1.016823in}}%
\pgfpathlineto{\pgfqpoint{3.402422in}{1.016823in}}%
\pgfusepath{stroke}%
\end{pgfscope}%
\begin{pgfscope}%
\pgfpathrectangle{\pgfqpoint{0.552773in}{0.431673in}}{\pgfqpoint{3.738807in}{1.765244in}}%
\pgfusepath{clip}%
\pgfsetbuttcap%
\pgfsetroundjoin%
\pgfsetlinewidth{1.003750pt}%
\definecolor{currentstroke}{rgb}{0.667253,0.779176,0.992959}%
\pgfsetstrokecolor{currentstroke}%
\pgfsetdash{}{0pt}%
\pgfpathmoveto{\pgfqpoint{3.426281in}{1.000369in}}%
\pgfpathlineto{\pgfqpoint{3.426281in}{1.000369in}}%
\pgfusepath{stroke}%
\end{pgfscope}%
\begin{pgfscope}%
\pgfpathrectangle{\pgfqpoint{0.552773in}{0.431673in}}{\pgfqpoint{3.738807in}{1.765244in}}%
\pgfusepath{clip}%
\pgfsetbuttcap%
\pgfsetroundjoin%
\pgfsetlinewidth{1.003750pt}%
\definecolor{currentstroke}{rgb}{0.667253,0.779176,0.992959}%
\pgfsetstrokecolor{currentstroke}%
\pgfsetdash{}{0pt}%
\pgfpathmoveto{\pgfqpoint{3.431078in}{0.949209in}}%
\pgfpathlineto{\pgfqpoint{3.431078in}{0.949209in}}%
\pgfusepath{stroke}%
\end{pgfscope}%
\begin{pgfscope}%
\pgfpathrectangle{\pgfqpoint{0.552773in}{0.431673in}}{\pgfqpoint{3.738807in}{1.765244in}}%
\pgfusepath{clip}%
\pgfsetbuttcap%
\pgfsetroundjoin%
\pgfsetlinewidth{1.003750pt}%
\definecolor{currentstroke}{rgb}{0.667253,0.779176,0.992959}%
\pgfsetstrokecolor{currentstroke}%
\pgfsetdash{}{0pt}%
\pgfpathmoveto{\pgfqpoint{3.439962in}{0.905491in}}%
\pgfpathlineto{\pgfqpoint{3.439962in}{0.905491in}}%
\pgfusepath{stroke}%
\end{pgfscope}%
\begin{pgfscope}%
\pgfpathrectangle{\pgfqpoint{0.552773in}{0.431673in}}{\pgfqpoint{3.738807in}{1.765244in}}%
\pgfusepath{clip}%
\pgfsetbuttcap%
\pgfsetroundjoin%
\pgfsetlinewidth{1.003750pt}%
\definecolor{currentstroke}{rgb}{0.667253,0.779176,0.992959}%
\pgfsetstrokecolor{currentstroke}%
\pgfsetdash{}{0pt}%
\pgfpathmoveto{\pgfqpoint{3.446892in}{0.858216in}}%
\pgfpathlineto{\pgfqpoint{3.446892in}{0.858216in}}%
\pgfusepath{stroke}%
\end{pgfscope}%
\begin{pgfscope}%
\pgfpathrectangle{\pgfqpoint{0.552773in}{0.431673in}}{\pgfqpoint{3.738807in}{1.765244in}}%
\pgfusepath{clip}%
\pgfsetbuttcap%
\pgfsetroundjoin%
\pgfsetlinewidth{1.003750pt}%
\definecolor{currentstroke}{rgb}{0.667253,0.779176,0.992959}%
\pgfsetstrokecolor{currentstroke}%
\pgfsetdash{}{0pt}%
\pgfpathmoveto{\pgfqpoint{3.456265in}{0.815386in}}%
\pgfpathlineto{\pgfqpoint{3.456265in}{0.815386in}}%
\pgfusepath{stroke}%
\end{pgfscope}%
\begin{pgfscope}%
\pgfpathrectangle{\pgfqpoint{0.552773in}{0.431673in}}{\pgfqpoint{3.738807in}{1.765244in}}%
\pgfusepath{clip}%
\pgfsetbuttcap%
\pgfsetroundjoin%
\pgfsetlinewidth{1.003750pt}%
\definecolor{currentstroke}{rgb}{0.667253,0.779176,0.992959}%
\pgfsetstrokecolor{currentstroke}%
\pgfsetdash{}{0pt}%
\pgfpathmoveto{\pgfqpoint{3.455917in}{0.754860in}}%
\pgfpathlineto{\pgfqpoint{3.455917in}{0.754860in}}%
\pgfusepath{stroke}%
\end{pgfscope}%
\begin{pgfscope}%
\pgfpathrectangle{\pgfqpoint{0.552773in}{0.431673in}}{\pgfqpoint{3.738807in}{1.765244in}}%
\pgfusepath{clip}%
\pgfsetbuttcap%
\pgfsetroundjoin%
\pgfsetlinewidth{1.003750pt}%
\definecolor{currentstroke}{rgb}{0.667253,0.779176,0.992959}%
\pgfsetstrokecolor{currentstroke}%
\pgfsetdash{}{0pt}%
\pgfpathmoveto{\pgfqpoint{3.471332in}{0.723033in}}%
\pgfpathlineto{\pgfqpoint{3.471332in}{0.723033in}}%
\pgfusepath{stroke}%
\end{pgfscope}%
\begin{pgfscope}%
\pgfpathrectangle{\pgfqpoint{0.552773in}{0.431673in}}{\pgfqpoint{3.738807in}{1.765244in}}%
\pgfusepath{clip}%
\pgfsetbuttcap%
\pgfsetroundjoin%
\pgfsetlinewidth{1.003750pt}%
\definecolor{currentstroke}{rgb}{0.667253,0.779176,0.992959}%
\pgfsetstrokecolor{currentstroke}%
\pgfsetdash{}{0pt}%
\pgfpathmoveto{\pgfqpoint{3.492791in}{0.702209in}}%
\pgfpathlineto{\pgfqpoint{3.492791in}{0.702209in}}%
\pgfusepath{stroke}%
\end{pgfscope}%
\begin{pgfscope}%
\pgfpathrectangle{\pgfqpoint{0.552773in}{0.431673in}}{\pgfqpoint{3.738807in}{1.765244in}}%
\pgfusepath{clip}%
\pgfsetbuttcap%
\pgfsetroundjoin%
\pgfsetlinewidth{1.003750pt}%
\definecolor{currentstroke}{rgb}{0.667253,0.779176,0.992959}%
\pgfsetstrokecolor{currentstroke}%
\pgfsetdash{}{0pt}%
\pgfpathmoveto{\pgfqpoint{3.581754in}{0.684504in}}%
\pgfpathlineto{\pgfqpoint{3.581754in}{0.684504in}}%
\pgfusepath{stroke}%
\end{pgfscope}%
\begin{pgfscope}%
\pgfpathrectangle{\pgfqpoint{0.552773in}{0.431673in}}{\pgfqpoint{3.738807in}{1.765244in}}%
\pgfusepath{clip}%
\pgfsetbuttcap%
\pgfsetroundjoin%
\pgfsetlinewidth{1.003750pt}%
\definecolor{currentstroke}{rgb}{0.667253,0.779176,0.992959}%
\pgfsetstrokecolor{currentstroke}%
\pgfsetdash{}{0pt}%
\pgfpathmoveto{\pgfqpoint{3.619118in}{0.692639in}}%
\pgfpathlineto{\pgfqpoint{3.619118in}{0.692639in}}%
\pgfusepath{stroke}%
\end{pgfscope}%
\begin{pgfscope}%
\pgfpathrectangle{\pgfqpoint{0.552773in}{0.431673in}}{\pgfqpoint{3.738807in}{1.765244in}}%
\pgfusepath{clip}%
\pgfsetbuttcap%
\pgfsetroundjoin%
\pgfsetlinewidth{1.003750pt}%
\definecolor{currentstroke}{rgb}{0.667253,0.779176,0.992959}%
\pgfsetstrokecolor{currentstroke}%
\pgfsetdash{}{0pt}%
\pgfpathmoveto{\pgfqpoint{3.681794in}{0.746858in}}%
\pgfpathlineto{\pgfqpoint{3.681794in}{0.746858in}}%
\pgfusepath{stroke}%
\end{pgfscope}%
\begin{pgfscope}%
\pgfpathrectangle{\pgfqpoint{0.552773in}{0.431673in}}{\pgfqpoint{3.738807in}{1.765244in}}%
\pgfusepath{clip}%
\pgfsetbuttcap%
\pgfsetroundjoin%
\pgfsetlinewidth{1.003750pt}%
\definecolor{currentstroke}{rgb}{0.667253,0.779176,0.992959}%
\pgfsetstrokecolor{currentstroke}%
\pgfsetdash{}{0pt}%
\pgfpathmoveto{\pgfqpoint{0.946281in}{0.917169in}}%
\pgfpathlineto{\pgfqpoint{0.946281in}{0.917169in}}%
\pgfusepath{stroke}%
\end{pgfscope}%
\begin{pgfscope}%
\pgfpathrectangle{\pgfqpoint{0.552773in}{0.431673in}}{\pgfqpoint{3.738807in}{1.765244in}}%
\pgfusepath{clip}%
\pgfsetbuttcap%
\pgfsetroundjoin%
\pgfsetlinewidth{1.003750pt}%
\definecolor{currentstroke}{rgb}{0.667253,0.779176,0.992959}%
\pgfsetstrokecolor{currentstroke}%
\pgfsetdash{}{0pt}%
\pgfpathmoveto{\pgfqpoint{0.968004in}{0.896826in}}%
\pgfpathlineto{\pgfqpoint{0.968004in}{0.896826in}}%
\pgfusepath{stroke}%
\end{pgfscope}%
\begin{pgfscope}%
\pgfpathrectangle{\pgfqpoint{0.552773in}{0.431673in}}{\pgfqpoint{3.738807in}{1.765244in}}%
\pgfusepath{clip}%
\pgfsetbuttcap%
\pgfsetroundjoin%
\pgfsetlinewidth{1.003750pt}%
\definecolor{currentstroke}{rgb}{0.667253,0.779176,0.992959}%
\pgfsetstrokecolor{currentstroke}%
\pgfsetdash{}{0pt}%
\pgfpathmoveto{\pgfqpoint{0.983367in}{0.864904in}}%
\pgfpathlineto{\pgfqpoint{0.983367in}{0.864904in}}%
\pgfusepath{stroke}%
\end{pgfscope}%
\begin{pgfscope}%
\pgfpathrectangle{\pgfqpoint{0.552773in}{0.431673in}}{\pgfqpoint{3.738807in}{1.765244in}}%
\pgfusepath{clip}%
\pgfsetbuttcap%
\pgfsetroundjoin%
\pgfsetlinewidth{1.003750pt}%
\definecolor{currentstroke}{rgb}{0.667253,0.779176,0.992959}%
\pgfsetstrokecolor{currentstroke}%
\pgfsetdash{}{0pt}%
\pgfpathmoveto{\pgfqpoint{0.997288in}{0.830357in}}%
\pgfpathlineto{\pgfqpoint{0.997288in}{0.830357in}}%
\pgfusepath{stroke}%
\end{pgfscope}%
\begin{pgfscope}%
\pgfpathrectangle{\pgfqpoint{0.552773in}{0.431673in}}{\pgfqpoint{3.738807in}{1.765244in}}%
\pgfusepath{clip}%
\pgfsetbuttcap%
\pgfsetroundjoin%
\pgfsetlinewidth{1.003750pt}%
\definecolor{currentstroke}{rgb}{0.667253,0.779176,0.992959}%
\pgfsetstrokecolor{currentstroke}%
\pgfsetdash{}{0pt}%
\pgfpathmoveto{\pgfqpoint{1.003998in}{0.782680in}}%
\pgfpathlineto{\pgfqpoint{1.003998in}{0.782680in}}%
\pgfusepath{stroke}%
\end{pgfscope}%
\begin{pgfscope}%
\pgfpathrectangle{\pgfqpoint{0.552773in}{0.431673in}}{\pgfqpoint{3.738807in}{1.765244in}}%
\pgfusepath{clip}%
\pgfsetbuttcap%
\pgfsetroundjoin%
\pgfsetlinewidth{1.003750pt}%
\definecolor{currentstroke}{rgb}{0.667253,0.779176,0.992959}%
\pgfsetstrokecolor{currentstroke}%
\pgfsetdash{}{0pt}%
\pgfpathmoveto{\pgfqpoint{1.002620in}{0.720278in}}%
\pgfpathlineto{\pgfqpoint{1.002620in}{0.720278in}}%
\pgfusepath{stroke}%
\end{pgfscope}%
\begin{pgfscope}%
\pgfpathrectangle{\pgfqpoint{0.552773in}{0.431673in}}{\pgfqpoint{3.738807in}{1.765244in}}%
\pgfusepath{clip}%
\pgfsetbuttcap%
\pgfsetroundjoin%
\pgfsetlinewidth{1.003750pt}%
\definecolor{currentstroke}{rgb}{0.667253,0.779176,0.992959}%
\pgfsetstrokecolor{currentstroke}%
\pgfsetdash{}{0pt}%
\pgfpathmoveto{\pgfqpoint{1.016428in}{0.685524in}}%
\pgfpathlineto{\pgfqpoint{1.016428in}{0.685524in}}%
\pgfusepath{stroke}%
\end{pgfscope}%
\begin{pgfscope}%
\pgfpathrectangle{\pgfqpoint{0.552773in}{0.431673in}}{\pgfqpoint{3.738807in}{1.765244in}}%
\pgfusepath{clip}%
\pgfsetbuttcap%
\pgfsetroundjoin%
\pgfsetlinewidth{1.003750pt}%
\definecolor{currentstroke}{rgb}{0.667253,0.779176,0.992959}%
\pgfsetstrokecolor{currentstroke}%
\pgfsetdash{}{0pt}%
\pgfpathmoveto{\pgfqpoint{1.016366in}{0.625519in}}%
\pgfpathlineto{\pgfqpoint{1.016366in}{0.625519in}}%
\pgfusepath{stroke}%
\end{pgfscope}%
\begin{pgfscope}%
\pgfpathrectangle{\pgfqpoint{0.552773in}{0.431673in}}{\pgfqpoint{3.738807in}{1.765244in}}%
\pgfusepath{clip}%
\pgfsetbuttcap%
\pgfsetroundjoin%
\pgfsetlinewidth{1.003750pt}%
\definecolor{currentstroke}{rgb}{0.667253,0.779176,0.992959}%
\pgfsetstrokecolor{currentstroke}%
\pgfsetdash{}{0pt}%
\pgfpathmoveto{\pgfqpoint{1.044197in}{0.616297in}}%
\pgfpathlineto{\pgfqpoint{1.044197in}{0.616297in}}%
\pgfusepath{stroke}%
\end{pgfscope}%
\begin{pgfscope}%
\pgfpathrectangle{\pgfqpoint{0.552773in}{0.431673in}}{\pgfqpoint{3.738807in}{1.765244in}}%
\pgfusepath{clip}%
\pgfsetbuttcap%
\pgfsetroundjoin%
\pgfsetlinewidth{1.003750pt}%
\definecolor{currentstroke}{rgb}{0.667253,0.779176,0.992959}%
\pgfsetstrokecolor{currentstroke}%
\pgfsetdash{}{0pt}%
\pgfpathmoveto{\pgfqpoint{1.061881in}{0.588600in}}%
\pgfpathlineto{\pgfqpoint{1.061881in}{0.588600in}}%
\pgfusepath{stroke}%
\end{pgfscope}%
\begin{pgfscope}%
\pgfpathrectangle{\pgfqpoint{0.552773in}{0.431673in}}{\pgfqpoint{3.738807in}{1.765244in}}%
\pgfusepath{clip}%
\pgfsetbuttcap%
\pgfsetroundjoin%
\pgfsetlinewidth{1.003750pt}%
\definecolor{currentstroke}{rgb}{0.667253,0.779176,0.992959}%
\pgfsetstrokecolor{currentstroke}%
\pgfsetdash{}{0pt}%
\pgfpathmoveto{\pgfqpoint{1.100011in}{0.598129in}}%
\pgfpathlineto{\pgfqpoint{1.100011in}{0.598129in}}%
\pgfusepath{stroke}%
\end{pgfscope}%
\begin{pgfscope}%
\pgfpathrectangle{\pgfqpoint{0.552773in}{0.431673in}}{\pgfqpoint{3.738807in}{1.765244in}}%
\pgfusepath{clip}%
\pgfsetbuttcap%
\pgfsetroundjoin%
\pgfsetlinewidth{1.003750pt}%
\definecolor{currentstroke}{rgb}{0.667253,0.779176,0.992959}%
\pgfsetstrokecolor{currentstroke}%
\pgfsetdash{}{0pt}%
\pgfpathmoveto{\pgfqpoint{1.133830in}{0.599811in}}%
\pgfpathlineto{\pgfqpoint{1.133830in}{0.599811in}}%
\pgfusepath{stroke}%
\end{pgfscope}%
\begin{pgfscope}%
\pgfpathrectangle{\pgfqpoint{0.552773in}{0.431673in}}{\pgfqpoint{3.738807in}{1.765244in}}%
\pgfusepath{clip}%
\pgfsetbuttcap%
\pgfsetroundjoin%
\pgfsetlinewidth{1.003750pt}%
\definecolor{currentstroke}{rgb}{0.667253,0.779176,0.992959}%
\pgfsetstrokecolor{currentstroke}%
\pgfsetdash{}{0pt}%
\pgfpathmoveto{\pgfqpoint{1.178595in}{0.621420in}}%
\pgfpathlineto{\pgfqpoint{1.178595in}{0.621420in}}%
\pgfusepath{stroke}%
\end{pgfscope}%
\begin{pgfscope}%
\pgfpathrectangle{\pgfqpoint{0.552773in}{0.431673in}}{\pgfqpoint{3.738807in}{1.765244in}}%
\pgfusepath{clip}%
\pgfsetbuttcap%
\pgfsetroundjoin%
\pgfsetlinewidth{1.003750pt}%
\definecolor{currentstroke}{rgb}{0.667253,0.779176,0.992959}%
\pgfsetstrokecolor{currentstroke}%
\pgfsetdash{}{0pt}%
\pgfpathmoveto{\pgfqpoint{1.241991in}{0.676951in}}%
\pgfpathlineto{\pgfqpoint{1.241991in}{0.676951in}}%
\pgfusepath{stroke}%
\end{pgfscope}%
\begin{pgfscope}%
\pgfpathrectangle{\pgfqpoint{0.552773in}{0.431673in}}{\pgfqpoint{3.738807in}{1.765244in}}%
\pgfusepath{clip}%
\pgfsetbuttcap%
\pgfsetroundjoin%
\pgfsetlinewidth{1.003750pt}%
\definecolor{currentstroke}{rgb}{0.667253,0.779176,0.992959}%
\pgfsetstrokecolor{currentstroke}%
\pgfsetdash{}{0pt}%
\pgfpathmoveto{\pgfqpoint{1.316482in}{0.752683in}}%
\pgfpathlineto{\pgfqpoint{1.316482in}{0.752683in}}%
\pgfusepath{stroke}%
\end{pgfscope}%
\begin{pgfscope}%
\pgfpathrectangle{\pgfqpoint{0.552773in}{0.431673in}}{\pgfqpoint{3.738807in}{1.765244in}}%
\pgfusepath{clip}%
\pgfsetbuttcap%
\pgfsetroundjoin%
\pgfsetlinewidth{1.003750pt}%
\definecolor{currentstroke}{rgb}{0.667253,0.779176,0.992959}%
\pgfsetstrokecolor{currentstroke}%
\pgfsetdash{}{0pt}%
\pgfpathmoveto{\pgfqpoint{1.386975in}{0.821134in}}%
\pgfpathlineto{\pgfqpoint{1.386975in}{0.821134in}}%
\pgfusepath{stroke}%
\end{pgfscope}%
\begin{pgfscope}%
\pgfpathrectangle{\pgfqpoint{0.552773in}{0.431673in}}{\pgfqpoint{3.738807in}{1.765244in}}%
\pgfusepath{clip}%
\pgfsetbuttcap%
\pgfsetroundjoin%
\pgfsetlinewidth{1.003750pt}%
\definecolor{currentstroke}{rgb}{0.667253,0.779176,0.992959}%
\pgfsetstrokecolor{currentstroke}%
\pgfsetdash{}{0pt}%
\pgfpathmoveto{\pgfqpoint{1.458963in}{0.892310in}}%
\pgfpathlineto{\pgfqpoint{1.458963in}{0.892310in}}%
\pgfusepath{stroke}%
\end{pgfscope}%
\begin{pgfscope}%
\pgfpathrectangle{\pgfqpoint{0.552773in}{0.431673in}}{\pgfqpoint{3.738807in}{1.765244in}}%
\pgfusepath{clip}%
\pgfsetbuttcap%
\pgfsetroundjoin%
\pgfsetlinewidth{1.003750pt}%
\definecolor{currentstroke}{rgb}{0.667253,0.779176,0.992959}%
\pgfsetstrokecolor{currentstroke}%
\pgfsetdash{}{0pt}%
\pgfpathmoveto{\pgfqpoint{1.515751in}{0.935810in}}%
\pgfpathlineto{\pgfqpoint{1.515751in}{0.935810in}}%
\pgfusepath{stroke}%
\end{pgfscope}%
\begin{pgfscope}%
\pgfpathrectangle{\pgfqpoint{0.552773in}{0.431673in}}{\pgfqpoint{3.738807in}{1.765244in}}%
\pgfusepath{clip}%
\pgfsetbuttcap%
\pgfsetroundjoin%
\pgfsetlinewidth{1.003750pt}%
\definecolor{currentstroke}{rgb}{0.667253,0.779176,0.992959}%
\pgfsetstrokecolor{currentstroke}%
\pgfsetdash{}{0pt}%
\pgfpathmoveto{\pgfqpoint{1.552230in}{0.942332in}}%
\pgfpathlineto{\pgfqpoint{1.552230in}{0.942332in}}%
\pgfusepath{stroke}%
\end{pgfscope}%
\begin{pgfscope}%
\pgfpathrectangle{\pgfqpoint{0.552773in}{0.431673in}}{\pgfqpoint{3.738807in}{1.765244in}}%
\pgfusepath{clip}%
\pgfsetbuttcap%
\pgfsetroundjoin%
\pgfsetlinewidth{1.003750pt}%
\definecolor{currentstroke}{rgb}{0.667253,0.779176,0.992959}%
\pgfsetstrokecolor{currentstroke}%
\pgfsetdash{}{0pt}%
\pgfpathmoveto{\pgfqpoint{1.571226in}{0.917025in}}%
\pgfpathlineto{\pgfqpoint{1.571226in}{0.917025in}}%
\pgfusepath{stroke}%
\end{pgfscope}%
\begin{pgfscope}%
\pgfpathrectangle{\pgfqpoint{0.552773in}{0.431673in}}{\pgfqpoint{3.738807in}{1.765244in}}%
\pgfusepath{clip}%
\pgfsetbuttcap%
\pgfsetroundjoin%
\pgfsetlinewidth{1.003750pt}%
\definecolor{currentstroke}{rgb}{0.667253,0.779176,0.992959}%
\pgfsetstrokecolor{currentstroke}%
\pgfsetdash{}{0pt}%
\pgfpathmoveto{\pgfqpoint{1.578686in}{0.870715in}}%
\pgfpathlineto{\pgfqpoint{1.578686in}{0.870715in}}%
\pgfusepath{stroke}%
\end{pgfscope}%
\begin{pgfscope}%
\pgfpathrectangle{\pgfqpoint{0.552773in}{0.431673in}}{\pgfqpoint{3.738807in}{1.765244in}}%
\pgfusepath{clip}%
\pgfsetbuttcap%
\pgfsetroundjoin%
\pgfsetlinewidth{1.003750pt}%
\definecolor{currentstroke}{rgb}{0.667253,0.779176,0.992959}%
\pgfsetstrokecolor{currentstroke}%
\pgfsetdash{}{0pt}%
\pgfpathmoveto{\pgfqpoint{1.598839in}{0.847513in}}%
\pgfpathlineto{\pgfqpoint{1.598839in}{0.847513in}}%
\pgfusepath{stroke}%
\end{pgfscope}%
\begin{pgfscope}%
\pgfpathrectangle{\pgfqpoint{0.552773in}{0.431673in}}{\pgfqpoint{3.738807in}{1.765244in}}%
\pgfusepath{clip}%
\pgfsetbuttcap%
\pgfsetroundjoin%
\pgfsetlinewidth{1.003750pt}%
\definecolor{currentstroke}{rgb}{0.667253,0.779176,0.992959}%
\pgfsetstrokecolor{currentstroke}%
\pgfsetdash{}{0pt}%
\pgfpathmoveto{\pgfqpoint{1.615264in}{0.817524in}}%
\pgfpathlineto{\pgfqpoint{1.615264in}{0.817524in}}%
\pgfusepath{stroke}%
\end{pgfscope}%
\begin{pgfscope}%
\pgfpathrectangle{\pgfqpoint{0.552773in}{0.431673in}}{\pgfqpoint{3.738807in}{1.765244in}}%
\pgfusepath{clip}%
\pgfsetbuttcap%
\pgfsetroundjoin%
\pgfsetlinewidth{1.003750pt}%
\definecolor{currentstroke}{rgb}{0.667253,0.779176,0.992959}%
\pgfsetstrokecolor{currentstroke}%
\pgfsetdash{}{0pt}%
\pgfpathmoveto{\pgfqpoint{1.629986in}{0.784436in}}%
\pgfpathlineto{\pgfqpoint{1.629986in}{0.784436in}}%
\pgfusepath{stroke}%
\end{pgfscope}%
\begin{pgfscope}%
\pgfpathrectangle{\pgfqpoint{0.552773in}{0.431673in}}{\pgfqpoint{3.738807in}{1.765244in}}%
\pgfusepath{clip}%
\pgfsetbuttcap%
\pgfsetroundjoin%
\pgfsetlinewidth{1.003750pt}%
\definecolor{currentstroke}{rgb}{0.667253,0.779176,0.992959}%
\pgfsetstrokecolor{currentstroke}%
\pgfsetdash{}{0pt}%
\pgfpathmoveto{\pgfqpoint{1.633661in}{0.731234in}}%
\pgfpathlineto{\pgfqpoint{1.633661in}{0.731234in}}%
\pgfusepath{stroke}%
\end{pgfscope}%
\begin{pgfscope}%
\pgfpathrectangle{\pgfqpoint{0.552773in}{0.431673in}}{\pgfqpoint{3.738807in}{1.765244in}}%
\pgfusepath{clip}%
\pgfsetbuttcap%
\pgfsetroundjoin%
\pgfsetlinewidth{1.003750pt}%
\definecolor{currentstroke}{rgb}{0.667253,0.779176,0.992959}%
\pgfsetstrokecolor{currentstroke}%
\pgfsetdash{}{0pt}%
\pgfpathmoveto{\pgfqpoint{1.644849in}{0.691711in}}%
\pgfpathlineto{\pgfqpoint{1.644849in}{0.691711in}}%
\pgfusepath{stroke}%
\end{pgfscope}%
\begin{pgfscope}%
\pgfpathrectangle{\pgfqpoint{0.552773in}{0.431673in}}{\pgfqpoint{3.738807in}{1.765244in}}%
\pgfusepath{clip}%
\pgfsetbuttcap%
\pgfsetroundjoin%
\pgfsetlinewidth{1.003750pt}%
\definecolor{currentstroke}{rgb}{0.667253,0.779176,0.992959}%
\pgfsetstrokecolor{currentstroke}%
\pgfsetdash{}{0pt}%
\pgfpathmoveto{\pgfqpoint{1.654597in}{0.649565in}}%
\pgfpathlineto{\pgfqpoint{1.654597in}{0.649565in}}%
\pgfusepath{stroke}%
\end{pgfscope}%
\begin{pgfscope}%
\pgfpathrectangle{\pgfqpoint{0.552773in}{0.431673in}}{\pgfqpoint{3.738807in}{1.765244in}}%
\pgfusepath{clip}%
\pgfsetbuttcap%
\pgfsetroundjoin%
\pgfsetlinewidth{1.003750pt}%
\definecolor{currentstroke}{rgb}{0.667253,0.779176,0.992959}%
\pgfsetstrokecolor{currentstroke}%
\pgfsetdash{}{0pt}%
\pgfpathmoveto{\pgfqpoint{1.665591in}{0.609687in}}%
\pgfpathlineto{\pgfqpoint{1.665591in}{0.609687in}}%
\pgfusepath{stroke}%
\end{pgfscope}%
\begin{pgfscope}%
\pgfpathrectangle{\pgfqpoint{0.552773in}{0.431673in}}{\pgfqpoint{3.738807in}{1.765244in}}%
\pgfusepath{clip}%
\pgfsetbuttcap%
\pgfsetroundjoin%
\pgfsetlinewidth{1.003750pt}%
\definecolor{currentstroke}{rgb}{0.667253,0.779176,0.992959}%
\pgfsetstrokecolor{currentstroke}%
\pgfsetdash{}{0pt}%
\pgfpathmoveto{\pgfqpoint{1.686427in}{0.587731in}}%
\pgfpathlineto{\pgfqpoint{1.686427in}{0.587731in}}%
\pgfusepath{stroke}%
\end{pgfscope}%
\begin{pgfscope}%
\pgfpathrectangle{\pgfqpoint{0.552773in}{0.431673in}}{\pgfqpoint{3.738807in}{1.765244in}}%
\pgfusepath{clip}%
\pgfsetbuttcap%
\pgfsetroundjoin%
\pgfsetlinewidth{1.003750pt}%
\definecolor{currentstroke}{rgb}{0.667253,0.779176,0.992959}%
\pgfsetstrokecolor{currentstroke}%
\pgfsetdash{}{0pt}%
\pgfpathmoveto{\pgfqpoint{1.711171in}{0.572888in}}%
\pgfpathlineto{\pgfqpoint{1.711171in}{0.572888in}}%
\pgfusepath{stroke}%
\end{pgfscope}%
\begin{pgfscope}%
\pgfpathrectangle{\pgfqpoint{0.552773in}{0.431673in}}{\pgfqpoint{3.738807in}{1.765244in}}%
\pgfusepath{clip}%
\pgfsetbuttcap%
\pgfsetroundjoin%
\pgfsetlinewidth{1.003750pt}%
\definecolor{currentstroke}{rgb}{0.667253,0.779176,0.992959}%
\pgfsetstrokecolor{currentstroke}%
\pgfsetdash{}{0pt}%
\pgfpathmoveto{\pgfqpoint{1.758001in}{0.598259in}}%
\pgfpathlineto{\pgfqpoint{1.758001in}{0.598259in}}%
\pgfusepath{stroke}%
\end{pgfscope}%
\begin{pgfscope}%
\pgfpathrectangle{\pgfqpoint{0.552773in}{0.431673in}}{\pgfqpoint{3.738807in}{1.765244in}}%
\pgfusepath{clip}%
\pgfsetbuttcap%
\pgfsetroundjoin%
\pgfsetlinewidth{1.003750pt}%
\definecolor{currentstroke}{rgb}{0.667253,0.779176,0.992959}%
\pgfsetstrokecolor{currentstroke}%
\pgfsetdash{}{0pt}%
\pgfpathmoveto{\pgfqpoint{1.816028in}{0.644014in}}%
\pgfpathlineto{\pgfqpoint{1.816028in}{0.644014in}}%
\pgfusepath{stroke}%
\end{pgfscope}%
\begin{pgfscope}%
\pgfpathrectangle{\pgfqpoint{0.552773in}{0.431673in}}{\pgfqpoint{3.738807in}{1.765244in}}%
\pgfusepath{clip}%
\pgfsetbuttcap%
\pgfsetroundjoin%
\pgfsetlinewidth{1.003750pt}%
\definecolor{currentstroke}{rgb}{0.667253,0.779176,0.992959}%
\pgfsetstrokecolor{currentstroke}%
\pgfsetdash{}{0pt}%
\pgfpathmoveto{\pgfqpoint{1.879012in}{0.698794in}}%
\pgfpathlineto{\pgfqpoint{1.879012in}{0.698794in}}%
\pgfusepath{stroke}%
\end{pgfscope}%
\begin{pgfscope}%
\pgfpathrectangle{\pgfqpoint{0.552773in}{0.431673in}}{\pgfqpoint{3.738807in}{1.765244in}}%
\pgfusepath{clip}%
\pgfsetbuttcap%
\pgfsetroundjoin%
\pgfsetlinewidth{1.003750pt}%
\definecolor{currentstroke}{rgb}{0.667253,0.779176,0.992959}%
\pgfsetstrokecolor{currentstroke}%
\pgfsetdash{}{0pt}%
\pgfpathmoveto{\pgfqpoint{1.944926in}{0.758909in}}%
\pgfpathlineto{\pgfqpoint{1.944926in}{0.758909in}}%
\pgfusepath{stroke}%
\end{pgfscope}%
\begin{pgfscope}%
\pgfpathrectangle{\pgfqpoint{0.552773in}{0.431673in}}{\pgfqpoint{3.738807in}{1.765244in}}%
\pgfusepath{clip}%
\pgfsetbuttcap%
\pgfsetroundjoin%
\pgfsetlinewidth{1.003750pt}%
\definecolor{currentstroke}{rgb}{0.667253,0.779176,0.992959}%
\pgfsetstrokecolor{currentstroke}%
\pgfsetdash{}{0pt}%
\pgfpathmoveto{\pgfqpoint{2.017564in}{0.831267in}}%
\pgfpathlineto{\pgfqpoint{2.017564in}{0.831267in}}%
\pgfusepath{stroke}%
\end{pgfscope}%
\begin{pgfscope}%
\pgfpathrectangle{\pgfqpoint{0.552773in}{0.431673in}}{\pgfqpoint{3.738807in}{1.765244in}}%
\pgfusepath{clip}%
\pgfsetbuttcap%
\pgfsetroundjoin%
\pgfsetlinewidth{1.003750pt}%
\definecolor{currentstroke}{rgb}{0.667253,0.779176,0.992959}%
\pgfsetstrokecolor{currentstroke}%
\pgfsetdash{}{0pt}%
\pgfpathmoveto{\pgfqpoint{2.077394in}{0.880307in}}%
\pgfpathlineto{\pgfqpoint{2.077394in}{0.880307in}}%
\pgfusepath{stroke}%
\end{pgfscope}%
\begin{pgfscope}%
\pgfpathrectangle{\pgfqpoint{0.552773in}{0.431673in}}{\pgfqpoint{3.738807in}{1.765244in}}%
\pgfusepath{clip}%
\pgfsetbuttcap%
\pgfsetroundjoin%
\pgfsetlinewidth{1.003750pt}%
\definecolor{currentstroke}{rgb}{0.667253,0.779176,0.992959}%
\pgfsetstrokecolor{currentstroke}%
\pgfsetdash{}{0pt}%
\pgfpathmoveto{\pgfqpoint{2.120452in}{0.898808in}}%
\pgfpathlineto{\pgfqpoint{2.120452in}{0.898808in}}%
\pgfusepath{stroke}%
\end{pgfscope}%
\begin{pgfscope}%
\pgfpathrectangle{\pgfqpoint{0.552773in}{0.431673in}}{\pgfqpoint{3.738807in}{1.765244in}}%
\pgfusepath{clip}%
\pgfsetbuttcap%
\pgfsetroundjoin%
\pgfsetlinewidth{1.003750pt}%
\definecolor{currentstroke}{rgb}{0.667253,0.779176,0.992959}%
\pgfsetstrokecolor{currentstroke}%
\pgfsetdash{}{0pt}%
\pgfpathmoveto{\pgfqpoint{2.148843in}{0.890606in}}%
\pgfpathlineto{\pgfqpoint{2.148843in}{0.890606in}}%
\pgfusepath{stroke}%
\end{pgfscope}%
\begin{pgfscope}%
\pgfpathrectangle{\pgfqpoint{0.552773in}{0.431673in}}{\pgfqpoint{3.738807in}{1.765244in}}%
\pgfusepath{clip}%
\pgfsetbuttcap%
\pgfsetroundjoin%
\pgfsetlinewidth{1.003750pt}%
\definecolor{currentstroke}{rgb}{0.667253,0.779176,0.992959}%
\pgfsetstrokecolor{currentstroke}%
\pgfsetdash{}{0pt}%
\pgfpathmoveto{\pgfqpoint{2.169787in}{0.868845in}}%
\pgfpathlineto{\pgfqpoint{2.169787in}{0.868845in}}%
\pgfusepath{stroke}%
\end{pgfscope}%
\begin{pgfscope}%
\pgfpathrectangle{\pgfqpoint{0.552773in}{0.431673in}}{\pgfqpoint{3.738807in}{1.765244in}}%
\pgfusepath{clip}%
\pgfsetbuttcap%
\pgfsetroundjoin%
\pgfsetlinewidth{1.003750pt}%
\definecolor{currentstroke}{rgb}{0.667253,0.779176,0.992959}%
\pgfsetstrokecolor{currentstroke}%
\pgfsetdash{}{0pt}%
\pgfpathmoveto{\pgfqpoint{2.191999in}{0.849394in}}%
\pgfpathlineto{\pgfqpoint{2.191999in}{0.849394in}}%
\pgfusepath{stroke}%
\end{pgfscope}%
\begin{pgfscope}%
\pgfpathrectangle{\pgfqpoint{0.552773in}{0.431673in}}{\pgfqpoint{3.738807in}{1.765244in}}%
\pgfusepath{clip}%
\pgfsetbuttcap%
\pgfsetroundjoin%
\pgfsetlinewidth{1.003750pt}%
\definecolor{currentstroke}{rgb}{0.667253,0.779176,0.992959}%
\pgfsetstrokecolor{currentstroke}%
\pgfsetdash{}{0pt}%
\pgfpathmoveto{\pgfqpoint{2.201330in}{0.806489in}}%
\pgfpathlineto{\pgfqpoint{2.201330in}{0.806489in}}%
\pgfusepath{stroke}%
\end{pgfscope}%
\begin{pgfscope}%
\pgfpathrectangle{\pgfqpoint{0.552773in}{0.431673in}}{\pgfqpoint{3.738807in}{1.765244in}}%
\pgfusepath{clip}%
\pgfsetbuttcap%
\pgfsetroundjoin%
\pgfsetlinewidth{1.003750pt}%
\definecolor{currentstroke}{rgb}{0.667253,0.779176,0.992959}%
\pgfsetstrokecolor{currentstroke}%
\pgfsetdash{}{0pt}%
\pgfpathmoveto{\pgfqpoint{2.202600in}{0.748907in}}%
\pgfpathlineto{\pgfqpoint{2.202600in}{0.748907in}}%
\pgfusepath{stroke}%
\end{pgfscope}%
\begin{pgfscope}%
\pgfpathrectangle{\pgfqpoint{0.552773in}{0.431673in}}{\pgfqpoint{3.738807in}{1.765244in}}%
\pgfusepath{clip}%
\pgfsetbuttcap%
\pgfsetroundjoin%
\pgfsetlinewidth{1.003750pt}%
\definecolor{currentstroke}{rgb}{0.667253,0.779176,0.992959}%
\pgfsetstrokecolor{currentstroke}%
\pgfsetdash{}{0pt}%
\pgfpathmoveto{\pgfqpoint{2.202588in}{0.688992in}}%
\pgfpathlineto{\pgfqpoint{2.202588in}{0.688992in}}%
\pgfusepath{stroke}%
\end{pgfscope}%
\begin{pgfscope}%
\pgfpathrectangle{\pgfqpoint{0.552773in}{0.431673in}}{\pgfqpoint{3.738807in}{1.765244in}}%
\pgfusepath{clip}%
\pgfsetbuttcap%
\pgfsetroundjoin%
\pgfsetlinewidth{1.003750pt}%
\definecolor{currentstroke}{rgb}{0.667253,0.779176,0.992959}%
\pgfsetstrokecolor{currentstroke}%
\pgfsetdash{}{0pt}%
\pgfpathmoveto{\pgfqpoint{2.206526in}{0.636268in}}%
\pgfpathlineto{\pgfqpoint{2.206526in}{0.636268in}}%
\pgfusepath{stroke}%
\end{pgfscope}%
\begin{pgfscope}%
\pgfpathrectangle{\pgfqpoint{0.552773in}{0.431673in}}{\pgfqpoint{3.738807in}{1.765244in}}%
\pgfusepath{clip}%
\pgfsetbuttcap%
\pgfsetroundjoin%
\pgfsetlinewidth{1.003750pt}%
\definecolor{currentstroke}{rgb}{0.667253,0.779176,0.992959}%
\pgfsetstrokecolor{currentstroke}%
\pgfsetdash{}{0pt}%
\pgfpathmoveto{\pgfqpoint{2.205818in}{0.575086in}}%
\pgfpathlineto{\pgfqpoint{2.205818in}{0.575086in}}%
\pgfusepath{stroke}%
\end{pgfscope}%
\begin{pgfscope}%
\pgfpathrectangle{\pgfqpoint{0.552773in}{0.431673in}}{\pgfqpoint{3.738807in}{1.765244in}}%
\pgfusepath{clip}%
\pgfsetbuttcap%
\pgfsetroundjoin%
\pgfsetlinewidth{1.003750pt}%
\definecolor{currentstroke}{rgb}{0.667253,0.779176,0.992959}%
\pgfsetstrokecolor{currentstroke}%
\pgfsetdash{}{0pt}%
\pgfpathmoveto{\pgfqpoint{2.225473in}{0.550978in}}%
\pgfpathlineto{\pgfqpoint{2.225473in}{0.550978in}}%
\pgfusepath{stroke}%
\end{pgfscope}%
\begin{pgfscope}%
\pgfpathrectangle{\pgfqpoint{0.552773in}{0.431673in}}{\pgfqpoint{3.738807in}{1.765244in}}%
\pgfusepath{clip}%
\pgfsetbuttcap%
\pgfsetroundjoin%
\pgfsetlinewidth{1.003750pt}%
\definecolor{currentstroke}{rgb}{0.667253,0.779176,0.992959}%
\pgfsetstrokecolor{currentstroke}%
\pgfsetdash{}{0pt}%
\pgfpathmoveto{\pgfqpoint{2.247923in}{0.531960in}}%
\pgfpathlineto{\pgfqpoint{2.247923in}{0.531960in}}%
\pgfusepath{stroke}%
\end{pgfscope}%
\begin{pgfscope}%
\pgfpathrectangle{\pgfqpoint{0.552773in}{0.431673in}}{\pgfqpoint{3.738807in}{1.765244in}}%
\pgfusepath{clip}%
\pgfsetbuttcap%
\pgfsetroundjoin%
\pgfsetlinewidth{1.003750pt}%
\definecolor{currentstroke}{rgb}{0.667253,0.779176,0.992959}%
\pgfsetstrokecolor{currentstroke}%
\pgfsetdash{}{0pt}%
\pgfpathmoveto{\pgfqpoint{2.269807in}{0.511911in}}%
\pgfpathlineto{\pgfqpoint{2.269807in}{0.511911in}}%
\pgfusepath{stroke}%
\end{pgfscope}%
\begin{pgfscope}%
\pgfpathrectangle{\pgfqpoint{0.552773in}{0.431673in}}{\pgfqpoint{3.738807in}{1.765244in}}%
\pgfusepath{clip}%
\pgfsetbuttcap%
\pgfsetroundjoin%
\pgfsetlinewidth{1.003750pt}%
\definecolor{currentstroke}{rgb}{0.667253,0.779176,0.992959}%
\pgfsetstrokecolor{currentstroke}%
\pgfsetdash{}{0pt}%
\pgfpathmoveto{\pgfqpoint{2.310165in}{0.525497in}}%
\pgfpathlineto{\pgfqpoint{2.310165in}{0.525497in}}%
\pgfusepath{stroke}%
\end{pgfscope}%
\begin{pgfscope}%
\pgfpathrectangle{\pgfqpoint{0.552773in}{0.431673in}}{\pgfqpoint{3.738807in}{1.765244in}}%
\pgfusepath{clip}%
\pgfsetbuttcap%
\pgfsetroundjoin%
\pgfsetlinewidth{1.003750pt}%
\definecolor{currentstroke}{rgb}{0.667253,0.779176,0.992959}%
\pgfsetstrokecolor{currentstroke}%
\pgfsetdash{}{0pt}%
\pgfpathmoveto{\pgfqpoint{2.369172in}{0.573036in}}%
\pgfpathlineto{\pgfqpoint{2.369172in}{0.573036in}}%
\pgfusepath{stroke}%
\end{pgfscope}%
\begin{pgfscope}%
\pgfpathrectangle{\pgfqpoint{0.552773in}{0.431673in}}{\pgfqpoint{3.738807in}{1.765244in}}%
\pgfusepath{clip}%
\pgfsetbuttcap%
\pgfsetroundjoin%
\pgfsetlinewidth{1.003750pt}%
\definecolor{currentstroke}{rgb}{0.667253,0.779176,0.992959}%
\pgfsetstrokecolor{currentstroke}%
\pgfsetdash{}{0pt}%
\pgfpathmoveto{\pgfqpoint{2.433038in}{0.629423in}}%
\pgfpathlineto{\pgfqpoint{2.433038in}{0.629423in}}%
\pgfusepath{stroke}%
\end{pgfscope}%
\begin{pgfscope}%
\pgfpathrectangle{\pgfqpoint{0.552773in}{0.431673in}}{\pgfqpoint{3.738807in}{1.765244in}}%
\pgfusepath{clip}%
\pgfsetbuttcap%
\pgfsetroundjoin%
\pgfsetlinewidth{1.003750pt}%
\definecolor{currentstroke}{rgb}{0.667253,0.779176,0.992959}%
\pgfsetstrokecolor{currentstroke}%
\pgfsetdash{}{0pt}%
\pgfpathmoveto{\pgfqpoint{2.512551in}{0.714298in}}%
\pgfpathlineto{\pgfqpoint{2.512551in}{0.714298in}}%
\pgfusepath{stroke}%
\end{pgfscope}%
\begin{pgfscope}%
\pgfpathrectangle{\pgfqpoint{0.552773in}{0.431673in}}{\pgfqpoint{3.738807in}{1.765244in}}%
\pgfusepath{clip}%
\pgfsetbuttcap%
\pgfsetroundjoin%
\pgfsetlinewidth{1.003750pt}%
\definecolor{currentstroke}{rgb}{0.667253,0.779176,0.992959}%
\pgfsetstrokecolor{currentstroke}%
\pgfsetdash{}{0pt}%
\pgfpathmoveto{\pgfqpoint{2.584448in}{0.785307in}}%
\pgfpathlineto{\pgfqpoint{2.584448in}{0.785307in}}%
\pgfusepath{stroke}%
\end{pgfscope}%
\begin{pgfscope}%
\pgfpathrectangle{\pgfqpoint{0.552773in}{0.431673in}}{\pgfqpoint{3.738807in}{1.765244in}}%
\pgfusepath{clip}%
\pgfsetbuttcap%
\pgfsetroundjoin%
\pgfsetlinewidth{1.003750pt}%
\definecolor{currentstroke}{rgb}{0.667253,0.779176,0.992959}%
\pgfsetstrokecolor{currentstroke}%
\pgfsetdash{}{0pt}%
\pgfpathmoveto{\pgfqpoint{2.654396in}{0.852767in}}%
\pgfpathlineto{\pgfqpoint{2.654396in}{0.852767in}}%
\pgfusepath{stroke}%
\end{pgfscope}%
\begin{pgfscope}%
\pgfpathrectangle{\pgfqpoint{0.552773in}{0.431673in}}{\pgfqpoint{3.738807in}{1.765244in}}%
\pgfusepath{clip}%
\pgfsetbuttcap%
\pgfsetroundjoin%
\pgfsetlinewidth{1.003750pt}%
\definecolor{currentstroke}{rgb}{0.667253,0.779176,0.992959}%
\pgfsetstrokecolor{currentstroke}%
\pgfsetdash{}{0pt}%
\pgfpathmoveto{\pgfqpoint{2.709975in}{0.894066in}}%
\pgfpathlineto{\pgfqpoint{2.709975in}{0.894066in}}%
\pgfusepath{stroke}%
\end{pgfscope}%
\begin{pgfscope}%
\pgfpathrectangle{\pgfqpoint{0.552773in}{0.431673in}}{\pgfqpoint{3.738807in}{1.765244in}}%
\pgfusepath{clip}%
\pgfsetbuttcap%
\pgfsetroundjoin%
\pgfsetlinewidth{1.003750pt}%
\definecolor{currentstroke}{rgb}{0.667253,0.779176,0.992959}%
\pgfsetstrokecolor{currentstroke}%
\pgfsetdash{}{0pt}%
\pgfpathmoveto{\pgfqpoint{2.745485in}{0.898825in}}%
\pgfpathlineto{\pgfqpoint{2.745485in}{0.898825in}}%
\pgfusepath{stroke}%
\end{pgfscope}%
\begin{pgfscope}%
\pgfpathrectangle{\pgfqpoint{0.552773in}{0.431673in}}{\pgfqpoint{3.738807in}{1.765244in}}%
\pgfusepath{clip}%
\pgfsetbuttcap%
\pgfsetroundjoin%
\pgfsetlinewidth{1.003750pt}%
\definecolor{currentstroke}{rgb}{0.667253,0.779176,0.992959}%
\pgfsetstrokecolor{currentstroke}%
\pgfsetdash{}{0pt}%
\pgfpathmoveto{\pgfqpoint{2.765939in}{0.876173in}}%
\pgfpathlineto{\pgfqpoint{2.765939in}{0.876173in}}%
\pgfusepath{stroke}%
\end{pgfscope}%
\begin{pgfscope}%
\pgfpathrectangle{\pgfqpoint{0.552773in}{0.431673in}}{\pgfqpoint{3.738807in}{1.765244in}}%
\pgfusepath{clip}%
\pgfsetbuttcap%
\pgfsetroundjoin%
\pgfsetlinewidth{1.003750pt}%
\definecolor{currentstroke}{rgb}{0.667253,0.779176,0.992959}%
\pgfsetstrokecolor{currentstroke}%
\pgfsetdash{}{0pt}%
\pgfpathmoveto{\pgfqpoint{2.777619in}{0.837544in}}%
\pgfpathlineto{\pgfqpoint{2.777619in}{0.837544in}}%
\pgfusepath{stroke}%
\end{pgfscope}%
\begin{pgfscope}%
\pgfpathrectangle{\pgfqpoint{0.552773in}{0.431673in}}{\pgfqpoint{3.738807in}{1.765244in}}%
\pgfusepath{clip}%
\pgfsetbuttcap%
\pgfsetroundjoin%
\pgfsetlinewidth{1.003750pt}%
\definecolor{currentstroke}{rgb}{0.667253,0.779176,0.992959}%
\pgfsetstrokecolor{currentstroke}%
\pgfsetdash{}{0pt}%
\pgfpathmoveto{\pgfqpoint{2.773219in}{0.769640in}}%
\pgfpathlineto{\pgfqpoint{2.773219in}{0.769640in}}%
\pgfusepath{stroke}%
\end{pgfscope}%
\begin{pgfscope}%
\pgfpathrectangle{\pgfqpoint{0.552773in}{0.431673in}}{\pgfqpoint{3.738807in}{1.765244in}}%
\pgfusepath{clip}%
\pgfsetbuttcap%
\pgfsetroundjoin%
\pgfsetlinewidth{1.003750pt}%
\definecolor{currentstroke}{rgb}{0.667253,0.779176,0.992959}%
\pgfsetstrokecolor{currentstroke}%
\pgfsetdash{}{0pt}%
\pgfpathmoveto{\pgfqpoint{2.773093in}{0.709516in}}%
\pgfpathlineto{\pgfqpoint{2.773093in}{0.709516in}}%
\pgfusepath{stroke}%
\end{pgfscope}%
\begin{pgfscope}%
\pgfpathrectangle{\pgfqpoint{0.552773in}{0.431673in}}{\pgfqpoint{3.738807in}{1.765244in}}%
\pgfusepath{clip}%
\pgfsetbuttcap%
\pgfsetroundjoin%
\pgfsetlinewidth{1.003750pt}%
\definecolor{currentstroke}{rgb}{0.667253,0.779176,0.992959}%
\pgfsetstrokecolor{currentstroke}%
\pgfsetdash{}{0pt}%
\pgfpathmoveto{\pgfqpoint{2.775290in}{0.653624in}}%
\pgfpathlineto{\pgfqpoint{2.775290in}{0.653624in}}%
\pgfusepath{stroke}%
\end{pgfscope}%
\begin{pgfscope}%
\pgfpathrectangle{\pgfqpoint{0.552773in}{0.431673in}}{\pgfqpoint{3.738807in}{1.765244in}}%
\pgfusepath{clip}%
\pgfsetbuttcap%
\pgfsetroundjoin%
\pgfsetlinewidth{1.003750pt}%
\definecolor{currentstroke}{rgb}{0.667253,0.779176,0.992959}%
\pgfsetstrokecolor{currentstroke}%
\pgfsetdash{}{0pt}%
\pgfpathmoveto{\pgfqpoint{2.792438in}{0.624951in}}%
\pgfpathlineto{\pgfqpoint{2.792438in}{0.624951in}}%
\pgfusepath{stroke}%
\end{pgfscope}%
\begin{pgfscope}%
\pgfpathrectangle{\pgfqpoint{0.552773in}{0.431673in}}{\pgfqpoint{3.738807in}{1.765244in}}%
\pgfusepath{clip}%
\pgfsetbuttcap%
\pgfsetroundjoin%
\pgfsetlinewidth{1.003750pt}%
\definecolor{currentstroke}{rgb}{0.667253,0.779176,0.992959}%
\pgfsetstrokecolor{currentstroke}%
\pgfsetdash{}{0pt}%
\pgfpathmoveto{\pgfqpoint{2.801363in}{0.581308in}}%
\pgfpathlineto{\pgfqpoint{2.801363in}{0.581308in}}%
\pgfusepath{stroke}%
\end{pgfscope}%
\begin{pgfscope}%
\pgfpathrectangle{\pgfqpoint{0.552773in}{0.431673in}}{\pgfqpoint{3.738807in}{1.765244in}}%
\pgfusepath{clip}%
\pgfsetbuttcap%
\pgfsetroundjoin%
\pgfsetlinewidth{1.003750pt}%
\definecolor{currentstroke}{rgb}{0.667253,0.779176,0.992959}%
\pgfsetstrokecolor{currentstroke}%
\pgfsetdash{}{0pt}%
\pgfpathmoveto{\pgfqpoint{2.816359in}{0.548717in}}%
\pgfpathlineto{\pgfqpoint{2.816359in}{0.548717in}}%
\pgfusepath{stroke}%
\end{pgfscope}%
\begin{pgfscope}%
\pgfpathrectangle{\pgfqpoint{0.552773in}{0.431673in}}{\pgfqpoint{3.738807in}{1.765244in}}%
\pgfusepath{clip}%
\pgfsetbuttcap%
\pgfsetroundjoin%
\pgfsetlinewidth{1.003750pt}%
\definecolor{currentstroke}{rgb}{0.667253,0.779176,0.992959}%
\pgfsetstrokecolor{currentstroke}%
\pgfsetdash{}{0pt}%
\pgfpathmoveto{\pgfqpoint{2.831119in}{0.515697in}}%
\pgfpathlineto{\pgfqpoint{2.831119in}{0.515697in}}%
\pgfusepath{stroke}%
\end{pgfscope}%
\begin{pgfscope}%
\pgfpathrectangle{\pgfqpoint{0.552773in}{0.431673in}}{\pgfqpoint{3.738807in}{1.765244in}}%
\pgfusepath{clip}%
\pgfsetbuttcap%
\pgfsetroundjoin%
\pgfsetlinewidth{1.003750pt}%
\definecolor{currentstroke}{rgb}{0.667253,0.779176,0.992959}%
\pgfsetstrokecolor{currentstroke}%
\pgfsetdash{}{0pt}%
\pgfpathmoveto{\pgfqpoint{2.870334in}{0.527203in}}%
\pgfpathlineto{\pgfqpoint{2.870334in}{0.527203in}}%
\pgfusepath{stroke}%
\end{pgfscope}%
\begin{pgfscope}%
\pgfpathrectangle{\pgfqpoint{0.552773in}{0.431673in}}{\pgfqpoint{3.738807in}{1.765244in}}%
\pgfusepath{clip}%
\pgfsetbuttcap%
\pgfsetroundjoin%
\pgfsetlinewidth{1.003750pt}%
\definecolor{currentstroke}{rgb}{0.667253,0.779176,0.992959}%
\pgfsetstrokecolor{currentstroke}%
\pgfsetdash{}{0pt}%
\pgfpathmoveto{\pgfqpoint{2.914301in}{0.547359in}}%
\pgfpathlineto{\pgfqpoint{2.914301in}{0.547359in}}%
\pgfusepath{stroke}%
\end{pgfscope}%
\begin{pgfscope}%
\pgfpathrectangle{\pgfqpoint{0.552773in}{0.431673in}}{\pgfqpoint{3.738807in}{1.765244in}}%
\pgfusepath{clip}%
\pgfsetbuttcap%
\pgfsetroundjoin%
\pgfsetlinewidth{1.003750pt}%
\definecolor{currentstroke}{rgb}{0.667253,0.779176,0.992959}%
\pgfsetstrokecolor{currentstroke}%
\pgfsetdash{}{0pt}%
\pgfpathmoveto{\pgfqpoint{2.968424in}{0.586007in}}%
\pgfpathlineto{\pgfqpoint{2.968424in}{0.586007in}}%
\pgfusepath{stroke}%
\end{pgfscope}%
\begin{pgfscope}%
\pgfpathrectangle{\pgfqpoint{0.552773in}{0.431673in}}{\pgfqpoint{3.738807in}{1.765244in}}%
\pgfusepath{clip}%
\pgfsetbuttcap%
\pgfsetroundjoin%
\pgfsetlinewidth{1.003750pt}%
\definecolor{currentstroke}{rgb}{0.667253,0.779176,0.992959}%
\pgfsetstrokecolor{currentstroke}%
\pgfsetdash{}{0pt}%
\pgfpathmoveto{\pgfqpoint{3.031090in}{0.640210in}}%
\pgfpathlineto{\pgfqpoint{3.031090in}{0.640210in}}%
\pgfusepath{stroke}%
\end{pgfscope}%
\begin{pgfscope}%
\pgfpathrectangle{\pgfqpoint{0.552773in}{0.431673in}}{\pgfqpoint{3.738807in}{1.765244in}}%
\pgfusepath{clip}%
\pgfsetbuttcap%
\pgfsetroundjoin%
\pgfsetlinewidth{1.003750pt}%
\definecolor{currentstroke}{rgb}{0.667253,0.779176,0.992959}%
\pgfsetstrokecolor{currentstroke}%
\pgfsetdash{}{0pt}%
\pgfpathmoveto{\pgfqpoint{3.108456in}{0.721175in}}%
\pgfpathlineto{\pgfqpoint{3.108456in}{0.721175in}}%
\pgfusepath{stroke}%
\end{pgfscope}%
\begin{pgfscope}%
\pgfpathrectangle{\pgfqpoint{0.552773in}{0.431673in}}{\pgfqpoint{3.738807in}{1.765244in}}%
\pgfusepath{clip}%
\pgfsetbuttcap%
\pgfsetroundjoin%
\pgfsetlinewidth{1.003750pt}%
\definecolor{currentstroke}{rgb}{0.667253,0.779176,0.992959}%
\pgfsetstrokecolor{currentstroke}%
\pgfsetdash{}{0pt}%
\pgfpathmoveto{\pgfqpoint{3.183879in}{0.798603in}}%
\pgfpathlineto{\pgfqpoint{3.183879in}{0.798603in}}%
\pgfusepath{stroke}%
\end{pgfscope}%
\begin{pgfscope}%
\pgfpathrectangle{\pgfqpoint{0.552773in}{0.431673in}}{\pgfqpoint{3.738807in}{1.765244in}}%
\pgfusepath{clip}%
\pgfsetbuttcap%
\pgfsetroundjoin%
\pgfsetlinewidth{1.003750pt}%
\definecolor{currentstroke}{rgb}{0.667253,0.779176,0.992959}%
\pgfsetstrokecolor{currentstroke}%
\pgfsetdash{}{0pt}%
\pgfpathmoveto{\pgfqpoint{3.274791in}{0.904233in}}%
\pgfpathlineto{\pgfqpoint{3.274791in}{0.904233in}}%
\pgfusepath{stroke}%
\end{pgfscope}%
\begin{pgfscope}%
\pgfpathrectangle{\pgfqpoint{0.552773in}{0.431673in}}{\pgfqpoint{3.738807in}{1.765244in}}%
\pgfusepath{clip}%
\pgfsetbuttcap%
\pgfsetroundjoin%
\pgfsetlinewidth{1.003750pt}%
\definecolor{currentstroke}{rgb}{0.667253,0.779176,0.992959}%
\pgfsetstrokecolor{currentstroke}%
\pgfsetdash{}{0pt}%
\pgfpathmoveto{\pgfqpoint{3.351275in}{0.983593in}}%
\pgfpathlineto{\pgfqpoint{3.351275in}{0.983593in}}%
\pgfusepath{stroke}%
\end{pgfscope}%
\begin{pgfscope}%
\pgfpathrectangle{\pgfqpoint{0.552773in}{0.431673in}}{\pgfqpoint{3.738807in}{1.765244in}}%
\pgfusepath{clip}%
\pgfsetbuttcap%
\pgfsetroundjoin%
\pgfsetlinewidth{1.003750pt}%
\definecolor{currentstroke}{rgb}{0.667253,0.779176,0.992959}%
\pgfsetstrokecolor{currentstroke}%
\pgfsetdash{}{0pt}%
\pgfpathmoveto{\pgfqpoint{3.402422in}{1.016823in}}%
\pgfpathlineto{\pgfqpoint{3.402422in}{1.016823in}}%
\pgfusepath{stroke}%
\end{pgfscope}%
\begin{pgfscope}%
\pgfpathrectangle{\pgfqpoint{0.552773in}{0.431673in}}{\pgfqpoint{3.738807in}{1.765244in}}%
\pgfusepath{clip}%
\pgfsetbuttcap%
\pgfsetroundjoin%
\pgfsetlinewidth{1.003750pt}%
\definecolor{currentstroke}{rgb}{0.667253,0.779176,0.992959}%
\pgfsetstrokecolor{currentstroke}%
\pgfsetdash{}{0pt}%
\pgfpathmoveto{\pgfqpoint{3.426281in}{1.000369in}}%
\pgfpathlineto{\pgfqpoint{3.426281in}{1.000369in}}%
\pgfusepath{stroke}%
\end{pgfscope}%
\begin{pgfscope}%
\pgfpathrectangle{\pgfqpoint{0.552773in}{0.431673in}}{\pgfqpoint{3.738807in}{1.765244in}}%
\pgfusepath{clip}%
\pgfsetbuttcap%
\pgfsetroundjoin%
\pgfsetlinewidth{1.003750pt}%
\definecolor{currentstroke}{rgb}{0.667253,0.779176,0.992959}%
\pgfsetstrokecolor{currentstroke}%
\pgfsetdash{}{0pt}%
\pgfpathmoveto{\pgfqpoint{3.431078in}{0.949209in}}%
\pgfpathlineto{\pgfqpoint{3.431078in}{0.949209in}}%
\pgfusepath{stroke}%
\end{pgfscope}%
\begin{pgfscope}%
\pgfpathrectangle{\pgfqpoint{0.552773in}{0.431673in}}{\pgfqpoint{3.738807in}{1.765244in}}%
\pgfusepath{clip}%
\pgfsetbuttcap%
\pgfsetroundjoin%
\pgfsetlinewidth{1.003750pt}%
\definecolor{currentstroke}{rgb}{0.667253,0.779176,0.992959}%
\pgfsetstrokecolor{currentstroke}%
\pgfsetdash{}{0pt}%
\pgfpathmoveto{\pgfqpoint{3.439962in}{0.905491in}}%
\pgfpathlineto{\pgfqpoint{3.439962in}{0.905491in}}%
\pgfusepath{stroke}%
\end{pgfscope}%
\begin{pgfscope}%
\pgfpathrectangle{\pgfqpoint{0.552773in}{0.431673in}}{\pgfqpoint{3.738807in}{1.765244in}}%
\pgfusepath{clip}%
\pgfsetbuttcap%
\pgfsetroundjoin%
\pgfsetlinewidth{1.003750pt}%
\definecolor{currentstroke}{rgb}{0.667253,0.779176,0.992959}%
\pgfsetstrokecolor{currentstroke}%
\pgfsetdash{}{0pt}%
\pgfpathmoveto{\pgfqpoint{3.446892in}{0.858216in}}%
\pgfpathlineto{\pgfqpoint{3.446892in}{0.858216in}}%
\pgfusepath{stroke}%
\end{pgfscope}%
\begin{pgfscope}%
\pgfpathrectangle{\pgfqpoint{0.552773in}{0.431673in}}{\pgfqpoint{3.738807in}{1.765244in}}%
\pgfusepath{clip}%
\pgfsetbuttcap%
\pgfsetroundjoin%
\pgfsetlinewidth{1.003750pt}%
\definecolor{currentstroke}{rgb}{0.667253,0.779176,0.992959}%
\pgfsetstrokecolor{currentstroke}%
\pgfsetdash{}{0pt}%
\pgfpathmoveto{\pgfqpoint{3.456265in}{0.815386in}}%
\pgfpathlineto{\pgfqpoint{3.456265in}{0.815386in}}%
\pgfusepath{stroke}%
\end{pgfscope}%
\begin{pgfscope}%
\pgfpathrectangle{\pgfqpoint{0.552773in}{0.431673in}}{\pgfqpoint{3.738807in}{1.765244in}}%
\pgfusepath{clip}%
\pgfsetbuttcap%
\pgfsetroundjoin%
\pgfsetlinewidth{1.003750pt}%
\definecolor{currentstroke}{rgb}{0.667253,0.779176,0.992959}%
\pgfsetstrokecolor{currentstroke}%
\pgfsetdash{}{0pt}%
\pgfpathmoveto{\pgfqpoint{3.455917in}{0.754860in}}%
\pgfpathlineto{\pgfqpoint{3.455917in}{0.754860in}}%
\pgfusepath{stroke}%
\end{pgfscope}%
\begin{pgfscope}%
\pgfpathrectangle{\pgfqpoint{0.552773in}{0.431673in}}{\pgfqpoint{3.738807in}{1.765244in}}%
\pgfusepath{clip}%
\pgfsetbuttcap%
\pgfsetroundjoin%
\pgfsetlinewidth{1.003750pt}%
\definecolor{currentstroke}{rgb}{0.667253,0.779176,0.992959}%
\pgfsetstrokecolor{currentstroke}%
\pgfsetdash{}{0pt}%
\pgfpathmoveto{\pgfqpoint{3.471332in}{0.723033in}}%
\pgfpathlineto{\pgfqpoint{3.471332in}{0.723033in}}%
\pgfusepath{stroke}%
\end{pgfscope}%
\begin{pgfscope}%
\pgfpathrectangle{\pgfqpoint{0.552773in}{0.431673in}}{\pgfqpoint{3.738807in}{1.765244in}}%
\pgfusepath{clip}%
\pgfsetbuttcap%
\pgfsetroundjoin%
\pgfsetlinewidth{1.003750pt}%
\definecolor{currentstroke}{rgb}{0.667253,0.779176,0.992959}%
\pgfsetstrokecolor{currentstroke}%
\pgfsetdash{}{0pt}%
\pgfpathmoveto{\pgfqpoint{3.492791in}{0.702209in}}%
\pgfpathlineto{\pgfqpoint{3.492791in}{0.702209in}}%
\pgfusepath{stroke}%
\end{pgfscope}%
\begin{pgfscope}%
\pgfpathrectangle{\pgfqpoint{0.552773in}{0.431673in}}{\pgfqpoint{3.738807in}{1.765244in}}%
\pgfusepath{clip}%
\pgfsetbuttcap%
\pgfsetroundjoin%
\pgfsetlinewidth{1.003750pt}%
\definecolor{currentstroke}{rgb}{0.667253,0.779176,0.992959}%
\pgfsetstrokecolor{currentstroke}%
\pgfsetdash{}{0pt}%
\pgfpathmoveto{\pgfqpoint{3.581754in}{0.684504in}}%
\pgfpathlineto{\pgfqpoint{3.581754in}{0.684504in}}%
\pgfusepath{stroke}%
\end{pgfscope}%
\begin{pgfscope}%
\pgfpathrectangle{\pgfqpoint{0.552773in}{0.431673in}}{\pgfqpoint{3.738807in}{1.765244in}}%
\pgfusepath{clip}%
\pgfsetbuttcap%
\pgfsetroundjoin%
\pgfsetlinewidth{1.003750pt}%
\definecolor{currentstroke}{rgb}{0.667253,0.779176,0.992959}%
\pgfsetstrokecolor{currentstroke}%
\pgfsetdash{}{0pt}%
\pgfpathmoveto{\pgfqpoint{3.619118in}{0.692639in}}%
\pgfpathlineto{\pgfqpoint{3.619118in}{0.692639in}}%
\pgfusepath{stroke}%
\end{pgfscope}%
\begin{pgfscope}%
\pgfpathrectangle{\pgfqpoint{0.552773in}{0.431673in}}{\pgfqpoint{3.738807in}{1.765244in}}%
\pgfusepath{clip}%
\pgfsetbuttcap%
\pgfsetroundjoin%
\pgfsetlinewidth{1.003750pt}%
\definecolor{currentstroke}{rgb}{0.667253,0.779176,0.992959}%
\pgfsetstrokecolor{currentstroke}%
\pgfsetdash{}{0pt}%
\pgfpathmoveto{\pgfqpoint{3.681794in}{0.746858in}}%
\pgfpathlineto{\pgfqpoint{3.681794in}{0.746858in}}%
\pgfusepath{stroke}%
\end{pgfscope}%
\begin{pgfscope}%
\pgfpathrectangle{\pgfqpoint{0.552773in}{0.431673in}}{\pgfqpoint{3.738807in}{1.765244in}}%
\pgfusepath{clip}%
\pgfsetbuttcap%
\pgfsetroundjoin%
\definecolor{currentfill}{rgb}{0.667253,0.779176,0.992959}%
\pgfsetfillcolor{currentfill}%
\pgfsetlinewidth{1.003750pt}%
\definecolor{currentstroke}{rgb}{0.667253,0.779176,0.992959}%
\pgfsetstrokecolor{currentstroke}%
\pgfsetdash{}{0pt}%
\pgfsys@defobject{currentmarker}{\pgfqpoint{0.000000in}{-0.027778in}}{\pgfqpoint{0.000000in}{0.027778in}}{%
\pgfpathmoveto{\pgfqpoint{0.000000in}{-0.027778in}}%
\pgfpathlineto{\pgfqpoint{0.000000in}{0.027778in}}%
\pgfusepath{stroke,fill}%
}%
\begin{pgfscope}%
\pgfsys@transformshift{0.946281in}{0.917169in}%
\pgfsys@useobject{currentmarker}{}%
\end{pgfscope}%
\begin{pgfscope}%
\pgfsys@transformshift{0.968004in}{0.896826in}%
\pgfsys@useobject{currentmarker}{}%
\end{pgfscope}%
\begin{pgfscope}%
\pgfsys@transformshift{0.983367in}{0.864904in}%
\pgfsys@useobject{currentmarker}{}%
\end{pgfscope}%
\begin{pgfscope}%
\pgfsys@transformshift{0.997288in}{0.830357in}%
\pgfsys@useobject{currentmarker}{}%
\end{pgfscope}%
\begin{pgfscope}%
\pgfsys@transformshift{1.003998in}{0.782680in}%
\pgfsys@useobject{currentmarker}{}%
\end{pgfscope}%
\begin{pgfscope}%
\pgfsys@transformshift{1.002620in}{0.720278in}%
\pgfsys@useobject{currentmarker}{}%
\end{pgfscope}%
\begin{pgfscope}%
\pgfsys@transformshift{1.016428in}{0.685524in}%
\pgfsys@useobject{currentmarker}{}%
\end{pgfscope}%
\begin{pgfscope}%
\pgfsys@transformshift{1.016366in}{0.625519in}%
\pgfsys@useobject{currentmarker}{}%
\end{pgfscope}%
\begin{pgfscope}%
\pgfsys@transformshift{1.044197in}{0.616297in}%
\pgfsys@useobject{currentmarker}{}%
\end{pgfscope}%
\begin{pgfscope}%
\pgfsys@transformshift{1.061881in}{0.588600in}%
\pgfsys@useobject{currentmarker}{}%
\end{pgfscope}%
\begin{pgfscope}%
\pgfsys@transformshift{1.100011in}{0.598129in}%
\pgfsys@useobject{currentmarker}{}%
\end{pgfscope}%
\begin{pgfscope}%
\pgfsys@transformshift{1.133830in}{0.599811in}%
\pgfsys@useobject{currentmarker}{}%
\end{pgfscope}%
\begin{pgfscope}%
\pgfsys@transformshift{1.178595in}{0.621420in}%
\pgfsys@useobject{currentmarker}{}%
\end{pgfscope}%
\begin{pgfscope}%
\pgfsys@transformshift{1.241991in}{0.676951in}%
\pgfsys@useobject{currentmarker}{}%
\end{pgfscope}%
\begin{pgfscope}%
\pgfsys@transformshift{1.316482in}{0.752683in}%
\pgfsys@useobject{currentmarker}{}%
\end{pgfscope}%
\begin{pgfscope}%
\pgfsys@transformshift{1.386975in}{0.821134in}%
\pgfsys@useobject{currentmarker}{}%
\end{pgfscope}%
\begin{pgfscope}%
\pgfsys@transformshift{1.458963in}{0.892310in}%
\pgfsys@useobject{currentmarker}{}%
\end{pgfscope}%
\begin{pgfscope}%
\pgfsys@transformshift{1.515751in}{0.935810in}%
\pgfsys@useobject{currentmarker}{}%
\end{pgfscope}%
\begin{pgfscope}%
\pgfsys@transformshift{1.552230in}{0.942332in}%
\pgfsys@useobject{currentmarker}{}%
\end{pgfscope}%
\begin{pgfscope}%
\pgfsys@transformshift{1.571226in}{0.917025in}%
\pgfsys@useobject{currentmarker}{}%
\end{pgfscope}%
\begin{pgfscope}%
\pgfsys@transformshift{1.578686in}{0.870715in}%
\pgfsys@useobject{currentmarker}{}%
\end{pgfscope}%
\begin{pgfscope}%
\pgfsys@transformshift{1.598839in}{0.847513in}%
\pgfsys@useobject{currentmarker}{}%
\end{pgfscope}%
\begin{pgfscope}%
\pgfsys@transformshift{1.615264in}{0.817524in}%
\pgfsys@useobject{currentmarker}{}%
\end{pgfscope}%
\begin{pgfscope}%
\pgfsys@transformshift{1.629986in}{0.784436in}%
\pgfsys@useobject{currentmarker}{}%
\end{pgfscope}%
\begin{pgfscope}%
\pgfsys@transformshift{1.633661in}{0.731234in}%
\pgfsys@useobject{currentmarker}{}%
\end{pgfscope}%
\begin{pgfscope}%
\pgfsys@transformshift{1.644849in}{0.691711in}%
\pgfsys@useobject{currentmarker}{}%
\end{pgfscope}%
\begin{pgfscope}%
\pgfsys@transformshift{1.654597in}{0.649565in}%
\pgfsys@useobject{currentmarker}{}%
\end{pgfscope}%
\begin{pgfscope}%
\pgfsys@transformshift{1.665591in}{0.609687in}%
\pgfsys@useobject{currentmarker}{}%
\end{pgfscope}%
\begin{pgfscope}%
\pgfsys@transformshift{1.686427in}{0.587731in}%
\pgfsys@useobject{currentmarker}{}%
\end{pgfscope}%
\begin{pgfscope}%
\pgfsys@transformshift{1.711171in}{0.572888in}%
\pgfsys@useobject{currentmarker}{}%
\end{pgfscope}%
\begin{pgfscope}%
\pgfsys@transformshift{1.758001in}{0.598259in}%
\pgfsys@useobject{currentmarker}{}%
\end{pgfscope}%
\begin{pgfscope}%
\pgfsys@transformshift{1.816028in}{0.644014in}%
\pgfsys@useobject{currentmarker}{}%
\end{pgfscope}%
\begin{pgfscope}%
\pgfsys@transformshift{1.879012in}{0.698794in}%
\pgfsys@useobject{currentmarker}{}%
\end{pgfscope}%
\begin{pgfscope}%
\pgfsys@transformshift{1.944926in}{0.758909in}%
\pgfsys@useobject{currentmarker}{}%
\end{pgfscope}%
\begin{pgfscope}%
\pgfsys@transformshift{2.017564in}{0.831267in}%
\pgfsys@useobject{currentmarker}{}%
\end{pgfscope}%
\begin{pgfscope}%
\pgfsys@transformshift{2.077394in}{0.880307in}%
\pgfsys@useobject{currentmarker}{}%
\end{pgfscope}%
\begin{pgfscope}%
\pgfsys@transformshift{2.120452in}{0.898808in}%
\pgfsys@useobject{currentmarker}{}%
\end{pgfscope}%
\begin{pgfscope}%
\pgfsys@transformshift{2.148843in}{0.890606in}%
\pgfsys@useobject{currentmarker}{}%
\end{pgfscope}%
\begin{pgfscope}%
\pgfsys@transformshift{2.169787in}{0.868845in}%
\pgfsys@useobject{currentmarker}{}%
\end{pgfscope}%
\begin{pgfscope}%
\pgfsys@transformshift{2.191999in}{0.849394in}%
\pgfsys@useobject{currentmarker}{}%
\end{pgfscope}%
\begin{pgfscope}%
\pgfsys@transformshift{2.201330in}{0.806489in}%
\pgfsys@useobject{currentmarker}{}%
\end{pgfscope}%
\begin{pgfscope}%
\pgfsys@transformshift{2.202600in}{0.748907in}%
\pgfsys@useobject{currentmarker}{}%
\end{pgfscope}%
\begin{pgfscope}%
\pgfsys@transformshift{2.202588in}{0.688992in}%
\pgfsys@useobject{currentmarker}{}%
\end{pgfscope}%
\begin{pgfscope}%
\pgfsys@transformshift{2.206526in}{0.636268in}%
\pgfsys@useobject{currentmarker}{}%
\end{pgfscope}%
\begin{pgfscope}%
\pgfsys@transformshift{2.205818in}{0.575086in}%
\pgfsys@useobject{currentmarker}{}%
\end{pgfscope}%
\begin{pgfscope}%
\pgfsys@transformshift{2.225473in}{0.550978in}%
\pgfsys@useobject{currentmarker}{}%
\end{pgfscope}%
\begin{pgfscope}%
\pgfsys@transformshift{2.247923in}{0.531960in}%
\pgfsys@useobject{currentmarker}{}%
\end{pgfscope}%
\begin{pgfscope}%
\pgfsys@transformshift{2.269807in}{0.511911in}%
\pgfsys@useobject{currentmarker}{}%
\end{pgfscope}%
\begin{pgfscope}%
\pgfsys@transformshift{2.310165in}{0.525497in}%
\pgfsys@useobject{currentmarker}{}%
\end{pgfscope}%
\begin{pgfscope}%
\pgfsys@transformshift{2.369172in}{0.573036in}%
\pgfsys@useobject{currentmarker}{}%
\end{pgfscope}%
\begin{pgfscope}%
\pgfsys@transformshift{2.433038in}{0.629423in}%
\pgfsys@useobject{currentmarker}{}%
\end{pgfscope}%
\begin{pgfscope}%
\pgfsys@transformshift{2.512551in}{0.714298in}%
\pgfsys@useobject{currentmarker}{}%
\end{pgfscope}%
\begin{pgfscope}%
\pgfsys@transformshift{2.584448in}{0.785307in}%
\pgfsys@useobject{currentmarker}{}%
\end{pgfscope}%
\begin{pgfscope}%
\pgfsys@transformshift{2.654396in}{0.852767in}%
\pgfsys@useobject{currentmarker}{}%
\end{pgfscope}%
\begin{pgfscope}%
\pgfsys@transformshift{2.709975in}{0.894066in}%
\pgfsys@useobject{currentmarker}{}%
\end{pgfscope}%
\begin{pgfscope}%
\pgfsys@transformshift{2.745485in}{0.898825in}%
\pgfsys@useobject{currentmarker}{}%
\end{pgfscope}%
\begin{pgfscope}%
\pgfsys@transformshift{2.765939in}{0.876173in}%
\pgfsys@useobject{currentmarker}{}%
\end{pgfscope}%
\begin{pgfscope}%
\pgfsys@transformshift{2.777619in}{0.837544in}%
\pgfsys@useobject{currentmarker}{}%
\end{pgfscope}%
\begin{pgfscope}%
\pgfsys@transformshift{2.773219in}{0.769640in}%
\pgfsys@useobject{currentmarker}{}%
\end{pgfscope}%
\begin{pgfscope}%
\pgfsys@transformshift{2.773093in}{0.709516in}%
\pgfsys@useobject{currentmarker}{}%
\end{pgfscope}%
\begin{pgfscope}%
\pgfsys@transformshift{2.775290in}{0.653624in}%
\pgfsys@useobject{currentmarker}{}%
\end{pgfscope}%
\begin{pgfscope}%
\pgfsys@transformshift{2.792438in}{0.624951in}%
\pgfsys@useobject{currentmarker}{}%
\end{pgfscope}%
\begin{pgfscope}%
\pgfsys@transformshift{2.801363in}{0.581308in}%
\pgfsys@useobject{currentmarker}{}%
\end{pgfscope}%
\begin{pgfscope}%
\pgfsys@transformshift{2.816359in}{0.548717in}%
\pgfsys@useobject{currentmarker}{}%
\end{pgfscope}%
\begin{pgfscope}%
\pgfsys@transformshift{2.831119in}{0.515697in}%
\pgfsys@useobject{currentmarker}{}%
\end{pgfscope}%
\begin{pgfscope}%
\pgfsys@transformshift{2.870334in}{0.527203in}%
\pgfsys@useobject{currentmarker}{}%
\end{pgfscope}%
\begin{pgfscope}%
\pgfsys@transformshift{2.914301in}{0.547359in}%
\pgfsys@useobject{currentmarker}{}%
\end{pgfscope}%
\begin{pgfscope}%
\pgfsys@transformshift{2.968424in}{0.586007in}%
\pgfsys@useobject{currentmarker}{}%
\end{pgfscope}%
\begin{pgfscope}%
\pgfsys@transformshift{3.031090in}{0.640210in}%
\pgfsys@useobject{currentmarker}{}%
\end{pgfscope}%
\begin{pgfscope}%
\pgfsys@transformshift{3.108456in}{0.721175in}%
\pgfsys@useobject{currentmarker}{}%
\end{pgfscope}%
\begin{pgfscope}%
\pgfsys@transformshift{3.183879in}{0.798603in}%
\pgfsys@useobject{currentmarker}{}%
\end{pgfscope}%
\begin{pgfscope}%
\pgfsys@transformshift{3.274791in}{0.904233in}%
\pgfsys@useobject{currentmarker}{}%
\end{pgfscope}%
\begin{pgfscope}%
\pgfsys@transformshift{3.351275in}{0.983593in}%
\pgfsys@useobject{currentmarker}{}%
\end{pgfscope}%
\begin{pgfscope}%
\pgfsys@transformshift{3.402422in}{1.016823in}%
\pgfsys@useobject{currentmarker}{}%
\end{pgfscope}%
\begin{pgfscope}%
\pgfsys@transformshift{3.426281in}{1.000369in}%
\pgfsys@useobject{currentmarker}{}%
\end{pgfscope}%
\begin{pgfscope}%
\pgfsys@transformshift{3.431078in}{0.949209in}%
\pgfsys@useobject{currentmarker}{}%
\end{pgfscope}%
\begin{pgfscope}%
\pgfsys@transformshift{3.439962in}{0.905491in}%
\pgfsys@useobject{currentmarker}{}%
\end{pgfscope}%
\begin{pgfscope}%
\pgfsys@transformshift{3.446892in}{0.858216in}%
\pgfsys@useobject{currentmarker}{}%
\end{pgfscope}%
\begin{pgfscope}%
\pgfsys@transformshift{3.456265in}{0.815386in}%
\pgfsys@useobject{currentmarker}{}%
\end{pgfscope}%
\begin{pgfscope}%
\pgfsys@transformshift{3.455917in}{0.754860in}%
\pgfsys@useobject{currentmarker}{}%
\end{pgfscope}%
\begin{pgfscope}%
\pgfsys@transformshift{3.471332in}{0.723033in}%
\pgfsys@useobject{currentmarker}{}%
\end{pgfscope}%
\begin{pgfscope}%
\pgfsys@transformshift{3.492791in}{0.702209in}%
\pgfsys@useobject{currentmarker}{}%
\end{pgfscope}%
\begin{pgfscope}%
\pgfsys@transformshift{3.581754in}{0.684504in}%
\pgfsys@useobject{currentmarker}{}%
\end{pgfscope}%
\begin{pgfscope}%
\pgfsys@transformshift{3.619118in}{0.692639in}%
\pgfsys@useobject{currentmarker}{}%
\end{pgfscope}%
\begin{pgfscope}%
\pgfsys@transformshift{3.681794in}{0.746858in}%
\pgfsys@useobject{currentmarker}{}%
\end{pgfscope}%
\end{pgfscope}%
\begin{pgfscope}%
\pgfpathrectangle{\pgfqpoint{0.552773in}{0.431673in}}{\pgfqpoint{3.738807in}{1.765244in}}%
\pgfusepath{clip}%
\pgfsetbuttcap%
\pgfsetroundjoin%
\definecolor{currentfill}{rgb}{0.667253,0.779176,0.992959}%
\pgfsetfillcolor{currentfill}%
\pgfsetlinewidth{1.003750pt}%
\definecolor{currentstroke}{rgb}{0.667253,0.779176,0.992959}%
\pgfsetstrokecolor{currentstroke}%
\pgfsetdash{}{0pt}%
\pgfsys@defobject{currentmarker}{\pgfqpoint{0.000000in}{-0.027778in}}{\pgfqpoint{0.000000in}{0.027778in}}{%
\pgfpathmoveto{\pgfqpoint{0.000000in}{-0.027778in}}%
\pgfpathlineto{\pgfqpoint{0.000000in}{0.027778in}}%
\pgfusepath{stroke,fill}%
}%
\begin{pgfscope}%
\pgfsys@transformshift{0.946281in}{0.917169in}%
\pgfsys@useobject{currentmarker}{}%
\end{pgfscope}%
\begin{pgfscope}%
\pgfsys@transformshift{0.968004in}{0.896826in}%
\pgfsys@useobject{currentmarker}{}%
\end{pgfscope}%
\begin{pgfscope}%
\pgfsys@transformshift{0.983367in}{0.864904in}%
\pgfsys@useobject{currentmarker}{}%
\end{pgfscope}%
\begin{pgfscope}%
\pgfsys@transformshift{0.997288in}{0.830357in}%
\pgfsys@useobject{currentmarker}{}%
\end{pgfscope}%
\begin{pgfscope}%
\pgfsys@transformshift{1.003998in}{0.782680in}%
\pgfsys@useobject{currentmarker}{}%
\end{pgfscope}%
\begin{pgfscope}%
\pgfsys@transformshift{1.002620in}{0.720278in}%
\pgfsys@useobject{currentmarker}{}%
\end{pgfscope}%
\begin{pgfscope}%
\pgfsys@transformshift{1.016428in}{0.685524in}%
\pgfsys@useobject{currentmarker}{}%
\end{pgfscope}%
\begin{pgfscope}%
\pgfsys@transformshift{1.016366in}{0.625519in}%
\pgfsys@useobject{currentmarker}{}%
\end{pgfscope}%
\begin{pgfscope}%
\pgfsys@transformshift{1.044197in}{0.616297in}%
\pgfsys@useobject{currentmarker}{}%
\end{pgfscope}%
\begin{pgfscope}%
\pgfsys@transformshift{1.061881in}{0.588600in}%
\pgfsys@useobject{currentmarker}{}%
\end{pgfscope}%
\begin{pgfscope}%
\pgfsys@transformshift{1.100011in}{0.598129in}%
\pgfsys@useobject{currentmarker}{}%
\end{pgfscope}%
\begin{pgfscope}%
\pgfsys@transformshift{1.133830in}{0.599811in}%
\pgfsys@useobject{currentmarker}{}%
\end{pgfscope}%
\begin{pgfscope}%
\pgfsys@transformshift{1.178595in}{0.621420in}%
\pgfsys@useobject{currentmarker}{}%
\end{pgfscope}%
\begin{pgfscope}%
\pgfsys@transformshift{1.241991in}{0.676951in}%
\pgfsys@useobject{currentmarker}{}%
\end{pgfscope}%
\begin{pgfscope}%
\pgfsys@transformshift{1.316482in}{0.752683in}%
\pgfsys@useobject{currentmarker}{}%
\end{pgfscope}%
\begin{pgfscope}%
\pgfsys@transformshift{1.386975in}{0.821134in}%
\pgfsys@useobject{currentmarker}{}%
\end{pgfscope}%
\begin{pgfscope}%
\pgfsys@transformshift{1.458963in}{0.892310in}%
\pgfsys@useobject{currentmarker}{}%
\end{pgfscope}%
\begin{pgfscope}%
\pgfsys@transformshift{1.515751in}{0.935810in}%
\pgfsys@useobject{currentmarker}{}%
\end{pgfscope}%
\begin{pgfscope}%
\pgfsys@transformshift{1.552230in}{0.942332in}%
\pgfsys@useobject{currentmarker}{}%
\end{pgfscope}%
\begin{pgfscope}%
\pgfsys@transformshift{1.571226in}{0.917025in}%
\pgfsys@useobject{currentmarker}{}%
\end{pgfscope}%
\begin{pgfscope}%
\pgfsys@transformshift{1.578686in}{0.870715in}%
\pgfsys@useobject{currentmarker}{}%
\end{pgfscope}%
\begin{pgfscope}%
\pgfsys@transformshift{1.598839in}{0.847513in}%
\pgfsys@useobject{currentmarker}{}%
\end{pgfscope}%
\begin{pgfscope}%
\pgfsys@transformshift{1.615264in}{0.817524in}%
\pgfsys@useobject{currentmarker}{}%
\end{pgfscope}%
\begin{pgfscope}%
\pgfsys@transformshift{1.629986in}{0.784436in}%
\pgfsys@useobject{currentmarker}{}%
\end{pgfscope}%
\begin{pgfscope}%
\pgfsys@transformshift{1.633661in}{0.731234in}%
\pgfsys@useobject{currentmarker}{}%
\end{pgfscope}%
\begin{pgfscope}%
\pgfsys@transformshift{1.644849in}{0.691711in}%
\pgfsys@useobject{currentmarker}{}%
\end{pgfscope}%
\begin{pgfscope}%
\pgfsys@transformshift{1.654597in}{0.649565in}%
\pgfsys@useobject{currentmarker}{}%
\end{pgfscope}%
\begin{pgfscope}%
\pgfsys@transformshift{1.665591in}{0.609687in}%
\pgfsys@useobject{currentmarker}{}%
\end{pgfscope}%
\begin{pgfscope}%
\pgfsys@transformshift{1.686427in}{0.587731in}%
\pgfsys@useobject{currentmarker}{}%
\end{pgfscope}%
\begin{pgfscope}%
\pgfsys@transformshift{1.711171in}{0.572888in}%
\pgfsys@useobject{currentmarker}{}%
\end{pgfscope}%
\begin{pgfscope}%
\pgfsys@transformshift{1.758001in}{0.598259in}%
\pgfsys@useobject{currentmarker}{}%
\end{pgfscope}%
\begin{pgfscope}%
\pgfsys@transformshift{1.816028in}{0.644014in}%
\pgfsys@useobject{currentmarker}{}%
\end{pgfscope}%
\begin{pgfscope}%
\pgfsys@transformshift{1.879012in}{0.698794in}%
\pgfsys@useobject{currentmarker}{}%
\end{pgfscope}%
\begin{pgfscope}%
\pgfsys@transformshift{1.944926in}{0.758909in}%
\pgfsys@useobject{currentmarker}{}%
\end{pgfscope}%
\begin{pgfscope}%
\pgfsys@transformshift{2.017564in}{0.831267in}%
\pgfsys@useobject{currentmarker}{}%
\end{pgfscope}%
\begin{pgfscope}%
\pgfsys@transformshift{2.077394in}{0.880307in}%
\pgfsys@useobject{currentmarker}{}%
\end{pgfscope}%
\begin{pgfscope}%
\pgfsys@transformshift{2.120452in}{0.898808in}%
\pgfsys@useobject{currentmarker}{}%
\end{pgfscope}%
\begin{pgfscope}%
\pgfsys@transformshift{2.148843in}{0.890606in}%
\pgfsys@useobject{currentmarker}{}%
\end{pgfscope}%
\begin{pgfscope}%
\pgfsys@transformshift{2.169787in}{0.868845in}%
\pgfsys@useobject{currentmarker}{}%
\end{pgfscope}%
\begin{pgfscope}%
\pgfsys@transformshift{2.191999in}{0.849394in}%
\pgfsys@useobject{currentmarker}{}%
\end{pgfscope}%
\begin{pgfscope}%
\pgfsys@transformshift{2.201330in}{0.806489in}%
\pgfsys@useobject{currentmarker}{}%
\end{pgfscope}%
\begin{pgfscope}%
\pgfsys@transformshift{2.202600in}{0.748907in}%
\pgfsys@useobject{currentmarker}{}%
\end{pgfscope}%
\begin{pgfscope}%
\pgfsys@transformshift{2.202588in}{0.688992in}%
\pgfsys@useobject{currentmarker}{}%
\end{pgfscope}%
\begin{pgfscope}%
\pgfsys@transformshift{2.206526in}{0.636268in}%
\pgfsys@useobject{currentmarker}{}%
\end{pgfscope}%
\begin{pgfscope}%
\pgfsys@transformshift{2.205818in}{0.575086in}%
\pgfsys@useobject{currentmarker}{}%
\end{pgfscope}%
\begin{pgfscope}%
\pgfsys@transformshift{2.225473in}{0.550978in}%
\pgfsys@useobject{currentmarker}{}%
\end{pgfscope}%
\begin{pgfscope}%
\pgfsys@transformshift{2.247923in}{0.531960in}%
\pgfsys@useobject{currentmarker}{}%
\end{pgfscope}%
\begin{pgfscope}%
\pgfsys@transformshift{2.269807in}{0.511911in}%
\pgfsys@useobject{currentmarker}{}%
\end{pgfscope}%
\begin{pgfscope}%
\pgfsys@transformshift{2.310165in}{0.525497in}%
\pgfsys@useobject{currentmarker}{}%
\end{pgfscope}%
\begin{pgfscope}%
\pgfsys@transformshift{2.369172in}{0.573036in}%
\pgfsys@useobject{currentmarker}{}%
\end{pgfscope}%
\begin{pgfscope}%
\pgfsys@transformshift{2.433038in}{0.629423in}%
\pgfsys@useobject{currentmarker}{}%
\end{pgfscope}%
\begin{pgfscope}%
\pgfsys@transformshift{2.512551in}{0.714298in}%
\pgfsys@useobject{currentmarker}{}%
\end{pgfscope}%
\begin{pgfscope}%
\pgfsys@transformshift{2.584448in}{0.785307in}%
\pgfsys@useobject{currentmarker}{}%
\end{pgfscope}%
\begin{pgfscope}%
\pgfsys@transformshift{2.654396in}{0.852767in}%
\pgfsys@useobject{currentmarker}{}%
\end{pgfscope}%
\begin{pgfscope}%
\pgfsys@transformshift{2.709975in}{0.894066in}%
\pgfsys@useobject{currentmarker}{}%
\end{pgfscope}%
\begin{pgfscope}%
\pgfsys@transformshift{2.745485in}{0.898825in}%
\pgfsys@useobject{currentmarker}{}%
\end{pgfscope}%
\begin{pgfscope}%
\pgfsys@transformshift{2.765939in}{0.876173in}%
\pgfsys@useobject{currentmarker}{}%
\end{pgfscope}%
\begin{pgfscope}%
\pgfsys@transformshift{2.777619in}{0.837544in}%
\pgfsys@useobject{currentmarker}{}%
\end{pgfscope}%
\begin{pgfscope}%
\pgfsys@transformshift{2.773219in}{0.769640in}%
\pgfsys@useobject{currentmarker}{}%
\end{pgfscope}%
\begin{pgfscope}%
\pgfsys@transformshift{2.773093in}{0.709516in}%
\pgfsys@useobject{currentmarker}{}%
\end{pgfscope}%
\begin{pgfscope}%
\pgfsys@transformshift{2.775290in}{0.653624in}%
\pgfsys@useobject{currentmarker}{}%
\end{pgfscope}%
\begin{pgfscope}%
\pgfsys@transformshift{2.792438in}{0.624951in}%
\pgfsys@useobject{currentmarker}{}%
\end{pgfscope}%
\begin{pgfscope}%
\pgfsys@transformshift{2.801363in}{0.581308in}%
\pgfsys@useobject{currentmarker}{}%
\end{pgfscope}%
\begin{pgfscope}%
\pgfsys@transformshift{2.816359in}{0.548717in}%
\pgfsys@useobject{currentmarker}{}%
\end{pgfscope}%
\begin{pgfscope}%
\pgfsys@transformshift{2.831119in}{0.515697in}%
\pgfsys@useobject{currentmarker}{}%
\end{pgfscope}%
\begin{pgfscope}%
\pgfsys@transformshift{2.870334in}{0.527203in}%
\pgfsys@useobject{currentmarker}{}%
\end{pgfscope}%
\begin{pgfscope}%
\pgfsys@transformshift{2.914301in}{0.547359in}%
\pgfsys@useobject{currentmarker}{}%
\end{pgfscope}%
\begin{pgfscope}%
\pgfsys@transformshift{2.968424in}{0.586007in}%
\pgfsys@useobject{currentmarker}{}%
\end{pgfscope}%
\begin{pgfscope}%
\pgfsys@transformshift{3.031090in}{0.640210in}%
\pgfsys@useobject{currentmarker}{}%
\end{pgfscope}%
\begin{pgfscope}%
\pgfsys@transformshift{3.108456in}{0.721175in}%
\pgfsys@useobject{currentmarker}{}%
\end{pgfscope}%
\begin{pgfscope}%
\pgfsys@transformshift{3.183879in}{0.798603in}%
\pgfsys@useobject{currentmarker}{}%
\end{pgfscope}%
\begin{pgfscope}%
\pgfsys@transformshift{3.274791in}{0.904233in}%
\pgfsys@useobject{currentmarker}{}%
\end{pgfscope}%
\begin{pgfscope}%
\pgfsys@transformshift{3.351275in}{0.983593in}%
\pgfsys@useobject{currentmarker}{}%
\end{pgfscope}%
\begin{pgfscope}%
\pgfsys@transformshift{3.402422in}{1.016823in}%
\pgfsys@useobject{currentmarker}{}%
\end{pgfscope}%
\begin{pgfscope}%
\pgfsys@transformshift{3.426281in}{1.000369in}%
\pgfsys@useobject{currentmarker}{}%
\end{pgfscope}%
\begin{pgfscope}%
\pgfsys@transformshift{3.431078in}{0.949209in}%
\pgfsys@useobject{currentmarker}{}%
\end{pgfscope}%
\begin{pgfscope}%
\pgfsys@transformshift{3.439962in}{0.905491in}%
\pgfsys@useobject{currentmarker}{}%
\end{pgfscope}%
\begin{pgfscope}%
\pgfsys@transformshift{3.446892in}{0.858216in}%
\pgfsys@useobject{currentmarker}{}%
\end{pgfscope}%
\begin{pgfscope}%
\pgfsys@transformshift{3.456265in}{0.815386in}%
\pgfsys@useobject{currentmarker}{}%
\end{pgfscope}%
\begin{pgfscope}%
\pgfsys@transformshift{3.455917in}{0.754860in}%
\pgfsys@useobject{currentmarker}{}%
\end{pgfscope}%
\begin{pgfscope}%
\pgfsys@transformshift{3.471332in}{0.723033in}%
\pgfsys@useobject{currentmarker}{}%
\end{pgfscope}%
\begin{pgfscope}%
\pgfsys@transformshift{3.492791in}{0.702209in}%
\pgfsys@useobject{currentmarker}{}%
\end{pgfscope}%
\begin{pgfscope}%
\pgfsys@transformshift{3.581754in}{0.684504in}%
\pgfsys@useobject{currentmarker}{}%
\end{pgfscope}%
\begin{pgfscope}%
\pgfsys@transformshift{3.619118in}{0.692639in}%
\pgfsys@useobject{currentmarker}{}%
\end{pgfscope}%
\begin{pgfscope}%
\pgfsys@transformshift{3.681794in}{0.746858in}%
\pgfsys@useobject{currentmarker}{}%
\end{pgfscope}%
\end{pgfscope}%
\begin{pgfscope}%
\pgfpathrectangle{\pgfqpoint{0.552773in}{0.431673in}}{\pgfqpoint{3.738807in}{1.765244in}}%
\pgfusepath{clip}%
\pgfsetbuttcap%
\pgfsetroundjoin%
\definecolor{currentfill}{rgb}{0.667253,0.779176,0.992959}%
\pgfsetfillcolor{currentfill}%
\pgfsetlinewidth{1.003750pt}%
\definecolor{currentstroke}{rgb}{0.667253,0.779176,0.992959}%
\pgfsetstrokecolor{currentstroke}%
\pgfsetdash{}{0pt}%
\pgfsys@defobject{currentmarker}{\pgfqpoint{-0.027778in}{-0.000000in}}{\pgfqpoint{0.027778in}{0.000000in}}{%
\pgfpathmoveto{\pgfqpoint{0.027778in}{-0.000000in}}%
\pgfpathlineto{\pgfqpoint{-0.027778in}{0.000000in}}%
\pgfusepath{stroke,fill}%
}%
\begin{pgfscope}%
\pgfsys@transformshift{0.946281in}{0.917169in}%
\pgfsys@useobject{currentmarker}{}%
\end{pgfscope}%
\begin{pgfscope}%
\pgfsys@transformshift{0.968004in}{0.896826in}%
\pgfsys@useobject{currentmarker}{}%
\end{pgfscope}%
\begin{pgfscope}%
\pgfsys@transformshift{0.983367in}{0.864904in}%
\pgfsys@useobject{currentmarker}{}%
\end{pgfscope}%
\begin{pgfscope}%
\pgfsys@transformshift{0.997288in}{0.830357in}%
\pgfsys@useobject{currentmarker}{}%
\end{pgfscope}%
\begin{pgfscope}%
\pgfsys@transformshift{1.003998in}{0.782680in}%
\pgfsys@useobject{currentmarker}{}%
\end{pgfscope}%
\begin{pgfscope}%
\pgfsys@transformshift{1.002620in}{0.720278in}%
\pgfsys@useobject{currentmarker}{}%
\end{pgfscope}%
\begin{pgfscope}%
\pgfsys@transformshift{1.016428in}{0.685524in}%
\pgfsys@useobject{currentmarker}{}%
\end{pgfscope}%
\begin{pgfscope}%
\pgfsys@transformshift{1.016366in}{0.625519in}%
\pgfsys@useobject{currentmarker}{}%
\end{pgfscope}%
\begin{pgfscope}%
\pgfsys@transformshift{1.044197in}{0.616297in}%
\pgfsys@useobject{currentmarker}{}%
\end{pgfscope}%
\begin{pgfscope}%
\pgfsys@transformshift{1.061881in}{0.588600in}%
\pgfsys@useobject{currentmarker}{}%
\end{pgfscope}%
\begin{pgfscope}%
\pgfsys@transformshift{1.100011in}{0.598129in}%
\pgfsys@useobject{currentmarker}{}%
\end{pgfscope}%
\begin{pgfscope}%
\pgfsys@transformshift{1.133830in}{0.599811in}%
\pgfsys@useobject{currentmarker}{}%
\end{pgfscope}%
\begin{pgfscope}%
\pgfsys@transformshift{1.178595in}{0.621420in}%
\pgfsys@useobject{currentmarker}{}%
\end{pgfscope}%
\begin{pgfscope}%
\pgfsys@transformshift{1.241991in}{0.676951in}%
\pgfsys@useobject{currentmarker}{}%
\end{pgfscope}%
\begin{pgfscope}%
\pgfsys@transformshift{1.316482in}{0.752683in}%
\pgfsys@useobject{currentmarker}{}%
\end{pgfscope}%
\begin{pgfscope}%
\pgfsys@transformshift{1.386975in}{0.821134in}%
\pgfsys@useobject{currentmarker}{}%
\end{pgfscope}%
\begin{pgfscope}%
\pgfsys@transformshift{1.458963in}{0.892310in}%
\pgfsys@useobject{currentmarker}{}%
\end{pgfscope}%
\begin{pgfscope}%
\pgfsys@transformshift{1.515751in}{0.935810in}%
\pgfsys@useobject{currentmarker}{}%
\end{pgfscope}%
\begin{pgfscope}%
\pgfsys@transformshift{1.552230in}{0.942332in}%
\pgfsys@useobject{currentmarker}{}%
\end{pgfscope}%
\begin{pgfscope}%
\pgfsys@transformshift{1.571226in}{0.917025in}%
\pgfsys@useobject{currentmarker}{}%
\end{pgfscope}%
\begin{pgfscope}%
\pgfsys@transformshift{1.578686in}{0.870715in}%
\pgfsys@useobject{currentmarker}{}%
\end{pgfscope}%
\begin{pgfscope}%
\pgfsys@transformshift{1.598839in}{0.847513in}%
\pgfsys@useobject{currentmarker}{}%
\end{pgfscope}%
\begin{pgfscope}%
\pgfsys@transformshift{1.615264in}{0.817524in}%
\pgfsys@useobject{currentmarker}{}%
\end{pgfscope}%
\begin{pgfscope}%
\pgfsys@transformshift{1.629986in}{0.784436in}%
\pgfsys@useobject{currentmarker}{}%
\end{pgfscope}%
\begin{pgfscope}%
\pgfsys@transformshift{1.633661in}{0.731234in}%
\pgfsys@useobject{currentmarker}{}%
\end{pgfscope}%
\begin{pgfscope}%
\pgfsys@transformshift{1.644849in}{0.691711in}%
\pgfsys@useobject{currentmarker}{}%
\end{pgfscope}%
\begin{pgfscope}%
\pgfsys@transformshift{1.654597in}{0.649565in}%
\pgfsys@useobject{currentmarker}{}%
\end{pgfscope}%
\begin{pgfscope}%
\pgfsys@transformshift{1.665591in}{0.609687in}%
\pgfsys@useobject{currentmarker}{}%
\end{pgfscope}%
\begin{pgfscope}%
\pgfsys@transformshift{1.686427in}{0.587731in}%
\pgfsys@useobject{currentmarker}{}%
\end{pgfscope}%
\begin{pgfscope}%
\pgfsys@transformshift{1.711171in}{0.572888in}%
\pgfsys@useobject{currentmarker}{}%
\end{pgfscope}%
\begin{pgfscope}%
\pgfsys@transformshift{1.758001in}{0.598259in}%
\pgfsys@useobject{currentmarker}{}%
\end{pgfscope}%
\begin{pgfscope}%
\pgfsys@transformshift{1.816028in}{0.644014in}%
\pgfsys@useobject{currentmarker}{}%
\end{pgfscope}%
\begin{pgfscope}%
\pgfsys@transformshift{1.879012in}{0.698794in}%
\pgfsys@useobject{currentmarker}{}%
\end{pgfscope}%
\begin{pgfscope}%
\pgfsys@transformshift{1.944926in}{0.758909in}%
\pgfsys@useobject{currentmarker}{}%
\end{pgfscope}%
\begin{pgfscope}%
\pgfsys@transformshift{2.017564in}{0.831267in}%
\pgfsys@useobject{currentmarker}{}%
\end{pgfscope}%
\begin{pgfscope}%
\pgfsys@transformshift{2.077394in}{0.880307in}%
\pgfsys@useobject{currentmarker}{}%
\end{pgfscope}%
\begin{pgfscope}%
\pgfsys@transformshift{2.120452in}{0.898808in}%
\pgfsys@useobject{currentmarker}{}%
\end{pgfscope}%
\begin{pgfscope}%
\pgfsys@transformshift{2.148843in}{0.890606in}%
\pgfsys@useobject{currentmarker}{}%
\end{pgfscope}%
\begin{pgfscope}%
\pgfsys@transformshift{2.169787in}{0.868845in}%
\pgfsys@useobject{currentmarker}{}%
\end{pgfscope}%
\begin{pgfscope}%
\pgfsys@transformshift{2.191999in}{0.849394in}%
\pgfsys@useobject{currentmarker}{}%
\end{pgfscope}%
\begin{pgfscope}%
\pgfsys@transformshift{2.201330in}{0.806489in}%
\pgfsys@useobject{currentmarker}{}%
\end{pgfscope}%
\begin{pgfscope}%
\pgfsys@transformshift{2.202600in}{0.748907in}%
\pgfsys@useobject{currentmarker}{}%
\end{pgfscope}%
\begin{pgfscope}%
\pgfsys@transformshift{2.202588in}{0.688992in}%
\pgfsys@useobject{currentmarker}{}%
\end{pgfscope}%
\begin{pgfscope}%
\pgfsys@transformshift{2.206526in}{0.636268in}%
\pgfsys@useobject{currentmarker}{}%
\end{pgfscope}%
\begin{pgfscope}%
\pgfsys@transformshift{2.205818in}{0.575086in}%
\pgfsys@useobject{currentmarker}{}%
\end{pgfscope}%
\begin{pgfscope}%
\pgfsys@transformshift{2.225473in}{0.550978in}%
\pgfsys@useobject{currentmarker}{}%
\end{pgfscope}%
\begin{pgfscope}%
\pgfsys@transformshift{2.247923in}{0.531960in}%
\pgfsys@useobject{currentmarker}{}%
\end{pgfscope}%
\begin{pgfscope}%
\pgfsys@transformshift{2.269807in}{0.511911in}%
\pgfsys@useobject{currentmarker}{}%
\end{pgfscope}%
\begin{pgfscope}%
\pgfsys@transformshift{2.310165in}{0.525497in}%
\pgfsys@useobject{currentmarker}{}%
\end{pgfscope}%
\begin{pgfscope}%
\pgfsys@transformshift{2.369172in}{0.573036in}%
\pgfsys@useobject{currentmarker}{}%
\end{pgfscope}%
\begin{pgfscope}%
\pgfsys@transformshift{2.433038in}{0.629423in}%
\pgfsys@useobject{currentmarker}{}%
\end{pgfscope}%
\begin{pgfscope}%
\pgfsys@transformshift{2.512551in}{0.714298in}%
\pgfsys@useobject{currentmarker}{}%
\end{pgfscope}%
\begin{pgfscope}%
\pgfsys@transformshift{2.584448in}{0.785307in}%
\pgfsys@useobject{currentmarker}{}%
\end{pgfscope}%
\begin{pgfscope}%
\pgfsys@transformshift{2.654396in}{0.852767in}%
\pgfsys@useobject{currentmarker}{}%
\end{pgfscope}%
\begin{pgfscope}%
\pgfsys@transformshift{2.709975in}{0.894066in}%
\pgfsys@useobject{currentmarker}{}%
\end{pgfscope}%
\begin{pgfscope}%
\pgfsys@transformshift{2.745485in}{0.898825in}%
\pgfsys@useobject{currentmarker}{}%
\end{pgfscope}%
\begin{pgfscope}%
\pgfsys@transformshift{2.765939in}{0.876173in}%
\pgfsys@useobject{currentmarker}{}%
\end{pgfscope}%
\begin{pgfscope}%
\pgfsys@transformshift{2.777619in}{0.837544in}%
\pgfsys@useobject{currentmarker}{}%
\end{pgfscope}%
\begin{pgfscope}%
\pgfsys@transformshift{2.773219in}{0.769640in}%
\pgfsys@useobject{currentmarker}{}%
\end{pgfscope}%
\begin{pgfscope}%
\pgfsys@transformshift{2.773093in}{0.709516in}%
\pgfsys@useobject{currentmarker}{}%
\end{pgfscope}%
\begin{pgfscope}%
\pgfsys@transformshift{2.775290in}{0.653624in}%
\pgfsys@useobject{currentmarker}{}%
\end{pgfscope}%
\begin{pgfscope}%
\pgfsys@transformshift{2.792438in}{0.624951in}%
\pgfsys@useobject{currentmarker}{}%
\end{pgfscope}%
\begin{pgfscope}%
\pgfsys@transformshift{2.801363in}{0.581308in}%
\pgfsys@useobject{currentmarker}{}%
\end{pgfscope}%
\begin{pgfscope}%
\pgfsys@transformshift{2.816359in}{0.548717in}%
\pgfsys@useobject{currentmarker}{}%
\end{pgfscope}%
\begin{pgfscope}%
\pgfsys@transformshift{2.831119in}{0.515697in}%
\pgfsys@useobject{currentmarker}{}%
\end{pgfscope}%
\begin{pgfscope}%
\pgfsys@transformshift{2.870334in}{0.527203in}%
\pgfsys@useobject{currentmarker}{}%
\end{pgfscope}%
\begin{pgfscope}%
\pgfsys@transformshift{2.914301in}{0.547359in}%
\pgfsys@useobject{currentmarker}{}%
\end{pgfscope}%
\begin{pgfscope}%
\pgfsys@transformshift{2.968424in}{0.586007in}%
\pgfsys@useobject{currentmarker}{}%
\end{pgfscope}%
\begin{pgfscope}%
\pgfsys@transformshift{3.031090in}{0.640210in}%
\pgfsys@useobject{currentmarker}{}%
\end{pgfscope}%
\begin{pgfscope}%
\pgfsys@transformshift{3.108456in}{0.721175in}%
\pgfsys@useobject{currentmarker}{}%
\end{pgfscope}%
\begin{pgfscope}%
\pgfsys@transformshift{3.183879in}{0.798603in}%
\pgfsys@useobject{currentmarker}{}%
\end{pgfscope}%
\begin{pgfscope}%
\pgfsys@transformshift{3.274791in}{0.904233in}%
\pgfsys@useobject{currentmarker}{}%
\end{pgfscope}%
\begin{pgfscope}%
\pgfsys@transformshift{3.351275in}{0.983593in}%
\pgfsys@useobject{currentmarker}{}%
\end{pgfscope}%
\begin{pgfscope}%
\pgfsys@transformshift{3.402422in}{1.016823in}%
\pgfsys@useobject{currentmarker}{}%
\end{pgfscope}%
\begin{pgfscope}%
\pgfsys@transformshift{3.426281in}{1.000369in}%
\pgfsys@useobject{currentmarker}{}%
\end{pgfscope}%
\begin{pgfscope}%
\pgfsys@transformshift{3.431078in}{0.949209in}%
\pgfsys@useobject{currentmarker}{}%
\end{pgfscope}%
\begin{pgfscope}%
\pgfsys@transformshift{3.439962in}{0.905491in}%
\pgfsys@useobject{currentmarker}{}%
\end{pgfscope}%
\begin{pgfscope}%
\pgfsys@transformshift{3.446892in}{0.858216in}%
\pgfsys@useobject{currentmarker}{}%
\end{pgfscope}%
\begin{pgfscope}%
\pgfsys@transformshift{3.456265in}{0.815386in}%
\pgfsys@useobject{currentmarker}{}%
\end{pgfscope}%
\begin{pgfscope}%
\pgfsys@transformshift{3.455917in}{0.754860in}%
\pgfsys@useobject{currentmarker}{}%
\end{pgfscope}%
\begin{pgfscope}%
\pgfsys@transformshift{3.471332in}{0.723033in}%
\pgfsys@useobject{currentmarker}{}%
\end{pgfscope}%
\begin{pgfscope}%
\pgfsys@transformshift{3.492791in}{0.702209in}%
\pgfsys@useobject{currentmarker}{}%
\end{pgfscope}%
\begin{pgfscope}%
\pgfsys@transformshift{3.581754in}{0.684504in}%
\pgfsys@useobject{currentmarker}{}%
\end{pgfscope}%
\begin{pgfscope}%
\pgfsys@transformshift{3.619118in}{0.692639in}%
\pgfsys@useobject{currentmarker}{}%
\end{pgfscope}%
\begin{pgfscope}%
\pgfsys@transformshift{3.681794in}{0.746858in}%
\pgfsys@useobject{currentmarker}{}%
\end{pgfscope}%
\end{pgfscope}%
\begin{pgfscope}%
\pgfpathrectangle{\pgfqpoint{0.552773in}{0.431673in}}{\pgfqpoint{3.738807in}{1.765244in}}%
\pgfusepath{clip}%
\pgfsetbuttcap%
\pgfsetroundjoin%
\definecolor{currentfill}{rgb}{0.667253,0.779176,0.992959}%
\pgfsetfillcolor{currentfill}%
\pgfsetlinewidth{1.003750pt}%
\definecolor{currentstroke}{rgb}{0.667253,0.779176,0.992959}%
\pgfsetstrokecolor{currentstroke}%
\pgfsetdash{}{0pt}%
\pgfsys@defobject{currentmarker}{\pgfqpoint{-0.027778in}{-0.000000in}}{\pgfqpoint{0.027778in}{0.000000in}}{%
\pgfpathmoveto{\pgfqpoint{0.027778in}{-0.000000in}}%
\pgfpathlineto{\pgfqpoint{-0.027778in}{0.000000in}}%
\pgfusepath{stroke,fill}%
}%
\begin{pgfscope}%
\pgfsys@transformshift{0.946281in}{0.917169in}%
\pgfsys@useobject{currentmarker}{}%
\end{pgfscope}%
\begin{pgfscope}%
\pgfsys@transformshift{0.968004in}{0.896826in}%
\pgfsys@useobject{currentmarker}{}%
\end{pgfscope}%
\begin{pgfscope}%
\pgfsys@transformshift{0.983367in}{0.864904in}%
\pgfsys@useobject{currentmarker}{}%
\end{pgfscope}%
\begin{pgfscope}%
\pgfsys@transformshift{0.997288in}{0.830357in}%
\pgfsys@useobject{currentmarker}{}%
\end{pgfscope}%
\begin{pgfscope}%
\pgfsys@transformshift{1.003998in}{0.782680in}%
\pgfsys@useobject{currentmarker}{}%
\end{pgfscope}%
\begin{pgfscope}%
\pgfsys@transformshift{1.002620in}{0.720278in}%
\pgfsys@useobject{currentmarker}{}%
\end{pgfscope}%
\begin{pgfscope}%
\pgfsys@transformshift{1.016428in}{0.685524in}%
\pgfsys@useobject{currentmarker}{}%
\end{pgfscope}%
\begin{pgfscope}%
\pgfsys@transformshift{1.016366in}{0.625519in}%
\pgfsys@useobject{currentmarker}{}%
\end{pgfscope}%
\begin{pgfscope}%
\pgfsys@transformshift{1.044197in}{0.616297in}%
\pgfsys@useobject{currentmarker}{}%
\end{pgfscope}%
\begin{pgfscope}%
\pgfsys@transformshift{1.061881in}{0.588600in}%
\pgfsys@useobject{currentmarker}{}%
\end{pgfscope}%
\begin{pgfscope}%
\pgfsys@transformshift{1.100011in}{0.598129in}%
\pgfsys@useobject{currentmarker}{}%
\end{pgfscope}%
\begin{pgfscope}%
\pgfsys@transformshift{1.133830in}{0.599811in}%
\pgfsys@useobject{currentmarker}{}%
\end{pgfscope}%
\begin{pgfscope}%
\pgfsys@transformshift{1.178595in}{0.621420in}%
\pgfsys@useobject{currentmarker}{}%
\end{pgfscope}%
\begin{pgfscope}%
\pgfsys@transformshift{1.241991in}{0.676951in}%
\pgfsys@useobject{currentmarker}{}%
\end{pgfscope}%
\begin{pgfscope}%
\pgfsys@transformshift{1.316482in}{0.752683in}%
\pgfsys@useobject{currentmarker}{}%
\end{pgfscope}%
\begin{pgfscope}%
\pgfsys@transformshift{1.386975in}{0.821134in}%
\pgfsys@useobject{currentmarker}{}%
\end{pgfscope}%
\begin{pgfscope}%
\pgfsys@transformshift{1.458963in}{0.892310in}%
\pgfsys@useobject{currentmarker}{}%
\end{pgfscope}%
\begin{pgfscope}%
\pgfsys@transformshift{1.515751in}{0.935810in}%
\pgfsys@useobject{currentmarker}{}%
\end{pgfscope}%
\begin{pgfscope}%
\pgfsys@transformshift{1.552230in}{0.942332in}%
\pgfsys@useobject{currentmarker}{}%
\end{pgfscope}%
\begin{pgfscope}%
\pgfsys@transformshift{1.571226in}{0.917025in}%
\pgfsys@useobject{currentmarker}{}%
\end{pgfscope}%
\begin{pgfscope}%
\pgfsys@transformshift{1.578686in}{0.870715in}%
\pgfsys@useobject{currentmarker}{}%
\end{pgfscope}%
\begin{pgfscope}%
\pgfsys@transformshift{1.598839in}{0.847513in}%
\pgfsys@useobject{currentmarker}{}%
\end{pgfscope}%
\begin{pgfscope}%
\pgfsys@transformshift{1.615264in}{0.817524in}%
\pgfsys@useobject{currentmarker}{}%
\end{pgfscope}%
\begin{pgfscope}%
\pgfsys@transformshift{1.629986in}{0.784436in}%
\pgfsys@useobject{currentmarker}{}%
\end{pgfscope}%
\begin{pgfscope}%
\pgfsys@transformshift{1.633661in}{0.731234in}%
\pgfsys@useobject{currentmarker}{}%
\end{pgfscope}%
\begin{pgfscope}%
\pgfsys@transformshift{1.644849in}{0.691711in}%
\pgfsys@useobject{currentmarker}{}%
\end{pgfscope}%
\begin{pgfscope}%
\pgfsys@transformshift{1.654597in}{0.649565in}%
\pgfsys@useobject{currentmarker}{}%
\end{pgfscope}%
\begin{pgfscope}%
\pgfsys@transformshift{1.665591in}{0.609687in}%
\pgfsys@useobject{currentmarker}{}%
\end{pgfscope}%
\begin{pgfscope}%
\pgfsys@transformshift{1.686427in}{0.587731in}%
\pgfsys@useobject{currentmarker}{}%
\end{pgfscope}%
\begin{pgfscope}%
\pgfsys@transformshift{1.711171in}{0.572888in}%
\pgfsys@useobject{currentmarker}{}%
\end{pgfscope}%
\begin{pgfscope}%
\pgfsys@transformshift{1.758001in}{0.598259in}%
\pgfsys@useobject{currentmarker}{}%
\end{pgfscope}%
\begin{pgfscope}%
\pgfsys@transformshift{1.816028in}{0.644014in}%
\pgfsys@useobject{currentmarker}{}%
\end{pgfscope}%
\begin{pgfscope}%
\pgfsys@transformshift{1.879012in}{0.698794in}%
\pgfsys@useobject{currentmarker}{}%
\end{pgfscope}%
\begin{pgfscope}%
\pgfsys@transformshift{1.944926in}{0.758909in}%
\pgfsys@useobject{currentmarker}{}%
\end{pgfscope}%
\begin{pgfscope}%
\pgfsys@transformshift{2.017564in}{0.831267in}%
\pgfsys@useobject{currentmarker}{}%
\end{pgfscope}%
\begin{pgfscope}%
\pgfsys@transformshift{2.077394in}{0.880307in}%
\pgfsys@useobject{currentmarker}{}%
\end{pgfscope}%
\begin{pgfscope}%
\pgfsys@transformshift{2.120452in}{0.898808in}%
\pgfsys@useobject{currentmarker}{}%
\end{pgfscope}%
\begin{pgfscope}%
\pgfsys@transformshift{2.148843in}{0.890606in}%
\pgfsys@useobject{currentmarker}{}%
\end{pgfscope}%
\begin{pgfscope}%
\pgfsys@transformshift{2.169787in}{0.868845in}%
\pgfsys@useobject{currentmarker}{}%
\end{pgfscope}%
\begin{pgfscope}%
\pgfsys@transformshift{2.191999in}{0.849394in}%
\pgfsys@useobject{currentmarker}{}%
\end{pgfscope}%
\begin{pgfscope}%
\pgfsys@transformshift{2.201330in}{0.806489in}%
\pgfsys@useobject{currentmarker}{}%
\end{pgfscope}%
\begin{pgfscope}%
\pgfsys@transformshift{2.202600in}{0.748907in}%
\pgfsys@useobject{currentmarker}{}%
\end{pgfscope}%
\begin{pgfscope}%
\pgfsys@transformshift{2.202588in}{0.688992in}%
\pgfsys@useobject{currentmarker}{}%
\end{pgfscope}%
\begin{pgfscope}%
\pgfsys@transformshift{2.206526in}{0.636268in}%
\pgfsys@useobject{currentmarker}{}%
\end{pgfscope}%
\begin{pgfscope}%
\pgfsys@transformshift{2.205818in}{0.575086in}%
\pgfsys@useobject{currentmarker}{}%
\end{pgfscope}%
\begin{pgfscope}%
\pgfsys@transformshift{2.225473in}{0.550978in}%
\pgfsys@useobject{currentmarker}{}%
\end{pgfscope}%
\begin{pgfscope}%
\pgfsys@transformshift{2.247923in}{0.531960in}%
\pgfsys@useobject{currentmarker}{}%
\end{pgfscope}%
\begin{pgfscope}%
\pgfsys@transformshift{2.269807in}{0.511911in}%
\pgfsys@useobject{currentmarker}{}%
\end{pgfscope}%
\begin{pgfscope}%
\pgfsys@transformshift{2.310165in}{0.525497in}%
\pgfsys@useobject{currentmarker}{}%
\end{pgfscope}%
\begin{pgfscope}%
\pgfsys@transformshift{2.369172in}{0.573036in}%
\pgfsys@useobject{currentmarker}{}%
\end{pgfscope}%
\begin{pgfscope}%
\pgfsys@transformshift{2.433038in}{0.629423in}%
\pgfsys@useobject{currentmarker}{}%
\end{pgfscope}%
\begin{pgfscope}%
\pgfsys@transformshift{2.512551in}{0.714298in}%
\pgfsys@useobject{currentmarker}{}%
\end{pgfscope}%
\begin{pgfscope}%
\pgfsys@transformshift{2.584448in}{0.785307in}%
\pgfsys@useobject{currentmarker}{}%
\end{pgfscope}%
\begin{pgfscope}%
\pgfsys@transformshift{2.654396in}{0.852767in}%
\pgfsys@useobject{currentmarker}{}%
\end{pgfscope}%
\begin{pgfscope}%
\pgfsys@transformshift{2.709975in}{0.894066in}%
\pgfsys@useobject{currentmarker}{}%
\end{pgfscope}%
\begin{pgfscope}%
\pgfsys@transformshift{2.745485in}{0.898825in}%
\pgfsys@useobject{currentmarker}{}%
\end{pgfscope}%
\begin{pgfscope}%
\pgfsys@transformshift{2.765939in}{0.876173in}%
\pgfsys@useobject{currentmarker}{}%
\end{pgfscope}%
\begin{pgfscope}%
\pgfsys@transformshift{2.777619in}{0.837544in}%
\pgfsys@useobject{currentmarker}{}%
\end{pgfscope}%
\begin{pgfscope}%
\pgfsys@transformshift{2.773219in}{0.769640in}%
\pgfsys@useobject{currentmarker}{}%
\end{pgfscope}%
\begin{pgfscope}%
\pgfsys@transformshift{2.773093in}{0.709516in}%
\pgfsys@useobject{currentmarker}{}%
\end{pgfscope}%
\begin{pgfscope}%
\pgfsys@transformshift{2.775290in}{0.653624in}%
\pgfsys@useobject{currentmarker}{}%
\end{pgfscope}%
\begin{pgfscope}%
\pgfsys@transformshift{2.792438in}{0.624951in}%
\pgfsys@useobject{currentmarker}{}%
\end{pgfscope}%
\begin{pgfscope}%
\pgfsys@transformshift{2.801363in}{0.581308in}%
\pgfsys@useobject{currentmarker}{}%
\end{pgfscope}%
\begin{pgfscope}%
\pgfsys@transformshift{2.816359in}{0.548717in}%
\pgfsys@useobject{currentmarker}{}%
\end{pgfscope}%
\begin{pgfscope}%
\pgfsys@transformshift{2.831119in}{0.515697in}%
\pgfsys@useobject{currentmarker}{}%
\end{pgfscope}%
\begin{pgfscope}%
\pgfsys@transformshift{2.870334in}{0.527203in}%
\pgfsys@useobject{currentmarker}{}%
\end{pgfscope}%
\begin{pgfscope}%
\pgfsys@transformshift{2.914301in}{0.547359in}%
\pgfsys@useobject{currentmarker}{}%
\end{pgfscope}%
\begin{pgfscope}%
\pgfsys@transformshift{2.968424in}{0.586007in}%
\pgfsys@useobject{currentmarker}{}%
\end{pgfscope}%
\begin{pgfscope}%
\pgfsys@transformshift{3.031090in}{0.640210in}%
\pgfsys@useobject{currentmarker}{}%
\end{pgfscope}%
\begin{pgfscope}%
\pgfsys@transformshift{3.108456in}{0.721175in}%
\pgfsys@useobject{currentmarker}{}%
\end{pgfscope}%
\begin{pgfscope}%
\pgfsys@transformshift{3.183879in}{0.798603in}%
\pgfsys@useobject{currentmarker}{}%
\end{pgfscope}%
\begin{pgfscope}%
\pgfsys@transformshift{3.274791in}{0.904233in}%
\pgfsys@useobject{currentmarker}{}%
\end{pgfscope}%
\begin{pgfscope}%
\pgfsys@transformshift{3.351275in}{0.983593in}%
\pgfsys@useobject{currentmarker}{}%
\end{pgfscope}%
\begin{pgfscope}%
\pgfsys@transformshift{3.402422in}{1.016823in}%
\pgfsys@useobject{currentmarker}{}%
\end{pgfscope}%
\begin{pgfscope}%
\pgfsys@transformshift{3.426281in}{1.000369in}%
\pgfsys@useobject{currentmarker}{}%
\end{pgfscope}%
\begin{pgfscope}%
\pgfsys@transformshift{3.431078in}{0.949209in}%
\pgfsys@useobject{currentmarker}{}%
\end{pgfscope}%
\begin{pgfscope}%
\pgfsys@transformshift{3.439962in}{0.905491in}%
\pgfsys@useobject{currentmarker}{}%
\end{pgfscope}%
\begin{pgfscope}%
\pgfsys@transformshift{3.446892in}{0.858216in}%
\pgfsys@useobject{currentmarker}{}%
\end{pgfscope}%
\begin{pgfscope}%
\pgfsys@transformshift{3.456265in}{0.815386in}%
\pgfsys@useobject{currentmarker}{}%
\end{pgfscope}%
\begin{pgfscope}%
\pgfsys@transformshift{3.455917in}{0.754860in}%
\pgfsys@useobject{currentmarker}{}%
\end{pgfscope}%
\begin{pgfscope}%
\pgfsys@transformshift{3.471332in}{0.723033in}%
\pgfsys@useobject{currentmarker}{}%
\end{pgfscope}%
\begin{pgfscope}%
\pgfsys@transformshift{3.492791in}{0.702209in}%
\pgfsys@useobject{currentmarker}{}%
\end{pgfscope}%
\begin{pgfscope}%
\pgfsys@transformshift{3.581754in}{0.684504in}%
\pgfsys@useobject{currentmarker}{}%
\end{pgfscope}%
\begin{pgfscope}%
\pgfsys@transformshift{3.619118in}{0.692639in}%
\pgfsys@useobject{currentmarker}{}%
\end{pgfscope}%
\begin{pgfscope}%
\pgfsys@transformshift{3.681794in}{0.746858in}%
\pgfsys@useobject{currentmarker}{}%
\end{pgfscope}%
\end{pgfscope}%
\begin{pgfscope}%
\pgfpathrectangle{\pgfqpoint{0.552773in}{0.431673in}}{\pgfqpoint{3.738807in}{1.765244in}}%
\pgfusepath{clip}%
\pgfsetbuttcap%
\pgfsetroundjoin%
\pgfsetlinewidth{1.003750pt}%
\definecolor{currentstroke}{rgb}{0.667253,0.779176,0.992959}%
\pgfsetstrokecolor{currentstroke}%
\pgfsetstrokeopacity{0.500000}%
\pgfsetdash{{3.700000pt}{1.600000pt}}{0.000000pt}%
\pgfpathmoveto{\pgfqpoint{0.552773in}{0.740787in}}%
\pgfpathlineto{\pgfqpoint{4.291580in}{0.740787in}}%
\pgfusepath{stroke}%
\end{pgfscope}%
\begin{pgfscope}%
\pgfpathrectangle{\pgfqpoint{0.552773in}{0.431673in}}{\pgfqpoint{3.738807in}{1.765244in}}%
\pgfusepath{clip}%
\pgfsetbuttcap%
\pgfsetroundjoin%
\pgfsetlinewidth{1.003750pt}%
\definecolor{currentstroke}{rgb}{0.968203,0.720844,0.612293}%
\pgfsetstrokecolor{currentstroke}%
\pgfsetdash{}{0pt}%
\pgfpathmoveto{\pgfqpoint{1.234882in}{1.346212in}}%
\pgfpathlineto{\pgfqpoint{1.234882in}{1.346212in}}%
\pgfusepath{stroke}%
\end{pgfscope}%
\begin{pgfscope}%
\pgfpathrectangle{\pgfqpoint{0.552773in}{0.431673in}}{\pgfqpoint{3.738807in}{1.765244in}}%
\pgfusepath{clip}%
\pgfsetbuttcap%
\pgfsetroundjoin%
\pgfsetlinewidth{1.003750pt}%
\definecolor{currentstroke}{rgb}{0.968203,0.720844,0.612293}%
\pgfsetstrokecolor{currentstroke}%
\pgfsetdash{}{0pt}%
\pgfpathmoveto{\pgfqpoint{1.295052in}{1.380370in}}%
\pgfpathlineto{\pgfqpoint{1.295052in}{1.380370in}}%
\pgfusepath{stroke}%
\end{pgfscope}%
\begin{pgfscope}%
\pgfpathrectangle{\pgfqpoint{0.552773in}{0.431673in}}{\pgfqpoint{3.738807in}{1.765244in}}%
\pgfusepath{clip}%
\pgfsetbuttcap%
\pgfsetroundjoin%
\pgfsetlinewidth{1.003750pt}%
\definecolor{currentstroke}{rgb}{0.968203,0.720844,0.612293}%
\pgfsetstrokecolor{currentstroke}%
\pgfsetdash{}{0pt}%
\pgfpathmoveto{\pgfqpoint{1.333712in}{1.380908in}}%
\pgfpathlineto{\pgfqpoint{1.333712in}{1.380908in}}%
\pgfusepath{stroke}%
\end{pgfscope}%
\begin{pgfscope}%
\pgfpathrectangle{\pgfqpoint{0.552773in}{0.431673in}}{\pgfqpoint{3.738807in}{1.765244in}}%
\pgfusepath{clip}%
\pgfsetbuttcap%
\pgfsetroundjoin%
\pgfsetlinewidth{1.003750pt}%
\definecolor{currentstroke}{rgb}{0.968203,0.720844,0.612293}%
\pgfsetstrokecolor{currentstroke}%
\pgfsetdash{}{0pt}%
\pgfpathmoveto{\pgfqpoint{1.360563in}{1.362986in}}%
\pgfpathlineto{\pgfqpoint{1.360563in}{1.362986in}}%
\pgfusepath{stroke}%
\end{pgfscope}%
\begin{pgfscope}%
\pgfpathrectangle{\pgfqpoint{0.552773in}{0.431673in}}{\pgfqpoint{3.738807in}{1.765244in}}%
\pgfusepath{clip}%
\pgfsetbuttcap%
\pgfsetroundjoin%
\pgfsetlinewidth{1.003750pt}%
\definecolor{currentstroke}{rgb}{0.968203,0.720844,0.612293}%
\pgfsetstrokecolor{currentstroke}%
\pgfsetdash{}{0pt}%
\pgfpathmoveto{\pgfqpoint{1.364476in}{1.309209in}}%
\pgfpathlineto{\pgfqpoint{1.364476in}{1.309209in}}%
\pgfusepath{stroke}%
\end{pgfscope}%
\begin{pgfscope}%
\pgfpathrectangle{\pgfqpoint{0.552773in}{0.431673in}}{\pgfqpoint{3.738807in}{1.765244in}}%
\pgfusepath{clip}%
\pgfsetbuttcap%
\pgfsetroundjoin%
\pgfsetlinewidth{1.003750pt}%
\definecolor{currentstroke}{rgb}{0.968203,0.720844,0.612293}%
\pgfsetstrokecolor{currentstroke}%
\pgfsetdash{}{0pt}%
\pgfpathmoveto{\pgfqpoint{1.369782in}{1.257608in}}%
\pgfpathlineto{\pgfqpoint{1.369782in}{1.257608in}}%
\pgfusepath{stroke}%
\end{pgfscope}%
\begin{pgfscope}%
\pgfpathrectangle{\pgfqpoint{0.552773in}{0.431673in}}{\pgfqpoint{3.738807in}{1.765244in}}%
\pgfusepath{clip}%
\pgfsetbuttcap%
\pgfsetroundjoin%
\pgfsetlinewidth{1.003750pt}%
\definecolor{currentstroke}{rgb}{0.968203,0.720844,0.612293}%
\pgfsetstrokecolor{currentstroke}%
\pgfsetdash{}{0pt}%
\pgfpathmoveto{\pgfqpoint{1.424566in}{1.283350in}}%
\pgfpathlineto{\pgfqpoint{1.424566in}{1.283350in}}%
\pgfusepath{stroke}%
\end{pgfscope}%
\begin{pgfscope}%
\pgfpathrectangle{\pgfqpoint{0.552773in}{0.431673in}}{\pgfqpoint{3.738807in}{1.765244in}}%
\pgfusepath{clip}%
\pgfsetbuttcap%
\pgfsetroundjoin%
\pgfsetlinewidth{1.003750pt}%
\definecolor{currentstroke}{rgb}{0.968203,0.720844,0.612293}%
\pgfsetstrokecolor{currentstroke}%
\pgfsetdash{}{0pt}%
\pgfpathmoveto{\pgfqpoint{1.443716in}{1.253390in}}%
\pgfpathlineto{\pgfqpoint{1.443716in}{1.253390in}}%
\pgfusepath{stroke}%
\end{pgfscope}%
\begin{pgfscope}%
\pgfpathrectangle{\pgfqpoint{0.552773in}{0.431673in}}{\pgfqpoint{3.738807in}{1.765244in}}%
\pgfusepath{clip}%
\pgfsetbuttcap%
\pgfsetroundjoin%
\pgfsetlinewidth{1.003750pt}%
\definecolor{currentstroke}{rgb}{0.968203,0.720844,0.612293}%
\pgfsetstrokecolor{currentstroke}%
\pgfsetdash{}{0pt}%
\pgfpathmoveto{\pgfqpoint{1.466969in}{1.229844in}}%
\pgfpathlineto{\pgfqpoint{1.466969in}{1.229844in}}%
\pgfusepath{stroke}%
\end{pgfscope}%
\begin{pgfscope}%
\pgfpathrectangle{\pgfqpoint{0.552773in}{0.431673in}}{\pgfqpoint{3.738807in}{1.765244in}}%
\pgfusepath{clip}%
\pgfsetbuttcap%
\pgfsetroundjoin%
\pgfsetlinewidth{1.003750pt}%
\definecolor{currentstroke}{rgb}{0.968203,0.720844,0.612293}%
\pgfsetstrokecolor{currentstroke}%
\pgfsetdash{}{0pt}%
\pgfpathmoveto{\pgfqpoint{1.490856in}{1.207288in}}%
\pgfpathlineto{\pgfqpoint{1.490856in}{1.207288in}}%
\pgfusepath{stroke}%
\end{pgfscope}%
\begin{pgfscope}%
\pgfpathrectangle{\pgfqpoint{0.552773in}{0.431673in}}{\pgfqpoint{3.738807in}{1.765244in}}%
\pgfusepath{clip}%
\pgfsetbuttcap%
\pgfsetroundjoin%
\pgfsetlinewidth{1.003750pt}%
\definecolor{currentstroke}{rgb}{0.968203,0.720844,0.612293}%
\pgfsetstrokecolor{currentstroke}%
\pgfsetdash{}{0pt}%
\pgfpathmoveto{\pgfqpoint{1.503289in}{1.166829in}}%
\pgfpathlineto{\pgfqpoint{1.503289in}{1.166829in}}%
\pgfusepath{stroke}%
\end{pgfscope}%
\begin{pgfscope}%
\pgfpathrectangle{\pgfqpoint{0.552773in}{0.431673in}}{\pgfqpoint{3.738807in}{1.765244in}}%
\pgfusepath{clip}%
\pgfsetbuttcap%
\pgfsetroundjoin%
\pgfsetlinewidth{1.003750pt}%
\definecolor{currentstroke}{rgb}{0.968203,0.720844,0.612293}%
\pgfsetstrokecolor{currentstroke}%
\pgfsetdash{}{0pt}%
\pgfpathmoveto{\pgfqpoint{1.551438in}{1.182198in}}%
\pgfpathlineto{\pgfqpoint{1.551438in}{1.182198in}}%
\pgfusepath{stroke}%
\end{pgfscope}%
\begin{pgfscope}%
\pgfpathrectangle{\pgfqpoint{0.552773in}{0.431673in}}{\pgfqpoint{3.738807in}{1.765244in}}%
\pgfusepath{clip}%
\pgfsetbuttcap%
\pgfsetroundjoin%
\pgfsetlinewidth{1.003750pt}%
\definecolor{currentstroke}{rgb}{0.968203,0.720844,0.612293}%
\pgfsetstrokecolor{currentstroke}%
\pgfsetdash{}{0pt}%
\pgfpathmoveto{\pgfqpoint{1.607593in}{1.210082in}}%
\pgfpathlineto{\pgfqpoint{1.607593in}{1.210082in}}%
\pgfusepath{stroke}%
\end{pgfscope}%
\begin{pgfscope}%
\pgfpathrectangle{\pgfqpoint{0.552773in}{0.431673in}}{\pgfqpoint{3.738807in}{1.765244in}}%
\pgfusepath{clip}%
\pgfsetbuttcap%
\pgfsetroundjoin%
\pgfsetlinewidth{1.003750pt}%
\definecolor{currentstroke}{rgb}{0.968203,0.720844,0.612293}%
\pgfsetstrokecolor{currentstroke}%
\pgfsetdash{}{0pt}%
\pgfpathmoveto{\pgfqpoint{1.658864in}{1.230331in}}%
\pgfpathlineto{\pgfqpoint{1.658864in}{1.230331in}}%
\pgfusepath{stroke}%
\end{pgfscope}%
\begin{pgfscope}%
\pgfpathrectangle{\pgfqpoint{0.552773in}{0.431673in}}{\pgfqpoint{3.738807in}{1.765244in}}%
\pgfusepath{clip}%
\pgfsetbuttcap%
\pgfsetroundjoin%
\pgfsetlinewidth{1.003750pt}%
\definecolor{currentstroke}{rgb}{0.968203,0.720844,0.612293}%
\pgfsetstrokecolor{currentstroke}%
\pgfsetdash{}{0pt}%
\pgfpathmoveto{\pgfqpoint{1.722476in}{1.269870in}}%
\pgfpathlineto{\pgfqpoint{1.722476in}{1.269870in}}%
\pgfusepath{stroke}%
\end{pgfscope}%
\begin{pgfscope}%
\pgfpathrectangle{\pgfqpoint{0.552773in}{0.431673in}}{\pgfqpoint{3.738807in}{1.765244in}}%
\pgfusepath{clip}%
\pgfsetbuttcap%
\pgfsetroundjoin%
\pgfsetlinewidth{1.003750pt}%
\definecolor{currentstroke}{rgb}{0.968203,0.720844,0.612293}%
\pgfsetstrokecolor{currentstroke}%
\pgfsetdash{}{0pt}%
\pgfpathmoveto{\pgfqpoint{1.797820in}{1.327748in}}%
\pgfpathlineto{\pgfqpoint{1.797820in}{1.327748in}}%
\pgfusepath{stroke}%
\end{pgfscope}%
\begin{pgfscope}%
\pgfpathrectangle{\pgfqpoint{0.552773in}{0.431673in}}{\pgfqpoint{3.738807in}{1.765244in}}%
\pgfusepath{clip}%
\pgfsetbuttcap%
\pgfsetroundjoin%
\pgfsetlinewidth{1.003750pt}%
\definecolor{currentstroke}{rgb}{0.968203,0.720844,0.612293}%
\pgfsetstrokecolor{currentstroke}%
\pgfsetdash{}{0pt}%
\pgfpathmoveto{\pgfqpoint{1.877516in}{1.392430in}}%
\pgfpathlineto{\pgfqpoint{1.877516in}{1.392430in}}%
\pgfusepath{stroke}%
\end{pgfscope}%
\begin{pgfscope}%
\pgfpathrectangle{\pgfqpoint{0.552773in}{0.431673in}}{\pgfqpoint{3.738807in}{1.765244in}}%
\pgfusepath{clip}%
\pgfsetbuttcap%
\pgfsetroundjoin%
\pgfsetlinewidth{1.003750pt}%
\definecolor{currentstroke}{rgb}{0.968203,0.720844,0.612293}%
\pgfsetstrokecolor{currentstroke}%
\pgfsetdash{}{0pt}%
\pgfpathmoveto{\pgfqpoint{1.954391in}{1.452701in}}%
\pgfpathlineto{\pgfqpoint{1.954391in}{1.452701in}}%
\pgfusepath{stroke}%
\end{pgfscope}%
\begin{pgfscope}%
\pgfpathrectangle{\pgfqpoint{0.552773in}{0.431673in}}{\pgfqpoint{3.738807in}{1.765244in}}%
\pgfusepath{clip}%
\pgfsetbuttcap%
\pgfsetroundjoin%
\pgfsetlinewidth{1.003750pt}%
\definecolor{currentstroke}{rgb}{0.968203,0.720844,0.612293}%
\pgfsetstrokecolor{currentstroke}%
\pgfsetdash{}{0pt}%
\pgfpathmoveto{\pgfqpoint{2.014562in}{1.486861in}}%
\pgfpathlineto{\pgfqpoint{2.014562in}{1.486861in}}%
\pgfusepath{stroke}%
\end{pgfscope}%
\begin{pgfscope}%
\pgfpathrectangle{\pgfqpoint{0.552773in}{0.431673in}}{\pgfqpoint{3.738807in}{1.765244in}}%
\pgfusepath{clip}%
\pgfsetbuttcap%
\pgfsetroundjoin%
\pgfsetlinewidth{1.003750pt}%
\definecolor{currentstroke}{rgb}{0.968203,0.720844,0.612293}%
\pgfsetstrokecolor{currentstroke}%
\pgfsetdash{}{0pt}%
\pgfpathmoveto{\pgfqpoint{2.061371in}{1.500136in}}%
\pgfpathlineto{\pgfqpoint{2.061371in}{1.500136in}}%
\pgfusepath{stroke}%
\end{pgfscope}%
\begin{pgfscope}%
\pgfpathrectangle{\pgfqpoint{0.552773in}{0.431673in}}{\pgfqpoint{3.738807in}{1.765244in}}%
\pgfusepath{clip}%
\pgfsetbuttcap%
\pgfsetroundjoin%
\pgfsetlinewidth{1.003750pt}%
\definecolor{currentstroke}{rgb}{0.968203,0.720844,0.612293}%
\pgfsetstrokecolor{currentstroke}%
\pgfsetdash{}{0pt}%
\pgfpathmoveto{\pgfqpoint{2.083864in}{1.475402in}}%
\pgfpathlineto{\pgfqpoint{2.083864in}{1.475402in}}%
\pgfusepath{stroke}%
\end{pgfscope}%
\begin{pgfscope}%
\pgfpathrectangle{\pgfqpoint{0.552773in}{0.431673in}}{\pgfqpoint{3.738807in}{1.765244in}}%
\pgfusepath{clip}%
\pgfsetbuttcap%
\pgfsetroundjoin%
\pgfsetlinewidth{1.003750pt}%
\definecolor{currentstroke}{rgb}{0.968203,0.720844,0.612293}%
\pgfsetstrokecolor{currentstroke}%
\pgfsetdash{}{0pt}%
\pgfpathmoveto{\pgfqpoint{2.105964in}{1.450053in}}%
\pgfpathlineto{\pgfqpoint{2.105964in}{1.450053in}}%
\pgfusepath{stroke}%
\end{pgfscope}%
\begin{pgfscope}%
\pgfpathrectangle{\pgfqpoint{0.552773in}{0.431673in}}{\pgfqpoint{3.738807in}{1.765244in}}%
\pgfusepath{clip}%
\pgfsetbuttcap%
\pgfsetroundjoin%
\pgfsetlinewidth{1.003750pt}%
\definecolor{currentstroke}{rgb}{0.968203,0.720844,0.612293}%
\pgfsetstrokecolor{currentstroke}%
\pgfsetdash{}{0pt}%
\pgfpathmoveto{\pgfqpoint{2.116436in}{1.406528in}}%
\pgfpathlineto{\pgfqpoint{2.116436in}{1.406528in}}%
\pgfusepath{stroke}%
\end{pgfscope}%
\begin{pgfscope}%
\pgfpathrectangle{\pgfqpoint{0.552773in}{0.431673in}}{\pgfqpoint{3.738807in}{1.765244in}}%
\pgfusepath{clip}%
\pgfsetbuttcap%
\pgfsetroundjoin%
\pgfsetlinewidth{1.003750pt}%
\definecolor{currentstroke}{rgb}{0.968203,0.720844,0.612293}%
\pgfsetstrokecolor{currentstroke}%
\pgfsetdash{}{0pt}%
\pgfpathmoveto{\pgfqpoint{2.141673in}{1.386083in}}%
\pgfpathlineto{\pgfqpoint{2.141673in}{1.386083in}}%
\pgfusepath{stroke}%
\end{pgfscope}%
\begin{pgfscope}%
\pgfpathrectangle{\pgfqpoint{0.552773in}{0.431673in}}{\pgfqpoint{3.738807in}{1.765244in}}%
\pgfusepath{clip}%
\pgfsetbuttcap%
\pgfsetroundjoin%
\pgfsetlinewidth{1.003750pt}%
\definecolor{currentstroke}{rgb}{0.968203,0.720844,0.612293}%
\pgfsetstrokecolor{currentstroke}%
\pgfsetdash{}{0pt}%
\pgfpathmoveto{\pgfqpoint{2.156766in}{1.349782in}}%
\pgfpathlineto{\pgfqpoint{2.156766in}{1.349782in}}%
\pgfusepath{stroke}%
\end{pgfscope}%
\begin{pgfscope}%
\pgfpathrectangle{\pgfqpoint{0.552773in}{0.431673in}}{\pgfqpoint{3.738807in}{1.765244in}}%
\pgfusepath{clip}%
\pgfsetbuttcap%
\pgfsetroundjoin%
\pgfsetlinewidth{1.003750pt}%
\definecolor{currentstroke}{rgb}{0.968203,0.720844,0.612293}%
\pgfsetstrokecolor{currentstroke}%
\pgfsetdash{}{0pt}%
\pgfpathmoveto{\pgfqpoint{2.171473in}{1.312877in}}%
\pgfpathlineto{\pgfqpoint{2.171473in}{1.312877in}}%
\pgfusepath{stroke}%
\end{pgfscope}%
\begin{pgfscope}%
\pgfpathrectangle{\pgfqpoint{0.552773in}{0.431673in}}{\pgfqpoint{3.738807in}{1.765244in}}%
\pgfusepath{clip}%
\pgfsetbuttcap%
\pgfsetroundjoin%
\pgfsetlinewidth{1.003750pt}%
\definecolor{currentstroke}{rgb}{0.968203,0.720844,0.612293}%
\pgfsetstrokecolor{currentstroke}%
\pgfsetdash{}{0pt}%
\pgfpathmoveto{\pgfqpoint{2.192191in}{1.285369in}}%
\pgfpathlineto{\pgfqpoint{2.192191in}{1.285369in}}%
\pgfusepath{stroke}%
\end{pgfscope}%
\begin{pgfscope}%
\pgfpathrectangle{\pgfqpoint{0.552773in}{0.431673in}}{\pgfqpoint{3.738807in}{1.765244in}}%
\pgfusepath{clip}%
\pgfsetbuttcap%
\pgfsetroundjoin%
\pgfsetlinewidth{1.003750pt}%
\definecolor{currentstroke}{rgb}{0.968203,0.720844,0.612293}%
\pgfsetstrokecolor{currentstroke}%
\pgfsetdash{}{0pt}%
\pgfpathmoveto{\pgfqpoint{2.223452in}{1.274341in}}%
\pgfpathlineto{\pgfqpoint{2.223452in}{1.274341in}}%
\pgfusepath{stroke}%
\end{pgfscope}%
\begin{pgfscope}%
\pgfpathrectangle{\pgfqpoint{0.552773in}{0.431673in}}{\pgfqpoint{3.738807in}{1.765244in}}%
\pgfusepath{clip}%
\pgfsetbuttcap%
\pgfsetroundjoin%
\pgfsetlinewidth{1.003750pt}%
\definecolor{currentstroke}{rgb}{0.968203,0.720844,0.612293}%
\pgfsetstrokecolor{currentstroke}%
\pgfsetdash{}{0pt}%
\pgfpathmoveto{\pgfqpoint{2.256793in}{1.266562in}}%
\pgfpathlineto{\pgfqpoint{2.256793in}{1.266562in}}%
\pgfusepath{stroke}%
\end{pgfscope}%
\begin{pgfscope}%
\pgfpathrectangle{\pgfqpoint{0.552773in}{0.431673in}}{\pgfqpoint{3.738807in}{1.765244in}}%
\pgfusepath{clip}%
\pgfsetbuttcap%
\pgfsetroundjoin%
\pgfsetlinewidth{1.003750pt}%
\definecolor{currentstroke}{rgb}{0.968203,0.720844,0.612293}%
\pgfsetstrokecolor{currentstroke}%
\pgfsetdash{}{0pt}%
\pgfpathmoveto{\pgfqpoint{2.298519in}{1.271893in}}%
\pgfpathlineto{\pgfqpoint{2.298519in}{1.271893in}}%
\pgfusepath{stroke}%
\end{pgfscope}%
\begin{pgfscope}%
\pgfpathrectangle{\pgfqpoint{0.552773in}{0.431673in}}{\pgfqpoint{3.738807in}{1.765244in}}%
\pgfusepath{clip}%
\pgfsetbuttcap%
\pgfsetroundjoin%
\pgfsetlinewidth{1.003750pt}%
\definecolor{currentstroke}{rgb}{0.968203,0.720844,0.612293}%
\pgfsetstrokecolor{currentstroke}%
\pgfsetdash{}{0pt}%
\pgfpathmoveto{\pgfqpoint{2.360130in}{1.308304in}}%
\pgfpathlineto{\pgfqpoint{2.360130in}{1.308304in}}%
\pgfusepath{stroke}%
\end{pgfscope}%
\begin{pgfscope}%
\pgfpathrectangle{\pgfqpoint{0.552773in}{0.431673in}}{\pgfqpoint{3.738807in}{1.765244in}}%
\pgfusepath{clip}%
\pgfsetbuttcap%
\pgfsetroundjoin%
\pgfsetlinewidth{1.003750pt}%
\definecolor{currentstroke}{rgb}{0.968203,0.720844,0.612293}%
\pgfsetstrokecolor{currentstroke}%
\pgfsetdash{}{0pt}%
\pgfpathmoveto{\pgfqpoint{2.402956in}{1.315352in}}%
\pgfpathlineto{\pgfqpoint{2.402956in}{1.315352in}}%
\pgfusepath{stroke}%
\end{pgfscope}%
\begin{pgfscope}%
\pgfpathrectangle{\pgfqpoint{0.552773in}{0.431673in}}{\pgfqpoint{3.738807in}{1.765244in}}%
\pgfusepath{clip}%
\pgfsetbuttcap%
\pgfsetroundjoin%
\pgfsetlinewidth{1.003750pt}%
\definecolor{currentstroke}{rgb}{0.968203,0.720844,0.612293}%
\pgfsetstrokecolor{currentstroke}%
\pgfsetdash{}{0pt}%
\pgfpathmoveto{\pgfqpoint{2.474900in}{1.367916in}}%
\pgfpathlineto{\pgfqpoint{2.474900in}{1.367916in}}%
\pgfusepath{stroke}%
\end{pgfscope}%
\begin{pgfscope}%
\pgfpathrectangle{\pgfqpoint{0.552773in}{0.431673in}}{\pgfqpoint{3.738807in}{1.765244in}}%
\pgfusepath{clip}%
\pgfsetbuttcap%
\pgfsetroundjoin%
\pgfsetlinewidth{1.003750pt}%
\definecolor{currentstroke}{rgb}{0.968203,0.720844,0.612293}%
\pgfsetstrokecolor{currentstroke}%
\pgfsetdash{}{0pt}%
\pgfpathmoveto{\pgfqpoint{2.549428in}{1.424519in}}%
\pgfpathlineto{\pgfqpoint{2.549428in}{1.424519in}}%
\pgfusepath{stroke}%
\end{pgfscope}%
\begin{pgfscope}%
\pgfpathrectangle{\pgfqpoint{0.552773in}{0.431673in}}{\pgfqpoint{3.738807in}{1.765244in}}%
\pgfusepath{clip}%
\pgfsetbuttcap%
\pgfsetroundjoin%
\pgfsetlinewidth{1.003750pt}%
\definecolor{currentstroke}{rgb}{0.968203,0.720844,0.612293}%
\pgfsetstrokecolor{currentstroke}%
\pgfsetdash{}{0pt}%
\pgfpathmoveto{\pgfqpoint{2.619369in}{1.473950in}}%
\pgfpathlineto{\pgfqpoint{2.619369in}{1.473950in}}%
\pgfusepath{stroke}%
\end{pgfscope}%
\begin{pgfscope}%
\pgfpathrectangle{\pgfqpoint{0.552773in}{0.431673in}}{\pgfqpoint{3.738807in}{1.765244in}}%
\pgfusepath{clip}%
\pgfsetbuttcap%
\pgfsetroundjoin%
\pgfsetlinewidth{1.003750pt}%
\definecolor{currentstroke}{rgb}{0.968203,0.720844,0.612293}%
\pgfsetstrokecolor{currentstroke}%
\pgfsetdash{}{0pt}%
\pgfpathmoveto{\pgfqpoint{2.688119in}{1.521522in}}%
\pgfpathlineto{\pgfqpoint{2.688119in}{1.521522in}}%
\pgfusepath{stroke}%
\end{pgfscope}%
\begin{pgfscope}%
\pgfpathrectangle{\pgfqpoint{0.552773in}{0.431673in}}{\pgfqpoint{3.738807in}{1.765244in}}%
\pgfusepath{clip}%
\pgfsetbuttcap%
\pgfsetroundjoin%
\pgfsetlinewidth{1.003750pt}%
\definecolor{currentstroke}{rgb}{0.968203,0.720844,0.612293}%
\pgfsetstrokecolor{currentstroke}%
\pgfsetdash{}{0pt}%
\pgfpathmoveto{\pgfqpoint{2.745485in}{1.551298in}}%
\pgfpathlineto{\pgfqpoint{2.745485in}{1.551298in}}%
\pgfusepath{stroke}%
\end{pgfscope}%
\begin{pgfscope}%
\pgfpathrectangle{\pgfqpoint{0.552773in}{0.431673in}}{\pgfqpoint{3.738807in}{1.765244in}}%
\pgfusepath{clip}%
\pgfsetbuttcap%
\pgfsetroundjoin%
\pgfsetlinewidth{1.003750pt}%
\definecolor{currentstroke}{rgb}{0.968203,0.720844,0.612293}%
\pgfsetstrokecolor{currentstroke}%
\pgfsetdash{}{0pt}%
\pgfpathmoveto{\pgfqpoint{2.770849in}{1.531052in}}%
\pgfpathlineto{\pgfqpoint{2.770849in}{1.531052in}}%
\pgfusepath{stroke}%
\end{pgfscope}%
\begin{pgfscope}%
\pgfpathrectangle{\pgfqpoint{0.552773in}{0.431673in}}{\pgfqpoint{3.738807in}{1.765244in}}%
\pgfusepath{clip}%
\pgfsetbuttcap%
\pgfsetroundjoin%
\pgfsetlinewidth{1.003750pt}%
\definecolor{currentstroke}{rgb}{0.968203,0.720844,0.612293}%
\pgfsetstrokecolor{currentstroke}%
\pgfsetdash{}{0pt}%
\pgfpathmoveto{\pgfqpoint{2.791502in}{1.503442in}}%
\pgfpathlineto{\pgfqpoint{2.791502in}{1.503442in}}%
\pgfusepath{stroke}%
\end{pgfscope}%
\begin{pgfscope}%
\pgfpathrectangle{\pgfqpoint{0.552773in}{0.431673in}}{\pgfqpoint{3.738807in}{1.765244in}}%
\pgfusepath{clip}%
\pgfsetbuttcap%
\pgfsetroundjoin%
\pgfsetlinewidth{1.003750pt}%
\definecolor{currentstroke}{rgb}{0.968203,0.720844,0.612293}%
\pgfsetstrokecolor{currentstroke}%
\pgfsetdash{}{0pt}%
\pgfpathmoveto{\pgfqpoint{2.801394in}{1.459011in}}%
\pgfpathlineto{\pgfqpoint{2.801394in}{1.459011in}}%
\pgfusepath{stroke}%
\end{pgfscope}%
\begin{pgfscope}%
\pgfpathrectangle{\pgfqpoint{0.552773in}{0.431673in}}{\pgfqpoint{3.738807in}{1.765244in}}%
\pgfusepath{clip}%
\pgfsetbuttcap%
\pgfsetroundjoin%
\pgfsetlinewidth{1.003750pt}%
\definecolor{currentstroke}{rgb}{0.968203,0.720844,0.612293}%
\pgfsetstrokecolor{currentstroke}%
\pgfsetdash{}{0pt}%
\pgfpathmoveto{\pgfqpoint{2.825889in}{1.437406in}}%
\pgfpathlineto{\pgfqpoint{2.825889in}{1.437406in}}%
\pgfusepath{stroke}%
\end{pgfscope}%
\begin{pgfscope}%
\pgfpathrectangle{\pgfqpoint{0.552773in}{0.431673in}}{\pgfqpoint{3.738807in}{1.765244in}}%
\pgfusepath{clip}%
\pgfsetbuttcap%
\pgfsetroundjoin%
\pgfsetlinewidth{1.003750pt}%
\definecolor{currentstroke}{rgb}{0.968203,0.720844,0.612293}%
\pgfsetstrokecolor{currentstroke}%
\pgfsetdash{}{0pt}%
\pgfpathmoveto{\pgfqpoint{2.823590in}{1.373918in}}%
\pgfpathlineto{\pgfqpoint{2.823590in}{1.373918in}}%
\pgfusepath{stroke}%
\end{pgfscope}%
\begin{pgfscope}%
\pgfpathrectangle{\pgfqpoint{0.552773in}{0.431673in}}{\pgfqpoint{3.738807in}{1.765244in}}%
\pgfusepath{clip}%
\pgfsetbuttcap%
\pgfsetroundjoin%
\pgfsetlinewidth{1.003750pt}%
\definecolor{currentstroke}{rgb}{0.968203,0.720844,0.612293}%
\pgfsetstrokecolor{currentstroke}%
\pgfsetdash{}{0pt}%
\pgfpathmoveto{\pgfqpoint{2.848428in}{1.352850in}}%
\pgfpathlineto{\pgfqpoint{2.848428in}{1.352850in}}%
\pgfusepath{stroke}%
\end{pgfscope}%
\begin{pgfscope}%
\pgfpathrectangle{\pgfqpoint{0.552773in}{0.431673in}}{\pgfqpoint{3.738807in}{1.765244in}}%
\pgfusepath{clip}%
\pgfsetbuttcap%
\pgfsetroundjoin%
\pgfsetlinewidth{1.003750pt}%
\definecolor{currentstroke}{rgb}{0.968203,0.720844,0.612293}%
\pgfsetstrokecolor{currentstroke}%
\pgfsetdash{}{0pt}%
\pgfpathmoveto{\pgfqpoint{2.859935in}{1.310943in}}%
\pgfpathlineto{\pgfqpoint{2.859935in}{1.310943in}}%
\pgfusepath{stroke}%
\end{pgfscope}%
\begin{pgfscope}%
\pgfpathrectangle{\pgfqpoint{0.552773in}{0.431673in}}{\pgfqpoint{3.738807in}{1.765244in}}%
\pgfusepath{clip}%
\pgfsetbuttcap%
\pgfsetroundjoin%
\pgfsetlinewidth{1.003750pt}%
\definecolor{currentstroke}{rgb}{0.968203,0.720844,0.612293}%
\pgfsetstrokecolor{currentstroke}%
\pgfsetdash{}{0pt}%
\pgfpathmoveto{\pgfqpoint{2.877851in}{1.279055in}}%
\pgfpathlineto{\pgfqpoint{2.877851in}{1.279055in}}%
\pgfusepath{stroke}%
\end{pgfscope}%
\begin{pgfscope}%
\pgfpathrectangle{\pgfqpoint{0.552773in}{0.431673in}}{\pgfqpoint{3.738807in}{1.765244in}}%
\pgfusepath{clip}%
\pgfsetbuttcap%
\pgfsetroundjoin%
\pgfsetlinewidth{1.003750pt}%
\definecolor{currentstroke}{rgb}{0.968203,0.720844,0.612293}%
\pgfsetstrokecolor{currentstroke}%
\pgfsetdash{}{0pt}%
\pgfpathmoveto{\pgfqpoint{2.897792in}{1.250331in}}%
\pgfpathlineto{\pgfqpoint{2.897792in}{1.250331in}}%
\pgfusepath{stroke}%
\end{pgfscope}%
\begin{pgfscope}%
\pgfpathrectangle{\pgfqpoint{0.552773in}{0.431673in}}{\pgfqpoint{3.738807in}{1.765244in}}%
\pgfusepath{clip}%
\pgfsetbuttcap%
\pgfsetroundjoin%
\pgfsetlinewidth{1.003750pt}%
\definecolor{currentstroke}{rgb}{0.968203,0.720844,0.612293}%
\pgfsetstrokecolor{currentstroke}%
\pgfsetdash{}{0pt}%
\pgfpathmoveto{\pgfqpoint{2.906245in}{1.203652in}}%
\pgfpathlineto{\pgfqpoint{2.906245in}{1.203652in}}%
\pgfusepath{stroke}%
\end{pgfscope}%
\begin{pgfscope}%
\pgfpathrectangle{\pgfqpoint{0.552773in}{0.431673in}}{\pgfqpoint{3.738807in}{1.765244in}}%
\pgfusepath{clip}%
\pgfsetbuttcap%
\pgfsetroundjoin%
\pgfsetlinewidth{1.003750pt}%
\definecolor{currentstroke}{rgb}{0.968203,0.720844,0.612293}%
\pgfsetstrokecolor{currentstroke}%
\pgfsetdash{}{0pt}%
\pgfpathmoveto{\pgfqpoint{2.946444in}{1.206593in}}%
\pgfpathlineto{\pgfqpoint{2.946444in}{1.206593in}}%
\pgfusepath{stroke}%
\end{pgfscope}%
\begin{pgfscope}%
\pgfpathrectangle{\pgfqpoint{0.552773in}{0.431673in}}{\pgfqpoint{3.738807in}{1.765244in}}%
\pgfusepath{clip}%
\pgfsetbuttcap%
\pgfsetroundjoin%
\pgfsetlinewidth{1.003750pt}%
\definecolor{currentstroke}{rgb}{0.968203,0.720844,0.612293}%
\pgfsetstrokecolor{currentstroke}%
\pgfsetdash{}{0pt}%
\pgfpathmoveto{\pgfqpoint{3.013245in}{1.251118in}}%
\pgfpathlineto{\pgfqpoint{3.013245in}{1.251118in}}%
\pgfusepath{stroke}%
\end{pgfscope}%
\begin{pgfscope}%
\pgfpathrectangle{\pgfqpoint{0.552773in}{0.431673in}}{\pgfqpoint{3.738807in}{1.765244in}}%
\pgfusepath{clip}%
\pgfsetbuttcap%
\pgfsetroundjoin%
\pgfsetlinewidth{1.003750pt}%
\definecolor{currentstroke}{rgb}{0.968203,0.720844,0.612293}%
\pgfsetstrokecolor{currentstroke}%
\pgfsetdash{}{0pt}%
\pgfpathmoveto{\pgfqpoint{3.071632in}{1.282490in}}%
\pgfpathlineto{\pgfqpoint{3.071632in}{1.282490in}}%
\pgfusepath{stroke}%
\end{pgfscope}%
\begin{pgfscope}%
\pgfpathrectangle{\pgfqpoint{0.552773in}{0.431673in}}{\pgfqpoint{3.738807in}{1.765244in}}%
\pgfusepath{clip}%
\pgfsetbuttcap%
\pgfsetroundjoin%
\pgfsetlinewidth{1.003750pt}%
\definecolor{currentstroke}{rgb}{0.968203,0.720844,0.612293}%
\pgfsetstrokecolor{currentstroke}%
\pgfsetdash{}{0pt}%
\pgfpathmoveto{\pgfqpoint{3.138985in}{1.327877in}}%
\pgfpathlineto{\pgfqpoint{3.138985in}{1.327877in}}%
\pgfusepath{stroke}%
\end{pgfscope}%
\begin{pgfscope}%
\pgfpathrectangle{\pgfqpoint{0.552773in}{0.431673in}}{\pgfqpoint{3.738807in}{1.765244in}}%
\pgfusepath{clip}%
\pgfsetbuttcap%
\pgfsetroundjoin%
\pgfsetlinewidth{1.003750pt}%
\definecolor{currentstroke}{rgb}{0.968203,0.720844,0.612293}%
\pgfsetstrokecolor{currentstroke}%
\pgfsetdash{}{0pt}%
\pgfpathmoveto{\pgfqpoint{3.210642in}{1.379991in}}%
\pgfpathlineto{\pgfqpoint{3.210642in}{1.379991in}}%
\pgfusepath{stroke}%
\end{pgfscope}%
\begin{pgfscope}%
\pgfpathrectangle{\pgfqpoint{0.552773in}{0.431673in}}{\pgfqpoint{3.738807in}{1.765244in}}%
\pgfusepath{clip}%
\pgfsetbuttcap%
\pgfsetroundjoin%
\pgfsetlinewidth{1.003750pt}%
\definecolor{currentstroke}{rgb}{0.968203,0.720844,0.612293}%
\pgfsetstrokecolor{currentstroke}%
\pgfsetdash{}{0pt}%
\pgfpathmoveto{\pgfqpoint{3.288346in}{1.441559in}}%
\pgfpathlineto{\pgfqpoint{3.288346in}{1.441559in}}%
\pgfusepath{stroke}%
\end{pgfscope}%
\begin{pgfscope}%
\pgfpathrectangle{\pgfqpoint{0.552773in}{0.431673in}}{\pgfqpoint{3.738807in}{1.765244in}}%
\pgfusepath{clip}%
\pgfsetbuttcap%
\pgfsetroundjoin%
\pgfsetlinewidth{1.003750pt}%
\definecolor{currentstroke}{rgb}{0.968203,0.720844,0.612293}%
\pgfsetstrokecolor{currentstroke}%
\pgfsetdash{}{0pt}%
\pgfpathmoveto{\pgfqpoint{3.354079in}{1.484414in}}%
\pgfpathlineto{\pgfqpoint{3.354079in}{1.484414in}}%
\pgfusepath{stroke}%
\end{pgfscope}%
\begin{pgfscope}%
\pgfpathrectangle{\pgfqpoint{0.552773in}{0.431673in}}{\pgfqpoint{3.738807in}{1.765244in}}%
\pgfusepath{clip}%
\pgfsetbuttcap%
\pgfsetroundjoin%
\pgfsetlinewidth{1.003750pt}%
\definecolor{currentstroke}{rgb}{0.968203,0.720844,0.612293}%
\pgfsetstrokecolor{currentstroke}%
\pgfsetdash{}{0pt}%
\pgfpathmoveto{\pgfqpoint{3.395162in}{1.488738in}}%
\pgfpathlineto{\pgfqpoint{3.395162in}{1.488738in}}%
\pgfusepath{stroke}%
\end{pgfscope}%
\begin{pgfscope}%
\pgfpathrectangle{\pgfqpoint{0.552773in}{0.431673in}}{\pgfqpoint{3.738807in}{1.765244in}}%
\pgfusepath{clip}%
\pgfsetbuttcap%
\pgfsetroundjoin%
\pgfsetlinewidth{1.003750pt}%
\definecolor{currentstroke}{rgb}{0.968203,0.720844,0.612293}%
\pgfsetstrokecolor{currentstroke}%
\pgfsetdash{}{0pt}%
\pgfpathmoveto{\pgfqpoint{3.423246in}{1.472743in}}%
\pgfpathlineto{\pgfqpoint{3.423246in}{1.472743in}}%
\pgfusepath{stroke}%
\end{pgfscope}%
\begin{pgfscope}%
\pgfpathrectangle{\pgfqpoint{0.552773in}{0.431673in}}{\pgfqpoint{3.738807in}{1.765244in}}%
\pgfusepath{clip}%
\pgfsetbuttcap%
\pgfsetroundjoin%
\pgfsetlinewidth{1.003750pt}%
\definecolor{currentstroke}{rgb}{0.968203,0.720844,0.612293}%
\pgfsetstrokecolor{currentstroke}%
\pgfsetdash{}{0pt}%
\pgfpathmoveto{\pgfqpoint{3.450046in}{1.454742in}}%
\pgfpathlineto{\pgfqpoint{3.450046in}{1.454742in}}%
\pgfusepath{stroke}%
\end{pgfscope}%
\begin{pgfscope}%
\pgfpathrectangle{\pgfqpoint{0.552773in}{0.431673in}}{\pgfqpoint{3.738807in}{1.765244in}}%
\pgfusepath{clip}%
\pgfsetbuttcap%
\pgfsetroundjoin%
\pgfsetlinewidth{1.003750pt}%
\definecolor{currentstroke}{rgb}{0.968203,0.720844,0.612293}%
\pgfsetstrokecolor{currentstroke}%
\pgfsetdash{}{0pt}%
\pgfpathmoveto{\pgfqpoint{3.453543in}{1.400314in}}%
\pgfpathlineto{\pgfqpoint{3.453543in}{1.400314in}}%
\pgfusepath{stroke}%
\end{pgfscope}%
\begin{pgfscope}%
\pgfpathrectangle{\pgfqpoint{0.552773in}{0.431673in}}{\pgfqpoint{3.738807in}{1.765244in}}%
\pgfusepath{clip}%
\pgfsetbuttcap%
\pgfsetroundjoin%
\pgfsetlinewidth{1.003750pt}%
\definecolor{currentstroke}{rgb}{0.968203,0.720844,0.612293}%
\pgfsetstrokecolor{currentstroke}%
\pgfsetdash{}{0pt}%
\pgfpathmoveto{\pgfqpoint{3.455571in}{1.343591in}}%
\pgfpathlineto{\pgfqpoint{3.455571in}{1.343591in}}%
\pgfusepath{stroke}%
\end{pgfscope}%
\begin{pgfscope}%
\pgfpathrectangle{\pgfqpoint{0.552773in}{0.431673in}}{\pgfqpoint{3.738807in}{1.765244in}}%
\pgfusepath{clip}%
\pgfsetbuttcap%
\pgfsetroundjoin%
\pgfsetlinewidth{1.003750pt}%
\definecolor{currentstroke}{rgb}{0.968203,0.720844,0.612293}%
\pgfsetstrokecolor{currentstroke}%
\pgfsetdash{}{0pt}%
\pgfpathmoveto{\pgfqpoint{3.477372in}{1.317776in}}%
\pgfpathlineto{\pgfqpoint{3.477372in}{1.317776in}}%
\pgfusepath{stroke}%
\end{pgfscope}%
\begin{pgfscope}%
\pgfpathrectangle{\pgfqpoint{0.552773in}{0.431673in}}{\pgfqpoint{3.738807in}{1.765244in}}%
\pgfusepath{clip}%
\pgfsetbuttcap%
\pgfsetroundjoin%
\pgfsetlinewidth{1.003750pt}%
\definecolor{currentstroke}{rgb}{0.968203,0.720844,0.612293}%
\pgfsetstrokecolor{currentstroke}%
\pgfsetdash{}{0pt}%
\pgfpathmoveto{\pgfqpoint{3.481371in}{1.264133in}}%
\pgfpathlineto{\pgfqpoint{3.481371in}{1.264133in}}%
\pgfusepath{stroke}%
\end{pgfscope}%
\begin{pgfscope}%
\pgfpathrectangle{\pgfqpoint{0.552773in}{0.431673in}}{\pgfqpoint{3.738807in}{1.765244in}}%
\pgfusepath{clip}%
\pgfsetbuttcap%
\pgfsetroundjoin%
\pgfsetlinewidth{1.003750pt}%
\definecolor{currentstroke}{rgb}{0.968203,0.720844,0.612293}%
\pgfsetstrokecolor{currentstroke}%
\pgfsetdash{}{0pt}%
\pgfpathmoveto{\pgfqpoint{3.505373in}{1.241757in}}%
\pgfpathlineto{\pgfqpoint{3.505373in}{1.241757in}}%
\pgfusepath{stroke}%
\end{pgfscope}%
\begin{pgfscope}%
\pgfpathrectangle{\pgfqpoint{0.552773in}{0.431673in}}{\pgfqpoint{3.738807in}{1.765244in}}%
\pgfusepath{clip}%
\pgfsetbuttcap%
\pgfsetroundjoin%
\pgfsetlinewidth{1.003750pt}%
\definecolor{currentstroke}{rgb}{0.968203,0.720844,0.612293}%
\pgfsetstrokecolor{currentstroke}%
\pgfsetdash{}{0pt}%
\pgfpathmoveto{\pgfqpoint{3.526998in}{1.215666in}}%
\pgfpathlineto{\pgfqpoint{3.526998in}{1.215666in}}%
\pgfusepath{stroke}%
\end{pgfscope}%
\begin{pgfscope}%
\pgfpathrectangle{\pgfqpoint{0.552773in}{0.431673in}}{\pgfqpoint{3.738807in}{1.765244in}}%
\pgfusepath{clip}%
\pgfsetbuttcap%
\pgfsetroundjoin%
\pgfsetlinewidth{1.003750pt}%
\definecolor{currentstroke}{rgb}{0.968203,0.720844,0.612293}%
\pgfsetstrokecolor{currentstroke}%
\pgfsetdash{}{0pt}%
\pgfpathmoveto{\pgfqpoint{3.562203in}{1.210802in}}%
\pgfpathlineto{\pgfqpoint{3.562203in}{1.210802in}}%
\pgfusepath{stroke}%
\end{pgfscope}%
\begin{pgfscope}%
\pgfpathrectangle{\pgfqpoint{0.552773in}{0.431673in}}{\pgfqpoint{3.738807in}{1.765244in}}%
\pgfusepath{clip}%
\pgfsetbuttcap%
\pgfsetroundjoin%
\pgfsetlinewidth{1.003750pt}%
\definecolor{currentstroke}{rgb}{0.968203,0.720844,0.612293}%
\pgfsetstrokecolor{currentstroke}%
\pgfsetdash{}{0pt}%
\pgfpathmoveto{\pgfqpoint{3.586856in}{1.189444in}}%
\pgfpathlineto{\pgfqpoint{3.586856in}{1.189444in}}%
\pgfusepath{stroke}%
\end{pgfscope}%
\begin{pgfscope}%
\pgfpathrectangle{\pgfqpoint{0.552773in}{0.431673in}}{\pgfqpoint{3.738807in}{1.765244in}}%
\pgfusepath{clip}%
\pgfsetbuttcap%
\pgfsetroundjoin%
\pgfsetlinewidth{1.003750pt}%
\definecolor{currentstroke}{rgb}{0.968203,0.720844,0.612293}%
\pgfsetstrokecolor{currentstroke}%
\pgfsetdash{}{0pt}%
\pgfpathmoveto{\pgfqpoint{3.629765in}{1.196622in}}%
\pgfpathlineto{\pgfqpoint{3.629765in}{1.196622in}}%
\pgfusepath{stroke}%
\end{pgfscope}%
\begin{pgfscope}%
\pgfpathrectangle{\pgfqpoint{0.552773in}{0.431673in}}{\pgfqpoint{3.738807in}{1.765244in}}%
\pgfusepath{clip}%
\pgfsetbuttcap%
\pgfsetroundjoin%
\pgfsetlinewidth{1.003750pt}%
\definecolor{currentstroke}{rgb}{0.968203,0.720844,0.612293}%
\pgfsetstrokecolor{currentstroke}%
\pgfsetdash{}{0pt}%
\pgfpathmoveto{\pgfqpoint{3.676730in}{1.210141in}}%
\pgfpathlineto{\pgfqpoint{3.676730in}{1.210141in}}%
\pgfusepath{stroke}%
\end{pgfscope}%
\begin{pgfscope}%
\pgfpathrectangle{\pgfqpoint{0.552773in}{0.431673in}}{\pgfqpoint{3.738807in}{1.765244in}}%
\pgfusepath{clip}%
\pgfsetbuttcap%
\pgfsetroundjoin%
\pgfsetlinewidth{1.003750pt}%
\definecolor{currentstroke}{rgb}{0.968203,0.720844,0.612293}%
\pgfsetstrokecolor{currentstroke}%
\pgfsetdash{}{0pt}%
\pgfpathmoveto{\pgfqpoint{3.745250in}{1.257352in}}%
\pgfpathlineto{\pgfqpoint{3.745250in}{1.257352in}}%
\pgfusepath{stroke}%
\end{pgfscope}%
\begin{pgfscope}%
\pgfpathrectangle{\pgfqpoint{0.552773in}{0.431673in}}{\pgfqpoint{3.738807in}{1.765244in}}%
\pgfusepath{clip}%
\pgfsetbuttcap%
\pgfsetroundjoin%
\pgfsetlinewidth{1.003750pt}%
\definecolor{currentstroke}{rgb}{0.968203,0.720844,0.612293}%
\pgfsetstrokecolor{currentstroke}%
\pgfsetdash{}{0pt}%
\pgfpathmoveto{\pgfqpoint{3.816743in}{1.309211in}}%
\pgfpathlineto{\pgfqpoint{3.816743in}{1.309211in}}%
\pgfusepath{stroke}%
\end{pgfscope}%
\begin{pgfscope}%
\pgfpathrectangle{\pgfqpoint{0.552773in}{0.431673in}}{\pgfqpoint{3.738807in}{1.765244in}}%
\pgfusepath{clip}%
\pgfsetbuttcap%
\pgfsetroundjoin%
\pgfsetlinewidth{1.003750pt}%
\definecolor{currentstroke}{rgb}{0.968203,0.720844,0.612293}%
\pgfsetstrokecolor{currentstroke}%
\pgfsetdash{}{0pt}%
\pgfpathmoveto{\pgfqpoint{3.871459in}{1.334845in}}%
\pgfpathlineto{\pgfqpoint{3.871459in}{1.334845in}}%
\pgfusepath{stroke}%
\end{pgfscope}%
\begin{pgfscope}%
\pgfpathrectangle{\pgfqpoint{0.552773in}{0.431673in}}{\pgfqpoint{3.738807in}{1.765244in}}%
\pgfusepath{clip}%
\pgfsetbuttcap%
\pgfsetroundjoin%
\pgfsetlinewidth{1.003750pt}%
\definecolor{currentstroke}{rgb}{0.968203,0.720844,0.612293}%
\pgfsetstrokecolor{currentstroke}%
\pgfsetdash{}{0pt}%
\pgfpathmoveto{\pgfqpoint{3.950363in}{1.398288in}}%
\pgfpathlineto{\pgfqpoint{3.950363in}{1.398288in}}%
\pgfusepath{stroke}%
\end{pgfscope}%
\begin{pgfscope}%
\pgfpathrectangle{\pgfqpoint{0.552773in}{0.431673in}}{\pgfqpoint{3.738807in}{1.765244in}}%
\pgfusepath{clip}%
\pgfsetbuttcap%
\pgfsetroundjoin%
\pgfsetlinewidth{1.003750pt}%
\definecolor{currentstroke}{rgb}{0.968203,0.720844,0.612293}%
\pgfsetstrokecolor{currentstroke}%
\pgfsetdash{}{0pt}%
\pgfpathmoveto{\pgfqpoint{3.993573in}{1.405936in}}%
\pgfpathlineto{\pgfqpoint{3.993573in}{1.405936in}}%
\pgfusepath{stroke}%
\end{pgfscope}%
\begin{pgfscope}%
\pgfpathrectangle{\pgfqpoint{0.552773in}{0.431673in}}{\pgfqpoint{3.738807in}{1.765244in}}%
\pgfusepath{clip}%
\pgfsetbuttcap%
\pgfsetroundjoin%
\pgfsetlinewidth{1.003750pt}%
\definecolor{currentstroke}{rgb}{0.968203,0.720844,0.612293}%
\pgfsetstrokecolor{currentstroke}%
\pgfsetdash{}{0pt}%
\pgfpathmoveto{\pgfqpoint{1.234882in}{1.346212in}}%
\pgfpathlineto{\pgfqpoint{1.234882in}{1.346212in}}%
\pgfusepath{stroke}%
\end{pgfscope}%
\begin{pgfscope}%
\pgfpathrectangle{\pgfqpoint{0.552773in}{0.431673in}}{\pgfqpoint{3.738807in}{1.765244in}}%
\pgfusepath{clip}%
\pgfsetbuttcap%
\pgfsetroundjoin%
\pgfsetlinewidth{1.003750pt}%
\definecolor{currentstroke}{rgb}{0.968203,0.720844,0.612293}%
\pgfsetstrokecolor{currentstroke}%
\pgfsetdash{}{0pt}%
\pgfpathmoveto{\pgfqpoint{1.295052in}{1.380370in}}%
\pgfpathlineto{\pgfqpoint{1.295052in}{1.380370in}}%
\pgfusepath{stroke}%
\end{pgfscope}%
\begin{pgfscope}%
\pgfpathrectangle{\pgfqpoint{0.552773in}{0.431673in}}{\pgfqpoint{3.738807in}{1.765244in}}%
\pgfusepath{clip}%
\pgfsetbuttcap%
\pgfsetroundjoin%
\pgfsetlinewidth{1.003750pt}%
\definecolor{currentstroke}{rgb}{0.968203,0.720844,0.612293}%
\pgfsetstrokecolor{currentstroke}%
\pgfsetdash{}{0pt}%
\pgfpathmoveto{\pgfqpoint{1.333712in}{1.380908in}}%
\pgfpathlineto{\pgfqpoint{1.333712in}{1.380908in}}%
\pgfusepath{stroke}%
\end{pgfscope}%
\begin{pgfscope}%
\pgfpathrectangle{\pgfqpoint{0.552773in}{0.431673in}}{\pgfqpoint{3.738807in}{1.765244in}}%
\pgfusepath{clip}%
\pgfsetbuttcap%
\pgfsetroundjoin%
\pgfsetlinewidth{1.003750pt}%
\definecolor{currentstroke}{rgb}{0.968203,0.720844,0.612293}%
\pgfsetstrokecolor{currentstroke}%
\pgfsetdash{}{0pt}%
\pgfpathmoveto{\pgfqpoint{1.360563in}{1.362986in}}%
\pgfpathlineto{\pgfqpoint{1.360563in}{1.362986in}}%
\pgfusepath{stroke}%
\end{pgfscope}%
\begin{pgfscope}%
\pgfpathrectangle{\pgfqpoint{0.552773in}{0.431673in}}{\pgfqpoint{3.738807in}{1.765244in}}%
\pgfusepath{clip}%
\pgfsetbuttcap%
\pgfsetroundjoin%
\pgfsetlinewidth{1.003750pt}%
\definecolor{currentstroke}{rgb}{0.968203,0.720844,0.612293}%
\pgfsetstrokecolor{currentstroke}%
\pgfsetdash{}{0pt}%
\pgfpathmoveto{\pgfqpoint{1.364476in}{1.309209in}}%
\pgfpathlineto{\pgfqpoint{1.364476in}{1.309209in}}%
\pgfusepath{stroke}%
\end{pgfscope}%
\begin{pgfscope}%
\pgfpathrectangle{\pgfqpoint{0.552773in}{0.431673in}}{\pgfqpoint{3.738807in}{1.765244in}}%
\pgfusepath{clip}%
\pgfsetbuttcap%
\pgfsetroundjoin%
\pgfsetlinewidth{1.003750pt}%
\definecolor{currentstroke}{rgb}{0.968203,0.720844,0.612293}%
\pgfsetstrokecolor{currentstroke}%
\pgfsetdash{}{0pt}%
\pgfpathmoveto{\pgfqpoint{1.369782in}{1.257608in}}%
\pgfpathlineto{\pgfqpoint{1.369782in}{1.257608in}}%
\pgfusepath{stroke}%
\end{pgfscope}%
\begin{pgfscope}%
\pgfpathrectangle{\pgfqpoint{0.552773in}{0.431673in}}{\pgfqpoint{3.738807in}{1.765244in}}%
\pgfusepath{clip}%
\pgfsetbuttcap%
\pgfsetroundjoin%
\pgfsetlinewidth{1.003750pt}%
\definecolor{currentstroke}{rgb}{0.968203,0.720844,0.612293}%
\pgfsetstrokecolor{currentstroke}%
\pgfsetdash{}{0pt}%
\pgfpathmoveto{\pgfqpoint{1.424566in}{1.283350in}}%
\pgfpathlineto{\pgfqpoint{1.424566in}{1.283350in}}%
\pgfusepath{stroke}%
\end{pgfscope}%
\begin{pgfscope}%
\pgfpathrectangle{\pgfqpoint{0.552773in}{0.431673in}}{\pgfqpoint{3.738807in}{1.765244in}}%
\pgfusepath{clip}%
\pgfsetbuttcap%
\pgfsetroundjoin%
\pgfsetlinewidth{1.003750pt}%
\definecolor{currentstroke}{rgb}{0.968203,0.720844,0.612293}%
\pgfsetstrokecolor{currentstroke}%
\pgfsetdash{}{0pt}%
\pgfpathmoveto{\pgfqpoint{1.443716in}{1.253390in}}%
\pgfpathlineto{\pgfqpoint{1.443716in}{1.253390in}}%
\pgfusepath{stroke}%
\end{pgfscope}%
\begin{pgfscope}%
\pgfpathrectangle{\pgfqpoint{0.552773in}{0.431673in}}{\pgfqpoint{3.738807in}{1.765244in}}%
\pgfusepath{clip}%
\pgfsetbuttcap%
\pgfsetroundjoin%
\pgfsetlinewidth{1.003750pt}%
\definecolor{currentstroke}{rgb}{0.968203,0.720844,0.612293}%
\pgfsetstrokecolor{currentstroke}%
\pgfsetdash{}{0pt}%
\pgfpathmoveto{\pgfqpoint{1.466969in}{1.229844in}}%
\pgfpathlineto{\pgfqpoint{1.466969in}{1.229844in}}%
\pgfusepath{stroke}%
\end{pgfscope}%
\begin{pgfscope}%
\pgfpathrectangle{\pgfqpoint{0.552773in}{0.431673in}}{\pgfqpoint{3.738807in}{1.765244in}}%
\pgfusepath{clip}%
\pgfsetbuttcap%
\pgfsetroundjoin%
\pgfsetlinewidth{1.003750pt}%
\definecolor{currentstroke}{rgb}{0.968203,0.720844,0.612293}%
\pgfsetstrokecolor{currentstroke}%
\pgfsetdash{}{0pt}%
\pgfpathmoveto{\pgfqpoint{1.490856in}{1.207288in}}%
\pgfpathlineto{\pgfqpoint{1.490856in}{1.207288in}}%
\pgfusepath{stroke}%
\end{pgfscope}%
\begin{pgfscope}%
\pgfpathrectangle{\pgfqpoint{0.552773in}{0.431673in}}{\pgfqpoint{3.738807in}{1.765244in}}%
\pgfusepath{clip}%
\pgfsetbuttcap%
\pgfsetroundjoin%
\pgfsetlinewidth{1.003750pt}%
\definecolor{currentstroke}{rgb}{0.968203,0.720844,0.612293}%
\pgfsetstrokecolor{currentstroke}%
\pgfsetdash{}{0pt}%
\pgfpathmoveto{\pgfqpoint{1.503289in}{1.166829in}}%
\pgfpathlineto{\pgfqpoint{1.503289in}{1.166829in}}%
\pgfusepath{stroke}%
\end{pgfscope}%
\begin{pgfscope}%
\pgfpathrectangle{\pgfqpoint{0.552773in}{0.431673in}}{\pgfqpoint{3.738807in}{1.765244in}}%
\pgfusepath{clip}%
\pgfsetbuttcap%
\pgfsetroundjoin%
\pgfsetlinewidth{1.003750pt}%
\definecolor{currentstroke}{rgb}{0.968203,0.720844,0.612293}%
\pgfsetstrokecolor{currentstroke}%
\pgfsetdash{}{0pt}%
\pgfpathmoveto{\pgfqpoint{1.551438in}{1.182198in}}%
\pgfpathlineto{\pgfqpoint{1.551438in}{1.182198in}}%
\pgfusepath{stroke}%
\end{pgfscope}%
\begin{pgfscope}%
\pgfpathrectangle{\pgfqpoint{0.552773in}{0.431673in}}{\pgfqpoint{3.738807in}{1.765244in}}%
\pgfusepath{clip}%
\pgfsetbuttcap%
\pgfsetroundjoin%
\pgfsetlinewidth{1.003750pt}%
\definecolor{currentstroke}{rgb}{0.968203,0.720844,0.612293}%
\pgfsetstrokecolor{currentstroke}%
\pgfsetdash{}{0pt}%
\pgfpathmoveto{\pgfqpoint{1.607593in}{1.210082in}}%
\pgfpathlineto{\pgfqpoint{1.607593in}{1.210082in}}%
\pgfusepath{stroke}%
\end{pgfscope}%
\begin{pgfscope}%
\pgfpathrectangle{\pgfqpoint{0.552773in}{0.431673in}}{\pgfqpoint{3.738807in}{1.765244in}}%
\pgfusepath{clip}%
\pgfsetbuttcap%
\pgfsetroundjoin%
\pgfsetlinewidth{1.003750pt}%
\definecolor{currentstroke}{rgb}{0.968203,0.720844,0.612293}%
\pgfsetstrokecolor{currentstroke}%
\pgfsetdash{}{0pt}%
\pgfpathmoveto{\pgfqpoint{1.658864in}{1.230331in}}%
\pgfpathlineto{\pgfqpoint{1.658864in}{1.230331in}}%
\pgfusepath{stroke}%
\end{pgfscope}%
\begin{pgfscope}%
\pgfpathrectangle{\pgfqpoint{0.552773in}{0.431673in}}{\pgfqpoint{3.738807in}{1.765244in}}%
\pgfusepath{clip}%
\pgfsetbuttcap%
\pgfsetroundjoin%
\pgfsetlinewidth{1.003750pt}%
\definecolor{currentstroke}{rgb}{0.968203,0.720844,0.612293}%
\pgfsetstrokecolor{currentstroke}%
\pgfsetdash{}{0pt}%
\pgfpathmoveto{\pgfqpoint{1.722476in}{1.269870in}}%
\pgfpathlineto{\pgfqpoint{1.722476in}{1.269870in}}%
\pgfusepath{stroke}%
\end{pgfscope}%
\begin{pgfscope}%
\pgfpathrectangle{\pgfqpoint{0.552773in}{0.431673in}}{\pgfqpoint{3.738807in}{1.765244in}}%
\pgfusepath{clip}%
\pgfsetbuttcap%
\pgfsetroundjoin%
\pgfsetlinewidth{1.003750pt}%
\definecolor{currentstroke}{rgb}{0.968203,0.720844,0.612293}%
\pgfsetstrokecolor{currentstroke}%
\pgfsetdash{}{0pt}%
\pgfpathmoveto{\pgfqpoint{1.797820in}{1.327748in}}%
\pgfpathlineto{\pgfqpoint{1.797820in}{1.327748in}}%
\pgfusepath{stroke}%
\end{pgfscope}%
\begin{pgfscope}%
\pgfpathrectangle{\pgfqpoint{0.552773in}{0.431673in}}{\pgfqpoint{3.738807in}{1.765244in}}%
\pgfusepath{clip}%
\pgfsetbuttcap%
\pgfsetroundjoin%
\pgfsetlinewidth{1.003750pt}%
\definecolor{currentstroke}{rgb}{0.968203,0.720844,0.612293}%
\pgfsetstrokecolor{currentstroke}%
\pgfsetdash{}{0pt}%
\pgfpathmoveto{\pgfqpoint{1.877516in}{1.392430in}}%
\pgfpathlineto{\pgfqpoint{1.877516in}{1.392430in}}%
\pgfusepath{stroke}%
\end{pgfscope}%
\begin{pgfscope}%
\pgfpathrectangle{\pgfqpoint{0.552773in}{0.431673in}}{\pgfqpoint{3.738807in}{1.765244in}}%
\pgfusepath{clip}%
\pgfsetbuttcap%
\pgfsetroundjoin%
\pgfsetlinewidth{1.003750pt}%
\definecolor{currentstroke}{rgb}{0.968203,0.720844,0.612293}%
\pgfsetstrokecolor{currentstroke}%
\pgfsetdash{}{0pt}%
\pgfpathmoveto{\pgfqpoint{1.954391in}{1.452701in}}%
\pgfpathlineto{\pgfqpoint{1.954391in}{1.452701in}}%
\pgfusepath{stroke}%
\end{pgfscope}%
\begin{pgfscope}%
\pgfpathrectangle{\pgfqpoint{0.552773in}{0.431673in}}{\pgfqpoint{3.738807in}{1.765244in}}%
\pgfusepath{clip}%
\pgfsetbuttcap%
\pgfsetroundjoin%
\pgfsetlinewidth{1.003750pt}%
\definecolor{currentstroke}{rgb}{0.968203,0.720844,0.612293}%
\pgfsetstrokecolor{currentstroke}%
\pgfsetdash{}{0pt}%
\pgfpathmoveto{\pgfqpoint{2.014562in}{1.486861in}}%
\pgfpathlineto{\pgfqpoint{2.014562in}{1.486861in}}%
\pgfusepath{stroke}%
\end{pgfscope}%
\begin{pgfscope}%
\pgfpathrectangle{\pgfqpoint{0.552773in}{0.431673in}}{\pgfqpoint{3.738807in}{1.765244in}}%
\pgfusepath{clip}%
\pgfsetbuttcap%
\pgfsetroundjoin%
\pgfsetlinewidth{1.003750pt}%
\definecolor{currentstroke}{rgb}{0.968203,0.720844,0.612293}%
\pgfsetstrokecolor{currentstroke}%
\pgfsetdash{}{0pt}%
\pgfpathmoveto{\pgfqpoint{2.061371in}{1.500136in}}%
\pgfpathlineto{\pgfqpoint{2.061371in}{1.500136in}}%
\pgfusepath{stroke}%
\end{pgfscope}%
\begin{pgfscope}%
\pgfpathrectangle{\pgfqpoint{0.552773in}{0.431673in}}{\pgfqpoint{3.738807in}{1.765244in}}%
\pgfusepath{clip}%
\pgfsetbuttcap%
\pgfsetroundjoin%
\pgfsetlinewidth{1.003750pt}%
\definecolor{currentstroke}{rgb}{0.968203,0.720844,0.612293}%
\pgfsetstrokecolor{currentstroke}%
\pgfsetdash{}{0pt}%
\pgfpathmoveto{\pgfqpoint{2.083864in}{1.475402in}}%
\pgfpathlineto{\pgfqpoint{2.083864in}{1.475402in}}%
\pgfusepath{stroke}%
\end{pgfscope}%
\begin{pgfscope}%
\pgfpathrectangle{\pgfqpoint{0.552773in}{0.431673in}}{\pgfqpoint{3.738807in}{1.765244in}}%
\pgfusepath{clip}%
\pgfsetbuttcap%
\pgfsetroundjoin%
\pgfsetlinewidth{1.003750pt}%
\definecolor{currentstroke}{rgb}{0.968203,0.720844,0.612293}%
\pgfsetstrokecolor{currentstroke}%
\pgfsetdash{}{0pt}%
\pgfpathmoveto{\pgfqpoint{2.105964in}{1.450053in}}%
\pgfpathlineto{\pgfqpoint{2.105964in}{1.450053in}}%
\pgfusepath{stroke}%
\end{pgfscope}%
\begin{pgfscope}%
\pgfpathrectangle{\pgfqpoint{0.552773in}{0.431673in}}{\pgfqpoint{3.738807in}{1.765244in}}%
\pgfusepath{clip}%
\pgfsetbuttcap%
\pgfsetroundjoin%
\pgfsetlinewidth{1.003750pt}%
\definecolor{currentstroke}{rgb}{0.968203,0.720844,0.612293}%
\pgfsetstrokecolor{currentstroke}%
\pgfsetdash{}{0pt}%
\pgfpathmoveto{\pgfqpoint{2.116436in}{1.406528in}}%
\pgfpathlineto{\pgfqpoint{2.116436in}{1.406528in}}%
\pgfusepath{stroke}%
\end{pgfscope}%
\begin{pgfscope}%
\pgfpathrectangle{\pgfqpoint{0.552773in}{0.431673in}}{\pgfqpoint{3.738807in}{1.765244in}}%
\pgfusepath{clip}%
\pgfsetbuttcap%
\pgfsetroundjoin%
\pgfsetlinewidth{1.003750pt}%
\definecolor{currentstroke}{rgb}{0.968203,0.720844,0.612293}%
\pgfsetstrokecolor{currentstroke}%
\pgfsetdash{}{0pt}%
\pgfpathmoveto{\pgfqpoint{2.141673in}{1.386083in}}%
\pgfpathlineto{\pgfqpoint{2.141673in}{1.386083in}}%
\pgfusepath{stroke}%
\end{pgfscope}%
\begin{pgfscope}%
\pgfpathrectangle{\pgfqpoint{0.552773in}{0.431673in}}{\pgfqpoint{3.738807in}{1.765244in}}%
\pgfusepath{clip}%
\pgfsetbuttcap%
\pgfsetroundjoin%
\pgfsetlinewidth{1.003750pt}%
\definecolor{currentstroke}{rgb}{0.968203,0.720844,0.612293}%
\pgfsetstrokecolor{currentstroke}%
\pgfsetdash{}{0pt}%
\pgfpathmoveto{\pgfqpoint{2.156766in}{1.349782in}}%
\pgfpathlineto{\pgfqpoint{2.156766in}{1.349782in}}%
\pgfusepath{stroke}%
\end{pgfscope}%
\begin{pgfscope}%
\pgfpathrectangle{\pgfqpoint{0.552773in}{0.431673in}}{\pgfqpoint{3.738807in}{1.765244in}}%
\pgfusepath{clip}%
\pgfsetbuttcap%
\pgfsetroundjoin%
\pgfsetlinewidth{1.003750pt}%
\definecolor{currentstroke}{rgb}{0.968203,0.720844,0.612293}%
\pgfsetstrokecolor{currentstroke}%
\pgfsetdash{}{0pt}%
\pgfpathmoveto{\pgfqpoint{2.171473in}{1.312877in}}%
\pgfpathlineto{\pgfqpoint{2.171473in}{1.312877in}}%
\pgfusepath{stroke}%
\end{pgfscope}%
\begin{pgfscope}%
\pgfpathrectangle{\pgfqpoint{0.552773in}{0.431673in}}{\pgfqpoint{3.738807in}{1.765244in}}%
\pgfusepath{clip}%
\pgfsetbuttcap%
\pgfsetroundjoin%
\pgfsetlinewidth{1.003750pt}%
\definecolor{currentstroke}{rgb}{0.968203,0.720844,0.612293}%
\pgfsetstrokecolor{currentstroke}%
\pgfsetdash{}{0pt}%
\pgfpathmoveto{\pgfqpoint{2.192191in}{1.285369in}}%
\pgfpathlineto{\pgfqpoint{2.192191in}{1.285369in}}%
\pgfusepath{stroke}%
\end{pgfscope}%
\begin{pgfscope}%
\pgfpathrectangle{\pgfqpoint{0.552773in}{0.431673in}}{\pgfqpoint{3.738807in}{1.765244in}}%
\pgfusepath{clip}%
\pgfsetbuttcap%
\pgfsetroundjoin%
\pgfsetlinewidth{1.003750pt}%
\definecolor{currentstroke}{rgb}{0.968203,0.720844,0.612293}%
\pgfsetstrokecolor{currentstroke}%
\pgfsetdash{}{0pt}%
\pgfpathmoveto{\pgfqpoint{2.223452in}{1.274341in}}%
\pgfpathlineto{\pgfqpoint{2.223452in}{1.274341in}}%
\pgfusepath{stroke}%
\end{pgfscope}%
\begin{pgfscope}%
\pgfpathrectangle{\pgfqpoint{0.552773in}{0.431673in}}{\pgfqpoint{3.738807in}{1.765244in}}%
\pgfusepath{clip}%
\pgfsetbuttcap%
\pgfsetroundjoin%
\pgfsetlinewidth{1.003750pt}%
\definecolor{currentstroke}{rgb}{0.968203,0.720844,0.612293}%
\pgfsetstrokecolor{currentstroke}%
\pgfsetdash{}{0pt}%
\pgfpathmoveto{\pgfqpoint{2.256793in}{1.266562in}}%
\pgfpathlineto{\pgfqpoint{2.256793in}{1.266562in}}%
\pgfusepath{stroke}%
\end{pgfscope}%
\begin{pgfscope}%
\pgfpathrectangle{\pgfqpoint{0.552773in}{0.431673in}}{\pgfqpoint{3.738807in}{1.765244in}}%
\pgfusepath{clip}%
\pgfsetbuttcap%
\pgfsetroundjoin%
\pgfsetlinewidth{1.003750pt}%
\definecolor{currentstroke}{rgb}{0.968203,0.720844,0.612293}%
\pgfsetstrokecolor{currentstroke}%
\pgfsetdash{}{0pt}%
\pgfpathmoveto{\pgfqpoint{2.298519in}{1.271893in}}%
\pgfpathlineto{\pgfqpoint{2.298519in}{1.271893in}}%
\pgfusepath{stroke}%
\end{pgfscope}%
\begin{pgfscope}%
\pgfpathrectangle{\pgfqpoint{0.552773in}{0.431673in}}{\pgfqpoint{3.738807in}{1.765244in}}%
\pgfusepath{clip}%
\pgfsetbuttcap%
\pgfsetroundjoin%
\pgfsetlinewidth{1.003750pt}%
\definecolor{currentstroke}{rgb}{0.968203,0.720844,0.612293}%
\pgfsetstrokecolor{currentstroke}%
\pgfsetdash{}{0pt}%
\pgfpathmoveto{\pgfqpoint{2.360130in}{1.308304in}}%
\pgfpathlineto{\pgfqpoint{2.360130in}{1.308304in}}%
\pgfusepath{stroke}%
\end{pgfscope}%
\begin{pgfscope}%
\pgfpathrectangle{\pgfqpoint{0.552773in}{0.431673in}}{\pgfqpoint{3.738807in}{1.765244in}}%
\pgfusepath{clip}%
\pgfsetbuttcap%
\pgfsetroundjoin%
\pgfsetlinewidth{1.003750pt}%
\definecolor{currentstroke}{rgb}{0.968203,0.720844,0.612293}%
\pgfsetstrokecolor{currentstroke}%
\pgfsetdash{}{0pt}%
\pgfpathmoveto{\pgfqpoint{2.402956in}{1.315352in}}%
\pgfpathlineto{\pgfqpoint{2.402956in}{1.315352in}}%
\pgfusepath{stroke}%
\end{pgfscope}%
\begin{pgfscope}%
\pgfpathrectangle{\pgfqpoint{0.552773in}{0.431673in}}{\pgfqpoint{3.738807in}{1.765244in}}%
\pgfusepath{clip}%
\pgfsetbuttcap%
\pgfsetroundjoin%
\pgfsetlinewidth{1.003750pt}%
\definecolor{currentstroke}{rgb}{0.968203,0.720844,0.612293}%
\pgfsetstrokecolor{currentstroke}%
\pgfsetdash{}{0pt}%
\pgfpathmoveto{\pgfqpoint{2.474900in}{1.367916in}}%
\pgfpathlineto{\pgfqpoint{2.474900in}{1.367916in}}%
\pgfusepath{stroke}%
\end{pgfscope}%
\begin{pgfscope}%
\pgfpathrectangle{\pgfqpoint{0.552773in}{0.431673in}}{\pgfqpoint{3.738807in}{1.765244in}}%
\pgfusepath{clip}%
\pgfsetbuttcap%
\pgfsetroundjoin%
\pgfsetlinewidth{1.003750pt}%
\definecolor{currentstroke}{rgb}{0.968203,0.720844,0.612293}%
\pgfsetstrokecolor{currentstroke}%
\pgfsetdash{}{0pt}%
\pgfpathmoveto{\pgfqpoint{2.549428in}{1.424519in}}%
\pgfpathlineto{\pgfqpoint{2.549428in}{1.424519in}}%
\pgfusepath{stroke}%
\end{pgfscope}%
\begin{pgfscope}%
\pgfpathrectangle{\pgfqpoint{0.552773in}{0.431673in}}{\pgfqpoint{3.738807in}{1.765244in}}%
\pgfusepath{clip}%
\pgfsetbuttcap%
\pgfsetroundjoin%
\pgfsetlinewidth{1.003750pt}%
\definecolor{currentstroke}{rgb}{0.968203,0.720844,0.612293}%
\pgfsetstrokecolor{currentstroke}%
\pgfsetdash{}{0pt}%
\pgfpathmoveto{\pgfqpoint{2.619369in}{1.473950in}}%
\pgfpathlineto{\pgfqpoint{2.619369in}{1.473950in}}%
\pgfusepath{stroke}%
\end{pgfscope}%
\begin{pgfscope}%
\pgfpathrectangle{\pgfqpoint{0.552773in}{0.431673in}}{\pgfqpoint{3.738807in}{1.765244in}}%
\pgfusepath{clip}%
\pgfsetbuttcap%
\pgfsetroundjoin%
\pgfsetlinewidth{1.003750pt}%
\definecolor{currentstroke}{rgb}{0.968203,0.720844,0.612293}%
\pgfsetstrokecolor{currentstroke}%
\pgfsetdash{}{0pt}%
\pgfpathmoveto{\pgfqpoint{2.688119in}{1.521522in}}%
\pgfpathlineto{\pgfqpoint{2.688119in}{1.521522in}}%
\pgfusepath{stroke}%
\end{pgfscope}%
\begin{pgfscope}%
\pgfpathrectangle{\pgfqpoint{0.552773in}{0.431673in}}{\pgfqpoint{3.738807in}{1.765244in}}%
\pgfusepath{clip}%
\pgfsetbuttcap%
\pgfsetroundjoin%
\pgfsetlinewidth{1.003750pt}%
\definecolor{currentstroke}{rgb}{0.968203,0.720844,0.612293}%
\pgfsetstrokecolor{currentstroke}%
\pgfsetdash{}{0pt}%
\pgfpathmoveto{\pgfqpoint{2.745485in}{1.551298in}}%
\pgfpathlineto{\pgfqpoint{2.745485in}{1.551298in}}%
\pgfusepath{stroke}%
\end{pgfscope}%
\begin{pgfscope}%
\pgfpathrectangle{\pgfqpoint{0.552773in}{0.431673in}}{\pgfqpoint{3.738807in}{1.765244in}}%
\pgfusepath{clip}%
\pgfsetbuttcap%
\pgfsetroundjoin%
\pgfsetlinewidth{1.003750pt}%
\definecolor{currentstroke}{rgb}{0.968203,0.720844,0.612293}%
\pgfsetstrokecolor{currentstroke}%
\pgfsetdash{}{0pt}%
\pgfpathmoveto{\pgfqpoint{2.770849in}{1.531052in}}%
\pgfpathlineto{\pgfqpoint{2.770849in}{1.531052in}}%
\pgfusepath{stroke}%
\end{pgfscope}%
\begin{pgfscope}%
\pgfpathrectangle{\pgfqpoint{0.552773in}{0.431673in}}{\pgfqpoint{3.738807in}{1.765244in}}%
\pgfusepath{clip}%
\pgfsetbuttcap%
\pgfsetroundjoin%
\pgfsetlinewidth{1.003750pt}%
\definecolor{currentstroke}{rgb}{0.968203,0.720844,0.612293}%
\pgfsetstrokecolor{currentstroke}%
\pgfsetdash{}{0pt}%
\pgfpathmoveto{\pgfqpoint{2.791502in}{1.503442in}}%
\pgfpathlineto{\pgfqpoint{2.791502in}{1.503442in}}%
\pgfusepath{stroke}%
\end{pgfscope}%
\begin{pgfscope}%
\pgfpathrectangle{\pgfqpoint{0.552773in}{0.431673in}}{\pgfqpoint{3.738807in}{1.765244in}}%
\pgfusepath{clip}%
\pgfsetbuttcap%
\pgfsetroundjoin%
\pgfsetlinewidth{1.003750pt}%
\definecolor{currentstroke}{rgb}{0.968203,0.720844,0.612293}%
\pgfsetstrokecolor{currentstroke}%
\pgfsetdash{}{0pt}%
\pgfpathmoveto{\pgfqpoint{2.801394in}{1.459011in}}%
\pgfpathlineto{\pgfqpoint{2.801394in}{1.459011in}}%
\pgfusepath{stroke}%
\end{pgfscope}%
\begin{pgfscope}%
\pgfpathrectangle{\pgfqpoint{0.552773in}{0.431673in}}{\pgfqpoint{3.738807in}{1.765244in}}%
\pgfusepath{clip}%
\pgfsetbuttcap%
\pgfsetroundjoin%
\pgfsetlinewidth{1.003750pt}%
\definecolor{currentstroke}{rgb}{0.968203,0.720844,0.612293}%
\pgfsetstrokecolor{currentstroke}%
\pgfsetdash{}{0pt}%
\pgfpathmoveto{\pgfqpoint{2.825889in}{1.437406in}}%
\pgfpathlineto{\pgfqpoint{2.825889in}{1.437406in}}%
\pgfusepath{stroke}%
\end{pgfscope}%
\begin{pgfscope}%
\pgfpathrectangle{\pgfqpoint{0.552773in}{0.431673in}}{\pgfqpoint{3.738807in}{1.765244in}}%
\pgfusepath{clip}%
\pgfsetbuttcap%
\pgfsetroundjoin%
\pgfsetlinewidth{1.003750pt}%
\definecolor{currentstroke}{rgb}{0.968203,0.720844,0.612293}%
\pgfsetstrokecolor{currentstroke}%
\pgfsetdash{}{0pt}%
\pgfpathmoveto{\pgfqpoint{2.823590in}{1.373918in}}%
\pgfpathlineto{\pgfqpoint{2.823590in}{1.373918in}}%
\pgfusepath{stroke}%
\end{pgfscope}%
\begin{pgfscope}%
\pgfpathrectangle{\pgfqpoint{0.552773in}{0.431673in}}{\pgfqpoint{3.738807in}{1.765244in}}%
\pgfusepath{clip}%
\pgfsetbuttcap%
\pgfsetroundjoin%
\pgfsetlinewidth{1.003750pt}%
\definecolor{currentstroke}{rgb}{0.968203,0.720844,0.612293}%
\pgfsetstrokecolor{currentstroke}%
\pgfsetdash{}{0pt}%
\pgfpathmoveto{\pgfqpoint{2.848428in}{1.352850in}}%
\pgfpathlineto{\pgfqpoint{2.848428in}{1.352850in}}%
\pgfusepath{stroke}%
\end{pgfscope}%
\begin{pgfscope}%
\pgfpathrectangle{\pgfqpoint{0.552773in}{0.431673in}}{\pgfqpoint{3.738807in}{1.765244in}}%
\pgfusepath{clip}%
\pgfsetbuttcap%
\pgfsetroundjoin%
\pgfsetlinewidth{1.003750pt}%
\definecolor{currentstroke}{rgb}{0.968203,0.720844,0.612293}%
\pgfsetstrokecolor{currentstroke}%
\pgfsetdash{}{0pt}%
\pgfpathmoveto{\pgfqpoint{2.859935in}{1.310943in}}%
\pgfpathlineto{\pgfqpoint{2.859935in}{1.310943in}}%
\pgfusepath{stroke}%
\end{pgfscope}%
\begin{pgfscope}%
\pgfpathrectangle{\pgfqpoint{0.552773in}{0.431673in}}{\pgfqpoint{3.738807in}{1.765244in}}%
\pgfusepath{clip}%
\pgfsetbuttcap%
\pgfsetroundjoin%
\pgfsetlinewidth{1.003750pt}%
\definecolor{currentstroke}{rgb}{0.968203,0.720844,0.612293}%
\pgfsetstrokecolor{currentstroke}%
\pgfsetdash{}{0pt}%
\pgfpathmoveto{\pgfqpoint{2.877851in}{1.279055in}}%
\pgfpathlineto{\pgfqpoint{2.877851in}{1.279055in}}%
\pgfusepath{stroke}%
\end{pgfscope}%
\begin{pgfscope}%
\pgfpathrectangle{\pgfqpoint{0.552773in}{0.431673in}}{\pgfqpoint{3.738807in}{1.765244in}}%
\pgfusepath{clip}%
\pgfsetbuttcap%
\pgfsetroundjoin%
\pgfsetlinewidth{1.003750pt}%
\definecolor{currentstroke}{rgb}{0.968203,0.720844,0.612293}%
\pgfsetstrokecolor{currentstroke}%
\pgfsetdash{}{0pt}%
\pgfpathmoveto{\pgfqpoint{2.897792in}{1.250331in}}%
\pgfpathlineto{\pgfqpoint{2.897792in}{1.250331in}}%
\pgfusepath{stroke}%
\end{pgfscope}%
\begin{pgfscope}%
\pgfpathrectangle{\pgfqpoint{0.552773in}{0.431673in}}{\pgfqpoint{3.738807in}{1.765244in}}%
\pgfusepath{clip}%
\pgfsetbuttcap%
\pgfsetroundjoin%
\pgfsetlinewidth{1.003750pt}%
\definecolor{currentstroke}{rgb}{0.968203,0.720844,0.612293}%
\pgfsetstrokecolor{currentstroke}%
\pgfsetdash{}{0pt}%
\pgfpathmoveto{\pgfqpoint{2.906245in}{1.203652in}}%
\pgfpathlineto{\pgfqpoint{2.906245in}{1.203652in}}%
\pgfusepath{stroke}%
\end{pgfscope}%
\begin{pgfscope}%
\pgfpathrectangle{\pgfqpoint{0.552773in}{0.431673in}}{\pgfqpoint{3.738807in}{1.765244in}}%
\pgfusepath{clip}%
\pgfsetbuttcap%
\pgfsetroundjoin%
\pgfsetlinewidth{1.003750pt}%
\definecolor{currentstroke}{rgb}{0.968203,0.720844,0.612293}%
\pgfsetstrokecolor{currentstroke}%
\pgfsetdash{}{0pt}%
\pgfpathmoveto{\pgfqpoint{2.946444in}{1.206593in}}%
\pgfpathlineto{\pgfqpoint{2.946444in}{1.206593in}}%
\pgfusepath{stroke}%
\end{pgfscope}%
\begin{pgfscope}%
\pgfpathrectangle{\pgfqpoint{0.552773in}{0.431673in}}{\pgfqpoint{3.738807in}{1.765244in}}%
\pgfusepath{clip}%
\pgfsetbuttcap%
\pgfsetroundjoin%
\pgfsetlinewidth{1.003750pt}%
\definecolor{currentstroke}{rgb}{0.968203,0.720844,0.612293}%
\pgfsetstrokecolor{currentstroke}%
\pgfsetdash{}{0pt}%
\pgfpathmoveto{\pgfqpoint{3.013245in}{1.251118in}}%
\pgfpathlineto{\pgfqpoint{3.013245in}{1.251118in}}%
\pgfusepath{stroke}%
\end{pgfscope}%
\begin{pgfscope}%
\pgfpathrectangle{\pgfqpoint{0.552773in}{0.431673in}}{\pgfqpoint{3.738807in}{1.765244in}}%
\pgfusepath{clip}%
\pgfsetbuttcap%
\pgfsetroundjoin%
\pgfsetlinewidth{1.003750pt}%
\definecolor{currentstroke}{rgb}{0.968203,0.720844,0.612293}%
\pgfsetstrokecolor{currentstroke}%
\pgfsetdash{}{0pt}%
\pgfpathmoveto{\pgfqpoint{3.071632in}{1.282490in}}%
\pgfpathlineto{\pgfqpoint{3.071632in}{1.282490in}}%
\pgfusepath{stroke}%
\end{pgfscope}%
\begin{pgfscope}%
\pgfpathrectangle{\pgfqpoint{0.552773in}{0.431673in}}{\pgfqpoint{3.738807in}{1.765244in}}%
\pgfusepath{clip}%
\pgfsetbuttcap%
\pgfsetroundjoin%
\pgfsetlinewidth{1.003750pt}%
\definecolor{currentstroke}{rgb}{0.968203,0.720844,0.612293}%
\pgfsetstrokecolor{currentstroke}%
\pgfsetdash{}{0pt}%
\pgfpathmoveto{\pgfqpoint{3.138985in}{1.327877in}}%
\pgfpathlineto{\pgfqpoint{3.138985in}{1.327877in}}%
\pgfusepath{stroke}%
\end{pgfscope}%
\begin{pgfscope}%
\pgfpathrectangle{\pgfqpoint{0.552773in}{0.431673in}}{\pgfqpoint{3.738807in}{1.765244in}}%
\pgfusepath{clip}%
\pgfsetbuttcap%
\pgfsetroundjoin%
\pgfsetlinewidth{1.003750pt}%
\definecolor{currentstroke}{rgb}{0.968203,0.720844,0.612293}%
\pgfsetstrokecolor{currentstroke}%
\pgfsetdash{}{0pt}%
\pgfpathmoveto{\pgfqpoint{3.210642in}{1.379991in}}%
\pgfpathlineto{\pgfqpoint{3.210642in}{1.379991in}}%
\pgfusepath{stroke}%
\end{pgfscope}%
\begin{pgfscope}%
\pgfpathrectangle{\pgfqpoint{0.552773in}{0.431673in}}{\pgfqpoint{3.738807in}{1.765244in}}%
\pgfusepath{clip}%
\pgfsetbuttcap%
\pgfsetroundjoin%
\pgfsetlinewidth{1.003750pt}%
\definecolor{currentstroke}{rgb}{0.968203,0.720844,0.612293}%
\pgfsetstrokecolor{currentstroke}%
\pgfsetdash{}{0pt}%
\pgfpathmoveto{\pgfqpoint{3.288346in}{1.441559in}}%
\pgfpathlineto{\pgfqpoint{3.288346in}{1.441559in}}%
\pgfusepath{stroke}%
\end{pgfscope}%
\begin{pgfscope}%
\pgfpathrectangle{\pgfqpoint{0.552773in}{0.431673in}}{\pgfqpoint{3.738807in}{1.765244in}}%
\pgfusepath{clip}%
\pgfsetbuttcap%
\pgfsetroundjoin%
\pgfsetlinewidth{1.003750pt}%
\definecolor{currentstroke}{rgb}{0.968203,0.720844,0.612293}%
\pgfsetstrokecolor{currentstroke}%
\pgfsetdash{}{0pt}%
\pgfpathmoveto{\pgfqpoint{3.354079in}{1.484414in}}%
\pgfpathlineto{\pgfqpoint{3.354079in}{1.484414in}}%
\pgfusepath{stroke}%
\end{pgfscope}%
\begin{pgfscope}%
\pgfpathrectangle{\pgfqpoint{0.552773in}{0.431673in}}{\pgfqpoint{3.738807in}{1.765244in}}%
\pgfusepath{clip}%
\pgfsetbuttcap%
\pgfsetroundjoin%
\pgfsetlinewidth{1.003750pt}%
\definecolor{currentstroke}{rgb}{0.968203,0.720844,0.612293}%
\pgfsetstrokecolor{currentstroke}%
\pgfsetdash{}{0pt}%
\pgfpathmoveto{\pgfqpoint{3.395162in}{1.488738in}}%
\pgfpathlineto{\pgfqpoint{3.395162in}{1.488738in}}%
\pgfusepath{stroke}%
\end{pgfscope}%
\begin{pgfscope}%
\pgfpathrectangle{\pgfqpoint{0.552773in}{0.431673in}}{\pgfqpoint{3.738807in}{1.765244in}}%
\pgfusepath{clip}%
\pgfsetbuttcap%
\pgfsetroundjoin%
\pgfsetlinewidth{1.003750pt}%
\definecolor{currentstroke}{rgb}{0.968203,0.720844,0.612293}%
\pgfsetstrokecolor{currentstroke}%
\pgfsetdash{}{0pt}%
\pgfpathmoveto{\pgfqpoint{3.423246in}{1.472743in}}%
\pgfpathlineto{\pgfqpoint{3.423246in}{1.472743in}}%
\pgfusepath{stroke}%
\end{pgfscope}%
\begin{pgfscope}%
\pgfpathrectangle{\pgfqpoint{0.552773in}{0.431673in}}{\pgfqpoint{3.738807in}{1.765244in}}%
\pgfusepath{clip}%
\pgfsetbuttcap%
\pgfsetroundjoin%
\pgfsetlinewidth{1.003750pt}%
\definecolor{currentstroke}{rgb}{0.968203,0.720844,0.612293}%
\pgfsetstrokecolor{currentstroke}%
\pgfsetdash{}{0pt}%
\pgfpathmoveto{\pgfqpoint{3.450046in}{1.454742in}}%
\pgfpathlineto{\pgfqpoint{3.450046in}{1.454742in}}%
\pgfusepath{stroke}%
\end{pgfscope}%
\begin{pgfscope}%
\pgfpathrectangle{\pgfqpoint{0.552773in}{0.431673in}}{\pgfqpoint{3.738807in}{1.765244in}}%
\pgfusepath{clip}%
\pgfsetbuttcap%
\pgfsetroundjoin%
\pgfsetlinewidth{1.003750pt}%
\definecolor{currentstroke}{rgb}{0.968203,0.720844,0.612293}%
\pgfsetstrokecolor{currentstroke}%
\pgfsetdash{}{0pt}%
\pgfpathmoveto{\pgfqpoint{3.453543in}{1.400314in}}%
\pgfpathlineto{\pgfqpoint{3.453543in}{1.400314in}}%
\pgfusepath{stroke}%
\end{pgfscope}%
\begin{pgfscope}%
\pgfpathrectangle{\pgfqpoint{0.552773in}{0.431673in}}{\pgfqpoint{3.738807in}{1.765244in}}%
\pgfusepath{clip}%
\pgfsetbuttcap%
\pgfsetroundjoin%
\pgfsetlinewidth{1.003750pt}%
\definecolor{currentstroke}{rgb}{0.968203,0.720844,0.612293}%
\pgfsetstrokecolor{currentstroke}%
\pgfsetdash{}{0pt}%
\pgfpathmoveto{\pgfqpoint{3.455571in}{1.343591in}}%
\pgfpathlineto{\pgfqpoint{3.455571in}{1.343591in}}%
\pgfusepath{stroke}%
\end{pgfscope}%
\begin{pgfscope}%
\pgfpathrectangle{\pgfqpoint{0.552773in}{0.431673in}}{\pgfqpoint{3.738807in}{1.765244in}}%
\pgfusepath{clip}%
\pgfsetbuttcap%
\pgfsetroundjoin%
\pgfsetlinewidth{1.003750pt}%
\definecolor{currentstroke}{rgb}{0.968203,0.720844,0.612293}%
\pgfsetstrokecolor{currentstroke}%
\pgfsetdash{}{0pt}%
\pgfpathmoveto{\pgfqpoint{3.477372in}{1.317776in}}%
\pgfpathlineto{\pgfqpoint{3.477372in}{1.317776in}}%
\pgfusepath{stroke}%
\end{pgfscope}%
\begin{pgfscope}%
\pgfpathrectangle{\pgfqpoint{0.552773in}{0.431673in}}{\pgfqpoint{3.738807in}{1.765244in}}%
\pgfusepath{clip}%
\pgfsetbuttcap%
\pgfsetroundjoin%
\pgfsetlinewidth{1.003750pt}%
\definecolor{currentstroke}{rgb}{0.968203,0.720844,0.612293}%
\pgfsetstrokecolor{currentstroke}%
\pgfsetdash{}{0pt}%
\pgfpathmoveto{\pgfqpoint{3.481371in}{1.264133in}}%
\pgfpathlineto{\pgfqpoint{3.481371in}{1.264133in}}%
\pgfusepath{stroke}%
\end{pgfscope}%
\begin{pgfscope}%
\pgfpathrectangle{\pgfqpoint{0.552773in}{0.431673in}}{\pgfqpoint{3.738807in}{1.765244in}}%
\pgfusepath{clip}%
\pgfsetbuttcap%
\pgfsetroundjoin%
\pgfsetlinewidth{1.003750pt}%
\definecolor{currentstroke}{rgb}{0.968203,0.720844,0.612293}%
\pgfsetstrokecolor{currentstroke}%
\pgfsetdash{}{0pt}%
\pgfpathmoveto{\pgfqpoint{3.505373in}{1.241757in}}%
\pgfpathlineto{\pgfqpoint{3.505373in}{1.241757in}}%
\pgfusepath{stroke}%
\end{pgfscope}%
\begin{pgfscope}%
\pgfpathrectangle{\pgfqpoint{0.552773in}{0.431673in}}{\pgfqpoint{3.738807in}{1.765244in}}%
\pgfusepath{clip}%
\pgfsetbuttcap%
\pgfsetroundjoin%
\pgfsetlinewidth{1.003750pt}%
\definecolor{currentstroke}{rgb}{0.968203,0.720844,0.612293}%
\pgfsetstrokecolor{currentstroke}%
\pgfsetdash{}{0pt}%
\pgfpathmoveto{\pgfqpoint{3.526998in}{1.215666in}}%
\pgfpathlineto{\pgfqpoint{3.526998in}{1.215666in}}%
\pgfusepath{stroke}%
\end{pgfscope}%
\begin{pgfscope}%
\pgfpathrectangle{\pgfqpoint{0.552773in}{0.431673in}}{\pgfqpoint{3.738807in}{1.765244in}}%
\pgfusepath{clip}%
\pgfsetbuttcap%
\pgfsetroundjoin%
\pgfsetlinewidth{1.003750pt}%
\definecolor{currentstroke}{rgb}{0.968203,0.720844,0.612293}%
\pgfsetstrokecolor{currentstroke}%
\pgfsetdash{}{0pt}%
\pgfpathmoveto{\pgfqpoint{3.562203in}{1.210802in}}%
\pgfpathlineto{\pgfqpoint{3.562203in}{1.210802in}}%
\pgfusepath{stroke}%
\end{pgfscope}%
\begin{pgfscope}%
\pgfpathrectangle{\pgfqpoint{0.552773in}{0.431673in}}{\pgfqpoint{3.738807in}{1.765244in}}%
\pgfusepath{clip}%
\pgfsetbuttcap%
\pgfsetroundjoin%
\pgfsetlinewidth{1.003750pt}%
\definecolor{currentstroke}{rgb}{0.968203,0.720844,0.612293}%
\pgfsetstrokecolor{currentstroke}%
\pgfsetdash{}{0pt}%
\pgfpathmoveto{\pgfqpoint{3.586856in}{1.189444in}}%
\pgfpathlineto{\pgfqpoint{3.586856in}{1.189444in}}%
\pgfusepath{stroke}%
\end{pgfscope}%
\begin{pgfscope}%
\pgfpathrectangle{\pgfqpoint{0.552773in}{0.431673in}}{\pgfqpoint{3.738807in}{1.765244in}}%
\pgfusepath{clip}%
\pgfsetbuttcap%
\pgfsetroundjoin%
\pgfsetlinewidth{1.003750pt}%
\definecolor{currentstroke}{rgb}{0.968203,0.720844,0.612293}%
\pgfsetstrokecolor{currentstroke}%
\pgfsetdash{}{0pt}%
\pgfpathmoveto{\pgfqpoint{3.629765in}{1.196622in}}%
\pgfpathlineto{\pgfqpoint{3.629765in}{1.196622in}}%
\pgfusepath{stroke}%
\end{pgfscope}%
\begin{pgfscope}%
\pgfpathrectangle{\pgfqpoint{0.552773in}{0.431673in}}{\pgfqpoint{3.738807in}{1.765244in}}%
\pgfusepath{clip}%
\pgfsetbuttcap%
\pgfsetroundjoin%
\pgfsetlinewidth{1.003750pt}%
\definecolor{currentstroke}{rgb}{0.968203,0.720844,0.612293}%
\pgfsetstrokecolor{currentstroke}%
\pgfsetdash{}{0pt}%
\pgfpathmoveto{\pgfqpoint{3.676730in}{1.210141in}}%
\pgfpathlineto{\pgfqpoint{3.676730in}{1.210141in}}%
\pgfusepath{stroke}%
\end{pgfscope}%
\begin{pgfscope}%
\pgfpathrectangle{\pgfqpoint{0.552773in}{0.431673in}}{\pgfqpoint{3.738807in}{1.765244in}}%
\pgfusepath{clip}%
\pgfsetbuttcap%
\pgfsetroundjoin%
\pgfsetlinewidth{1.003750pt}%
\definecolor{currentstroke}{rgb}{0.968203,0.720844,0.612293}%
\pgfsetstrokecolor{currentstroke}%
\pgfsetdash{}{0pt}%
\pgfpathmoveto{\pgfqpoint{3.745250in}{1.257352in}}%
\pgfpathlineto{\pgfqpoint{3.745250in}{1.257352in}}%
\pgfusepath{stroke}%
\end{pgfscope}%
\begin{pgfscope}%
\pgfpathrectangle{\pgfqpoint{0.552773in}{0.431673in}}{\pgfqpoint{3.738807in}{1.765244in}}%
\pgfusepath{clip}%
\pgfsetbuttcap%
\pgfsetroundjoin%
\pgfsetlinewidth{1.003750pt}%
\definecolor{currentstroke}{rgb}{0.968203,0.720844,0.612293}%
\pgfsetstrokecolor{currentstroke}%
\pgfsetdash{}{0pt}%
\pgfpathmoveto{\pgfqpoint{3.816743in}{1.309211in}}%
\pgfpathlineto{\pgfqpoint{3.816743in}{1.309211in}}%
\pgfusepath{stroke}%
\end{pgfscope}%
\begin{pgfscope}%
\pgfpathrectangle{\pgfqpoint{0.552773in}{0.431673in}}{\pgfqpoint{3.738807in}{1.765244in}}%
\pgfusepath{clip}%
\pgfsetbuttcap%
\pgfsetroundjoin%
\pgfsetlinewidth{1.003750pt}%
\definecolor{currentstroke}{rgb}{0.968203,0.720844,0.612293}%
\pgfsetstrokecolor{currentstroke}%
\pgfsetdash{}{0pt}%
\pgfpathmoveto{\pgfqpoint{3.871459in}{1.334845in}}%
\pgfpathlineto{\pgfqpoint{3.871459in}{1.334845in}}%
\pgfusepath{stroke}%
\end{pgfscope}%
\begin{pgfscope}%
\pgfpathrectangle{\pgfqpoint{0.552773in}{0.431673in}}{\pgfqpoint{3.738807in}{1.765244in}}%
\pgfusepath{clip}%
\pgfsetbuttcap%
\pgfsetroundjoin%
\pgfsetlinewidth{1.003750pt}%
\definecolor{currentstroke}{rgb}{0.968203,0.720844,0.612293}%
\pgfsetstrokecolor{currentstroke}%
\pgfsetdash{}{0pt}%
\pgfpathmoveto{\pgfqpoint{3.950363in}{1.398288in}}%
\pgfpathlineto{\pgfqpoint{3.950363in}{1.398288in}}%
\pgfusepath{stroke}%
\end{pgfscope}%
\begin{pgfscope}%
\pgfpathrectangle{\pgfqpoint{0.552773in}{0.431673in}}{\pgfqpoint{3.738807in}{1.765244in}}%
\pgfusepath{clip}%
\pgfsetbuttcap%
\pgfsetroundjoin%
\pgfsetlinewidth{1.003750pt}%
\definecolor{currentstroke}{rgb}{0.968203,0.720844,0.612293}%
\pgfsetstrokecolor{currentstroke}%
\pgfsetdash{}{0pt}%
\pgfpathmoveto{\pgfqpoint{3.993573in}{1.405936in}}%
\pgfpathlineto{\pgfqpoint{3.993573in}{1.405936in}}%
\pgfusepath{stroke}%
\end{pgfscope}%
\begin{pgfscope}%
\pgfpathrectangle{\pgfqpoint{0.552773in}{0.431673in}}{\pgfqpoint{3.738807in}{1.765244in}}%
\pgfusepath{clip}%
\pgfsetbuttcap%
\pgfsetroundjoin%
\definecolor{currentfill}{rgb}{0.968203,0.720844,0.612293}%
\pgfsetfillcolor{currentfill}%
\pgfsetlinewidth{1.003750pt}%
\definecolor{currentstroke}{rgb}{0.968203,0.720844,0.612293}%
\pgfsetstrokecolor{currentstroke}%
\pgfsetdash{}{0pt}%
\pgfsys@defobject{currentmarker}{\pgfqpoint{0.000000in}{-0.027778in}}{\pgfqpoint{0.000000in}{0.027778in}}{%
\pgfpathmoveto{\pgfqpoint{0.000000in}{-0.027778in}}%
\pgfpathlineto{\pgfqpoint{0.000000in}{0.027778in}}%
\pgfusepath{stroke,fill}%
}%
\begin{pgfscope}%
\pgfsys@transformshift{1.234882in}{1.346212in}%
\pgfsys@useobject{currentmarker}{}%
\end{pgfscope}%
\begin{pgfscope}%
\pgfsys@transformshift{1.295052in}{1.380370in}%
\pgfsys@useobject{currentmarker}{}%
\end{pgfscope}%
\begin{pgfscope}%
\pgfsys@transformshift{1.333712in}{1.380908in}%
\pgfsys@useobject{currentmarker}{}%
\end{pgfscope}%
\begin{pgfscope}%
\pgfsys@transformshift{1.360563in}{1.362986in}%
\pgfsys@useobject{currentmarker}{}%
\end{pgfscope}%
\begin{pgfscope}%
\pgfsys@transformshift{1.364476in}{1.309209in}%
\pgfsys@useobject{currentmarker}{}%
\end{pgfscope}%
\begin{pgfscope}%
\pgfsys@transformshift{1.369782in}{1.257608in}%
\pgfsys@useobject{currentmarker}{}%
\end{pgfscope}%
\begin{pgfscope}%
\pgfsys@transformshift{1.424566in}{1.283350in}%
\pgfsys@useobject{currentmarker}{}%
\end{pgfscope}%
\begin{pgfscope}%
\pgfsys@transformshift{1.443716in}{1.253390in}%
\pgfsys@useobject{currentmarker}{}%
\end{pgfscope}%
\begin{pgfscope}%
\pgfsys@transformshift{1.466969in}{1.229844in}%
\pgfsys@useobject{currentmarker}{}%
\end{pgfscope}%
\begin{pgfscope}%
\pgfsys@transformshift{1.490856in}{1.207288in}%
\pgfsys@useobject{currentmarker}{}%
\end{pgfscope}%
\begin{pgfscope}%
\pgfsys@transformshift{1.503289in}{1.166829in}%
\pgfsys@useobject{currentmarker}{}%
\end{pgfscope}%
\begin{pgfscope}%
\pgfsys@transformshift{1.551438in}{1.182198in}%
\pgfsys@useobject{currentmarker}{}%
\end{pgfscope}%
\begin{pgfscope}%
\pgfsys@transformshift{1.607593in}{1.210082in}%
\pgfsys@useobject{currentmarker}{}%
\end{pgfscope}%
\begin{pgfscope}%
\pgfsys@transformshift{1.658864in}{1.230331in}%
\pgfsys@useobject{currentmarker}{}%
\end{pgfscope}%
\begin{pgfscope}%
\pgfsys@transformshift{1.722476in}{1.269870in}%
\pgfsys@useobject{currentmarker}{}%
\end{pgfscope}%
\begin{pgfscope}%
\pgfsys@transformshift{1.797820in}{1.327748in}%
\pgfsys@useobject{currentmarker}{}%
\end{pgfscope}%
\begin{pgfscope}%
\pgfsys@transformshift{1.877516in}{1.392430in}%
\pgfsys@useobject{currentmarker}{}%
\end{pgfscope}%
\begin{pgfscope}%
\pgfsys@transformshift{1.954391in}{1.452701in}%
\pgfsys@useobject{currentmarker}{}%
\end{pgfscope}%
\begin{pgfscope}%
\pgfsys@transformshift{2.014562in}{1.486861in}%
\pgfsys@useobject{currentmarker}{}%
\end{pgfscope}%
\begin{pgfscope}%
\pgfsys@transformshift{2.061371in}{1.500136in}%
\pgfsys@useobject{currentmarker}{}%
\end{pgfscope}%
\begin{pgfscope}%
\pgfsys@transformshift{2.083864in}{1.475402in}%
\pgfsys@useobject{currentmarker}{}%
\end{pgfscope}%
\begin{pgfscope}%
\pgfsys@transformshift{2.105964in}{1.450053in}%
\pgfsys@useobject{currentmarker}{}%
\end{pgfscope}%
\begin{pgfscope}%
\pgfsys@transformshift{2.116436in}{1.406528in}%
\pgfsys@useobject{currentmarker}{}%
\end{pgfscope}%
\begin{pgfscope}%
\pgfsys@transformshift{2.141673in}{1.386083in}%
\pgfsys@useobject{currentmarker}{}%
\end{pgfscope}%
\begin{pgfscope}%
\pgfsys@transformshift{2.156766in}{1.349782in}%
\pgfsys@useobject{currentmarker}{}%
\end{pgfscope}%
\begin{pgfscope}%
\pgfsys@transformshift{2.171473in}{1.312877in}%
\pgfsys@useobject{currentmarker}{}%
\end{pgfscope}%
\begin{pgfscope}%
\pgfsys@transformshift{2.192191in}{1.285369in}%
\pgfsys@useobject{currentmarker}{}%
\end{pgfscope}%
\begin{pgfscope}%
\pgfsys@transformshift{2.223452in}{1.274341in}%
\pgfsys@useobject{currentmarker}{}%
\end{pgfscope}%
\begin{pgfscope}%
\pgfsys@transformshift{2.256793in}{1.266562in}%
\pgfsys@useobject{currentmarker}{}%
\end{pgfscope}%
\begin{pgfscope}%
\pgfsys@transformshift{2.298519in}{1.271893in}%
\pgfsys@useobject{currentmarker}{}%
\end{pgfscope}%
\begin{pgfscope}%
\pgfsys@transformshift{2.360130in}{1.308304in}%
\pgfsys@useobject{currentmarker}{}%
\end{pgfscope}%
\begin{pgfscope}%
\pgfsys@transformshift{2.402956in}{1.315352in}%
\pgfsys@useobject{currentmarker}{}%
\end{pgfscope}%
\begin{pgfscope}%
\pgfsys@transformshift{2.474900in}{1.367916in}%
\pgfsys@useobject{currentmarker}{}%
\end{pgfscope}%
\begin{pgfscope}%
\pgfsys@transformshift{2.549428in}{1.424519in}%
\pgfsys@useobject{currentmarker}{}%
\end{pgfscope}%
\begin{pgfscope}%
\pgfsys@transformshift{2.619369in}{1.473950in}%
\pgfsys@useobject{currentmarker}{}%
\end{pgfscope}%
\begin{pgfscope}%
\pgfsys@transformshift{2.688119in}{1.521522in}%
\pgfsys@useobject{currentmarker}{}%
\end{pgfscope}%
\begin{pgfscope}%
\pgfsys@transformshift{2.745485in}{1.551298in}%
\pgfsys@useobject{currentmarker}{}%
\end{pgfscope}%
\begin{pgfscope}%
\pgfsys@transformshift{2.770849in}{1.531052in}%
\pgfsys@useobject{currentmarker}{}%
\end{pgfscope}%
\begin{pgfscope}%
\pgfsys@transformshift{2.791502in}{1.503442in}%
\pgfsys@useobject{currentmarker}{}%
\end{pgfscope}%
\begin{pgfscope}%
\pgfsys@transformshift{2.801394in}{1.459011in}%
\pgfsys@useobject{currentmarker}{}%
\end{pgfscope}%
\begin{pgfscope}%
\pgfsys@transformshift{2.825889in}{1.437406in}%
\pgfsys@useobject{currentmarker}{}%
\end{pgfscope}%
\begin{pgfscope}%
\pgfsys@transformshift{2.823590in}{1.373918in}%
\pgfsys@useobject{currentmarker}{}%
\end{pgfscope}%
\begin{pgfscope}%
\pgfsys@transformshift{2.848428in}{1.352850in}%
\pgfsys@useobject{currentmarker}{}%
\end{pgfscope}%
\begin{pgfscope}%
\pgfsys@transformshift{2.859935in}{1.310943in}%
\pgfsys@useobject{currentmarker}{}%
\end{pgfscope}%
\begin{pgfscope}%
\pgfsys@transformshift{2.877851in}{1.279055in}%
\pgfsys@useobject{currentmarker}{}%
\end{pgfscope}%
\begin{pgfscope}%
\pgfsys@transformshift{2.897792in}{1.250331in}%
\pgfsys@useobject{currentmarker}{}%
\end{pgfscope}%
\begin{pgfscope}%
\pgfsys@transformshift{2.906245in}{1.203652in}%
\pgfsys@useobject{currentmarker}{}%
\end{pgfscope}%
\begin{pgfscope}%
\pgfsys@transformshift{2.946444in}{1.206593in}%
\pgfsys@useobject{currentmarker}{}%
\end{pgfscope}%
\begin{pgfscope}%
\pgfsys@transformshift{3.013245in}{1.251118in}%
\pgfsys@useobject{currentmarker}{}%
\end{pgfscope}%
\begin{pgfscope}%
\pgfsys@transformshift{3.071632in}{1.282490in}%
\pgfsys@useobject{currentmarker}{}%
\end{pgfscope}%
\begin{pgfscope}%
\pgfsys@transformshift{3.138985in}{1.327877in}%
\pgfsys@useobject{currentmarker}{}%
\end{pgfscope}%
\begin{pgfscope}%
\pgfsys@transformshift{3.210642in}{1.379991in}%
\pgfsys@useobject{currentmarker}{}%
\end{pgfscope}%
\begin{pgfscope}%
\pgfsys@transformshift{3.288346in}{1.441559in}%
\pgfsys@useobject{currentmarker}{}%
\end{pgfscope}%
\begin{pgfscope}%
\pgfsys@transformshift{3.354079in}{1.484414in}%
\pgfsys@useobject{currentmarker}{}%
\end{pgfscope}%
\begin{pgfscope}%
\pgfsys@transformshift{3.395162in}{1.488738in}%
\pgfsys@useobject{currentmarker}{}%
\end{pgfscope}%
\begin{pgfscope}%
\pgfsys@transformshift{3.423246in}{1.472743in}%
\pgfsys@useobject{currentmarker}{}%
\end{pgfscope}%
\begin{pgfscope}%
\pgfsys@transformshift{3.450046in}{1.454742in}%
\pgfsys@useobject{currentmarker}{}%
\end{pgfscope}%
\begin{pgfscope}%
\pgfsys@transformshift{3.453543in}{1.400314in}%
\pgfsys@useobject{currentmarker}{}%
\end{pgfscope}%
\begin{pgfscope}%
\pgfsys@transformshift{3.455571in}{1.343591in}%
\pgfsys@useobject{currentmarker}{}%
\end{pgfscope}%
\begin{pgfscope}%
\pgfsys@transformshift{3.477372in}{1.317776in}%
\pgfsys@useobject{currentmarker}{}%
\end{pgfscope}%
\begin{pgfscope}%
\pgfsys@transformshift{3.481371in}{1.264133in}%
\pgfsys@useobject{currentmarker}{}%
\end{pgfscope}%
\begin{pgfscope}%
\pgfsys@transformshift{3.505373in}{1.241757in}%
\pgfsys@useobject{currentmarker}{}%
\end{pgfscope}%
\begin{pgfscope}%
\pgfsys@transformshift{3.526998in}{1.215666in}%
\pgfsys@useobject{currentmarker}{}%
\end{pgfscope}%
\begin{pgfscope}%
\pgfsys@transformshift{3.562203in}{1.210802in}%
\pgfsys@useobject{currentmarker}{}%
\end{pgfscope}%
\begin{pgfscope}%
\pgfsys@transformshift{3.586856in}{1.189444in}%
\pgfsys@useobject{currentmarker}{}%
\end{pgfscope}%
\begin{pgfscope}%
\pgfsys@transformshift{3.629765in}{1.196622in}%
\pgfsys@useobject{currentmarker}{}%
\end{pgfscope}%
\begin{pgfscope}%
\pgfsys@transformshift{3.676730in}{1.210141in}%
\pgfsys@useobject{currentmarker}{}%
\end{pgfscope}%
\begin{pgfscope}%
\pgfsys@transformshift{3.745250in}{1.257352in}%
\pgfsys@useobject{currentmarker}{}%
\end{pgfscope}%
\begin{pgfscope}%
\pgfsys@transformshift{3.816743in}{1.309211in}%
\pgfsys@useobject{currentmarker}{}%
\end{pgfscope}%
\begin{pgfscope}%
\pgfsys@transformshift{3.871459in}{1.334845in}%
\pgfsys@useobject{currentmarker}{}%
\end{pgfscope}%
\begin{pgfscope}%
\pgfsys@transformshift{3.950363in}{1.398288in}%
\pgfsys@useobject{currentmarker}{}%
\end{pgfscope}%
\begin{pgfscope}%
\pgfsys@transformshift{3.993573in}{1.405936in}%
\pgfsys@useobject{currentmarker}{}%
\end{pgfscope}%
\end{pgfscope}%
\begin{pgfscope}%
\pgfpathrectangle{\pgfqpoint{0.552773in}{0.431673in}}{\pgfqpoint{3.738807in}{1.765244in}}%
\pgfusepath{clip}%
\pgfsetbuttcap%
\pgfsetroundjoin%
\definecolor{currentfill}{rgb}{0.968203,0.720844,0.612293}%
\pgfsetfillcolor{currentfill}%
\pgfsetlinewidth{1.003750pt}%
\definecolor{currentstroke}{rgb}{0.968203,0.720844,0.612293}%
\pgfsetstrokecolor{currentstroke}%
\pgfsetdash{}{0pt}%
\pgfsys@defobject{currentmarker}{\pgfqpoint{0.000000in}{-0.027778in}}{\pgfqpoint{0.000000in}{0.027778in}}{%
\pgfpathmoveto{\pgfqpoint{0.000000in}{-0.027778in}}%
\pgfpathlineto{\pgfqpoint{0.000000in}{0.027778in}}%
\pgfusepath{stroke,fill}%
}%
\begin{pgfscope}%
\pgfsys@transformshift{1.234882in}{1.346212in}%
\pgfsys@useobject{currentmarker}{}%
\end{pgfscope}%
\begin{pgfscope}%
\pgfsys@transformshift{1.295052in}{1.380370in}%
\pgfsys@useobject{currentmarker}{}%
\end{pgfscope}%
\begin{pgfscope}%
\pgfsys@transformshift{1.333712in}{1.380908in}%
\pgfsys@useobject{currentmarker}{}%
\end{pgfscope}%
\begin{pgfscope}%
\pgfsys@transformshift{1.360563in}{1.362986in}%
\pgfsys@useobject{currentmarker}{}%
\end{pgfscope}%
\begin{pgfscope}%
\pgfsys@transformshift{1.364476in}{1.309209in}%
\pgfsys@useobject{currentmarker}{}%
\end{pgfscope}%
\begin{pgfscope}%
\pgfsys@transformshift{1.369782in}{1.257608in}%
\pgfsys@useobject{currentmarker}{}%
\end{pgfscope}%
\begin{pgfscope}%
\pgfsys@transformshift{1.424566in}{1.283350in}%
\pgfsys@useobject{currentmarker}{}%
\end{pgfscope}%
\begin{pgfscope}%
\pgfsys@transformshift{1.443716in}{1.253390in}%
\pgfsys@useobject{currentmarker}{}%
\end{pgfscope}%
\begin{pgfscope}%
\pgfsys@transformshift{1.466969in}{1.229844in}%
\pgfsys@useobject{currentmarker}{}%
\end{pgfscope}%
\begin{pgfscope}%
\pgfsys@transformshift{1.490856in}{1.207288in}%
\pgfsys@useobject{currentmarker}{}%
\end{pgfscope}%
\begin{pgfscope}%
\pgfsys@transformshift{1.503289in}{1.166829in}%
\pgfsys@useobject{currentmarker}{}%
\end{pgfscope}%
\begin{pgfscope}%
\pgfsys@transformshift{1.551438in}{1.182198in}%
\pgfsys@useobject{currentmarker}{}%
\end{pgfscope}%
\begin{pgfscope}%
\pgfsys@transformshift{1.607593in}{1.210082in}%
\pgfsys@useobject{currentmarker}{}%
\end{pgfscope}%
\begin{pgfscope}%
\pgfsys@transformshift{1.658864in}{1.230331in}%
\pgfsys@useobject{currentmarker}{}%
\end{pgfscope}%
\begin{pgfscope}%
\pgfsys@transformshift{1.722476in}{1.269870in}%
\pgfsys@useobject{currentmarker}{}%
\end{pgfscope}%
\begin{pgfscope}%
\pgfsys@transformshift{1.797820in}{1.327748in}%
\pgfsys@useobject{currentmarker}{}%
\end{pgfscope}%
\begin{pgfscope}%
\pgfsys@transformshift{1.877516in}{1.392430in}%
\pgfsys@useobject{currentmarker}{}%
\end{pgfscope}%
\begin{pgfscope}%
\pgfsys@transformshift{1.954391in}{1.452701in}%
\pgfsys@useobject{currentmarker}{}%
\end{pgfscope}%
\begin{pgfscope}%
\pgfsys@transformshift{2.014562in}{1.486861in}%
\pgfsys@useobject{currentmarker}{}%
\end{pgfscope}%
\begin{pgfscope}%
\pgfsys@transformshift{2.061371in}{1.500136in}%
\pgfsys@useobject{currentmarker}{}%
\end{pgfscope}%
\begin{pgfscope}%
\pgfsys@transformshift{2.083864in}{1.475402in}%
\pgfsys@useobject{currentmarker}{}%
\end{pgfscope}%
\begin{pgfscope}%
\pgfsys@transformshift{2.105964in}{1.450053in}%
\pgfsys@useobject{currentmarker}{}%
\end{pgfscope}%
\begin{pgfscope}%
\pgfsys@transformshift{2.116436in}{1.406528in}%
\pgfsys@useobject{currentmarker}{}%
\end{pgfscope}%
\begin{pgfscope}%
\pgfsys@transformshift{2.141673in}{1.386083in}%
\pgfsys@useobject{currentmarker}{}%
\end{pgfscope}%
\begin{pgfscope}%
\pgfsys@transformshift{2.156766in}{1.349782in}%
\pgfsys@useobject{currentmarker}{}%
\end{pgfscope}%
\begin{pgfscope}%
\pgfsys@transformshift{2.171473in}{1.312877in}%
\pgfsys@useobject{currentmarker}{}%
\end{pgfscope}%
\begin{pgfscope}%
\pgfsys@transformshift{2.192191in}{1.285369in}%
\pgfsys@useobject{currentmarker}{}%
\end{pgfscope}%
\begin{pgfscope}%
\pgfsys@transformshift{2.223452in}{1.274341in}%
\pgfsys@useobject{currentmarker}{}%
\end{pgfscope}%
\begin{pgfscope}%
\pgfsys@transformshift{2.256793in}{1.266562in}%
\pgfsys@useobject{currentmarker}{}%
\end{pgfscope}%
\begin{pgfscope}%
\pgfsys@transformshift{2.298519in}{1.271893in}%
\pgfsys@useobject{currentmarker}{}%
\end{pgfscope}%
\begin{pgfscope}%
\pgfsys@transformshift{2.360130in}{1.308304in}%
\pgfsys@useobject{currentmarker}{}%
\end{pgfscope}%
\begin{pgfscope}%
\pgfsys@transformshift{2.402956in}{1.315352in}%
\pgfsys@useobject{currentmarker}{}%
\end{pgfscope}%
\begin{pgfscope}%
\pgfsys@transformshift{2.474900in}{1.367916in}%
\pgfsys@useobject{currentmarker}{}%
\end{pgfscope}%
\begin{pgfscope}%
\pgfsys@transformshift{2.549428in}{1.424519in}%
\pgfsys@useobject{currentmarker}{}%
\end{pgfscope}%
\begin{pgfscope}%
\pgfsys@transformshift{2.619369in}{1.473950in}%
\pgfsys@useobject{currentmarker}{}%
\end{pgfscope}%
\begin{pgfscope}%
\pgfsys@transformshift{2.688119in}{1.521522in}%
\pgfsys@useobject{currentmarker}{}%
\end{pgfscope}%
\begin{pgfscope}%
\pgfsys@transformshift{2.745485in}{1.551298in}%
\pgfsys@useobject{currentmarker}{}%
\end{pgfscope}%
\begin{pgfscope}%
\pgfsys@transformshift{2.770849in}{1.531052in}%
\pgfsys@useobject{currentmarker}{}%
\end{pgfscope}%
\begin{pgfscope}%
\pgfsys@transformshift{2.791502in}{1.503442in}%
\pgfsys@useobject{currentmarker}{}%
\end{pgfscope}%
\begin{pgfscope}%
\pgfsys@transformshift{2.801394in}{1.459011in}%
\pgfsys@useobject{currentmarker}{}%
\end{pgfscope}%
\begin{pgfscope}%
\pgfsys@transformshift{2.825889in}{1.437406in}%
\pgfsys@useobject{currentmarker}{}%
\end{pgfscope}%
\begin{pgfscope}%
\pgfsys@transformshift{2.823590in}{1.373918in}%
\pgfsys@useobject{currentmarker}{}%
\end{pgfscope}%
\begin{pgfscope}%
\pgfsys@transformshift{2.848428in}{1.352850in}%
\pgfsys@useobject{currentmarker}{}%
\end{pgfscope}%
\begin{pgfscope}%
\pgfsys@transformshift{2.859935in}{1.310943in}%
\pgfsys@useobject{currentmarker}{}%
\end{pgfscope}%
\begin{pgfscope}%
\pgfsys@transformshift{2.877851in}{1.279055in}%
\pgfsys@useobject{currentmarker}{}%
\end{pgfscope}%
\begin{pgfscope}%
\pgfsys@transformshift{2.897792in}{1.250331in}%
\pgfsys@useobject{currentmarker}{}%
\end{pgfscope}%
\begin{pgfscope}%
\pgfsys@transformshift{2.906245in}{1.203652in}%
\pgfsys@useobject{currentmarker}{}%
\end{pgfscope}%
\begin{pgfscope}%
\pgfsys@transformshift{2.946444in}{1.206593in}%
\pgfsys@useobject{currentmarker}{}%
\end{pgfscope}%
\begin{pgfscope}%
\pgfsys@transformshift{3.013245in}{1.251118in}%
\pgfsys@useobject{currentmarker}{}%
\end{pgfscope}%
\begin{pgfscope}%
\pgfsys@transformshift{3.071632in}{1.282490in}%
\pgfsys@useobject{currentmarker}{}%
\end{pgfscope}%
\begin{pgfscope}%
\pgfsys@transformshift{3.138985in}{1.327877in}%
\pgfsys@useobject{currentmarker}{}%
\end{pgfscope}%
\begin{pgfscope}%
\pgfsys@transformshift{3.210642in}{1.379991in}%
\pgfsys@useobject{currentmarker}{}%
\end{pgfscope}%
\begin{pgfscope}%
\pgfsys@transformshift{3.288346in}{1.441559in}%
\pgfsys@useobject{currentmarker}{}%
\end{pgfscope}%
\begin{pgfscope}%
\pgfsys@transformshift{3.354079in}{1.484414in}%
\pgfsys@useobject{currentmarker}{}%
\end{pgfscope}%
\begin{pgfscope}%
\pgfsys@transformshift{3.395162in}{1.488738in}%
\pgfsys@useobject{currentmarker}{}%
\end{pgfscope}%
\begin{pgfscope}%
\pgfsys@transformshift{3.423246in}{1.472743in}%
\pgfsys@useobject{currentmarker}{}%
\end{pgfscope}%
\begin{pgfscope}%
\pgfsys@transformshift{3.450046in}{1.454742in}%
\pgfsys@useobject{currentmarker}{}%
\end{pgfscope}%
\begin{pgfscope}%
\pgfsys@transformshift{3.453543in}{1.400314in}%
\pgfsys@useobject{currentmarker}{}%
\end{pgfscope}%
\begin{pgfscope}%
\pgfsys@transformshift{3.455571in}{1.343591in}%
\pgfsys@useobject{currentmarker}{}%
\end{pgfscope}%
\begin{pgfscope}%
\pgfsys@transformshift{3.477372in}{1.317776in}%
\pgfsys@useobject{currentmarker}{}%
\end{pgfscope}%
\begin{pgfscope}%
\pgfsys@transformshift{3.481371in}{1.264133in}%
\pgfsys@useobject{currentmarker}{}%
\end{pgfscope}%
\begin{pgfscope}%
\pgfsys@transformshift{3.505373in}{1.241757in}%
\pgfsys@useobject{currentmarker}{}%
\end{pgfscope}%
\begin{pgfscope}%
\pgfsys@transformshift{3.526998in}{1.215666in}%
\pgfsys@useobject{currentmarker}{}%
\end{pgfscope}%
\begin{pgfscope}%
\pgfsys@transformshift{3.562203in}{1.210802in}%
\pgfsys@useobject{currentmarker}{}%
\end{pgfscope}%
\begin{pgfscope}%
\pgfsys@transformshift{3.586856in}{1.189444in}%
\pgfsys@useobject{currentmarker}{}%
\end{pgfscope}%
\begin{pgfscope}%
\pgfsys@transformshift{3.629765in}{1.196622in}%
\pgfsys@useobject{currentmarker}{}%
\end{pgfscope}%
\begin{pgfscope}%
\pgfsys@transformshift{3.676730in}{1.210141in}%
\pgfsys@useobject{currentmarker}{}%
\end{pgfscope}%
\begin{pgfscope}%
\pgfsys@transformshift{3.745250in}{1.257352in}%
\pgfsys@useobject{currentmarker}{}%
\end{pgfscope}%
\begin{pgfscope}%
\pgfsys@transformshift{3.816743in}{1.309211in}%
\pgfsys@useobject{currentmarker}{}%
\end{pgfscope}%
\begin{pgfscope}%
\pgfsys@transformshift{3.871459in}{1.334845in}%
\pgfsys@useobject{currentmarker}{}%
\end{pgfscope}%
\begin{pgfscope}%
\pgfsys@transformshift{3.950363in}{1.398288in}%
\pgfsys@useobject{currentmarker}{}%
\end{pgfscope}%
\begin{pgfscope}%
\pgfsys@transformshift{3.993573in}{1.405936in}%
\pgfsys@useobject{currentmarker}{}%
\end{pgfscope}%
\end{pgfscope}%
\begin{pgfscope}%
\pgfpathrectangle{\pgfqpoint{0.552773in}{0.431673in}}{\pgfqpoint{3.738807in}{1.765244in}}%
\pgfusepath{clip}%
\pgfsetbuttcap%
\pgfsetroundjoin%
\definecolor{currentfill}{rgb}{0.968203,0.720844,0.612293}%
\pgfsetfillcolor{currentfill}%
\pgfsetlinewidth{1.003750pt}%
\definecolor{currentstroke}{rgb}{0.968203,0.720844,0.612293}%
\pgfsetstrokecolor{currentstroke}%
\pgfsetdash{}{0pt}%
\pgfsys@defobject{currentmarker}{\pgfqpoint{-0.027778in}{-0.000000in}}{\pgfqpoint{0.027778in}{0.000000in}}{%
\pgfpathmoveto{\pgfqpoint{0.027778in}{-0.000000in}}%
\pgfpathlineto{\pgfqpoint{-0.027778in}{0.000000in}}%
\pgfusepath{stroke,fill}%
}%
\begin{pgfscope}%
\pgfsys@transformshift{1.234882in}{1.346212in}%
\pgfsys@useobject{currentmarker}{}%
\end{pgfscope}%
\begin{pgfscope}%
\pgfsys@transformshift{1.295052in}{1.380370in}%
\pgfsys@useobject{currentmarker}{}%
\end{pgfscope}%
\begin{pgfscope}%
\pgfsys@transformshift{1.333712in}{1.380908in}%
\pgfsys@useobject{currentmarker}{}%
\end{pgfscope}%
\begin{pgfscope}%
\pgfsys@transformshift{1.360563in}{1.362986in}%
\pgfsys@useobject{currentmarker}{}%
\end{pgfscope}%
\begin{pgfscope}%
\pgfsys@transformshift{1.364476in}{1.309209in}%
\pgfsys@useobject{currentmarker}{}%
\end{pgfscope}%
\begin{pgfscope}%
\pgfsys@transformshift{1.369782in}{1.257608in}%
\pgfsys@useobject{currentmarker}{}%
\end{pgfscope}%
\begin{pgfscope}%
\pgfsys@transformshift{1.424566in}{1.283350in}%
\pgfsys@useobject{currentmarker}{}%
\end{pgfscope}%
\begin{pgfscope}%
\pgfsys@transformshift{1.443716in}{1.253390in}%
\pgfsys@useobject{currentmarker}{}%
\end{pgfscope}%
\begin{pgfscope}%
\pgfsys@transformshift{1.466969in}{1.229844in}%
\pgfsys@useobject{currentmarker}{}%
\end{pgfscope}%
\begin{pgfscope}%
\pgfsys@transformshift{1.490856in}{1.207288in}%
\pgfsys@useobject{currentmarker}{}%
\end{pgfscope}%
\begin{pgfscope}%
\pgfsys@transformshift{1.503289in}{1.166829in}%
\pgfsys@useobject{currentmarker}{}%
\end{pgfscope}%
\begin{pgfscope}%
\pgfsys@transformshift{1.551438in}{1.182198in}%
\pgfsys@useobject{currentmarker}{}%
\end{pgfscope}%
\begin{pgfscope}%
\pgfsys@transformshift{1.607593in}{1.210082in}%
\pgfsys@useobject{currentmarker}{}%
\end{pgfscope}%
\begin{pgfscope}%
\pgfsys@transformshift{1.658864in}{1.230331in}%
\pgfsys@useobject{currentmarker}{}%
\end{pgfscope}%
\begin{pgfscope}%
\pgfsys@transformshift{1.722476in}{1.269870in}%
\pgfsys@useobject{currentmarker}{}%
\end{pgfscope}%
\begin{pgfscope}%
\pgfsys@transformshift{1.797820in}{1.327748in}%
\pgfsys@useobject{currentmarker}{}%
\end{pgfscope}%
\begin{pgfscope}%
\pgfsys@transformshift{1.877516in}{1.392430in}%
\pgfsys@useobject{currentmarker}{}%
\end{pgfscope}%
\begin{pgfscope}%
\pgfsys@transformshift{1.954391in}{1.452701in}%
\pgfsys@useobject{currentmarker}{}%
\end{pgfscope}%
\begin{pgfscope}%
\pgfsys@transformshift{2.014562in}{1.486861in}%
\pgfsys@useobject{currentmarker}{}%
\end{pgfscope}%
\begin{pgfscope}%
\pgfsys@transformshift{2.061371in}{1.500136in}%
\pgfsys@useobject{currentmarker}{}%
\end{pgfscope}%
\begin{pgfscope}%
\pgfsys@transformshift{2.083864in}{1.475402in}%
\pgfsys@useobject{currentmarker}{}%
\end{pgfscope}%
\begin{pgfscope}%
\pgfsys@transformshift{2.105964in}{1.450053in}%
\pgfsys@useobject{currentmarker}{}%
\end{pgfscope}%
\begin{pgfscope}%
\pgfsys@transformshift{2.116436in}{1.406528in}%
\pgfsys@useobject{currentmarker}{}%
\end{pgfscope}%
\begin{pgfscope}%
\pgfsys@transformshift{2.141673in}{1.386083in}%
\pgfsys@useobject{currentmarker}{}%
\end{pgfscope}%
\begin{pgfscope}%
\pgfsys@transformshift{2.156766in}{1.349782in}%
\pgfsys@useobject{currentmarker}{}%
\end{pgfscope}%
\begin{pgfscope}%
\pgfsys@transformshift{2.171473in}{1.312877in}%
\pgfsys@useobject{currentmarker}{}%
\end{pgfscope}%
\begin{pgfscope}%
\pgfsys@transformshift{2.192191in}{1.285369in}%
\pgfsys@useobject{currentmarker}{}%
\end{pgfscope}%
\begin{pgfscope}%
\pgfsys@transformshift{2.223452in}{1.274341in}%
\pgfsys@useobject{currentmarker}{}%
\end{pgfscope}%
\begin{pgfscope}%
\pgfsys@transformshift{2.256793in}{1.266562in}%
\pgfsys@useobject{currentmarker}{}%
\end{pgfscope}%
\begin{pgfscope}%
\pgfsys@transformshift{2.298519in}{1.271893in}%
\pgfsys@useobject{currentmarker}{}%
\end{pgfscope}%
\begin{pgfscope}%
\pgfsys@transformshift{2.360130in}{1.308304in}%
\pgfsys@useobject{currentmarker}{}%
\end{pgfscope}%
\begin{pgfscope}%
\pgfsys@transformshift{2.402956in}{1.315352in}%
\pgfsys@useobject{currentmarker}{}%
\end{pgfscope}%
\begin{pgfscope}%
\pgfsys@transformshift{2.474900in}{1.367916in}%
\pgfsys@useobject{currentmarker}{}%
\end{pgfscope}%
\begin{pgfscope}%
\pgfsys@transformshift{2.549428in}{1.424519in}%
\pgfsys@useobject{currentmarker}{}%
\end{pgfscope}%
\begin{pgfscope}%
\pgfsys@transformshift{2.619369in}{1.473950in}%
\pgfsys@useobject{currentmarker}{}%
\end{pgfscope}%
\begin{pgfscope}%
\pgfsys@transformshift{2.688119in}{1.521522in}%
\pgfsys@useobject{currentmarker}{}%
\end{pgfscope}%
\begin{pgfscope}%
\pgfsys@transformshift{2.745485in}{1.551298in}%
\pgfsys@useobject{currentmarker}{}%
\end{pgfscope}%
\begin{pgfscope}%
\pgfsys@transformshift{2.770849in}{1.531052in}%
\pgfsys@useobject{currentmarker}{}%
\end{pgfscope}%
\begin{pgfscope}%
\pgfsys@transformshift{2.791502in}{1.503442in}%
\pgfsys@useobject{currentmarker}{}%
\end{pgfscope}%
\begin{pgfscope}%
\pgfsys@transformshift{2.801394in}{1.459011in}%
\pgfsys@useobject{currentmarker}{}%
\end{pgfscope}%
\begin{pgfscope}%
\pgfsys@transformshift{2.825889in}{1.437406in}%
\pgfsys@useobject{currentmarker}{}%
\end{pgfscope}%
\begin{pgfscope}%
\pgfsys@transformshift{2.823590in}{1.373918in}%
\pgfsys@useobject{currentmarker}{}%
\end{pgfscope}%
\begin{pgfscope}%
\pgfsys@transformshift{2.848428in}{1.352850in}%
\pgfsys@useobject{currentmarker}{}%
\end{pgfscope}%
\begin{pgfscope}%
\pgfsys@transformshift{2.859935in}{1.310943in}%
\pgfsys@useobject{currentmarker}{}%
\end{pgfscope}%
\begin{pgfscope}%
\pgfsys@transformshift{2.877851in}{1.279055in}%
\pgfsys@useobject{currentmarker}{}%
\end{pgfscope}%
\begin{pgfscope}%
\pgfsys@transformshift{2.897792in}{1.250331in}%
\pgfsys@useobject{currentmarker}{}%
\end{pgfscope}%
\begin{pgfscope}%
\pgfsys@transformshift{2.906245in}{1.203652in}%
\pgfsys@useobject{currentmarker}{}%
\end{pgfscope}%
\begin{pgfscope}%
\pgfsys@transformshift{2.946444in}{1.206593in}%
\pgfsys@useobject{currentmarker}{}%
\end{pgfscope}%
\begin{pgfscope}%
\pgfsys@transformshift{3.013245in}{1.251118in}%
\pgfsys@useobject{currentmarker}{}%
\end{pgfscope}%
\begin{pgfscope}%
\pgfsys@transformshift{3.071632in}{1.282490in}%
\pgfsys@useobject{currentmarker}{}%
\end{pgfscope}%
\begin{pgfscope}%
\pgfsys@transformshift{3.138985in}{1.327877in}%
\pgfsys@useobject{currentmarker}{}%
\end{pgfscope}%
\begin{pgfscope}%
\pgfsys@transformshift{3.210642in}{1.379991in}%
\pgfsys@useobject{currentmarker}{}%
\end{pgfscope}%
\begin{pgfscope}%
\pgfsys@transformshift{3.288346in}{1.441559in}%
\pgfsys@useobject{currentmarker}{}%
\end{pgfscope}%
\begin{pgfscope}%
\pgfsys@transformshift{3.354079in}{1.484414in}%
\pgfsys@useobject{currentmarker}{}%
\end{pgfscope}%
\begin{pgfscope}%
\pgfsys@transformshift{3.395162in}{1.488738in}%
\pgfsys@useobject{currentmarker}{}%
\end{pgfscope}%
\begin{pgfscope}%
\pgfsys@transformshift{3.423246in}{1.472743in}%
\pgfsys@useobject{currentmarker}{}%
\end{pgfscope}%
\begin{pgfscope}%
\pgfsys@transformshift{3.450046in}{1.454742in}%
\pgfsys@useobject{currentmarker}{}%
\end{pgfscope}%
\begin{pgfscope}%
\pgfsys@transformshift{3.453543in}{1.400314in}%
\pgfsys@useobject{currentmarker}{}%
\end{pgfscope}%
\begin{pgfscope}%
\pgfsys@transformshift{3.455571in}{1.343591in}%
\pgfsys@useobject{currentmarker}{}%
\end{pgfscope}%
\begin{pgfscope}%
\pgfsys@transformshift{3.477372in}{1.317776in}%
\pgfsys@useobject{currentmarker}{}%
\end{pgfscope}%
\begin{pgfscope}%
\pgfsys@transformshift{3.481371in}{1.264133in}%
\pgfsys@useobject{currentmarker}{}%
\end{pgfscope}%
\begin{pgfscope}%
\pgfsys@transformshift{3.505373in}{1.241757in}%
\pgfsys@useobject{currentmarker}{}%
\end{pgfscope}%
\begin{pgfscope}%
\pgfsys@transformshift{3.526998in}{1.215666in}%
\pgfsys@useobject{currentmarker}{}%
\end{pgfscope}%
\begin{pgfscope}%
\pgfsys@transformshift{3.562203in}{1.210802in}%
\pgfsys@useobject{currentmarker}{}%
\end{pgfscope}%
\begin{pgfscope}%
\pgfsys@transformshift{3.586856in}{1.189444in}%
\pgfsys@useobject{currentmarker}{}%
\end{pgfscope}%
\begin{pgfscope}%
\pgfsys@transformshift{3.629765in}{1.196622in}%
\pgfsys@useobject{currentmarker}{}%
\end{pgfscope}%
\begin{pgfscope}%
\pgfsys@transformshift{3.676730in}{1.210141in}%
\pgfsys@useobject{currentmarker}{}%
\end{pgfscope}%
\begin{pgfscope}%
\pgfsys@transformshift{3.745250in}{1.257352in}%
\pgfsys@useobject{currentmarker}{}%
\end{pgfscope}%
\begin{pgfscope}%
\pgfsys@transformshift{3.816743in}{1.309211in}%
\pgfsys@useobject{currentmarker}{}%
\end{pgfscope}%
\begin{pgfscope}%
\pgfsys@transformshift{3.871459in}{1.334845in}%
\pgfsys@useobject{currentmarker}{}%
\end{pgfscope}%
\begin{pgfscope}%
\pgfsys@transformshift{3.950363in}{1.398288in}%
\pgfsys@useobject{currentmarker}{}%
\end{pgfscope}%
\begin{pgfscope}%
\pgfsys@transformshift{3.993573in}{1.405936in}%
\pgfsys@useobject{currentmarker}{}%
\end{pgfscope}%
\end{pgfscope}%
\begin{pgfscope}%
\pgfpathrectangle{\pgfqpoint{0.552773in}{0.431673in}}{\pgfqpoint{3.738807in}{1.765244in}}%
\pgfusepath{clip}%
\pgfsetbuttcap%
\pgfsetroundjoin%
\definecolor{currentfill}{rgb}{0.968203,0.720844,0.612293}%
\pgfsetfillcolor{currentfill}%
\pgfsetlinewidth{1.003750pt}%
\definecolor{currentstroke}{rgb}{0.968203,0.720844,0.612293}%
\pgfsetstrokecolor{currentstroke}%
\pgfsetdash{}{0pt}%
\pgfsys@defobject{currentmarker}{\pgfqpoint{-0.027778in}{-0.000000in}}{\pgfqpoint{0.027778in}{0.000000in}}{%
\pgfpathmoveto{\pgfqpoint{0.027778in}{-0.000000in}}%
\pgfpathlineto{\pgfqpoint{-0.027778in}{0.000000in}}%
\pgfusepath{stroke,fill}%
}%
\begin{pgfscope}%
\pgfsys@transformshift{1.234882in}{1.346212in}%
\pgfsys@useobject{currentmarker}{}%
\end{pgfscope}%
\begin{pgfscope}%
\pgfsys@transformshift{1.295052in}{1.380370in}%
\pgfsys@useobject{currentmarker}{}%
\end{pgfscope}%
\begin{pgfscope}%
\pgfsys@transformshift{1.333712in}{1.380908in}%
\pgfsys@useobject{currentmarker}{}%
\end{pgfscope}%
\begin{pgfscope}%
\pgfsys@transformshift{1.360563in}{1.362986in}%
\pgfsys@useobject{currentmarker}{}%
\end{pgfscope}%
\begin{pgfscope}%
\pgfsys@transformshift{1.364476in}{1.309209in}%
\pgfsys@useobject{currentmarker}{}%
\end{pgfscope}%
\begin{pgfscope}%
\pgfsys@transformshift{1.369782in}{1.257608in}%
\pgfsys@useobject{currentmarker}{}%
\end{pgfscope}%
\begin{pgfscope}%
\pgfsys@transformshift{1.424566in}{1.283350in}%
\pgfsys@useobject{currentmarker}{}%
\end{pgfscope}%
\begin{pgfscope}%
\pgfsys@transformshift{1.443716in}{1.253390in}%
\pgfsys@useobject{currentmarker}{}%
\end{pgfscope}%
\begin{pgfscope}%
\pgfsys@transformshift{1.466969in}{1.229844in}%
\pgfsys@useobject{currentmarker}{}%
\end{pgfscope}%
\begin{pgfscope}%
\pgfsys@transformshift{1.490856in}{1.207288in}%
\pgfsys@useobject{currentmarker}{}%
\end{pgfscope}%
\begin{pgfscope}%
\pgfsys@transformshift{1.503289in}{1.166829in}%
\pgfsys@useobject{currentmarker}{}%
\end{pgfscope}%
\begin{pgfscope}%
\pgfsys@transformshift{1.551438in}{1.182198in}%
\pgfsys@useobject{currentmarker}{}%
\end{pgfscope}%
\begin{pgfscope}%
\pgfsys@transformshift{1.607593in}{1.210082in}%
\pgfsys@useobject{currentmarker}{}%
\end{pgfscope}%
\begin{pgfscope}%
\pgfsys@transformshift{1.658864in}{1.230331in}%
\pgfsys@useobject{currentmarker}{}%
\end{pgfscope}%
\begin{pgfscope}%
\pgfsys@transformshift{1.722476in}{1.269870in}%
\pgfsys@useobject{currentmarker}{}%
\end{pgfscope}%
\begin{pgfscope}%
\pgfsys@transformshift{1.797820in}{1.327748in}%
\pgfsys@useobject{currentmarker}{}%
\end{pgfscope}%
\begin{pgfscope}%
\pgfsys@transformshift{1.877516in}{1.392430in}%
\pgfsys@useobject{currentmarker}{}%
\end{pgfscope}%
\begin{pgfscope}%
\pgfsys@transformshift{1.954391in}{1.452701in}%
\pgfsys@useobject{currentmarker}{}%
\end{pgfscope}%
\begin{pgfscope}%
\pgfsys@transformshift{2.014562in}{1.486861in}%
\pgfsys@useobject{currentmarker}{}%
\end{pgfscope}%
\begin{pgfscope}%
\pgfsys@transformshift{2.061371in}{1.500136in}%
\pgfsys@useobject{currentmarker}{}%
\end{pgfscope}%
\begin{pgfscope}%
\pgfsys@transformshift{2.083864in}{1.475402in}%
\pgfsys@useobject{currentmarker}{}%
\end{pgfscope}%
\begin{pgfscope}%
\pgfsys@transformshift{2.105964in}{1.450053in}%
\pgfsys@useobject{currentmarker}{}%
\end{pgfscope}%
\begin{pgfscope}%
\pgfsys@transformshift{2.116436in}{1.406528in}%
\pgfsys@useobject{currentmarker}{}%
\end{pgfscope}%
\begin{pgfscope}%
\pgfsys@transformshift{2.141673in}{1.386083in}%
\pgfsys@useobject{currentmarker}{}%
\end{pgfscope}%
\begin{pgfscope}%
\pgfsys@transformshift{2.156766in}{1.349782in}%
\pgfsys@useobject{currentmarker}{}%
\end{pgfscope}%
\begin{pgfscope}%
\pgfsys@transformshift{2.171473in}{1.312877in}%
\pgfsys@useobject{currentmarker}{}%
\end{pgfscope}%
\begin{pgfscope}%
\pgfsys@transformshift{2.192191in}{1.285369in}%
\pgfsys@useobject{currentmarker}{}%
\end{pgfscope}%
\begin{pgfscope}%
\pgfsys@transformshift{2.223452in}{1.274341in}%
\pgfsys@useobject{currentmarker}{}%
\end{pgfscope}%
\begin{pgfscope}%
\pgfsys@transformshift{2.256793in}{1.266562in}%
\pgfsys@useobject{currentmarker}{}%
\end{pgfscope}%
\begin{pgfscope}%
\pgfsys@transformshift{2.298519in}{1.271893in}%
\pgfsys@useobject{currentmarker}{}%
\end{pgfscope}%
\begin{pgfscope}%
\pgfsys@transformshift{2.360130in}{1.308304in}%
\pgfsys@useobject{currentmarker}{}%
\end{pgfscope}%
\begin{pgfscope}%
\pgfsys@transformshift{2.402956in}{1.315352in}%
\pgfsys@useobject{currentmarker}{}%
\end{pgfscope}%
\begin{pgfscope}%
\pgfsys@transformshift{2.474900in}{1.367916in}%
\pgfsys@useobject{currentmarker}{}%
\end{pgfscope}%
\begin{pgfscope}%
\pgfsys@transformshift{2.549428in}{1.424519in}%
\pgfsys@useobject{currentmarker}{}%
\end{pgfscope}%
\begin{pgfscope}%
\pgfsys@transformshift{2.619369in}{1.473950in}%
\pgfsys@useobject{currentmarker}{}%
\end{pgfscope}%
\begin{pgfscope}%
\pgfsys@transformshift{2.688119in}{1.521522in}%
\pgfsys@useobject{currentmarker}{}%
\end{pgfscope}%
\begin{pgfscope}%
\pgfsys@transformshift{2.745485in}{1.551298in}%
\pgfsys@useobject{currentmarker}{}%
\end{pgfscope}%
\begin{pgfscope}%
\pgfsys@transformshift{2.770849in}{1.531052in}%
\pgfsys@useobject{currentmarker}{}%
\end{pgfscope}%
\begin{pgfscope}%
\pgfsys@transformshift{2.791502in}{1.503442in}%
\pgfsys@useobject{currentmarker}{}%
\end{pgfscope}%
\begin{pgfscope}%
\pgfsys@transformshift{2.801394in}{1.459011in}%
\pgfsys@useobject{currentmarker}{}%
\end{pgfscope}%
\begin{pgfscope}%
\pgfsys@transformshift{2.825889in}{1.437406in}%
\pgfsys@useobject{currentmarker}{}%
\end{pgfscope}%
\begin{pgfscope}%
\pgfsys@transformshift{2.823590in}{1.373918in}%
\pgfsys@useobject{currentmarker}{}%
\end{pgfscope}%
\begin{pgfscope}%
\pgfsys@transformshift{2.848428in}{1.352850in}%
\pgfsys@useobject{currentmarker}{}%
\end{pgfscope}%
\begin{pgfscope}%
\pgfsys@transformshift{2.859935in}{1.310943in}%
\pgfsys@useobject{currentmarker}{}%
\end{pgfscope}%
\begin{pgfscope}%
\pgfsys@transformshift{2.877851in}{1.279055in}%
\pgfsys@useobject{currentmarker}{}%
\end{pgfscope}%
\begin{pgfscope}%
\pgfsys@transformshift{2.897792in}{1.250331in}%
\pgfsys@useobject{currentmarker}{}%
\end{pgfscope}%
\begin{pgfscope}%
\pgfsys@transformshift{2.906245in}{1.203652in}%
\pgfsys@useobject{currentmarker}{}%
\end{pgfscope}%
\begin{pgfscope}%
\pgfsys@transformshift{2.946444in}{1.206593in}%
\pgfsys@useobject{currentmarker}{}%
\end{pgfscope}%
\begin{pgfscope}%
\pgfsys@transformshift{3.013245in}{1.251118in}%
\pgfsys@useobject{currentmarker}{}%
\end{pgfscope}%
\begin{pgfscope}%
\pgfsys@transformshift{3.071632in}{1.282490in}%
\pgfsys@useobject{currentmarker}{}%
\end{pgfscope}%
\begin{pgfscope}%
\pgfsys@transformshift{3.138985in}{1.327877in}%
\pgfsys@useobject{currentmarker}{}%
\end{pgfscope}%
\begin{pgfscope}%
\pgfsys@transformshift{3.210642in}{1.379991in}%
\pgfsys@useobject{currentmarker}{}%
\end{pgfscope}%
\begin{pgfscope}%
\pgfsys@transformshift{3.288346in}{1.441559in}%
\pgfsys@useobject{currentmarker}{}%
\end{pgfscope}%
\begin{pgfscope}%
\pgfsys@transformshift{3.354079in}{1.484414in}%
\pgfsys@useobject{currentmarker}{}%
\end{pgfscope}%
\begin{pgfscope}%
\pgfsys@transformshift{3.395162in}{1.488738in}%
\pgfsys@useobject{currentmarker}{}%
\end{pgfscope}%
\begin{pgfscope}%
\pgfsys@transformshift{3.423246in}{1.472743in}%
\pgfsys@useobject{currentmarker}{}%
\end{pgfscope}%
\begin{pgfscope}%
\pgfsys@transformshift{3.450046in}{1.454742in}%
\pgfsys@useobject{currentmarker}{}%
\end{pgfscope}%
\begin{pgfscope}%
\pgfsys@transformshift{3.453543in}{1.400314in}%
\pgfsys@useobject{currentmarker}{}%
\end{pgfscope}%
\begin{pgfscope}%
\pgfsys@transformshift{3.455571in}{1.343591in}%
\pgfsys@useobject{currentmarker}{}%
\end{pgfscope}%
\begin{pgfscope}%
\pgfsys@transformshift{3.477372in}{1.317776in}%
\pgfsys@useobject{currentmarker}{}%
\end{pgfscope}%
\begin{pgfscope}%
\pgfsys@transformshift{3.481371in}{1.264133in}%
\pgfsys@useobject{currentmarker}{}%
\end{pgfscope}%
\begin{pgfscope}%
\pgfsys@transformshift{3.505373in}{1.241757in}%
\pgfsys@useobject{currentmarker}{}%
\end{pgfscope}%
\begin{pgfscope}%
\pgfsys@transformshift{3.526998in}{1.215666in}%
\pgfsys@useobject{currentmarker}{}%
\end{pgfscope}%
\begin{pgfscope}%
\pgfsys@transformshift{3.562203in}{1.210802in}%
\pgfsys@useobject{currentmarker}{}%
\end{pgfscope}%
\begin{pgfscope}%
\pgfsys@transformshift{3.586856in}{1.189444in}%
\pgfsys@useobject{currentmarker}{}%
\end{pgfscope}%
\begin{pgfscope}%
\pgfsys@transformshift{3.629765in}{1.196622in}%
\pgfsys@useobject{currentmarker}{}%
\end{pgfscope}%
\begin{pgfscope}%
\pgfsys@transformshift{3.676730in}{1.210141in}%
\pgfsys@useobject{currentmarker}{}%
\end{pgfscope}%
\begin{pgfscope}%
\pgfsys@transformshift{3.745250in}{1.257352in}%
\pgfsys@useobject{currentmarker}{}%
\end{pgfscope}%
\begin{pgfscope}%
\pgfsys@transformshift{3.816743in}{1.309211in}%
\pgfsys@useobject{currentmarker}{}%
\end{pgfscope}%
\begin{pgfscope}%
\pgfsys@transformshift{3.871459in}{1.334845in}%
\pgfsys@useobject{currentmarker}{}%
\end{pgfscope}%
\begin{pgfscope}%
\pgfsys@transformshift{3.950363in}{1.398288in}%
\pgfsys@useobject{currentmarker}{}%
\end{pgfscope}%
\begin{pgfscope}%
\pgfsys@transformshift{3.993573in}{1.405936in}%
\pgfsys@useobject{currentmarker}{}%
\end{pgfscope}%
\end{pgfscope}%
\begin{pgfscope}%
\pgfpathrectangle{\pgfqpoint{0.552773in}{0.431673in}}{\pgfqpoint{3.738807in}{1.765244in}}%
\pgfusepath{clip}%
\pgfsetbuttcap%
\pgfsetroundjoin%
\pgfsetlinewidth{1.003750pt}%
\definecolor{currentstroke}{rgb}{0.968203,0.720844,0.612293}%
\pgfsetstrokecolor{currentstroke}%
\pgfsetstrokeopacity{0.500000}%
\pgfsetdash{{3.700000pt}{1.600000pt}}{0.000000pt}%
\pgfpathmoveto{\pgfqpoint{0.552773in}{1.339721in}}%
\pgfpathlineto{\pgfqpoint{4.291580in}{1.339721in}}%
\pgfusepath{stroke}%
\end{pgfscope}%
\begin{pgfscope}%
\pgfpathrectangle{\pgfqpoint{0.552773in}{0.431673in}}{\pgfqpoint{3.738807in}{1.765244in}}%
\pgfusepath{clip}%
\pgfsetbuttcap%
\pgfsetroundjoin%
\pgfsetlinewidth{1.003750pt}%
\definecolor{currentstroke}{rgb}{0.705673,0.015556,0.150233}%
\pgfsetstrokecolor{currentstroke}%
\pgfsetdash{}{0pt}%
\pgfpathmoveto{\pgfqpoint{0.722719in}{2.012117in}}%
\pgfpathlineto{\pgfqpoint{0.722719in}{2.012117in}}%
\pgfusepath{stroke}%
\end{pgfscope}%
\begin{pgfscope}%
\pgfpathrectangle{\pgfqpoint{0.552773in}{0.431673in}}{\pgfqpoint{3.738807in}{1.765244in}}%
\pgfusepath{clip}%
\pgfsetbuttcap%
\pgfsetroundjoin%
\pgfsetlinewidth{1.003750pt}%
\definecolor{currentstroke}{rgb}{0.705673,0.015556,0.150233}%
\pgfsetstrokecolor{currentstroke}%
\pgfsetdash{}{0pt}%
\pgfpathmoveto{\pgfqpoint{0.786936in}{1.992022in}}%
\pgfpathlineto{\pgfqpoint{0.786936in}{1.992022in}}%
\pgfusepath{stroke}%
\end{pgfscope}%
\begin{pgfscope}%
\pgfpathrectangle{\pgfqpoint{0.552773in}{0.431673in}}{\pgfqpoint{3.738807in}{1.765244in}}%
\pgfusepath{clip}%
\pgfsetbuttcap%
\pgfsetroundjoin%
\pgfsetlinewidth{1.003750pt}%
\definecolor{currentstroke}{rgb}{0.705673,0.015556,0.150233}%
\pgfsetstrokecolor{currentstroke}%
\pgfsetdash{}{0pt}%
\pgfpathmoveto{\pgfqpoint{0.819797in}{1.983144in}}%
\pgfpathlineto{\pgfqpoint{0.819797in}{1.983144in}}%
\pgfusepath{stroke}%
\end{pgfscope}%
\begin{pgfscope}%
\pgfpathrectangle{\pgfqpoint{0.552773in}{0.431673in}}{\pgfqpoint{3.738807in}{1.765244in}}%
\pgfusepath{clip}%
\pgfsetbuttcap%
\pgfsetroundjoin%
\pgfsetlinewidth{1.003750pt}%
\definecolor{currentstroke}{rgb}{0.705673,0.015556,0.150233}%
\pgfsetstrokecolor{currentstroke}%
\pgfsetdash{}{0pt}%
\pgfpathmoveto{\pgfqpoint{0.831560in}{1.941512in}}%
\pgfpathlineto{\pgfqpoint{0.831560in}{1.941512in}}%
\pgfusepath{stroke}%
\end{pgfscope}%
\begin{pgfscope}%
\pgfpathrectangle{\pgfqpoint{0.552773in}{0.431673in}}{\pgfqpoint{3.738807in}{1.765244in}}%
\pgfusepath{clip}%
\pgfsetbuttcap%
\pgfsetroundjoin%
\pgfsetlinewidth{1.003750pt}%
\definecolor{currentstroke}{rgb}{0.705673,0.015556,0.150233}%
\pgfsetstrokecolor{currentstroke}%
\pgfsetdash{}{0pt}%
\pgfpathmoveto{\pgfqpoint{0.837884in}{1.891436in}}%
\pgfpathlineto{\pgfqpoint{0.837884in}{1.891436in}}%
\pgfusepath{stroke}%
\end{pgfscope}%
\begin{pgfscope}%
\pgfpathrectangle{\pgfqpoint{0.552773in}{0.431673in}}{\pgfqpoint{3.738807in}{1.765244in}}%
\pgfusepath{clip}%
\pgfsetbuttcap%
\pgfsetroundjoin%
\pgfsetlinewidth{1.003750pt}%
\definecolor{currentstroke}{rgb}{0.705673,0.015556,0.150233}%
\pgfsetstrokecolor{currentstroke}%
\pgfsetdash{}{0pt}%
\pgfpathmoveto{\pgfqpoint{0.853441in}{1.855694in}}%
\pgfpathlineto{\pgfqpoint{0.853441in}{1.855694in}}%
\pgfusepath{stroke}%
\end{pgfscope}%
\begin{pgfscope}%
\pgfpathrectangle{\pgfqpoint{0.552773in}{0.431673in}}{\pgfqpoint{3.738807in}{1.765244in}}%
\pgfusepath{clip}%
\pgfsetbuttcap%
\pgfsetroundjoin%
\pgfsetlinewidth{1.003750pt}%
\definecolor{currentstroke}{rgb}{0.705673,0.015556,0.150233}%
\pgfsetstrokecolor{currentstroke}%
\pgfsetdash{}{0pt}%
\pgfpathmoveto{\pgfqpoint{0.871877in}{1.824422in}}%
\pgfpathlineto{\pgfqpoint{0.871877in}{1.824422in}}%
\pgfusepath{stroke}%
\end{pgfscope}%
\begin{pgfscope}%
\pgfpathrectangle{\pgfqpoint{0.552773in}{0.431673in}}{\pgfqpoint{3.738807in}{1.765244in}}%
\pgfusepath{clip}%
\pgfsetbuttcap%
\pgfsetroundjoin%
\pgfsetlinewidth{1.003750pt}%
\definecolor{currentstroke}{rgb}{0.705673,0.015556,0.150233}%
\pgfsetstrokecolor{currentstroke}%
\pgfsetdash{}{0pt}%
\pgfpathmoveto{\pgfqpoint{0.894231in}{1.799231in}}%
\pgfpathlineto{\pgfqpoint{0.894231in}{1.799231in}}%
\pgfusepath{stroke}%
\end{pgfscope}%
\begin{pgfscope}%
\pgfpathrectangle{\pgfqpoint{0.552773in}{0.431673in}}{\pgfqpoint{3.738807in}{1.765244in}}%
\pgfusepath{clip}%
\pgfsetbuttcap%
\pgfsetroundjoin%
\pgfsetlinewidth{1.003750pt}%
\definecolor{currentstroke}{rgb}{0.705673,0.015556,0.150233}%
\pgfsetstrokecolor{currentstroke}%
\pgfsetdash{}{0pt}%
\pgfpathmoveto{\pgfqpoint{0.917097in}{1.774835in}}%
\pgfpathlineto{\pgfqpoint{0.917097in}{1.774835in}}%
\pgfusepath{stroke}%
\end{pgfscope}%
\begin{pgfscope}%
\pgfpathrectangle{\pgfqpoint{0.552773in}{0.431673in}}{\pgfqpoint{3.738807in}{1.765244in}}%
\pgfusepath{clip}%
\pgfsetbuttcap%
\pgfsetroundjoin%
\pgfsetlinewidth{1.003750pt}%
\definecolor{currentstroke}{rgb}{0.705673,0.015556,0.150233}%
\pgfsetstrokecolor{currentstroke}%
\pgfsetdash{}{0pt}%
\pgfpathmoveto{\pgfqpoint{0.954263in}{1.772639in}}%
\pgfpathlineto{\pgfqpoint{0.954263in}{1.772639in}}%
\pgfusepath{stroke}%
\end{pgfscope}%
\begin{pgfscope}%
\pgfpathrectangle{\pgfqpoint{0.552773in}{0.431673in}}{\pgfqpoint{3.738807in}{1.765244in}}%
\pgfusepath{clip}%
\pgfsetbuttcap%
\pgfsetroundjoin%
\pgfsetlinewidth{1.003750pt}%
\definecolor{currentstroke}{rgb}{0.705673,0.015556,0.150233}%
\pgfsetstrokecolor{currentstroke}%
\pgfsetdash{}{0pt}%
\pgfpathmoveto{\pgfqpoint{1.054856in}{1.809017in}}%
\pgfpathlineto{\pgfqpoint{1.054856in}{1.809017in}}%
\pgfusepath{stroke}%
\end{pgfscope}%
\begin{pgfscope}%
\pgfpathrectangle{\pgfqpoint{0.552773in}{0.431673in}}{\pgfqpoint{3.738807in}{1.765244in}}%
\pgfusepath{clip}%
\pgfsetbuttcap%
\pgfsetroundjoin%
\pgfsetlinewidth{1.003750pt}%
\definecolor{currentstroke}{rgb}{0.705673,0.015556,0.150233}%
\pgfsetstrokecolor{currentstroke}%
\pgfsetdash{}{0pt}%
\pgfpathmoveto{\pgfqpoint{1.127913in}{1.862539in}}%
\pgfpathlineto{\pgfqpoint{1.127913in}{1.862539in}}%
\pgfusepath{stroke}%
\end{pgfscope}%
\begin{pgfscope}%
\pgfpathrectangle{\pgfqpoint{0.552773in}{0.431673in}}{\pgfqpoint{3.738807in}{1.765244in}}%
\pgfusepath{clip}%
\pgfsetbuttcap%
\pgfsetroundjoin%
\pgfsetlinewidth{1.003750pt}%
\definecolor{currentstroke}{rgb}{0.705673,0.015556,0.150233}%
\pgfsetstrokecolor{currentstroke}%
\pgfsetdash{}{0pt}%
\pgfpathmoveto{\pgfqpoint{1.216245in}{1.939776in}}%
\pgfpathlineto{\pgfqpoint{1.216245in}{1.939776in}}%
\pgfusepath{stroke}%
\end{pgfscope}%
\begin{pgfscope}%
\pgfpathrectangle{\pgfqpoint{0.552773in}{0.431673in}}{\pgfqpoint{3.738807in}{1.765244in}}%
\pgfusepath{clip}%
\pgfsetbuttcap%
\pgfsetroundjoin%
\pgfsetlinewidth{1.003750pt}%
\definecolor{currentstroke}{rgb}{0.705673,0.015556,0.150233}%
\pgfsetstrokecolor{currentstroke}%
\pgfsetdash{}{0pt}%
\pgfpathmoveto{\pgfqpoint{1.299586in}{2.009263in}}%
\pgfpathlineto{\pgfqpoint{1.299586in}{2.009263in}}%
\pgfusepath{stroke}%
\end{pgfscope}%
\begin{pgfscope}%
\pgfpathrectangle{\pgfqpoint{0.552773in}{0.431673in}}{\pgfqpoint{3.738807in}{1.765244in}}%
\pgfusepath{clip}%
\pgfsetbuttcap%
\pgfsetroundjoin%
\pgfsetlinewidth{1.003750pt}%
\definecolor{currentstroke}{rgb}{0.705673,0.015556,0.150233}%
\pgfsetstrokecolor{currentstroke}%
\pgfsetdash{}{0pt}%
\pgfpathmoveto{\pgfqpoint{1.383835in}{2.080160in}}%
\pgfpathlineto{\pgfqpoint{1.383835in}{2.080160in}}%
\pgfusepath{stroke}%
\end{pgfscope}%
\begin{pgfscope}%
\pgfpathrectangle{\pgfqpoint{0.552773in}{0.431673in}}{\pgfqpoint{3.738807in}{1.765244in}}%
\pgfusepath{clip}%
\pgfsetbuttcap%
\pgfsetroundjoin%
\pgfsetlinewidth{1.003750pt}%
\definecolor{currentstroke}{rgb}{0.705673,0.015556,0.150233}%
\pgfsetstrokecolor{currentstroke}%
\pgfsetdash{}{0pt}%
\pgfpathmoveto{\pgfqpoint{1.428876in}{2.090189in}}%
\pgfpathlineto{\pgfqpoint{1.428876in}{2.090189in}}%
\pgfusepath{stroke}%
\end{pgfscope}%
\begin{pgfscope}%
\pgfpathrectangle{\pgfqpoint{0.552773in}{0.431673in}}{\pgfqpoint{3.738807in}{1.765244in}}%
\pgfusepath{clip}%
\pgfsetbuttcap%
\pgfsetroundjoin%
\pgfsetlinewidth{1.003750pt}%
\definecolor{currentstroke}{rgb}{0.705673,0.015556,0.150233}%
\pgfsetstrokecolor{currentstroke}%
\pgfsetdash{}{0pt}%
\pgfpathmoveto{\pgfqpoint{1.461920in}{2.081596in}}%
\pgfpathlineto{\pgfqpoint{1.461920in}{2.081596in}}%
\pgfusepath{stroke}%
\end{pgfscope}%
\begin{pgfscope}%
\pgfpathrectangle{\pgfqpoint{0.552773in}{0.431673in}}{\pgfqpoint{3.738807in}{1.765244in}}%
\pgfusepath{clip}%
\pgfsetbuttcap%
\pgfsetroundjoin%
\pgfsetlinewidth{1.003750pt}%
\definecolor{currentstroke}{rgb}{0.705673,0.015556,0.150233}%
\pgfsetstrokecolor{currentstroke}%
\pgfsetdash{}{0pt}%
\pgfpathmoveto{\pgfqpoint{1.470951in}{2.035721in}}%
\pgfpathlineto{\pgfqpoint{1.470951in}{2.035721in}}%
\pgfusepath{stroke}%
\end{pgfscope}%
\begin{pgfscope}%
\pgfpathrectangle{\pgfqpoint{0.552773in}{0.431673in}}{\pgfqpoint{3.738807in}{1.765244in}}%
\pgfusepath{clip}%
\pgfsetbuttcap%
\pgfsetroundjoin%
\pgfsetlinewidth{1.003750pt}%
\definecolor{currentstroke}{rgb}{0.705673,0.015556,0.150233}%
\pgfsetstrokecolor{currentstroke}%
\pgfsetdash{}{0pt}%
\pgfpathmoveto{\pgfqpoint{1.500704in}{2.022017in}}%
\pgfpathlineto{\pgfqpoint{1.500704in}{2.022017in}}%
\pgfusepath{stroke}%
\end{pgfscope}%
\begin{pgfscope}%
\pgfpathrectangle{\pgfqpoint{0.552773in}{0.431673in}}{\pgfqpoint{3.738807in}{1.765244in}}%
\pgfusepath{clip}%
\pgfsetbuttcap%
\pgfsetroundjoin%
\pgfsetlinewidth{1.003750pt}%
\definecolor{currentstroke}{rgb}{0.705673,0.015556,0.150233}%
\pgfsetstrokecolor{currentstroke}%
\pgfsetdash{}{0pt}%
\pgfpathmoveto{\pgfqpoint{1.498835in}{1.959223in}}%
\pgfpathlineto{\pgfqpoint{1.498835in}{1.959223in}}%
\pgfusepath{stroke}%
\end{pgfscope}%
\begin{pgfscope}%
\pgfpathrectangle{\pgfqpoint{0.552773in}{0.431673in}}{\pgfqpoint{3.738807in}{1.765244in}}%
\pgfusepath{clip}%
\pgfsetbuttcap%
\pgfsetroundjoin%
\pgfsetlinewidth{1.003750pt}%
\definecolor{currentstroke}{rgb}{0.705673,0.015556,0.150233}%
\pgfsetstrokecolor{currentstroke}%
\pgfsetdash{}{0pt}%
\pgfpathmoveto{\pgfqpoint{1.510085in}{1.916795in}}%
\pgfpathlineto{\pgfqpoint{1.510085in}{1.916795in}}%
\pgfusepath{stroke}%
\end{pgfscope}%
\begin{pgfscope}%
\pgfpathrectangle{\pgfqpoint{0.552773in}{0.431673in}}{\pgfqpoint{3.738807in}{1.765244in}}%
\pgfusepath{clip}%
\pgfsetbuttcap%
\pgfsetroundjoin%
\pgfsetlinewidth{1.003750pt}%
\definecolor{currentstroke}{rgb}{0.705673,0.015556,0.150233}%
\pgfsetstrokecolor{currentstroke}%
\pgfsetdash{}{0pt}%
\pgfpathmoveto{\pgfqpoint{1.536220in}{1.897473in}}%
\pgfpathlineto{\pgfqpoint{1.536220in}{1.897473in}}%
\pgfusepath{stroke}%
\end{pgfscope}%
\begin{pgfscope}%
\pgfpathrectangle{\pgfqpoint{0.552773in}{0.431673in}}{\pgfqpoint{3.738807in}{1.765244in}}%
\pgfusepath{clip}%
\pgfsetbuttcap%
\pgfsetroundjoin%
\pgfsetlinewidth{1.003750pt}%
\definecolor{currentstroke}{rgb}{0.705673,0.015556,0.150233}%
\pgfsetstrokecolor{currentstroke}%
\pgfsetdash{}{0pt}%
\pgfpathmoveto{\pgfqpoint{1.565209in}{1.882584in}}%
\pgfpathlineto{\pgfqpoint{1.565209in}{1.882584in}}%
\pgfusepath{stroke}%
\end{pgfscope}%
\begin{pgfscope}%
\pgfpathrectangle{\pgfqpoint{0.552773in}{0.431673in}}{\pgfqpoint{3.738807in}{1.765244in}}%
\pgfusepath{clip}%
\pgfsetbuttcap%
\pgfsetroundjoin%
\pgfsetlinewidth{1.003750pt}%
\definecolor{currentstroke}{rgb}{0.705673,0.015556,0.150233}%
\pgfsetstrokecolor{currentstroke}%
\pgfsetdash{}{0pt}%
\pgfpathmoveto{\pgfqpoint{1.587354in}{1.857070in}}%
\pgfpathlineto{\pgfqpoint{1.587354in}{1.857070in}}%
\pgfusepath{stroke}%
\end{pgfscope}%
\begin{pgfscope}%
\pgfpathrectangle{\pgfqpoint{0.552773in}{0.431673in}}{\pgfqpoint{3.738807in}{1.765244in}}%
\pgfusepath{clip}%
\pgfsetbuttcap%
\pgfsetroundjoin%
\pgfsetlinewidth{1.003750pt}%
\definecolor{currentstroke}{rgb}{0.705673,0.015556,0.150233}%
\pgfsetstrokecolor{currentstroke}%
\pgfsetdash{}{0pt}%
\pgfpathmoveto{\pgfqpoint{1.624003in}{1.854071in}}%
\pgfpathlineto{\pgfqpoint{1.624003in}{1.854071in}}%
\pgfusepath{stroke}%
\end{pgfscope}%
\begin{pgfscope}%
\pgfpathrectangle{\pgfqpoint{0.552773in}{0.431673in}}{\pgfqpoint{3.738807in}{1.765244in}}%
\pgfusepath{clip}%
\pgfsetbuttcap%
\pgfsetroundjoin%
\pgfsetlinewidth{1.003750pt}%
\definecolor{currentstroke}{rgb}{0.705673,0.015556,0.150233}%
\pgfsetstrokecolor{currentstroke}%
\pgfsetdash{}{0pt}%
\pgfpathmoveto{\pgfqpoint{1.656435in}{1.844526in}}%
\pgfpathlineto{\pgfqpoint{1.656435in}{1.844526in}}%
\pgfusepath{stroke}%
\end{pgfscope}%
\begin{pgfscope}%
\pgfpathrectangle{\pgfqpoint{0.552773in}{0.431673in}}{\pgfqpoint{3.738807in}{1.765244in}}%
\pgfusepath{clip}%
\pgfsetbuttcap%
\pgfsetroundjoin%
\pgfsetlinewidth{1.003750pt}%
\definecolor{currentstroke}{rgb}{0.705673,0.015556,0.150233}%
\pgfsetstrokecolor{currentstroke}%
\pgfsetdash{}{0pt}%
\pgfpathmoveto{\pgfqpoint{1.704526in}{1.859291in}}%
\pgfpathlineto{\pgfqpoint{1.704526in}{1.859291in}}%
\pgfusepath{stroke}%
\end{pgfscope}%
\begin{pgfscope}%
\pgfpathrectangle{\pgfqpoint{0.552773in}{0.431673in}}{\pgfqpoint{3.738807in}{1.765244in}}%
\pgfusepath{clip}%
\pgfsetbuttcap%
\pgfsetroundjoin%
\pgfsetlinewidth{1.003750pt}%
\definecolor{currentstroke}{rgb}{0.705673,0.015556,0.150233}%
\pgfsetstrokecolor{currentstroke}%
\pgfsetdash{}{0pt}%
\pgfpathmoveto{\pgfqpoint{1.761295in}{1.887527in}}%
\pgfpathlineto{\pgfqpoint{1.761295in}{1.887527in}}%
\pgfusepath{stroke}%
\end{pgfscope}%
\begin{pgfscope}%
\pgfpathrectangle{\pgfqpoint{0.552773in}{0.431673in}}{\pgfqpoint{3.738807in}{1.765244in}}%
\pgfusepath{clip}%
\pgfsetbuttcap%
\pgfsetroundjoin%
\pgfsetlinewidth{1.003750pt}%
\definecolor{currentstroke}{rgb}{0.705673,0.015556,0.150233}%
\pgfsetstrokecolor{currentstroke}%
\pgfsetdash{}{0pt}%
\pgfpathmoveto{\pgfqpoint{1.828491in}{1.931951in}}%
\pgfpathlineto{\pgfqpoint{1.828491in}{1.931951in}}%
\pgfusepath{stroke}%
\end{pgfscope}%
\begin{pgfscope}%
\pgfpathrectangle{\pgfqpoint{0.552773in}{0.431673in}}{\pgfqpoint{3.738807in}{1.765244in}}%
\pgfusepath{clip}%
\pgfsetbuttcap%
\pgfsetroundjoin%
\pgfsetlinewidth{1.003750pt}%
\definecolor{currentstroke}{rgb}{0.705673,0.015556,0.150233}%
\pgfsetstrokecolor{currentstroke}%
\pgfsetdash{}{0pt}%
\pgfpathmoveto{\pgfqpoint{1.902753in}{1.987345in}}%
\pgfpathlineto{\pgfqpoint{1.902753in}{1.987345in}}%
\pgfusepath{stroke}%
\end{pgfscope}%
\begin{pgfscope}%
\pgfpathrectangle{\pgfqpoint{0.552773in}{0.431673in}}{\pgfqpoint{3.738807in}{1.765244in}}%
\pgfusepath{clip}%
\pgfsetbuttcap%
\pgfsetroundjoin%
\pgfsetlinewidth{1.003750pt}%
\definecolor{currentstroke}{rgb}{0.705673,0.015556,0.150233}%
\pgfsetstrokecolor{currentstroke}%
\pgfsetdash{}{0pt}%
\pgfpathmoveto{\pgfqpoint{1.989053in}{2.061425in}}%
\pgfpathlineto{\pgfqpoint{1.989053in}{2.061425in}}%
\pgfusepath{stroke}%
\end{pgfscope}%
\begin{pgfscope}%
\pgfpathrectangle{\pgfqpoint{0.552773in}{0.431673in}}{\pgfqpoint{3.738807in}{1.765244in}}%
\pgfusepath{clip}%
\pgfsetbuttcap%
\pgfsetroundjoin%
\pgfsetlinewidth{1.003750pt}%
\definecolor{currentstroke}{rgb}{0.705673,0.015556,0.150233}%
\pgfsetstrokecolor{currentstroke}%
\pgfsetdash{}{0pt}%
\pgfpathmoveto{\pgfqpoint{2.045418in}{2.089036in}}%
\pgfpathlineto{\pgfqpoint{2.045418in}{2.089036in}}%
\pgfusepath{stroke}%
\end{pgfscope}%
\begin{pgfscope}%
\pgfpathrectangle{\pgfqpoint{0.552773in}{0.431673in}}{\pgfqpoint{3.738807in}{1.765244in}}%
\pgfusepath{clip}%
\pgfsetbuttcap%
\pgfsetroundjoin%
\pgfsetlinewidth{1.003750pt}%
\definecolor{currentstroke}{rgb}{0.705673,0.015556,0.150233}%
\pgfsetstrokecolor{currentstroke}%
\pgfsetdash{}{0pt}%
\pgfpathmoveto{\pgfqpoint{2.101804in}{2.116678in}}%
\pgfpathlineto{\pgfqpoint{2.101804in}{2.116678in}}%
\pgfusepath{stroke}%
\end{pgfscope}%
\begin{pgfscope}%
\pgfpathrectangle{\pgfqpoint{0.552773in}{0.431673in}}{\pgfqpoint{3.738807in}{1.765244in}}%
\pgfusepath{clip}%
\pgfsetbuttcap%
\pgfsetroundjoin%
\pgfsetlinewidth{1.003750pt}%
\definecolor{currentstroke}{rgb}{0.705673,0.015556,0.150233}%
\pgfsetstrokecolor{currentstroke}%
\pgfsetdash{}{0pt}%
\pgfpathmoveto{\pgfqpoint{2.115398in}{2.077888in}}%
\pgfpathlineto{\pgfqpoint{2.115398in}{2.077888in}}%
\pgfusepath{stroke}%
\end{pgfscope}%
\begin{pgfscope}%
\pgfpathrectangle{\pgfqpoint{0.552773in}{0.431673in}}{\pgfqpoint{3.738807in}{1.765244in}}%
\pgfusepath{clip}%
\pgfsetbuttcap%
\pgfsetroundjoin%
\pgfsetlinewidth{1.003750pt}%
\definecolor{currentstroke}{rgb}{0.705673,0.015556,0.150233}%
\pgfsetstrokecolor{currentstroke}%
\pgfsetdash{}{0pt}%
\pgfpathmoveto{\pgfqpoint{2.157435in}{2.083254in}}%
\pgfpathlineto{\pgfqpoint{2.157435in}{2.083254in}}%
\pgfusepath{stroke}%
\end{pgfscope}%
\begin{pgfscope}%
\pgfpathrectangle{\pgfqpoint{0.552773in}{0.431673in}}{\pgfqpoint{3.738807in}{1.765244in}}%
\pgfusepath{clip}%
\pgfsetbuttcap%
\pgfsetroundjoin%
\pgfsetlinewidth{1.003750pt}%
\definecolor{currentstroke}{rgb}{0.705673,0.015556,0.150233}%
\pgfsetstrokecolor{currentstroke}%
\pgfsetdash{}{0pt}%
\pgfpathmoveto{\pgfqpoint{2.162229in}{2.030803in}}%
\pgfpathlineto{\pgfqpoint{2.162229in}{2.030803in}}%
\pgfusepath{stroke}%
\end{pgfscope}%
\begin{pgfscope}%
\pgfpathrectangle{\pgfqpoint{0.552773in}{0.431673in}}{\pgfqpoint{3.738807in}{1.765244in}}%
\pgfusepath{clip}%
\pgfsetbuttcap%
\pgfsetroundjoin%
\pgfsetlinewidth{1.003750pt}%
\definecolor{currentstroke}{rgb}{0.705673,0.015556,0.150233}%
\pgfsetstrokecolor{currentstroke}%
\pgfsetdash{}{0pt}%
\pgfpathmoveto{\pgfqpoint{2.178795in}{1.996627in}}%
\pgfpathlineto{\pgfqpoint{2.178795in}{1.996627in}}%
\pgfusepath{stroke}%
\end{pgfscope}%
\begin{pgfscope}%
\pgfpathrectangle{\pgfqpoint{0.552773in}{0.431673in}}{\pgfqpoint{3.738807in}{1.765244in}}%
\pgfusepath{clip}%
\pgfsetbuttcap%
\pgfsetroundjoin%
\pgfsetlinewidth{1.003750pt}%
\definecolor{currentstroke}{rgb}{0.705673,0.015556,0.150233}%
\pgfsetstrokecolor{currentstroke}%
\pgfsetdash{}{0pt}%
\pgfpathmoveto{\pgfqpoint{2.190577in}{1.955024in}}%
\pgfpathlineto{\pgfqpoint{2.190577in}{1.955024in}}%
\pgfusepath{stroke}%
\end{pgfscope}%
\begin{pgfscope}%
\pgfpathrectangle{\pgfqpoint{0.552773in}{0.431673in}}{\pgfqpoint{3.738807in}{1.765244in}}%
\pgfusepath{clip}%
\pgfsetbuttcap%
\pgfsetroundjoin%
\pgfsetlinewidth{1.003750pt}%
\definecolor{currentstroke}{rgb}{0.705673,0.015556,0.150233}%
\pgfsetstrokecolor{currentstroke}%
\pgfsetdash{}{0pt}%
\pgfpathmoveto{\pgfqpoint{2.210306in}{1.925759in}}%
\pgfpathlineto{\pgfqpoint{2.210306in}{1.925759in}}%
\pgfusepath{stroke}%
\end{pgfscope}%
\begin{pgfscope}%
\pgfpathrectangle{\pgfqpoint{0.552773in}{0.431673in}}{\pgfqpoint{3.738807in}{1.765244in}}%
\pgfusepath{clip}%
\pgfsetbuttcap%
\pgfsetroundjoin%
\pgfsetlinewidth{1.003750pt}%
\definecolor{currentstroke}{rgb}{0.705673,0.015556,0.150233}%
\pgfsetstrokecolor{currentstroke}%
\pgfsetdash{}{0pt}%
\pgfpathmoveto{\pgfqpoint{2.230408in}{1.897072in}}%
\pgfpathlineto{\pgfqpoint{2.230408in}{1.897072in}}%
\pgfusepath{stroke}%
\end{pgfscope}%
\begin{pgfscope}%
\pgfpathrectangle{\pgfqpoint{0.552773in}{0.431673in}}{\pgfqpoint{3.738807in}{1.765244in}}%
\pgfusepath{clip}%
\pgfsetbuttcap%
\pgfsetroundjoin%
\pgfsetlinewidth{1.003750pt}%
\definecolor{currentstroke}{rgb}{0.705673,0.015556,0.150233}%
\pgfsetstrokecolor{currentstroke}%
\pgfsetdash{}{0pt}%
\pgfpathmoveto{\pgfqpoint{2.261554in}{1.885531in}}%
\pgfpathlineto{\pgfqpoint{2.261554in}{1.885531in}}%
\pgfusepath{stroke}%
\end{pgfscope}%
\begin{pgfscope}%
\pgfpathrectangle{\pgfqpoint{0.552773in}{0.431673in}}{\pgfqpoint{3.738807in}{1.765244in}}%
\pgfusepath{clip}%
\pgfsetbuttcap%
\pgfsetroundjoin%
\pgfsetlinewidth{1.003750pt}%
\definecolor{currentstroke}{rgb}{0.705673,0.015556,0.150233}%
\pgfsetstrokecolor{currentstroke}%
\pgfsetdash{}{0pt}%
\pgfpathmoveto{\pgfqpoint{2.282599in}{1.858308in}}%
\pgfpathlineto{\pgfqpoint{2.282599in}{1.858308in}}%
\pgfusepath{stroke}%
\end{pgfscope}%
\begin{pgfscope}%
\pgfpathrectangle{\pgfqpoint{0.552773in}{0.431673in}}{\pgfqpoint{3.738807in}{1.765244in}}%
\pgfusepath{clip}%
\pgfsetbuttcap%
\pgfsetroundjoin%
\pgfsetlinewidth{1.003750pt}%
\definecolor{currentstroke}{rgb}{0.705673,0.015556,0.150233}%
\pgfsetstrokecolor{currentstroke}%
\pgfsetdash{}{0pt}%
\pgfpathmoveto{\pgfqpoint{2.311609in}{1.843452in}}%
\pgfpathlineto{\pgfqpoint{2.311609in}{1.843452in}}%
\pgfusepath{stroke}%
\end{pgfscope}%
\begin{pgfscope}%
\pgfpathrectangle{\pgfqpoint{0.552773in}{0.431673in}}{\pgfqpoint{3.738807in}{1.765244in}}%
\pgfusepath{clip}%
\pgfsetbuttcap%
\pgfsetroundjoin%
\pgfsetlinewidth{1.003750pt}%
\definecolor{currentstroke}{rgb}{0.705673,0.015556,0.150233}%
\pgfsetstrokecolor{currentstroke}%
\pgfsetdash{}{0pt}%
\pgfpathmoveto{\pgfqpoint{2.360121in}{1.858871in}}%
\pgfpathlineto{\pgfqpoint{2.360121in}{1.858871in}}%
\pgfusepath{stroke}%
\end{pgfscope}%
\begin{pgfscope}%
\pgfpathrectangle{\pgfqpoint{0.552773in}{0.431673in}}{\pgfqpoint{3.738807in}{1.765244in}}%
\pgfusepath{clip}%
\pgfsetbuttcap%
\pgfsetroundjoin%
\pgfsetlinewidth{1.003750pt}%
\definecolor{currentstroke}{rgb}{0.705673,0.015556,0.150233}%
\pgfsetstrokecolor{currentstroke}%
\pgfsetdash{}{0pt}%
\pgfpathmoveto{\pgfqpoint{2.409930in}{1.876302in}}%
\pgfpathlineto{\pgfqpoint{2.409930in}{1.876302in}}%
\pgfusepath{stroke}%
\end{pgfscope}%
\begin{pgfscope}%
\pgfpathrectangle{\pgfqpoint{0.552773in}{0.431673in}}{\pgfqpoint{3.738807in}{1.765244in}}%
\pgfusepath{clip}%
\pgfsetbuttcap%
\pgfsetroundjoin%
\pgfsetlinewidth{1.003750pt}%
\definecolor{currentstroke}{rgb}{0.705673,0.015556,0.150233}%
\pgfsetstrokecolor{currentstroke}%
\pgfsetdash{}{0pt}%
\pgfpathmoveto{\pgfqpoint{2.475675in}{1.918474in}}%
\pgfpathlineto{\pgfqpoint{2.475675in}{1.918474in}}%
\pgfusepath{stroke}%
\end{pgfscope}%
\begin{pgfscope}%
\pgfpathrectangle{\pgfqpoint{0.552773in}{0.431673in}}{\pgfqpoint{3.738807in}{1.765244in}}%
\pgfusepath{clip}%
\pgfsetbuttcap%
\pgfsetroundjoin%
\pgfsetlinewidth{1.003750pt}%
\definecolor{currentstroke}{rgb}{0.705673,0.015556,0.150233}%
\pgfsetstrokecolor{currentstroke}%
\pgfsetdash{}{0pt}%
\pgfpathmoveto{\pgfqpoint{2.551377in}{1.976101in}}%
\pgfpathlineto{\pgfqpoint{2.551377in}{1.976101in}}%
\pgfusepath{stroke}%
\end{pgfscope}%
\begin{pgfscope}%
\pgfpathrectangle{\pgfqpoint{0.552773in}{0.431673in}}{\pgfqpoint{3.738807in}{1.765244in}}%
\pgfusepath{clip}%
\pgfsetbuttcap%
\pgfsetroundjoin%
\pgfsetlinewidth{1.003750pt}%
\definecolor{currentstroke}{rgb}{0.705673,0.015556,0.150233}%
\pgfsetstrokecolor{currentstroke}%
\pgfsetdash{}{0pt}%
\pgfpathmoveto{\pgfqpoint{2.632372in}{2.041948in}}%
\pgfpathlineto{\pgfqpoint{2.632372in}{2.041948in}}%
\pgfusepath{stroke}%
\end{pgfscope}%
\begin{pgfscope}%
\pgfpathrectangle{\pgfqpoint{0.552773in}{0.431673in}}{\pgfqpoint{3.738807in}{1.765244in}}%
\pgfusepath{clip}%
\pgfsetbuttcap%
\pgfsetroundjoin%
\pgfsetlinewidth{1.003750pt}%
\definecolor{currentstroke}{rgb}{0.705673,0.015556,0.150233}%
\pgfsetstrokecolor{currentstroke}%
\pgfsetdash{}{0pt}%
\pgfpathmoveto{\pgfqpoint{2.694363in}{2.078291in}}%
\pgfpathlineto{\pgfqpoint{2.694363in}{2.078291in}}%
\pgfusepath{stroke}%
\end{pgfscope}%
\begin{pgfscope}%
\pgfpathrectangle{\pgfqpoint{0.552773in}{0.431673in}}{\pgfqpoint{3.738807in}{1.765244in}}%
\pgfusepath{clip}%
\pgfsetbuttcap%
\pgfsetroundjoin%
\pgfsetlinewidth{1.003750pt}%
\definecolor{currentstroke}{rgb}{0.705673,0.015556,0.150233}%
\pgfsetstrokecolor{currentstroke}%
\pgfsetdash{}{0pt}%
\pgfpathmoveto{\pgfqpoint{2.745485in}{2.097761in}}%
\pgfpathlineto{\pgfqpoint{2.745485in}{2.097761in}}%
\pgfusepath{stroke}%
\end{pgfscope}%
\begin{pgfscope}%
\pgfpathrectangle{\pgfqpoint{0.552773in}{0.431673in}}{\pgfqpoint{3.738807in}{1.765244in}}%
\pgfusepath{clip}%
\pgfsetbuttcap%
\pgfsetroundjoin%
\pgfsetlinewidth{1.003750pt}%
\definecolor{currentstroke}{rgb}{0.705673,0.015556,0.150233}%
\pgfsetstrokecolor{currentstroke}%
\pgfsetdash{}{0pt}%
\pgfpathmoveto{\pgfqpoint{2.762412in}{2.064146in}}%
\pgfpathlineto{\pgfqpoint{2.762412in}{2.064146in}}%
\pgfusepath{stroke}%
\end{pgfscope}%
\begin{pgfscope}%
\pgfpathrectangle{\pgfqpoint{0.552773in}{0.431673in}}{\pgfqpoint{3.738807in}{1.765244in}}%
\pgfusepath{clip}%
\pgfsetbuttcap%
\pgfsetroundjoin%
\pgfsetlinewidth{1.003750pt}%
\definecolor{currentstroke}{rgb}{0.705673,0.015556,0.150233}%
\pgfsetstrokecolor{currentstroke}%
\pgfsetdash{}{0pt}%
\pgfpathmoveto{\pgfqpoint{2.787529in}{2.043244in}}%
\pgfpathlineto{\pgfqpoint{2.787529in}{2.043244in}}%
\pgfusepath{stroke}%
\end{pgfscope}%
\begin{pgfscope}%
\pgfpathrectangle{\pgfqpoint{0.552773in}{0.431673in}}{\pgfqpoint{3.738807in}{1.765244in}}%
\pgfusepath{clip}%
\pgfsetbuttcap%
\pgfsetroundjoin%
\pgfsetlinewidth{1.003750pt}%
\definecolor{currentstroke}{rgb}{0.705673,0.015556,0.150233}%
\pgfsetstrokecolor{currentstroke}%
\pgfsetdash{}{0pt}%
\pgfpathmoveto{\pgfqpoint{2.777791in}{1.968233in}}%
\pgfpathlineto{\pgfqpoint{2.777791in}{1.968233in}}%
\pgfusepath{stroke}%
\end{pgfscope}%
\begin{pgfscope}%
\pgfpathrectangle{\pgfqpoint{0.552773in}{0.431673in}}{\pgfqpoint{3.738807in}{1.765244in}}%
\pgfusepath{clip}%
\pgfsetbuttcap%
\pgfsetroundjoin%
\pgfsetlinewidth{1.003750pt}%
\definecolor{currentstroke}{rgb}{0.705673,0.015556,0.150233}%
\pgfsetstrokecolor{currentstroke}%
\pgfsetdash{}{0pt}%
\pgfpathmoveto{\pgfqpoint{2.788868in}{1.925536in}}%
\pgfpathlineto{\pgfqpoint{2.788868in}{1.925536in}}%
\pgfusepath{stroke}%
\end{pgfscope}%
\begin{pgfscope}%
\pgfpathrectangle{\pgfqpoint{0.552773in}{0.431673in}}{\pgfqpoint{3.738807in}{1.765244in}}%
\pgfusepath{clip}%
\pgfsetbuttcap%
\pgfsetroundjoin%
\pgfsetlinewidth{1.003750pt}%
\definecolor{currentstroke}{rgb}{0.705673,0.015556,0.150233}%
\pgfsetstrokecolor{currentstroke}%
\pgfsetdash{}{0pt}%
\pgfpathmoveto{\pgfqpoint{2.856282in}{1.850512in}}%
\pgfpathlineto{\pgfqpoint{2.856282in}{1.850512in}}%
\pgfusepath{stroke}%
\end{pgfscope}%
\begin{pgfscope}%
\pgfpathrectangle{\pgfqpoint{0.552773in}{0.431673in}}{\pgfqpoint{3.738807in}{1.765244in}}%
\pgfusepath{clip}%
\pgfsetbuttcap%
\pgfsetroundjoin%
\pgfsetlinewidth{1.003750pt}%
\definecolor{currentstroke}{rgb}{0.705673,0.015556,0.150233}%
\pgfsetstrokecolor{currentstroke}%
\pgfsetdash{}{0pt}%
\pgfpathmoveto{\pgfqpoint{2.870607in}{1.812857in}}%
\pgfpathlineto{\pgfqpoint{2.870607in}{1.812857in}}%
\pgfusepath{stroke}%
\end{pgfscope}%
\begin{pgfscope}%
\pgfpathrectangle{\pgfqpoint{0.552773in}{0.431673in}}{\pgfqpoint{3.738807in}{1.765244in}}%
\pgfusepath{clip}%
\pgfsetbuttcap%
\pgfsetroundjoin%
\pgfsetlinewidth{1.003750pt}%
\definecolor{currentstroke}{rgb}{0.705673,0.015556,0.150233}%
\pgfsetstrokecolor{currentstroke}%
\pgfsetdash{}{0pt}%
\pgfpathmoveto{\pgfqpoint{2.897552in}{1.794794in}}%
\pgfpathlineto{\pgfqpoint{2.897552in}{1.794794in}}%
\pgfusepath{stroke}%
\end{pgfscope}%
\begin{pgfscope}%
\pgfpathrectangle{\pgfqpoint{0.552773in}{0.431673in}}{\pgfqpoint{3.738807in}{1.765244in}}%
\pgfusepath{clip}%
\pgfsetbuttcap%
\pgfsetroundjoin%
\pgfsetlinewidth{1.003750pt}%
\definecolor{currentstroke}{rgb}{0.705673,0.015556,0.150233}%
\pgfsetstrokecolor{currentstroke}%
\pgfsetdash{}{0pt}%
\pgfpathmoveto{\pgfqpoint{2.926742in}{1.780216in}}%
\pgfpathlineto{\pgfqpoint{2.926742in}{1.780216in}}%
\pgfusepath{stroke}%
\end{pgfscope}%
\begin{pgfscope}%
\pgfpathrectangle{\pgfqpoint{0.552773in}{0.431673in}}{\pgfqpoint{3.738807in}{1.765244in}}%
\pgfusepath{clip}%
\pgfsetbuttcap%
\pgfsetroundjoin%
\pgfsetlinewidth{1.003750pt}%
\definecolor{currentstroke}{rgb}{0.705673,0.015556,0.150233}%
\pgfsetstrokecolor{currentstroke}%
\pgfsetdash{}{0pt}%
\pgfpathmoveto{\pgfqpoint{3.036531in}{1.830870in}}%
\pgfpathlineto{\pgfqpoint{3.036531in}{1.830870in}}%
\pgfusepath{stroke}%
\end{pgfscope}%
\begin{pgfscope}%
\pgfpathrectangle{\pgfqpoint{0.552773in}{0.431673in}}{\pgfqpoint{3.738807in}{1.765244in}}%
\pgfusepath{clip}%
\pgfsetbuttcap%
\pgfsetroundjoin%
\pgfsetlinewidth{1.003750pt}%
\definecolor{currentstroke}{rgb}{0.705673,0.015556,0.150233}%
\pgfsetstrokecolor{currentstroke}%
\pgfsetdash{}{0pt}%
\pgfpathmoveto{\pgfqpoint{3.326910in}{2.042090in}}%
\pgfpathlineto{\pgfqpoint{3.326910in}{2.042090in}}%
\pgfusepath{stroke}%
\end{pgfscope}%
\begin{pgfscope}%
\pgfpathrectangle{\pgfqpoint{0.552773in}{0.431673in}}{\pgfqpoint{3.738807in}{1.765244in}}%
\pgfusepath{clip}%
\pgfsetbuttcap%
\pgfsetroundjoin%
\pgfsetlinewidth{1.003750pt}%
\definecolor{currentstroke}{rgb}{0.705673,0.015556,0.150233}%
\pgfsetstrokecolor{currentstroke}%
\pgfsetdash{}{0pt}%
\pgfpathmoveto{\pgfqpoint{3.367472in}{2.045166in}}%
\pgfpathlineto{\pgfqpoint{3.367472in}{2.045166in}}%
\pgfusepath{stroke}%
\end{pgfscope}%
\begin{pgfscope}%
\pgfpathrectangle{\pgfqpoint{0.552773in}{0.431673in}}{\pgfqpoint{3.738807in}{1.765244in}}%
\pgfusepath{clip}%
\pgfsetbuttcap%
\pgfsetroundjoin%
\pgfsetlinewidth{1.003750pt}%
\definecolor{currentstroke}{rgb}{0.705673,0.015556,0.150233}%
\pgfsetstrokecolor{currentstroke}%
\pgfsetdash{}{0pt}%
\pgfpathmoveto{\pgfqpoint{3.396939in}{2.031018in}}%
\pgfpathlineto{\pgfqpoint{3.396939in}{2.031018in}}%
\pgfusepath{stroke}%
\end{pgfscope}%
\begin{pgfscope}%
\pgfpathrectangle{\pgfqpoint{0.552773in}{0.431673in}}{\pgfqpoint{3.738807in}{1.765244in}}%
\pgfusepath{clip}%
\pgfsetbuttcap%
\pgfsetroundjoin%
\pgfsetlinewidth{1.003750pt}%
\definecolor{currentstroke}{rgb}{0.705673,0.015556,0.150233}%
\pgfsetstrokecolor{currentstroke}%
\pgfsetdash{}{0pt}%
\pgfpathmoveto{\pgfqpoint{3.415993in}{2.000704in}}%
\pgfpathlineto{\pgfqpoint{3.415993in}{2.000704in}}%
\pgfusepath{stroke}%
\end{pgfscope}%
\begin{pgfscope}%
\pgfpathrectangle{\pgfqpoint{0.552773in}{0.431673in}}{\pgfqpoint{3.738807in}{1.765244in}}%
\pgfusepath{clip}%
\pgfsetbuttcap%
\pgfsetroundjoin%
\pgfsetlinewidth{1.003750pt}%
\definecolor{currentstroke}{rgb}{0.705673,0.015556,0.150233}%
\pgfsetstrokecolor{currentstroke}%
\pgfsetdash{}{0pt}%
\pgfpathmoveto{\pgfqpoint{3.427718in}{1.959014in}}%
\pgfpathlineto{\pgfqpoint{3.427718in}{1.959014in}}%
\pgfusepath{stroke}%
\end{pgfscope}%
\begin{pgfscope}%
\pgfpathrectangle{\pgfqpoint{0.552773in}{0.431673in}}{\pgfqpoint{3.738807in}{1.765244in}}%
\pgfusepath{clip}%
\pgfsetbuttcap%
\pgfsetroundjoin%
\pgfsetlinewidth{1.003750pt}%
\definecolor{currentstroke}{rgb}{0.705673,0.015556,0.150233}%
\pgfsetstrokecolor{currentstroke}%
\pgfsetdash{}{0pt}%
\pgfpathmoveto{\pgfqpoint{3.450369in}{1.934284in}}%
\pgfpathlineto{\pgfqpoint{3.450369in}{1.934284in}}%
\pgfusepath{stroke}%
\end{pgfscope}%
\begin{pgfscope}%
\pgfpathrectangle{\pgfqpoint{0.552773in}{0.431673in}}{\pgfqpoint{3.738807in}{1.765244in}}%
\pgfusepath{clip}%
\pgfsetbuttcap%
\pgfsetroundjoin%
\pgfsetlinewidth{1.003750pt}%
\definecolor{currentstroke}{rgb}{0.705673,0.015556,0.150233}%
\pgfsetstrokecolor{currentstroke}%
\pgfsetdash{}{0pt}%
\pgfpathmoveto{\pgfqpoint{3.478306in}{1.917762in}}%
\pgfpathlineto{\pgfqpoint{3.478306in}{1.917762in}}%
\pgfusepath{stroke}%
\end{pgfscope}%
\begin{pgfscope}%
\pgfpathrectangle{\pgfqpoint{0.552773in}{0.431673in}}{\pgfqpoint{3.738807in}{1.765244in}}%
\pgfusepath{clip}%
\pgfsetbuttcap%
\pgfsetroundjoin%
\pgfsetlinewidth{1.003750pt}%
\definecolor{currentstroke}{rgb}{0.705673,0.015556,0.150233}%
\pgfsetstrokecolor{currentstroke}%
\pgfsetdash{}{0pt}%
\pgfpathmoveto{\pgfqpoint{3.477156in}{1.856083in}}%
\pgfpathlineto{\pgfqpoint{3.477156in}{1.856083in}}%
\pgfusepath{stroke}%
\end{pgfscope}%
\begin{pgfscope}%
\pgfpathrectangle{\pgfqpoint{0.552773in}{0.431673in}}{\pgfqpoint{3.738807in}{1.765244in}}%
\pgfusepath{clip}%
\pgfsetbuttcap%
\pgfsetroundjoin%
\pgfsetlinewidth{1.003750pt}%
\definecolor{currentstroke}{rgb}{0.705673,0.015556,0.150233}%
\pgfsetstrokecolor{currentstroke}%
\pgfsetdash{}{0pt}%
\pgfpathmoveto{\pgfqpoint{3.509263in}{1.846033in}}%
\pgfpathlineto{\pgfqpoint{3.509263in}{1.846033in}}%
\pgfusepath{stroke}%
\end{pgfscope}%
\begin{pgfscope}%
\pgfpathrectangle{\pgfqpoint{0.552773in}{0.431673in}}{\pgfqpoint{3.738807in}{1.765244in}}%
\pgfusepath{clip}%
\pgfsetbuttcap%
\pgfsetroundjoin%
\pgfsetlinewidth{1.003750pt}%
\definecolor{currentstroke}{rgb}{0.705673,0.015556,0.150233}%
\pgfsetstrokecolor{currentstroke}%
\pgfsetdash{}{0pt}%
\pgfpathmoveto{\pgfqpoint{3.529555in}{1.817642in}}%
\pgfpathlineto{\pgfqpoint{3.529555in}{1.817642in}}%
\pgfusepath{stroke}%
\end{pgfscope}%
\begin{pgfscope}%
\pgfpathrectangle{\pgfqpoint{0.552773in}{0.431673in}}{\pgfqpoint{3.738807in}{1.765244in}}%
\pgfusepath{clip}%
\pgfsetbuttcap%
\pgfsetroundjoin%
\pgfsetlinewidth{1.003750pt}%
\definecolor{currentstroke}{rgb}{0.705673,0.015556,0.150233}%
\pgfsetstrokecolor{currentstroke}%
\pgfsetdash{}{0pt}%
\pgfpathmoveto{\pgfqpoint{3.565162in}{1.813026in}}%
\pgfpathlineto{\pgfqpoint{3.565162in}{1.813026in}}%
\pgfusepath{stroke}%
\end{pgfscope}%
\begin{pgfscope}%
\pgfpathrectangle{\pgfqpoint{0.552773in}{0.431673in}}{\pgfqpoint{3.738807in}{1.765244in}}%
\pgfusepath{clip}%
\pgfsetbuttcap%
\pgfsetroundjoin%
\pgfsetlinewidth{1.003750pt}%
\definecolor{currentstroke}{rgb}{0.705673,0.015556,0.150233}%
\pgfsetstrokecolor{currentstroke}%
\pgfsetdash{}{0pt}%
\pgfpathmoveto{\pgfqpoint{3.597068in}{1.802664in}}%
\pgfpathlineto{\pgfqpoint{3.597068in}{1.802664in}}%
\pgfusepath{stroke}%
\end{pgfscope}%
\begin{pgfscope}%
\pgfpathrectangle{\pgfqpoint{0.552773in}{0.431673in}}{\pgfqpoint{3.738807in}{1.765244in}}%
\pgfusepath{clip}%
\pgfsetbuttcap%
\pgfsetroundjoin%
\pgfsetlinewidth{1.003750pt}%
\definecolor{currentstroke}{rgb}{0.705673,0.015556,0.150233}%
\pgfsetstrokecolor{currentstroke}%
\pgfsetdash{}{0pt}%
\pgfpathmoveto{\pgfqpoint{3.641696in}{1.812053in}}%
\pgfpathlineto{\pgfqpoint{3.641696in}{1.812053in}}%
\pgfusepath{stroke}%
\end{pgfscope}%
\begin{pgfscope}%
\pgfpathrectangle{\pgfqpoint{0.552773in}{0.431673in}}{\pgfqpoint{3.738807in}{1.765244in}}%
\pgfusepath{clip}%
\pgfsetbuttcap%
\pgfsetroundjoin%
\pgfsetlinewidth{1.003750pt}%
\definecolor{currentstroke}{rgb}{0.705673,0.015556,0.150233}%
\pgfsetstrokecolor{currentstroke}%
\pgfsetdash{}{0pt}%
\pgfpathmoveto{\pgfqpoint{3.718669in}{1.871655in}}%
\pgfpathlineto{\pgfqpoint{3.718669in}{1.871655in}}%
\pgfusepath{stroke}%
\end{pgfscope}%
\begin{pgfscope}%
\pgfpathrectangle{\pgfqpoint{0.552773in}{0.431673in}}{\pgfqpoint{3.738807in}{1.765244in}}%
\pgfusepath{clip}%
\pgfsetbuttcap%
\pgfsetroundjoin%
\pgfsetlinewidth{1.003750pt}%
\definecolor{currentstroke}{rgb}{0.705673,0.015556,0.150233}%
\pgfsetstrokecolor{currentstroke}%
\pgfsetdash{}{0pt}%
\pgfpathmoveto{\pgfqpoint{3.775425in}{1.899872in}}%
\pgfpathlineto{\pgfqpoint{3.775425in}{1.899872in}}%
\pgfusepath{stroke}%
\end{pgfscope}%
\begin{pgfscope}%
\pgfpathrectangle{\pgfqpoint{0.552773in}{0.431673in}}{\pgfqpoint{3.738807in}{1.765244in}}%
\pgfusepath{clip}%
\pgfsetbuttcap%
\pgfsetroundjoin%
\pgfsetlinewidth{1.003750pt}%
\definecolor{currentstroke}{rgb}{0.705673,0.015556,0.150233}%
\pgfsetstrokecolor{currentstroke}%
\pgfsetdash{}{0pt}%
\pgfpathmoveto{\pgfqpoint{3.852269in}{1.959273in}}%
\pgfpathlineto{\pgfqpoint{3.852269in}{1.959273in}}%
\pgfusepath{stroke}%
\end{pgfscope}%
\begin{pgfscope}%
\pgfpathrectangle{\pgfqpoint{0.552773in}{0.431673in}}{\pgfqpoint{3.738807in}{1.765244in}}%
\pgfusepath{clip}%
\pgfsetbuttcap%
\pgfsetroundjoin%
\pgfsetlinewidth{1.003750pt}%
\definecolor{currentstroke}{rgb}{0.705673,0.015556,0.150233}%
\pgfsetstrokecolor{currentstroke}%
\pgfsetdash{}{0pt}%
\pgfpathmoveto{\pgfqpoint{3.927500in}{2.016170in}}%
\pgfpathlineto{\pgfqpoint{3.927500in}{2.016170in}}%
\pgfusepath{stroke}%
\end{pgfscope}%
\begin{pgfscope}%
\pgfpathrectangle{\pgfqpoint{0.552773in}{0.431673in}}{\pgfqpoint{3.738807in}{1.765244in}}%
\pgfusepath{clip}%
\pgfsetbuttcap%
\pgfsetroundjoin%
\pgfsetlinewidth{1.003750pt}%
\definecolor{currentstroke}{rgb}{0.705673,0.015556,0.150233}%
\pgfsetstrokecolor{currentstroke}%
\pgfsetdash{}{0pt}%
\pgfpathmoveto{\pgfqpoint{3.985669in}{2.046581in}}%
\pgfpathlineto{\pgfqpoint{3.985669in}{2.046581in}}%
\pgfusepath{stroke}%
\end{pgfscope}%
\begin{pgfscope}%
\pgfpathrectangle{\pgfqpoint{0.552773in}{0.431673in}}{\pgfqpoint{3.738807in}{1.765244in}}%
\pgfusepath{clip}%
\pgfsetbuttcap%
\pgfsetroundjoin%
\pgfsetlinewidth{1.003750pt}%
\definecolor{currentstroke}{rgb}{0.705673,0.015556,0.150233}%
\pgfsetstrokecolor{currentstroke}%
\pgfsetdash{}{0pt}%
\pgfpathmoveto{\pgfqpoint{4.026778in}{2.050506in}}%
\pgfpathlineto{\pgfqpoint{4.026778in}{2.050506in}}%
\pgfusepath{stroke}%
\end{pgfscope}%
\begin{pgfscope}%
\pgfpathrectangle{\pgfqpoint{0.552773in}{0.431673in}}{\pgfqpoint{3.738807in}{1.765244in}}%
\pgfusepath{clip}%
\pgfsetbuttcap%
\pgfsetroundjoin%
\pgfsetlinewidth{1.003750pt}%
\definecolor{currentstroke}{rgb}{0.705673,0.015556,0.150233}%
\pgfsetstrokecolor{currentstroke}%
\pgfsetdash{}{0pt}%
\pgfpathmoveto{\pgfqpoint{4.053468in}{2.032048in}}%
\pgfpathlineto{\pgfqpoint{4.053468in}{2.032048in}}%
\pgfusepath{stroke}%
\end{pgfscope}%
\begin{pgfscope}%
\pgfpathrectangle{\pgfqpoint{0.552773in}{0.431673in}}{\pgfqpoint{3.738807in}{1.765244in}}%
\pgfusepath{clip}%
\pgfsetbuttcap%
\pgfsetroundjoin%
\pgfsetlinewidth{1.003750pt}%
\definecolor{currentstroke}{rgb}{0.705673,0.015556,0.150233}%
\pgfsetstrokecolor{currentstroke}%
\pgfsetdash{}{0pt}%
\pgfpathmoveto{\pgfqpoint{4.068392in}{1.995322in}}%
\pgfpathlineto{\pgfqpoint{4.068392in}{1.995322in}}%
\pgfusepath{stroke}%
\end{pgfscope}%
\begin{pgfscope}%
\pgfpathrectangle{\pgfqpoint{0.552773in}{0.431673in}}{\pgfqpoint{3.738807in}{1.765244in}}%
\pgfusepath{clip}%
\pgfsetbuttcap%
\pgfsetroundjoin%
\pgfsetlinewidth{1.003750pt}%
\definecolor{currentstroke}{rgb}{0.705673,0.015556,0.150233}%
\pgfsetstrokecolor{currentstroke}%
\pgfsetdash{}{0pt}%
\pgfpathmoveto{\pgfqpoint{4.075955in}{1.947171in}}%
\pgfpathlineto{\pgfqpoint{4.075955in}{1.947171in}}%
\pgfusepath{stroke}%
\end{pgfscope}%
\begin{pgfscope}%
\pgfpathrectangle{\pgfqpoint{0.552773in}{0.431673in}}{\pgfqpoint{3.738807in}{1.765244in}}%
\pgfusepath{clip}%
\pgfsetbuttcap%
\pgfsetroundjoin%
\pgfsetlinewidth{1.003750pt}%
\definecolor{currentstroke}{rgb}{0.705673,0.015556,0.150233}%
\pgfsetstrokecolor{currentstroke}%
\pgfsetdash{}{0pt}%
\pgfpathmoveto{\pgfqpoint{4.096830in}{1.919684in}}%
\pgfpathlineto{\pgfqpoint{4.096830in}{1.919684in}}%
\pgfusepath{stroke}%
\end{pgfscope}%
\begin{pgfscope}%
\pgfpathrectangle{\pgfqpoint{0.552773in}{0.431673in}}{\pgfqpoint{3.738807in}{1.765244in}}%
\pgfusepath{clip}%
\pgfsetbuttcap%
\pgfsetroundjoin%
\pgfsetlinewidth{1.003750pt}%
\definecolor{currentstroke}{rgb}{0.705673,0.015556,0.150233}%
\pgfsetstrokecolor{currentstroke}%
\pgfsetdash{}{0pt}%
\pgfpathmoveto{\pgfqpoint{4.121635in}{1.898298in}}%
\pgfpathlineto{\pgfqpoint{4.121635in}{1.898298in}}%
\pgfusepath{stroke}%
\end{pgfscope}%
\begin{pgfscope}%
\pgfpathrectangle{\pgfqpoint{0.552773in}{0.431673in}}{\pgfqpoint{3.738807in}{1.765244in}}%
\pgfusepath{clip}%
\pgfsetbuttcap%
\pgfsetroundjoin%
\pgfsetlinewidth{1.003750pt}%
\definecolor{currentstroke}{rgb}{0.705673,0.015556,0.150233}%
\pgfsetstrokecolor{currentstroke}%
\pgfsetdash{}{0pt}%
\pgfpathmoveto{\pgfqpoint{0.722719in}{2.012117in}}%
\pgfpathlineto{\pgfqpoint{0.722719in}{2.012117in}}%
\pgfusepath{stroke}%
\end{pgfscope}%
\begin{pgfscope}%
\pgfpathrectangle{\pgfqpoint{0.552773in}{0.431673in}}{\pgfqpoint{3.738807in}{1.765244in}}%
\pgfusepath{clip}%
\pgfsetbuttcap%
\pgfsetroundjoin%
\pgfsetlinewidth{1.003750pt}%
\definecolor{currentstroke}{rgb}{0.705673,0.015556,0.150233}%
\pgfsetstrokecolor{currentstroke}%
\pgfsetdash{}{0pt}%
\pgfpathmoveto{\pgfqpoint{0.786936in}{1.992022in}}%
\pgfpathlineto{\pgfqpoint{0.786936in}{1.992022in}}%
\pgfusepath{stroke}%
\end{pgfscope}%
\begin{pgfscope}%
\pgfpathrectangle{\pgfqpoint{0.552773in}{0.431673in}}{\pgfqpoint{3.738807in}{1.765244in}}%
\pgfusepath{clip}%
\pgfsetbuttcap%
\pgfsetroundjoin%
\pgfsetlinewidth{1.003750pt}%
\definecolor{currentstroke}{rgb}{0.705673,0.015556,0.150233}%
\pgfsetstrokecolor{currentstroke}%
\pgfsetdash{}{0pt}%
\pgfpathmoveto{\pgfqpoint{0.819797in}{1.983144in}}%
\pgfpathlineto{\pgfqpoint{0.819797in}{1.983144in}}%
\pgfusepath{stroke}%
\end{pgfscope}%
\begin{pgfscope}%
\pgfpathrectangle{\pgfqpoint{0.552773in}{0.431673in}}{\pgfqpoint{3.738807in}{1.765244in}}%
\pgfusepath{clip}%
\pgfsetbuttcap%
\pgfsetroundjoin%
\pgfsetlinewidth{1.003750pt}%
\definecolor{currentstroke}{rgb}{0.705673,0.015556,0.150233}%
\pgfsetstrokecolor{currentstroke}%
\pgfsetdash{}{0pt}%
\pgfpathmoveto{\pgfqpoint{0.831560in}{1.941512in}}%
\pgfpathlineto{\pgfqpoint{0.831560in}{1.941512in}}%
\pgfusepath{stroke}%
\end{pgfscope}%
\begin{pgfscope}%
\pgfpathrectangle{\pgfqpoint{0.552773in}{0.431673in}}{\pgfqpoint{3.738807in}{1.765244in}}%
\pgfusepath{clip}%
\pgfsetbuttcap%
\pgfsetroundjoin%
\pgfsetlinewidth{1.003750pt}%
\definecolor{currentstroke}{rgb}{0.705673,0.015556,0.150233}%
\pgfsetstrokecolor{currentstroke}%
\pgfsetdash{}{0pt}%
\pgfpathmoveto{\pgfqpoint{0.837884in}{1.891436in}}%
\pgfpathlineto{\pgfqpoint{0.837884in}{1.891436in}}%
\pgfusepath{stroke}%
\end{pgfscope}%
\begin{pgfscope}%
\pgfpathrectangle{\pgfqpoint{0.552773in}{0.431673in}}{\pgfqpoint{3.738807in}{1.765244in}}%
\pgfusepath{clip}%
\pgfsetbuttcap%
\pgfsetroundjoin%
\pgfsetlinewidth{1.003750pt}%
\definecolor{currentstroke}{rgb}{0.705673,0.015556,0.150233}%
\pgfsetstrokecolor{currentstroke}%
\pgfsetdash{}{0pt}%
\pgfpathmoveto{\pgfqpoint{0.853441in}{1.855694in}}%
\pgfpathlineto{\pgfqpoint{0.853441in}{1.855694in}}%
\pgfusepath{stroke}%
\end{pgfscope}%
\begin{pgfscope}%
\pgfpathrectangle{\pgfqpoint{0.552773in}{0.431673in}}{\pgfqpoint{3.738807in}{1.765244in}}%
\pgfusepath{clip}%
\pgfsetbuttcap%
\pgfsetroundjoin%
\pgfsetlinewidth{1.003750pt}%
\definecolor{currentstroke}{rgb}{0.705673,0.015556,0.150233}%
\pgfsetstrokecolor{currentstroke}%
\pgfsetdash{}{0pt}%
\pgfpathmoveto{\pgfqpoint{0.871877in}{1.824422in}}%
\pgfpathlineto{\pgfqpoint{0.871877in}{1.824422in}}%
\pgfusepath{stroke}%
\end{pgfscope}%
\begin{pgfscope}%
\pgfpathrectangle{\pgfqpoint{0.552773in}{0.431673in}}{\pgfqpoint{3.738807in}{1.765244in}}%
\pgfusepath{clip}%
\pgfsetbuttcap%
\pgfsetroundjoin%
\pgfsetlinewidth{1.003750pt}%
\definecolor{currentstroke}{rgb}{0.705673,0.015556,0.150233}%
\pgfsetstrokecolor{currentstroke}%
\pgfsetdash{}{0pt}%
\pgfpathmoveto{\pgfqpoint{0.894231in}{1.799231in}}%
\pgfpathlineto{\pgfqpoint{0.894231in}{1.799231in}}%
\pgfusepath{stroke}%
\end{pgfscope}%
\begin{pgfscope}%
\pgfpathrectangle{\pgfqpoint{0.552773in}{0.431673in}}{\pgfqpoint{3.738807in}{1.765244in}}%
\pgfusepath{clip}%
\pgfsetbuttcap%
\pgfsetroundjoin%
\pgfsetlinewidth{1.003750pt}%
\definecolor{currentstroke}{rgb}{0.705673,0.015556,0.150233}%
\pgfsetstrokecolor{currentstroke}%
\pgfsetdash{}{0pt}%
\pgfpathmoveto{\pgfqpoint{0.917097in}{1.774835in}}%
\pgfpathlineto{\pgfqpoint{0.917097in}{1.774835in}}%
\pgfusepath{stroke}%
\end{pgfscope}%
\begin{pgfscope}%
\pgfpathrectangle{\pgfqpoint{0.552773in}{0.431673in}}{\pgfqpoint{3.738807in}{1.765244in}}%
\pgfusepath{clip}%
\pgfsetbuttcap%
\pgfsetroundjoin%
\pgfsetlinewidth{1.003750pt}%
\definecolor{currentstroke}{rgb}{0.705673,0.015556,0.150233}%
\pgfsetstrokecolor{currentstroke}%
\pgfsetdash{}{0pt}%
\pgfpathmoveto{\pgfqpoint{0.954263in}{1.772639in}}%
\pgfpathlineto{\pgfqpoint{0.954263in}{1.772639in}}%
\pgfusepath{stroke}%
\end{pgfscope}%
\begin{pgfscope}%
\pgfpathrectangle{\pgfqpoint{0.552773in}{0.431673in}}{\pgfqpoint{3.738807in}{1.765244in}}%
\pgfusepath{clip}%
\pgfsetbuttcap%
\pgfsetroundjoin%
\pgfsetlinewidth{1.003750pt}%
\definecolor{currentstroke}{rgb}{0.705673,0.015556,0.150233}%
\pgfsetstrokecolor{currentstroke}%
\pgfsetdash{}{0pt}%
\pgfpathmoveto{\pgfqpoint{1.054856in}{1.809017in}}%
\pgfpathlineto{\pgfqpoint{1.054856in}{1.809017in}}%
\pgfusepath{stroke}%
\end{pgfscope}%
\begin{pgfscope}%
\pgfpathrectangle{\pgfqpoint{0.552773in}{0.431673in}}{\pgfqpoint{3.738807in}{1.765244in}}%
\pgfusepath{clip}%
\pgfsetbuttcap%
\pgfsetroundjoin%
\pgfsetlinewidth{1.003750pt}%
\definecolor{currentstroke}{rgb}{0.705673,0.015556,0.150233}%
\pgfsetstrokecolor{currentstroke}%
\pgfsetdash{}{0pt}%
\pgfpathmoveto{\pgfqpoint{1.127913in}{1.862539in}}%
\pgfpathlineto{\pgfqpoint{1.127913in}{1.862539in}}%
\pgfusepath{stroke}%
\end{pgfscope}%
\begin{pgfscope}%
\pgfpathrectangle{\pgfqpoint{0.552773in}{0.431673in}}{\pgfqpoint{3.738807in}{1.765244in}}%
\pgfusepath{clip}%
\pgfsetbuttcap%
\pgfsetroundjoin%
\pgfsetlinewidth{1.003750pt}%
\definecolor{currentstroke}{rgb}{0.705673,0.015556,0.150233}%
\pgfsetstrokecolor{currentstroke}%
\pgfsetdash{}{0pt}%
\pgfpathmoveto{\pgfqpoint{1.216245in}{1.939776in}}%
\pgfpathlineto{\pgfqpoint{1.216245in}{1.939776in}}%
\pgfusepath{stroke}%
\end{pgfscope}%
\begin{pgfscope}%
\pgfpathrectangle{\pgfqpoint{0.552773in}{0.431673in}}{\pgfqpoint{3.738807in}{1.765244in}}%
\pgfusepath{clip}%
\pgfsetbuttcap%
\pgfsetroundjoin%
\pgfsetlinewidth{1.003750pt}%
\definecolor{currentstroke}{rgb}{0.705673,0.015556,0.150233}%
\pgfsetstrokecolor{currentstroke}%
\pgfsetdash{}{0pt}%
\pgfpathmoveto{\pgfqpoint{1.299586in}{2.009263in}}%
\pgfpathlineto{\pgfqpoint{1.299586in}{2.009263in}}%
\pgfusepath{stroke}%
\end{pgfscope}%
\begin{pgfscope}%
\pgfpathrectangle{\pgfqpoint{0.552773in}{0.431673in}}{\pgfqpoint{3.738807in}{1.765244in}}%
\pgfusepath{clip}%
\pgfsetbuttcap%
\pgfsetroundjoin%
\pgfsetlinewidth{1.003750pt}%
\definecolor{currentstroke}{rgb}{0.705673,0.015556,0.150233}%
\pgfsetstrokecolor{currentstroke}%
\pgfsetdash{}{0pt}%
\pgfpathmoveto{\pgfqpoint{1.383835in}{2.080160in}}%
\pgfpathlineto{\pgfqpoint{1.383835in}{2.080160in}}%
\pgfusepath{stroke}%
\end{pgfscope}%
\begin{pgfscope}%
\pgfpathrectangle{\pgfqpoint{0.552773in}{0.431673in}}{\pgfqpoint{3.738807in}{1.765244in}}%
\pgfusepath{clip}%
\pgfsetbuttcap%
\pgfsetroundjoin%
\pgfsetlinewidth{1.003750pt}%
\definecolor{currentstroke}{rgb}{0.705673,0.015556,0.150233}%
\pgfsetstrokecolor{currentstroke}%
\pgfsetdash{}{0pt}%
\pgfpathmoveto{\pgfqpoint{1.428876in}{2.090189in}}%
\pgfpathlineto{\pgfqpoint{1.428876in}{2.090189in}}%
\pgfusepath{stroke}%
\end{pgfscope}%
\begin{pgfscope}%
\pgfpathrectangle{\pgfqpoint{0.552773in}{0.431673in}}{\pgfqpoint{3.738807in}{1.765244in}}%
\pgfusepath{clip}%
\pgfsetbuttcap%
\pgfsetroundjoin%
\pgfsetlinewidth{1.003750pt}%
\definecolor{currentstroke}{rgb}{0.705673,0.015556,0.150233}%
\pgfsetstrokecolor{currentstroke}%
\pgfsetdash{}{0pt}%
\pgfpathmoveto{\pgfqpoint{1.461920in}{2.081596in}}%
\pgfpathlineto{\pgfqpoint{1.461920in}{2.081596in}}%
\pgfusepath{stroke}%
\end{pgfscope}%
\begin{pgfscope}%
\pgfpathrectangle{\pgfqpoint{0.552773in}{0.431673in}}{\pgfqpoint{3.738807in}{1.765244in}}%
\pgfusepath{clip}%
\pgfsetbuttcap%
\pgfsetroundjoin%
\pgfsetlinewidth{1.003750pt}%
\definecolor{currentstroke}{rgb}{0.705673,0.015556,0.150233}%
\pgfsetstrokecolor{currentstroke}%
\pgfsetdash{}{0pt}%
\pgfpathmoveto{\pgfqpoint{1.470951in}{2.035721in}}%
\pgfpathlineto{\pgfqpoint{1.470951in}{2.035721in}}%
\pgfusepath{stroke}%
\end{pgfscope}%
\begin{pgfscope}%
\pgfpathrectangle{\pgfqpoint{0.552773in}{0.431673in}}{\pgfqpoint{3.738807in}{1.765244in}}%
\pgfusepath{clip}%
\pgfsetbuttcap%
\pgfsetroundjoin%
\pgfsetlinewidth{1.003750pt}%
\definecolor{currentstroke}{rgb}{0.705673,0.015556,0.150233}%
\pgfsetstrokecolor{currentstroke}%
\pgfsetdash{}{0pt}%
\pgfpathmoveto{\pgfqpoint{1.500704in}{2.022017in}}%
\pgfpathlineto{\pgfqpoint{1.500704in}{2.022017in}}%
\pgfusepath{stroke}%
\end{pgfscope}%
\begin{pgfscope}%
\pgfpathrectangle{\pgfqpoint{0.552773in}{0.431673in}}{\pgfqpoint{3.738807in}{1.765244in}}%
\pgfusepath{clip}%
\pgfsetbuttcap%
\pgfsetroundjoin%
\pgfsetlinewidth{1.003750pt}%
\definecolor{currentstroke}{rgb}{0.705673,0.015556,0.150233}%
\pgfsetstrokecolor{currentstroke}%
\pgfsetdash{}{0pt}%
\pgfpathmoveto{\pgfqpoint{1.498835in}{1.959223in}}%
\pgfpathlineto{\pgfqpoint{1.498835in}{1.959223in}}%
\pgfusepath{stroke}%
\end{pgfscope}%
\begin{pgfscope}%
\pgfpathrectangle{\pgfqpoint{0.552773in}{0.431673in}}{\pgfqpoint{3.738807in}{1.765244in}}%
\pgfusepath{clip}%
\pgfsetbuttcap%
\pgfsetroundjoin%
\pgfsetlinewidth{1.003750pt}%
\definecolor{currentstroke}{rgb}{0.705673,0.015556,0.150233}%
\pgfsetstrokecolor{currentstroke}%
\pgfsetdash{}{0pt}%
\pgfpathmoveto{\pgfqpoint{1.510085in}{1.916795in}}%
\pgfpathlineto{\pgfqpoint{1.510085in}{1.916795in}}%
\pgfusepath{stroke}%
\end{pgfscope}%
\begin{pgfscope}%
\pgfpathrectangle{\pgfqpoint{0.552773in}{0.431673in}}{\pgfqpoint{3.738807in}{1.765244in}}%
\pgfusepath{clip}%
\pgfsetbuttcap%
\pgfsetroundjoin%
\pgfsetlinewidth{1.003750pt}%
\definecolor{currentstroke}{rgb}{0.705673,0.015556,0.150233}%
\pgfsetstrokecolor{currentstroke}%
\pgfsetdash{}{0pt}%
\pgfpathmoveto{\pgfqpoint{1.536220in}{1.897473in}}%
\pgfpathlineto{\pgfqpoint{1.536220in}{1.897473in}}%
\pgfusepath{stroke}%
\end{pgfscope}%
\begin{pgfscope}%
\pgfpathrectangle{\pgfqpoint{0.552773in}{0.431673in}}{\pgfqpoint{3.738807in}{1.765244in}}%
\pgfusepath{clip}%
\pgfsetbuttcap%
\pgfsetroundjoin%
\pgfsetlinewidth{1.003750pt}%
\definecolor{currentstroke}{rgb}{0.705673,0.015556,0.150233}%
\pgfsetstrokecolor{currentstroke}%
\pgfsetdash{}{0pt}%
\pgfpathmoveto{\pgfqpoint{1.565209in}{1.882584in}}%
\pgfpathlineto{\pgfqpoint{1.565209in}{1.882584in}}%
\pgfusepath{stroke}%
\end{pgfscope}%
\begin{pgfscope}%
\pgfpathrectangle{\pgfqpoint{0.552773in}{0.431673in}}{\pgfqpoint{3.738807in}{1.765244in}}%
\pgfusepath{clip}%
\pgfsetbuttcap%
\pgfsetroundjoin%
\pgfsetlinewidth{1.003750pt}%
\definecolor{currentstroke}{rgb}{0.705673,0.015556,0.150233}%
\pgfsetstrokecolor{currentstroke}%
\pgfsetdash{}{0pt}%
\pgfpathmoveto{\pgfqpoint{1.587354in}{1.857070in}}%
\pgfpathlineto{\pgfqpoint{1.587354in}{1.857070in}}%
\pgfusepath{stroke}%
\end{pgfscope}%
\begin{pgfscope}%
\pgfpathrectangle{\pgfqpoint{0.552773in}{0.431673in}}{\pgfqpoint{3.738807in}{1.765244in}}%
\pgfusepath{clip}%
\pgfsetbuttcap%
\pgfsetroundjoin%
\pgfsetlinewidth{1.003750pt}%
\definecolor{currentstroke}{rgb}{0.705673,0.015556,0.150233}%
\pgfsetstrokecolor{currentstroke}%
\pgfsetdash{}{0pt}%
\pgfpathmoveto{\pgfqpoint{1.624003in}{1.854071in}}%
\pgfpathlineto{\pgfqpoint{1.624003in}{1.854071in}}%
\pgfusepath{stroke}%
\end{pgfscope}%
\begin{pgfscope}%
\pgfpathrectangle{\pgfqpoint{0.552773in}{0.431673in}}{\pgfqpoint{3.738807in}{1.765244in}}%
\pgfusepath{clip}%
\pgfsetbuttcap%
\pgfsetroundjoin%
\pgfsetlinewidth{1.003750pt}%
\definecolor{currentstroke}{rgb}{0.705673,0.015556,0.150233}%
\pgfsetstrokecolor{currentstroke}%
\pgfsetdash{}{0pt}%
\pgfpathmoveto{\pgfqpoint{1.656435in}{1.844526in}}%
\pgfpathlineto{\pgfqpoint{1.656435in}{1.844526in}}%
\pgfusepath{stroke}%
\end{pgfscope}%
\begin{pgfscope}%
\pgfpathrectangle{\pgfqpoint{0.552773in}{0.431673in}}{\pgfqpoint{3.738807in}{1.765244in}}%
\pgfusepath{clip}%
\pgfsetbuttcap%
\pgfsetroundjoin%
\pgfsetlinewidth{1.003750pt}%
\definecolor{currentstroke}{rgb}{0.705673,0.015556,0.150233}%
\pgfsetstrokecolor{currentstroke}%
\pgfsetdash{}{0pt}%
\pgfpathmoveto{\pgfqpoint{1.704526in}{1.859291in}}%
\pgfpathlineto{\pgfqpoint{1.704526in}{1.859291in}}%
\pgfusepath{stroke}%
\end{pgfscope}%
\begin{pgfscope}%
\pgfpathrectangle{\pgfqpoint{0.552773in}{0.431673in}}{\pgfqpoint{3.738807in}{1.765244in}}%
\pgfusepath{clip}%
\pgfsetbuttcap%
\pgfsetroundjoin%
\pgfsetlinewidth{1.003750pt}%
\definecolor{currentstroke}{rgb}{0.705673,0.015556,0.150233}%
\pgfsetstrokecolor{currentstroke}%
\pgfsetdash{}{0pt}%
\pgfpathmoveto{\pgfqpoint{1.761295in}{1.887527in}}%
\pgfpathlineto{\pgfqpoint{1.761295in}{1.887527in}}%
\pgfusepath{stroke}%
\end{pgfscope}%
\begin{pgfscope}%
\pgfpathrectangle{\pgfqpoint{0.552773in}{0.431673in}}{\pgfqpoint{3.738807in}{1.765244in}}%
\pgfusepath{clip}%
\pgfsetbuttcap%
\pgfsetroundjoin%
\pgfsetlinewidth{1.003750pt}%
\definecolor{currentstroke}{rgb}{0.705673,0.015556,0.150233}%
\pgfsetstrokecolor{currentstroke}%
\pgfsetdash{}{0pt}%
\pgfpathmoveto{\pgfqpoint{1.828491in}{1.931951in}}%
\pgfpathlineto{\pgfqpoint{1.828491in}{1.931951in}}%
\pgfusepath{stroke}%
\end{pgfscope}%
\begin{pgfscope}%
\pgfpathrectangle{\pgfqpoint{0.552773in}{0.431673in}}{\pgfqpoint{3.738807in}{1.765244in}}%
\pgfusepath{clip}%
\pgfsetbuttcap%
\pgfsetroundjoin%
\pgfsetlinewidth{1.003750pt}%
\definecolor{currentstroke}{rgb}{0.705673,0.015556,0.150233}%
\pgfsetstrokecolor{currentstroke}%
\pgfsetdash{}{0pt}%
\pgfpathmoveto{\pgfqpoint{1.902753in}{1.987345in}}%
\pgfpathlineto{\pgfqpoint{1.902753in}{1.987345in}}%
\pgfusepath{stroke}%
\end{pgfscope}%
\begin{pgfscope}%
\pgfpathrectangle{\pgfqpoint{0.552773in}{0.431673in}}{\pgfqpoint{3.738807in}{1.765244in}}%
\pgfusepath{clip}%
\pgfsetbuttcap%
\pgfsetroundjoin%
\pgfsetlinewidth{1.003750pt}%
\definecolor{currentstroke}{rgb}{0.705673,0.015556,0.150233}%
\pgfsetstrokecolor{currentstroke}%
\pgfsetdash{}{0pt}%
\pgfpathmoveto{\pgfqpoint{1.989053in}{2.061425in}}%
\pgfpathlineto{\pgfqpoint{1.989053in}{2.061425in}}%
\pgfusepath{stroke}%
\end{pgfscope}%
\begin{pgfscope}%
\pgfpathrectangle{\pgfqpoint{0.552773in}{0.431673in}}{\pgfqpoint{3.738807in}{1.765244in}}%
\pgfusepath{clip}%
\pgfsetbuttcap%
\pgfsetroundjoin%
\pgfsetlinewidth{1.003750pt}%
\definecolor{currentstroke}{rgb}{0.705673,0.015556,0.150233}%
\pgfsetstrokecolor{currentstroke}%
\pgfsetdash{}{0pt}%
\pgfpathmoveto{\pgfqpoint{2.045418in}{2.089036in}}%
\pgfpathlineto{\pgfqpoint{2.045418in}{2.089036in}}%
\pgfusepath{stroke}%
\end{pgfscope}%
\begin{pgfscope}%
\pgfpathrectangle{\pgfqpoint{0.552773in}{0.431673in}}{\pgfqpoint{3.738807in}{1.765244in}}%
\pgfusepath{clip}%
\pgfsetbuttcap%
\pgfsetroundjoin%
\pgfsetlinewidth{1.003750pt}%
\definecolor{currentstroke}{rgb}{0.705673,0.015556,0.150233}%
\pgfsetstrokecolor{currentstroke}%
\pgfsetdash{}{0pt}%
\pgfpathmoveto{\pgfqpoint{2.101804in}{2.116678in}}%
\pgfpathlineto{\pgfqpoint{2.101804in}{2.116678in}}%
\pgfusepath{stroke}%
\end{pgfscope}%
\begin{pgfscope}%
\pgfpathrectangle{\pgfqpoint{0.552773in}{0.431673in}}{\pgfqpoint{3.738807in}{1.765244in}}%
\pgfusepath{clip}%
\pgfsetbuttcap%
\pgfsetroundjoin%
\pgfsetlinewidth{1.003750pt}%
\definecolor{currentstroke}{rgb}{0.705673,0.015556,0.150233}%
\pgfsetstrokecolor{currentstroke}%
\pgfsetdash{}{0pt}%
\pgfpathmoveto{\pgfqpoint{2.115398in}{2.077888in}}%
\pgfpathlineto{\pgfqpoint{2.115398in}{2.077888in}}%
\pgfusepath{stroke}%
\end{pgfscope}%
\begin{pgfscope}%
\pgfpathrectangle{\pgfqpoint{0.552773in}{0.431673in}}{\pgfqpoint{3.738807in}{1.765244in}}%
\pgfusepath{clip}%
\pgfsetbuttcap%
\pgfsetroundjoin%
\pgfsetlinewidth{1.003750pt}%
\definecolor{currentstroke}{rgb}{0.705673,0.015556,0.150233}%
\pgfsetstrokecolor{currentstroke}%
\pgfsetdash{}{0pt}%
\pgfpathmoveto{\pgfqpoint{2.157435in}{2.083254in}}%
\pgfpathlineto{\pgfqpoint{2.157435in}{2.083254in}}%
\pgfusepath{stroke}%
\end{pgfscope}%
\begin{pgfscope}%
\pgfpathrectangle{\pgfqpoint{0.552773in}{0.431673in}}{\pgfqpoint{3.738807in}{1.765244in}}%
\pgfusepath{clip}%
\pgfsetbuttcap%
\pgfsetroundjoin%
\pgfsetlinewidth{1.003750pt}%
\definecolor{currentstroke}{rgb}{0.705673,0.015556,0.150233}%
\pgfsetstrokecolor{currentstroke}%
\pgfsetdash{}{0pt}%
\pgfpathmoveto{\pgfqpoint{2.162229in}{2.030803in}}%
\pgfpathlineto{\pgfqpoint{2.162229in}{2.030803in}}%
\pgfusepath{stroke}%
\end{pgfscope}%
\begin{pgfscope}%
\pgfpathrectangle{\pgfqpoint{0.552773in}{0.431673in}}{\pgfqpoint{3.738807in}{1.765244in}}%
\pgfusepath{clip}%
\pgfsetbuttcap%
\pgfsetroundjoin%
\pgfsetlinewidth{1.003750pt}%
\definecolor{currentstroke}{rgb}{0.705673,0.015556,0.150233}%
\pgfsetstrokecolor{currentstroke}%
\pgfsetdash{}{0pt}%
\pgfpathmoveto{\pgfqpoint{2.178795in}{1.996627in}}%
\pgfpathlineto{\pgfqpoint{2.178795in}{1.996627in}}%
\pgfusepath{stroke}%
\end{pgfscope}%
\begin{pgfscope}%
\pgfpathrectangle{\pgfqpoint{0.552773in}{0.431673in}}{\pgfqpoint{3.738807in}{1.765244in}}%
\pgfusepath{clip}%
\pgfsetbuttcap%
\pgfsetroundjoin%
\pgfsetlinewidth{1.003750pt}%
\definecolor{currentstroke}{rgb}{0.705673,0.015556,0.150233}%
\pgfsetstrokecolor{currentstroke}%
\pgfsetdash{}{0pt}%
\pgfpathmoveto{\pgfqpoint{2.190577in}{1.955024in}}%
\pgfpathlineto{\pgfqpoint{2.190577in}{1.955024in}}%
\pgfusepath{stroke}%
\end{pgfscope}%
\begin{pgfscope}%
\pgfpathrectangle{\pgfqpoint{0.552773in}{0.431673in}}{\pgfqpoint{3.738807in}{1.765244in}}%
\pgfusepath{clip}%
\pgfsetbuttcap%
\pgfsetroundjoin%
\pgfsetlinewidth{1.003750pt}%
\definecolor{currentstroke}{rgb}{0.705673,0.015556,0.150233}%
\pgfsetstrokecolor{currentstroke}%
\pgfsetdash{}{0pt}%
\pgfpathmoveto{\pgfqpoint{2.210306in}{1.925759in}}%
\pgfpathlineto{\pgfqpoint{2.210306in}{1.925759in}}%
\pgfusepath{stroke}%
\end{pgfscope}%
\begin{pgfscope}%
\pgfpathrectangle{\pgfqpoint{0.552773in}{0.431673in}}{\pgfqpoint{3.738807in}{1.765244in}}%
\pgfusepath{clip}%
\pgfsetbuttcap%
\pgfsetroundjoin%
\pgfsetlinewidth{1.003750pt}%
\definecolor{currentstroke}{rgb}{0.705673,0.015556,0.150233}%
\pgfsetstrokecolor{currentstroke}%
\pgfsetdash{}{0pt}%
\pgfpathmoveto{\pgfqpoint{2.230408in}{1.897072in}}%
\pgfpathlineto{\pgfqpoint{2.230408in}{1.897072in}}%
\pgfusepath{stroke}%
\end{pgfscope}%
\begin{pgfscope}%
\pgfpathrectangle{\pgfqpoint{0.552773in}{0.431673in}}{\pgfqpoint{3.738807in}{1.765244in}}%
\pgfusepath{clip}%
\pgfsetbuttcap%
\pgfsetroundjoin%
\pgfsetlinewidth{1.003750pt}%
\definecolor{currentstroke}{rgb}{0.705673,0.015556,0.150233}%
\pgfsetstrokecolor{currentstroke}%
\pgfsetdash{}{0pt}%
\pgfpathmoveto{\pgfqpoint{2.261554in}{1.885531in}}%
\pgfpathlineto{\pgfqpoint{2.261554in}{1.885531in}}%
\pgfusepath{stroke}%
\end{pgfscope}%
\begin{pgfscope}%
\pgfpathrectangle{\pgfqpoint{0.552773in}{0.431673in}}{\pgfqpoint{3.738807in}{1.765244in}}%
\pgfusepath{clip}%
\pgfsetbuttcap%
\pgfsetroundjoin%
\pgfsetlinewidth{1.003750pt}%
\definecolor{currentstroke}{rgb}{0.705673,0.015556,0.150233}%
\pgfsetstrokecolor{currentstroke}%
\pgfsetdash{}{0pt}%
\pgfpathmoveto{\pgfqpoint{2.282599in}{1.858308in}}%
\pgfpathlineto{\pgfqpoint{2.282599in}{1.858308in}}%
\pgfusepath{stroke}%
\end{pgfscope}%
\begin{pgfscope}%
\pgfpathrectangle{\pgfqpoint{0.552773in}{0.431673in}}{\pgfqpoint{3.738807in}{1.765244in}}%
\pgfusepath{clip}%
\pgfsetbuttcap%
\pgfsetroundjoin%
\pgfsetlinewidth{1.003750pt}%
\definecolor{currentstroke}{rgb}{0.705673,0.015556,0.150233}%
\pgfsetstrokecolor{currentstroke}%
\pgfsetdash{}{0pt}%
\pgfpathmoveto{\pgfqpoint{2.311609in}{1.843452in}}%
\pgfpathlineto{\pgfqpoint{2.311609in}{1.843452in}}%
\pgfusepath{stroke}%
\end{pgfscope}%
\begin{pgfscope}%
\pgfpathrectangle{\pgfqpoint{0.552773in}{0.431673in}}{\pgfqpoint{3.738807in}{1.765244in}}%
\pgfusepath{clip}%
\pgfsetbuttcap%
\pgfsetroundjoin%
\pgfsetlinewidth{1.003750pt}%
\definecolor{currentstroke}{rgb}{0.705673,0.015556,0.150233}%
\pgfsetstrokecolor{currentstroke}%
\pgfsetdash{}{0pt}%
\pgfpathmoveto{\pgfqpoint{2.360121in}{1.858871in}}%
\pgfpathlineto{\pgfqpoint{2.360121in}{1.858871in}}%
\pgfusepath{stroke}%
\end{pgfscope}%
\begin{pgfscope}%
\pgfpathrectangle{\pgfqpoint{0.552773in}{0.431673in}}{\pgfqpoint{3.738807in}{1.765244in}}%
\pgfusepath{clip}%
\pgfsetbuttcap%
\pgfsetroundjoin%
\pgfsetlinewidth{1.003750pt}%
\definecolor{currentstroke}{rgb}{0.705673,0.015556,0.150233}%
\pgfsetstrokecolor{currentstroke}%
\pgfsetdash{}{0pt}%
\pgfpathmoveto{\pgfqpoint{2.409930in}{1.876302in}}%
\pgfpathlineto{\pgfqpoint{2.409930in}{1.876302in}}%
\pgfusepath{stroke}%
\end{pgfscope}%
\begin{pgfscope}%
\pgfpathrectangle{\pgfqpoint{0.552773in}{0.431673in}}{\pgfqpoint{3.738807in}{1.765244in}}%
\pgfusepath{clip}%
\pgfsetbuttcap%
\pgfsetroundjoin%
\pgfsetlinewidth{1.003750pt}%
\definecolor{currentstroke}{rgb}{0.705673,0.015556,0.150233}%
\pgfsetstrokecolor{currentstroke}%
\pgfsetdash{}{0pt}%
\pgfpathmoveto{\pgfqpoint{2.475675in}{1.918474in}}%
\pgfpathlineto{\pgfqpoint{2.475675in}{1.918474in}}%
\pgfusepath{stroke}%
\end{pgfscope}%
\begin{pgfscope}%
\pgfpathrectangle{\pgfqpoint{0.552773in}{0.431673in}}{\pgfqpoint{3.738807in}{1.765244in}}%
\pgfusepath{clip}%
\pgfsetbuttcap%
\pgfsetroundjoin%
\pgfsetlinewidth{1.003750pt}%
\definecolor{currentstroke}{rgb}{0.705673,0.015556,0.150233}%
\pgfsetstrokecolor{currentstroke}%
\pgfsetdash{}{0pt}%
\pgfpathmoveto{\pgfqpoint{2.551377in}{1.976101in}}%
\pgfpathlineto{\pgfqpoint{2.551377in}{1.976101in}}%
\pgfusepath{stroke}%
\end{pgfscope}%
\begin{pgfscope}%
\pgfpathrectangle{\pgfqpoint{0.552773in}{0.431673in}}{\pgfqpoint{3.738807in}{1.765244in}}%
\pgfusepath{clip}%
\pgfsetbuttcap%
\pgfsetroundjoin%
\pgfsetlinewidth{1.003750pt}%
\definecolor{currentstroke}{rgb}{0.705673,0.015556,0.150233}%
\pgfsetstrokecolor{currentstroke}%
\pgfsetdash{}{0pt}%
\pgfpathmoveto{\pgfqpoint{2.632372in}{2.041948in}}%
\pgfpathlineto{\pgfqpoint{2.632372in}{2.041948in}}%
\pgfusepath{stroke}%
\end{pgfscope}%
\begin{pgfscope}%
\pgfpathrectangle{\pgfqpoint{0.552773in}{0.431673in}}{\pgfqpoint{3.738807in}{1.765244in}}%
\pgfusepath{clip}%
\pgfsetbuttcap%
\pgfsetroundjoin%
\pgfsetlinewidth{1.003750pt}%
\definecolor{currentstroke}{rgb}{0.705673,0.015556,0.150233}%
\pgfsetstrokecolor{currentstroke}%
\pgfsetdash{}{0pt}%
\pgfpathmoveto{\pgfqpoint{2.694363in}{2.078291in}}%
\pgfpathlineto{\pgfqpoint{2.694363in}{2.078291in}}%
\pgfusepath{stroke}%
\end{pgfscope}%
\begin{pgfscope}%
\pgfpathrectangle{\pgfqpoint{0.552773in}{0.431673in}}{\pgfqpoint{3.738807in}{1.765244in}}%
\pgfusepath{clip}%
\pgfsetbuttcap%
\pgfsetroundjoin%
\pgfsetlinewidth{1.003750pt}%
\definecolor{currentstroke}{rgb}{0.705673,0.015556,0.150233}%
\pgfsetstrokecolor{currentstroke}%
\pgfsetdash{}{0pt}%
\pgfpathmoveto{\pgfqpoint{2.745485in}{2.097761in}}%
\pgfpathlineto{\pgfqpoint{2.745485in}{2.097761in}}%
\pgfusepath{stroke}%
\end{pgfscope}%
\begin{pgfscope}%
\pgfpathrectangle{\pgfqpoint{0.552773in}{0.431673in}}{\pgfqpoint{3.738807in}{1.765244in}}%
\pgfusepath{clip}%
\pgfsetbuttcap%
\pgfsetroundjoin%
\pgfsetlinewidth{1.003750pt}%
\definecolor{currentstroke}{rgb}{0.705673,0.015556,0.150233}%
\pgfsetstrokecolor{currentstroke}%
\pgfsetdash{}{0pt}%
\pgfpathmoveto{\pgfqpoint{2.762412in}{2.064146in}}%
\pgfpathlineto{\pgfqpoint{2.762412in}{2.064146in}}%
\pgfusepath{stroke}%
\end{pgfscope}%
\begin{pgfscope}%
\pgfpathrectangle{\pgfqpoint{0.552773in}{0.431673in}}{\pgfqpoint{3.738807in}{1.765244in}}%
\pgfusepath{clip}%
\pgfsetbuttcap%
\pgfsetroundjoin%
\pgfsetlinewidth{1.003750pt}%
\definecolor{currentstroke}{rgb}{0.705673,0.015556,0.150233}%
\pgfsetstrokecolor{currentstroke}%
\pgfsetdash{}{0pt}%
\pgfpathmoveto{\pgfqpoint{2.787529in}{2.043244in}}%
\pgfpathlineto{\pgfqpoint{2.787529in}{2.043244in}}%
\pgfusepath{stroke}%
\end{pgfscope}%
\begin{pgfscope}%
\pgfpathrectangle{\pgfqpoint{0.552773in}{0.431673in}}{\pgfqpoint{3.738807in}{1.765244in}}%
\pgfusepath{clip}%
\pgfsetbuttcap%
\pgfsetroundjoin%
\pgfsetlinewidth{1.003750pt}%
\definecolor{currentstroke}{rgb}{0.705673,0.015556,0.150233}%
\pgfsetstrokecolor{currentstroke}%
\pgfsetdash{}{0pt}%
\pgfpathmoveto{\pgfqpoint{2.777791in}{1.968233in}}%
\pgfpathlineto{\pgfqpoint{2.777791in}{1.968233in}}%
\pgfusepath{stroke}%
\end{pgfscope}%
\begin{pgfscope}%
\pgfpathrectangle{\pgfqpoint{0.552773in}{0.431673in}}{\pgfqpoint{3.738807in}{1.765244in}}%
\pgfusepath{clip}%
\pgfsetbuttcap%
\pgfsetroundjoin%
\pgfsetlinewidth{1.003750pt}%
\definecolor{currentstroke}{rgb}{0.705673,0.015556,0.150233}%
\pgfsetstrokecolor{currentstroke}%
\pgfsetdash{}{0pt}%
\pgfpathmoveto{\pgfqpoint{2.788868in}{1.925536in}}%
\pgfpathlineto{\pgfqpoint{2.788868in}{1.925536in}}%
\pgfusepath{stroke}%
\end{pgfscope}%
\begin{pgfscope}%
\pgfpathrectangle{\pgfqpoint{0.552773in}{0.431673in}}{\pgfqpoint{3.738807in}{1.765244in}}%
\pgfusepath{clip}%
\pgfsetbuttcap%
\pgfsetroundjoin%
\pgfsetlinewidth{1.003750pt}%
\definecolor{currentstroke}{rgb}{0.705673,0.015556,0.150233}%
\pgfsetstrokecolor{currentstroke}%
\pgfsetdash{}{0pt}%
\pgfpathmoveto{\pgfqpoint{2.856282in}{1.850512in}}%
\pgfpathlineto{\pgfqpoint{2.856282in}{1.850512in}}%
\pgfusepath{stroke}%
\end{pgfscope}%
\begin{pgfscope}%
\pgfpathrectangle{\pgfqpoint{0.552773in}{0.431673in}}{\pgfqpoint{3.738807in}{1.765244in}}%
\pgfusepath{clip}%
\pgfsetbuttcap%
\pgfsetroundjoin%
\pgfsetlinewidth{1.003750pt}%
\definecolor{currentstroke}{rgb}{0.705673,0.015556,0.150233}%
\pgfsetstrokecolor{currentstroke}%
\pgfsetdash{}{0pt}%
\pgfpathmoveto{\pgfqpoint{2.870607in}{1.812857in}}%
\pgfpathlineto{\pgfqpoint{2.870607in}{1.812857in}}%
\pgfusepath{stroke}%
\end{pgfscope}%
\begin{pgfscope}%
\pgfpathrectangle{\pgfqpoint{0.552773in}{0.431673in}}{\pgfqpoint{3.738807in}{1.765244in}}%
\pgfusepath{clip}%
\pgfsetbuttcap%
\pgfsetroundjoin%
\pgfsetlinewidth{1.003750pt}%
\definecolor{currentstroke}{rgb}{0.705673,0.015556,0.150233}%
\pgfsetstrokecolor{currentstroke}%
\pgfsetdash{}{0pt}%
\pgfpathmoveto{\pgfqpoint{2.897552in}{1.794794in}}%
\pgfpathlineto{\pgfqpoint{2.897552in}{1.794794in}}%
\pgfusepath{stroke}%
\end{pgfscope}%
\begin{pgfscope}%
\pgfpathrectangle{\pgfqpoint{0.552773in}{0.431673in}}{\pgfqpoint{3.738807in}{1.765244in}}%
\pgfusepath{clip}%
\pgfsetbuttcap%
\pgfsetroundjoin%
\pgfsetlinewidth{1.003750pt}%
\definecolor{currentstroke}{rgb}{0.705673,0.015556,0.150233}%
\pgfsetstrokecolor{currentstroke}%
\pgfsetdash{}{0pt}%
\pgfpathmoveto{\pgfqpoint{2.926742in}{1.780216in}}%
\pgfpathlineto{\pgfqpoint{2.926742in}{1.780216in}}%
\pgfusepath{stroke}%
\end{pgfscope}%
\begin{pgfscope}%
\pgfpathrectangle{\pgfqpoint{0.552773in}{0.431673in}}{\pgfqpoint{3.738807in}{1.765244in}}%
\pgfusepath{clip}%
\pgfsetbuttcap%
\pgfsetroundjoin%
\pgfsetlinewidth{1.003750pt}%
\definecolor{currentstroke}{rgb}{0.705673,0.015556,0.150233}%
\pgfsetstrokecolor{currentstroke}%
\pgfsetdash{}{0pt}%
\pgfpathmoveto{\pgfqpoint{3.036531in}{1.830870in}}%
\pgfpathlineto{\pgfqpoint{3.036531in}{1.830870in}}%
\pgfusepath{stroke}%
\end{pgfscope}%
\begin{pgfscope}%
\pgfpathrectangle{\pgfqpoint{0.552773in}{0.431673in}}{\pgfqpoint{3.738807in}{1.765244in}}%
\pgfusepath{clip}%
\pgfsetbuttcap%
\pgfsetroundjoin%
\pgfsetlinewidth{1.003750pt}%
\definecolor{currentstroke}{rgb}{0.705673,0.015556,0.150233}%
\pgfsetstrokecolor{currentstroke}%
\pgfsetdash{}{0pt}%
\pgfpathmoveto{\pgfqpoint{3.326910in}{2.042090in}}%
\pgfpathlineto{\pgfqpoint{3.326910in}{2.042090in}}%
\pgfusepath{stroke}%
\end{pgfscope}%
\begin{pgfscope}%
\pgfpathrectangle{\pgfqpoint{0.552773in}{0.431673in}}{\pgfqpoint{3.738807in}{1.765244in}}%
\pgfusepath{clip}%
\pgfsetbuttcap%
\pgfsetroundjoin%
\pgfsetlinewidth{1.003750pt}%
\definecolor{currentstroke}{rgb}{0.705673,0.015556,0.150233}%
\pgfsetstrokecolor{currentstroke}%
\pgfsetdash{}{0pt}%
\pgfpathmoveto{\pgfqpoint{3.367472in}{2.045166in}}%
\pgfpathlineto{\pgfqpoint{3.367472in}{2.045166in}}%
\pgfusepath{stroke}%
\end{pgfscope}%
\begin{pgfscope}%
\pgfpathrectangle{\pgfqpoint{0.552773in}{0.431673in}}{\pgfqpoint{3.738807in}{1.765244in}}%
\pgfusepath{clip}%
\pgfsetbuttcap%
\pgfsetroundjoin%
\pgfsetlinewidth{1.003750pt}%
\definecolor{currentstroke}{rgb}{0.705673,0.015556,0.150233}%
\pgfsetstrokecolor{currentstroke}%
\pgfsetdash{}{0pt}%
\pgfpathmoveto{\pgfqpoint{3.396939in}{2.031018in}}%
\pgfpathlineto{\pgfqpoint{3.396939in}{2.031018in}}%
\pgfusepath{stroke}%
\end{pgfscope}%
\begin{pgfscope}%
\pgfpathrectangle{\pgfqpoint{0.552773in}{0.431673in}}{\pgfqpoint{3.738807in}{1.765244in}}%
\pgfusepath{clip}%
\pgfsetbuttcap%
\pgfsetroundjoin%
\pgfsetlinewidth{1.003750pt}%
\definecolor{currentstroke}{rgb}{0.705673,0.015556,0.150233}%
\pgfsetstrokecolor{currentstroke}%
\pgfsetdash{}{0pt}%
\pgfpathmoveto{\pgfqpoint{3.415993in}{2.000704in}}%
\pgfpathlineto{\pgfqpoint{3.415993in}{2.000704in}}%
\pgfusepath{stroke}%
\end{pgfscope}%
\begin{pgfscope}%
\pgfpathrectangle{\pgfqpoint{0.552773in}{0.431673in}}{\pgfqpoint{3.738807in}{1.765244in}}%
\pgfusepath{clip}%
\pgfsetbuttcap%
\pgfsetroundjoin%
\pgfsetlinewidth{1.003750pt}%
\definecolor{currentstroke}{rgb}{0.705673,0.015556,0.150233}%
\pgfsetstrokecolor{currentstroke}%
\pgfsetdash{}{0pt}%
\pgfpathmoveto{\pgfqpoint{3.427718in}{1.959014in}}%
\pgfpathlineto{\pgfqpoint{3.427718in}{1.959014in}}%
\pgfusepath{stroke}%
\end{pgfscope}%
\begin{pgfscope}%
\pgfpathrectangle{\pgfqpoint{0.552773in}{0.431673in}}{\pgfqpoint{3.738807in}{1.765244in}}%
\pgfusepath{clip}%
\pgfsetbuttcap%
\pgfsetroundjoin%
\pgfsetlinewidth{1.003750pt}%
\definecolor{currentstroke}{rgb}{0.705673,0.015556,0.150233}%
\pgfsetstrokecolor{currentstroke}%
\pgfsetdash{}{0pt}%
\pgfpathmoveto{\pgfqpoint{3.450369in}{1.934284in}}%
\pgfpathlineto{\pgfqpoint{3.450369in}{1.934284in}}%
\pgfusepath{stroke}%
\end{pgfscope}%
\begin{pgfscope}%
\pgfpathrectangle{\pgfqpoint{0.552773in}{0.431673in}}{\pgfqpoint{3.738807in}{1.765244in}}%
\pgfusepath{clip}%
\pgfsetbuttcap%
\pgfsetroundjoin%
\pgfsetlinewidth{1.003750pt}%
\definecolor{currentstroke}{rgb}{0.705673,0.015556,0.150233}%
\pgfsetstrokecolor{currentstroke}%
\pgfsetdash{}{0pt}%
\pgfpathmoveto{\pgfqpoint{3.478306in}{1.917762in}}%
\pgfpathlineto{\pgfqpoint{3.478306in}{1.917762in}}%
\pgfusepath{stroke}%
\end{pgfscope}%
\begin{pgfscope}%
\pgfpathrectangle{\pgfqpoint{0.552773in}{0.431673in}}{\pgfqpoint{3.738807in}{1.765244in}}%
\pgfusepath{clip}%
\pgfsetbuttcap%
\pgfsetroundjoin%
\pgfsetlinewidth{1.003750pt}%
\definecolor{currentstroke}{rgb}{0.705673,0.015556,0.150233}%
\pgfsetstrokecolor{currentstroke}%
\pgfsetdash{}{0pt}%
\pgfpathmoveto{\pgfqpoint{3.477156in}{1.856083in}}%
\pgfpathlineto{\pgfqpoint{3.477156in}{1.856083in}}%
\pgfusepath{stroke}%
\end{pgfscope}%
\begin{pgfscope}%
\pgfpathrectangle{\pgfqpoint{0.552773in}{0.431673in}}{\pgfqpoint{3.738807in}{1.765244in}}%
\pgfusepath{clip}%
\pgfsetbuttcap%
\pgfsetroundjoin%
\pgfsetlinewidth{1.003750pt}%
\definecolor{currentstroke}{rgb}{0.705673,0.015556,0.150233}%
\pgfsetstrokecolor{currentstroke}%
\pgfsetdash{}{0pt}%
\pgfpathmoveto{\pgfqpoint{3.509263in}{1.846033in}}%
\pgfpathlineto{\pgfqpoint{3.509263in}{1.846033in}}%
\pgfusepath{stroke}%
\end{pgfscope}%
\begin{pgfscope}%
\pgfpathrectangle{\pgfqpoint{0.552773in}{0.431673in}}{\pgfqpoint{3.738807in}{1.765244in}}%
\pgfusepath{clip}%
\pgfsetbuttcap%
\pgfsetroundjoin%
\pgfsetlinewidth{1.003750pt}%
\definecolor{currentstroke}{rgb}{0.705673,0.015556,0.150233}%
\pgfsetstrokecolor{currentstroke}%
\pgfsetdash{}{0pt}%
\pgfpathmoveto{\pgfqpoint{3.529555in}{1.817642in}}%
\pgfpathlineto{\pgfqpoint{3.529555in}{1.817642in}}%
\pgfusepath{stroke}%
\end{pgfscope}%
\begin{pgfscope}%
\pgfpathrectangle{\pgfqpoint{0.552773in}{0.431673in}}{\pgfqpoint{3.738807in}{1.765244in}}%
\pgfusepath{clip}%
\pgfsetbuttcap%
\pgfsetroundjoin%
\pgfsetlinewidth{1.003750pt}%
\definecolor{currentstroke}{rgb}{0.705673,0.015556,0.150233}%
\pgfsetstrokecolor{currentstroke}%
\pgfsetdash{}{0pt}%
\pgfpathmoveto{\pgfqpoint{3.565162in}{1.813026in}}%
\pgfpathlineto{\pgfqpoint{3.565162in}{1.813026in}}%
\pgfusepath{stroke}%
\end{pgfscope}%
\begin{pgfscope}%
\pgfpathrectangle{\pgfqpoint{0.552773in}{0.431673in}}{\pgfqpoint{3.738807in}{1.765244in}}%
\pgfusepath{clip}%
\pgfsetbuttcap%
\pgfsetroundjoin%
\pgfsetlinewidth{1.003750pt}%
\definecolor{currentstroke}{rgb}{0.705673,0.015556,0.150233}%
\pgfsetstrokecolor{currentstroke}%
\pgfsetdash{}{0pt}%
\pgfpathmoveto{\pgfqpoint{3.597068in}{1.802664in}}%
\pgfpathlineto{\pgfqpoint{3.597068in}{1.802664in}}%
\pgfusepath{stroke}%
\end{pgfscope}%
\begin{pgfscope}%
\pgfpathrectangle{\pgfqpoint{0.552773in}{0.431673in}}{\pgfqpoint{3.738807in}{1.765244in}}%
\pgfusepath{clip}%
\pgfsetbuttcap%
\pgfsetroundjoin%
\pgfsetlinewidth{1.003750pt}%
\definecolor{currentstroke}{rgb}{0.705673,0.015556,0.150233}%
\pgfsetstrokecolor{currentstroke}%
\pgfsetdash{}{0pt}%
\pgfpathmoveto{\pgfqpoint{3.641696in}{1.812053in}}%
\pgfpathlineto{\pgfqpoint{3.641696in}{1.812053in}}%
\pgfusepath{stroke}%
\end{pgfscope}%
\begin{pgfscope}%
\pgfpathrectangle{\pgfqpoint{0.552773in}{0.431673in}}{\pgfqpoint{3.738807in}{1.765244in}}%
\pgfusepath{clip}%
\pgfsetbuttcap%
\pgfsetroundjoin%
\pgfsetlinewidth{1.003750pt}%
\definecolor{currentstroke}{rgb}{0.705673,0.015556,0.150233}%
\pgfsetstrokecolor{currentstroke}%
\pgfsetdash{}{0pt}%
\pgfpathmoveto{\pgfqpoint{3.718669in}{1.871655in}}%
\pgfpathlineto{\pgfqpoint{3.718669in}{1.871655in}}%
\pgfusepath{stroke}%
\end{pgfscope}%
\begin{pgfscope}%
\pgfpathrectangle{\pgfqpoint{0.552773in}{0.431673in}}{\pgfqpoint{3.738807in}{1.765244in}}%
\pgfusepath{clip}%
\pgfsetbuttcap%
\pgfsetroundjoin%
\pgfsetlinewidth{1.003750pt}%
\definecolor{currentstroke}{rgb}{0.705673,0.015556,0.150233}%
\pgfsetstrokecolor{currentstroke}%
\pgfsetdash{}{0pt}%
\pgfpathmoveto{\pgfqpoint{3.775425in}{1.899872in}}%
\pgfpathlineto{\pgfqpoint{3.775425in}{1.899872in}}%
\pgfusepath{stroke}%
\end{pgfscope}%
\begin{pgfscope}%
\pgfpathrectangle{\pgfqpoint{0.552773in}{0.431673in}}{\pgfqpoint{3.738807in}{1.765244in}}%
\pgfusepath{clip}%
\pgfsetbuttcap%
\pgfsetroundjoin%
\pgfsetlinewidth{1.003750pt}%
\definecolor{currentstroke}{rgb}{0.705673,0.015556,0.150233}%
\pgfsetstrokecolor{currentstroke}%
\pgfsetdash{}{0pt}%
\pgfpathmoveto{\pgfqpoint{3.852269in}{1.959273in}}%
\pgfpathlineto{\pgfqpoint{3.852269in}{1.959273in}}%
\pgfusepath{stroke}%
\end{pgfscope}%
\begin{pgfscope}%
\pgfpathrectangle{\pgfqpoint{0.552773in}{0.431673in}}{\pgfqpoint{3.738807in}{1.765244in}}%
\pgfusepath{clip}%
\pgfsetbuttcap%
\pgfsetroundjoin%
\pgfsetlinewidth{1.003750pt}%
\definecolor{currentstroke}{rgb}{0.705673,0.015556,0.150233}%
\pgfsetstrokecolor{currentstroke}%
\pgfsetdash{}{0pt}%
\pgfpathmoveto{\pgfqpoint{3.927500in}{2.016170in}}%
\pgfpathlineto{\pgfqpoint{3.927500in}{2.016170in}}%
\pgfusepath{stroke}%
\end{pgfscope}%
\begin{pgfscope}%
\pgfpathrectangle{\pgfqpoint{0.552773in}{0.431673in}}{\pgfqpoint{3.738807in}{1.765244in}}%
\pgfusepath{clip}%
\pgfsetbuttcap%
\pgfsetroundjoin%
\pgfsetlinewidth{1.003750pt}%
\definecolor{currentstroke}{rgb}{0.705673,0.015556,0.150233}%
\pgfsetstrokecolor{currentstroke}%
\pgfsetdash{}{0pt}%
\pgfpathmoveto{\pgfqpoint{3.985669in}{2.046581in}}%
\pgfpathlineto{\pgfqpoint{3.985669in}{2.046581in}}%
\pgfusepath{stroke}%
\end{pgfscope}%
\begin{pgfscope}%
\pgfpathrectangle{\pgfqpoint{0.552773in}{0.431673in}}{\pgfqpoint{3.738807in}{1.765244in}}%
\pgfusepath{clip}%
\pgfsetbuttcap%
\pgfsetroundjoin%
\pgfsetlinewidth{1.003750pt}%
\definecolor{currentstroke}{rgb}{0.705673,0.015556,0.150233}%
\pgfsetstrokecolor{currentstroke}%
\pgfsetdash{}{0pt}%
\pgfpathmoveto{\pgfqpoint{4.026778in}{2.050506in}}%
\pgfpathlineto{\pgfqpoint{4.026778in}{2.050506in}}%
\pgfusepath{stroke}%
\end{pgfscope}%
\begin{pgfscope}%
\pgfpathrectangle{\pgfqpoint{0.552773in}{0.431673in}}{\pgfqpoint{3.738807in}{1.765244in}}%
\pgfusepath{clip}%
\pgfsetbuttcap%
\pgfsetroundjoin%
\pgfsetlinewidth{1.003750pt}%
\definecolor{currentstroke}{rgb}{0.705673,0.015556,0.150233}%
\pgfsetstrokecolor{currentstroke}%
\pgfsetdash{}{0pt}%
\pgfpathmoveto{\pgfqpoint{4.053468in}{2.032048in}}%
\pgfpathlineto{\pgfqpoint{4.053468in}{2.032048in}}%
\pgfusepath{stroke}%
\end{pgfscope}%
\begin{pgfscope}%
\pgfpathrectangle{\pgfqpoint{0.552773in}{0.431673in}}{\pgfqpoint{3.738807in}{1.765244in}}%
\pgfusepath{clip}%
\pgfsetbuttcap%
\pgfsetroundjoin%
\pgfsetlinewidth{1.003750pt}%
\definecolor{currentstroke}{rgb}{0.705673,0.015556,0.150233}%
\pgfsetstrokecolor{currentstroke}%
\pgfsetdash{}{0pt}%
\pgfpathmoveto{\pgfqpoint{4.068392in}{1.995322in}}%
\pgfpathlineto{\pgfqpoint{4.068392in}{1.995322in}}%
\pgfusepath{stroke}%
\end{pgfscope}%
\begin{pgfscope}%
\pgfpathrectangle{\pgfqpoint{0.552773in}{0.431673in}}{\pgfqpoint{3.738807in}{1.765244in}}%
\pgfusepath{clip}%
\pgfsetbuttcap%
\pgfsetroundjoin%
\pgfsetlinewidth{1.003750pt}%
\definecolor{currentstroke}{rgb}{0.705673,0.015556,0.150233}%
\pgfsetstrokecolor{currentstroke}%
\pgfsetdash{}{0pt}%
\pgfpathmoveto{\pgfqpoint{4.075955in}{1.947171in}}%
\pgfpathlineto{\pgfqpoint{4.075955in}{1.947171in}}%
\pgfusepath{stroke}%
\end{pgfscope}%
\begin{pgfscope}%
\pgfpathrectangle{\pgfqpoint{0.552773in}{0.431673in}}{\pgfqpoint{3.738807in}{1.765244in}}%
\pgfusepath{clip}%
\pgfsetbuttcap%
\pgfsetroundjoin%
\pgfsetlinewidth{1.003750pt}%
\definecolor{currentstroke}{rgb}{0.705673,0.015556,0.150233}%
\pgfsetstrokecolor{currentstroke}%
\pgfsetdash{}{0pt}%
\pgfpathmoveto{\pgfqpoint{4.096830in}{1.919684in}}%
\pgfpathlineto{\pgfqpoint{4.096830in}{1.919684in}}%
\pgfusepath{stroke}%
\end{pgfscope}%
\begin{pgfscope}%
\pgfpathrectangle{\pgfqpoint{0.552773in}{0.431673in}}{\pgfqpoint{3.738807in}{1.765244in}}%
\pgfusepath{clip}%
\pgfsetbuttcap%
\pgfsetroundjoin%
\pgfsetlinewidth{1.003750pt}%
\definecolor{currentstroke}{rgb}{0.705673,0.015556,0.150233}%
\pgfsetstrokecolor{currentstroke}%
\pgfsetdash{}{0pt}%
\pgfpathmoveto{\pgfqpoint{4.121635in}{1.898298in}}%
\pgfpathlineto{\pgfqpoint{4.121635in}{1.898298in}}%
\pgfusepath{stroke}%
\end{pgfscope}%
\begin{pgfscope}%
\pgfpathrectangle{\pgfqpoint{0.552773in}{0.431673in}}{\pgfqpoint{3.738807in}{1.765244in}}%
\pgfusepath{clip}%
\pgfsetbuttcap%
\pgfsetroundjoin%
\definecolor{currentfill}{rgb}{0.705673,0.015556,0.150233}%
\pgfsetfillcolor{currentfill}%
\pgfsetlinewidth{1.003750pt}%
\definecolor{currentstroke}{rgb}{0.705673,0.015556,0.150233}%
\pgfsetstrokecolor{currentstroke}%
\pgfsetdash{}{0pt}%
\pgfsys@defobject{currentmarker}{\pgfqpoint{0.000000in}{-0.027778in}}{\pgfqpoint{0.000000in}{0.027778in}}{%
\pgfpathmoveto{\pgfqpoint{0.000000in}{-0.027778in}}%
\pgfpathlineto{\pgfqpoint{0.000000in}{0.027778in}}%
\pgfusepath{stroke,fill}%
}%
\begin{pgfscope}%
\pgfsys@transformshift{0.722719in}{2.012117in}%
\pgfsys@useobject{currentmarker}{}%
\end{pgfscope}%
\begin{pgfscope}%
\pgfsys@transformshift{0.786936in}{1.992022in}%
\pgfsys@useobject{currentmarker}{}%
\end{pgfscope}%
\begin{pgfscope}%
\pgfsys@transformshift{0.819797in}{1.983144in}%
\pgfsys@useobject{currentmarker}{}%
\end{pgfscope}%
\begin{pgfscope}%
\pgfsys@transformshift{0.831560in}{1.941512in}%
\pgfsys@useobject{currentmarker}{}%
\end{pgfscope}%
\begin{pgfscope}%
\pgfsys@transformshift{0.837884in}{1.891436in}%
\pgfsys@useobject{currentmarker}{}%
\end{pgfscope}%
\begin{pgfscope}%
\pgfsys@transformshift{0.853441in}{1.855694in}%
\pgfsys@useobject{currentmarker}{}%
\end{pgfscope}%
\begin{pgfscope}%
\pgfsys@transformshift{0.871877in}{1.824422in}%
\pgfsys@useobject{currentmarker}{}%
\end{pgfscope}%
\begin{pgfscope}%
\pgfsys@transformshift{0.894231in}{1.799231in}%
\pgfsys@useobject{currentmarker}{}%
\end{pgfscope}%
\begin{pgfscope}%
\pgfsys@transformshift{0.917097in}{1.774835in}%
\pgfsys@useobject{currentmarker}{}%
\end{pgfscope}%
\begin{pgfscope}%
\pgfsys@transformshift{0.954263in}{1.772639in}%
\pgfsys@useobject{currentmarker}{}%
\end{pgfscope}%
\begin{pgfscope}%
\pgfsys@transformshift{1.054856in}{1.809017in}%
\pgfsys@useobject{currentmarker}{}%
\end{pgfscope}%
\begin{pgfscope}%
\pgfsys@transformshift{1.127913in}{1.862539in}%
\pgfsys@useobject{currentmarker}{}%
\end{pgfscope}%
\begin{pgfscope}%
\pgfsys@transformshift{1.216245in}{1.939776in}%
\pgfsys@useobject{currentmarker}{}%
\end{pgfscope}%
\begin{pgfscope}%
\pgfsys@transformshift{1.299586in}{2.009263in}%
\pgfsys@useobject{currentmarker}{}%
\end{pgfscope}%
\begin{pgfscope}%
\pgfsys@transformshift{1.383835in}{2.080160in}%
\pgfsys@useobject{currentmarker}{}%
\end{pgfscope}%
\begin{pgfscope}%
\pgfsys@transformshift{1.428876in}{2.090189in}%
\pgfsys@useobject{currentmarker}{}%
\end{pgfscope}%
\begin{pgfscope}%
\pgfsys@transformshift{1.461920in}{2.081596in}%
\pgfsys@useobject{currentmarker}{}%
\end{pgfscope}%
\begin{pgfscope}%
\pgfsys@transformshift{1.470951in}{2.035721in}%
\pgfsys@useobject{currentmarker}{}%
\end{pgfscope}%
\begin{pgfscope}%
\pgfsys@transformshift{1.500704in}{2.022017in}%
\pgfsys@useobject{currentmarker}{}%
\end{pgfscope}%
\begin{pgfscope}%
\pgfsys@transformshift{1.498835in}{1.959223in}%
\pgfsys@useobject{currentmarker}{}%
\end{pgfscope}%
\begin{pgfscope}%
\pgfsys@transformshift{1.510085in}{1.916795in}%
\pgfsys@useobject{currentmarker}{}%
\end{pgfscope}%
\begin{pgfscope}%
\pgfsys@transformshift{1.536220in}{1.897473in}%
\pgfsys@useobject{currentmarker}{}%
\end{pgfscope}%
\begin{pgfscope}%
\pgfsys@transformshift{1.565209in}{1.882584in}%
\pgfsys@useobject{currentmarker}{}%
\end{pgfscope}%
\begin{pgfscope}%
\pgfsys@transformshift{1.587354in}{1.857070in}%
\pgfsys@useobject{currentmarker}{}%
\end{pgfscope}%
\begin{pgfscope}%
\pgfsys@transformshift{1.624003in}{1.854071in}%
\pgfsys@useobject{currentmarker}{}%
\end{pgfscope}%
\begin{pgfscope}%
\pgfsys@transformshift{1.656435in}{1.844526in}%
\pgfsys@useobject{currentmarker}{}%
\end{pgfscope}%
\begin{pgfscope}%
\pgfsys@transformshift{1.704526in}{1.859291in}%
\pgfsys@useobject{currentmarker}{}%
\end{pgfscope}%
\begin{pgfscope}%
\pgfsys@transformshift{1.761295in}{1.887527in}%
\pgfsys@useobject{currentmarker}{}%
\end{pgfscope}%
\begin{pgfscope}%
\pgfsys@transformshift{1.828491in}{1.931951in}%
\pgfsys@useobject{currentmarker}{}%
\end{pgfscope}%
\begin{pgfscope}%
\pgfsys@transformshift{1.902753in}{1.987345in}%
\pgfsys@useobject{currentmarker}{}%
\end{pgfscope}%
\begin{pgfscope}%
\pgfsys@transformshift{1.989053in}{2.061425in}%
\pgfsys@useobject{currentmarker}{}%
\end{pgfscope}%
\begin{pgfscope}%
\pgfsys@transformshift{2.045418in}{2.089036in}%
\pgfsys@useobject{currentmarker}{}%
\end{pgfscope}%
\begin{pgfscope}%
\pgfsys@transformshift{2.101804in}{2.116678in}%
\pgfsys@useobject{currentmarker}{}%
\end{pgfscope}%
\begin{pgfscope}%
\pgfsys@transformshift{2.115398in}{2.077888in}%
\pgfsys@useobject{currentmarker}{}%
\end{pgfscope}%
\begin{pgfscope}%
\pgfsys@transformshift{2.157435in}{2.083254in}%
\pgfsys@useobject{currentmarker}{}%
\end{pgfscope}%
\begin{pgfscope}%
\pgfsys@transformshift{2.162229in}{2.030803in}%
\pgfsys@useobject{currentmarker}{}%
\end{pgfscope}%
\begin{pgfscope}%
\pgfsys@transformshift{2.178795in}{1.996627in}%
\pgfsys@useobject{currentmarker}{}%
\end{pgfscope}%
\begin{pgfscope}%
\pgfsys@transformshift{2.190577in}{1.955024in}%
\pgfsys@useobject{currentmarker}{}%
\end{pgfscope}%
\begin{pgfscope}%
\pgfsys@transformshift{2.210306in}{1.925759in}%
\pgfsys@useobject{currentmarker}{}%
\end{pgfscope}%
\begin{pgfscope}%
\pgfsys@transformshift{2.230408in}{1.897072in}%
\pgfsys@useobject{currentmarker}{}%
\end{pgfscope}%
\begin{pgfscope}%
\pgfsys@transformshift{2.261554in}{1.885531in}%
\pgfsys@useobject{currentmarker}{}%
\end{pgfscope}%
\begin{pgfscope}%
\pgfsys@transformshift{2.282599in}{1.858308in}%
\pgfsys@useobject{currentmarker}{}%
\end{pgfscope}%
\begin{pgfscope}%
\pgfsys@transformshift{2.311609in}{1.843452in}%
\pgfsys@useobject{currentmarker}{}%
\end{pgfscope}%
\begin{pgfscope}%
\pgfsys@transformshift{2.360121in}{1.858871in}%
\pgfsys@useobject{currentmarker}{}%
\end{pgfscope}%
\begin{pgfscope}%
\pgfsys@transformshift{2.409930in}{1.876302in}%
\pgfsys@useobject{currentmarker}{}%
\end{pgfscope}%
\begin{pgfscope}%
\pgfsys@transformshift{2.475675in}{1.918474in}%
\pgfsys@useobject{currentmarker}{}%
\end{pgfscope}%
\begin{pgfscope}%
\pgfsys@transformshift{2.551377in}{1.976101in}%
\pgfsys@useobject{currentmarker}{}%
\end{pgfscope}%
\begin{pgfscope}%
\pgfsys@transformshift{2.632372in}{2.041948in}%
\pgfsys@useobject{currentmarker}{}%
\end{pgfscope}%
\begin{pgfscope}%
\pgfsys@transformshift{2.694363in}{2.078291in}%
\pgfsys@useobject{currentmarker}{}%
\end{pgfscope}%
\begin{pgfscope}%
\pgfsys@transformshift{2.745485in}{2.097761in}%
\pgfsys@useobject{currentmarker}{}%
\end{pgfscope}%
\begin{pgfscope}%
\pgfsys@transformshift{2.762412in}{2.064146in}%
\pgfsys@useobject{currentmarker}{}%
\end{pgfscope}%
\begin{pgfscope}%
\pgfsys@transformshift{2.787529in}{2.043244in}%
\pgfsys@useobject{currentmarker}{}%
\end{pgfscope}%
\begin{pgfscope}%
\pgfsys@transformshift{2.777791in}{1.968233in}%
\pgfsys@useobject{currentmarker}{}%
\end{pgfscope}%
\begin{pgfscope}%
\pgfsys@transformshift{2.788868in}{1.925536in}%
\pgfsys@useobject{currentmarker}{}%
\end{pgfscope}%
\begin{pgfscope}%
\pgfsys@transformshift{2.856282in}{1.850512in}%
\pgfsys@useobject{currentmarker}{}%
\end{pgfscope}%
\begin{pgfscope}%
\pgfsys@transformshift{2.870607in}{1.812857in}%
\pgfsys@useobject{currentmarker}{}%
\end{pgfscope}%
\begin{pgfscope}%
\pgfsys@transformshift{2.897552in}{1.794794in}%
\pgfsys@useobject{currentmarker}{}%
\end{pgfscope}%
\begin{pgfscope}%
\pgfsys@transformshift{2.926742in}{1.780216in}%
\pgfsys@useobject{currentmarker}{}%
\end{pgfscope}%
\begin{pgfscope}%
\pgfsys@transformshift{3.036531in}{1.830870in}%
\pgfsys@useobject{currentmarker}{}%
\end{pgfscope}%
\begin{pgfscope}%
\pgfsys@transformshift{3.326910in}{2.042090in}%
\pgfsys@useobject{currentmarker}{}%
\end{pgfscope}%
\begin{pgfscope}%
\pgfsys@transformshift{3.367472in}{2.045166in}%
\pgfsys@useobject{currentmarker}{}%
\end{pgfscope}%
\begin{pgfscope}%
\pgfsys@transformshift{3.396939in}{2.031018in}%
\pgfsys@useobject{currentmarker}{}%
\end{pgfscope}%
\begin{pgfscope}%
\pgfsys@transformshift{3.415993in}{2.000704in}%
\pgfsys@useobject{currentmarker}{}%
\end{pgfscope}%
\begin{pgfscope}%
\pgfsys@transformshift{3.427718in}{1.959014in}%
\pgfsys@useobject{currentmarker}{}%
\end{pgfscope}%
\begin{pgfscope}%
\pgfsys@transformshift{3.450369in}{1.934284in}%
\pgfsys@useobject{currentmarker}{}%
\end{pgfscope}%
\begin{pgfscope}%
\pgfsys@transformshift{3.478306in}{1.917762in}%
\pgfsys@useobject{currentmarker}{}%
\end{pgfscope}%
\begin{pgfscope}%
\pgfsys@transformshift{3.477156in}{1.856083in}%
\pgfsys@useobject{currentmarker}{}%
\end{pgfscope}%
\begin{pgfscope}%
\pgfsys@transformshift{3.509263in}{1.846033in}%
\pgfsys@useobject{currentmarker}{}%
\end{pgfscope}%
\begin{pgfscope}%
\pgfsys@transformshift{3.529555in}{1.817642in}%
\pgfsys@useobject{currentmarker}{}%
\end{pgfscope}%
\begin{pgfscope}%
\pgfsys@transformshift{3.565162in}{1.813026in}%
\pgfsys@useobject{currentmarker}{}%
\end{pgfscope}%
\begin{pgfscope}%
\pgfsys@transformshift{3.597068in}{1.802664in}%
\pgfsys@useobject{currentmarker}{}%
\end{pgfscope}%
\begin{pgfscope}%
\pgfsys@transformshift{3.641696in}{1.812053in}%
\pgfsys@useobject{currentmarker}{}%
\end{pgfscope}%
\begin{pgfscope}%
\pgfsys@transformshift{3.718669in}{1.871655in}%
\pgfsys@useobject{currentmarker}{}%
\end{pgfscope}%
\begin{pgfscope}%
\pgfsys@transformshift{3.775425in}{1.899872in}%
\pgfsys@useobject{currentmarker}{}%
\end{pgfscope}%
\begin{pgfscope}%
\pgfsys@transformshift{3.852269in}{1.959273in}%
\pgfsys@useobject{currentmarker}{}%
\end{pgfscope}%
\begin{pgfscope}%
\pgfsys@transformshift{3.927500in}{2.016170in}%
\pgfsys@useobject{currentmarker}{}%
\end{pgfscope}%
\begin{pgfscope}%
\pgfsys@transformshift{3.985669in}{2.046581in}%
\pgfsys@useobject{currentmarker}{}%
\end{pgfscope}%
\begin{pgfscope}%
\pgfsys@transformshift{4.026778in}{2.050506in}%
\pgfsys@useobject{currentmarker}{}%
\end{pgfscope}%
\begin{pgfscope}%
\pgfsys@transformshift{4.053468in}{2.032048in}%
\pgfsys@useobject{currentmarker}{}%
\end{pgfscope}%
\begin{pgfscope}%
\pgfsys@transformshift{4.068392in}{1.995322in}%
\pgfsys@useobject{currentmarker}{}%
\end{pgfscope}%
\begin{pgfscope}%
\pgfsys@transformshift{4.075955in}{1.947171in}%
\pgfsys@useobject{currentmarker}{}%
\end{pgfscope}%
\begin{pgfscope}%
\pgfsys@transformshift{4.096830in}{1.919684in}%
\pgfsys@useobject{currentmarker}{}%
\end{pgfscope}%
\begin{pgfscope}%
\pgfsys@transformshift{4.121635in}{1.898298in}%
\pgfsys@useobject{currentmarker}{}%
\end{pgfscope}%
\end{pgfscope}%
\begin{pgfscope}%
\pgfpathrectangle{\pgfqpoint{0.552773in}{0.431673in}}{\pgfqpoint{3.738807in}{1.765244in}}%
\pgfusepath{clip}%
\pgfsetbuttcap%
\pgfsetroundjoin%
\definecolor{currentfill}{rgb}{0.705673,0.015556,0.150233}%
\pgfsetfillcolor{currentfill}%
\pgfsetlinewidth{1.003750pt}%
\definecolor{currentstroke}{rgb}{0.705673,0.015556,0.150233}%
\pgfsetstrokecolor{currentstroke}%
\pgfsetdash{}{0pt}%
\pgfsys@defobject{currentmarker}{\pgfqpoint{0.000000in}{-0.027778in}}{\pgfqpoint{0.000000in}{0.027778in}}{%
\pgfpathmoveto{\pgfqpoint{0.000000in}{-0.027778in}}%
\pgfpathlineto{\pgfqpoint{0.000000in}{0.027778in}}%
\pgfusepath{stroke,fill}%
}%
\begin{pgfscope}%
\pgfsys@transformshift{0.722719in}{2.012117in}%
\pgfsys@useobject{currentmarker}{}%
\end{pgfscope}%
\begin{pgfscope}%
\pgfsys@transformshift{0.786936in}{1.992022in}%
\pgfsys@useobject{currentmarker}{}%
\end{pgfscope}%
\begin{pgfscope}%
\pgfsys@transformshift{0.819797in}{1.983144in}%
\pgfsys@useobject{currentmarker}{}%
\end{pgfscope}%
\begin{pgfscope}%
\pgfsys@transformshift{0.831560in}{1.941512in}%
\pgfsys@useobject{currentmarker}{}%
\end{pgfscope}%
\begin{pgfscope}%
\pgfsys@transformshift{0.837884in}{1.891436in}%
\pgfsys@useobject{currentmarker}{}%
\end{pgfscope}%
\begin{pgfscope}%
\pgfsys@transformshift{0.853441in}{1.855694in}%
\pgfsys@useobject{currentmarker}{}%
\end{pgfscope}%
\begin{pgfscope}%
\pgfsys@transformshift{0.871877in}{1.824422in}%
\pgfsys@useobject{currentmarker}{}%
\end{pgfscope}%
\begin{pgfscope}%
\pgfsys@transformshift{0.894231in}{1.799231in}%
\pgfsys@useobject{currentmarker}{}%
\end{pgfscope}%
\begin{pgfscope}%
\pgfsys@transformshift{0.917097in}{1.774835in}%
\pgfsys@useobject{currentmarker}{}%
\end{pgfscope}%
\begin{pgfscope}%
\pgfsys@transformshift{0.954263in}{1.772639in}%
\pgfsys@useobject{currentmarker}{}%
\end{pgfscope}%
\begin{pgfscope}%
\pgfsys@transformshift{1.054856in}{1.809017in}%
\pgfsys@useobject{currentmarker}{}%
\end{pgfscope}%
\begin{pgfscope}%
\pgfsys@transformshift{1.127913in}{1.862539in}%
\pgfsys@useobject{currentmarker}{}%
\end{pgfscope}%
\begin{pgfscope}%
\pgfsys@transformshift{1.216245in}{1.939776in}%
\pgfsys@useobject{currentmarker}{}%
\end{pgfscope}%
\begin{pgfscope}%
\pgfsys@transformshift{1.299586in}{2.009263in}%
\pgfsys@useobject{currentmarker}{}%
\end{pgfscope}%
\begin{pgfscope}%
\pgfsys@transformshift{1.383835in}{2.080160in}%
\pgfsys@useobject{currentmarker}{}%
\end{pgfscope}%
\begin{pgfscope}%
\pgfsys@transformshift{1.428876in}{2.090189in}%
\pgfsys@useobject{currentmarker}{}%
\end{pgfscope}%
\begin{pgfscope}%
\pgfsys@transformshift{1.461920in}{2.081596in}%
\pgfsys@useobject{currentmarker}{}%
\end{pgfscope}%
\begin{pgfscope}%
\pgfsys@transformshift{1.470951in}{2.035721in}%
\pgfsys@useobject{currentmarker}{}%
\end{pgfscope}%
\begin{pgfscope}%
\pgfsys@transformshift{1.500704in}{2.022017in}%
\pgfsys@useobject{currentmarker}{}%
\end{pgfscope}%
\begin{pgfscope}%
\pgfsys@transformshift{1.498835in}{1.959223in}%
\pgfsys@useobject{currentmarker}{}%
\end{pgfscope}%
\begin{pgfscope}%
\pgfsys@transformshift{1.510085in}{1.916795in}%
\pgfsys@useobject{currentmarker}{}%
\end{pgfscope}%
\begin{pgfscope}%
\pgfsys@transformshift{1.536220in}{1.897473in}%
\pgfsys@useobject{currentmarker}{}%
\end{pgfscope}%
\begin{pgfscope}%
\pgfsys@transformshift{1.565209in}{1.882584in}%
\pgfsys@useobject{currentmarker}{}%
\end{pgfscope}%
\begin{pgfscope}%
\pgfsys@transformshift{1.587354in}{1.857070in}%
\pgfsys@useobject{currentmarker}{}%
\end{pgfscope}%
\begin{pgfscope}%
\pgfsys@transformshift{1.624003in}{1.854071in}%
\pgfsys@useobject{currentmarker}{}%
\end{pgfscope}%
\begin{pgfscope}%
\pgfsys@transformshift{1.656435in}{1.844526in}%
\pgfsys@useobject{currentmarker}{}%
\end{pgfscope}%
\begin{pgfscope}%
\pgfsys@transformshift{1.704526in}{1.859291in}%
\pgfsys@useobject{currentmarker}{}%
\end{pgfscope}%
\begin{pgfscope}%
\pgfsys@transformshift{1.761295in}{1.887527in}%
\pgfsys@useobject{currentmarker}{}%
\end{pgfscope}%
\begin{pgfscope}%
\pgfsys@transformshift{1.828491in}{1.931951in}%
\pgfsys@useobject{currentmarker}{}%
\end{pgfscope}%
\begin{pgfscope}%
\pgfsys@transformshift{1.902753in}{1.987345in}%
\pgfsys@useobject{currentmarker}{}%
\end{pgfscope}%
\begin{pgfscope}%
\pgfsys@transformshift{1.989053in}{2.061425in}%
\pgfsys@useobject{currentmarker}{}%
\end{pgfscope}%
\begin{pgfscope}%
\pgfsys@transformshift{2.045418in}{2.089036in}%
\pgfsys@useobject{currentmarker}{}%
\end{pgfscope}%
\begin{pgfscope}%
\pgfsys@transformshift{2.101804in}{2.116678in}%
\pgfsys@useobject{currentmarker}{}%
\end{pgfscope}%
\begin{pgfscope}%
\pgfsys@transformshift{2.115398in}{2.077888in}%
\pgfsys@useobject{currentmarker}{}%
\end{pgfscope}%
\begin{pgfscope}%
\pgfsys@transformshift{2.157435in}{2.083254in}%
\pgfsys@useobject{currentmarker}{}%
\end{pgfscope}%
\begin{pgfscope}%
\pgfsys@transformshift{2.162229in}{2.030803in}%
\pgfsys@useobject{currentmarker}{}%
\end{pgfscope}%
\begin{pgfscope}%
\pgfsys@transformshift{2.178795in}{1.996627in}%
\pgfsys@useobject{currentmarker}{}%
\end{pgfscope}%
\begin{pgfscope}%
\pgfsys@transformshift{2.190577in}{1.955024in}%
\pgfsys@useobject{currentmarker}{}%
\end{pgfscope}%
\begin{pgfscope}%
\pgfsys@transformshift{2.210306in}{1.925759in}%
\pgfsys@useobject{currentmarker}{}%
\end{pgfscope}%
\begin{pgfscope}%
\pgfsys@transformshift{2.230408in}{1.897072in}%
\pgfsys@useobject{currentmarker}{}%
\end{pgfscope}%
\begin{pgfscope}%
\pgfsys@transformshift{2.261554in}{1.885531in}%
\pgfsys@useobject{currentmarker}{}%
\end{pgfscope}%
\begin{pgfscope}%
\pgfsys@transformshift{2.282599in}{1.858308in}%
\pgfsys@useobject{currentmarker}{}%
\end{pgfscope}%
\begin{pgfscope}%
\pgfsys@transformshift{2.311609in}{1.843452in}%
\pgfsys@useobject{currentmarker}{}%
\end{pgfscope}%
\begin{pgfscope}%
\pgfsys@transformshift{2.360121in}{1.858871in}%
\pgfsys@useobject{currentmarker}{}%
\end{pgfscope}%
\begin{pgfscope}%
\pgfsys@transformshift{2.409930in}{1.876302in}%
\pgfsys@useobject{currentmarker}{}%
\end{pgfscope}%
\begin{pgfscope}%
\pgfsys@transformshift{2.475675in}{1.918474in}%
\pgfsys@useobject{currentmarker}{}%
\end{pgfscope}%
\begin{pgfscope}%
\pgfsys@transformshift{2.551377in}{1.976101in}%
\pgfsys@useobject{currentmarker}{}%
\end{pgfscope}%
\begin{pgfscope}%
\pgfsys@transformshift{2.632372in}{2.041948in}%
\pgfsys@useobject{currentmarker}{}%
\end{pgfscope}%
\begin{pgfscope}%
\pgfsys@transformshift{2.694363in}{2.078291in}%
\pgfsys@useobject{currentmarker}{}%
\end{pgfscope}%
\begin{pgfscope}%
\pgfsys@transformshift{2.745485in}{2.097761in}%
\pgfsys@useobject{currentmarker}{}%
\end{pgfscope}%
\begin{pgfscope}%
\pgfsys@transformshift{2.762412in}{2.064146in}%
\pgfsys@useobject{currentmarker}{}%
\end{pgfscope}%
\begin{pgfscope}%
\pgfsys@transformshift{2.787529in}{2.043244in}%
\pgfsys@useobject{currentmarker}{}%
\end{pgfscope}%
\begin{pgfscope}%
\pgfsys@transformshift{2.777791in}{1.968233in}%
\pgfsys@useobject{currentmarker}{}%
\end{pgfscope}%
\begin{pgfscope}%
\pgfsys@transformshift{2.788868in}{1.925536in}%
\pgfsys@useobject{currentmarker}{}%
\end{pgfscope}%
\begin{pgfscope}%
\pgfsys@transformshift{2.856282in}{1.850512in}%
\pgfsys@useobject{currentmarker}{}%
\end{pgfscope}%
\begin{pgfscope}%
\pgfsys@transformshift{2.870607in}{1.812857in}%
\pgfsys@useobject{currentmarker}{}%
\end{pgfscope}%
\begin{pgfscope}%
\pgfsys@transformshift{2.897552in}{1.794794in}%
\pgfsys@useobject{currentmarker}{}%
\end{pgfscope}%
\begin{pgfscope}%
\pgfsys@transformshift{2.926742in}{1.780216in}%
\pgfsys@useobject{currentmarker}{}%
\end{pgfscope}%
\begin{pgfscope}%
\pgfsys@transformshift{3.036531in}{1.830870in}%
\pgfsys@useobject{currentmarker}{}%
\end{pgfscope}%
\begin{pgfscope}%
\pgfsys@transformshift{3.326910in}{2.042090in}%
\pgfsys@useobject{currentmarker}{}%
\end{pgfscope}%
\begin{pgfscope}%
\pgfsys@transformshift{3.367472in}{2.045166in}%
\pgfsys@useobject{currentmarker}{}%
\end{pgfscope}%
\begin{pgfscope}%
\pgfsys@transformshift{3.396939in}{2.031018in}%
\pgfsys@useobject{currentmarker}{}%
\end{pgfscope}%
\begin{pgfscope}%
\pgfsys@transformshift{3.415993in}{2.000704in}%
\pgfsys@useobject{currentmarker}{}%
\end{pgfscope}%
\begin{pgfscope}%
\pgfsys@transformshift{3.427718in}{1.959014in}%
\pgfsys@useobject{currentmarker}{}%
\end{pgfscope}%
\begin{pgfscope}%
\pgfsys@transformshift{3.450369in}{1.934284in}%
\pgfsys@useobject{currentmarker}{}%
\end{pgfscope}%
\begin{pgfscope}%
\pgfsys@transformshift{3.478306in}{1.917762in}%
\pgfsys@useobject{currentmarker}{}%
\end{pgfscope}%
\begin{pgfscope}%
\pgfsys@transformshift{3.477156in}{1.856083in}%
\pgfsys@useobject{currentmarker}{}%
\end{pgfscope}%
\begin{pgfscope}%
\pgfsys@transformshift{3.509263in}{1.846033in}%
\pgfsys@useobject{currentmarker}{}%
\end{pgfscope}%
\begin{pgfscope}%
\pgfsys@transformshift{3.529555in}{1.817642in}%
\pgfsys@useobject{currentmarker}{}%
\end{pgfscope}%
\begin{pgfscope}%
\pgfsys@transformshift{3.565162in}{1.813026in}%
\pgfsys@useobject{currentmarker}{}%
\end{pgfscope}%
\begin{pgfscope}%
\pgfsys@transformshift{3.597068in}{1.802664in}%
\pgfsys@useobject{currentmarker}{}%
\end{pgfscope}%
\begin{pgfscope}%
\pgfsys@transformshift{3.641696in}{1.812053in}%
\pgfsys@useobject{currentmarker}{}%
\end{pgfscope}%
\begin{pgfscope}%
\pgfsys@transformshift{3.718669in}{1.871655in}%
\pgfsys@useobject{currentmarker}{}%
\end{pgfscope}%
\begin{pgfscope}%
\pgfsys@transformshift{3.775425in}{1.899872in}%
\pgfsys@useobject{currentmarker}{}%
\end{pgfscope}%
\begin{pgfscope}%
\pgfsys@transformshift{3.852269in}{1.959273in}%
\pgfsys@useobject{currentmarker}{}%
\end{pgfscope}%
\begin{pgfscope}%
\pgfsys@transformshift{3.927500in}{2.016170in}%
\pgfsys@useobject{currentmarker}{}%
\end{pgfscope}%
\begin{pgfscope}%
\pgfsys@transformshift{3.985669in}{2.046581in}%
\pgfsys@useobject{currentmarker}{}%
\end{pgfscope}%
\begin{pgfscope}%
\pgfsys@transformshift{4.026778in}{2.050506in}%
\pgfsys@useobject{currentmarker}{}%
\end{pgfscope}%
\begin{pgfscope}%
\pgfsys@transformshift{4.053468in}{2.032048in}%
\pgfsys@useobject{currentmarker}{}%
\end{pgfscope}%
\begin{pgfscope}%
\pgfsys@transformshift{4.068392in}{1.995322in}%
\pgfsys@useobject{currentmarker}{}%
\end{pgfscope}%
\begin{pgfscope}%
\pgfsys@transformshift{4.075955in}{1.947171in}%
\pgfsys@useobject{currentmarker}{}%
\end{pgfscope}%
\begin{pgfscope}%
\pgfsys@transformshift{4.096830in}{1.919684in}%
\pgfsys@useobject{currentmarker}{}%
\end{pgfscope}%
\begin{pgfscope}%
\pgfsys@transformshift{4.121635in}{1.898298in}%
\pgfsys@useobject{currentmarker}{}%
\end{pgfscope}%
\end{pgfscope}%
\begin{pgfscope}%
\pgfpathrectangle{\pgfqpoint{0.552773in}{0.431673in}}{\pgfqpoint{3.738807in}{1.765244in}}%
\pgfusepath{clip}%
\pgfsetbuttcap%
\pgfsetroundjoin%
\definecolor{currentfill}{rgb}{0.705673,0.015556,0.150233}%
\pgfsetfillcolor{currentfill}%
\pgfsetlinewidth{1.003750pt}%
\definecolor{currentstroke}{rgb}{0.705673,0.015556,0.150233}%
\pgfsetstrokecolor{currentstroke}%
\pgfsetdash{}{0pt}%
\pgfsys@defobject{currentmarker}{\pgfqpoint{-0.027778in}{-0.000000in}}{\pgfqpoint{0.027778in}{0.000000in}}{%
\pgfpathmoveto{\pgfqpoint{0.027778in}{-0.000000in}}%
\pgfpathlineto{\pgfqpoint{-0.027778in}{0.000000in}}%
\pgfusepath{stroke,fill}%
}%
\begin{pgfscope}%
\pgfsys@transformshift{0.722719in}{2.012117in}%
\pgfsys@useobject{currentmarker}{}%
\end{pgfscope}%
\begin{pgfscope}%
\pgfsys@transformshift{0.786936in}{1.992022in}%
\pgfsys@useobject{currentmarker}{}%
\end{pgfscope}%
\begin{pgfscope}%
\pgfsys@transformshift{0.819797in}{1.983144in}%
\pgfsys@useobject{currentmarker}{}%
\end{pgfscope}%
\begin{pgfscope}%
\pgfsys@transformshift{0.831560in}{1.941512in}%
\pgfsys@useobject{currentmarker}{}%
\end{pgfscope}%
\begin{pgfscope}%
\pgfsys@transformshift{0.837884in}{1.891436in}%
\pgfsys@useobject{currentmarker}{}%
\end{pgfscope}%
\begin{pgfscope}%
\pgfsys@transformshift{0.853441in}{1.855694in}%
\pgfsys@useobject{currentmarker}{}%
\end{pgfscope}%
\begin{pgfscope}%
\pgfsys@transformshift{0.871877in}{1.824422in}%
\pgfsys@useobject{currentmarker}{}%
\end{pgfscope}%
\begin{pgfscope}%
\pgfsys@transformshift{0.894231in}{1.799231in}%
\pgfsys@useobject{currentmarker}{}%
\end{pgfscope}%
\begin{pgfscope}%
\pgfsys@transformshift{0.917097in}{1.774835in}%
\pgfsys@useobject{currentmarker}{}%
\end{pgfscope}%
\begin{pgfscope}%
\pgfsys@transformshift{0.954263in}{1.772639in}%
\pgfsys@useobject{currentmarker}{}%
\end{pgfscope}%
\begin{pgfscope}%
\pgfsys@transformshift{1.054856in}{1.809017in}%
\pgfsys@useobject{currentmarker}{}%
\end{pgfscope}%
\begin{pgfscope}%
\pgfsys@transformshift{1.127913in}{1.862539in}%
\pgfsys@useobject{currentmarker}{}%
\end{pgfscope}%
\begin{pgfscope}%
\pgfsys@transformshift{1.216245in}{1.939776in}%
\pgfsys@useobject{currentmarker}{}%
\end{pgfscope}%
\begin{pgfscope}%
\pgfsys@transformshift{1.299586in}{2.009263in}%
\pgfsys@useobject{currentmarker}{}%
\end{pgfscope}%
\begin{pgfscope}%
\pgfsys@transformshift{1.383835in}{2.080160in}%
\pgfsys@useobject{currentmarker}{}%
\end{pgfscope}%
\begin{pgfscope}%
\pgfsys@transformshift{1.428876in}{2.090189in}%
\pgfsys@useobject{currentmarker}{}%
\end{pgfscope}%
\begin{pgfscope}%
\pgfsys@transformshift{1.461920in}{2.081596in}%
\pgfsys@useobject{currentmarker}{}%
\end{pgfscope}%
\begin{pgfscope}%
\pgfsys@transformshift{1.470951in}{2.035721in}%
\pgfsys@useobject{currentmarker}{}%
\end{pgfscope}%
\begin{pgfscope}%
\pgfsys@transformshift{1.500704in}{2.022017in}%
\pgfsys@useobject{currentmarker}{}%
\end{pgfscope}%
\begin{pgfscope}%
\pgfsys@transformshift{1.498835in}{1.959223in}%
\pgfsys@useobject{currentmarker}{}%
\end{pgfscope}%
\begin{pgfscope}%
\pgfsys@transformshift{1.510085in}{1.916795in}%
\pgfsys@useobject{currentmarker}{}%
\end{pgfscope}%
\begin{pgfscope}%
\pgfsys@transformshift{1.536220in}{1.897473in}%
\pgfsys@useobject{currentmarker}{}%
\end{pgfscope}%
\begin{pgfscope}%
\pgfsys@transformshift{1.565209in}{1.882584in}%
\pgfsys@useobject{currentmarker}{}%
\end{pgfscope}%
\begin{pgfscope}%
\pgfsys@transformshift{1.587354in}{1.857070in}%
\pgfsys@useobject{currentmarker}{}%
\end{pgfscope}%
\begin{pgfscope}%
\pgfsys@transformshift{1.624003in}{1.854071in}%
\pgfsys@useobject{currentmarker}{}%
\end{pgfscope}%
\begin{pgfscope}%
\pgfsys@transformshift{1.656435in}{1.844526in}%
\pgfsys@useobject{currentmarker}{}%
\end{pgfscope}%
\begin{pgfscope}%
\pgfsys@transformshift{1.704526in}{1.859291in}%
\pgfsys@useobject{currentmarker}{}%
\end{pgfscope}%
\begin{pgfscope}%
\pgfsys@transformshift{1.761295in}{1.887527in}%
\pgfsys@useobject{currentmarker}{}%
\end{pgfscope}%
\begin{pgfscope}%
\pgfsys@transformshift{1.828491in}{1.931951in}%
\pgfsys@useobject{currentmarker}{}%
\end{pgfscope}%
\begin{pgfscope}%
\pgfsys@transformshift{1.902753in}{1.987345in}%
\pgfsys@useobject{currentmarker}{}%
\end{pgfscope}%
\begin{pgfscope}%
\pgfsys@transformshift{1.989053in}{2.061425in}%
\pgfsys@useobject{currentmarker}{}%
\end{pgfscope}%
\begin{pgfscope}%
\pgfsys@transformshift{2.045418in}{2.089036in}%
\pgfsys@useobject{currentmarker}{}%
\end{pgfscope}%
\begin{pgfscope}%
\pgfsys@transformshift{2.101804in}{2.116678in}%
\pgfsys@useobject{currentmarker}{}%
\end{pgfscope}%
\begin{pgfscope}%
\pgfsys@transformshift{2.115398in}{2.077888in}%
\pgfsys@useobject{currentmarker}{}%
\end{pgfscope}%
\begin{pgfscope}%
\pgfsys@transformshift{2.157435in}{2.083254in}%
\pgfsys@useobject{currentmarker}{}%
\end{pgfscope}%
\begin{pgfscope}%
\pgfsys@transformshift{2.162229in}{2.030803in}%
\pgfsys@useobject{currentmarker}{}%
\end{pgfscope}%
\begin{pgfscope}%
\pgfsys@transformshift{2.178795in}{1.996627in}%
\pgfsys@useobject{currentmarker}{}%
\end{pgfscope}%
\begin{pgfscope}%
\pgfsys@transformshift{2.190577in}{1.955024in}%
\pgfsys@useobject{currentmarker}{}%
\end{pgfscope}%
\begin{pgfscope}%
\pgfsys@transformshift{2.210306in}{1.925759in}%
\pgfsys@useobject{currentmarker}{}%
\end{pgfscope}%
\begin{pgfscope}%
\pgfsys@transformshift{2.230408in}{1.897072in}%
\pgfsys@useobject{currentmarker}{}%
\end{pgfscope}%
\begin{pgfscope}%
\pgfsys@transformshift{2.261554in}{1.885531in}%
\pgfsys@useobject{currentmarker}{}%
\end{pgfscope}%
\begin{pgfscope}%
\pgfsys@transformshift{2.282599in}{1.858308in}%
\pgfsys@useobject{currentmarker}{}%
\end{pgfscope}%
\begin{pgfscope}%
\pgfsys@transformshift{2.311609in}{1.843452in}%
\pgfsys@useobject{currentmarker}{}%
\end{pgfscope}%
\begin{pgfscope}%
\pgfsys@transformshift{2.360121in}{1.858871in}%
\pgfsys@useobject{currentmarker}{}%
\end{pgfscope}%
\begin{pgfscope}%
\pgfsys@transformshift{2.409930in}{1.876302in}%
\pgfsys@useobject{currentmarker}{}%
\end{pgfscope}%
\begin{pgfscope}%
\pgfsys@transformshift{2.475675in}{1.918474in}%
\pgfsys@useobject{currentmarker}{}%
\end{pgfscope}%
\begin{pgfscope}%
\pgfsys@transformshift{2.551377in}{1.976101in}%
\pgfsys@useobject{currentmarker}{}%
\end{pgfscope}%
\begin{pgfscope}%
\pgfsys@transformshift{2.632372in}{2.041948in}%
\pgfsys@useobject{currentmarker}{}%
\end{pgfscope}%
\begin{pgfscope}%
\pgfsys@transformshift{2.694363in}{2.078291in}%
\pgfsys@useobject{currentmarker}{}%
\end{pgfscope}%
\begin{pgfscope}%
\pgfsys@transformshift{2.745485in}{2.097761in}%
\pgfsys@useobject{currentmarker}{}%
\end{pgfscope}%
\begin{pgfscope}%
\pgfsys@transformshift{2.762412in}{2.064146in}%
\pgfsys@useobject{currentmarker}{}%
\end{pgfscope}%
\begin{pgfscope}%
\pgfsys@transformshift{2.787529in}{2.043244in}%
\pgfsys@useobject{currentmarker}{}%
\end{pgfscope}%
\begin{pgfscope}%
\pgfsys@transformshift{2.777791in}{1.968233in}%
\pgfsys@useobject{currentmarker}{}%
\end{pgfscope}%
\begin{pgfscope}%
\pgfsys@transformshift{2.788868in}{1.925536in}%
\pgfsys@useobject{currentmarker}{}%
\end{pgfscope}%
\begin{pgfscope}%
\pgfsys@transformshift{2.856282in}{1.850512in}%
\pgfsys@useobject{currentmarker}{}%
\end{pgfscope}%
\begin{pgfscope}%
\pgfsys@transformshift{2.870607in}{1.812857in}%
\pgfsys@useobject{currentmarker}{}%
\end{pgfscope}%
\begin{pgfscope}%
\pgfsys@transformshift{2.897552in}{1.794794in}%
\pgfsys@useobject{currentmarker}{}%
\end{pgfscope}%
\begin{pgfscope}%
\pgfsys@transformshift{2.926742in}{1.780216in}%
\pgfsys@useobject{currentmarker}{}%
\end{pgfscope}%
\begin{pgfscope}%
\pgfsys@transformshift{3.036531in}{1.830870in}%
\pgfsys@useobject{currentmarker}{}%
\end{pgfscope}%
\begin{pgfscope}%
\pgfsys@transformshift{3.326910in}{2.042090in}%
\pgfsys@useobject{currentmarker}{}%
\end{pgfscope}%
\begin{pgfscope}%
\pgfsys@transformshift{3.367472in}{2.045166in}%
\pgfsys@useobject{currentmarker}{}%
\end{pgfscope}%
\begin{pgfscope}%
\pgfsys@transformshift{3.396939in}{2.031018in}%
\pgfsys@useobject{currentmarker}{}%
\end{pgfscope}%
\begin{pgfscope}%
\pgfsys@transformshift{3.415993in}{2.000704in}%
\pgfsys@useobject{currentmarker}{}%
\end{pgfscope}%
\begin{pgfscope}%
\pgfsys@transformshift{3.427718in}{1.959014in}%
\pgfsys@useobject{currentmarker}{}%
\end{pgfscope}%
\begin{pgfscope}%
\pgfsys@transformshift{3.450369in}{1.934284in}%
\pgfsys@useobject{currentmarker}{}%
\end{pgfscope}%
\begin{pgfscope}%
\pgfsys@transformshift{3.478306in}{1.917762in}%
\pgfsys@useobject{currentmarker}{}%
\end{pgfscope}%
\begin{pgfscope}%
\pgfsys@transformshift{3.477156in}{1.856083in}%
\pgfsys@useobject{currentmarker}{}%
\end{pgfscope}%
\begin{pgfscope}%
\pgfsys@transformshift{3.509263in}{1.846033in}%
\pgfsys@useobject{currentmarker}{}%
\end{pgfscope}%
\begin{pgfscope}%
\pgfsys@transformshift{3.529555in}{1.817642in}%
\pgfsys@useobject{currentmarker}{}%
\end{pgfscope}%
\begin{pgfscope}%
\pgfsys@transformshift{3.565162in}{1.813026in}%
\pgfsys@useobject{currentmarker}{}%
\end{pgfscope}%
\begin{pgfscope}%
\pgfsys@transformshift{3.597068in}{1.802664in}%
\pgfsys@useobject{currentmarker}{}%
\end{pgfscope}%
\begin{pgfscope}%
\pgfsys@transformshift{3.641696in}{1.812053in}%
\pgfsys@useobject{currentmarker}{}%
\end{pgfscope}%
\begin{pgfscope}%
\pgfsys@transformshift{3.718669in}{1.871655in}%
\pgfsys@useobject{currentmarker}{}%
\end{pgfscope}%
\begin{pgfscope}%
\pgfsys@transformshift{3.775425in}{1.899872in}%
\pgfsys@useobject{currentmarker}{}%
\end{pgfscope}%
\begin{pgfscope}%
\pgfsys@transformshift{3.852269in}{1.959273in}%
\pgfsys@useobject{currentmarker}{}%
\end{pgfscope}%
\begin{pgfscope}%
\pgfsys@transformshift{3.927500in}{2.016170in}%
\pgfsys@useobject{currentmarker}{}%
\end{pgfscope}%
\begin{pgfscope}%
\pgfsys@transformshift{3.985669in}{2.046581in}%
\pgfsys@useobject{currentmarker}{}%
\end{pgfscope}%
\begin{pgfscope}%
\pgfsys@transformshift{4.026778in}{2.050506in}%
\pgfsys@useobject{currentmarker}{}%
\end{pgfscope}%
\begin{pgfscope}%
\pgfsys@transformshift{4.053468in}{2.032048in}%
\pgfsys@useobject{currentmarker}{}%
\end{pgfscope}%
\begin{pgfscope}%
\pgfsys@transformshift{4.068392in}{1.995322in}%
\pgfsys@useobject{currentmarker}{}%
\end{pgfscope}%
\begin{pgfscope}%
\pgfsys@transformshift{4.075955in}{1.947171in}%
\pgfsys@useobject{currentmarker}{}%
\end{pgfscope}%
\begin{pgfscope}%
\pgfsys@transformshift{4.096830in}{1.919684in}%
\pgfsys@useobject{currentmarker}{}%
\end{pgfscope}%
\begin{pgfscope}%
\pgfsys@transformshift{4.121635in}{1.898298in}%
\pgfsys@useobject{currentmarker}{}%
\end{pgfscope}%
\end{pgfscope}%
\begin{pgfscope}%
\pgfpathrectangle{\pgfqpoint{0.552773in}{0.431673in}}{\pgfqpoint{3.738807in}{1.765244in}}%
\pgfusepath{clip}%
\pgfsetbuttcap%
\pgfsetroundjoin%
\definecolor{currentfill}{rgb}{0.705673,0.015556,0.150233}%
\pgfsetfillcolor{currentfill}%
\pgfsetlinewidth{1.003750pt}%
\definecolor{currentstroke}{rgb}{0.705673,0.015556,0.150233}%
\pgfsetstrokecolor{currentstroke}%
\pgfsetdash{}{0pt}%
\pgfsys@defobject{currentmarker}{\pgfqpoint{-0.027778in}{-0.000000in}}{\pgfqpoint{0.027778in}{0.000000in}}{%
\pgfpathmoveto{\pgfqpoint{0.027778in}{-0.000000in}}%
\pgfpathlineto{\pgfqpoint{-0.027778in}{0.000000in}}%
\pgfusepath{stroke,fill}%
}%
\begin{pgfscope}%
\pgfsys@transformshift{0.722719in}{2.012117in}%
\pgfsys@useobject{currentmarker}{}%
\end{pgfscope}%
\begin{pgfscope}%
\pgfsys@transformshift{0.786936in}{1.992022in}%
\pgfsys@useobject{currentmarker}{}%
\end{pgfscope}%
\begin{pgfscope}%
\pgfsys@transformshift{0.819797in}{1.983144in}%
\pgfsys@useobject{currentmarker}{}%
\end{pgfscope}%
\begin{pgfscope}%
\pgfsys@transformshift{0.831560in}{1.941512in}%
\pgfsys@useobject{currentmarker}{}%
\end{pgfscope}%
\begin{pgfscope}%
\pgfsys@transformshift{0.837884in}{1.891436in}%
\pgfsys@useobject{currentmarker}{}%
\end{pgfscope}%
\begin{pgfscope}%
\pgfsys@transformshift{0.853441in}{1.855694in}%
\pgfsys@useobject{currentmarker}{}%
\end{pgfscope}%
\begin{pgfscope}%
\pgfsys@transformshift{0.871877in}{1.824422in}%
\pgfsys@useobject{currentmarker}{}%
\end{pgfscope}%
\begin{pgfscope}%
\pgfsys@transformshift{0.894231in}{1.799231in}%
\pgfsys@useobject{currentmarker}{}%
\end{pgfscope}%
\begin{pgfscope}%
\pgfsys@transformshift{0.917097in}{1.774835in}%
\pgfsys@useobject{currentmarker}{}%
\end{pgfscope}%
\begin{pgfscope}%
\pgfsys@transformshift{0.954263in}{1.772639in}%
\pgfsys@useobject{currentmarker}{}%
\end{pgfscope}%
\begin{pgfscope}%
\pgfsys@transformshift{1.054856in}{1.809017in}%
\pgfsys@useobject{currentmarker}{}%
\end{pgfscope}%
\begin{pgfscope}%
\pgfsys@transformshift{1.127913in}{1.862539in}%
\pgfsys@useobject{currentmarker}{}%
\end{pgfscope}%
\begin{pgfscope}%
\pgfsys@transformshift{1.216245in}{1.939776in}%
\pgfsys@useobject{currentmarker}{}%
\end{pgfscope}%
\begin{pgfscope}%
\pgfsys@transformshift{1.299586in}{2.009263in}%
\pgfsys@useobject{currentmarker}{}%
\end{pgfscope}%
\begin{pgfscope}%
\pgfsys@transformshift{1.383835in}{2.080160in}%
\pgfsys@useobject{currentmarker}{}%
\end{pgfscope}%
\begin{pgfscope}%
\pgfsys@transformshift{1.428876in}{2.090189in}%
\pgfsys@useobject{currentmarker}{}%
\end{pgfscope}%
\begin{pgfscope}%
\pgfsys@transformshift{1.461920in}{2.081596in}%
\pgfsys@useobject{currentmarker}{}%
\end{pgfscope}%
\begin{pgfscope}%
\pgfsys@transformshift{1.470951in}{2.035721in}%
\pgfsys@useobject{currentmarker}{}%
\end{pgfscope}%
\begin{pgfscope}%
\pgfsys@transformshift{1.500704in}{2.022017in}%
\pgfsys@useobject{currentmarker}{}%
\end{pgfscope}%
\begin{pgfscope}%
\pgfsys@transformshift{1.498835in}{1.959223in}%
\pgfsys@useobject{currentmarker}{}%
\end{pgfscope}%
\begin{pgfscope}%
\pgfsys@transformshift{1.510085in}{1.916795in}%
\pgfsys@useobject{currentmarker}{}%
\end{pgfscope}%
\begin{pgfscope}%
\pgfsys@transformshift{1.536220in}{1.897473in}%
\pgfsys@useobject{currentmarker}{}%
\end{pgfscope}%
\begin{pgfscope}%
\pgfsys@transformshift{1.565209in}{1.882584in}%
\pgfsys@useobject{currentmarker}{}%
\end{pgfscope}%
\begin{pgfscope}%
\pgfsys@transformshift{1.587354in}{1.857070in}%
\pgfsys@useobject{currentmarker}{}%
\end{pgfscope}%
\begin{pgfscope}%
\pgfsys@transformshift{1.624003in}{1.854071in}%
\pgfsys@useobject{currentmarker}{}%
\end{pgfscope}%
\begin{pgfscope}%
\pgfsys@transformshift{1.656435in}{1.844526in}%
\pgfsys@useobject{currentmarker}{}%
\end{pgfscope}%
\begin{pgfscope}%
\pgfsys@transformshift{1.704526in}{1.859291in}%
\pgfsys@useobject{currentmarker}{}%
\end{pgfscope}%
\begin{pgfscope}%
\pgfsys@transformshift{1.761295in}{1.887527in}%
\pgfsys@useobject{currentmarker}{}%
\end{pgfscope}%
\begin{pgfscope}%
\pgfsys@transformshift{1.828491in}{1.931951in}%
\pgfsys@useobject{currentmarker}{}%
\end{pgfscope}%
\begin{pgfscope}%
\pgfsys@transformshift{1.902753in}{1.987345in}%
\pgfsys@useobject{currentmarker}{}%
\end{pgfscope}%
\begin{pgfscope}%
\pgfsys@transformshift{1.989053in}{2.061425in}%
\pgfsys@useobject{currentmarker}{}%
\end{pgfscope}%
\begin{pgfscope}%
\pgfsys@transformshift{2.045418in}{2.089036in}%
\pgfsys@useobject{currentmarker}{}%
\end{pgfscope}%
\begin{pgfscope}%
\pgfsys@transformshift{2.101804in}{2.116678in}%
\pgfsys@useobject{currentmarker}{}%
\end{pgfscope}%
\begin{pgfscope}%
\pgfsys@transformshift{2.115398in}{2.077888in}%
\pgfsys@useobject{currentmarker}{}%
\end{pgfscope}%
\begin{pgfscope}%
\pgfsys@transformshift{2.157435in}{2.083254in}%
\pgfsys@useobject{currentmarker}{}%
\end{pgfscope}%
\begin{pgfscope}%
\pgfsys@transformshift{2.162229in}{2.030803in}%
\pgfsys@useobject{currentmarker}{}%
\end{pgfscope}%
\begin{pgfscope}%
\pgfsys@transformshift{2.178795in}{1.996627in}%
\pgfsys@useobject{currentmarker}{}%
\end{pgfscope}%
\begin{pgfscope}%
\pgfsys@transformshift{2.190577in}{1.955024in}%
\pgfsys@useobject{currentmarker}{}%
\end{pgfscope}%
\begin{pgfscope}%
\pgfsys@transformshift{2.210306in}{1.925759in}%
\pgfsys@useobject{currentmarker}{}%
\end{pgfscope}%
\begin{pgfscope}%
\pgfsys@transformshift{2.230408in}{1.897072in}%
\pgfsys@useobject{currentmarker}{}%
\end{pgfscope}%
\begin{pgfscope}%
\pgfsys@transformshift{2.261554in}{1.885531in}%
\pgfsys@useobject{currentmarker}{}%
\end{pgfscope}%
\begin{pgfscope}%
\pgfsys@transformshift{2.282599in}{1.858308in}%
\pgfsys@useobject{currentmarker}{}%
\end{pgfscope}%
\begin{pgfscope}%
\pgfsys@transformshift{2.311609in}{1.843452in}%
\pgfsys@useobject{currentmarker}{}%
\end{pgfscope}%
\begin{pgfscope}%
\pgfsys@transformshift{2.360121in}{1.858871in}%
\pgfsys@useobject{currentmarker}{}%
\end{pgfscope}%
\begin{pgfscope}%
\pgfsys@transformshift{2.409930in}{1.876302in}%
\pgfsys@useobject{currentmarker}{}%
\end{pgfscope}%
\begin{pgfscope}%
\pgfsys@transformshift{2.475675in}{1.918474in}%
\pgfsys@useobject{currentmarker}{}%
\end{pgfscope}%
\begin{pgfscope}%
\pgfsys@transformshift{2.551377in}{1.976101in}%
\pgfsys@useobject{currentmarker}{}%
\end{pgfscope}%
\begin{pgfscope}%
\pgfsys@transformshift{2.632372in}{2.041948in}%
\pgfsys@useobject{currentmarker}{}%
\end{pgfscope}%
\begin{pgfscope}%
\pgfsys@transformshift{2.694363in}{2.078291in}%
\pgfsys@useobject{currentmarker}{}%
\end{pgfscope}%
\begin{pgfscope}%
\pgfsys@transformshift{2.745485in}{2.097761in}%
\pgfsys@useobject{currentmarker}{}%
\end{pgfscope}%
\begin{pgfscope}%
\pgfsys@transformshift{2.762412in}{2.064146in}%
\pgfsys@useobject{currentmarker}{}%
\end{pgfscope}%
\begin{pgfscope}%
\pgfsys@transformshift{2.787529in}{2.043244in}%
\pgfsys@useobject{currentmarker}{}%
\end{pgfscope}%
\begin{pgfscope}%
\pgfsys@transformshift{2.777791in}{1.968233in}%
\pgfsys@useobject{currentmarker}{}%
\end{pgfscope}%
\begin{pgfscope}%
\pgfsys@transformshift{2.788868in}{1.925536in}%
\pgfsys@useobject{currentmarker}{}%
\end{pgfscope}%
\begin{pgfscope}%
\pgfsys@transformshift{2.856282in}{1.850512in}%
\pgfsys@useobject{currentmarker}{}%
\end{pgfscope}%
\begin{pgfscope}%
\pgfsys@transformshift{2.870607in}{1.812857in}%
\pgfsys@useobject{currentmarker}{}%
\end{pgfscope}%
\begin{pgfscope}%
\pgfsys@transformshift{2.897552in}{1.794794in}%
\pgfsys@useobject{currentmarker}{}%
\end{pgfscope}%
\begin{pgfscope}%
\pgfsys@transformshift{2.926742in}{1.780216in}%
\pgfsys@useobject{currentmarker}{}%
\end{pgfscope}%
\begin{pgfscope}%
\pgfsys@transformshift{3.036531in}{1.830870in}%
\pgfsys@useobject{currentmarker}{}%
\end{pgfscope}%
\begin{pgfscope}%
\pgfsys@transformshift{3.326910in}{2.042090in}%
\pgfsys@useobject{currentmarker}{}%
\end{pgfscope}%
\begin{pgfscope}%
\pgfsys@transformshift{3.367472in}{2.045166in}%
\pgfsys@useobject{currentmarker}{}%
\end{pgfscope}%
\begin{pgfscope}%
\pgfsys@transformshift{3.396939in}{2.031018in}%
\pgfsys@useobject{currentmarker}{}%
\end{pgfscope}%
\begin{pgfscope}%
\pgfsys@transformshift{3.415993in}{2.000704in}%
\pgfsys@useobject{currentmarker}{}%
\end{pgfscope}%
\begin{pgfscope}%
\pgfsys@transformshift{3.427718in}{1.959014in}%
\pgfsys@useobject{currentmarker}{}%
\end{pgfscope}%
\begin{pgfscope}%
\pgfsys@transformshift{3.450369in}{1.934284in}%
\pgfsys@useobject{currentmarker}{}%
\end{pgfscope}%
\begin{pgfscope}%
\pgfsys@transformshift{3.478306in}{1.917762in}%
\pgfsys@useobject{currentmarker}{}%
\end{pgfscope}%
\begin{pgfscope}%
\pgfsys@transformshift{3.477156in}{1.856083in}%
\pgfsys@useobject{currentmarker}{}%
\end{pgfscope}%
\begin{pgfscope}%
\pgfsys@transformshift{3.509263in}{1.846033in}%
\pgfsys@useobject{currentmarker}{}%
\end{pgfscope}%
\begin{pgfscope}%
\pgfsys@transformshift{3.529555in}{1.817642in}%
\pgfsys@useobject{currentmarker}{}%
\end{pgfscope}%
\begin{pgfscope}%
\pgfsys@transformshift{3.565162in}{1.813026in}%
\pgfsys@useobject{currentmarker}{}%
\end{pgfscope}%
\begin{pgfscope}%
\pgfsys@transformshift{3.597068in}{1.802664in}%
\pgfsys@useobject{currentmarker}{}%
\end{pgfscope}%
\begin{pgfscope}%
\pgfsys@transformshift{3.641696in}{1.812053in}%
\pgfsys@useobject{currentmarker}{}%
\end{pgfscope}%
\begin{pgfscope}%
\pgfsys@transformshift{3.718669in}{1.871655in}%
\pgfsys@useobject{currentmarker}{}%
\end{pgfscope}%
\begin{pgfscope}%
\pgfsys@transformshift{3.775425in}{1.899872in}%
\pgfsys@useobject{currentmarker}{}%
\end{pgfscope}%
\begin{pgfscope}%
\pgfsys@transformshift{3.852269in}{1.959273in}%
\pgfsys@useobject{currentmarker}{}%
\end{pgfscope}%
\begin{pgfscope}%
\pgfsys@transformshift{3.927500in}{2.016170in}%
\pgfsys@useobject{currentmarker}{}%
\end{pgfscope}%
\begin{pgfscope}%
\pgfsys@transformshift{3.985669in}{2.046581in}%
\pgfsys@useobject{currentmarker}{}%
\end{pgfscope}%
\begin{pgfscope}%
\pgfsys@transformshift{4.026778in}{2.050506in}%
\pgfsys@useobject{currentmarker}{}%
\end{pgfscope}%
\begin{pgfscope}%
\pgfsys@transformshift{4.053468in}{2.032048in}%
\pgfsys@useobject{currentmarker}{}%
\end{pgfscope}%
\begin{pgfscope}%
\pgfsys@transformshift{4.068392in}{1.995322in}%
\pgfsys@useobject{currentmarker}{}%
\end{pgfscope}%
\begin{pgfscope}%
\pgfsys@transformshift{4.075955in}{1.947171in}%
\pgfsys@useobject{currentmarker}{}%
\end{pgfscope}%
\begin{pgfscope}%
\pgfsys@transformshift{4.096830in}{1.919684in}%
\pgfsys@useobject{currentmarker}{}%
\end{pgfscope}%
\begin{pgfscope}%
\pgfsys@transformshift{4.121635in}{1.898298in}%
\pgfsys@useobject{currentmarker}{}%
\end{pgfscope}%
\end{pgfscope}%
\begin{pgfscope}%
\pgfpathrectangle{\pgfqpoint{0.552773in}{0.431673in}}{\pgfqpoint{3.738807in}{1.765244in}}%
\pgfusepath{clip}%
\pgfsetbuttcap%
\pgfsetroundjoin%
\pgfsetlinewidth{1.003750pt}%
\definecolor{currentstroke}{rgb}{0.705673,0.015556,0.150233}%
\pgfsetstrokecolor{currentstroke}%
\pgfsetstrokeopacity{0.500000}%
\pgfsetdash{{3.700000pt}{1.600000pt}}{0.000000pt}%
\pgfpathmoveto{\pgfqpoint{0.552773in}{1.938654in}}%
\pgfpathlineto{\pgfqpoint{4.291580in}{1.938654in}}%
\pgfusepath{stroke}%
\end{pgfscope}%
\begin{pgfscope}%
\pgfpathrectangle{\pgfqpoint{0.552773in}{0.431673in}}{\pgfqpoint{3.738807in}{1.765244in}}%
\pgfusepath{clip}%
\pgfsetbuttcap%
\pgfsetroundjoin%
\pgfsetlinewidth{1.003750pt}%
\definecolor{currentstroke}{rgb}{0.667253,0.779176,0.992959}%
\pgfsetstrokecolor{currentstroke}%
\pgfsetdash{{3.700000pt}{1.600000pt}}{0.000000pt}%
\pgfpathmoveto{\pgfqpoint{0.946281in}{0.917169in}}%
\pgfpathlineto{\pgfqpoint{0.968004in}{0.896826in}}%
\pgfpathlineto{\pgfqpoint{0.983367in}{0.864904in}}%
\pgfpathlineto{\pgfqpoint{0.997288in}{0.830357in}}%
\pgfpathlineto{\pgfqpoint{1.003998in}{0.782680in}}%
\pgfpathlineto{\pgfqpoint{1.002620in}{0.720278in}}%
\pgfpathlineto{\pgfqpoint{1.016428in}{0.685524in}}%
\pgfpathlineto{\pgfqpoint{1.016366in}{0.625519in}}%
\pgfpathlineto{\pgfqpoint{1.044197in}{0.616297in}}%
\pgfpathlineto{\pgfqpoint{1.061881in}{0.588600in}}%
\pgfpathlineto{\pgfqpoint{1.100011in}{0.598129in}}%
\pgfpathlineto{\pgfqpoint{1.133830in}{0.599811in}}%
\pgfpathlineto{\pgfqpoint{1.178595in}{0.621420in}}%
\pgfpathlineto{\pgfqpoint{1.241991in}{0.676951in}}%
\pgfpathlineto{\pgfqpoint{1.316482in}{0.752683in}}%
\pgfpathlineto{\pgfqpoint{1.386975in}{0.821134in}}%
\pgfpathlineto{\pgfqpoint{1.458963in}{0.892310in}}%
\pgfpathlineto{\pgfqpoint{1.515751in}{0.935810in}}%
\pgfpathlineto{\pgfqpoint{1.552230in}{0.942332in}}%
\pgfpathlineto{\pgfqpoint{1.571226in}{0.917025in}}%
\pgfpathlineto{\pgfqpoint{1.578686in}{0.870715in}}%
\pgfpathlineto{\pgfqpoint{1.598839in}{0.847513in}}%
\pgfpathlineto{\pgfqpoint{1.615264in}{0.817524in}}%
\pgfpathlineto{\pgfqpoint{1.629986in}{0.784436in}}%
\pgfpathlineto{\pgfqpoint{1.633661in}{0.731234in}}%
\pgfpathlineto{\pgfqpoint{1.644849in}{0.691711in}}%
\pgfpathlineto{\pgfqpoint{1.654597in}{0.649565in}}%
\pgfpathlineto{\pgfqpoint{1.665591in}{0.609687in}}%
\pgfpathlineto{\pgfqpoint{1.686427in}{0.587731in}}%
\pgfpathlineto{\pgfqpoint{1.711171in}{0.572888in}}%
\pgfpathlineto{\pgfqpoint{1.758001in}{0.598259in}}%
\pgfpathlineto{\pgfqpoint{1.816028in}{0.644014in}}%
\pgfpathlineto{\pgfqpoint{1.879012in}{0.698794in}}%
\pgfpathlineto{\pgfqpoint{1.944926in}{0.758909in}}%
\pgfpathlineto{\pgfqpoint{2.017564in}{0.831267in}}%
\pgfpathlineto{\pgfqpoint{2.077394in}{0.880307in}}%
\pgfpathlineto{\pgfqpoint{2.120452in}{0.898808in}}%
\pgfpathlineto{\pgfqpoint{2.148843in}{0.890606in}}%
\pgfpathlineto{\pgfqpoint{2.169787in}{0.868845in}}%
\pgfpathlineto{\pgfqpoint{2.191999in}{0.849394in}}%
\pgfpathlineto{\pgfqpoint{2.201330in}{0.806489in}}%
\pgfpathlineto{\pgfqpoint{2.202600in}{0.748907in}}%
\pgfpathlineto{\pgfqpoint{2.202588in}{0.688992in}}%
\pgfpathlineto{\pgfqpoint{2.206526in}{0.636268in}}%
\pgfpathlineto{\pgfqpoint{2.205818in}{0.575086in}}%
\pgfpathlineto{\pgfqpoint{2.225473in}{0.550978in}}%
\pgfpathlineto{\pgfqpoint{2.247923in}{0.531960in}}%
\pgfpathlineto{\pgfqpoint{2.269807in}{0.511911in}}%
\pgfpathlineto{\pgfqpoint{2.310165in}{0.525497in}}%
\pgfpathlineto{\pgfqpoint{2.369172in}{0.573036in}}%
\pgfpathlineto{\pgfqpoint{2.433038in}{0.629423in}}%
\pgfpathlineto{\pgfqpoint{2.512551in}{0.714298in}}%
\pgfpathlineto{\pgfqpoint{2.584448in}{0.785307in}}%
\pgfpathlineto{\pgfqpoint{2.654396in}{0.852767in}}%
\pgfpathlineto{\pgfqpoint{2.709975in}{0.894066in}}%
\pgfpathlineto{\pgfqpoint{2.745485in}{0.898825in}}%
\pgfpathlineto{\pgfqpoint{2.765939in}{0.876173in}}%
\pgfpathlineto{\pgfqpoint{2.777619in}{0.837544in}}%
\pgfpathlineto{\pgfqpoint{2.773219in}{0.769640in}}%
\pgfpathlineto{\pgfqpoint{2.773093in}{0.709516in}}%
\pgfpathlineto{\pgfqpoint{2.775290in}{0.653624in}}%
\pgfpathlineto{\pgfqpoint{2.792438in}{0.624951in}}%
\pgfpathlineto{\pgfqpoint{2.801363in}{0.581308in}}%
\pgfpathlineto{\pgfqpoint{2.816359in}{0.548717in}}%
\pgfpathlineto{\pgfqpoint{2.831119in}{0.515697in}}%
\pgfpathlineto{\pgfqpoint{2.870334in}{0.527203in}}%
\pgfpathlineto{\pgfqpoint{2.914301in}{0.547359in}}%
\pgfpathlineto{\pgfqpoint{2.968424in}{0.586007in}}%
\pgfpathlineto{\pgfqpoint{3.031090in}{0.640210in}}%
\pgfpathlineto{\pgfqpoint{3.108456in}{0.721175in}}%
\pgfpathlineto{\pgfqpoint{3.183879in}{0.798603in}}%
\pgfpathlineto{\pgfqpoint{3.274791in}{0.904233in}}%
\pgfpathlineto{\pgfqpoint{3.351275in}{0.983593in}}%
\pgfpathlineto{\pgfqpoint{3.402422in}{1.016823in}}%
\pgfpathlineto{\pgfqpoint{3.426281in}{1.000369in}}%
\pgfpathlineto{\pgfqpoint{3.431078in}{0.949209in}}%
\pgfpathlineto{\pgfqpoint{3.439962in}{0.905491in}}%
\pgfpathlineto{\pgfqpoint{3.446892in}{0.858216in}}%
\pgfpathlineto{\pgfqpoint{3.456265in}{0.815386in}}%
\pgfpathlineto{\pgfqpoint{3.455917in}{0.754860in}}%
\pgfpathlineto{\pgfqpoint{3.471332in}{0.723033in}}%
\pgfpathlineto{\pgfqpoint{3.492791in}{0.702209in}}%
\pgfpathlineto{\pgfqpoint{3.581754in}{0.684504in}}%
\pgfpathlineto{\pgfqpoint{3.619118in}{0.692639in}}%
\pgfpathlineto{\pgfqpoint{3.681794in}{0.746858in}}%
\pgfusepath{stroke}%
\end{pgfscope}%
\begin{pgfscope}%
\pgfpathrectangle{\pgfqpoint{0.552773in}{0.431673in}}{\pgfqpoint{3.738807in}{1.765244in}}%
\pgfusepath{clip}%
\pgfsetbuttcap%
\pgfsetroundjoin%
\definecolor{currentfill}{rgb}{0.667253,0.779176,0.992959}%
\pgfsetfillcolor{currentfill}%
\pgfsetlinewidth{1.003750pt}%
\definecolor{currentstroke}{rgb}{0.667253,0.779176,0.992959}%
\pgfsetstrokecolor{currentstroke}%
\pgfsetdash{}{0pt}%
\pgfsys@defobject{currentmarker}{\pgfqpoint{-0.020833in}{-0.020833in}}{\pgfqpoint{0.020833in}{0.020833in}}{%
\pgfpathmoveto{\pgfqpoint{0.000000in}{-0.020833in}}%
\pgfpathcurveto{\pgfqpoint{0.005525in}{-0.020833in}}{\pgfqpoint{0.010825in}{-0.018638in}}{\pgfqpoint{0.014731in}{-0.014731in}}%
\pgfpathcurveto{\pgfqpoint{0.018638in}{-0.010825in}}{\pgfqpoint{0.020833in}{-0.005525in}}{\pgfqpoint{0.020833in}{0.000000in}}%
\pgfpathcurveto{\pgfqpoint{0.020833in}{0.005525in}}{\pgfqpoint{0.018638in}{0.010825in}}{\pgfqpoint{0.014731in}{0.014731in}}%
\pgfpathcurveto{\pgfqpoint{0.010825in}{0.018638in}}{\pgfqpoint{0.005525in}{0.020833in}}{\pgfqpoint{0.000000in}{0.020833in}}%
\pgfpathcurveto{\pgfqpoint{-0.005525in}{0.020833in}}{\pgfqpoint{-0.010825in}{0.018638in}}{\pgfqpoint{-0.014731in}{0.014731in}}%
\pgfpathcurveto{\pgfqpoint{-0.018638in}{0.010825in}}{\pgfqpoint{-0.020833in}{0.005525in}}{\pgfqpoint{-0.020833in}{0.000000in}}%
\pgfpathcurveto{\pgfqpoint{-0.020833in}{-0.005525in}}{\pgfqpoint{-0.018638in}{-0.010825in}}{\pgfqpoint{-0.014731in}{-0.014731in}}%
\pgfpathcurveto{\pgfqpoint{-0.010825in}{-0.018638in}}{\pgfqpoint{-0.005525in}{-0.020833in}}{\pgfqpoint{0.000000in}{-0.020833in}}%
\pgfpathlineto{\pgfqpoint{0.000000in}{-0.020833in}}%
\pgfpathclose%
\pgfusepath{stroke,fill}%
}%
\begin{pgfscope}%
\pgfsys@transformshift{0.946281in}{0.917169in}%
\pgfsys@useobject{currentmarker}{}%
\end{pgfscope}%
\begin{pgfscope}%
\pgfsys@transformshift{0.968004in}{0.896826in}%
\pgfsys@useobject{currentmarker}{}%
\end{pgfscope}%
\begin{pgfscope}%
\pgfsys@transformshift{0.983367in}{0.864904in}%
\pgfsys@useobject{currentmarker}{}%
\end{pgfscope}%
\begin{pgfscope}%
\pgfsys@transformshift{0.997288in}{0.830357in}%
\pgfsys@useobject{currentmarker}{}%
\end{pgfscope}%
\begin{pgfscope}%
\pgfsys@transformshift{1.003998in}{0.782680in}%
\pgfsys@useobject{currentmarker}{}%
\end{pgfscope}%
\begin{pgfscope}%
\pgfsys@transformshift{1.002620in}{0.720278in}%
\pgfsys@useobject{currentmarker}{}%
\end{pgfscope}%
\begin{pgfscope}%
\pgfsys@transformshift{1.016428in}{0.685524in}%
\pgfsys@useobject{currentmarker}{}%
\end{pgfscope}%
\begin{pgfscope}%
\pgfsys@transformshift{1.016366in}{0.625519in}%
\pgfsys@useobject{currentmarker}{}%
\end{pgfscope}%
\begin{pgfscope}%
\pgfsys@transformshift{1.044197in}{0.616297in}%
\pgfsys@useobject{currentmarker}{}%
\end{pgfscope}%
\begin{pgfscope}%
\pgfsys@transformshift{1.061881in}{0.588600in}%
\pgfsys@useobject{currentmarker}{}%
\end{pgfscope}%
\begin{pgfscope}%
\pgfsys@transformshift{1.100011in}{0.598129in}%
\pgfsys@useobject{currentmarker}{}%
\end{pgfscope}%
\begin{pgfscope}%
\pgfsys@transformshift{1.133830in}{0.599811in}%
\pgfsys@useobject{currentmarker}{}%
\end{pgfscope}%
\begin{pgfscope}%
\pgfsys@transformshift{1.178595in}{0.621420in}%
\pgfsys@useobject{currentmarker}{}%
\end{pgfscope}%
\begin{pgfscope}%
\pgfsys@transformshift{1.241991in}{0.676951in}%
\pgfsys@useobject{currentmarker}{}%
\end{pgfscope}%
\begin{pgfscope}%
\pgfsys@transformshift{1.316482in}{0.752683in}%
\pgfsys@useobject{currentmarker}{}%
\end{pgfscope}%
\begin{pgfscope}%
\pgfsys@transformshift{1.386975in}{0.821134in}%
\pgfsys@useobject{currentmarker}{}%
\end{pgfscope}%
\begin{pgfscope}%
\pgfsys@transformshift{1.458963in}{0.892310in}%
\pgfsys@useobject{currentmarker}{}%
\end{pgfscope}%
\begin{pgfscope}%
\pgfsys@transformshift{1.515751in}{0.935810in}%
\pgfsys@useobject{currentmarker}{}%
\end{pgfscope}%
\begin{pgfscope}%
\pgfsys@transformshift{1.552230in}{0.942332in}%
\pgfsys@useobject{currentmarker}{}%
\end{pgfscope}%
\begin{pgfscope}%
\pgfsys@transformshift{1.571226in}{0.917025in}%
\pgfsys@useobject{currentmarker}{}%
\end{pgfscope}%
\begin{pgfscope}%
\pgfsys@transformshift{1.578686in}{0.870715in}%
\pgfsys@useobject{currentmarker}{}%
\end{pgfscope}%
\begin{pgfscope}%
\pgfsys@transformshift{1.598839in}{0.847513in}%
\pgfsys@useobject{currentmarker}{}%
\end{pgfscope}%
\begin{pgfscope}%
\pgfsys@transformshift{1.615264in}{0.817524in}%
\pgfsys@useobject{currentmarker}{}%
\end{pgfscope}%
\begin{pgfscope}%
\pgfsys@transformshift{1.629986in}{0.784436in}%
\pgfsys@useobject{currentmarker}{}%
\end{pgfscope}%
\begin{pgfscope}%
\pgfsys@transformshift{1.633661in}{0.731234in}%
\pgfsys@useobject{currentmarker}{}%
\end{pgfscope}%
\begin{pgfscope}%
\pgfsys@transformshift{1.644849in}{0.691711in}%
\pgfsys@useobject{currentmarker}{}%
\end{pgfscope}%
\begin{pgfscope}%
\pgfsys@transformshift{1.654597in}{0.649565in}%
\pgfsys@useobject{currentmarker}{}%
\end{pgfscope}%
\begin{pgfscope}%
\pgfsys@transformshift{1.665591in}{0.609687in}%
\pgfsys@useobject{currentmarker}{}%
\end{pgfscope}%
\begin{pgfscope}%
\pgfsys@transformshift{1.686427in}{0.587731in}%
\pgfsys@useobject{currentmarker}{}%
\end{pgfscope}%
\begin{pgfscope}%
\pgfsys@transformshift{1.711171in}{0.572888in}%
\pgfsys@useobject{currentmarker}{}%
\end{pgfscope}%
\begin{pgfscope}%
\pgfsys@transformshift{1.758001in}{0.598259in}%
\pgfsys@useobject{currentmarker}{}%
\end{pgfscope}%
\begin{pgfscope}%
\pgfsys@transformshift{1.816028in}{0.644014in}%
\pgfsys@useobject{currentmarker}{}%
\end{pgfscope}%
\begin{pgfscope}%
\pgfsys@transformshift{1.879012in}{0.698794in}%
\pgfsys@useobject{currentmarker}{}%
\end{pgfscope}%
\begin{pgfscope}%
\pgfsys@transformshift{1.944926in}{0.758909in}%
\pgfsys@useobject{currentmarker}{}%
\end{pgfscope}%
\begin{pgfscope}%
\pgfsys@transformshift{2.017564in}{0.831267in}%
\pgfsys@useobject{currentmarker}{}%
\end{pgfscope}%
\begin{pgfscope}%
\pgfsys@transformshift{2.077394in}{0.880307in}%
\pgfsys@useobject{currentmarker}{}%
\end{pgfscope}%
\begin{pgfscope}%
\pgfsys@transformshift{2.120452in}{0.898808in}%
\pgfsys@useobject{currentmarker}{}%
\end{pgfscope}%
\begin{pgfscope}%
\pgfsys@transformshift{2.148843in}{0.890606in}%
\pgfsys@useobject{currentmarker}{}%
\end{pgfscope}%
\begin{pgfscope}%
\pgfsys@transformshift{2.169787in}{0.868845in}%
\pgfsys@useobject{currentmarker}{}%
\end{pgfscope}%
\begin{pgfscope}%
\pgfsys@transformshift{2.191999in}{0.849394in}%
\pgfsys@useobject{currentmarker}{}%
\end{pgfscope}%
\begin{pgfscope}%
\pgfsys@transformshift{2.201330in}{0.806489in}%
\pgfsys@useobject{currentmarker}{}%
\end{pgfscope}%
\begin{pgfscope}%
\pgfsys@transformshift{2.202600in}{0.748907in}%
\pgfsys@useobject{currentmarker}{}%
\end{pgfscope}%
\begin{pgfscope}%
\pgfsys@transformshift{2.202588in}{0.688992in}%
\pgfsys@useobject{currentmarker}{}%
\end{pgfscope}%
\begin{pgfscope}%
\pgfsys@transformshift{2.206526in}{0.636268in}%
\pgfsys@useobject{currentmarker}{}%
\end{pgfscope}%
\begin{pgfscope}%
\pgfsys@transformshift{2.205818in}{0.575086in}%
\pgfsys@useobject{currentmarker}{}%
\end{pgfscope}%
\begin{pgfscope}%
\pgfsys@transformshift{2.225473in}{0.550978in}%
\pgfsys@useobject{currentmarker}{}%
\end{pgfscope}%
\begin{pgfscope}%
\pgfsys@transformshift{2.247923in}{0.531960in}%
\pgfsys@useobject{currentmarker}{}%
\end{pgfscope}%
\begin{pgfscope}%
\pgfsys@transformshift{2.269807in}{0.511911in}%
\pgfsys@useobject{currentmarker}{}%
\end{pgfscope}%
\begin{pgfscope}%
\pgfsys@transformshift{2.310165in}{0.525497in}%
\pgfsys@useobject{currentmarker}{}%
\end{pgfscope}%
\begin{pgfscope}%
\pgfsys@transformshift{2.369172in}{0.573036in}%
\pgfsys@useobject{currentmarker}{}%
\end{pgfscope}%
\begin{pgfscope}%
\pgfsys@transformshift{2.433038in}{0.629423in}%
\pgfsys@useobject{currentmarker}{}%
\end{pgfscope}%
\begin{pgfscope}%
\pgfsys@transformshift{2.512551in}{0.714298in}%
\pgfsys@useobject{currentmarker}{}%
\end{pgfscope}%
\begin{pgfscope}%
\pgfsys@transformshift{2.584448in}{0.785307in}%
\pgfsys@useobject{currentmarker}{}%
\end{pgfscope}%
\begin{pgfscope}%
\pgfsys@transformshift{2.654396in}{0.852767in}%
\pgfsys@useobject{currentmarker}{}%
\end{pgfscope}%
\begin{pgfscope}%
\pgfsys@transformshift{2.709975in}{0.894066in}%
\pgfsys@useobject{currentmarker}{}%
\end{pgfscope}%
\begin{pgfscope}%
\pgfsys@transformshift{2.745485in}{0.898825in}%
\pgfsys@useobject{currentmarker}{}%
\end{pgfscope}%
\begin{pgfscope}%
\pgfsys@transformshift{2.765939in}{0.876173in}%
\pgfsys@useobject{currentmarker}{}%
\end{pgfscope}%
\begin{pgfscope}%
\pgfsys@transformshift{2.777619in}{0.837544in}%
\pgfsys@useobject{currentmarker}{}%
\end{pgfscope}%
\begin{pgfscope}%
\pgfsys@transformshift{2.773219in}{0.769640in}%
\pgfsys@useobject{currentmarker}{}%
\end{pgfscope}%
\begin{pgfscope}%
\pgfsys@transformshift{2.773093in}{0.709516in}%
\pgfsys@useobject{currentmarker}{}%
\end{pgfscope}%
\begin{pgfscope}%
\pgfsys@transformshift{2.775290in}{0.653624in}%
\pgfsys@useobject{currentmarker}{}%
\end{pgfscope}%
\begin{pgfscope}%
\pgfsys@transformshift{2.792438in}{0.624951in}%
\pgfsys@useobject{currentmarker}{}%
\end{pgfscope}%
\begin{pgfscope}%
\pgfsys@transformshift{2.801363in}{0.581308in}%
\pgfsys@useobject{currentmarker}{}%
\end{pgfscope}%
\begin{pgfscope}%
\pgfsys@transformshift{2.816359in}{0.548717in}%
\pgfsys@useobject{currentmarker}{}%
\end{pgfscope}%
\begin{pgfscope}%
\pgfsys@transformshift{2.831119in}{0.515697in}%
\pgfsys@useobject{currentmarker}{}%
\end{pgfscope}%
\begin{pgfscope}%
\pgfsys@transformshift{2.870334in}{0.527203in}%
\pgfsys@useobject{currentmarker}{}%
\end{pgfscope}%
\begin{pgfscope}%
\pgfsys@transformshift{2.914301in}{0.547359in}%
\pgfsys@useobject{currentmarker}{}%
\end{pgfscope}%
\begin{pgfscope}%
\pgfsys@transformshift{2.968424in}{0.586007in}%
\pgfsys@useobject{currentmarker}{}%
\end{pgfscope}%
\begin{pgfscope}%
\pgfsys@transformshift{3.031090in}{0.640210in}%
\pgfsys@useobject{currentmarker}{}%
\end{pgfscope}%
\begin{pgfscope}%
\pgfsys@transformshift{3.108456in}{0.721175in}%
\pgfsys@useobject{currentmarker}{}%
\end{pgfscope}%
\begin{pgfscope}%
\pgfsys@transformshift{3.183879in}{0.798603in}%
\pgfsys@useobject{currentmarker}{}%
\end{pgfscope}%
\begin{pgfscope}%
\pgfsys@transformshift{3.274791in}{0.904233in}%
\pgfsys@useobject{currentmarker}{}%
\end{pgfscope}%
\begin{pgfscope}%
\pgfsys@transformshift{3.351275in}{0.983593in}%
\pgfsys@useobject{currentmarker}{}%
\end{pgfscope}%
\begin{pgfscope}%
\pgfsys@transformshift{3.402422in}{1.016823in}%
\pgfsys@useobject{currentmarker}{}%
\end{pgfscope}%
\begin{pgfscope}%
\pgfsys@transformshift{3.426281in}{1.000369in}%
\pgfsys@useobject{currentmarker}{}%
\end{pgfscope}%
\begin{pgfscope}%
\pgfsys@transformshift{3.431078in}{0.949209in}%
\pgfsys@useobject{currentmarker}{}%
\end{pgfscope}%
\begin{pgfscope}%
\pgfsys@transformshift{3.439962in}{0.905491in}%
\pgfsys@useobject{currentmarker}{}%
\end{pgfscope}%
\begin{pgfscope}%
\pgfsys@transformshift{3.446892in}{0.858216in}%
\pgfsys@useobject{currentmarker}{}%
\end{pgfscope}%
\begin{pgfscope}%
\pgfsys@transformshift{3.456265in}{0.815386in}%
\pgfsys@useobject{currentmarker}{}%
\end{pgfscope}%
\begin{pgfscope}%
\pgfsys@transformshift{3.455917in}{0.754860in}%
\pgfsys@useobject{currentmarker}{}%
\end{pgfscope}%
\begin{pgfscope}%
\pgfsys@transformshift{3.471332in}{0.723033in}%
\pgfsys@useobject{currentmarker}{}%
\end{pgfscope}%
\begin{pgfscope}%
\pgfsys@transformshift{3.492791in}{0.702209in}%
\pgfsys@useobject{currentmarker}{}%
\end{pgfscope}%
\begin{pgfscope}%
\pgfsys@transformshift{3.581754in}{0.684504in}%
\pgfsys@useobject{currentmarker}{}%
\end{pgfscope}%
\begin{pgfscope}%
\pgfsys@transformshift{3.619118in}{0.692639in}%
\pgfsys@useobject{currentmarker}{}%
\end{pgfscope}%
\begin{pgfscope}%
\pgfsys@transformshift{3.681794in}{0.746858in}%
\pgfsys@useobject{currentmarker}{}%
\end{pgfscope}%
\end{pgfscope}%
\begin{pgfscope}%
\pgfpathrectangle{\pgfqpoint{0.552773in}{0.431673in}}{\pgfqpoint{3.738807in}{1.765244in}}%
\pgfusepath{clip}%
\pgfsetbuttcap%
\pgfsetroundjoin%
\pgfsetlinewidth{1.003750pt}%
\definecolor{currentstroke}{rgb}{0.968203,0.720844,0.612293}%
\pgfsetstrokecolor{currentstroke}%
\pgfsetdash{{3.700000pt}{1.600000pt}}{0.000000pt}%
\pgfpathmoveto{\pgfqpoint{1.234882in}{1.346212in}}%
\pgfpathlineto{\pgfqpoint{1.295052in}{1.380370in}}%
\pgfpathlineto{\pgfqpoint{1.333712in}{1.380908in}}%
\pgfpathlineto{\pgfqpoint{1.360563in}{1.362986in}}%
\pgfpathlineto{\pgfqpoint{1.364476in}{1.309209in}}%
\pgfpathlineto{\pgfqpoint{1.369782in}{1.257608in}}%
\pgfpathlineto{\pgfqpoint{1.424566in}{1.283350in}}%
\pgfpathlineto{\pgfqpoint{1.443716in}{1.253390in}}%
\pgfpathlineto{\pgfqpoint{1.466969in}{1.229844in}}%
\pgfpathlineto{\pgfqpoint{1.490856in}{1.207288in}}%
\pgfpathlineto{\pgfqpoint{1.503289in}{1.166829in}}%
\pgfpathlineto{\pgfqpoint{1.551438in}{1.182198in}}%
\pgfpathlineto{\pgfqpoint{1.607593in}{1.210082in}}%
\pgfpathlineto{\pgfqpoint{1.658864in}{1.230331in}}%
\pgfpathlineto{\pgfqpoint{1.722476in}{1.269870in}}%
\pgfpathlineto{\pgfqpoint{1.797820in}{1.327748in}}%
\pgfpathlineto{\pgfqpoint{1.877516in}{1.392430in}}%
\pgfpathlineto{\pgfqpoint{1.954391in}{1.452701in}}%
\pgfpathlineto{\pgfqpoint{2.014562in}{1.486861in}}%
\pgfpathlineto{\pgfqpoint{2.061371in}{1.500136in}}%
\pgfpathlineto{\pgfqpoint{2.083864in}{1.475402in}}%
\pgfpathlineto{\pgfqpoint{2.105964in}{1.450053in}}%
\pgfpathlineto{\pgfqpoint{2.116436in}{1.406528in}}%
\pgfpathlineto{\pgfqpoint{2.141673in}{1.386083in}}%
\pgfpathlineto{\pgfqpoint{2.156766in}{1.349782in}}%
\pgfpathlineto{\pgfqpoint{2.171473in}{1.312877in}}%
\pgfpathlineto{\pgfqpoint{2.192191in}{1.285369in}}%
\pgfpathlineto{\pgfqpoint{2.223452in}{1.274341in}}%
\pgfpathlineto{\pgfqpoint{2.256793in}{1.266562in}}%
\pgfpathlineto{\pgfqpoint{2.298519in}{1.271893in}}%
\pgfpathlineto{\pgfqpoint{2.360130in}{1.308304in}}%
\pgfpathlineto{\pgfqpoint{2.402956in}{1.315352in}}%
\pgfpathlineto{\pgfqpoint{2.474900in}{1.367916in}}%
\pgfpathlineto{\pgfqpoint{2.549428in}{1.424519in}}%
\pgfpathlineto{\pgfqpoint{2.619369in}{1.473950in}}%
\pgfpathlineto{\pgfqpoint{2.688119in}{1.521522in}}%
\pgfpathlineto{\pgfqpoint{2.745485in}{1.551298in}}%
\pgfpathlineto{\pgfqpoint{2.770849in}{1.531052in}}%
\pgfpathlineto{\pgfqpoint{2.791502in}{1.503442in}}%
\pgfpathlineto{\pgfqpoint{2.801394in}{1.459011in}}%
\pgfpathlineto{\pgfqpoint{2.825889in}{1.437406in}}%
\pgfpathlineto{\pgfqpoint{2.823590in}{1.373918in}}%
\pgfpathlineto{\pgfqpoint{2.848428in}{1.352850in}}%
\pgfpathlineto{\pgfqpoint{2.859935in}{1.310943in}}%
\pgfpathlineto{\pgfqpoint{2.877851in}{1.279055in}}%
\pgfpathlineto{\pgfqpoint{2.897792in}{1.250331in}}%
\pgfpathlineto{\pgfqpoint{2.906245in}{1.203652in}}%
\pgfpathlineto{\pgfqpoint{2.946444in}{1.206593in}}%
\pgfpathlineto{\pgfqpoint{3.013245in}{1.251118in}}%
\pgfpathlineto{\pgfqpoint{3.071632in}{1.282490in}}%
\pgfpathlineto{\pgfqpoint{3.138985in}{1.327877in}}%
\pgfpathlineto{\pgfqpoint{3.210642in}{1.379991in}}%
\pgfpathlineto{\pgfqpoint{3.288346in}{1.441559in}}%
\pgfpathlineto{\pgfqpoint{3.354079in}{1.484414in}}%
\pgfpathlineto{\pgfqpoint{3.395162in}{1.488738in}}%
\pgfpathlineto{\pgfqpoint{3.423246in}{1.472743in}}%
\pgfpathlineto{\pgfqpoint{3.450046in}{1.454742in}}%
\pgfpathlineto{\pgfqpoint{3.453543in}{1.400314in}}%
\pgfpathlineto{\pgfqpoint{3.455571in}{1.343591in}}%
\pgfpathlineto{\pgfqpoint{3.477372in}{1.317776in}}%
\pgfpathlineto{\pgfqpoint{3.481371in}{1.264133in}}%
\pgfpathlineto{\pgfqpoint{3.505373in}{1.241757in}}%
\pgfpathlineto{\pgfqpoint{3.526998in}{1.215666in}}%
\pgfpathlineto{\pgfqpoint{3.562203in}{1.210802in}}%
\pgfpathlineto{\pgfqpoint{3.586856in}{1.189444in}}%
\pgfpathlineto{\pgfqpoint{3.629765in}{1.196622in}}%
\pgfpathlineto{\pgfqpoint{3.676730in}{1.210141in}}%
\pgfpathlineto{\pgfqpoint{3.745250in}{1.257352in}}%
\pgfpathlineto{\pgfqpoint{3.816743in}{1.309211in}}%
\pgfpathlineto{\pgfqpoint{3.871459in}{1.334845in}}%
\pgfpathlineto{\pgfqpoint{3.950363in}{1.398288in}}%
\pgfpathlineto{\pgfqpoint{3.993573in}{1.405936in}}%
\pgfusepath{stroke}%
\end{pgfscope}%
\begin{pgfscope}%
\pgfpathrectangle{\pgfqpoint{0.552773in}{0.431673in}}{\pgfqpoint{3.738807in}{1.765244in}}%
\pgfusepath{clip}%
\pgfsetbuttcap%
\pgfsetroundjoin%
\definecolor{currentfill}{rgb}{0.968203,0.720844,0.612293}%
\pgfsetfillcolor{currentfill}%
\pgfsetlinewidth{1.003750pt}%
\definecolor{currentstroke}{rgb}{0.968203,0.720844,0.612293}%
\pgfsetstrokecolor{currentstroke}%
\pgfsetdash{}{0pt}%
\pgfsys@defobject{currentmarker}{\pgfqpoint{-0.020833in}{-0.020833in}}{\pgfqpoint{0.020833in}{0.020833in}}{%
\pgfpathmoveto{\pgfqpoint{0.000000in}{-0.020833in}}%
\pgfpathcurveto{\pgfqpoint{0.005525in}{-0.020833in}}{\pgfqpoint{0.010825in}{-0.018638in}}{\pgfqpoint{0.014731in}{-0.014731in}}%
\pgfpathcurveto{\pgfqpoint{0.018638in}{-0.010825in}}{\pgfqpoint{0.020833in}{-0.005525in}}{\pgfqpoint{0.020833in}{0.000000in}}%
\pgfpathcurveto{\pgfqpoint{0.020833in}{0.005525in}}{\pgfqpoint{0.018638in}{0.010825in}}{\pgfqpoint{0.014731in}{0.014731in}}%
\pgfpathcurveto{\pgfqpoint{0.010825in}{0.018638in}}{\pgfqpoint{0.005525in}{0.020833in}}{\pgfqpoint{0.000000in}{0.020833in}}%
\pgfpathcurveto{\pgfqpoint{-0.005525in}{0.020833in}}{\pgfqpoint{-0.010825in}{0.018638in}}{\pgfqpoint{-0.014731in}{0.014731in}}%
\pgfpathcurveto{\pgfqpoint{-0.018638in}{0.010825in}}{\pgfqpoint{-0.020833in}{0.005525in}}{\pgfqpoint{-0.020833in}{0.000000in}}%
\pgfpathcurveto{\pgfqpoint{-0.020833in}{-0.005525in}}{\pgfqpoint{-0.018638in}{-0.010825in}}{\pgfqpoint{-0.014731in}{-0.014731in}}%
\pgfpathcurveto{\pgfqpoint{-0.010825in}{-0.018638in}}{\pgfqpoint{-0.005525in}{-0.020833in}}{\pgfqpoint{0.000000in}{-0.020833in}}%
\pgfpathlineto{\pgfqpoint{0.000000in}{-0.020833in}}%
\pgfpathclose%
\pgfusepath{stroke,fill}%
}%
\begin{pgfscope}%
\pgfsys@transformshift{1.234882in}{1.346212in}%
\pgfsys@useobject{currentmarker}{}%
\end{pgfscope}%
\begin{pgfscope}%
\pgfsys@transformshift{1.295052in}{1.380370in}%
\pgfsys@useobject{currentmarker}{}%
\end{pgfscope}%
\begin{pgfscope}%
\pgfsys@transformshift{1.333712in}{1.380908in}%
\pgfsys@useobject{currentmarker}{}%
\end{pgfscope}%
\begin{pgfscope}%
\pgfsys@transformshift{1.360563in}{1.362986in}%
\pgfsys@useobject{currentmarker}{}%
\end{pgfscope}%
\begin{pgfscope}%
\pgfsys@transformshift{1.364476in}{1.309209in}%
\pgfsys@useobject{currentmarker}{}%
\end{pgfscope}%
\begin{pgfscope}%
\pgfsys@transformshift{1.369782in}{1.257608in}%
\pgfsys@useobject{currentmarker}{}%
\end{pgfscope}%
\begin{pgfscope}%
\pgfsys@transformshift{1.424566in}{1.283350in}%
\pgfsys@useobject{currentmarker}{}%
\end{pgfscope}%
\begin{pgfscope}%
\pgfsys@transformshift{1.443716in}{1.253390in}%
\pgfsys@useobject{currentmarker}{}%
\end{pgfscope}%
\begin{pgfscope}%
\pgfsys@transformshift{1.466969in}{1.229844in}%
\pgfsys@useobject{currentmarker}{}%
\end{pgfscope}%
\begin{pgfscope}%
\pgfsys@transformshift{1.490856in}{1.207288in}%
\pgfsys@useobject{currentmarker}{}%
\end{pgfscope}%
\begin{pgfscope}%
\pgfsys@transformshift{1.503289in}{1.166829in}%
\pgfsys@useobject{currentmarker}{}%
\end{pgfscope}%
\begin{pgfscope}%
\pgfsys@transformshift{1.551438in}{1.182198in}%
\pgfsys@useobject{currentmarker}{}%
\end{pgfscope}%
\begin{pgfscope}%
\pgfsys@transformshift{1.607593in}{1.210082in}%
\pgfsys@useobject{currentmarker}{}%
\end{pgfscope}%
\begin{pgfscope}%
\pgfsys@transformshift{1.658864in}{1.230331in}%
\pgfsys@useobject{currentmarker}{}%
\end{pgfscope}%
\begin{pgfscope}%
\pgfsys@transformshift{1.722476in}{1.269870in}%
\pgfsys@useobject{currentmarker}{}%
\end{pgfscope}%
\begin{pgfscope}%
\pgfsys@transformshift{1.797820in}{1.327748in}%
\pgfsys@useobject{currentmarker}{}%
\end{pgfscope}%
\begin{pgfscope}%
\pgfsys@transformshift{1.877516in}{1.392430in}%
\pgfsys@useobject{currentmarker}{}%
\end{pgfscope}%
\begin{pgfscope}%
\pgfsys@transformshift{1.954391in}{1.452701in}%
\pgfsys@useobject{currentmarker}{}%
\end{pgfscope}%
\begin{pgfscope}%
\pgfsys@transformshift{2.014562in}{1.486861in}%
\pgfsys@useobject{currentmarker}{}%
\end{pgfscope}%
\begin{pgfscope}%
\pgfsys@transformshift{2.061371in}{1.500136in}%
\pgfsys@useobject{currentmarker}{}%
\end{pgfscope}%
\begin{pgfscope}%
\pgfsys@transformshift{2.083864in}{1.475402in}%
\pgfsys@useobject{currentmarker}{}%
\end{pgfscope}%
\begin{pgfscope}%
\pgfsys@transformshift{2.105964in}{1.450053in}%
\pgfsys@useobject{currentmarker}{}%
\end{pgfscope}%
\begin{pgfscope}%
\pgfsys@transformshift{2.116436in}{1.406528in}%
\pgfsys@useobject{currentmarker}{}%
\end{pgfscope}%
\begin{pgfscope}%
\pgfsys@transformshift{2.141673in}{1.386083in}%
\pgfsys@useobject{currentmarker}{}%
\end{pgfscope}%
\begin{pgfscope}%
\pgfsys@transformshift{2.156766in}{1.349782in}%
\pgfsys@useobject{currentmarker}{}%
\end{pgfscope}%
\begin{pgfscope}%
\pgfsys@transformshift{2.171473in}{1.312877in}%
\pgfsys@useobject{currentmarker}{}%
\end{pgfscope}%
\begin{pgfscope}%
\pgfsys@transformshift{2.192191in}{1.285369in}%
\pgfsys@useobject{currentmarker}{}%
\end{pgfscope}%
\begin{pgfscope}%
\pgfsys@transformshift{2.223452in}{1.274341in}%
\pgfsys@useobject{currentmarker}{}%
\end{pgfscope}%
\begin{pgfscope}%
\pgfsys@transformshift{2.256793in}{1.266562in}%
\pgfsys@useobject{currentmarker}{}%
\end{pgfscope}%
\begin{pgfscope}%
\pgfsys@transformshift{2.298519in}{1.271893in}%
\pgfsys@useobject{currentmarker}{}%
\end{pgfscope}%
\begin{pgfscope}%
\pgfsys@transformshift{2.360130in}{1.308304in}%
\pgfsys@useobject{currentmarker}{}%
\end{pgfscope}%
\begin{pgfscope}%
\pgfsys@transformshift{2.402956in}{1.315352in}%
\pgfsys@useobject{currentmarker}{}%
\end{pgfscope}%
\begin{pgfscope}%
\pgfsys@transformshift{2.474900in}{1.367916in}%
\pgfsys@useobject{currentmarker}{}%
\end{pgfscope}%
\begin{pgfscope}%
\pgfsys@transformshift{2.549428in}{1.424519in}%
\pgfsys@useobject{currentmarker}{}%
\end{pgfscope}%
\begin{pgfscope}%
\pgfsys@transformshift{2.619369in}{1.473950in}%
\pgfsys@useobject{currentmarker}{}%
\end{pgfscope}%
\begin{pgfscope}%
\pgfsys@transformshift{2.688119in}{1.521522in}%
\pgfsys@useobject{currentmarker}{}%
\end{pgfscope}%
\begin{pgfscope}%
\pgfsys@transformshift{2.745485in}{1.551298in}%
\pgfsys@useobject{currentmarker}{}%
\end{pgfscope}%
\begin{pgfscope}%
\pgfsys@transformshift{2.770849in}{1.531052in}%
\pgfsys@useobject{currentmarker}{}%
\end{pgfscope}%
\begin{pgfscope}%
\pgfsys@transformshift{2.791502in}{1.503442in}%
\pgfsys@useobject{currentmarker}{}%
\end{pgfscope}%
\begin{pgfscope}%
\pgfsys@transformshift{2.801394in}{1.459011in}%
\pgfsys@useobject{currentmarker}{}%
\end{pgfscope}%
\begin{pgfscope}%
\pgfsys@transformshift{2.825889in}{1.437406in}%
\pgfsys@useobject{currentmarker}{}%
\end{pgfscope}%
\begin{pgfscope}%
\pgfsys@transformshift{2.823590in}{1.373918in}%
\pgfsys@useobject{currentmarker}{}%
\end{pgfscope}%
\begin{pgfscope}%
\pgfsys@transformshift{2.848428in}{1.352850in}%
\pgfsys@useobject{currentmarker}{}%
\end{pgfscope}%
\begin{pgfscope}%
\pgfsys@transformshift{2.859935in}{1.310943in}%
\pgfsys@useobject{currentmarker}{}%
\end{pgfscope}%
\begin{pgfscope}%
\pgfsys@transformshift{2.877851in}{1.279055in}%
\pgfsys@useobject{currentmarker}{}%
\end{pgfscope}%
\begin{pgfscope}%
\pgfsys@transformshift{2.897792in}{1.250331in}%
\pgfsys@useobject{currentmarker}{}%
\end{pgfscope}%
\begin{pgfscope}%
\pgfsys@transformshift{2.906245in}{1.203652in}%
\pgfsys@useobject{currentmarker}{}%
\end{pgfscope}%
\begin{pgfscope}%
\pgfsys@transformshift{2.946444in}{1.206593in}%
\pgfsys@useobject{currentmarker}{}%
\end{pgfscope}%
\begin{pgfscope}%
\pgfsys@transformshift{3.013245in}{1.251118in}%
\pgfsys@useobject{currentmarker}{}%
\end{pgfscope}%
\begin{pgfscope}%
\pgfsys@transformshift{3.071632in}{1.282490in}%
\pgfsys@useobject{currentmarker}{}%
\end{pgfscope}%
\begin{pgfscope}%
\pgfsys@transformshift{3.138985in}{1.327877in}%
\pgfsys@useobject{currentmarker}{}%
\end{pgfscope}%
\begin{pgfscope}%
\pgfsys@transformshift{3.210642in}{1.379991in}%
\pgfsys@useobject{currentmarker}{}%
\end{pgfscope}%
\begin{pgfscope}%
\pgfsys@transformshift{3.288346in}{1.441559in}%
\pgfsys@useobject{currentmarker}{}%
\end{pgfscope}%
\begin{pgfscope}%
\pgfsys@transformshift{3.354079in}{1.484414in}%
\pgfsys@useobject{currentmarker}{}%
\end{pgfscope}%
\begin{pgfscope}%
\pgfsys@transformshift{3.395162in}{1.488738in}%
\pgfsys@useobject{currentmarker}{}%
\end{pgfscope}%
\begin{pgfscope}%
\pgfsys@transformshift{3.423246in}{1.472743in}%
\pgfsys@useobject{currentmarker}{}%
\end{pgfscope}%
\begin{pgfscope}%
\pgfsys@transformshift{3.450046in}{1.454742in}%
\pgfsys@useobject{currentmarker}{}%
\end{pgfscope}%
\begin{pgfscope}%
\pgfsys@transformshift{3.453543in}{1.400314in}%
\pgfsys@useobject{currentmarker}{}%
\end{pgfscope}%
\begin{pgfscope}%
\pgfsys@transformshift{3.455571in}{1.343591in}%
\pgfsys@useobject{currentmarker}{}%
\end{pgfscope}%
\begin{pgfscope}%
\pgfsys@transformshift{3.477372in}{1.317776in}%
\pgfsys@useobject{currentmarker}{}%
\end{pgfscope}%
\begin{pgfscope}%
\pgfsys@transformshift{3.481371in}{1.264133in}%
\pgfsys@useobject{currentmarker}{}%
\end{pgfscope}%
\begin{pgfscope}%
\pgfsys@transformshift{3.505373in}{1.241757in}%
\pgfsys@useobject{currentmarker}{}%
\end{pgfscope}%
\begin{pgfscope}%
\pgfsys@transformshift{3.526998in}{1.215666in}%
\pgfsys@useobject{currentmarker}{}%
\end{pgfscope}%
\begin{pgfscope}%
\pgfsys@transformshift{3.562203in}{1.210802in}%
\pgfsys@useobject{currentmarker}{}%
\end{pgfscope}%
\begin{pgfscope}%
\pgfsys@transformshift{3.586856in}{1.189444in}%
\pgfsys@useobject{currentmarker}{}%
\end{pgfscope}%
\begin{pgfscope}%
\pgfsys@transformshift{3.629765in}{1.196622in}%
\pgfsys@useobject{currentmarker}{}%
\end{pgfscope}%
\begin{pgfscope}%
\pgfsys@transformshift{3.676730in}{1.210141in}%
\pgfsys@useobject{currentmarker}{}%
\end{pgfscope}%
\begin{pgfscope}%
\pgfsys@transformshift{3.745250in}{1.257352in}%
\pgfsys@useobject{currentmarker}{}%
\end{pgfscope}%
\begin{pgfscope}%
\pgfsys@transformshift{3.816743in}{1.309211in}%
\pgfsys@useobject{currentmarker}{}%
\end{pgfscope}%
\begin{pgfscope}%
\pgfsys@transformshift{3.871459in}{1.334845in}%
\pgfsys@useobject{currentmarker}{}%
\end{pgfscope}%
\begin{pgfscope}%
\pgfsys@transformshift{3.950363in}{1.398288in}%
\pgfsys@useobject{currentmarker}{}%
\end{pgfscope}%
\begin{pgfscope}%
\pgfsys@transformshift{3.993573in}{1.405936in}%
\pgfsys@useobject{currentmarker}{}%
\end{pgfscope}%
\end{pgfscope}%
\begin{pgfscope}%
\pgfpathrectangle{\pgfqpoint{0.552773in}{0.431673in}}{\pgfqpoint{3.738807in}{1.765244in}}%
\pgfusepath{clip}%
\pgfsetbuttcap%
\pgfsetroundjoin%
\pgfsetlinewidth{1.003750pt}%
\definecolor{currentstroke}{rgb}{0.705673,0.015556,0.150233}%
\pgfsetstrokecolor{currentstroke}%
\pgfsetdash{{3.700000pt}{1.600000pt}}{0.000000pt}%
\pgfpathmoveto{\pgfqpoint{0.722719in}{2.012117in}}%
\pgfpathlineto{\pgfqpoint{0.786936in}{1.992022in}}%
\pgfpathlineto{\pgfqpoint{0.819797in}{1.983144in}}%
\pgfpathlineto{\pgfqpoint{0.831560in}{1.941512in}}%
\pgfpathlineto{\pgfqpoint{0.837884in}{1.891436in}}%
\pgfpathlineto{\pgfqpoint{0.853441in}{1.855694in}}%
\pgfpathlineto{\pgfqpoint{0.871877in}{1.824422in}}%
\pgfpathlineto{\pgfqpoint{0.894231in}{1.799231in}}%
\pgfpathlineto{\pgfqpoint{0.917097in}{1.774835in}}%
\pgfpathlineto{\pgfqpoint{0.954263in}{1.772639in}}%
\pgfpathlineto{\pgfqpoint{1.054856in}{1.809017in}}%
\pgfpathlineto{\pgfqpoint{1.127913in}{1.862539in}}%
\pgfpathlineto{\pgfqpoint{1.216245in}{1.939776in}}%
\pgfpathlineto{\pgfqpoint{1.299586in}{2.009263in}}%
\pgfpathlineto{\pgfqpoint{1.383835in}{2.080160in}}%
\pgfpathlineto{\pgfqpoint{1.428876in}{2.090189in}}%
\pgfpathlineto{\pgfqpoint{1.461920in}{2.081596in}}%
\pgfpathlineto{\pgfqpoint{1.470951in}{2.035721in}}%
\pgfpathlineto{\pgfqpoint{1.500704in}{2.022017in}}%
\pgfpathlineto{\pgfqpoint{1.498835in}{1.959223in}}%
\pgfpathlineto{\pgfqpoint{1.510085in}{1.916795in}}%
\pgfpathlineto{\pgfqpoint{1.536220in}{1.897473in}}%
\pgfpathlineto{\pgfqpoint{1.565209in}{1.882584in}}%
\pgfpathlineto{\pgfqpoint{1.587354in}{1.857070in}}%
\pgfpathlineto{\pgfqpoint{1.624003in}{1.854071in}}%
\pgfpathlineto{\pgfqpoint{1.656435in}{1.844526in}}%
\pgfpathlineto{\pgfqpoint{1.704526in}{1.859291in}}%
\pgfpathlineto{\pgfqpoint{1.761295in}{1.887527in}}%
\pgfpathlineto{\pgfqpoint{1.828491in}{1.931951in}}%
\pgfpathlineto{\pgfqpoint{1.902753in}{1.987345in}}%
\pgfpathlineto{\pgfqpoint{1.989053in}{2.061425in}}%
\pgfpathlineto{\pgfqpoint{2.045418in}{2.089036in}}%
\pgfpathlineto{\pgfqpoint{2.101804in}{2.116678in}}%
\pgfpathlineto{\pgfqpoint{2.115398in}{2.077888in}}%
\pgfpathlineto{\pgfqpoint{2.157435in}{2.083254in}}%
\pgfpathlineto{\pgfqpoint{2.162229in}{2.030803in}}%
\pgfpathlineto{\pgfqpoint{2.178795in}{1.996627in}}%
\pgfpathlineto{\pgfqpoint{2.190577in}{1.955024in}}%
\pgfpathlineto{\pgfqpoint{2.210306in}{1.925759in}}%
\pgfpathlineto{\pgfqpoint{2.230408in}{1.897072in}}%
\pgfpathlineto{\pgfqpoint{2.261554in}{1.885531in}}%
\pgfpathlineto{\pgfqpoint{2.282599in}{1.858308in}}%
\pgfpathlineto{\pgfqpoint{2.311609in}{1.843452in}}%
\pgfpathlineto{\pgfqpoint{2.360121in}{1.858871in}}%
\pgfpathlineto{\pgfqpoint{2.409930in}{1.876302in}}%
\pgfpathlineto{\pgfqpoint{2.475675in}{1.918474in}}%
\pgfpathlineto{\pgfqpoint{2.551377in}{1.976101in}}%
\pgfpathlineto{\pgfqpoint{2.632372in}{2.041948in}}%
\pgfpathlineto{\pgfqpoint{2.694363in}{2.078291in}}%
\pgfpathlineto{\pgfqpoint{2.745485in}{2.097761in}}%
\pgfpathlineto{\pgfqpoint{2.762412in}{2.064146in}}%
\pgfpathlineto{\pgfqpoint{2.787529in}{2.043244in}}%
\pgfpathlineto{\pgfqpoint{2.777791in}{1.968233in}}%
\pgfpathlineto{\pgfqpoint{2.788868in}{1.925536in}}%
\pgfpathlineto{\pgfqpoint{2.856282in}{1.850512in}}%
\pgfpathlineto{\pgfqpoint{2.870607in}{1.812857in}}%
\pgfpathlineto{\pgfqpoint{2.897552in}{1.794794in}}%
\pgfpathlineto{\pgfqpoint{2.926742in}{1.780216in}}%
\pgfpathlineto{\pgfqpoint{3.036531in}{1.830870in}}%
\pgfpathlineto{\pgfqpoint{3.326910in}{2.042090in}}%
\pgfpathlineto{\pgfqpoint{3.367472in}{2.045166in}}%
\pgfpathlineto{\pgfqpoint{3.396939in}{2.031018in}}%
\pgfpathlineto{\pgfqpoint{3.415993in}{2.000704in}}%
\pgfpathlineto{\pgfqpoint{3.427718in}{1.959014in}}%
\pgfpathlineto{\pgfqpoint{3.450369in}{1.934284in}}%
\pgfpathlineto{\pgfqpoint{3.478306in}{1.917762in}}%
\pgfpathlineto{\pgfqpoint{3.477156in}{1.856083in}}%
\pgfpathlineto{\pgfqpoint{3.509263in}{1.846033in}}%
\pgfpathlineto{\pgfqpoint{3.529555in}{1.817642in}}%
\pgfpathlineto{\pgfqpoint{3.565162in}{1.813026in}}%
\pgfpathlineto{\pgfqpoint{3.597068in}{1.802664in}}%
\pgfpathlineto{\pgfqpoint{3.641696in}{1.812053in}}%
\pgfpathlineto{\pgfqpoint{3.718669in}{1.871655in}}%
\pgfpathlineto{\pgfqpoint{3.775425in}{1.899872in}}%
\pgfpathlineto{\pgfqpoint{3.852269in}{1.959273in}}%
\pgfpathlineto{\pgfqpoint{3.927500in}{2.016170in}}%
\pgfpathlineto{\pgfqpoint{3.985669in}{2.046581in}}%
\pgfpathlineto{\pgfqpoint{4.026778in}{2.050506in}}%
\pgfpathlineto{\pgfqpoint{4.053468in}{2.032048in}}%
\pgfpathlineto{\pgfqpoint{4.068392in}{1.995322in}}%
\pgfpathlineto{\pgfqpoint{4.075955in}{1.947171in}}%
\pgfpathlineto{\pgfqpoint{4.096830in}{1.919684in}}%
\pgfpathlineto{\pgfqpoint{4.121635in}{1.898298in}}%
\pgfusepath{stroke}%
\end{pgfscope}%
\begin{pgfscope}%
\pgfpathrectangle{\pgfqpoint{0.552773in}{0.431673in}}{\pgfqpoint{3.738807in}{1.765244in}}%
\pgfusepath{clip}%
\pgfsetbuttcap%
\pgfsetroundjoin%
\definecolor{currentfill}{rgb}{0.705673,0.015556,0.150233}%
\pgfsetfillcolor{currentfill}%
\pgfsetlinewidth{1.003750pt}%
\definecolor{currentstroke}{rgb}{0.705673,0.015556,0.150233}%
\pgfsetstrokecolor{currentstroke}%
\pgfsetdash{}{0pt}%
\pgfsys@defobject{currentmarker}{\pgfqpoint{-0.020833in}{-0.020833in}}{\pgfqpoint{0.020833in}{0.020833in}}{%
\pgfpathmoveto{\pgfqpoint{0.000000in}{-0.020833in}}%
\pgfpathcurveto{\pgfqpoint{0.005525in}{-0.020833in}}{\pgfqpoint{0.010825in}{-0.018638in}}{\pgfqpoint{0.014731in}{-0.014731in}}%
\pgfpathcurveto{\pgfqpoint{0.018638in}{-0.010825in}}{\pgfqpoint{0.020833in}{-0.005525in}}{\pgfqpoint{0.020833in}{0.000000in}}%
\pgfpathcurveto{\pgfqpoint{0.020833in}{0.005525in}}{\pgfqpoint{0.018638in}{0.010825in}}{\pgfqpoint{0.014731in}{0.014731in}}%
\pgfpathcurveto{\pgfqpoint{0.010825in}{0.018638in}}{\pgfqpoint{0.005525in}{0.020833in}}{\pgfqpoint{0.000000in}{0.020833in}}%
\pgfpathcurveto{\pgfqpoint{-0.005525in}{0.020833in}}{\pgfqpoint{-0.010825in}{0.018638in}}{\pgfqpoint{-0.014731in}{0.014731in}}%
\pgfpathcurveto{\pgfqpoint{-0.018638in}{0.010825in}}{\pgfqpoint{-0.020833in}{0.005525in}}{\pgfqpoint{-0.020833in}{0.000000in}}%
\pgfpathcurveto{\pgfqpoint{-0.020833in}{-0.005525in}}{\pgfqpoint{-0.018638in}{-0.010825in}}{\pgfqpoint{-0.014731in}{-0.014731in}}%
\pgfpathcurveto{\pgfqpoint{-0.010825in}{-0.018638in}}{\pgfqpoint{-0.005525in}{-0.020833in}}{\pgfqpoint{0.000000in}{-0.020833in}}%
\pgfpathlineto{\pgfqpoint{0.000000in}{-0.020833in}}%
\pgfpathclose%
\pgfusepath{stroke,fill}%
}%
\begin{pgfscope}%
\pgfsys@transformshift{0.722719in}{2.012117in}%
\pgfsys@useobject{currentmarker}{}%
\end{pgfscope}%
\begin{pgfscope}%
\pgfsys@transformshift{0.786936in}{1.992022in}%
\pgfsys@useobject{currentmarker}{}%
\end{pgfscope}%
\begin{pgfscope}%
\pgfsys@transformshift{0.819797in}{1.983144in}%
\pgfsys@useobject{currentmarker}{}%
\end{pgfscope}%
\begin{pgfscope}%
\pgfsys@transformshift{0.831560in}{1.941512in}%
\pgfsys@useobject{currentmarker}{}%
\end{pgfscope}%
\begin{pgfscope}%
\pgfsys@transformshift{0.837884in}{1.891436in}%
\pgfsys@useobject{currentmarker}{}%
\end{pgfscope}%
\begin{pgfscope}%
\pgfsys@transformshift{0.853441in}{1.855694in}%
\pgfsys@useobject{currentmarker}{}%
\end{pgfscope}%
\begin{pgfscope}%
\pgfsys@transformshift{0.871877in}{1.824422in}%
\pgfsys@useobject{currentmarker}{}%
\end{pgfscope}%
\begin{pgfscope}%
\pgfsys@transformshift{0.894231in}{1.799231in}%
\pgfsys@useobject{currentmarker}{}%
\end{pgfscope}%
\begin{pgfscope}%
\pgfsys@transformshift{0.917097in}{1.774835in}%
\pgfsys@useobject{currentmarker}{}%
\end{pgfscope}%
\begin{pgfscope}%
\pgfsys@transformshift{0.954263in}{1.772639in}%
\pgfsys@useobject{currentmarker}{}%
\end{pgfscope}%
\begin{pgfscope}%
\pgfsys@transformshift{1.054856in}{1.809017in}%
\pgfsys@useobject{currentmarker}{}%
\end{pgfscope}%
\begin{pgfscope}%
\pgfsys@transformshift{1.127913in}{1.862539in}%
\pgfsys@useobject{currentmarker}{}%
\end{pgfscope}%
\begin{pgfscope}%
\pgfsys@transformshift{1.216245in}{1.939776in}%
\pgfsys@useobject{currentmarker}{}%
\end{pgfscope}%
\begin{pgfscope}%
\pgfsys@transformshift{1.299586in}{2.009263in}%
\pgfsys@useobject{currentmarker}{}%
\end{pgfscope}%
\begin{pgfscope}%
\pgfsys@transformshift{1.383835in}{2.080160in}%
\pgfsys@useobject{currentmarker}{}%
\end{pgfscope}%
\begin{pgfscope}%
\pgfsys@transformshift{1.428876in}{2.090189in}%
\pgfsys@useobject{currentmarker}{}%
\end{pgfscope}%
\begin{pgfscope}%
\pgfsys@transformshift{1.461920in}{2.081596in}%
\pgfsys@useobject{currentmarker}{}%
\end{pgfscope}%
\begin{pgfscope}%
\pgfsys@transformshift{1.470951in}{2.035721in}%
\pgfsys@useobject{currentmarker}{}%
\end{pgfscope}%
\begin{pgfscope}%
\pgfsys@transformshift{1.500704in}{2.022017in}%
\pgfsys@useobject{currentmarker}{}%
\end{pgfscope}%
\begin{pgfscope}%
\pgfsys@transformshift{1.498835in}{1.959223in}%
\pgfsys@useobject{currentmarker}{}%
\end{pgfscope}%
\begin{pgfscope}%
\pgfsys@transformshift{1.510085in}{1.916795in}%
\pgfsys@useobject{currentmarker}{}%
\end{pgfscope}%
\begin{pgfscope}%
\pgfsys@transformshift{1.536220in}{1.897473in}%
\pgfsys@useobject{currentmarker}{}%
\end{pgfscope}%
\begin{pgfscope}%
\pgfsys@transformshift{1.565209in}{1.882584in}%
\pgfsys@useobject{currentmarker}{}%
\end{pgfscope}%
\begin{pgfscope}%
\pgfsys@transformshift{1.587354in}{1.857070in}%
\pgfsys@useobject{currentmarker}{}%
\end{pgfscope}%
\begin{pgfscope}%
\pgfsys@transformshift{1.624003in}{1.854071in}%
\pgfsys@useobject{currentmarker}{}%
\end{pgfscope}%
\begin{pgfscope}%
\pgfsys@transformshift{1.656435in}{1.844526in}%
\pgfsys@useobject{currentmarker}{}%
\end{pgfscope}%
\begin{pgfscope}%
\pgfsys@transformshift{1.704526in}{1.859291in}%
\pgfsys@useobject{currentmarker}{}%
\end{pgfscope}%
\begin{pgfscope}%
\pgfsys@transformshift{1.761295in}{1.887527in}%
\pgfsys@useobject{currentmarker}{}%
\end{pgfscope}%
\begin{pgfscope}%
\pgfsys@transformshift{1.828491in}{1.931951in}%
\pgfsys@useobject{currentmarker}{}%
\end{pgfscope}%
\begin{pgfscope}%
\pgfsys@transformshift{1.902753in}{1.987345in}%
\pgfsys@useobject{currentmarker}{}%
\end{pgfscope}%
\begin{pgfscope}%
\pgfsys@transformshift{1.989053in}{2.061425in}%
\pgfsys@useobject{currentmarker}{}%
\end{pgfscope}%
\begin{pgfscope}%
\pgfsys@transformshift{2.045418in}{2.089036in}%
\pgfsys@useobject{currentmarker}{}%
\end{pgfscope}%
\begin{pgfscope}%
\pgfsys@transformshift{2.101804in}{2.116678in}%
\pgfsys@useobject{currentmarker}{}%
\end{pgfscope}%
\begin{pgfscope}%
\pgfsys@transformshift{2.115398in}{2.077888in}%
\pgfsys@useobject{currentmarker}{}%
\end{pgfscope}%
\begin{pgfscope}%
\pgfsys@transformshift{2.157435in}{2.083254in}%
\pgfsys@useobject{currentmarker}{}%
\end{pgfscope}%
\begin{pgfscope}%
\pgfsys@transformshift{2.162229in}{2.030803in}%
\pgfsys@useobject{currentmarker}{}%
\end{pgfscope}%
\begin{pgfscope}%
\pgfsys@transformshift{2.178795in}{1.996627in}%
\pgfsys@useobject{currentmarker}{}%
\end{pgfscope}%
\begin{pgfscope}%
\pgfsys@transformshift{2.190577in}{1.955024in}%
\pgfsys@useobject{currentmarker}{}%
\end{pgfscope}%
\begin{pgfscope}%
\pgfsys@transformshift{2.210306in}{1.925759in}%
\pgfsys@useobject{currentmarker}{}%
\end{pgfscope}%
\begin{pgfscope}%
\pgfsys@transformshift{2.230408in}{1.897072in}%
\pgfsys@useobject{currentmarker}{}%
\end{pgfscope}%
\begin{pgfscope}%
\pgfsys@transformshift{2.261554in}{1.885531in}%
\pgfsys@useobject{currentmarker}{}%
\end{pgfscope}%
\begin{pgfscope}%
\pgfsys@transformshift{2.282599in}{1.858308in}%
\pgfsys@useobject{currentmarker}{}%
\end{pgfscope}%
\begin{pgfscope}%
\pgfsys@transformshift{2.311609in}{1.843452in}%
\pgfsys@useobject{currentmarker}{}%
\end{pgfscope}%
\begin{pgfscope}%
\pgfsys@transformshift{2.360121in}{1.858871in}%
\pgfsys@useobject{currentmarker}{}%
\end{pgfscope}%
\begin{pgfscope}%
\pgfsys@transformshift{2.409930in}{1.876302in}%
\pgfsys@useobject{currentmarker}{}%
\end{pgfscope}%
\begin{pgfscope}%
\pgfsys@transformshift{2.475675in}{1.918474in}%
\pgfsys@useobject{currentmarker}{}%
\end{pgfscope}%
\begin{pgfscope}%
\pgfsys@transformshift{2.551377in}{1.976101in}%
\pgfsys@useobject{currentmarker}{}%
\end{pgfscope}%
\begin{pgfscope}%
\pgfsys@transformshift{2.632372in}{2.041948in}%
\pgfsys@useobject{currentmarker}{}%
\end{pgfscope}%
\begin{pgfscope}%
\pgfsys@transformshift{2.694363in}{2.078291in}%
\pgfsys@useobject{currentmarker}{}%
\end{pgfscope}%
\begin{pgfscope}%
\pgfsys@transformshift{2.745485in}{2.097761in}%
\pgfsys@useobject{currentmarker}{}%
\end{pgfscope}%
\begin{pgfscope}%
\pgfsys@transformshift{2.762412in}{2.064146in}%
\pgfsys@useobject{currentmarker}{}%
\end{pgfscope}%
\begin{pgfscope}%
\pgfsys@transformshift{2.787529in}{2.043244in}%
\pgfsys@useobject{currentmarker}{}%
\end{pgfscope}%
\begin{pgfscope}%
\pgfsys@transformshift{2.777791in}{1.968233in}%
\pgfsys@useobject{currentmarker}{}%
\end{pgfscope}%
\begin{pgfscope}%
\pgfsys@transformshift{2.788868in}{1.925536in}%
\pgfsys@useobject{currentmarker}{}%
\end{pgfscope}%
\begin{pgfscope}%
\pgfsys@transformshift{2.856282in}{1.850512in}%
\pgfsys@useobject{currentmarker}{}%
\end{pgfscope}%
\begin{pgfscope}%
\pgfsys@transformshift{2.870607in}{1.812857in}%
\pgfsys@useobject{currentmarker}{}%
\end{pgfscope}%
\begin{pgfscope}%
\pgfsys@transformshift{2.897552in}{1.794794in}%
\pgfsys@useobject{currentmarker}{}%
\end{pgfscope}%
\begin{pgfscope}%
\pgfsys@transformshift{2.926742in}{1.780216in}%
\pgfsys@useobject{currentmarker}{}%
\end{pgfscope}%
\begin{pgfscope}%
\pgfsys@transformshift{3.036531in}{1.830870in}%
\pgfsys@useobject{currentmarker}{}%
\end{pgfscope}%
\begin{pgfscope}%
\pgfsys@transformshift{3.326910in}{2.042090in}%
\pgfsys@useobject{currentmarker}{}%
\end{pgfscope}%
\begin{pgfscope}%
\pgfsys@transformshift{3.367472in}{2.045166in}%
\pgfsys@useobject{currentmarker}{}%
\end{pgfscope}%
\begin{pgfscope}%
\pgfsys@transformshift{3.396939in}{2.031018in}%
\pgfsys@useobject{currentmarker}{}%
\end{pgfscope}%
\begin{pgfscope}%
\pgfsys@transformshift{3.415993in}{2.000704in}%
\pgfsys@useobject{currentmarker}{}%
\end{pgfscope}%
\begin{pgfscope}%
\pgfsys@transformshift{3.427718in}{1.959014in}%
\pgfsys@useobject{currentmarker}{}%
\end{pgfscope}%
\begin{pgfscope}%
\pgfsys@transformshift{3.450369in}{1.934284in}%
\pgfsys@useobject{currentmarker}{}%
\end{pgfscope}%
\begin{pgfscope}%
\pgfsys@transformshift{3.478306in}{1.917762in}%
\pgfsys@useobject{currentmarker}{}%
\end{pgfscope}%
\begin{pgfscope}%
\pgfsys@transformshift{3.477156in}{1.856083in}%
\pgfsys@useobject{currentmarker}{}%
\end{pgfscope}%
\begin{pgfscope}%
\pgfsys@transformshift{3.509263in}{1.846033in}%
\pgfsys@useobject{currentmarker}{}%
\end{pgfscope}%
\begin{pgfscope}%
\pgfsys@transformshift{3.529555in}{1.817642in}%
\pgfsys@useobject{currentmarker}{}%
\end{pgfscope}%
\begin{pgfscope}%
\pgfsys@transformshift{3.565162in}{1.813026in}%
\pgfsys@useobject{currentmarker}{}%
\end{pgfscope}%
\begin{pgfscope}%
\pgfsys@transformshift{3.597068in}{1.802664in}%
\pgfsys@useobject{currentmarker}{}%
\end{pgfscope}%
\begin{pgfscope}%
\pgfsys@transformshift{3.641696in}{1.812053in}%
\pgfsys@useobject{currentmarker}{}%
\end{pgfscope}%
\begin{pgfscope}%
\pgfsys@transformshift{3.718669in}{1.871655in}%
\pgfsys@useobject{currentmarker}{}%
\end{pgfscope}%
\begin{pgfscope}%
\pgfsys@transformshift{3.775425in}{1.899872in}%
\pgfsys@useobject{currentmarker}{}%
\end{pgfscope}%
\begin{pgfscope}%
\pgfsys@transformshift{3.852269in}{1.959273in}%
\pgfsys@useobject{currentmarker}{}%
\end{pgfscope}%
\begin{pgfscope}%
\pgfsys@transformshift{3.927500in}{2.016170in}%
\pgfsys@useobject{currentmarker}{}%
\end{pgfscope}%
\begin{pgfscope}%
\pgfsys@transformshift{3.985669in}{2.046581in}%
\pgfsys@useobject{currentmarker}{}%
\end{pgfscope}%
\begin{pgfscope}%
\pgfsys@transformshift{4.026778in}{2.050506in}%
\pgfsys@useobject{currentmarker}{}%
\end{pgfscope}%
\begin{pgfscope}%
\pgfsys@transformshift{4.053468in}{2.032048in}%
\pgfsys@useobject{currentmarker}{}%
\end{pgfscope}%
\begin{pgfscope}%
\pgfsys@transformshift{4.068392in}{1.995322in}%
\pgfsys@useobject{currentmarker}{}%
\end{pgfscope}%
\begin{pgfscope}%
\pgfsys@transformshift{4.075955in}{1.947171in}%
\pgfsys@useobject{currentmarker}{}%
\end{pgfscope}%
\begin{pgfscope}%
\pgfsys@transformshift{4.096830in}{1.919684in}%
\pgfsys@useobject{currentmarker}{}%
\end{pgfscope}%
\begin{pgfscope}%
\pgfsys@transformshift{4.121635in}{1.898298in}%
\pgfsys@useobject{currentmarker}{}%
\end{pgfscope}%
\end{pgfscope}%
\begin{pgfscope}%
\pgfsetrectcap%
\pgfsetmiterjoin%
\pgfsetlinewidth{0.501875pt}%
\definecolor{currentstroke}{rgb}{0.000000,0.000000,0.000000}%
\pgfsetstrokecolor{currentstroke}%
\pgfsetdash{}{0pt}%
\pgfpathmoveto{\pgfqpoint{0.552773in}{0.431673in}}%
\pgfpathlineto{\pgfqpoint{0.552773in}{2.196916in}}%
\pgfusepath{stroke}%
\end{pgfscope}%
\begin{pgfscope}%
\pgfsetrectcap%
\pgfsetmiterjoin%
\pgfsetlinewidth{0.501875pt}%
\definecolor{currentstroke}{rgb}{0.000000,0.000000,0.000000}%
\pgfsetstrokecolor{currentstroke}%
\pgfsetdash{}{0pt}%
\pgfpathmoveto{\pgfqpoint{4.291580in}{0.431673in}}%
\pgfpathlineto{\pgfqpoint{4.291580in}{2.196916in}}%
\pgfusepath{stroke}%
\end{pgfscope}%
\begin{pgfscope}%
\pgfsetrectcap%
\pgfsetmiterjoin%
\pgfsetlinewidth{0.501875pt}%
\definecolor{currentstroke}{rgb}{0.000000,0.000000,0.000000}%
\pgfsetstrokecolor{currentstroke}%
\pgfsetdash{}{0pt}%
\pgfpathmoveto{\pgfqpoint{0.552773in}{0.431673in}}%
\pgfpathlineto{\pgfqpoint{4.291580in}{0.431673in}}%
\pgfusepath{stroke}%
\end{pgfscope}%
\begin{pgfscope}%
\pgfsetrectcap%
\pgfsetmiterjoin%
\pgfsetlinewidth{0.501875pt}%
\definecolor{currentstroke}{rgb}{0.000000,0.000000,0.000000}%
\pgfsetstrokecolor{currentstroke}%
\pgfsetdash{}{0pt}%
\pgfpathmoveto{\pgfqpoint{0.552773in}{2.196916in}}%
\pgfpathlineto{\pgfqpoint{4.291580in}{2.196916in}}%
\pgfusepath{stroke}%
\end{pgfscope}%
\begin{pgfscope}%
\definecolor{textcolor}{rgb}{0.667253,0.779176,0.992959}%
\pgfsetstrokecolor{textcolor}%
\pgfsetfillcolor{textcolor}%
\pgftext[x=4.366357in,y=0.740787in,left,]{\color{textcolor}\rmfamily\fontsize{10.000000}{12.000000}\selectfont \qty{3.0}{\kelvin}}%
\end{pgfscope}%
\begin{pgfscope}%
\definecolor{textcolor}{rgb}{0.968203,0.720844,0.612293}%
\pgfsetstrokecolor{textcolor}%
\pgfsetfillcolor{textcolor}%
\pgftext[x=4.366357in,y=1.339721in,left,]{\color{textcolor}\rmfamily\fontsize{10.000000}{12.000000}\selectfont \qty{3.2}{\kelvin}}%
\end{pgfscope}%
\begin{pgfscope}%
\definecolor{textcolor}{rgb}{0.705673,0.015556,0.150233}%
\pgfsetstrokecolor{textcolor}%
\pgfsetfillcolor{textcolor}%
\pgftext[x=4.366357in,y=1.938654in,left,]{\color{textcolor}\rmfamily\fontsize{10.000000}{12.000000}\selectfont \qty{3.4}{\kelvin}}%
\end{pgfscope}%
\end{pgfpicture}%
\makeatother%
\endgroup%

	\caption{Various current-phase relations measured on sample CP2 in a range of \qtyrange{2.8}{3.6}{\kelvin}. The CPRs have been visually offset by \qty{150}{\micro\ampere}. The horizontal dashed lines denote zero current. The lines through the data points are for visual clarity.}
	\label{fig:CP2.6B_revisited_CPRs}
\end{figure}

Having to fit the mutual and loop inductance is one of the weaknesses of our method. In the linear regime of the dc-SQUID extracting the mutual inductance should be trivial. The loop inductance however is determined by the periodicity in the data. Whilst possible to compare the loop inductance to a simulated value, a factor two difference already changes a $2\pi$-periodic CPR to a $4\pi$-periodic current-phase relation. To overcome this, a reference loop without any junction could be added. Alternatively, since the loop inductance also in part determines the amplitude of the junctions critical current, it is possible to later cut the junction's loop and measure the critical current of just the junction.

Ignoring the measurement at \qty{3.6}{\kelvin}, the critical current goes from \qtyrange{90}{45}{\micro\ampere} between \qtyrange{2.8}{3.4}{\kelvin}. Figure~\ref{fig:SNS_junction_predictions} shows the CPR at different temperatures for both a short diffusive and short ballistic SNS junction. Qualitatively we see that our results best match the short ballistic CPR. In either case we see that the CPR becomes more sinusoidal and that the amplitudes decrease as the temperature increases.

\begin{figure}[ht!]
	\begin{subfigure}[t]{\textwidth}
		\centering
		\import{figures/simulations}{SNS_ballistic_junction_CPRs.pgf}
		\subcaption{Theoretical CPRs at different temperatures for a short ballistic SNS junction, see Equation~2.13 in \cite{vermeerSTMbasedScanningSQUID2021}.}
	\end{subfigure}
	\hfill
	\begin{subfigure}[t]{\textwidth}
		\centering
		\import{figures/simulations}{SNS_diffusive_junction_CPRs.pgf}
		\subcaption{Theoretical CPRs at different temperatures for a short diffusive SNS junction, see Equation~2.14 in \cite{vermeerSTMbasedScanningSQUID2021}.}
	\end{subfigure}

	\caption{Theoretical predictions for the CPR of both a short ballistic and short diffusive SNS junction. In both cases $\Delta$ has been estimated at \qty{1}{\milli\electronvolt}.}
	\label{fig:SNS_junction_predictions}
\end{figure}

It is odd that that the short ballistic model qualitatively matches the data better. Especially since a gaussian pattern was observed when doing higher field measurements on the dc-SQUID. A gaussian pattern is expected with diffusive behaviour. Since the junctions are very similar they all should have the same behaviour. Another possibility is that our junction is diffusive but that we are in a multi-valued regime.

Clearly the results show a few artefacts most notably at \qtylist{2.8;3.0;3.4}{\kelvin}. We attribute this to the fact that we are not in the linear regime of our dc-SQUID. We are confident that if the dc-SQUID would be biased in the linear regime that these artefacts would disappear. Additionally because we did not control our dc-SQUID's bias, it is not possible to say what point $\gamma = 0$. As such we have artificially centred the curves such that $\gamma = 0$ for $I_s=0$.