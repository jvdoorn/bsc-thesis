% !TEX root = ../../../thesis.tex
The sample had not been permanently stored in a desiccator. An important check was thus if the sample had degraded. The RT-curve of the dc-SQUID showed no significant changes. As such it is unlikely that the sample degraded significantly. It also confirmed that all the contacts are still good.

Previously the CPR was measured at \qty{3}{\kelvin}. An increase in temperature degrades the sensitivity of our dc-SQUID and the critical current of the junction loop. We note that our measurement was approximately \qty{200}{\micro\ampere} periodic. In order to seem multiple periods it is preferable to have a critical junction around \qty{400}{\micro\ampere}. This allows us to see around 4 to 5 periods. Through trial and error we found the highest usable temperature to be \qty{3.4}{\kelvin}.

Similarly to last time we then measured SQIs for several bias currents. We found the optimal bias current to be \qty{300}{\micro\ampere}. We measured the SQIs for \qtylist{3.0;3.2;3.4}{\kelvin}. They are shown in Figure~\ref{fig:CP2.6B_revisited_SQIs}. The curved background might be attributed to a crossover between dc-SQUID oscillations and the Little-Parks effect\cite{sharonCurrentinducedSQUIDBehavior2016}.

\begin{figure}[ht!]
	\centering
	%% Creator: Matplotlib, PGF backend
%%
%% To include the figure in your LaTeX document, write
%%   \input{<filename>.pgf}
%%
%% Make sure the required packages are loaded in your preamble
%%   \usepackage{pgf}
%%
%% Also ensure that all the required font packages are loaded; for instance,
%% the lmodern package is sometimes necessary when using math font.
%%   \usepackage{lmodern}
%%
%% Figures using additional raster images can only be included by \input if
%% they are in the same directory as the main LaTeX file. For loading figures
%% from other directories you can use the `import` package
%%   \usepackage{import}
%%
%% and then include the figures with
%%   \import{<path to file>}{<filename>.pgf}
%%
%% Matplotlib used the following preamble
%%   \usepackage{siunitx}
%%   \usepackage{fontspec}
%%   \setmainfont{Times New Roman.ttf}[Path=\detokenize{/System/Library/Fonts/Supplemental/}]
%%   \setsansfont{DejaVuSans.ttf}[Path=\detokenize{/Users/julian/UL-BRP-analysis/venv/lib/python3.10/site-packages/matplotlib/mpl-data/fonts/ttf/}]
%%   \setmonofont{DejaVuSansMono.ttf}[Path=\detokenize{/Users/julian/UL-BRP-analysis/venv/lib/python3.10/site-packages/matplotlib/mpl-data/fonts/ttf/}]
%%   \makeatletter\@ifpackageloaded{underscore}{}{\usepackage[strings]{underscore}}\makeatother
%%
\begingroup%
\makeatletter%
\begin{pgfpicture}%
\pgfpathrectangle{\pgfpointorigin}{\pgfqpoint{4.726167in}{4.728881in}}%
\pgfusepath{use as bounding box, clip}%
\begin{pgfscope}%
\pgfsetbuttcap%
\pgfsetmiterjoin%
\definecolor{currentfill}{rgb}{1.000000,1.000000,1.000000}%
\pgfsetfillcolor{currentfill}%
\pgfsetlinewidth{0.000000pt}%
\definecolor{currentstroke}{rgb}{1.000000,1.000000,1.000000}%
\pgfsetstrokecolor{currentstroke}%
\pgfsetdash{}{0pt}%
\pgfpathmoveto{\pgfqpoint{0.000000in}{0.000000in}}%
\pgfpathlineto{\pgfqpoint{4.726167in}{0.000000in}}%
\pgfpathlineto{\pgfqpoint{4.726167in}{4.728881in}}%
\pgfpathlineto{\pgfqpoint{0.000000in}{4.728881in}}%
\pgfpathlineto{\pgfqpoint{0.000000in}{0.000000in}}%
\pgfpathclose%
\pgfusepath{fill}%
\end{pgfscope}%
\begin{pgfscope}%
\pgfsetbuttcap%
\pgfsetmiterjoin%
\definecolor{currentfill}{rgb}{1.000000,1.000000,1.000000}%
\pgfsetfillcolor{currentfill}%
\pgfsetlinewidth{0.000000pt}%
\definecolor{currentstroke}{rgb}{0.000000,0.000000,0.000000}%
\pgfsetstrokecolor{currentstroke}%
\pgfsetstrokeopacity{0.000000}%
\pgfsetdash{}{0pt}%
\pgfpathmoveto{\pgfqpoint{0.444748in}{4.012575in}}%
\pgfpathlineto{\pgfqpoint{4.676167in}{4.012575in}}%
\pgfpathlineto{\pgfqpoint{4.676167in}{4.479825in}}%
\pgfpathlineto{\pgfqpoint{0.444748in}{4.479825in}}%
\pgfpathlineto{\pgfqpoint{0.444748in}{4.012575in}}%
\pgfpathclose%
\pgfusepath{fill}%
\end{pgfscope}%
\begin{pgfscope}%
\pgfsetbuttcap%
\pgfsetroundjoin%
\definecolor{currentfill}{rgb}{0.000000,0.000000,0.000000}%
\pgfsetfillcolor{currentfill}%
\pgfsetlinewidth{0.501875pt}%
\definecolor{currentstroke}{rgb}{0.000000,0.000000,0.000000}%
\pgfsetstrokecolor{currentstroke}%
\pgfsetdash{}{0pt}%
\pgfsys@defobject{currentmarker}{\pgfqpoint{0.000000in}{0.000000in}}{\pgfqpoint{0.000000in}{0.041667in}}{%
\pgfpathmoveto{\pgfqpoint{0.000000in}{0.000000in}}%
\pgfpathlineto{\pgfqpoint{0.000000in}{0.041667in}}%
\pgfusepath{stroke,fill}%
}%
\begin{pgfscope}%
\pgfsys@transformshift{0.643886in}{4.012575in}%
\pgfsys@useobject{currentmarker}{}%
\end{pgfscope}%
\end{pgfscope}%
\begin{pgfscope}%
\pgfsetbuttcap%
\pgfsetroundjoin%
\definecolor{currentfill}{rgb}{0.000000,0.000000,0.000000}%
\pgfsetfillcolor{currentfill}%
\pgfsetlinewidth{0.501875pt}%
\definecolor{currentstroke}{rgb}{0.000000,0.000000,0.000000}%
\pgfsetstrokecolor{currentstroke}%
\pgfsetdash{}{0pt}%
\pgfsys@defobject{currentmarker}{\pgfqpoint{0.000000in}{-0.041667in}}{\pgfqpoint{0.000000in}{0.000000in}}{%
\pgfpathmoveto{\pgfqpoint{0.000000in}{0.000000in}}%
\pgfpathlineto{\pgfqpoint{0.000000in}{-0.041667in}}%
\pgfusepath{stroke,fill}%
}%
\begin{pgfscope}%
\pgfsys@transformshift{0.643886in}{4.479825in}%
\pgfsys@useobject{currentmarker}{}%
\end{pgfscope}%
\end{pgfscope}%
\begin{pgfscope}%
\pgfsetbuttcap%
\pgfsetroundjoin%
\definecolor{currentfill}{rgb}{0.000000,0.000000,0.000000}%
\pgfsetfillcolor{currentfill}%
\pgfsetlinewidth{0.501875pt}%
\definecolor{currentstroke}{rgb}{0.000000,0.000000,0.000000}%
\pgfsetstrokecolor{currentstroke}%
\pgfsetdash{}{0pt}%
\pgfsys@defobject{currentmarker}{\pgfqpoint{0.000000in}{0.000000in}}{\pgfqpoint{0.000000in}{0.041667in}}{%
\pgfpathmoveto{\pgfqpoint{0.000000in}{0.000000in}}%
\pgfpathlineto{\pgfqpoint{0.000000in}{0.041667in}}%
\pgfusepath{stroke,fill}%
}%
\begin{pgfscope}%
\pgfsys@transformshift{1.124261in}{4.012575in}%
\pgfsys@useobject{currentmarker}{}%
\end{pgfscope}%
\end{pgfscope}%
\begin{pgfscope}%
\pgfsetbuttcap%
\pgfsetroundjoin%
\definecolor{currentfill}{rgb}{0.000000,0.000000,0.000000}%
\pgfsetfillcolor{currentfill}%
\pgfsetlinewidth{0.501875pt}%
\definecolor{currentstroke}{rgb}{0.000000,0.000000,0.000000}%
\pgfsetstrokecolor{currentstroke}%
\pgfsetdash{}{0pt}%
\pgfsys@defobject{currentmarker}{\pgfqpoint{0.000000in}{-0.041667in}}{\pgfqpoint{0.000000in}{0.000000in}}{%
\pgfpathmoveto{\pgfqpoint{0.000000in}{0.000000in}}%
\pgfpathlineto{\pgfqpoint{0.000000in}{-0.041667in}}%
\pgfusepath{stroke,fill}%
}%
\begin{pgfscope}%
\pgfsys@transformshift{1.124261in}{4.479825in}%
\pgfsys@useobject{currentmarker}{}%
\end{pgfscope}%
\end{pgfscope}%
\begin{pgfscope}%
\pgfsetbuttcap%
\pgfsetroundjoin%
\definecolor{currentfill}{rgb}{0.000000,0.000000,0.000000}%
\pgfsetfillcolor{currentfill}%
\pgfsetlinewidth{0.501875pt}%
\definecolor{currentstroke}{rgb}{0.000000,0.000000,0.000000}%
\pgfsetstrokecolor{currentstroke}%
\pgfsetdash{}{0pt}%
\pgfsys@defobject{currentmarker}{\pgfqpoint{0.000000in}{0.000000in}}{\pgfqpoint{0.000000in}{0.041667in}}{%
\pgfpathmoveto{\pgfqpoint{0.000000in}{0.000000in}}%
\pgfpathlineto{\pgfqpoint{0.000000in}{0.041667in}}%
\pgfusepath{stroke,fill}%
}%
\begin{pgfscope}%
\pgfsys@transformshift{1.604637in}{4.012575in}%
\pgfsys@useobject{currentmarker}{}%
\end{pgfscope}%
\end{pgfscope}%
\begin{pgfscope}%
\pgfsetbuttcap%
\pgfsetroundjoin%
\definecolor{currentfill}{rgb}{0.000000,0.000000,0.000000}%
\pgfsetfillcolor{currentfill}%
\pgfsetlinewidth{0.501875pt}%
\definecolor{currentstroke}{rgb}{0.000000,0.000000,0.000000}%
\pgfsetstrokecolor{currentstroke}%
\pgfsetdash{}{0pt}%
\pgfsys@defobject{currentmarker}{\pgfqpoint{0.000000in}{-0.041667in}}{\pgfqpoint{0.000000in}{0.000000in}}{%
\pgfpathmoveto{\pgfqpoint{0.000000in}{0.000000in}}%
\pgfpathlineto{\pgfqpoint{0.000000in}{-0.041667in}}%
\pgfusepath{stroke,fill}%
}%
\begin{pgfscope}%
\pgfsys@transformshift{1.604637in}{4.479825in}%
\pgfsys@useobject{currentmarker}{}%
\end{pgfscope}%
\end{pgfscope}%
\begin{pgfscope}%
\pgfsetbuttcap%
\pgfsetroundjoin%
\definecolor{currentfill}{rgb}{0.000000,0.000000,0.000000}%
\pgfsetfillcolor{currentfill}%
\pgfsetlinewidth{0.501875pt}%
\definecolor{currentstroke}{rgb}{0.000000,0.000000,0.000000}%
\pgfsetstrokecolor{currentstroke}%
\pgfsetdash{}{0pt}%
\pgfsys@defobject{currentmarker}{\pgfqpoint{0.000000in}{0.000000in}}{\pgfqpoint{0.000000in}{0.041667in}}{%
\pgfpathmoveto{\pgfqpoint{0.000000in}{0.000000in}}%
\pgfpathlineto{\pgfqpoint{0.000000in}{0.041667in}}%
\pgfusepath{stroke,fill}%
}%
\begin{pgfscope}%
\pgfsys@transformshift{2.085012in}{4.012575in}%
\pgfsys@useobject{currentmarker}{}%
\end{pgfscope}%
\end{pgfscope}%
\begin{pgfscope}%
\pgfsetbuttcap%
\pgfsetroundjoin%
\definecolor{currentfill}{rgb}{0.000000,0.000000,0.000000}%
\pgfsetfillcolor{currentfill}%
\pgfsetlinewidth{0.501875pt}%
\definecolor{currentstroke}{rgb}{0.000000,0.000000,0.000000}%
\pgfsetstrokecolor{currentstroke}%
\pgfsetdash{}{0pt}%
\pgfsys@defobject{currentmarker}{\pgfqpoint{0.000000in}{-0.041667in}}{\pgfqpoint{0.000000in}{0.000000in}}{%
\pgfpathmoveto{\pgfqpoint{0.000000in}{0.000000in}}%
\pgfpathlineto{\pgfqpoint{0.000000in}{-0.041667in}}%
\pgfusepath{stroke,fill}%
}%
\begin{pgfscope}%
\pgfsys@transformshift{2.085012in}{4.479825in}%
\pgfsys@useobject{currentmarker}{}%
\end{pgfscope}%
\end{pgfscope}%
\begin{pgfscope}%
\pgfsetbuttcap%
\pgfsetroundjoin%
\definecolor{currentfill}{rgb}{0.000000,0.000000,0.000000}%
\pgfsetfillcolor{currentfill}%
\pgfsetlinewidth{0.501875pt}%
\definecolor{currentstroke}{rgb}{0.000000,0.000000,0.000000}%
\pgfsetstrokecolor{currentstroke}%
\pgfsetdash{}{0pt}%
\pgfsys@defobject{currentmarker}{\pgfqpoint{0.000000in}{0.000000in}}{\pgfqpoint{0.000000in}{0.041667in}}{%
\pgfpathmoveto{\pgfqpoint{0.000000in}{0.000000in}}%
\pgfpathlineto{\pgfqpoint{0.000000in}{0.041667in}}%
\pgfusepath{stroke,fill}%
}%
\begin{pgfscope}%
\pgfsys@transformshift{2.565388in}{4.012575in}%
\pgfsys@useobject{currentmarker}{}%
\end{pgfscope}%
\end{pgfscope}%
\begin{pgfscope}%
\pgfsetbuttcap%
\pgfsetroundjoin%
\definecolor{currentfill}{rgb}{0.000000,0.000000,0.000000}%
\pgfsetfillcolor{currentfill}%
\pgfsetlinewidth{0.501875pt}%
\definecolor{currentstroke}{rgb}{0.000000,0.000000,0.000000}%
\pgfsetstrokecolor{currentstroke}%
\pgfsetdash{}{0pt}%
\pgfsys@defobject{currentmarker}{\pgfqpoint{0.000000in}{-0.041667in}}{\pgfqpoint{0.000000in}{0.000000in}}{%
\pgfpathmoveto{\pgfqpoint{0.000000in}{0.000000in}}%
\pgfpathlineto{\pgfqpoint{0.000000in}{-0.041667in}}%
\pgfusepath{stroke,fill}%
}%
\begin{pgfscope}%
\pgfsys@transformshift{2.565388in}{4.479825in}%
\pgfsys@useobject{currentmarker}{}%
\end{pgfscope}%
\end{pgfscope}%
\begin{pgfscope}%
\pgfsetbuttcap%
\pgfsetroundjoin%
\definecolor{currentfill}{rgb}{0.000000,0.000000,0.000000}%
\pgfsetfillcolor{currentfill}%
\pgfsetlinewidth{0.501875pt}%
\definecolor{currentstroke}{rgb}{0.000000,0.000000,0.000000}%
\pgfsetstrokecolor{currentstroke}%
\pgfsetdash{}{0pt}%
\pgfsys@defobject{currentmarker}{\pgfqpoint{0.000000in}{0.000000in}}{\pgfqpoint{0.000000in}{0.041667in}}{%
\pgfpathmoveto{\pgfqpoint{0.000000in}{0.000000in}}%
\pgfpathlineto{\pgfqpoint{0.000000in}{0.041667in}}%
\pgfusepath{stroke,fill}%
}%
\begin{pgfscope}%
\pgfsys@transformshift{3.045763in}{4.012575in}%
\pgfsys@useobject{currentmarker}{}%
\end{pgfscope}%
\end{pgfscope}%
\begin{pgfscope}%
\pgfsetbuttcap%
\pgfsetroundjoin%
\definecolor{currentfill}{rgb}{0.000000,0.000000,0.000000}%
\pgfsetfillcolor{currentfill}%
\pgfsetlinewidth{0.501875pt}%
\definecolor{currentstroke}{rgb}{0.000000,0.000000,0.000000}%
\pgfsetstrokecolor{currentstroke}%
\pgfsetdash{}{0pt}%
\pgfsys@defobject{currentmarker}{\pgfqpoint{0.000000in}{-0.041667in}}{\pgfqpoint{0.000000in}{0.000000in}}{%
\pgfpathmoveto{\pgfqpoint{0.000000in}{0.000000in}}%
\pgfpathlineto{\pgfqpoint{0.000000in}{-0.041667in}}%
\pgfusepath{stroke,fill}%
}%
\begin{pgfscope}%
\pgfsys@transformshift{3.045763in}{4.479825in}%
\pgfsys@useobject{currentmarker}{}%
\end{pgfscope}%
\end{pgfscope}%
\begin{pgfscope}%
\pgfsetbuttcap%
\pgfsetroundjoin%
\definecolor{currentfill}{rgb}{0.000000,0.000000,0.000000}%
\pgfsetfillcolor{currentfill}%
\pgfsetlinewidth{0.501875pt}%
\definecolor{currentstroke}{rgb}{0.000000,0.000000,0.000000}%
\pgfsetstrokecolor{currentstroke}%
\pgfsetdash{}{0pt}%
\pgfsys@defobject{currentmarker}{\pgfqpoint{0.000000in}{0.000000in}}{\pgfqpoint{0.000000in}{0.041667in}}{%
\pgfpathmoveto{\pgfqpoint{0.000000in}{0.000000in}}%
\pgfpathlineto{\pgfqpoint{0.000000in}{0.041667in}}%
\pgfusepath{stroke,fill}%
}%
\begin{pgfscope}%
\pgfsys@transformshift{3.526138in}{4.012575in}%
\pgfsys@useobject{currentmarker}{}%
\end{pgfscope}%
\end{pgfscope}%
\begin{pgfscope}%
\pgfsetbuttcap%
\pgfsetroundjoin%
\definecolor{currentfill}{rgb}{0.000000,0.000000,0.000000}%
\pgfsetfillcolor{currentfill}%
\pgfsetlinewidth{0.501875pt}%
\definecolor{currentstroke}{rgb}{0.000000,0.000000,0.000000}%
\pgfsetstrokecolor{currentstroke}%
\pgfsetdash{}{0pt}%
\pgfsys@defobject{currentmarker}{\pgfqpoint{0.000000in}{-0.041667in}}{\pgfqpoint{0.000000in}{0.000000in}}{%
\pgfpathmoveto{\pgfqpoint{0.000000in}{0.000000in}}%
\pgfpathlineto{\pgfqpoint{0.000000in}{-0.041667in}}%
\pgfusepath{stroke,fill}%
}%
\begin{pgfscope}%
\pgfsys@transformshift{3.526138in}{4.479825in}%
\pgfsys@useobject{currentmarker}{}%
\end{pgfscope}%
\end{pgfscope}%
\begin{pgfscope}%
\pgfsetbuttcap%
\pgfsetroundjoin%
\definecolor{currentfill}{rgb}{0.000000,0.000000,0.000000}%
\pgfsetfillcolor{currentfill}%
\pgfsetlinewidth{0.501875pt}%
\definecolor{currentstroke}{rgb}{0.000000,0.000000,0.000000}%
\pgfsetstrokecolor{currentstroke}%
\pgfsetdash{}{0pt}%
\pgfsys@defobject{currentmarker}{\pgfqpoint{0.000000in}{0.000000in}}{\pgfqpoint{0.000000in}{0.041667in}}{%
\pgfpathmoveto{\pgfqpoint{0.000000in}{0.000000in}}%
\pgfpathlineto{\pgfqpoint{0.000000in}{0.041667in}}%
\pgfusepath{stroke,fill}%
}%
\begin{pgfscope}%
\pgfsys@transformshift{4.006514in}{4.012575in}%
\pgfsys@useobject{currentmarker}{}%
\end{pgfscope}%
\end{pgfscope}%
\begin{pgfscope}%
\pgfsetbuttcap%
\pgfsetroundjoin%
\definecolor{currentfill}{rgb}{0.000000,0.000000,0.000000}%
\pgfsetfillcolor{currentfill}%
\pgfsetlinewidth{0.501875pt}%
\definecolor{currentstroke}{rgb}{0.000000,0.000000,0.000000}%
\pgfsetstrokecolor{currentstroke}%
\pgfsetdash{}{0pt}%
\pgfsys@defobject{currentmarker}{\pgfqpoint{0.000000in}{-0.041667in}}{\pgfqpoint{0.000000in}{0.000000in}}{%
\pgfpathmoveto{\pgfqpoint{0.000000in}{0.000000in}}%
\pgfpathlineto{\pgfqpoint{0.000000in}{-0.041667in}}%
\pgfusepath{stroke,fill}%
}%
\begin{pgfscope}%
\pgfsys@transformshift{4.006514in}{4.479825in}%
\pgfsys@useobject{currentmarker}{}%
\end{pgfscope}%
\end{pgfscope}%
\begin{pgfscope}%
\pgfsetbuttcap%
\pgfsetroundjoin%
\definecolor{currentfill}{rgb}{0.000000,0.000000,0.000000}%
\pgfsetfillcolor{currentfill}%
\pgfsetlinewidth{0.501875pt}%
\definecolor{currentstroke}{rgb}{0.000000,0.000000,0.000000}%
\pgfsetstrokecolor{currentstroke}%
\pgfsetdash{}{0pt}%
\pgfsys@defobject{currentmarker}{\pgfqpoint{0.000000in}{0.000000in}}{\pgfqpoint{0.000000in}{0.041667in}}{%
\pgfpathmoveto{\pgfqpoint{0.000000in}{0.000000in}}%
\pgfpathlineto{\pgfqpoint{0.000000in}{0.041667in}}%
\pgfusepath{stroke,fill}%
}%
\begin{pgfscope}%
\pgfsys@transformshift{4.486889in}{4.012575in}%
\pgfsys@useobject{currentmarker}{}%
\end{pgfscope}%
\end{pgfscope}%
\begin{pgfscope}%
\pgfsetbuttcap%
\pgfsetroundjoin%
\definecolor{currentfill}{rgb}{0.000000,0.000000,0.000000}%
\pgfsetfillcolor{currentfill}%
\pgfsetlinewidth{0.501875pt}%
\definecolor{currentstroke}{rgb}{0.000000,0.000000,0.000000}%
\pgfsetstrokecolor{currentstroke}%
\pgfsetdash{}{0pt}%
\pgfsys@defobject{currentmarker}{\pgfqpoint{0.000000in}{-0.041667in}}{\pgfqpoint{0.000000in}{0.000000in}}{%
\pgfpathmoveto{\pgfqpoint{0.000000in}{0.000000in}}%
\pgfpathlineto{\pgfqpoint{0.000000in}{-0.041667in}}%
\pgfusepath{stroke,fill}%
}%
\begin{pgfscope}%
\pgfsys@transformshift{4.486889in}{4.479825in}%
\pgfsys@useobject{currentmarker}{}%
\end{pgfscope}%
\end{pgfscope}%
\begin{pgfscope}%
\pgfsetbuttcap%
\pgfsetroundjoin%
\definecolor{currentfill}{rgb}{0.000000,0.000000,0.000000}%
\pgfsetfillcolor{currentfill}%
\pgfsetlinewidth{0.501875pt}%
\definecolor{currentstroke}{rgb}{0.000000,0.000000,0.000000}%
\pgfsetstrokecolor{currentstroke}%
\pgfsetdash{}{0pt}%
\pgfsys@defobject{currentmarker}{\pgfqpoint{0.000000in}{0.000000in}}{\pgfqpoint{0.000000in}{0.020833in}}{%
\pgfpathmoveto{\pgfqpoint{0.000000in}{0.000000in}}%
\pgfpathlineto{\pgfqpoint{0.000000in}{0.020833in}}%
\pgfusepath{stroke,fill}%
}%
\begin{pgfscope}%
\pgfsys@transformshift{0.451736in}{4.012575in}%
\pgfsys@useobject{currentmarker}{}%
\end{pgfscope}%
\end{pgfscope}%
\begin{pgfscope}%
\pgfsetbuttcap%
\pgfsetroundjoin%
\definecolor{currentfill}{rgb}{0.000000,0.000000,0.000000}%
\pgfsetfillcolor{currentfill}%
\pgfsetlinewidth{0.501875pt}%
\definecolor{currentstroke}{rgb}{0.000000,0.000000,0.000000}%
\pgfsetstrokecolor{currentstroke}%
\pgfsetdash{}{0pt}%
\pgfsys@defobject{currentmarker}{\pgfqpoint{0.000000in}{-0.020833in}}{\pgfqpoint{0.000000in}{0.000000in}}{%
\pgfpathmoveto{\pgfqpoint{0.000000in}{0.000000in}}%
\pgfpathlineto{\pgfqpoint{0.000000in}{-0.020833in}}%
\pgfusepath{stroke,fill}%
}%
\begin{pgfscope}%
\pgfsys@transformshift{0.451736in}{4.479825in}%
\pgfsys@useobject{currentmarker}{}%
\end{pgfscope}%
\end{pgfscope}%
\begin{pgfscope}%
\pgfsetbuttcap%
\pgfsetroundjoin%
\definecolor{currentfill}{rgb}{0.000000,0.000000,0.000000}%
\pgfsetfillcolor{currentfill}%
\pgfsetlinewidth{0.501875pt}%
\definecolor{currentstroke}{rgb}{0.000000,0.000000,0.000000}%
\pgfsetstrokecolor{currentstroke}%
\pgfsetdash{}{0pt}%
\pgfsys@defobject{currentmarker}{\pgfqpoint{0.000000in}{0.000000in}}{\pgfqpoint{0.000000in}{0.020833in}}{%
\pgfpathmoveto{\pgfqpoint{0.000000in}{0.000000in}}%
\pgfpathlineto{\pgfqpoint{0.000000in}{0.020833in}}%
\pgfusepath{stroke,fill}%
}%
\begin{pgfscope}%
\pgfsys@transformshift{0.547811in}{4.012575in}%
\pgfsys@useobject{currentmarker}{}%
\end{pgfscope}%
\end{pgfscope}%
\begin{pgfscope}%
\pgfsetbuttcap%
\pgfsetroundjoin%
\definecolor{currentfill}{rgb}{0.000000,0.000000,0.000000}%
\pgfsetfillcolor{currentfill}%
\pgfsetlinewidth{0.501875pt}%
\definecolor{currentstroke}{rgb}{0.000000,0.000000,0.000000}%
\pgfsetstrokecolor{currentstroke}%
\pgfsetdash{}{0pt}%
\pgfsys@defobject{currentmarker}{\pgfqpoint{0.000000in}{-0.020833in}}{\pgfqpoint{0.000000in}{0.000000in}}{%
\pgfpathmoveto{\pgfqpoint{0.000000in}{0.000000in}}%
\pgfpathlineto{\pgfqpoint{0.000000in}{-0.020833in}}%
\pgfusepath{stroke,fill}%
}%
\begin{pgfscope}%
\pgfsys@transformshift{0.547811in}{4.479825in}%
\pgfsys@useobject{currentmarker}{}%
\end{pgfscope}%
\end{pgfscope}%
\begin{pgfscope}%
\pgfsetbuttcap%
\pgfsetroundjoin%
\definecolor{currentfill}{rgb}{0.000000,0.000000,0.000000}%
\pgfsetfillcolor{currentfill}%
\pgfsetlinewidth{0.501875pt}%
\definecolor{currentstroke}{rgb}{0.000000,0.000000,0.000000}%
\pgfsetstrokecolor{currentstroke}%
\pgfsetdash{}{0pt}%
\pgfsys@defobject{currentmarker}{\pgfqpoint{0.000000in}{0.000000in}}{\pgfqpoint{0.000000in}{0.020833in}}{%
\pgfpathmoveto{\pgfqpoint{0.000000in}{0.000000in}}%
\pgfpathlineto{\pgfqpoint{0.000000in}{0.020833in}}%
\pgfusepath{stroke,fill}%
}%
\begin{pgfscope}%
\pgfsys@transformshift{0.739961in}{4.012575in}%
\pgfsys@useobject{currentmarker}{}%
\end{pgfscope}%
\end{pgfscope}%
\begin{pgfscope}%
\pgfsetbuttcap%
\pgfsetroundjoin%
\definecolor{currentfill}{rgb}{0.000000,0.000000,0.000000}%
\pgfsetfillcolor{currentfill}%
\pgfsetlinewidth{0.501875pt}%
\definecolor{currentstroke}{rgb}{0.000000,0.000000,0.000000}%
\pgfsetstrokecolor{currentstroke}%
\pgfsetdash{}{0pt}%
\pgfsys@defobject{currentmarker}{\pgfqpoint{0.000000in}{-0.020833in}}{\pgfqpoint{0.000000in}{0.000000in}}{%
\pgfpathmoveto{\pgfqpoint{0.000000in}{0.000000in}}%
\pgfpathlineto{\pgfqpoint{0.000000in}{-0.020833in}}%
\pgfusepath{stroke,fill}%
}%
\begin{pgfscope}%
\pgfsys@transformshift{0.739961in}{4.479825in}%
\pgfsys@useobject{currentmarker}{}%
\end{pgfscope}%
\end{pgfscope}%
\begin{pgfscope}%
\pgfsetbuttcap%
\pgfsetroundjoin%
\definecolor{currentfill}{rgb}{0.000000,0.000000,0.000000}%
\pgfsetfillcolor{currentfill}%
\pgfsetlinewidth{0.501875pt}%
\definecolor{currentstroke}{rgb}{0.000000,0.000000,0.000000}%
\pgfsetstrokecolor{currentstroke}%
\pgfsetdash{}{0pt}%
\pgfsys@defobject{currentmarker}{\pgfqpoint{0.000000in}{0.000000in}}{\pgfqpoint{0.000000in}{0.020833in}}{%
\pgfpathmoveto{\pgfqpoint{0.000000in}{0.000000in}}%
\pgfpathlineto{\pgfqpoint{0.000000in}{0.020833in}}%
\pgfusepath{stroke,fill}%
}%
\begin{pgfscope}%
\pgfsys@transformshift{0.836036in}{4.012575in}%
\pgfsys@useobject{currentmarker}{}%
\end{pgfscope}%
\end{pgfscope}%
\begin{pgfscope}%
\pgfsetbuttcap%
\pgfsetroundjoin%
\definecolor{currentfill}{rgb}{0.000000,0.000000,0.000000}%
\pgfsetfillcolor{currentfill}%
\pgfsetlinewidth{0.501875pt}%
\definecolor{currentstroke}{rgb}{0.000000,0.000000,0.000000}%
\pgfsetstrokecolor{currentstroke}%
\pgfsetdash{}{0pt}%
\pgfsys@defobject{currentmarker}{\pgfqpoint{0.000000in}{-0.020833in}}{\pgfqpoint{0.000000in}{0.000000in}}{%
\pgfpathmoveto{\pgfqpoint{0.000000in}{0.000000in}}%
\pgfpathlineto{\pgfqpoint{0.000000in}{-0.020833in}}%
\pgfusepath{stroke,fill}%
}%
\begin{pgfscope}%
\pgfsys@transformshift{0.836036in}{4.479825in}%
\pgfsys@useobject{currentmarker}{}%
\end{pgfscope}%
\end{pgfscope}%
\begin{pgfscope}%
\pgfsetbuttcap%
\pgfsetroundjoin%
\definecolor{currentfill}{rgb}{0.000000,0.000000,0.000000}%
\pgfsetfillcolor{currentfill}%
\pgfsetlinewidth{0.501875pt}%
\definecolor{currentstroke}{rgb}{0.000000,0.000000,0.000000}%
\pgfsetstrokecolor{currentstroke}%
\pgfsetdash{}{0pt}%
\pgfsys@defobject{currentmarker}{\pgfqpoint{0.000000in}{0.000000in}}{\pgfqpoint{0.000000in}{0.020833in}}{%
\pgfpathmoveto{\pgfqpoint{0.000000in}{0.000000in}}%
\pgfpathlineto{\pgfqpoint{0.000000in}{0.020833in}}%
\pgfusepath{stroke,fill}%
}%
\begin{pgfscope}%
\pgfsys@transformshift{0.932111in}{4.012575in}%
\pgfsys@useobject{currentmarker}{}%
\end{pgfscope}%
\end{pgfscope}%
\begin{pgfscope}%
\pgfsetbuttcap%
\pgfsetroundjoin%
\definecolor{currentfill}{rgb}{0.000000,0.000000,0.000000}%
\pgfsetfillcolor{currentfill}%
\pgfsetlinewidth{0.501875pt}%
\definecolor{currentstroke}{rgb}{0.000000,0.000000,0.000000}%
\pgfsetstrokecolor{currentstroke}%
\pgfsetdash{}{0pt}%
\pgfsys@defobject{currentmarker}{\pgfqpoint{0.000000in}{-0.020833in}}{\pgfqpoint{0.000000in}{0.000000in}}{%
\pgfpathmoveto{\pgfqpoint{0.000000in}{0.000000in}}%
\pgfpathlineto{\pgfqpoint{0.000000in}{-0.020833in}}%
\pgfusepath{stroke,fill}%
}%
\begin{pgfscope}%
\pgfsys@transformshift{0.932111in}{4.479825in}%
\pgfsys@useobject{currentmarker}{}%
\end{pgfscope}%
\end{pgfscope}%
\begin{pgfscope}%
\pgfsetbuttcap%
\pgfsetroundjoin%
\definecolor{currentfill}{rgb}{0.000000,0.000000,0.000000}%
\pgfsetfillcolor{currentfill}%
\pgfsetlinewidth{0.501875pt}%
\definecolor{currentstroke}{rgb}{0.000000,0.000000,0.000000}%
\pgfsetstrokecolor{currentstroke}%
\pgfsetdash{}{0pt}%
\pgfsys@defobject{currentmarker}{\pgfqpoint{0.000000in}{0.000000in}}{\pgfqpoint{0.000000in}{0.020833in}}{%
\pgfpathmoveto{\pgfqpoint{0.000000in}{0.000000in}}%
\pgfpathlineto{\pgfqpoint{0.000000in}{0.020833in}}%
\pgfusepath{stroke,fill}%
}%
\begin{pgfscope}%
\pgfsys@transformshift{1.028186in}{4.012575in}%
\pgfsys@useobject{currentmarker}{}%
\end{pgfscope}%
\end{pgfscope}%
\begin{pgfscope}%
\pgfsetbuttcap%
\pgfsetroundjoin%
\definecolor{currentfill}{rgb}{0.000000,0.000000,0.000000}%
\pgfsetfillcolor{currentfill}%
\pgfsetlinewidth{0.501875pt}%
\definecolor{currentstroke}{rgb}{0.000000,0.000000,0.000000}%
\pgfsetstrokecolor{currentstroke}%
\pgfsetdash{}{0pt}%
\pgfsys@defobject{currentmarker}{\pgfqpoint{0.000000in}{-0.020833in}}{\pgfqpoint{0.000000in}{0.000000in}}{%
\pgfpathmoveto{\pgfqpoint{0.000000in}{0.000000in}}%
\pgfpathlineto{\pgfqpoint{0.000000in}{-0.020833in}}%
\pgfusepath{stroke,fill}%
}%
\begin{pgfscope}%
\pgfsys@transformshift{1.028186in}{4.479825in}%
\pgfsys@useobject{currentmarker}{}%
\end{pgfscope}%
\end{pgfscope}%
\begin{pgfscope}%
\pgfsetbuttcap%
\pgfsetroundjoin%
\definecolor{currentfill}{rgb}{0.000000,0.000000,0.000000}%
\pgfsetfillcolor{currentfill}%
\pgfsetlinewidth{0.501875pt}%
\definecolor{currentstroke}{rgb}{0.000000,0.000000,0.000000}%
\pgfsetstrokecolor{currentstroke}%
\pgfsetdash{}{0pt}%
\pgfsys@defobject{currentmarker}{\pgfqpoint{0.000000in}{0.000000in}}{\pgfqpoint{0.000000in}{0.020833in}}{%
\pgfpathmoveto{\pgfqpoint{0.000000in}{0.000000in}}%
\pgfpathlineto{\pgfqpoint{0.000000in}{0.020833in}}%
\pgfusepath{stroke,fill}%
}%
\begin{pgfscope}%
\pgfsys@transformshift{1.220336in}{4.012575in}%
\pgfsys@useobject{currentmarker}{}%
\end{pgfscope}%
\end{pgfscope}%
\begin{pgfscope}%
\pgfsetbuttcap%
\pgfsetroundjoin%
\definecolor{currentfill}{rgb}{0.000000,0.000000,0.000000}%
\pgfsetfillcolor{currentfill}%
\pgfsetlinewidth{0.501875pt}%
\definecolor{currentstroke}{rgb}{0.000000,0.000000,0.000000}%
\pgfsetstrokecolor{currentstroke}%
\pgfsetdash{}{0pt}%
\pgfsys@defobject{currentmarker}{\pgfqpoint{0.000000in}{-0.020833in}}{\pgfqpoint{0.000000in}{0.000000in}}{%
\pgfpathmoveto{\pgfqpoint{0.000000in}{0.000000in}}%
\pgfpathlineto{\pgfqpoint{0.000000in}{-0.020833in}}%
\pgfusepath{stroke,fill}%
}%
\begin{pgfscope}%
\pgfsys@transformshift{1.220336in}{4.479825in}%
\pgfsys@useobject{currentmarker}{}%
\end{pgfscope}%
\end{pgfscope}%
\begin{pgfscope}%
\pgfsetbuttcap%
\pgfsetroundjoin%
\definecolor{currentfill}{rgb}{0.000000,0.000000,0.000000}%
\pgfsetfillcolor{currentfill}%
\pgfsetlinewidth{0.501875pt}%
\definecolor{currentstroke}{rgb}{0.000000,0.000000,0.000000}%
\pgfsetstrokecolor{currentstroke}%
\pgfsetdash{}{0pt}%
\pgfsys@defobject{currentmarker}{\pgfqpoint{0.000000in}{0.000000in}}{\pgfqpoint{0.000000in}{0.020833in}}{%
\pgfpathmoveto{\pgfqpoint{0.000000in}{0.000000in}}%
\pgfpathlineto{\pgfqpoint{0.000000in}{0.020833in}}%
\pgfusepath{stroke,fill}%
}%
\begin{pgfscope}%
\pgfsys@transformshift{1.316411in}{4.012575in}%
\pgfsys@useobject{currentmarker}{}%
\end{pgfscope}%
\end{pgfscope}%
\begin{pgfscope}%
\pgfsetbuttcap%
\pgfsetroundjoin%
\definecolor{currentfill}{rgb}{0.000000,0.000000,0.000000}%
\pgfsetfillcolor{currentfill}%
\pgfsetlinewidth{0.501875pt}%
\definecolor{currentstroke}{rgb}{0.000000,0.000000,0.000000}%
\pgfsetstrokecolor{currentstroke}%
\pgfsetdash{}{0pt}%
\pgfsys@defobject{currentmarker}{\pgfqpoint{0.000000in}{-0.020833in}}{\pgfqpoint{0.000000in}{0.000000in}}{%
\pgfpathmoveto{\pgfqpoint{0.000000in}{0.000000in}}%
\pgfpathlineto{\pgfqpoint{0.000000in}{-0.020833in}}%
\pgfusepath{stroke,fill}%
}%
\begin{pgfscope}%
\pgfsys@transformshift{1.316411in}{4.479825in}%
\pgfsys@useobject{currentmarker}{}%
\end{pgfscope}%
\end{pgfscope}%
\begin{pgfscope}%
\pgfsetbuttcap%
\pgfsetroundjoin%
\definecolor{currentfill}{rgb}{0.000000,0.000000,0.000000}%
\pgfsetfillcolor{currentfill}%
\pgfsetlinewidth{0.501875pt}%
\definecolor{currentstroke}{rgb}{0.000000,0.000000,0.000000}%
\pgfsetstrokecolor{currentstroke}%
\pgfsetdash{}{0pt}%
\pgfsys@defobject{currentmarker}{\pgfqpoint{0.000000in}{0.000000in}}{\pgfqpoint{0.000000in}{0.020833in}}{%
\pgfpathmoveto{\pgfqpoint{0.000000in}{0.000000in}}%
\pgfpathlineto{\pgfqpoint{0.000000in}{0.020833in}}%
\pgfusepath{stroke,fill}%
}%
\begin{pgfscope}%
\pgfsys@transformshift{1.412487in}{4.012575in}%
\pgfsys@useobject{currentmarker}{}%
\end{pgfscope}%
\end{pgfscope}%
\begin{pgfscope}%
\pgfsetbuttcap%
\pgfsetroundjoin%
\definecolor{currentfill}{rgb}{0.000000,0.000000,0.000000}%
\pgfsetfillcolor{currentfill}%
\pgfsetlinewidth{0.501875pt}%
\definecolor{currentstroke}{rgb}{0.000000,0.000000,0.000000}%
\pgfsetstrokecolor{currentstroke}%
\pgfsetdash{}{0pt}%
\pgfsys@defobject{currentmarker}{\pgfqpoint{0.000000in}{-0.020833in}}{\pgfqpoint{0.000000in}{0.000000in}}{%
\pgfpathmoveto{\pgfqpoint{0.000000in}{0.000000in}}%
\pgfpathlineto{\pgfqpoint{0.000000in}{-0.020833in}}%
\pgfusepath{stroke,fill}%
}%
\begin{pgfscope}%
\pgfsys@transformshift{1.412487in}{4.479825in}%
\pgfsys@useobject{currentmarker}{}%
\end{pgfscope}%
\end{pgfscope}%
\begin{pgfscope}%
\pgfsetbuttcap%
\pgfsetroundjoin%
\definecolor{currentfill}{rgb}{0.000000,0.000000,0.000000}%
\pgfsetfillcolor{currentfill}%
\pgfsetlinewidth{0.501875pt}%
\definecolor{currentstroke}{rgb}{0.000000,0.000000,0.000000}%
\pgfsetstrokecolor{currentstroke}%
\pgfsetdash{}{0pt}%
\pgfsys@defobject{currentmarker}{\pgfqpoint{0.000000in}{0.000000in}}{\pgfqpoint{0.000000in}{0.020833in}}{%
\pgfpathmoveto{\pgfqpoint{0.000000in}{0.000000in}}%
\pgfpathlineto{\pgfqpoint{0.000000in}{0.020833in}}%
\pgfusepath{stroke,fill}%
}%
\begin{pgfscope}%
\pgfsys@transformshift{1.508562in}{4.012575in}%
\pgfsys@useobject{currentmarker}{}%
\end{pgfscope}%
\end{pgfscope}%
\begin{pgfscope}%
\pgfsetbuttcap%
\pgfsetroundjoin%
\definecolor{currentfill}{rgb}{0.000000,0.000000,0.000000}%
\pgfsetfillcolor{currentfill}%
\pgfsetlinewidth{0.501875pt}%
\definecolor{currentstroke}{rgb}{0.000000,0.000000,0.000000}%
\pgfsetstrokecolor{currentstroke}%
\pgfsetdash{}{0pt}%
\pgfsys@defobject{currentmarker}{\pgfqpoint{0.000000in}{-0.020833in}}{\pgfqpoint{0.000000in}{0.000000in}}{%
\pgfpathmoveto{\pgfqpoint{0.000000in}{0.000000in}}%
\pgfpathlineto{\pgfqpoint{0.000000in}{-0.020833in}}%
\pgfusepath{stroke,fill}%
}%
\begin{pgfscope}%
\pgfsys@transformshift{1.508562in}{4.479825in}%
\pgfsys@useobject{currentmarker}{}%
\end{pgfscope}%
\end{pgfscope}%
\begin{pgfscope}%
\pgfsetbuttcap%
\pgfsetroundjoin%
\definecolor{currentfill}{rgb}{0.000000,0.000000,0.000000}%
\pgfsetfillcolor{currentfill}%
\pgfsetlinewidth{0.501875pt}%
\definecolor{currentstroke}{rgb}{0.000000,0.000000,0.000000}%
\pgfsetstrokecolor{currentstroke}%
\pgfsetdash{}{0pt}%
\pgfsys@defobject{currentmarker}{\pgfqpoint{0.000000in}{0.000000in}}{\pgfqpoint{0.000000in}{0.020833in}}{%
\pgfpathmoveto{\pgfqpoint{0.000000in}{0.000000in}}%
\pgfpathlineto{\pgfqpoint{0.000000in}{0.020833in}}%
\pgfusepath{stroke,fill}%
}%
\begin{pgfscope}%
\pgfsys@transformshift{1.700712in}{4.012575in}%
\pgfsys@useobject{currentmarker}{}%
\end{pgfscope}%
\end{pgfscope}%
\begin{pgfscope}%
\pgfsetbuttcap%
\pgfsetroundjoin%
\definecolor{currentfill}{rgb}{0.000000,0.000000,0.000000}%
\pgfsetfillcolor{currentfill}%
\pgfsetlinewidth{0.501875pt}%
\definecolor{currentstroke}{rgb}{0.000000,0.000000,0.000000}%
\pgfsetstrokecolor{currentstroke}%
\pgfsetdash{}{0pt}%
\pgfsys@defobject{currentmarker}{\pgfqpoint{0.000000in}{-0.020833in}}{\pgfqpoint{0.000000in}{0.000000in}}{%
\pgfpathmoveto{\pgfqpoint{0.000000in}{0.000000in}}%
\pgfpathlineto{\pgfqpoint{0.000000in}{-0.020833in}}%
\pgfusepath{stroke,fill}%
}%
\begin{pgfscope}%
\pgfsys@transformshift{1.700712in}{4.479825in}%
\pgfsys@useobject{currentmarker}{}%
\end{pgfscope}%
\end{pgfscope}%
\begin{pgfscope}%
\pgfsetbuttcap%
\pgfsetroundjoin%
\definecolor{currentfill}{rgb}{0.000000,0.000000,0.000000}%
\pgfsetfillcolor{currentfill}%
\pgfsetlinewidth{0.501875pt}%
\definecolor{currentstroke}{rgb}{0.000000,0.000000,0.000000}%
\pgfsetstrokecolor{currentstroke}%
\pgfsetdash{}{0pt}%
\pgfsys@defobject{currentmarker}{\pgfqpoint{0.000000in}{0.000000in}}{\pgfqpoint{0.000000in}{0.020833in}}{%
\pgfpathmoveto{\pgfqpoint{0.000000in}{0.000000in}}%
\pgfpathlineto{\pgfqpoint{0.000000in}{0.020833in}}%
\pgfusepath{stroke,fill}%
}%
\begin{pgfscope}%
\pgfsys@transformshift{1.796787in}{4.012575in}%
\pgfsys@useobject{currentmarker}{}%
\end{pgfscope}%
\end{pgfscope}%
\begin{pgfscope}%
\pgfsetbuttcap%
\pgfsetroundjoin%
\definecolor{currentfill}{rgb}{0.000000,0.000000,0.000000}%
\pgfsetfillcolor{currentfill}%
\pgfsetlinewidth{0.501875pt}%
\definecolor{currentstroke}{rgb}{0.000000,0.000000,0.000000}%
\pgfsetstrokecolor{currentstroke}%
\pgfsetdash{}{0pt}%
\pgfsys@defobject{currentmarker}{\pgfqpoint{0.000000in}{-0.020833in}}{\pgfqpoint{0.000000in}{0.000000in}}{%
\pgfpathmoveto{\pgfqpoint{0.000000in}{0.000000in}}%
\pgfpathlineto{\pgfqpoint{0.000000in}{-0.020833in}}%
\pgfusepath{stroke,fill}%
}%
\begin{pgfscope}%
\pgfsys@transformshift{1.796787in}{4.479825in}%
\pgfsys@useobject{currentmarker}{}%
\end{pgfscope}%
\end{pgfscope}%
\begin{pgfscope}%
\pgfsetbuttcap%
\pgfsetroundjoin%
\definecolor{currentfill}{rgb}{0.000000,0.000000,0.000000}%
\pgfsetfillcolor{currentfill}%
\pgfsetlinewidth{0.501875pt}%
\definecolor{currentstroke}{rgb}{0.000000,0.000000,0.000000}%
\pgfsetstrokecolor{currentstroke}%
\pgfsetdash{}{0pt}%
\pgfsys@defobject{currentmarker}{\pgfqpoint{0.000000in}{0.000000in}}{\pgfqpoint{0.000000in}{0.020833in}}{%
\pgfpathmoveto{\pgfqpoint{0.000000in}{0.000000in}}%
\pgfpathlineto{\pgfqpoint{0.000000in}{0.020833in}}%
\pgfusepath{stroke,fill}%
}%
\begin{pgfscope}%
\pgfsys@transformshift{1.892862in}{4.012575in}%
\pgfsys@useobject{currentmarker}{}%
\end{pgfscope}%
\end{pgfscope}%
\begin{pgfscope}%
\pgfsetbuttcap%
\pgfsetroundjoin%
\definecolor{currentfill}{rgb}{0.000000,0.000000,0.000000}%
\pgfsetfillcolor{currentfill}%
\pgfsetlinewidth{0.501875pt}%
\definecolor{currentstroke}{rgb}{0.000000,0.000000,0.000000}%
\pgfsetstrokecolor{currentstroke}%
\pgfsetdash{}{0pt}%
\pgfsys@defobject{currentmarker}{\pgfqpoint{0.000000in}{-0.020833in}}{\pgfqpoint{0.000000in}{0.000000in}}{%
\pgfpathmoveto{\pgfqpoint{0.000000in}{0.000000in}}%
\pgfpathlineto{\pgfqpoint{0.000000in}{-0.020833in}}%
\pgfusepath{stroke,fill}%
}%
\begin{pgfscope}%
\pgfsys@transformshift{1.892862in}{4.479825in}%
\pgfsys@useobject{currentmarker}{}%
\end{pgfscope}%
\end{pgfscope}%
\begin{pgfscope}%
\pgfsetbuttcap%
\pgfsetroundjoin%
\definecolor{currentfill}{rgb}{0.000000,0.000000,0.000000}%
\pgfsetfillcolor{currentfill}%
\pgfsetlinewidth{0.501875pt}%
\definecolor{currentstroke}{rgb}{0.000000,0.000000,0.000000}%
\pgfsetstrokecolor{currentstroke}%
\pgfsetdash{}{0pt}%
\pgfsys@defobject{currentmarker}{\pgfqpoint{0.000000in}{0.000000in}}{\pgfqpoint{0.000000in}{0.020833in}}{%
\pgfpathmoveto{\pgfqpoint{0.000000in}{0.000000in}}%
\pgfpathlineto{\pgfqpoint{0.000000in}{0.020833in}}%
\pgfusepath{stroke,fill}%
}%
\begin{pgfscope}%
\pgfsys@transformshift{1.988937in}{4.012575in}%
\pgfsys@useobject{currentmarker}{}%
\end{pgfscope}%
\end{pgfscope}%
\begin{pgfscope}%
\pgfsetbuttcap%
\pgfsetroundjoin%
\definecolor{currentfill}{rgb}{0.000000,0.000000,0.000000}%
\pgfsetfillcolor{currentfill}%
\pgfsetlinewidth{0.501875pt}%
\definecolor{currentstroke}{rgb}{0.000000,0.000000,0.000000}%
\pgfsetstrokecolor{currentstroke}%
\pgfsetdash{}{0pt}%
\pgfsys@defobject{currentmarker}{\pgfqpoint{0.000000in}{-0.020833in}}{\pgfqpoint{0.000000in}{0.000000in}}{%
\pgfpathmoveto{\pgfqpoint{0.000000in}{0.000000in}}%
\pgfpathlineto{\pgfqpoint{0.000000in}{-0.020833in}}%
\pgfusepath{stroke,fill}%
}%
\begin{pgfscope}%
\pgfsys@transformshift{1.988937in}{4.479825in}%
\pgfsys@useobject{currentmarker}{}%
\end{pgfscope}%
\end{pgfscope}%
\begin{pgfscope}%
\pgfsetbuttcap%
\pgfsetroundjoin%
\definecolor{currentfill}{rgb}{0.000000,0.000000,0.000000}%
\pgfsetfillcolor{currentfill}%
\pgfsetlinewidth{0.501875pt}%
\definecolor{currentstroke}{rgb}{0.000000,0.000000,0.000000}%
\pgfsetstrokecolor{currentstroke}%
\pgfsetdash{}{0pt}%
\pgfsys@defobject{currentmarker}{\pgfqpoint{0.000000in}{0.000000in}}{\pgfqpoint{0.000000in}{0.020833in}}{%
\pgfpathmoveto{\pgfqpoint{0.000000in}{0.000000in}}%
\pgfpathlineto{\pgfqpoint{0.000000in}{0.020833in}}%
\pgfusepath{stroke,fill}%
}%
\begin{pgfscope}%
\pgfsys@transformshift{2.181087in}{4.012575in}%
\pgfsys@useobject{currentmarker}{}%
\end{pgfscope}%
\end{pgfscope}%
\begin{pgfscope}%
\pgfsetbuttcap%
\pgfsetroundjoin%
\definecolor{currentfill}{rgb}{0.000000,0.000000,0.000000}%
\pgfsetfillcolor{currentfill}%
\pgfsetlinewidth{0.501875pt}%
\definecolor{currentstroke}{rgb}{0.000000,0.000000,0.000000}%
\pgfsetstrokecolor{currentstroke}%
\pgfsetdash{}{0pt}%
\pgfsys@defobject{currentmarker}{\pgfqpoint{0.000000in}{-0.020833in}}{\pgfqpoint{0.000000in}{0.000000in}}{%
\pgfpathmoveto{\pgfqpoint{0.000000in}{0.000000in}}%
\pgfpathlineto{\pgfqpoint{0.000000in}{-0.020833in}}%
\pgfusepath{stroke,fill}%
}%
\begin{pgfscope}%
\pgfsys@transformshift{2.181087in}{4.479825in}%
\pgfsys@useobject{currentmarker}{}%
\end{pgfscope}%
\end{pgfscope}%
\begin{pgfscope}%
\pgfsetbuttcap%
\pgfsetroundjoin%
\definecolor{currentfill}{rgb}{0.000000,0.000000,0.000000}%
\pgfsetfillcolor{currentfill}%
\pgfsetlinewidth{0.501875pt}%
\definecolor{currentstroke}{rgb}{0.000000,0.000000,0.000000}%
\pgfsetstrokecolor{currentstroke}%
\pgfsetdash{}{0pt}%
\pgfsys@defobject{currentmarker}{\pgfqpoint{0.000000in}{0.000000in}}{\pgfqpoint{0.000000in}{0.020833in}}{%
\pgfpathmoveto{\pgfqpoint{0.000000in}{0.000000in}}%
\pgfpathlineto{\pgfqpoint{0.000000in}{0.020833in}}%
\pgfusepath{stroke,fill}%
}%
\begin{pgfscope}%
\pgfsys@transformshift{2.277162in}{4.012575in}%
\pgfsys@useobject{currentmarker}{}%
\end{pgfscope}%
\end{pgfscope}%
\begin{pgfscope}%
\pgfsetbuttcap%
\pgfsetroundjoin%
\definecolor{currentfill}{rgb}{0.000000,0.000000,0.000000}%
\pgfsetfillcolor{currentfill}%
\pgfsetlinewidth{0.501875pt}%
\definecolor{currentstroke}{rgb}{0.000000,0.000000,0.000000}%
\pgfsetstrokecolor{currentstroke}%
\pgfsetdash{}{0pt}%
\pgfsys@defobject{currentmarker}{\pgfqpoint{0.000000in}{-0.020833in}}{\pgfqpoint{0.000000in}{0.000000in}}{%
\pgfpathmoveto{\pgfqpoint{0.000000in}{0.000000in}}%
\pgfpathlineto{\pgfqpoint{0.000000in}{-0.020833in}}%
\pgfusepath{stroke,fill}%
}%
\begin{pgfscope}%
\pgfsys@transformshift{2.277162in}{4.479825in}%
\pgfsys@useobject{currentmarker}{}%
\end{pgfscope}%
\end{pgfscope}%
\begin{pgfscope}%
\pgfsetbuttcap%
\pgfsetroundjoin%
\definecolor{currentfill}{rgb}{0.000000,0.000000,0.000000}%
\pgfsetfillcolor{currentfill}%
\pgfsetlinewidth{0.501875pt}%
\definecolor{currentstroke}{rgb}{0.000000,0.000000,0.000000}%
\pgfsetstrokecolor{currentstroke}%
\pgfsetdash{}{0pt}%
\pgfsys@defobject{currentmarker}{\pgfqpoint{0.000000in}{0.000000in}}{\pgfqpoint{0.000000in}{0.020833in}}{%
\pgfpathmoveto{\pgfqpoint{0.000000in}{0.000000in}}%
\pgfpathlineto{\pgfqpoint{0.000000in}{0.020833in}}%
\pgfusepath{stroke,fill}%
}%
\begin{pgfscope}%
\pgfsys@transformshift{2.373237in}{4.012575in}%
\pgfsys@useobject{currentmarker}{}%
\end{pgfscope}%
\end{pgfscope}%
\begin{pgfscope}%
\pgfsetbuttcap%
\pgfsetroundjoin%
\definecolor{currentfill}{rgb}{0.000000,0.000000,0.000000}%
\pgfsetfillcolor{currentfill}%
\pgfsetlinewidth{0.501875pt}%
\definecolor{currentstroke}{rgb}{0.000000,0.000000,0.000000}%
\pgfsetstrokecolor{currentstroke}%
\pgfsetdash{}{0pt}%
\pgfsys@defobject{currentmarker}{\pgfqpoint{0.000000in}{-0.020833in}}{\pgfqpoint{0.000000in}{0.000000in}}{%
\pgfpathmoveto{\pgfqpoint{0.000000in}{0.000000in}}%
\pgfpathlineto{\pgfqpoint{0.000000in}{-0.020833in}}%
\pgfusepath{stroke,fill}%
}%
\begin{pgfscope}%
\pgfsys@transformshift{2.373237in}{4.479825in}%
\pgfsys@useobject{currentmarker}{}%
\end{pgfscope}%
\end{pgfscope}%
\begin{pgfscope}%
\pgfsetbuttcap%
\pgfsetroundjoin%
\definecolor{currentfill}{rgb}{0.000000,0.000000,0.000000}%
\pgfsetfillcolor{currentfill}%
\pgfsetlinewidth{0.501875pt}%
\definecolor{currentstroke}{rgb}{0.000000,0.000000,0.000000}%
\pgfsetstrokecolor{currentstroke}%
\pgfsetdash{}{0pt}%
\pgfsys@defobject{currentmarker}{\pgfqpoint{0.000000in}{0.000000in}}{\pgfqpoint{0.000000in}{0.020833in}}{%
\pgfpathmoveto{\pgfqpoint{0.000000in}{0.000000in}}%
\pgfpathlineto{\pgfqpoint{0.000000in}{0.020833in}}%
\pgfusepath{stroke,fill}%
}%
\begin{pgfscope}%
\pgfsys@transformshift{2.469312in}{4.012575in}%
\pgfsys@useobject{currentmarker}{}%
\end{pgfscope}%
\end{pgfscope}%
\begin{pgfscope}%
\pgfsetbuttcap%
\pgfsetroundjoin%
\definecolor{currentfill}{rgb}{0.000000,0.000000,0.000000}%
\pgfsetfillcolor{currentfill}%
\pgfsetlinewidth{0.501875pt}%
\definecolor{currentstroke}{rgb}{0.000000,0.000000,0.000000}%
\pgfsetstrokecolor{currentstroke}%
\pgfsetdash{}{0pt}%
\pgfsys@defobject{currentmarker}{\pgfqpoint{0.000000in}{-0.020833in}}{\pgfqpoint{0.000000in}{0.000000in}}{%
\pgfpathmoveto{\pgfqpoint{0.000000in}{0.000000in}}%
\pgfpathlineto{\pgfqpoint{0.000000in}{-0.020833in}}%
\pgfusepath{stroke,fill}%
}%
\begin{pgfscope}%
\pgfsys@transformshift{2.469312in}{4.479825in}%
\pgfsys@useobject{currentmarker}{}%
\end{pgfscope}%
\end{pgfscope}%
\begin{pgfscope}%
\pgfsetbuttcap%
\pgfsetroundjoin%
\definecolor{currentfill}{rgb}{0.000000,0.000000,0.000000}%
\pgfsetfillcolor{currentfill}%
\pgfsetlinewidth{0.501875pt}%
\definecolor{currentstroke}{rgb}{0.000000,0.000000,0.000000}%
\pgfsetstrokecolor{currentstroke}%
\pgfsetdash{}{0pt}%
\pgfsys@defobject{currentmarker}{\pgfqpoint{0.000000in}{0.000000in}}{\pgfqpoint{0.000000in}{0.020833in}}{%
\pgfpathmoveto{\pgfqpoint{0.000000in}{0.000000in}}%
\pgfpathlineto{\pgfqpoint{0.000000in}{0.020833in}}%
\pgfusepath{stroke,fill}%
}%
\begin{pgfscope}%
\pgfsys@transformshift{2.661463in}{4.012575in}%
\pgfsys@useobject{currentmarker}{}%
\end{pgfscope}%
\end{pgfscope}%
\begin{pgfscope}%
\pgfsetbuttcap%
\pgfsetroundjoin%
\definecolor{currentfill}{rgb}{0.000000,0.000000,0.000000}%
\pgfsetfillcolor{currentfill}%
\pgfsetlinewidth{0.501875pt}%
\definecolor{currentstroke}{rgb}{0.000000,0.000000,0.000000}%
\pgfsetstrokecolor{currentstroke}%
\pgfsetdash{}{0pt}%
\pgfsys@defobject{currentmarker}{\pgfqpoint{0.000000in}{-0.020833in}}{\pgfqpoint{0.000000in}{0.000000in}}{%
\pgfpathmoveto{\pgfqpoint{0.000000in}{0.000000in}}%
\pgfpathlineto{\pgfqpoint{0.000000in}{-0.020833in}}%
\pgfusepath{stroke,fill}%
}%
\begin{pgfscope}%
\pgfsys@transformshift{2.661463in}{4.479825in}%
\pgfsys@useobject{currentmarker}{}%
\end{pgfscope}%
\end{pgfscope}%
\begin{pgfscope}%
\pgfsetbuttcap%
\pgfsetroundjoin%
\definecolor{currentfill}{rgb}{0.000000,0.000000,0.000000}%
\pgfsetfillcolor{currentfill}%
\pgfsetlinewidth{0.501875pt}%
\definecolor{currentstroke}{rgb}{0.000000,0.000000,0.000000}%
\pgfsetstrokecolor{currentstroke}%
\pgfsetdash{}{0pt}%
\pgfsys@defobject{currentmarker}{\pgfqpoint{0.000000in}{0.000000in}}{\pgfqpoint{0.000000in}{0.020833in}}{%
\pgfpathmoveto{\pgfqpoint{0.000000in}{0.000000in}}%
\pgfpathlineto{\pgfqpoint{0.000000in}{0.020833in}}%
\pgfusepath{stroke,fill}%
}%
\begin{pgfscope}%
\pgfsys@transformshift{2.757538in}{4.012575in}%
\pgfsys@useobject{currentmarker}{}%
\end{pgfscope}%
\end{pgfscope}%
\begin{pgfscope}%
\pgfsetbuttcap%
\pgfsetroundjoin%
\definecolor{currentfill}{rgb}{0.000000,0.000000,0.000000}%
\pgfsetfillcolor{currentfill}%
\pgfsetlinewidth{0.501875pt}%
\definecolor{currentstroke}{rgb}{0.000000,0.000000,0.000000}%
\pgfsetstrokecolor{currentstroke}%
\pgfsetdash{}{0pt}%
\pgfsys@defobject{currentmarker}{\pgfqpoint{0.000000in}{-0.020833in}}{\pgfqpoint{0.000000in}{0.000000in}}{%
\pgfpathmoveto{\pgfqpoint{0.000000in}{0.000000in}}%
\pgfpathlineto{\pgfqpoint{0.000000in}{-0.020833in}}%
\pgfusepath{stroke,fill}%
}%
\begin{pgfscope}%
\pgfsys@transformshift{2.757538in}{4.479825in}%
\pgfsys@useobject{currentmarker}{}%
\end{pgfscope}%
\end{pgfscope}%
\begin{pgfscope}%
\pgfsetbuttcap%
\pgfsetroundjoin%
\definecolor{currentfill}{rgb}{0.000000,0.000000,0.000000}%
\pgfsetfillcolor{currentfill}%
\pgfsetlinewidth{0.501875pt}%
\definecolor{currentstroke}{rgb}{0.000000,0.000000,0.000000}%
\pgfsetstrokecolor{currentstroke}%
\pgfsetdash{}{0pt}%
\pgfsys@defobject{currentmarker}{\pgfqpoint{0.000000in}{0.000000in}}{\pgfqpoint{0.000000in}{0.020833in}}{%
\pgfpathmoveto{\pgfqpoint{0.000000in}{0.000000in}}%
\pgfpathlineto{\pgfqpoint{0.000000in}{0.020833in}}%
\pgfusepath{stroke,fill}%
}%
\begin{pgfscope}%
\pgfsys@transformshift{2.853613in}{4.012575in}%
\pgfsys@useobject{currentmarker}{}%
\end{pgfscope}%
\end{pgfscope}%
\begin{pgfscope}%
\pgfsetbuttcap%
\pgfsetroundjoin%
\definecolor{currentfill}{rgb}{0.000000,0.000000,0.000000}%
\pgfsetfillcolor{currentfill}%
\pgfsetlinewidth{0.501875pt}%
\definecolor{currentstroke}{rgb}{0.000000,0.000000,0.000000}%
\pgfsetstrokecolor{currentstroke}%
\pgfsetdash{}{0pt}%
\pgfsys@defobject{currentmarker}{\pgfqpoint{0.000000in}{-0.020833in}}{\pgfqpoint{0.000000in}{0.000000in}}{%
\pgfpathmoveto{\pgfqpoint{0.000000in}{0.000000in}}%
\pgfpathlineto{\pgfqpoint{0.000000in}{-0.020833in}}%
\pgfusepath{stroke,fill}%
}%
\begin{pgfscope}%
\pgfsys@transformshift{2.853613in}{4.479825in}%
\pgfsys@useobject{currentmarker}{}%
\end{pgfscope}%
\end{pgfscope}%
\begin{pgfscope}%
\pgfsetbuttcap%
\pgfsetroundjoin%
\definecolor{currentfill}{rgb}{0.000000,0.000000,0.000000}%
\pgfsetfillcolor{currentfill}%
\pgfsetlinewidth{0.501875pt}%
\definecolor{currentstroke}{rgb}{0.000000,0.000000,0.000000}%
\pgfsetstrokecolor{currentstroke}%
\pgfsetdash{}{0pt}%
\pgfsys@defobject{currentmarker}{\pgfqpoint{0.000000in}{0.000000in}}{\pgfqpoint{0.000000in}{0.020833in}}{%
\pgfpathmoveto{\pgfqpoint{0.000000in}{0.000000in}}%
\pgfpathlineto{\pgfqpoint{0.000000in}{0.020833in}}%
\pgfusepath{stroke,fill}%
}%
\begin{pgfscope}%
\pgfsys@transformshift{2.949688in}{4.012575in}%
\pgfsys@useobject{currentmarker}{}%
\end{pgfscope}%
\end{pgfscope}%
\begin{pgfscope}%
\pgfsetbuttcap%
\pgfsetroundjoin%
\definecolor{currentfill}{rgb}{0.000000,0.000000,0.000000}%
\pgfsetfillcolor{currentfill}%
\pgfsetlinewidth{0.501875pt}%
\definecolor{currentstroke}{rgb}{0.000000,0.000000,0.000000}%
\pgfsetstrokecolor{currentstroke}%
\pgfsetdash{}{0pt}%
\pgfsys@defobject{currentmarker}{\pgfqpoint{0.000000in}{-0.020833in}}{\pgfqpoint{0.000000in}{0.000000in}}{%
\pgfpathmoveto{\pgfqpoint{0.000000in}{0.000000in}}%
\pgfpathlineto{\pgfqpoint{0.000000in}{-0.020833in}}%
\pgfusepath{stroke,fill}%
}%
\begin{pgfscope}%
\pgfsys@transformshift{2.949688in}{4.479825in}%
\pgfsys@useobject{currentmarker}{}%
\end{pgfscope}%
\end{pgfscope}%
\begin{pgfscope}%
\pgfsetbuttcap%
\pgfsetroundjoin%
\definecolor{currentfill}{rgb}{0.000000,0.000000,0.000000}%
\pgfsetfillcolor{currentfill}%
\pgfsetlinewidth{0.501875pt}%
\definecolor{currentstroke}{rgb}{0.000000,0.000000,0.000000}%
\pgfsetstrokecolor{currentstroke}%
\pgfsetdash{}{0pt}%
\pgfsys@defobject{currentmarker}{\pgfqpoint{0.000000in}{0.000000in}}{\pgfqpoint{0.000000in}{0.020833in}}{%
\pgfpathmoveto{\pgfqpoint{0.000000in}{0.000000in}}%
\pgfpathlineto{\pgfqpoint{0.000000in}{0.020833in}}%
\pgfusepath{stroke,fill}%
}%
\begin{pgfscope}%
\pgfsys@transformshift{3.141838in}{4.012575in}%
\pgfsys@useobject{currentmarker}{}%
\end{pgfscope}%
\end{pgfscope}%
\begin{pgfscope}%
\pgfsetbuttcap%
\pgfsetroundjoin%
\definecolor{currentfill}{rgb}{0.000000,0.000000,0.000000}%
\pgfsetfillcolor{currentfill}%
\pgfsetlinewidth{0.501875pt}%
\definecolor{currentstroke}{rgb}{0.000000,0.000000,0.000000}%
\pgfsetstrokecolor{currentstroke}%
\pgfsetdash{}{0pt}%
\pgfsys@defobject{currentmarker}{\pgfqpoint{0.000000in}{-0.020833in}}{\pgfqpoint{0.000000in}{0.000000in}}{%
\pgfpathmoveto{\pgfqpoint{0.000000in}{0.000000in}}%
\pgfpathlineto{\pgfqpoint{0.000000in}{-0.020833in}}%
\pgfusepath{stroke,fill}%
}%
\begin{pgfscope}%
\pgfsys@transformshift{3.141838in}{4.479825in}%
\pgfsys@useobject{currentmarker}{}%
\end{pgfscope}%
\end{pgfscope}%
\begin{pgfscope}%
\pgfsetbuttcap%
\pgfsetroundjoin%
\definecolor{currentfill}{rgb}{0.000000,0.000000,0.000000}%
\pgfsetfillcolor{currentfill}%
\pgfsetlinewidth{0.501875pt}%
\definecolor{currentstroke}{rgb}{0.000000,0.000000,0.000000}%
\pgfsetstrokecolor{currentstroke}%
\pgfsetdash{}{0pt}%
\pgfsys@defobject{currentmarker}{\pgfqpoint{0.000000in}{0.000000in}}{\pgfqpoint{0.000000in}{0.020833in}}{%
\pgfpathmoveto{\pgfqpoint{0.000000in}{0.000000in}}%
\pgfpathlineto{\pgfqpoint{0.000000in}{0.020833in}}%
\pgfusepath{stroke,fill}%
}%
\begin{pgfscope}%
\pgfsys@transformshift{3.237913in}{4.012575in}%
\pgfsys@useobject{currentmarker}{}%
\end{pgfscope}%
\end{pgfscope}%
\begin{pgfscope}%
\pgfsetbuttcap%
\pgfsetroundjoin%
\definecolor{currentfill}{rgb}{0.000000,0.000000,0.000000}%
\pgfsetfillcolor{currentfill}%
\pgfsetlinewidth{0.501875pt}%
\definecolor{currentstroke}{rgb}{0.000000,0.000000,0.000000}%
\pgfsetstrokecolor{currentstroke}%
\pgfsetdash{}{0pt}%
\pgfsys@defobject{currentmarker}{\pgfqpoint{0.000000in}{-0.020833in}}{\pgfqpoint{0.000000in}{0.000000in}}{%
\pgfpathmoveto{\pgfqpoint{0.000000in}{0.000000in}}%
\pgfpathlineto{\pgfqpoint{0.000000in}{-0.020833in}}%
\pgfusepath{stroke,fill}%
}%
\begin{pgfscope}%
\pgfsys@transformshift{3.237913in}{4.479825in}%
\pgfsys@useobject{currentmarker}{}%
\end{pgfscope}%
\end{pgfscope}%
\begin{pgfscope}%
\pgfsetbuttcap%
\pgfsetroundjoin%
\definecolor{currentfill}{rgb}{0.000000,0.000000,0.000000}%
\pgfsetfillcolor{currentfill}%
\pgfsetlinewidth{0.501875pt}%
\definecolor{currentstroke}{rgb}{0.000000,0.000000,0.000000}%
\pgfsetstrokecolor{currentstroke}%
\pgfsetdash{}{0pt}%
\pgfsys@defobject{currentmarker}{\pgfqpoint{0.000000in}{0.000000in}}{\pgfqpoint{0.000000in}{0.020833in}}{%
\pgfpathmoveto{\pgfqpoint{0.000000in}{0.000000in}}%
\pgfpathlineto{\pgfqpoint{0.000000in}{0.020833in}}%
\pgfusepath{stroke,fill}%
}%
\begin{pgfscope}%
\pgfsys@transformshift{3.333988in}{4.012575in}%
\pgfsys@useobject{currentmarker}{}%
\end{pgfscope}%
\end{pgfscope}%
\begin{pgfscope}%
\pgfsetbuttcap%
\pgfsetroundjoin%
\definecolor{currentfill}{rgb}{0.000000,0.000000,0.000000}%
\pgfsetfillcolor{currentfill}%
\pgfsetlinewidth{0.501875pt}%
\definecolor{currentstroke}{rgb}{0.000000,0.000000,0.000000}%
\pgfsetstrokecolor{currentstroke}%
\pgfsetdash{}{0pt}%
\pgfsys@defobject{currentmarker}{\pgfqpoint{0.000000in}{-0.020833in}}{\pgfqpoint{0.000000in}{0.000000in}}{%
\pgfpathmoveto{\pgfqpoint{0.000000in}{0.000000in}}%
\pgfpathlineto{\pgfqpoint{0.000000in}{-0.020833in}}%
\pgfusepath{stroke,fill}%
}%
\begin{pgfscope}%
\pgfsys@transformshift{3.333988in}{4.479825in}%
\pgfsys@useobject{currentmarker}{}%
\end{pgfscope}%
\end{pgfscope}%
\begin{pgfscope}%
\pgfsetbuttcap%
\pgfsetroundjoin%
\definecolor{currentfill}{rgb}{0.000000,0.000000,0.000000}%
\pgfsetfillcolor{currentfill}%
\pgfsetlinewidth{0.501875pt}%
\definecolor{currentstroke}{rgb}{0.000000,0.000000,0.000000}%
\pgfsetstrokecolor{currentstroke}%
\pgfsetdash{}{0pt}%
\pgfsys@defobject{currentmarker}{\pgfqpoint{0.000000in}{0.000000in}}{\pgfqpoint{0.000000in}{0.020833in}}{%
\pgfpathmoveto{\pgfqpoint{0.000000in}{0.000000in}}%
\pgfpathlineto{\pgfqpoint{0.000000in}{0.020833in}}%
\pgfusepath{stroke,fill}%
}%
\begin{pgfscope}%
\pgfsys@transformshift{3.430063in}{4.012575in}%
\pgfsys@useobject{currentmarker}{}%
\end{pgfscope}%
\end{pgfscope}%
\begin{pgfscope}%
\pgfsetbuttcap%
\pgfsetroundjoin%
\definecolor{currentfill}{rgb}{0.000000,0.000000,0.000000}%
\pgfsetfillcolor{currentfill}%
\pgfsetlinewidth{0.501875pt}%
\definecolor{currentstroke}{rgb}{0.000000,0.000000,0.000000}%
\pgfsetstrokecolor{currentstroke}%
\pgfsetdash{}{0pt}%
\pgfsys@defobject{currentmarker}{\pgfqpoint{0.000000in}{-0.020833in}}{\pgfqpoint{0.000000in}{0.000000in}}{%
\pgfpathmoveto{\pgfqpoint{0.000000in}{0.000000in}}%
\pgfpathlineto{\pgfqpoint{0.000000in}{-0.020833in}}%
\pgfusepath{stroke,fill}%
}%
\begin{pgfscope}%
\pgfsys@transformshift{3.430063in}{4.479825in}%
\pgfsys@useobject{currentmarker}{}%
\end{pgfscope}%
\end{pgfscope}%
\begin{pgfscope}%
\pgfsetbuttcap%
\pgfsetroundjoin%
\definecolor{currentfill}{rgb}{0.000000,0.000000,0.000000}%
\pgfsetfillcolor{currentfill}%
\pgfsetlinewidth{0.501875pt}%
\definecolor{currentstroke}{rgb}{0.000000,0.000000,0.000000}%
\pgfsetstrokecolor{currentstroke}%
\pgfsetdash{}{0pt}%
\pgfsys@defobject{currentmarker}{\pgfqpoint{0.000000in}{0.000000in}}{\pgfqpoint{0.000000in}{0.020833in}}{%
\pgfpathmoveto{\pgfqpoint{0.000000in}{0.000000in}}%
\pgfpathlineto{\pgfqpoint{0.000000in}{0.020833in}}%
\pgfusepath{stroke,fill}%
}%
\begin{pgfscope}%
\pgfsys@transformshift{3.622213in}{4.012575in}%
\pgfsys@useobject{currentmarker}{}%
\end{pgfscope}%
\end{pgfscope}%
\begin{pgfscope}%
\pgfsetbuttcap%
\pgfsetroundjoin%
\definecolor{currentfill}{rgb}{0.000000,0.000000,0.000000}%
\pgfsetfillcolor{currentfill}%
\pgfsetlinewidth{0.501875pt}%
\definecolor{currentstroke}{rgb}{0.000000,0.000000,0.000000}%
\pgfsetstrokecolor{currentstroke}%
\pgfsetdash{}{0pt}%
\pgfsys@defobject{currentmarker}{\pgfqpoint{0.000000in}{-0.020833in}}{\pgfqpoint{0.000000in}{0.000000in}}{%
\pgfpathmoveto{\pgfqpoint{0.000000in}{0.000000in}}%
\pgfpathlineto{\pgfqpoint{0.000000in}{-0.020833in}}%
\pgfusepath{stroke,fill}%
}%
\begin{pgfscope}%
\pgfsys@transformshift{3.622213in}{4.479825in}%
\pgfsys@useobject{currentmarker}{}%
\end{pgfscope}%
\end{pgfscope}%
\begin{pgfscope}%
\pgfsetbuttcap%
\pgfsetroundjoin%
\definecolor{currentfill}{rgb}{0.000000,0.000000,0.000000}%
\pgfsetfillcolor{currentfill}%
\pgfsetlinewidth{0.501875pt}%
\definecolor{currentstroke}{rgb}{0.000000,0.000000,0.000000}%
\pgfsetstrokecolor{currentstroke}%
\pgfsetdash{}{0pt}%
\pgfsys@defobject{currentmarker}{\pgfqpoint{0.000000in}{0.000000in}}{\pgfqpoint{0.000000in}{0.020833in}}{%
\pgfpathmoveto{\pgfqpoint{0.000000in}{0.000000in}}%
\pgfpathlineto{\pgfqpoint{0.000000in}{0.020833in}}%
\pgfusepath{stroke,fill}%
}%
\begin{pgfscope}%
\pgfsys@transformshift{3.718289in}{4.012575in}%
\pgfsys@useobject{currentmarker}{}%
\end{pgfscope}%
\end{pgfscope}%
\begin{pgfscope}%
\pgfsetbuttcap%
\pgfsetroundjoin%
\definecolor{currentfill}{rgb}{0.000000,0.000000,0.000000}%
\pgfsetfillcolor{currentfill}%
\pgfsetlinewidth{0.501875pt}%
\definecolor{currentstroke}{rgb}{0.000000,0.000000,0.000000}%
\pgfsetstrokecolor{currentstroke}%
\pgfsetdash{}{0pt}%
\pgfsys@defobject{currentmarker}{\pgfqpoint{0.000000in}{-0.020833in}}{\pgfqpoint{0.000000in}{0.000000in}}{%
\pgfpathmoveto{\pgfqpoint{0.000000in}{0.000000in}}%
\pgfpathlineto{\pgfqpoint{0.000000in}{-0.020833in}}%
\pgfusepath{stroke,fill}%
}%
\begin{pgfscope}%
\pgfsys@transformshift{3.718289in}{4.479825in}%
\pgfsys@useobject{currentmarker}{}%
\end{pgfscope}%
\end{pgfscope}%
\begin{pgfscope}%
\pgfsetbuttcap%
\pgfsetroundjoin%
\definecolor{currentfill}{rgb}{0.000000,0.000000,0.000000}%
\pgfsetfillcolor{currentfill}%
\pgfsetlinewidth{0.501875pt}%
\definecolor{currentstroke}{rgb}{0.000000,0.000000,0.000000}%
\pgfsetstrokecolor{currentstroke}%
\pgfsetdash{}{0pt}%
\pgfsys@defobject{currentmarker}{\pgfqpoint{0.000000in}{0.000000in}}{\pgfqpoint{0.000000in}{0.020833in}}{%
\pgfpathmoveto{\pgfqpoint{0.000000in}{0.000000in}}%
\pgfpathlineto{\pgfqpoint{0.000000in}{0.020833in}}%
\pgfusepath{stroke,fill}%
}%
\begin{pgfscope}%
\pgfsys@transformshift{3.814364in}{4.012575in}%
\pgfsys@useobject{currentmarker}{}%
\end{pgfscope}%
\end{pgfscope}%
\begin{pgfscope}%
\pgfsetbuttcap%
\pgfsetroundjoin%
\definecolor{currentfill}{rgb}{0.000000,0.000000,0.000000}%
\pgfsetfillcolor{currentfill}%
\pgfsetlinewidth{0.501875pt}%
\definecolor{currentstroke}{rgb}{0.000000,0.000000,0.000000}%
\pgfsetstrokecolor{currentstroke}%
\pgfsetdash{}{0pt}%
\pgfsys@defobject{currentmarker}{\pgfqpoint{0.000000in}{-0.020833in}}{\pgfqpoint{0.000000in}{0.000000in}}{%
\pgfpathmoveto{\pgfqpoint{0.000000in}{0.000000in}}%
\pgfpathlineto{\pgfqpoint{0.000000in}{-0.020833in}}%
\pgfusepath{stroke,fill}%
}%
\begin{pgfscope}%
\pgfsys@transformshift{3.814364in}{4.479825in}%
\pgfsys@useobject{currentmarker}{}%
\end{pgfscope}%
\end{pgfscope}%
\begin{pgfscope}%
\pgfsetbuttcap%
\pgfsetroundjoin%
\definecolor{currentfill}{rgb}{0.000000,0.000000,0.000000}%
\pgfsetfillcolor{currentfill}%
\pgfsetlinewidth{0.501875pt}%
\definecolor{currentstroke}{rgb}{0.000000,0.000000,0.000000}%
\pgfsetstrokecolor{currentstroke}%
\pgfsetdash{}{0pt}%
\pgfsys@defobject{currentmarker}{\pgfqpoint{0.000000in}{0.000000in}}{\pgfqpoint{0.000000in}{0.020833in}}{%
\pgfpathmoveto{\pgfqpoint{0.000000in}{0.000000in}}%
\pgfpathlineto{\pgfqpoint{0.000000in}{0.020833in}}%
\pgfusepath{stroke,fill}%
}%
\begin{pgfscope}%
\pgfsys@transformshift{3.910439in}{4.012575in}%
\pgfsys@useobject{currentmarker}{}%
\end{pgfscope}%
\end{pgfscope}%
\begin{pgfscope}%
\pgfsetbuttcap%
\pgfsetroundjoin%
\definecolor{currentfill}{rgb}{0.000000,0.000000,0.000000}%
\pgfsetfillcolor{currentfill}%
\pgfsetlinewidth{0.501875pt}%
\definecolor{currentstroke}{rgb}{0.000000,0.000000,0.000000}%
\pgfsetstrokecolor{currentstroke}%
\pgfsetdash{}{0pt}%
\pgfsys@defobject{currentmarker}{\pgfqpoint{0.000000in}{-0.020833in}}{\pgfqpoint{0.000000in}{0.000000in}}{%
\pgfpathmoveto{\pgfqpoint{0.000000in}{0.000000in}}%
\pgfpathlineto{\pgfqpoint{0.000000in}{-0.020833in}}%
\pgfusepath{stroke,fill}%
}%
\begin{pgfscope}%
\pgfsys@transformshift{3.910439in}{4.479825in}%
\pgfsys@useobject{currentmarker}{}%
\end{pgfscope}%
\end{pgfscope}%
\begin{pgfscope}%
\pgfsetbuttcap%
\pgfsetroundjoin%
\definecolor{currentfill}{rgb}{0.000000,0.000000,0.000000}%
\pgfsetfillcolor{currentfill}%
\pgfsetlinewidth{0.501875pt}%
\definecolor{currentstroke}{rgb}{0.000000,0.000000,0.000000}%
\pgfsetstrokecolor{currentstroke}%
\pgfsetdash{}{0pt}%
\pgfsys@defobject{currentmarker}{\pgfqpoint{0.000000in}{0.000000in}}{\pgfqpoint{0.000000in}{0.020833in}}{%
\pgfpathmoveto{\pgfqpoint{0.000000in}{0.000000in}}%
\pgfpathlineto{\pgfqpoint{0.000000in}{0.020833in}}%
\pgfusepath{stroke,fill}%
}%
\begin{pgfscope}%
\pgfsys@transformshift{4.102589in}{4.012575in}%
\pgfsys@useobject{currentmarker}{}%
\end{pgfscope}%
\end{pgfscope}%
\begin{pgfscope}%
\pgfsetbuttcap%
\pgfsetroundjoin%
\definecolor{currentfill}{rgb}{0.000000,0.000000,0.000000}%
\pgfsetfillcolor{currentfill}%
\pgfsetlinewidth{0.501875pt}%
\definecolor{currentstroke}{rgb}{0.000000,0.000000,0.000000}%
\pgfsetstrokecolor{currentstroke}%
\pgfsetdash{}{0pt}%
\pgfsys@defobject{currentmarker}{\pgfqpoint{0.000000in}{-0.020833in}}{\pgfqpoint{0.000000in}{0.000000in}}{%
\pgfpathmoveto{\pgfqpoint{0.000000in}{0.000000in}}%
\pgfpathlineto{\pgfqpoint{0.000000in}{-0.020833in}}%
\pgfusepath{stroke,fill}%
}%
\begin{pgfscope}%
\pgfsys@transformshift{4.102589in}{4.479825in}%
\pgfsys@useobject{currentmarker}{}%
\end{pgfscope}%
\end{pgfscope}%
\begin{pgfscope}%
\pgfsetbuttcap%
\pgfsetroundjoin%
\definecolor{currentfill}{rgb}{0.000000,0.000000,0.000000}%
\pgfsetfillcolor{currentfill}%
\pgfsetlinewidth{0.501875pt}%
\definecolor{currentstroke}{rgb}{0.000000,0.000000,0.000000}%
\pgfsetstrokecolor{currentstroke}%
\pgfsetdash{}{0pt}%
\pgfsys@defobject{currentmarker}{\pgfqpoint{0.000000in}{0.000000in}}{\pgfqpoint{0.000000in}{0.020833in}}{%
\pgfpathmoveto{\pgfqpoint{0.000000in}{0.000000in}}%
\pgfpathlineto{\pgfqpoint{0.000000in}{0.020833in}}%
\pgfusepath{stroke,fill}%
}%
\begin{pgfscope}%
\pgfsys@transformshift{4.198664in}{4.012575in}%
\pgfsys@useobject{currentmarker}{}%
\end{pgfscope}%
\end{pgfscope}%
\begin{pgfscope}%
\pgfsetbuttcap%
\pgfsetroundjoin%
\definecolor{currentfill}{rgb}{0.000000,0.000000,0.000000}%
\pgfsetfillcolor{currentfill}%
\pgfsetlinewidth{0.501875pt}%
\definecolor{currentstroke}{rgb}{0.000000,0.000000,0.000000}%
\pgfsetstrokecolor{currentstroke}%
\pgfsetdash{}{0pt}%
\pgfsys@defobject{currentmarker}{\pgfqpoint{0.000000in}{-0.020833in}}{\pgfqpoint{0.000000in}{0.000000in}}{%
\pgfpathmoveto{\pgfqpoint{0.000000in}{0.000000in}}%
\pgfpathlineto{\pgfqpoint{0.000000in}{-0.020833in}}%
\pgfusepath{stroke,fill}%
}%
\begin{pgfscope}%
\pgfsys@transformshift{4.198664in}{4.479825in}%
\pgfsys@useobject{currentmarker}{}%
\end{pgfscope}%
\end{pgfscope}%
\begin{pgfscope}%
\pgfsetbuttcap%
\pgfsetroundjoin%
\definecolor{currentfill}{rgb}{0.000000,0.000000,0.000000}%
\pgfsetfillcolor{currentfill}%
\pgfsetlinewidth{0.501875pt}%
\definecolor{currentstroke}{rgb}{0.000000,0.000000,0.000000}%
\pgfsetstrokecolor{currentstroke}%
\pgfsetdash{}{0pt}%
\pgfsys@defobject{currentmarker}{\pgfqpoint{0.000000in}{0.000000in}}{\pgfqpoint{0.000000in}{0.020833in}}{%
\pgfpathmoveto{\pgfqpoint{0.000000in}{0.000000in}}%
\pgfpathlineto{\pgfqpoint{0.000000in}{0.020833in}}%
\pgfusepath{stroke,fill}%
}%
\begin{pgfscope}%
\pgfsys@transformshift{4.294739in}{4.012575in}%
\pgfsys@useobject{currentmarker}{}%
\end{pgfscope}%
\end{pgfscope}%
\begin{pgfscope}%
\pgfsetbuttcap%
\pgfsetroundjoin%
\definecolor{currentfill}{rgb}{0.000000,0.000000,0.000000}%
\pgfsetfillcolor{currentfill}%
\pgfsetlinewidth{0.501875pt}%
\definecolor{currentstroke}{rgb}{0.000000,0.000000,0.000000}%
\pgfsetstrokecolor{currentstroke}%
\pgfsetdash{}{0pt}%
\pgfsys@defobject{currentmarker}{\pgfqpoint{0.000000in}{-0.020833in}}{\pgfqpoint{0.000000in}{0.000000in}}{%
\pgfpathmoveto{\pgfqpoint{0.000000in}{0.000000in}}%
\pgfpathlineto{\pgfqpoint{0.000000in}{-0.020833in}}%
\pgfusepath{stroke,fill}%
}%
\begin{pgfscope}%
\pgfsys@transformshift{4.294739in}{4.479825in}%
\pgfsys@useobject{currentmarker}{}%
\end{pgfscope}%
\end{pgfscope}%
\begin{pgfscope}%
\pgfsetbuttcap%
\pgfsetroundjoin%
\definecolor{currentfill}{rgb}{0.000000,0.000000,0.000000}%
\pgfsetfillcolor{currentfill}%
\pgfsetlinewidth{0.501875pt}%
\definecolor{currentstroke}{rgb}{0.000000,0.000000,0.000000}%
\pgfsetstrokecolor{currentstroke}%
\pgfsetdash{}{0pt}%
\pgfsys@defobject{currentmarker}{\pgfqpoint{0.000000in}{0.000000in}}{\pgfqpoint{0.000000in}{0.020833in}}{%
\pgfpathmoveto{\pgfqpoint{0.000000in}{0.000000in}}%
\pgfpathlineto{\pgfqpoint{0.000000in}{0.020833in}}%
\pgfusepath{stroke,fill}%
}%
\begin{pgfscope}%
\pgfsys@transformshift{4.390814in}{4.012575in}%
\pgfsys@useobject{currentmarker}{}%
\end{pgfscope}%
\end{pgfscope}%
\begin{pgfscope}%
\pgfsetbuttcap%
\pgfsetroundjoin%
\definecolor{currentfill}{rgb}{0.000000,0.000000,0.000000}%
\pgfsetfillcolor{currentfill}%
\pgfsetlinewidth{0.501875pt}%
\definecolor{currentstroke}{rgb}{0.000000,0.000000,0.000000}%
\pgfsetstrokecolor{currentstroke}%
\pgfsetdash{}{0pt}%
\pgfsys@defobject{currentmarker}{\pgfqpoint{0.000000in}{-0.020833in}}{\pgfqpoint{0.000000in}{0.000000in}}{%
\pgfpathmoveto{\pgfqpoint{0.000000in}{0.000000in}}%
\pgfpathlineto{\pgfqpoint{0.000000in}{-0.020833in}}%
\pgfusepath{stroke,fill}%
}%
\begin{pgfscope}%
\pgfsys@transformshift{4.390814in}{4.479825in}%
\pgfsys@useobject{currentmarker}{}%
\end{pgfscope}%
\end{pgfscope}%
\begin{pgfscope}%
\pgfsetbuttcap%
\pgfsetroundjoin%
\definecolor{currentfill}{rgb}{0.000000,0.000000,0.000000}%
\pgfsetfillcolor{currentfill}%
\pgfsetlinewidth{0.501875pt}%
\definecolor{currentstroke}{rgb}{0.000000,0.000000,0.000000}%
\pgfsetstrokecolor{currentstroke}%
\pgfsetdash{}{0pt}%
\pgfsys@defobject{currentmarker}{\pgfqpoint{0.000000in}{0.000000in}}{\pgfqpoint{0.000000in}{0.020833in}}{%
\pgfpathmoveto{\pgfqpoint{0.000000in}{0.000000in}}%
\pgfpathlineto{\pgfqpoint{0.000000in}{0.020833in}}%
\pgfusepath{stroke,fill}%
}%
\begin{pgfscope}%
\pgfsys@transformshift{4.582964in}{4.012575in}%
\pgfsys@useobject{currentmarker}{}%
\end{pgfscope}%
\end{pgfscope}%
\begin{pgfscope}%
\pgfsetbuttcap%
\pgfsetroundjoin%
\definecolor{currentfill}{rgb}{0.000000,0.000000,0.000000}%
\pgfsetfillcolor{currentfill}%
\pgfsetlinewidth{0.501875pt}%
\definecolor{currentstroke}{rgb}{0.000000,0.000000,0.000000}%
\pgfsetstrokecolor{currentstroke}%
\pgfsetdash{}{0pt}%
\pgfsys@defobject{currentmarker}{\pgfqpoint{0.000000in}{-0.020833in}}{\pgfqpoint{0.000000in}{0.000000in}}{%
\pgfpathmoveto{\pgfqpoint{0.000000in}{0.000000in}}%
\pgfpathlineto{\pgfqpoint{0.000000in}{-0.020833in}}%
\pgfusepath{stroke,fill}%
}%
\begin{pgfscope}%
\pgfsys@transformshift{4.582964in}{4.479825in}%
\pgfsys@useobject{currentmarker}{}%
\end{pgfscope}%
\end{pgfscope}%
\begin{pgfscope}%
\pgfsetbuttcap%
\pgfsetroundjoin%
\definecolor{currentfill}{rgb}{0.000000,0.000000,0.000000}%
\pgfsetfillcolor{currentfill}%
\pgfsetlinewidth{0.501875pt}%
\definecolor{currentstroke}{rgb}{0.000000,0.000000,0.000000}%
\pgfsetstrokecolor{currentstroke}%
\pgfsetdash{}{0pt}%
\pgfsys@defobject{currentmarker}{\pgfqpoint{0.000000in}{0.000000in}}{\pgfqpoint{0.041667in}{0.000000in}}{%
\pgfpathmoveto{\pgfqpoint{0.000000in}{0.000000in}}%
\pgfpathlineto{\pgfqpoint{0.041667in}{0.000000in}}%
\pgfusepath{stroke,fill}%
}%
\begin{pgfscope}%
\pgfsys@transformshift{0.444748in}{4.038342in}%
\pgfsys@useobject{currentmarker}{}%
\end{pgfscope}%
\end{pgfscope}%
\begin{pgfscope}%
\pgfsetbuttcap%
\pgfsetroundjoin%
\definecolor{currentfill}{rgb}{0.000000,0.000000,0.000000}%
\pgfsetfillcolor{currentfill}%
\pgfsetlinewidth{0.501875pt}%
\definecolor{currentstroke}{rgb}{0.000000,0.000000,0.000000}%
\pgfsetstrokecolor{currentstroke}%
\pgfsetdash{}{0pt}%
\pgfsys@defobject{currentmarker}{\pgfqpoint{-0.041667in}{0.000000in}}{\pgfqpoint{-0.000000in}{0.000000in}}{%
\pgfpathmoveto{\pgfqpoint{-0.000000in}{0.000000in}}%
\pgfpathlineto{\pgfqpoint{-0.041667in}{0.000000in}}%
\pgfusepath{stroke,fill}%
}%
\begin{pgfscope}%
\pgfsys@transformshift{4.676167in}{4.038342in}%
\pgfsys@useobject{currentmarker}{}%
\end{pgfscope}%
\end{pgfscope}%
\begin{pgfscope}%
\definecolor{textcolor}{rgb}{0.000000,0.000000,0.000000}%
\pgfsetstrokecolor{textcolor}%
\pgfsetfillcolor{textcolor}%
\pgftext[x=0.326693in, y=3.990124in, left, base]{\color{textcolor}\rmfamily\fontsize{10.000000}{12.000000}\selectfont \(\displaystyle {0}\)}%
\end{pgfscope}%
\begin{pgfscope}%
\pgfsetbuttcap%
\pgfsetroundjoin%
\definecolor{currentfill}{rgb}{0.000000,0.000000,0.000000}%
\pgfsetfillcolor{currentfill}%
\pgfsetlinewidth{0.501875pt}%
\definecolor{currentstroke}{rgb}{0.000000,0.000000,0.000000}%
\pgfsetstrokecolor{currentstroke}%
\pgfsetdash{}{0pt}%
\pgfsys@defobject{currentmarker}{\pgfqpoint{0.000000in}{0.000000in}}{\pgfqpoint{0.041667in}{0.000000in}}{%
\pgfpathmoveto{\pgfqpoint{0.000000in}{0.000000in}}%
\pgfpathlineto{\pgfqpoint{0.041667in}{0.000000in}}%
\pgfusepath{stroke,fill}%
}%
\begin{pgfscope}%
\pgfsys@transformshift{0.444748in}{4.283979in}%
\pgfsys@useobject{currentmarker}{}%
\end{pgfscope}%
\end{pgfscope}%
\begin{pgfscope}%
\pgfsetbuttcap%
\pgfsetroundjoin%
\definecolor{currentfill}{rgb}{0.000000,0.000000,0.000000}%
\pgfsetfillcolor{currentfill}%
\pgfsetlinewidth{0.501875pt}%
\definecolor{currentstroke}{rgb}{0.000000,0.000000,0.000000}%
\pgfsetstrokecolor{currentstroke}%
\pgfsetdash{}{0pt}%
\pgfsys@defobject{currentmarker}{\pgfqpoint{-0.041667in}{0.000000in}}{\pgfqpoint{-0.000000in}{0.000000in}}{%
\pgfpathmoveto{\pgfqpoint{-0.000000in}{0.000000in}}%
\pgfpathlineto{\pgfqpoint{-0.041667in}{0.000000in}}%
\pgfusepath{stroke,fill}%
}%
\begin{pgfscope}%
\pgfsys@transformshift{4.676167in}{4.283979in}%
\pgfsys@useobject{currentmarker}{}%
\end{pgfscope}%
\end{pgfscope}%
\begin{pgfscope}%
\definecolor{textcolor}{rgb}{0.000000,0.000000,0.000000}%
\pgfsetstrokecolor{textcolor}%
\pgfsetfillcolor{textcolor}%
\pgftext[x=0.257248in, y=4.235761in, left, base]{\color{textcolor}\rmfamily\fontsize{10.000000}{12.000000}\selectfont \(\displaystyle {50}\)}%
\end{pgfscope}%
\begin{pgfscope}%
\pgfsetbuttcap%
\pgfsetroundjoin%
\definecolor{currentfill}{rgb}{0.000000,0.000000,0.000000}%
\pgfsetfillcolor{currentfill}%
\pgfsetlinewidth{0.501875pt}%
\definecolor{currentstroke}{rgb}{0.000000,0.000000,0.000000}%
\pgfsetstrokecolor{currentstroke}%
\pgfsetdash{}{0pt}%
\pgfsys@defobject{currentmarker}{\pgfqpoint{0.000000in}{0.000000in}}{\pgfqpoint{0.020833in}{0.000000in}}{%
\pgfpathmoveto{\pgfqpoint{0.000000in}{0.000000in}}%
\pgfpathlineto{\pgfqpoint{0.020833in}{0.000000in}}%
\pgfusepath{stroke,fill}%
}%
\begin{pgfscope}%
\pgfsys@transformshift{0.444748in}{4.087469in}%
\pgfsys@useobject{currentmarker}{}%
\end{pgfscope}%
\end{pgfscope}%
\begin{pgfscope}%
\pgfsetbuttcap%
\pgfsetroundjoin%
\definecolor{currentfill}{rgb}{0.000000,0.000000,0.000000}%
\pgfsetfillcolor{currentfill}%
\pgfsetlinewidth{0.501875pt}%
\definecolor{currentstroke}{rgb}{0.000000,0.000000,0.000000}%
\pgfsetstrokecolor{currentstroke}%
\pgfsetdash{}{0pt}%
\pgfsys@defobject{currentmarker}{\pgfqpoint{-0.020833in}{0.000000in}}{\pgfqpoint{-0.000000in}{0.000000in}}{%
\pgfpathmoveto{\pgfqpoint{-0.000000in}{0.000000in}}%
\pgfpathlineto{\pgfqpoint{-0.020833in}{0.000000in}}%
\pgfusepath{stroke,fill}%
}%
\begin{pgfscope}%
\pgfsys@transformshift{4.676167in}{4.087469in}%
\pgfsys@useobject{currentmarker}{}%
\end{pgfscope}%
\end{pgfscope}%
\begin{pgfscope}%
\pgfsetbuttcap%
\pgfsetroundjoin%
\definecolor{currentfill}{rgb}{0.000000,0.000000,0.000000}%
\pgfsetfillcolor{currentfill}%
\pgfsetlinewidth{0.501875pt}%
\definecolor{currentstroke}{rgb}{0.000000,0.000000,0.000000}%
\pgfsetstrokecolor{currentstroke}%
\pgfsetdash{}{0pt}%
\pgfsys@defobject{currentmarker}{\pgfqpoint{0.000000in}{0.000000in}}{\pgfqpoint{0.020833in}{0.000000in}}{%
\pgfpathmoveto{\pgfqpoint{0.000000in}{0.000000in}}%
\pgfpathlineto{\pgfqpoint{0.020833in}{0.000000in}}%
\pgfusepath{stroke,fill}%
}%
\begin{pgfscope}%
\pgfsys@transformshift{0.444748in}{4.136597in}%
\pgfsys@useobject{currentmarker}{}%
\end{pgfscope}%
\end{pgfscope}%
\begin{pgfscope}%
\pgfsetbuttcap%
\pgfsetroundjoin%
\definecolor{currentfill}{rgb}{0.000000,0.000000,0.000000}%
\pgfsetfillcolor{currentfill}%
\pgfsetlinewidth{0.501875pt}%
\definecolor{currentstroke}{rgb}{0.000000,0.000000,0.000000}%
\pgfsetstrokecolor{currentstroke}%
\pgfsetdash{}{0pt}%
\pgfsys@defobject{currentmarker}{\pgfqpoint{-0.020833in}{0.000000in}}{\pgfqpoint{-0.000000in}{0.000000in}}{%
\pgfpathmoveto{\pgfqpoint{-0.000000in}{0.000000in}}%
\pgfpathlineto{\pgfqpoint{-0.020833in}{0.000000in}}%
\pgfusepath{stroke,fill}%
}%
\begin{pgfscope}%
\pgfsys@transformshift{4.676167in}{4.136597in}%
\pgfsys@useobject{currentmarker}{}%
\end{pgfscope}%
\end{pgfscope}%
\begin{pgfscope}%
\pgfsetbuttcap%
\pgfsetroundjoin%
\definecolor{currentfill}{rgb}{0.000000,0.000000,0.000000}%
\pgfsetfillcolor{currentfill}%
\pgfsetlinewidth{0.501875pt}%
\definecolor{currentstroke}{rgb}{0.000000,0.000000,0.000000}%
\pgfsetstrokecolor{currentstroke}%
\pgfsetdash{}{0pt}%
\pgfsys@defobject{currentmarker}{\pgfqpoint{0.000000in}{0.000000in}}{\pgfqpoint{0.020833in}{0.000000in}}{%
\pgfpathmoveto{\pgfqpoint{0.000000in}{0.000000in}}%
\pgfpathlineto{\pgfqpoint{0.020833in}{0.000000in}}%
\pgfusepath{stroke,fill}%
}%
\begin{pgfscope}%
\pgfsys@transformshift{0.444748in}{4.185724in}%
\pgfsys@useobject{currentmarker}{}%
\end{pgfscope}%
\end{pgfscope}%
\begin{pgfscope}%
\pgfsetbuttcap%
\pgfsetroundjoin%
\definecolor{currentfill}{rgb}{0.000000,0.000000,0.000000}%
\pgfsetfillcolor{currentfill}%
\pgfsetlinewidth{0.501875pt}%
\definecolor{currentstroke}{rgb}{0.000000,0.000000,0.000000}%
\pgfsetstrokecolor{currentstroke}%
\pgfsetdash{}{0pt}%
\pgfsys@defobject{currentmarker}{\pgfqpoint{-0.020833in}{0.000000in}}{\pgfqpoint{-0.000000in}{0.000000in}}{%
\pgfpathmoveto{\pgfqpoint{-0.000000in}{0.000000in}}%
\pgfpathlineto{\pgfqpoint{-0.020833in}{0.000000in}}%
\pgfusepath{stroke,fill}%
}%
\begin{pgfscope}%
\pgfsys@transformshift{4.676167in}{4.185724in}%
\pgfsys@useobject{currentmarker}{}%
\end{pgfscope}%
\end{pgfscope}%
\begin{pgfscope}%
\pgfsetbuttcap%
\pgfsetroundjoin%
\definecolor{currentfill}{rgb}{0.000000,0.000000,0.000000}%
\pgfsetfillcolor{currentfill}%
\pgfsetlinewidth{0.501875pt}%
\definecolor{currentstroke}{rgb}{0.000000,0.000000,0.000000}%
\pgfsetstrokecolor{currentstroke}%
\pgfsetdash{}{0pt}%
\pgfsys@defobject{currentmarker}{\pgfqpoint{0.000000in}{0.000000in}}{\pgfqpoint{0.020833in}{0.000000in}}{%
\pgfpathmoveto{\pgfqpoint{0.000000in}{0.000000in}}%
\pgfpathlineto{\pgfqpoint{0.020833in}{0.000000in}}%
\pgfusepath{stroke,fill}%
}%
\begin{pgfscope}%
\pgfsys@transformshift{0.444748in}{4.234851in}%
\pgfsys@useobject{currentmarker}{}%
\end{pgfscope}%
\end{pgfscope}%
\begin{pgfscope}%
\pgfsetbuttcap%
\pgfsetroundjoin%
\definecolor{currentfill}{rgb}{0.000000,0.000000,0.000000}%
\pgfsetfillcolor{currentfill}%
\pgfsetlinewidth{0.501875pt}%
\definecolor{currentstroke}{rgb}{0.000000,0.000000,0.000000}%
\pgfsetstrokecolor{currentstroke}%
\pgfsetdash{}{0pt}%
\pgfsys@defobject{currentmarker}{\pgfqpoint{-0.020833in}{0.000000in}}{\pgfqpoint{-0.000000in}{0.000000in}}{%
\pgfpathmoveto{\pgfqpoint{-0.000000in}{0.000000in}}%
\pgfpathlineto{\pgfqpoint{-0.020833in}{0.000000in}}%
\pgfusepath{stroke,fill}%
}%
\begin{pgfscope}%
\pgfsys@transformshift{4.676167in}{4.234851in}%
\pgfsys@useobject{currentmarker}{}%
\end{pgfscope}%
\end{pgfscope}%
\begin{pgfscope}%
\pgfsetbuttcap%
\pgfsetroundjoin%
\definecolor{currentfill}{rgb}{0.000000,0.000000,0.000000}%
\pgfsetfillcolor{currentfill}%
\pgfsetlinewidth{0.501875pt}%
\definecolor{currentstroke}{rgb}{0.000000,0.000000,0.000000}%
\pgfsetstrokecolor{currentstroke}%
\pgfsetdash{}{0pt}%
\pgfsys@defobject{currentmarker}{\pgfqpoint{0.000000in}{0.000000in}}{\pgfqpoint{0.020833in}{0.000000in}}{%
\pgfpathmoveto{\pgfqpoint{0.000000in}{0.000000in}}%
\pgfpathlineto{\pgfqpoint{0.020833in}{0.000000in}}%
\pgfusepath{stroke,fill}%
}%
\begin{pgfscope}%
\pgfsys@transformshift{0.444748in}{4.333106in}%
\pgfsys@useobject{currentmarker}{}%
\end{pgfscope}%
\end{pgfscope}%
\begin{pgfscope}%
\pgfsetbuttcap%
\pgfsetroundjoin%
\definecolor{currentfill}{rgb}{0.000000,0.000000,0.000000}%
\pgfsetfillcolor{currentfill}%
\pgfsetlinewidth{0.501875pt}%
\definecolor{currentstroke}{rgb}{0.000000,0.000000,0.000000}%
\pgfsetstrokecolor{currentstroke}%
\pgfsetdash{}{0pt}%
\pgfsys@defobject{currentmarker}{\pgfqpoint{-0.020833in}{0.000000in}}{\pgfqpoint{-0.000000in}{0.000000in}}{%
\pgfpathmoveto{\pgfqpoint{-0.000000in}{0.000000in}}%
\pgfpathlineto{\pgfqpoint{-0.020833in}{0.000000in}}%
\pgfusepath{stroke,fill}%
}%
\begin{pgfscope}%
\pgfsys@transformshift{4.676167in}{4.333106in}%
\pgfsys@useobject{currentmarker}{}%
\end{pgfscope}%
\end{pgfscope}%
\begin{pgfscope}%
\pgfsetbuttcap%
\pgfsetroundjoin%
\definecolor{currentfill}{rgb}{0.000000,0.000000,0.000000}%
\pgfsetfillcolor{currentfill}%
\pgfsetlinewidth{0.501875pt}%
\definecolor{currentstroke}{rgb}{0.000000,0.000000,0.000000}%
\pgfsetstrokecolor{currentstroke}%
\pgfsetdash{}{0pt}%
\pgfsys@defobject{currentmarker}{\pgfqpoint{0.000000in}{0.000000in}}{\pgfqpoint{0.020833in}{0.000000in}}{%
\pgfpathmoveto{\pgfqpoint{0.000000in}{0.000000in}}%
\pgfpathlineto{\pgfqpoint{0.020833in}{0.000000in}}%
\pgfusepath{stroke,fill}%
}%
\begin{pgfscope}%
\pgfsys@transformshift{0.444748in}{4.382234in}%
\pgfsys@useobject{currentmarker}{}%
\end{pgfscope}%
\end{pgfscope}%
\begin{pgfscope}%
\pgfsetbuttcap%
\pgfsetroundjoin%
\definecolor{currentfill}{rgb}{0.000000,0.000000,0.000000}%
\pgfsetfillcolor{currentfill}%
\pgfsetlinewidth{0.501875pt}%
\definecolor{currentstroke}{rgb}{0.000000,0.000000,0.000000}%
\pgfsetstrokecolor{currentstroke}%
\pgfsetdash{}{0pt}%
\pgfsys@defobject{currentmarker}{\pgfqpoint{-0.020833in}{0.000000in}}{\pgfqpoint{-0.000000in}{0.000000in}}{%
\pgfpathmoveto{\pgfqpoint{-0.000000in}{0.000000in}}%
\pgfpathlineto{\pgfqpoint{-0.020833in}{0.000000in}}%
\pgfusepath{stroke,fill}%
}%
\begin{pgfscope}%
\pgfsys@transformshift{4.676167in}{4.382234in}%
\pgfsys@useobject{currentmarker}{}%
\end{pgfscope}%
\end{pgfscope}%
\begin{pgfscope}%
\pgfsetbuttcap%
\pgfsetroundjoin%
\definecolor{currentfill}{rgb}{0.000000,0.000000,0.000000}%
\pgfsetfillcolor{currentfill}%
\pgfsetlinewidth{0.501875pt}%
\definecolor{currentstroke}{rgb}{0.000000,0.000000,0.000000}%
\pgfsetstrokecolor{currentstroke}%
\pgfsetdash{}{0pt}%
\pgfsys@defobject{currentmarker}{\pgfqpoint{0.000000in}{0.000000in}}{\pgfqpoint{0.020833in}{0.000000in}}{%
\pgfpathmoveto{\pgfqpoint{0.000000in}{0.000000in}}%
\pgfpathlineto{\pgfqpoint{0.020833in}{0.000000in}}%
\pgfusepath{stroke,fill}%
}%
\begin{pgfscope}%
\pgfsys@transformshift{0.444748in}{4.431361in}%
\pgfsys@useobject{currentmarker}{}%
\end{pgfscope}%
\end{pgfscope}%
\begin{pgfscope}%
\pgfsetbuttcap%
\pgfsetroundjoin%
\definecolor{currentfill}{rgb}{0.000000,0.000000,0.000000}%
\pgfsetfillcolor{currentfill}%
\pgfsetlinewidth{0.501875pt}%
\definecolor{currentstroke}{rgb}{0.000000,0.000000,0.000000}%
\pgfsetstrokecolor{currentstroke}%
\pgfsetdash{}{0pt}%
\pgfsys@defobject{currentmarker}{\pgfqpoint{-0.020833in}{0.000000in}}{\pgfqpoint{-0.000000in}{0.000000in}}{%
\pgfpathmoveto{\pgfqpoint{-0.000000in}{0.000000in}}%
\pgfpathlineto{\pgfqpoint{-0.020833in}{0.000000in}}%
\pgfusepath{stroke,fill}%
}%
\begin{pgfscope}%
\pgfsys@transformshift{4.676167in}{4.431361in}%
\pgfsys@useobject{currentmarker}{}%
\end{pgfscope}%
\end{pgfscope}%
\begin{pgfscope}%
\definecolor{textcolor}{rgb}{0.000000,0.000000,0.000000}%
\pgfsetstrokecolor{textcolor}%
\pgfsetfillcolor{textcolor}%
\pgftext[x=0.201692in,y=4.246200in,,bottom,rotate=90.000000]{\color{textcolor}\rmfamily\fontsize{12.000000}{14.400000}\selectfont \(\displaystyle V_s\) (\unit{\micro\volt})}%
\end{pgfscope}%
\begin{pgfscope}%
\pgfpathrectangle{\pgfqpoint{0.444748in}{4.012575in}}{\pgfqpoint{4.231419in}{0.467251in}}%
\pgfusepath{clip}%
\pgfsetbuttcap%
\pgfsetroundjoin%
\pgfsetlinewidth{1.003750pt}%
\definecolor{currentstroke}{rgb}{0.047059,0.364706,0.647059}%
\pgfsetstrokecolor{currentstroke}%
\pgfsetdash{{3.700000pt}{1.600000pt}}{0.000000pt}%
\pgfpathmoveto{\pgfqpoint{0.637086in}{4.047001in}}%
\pgfpathlineto{\pgfqpoint{0.655396in}{4.037814in}}%
\pgfpathlineto{\pgfqpoint{0.676052in}{4.046150in}}%
\pgfpathlineto{\pgfqpoint{0.697177in}{4.085894in}}%
\pgfpathlineto{\pgfqpoint{0.718067in}{4.227070in}}%
\pgfpathlineto{\pgfqpoint{0.734029in}{4.423639in}}%
\pgfpathlineto{\pgfqpoint{0.754685in}{4.427741in}}%
\pgfpathlineto{\pgfqpoint{0.772994in}{4.276456in}}%
\pgfpathlineto{\pgfqpoint{0.791303in}{4.103346in}}%
\pgfpathlineto{\pgfqpoint{0.810317in}{4.055942in}}%
\pgfpathlineto{\pgfqpoint{0.829564in}{4.038485in}}%
\pgfpathlineto{\pgfqpoint{0.851629in}{4.042172in}}%
\pgfpathlineto{\pgfqpoint{0.868060in}{4.060973in}}%
\pgfpathlineto{\pgfqpoint{0.884725in}{4.115434in}}%
\pgfpathlineto{\pgfqpoint{0.905852in}{4.363447in}}%
\pgfpathlineto{\pgfqpoint{0.922283in}{4.440983in}}%
\pgfpathlineto{\pgfqpoint{0.945756in}{4.309659in}}%
\pgfpathlineto{\pgfqpoint{0.963829in}{4.120875in}}%
\pgfpathlineto{\pgfqpoint{0.981200in}{4.062183in}}%
\pgfpathlineto{\pgfqpoint{1.002559in}{4.039331in}}%
\pgfpathlineto{\pgfqpoint{1.021104in}{4.037443in}}%
\pgfpathlineto{\pgfqpoint{1.041994in}{4.051523in}}%
\pgfpathlineto{\pgfqpoint{1.058894in}{4.093891in}}%
\pgfpathlineto{\pgfqpoint{1.081664in}{4.274837in}}%
\pgfpathlineto{\pgfqpoint{1.097860in}{4.428299in}}%
\pgfpathlineto{\pgfqpoint{1.116873in}{4.392891in}}%
\pgfpathlineto{\pgfqpoint{1.136590in}{4.217478in}}%
\pgfpathlineto{\pgfqpoint{1.158420in}{4.072806in}}%
\pgfpathlineto{\pgfqpoint{1.174147in}{4.045768in}}%
\pgfpathlineto{\pgfqpoint{1.193864in}{4.036686in}}%
\pgfpathlineto{\pgfqpoint{1.213112in}{4.039233in}}%
\pgfpathlineto{\pgfqpoint{1.235411in}{4.062322in}}%
\pgfpathlineto{\pgfqpoint{1.251138in}{4.096982in}}%
\pgfpathlineto{\pgfqpoint{1.270386in}{4.244930in}}%
\pgfpathlineto{\pgfqpoint{1.292216in}{4.424644in}}%
\pgfpathlineto{\pgfqpoint{1.309821in}{4.395746in}}%
\pgfpathlineto{\pgfqpoint{1.329303in}{4.238839in}}%
\pgfpathlineto{\pgfqpoint{1.348316in}{4.130485in}}%
\pgfpathlineto{\pgfqpoint{1.368267in}{4.056144in}}%
\pgfpathlineto{\pgfqpoint{1.385403in}{4.269327in}}%
\pgfpathlineto{\pgfqpoint{1.406528in}{4.412665in}}%
\pgfpathlineto{\pgfqpoint{1.425776in}{4.274164in}}%
\pgfpathlineto{\pgfqpoint{1.463333in}{4.069383in}}%
\pgfpathlineto{\pgfqpoint{1.483521in}{4.044237in}}%
\pgfpathlineto{\pgfqpoint{1.502298in}{4.036198in}}%
\pgfpathlineto{\pgfqpoint{1.521545in}{4.038209in}}%
\pgfpathlineto{\pgfqpoint{1.539621in}{4.050614in}}%
\pgfpathlineto{\pgfqpoint{1.559103in}{4.098175in}}%
\pgfpathlineto{\pgfqpoint{1.580228in}{4.304649in}}%
\pgfpathlineto{\pgfqpoint{1.597598in}{4.418135in}}%
\pgfpathlineto{\pgfqpoint{1.616377in}{4.362066in}}%
\pgfpathlineto{\pgfqpoint{1.634920in}{4.232475in}}%
\pgfpathlineto{\pgfqpoint{1.654638in}{4.086567in}}%
\pgfpathlineto{\pgfqpoint{1.676232in}{4.044236in}}%
\pgfpathlineto{\pgfqpoint{1.697124in}{4.036287in}}%
\pgfpathlineto{\pgfqpoint{1.712850in}{4.036728in}}%
\pgfpathlineto{\pgfqpoint{1.735149in}{4.044706in}}%
\pgfpathlineto{\pgfqpoint{1.750408in}{4.061442in}}%
\pgfpathlineto{\pgfqpoint{1.772707in}{4.165543in}}%
\pgfpathlineto{\pgfqpoint{1.790546in}{4.385294in}}%
\pgfpathlineto{\pgfqpoint{1.809559in}{4.411837in}}%
\pgfpathlineto{\pgfqpoint{1.828572in}{4.281163in}}%
\pgfpathlineto{\pgfqpoint{1.848523in}{4.116923in}}%
\pgfpathlineto{\pgfqpoint{1.866599in}{4.056482in}}%
\pgfpathlineto{\pgfqpoint{1.887958in}{4.237204in}}%
\pgfpathlineto{\pgfqpoint{1.903920in}{4.101526in}}%
\pgfpathlineto{\pgfqpoint{1.925750in}{4.047764in}}%
\pgfpathlineto{\pgfqpoint{1.941947in}{4.119876in}}%
\pgfpathlineto{\pgfqpoint{1.963306in}{4.052055in}}%
\pgfpathlineto{\pgfqpoint{1.982319in}{4.038675in}}%
\pgfpathlineto{\pgfqpoint{2.000629in}{4.035770in}}%
\pgfpathlineto{\pgfqpoint{2.020346in}{4.041267in}}%
\pgfpathlineto{\pgfqpoint{2.041002in}{4.059157in}}%
\pgfpathlineto{\pgfqpoint{2.059546in}{4.125416in}}%
\pgfpathlineto{\pgfqpoint{2.078560in}{4.279329in}}%
\pgfpathlineto{\pgfqpoint{2.097102in}{4.415770in}}%
\pgfpathlineto{\pgfqpoint{2.116350in}{4.369652in}}%
\pgfpathlineto{\pgfqpoint{2.135129in}{4.197432in}}%
\pgfpathlineto{\pgfqpoint{2.157193in}{4.083194in}}%
\pgfpathlineto{\pgfqpoint{2.171747in}{4.047704in}}%
\pgfpathlineto{\pgfqpoint{2.192637in}{4.037441in}}%
\pgfpathlineto{\pgfqpoint{2.214468in}{4.036609in}}%
\pgfpathlineto{\pgfqpoint{2.233010in}{4.042798in}}%
\pgfpathlineto{\pgfqpoint{2.250380in}{4.065205in}}%
\pgfpathlineto{\pgfqpoint{2.271976in}{4.159862in}}%
\pgfpathlineto{\pgfqpoint{2.289581in}{4.366870in}}%
\pgfpathlineto{\pgfqpoint{2.307889in}{4.418491in}}%
\pgfpathlineto{\pgfqpoint{2.328311in}{4.308736in}}%
\pgfpathlineto{\pgfqpoint{2.349436in}{4.148565in}}%
\pgfpathlineto{\pgfqpoint{2.365868in}{4.070987in}}%
\pgfpathlineto{\pgfqpoint{2.385819in}{4.043731in}}%
\pgfpathlineto{\pgfqpoint{2.402721in}{4.036478in}}%
\pgfpathlineto{\pgfqpoint{2.424080in}{4.036751in}}%
\pgfpathlineto{\pgfqpoint{2.445442in}{4.046354in}}%
\pgfpathlineto{\pgfqpoint{2.462576in}{4.071884in}}%
\pgfpathlineto{\pgfqpoint{2.480886in}{4.148324in}}%
\pgfpathlineto{\pgfqpoint{2.501542in}{4.368394in}}%
\pgfpathlineto{\pgfqpoint{2.522198in}{4.413163in}}%
\pgfpathlineto{\pgfqpoint{2.543557in}{4.293095in}}%
\pgfpathlineto{\pgfqpoint{2.558580in}{4.228119in}}%
\pgfpathlineto{\pgfqpoint{2.579472in}{4.084277in}}%
\pgfpathlineto{\pgfqpoint{2.598015in}{4.049070in}}%
\pgfpathlineto{\pgfqpoint{2.614916in}{4.038596in}}%
\pgfpathlineto{\pgfqpoint{2.636276in}{4.035641in}}%
\pgfpathlineto{\pgfqpoint{2.653646in}{4.039124in}}%
\pgfpathlineto{\pgfqpoint{2.675005in}{4.055232in}}%
\pgfpathlineto{\pgfqpoint{2.693784in}{4.084092in}}%
\pgfpathlineto{\pgfqpoint{2.713032in}{4.225559in}}%
\pgfpathlineto{\pgfqpoint{2.732280in}{4.403116in}}%
\pgfpathlineto{\pgfqpoint{2.749416in}{4.412505in}}%
\pgfpathlineto{\pgfqpoint{2.771246in}{4.329880in}}%
\pgfpathlineto{\pgfqpoint{2.788850in}{4.180553in}}%
\pgfpathlineto{\pgfqpoint{2.807393in}{4.113524in}}%
\pgfpathlineto{\pgfqpoint{2.828049in}{4.053084in}}%
\pgfpathlineto{\pgfqpoint{2.846594in}{4.041123in}}%
\pgfpathlineto{\pgfqpoint{2.866544in}{4.035735in}}%
\pgfpathlineto{\pgfqpoint{2.884854in}{4.038494in}}%
\pgfpathlineto{\pgfqpoint{2.902459in}{4.045188in}}%
\pgfpathlineto{\pgfqpoint{2.924289in}{4.075015in}}%
\pgfpathlineto{\pgfqpoint{2.941658in}{4.151431in}}%
\pgfpathlineto{\pgfqpoint{2.962785in}{4.328320in}}%
\pgfpathlineto{\pgfqpoint{2.981327in}{4.412210in}}%
\pgfpathlineto{\pgfqpoint{2.999872in}{4.364225in}}%
\pgfpathlineto{\pgfqpoint{3.020997in}{4.201348in}}%
\pgfpathlineto{\pgfqpoint{3.038133in}{4.094185in}}%
\pgfpathlineto{\pgfqpoint{3.056912in}{4.055908in}}%
\pgfpathlineto{\pgfqpoint{3.078036in}{4.040859in}}%
\pgfpathlineto{\pgfqpoint{3.095407in}{4.035862in}}%
\pgfpathlineto{\pgfqpoint{3.116766in}{4.036503in}}%
\pgfpathlineto{\pgfqpoint{3.134606in}{4.043355in}}%
\pgfpathlineto{\pgfqpoint{3.155498in}{4.071233in}}%
\pgfpathlineto{\pgfqpoint{3.173103in}{4.129380in}}%
\pgfpathlineto{\pgfqpoint{3.191645in}{4.269201in}}%
\pgfpathlineto{\pgfqpoint{3.213007in}{4.414789in}}%
\pgfpathlineto{\pgfqpoint{3.232254in}{4.396585in}}%
\pgfpathlineto{\pgfqpoint{3.249154in}{4.320499in}}%
\pgfpathlineto{\pgfqpoint{3.268636in}{4.177506in}}%
\pgfpathlineto{\pgfqpoint{3.289763in}{4.209958in}}%
\pgfpathlineto{\pgfqpoint{3.307368in}{4.098674in}}%
\pgfpathlineto{\pgfqpoint{3.324502in}{4.058548in}}%
\pgfpathlineto{\pgfqpoint{3.347272in}{4.041056in}}%
\pgfpathlineto{\pgfqpoint{3.364642in}{4.036392in}}%
\pgfpathlineto{\pgfqpoint{3.385767in}{4.036851in}}%
\pgfpathlineto{\pgfqpoint{3.403606in}{4.044259in}}%
\pgfpathlineto{\pgfqpoint{3.424497in}{4.069552in}}%
\pgfpathlineto{\pgfqpoint{3.442101in}{4.105886in}}%
\pgfpathlineto{\pgfqpoint{3.459941in}{4.209607in}}%
\pgfpathlineto{\pgfqpoint{3.481068in}{4.407970in}}%
\pgfpathlineto{\pgfqpoint{3.501958in}{4.417691in}}%
\pgfpathlineto{\pgfqpoint{3.519329in}{4.334392in}}%
\pgfpathlineto{\pgfqpoint{3.540688in}{4.193995in}}%
\pgfpathlineto{\pgfqpoint{3.559467in}{4.099497in}}%
\pgfpathlineto{\pgfqpoint{3.576837in}{4.063253in}}%
\pgfpathlineto{\pgfqpoint{3.596788in}{4.043423in}}%
\pgfpathlineto{\pgfqpoint{3.616270in}{4.038350in}}%
\pgfpathlineto{\pgfqpoint{3.636926in}{4.036630in}}%
\pgfpathlineto{\pgfqpoint{3.654062in}{4.040693in}}%
\pgfpathlineto{\pgfqpoint{3.674953in}{4.054982in}}%
\pgfpathlineto{\pgfqpoint{3.691854in}{4.080979in}}%
\pgfpathlineto{\pgfqpoint{3.715796in}{4.186227in}}%
\pgfpathlineto{\pgfqpoint{3.731053in}{4.303862in}}%
\pgfpathlineto{\pgfqpoint{3.750535in}{4.423607in}}%
\pgfpathlineto{\pgfqpoint{3.768845in}{4.422927in}}%
\pgfpathlineto{\pgfqpoint{3.793256in}{4.300610in}}%
\pgfpathlineto{\pgfqpoint{3.808280in}{4.207037in}}%
\pgfpathlineto{\pgfqpoint{3.826588in}{4.104869in}}%
\pgfpathlineto{\pgfqpoint{3.847950in}{4.060277in}}%
\pgfpathlineto{\pgfqpoint{3.864849in}{4.047718in}}%
\pgfpathlineto{\pgfqpoint{3.883394in}{4.040101in}}%
\pgfpathlineto{\pgfqpoint{3.905693in}{4.036934in}}%
\pgfpathlineto{\pgfqpoint{3.922827in}{4.037598in}}%
\pgfpathlineto{\pgfqpoint{3.943483in}{4.043324in}}%
\pgfpathlineto{\pgfqpoint{3.962262in}{4.060249in}}%
\pgfpathlineto{\pgfqpoint{3.980806in}{4.093052in}}%
\pgfpathlineto{\pgfqpoint{4.000991in}{4.170057in}}%
\pgfpathlineto{\pgfqpoint{4.019301in}{4.320166in}}%
\pgfpathlineto{\pgfqpoint{4.037844in}{4.434634in}}%
\pgfpathlineto{\pgfqpoint{4.057562in}{4.438979in}}%
\pgfpathlineto{\pgfqpoint{4.076341in}{4.375410in}}%
\pgfpathlineto{\pgfqpoint{4.115540in}{4.139097in}}%
\pgfpathlineto{\pgfqpoint{4.132676in}{4.084415in}}%
\pgfpathlineto{\pgfqpoint{4.154740in}{4.052293in}}%
\pgfpathlineto{\pgfqpoint{4.190184in}{4.040165in}}%
\pgfpathlineto{\pgfqpoint{4.212249in}{4.038000in}}%
\pgfpathlineto{\pgfqpoint{4.229619in}{4.044182in}}%
\pgfpathlineto{\pgfqpoint{4.248162in}{4.056370in}}%
\pgfpathlineto{\pgfqpoint{4.268818in}{4.095893in}}%
\pgfpathlineto{\pgfqpoint{4.287128in}{4.182930in}}%
\pgfpathlineto{\pgfqpoint{4.308253in}{4.356689in}}%
\pgfpathlineto{\pgfqpoint{4.325858in}{4.447516in}}%
\pgfpathlineto{\pgfqpoint{4.343697in}{4.453071in}}%
\pgfpathlineto{\pgfqpoint{4.365762in}{4.387631in}}%
\pgfpathlineto{\pgfqpoint{4.383835in}{4.251733in}}%
\pgfpathlineto{\pgfqpoint{4.404491in}{4.138202in}}%
\pgfpathlineto{\pgfqpoint{4.423270in}{4.089155in}}%
\pgfpathlineto{\pgfqpoint{4.440875in}{4.058666in}}%
\pgfpathlineto{\pgfqpoint{4.461766in}{4.044514in}}%
\pgfpathlineto{\pgfqpoint{4.482187in}{4.038515in}}%
\pgfpathlineto{\pgfqpoint{4.475379in}{4.039896in}}%
\pgfpathlineto{\pgfqpoint{4.455194in}{4.057762in}}%
\pgfpathlineto{\pgfqpoint{4.434303in}{4.131518in}}%
\pgfpathlineto{\pgfqpoint{4.411534in}{4.335421in}}%
\pgfpathlineto{\pgfqpoint{4.396746in}{4.439247in}}%
\pgfpathlineto{\pgfqpoint{4.376795in}{4.440489in}}%
\pgfpathlineto{\pgfqpoint{4.359424in}{4.226292in}}%
\pgfpathlineto{\pgfqpoint{4.337360in}{4.072918in}}%
\pgfpathlineto{\pgfqpoint{4.319755in}{4.045382in}}%
\pgfpathlineto{\pgfqpoint{4.300273in}{4.037598in}}%
\pgfpathlineto{\pgfqpoint{4.280554in}{4.046811in}}%
\pgfpathlineto{\pgfqpoint{4.262715in}{4.079835in}}%
\pgfpathlineto{\pgfqpoint{4.242528in}{4.206440in}}%
\pgfpathlineto{\pgfqpoint{4.223517in}{4.390578in}}%
\pgfpathlineto{\pgfqpoint{4.205207in}{4.445990in}}%
\pgfpathlineto{\pgfqpoint{4.183847in}{4.276564in}}%
\pgfpathlineto{\pgfqpoint{4.165537in}{4.112593in}}%
\pgfpathlineto{\pgfqpoint{4.147698in}{4.056553in}}%
\pgfpathlineto{\pgfqpoint{4.129624in}{4.040644in}}%
\pgfpathlineto{\pgfqpoint{4.107089in}{4.038579in}}%
\pgfpathlineto{\pgfqpoint{4.090424in}{4.050974in}}%
\pgfpathlineto{\pgfqpoint{4.069533in}{4.103922in}}%
\pgfpathlineto{\pgfqpoint{4.051225in}{4.267852in}}%
\pgfpathlineto{\pgfqpoint{4.031038in}{4.420926in}}%
\pgfpathlineto{\pgfqpoint{4.013199in}{4.418776in}}%
\pgfpathlineto{\pgfqpoint{3.994654in}{4.210714in}}%
\pgfpathlineto{\pgfqpoint{3.974703in}{4.078408in}}%
\pgfpathlineto{\pgfqpoint{3.955924in}{4.046644in}}%
\pgfpathlineto{\pgfqpoint{3.935737in}{4.036662in}}%
\pgfpathlineto{\pgfqpoint{3.915315in}{4.041688in}}%
\pgfpathlineto{\pgfqpoint{3.897476in}{4.059329in}}%
\pgfpathlineto{\pgfqpoint{3.880577in}{4.156563in}}%
\pgfpathlineto{\pgfqpoint{3.856869in}{4.345680in}}%
\pgfpathlineto{\pgfqpoint{3.839030in}{4.427289in}}%
\pgfpathlineto{\pgfqpoint{3.819077in}{4.361433in}}%
\pgfpathlineto{\pgfqpoint{3.802646in}{4.365788in}}%
\pgfpathlineto{\pgfqpoint{3.784102in}{4.135751in}}%
\pgfpathlineto{\pgfqpoint{3.763211in}{4.059890in}}%
\pgfpathlineto{\pgfqpoint{3.742555in}{4.039927in}}%
\pgfpathlineto{\pgfqpoint{3.726593in}{4.036086in}}%
\pgfpathlineto{\pgfqpoint{3.704060in}{4.041476in}}%
\pgfpathlineto{\pgfqpoint{3.686689in}{4.059608in}}%
\pgfpathlineto{\pgfqpoint{3.670024in}{4.120137in}}%
\pgfpathlineto{\pgfqpoint{3.648663in}{4.291080in}}%
\pgfpathlineto{\pgfqpoint{3.629181in}{4.417290in}}%
\pgfpathlineto{\pgfqpoint{3.608525in}{4.360867in}}%
\pgfpathlineto{\pgfqpoint{3.590217in}{4.140543in}}%
\pgfpathlineto{\pgfqpoint{3.571203in}{4.061989in}}%
\pgfpathlineto{\pgfqpoint{3.552190in}{4.044433in}}%
\pgfpathlineto{\pgfqpoint{3.528951in}{4.036523in}}%
\pgfpathlineto{\pgfqpoint{3.514398in}{4.036077in}}%
\pgfpathlineto{\pgfqpoint{3.493507in}{4.044634in}}%
\pgfpathlineto{\pgfqpoint{3.473086in}{4.083183in}}%
\pgfpathlineto{\pgfqpoint{3.456655in}{4.146232in}}%
\pgfpathlineto{\pgfqpoint{3.436468in}{4.334575in}}%
\pgfpathlineto{\pgfqpoint{3.417925in}{4.418592in}}%
\pgfpathlineto{\pgfqpoint{3.397269in}{4.316843in}}%
\pgfpathlineto{\pgfqpoint{3.379430in}{4.138524in}}%
\pgfpathlineto{\pgfqpoint{3.361354in}{4.065370in}}%
\pgfpathlineto{\pgfqpoint{3.340933in}{4.044679in}}%
\pgfpathlineto{\pgfqpoint{3.319573in}{4.036050in}}%
\pgfpathlineto{\pgfqpoint{3.302672in}{4.038588in}}%
\pgfpathlineto{\pgfqpoint{3.284598in}{4.050208in}}%
\pgfpathlineto{\pgfqpoint{3.260890in}{4.045277in}}%
\pgfpathlineto{\pgfqpoint{3.243991in}{4.076092in}}%
\pgfpathlineto{\pgfqpoint{3.224978in}{4.160471in}}%
\pgfpathlineto{\pgfqpoint{3.204790in}{4.359561in}}%
\pgfpathlineto{\pgfqpoint{3.186717in}{4.419378in}}%
\pgfpathlineto{\pgfqpoint{3.169112in}{4.345018in}}%
\pgfpathlineto{\pgfqpoint{3.147985in}{4.130073in}}%
\pgfpathlineto{\pgfqpoint{3.130616in}{4.065659in}}%
\pgfpathlineto{\pgfqpoint{3.109255in}{4.041495in}}%
\pgfpathlineto{\pgfqpoint{3.090007in}{4.035659in}}%
\pgfpathlineto{\pgfqpoint{3.070760in}{4.037564in}}%
\pgfpathlineto{\pgfqpoint{3.053624in}{4.046162in}}%
\pgfpathlineto{\pgfqpoint{3.031796in}{4.081568in}}%
\pgfpathlineto{\pgfqpoint{3.013720in}{4.076075in}}%
\pgfpathlineto{\pgfqpoint{2.993064in}{4.092381in}}%
\pgfpathlineto{\pgfqpoint{2.975225in}{4.225966in}}%
\pgfpathlineto{\pgfqpoint{2.957620in}{4.079675in}}%
\pgfpathlineto{\pgfqpoint{2.939312in}{4.213680in}}%
\pgfpathlineto{\pgfqpoint{2.917950in}{4.393840in}}%
\pgfpathlineto{\pgfqpoint{2.897294in}{4.411580in}}%
\pgfpathlineto{\pgfqpoint{2.880629in}{4.258370in}}%
\pgfpathlineto{\pgfqpoint{2.857625in}{4.088066in}}%
\pgfpathlineto{\pgfqpoint{2.841665in}{4.052792in}}%
\pgfpathlineto{\pgfqpoint{2.823589in}{4.038584in}}%
\pgfpathlineto{\pgfqpoint{2.801525in}{4.035893in}}%
\pgfpathlineto{\pgfqpoint{2.782748in}{4.041309in}}%
\pgfpathlineto{\pgfqpoint{2.764438in}{4.055358in}}%
\pgfpathlineto{\pgfqpoint{2.745425in}{4.113621in}}%
\pgfpathlineto{\pgfqpoint{2.726882in}{4.274102in}}%
\pgfpathlineto{\pgfqpoint{2.707634in}{4.401230in}}%
\pgfpathlineto{\pgfqpoint{2.687213in}{4.382254in}}%
\pgfpathlineto{\pgfqpoint{2.670546in}{4.211789in}}%
\pgfpathlineto{\pgfqpoint{2.648952in}{4.110191in}}%
\pgfpathlineto{\pgfqpoint{2.630407in}{4.058083in}}%
\pgfpathlineto{\pgfqpoint{2.608343in}{4.038940in}}%
\pgfpathlineto{\pgfqpoint{2.589329in}{4.035416in}}%
\pgfpathlineto{\pgfqpoint{2.571256in}{4.038120in}}%
\pgfpathlineto{\pgfqpoint{2.551774in}{4.044918in}}%
\pgfpathlineto{\pgfqpoint{2.531352in}{4.073460in}}%
\pgfpathlineto{\pgfqpoint{2.514450in}{4.165569in}}%
\pgfpathlineto{\pgfqpoint{2.493091in}{4.367652in}}%
\pgfpathlineto{\pgfqpoint{2.477364in}{4.413313in}}%
\pgfpathlineto{\pgfqpoint{2.457413in}{4.334156in}}%
\pgfpathlineto{\pgfqpoint{2.438634in}{4.142352in}}%
\pgfpathlineto{\pgfqpoint{2.418446in}{4.065302in}}%
\pgfpathlineto{\pgfqpoint{2.399199in}{4.042240in}}%
\pgfpathlineto{\pgfqpoint{2.380186in}{4.036377in}}%
\pgfpathlineto{\pgfqpoint{2.361643in}{4.036310in}}%
\pgfpathlineto{\pgfqpoint{2.340282in}{4.041546in}}%
\pgfpathlineto{\pgfqpoint{2.321503in}{4.059753in}}%
\pgfpathlineto{\pgfqpoint{2.302726in}{4.084203in}}%
\pgfpathlineto{\pgfqpoint{2.284650in}{4.196286in}}%
\pgfpathlineto{\pgfqpoint{2.265639in}{4.346764in}}%
\pgfpathlineto{\pgfqpoint{2.246860in}{4.405021in}}%
\pgfpathlineto{\pgfqpoint{2.226438in}{4.341271in}}%
\pgfpathlineto{\pgfqpoint{2.207191in}{4.155338in}}%
\pgfpathlineto{\pgfqpoint{2.188178in}{4.082343in}}%
\pgfpathlineto{\pgfqpoint{2.166582in}{4.051611in}}%
\pgfpathlineto{\pgfqpoint{2.147805in}{4.038741in}}%
\pgfpathlineto{\pgfqpoint{2.129729in}{4.035703in}}%
\pgfpathlineto{\pgfqpoint{2.110716in}{4.038439in}}%
\pgfpathlineto{\pgfqpoint{2.089122in}{4.052332in}}%
\pgfpathlineto{\pgfqpoint{2.073395in}{4.071566in}}%
\pgfpathlineto{\pgfqpoint{2.052270in}{4.158185in}}%
\pgfpathlineto{\pgfqpoint{2.033725in}{4.313552in}}%
\pgfpathlineto{\pgfqpoint{2.016355in}{4.402958in}}%
\pgfpathlineto{\pgfqpoint{1.995933in}{4.397728in}}%
\pgfpathlineto{\pgfqpoint{1.959081in}{4.110250in}}%
\pgfpathlineto{\pgfqpoint{1.937016in}{4.060086in}}%
\pgfpathlineto{\pgfqpoint{1.915422in}{4.042209in}}%
\pgfpathlineto{\pgfqpoint{1.899695in}{4.037682in}}%
\pgfpathlineto{\pgfqpoint{1.881621in}{4.035693in}}%
\pgfpathlineto{\pgfqpoint{1.863077in}{4.039263in}}%
\pgfpathlineto{\pgfqpoint{1.842186in}{4.054691in}}%
\pgfpathlineto{\pgfqpoint{1.822470in}{4.085355in}}%
\pgfpathlineto{\pgfqpoint{1.804394in}{4.176221in}}%
\pgfpathlineto{\pgfqpoint{1.781861in}{4.325672in}}%
\pgfpathlineto{\pgfqpoint{1.763318in}{4.412941in}}%
\pgfpathlineto{\pgfqpoint{1.745713in}{4.393741in}}%
\pgfpathlineto{\pgfqpoint{1.727403in}{4.332246in}}%
\pgfpathlineto{\pgfqpoint{1.707687in}{4.410352in}}%
\pgfpathlineto{\pgfqpoint{1.689613in}{4.396382in}}%
\pgfpathlineto{\pgfqpoint{1.667783in}{4.178424in}}%
\pgfpathlineto{\pgfqpoint{1.648770in}{4.083907in}}%
\pgfpathlineto{\pgfqpoint{1.629757in}{4.056415in}}%
\pgfpathlineto{\pgfqpoint{1.608632in}{4.039603in}}%
\pgfpathlineto{\pgfqpoint{1.590790in}{4.035997in}}%
\pgfpathlineto{\pgfqpoint{1.572248in}{4.038428in}}%
\pgfpathlineto{\pgfqpoint{1.553703in}{4.045264in}}%
\pgfpathlineto{\pgfqpoint{1.531639in}{4.072882in}}%
\pgfpathlineto{\pgfqpoint{1.516851in}{4.119404in}}%
\pgfpathlineto{\pgfqpoint{1.495021in}{4.279824in}}%
\pgfpathlineto{\pgfqpoint{1.479295in}{4.379226in}}%
\pgfpathlineto{\pgfqpoint{1.457231in}{4.425033in}}%
\pgfpathlineto{\pgfqpoint{1.439391in}{4.385134in}}%
\pgfpathlineto{\pgfqpoint{1.418030in}{4.190091in}}%
\pgfpathlineto{\pgfqpoint{1.399488in}{4.094424in}}%
\pgfpathlineto{\pgfqpoint{1.380004in}{4.056374in}}%
\pgfpathlineto{\pgfqpoint{1.360756in}{4.043595in}}%
\pgfpathlineto{\pgfqpoint{1.339865in}{4.037020in}}%
\pgfpathlineto{\pgfqpoint{1.325077in}{4.036994in}}%
\pgfpathlineto{\pgfqpoint{1.303718in}{4.040852in}}%
\pgfpathlineto{\pgfqpoint{1.284705in}{4.127541in}}%
\pgfpathlineto{\pgfqpoint{1.266160in}{4.061933in}}%
\pgfpathlineto{\pgfqpoint{1.247382in}{4.049353in}}%
\pgfpathlineto{\pgfqpoint{1.226960in}{4.037237in}}%
\pgfpathlineto{\pgfqpoint{1.209591in}{4.037434in}}%
\pgfpathlineto{\pgfqpoint{1.187996in}{4.043400in}}%
\pgfpathlineto{\pgfqpoint{1.167574in}{4.068278in}}%
\pgfpathlineto{\pgfqpoint{1.149031in}{4.137696in}}%
\pgfpathlineto{\pgfqpoint{1.131190in}{4.286713in}}%
\pgfpathlineto{\pgfqpoint{1.111239in}{4.413556in}}%
\pgfpathlineto{\pgfqpoint{1.089878in}{4.430227in}}%
\pgfpathlineto{\pgfqpoint{1.074152in}{4.288633in}}%
\pgfpathlineto{\pgfqpoint{1.052557in}{4.121210in}}%
\pgfpathlineto{\pgfqpoint{1.034014in}{4.074175in}}%
\pgfpathlineto{\pgfqpoint{1.015235in}{4.049566in}}%
\pgfpathlineto{\pgfqpoint{0.996925in}{4.039261in}}%
\pgfpathlineto{\pgfqpoint{0.973923in}{4.038650in}}%
\pgfpathlineto{\pgfqpoint{0.954910in}{4.046086in}}%
\pgfpathlineto{\pgfqpoint{0.937539in}{4.069871in}}%
\pgfpathlineto{\pgfqpoint{0.917118in}{4.134870in}}%
\pgfpathlineto{\pgfqpoint{0.880031in}{4.364068in}}%
\pgfpathlineto{\pgfqpoint{0.861252in}{4.439117in}}%
\pgfpathlineto{\pgfqpoint{0.843178in}{4.433658in}}%
\pgfpathlineto{\pgfqpoint{0.824870in}{4.280367in}}%
\pgfpathlineto{\pgfqpoint{0.803040in}{4.122908in}}%
\pgfpathlineto{\pgfqpoint{0.784730in}{4.082102in}}%
\pgfpathlineto{\pgfqpoint{0.766188in}{4.053829in}}%
\pgfpathlineto{\pgfqpoint{0.747643in}{4.044997in}}%
\pgfpathlineto{\pgfqpoint{0.725579in}{4.037594in}}%
\pgfpathlineto{\pgfqpoint{0.708208in}{4.040004in}}%
\pgfpathlineto{\pgfqpoint{0.689431in}{4.048402in}}%
\pgfpathlineto{\pgfqpoint{0.671826in}{4.063942in}}%
\pgfpathlineto{\pgfqpoint{0.645536in}{4.144319in}}%
\pgfpathlineto{\pgfqpoint{0.645302in}{4.142270in}}%
\pgfpathlineto{\pgfqpoint{0.658211in}{4.079394in}}%
\pgfpathlineto{\pgfqpoint{0.676286in}{4.047515in}}%
\pgfpathlineto{\pgfqpoint{0.693420in}{4.037974in}}%
\pgfpathlineto{\pgfqpoint{0.712199in}{4.043657in}}%
\pgfpathlineto{\pgfqpoint{0.736141in}{4.081334in}}%
\pgfpathlineto{\pgfqpoint{0.754217in}{4.187446in}}%
\pgfpathlineto{\pgfqpoint{0.774636in}{4.430039in}}%
\pgfpathlineto{\pgfqpoint{0.790129in}{4.439660in}}%
\pgfpathlineto{\pgfqpoint{0.811020in}{4.296707in}}%
\pgfpathlineto{\pgfqpoint{0.829094in}{4.129642in}}%
\pgfpathlineto{\pgfqpoint{0.848107in}{4.057996in}}%
\pgfpathlineto{\pgfqpoint{0.868763in}{4.039375in}}%
\pgfpathlineto{\pgfqpoint{0.886133in}{4.038615in}}%
\pgfpathlineto{\pgfqpoint{0.907258in}{4.054117in}}%
\pgfpathlineto{\pgfqpoint{0.926037in}{4.099353in}}%
\pgfpathlineto{\pgfqpoint{0.946693in}{4.319222in}}%
\pgfpathlineto{\pgfqpoint{0.963361in}{4.438183in}}%
\pgfpathlineto{\pgfqpoint{0.984017in}{4.355569in}}%
\pgfpathlineto{\pgfqpoint{1.001385in}{4.168180in}}%
\pgfpathlineto{\pgfqpoint{1.020633in}{4.070178in}}%
\pgfpathlineto{\pgfqpoint{1.043871in}{4.040565in}}%
\pgfpathlineto{\pgfqpoint{1.060773in}{4.036661in}}%
\pgfpathlineto{\pgfqpoint{1.080255in}{4.046004in}}%
\pgfpathlineto{\pgfqpoint{1.099972in}{4.071048in}}%
\pgfpathlineto{\pgfqpoint{1.118282in}{4.174669in}}%
\pgfpathlineto{\pgfqpoint{1.139641in}{4.412157in}}%
\pgfpathlineto{\pgfqpoint{1.155134in}{4.419232in}}%
\pgfpathlineto{\pgfqpoint{1.191987in}{4.125119in}}%
\pgfpathlineto{\pgfqpoint{1.214989in}{4.051418in}}%
\pgfpathlineto{\pgfqpoint{1.233768in}{4.038256in}}%
\pgfpathlineto{\pgfqpoint{1.252781in}{4.037071in}}%
\pgfpathlineto{\pgfqpoint{1.269681in}{4.045016in}}%
\pgfpathlineto{\pgfqpoint{1.288225in}{4.082117in}}%
\pgfpathlineto{\pgfqpoint{1.310290in}{4.217712in}}%
\pgfpathlineto{\pgfqpoint{1.326955in}{4.406161in}}%
\pgfpathlineto{\pgfqpoint{1.349725in}{4.388107in}}%
\pgfpathlineto{\pgfqpoint{1.369207in}{4.207379in}}%
\pgfpathlineto{\pgfqpoint{1.385872in}{4.084999in}}%
\pgfpathlineto{\pgfqpoint{1.405825in}{4.055582in}}%
\pgfpathlineto{\pgfqpoint{1.424604in}{4.038870in}}%
\pgfpathlineto{\pgfqpoint{1.443380in}{4.036045in}}%
\pgfpathlineto{\pgfqpoint{1.462864in}{4.043370in}}%
\pgfpathlineto{\pgfqpoint{1.481641in}{4.054650in}}%
\pgfpathlineto{\pgfqpoint{1.501594in}{4.111148in}}%
\pgfpathlineto{\pgfqpoint{1.519902in}{4.266435in}}%
\pgfpathlineto{\pgfqpoint{1.542201in}{4.421975in}}%
\pgfpathlineto{\pgfqpoint{1.561685in}{4.327102in}}%
\pgfpathlineto{\pgfqpoint{1.579759in}{4.203168in}}%
\pgfpathlineto{\pgfqpoint{1.599007in}{4.081492in}}%
\pgfpathlineto{\pgfqpoint{1.618723in}{4.048281in}}%
\pgfpathlineto{\pgfqpoint{1.636094in}{4.038310in}}%
\pgfpathlineto{\pgfqpoint{1.654404in}{4.404883in}}%
\pgfpathlineto{\pgfqpoint{1.677640in}{4.371987in}}%
\pgfpathlineto{\pgfqpoint{1.715198in}{4.082529in}}%
\pgfpathlineto{\pgfqpoint{1.731160in}{4.048176in}}%
\pgfpathlineto{\pgfqpoint{1.751816in}{4.037083in}}%
\pgfpathlineto{\pgfqpoint{1.772001in}{4.037167in}}%
\pgfpathlineto{\pgfqpoint{1.792423in}{4.044779in}}%
\pgfpathlineto{\pgfqpoint{1.810499in}{4.075836in}}%
\pgfpathlineto{\pgfqpoint{1.829276in}{4.173873in}}%
\pgfpathlineto{\pgfqpoint{1.850403in}{4.390256in}}%
\pgfpathlineto{\pgfqpoint{1.865190in}{4.416515in}}%
\pgfpathlineto{\pgfqpoint{1.883967in}{4.348332in}}%
\pgfpathlineto{\pgfqpoint{1.903686in}{4.156489in}}%
\pgfpathlineto{\pgfqpoint{1.921525in}{4.063238in}}%
\pgfpathlineto{\pgfqpoint{1.944764in}{4.040033in}}%
\pgfpathlineto{\pgfqpoint{1.963072in}{4.035549in}}%
\pgfpathlineto{\pgfqpoint{1.980911in}{4.039333in}}%
\pgfpathlineto{\pgfqpoint{1.999455in}{4.055425in}}%
\pgfpathlineto{\pgfqpoint{2.022223in}{4.145587in}}%
\pgfpathlineto{\pgfqpoint{2.039125in}{4.104651in}}%
\pgfpathlineto{\pgfqpoint{2.059546in}{4.280183in}}%
\pgfpathlineto{\pgfqpoint{2.076915in}{4.417792in}}%
\pgfpathlineto{\pgfqpoint{2.096633in}{4.347446in}}%
\pgfpathlineto{\pgfqpoint{2.116115in}{4.166571in}}%
\pgfpathlineto{\pgfqpoint{2.135832in}{4.067651in}}%
\pgfpathlineto{\pgfqpoint{2.156019in}{4.042312in}}%
\pgfpathlineto{\pgfqpoint{2.173624in}{4.036010in}}%
\pgfpathlineto{\pgfqpoint{2.193106in}{4.035566in}}%
\pgfpathlineto{\pgfqpoint{2.214936in}{4.043560in}}%
\pgfpathlineto{\pgfqpoint{2.233246in}{4.045435in}}%
\pgfpathlineto{\pgfqpoint{2.251554in}{4.069259in}}%
\pgfpathlineto{\pgfqpoint{2.269394in}{4.159963in}}%
\pgfpathlineto{\pgfqpoint{2.286530in}{4.378489in}}%
\pgfpathlineto{\pgfqpoint{2.308594in}{4.412026in}}%
\pgfpathlineto{\pgfqpoint{2.327607in}{4.281649in}}%
\pgfpathlineto{\pgfqpoint{2.348498in}{4.100992in}}%
\pgfpathlineto{\pgfqpoint{2.366337in}{4.065538in}}%
\pgfpathlineto{\pgfqpoint{2.389105in}{4.043901in}}%
\pgfpathlineto{\pgfqpoint{2.408353in}{4.036128in}}%
\pgfpathlineto{\pgfqpoint{2.423611in}{4.035749in}}%
\pgfpathlineto{\pgfqpoint{2.443797in}{4.043302in}}%
\pgfpathlineto{\pgfqpoint{2.461872in}{4.067148in}}%
\pgfpathlineto{\pgfqpoint{2.483232in}{4.146092in}}%
\pgfpathlineto{\pgfqpoint{2.501307in}{4.329389in}}%
\pgfpathlineto{\pgfqpoint{2.519147in}{4.413851in}}%
\pgfpathlineto{\pgfqpoint{2.536986in}{4.396457in}}%
\pgfpathlineto{\pgfqpoint{2.558816in}{4.298548in}}%
\pgfpathlineto{\pgfqpoint{2.578533in}{4.112494in}}%
\pgfpathlineto{\pgfqpoint{2.598483in}{4.051544in}}%
\pgfpathlineto{\pgfqpoint{2.616793in}{4.039515in}}%
\pgfpathlineto{\pgfqpoint{2.638858in}{4.035488in}}%
\pgfpathlineto{\pgfqpoint{2.655054in}{4.039083in}}%
\pgfpathlineto{\pgfqpoint{2.673833in}{4.051574in}}%
\pgfpathlineto{\pgfqpoint{2.694253in}{4.091878in}}%
\pgfpathlineto{\pgfqpoint{2.730637in}{4.401262in}}%
\pgfpathlineto{\pgfqpoint{2.751293in}{4.387065in}}%
\pgfpathlineto{\pgfqpoint{2.769603in}{4.238593in}}%
\pgfpathlineto{\pgfqpoint{2.790728in}{4.145028in}}%
\pgfpathlineto{\pgfqpoint{2.808098in}{4.333451in}}%
\pgfpathlineto{\pgfqpoint{2.826172in}{4.157887in}}%
\pgfpathlineto{\pgfqpoint{2.847062in}{4.069510in}}%
\pgfpathlineto{\pgfqpoint{2.865841in}{4.043897in}}%
\pgfpathlineto{\pgfqpoint{2.886497in}{4.037460in}}%
\pgfpathlineto{\pgfqpoint{2.903868in}{4.035546in}}%
\pgfpathlineto{\pgfqpoint{2.921473in}{4.039665in}}%
\pgfpathlineto{\pgfqpoint{2.942598in}{4.058152in}}%
\pgfpathlineto{\pgfqpoint{2.963959in}{4.122099in}}%
\pgfpathlineto{\pgfqpoint{2.982736in}{4.287212in}}%
\pgfpathlineto{\pgfqpoint{3.000811in}{4.405405in}}%
\pgfpathlineto{\pgfqpoint{3.020762in}{4.374006in}}%
\pgfpathlineto{\pgfqpoint{3.039072in}{4.262452in}}%
\pgfpathlineto{\pgfqpoint{3.061371in}{4.097552in}}%
\pgfpathlineto{\pgfqpoint{3.077802in}{4.057325in}}%
\pgfpathlineto{\pgfqpoint{3.096110in}{4.043307in}}%
\pgfpathlineto{\pgfqpoint{3.116297in}{4.036111in}}%
\pgfpathlineto{\pgfqpoint{3.138596in}{4.037941in}}%
\pgfpathlineto{\pgfqpoint{3.153150in}{4.044348in}}%
\pgfpathlineto{\pgfqpoint{3.174040in}{4.069157in}}%
\pgfpathlineto{\pgfqpoint{3.194697in}{4.135923in}}%
\pgfpathlineto{\pgfqpoint{3.210424in}{4.273891in}}%
\pgfpathlineto{\pgfqpoint{3.231549in}{4.417641in}}%
\pgfpathlineto{\pgfqpoint{3.249623in}{4.378616in}}%
\pgfpathlineto{\pgfqpoint{3.271219in}{4.267096in}}%
\pgfpathlineto{\pgfqpoint{3.289058in}{4.126634in}}%
\pgfpathlineto{\pgfqpoint{3.309948in}{4.059963in}}%
\pgfpathlineto{\pgfqpoint{3.329196in}{4.124063in}}%
\pgfpathlineto{\pgfqpoint{3.346098in}{4.064788in}}%
\pgfpathlineto{\pgfqpoint{3.363468in}{4.044061in}}%
\pgfpathlineto{\pgfqpoint{3.387175in}{4.036139in}}%
\pgfpathlineto{\pgfqpoint{3.402432in}{4.037442in}}%
\pgfpathlineto{\pgfqpoint{3.424497in}{4.045544in}}%
\pgfpathlineto{\pgfqpoint{3.441867in}{4.062579in}}%
\pgfpathlineto{\pgfqpoint{3.464166in}{4.111450in}}%
\pgfpathlineto{\pgfqpoint{3.480831in}{4.244330in}}%
\pgfpathlineto{\pgfqpoint{3.500784in}{4.415737in}}%
\pgfpathlineto{\pgfqpoint{3.520032in}{4.410122in}}%
\pgfpathlineto{\pgfqpoint{3.558996in}{4.173101in}}%
\pgfpathlineto{\pgfqpoint{3.576837in}{4.082255in}}%
\pgfpathlineto{\pgfqpoint{3.597728in}{4.056574in}}%
\pgfpathlineto{\pgfqpoint{3.615801in}{4.041501in}}%
\pgfpathlineto{\pgfqpoint{3.634580in}{4.037471in}}%
\pgfpathlineto{\pgfqpoint{3.653828in}{4.036588in}}%
\pgfpathlineto{\pgfqpoint{3.676596in}{4.042512in}}%
\pgfpathlineto{\pgfqpoint{3.691149in}{4.054113in}}%
\pgfpathlineto{\pgfqpoint{3.715327in}{4.082556in}}%
\pgfpathlineto{\pgfqpoint{3.733401in}{4.151807in}}%
\pgfpathlineto{\pgfqpoint{3.750772in}{4.283675in}}%
\pgfpathlineto{\pgfqpoint{3.770722in}{4.426350in}}%
\pgfpathlineto{\pgfqpoint{3.786684in}{4.429874in}}%
\pgfpathlineto{\pgfqpoint{3.807341in}{4.332084in}}%
\pgfpathlineto{\pgfqpoint{3.826119in}{4.197596in}}%
\pgfpathlineto{\pgfqpoint{3.847010in}{4.093653in}}%
\pgfpathlineto{\pgfqpoint{3.866963in}{4.063963in}}%
\pgfpathlineto{\pgfqpoint{3.882923in}{4.046577in}}%
\pgfpathlineto{\pgfqpoint{3.903579in}{4.038905in}}%
\pgfpathlineto{\pgfqpoint{3.922358in}{4.036628in}}%
\pgfpathlineto{\pgfqpoint{3.946065in}{4.041913in}}%
\pgfpathlineto{\pgfqpoint{3.961324in}{4.049731in}}%
\pgfpathlineto{\pgfqpoint{3.979163in}{4.069250in}}%
\pgfpathlineto{\pgfqpoint{4.000288in}{4.124443in}}%
\pgfpathlineto{\pgfqpoint{4.018362in}{4.231108in}}%
\pgfpathlineto{\pgfqpoint{4.038783in}{4.425919in}}%
\pgfpathlineto{\pgfqpoint{4.056154in}{4.444742in}}%
\pgfpathlineto{\pgfqpoint{4.074462in}{4.403778in}}%
\pgfpathlineto{\pgfqpoint{4.096058in}{4.270305in}}%
\pgfpathlineto{\pgfqpoint{4.114366in}{4.192802in}}%
\pgfpathlineto{\pgfqpoint{4.136196in}{4.129124in}}%
\pgfpathlineto{\pgfqpoint{4.154975in}{4.069111in}}%
\pgfpathlineto{\pgfqpoint{4.171640in}{4.049252in}}%
\pgfpathlineto{\pgfqpoint{4.193001in}{4.038613in}}%
\pgfpathlineto{\pgfqpoint{4.210840in}{4.037813in}}%
\pgfpathlineto{\pgfqpoint{4.232905in}{4.045277in}}%
\pgfpathlineto{\pgfqpoint{4.250041in}{4.058529in}}%
\pgfpathlineto{\pgfqpoint{4.267644in}{4.094230in}}%
\pgfpathlineto{\pgfqpoint{4.288536in}{4.110860in}}%
\pgfpathlineto{\pgfqpoint{4.307079in}{4.222674in}}%
\pgfpathlineto{\pgfqpoint{4.327969in}{4.423479in}}%
\pgfpathlineto{\pgfqpoint{4.342054in}{4.450170in}}%
\pgfpathlineto{\pgfqpoint{4.365058in}{4.440284in}}%
\pgfpathlineto{\pgfqpoint{4.382427in}{4.375006in}}%
\pgfpathlineto{\pgfqpoint{4.404491in}{4.216280in}}%
\pgfpathlineto{\pgfqpoint{4.425147in}{4.102775in}}%
\pgfpathlineto{\pgfqpoint{4.441109in}{4.072198in}}%
\pgfpathlineto{\pgfqpoint{4.460123in}{4.048075in}}%
\pgfpathlineto{\pgfqpoint{4.481484in}{4.038592in}}%
\pgfpathlineto{\pgfqpoint{4.481013in}{4.042432in}}%
\pgfpathlineto{\pgfqpoint{4.471390in}{4.049912in}}%
\pgfpathlineto{\pgfqpoint{4.456837in}{4.079816in}}%
\pgfpathlineto{\pgfqpoint{4.436650in}{4.217371in}}%
\pgfpathlineto{\pgfqpoint{4.417636in}{4.399470in}}%
\pgfpathlineto{\pgfqpoint{4.398389in}{4.457929in}}%
\pgfpathlineto{\pgfqpoint{4.379375in}{4.039265in}}%
\pgfpathlineto{\pgfqpoint{4.357782in}{4.041109in}}%
\pgfpathlineto{\pgfqpoint{4.339472in}{4.062104in}}%
\pgfpathlineto{\pgfqpoint{4.320929in}{4.133763in}}%
\pgfpathlineto{\pgfqpoint{4.303090in}{4.328974in}}%
\pgfpathlineto{\pgfqpoint{4.283606in}{4.445046in}}%
\pgfpathlineto{\pgfqpoint{4.259430in}{4.345177in}}%
\pgfpathlineto{\pgfqpoint{4.239477in}{4.109459in}}%
\pgfpathlineto{\pgfqpoint{4.222108in}{4.061854in}}%
\pgfpathlineto{\pgfqpoint{4.203798in}{4.040316in}}%
\pgfpathlineto{\pgfqpoint{4.185490in}{4.037959in}}%
\pgfpathlineto{\pgfqpoint{4.166477in}{4.049681in}}%
\pgfpathlineto{\pgfqpoint{4.145116in}{4.105619in}}%
\pgfpathlineto{\pgfqpoint{4.129624in}{4.233942in}}%
\pgfpathlineto{\pgfqpoint{4.108734in}{4.420807in}}%
\pgfpathlineto{\pgfqpoint{4.089721in}{4.433001in}}%
\pgfpathlineto{\pgfqpoint{4.073290in}{4.271331in}}%
\pgfpathlineto{\pgfqpoint{4.050989in}{4.084442in}}%
\pgfpathlineto{\pgfqpoint{4.034793in}{4.048210in}}%
\pgfpathlineto{\pgfqpoint{4.013199in}{4.036658in}}%
\pgfpathlineto{\pgfqpoint{3.993951in}{4.040276in}}%
\pgfpathlineto{\pgfqpoint{3.975172in}{4.049967in}}%
\pgfpathlineto{\pgfqpoint{3.957802in}{4.093762in}}%
\pgfpathlineto{\pgfqpoint{3.937146in}{4.230630in}}%
\pgfpathlineto{\pgfqpoint{3.915786in}{4.409167in}}%
\pgfpathlineto{\pgfqpoint{3.896068in}{4.424908in}}%
\pgfpathlineto{\pgfqpoint{3.878697in}{4.252067in}}%
\pgfpathlineto{\pgfqpoint{3.858512in}{4.087108in}}%
\pgfpathlineto{\pgfqpoint{3.840673in}{4.049016in}}%
\pgfpathlineto{\pgfqpoint{3.820486in}{4.038838in}}%
\pgfpathlineto{\pgfqpoint{3.802412in}{4.036002in}}%
\pgfpathlineto{\pgfqpoint{3.782225in}{4.042025in}}%
\pgfpathlineto{\pgfqpoint{3.761803in}{4.067786in}}%
\pgfpathlineto{\pgfqpoint{3.744432in}{4.147045in}}%
\pgfpathlineto{\pgfqpoint{3.726828in}{4.314042in}}%
\pgfpathlineto{\pgfqpoint{3.704999in}{4.428262in}}%
\pgfpathlineto{\pgfqpoint{3.683169in}{4.377222in}}%
\pgfpathlineto{\pgfqpoint{3.666738in}{4.191143in}}%
\pgfpathlineto{\pgfqpoint{3.649603in}{4.094321in}}%
\pgfpathlineto{\pgfqpoint{3.629415in}{4.052723in}}%
\pgfpathlineto{\pgfqpoint{3.607351in}{4.037824in}}%
\pgfpathlineto{\pgfqpoint{3.589277in}{4.037815in}}%
\pgfpathlineto{\pgfqpoint{3.572377in}{4.036380in}}%
\pgfpathlineto{\pgfqpoint{3.552893in}{4.044667in}}%
\pgfpathlineto{\pgfqpoint{3.531300in}{4.071801in}}%
\pgfpathlineto{\pgfqpoint{3.513695in}{4.100491in}}%
\pgfpathlineto{\pgfqpoint{3.476137in}{4.400433in}}%
\pgfpathlineto{\pgfqpoint{3.455715in}{4.405047in}}%
\pgfpathlineto{\pgfqpoint{3.436702in}{4.202141in}}%
\pgfpathlineto{\pgfqpoint{3.412995in}{4.069644in}}%
\pgfpathlineto{\pgfqpoint{3.397972in}{4.278888in}}%
\pgfpathlineto{\pgfqpoint{3.378959in}{4.104593in}}%
\pgfpathlineto{\pgfqpoint{3.360885in}{4.061092in}}%
\pgfpathlineto{\pgfqpoint{3.340933in}{4.040934in}}%
\pgfpathlineto{\pgfqpoint{3.320276in}{4.035740in}}%
\pgfpathlineto{\pgfqpoint{3.301029in}{4.040385in}}%
\pgfpathlineto{\pgfqpoint{3.283424in}{4.054392in}}%
\pgfpathlineto{\pgfqpoint{3.262064in}{4.114717in}}%
\pgfpathlineto{\pgfqpoint{3.245163in}{4.266853in}}%
\pgfpathlineto{\pgfqpoint{3.225212in}{4.386373in}}%
\pgfpathlineto{\pgfqpoint{3.204556in}{4.410910in}}%
\pgfpathlineto{\pgfqpoint{3.186951in}{4.232854in}}%
\pgfpathlineto{\pgfqpoint{3.162538in}{4.096234in}}%
\pgfpathlineto{\pgfqpoint{3.148221in}{4.059448in}}%
\pgfpathlineto{\pgfqpoint{3.129911in}{4.040694in}}%
\pgfpathlineto{\pgfqpoint{3.109021in}{4.201119in}}%
\pgfpathlineto{\pgfqpoint{3.089068in}{4.076175in}}%
\pgfpathlineto{\pgfqpoint{3.070525in}{4.044973in}}%
\pgfpathlineto{\pgfqpoint{3.052452in}{4.036731in}}%
\pgfpathlineto{\pgfqpoint{3.031559in}{4.037581in}}%
\pgfpathlineto{\pgfqpoint{3.013720in}{4.045860in}}%
\pgfpathlineto{\pgfqpoint{2.996115in}{4.073080in}}%
\pgfpathlineto{\pgfqpoint{2.975695in}{4.193597in}}%
\pgfpathlineto{\pgfqpoint{2.957151in}{4.342199in}}%
\pgfpathlineto{\pgfqpoint{2.935790in}{4.415939in}}%
\pgfpathlineto{\pgfqpoint{2.919593in}{4.301190in}}%
\pgfpathlineto{\pgfqpoint{2.901285in}{4.129963in}}%
\pgfpathlineto{\pgfqpoint{2.878281in}{4.055555in}}%
\pgfpathlineto{\pgfqpoint{2.860207in}{4.042824in}}%
\pgfpathlineto{\pgfqpoint{2.838377in}{4.035727in}}%
\pgfpathlineto{\pgfqpoint{2.822181in}{4.036291in}}%
\pgfpathlineto{\pgfqpoint{2.804342in}{4.039586in}}%
\pgfpathlineto{\pgfqpoint{2.782748in}{4.052586in}}%
\pgfpathlineto{\pgfqpoint{2.764203in}{4.105237in}}%
\pgfpathlineto{\pgfqpoint{2.745895in}{4.221337in}}%
\pgfpathlineto{\pgfqpoint{2.726177in}{4.392270in}}%
\pgfpathlineto{\pgfqpoint{2.706226in}{4.412718in}}%
\pgfpathlineto{\pgfqpoint{2.682518in}{4.292369in}}%
\pgfpathlineto{\pgfqpoint{2.667260in}{4.140600in}}%
\pgfpathlineto{\pgfqpoint{2.648247in}{4.067089in}}%
\pgfpathlineto{\pgfqpoint{2.628999in}{4.058925in}}%
\pgfpathlineto{\pgfqpoint{2.610456in}{4.040685in}}%
\pgfpathlineto{\pgfqpoint{2.592615in}{4.035654in}}%
\pgfpathlineto{\pgfqpoint{2.570787in}{4.039050in}}%
\pgfpathlineto{\pgfqpoint{2.552711in}{4.053064in}}%
\pgfpathlineto{\pgfqpoint{2.533464in}{4.102852in}}%
\pgfpathlineto{\pgfqpoint{2.513278in}{4.216506in}}%
\pgfpathlineto{\pgfqpoint{2.494265in}{4.372296in}}%
\pgfpathlineto{\pgfqpoint{2.474547in}{4.410068in}}%
\pgfpathlineto{\pgfqpoint{2.457413in}{4.266425in}}%
\pgfpathlineto{\pgfqpoint{2.437460in}{4.109251in}}%
\pgfpathlineto{\pgfqpoint{2.415630in}{4.052652in}}%
\pgfpathlineto{\pgfqpoint{2.400373in}{4.040733in}}%
\pgfpathlineto{\pgfqpoint{2.379011in}{4.035937in}}%
\pgfpathlineto{\pgfqpoint{2.360000in}{4.037294in}}%
\pgfpathlineto{\pgfqpoint{2.342395in}{4.042641in}}%
\pgfpathlineto{\pgfqpoint{2.323617in}{4.064537in}}%
\pgfpathlineto{\pgfqpoint{2.304603in}{4.140841in}}%
\pgfpathlineto{\pgfqpoint{2.282070in}{4.272598in}}%
\pgfpathlineto{\pgfqpoint{2.263994in}{4.397384in}}%
\pgfpathlineto{\pgfqpoint{2.245452in}{4.406571in}}%
\pgfpathlineto{\pgfqpoint{2.224561in}{4.241480in}}%
\pgfpathlineto{\pgfqpoint{2.206017in}{4.106977in}}%
\pgfpathlineto{\pgfqpoint{2.187238in}{4.056791in}}%
\pgfpathlineto{\pgfqpoint{2.169399in}{4.042382in}}%
\pgfpathlineto{\pgfqpoint{2.148743in}{4.036228in}}%
\pgfpathlineto{\pgfqpoint{2.129026in}{4.036941in}}%
\pgfpathlineto{\pgfqpoint{2.110013in}{4.042784in}}%
\pgfpathlineto{\pgfqpoint{2.090765in}{4.063065in}}%
\pgfpathlineto{\pgfqpoint{2.069875in}{4.124264in}}%
\pgfpathlineto{\pgfqpoint{2.035839in}{4.390962in}}%
\pgfpathlineto{\pgfqpoint{2.015886in}{4.404688in}}%
\pgfpathlineto{\pgfqpoint{1.995699in}{4.311385in}}%
\pgfpathlineto{\pgfqpoint{1.974574in}{4.134276in}}%
\pgfpathlineto{\pgfqpoint{1.956969in}{4.066367in}}%
\pgfpathlineto{\pgfqpoint{1.940069in}{4.039090in}}%
\pgfpathlineto{\pgfqpoint{1.919882in}{4.048252in}}%
\pgfpathlineto{\pgfqpoint{1.900869in}{4.092408in}}%
\pgfpathlineto{\pgfqpoint{1.879508in}{4.228907in}}%
\pgfpathlineto{\pgfqpoint{1.861668in}{4.378286in}}%
\pgfpathlineto{\pgfqpoint{1.842186in}{4.410739in}}%
\pgfpathlineto{\pgfqpoint{1.820827in}{4.287310in}}%
\pgfpathlineto{\pgfqpoint{1.803925in}{4.113019in}}%
\pgfpathlineto{\pgfqpoint{1.786086in}{4.065684in}}%
\pgfpathlineto{\pgfqpoint{1.764727in}{4.045317in}}%
\pgfpathlineto{\pgfqpoint{1.746885in}{4.036136in}}%
\pgfpathlineto{\pgfqpoint{1.726700in}{4.037756in}}%
\pgfpathlineto{\pgfqpoint{1.707687in}{4.047610in}}%
\pgfpathlineto{\pgfqpoint{1.686326in}{4.099841in}}%
\pgfpathlineto{\pgfqpoint{1.664732in}{4.232366in}}%
\pgfpathlineto{\pgfqpoint{1.649707in}{4.370148in}}%
\pgfpathlineto{\pgfqpoint{1.628348in}{4.414188in}}%
\pgfpathlineto{\pgfqpoint{1.611212in}{4.391152in}}%
\pgfpathlineto{\pgfqpoint{1.592904in}{4.205638in}}%
\pgfpathlineto{\pgfqpoint{1.574830in}{4.083439in}}%
\pgfpathlineto{\pgfqpoint{1.554174in}{4.048226in}}%
\pgfpathlineto{\pgfqpoint{1.533047in}{4.038311in}}%
\pgfpathlineto{\pgfqpoint{1.516851in}{4.035960in}}%
\pgfpathlineto{\pgfqpoint{1.495726in}{4.042186in}}%
\pgfpathlineto{\pgfqpoint{1.476713in}{4.057584in}}%
\pgfpathlineto{\pgfqpoint{1.455353in}{4.126938in}}%
\pgfpathlineto{\pgfqpoint{1.439626in}{4.257343in}}%
\pgfpathlineto{\pgfqpoint{1.418501in}{4.396623in}}%
\pgfpathlineto{\pgfqpoint{1.402068in}{4.426168in}}%
\pgfpathlineto{\pgfqpoint{1.380709in}{4.362608in}}%
\pgfpathlineto{\pgfqpoint{1.360756in}{4.137168in}}%
\pgfpathlineto{\pgfqpoint{1.344560in}{4.070763in}}%
\pgfpathlineto{\pgfqpoint{1.324138in}{4.045266in}}%
\pgfpathlineto{\pgfqpoint{1.303952in}{4.038186in}}%
\pgfpathlineto{\pgfqpoint{1.284705in}{4.036995in}}%
\pgfpathlineto{\pgfqpoint{1.263343in}{4.046703in}}%
\pgfpathlineto{\pgfqpoint{1.248087in}{4.055092in}}%
\pgfpathlineto{\pgfqpoint{1.227665in}{4.103484in}}%
\pgfpathlineto{\pgfqpoint{1.186353in}{4.356282in}}%
\pgfpathlineto{\pgfqpoint{1.167808in}{4.428949in}}%
\pgfpathlineto{\pgfqpoint{1.149266in}{4.424035in}}%
\pgfpathlineto{\pgfqpoint{1.131661in}{4.272281in}}%
\pgfpathlineto{\pgfqpoint{1.112413in}{4.134866in}}%
\pgfpathlineto{\pgfqpoint{1.093166in}{4.148902in}}%
\pgfpathlineto{\pgfqpoint{1.071570in}{4.066516in}}%
\pgfpathlineto{\pgfqpoint{1.053496in}{4.047210in}}%
\pgfpathlineto{\pgfqpoint{1.031900in}{4.037441in}}%
\pgfpathlineto{\pgfqpoint{1.012887in}{4.037758in}}%
\pgfpathlineto{\pgfqpoint{0.995282in}{4.044704in}}%
\pgfpathlineto{\pgfqpoint{0.976740in}{4.061727in}}%
\pgfpathlineto{\pgfqpoint{0.956084in}{4.109329in}}%
\pgfpathlineto{\pgfqpoint{0.937774in}{4.207885in}}%
\pgfpathlineto{\pgfqpoint{0.921812in}{4.308987in}}%
\pgfpathlineto{\pgfqpoint{0.902330in}{4.418215in}}%
\pgfpathlineto{\pgfqpoint{0.883317in}{4.443092in}}%
\pgfpathlineto{\pgfqpoint{0.861723in}{4.330267in}}%
\pgfpathlineto{\pgfqpoint{0.843647in}{4.205836in}}%
\pgfpathlineto{\pgfqpoint{0.824870in}{4.102047in}}%
\pgfpathlineto{\pgfqpoint{0.803978in}{4.059602in}}%
\pgfpathlineto{\pgfqpoint{0.785904in}{4.043635in}}%
\pgfpathlineto{\pgfqpoint{0.765248in}{4.038400in}}%
\pgfpathlineto{\pgfqpoint{0.746235in}{4.039168in}}%
\pgfpathlineto{\pgfqpoint{0.726753in}{4.049380in}}%
\pgfpathlineto{\pgfqpoint{0.708679in}{4.038533in}}%
\pgfpathlineto{\pgfqpoint{0.689666in}{4.047704in}}%
\pgfpathlineto{\pgfqpoint{0.667835in}{4.079419in}}%
\pgfpathlineto{\pgfqpoint{0.649762in}{4.144974in}}%
\pgfpathlineto{\pgfqpoint{0.649525in}{4.143592in}}%
\pgfpathlineto{\pgfqpoint{0.658916in}{4.094525in}}%
\pgfpathlineto{\pgfqpoint{0.676521in}{4.051102in}}%
\pgfpathlineto{\pgfqpoint{0.694829in}{4.038486in}}%
\pgfpathlineto{\pgfqpoint{0.715485in}{4.042804in}}%
\pgfpathlineto{\pgfqpoint{0.733795in}{4.063200in}}%
\pgfpathlineto{\pgfqpoint{0.751165in}{4.128509in}}%
\pgfpathlineto{\pgfqpoint{0.772759in}{4.374820in}}%
\pgfpathlineto{\pgfqpoint{0.791069in}{4.447928in}}%
\pgfpathlineto{\pgfqpoint{0.810786in}{4.362002in}}%
\pgfpathlineto{\pgfqpoint{0.830268in}{4.150862in}}%
\pgfpathlineto{\pgfqpoint{0.848578in}{4.066941in}}%
\pgfpathlineto{\pgfqpoint{0.870406in}{4.040928in}}%
\pgfpathlineto{\pgfqpoint{0.884256in}{4.037037in}}%
\pgfpathlineto{\pgfqpoint{0.906321in}{4.047851in}}%
\pgfpathlineto{\pgfqpoint{0.924629in}{4.079637in}}%
\pgfpathlineto{\pgfqpoint{0.945051in}{4.222260in}}%
\pgfpathlineto{\pgfqpoint{0.965472in}{4.432528in}}%
\pgfpathlineto{\pgfqpoint{0.981903in}{4.401968in}}%
\pgfpathlineto{\pgfqpoint{1.003030in}{4.192935in}}%
\pgfpathlineto{\pgfqpoint{1.021338in}{4.104107in}}%
\pgfpathlineto{\pgfqpoint{1.041760in}{4.051507in}}%
\pgfpathlineto{\pgfqpoint{1.061007in}{4.037754in}}%
\pgfpathlineto{\pgfqpoint{1.082838in}{4.039554in}}%
\pgfpathlineto{\pgfqpoint{1.097155in}{4.053181in}}%
\pgfpathlineto{\pgfqpoint{1.119219in}{4.114561in}}%
\pgfpathlineto{\pgfqpoint{1.155837in}{4.430976in}}%
\pgfpathlineto{\pgfqpoint{1.176728in}{4.338384in}}%
\pgfpathlineto{\pgfqpoint{1.195741in}{4.147449in}}%
\pgfpathlineto{\pgfqpoint{1.213580in}{4.069107in}}%
\pgfpathlineto{\pgfqpoint{1.234707in}{4.040565in}}%
\pgfpathlineto{\pgfqpoint{1.254424in}{4.036346in}}%
\pgfpathlineto{\pgfqpoint{1.276254in}{4.044242in}}%
\pgfpathlineto{\pgfqpoint{1.291511in}{4.060943in}}%
\pgfpathlineto{\pgfqpoint{1.312638in}{4.139650in}}%
\pgfpathlineto{\pgfqpoint{1.327660in}{4.330767in}}%
\pgfpathlineto{\pgfqpoint{1.346439in}{4.425642in}}%
\pgfpathlineto{\pgfqpoint{1.365686in}{4.359130in}}%
\pgfpathlineto{\pgfqpoint{1.387517in}{4.153048in}}%
\pgfpathlineto{\pgfqpoint{1.407468in}{4.065972in}}%
\pgfpathlineto{\pgfqpoint{1.426246in}{4.041565in}}%
\pgfpathlineto{\pgfqpoint{1.445260in}{4.036073in}}%
\pgfpathlineto{\pgfqpoint{1.465681in}{4.040309in}}%
\pgfpathlineto{\pgfqpoint{1.483050in}{4.058843in}}%
\pgfpathlineto{\pgfqpoint{1.503706in}{4.120550in}}%
\pgfpathlineto{\pgfqpoint{1.521311in}{4.312609in}}%
\pgfpathlineto{\pgfqpoint{1.540793in}{4.422747in}}%
\pgfpathlineto{\pgfqpoint{1.559572in}{4.357981in}}%
\pgfpathlineto{\pgfqpoint{1.579993in}{4.181142in}}%
\pgfpathlineto{\pgfqpoint{1.600181in}{4.081489in}}%
\pgfpathlineto{\pgfqpoint{1.619663in}{4.046116in}}%
\pgfpathlineto{\pgfqpoint{1.634920in}{4.040897in}}%
\pgfpathlineto{\pgfqpoint{1.656046in}{4.035898in}}%
\pgfpathlineto{\pgfqpoint{1.675763in}{4.041117in}}%
\pgfpathlineto{\pgfqpoint{1.694542in}{4.055638in}}%
\pgfpathlineto{\pgfqpoint{1.713790in}{4.087216in}}%
\pgfpathlineto{\pgfqpoint{1.733740in}{4.162829in}}%
\pgfpathlineto{\pgfqpoint{1.751816in}{4.386513in}}%
\pgfpathlineto{\pgfqpoint{1.769655in}{4.415907in}}%
\pgfpathlineto{\pgfqpoint{1.793597in}{4.279890in}}%
\pgfpathlineto{\pgfqpoint{1.810028in}{4.117660in}}%
\pgfpathlineto{\pgfqpoint{1.829981in}{4.065540in}}%
\pgfpathlineto{\pgfqpoint{1.852983in}{4.041071in}}%
\pgfpathlineto{\pgfqpoint{1.886315in}{4.035595in}}%
\pgfpathlineto{\pgfqpoint{1.908380in}{4.041478in}}%
\pgfpathlineto{\pgfqpoint{1.927159in}{4.059296in}}%
\pgfpathlineto{\pgfqpoint{1.944529in}{4.083613in}}%
\pgfpathlineto{\pgfqpoint{1.963541in}{4.106343in}}%
\pgfpathlineto{\pgfqpoint{1.981382in}{4.278562in}}%
\pgfpathlineto{\pgfqpoint{2.003446in}{4.415677in}}%
\pgfpathlineto{\pgfqpoint{2.019877in}{4.378784in}}%
\pgfpathlineto{\pgfqpoint{2.041471in}{4.200422in}}%
\pgfpathlineto{\pgfqpoint{2.060250in}{4.080269in}}%
\pgfpathlineto{\pgfqpoint{2.078794in}{4.410049in}}%
\pgfpathlineto{\pgfqpoint{2.097807in}{4.289223in}}%
\pgfpathlineto{\pgfqpoint{2.118698in}{4.113949in}}%
\pgfpathlineto{\pgfqpoint{2.133720in}{4.061423in}}%
\pgfpathlineto{\pgfqpoint{2.155316in}{4.040321in}}%
\pgfpathlineto{\pgfqpoint{2.173390in}{4.037285in}}%
\pgfpathlineto{\pgfqpoint{2.192403in}{4.036168in}}%
\pgfpathlineto{\pgfqpoint{2.216110in}{4.047956in}}%
\pgfpathlineto{\pgfqpoint{2.230898in}{4.071836in}}%
\pgfpathlineto{\pgfqpoint{2.252023in}{4.183569in}}%
\pgfpathlineto{\pgfqpoint{2.270333in}{4.367271in}}%
\pgfpathlineto{\pgfqpoint{2.291458in}{4.417560in}}%
\pgfpathlineto{\pgfqpoint{2.309063in}{4.361518in}}%
\pgfpathlineto{\pgfqpoint{2.326199in}{4.197001in}}%
\pgfpathlineto{\pgfqpoint{2.347324in}{4.074181in}}%
\pgfpathlineto{\pgfqpoint{2.363989in}{4.043777in}}%
\pgfpathlineto{\pgfqpoint{2.388636in}{4.036067in}}%
\pgfpathlineto{\pgfqpoint{2.405772in}{4.036662in}}%
\pgfpathlineto{\pgfqpoint{2.424549in}{4.043796in}}%
\pgfpathlineto{\pgfqpoint{2.441216in}{4.066582in}}%
\pgfpathlineto{\pgfqpoint{2.463984in}{4.125433in}}%
\pgfpathlineto{\pgfqpoint{2.480415in}{4.270451in}}%
\pgfpathlineto{\pgfqpoint{2.500602in}{4.409658in}}%
\pgfpathlineto{\pgfqpoint{2.519850in}{4.377409in}}%
\pgfpathlineto{\pgfqpoint{2.540037in}{4.200252in}}%
\pgfpathlineto{\pgfqpoint{2.558816in}{4.094385in}}%
\pgfpathlineto{\pgfqpoint{2.579472in}{4.047446in}}%
\pgfpathlineto{\pgfqpoint{2.597780in}{4.039029in}}%
\pgfpathlineto{\pgfqpoint{2.616088in}{4.035501in}}%
\pgfpathlineto{\pgfqpoint{2.635572in}{4.036773in}}%
\pgfpathlineto{\pgfqpoint{2.657401in}{4.049650in}}%
\pgfpathlineto{\pgfqpoint{2.673597in}{4.066689in}}%
\pgfpathlineto{\pgfqpoint{2.693550in}{4.157115in}}%
\pgfpathlineto{\pgfqpoint{2.713503in}{4.337437in}}%
\pgfpathlineto{\pgfqpoint{2.732045in}{4.412149in}}%
\pgfpathlineto{\pgfqpoint{2.750355in}{4.362200in}}%
\pgfpathlineto{\pgfqpoint{2.769837in}{4.303548in}}%
\pgfpathlineto{\pgfqpoint{2.789788in}{4.141583in}}%
\pgfpathlineto{\pgfqpoint{2.813732in}{4.053203in}}%
\pgfpathlineto{\pgfqpoint{2.828754in}{4.040726in}}%
\pgfpathlineto{\pgfqpoint{2.847768in}{4.036082in}}%
\pgfpathlineto{\pgfqpoint{2.865841in}{4.050336in}}%
\pgfpathlineto{\pgfqpoint{2.887201in}{4.038002in}}%
\pgfpathlineto{\pgfqpoint{2.906685in}{4.035653in}}%
\pgfpathlineto{\pgfqpoint{2.924524in}{4.039397in}}%
\pgfpathlineto{\pgfqpoint{2.941894in}{4.052698in}}%
\pgfpathlineto{\pgfqpoint{2.963254in}{4.098739in}}%
\pgfpathlineto{\pgfqpoint{2.984615in}{4.207865in}}%
\pgfpathlineto{\pgfqpoint{3.001983in}{4.391115in}}%
\pgfpathlineto{\pgfqpoint{3.019588in}{4.404598in}}%
\pgfpathlineto{\pgfqpoint{3.037898in}{4.289522in}}%
\pgfpathlineto{\pgfqpoint{3.059728in}{4.118935in}}%
\pgfpathlineto{\pgfqpoint{3.077333in}{4.074447in}}%
\pgfpathlineto{\pgfqpoint{3.095641in}{4.046784in}}%
\pgfpathlineto{\pgfqpoint{3.116063in}{4.189053in}}%
\pgfpathlineto{\pgfqpoint{3.134371in}{4.084917in}}%
\pgfpathlineto{\pgfqpoint{3.155967in}{4.045699in}}%
\pgfpathlineto{\pgfqpoint{3.174746in}{4.036752in}}%
\pgfpathlineto{\pgfqpoint{3.194228in}{4.036572in}}%
\pgfpathlineto{\pgfqpoint{3.211362in}{4.039358in}}%
\pgfpathlineto{\pgfqpoint{3.232489in}{4.057109in}}%
\pgfpathlineto{\pgfqpoint{3.251736in}{4.102632in}}%
\pgfpathlineto{\pgfqpoint{3.268402in}{4.247203in}}%
\pgfpathlineto{\pgfqpoint{3.289528in}{4.404155in}}%
\pgfpathlineto{\pgfqpoint{3.308305in}{4.406066in}}%
\pgfpathlineto{\pgfqpoint{3.325676in}{4.286565in}}%
\pgfpathlineto{\pgfqpoint{3.346566in}{4.124482in}}%
\pgfpathlineto{\pgfqpoint{3.364642in}{4.070513in}}%
\pgfpathlineto{\pgfqpoint{3.383653in}{4.044584in}}%
\pgfpathlineto{\pgfqpoint{3.403372in}{4.038152in}}%
\pgfpathlineto{\pgfqpoint{3.424966in}{4.036098in}}%
\pgfpathlineto{\pgfqpoint{3.443276in}{4.040569in}}%
\pgfpathlineto{\pgfqpoint{3.461818in}{4.054285in}}%
\pgfpathlineto{\pgfqpoint{3.480597in}{4.086567in}}%
\pgfpathlineto{\pgfqpoint{3.501958in}{4.218000in}}%
\pgfpathlineto{\pgfqpoint{3.518155in}{4.388392in}}%
\pgfpathlineto{\pgfqpoint{3.539280in}{4.422388in}}%
\pgfpathlineto{\pgfqpoint{3.558293in}{4.335721in}}%
\pgfpathlineto{\pgfqpoint{3.578009in}{4.178177in}}%
\pgfpathlineto{\pgfqpoint{3.596319in}{4.085623in}}%
\pgfpathlineto{\pgfqpoint{3.614159in}{4.058965in}}%
\pgfpathlineto{\pgfqpoint{3.635049in}{4.042116in}}%
\pgfpathlineto{\pgfqpoint{3.653123in}{4.037451in}}%
\pgfpathlineto{\pgfqpoint{3.674953in}{4.037429in}}%
\pgfpathlineto{\pgfqpoint{3.690446in}{4.044182in}}%
\pgfpathlineto{\pgfqpoint{3.713214in}{4.069119in}}%
\pgfpathlineto{\pgfqpoint{3.729645in}{4.108206in}}%
\pgfpathlineto{\pgfqpoint{3.749832in}{4.202575in}}%
\pgfpathlineto{\pgfqpoint{3.768140in}{4.396994in}}%
\pgfpathlineto{\pgfqpoint{3.789736in}{4.434267in}}%
\pgfpathlineto{\pgfqpoint{3.809923in}{4.410232in}}%
\pgfpathlineto{\pgfqpoint{3.825649in}{4.307411in}}%
\pgfpathlineto{\pgfqpoint{3.847950in}{4.152963in}}%
\pgfpathlineto{\pgfqpoint{3.865084in}{4.107589in}}%
\pgfpathlineto{\pgfqpoint{3.883862in}{4.067237in}}%
\pgfpathlineto{\pgfqpoint{3.904284in}{4.043375in}}%
\pgfpathlineto{\pgfqpoint{3.922123in}{4.037387in}}%
\pgfpathlineto{\pgfqpoint{3.942545in}{4.038024in}}%
\pgfpathlineto{\pgfqpoint{3.961324in}{4.044992in}}%
\pgfpathlineto{\pgfqpoint{3.979866in}{4.063126in}}%
\pgfpathlineto{\pgfqpoint{4.002400in}{4.037024in}}%
\pgfpathlineto{\pgfqpoint{4.017424in}{4.041898in}}%
\pgfpathlineto{\pgfqpoint{4.039489in}{4.053039in}}%
\pgfpathlineto{\pgfqpoint{4.058502in}{4.091213in}}%
\pgfpathlineto{\pgfqpoint{4.076810in}{4.186562in}}%
\pgfpathlineto{\pgfqpoint{4.095118in}{4.403557in}}%
\pgfpathlineto{\pgfqpoint{4.114602in}{4.447336in}}%
\pgfpathlineto{\pgfqpoint{4.135493in}{4.398784in}}%
\pgfpathlineto{\pgfqpoint{4.153097in}{4.279556in}}%
\pgfpathlineto{\pgfqpoint{4.171406in}{4.146322in}}%
\pgfpathlineto{\pgfqpoint{4.192062in}{4.078125in}}%
\pgfpathlineto{\pgfqpoint{4.210372in}{4.050823in}}%
\pgfpathlineto{\pgfqpoint{4.231965in}{4.040901in}}%
\pgfpathlineto{\pgfqpoint{4.250510in}{4.037554in}}%
\pgfpathlineto{\pgfqpoint{4.270932in}{4.042972in}}%
\pgfpathlineto{\pgfqpoint{4.289005in}{4.060032in}}%
\pgfpathlineto{\pgfqpoint{4.307550in}{4.085237in}}%
\pgfpathlineto{\pgfqpoint{4.327501in}{4.187863in}}%
\pgfpathlineto{\pgfqpoint{4.343228in}{4.313889in}}%
\pgfpathlineto{\pgfqpoint{4.364588in}{4.446704in}}%
\pgfpathlineto{\pgfqpoint{4.382427in}{4.451753in}}%
\pgfpathlineto{\pgfqpoint{4.404257in}{4.371033in}}%
\pgfpathlineto{\pgfqpoint{4.441815in}{4.109622in}}%
\pgfpathlineto{\pgfqpoint{4.460828in}{4.069411in}}%
\pgfpathlineto{\pgfqpoint{4.480544in}{4.047355in}}%
\pgfpathlineto{\pgfqpoint{4.473973in}{4.061407in}}%
\pgfpathlineto{\pgfqpoint{4.453551in}{4.129151in}}%
\pgfpathlineto{\pgfqpoint{4.436181in}{4.307252in}}%
\pgfpathlineto{\pgfqpoint{4.418107in}{4.440826in}}%
\pgfpathlineto{\pgfqpoint{4.395337in}{4.432275in}}%
\pgfpathlineto{\pgfqpoint{4.379610in}{4.228270in}}%
\pgfpathlineto{\pgfqpoint{4.358016in}{4.085886in}}%
\pgfpathlineto{\pgfqpoint{4.340177in}{4.046505in}}%
\pgfpathlineto{\pgfqpoint{4.322101in}{4.037669in}}%
\pgfpathlineto{\pgfqpoint{4.300037in}{4.045134in}}%
\pgfpathlineto{\pgfqpoint{4.281729in}{4.073545in}}%
\pgfpathlineto{\pgfqpoint{4.264358in}{4.181880in}}%
\pgfpathlineto{\pgfqpoint{4.242059in}{4.394010in}}%
\pgfpathlineto{\pgfqpoint{4.223517in}{4.447858in}}%
\pgfpathlineto{\pgfqpoint{4.205675in}{4.350664in}}%
\pgfpathlineto{\pgfqpoint{4.186428in}{4.123976in}}%
\pgfpathlineto{\pgfqpoint{4.166008in}{4.057099in}}%
\pgfpathlineto{\pgfqpoint{4.147229in}{4.039503in}}%
\pgfpathlineto{\pgfqpoint{4.129154in}{4.037878in}}%
\pgfpathlineto{\pgfqpoint{4.108734in}{4.050067in}}%
\pgfpathlineto{\pgfqpoint{4.087607in}{4.107849in}}%
\pgfpathlineto{\pgfqpoint{4.069768in}{4.265099in}}%
\pgfpathlineto{\pgfqpoint{4.054277in}{4.427694in}}%
\pgfpathlineto{\pgfqpoint{4.033620in}{4.429469in}}%
\pgfpathlineto{\pgfqpoint{4.011790in}{4.192512in}}%
\pgfpathlineto{\pgfqpoint{3.994420in}{4.081117in}}%
\pgfpathlineto{\pgfqpoint{3.974467in}{4.044335in}}%
\pgfpathlineto{\pgfqpoint{3.957333in}{4.037021in}}%
\pgfpathlineto{\pgfqpoint{3.936442in}{4.040142in}}%
\pgfpathlineto{\pgfqpoint{3.915315in}{4.061483in}}%
\pgfpathlineto{\pgfqpoint{3.899119in}{4.084423in}}%
\pgfpathlineto{\pgfqpoint{3.878934in}{4.241646in}}%
\pgfpathlineto{\pgfqpoint{3.862501in}{4.393330in}}%
\pgfpathlineto{\pgfqpoint{3.839264in}{4.423954in}}%
\pgfpathlineto{\pgfqpoint{3.823537in}{4.254001in}}%
\pgfpathlineto{\pgfqpoint{3.802412in}{4.087454in}}%
\pgfpathlineto{\pgfqpoint{3.781990in}{4.047415in}}%
\pgfpathlineto{\pgfqpoint{3.765089in}{4.037347in}}%
\pgfpathlineto{\pgfqpoint{3.744432in}{4.037765in}}%
\pgfpathlineto{\pgfqpoint{3.724247in}{4.049684in}}%
\pgfpathlineto{\pgfqpoint{3.704060in}{4.093940in}}%
\pgfpathlineto{\pgfqpoint{3.686924in}{4.224782in}}%
\pgfpathlineto{\pgfqpoint{3.666973in}{4.395649in}}%
\pgfpathlineto{\pgfqpoint{3.646551in}{4.416598in}}%
\pgfpathlineto{\pgfqpoint{3.628712in}{4.274832in}}%
\pgfpathlineto{\pgfqpoint{3.611576in}{4.121843in}}%
\pgfpathlineto{\pgfqpoint{3.589511in}{4.051996in}}%
\pgfpathlineto{\pgfqpoint{3.571672in}{4.040212in}}%
\pgfpathlineto{\pgfqpoint{3.551721in}{4.035812in}}%
\pgfpathlineto{\pgfqpoint{3.533411in}{4.040778in}}%
\pgfpathlineto{\pgfqpoint{3.512990in}{4.065574in}}%
\pgfpathlineto{\pgfqpoint{3.494681in}{4.123053in}}%
\pgfpathlineto{\pgfqpoint{3.474963in}{4.267916in}}%
\pgfpathlineto{\pgfqpoint{3.457358in}{4.355614in}}%
\pgfpathlineto{\pgfqpoint{3.436233in}{4.418938in}}%
\pgfpathlineto{\pgfqpoint{3.414637in}{4.249261in}}%
\pgfpathlineto{\pgfqpoint{3.397503in}{4.098506in}}%
\pgfpathlineto{\pgfqpoint{3.379193in}{4.061926in}}%
\pgfpathlineto{\pgfqpoint{3.357834in}{4.070041in}}%
\pgfpathlineto{\pgfqpoint{3.338586in}{4.044506in}}%
\pgfpathlineto{\pgfqpoint{3.321216in}{4.036476in}}%
\pgfpathlineto{\pgfqpoint{3.302437in}{4.037639in}}%
\pgfpathlineto{\pgfqpoint{3.281078in}{4.050946in}}%
\pgfpathlineto{\pgfqpoint{3.263707in}{4.091809in}}%
\pgfpathlineto{\pgfqpoint{3.243754in}{4.190328in}}%
\pgfpathlineto{\pgfqpoint{3.228498in}{4.352588in}}%
\pgfpathlineto{\pgfqpoint{3.207842in}{4.419421in}}%
\pgfpathlineto{\pgfqpoint{3.187654in}{4.377732in}}%
\pgfpathlineto{\pgfqpoint{3.167235in}{4.138366in}}%
\pgfpathlineto{\pgfqpoint{3.148925in}{4.065828in}}%
\pgfpathlineto{\pgfqpoint{3.127565in}{4.042750in}}%
\pgfpathlineto{\pgfqpoint{3.110664in}{4.035917in}}%
\pgfpathlineto{\pgfqpoint{3.091181in}{4.036610in}}%
\pgfpathlineto{\pgfqpoint{3.069586in}{4.045159in}}%
\pgfpathlineto{\pgfqpoint{3.051981in}{4.071713in}}%
\pgfpathlineto{\pgfqpoint{3.030622in}{4.190887in}}%
\pgfpathlineto{\pgfqpoint{3.015363in}{4.314805in}}%
\pgfpathlineto{\pgfqpoint{2.995412in}{4.414271in}}%
\pgfpathlineto{\pgfqpoint{2.973816in}{4.295915in}}%
\pgfpathlineto{\pgfqpoint{2.956211in}{4.132323in}}%
\pgfpathlineto{\pgfqpoint{2.938138in}{4.099042in}}%
\pgfpathlineto{\pgfqpoint{2.916308in}{4.050847in}}%
\pgfpathlineto{\pgfqpoint{2.897529in}{4.039009in}}%
\pgfpathlineto{\pgfqpoint{2.878281in}{4.035402in}}%
\pgfpathlineto{\pgfqpoint{2.859033in}{4.038282in}}%
\pgfpathlineto{\pgfqpoint{2.840020in}{4.043410in}}%
\pgfpathlineto{\pgfqpoint{2.821478in}{4.068638in}}%
\pgfpathlineto{\pgfqpoint{2.802699in}{4.159507in}}%
\pgfpathlineto{\pgfqpoint{2.782043in}{4.345061in}}%
\pgfpathlineto{\pgfqpoint{2.765143in}{4.410868in}}%
\pgfpathlineto{\pgfqpoint{2.743782in}{4.369644in}}%
\pgfpathlineto{\pgfqpoint{2.725708in}{4.165377in}}%
\pgfpathlineto{\pgfqpoint{2.706226in}{4.076846in}}%
\pgfpathlineto{\pgfqpoint{2.688621in}{4.189357in}}%
\pgfpathlineto{\pgfqpoint{2.668668in}{4.359012in}}%
\pgfpathlineto{\pgfqpoint{2.649889in}{4.418341in}}%
\pgfpathlineto{\pgfqpoint{2.625244in}{4.254190in}}%
\pgfpathlineto{\pgfqpoint{2.609986in}{4.160968in}}%
\pgfpathlineto{\pgfqpoint{2.591678in}{4.067306in}}%
\pgfpathlineto{\pgfqpoint{2.568439in}{4.045767in}}%
\pgfpathlineto{\pgfqpoint{2.554356in}{4.037958in}}%
\pgfpathlineto{\pgfqpoint{2.536515in}{4.035988in}}%
\pgfpathlineto{\pgfqpoint{2.516095in}{4.042204in}}%
\pgfpathlineto{\pgfqpoint{2.493560in}{4.074644in}}%
\pgfpathlineto{\pgfqpoint{2.475955in}{4.172691in}}%
\pgfpathlineto{\pgfqpoint{2.457413in}{4.355605in}}%
\pgfpathlineto{\pgfqpoint{2.439573in}{4.413970in}}%
\pgfpathlineto{\pgfqpoint{2.418681in}{4.289729in}}%
\pgfpathlineto{\pgfqpoint{2.397087in}{4.103023in}}%
\pgfpathlineto{\pgfqpoint{2.379011in}{4.054589in}}%
\pgfpathlineto{\pgfqpoint{2.359764in}{4.042358in}}%
\pgfpathlineto{\pgfqpoint{2.340752in}{4.037238in}}%
\pgfpathlineto{\pgfqpoint{2.322208in}{4.035748in}}%
\pgfpathlineto{\pgfqpoint{2.304134in}{4.040858in}}%
\pgfpathlineto{\pgfqpoint{2.282304in}{4.063251in}}%
\pgfpathlineto{\pgfqpoint{2.264465in}{4.137065in}}%
\pgfpathlineto{\pgfqpoint{2.245452in}{4.295195in}}%
\pgfpathlineto{\pgfqpoint{2.227847in}{4.402810in}}%
\pgfpathlineto{\pgfqpoint{2.205314in}{4.375277in}}%
\pgfpathlineto{\pgfqpoint{2.190760in}{4.255670in}}%
\pgfpathlineto{\pgfqpoint{2.167521in}{4.093098in}}%
\pgfpathlineto{\pgfqpoint{2.149448in}{4.052208in}}%
\pgfpathlineto{\pgfqpoint{2.131138in}{4.039342in}}%
\pgfpathlineto{\pgfqpoint{2.110013in}{4.036493in}}%
\pgfpathlineto{\pgfqpoint{2.089357in}{4.035897in}}%
\pgfpathlineto{\pgfqpoint{2.073629in}{4.038779in}}%
\pgfpathlineto{\pgfqpoint{2.051565in}{4.054201in}}%
\pgfpathlineto{\pgfqpoint{2.030440in}{4.113772in}}%
\pgfpathlineto{\pgfqpoint{2.016355in}{4.222980in}}%
\pgfpathlineto{\pgfqpoint{1.996404in}{4.327721in}}%
\pgfpathlineto{\pgfqpoint{1.976922in}{4.413566in}}%
\pgfpathlineto{\pgfqpoint{1.957672in}{4.312261in}}%
\pgfpathlineto{\pgfqpoint{1.937487in}{4.155147in}}%
\pgfpathlineto{\pgfqpoint{1.919413in}{4.074519in}}%
\pgfpathlineto{\pgfqpoint{1.899929in}{4.047282in}}%
\pgfpathlineto{\pgfqpoint{1.879508in}{4.037026in}}%
\pgfpathlineto{\pgfqpoint{1.862608in}{4.057671in}}%
\pgfpathlineto{\pgfqpoint{1.843829in}{4.041527in}}%
\pgfpathlineto{\pgfqpoint{1.820827in}{4.035997in}}%
\pgfpathlineto{\pgfqpoint{1.802517in}{4.038206in}}%
\pgfpathlineto{\pgfqpoint{1.783269in}{4.050930in}}%
\pgfpathlineto{\pgfqpoint{1.764021in}{4.070835in}}%
\pgfpathlineto{\pgfqpoint{1.747356in}{4.145150in}}%
\pgfpathlineto{\pgfqpoint{1.725760in}{4.333862in}}%
\pgfpathlineto{\pgfqpoint{1.709095in}{4.405182in}}%
\pgfpathlineto{\pgfqpoint{1.687265in}{4.401978in}}%
\pgfpathlineto{\pgfqpoint{1.670129in}{4.246008in}}%
\pgfpathlineto{\pgfqpoint{1.651116in}{4.112101in}}%
\pgfpathlineto{\pgfqpoint{1.628114in}{4.053644in}}%
\pgfpathlineto{\pgfqpoint{1.613326in}{4.041847in}}%
\pgfpathlineto{\pgfqpoint{1.591496in}{4.036568in}}%
\pgfpathlineto{\pgfqpoint{1.573891in}{4.036695in}}%
\pgfpathlineto{\pgfqpoint{1.553235in}{4.044071in}}%
\pgfpathlineto{\pgfqpoint{1.534927in}{4.066708in}}%
\pgfpathlineto{\pgfqpoint{1.513331in}{4.117381in}}%
\pgfpathlineto{\pgfqpoint{1.495492in}{4.231480in}}%
\pgfpathlineto{\pgfqpoint{1.476713in}{4.359959in}}%
\pgfpathlineto{\pgfqpoint{1.457934in}{4.424188in}}%
\pgfpathlineto{\pgfqpoint{1.439626in}{4.407076in}}%
\pgfpathlineto{\pgfqpoint{1.418265in}{4.224259in}}%
\pgfpathlineto{\pgfqpoint{1.401130in}{4.114177in}}%
\pgfpathlineto{\pgfqpoint{1.382820in}{4.067494in}}%
\pgfpathlineto{\pgfqpoint{1.362399in}{4.049049in}}%
\pgfpathlineto{\pgfqpoint{1.341039in}{4.038779in}}%
\pgfpathlineto{\pgfqpoint{1.322495in}{4.036390in}}%
\pgfpathlineto{\pgfqpoint{1.304187in}{4.041428in}}%
\pgfpathlineto{\pgfqpoint{1.282122in}{4.055629in}}%
\pgfpathlineto{\pgfqpoint{1.267569in}{4.082991in}}%
\pgfpathlineto{\pgfqpoint{1.245035in}{4.111133in}}%
\pgfpathlineto{\pgfqpoint{1.208183in}{4.371788in}}%
\pgfpathlineto{\pgfqpoint{1.189170in}{4.430982in}}%
\pgfpathlineto{\pgfqpoint{1.167808in}{4.432752in}}%
\pgfpathlineto{\pgfqpoint{1.149500in}{4.316252in}}%
\pgfpathlineto{\pgfqpoint{1.130956in}{4.170133in}}%
\pgfpathlineto{\pgfqpoint{1.108891in}{4.268135in}}%
\pgfpathlineto{\pgfqpoint{1.091052in}{4.121293in}}%
\pgfpathlineto{\pgfqpoint{1.072039in}{4.067646in}}%
\pgfpathlineto{\pgfqpoint{1.054670in}{4.050429in}}%
\pgfpathlineto{\pgfqpoint{1.035421in}{4.038937in}}%
\pgfpathlineto{\pgfqpoint{1.014296in}{4.037797in}}%
\pgfpathlineto{\pgfqpoint{0.994345in}{4.046236in}}%
\pgfpathlineto{\pgfqpoint{0.977443in}{4.067261in}}%
\pgfpathlineto{\pgfqpoint{0.958196in}{4.103725in}}%
\pgfpathlineto{\pgfqpoint{0.936365in}{4.177443in}}%
\pgfpathlineto{\pgfqpoint{0.920874in}{4.297430in}}%
\pgfpathlineto{\pgfqpoint{0.899044in}{4.424578in}}%
\pgfpathlineto{\pgfqpoint{0.881908in}{4.445960in}}%
\pgfpathlineto{\pgfqpoint{0.862426in}{4.414637in}}%
\pgfpathlineto{\pgfqpoint{0.843882in}{4.206813in}}%
\pgfpathlineto{\pgfqpoint{0.822053in}{4.099555in}}%
\pgfpathlineto{\pgfqpoint{0.803509in}{4.060948in}}%
\pgfpathlineto{\pgfqpoint{0.784964in}{4.048247in}}%
\pgfpathlineto{\pgfqpoint{0.766422in}{4.038852in}}%
\pgfpathlineto{\pgfqpoint{0.744357in}{4.037511in}}%
\pgfpathlineto{\pgfqpoint{0.726518in}{4.043167in}}%
\pgfpathlineto{\pgfqpoint{0.708445in}{4.056957in}}%
\pgfpathlineto{\pgfqpoint{0.688960in}{4.085528in}}%
\pgfpathlineto{\pgfqpoint{0.669947in}{4.134390in}}%
\pgfpathlineto{\pgfqpoint{0.653048in}{4.254726in}}%
\pgfpathlineto{\pgfqpoint{0.653516in}{4.301977in}}%
\pgfpathlineto{\pgfqpoint{0.658681in}{4.037995in}}%
\pgfpathlineto{\pgfqpoint{0.676052in}{4.042638in}}%
\pgfpathlineto{\pgfqpoint{0.695768in}{4.066081in}}%
\pgfpathlineto{\pgfqpoint{0.715485in}{4.159491in}}%
\pgfpathlineto{\pgfqpoint{0.732855in}{4.389448in}}%
\pgfpathlineto{\pgfqpoint{0.751165in}{4.450271in}}%
\pgfpathlineto{\pgfqpoint{0.769004in}{4.345458in}}%
\pgfpathlineto{\pgfqpoint{0.789895in}{4.143641in}}%
\pgfpathlineto{\pgfqpoint{0.811254in}{4.059179in}}%
\pgfpathlineto{\pgfqpoint{0.829799in}{4.041218in}}%
\pgfpathlineto{\pgfqpoint{0.849047in}{4.037560in}}%
\pgfpathlineto{\pgfqpoint{0.865477in}{4.048903in}}%
\pgfpathlineto{\pgfqpoint{0.889890in}{4.100676in}}%
\pgfpathlineto{\pgfqpoint{0.904678in}{4.209585in}}%
\pgfpathlineto{\pgfqpoint{0.926037in}{4.435421in}}%
\pgfpathlineto{\pgfqpoint{0.944347in}{4.405584in}}%
\pgfpathlineto{\pgfqpoint{0.964064in}{4.420992in}}%
\pgfpathlineto{\pgfqpoint{0.981903in}{4.259064in}}%
\pgfpathlineto{\pgfqpoint{1.001856in}{4.095732in}}%
\pgfpathlineto{\pgfqpoint{1.023215in}{4.046197in}}%
\pgfpathlineto{\pgfqpoint{1.045985in}{4.036489in}}%
\pgfpathlineto{\pgfqpoint{1.060302in}{4.039991in}}%
\pgfpathlineto{\pgfqpoint{1.078847in}{4.057173in}}%
\pgfpathlineto{\pgfqpoint{1.099737in}{4.119048in}}%
\pgfpathlineto{\pgfqpoint{1.117342in}{4.340816in}}%
\pgfpathlineto{\pgfqpoint{1.138233in}{4.431077in}}%
\pgfpathlineto{\pgfqpoint{1.155837in}{4.320461in}}%
\pgfpathlineto{\pgfqpoint{1.174616in}{4.132005in}}%
\pgfpathlineto{\pgfqpoint{1.197384in}{4.053945in}}%
\pgfpathlineto{\pgfqpoint{1.214520in}{4.039785in}}%
\pgfpathlineto{\pgfqpoint{1.232359in}{4.036109in}}%
\pgfpathlineto{\pgfqpoint{1.251841in}{4.043703in}}%
\pgfpathlineto{\pgfqpoint{1.270855in}{4.069792in}}%
\pgfpathlineto{\pgfqpoint{1.290339in}{4.157199in}}%
\pgfpathlineto{\pgfqpoint{1.309116in}{4.371568in}}%
\pgfpathlineto{\pgfqpoint{1.332120in}{4.414187in}}%
\pgfpathlineto{\pgfqpoint{1.349725in}{4.270949in}}%
\pgfpathlineto{\pgfqpoint{1.365686in}{4.125159in}}%
\pgfpathlineto{\pgfqpoint{1.388454in}{4.052862in}}%
\pgfpathlineto{\pgfqpoint{1.406764in}{4.039553in}}%
\pgfpathlineto{\pgfqpoint{1.426012in}{4.035944in}}%
\pgfpathlineto{\pgfqpoint{1.445494in}{4.039917in}}%
\pgfpathlineto{\pgfqpoint{1.464742in}{4.056754in}}%
\pgfpathlineto{\pgfqpoint{1.483989in}{4.087609in}}%
\pgfpathlineto{\pgfqpoint{1.502768in}{4.221520in}}%
\pgfpathlineto{\pgfqpoint{1.522016in}{4.408731in}}%
\pgfpathlineto{\pgfqpoint{1.540558in}{4.406103in}}%
\pgfpathlineto{\pgfqpoint{1.559337in}{4.308517in}}%
\pgfpathlineto{\pgfqpoint{1.579054in}{4.138732in}}%
\pgfpathlineto{\pgfqpoint{1.597364in}{4.063508in}}%
\pgfpathlineto{\pgfqpoint{1.617080in}{4.042201in}}%
\pgfpathlineto{\pgfqpoint{1.636328in}{4.036029in}}%
\pgfpathlineto{\pgfqpoint{1.655107in}{4.040484in}}%
\pgfpathlineto{\pgfqpoint{1.676468in}{4.035736in}}%
\pgfpathlineto{\pgfqpoint{1.693837in}{4.039723in}}%
\pgfpathlineto{\pgfqpoint{1.712615in}{4.051773in}}%
\pgfpathlineto{\pgfqpoint{1.733740in}{4.110773in}}%
\pgfpathlineto{\pgfqpoint{1.749468in}{4.251008in}}%
\pgfpathlineto{\pgfqpoint{1.772001in}{4.419873in}}%
\pgfpathlineto{\pgfqpoint{1.791249in}{4.377716in}}%
\pgfpathlineto{\pgfqpoint{1.808854in}{4.224116in}}%
\pgfpathlineto{\pgfqpoint{1.828807in}{4.088808in}}%
\pgfpathlineto{\pgfqpoint{1.852280in}{4.045210in}}%
\pgfpathlineto{\pgfqpoint{1.868711in}{4.037044in}}%
\pgfpathlineto{\pgfqpoint{1.886550in}{4.036180in}}%
\pgfpathlineto{\pgfqpoint{1.906972in}{4.040642in}}%
\pgfpathlineto{\pgfqpoint{1.925280in}{4.055062in}}%
\pgfpathlineto{\pgfqpoint{1.943355in}{4.103815in}}%
\pgfpathlineto{\pgfqpoint{1.961898in}{4.241961in}}%
\pgfpathlineto{\pgfqpoint{1.981382in}{4.409792in}}%
\pgfpathlineto{\pgfqpoint{2.002272in}{4.384567in}}%
\pgfpathlineto{\pgfqpoint{2.022928in}{4.249972in}}%
\pgfpathlineto{\pgfqpoint{2.040768in}{4.103815in}}%
\pgfpathlineto{\pgfqpoint{2.059310in}{4.060253in}}%
\pgfpathlineto{\pgfqpoint{2.082080in}{4.041929in}}%
\pgfpathlineto{\pgfqpoint{2.098511in}{4.037224in}}%
\pgfpathlineto{\pgfqpoint{2.118227in}{4.036185in}}%
\pgfpathlineto{\pgfqpoint{2.137946in}{4.042043in}}%
\pgfpathlineto{\pgfqpoint{2.154845in}{4.059428in}}%
\pgfpathlineto{\pgfqpoint{2.175267in}{4.084038in}}%
\pgfpathlineto{\pgfqpoint{2.194046in}{4.147881in}}%
\pgfpathlineto{\pgfqpoint{2.216814in}{4.387935in}}%
\pgfpathlineto{\pgfqpoint{2.231602in}{4.412054in}}%
\pgfpathlineto{\pgfqpoint{2.253432in}{4.323602in}}%
\pgfpathlineto{\pgfqpoint{2.291458in}{4.086132in}}%
\pgfpathlineto{\pgfqpoint{2.310003in}{4.050127in}}%
\pgfpathlineto{\pgfqpoint{2.327371in}{4.038894in}}%
\pgfpathlineto{\pgfqpoint{2.347793in}{4.035333in}}%
\pgfpathlineto{\pgfqpoint{2.367980in}{4.038333in}}%
\pgfpathlineto{\pgfqpoint{2.385116in}{4.048977in}}%
\pgfpathlineto{\pgfqpoint{2.406710in}{4.090436in}}%
\pgfpathlineto{\pgfqpoint{2.423141in}{4.214647in}}%
\pgfpathlineto{\pgfqpoint{2.444033in}{4.385089in}}%
\pgfpathlineto{\pgfqpoint{2.462576in}{4.409061in}}%
\pgfpathlineto{\pgfqpoint{2.482997in}{4.341122in}}%
\pgfpathlineto{\pgfqpoint{2.502479in}{4.260472in}}%
\pgfpathlineto{\pgfqpoint{2.520555in}{4.175071in}}%
\pgfpathlineto{\pgfqpoint{2.540740in}{4.070574in}}%
\pgfpathlineto{\pgfqpoint{2.558111in}{4.046388in}}%
\pgfpathlineto{\pgfqpoint{2.579472in}{4.037851in}}%
\pgfpathlineto{\pgfqpoint{2.597077in}{4.035571in}}%
\pgfpathlineto{\pgfqpoint{2.618905in}{4.041162in}}%
\pgfpathlineto{\pgfqpoint{2.636041in}{4.054229in}}%
\pgfpathlineto{\pgfqpoint{2.655054in}{4.092416in}}%
\pgfpathlineto{\pgfqpoint{2.677353in}{4.219683in}}%
\pgfpathlineto{\pgfqpoint{2.694019in}{4.388737in}}%
\pgfpathlineto{\pgfqpoint{2.711858in}{4.404229in}}%
\pgfpathlineto{\pgfqpoint{2.730871in}{4.288981in}}%
\pgfpathlineto{\pgfqpoint{2.749884in}{4.206815in}}%
\pgfpathlineto{\pgfqpoint{2.773826in}{4.067218in}}%
\pgfpathlineto{\pgfqpoint{2.789085in}{4.048596in}}%
\pgfpathlineto{\pgfqpoint{2.808801in}{4.037837in}}%
\pgfpathlineto{\pgfqpoint{2.830163in}{4.035654in}}%
\pgfpathlineto{\pgfqpoint{2.845185in}{4.038821in}}%
\pgfpathlineto{\pgfqpoint{2.866310in}{4.045330in}}%
\pgfpathlineto{\pgfqpoint{2.883680in}{4.058099in}}%
\pgfpathlineto{\pgfqpoint{2.906685in}{4.124848in}}%
\pgfpathlineto{\pgfqpoint{2.923350in}{4.214488in}}%
\pgfpathlineto{\pgfqpoint{2.947292in}{4.406534in}}%
\pgfpathlineto{\pgfqpoint{2.964897in}{4.392443in}}%
\pgfpathlineto{\pgfqpoint{2.981562in}{4.303155in}}%
\pgfpathlineto{\pgfqpoint{3.001749in}{4.127758in}}%
\pgfpathlineto{\pgfqpoint{3.021702in}{4.063283in}}%
\pgfpathlineto{\pgfqpoint{3.039776in}{4.045670in}}%
\pgfpathlineto{\pgfqpoint{3.058084in}{4.038002in}}%
\pgfpathlineto{\pgfqpoint{3.078505in}{4.036270in}}%
\pgfpathlineto{\pgfqpoint{3.097753in}{4.041257in}}%
\pgfpathlineto{\pgfqpoint{3.114420in}{4.047214in}}%
\pgfpathlineto{\pgfqpoint{3.136014in}{4.040852in}}%
\pgfpathlineto{\pgfqpoint{3.154324in}{4.050505in}}%
\pgfpathlineto{\pgfqpoint{3.174509in}{4.086430in}}%
\pgfpathlineto{\pgfqpoint{3.193523in}{4.179940in}}%
\pgfpathlineto{\pgfqpoint{3.214649in}{4.374432in}}%
\pgfpathlineto{\pgfqpoint{3.232254in}{4.417867in}}%
\pgfpathlineto{\pgfqpoint{3.249859in}{4.345888in}}%
\pgfpathlineto{\pgfqpoint{3.270515in}{4.170093in}}%
\pgfpathlineto{\pgfqpoint{3.289528in}{4.083868in}}%
\pgfpathlineto{\pgfqpoint{3.306663in}{4.362461in}}%
\pgfpathlineto{\pgfqpoint{3.329667in}{4.186887in}}%
\pgfpathlineto{\pgfqpoint{3.347975in}{4.088672in}}%
\pgfpathlineto{\pgfqpoint{3.366285in}{4.052458in}}%
\pgfpathlineto{\pgfqpoint{3.384827in}{4.038950in}}%
\pgfpathlineto{\pgfqpoint{3.402667in}{4.036196in}}%
\pgfpathlineto{\pgfqpoint{3.423793in}{4.039875in}}%
\pgfpathlineto{\pgfqpoint{3.442336in}{4.050717in}}%
\pgfpathlineto{\pgfqpoint{3.460411in}{4.072128in}}%
\pgfpathlineto{\pgfqpoint{3.481536in}{4.160944in}}%
\pgfpathlineto{\pgfqpoint{3.502193in}{4.371605in}}%
\pgfpathlineto{\pgfqpoint{3.516980in}{4.409889in}}%
\pgfpathlineto{\pgfqpoint{3.538105in}{4.410796in}}%
\pgfpathlineto{\pgfqpoint{3.559232in}{4.288009in}}%
\pgfpathlineto{\pgfqpoint{3.577540in}{4.162638in}}%
\pgfpathlineto{\pgfqpoint{3.596788in}{4.083933in}}%
\pgfpathlineto{\pgfqpoint{3.616270in}{4.049906in}}%
\pgfpathlineto{\pgfqpoint{3.634580in}{4.039912in}}%
\pgfpathlineto{\pgfqpoint{3.652888in}{4.036278in}}%
\pgfpathlineto{\pgfqpoint{3.672136in}{4.039294in}}%
\pgfpathlineto{\pgfqpoint{3.691854in}{4.049205in}}%
\pgfpathlineto{\pgfqpoint{3.712274in}{4.076682in}}%
\pgfpathlineto{\pgfqpoint{3.734575in}{4.177381in}}%
\pgfpathlineto{\pgfqpoint{3.749597in}{4.308416in}}%
\pgfpathlineto{\pgfqpoint{3.770957in}{4.413723in}}%
\pgfpathlineto{\pgfqpoint{3.788327in}{4.433319in}}%
\pgfpathlineto{\pgfqpoint{3.808983in}{4.351352in}}%
\pgfpathlineto{\pgfqpoint{3.826119in}{4.215173in}}%
\pgfpathlineto{\pgfqpoint{3.844662in}{4.135853in}}%
\pgfpathlineto{\pgfqpoint{3.865318in}{4.071200in}}%
\pgfpathlineto{\pgfqpoint{3.883159in}{4.053226in}}%
\pgfpathlineto{\pgfqpoint{3.902876in}{4.039342in}}%
\pgfpathlineto{\pgfqpoint{3.922592in}{4.036914in}}%
\pgfpathlineto{\pgfqpoint{3.939728in}{4.040254in}}%
\pgfpathlineto{\pgfqpoint{3.961558in}{4.053011in}}%
\pgfpathlineto{\pgfqpoint{3.983857in}{4.085858in}}%
\pgfpathlineto{\pgfqpoint{4.000991in}{4.143983in}}%
\pgfpathlineto{\pgfqpoint{4.021413in}{4.304006in}}%
\pgfpathlineto{\pgfqpoint{4.036672in}{4.395993in}}%
\pgfpathlineto{\pgfqpoint{4.057328in}{4.440849in}}%
\pgfpathlineto{\pgfqpoint{4.078922in}{4.416458in}}%
\pgfpathlineto{\pgfqpoint{4.096997in}{4.313776in}}%
\pgfpathlineto{\pgfqpoint{4.115071in}{4.174918in}}%
\pgfpathlineto{\pgfqpoint{4.133145in}{4.102619in}}%
\pgfpathlineto{\pgfqpoint{4.151455in}{4.063171in}}%
\pgfpathlineto{\pgfqpoint{4.171640in}{4.048232in}}%
\pgfpathlineto{\pgfqpoint{4.189716in}{4.040758in}}%
\pgfpathlineto{\pgfqpoint{4.213657in}{4.038931in}}%
\pgfpathlineto{\pgfqpoint{4.229385in}{4.044730in}}%
\pgfpathlineto{\pgfqpoint{4.250041in}{4.057876in}}%
\pgfpathlineto{\pgfqpoint{4.271400in}{4.101772in}}%
\pgfpathlineto{\pgfqpoint{4.289945in}{4.169190in}}%
\pgfpathlineto{\pgfqpoint{4.307550in}{4.292457in}}%
\pgfpathlineto{\pgfqpoint{4.324684in}{4.422107in}}%
\pgfpathlineto{\pgfqpoint{4.346279in}{4.455974in}}%
\pgfpathlineto{\pgfqpoint{4.364588in}{4.424942in}}%
\pgfpathlineto{\pgfqpoint{4.386183in}{4.315318in}}%
\pgfpathlineto{\pgfqpoint{4.404728in}{4.200775in}}%
\pgfpathlineto{\pgfqpoint{4.422096in}{4.124877in}}%
\pgfpathlineto{\pgfqpoint{4.444161in}{4.063872in}}%
\pgfpathlineto{\pgfqpoint{4.461062in}{4.049166in}}%
\pgfpathlineto{\pgfqpoint{4.481013in}{4.039655in}}%
\pgfpathlineto{\pgfqpoint{4.475850in}{4.042843in}}%
\pgfpathlineto{\pgfqpoint{4.454254in}{4.070279in}}%
\pgfpathlineto{\pgfqpoint{4.435476in}{4.161760in}}%
\pgfpathlineto{\pgfqpoint{4.416933in}{4.357985in}}%
\pgfpathlineto{\pgfqpoint{4.398389in}{4.173361in}}%
\pgfpathlineto{\pgfqpoint{4.376090in}{4.066895in}}%
\pgfpathlineto{\pgfqpoint{4.358250in}{4.042830in}}%
\pgfpathlineto{\pgfqpoint{4.341351in}{4.037904in}}%
\pgfpathlineto{\pgfqpoint{4.321164in}{4.047927in}}%
\pgfpathlineto{\pgfqpoint{4.299802in}{4.100526in}}%
\pgfpathlineto{\pgfqpoint{4.280554in}{4.259581in}}%
\pgfpathlineto{\pgfqpoint{4.261072in}{4.433690in}}%
\pgfpathlineto{\pgfqpoint{4.244642in}{4.444603in}}%
\pgfpathlineto{\pgfqpoint{4.222811in}{4.242808in}}%
\pgfpathlineto{\pgfqpoint{4.204738in}{4.094751in}}%
\pgfpathlineto{\pgfqpoint{4.187367in}{4.051759in}}%
\pgfpathlineto{\pgfqpoint{4.165537in}{4.037602in}}%
\pgfpathlineto{\pgfqpoint{4.148167in}{4.039489in}}%
\pgfpathlineto{\pgfqpoint{4.128919in}{4.059672in}}%
\pgfpathlineto{\pgfqpoint{4.109906in}{4.121231in}}%
\pgfpathlineto{\pgfqpoint{4.085964in}{4.356605in}}%
\pgfpathlineto{\pgfqpoint{4.068830in}{4.440202in}}%
\pgfpathlineto{\pgfqpoint{4.053337in}{4.402083in}}%
\pgfpathlineto{\pgfqpoint{4.033150in}{4.194254in}}%
\pgfpathlineto{\pgfqpoint{4.010382in}{4.068719in}}%
\pgfpathlineto{\pgfqpoint{3.993246in}{4.043309in}}%
\pgfpathlineto{\pgfqpoint{3.976346in}{4.036424in}}%
\pgfpathlineto{\pgfqpoint{3.955219in}{4.040814in}}%
\pgfpathlineto{\pgfqpoint{3.934799in}{4.066534in}}%
\pgfpathlineto{\pgfqpoint{3.917195in}{4.142458in}}%
\pgfpathlineto{\pgfqpoint{3.898885in}{4.346570in}}%
\pgfpathlineto{\pgfqpoint{3.879403in}{4.432694in}}%
\pgfpathlineto{\pgfqpoint{3.858746in}{4.327857in}}%
\pgfpathlineto{\pgfqpoint{3.838559in}{4.109740in}}%
\pgfpathlineto{\pgfqpoint{3.822363in}{4.057135in}}%
\pgfpathlineto{\pgfqpoint{3.801238in}{4.038820in}}%
\pgfpathlineto{\pgfqpoint{3.782459in}{4.036199in}}%
\pgfpathlineto{\pgfqpoint{3.764854in}{4.041891in}}%
\pgfpathlineto{\pgfqpoint{3.745372in}{4.054789in}}%
\pgfpathlineto{\pgfqpoint{3.724716in}{4.133448in}}%
\pgfpathlineto{\pgfqpoint{3.687629in}{4.405473in}}%
\pgfpathlineto{\pgfqpoint{3.666738in}{4.403025in}}%
\pgfpathlineto{\pgfqpoint{3.649603in}{4.248064in}}%
\pgfpathlineto{\pgfqpoint{3.628241in}{4.096193in}}%
\pgfpathlineto{\pgfqpoint{3.611342in}{4.057085in}}%
\pgfpathlineto{\pgfqpoint{3.589982in}{4.040004in}}%
\pgfpathlineto{\pgfqpoint{3.572141in}{4.036079in}}%
\pgfpathlineto{\pgfqpoint{3.552190in}{4.038196in}}%
\pgfpathlineto{\pgfqpoint{3.533880in}{4.049059in}}%
\pgfpathlineto{\pgfqpoint{3.513929in}{4.099562in}}%
\pgfpathlineto{\pgfqpoint{3.495619in}{4.215032in}}%
\pgfpathlineto{\pgfqpoint{3.475434in}{4.350557in}}%
\pgfpathlineto{\pgfqpoint{3.451726in}{4.416976in}}%
\pgfpathlineto{\pgfqpoint{3.436702in}{4.272234in}}%
\pgfpathlineto{\pgfqpoint{3.420271in}{4.135292in}}%
\pgfpathlineto{\pgfqpoint{3.398441in}{4.060996in}}%
\pgfpathlineto{\pgfqpoint{3.380367in}{4.042066in}}%
\pgfpathlineto{\pgfqpoint{3.356660in}{4.036368in}}%
\pgfpathlineto{\pgfqpoint{3.341872in}{4.036285in}}%
\pgfpathlineto{\pgfqpoint{3.320982in}{4.044071in}}%
\pgfpathlineto{\pgfqpoint{3.301968in}{4.073979in}}%
\pgfpathlineto{\pgfqpoint{3.283660in}{4.149549in}}%
\pgfpathlineto{\pgfqpoint{3.263942in}{4.310150in}}%
\pgfpathlineto{\pgfqpoint{3.247042in}{4.109652in}}%
\pgfpathlineto{\pgfqpoint{3.223804in}{4.233044in}}%
\pgfpathlineto{\pgfqpoint{3.206668in}{4.387238in}}%
\pgfpathlineto{\pgfqpoint{3.186011in}{4.407492in}}%
\pgfpathlineto{\pgfqpoint{3.169112in}{4.220961in}}%
\pgfpathlineto{\pgfqpoint{3.148456in}{4.081811in}}%
\pgfpathlineto{\pgfqpoint{3.130851in}{4.273543in}}%
\pgfpathlineto{\pgfqpoint{3.111132in}{4.096127in}}%
\pgfpathlineto{\pgfqpoint{3.090242in}{4.052130in}}%
\pgfpathlineto{\pgfqpoint{3.072637in}{4.038546in}}%
\pgfpathlineto{\pgfqpoint{3.051512in}{4.036078in}}%
\pgfpathlineto{\pgfqpoint{3.033204in}{4.041587in}}%
\pgfpathlineto{\pgfqpoint{3.013017in}{4.059485in}}%
\pgfpathlineto{\pgfqpoint{2.993535in}{4.129668in}}%
\pgfpathlineto{\pgfqpoint{2.978276in}{4.300458in}}%
\pgfpathlineto{\pgfqpoint{2.956682in}{4.391802in}}%
\pgfpathlineto{\pgfqpoint{2.937669in}{4.412970in}}%
\pgfpathlineto{\pgfqpoint{2.915839in}{4.210339in}}%
\pgfpathlineto{\pgfqpoint{2.897529in}{4.092158in}}%
\pgfpathlineto{\pgfqpoint{2.879221in}{4.050661in}}%
\pgfpathlineto{\pgfqpoint{2.860207in}{4.038739in}}%
\pgfpathlineto{\pgfqpoint{2.841899in}{4.035656in}}%
\pgfpathlineto{\pgfqpoint{2.823120in}{4.037047in}}%
\pgfpathlineto{\pgfqpoint{2.801056in}{4.048656in}}%
\pgfpathlineto{\pgfqpoint{2.783685in}{4.068976in}}%
\pgfpathlineto{\pgfqpoint{2.763969in}{4.168703in}}%
\pgfpathlineto{\pgfqpoint{2.744956in}{4.326510in}}%
\pgfpathlineto{\pgfqpoint{2.722891in}{4.412927in}}%
\pgfpathlineto{\pgfqpoint{2.706226in}{4.307905in}}%
\pgfpathlineto{\pgfqpoint{2.686273in}{4.126138in}}%
\pgfpathlineto{\pgfqpoint{2.668668in}{4.061935in}}%
\pgfpathlineto{\pgfqpoint{2.649889in}{4.044196in}}%
\pgfpathlineto{\pgfqpoint{2.629468in}{4.036028in}}%
\pgfpathlineto{\pgfqpoint{2.607639in}{4.036134in}}%
\pgfpathlineto{\pgfqpoint{2.593086in}{4.039597in}}%
\pgfpathlineto{\pgfqpoint{2.567736in}{4.055075in}}%
\pgfpathlineto{\pgfqpoint{2.552243in}{4.089197in}}%
\pgfpathlineto{\pgfqpoint{2.533698in}{4.194621in}}%
\pgfpathlineto{\pgfqpoint{2.514216in}{4.349307in}}%
\pgfpathlineto{\pgfqpoint{2.496377in}{4.409102in}}%
\pgfpathlineto{\pgfqpoint{2.477364in}{4.357898in}}%
\pgfpathlineto{\pgfqpoint{2.455770in}{4.163049in}}%
\pgfpathlineto{\pgfqpoint{2.436991in}{4.118689in}}%
\pgfpathlineto{\pgfqpoint{2.419152in}{4.059229in}}%
\pgfpathlineto{\pgfqpoint{2.397087in}{4.039119in}}%
\pgfpathlineto{\pgfqpoint{2.381828in}{4.035821in}}%
\pgfpathlineto{\pgfqpoint{2.359764in}{4.037284in}}%
\pgfpathlineto{\pgfqpoint{2.340987in}{4.044578in}}%
\pgfpathlineto{\pgfqpoint{2.322208in}{4.070582in}}%
\pgfpathlineto{\pgfqpoint{2.302960in}{4.149156in}}%
\pgfpathlineto{\pgfqpoint{2.284650in}{4.308424in}}%
\pgfpathlineto{\pgfqpoint{2.266108in}{4.403382in}}%
\pgfpathlineto{\pgfqpoint{2.244747in}{4.374004in}}%
\pgfpathlineto{\pgfqpoint{2.225499in}{4.191142in}}%
\pgfpathlineto{\pgfqpoint{2.206486in}{4.087379in}}%
\pgfpathlineto{\pgfqpoint{2.187943in}{4.051898in}}%
\pgfpathlineto{\pgfqpoint{2.168695in}{4.041251in}}%
\pgfpathlineto{\pgfqpoint{2.148274in}{4.035470in}}%
\pgfpathlineto{\pgfqpoint{2.128555in}{4.037990in}}%
\pgfpathlineto{\pgfqpoint{2.110716in}{4.047637in}}%
\pgfpathlineto{\pgfqpoint{2.093111in}{4.037527in}}%
\pgfpathlineto{\pgfqpoint{2.071517in}{4.035649in}}%
\pgfpathlineto{\pgfqpoint{2.054850in}{4.036535in}}%
\pgfpathlineto{\pgfqpoint{2.033725in}{4.047131in}}%
\pgfpathlineto{\pgfqpoint{2.014009in}{4.083818in}}%
\pgfpathlineto{\pgfqpoint{1.977156in}{4.360288in}}%
\pgfpathlineto{\pgfqpoint{1.955326in}{4.414669in}}%
\pgfpathlineto{\pgfqpoint{1.936782in}{4.305831in}}%
\pgfpathlineto{\pgfqpoint{1.918005in}{4.143143in}}%
\pgfpathlineto{\pgfqpoint{1.900164in}{4.070341in}}%
\pgfpathlineto{\pgfqpoint{1.881621in}{4.046834in}}%
\pgfpathlineto{\pgfqpoint{1.863077in}{4.037633in}}%
\pgfpathlineto{\pgfqpoint{1.841247in}{4.036491in}}%
\pgfpathlineto{\pgfqpoint{1.822470in}{4.040641in}}%
\pgfpathlineto{\pgfqpoint{1.800874in}{4.061277in}}%
\pgfpathlineto{\pgfqpoint{1.781861in}{4.129509in}}%
\pgfpathlineto{\pgfqpoint{1.745243in}{4.395935in}}%
\pgfpathlineto{\pgfqpoint{1.725760in}{4.414581in}}%
\pgfpathlineto{\pgfqpoint{1.708156in}{4.307339in}}%
\pgfpathlineto{\pgfqpoint{1.688908in}{4.163316in}}%
\pgfpathlineto{\pgfqpoint{1.668017in}{4.076939in}}%
\pgfpathlineto{\pgfqpoint{1.649239in}{4.050179in}}%
\pgfpathlineto{\pgfqpoint{1.630696in}{4.041914in}}%
\pgfpathlineto{\pgfqpoint{1.612152in}{4.036151in}}%
\pgfpathlineto{\pgfqpoint{1.590556in}{4.038465in}}%
\pgfpathlineto{\pgfqpoint{1.571779in}{4.047401in}}%
\pgfpathlineto{\pgfqpoint{1.550418in}{4.076402in}}%
\pgfpathlineto{\pgfqpoint{1.535161in}{4.130005in}}%
\pgfpathlineto{\pgfqpoint{1.513800in}{4.275324in}}%
\pgfpathlineto{\pgfqpoint{1.495492in}{4.312489in}}%
\pgfpathlineto{\pgfqpoint{1.476713in}{4.405756in}}%
\pgfpathlineto{\pgfqpoint{1.457699in}{4.422020in}}%
\pgfpathlineto{\pgfqpoint{1.439391in}{4.340329in}}%
\pgfpathlineto{\pgfqpoint{1.421081in}{4.153177in}}%
\pgfpathlineto{\pgfqpoint{1.395731in}{4.067459in}}%
\pgfpathlineto{\pgfqpoint{1.380474in}{4.049075in}}%
\pgfpathlineto{\pgfqpoint{1.361461in}{4.039263in}}%
\pgfpathlineto{\pgfqpoint{1.339396in}{4.036684in}}%
\pgfpathlineto{\pgfqpoint{1.322026in}{4.039674in}}%
\pgfpathlineto{\pgfqpoint{1.303718in}{4.045447in}}%
\pgfpathlineto{\pgfqpoint{1.284939in}{4.061512in}}%
\pgfpathlineto{\pgfqpoint{1.266395in}{4.115137in}}%
\pgfpathlineto{\pgfqpoint{1.247616in}{4.199623in}}%
\pgfpathlineto{\pgfqpoint{1.226022in}{4.363865in}}%
\pgfpathlineto{\pgfqpoint{1.204426in}{4.362669in}}%
\pgfpathlineto{\pgfqpoint{1.188935in}{4.423802in}}%
\pgfpathlineto{\pgfqpoint{1.167105in}{4.418694in}}%
\pgfpathlineto{\pgfqpoint{1.148561in}{4.284058in}}%
\pgfpathlineto{\pgfqpoint{1.130018in}{4.136289in}}%
\pgfpathlineto{\pgfqpoint{1.112648in}{4.080549in}}%
\pgfpathlineto{\pgfqpoint{1.090114in}{4.048768in}}%
\pgfpathlineto{\pgfqpoint{1.071335in}{4.040918in}}%
\pgfpathlineto{\pgfqpoint{1.053262in}{4.036935in}}%
\pgfpathlineto{\pgfqpoint{1.015470in}{4.037295in}}%
\pgfpathlineto{\pgfqpoint{0.997865in}{4.040044in}}%
\pgfpathlineto{\pgfqpoint{0.975800in}{4.084953in}}%
\pgfpathlineto{\pgfqpoint{0.956787in}{4.069756in}}%
\pgfpathlineto{\pgfqpoint{0.938008in}{4.045705in}}%
\pgfpathlineto{\pgfqpoint{0.920403in}{4.037563in}}%
\pgfpathlineto{\pgfqpoint{0.898810in}{4.040982in}}%
\pgfpathlineto{\pgfqpoint{0.879796in}{4.054002in}}%
\pgfpathlineto{\pgfqpoint{0.861486in}{4.097538in}}%
\pgfpathlineto{\pgfqpoint{0.842944in}{4.212891in}}%
\pgfpathlineto{\pgfqpoint{0.822522in}{4.361772in}}%
\pgfpathlineto{\pgfqpoint{0.803040in}{4.440821in}}%
\pgfpathlineto{\pgfqpoint{0.784261in}{4.431890in}}%
\pgfpathlineto{\pgfqpoint{0.765717in}{4.264377in}}%
\pgfpathlineto{\pgfqpoint{0.748112in}{4.118530in}}%
\pgfpathlineto{\pgfqpoint{0.726047in}{4.069153in}}%
\pgfpathlineto{\pgfqpoint{0.708679in}{4.047498in}}%
\pgfpathlineto{\pgfqpoint{0.688492in}{4.038192in}}%
\pgfpathlineto{\pgfqpoint{0.670184in}{4.039928in}}%
\pgfpathlineto{\pgfqpoint{0.651405in}{4.050027in}}%
\pgfpathlineto{\pgfqpoint{0.654690in}{4.046434in}}%
\pgfpathlineto{\pgfqpoint{0.677224in}{4.039286in}}%
\pgfpathlineto{\pgfqpoint{0.694125in}{4.052818in}}%
\pgfpathlineto{\pgfqpoint{0.714782in}{4.107633in}}%
\pgfpathlineto{\pgfqpoint{0.733560in}{4.275533in}}%
\pgfpathlineto{\pgfqpoint{0.751165in}{4.446717in}}%
\pgfpathlineto{\pgfqpoint{0.772290in}{4.390156in}}%
\pgfpathlineto{\pgfqpoint{0.791538in}{4.201969in}}%
\pgfpathlineto{\pgfqpoint{0.809377in}{4.081950in}}%
\pgfpathlineto{\pgfqpoint{0.834024in}{4.040116in}}%
\pgfpathlineto{\pgfqpoint{0.850221in}{4.036087in}}%
\pgfpathlineto{\pgfqpoint{0.867825in}{4.043922in}}%
\pgfpathlineto{\pgfqpoint{0.887073in}{4.070585in}}%
\pgfpathlineto{\pgfqpoint{0.908667in}{4.222814in}}%
\pgfpathlineto{\pgfqpoint{0.924863in}{4.424039in}}%
\pgfpathlineto{\pgfqpoint{0.945051in}{4.405352in}}%
\pgfpathlineto{\pgfqpoint{0.962890in}{4.217593in}}%
\pgfpathlineto{\pgfqpoint{0.982843in}{4.084712in}}%
\pgfpathlineto{\pgfqpoint{1.001385in}{4.044903in}}%
\pgfpathlineto{\pgfqpoint{1.023686in}{4.035536in}}%
\pgfpathlineto{\pgfqpoint{1.041994in}{4.038986in}}%
\pgfpathlineto{\pgfqpoint{1.061711in}{4.054733in}}%
\pgfpathlineto{\pgfqpoint{1.080255in}{4.119462in}}%
\pgfpathlineto{\pgfqpoint{1.099972in}{4.307051in}}%
\pgfpathlineto{\pgfqpoint{1.115699in}{4.432037in}}%
\pgfpathlineto{\pgfqpoint{1.134712in}{4.357428in}}%
\pgfpathlineto{\pgfqpoint{1.157717in}{4.138711in}}%
\pgfpathlineto{\pgfqpoint{1.177433in}{4.056208in}}%
\pgfpathlineto{\pgfqpoint{1.195741in}{4.037705in}}%
\pgfpathlineto{\pgfqpoint{1.214286in}{4.035209in}}%
\pgfpathlineto{\pgfqpoint{1.233768in}{4.041281in}}%
\pgfpathlineto{\pgfqpoint{1.252781in}{4.061215in}}%
\pgfpathlineto{\pgfqpoint{1.270620in}{4.129218in}}%
\pgfpathlineto{\pgfqpoint{1.292450in}{4.333412in}}%
\pgfpathlineto{\pgfqpoint{1.310993in}{4.425687in}}%
\pgfpathlineto{\pgfqpoint{1.330006in}{4.307293in}}%
\pgfpathlineto{\pgfqpoint{1.349959in}{4.122816in}}%
\pgfpathlineto{\pgfqpoint{1.368033in}{4.057894in}}%
\pgfpathlineto{\pgfqpoint{1.386577in}{4.038894in}}%
\pgfpathlineto{\pgfqpoint{1.405356in}{4.035276in}}%
\pgfpathlineto{\pgfqpoint{1.429532in}{4.038579in}}%
\pgfpathlineto{\pgfqpoint{1.444086in}{4.045986in}}%
\pgfpathlineto{\pgfqpoint{1.462394in}{4.079402in}}%
\pgfpathlineto{\pgfqpoint{1.484693in}{4.239664in}}%
\pgfpathlineto{\pgfqpoint{1.503706in}{4.419982in}}%
\pgfpathlineto{\pgfqpoint{1.523190in}{4.204423in}}%
\pgfpathlineto{\pgfqpoint{1.541498in}{4.409724in}}%
\pgfpathlineto{\pgfqpoint{1.563328in}{4.398999in}}%
\pgfpathlineto{\pgfqpoint{1.598067in}{4.101207in}}%
\pgfpathlineto{\pgfqpoint{1.616377in}{4.050331in}}%
\pgfpathlineto{\pgfqpoint{1.638207in}{4.037094in}}%
\pgfpathlineto{\pgfqpoint{1.654638in}{4.035047in}}%
\pgfpathlineto{\pgfqpoint{1.676703in}{4.038411in}}%
\pgfpathlineto{\pgfqpoint{1.693837in}{4.049133in}}%
\pgfpathlineto{\pgfqpoint{1.714258in}{4.082035in}}%
\pgfpathlineto{\pgfqpoint{1.732568in}{4.212233in}}%
\pgfpathlineto{\pgfqpoint{1.751816in}{4.399336in}}%
\pgfpathlineto{\pgfqpoint{1.770359in}{4.410667in}}%
\pgfpathlineto{\pgfqpoint{1.789606in}{4.293762in}}%
\pgfpathlineto{\pgfqpoint{1.807682in}{4.116932in}}%
\pgfpathlineto{\pgfqpoint{1.832093in}{4.045982in}}%
\pgfpathlineto{\pgfqpoint{1.849697in}{4.038232in}}%
\pgfpathlineto{\pgfqpoint{1.868007in}{4.035235in}}%
\pgfpathlineto{\pgfqpoint{1.885612in}{4.035068in}}%
\pgfpathlineto{\pgfqpoint{1.907206in}{4.042126in}}%
\pgfpathlineto{\pgfqpoint{1.929741in}{4.074706in}}%
\pgfpathlineto{\pgfqpoint{1.945701in}{4.135074in}}%
\pgfpathlineto{\pgfqpoint{1.963777in}{4.332474in}}%
\pgfpathlineto{\pgfqpoint{1.983025in}{4.416375in}}%
\pgfpathlineto{\pgfqpoint{2.001098in}{4.331776in}}%
\pgfpathlineto{\pgfqpoint{2.020346in}{4.182242in}}%
\pgfpathlineto{\pgfqpoint{2.038420in}{4.104425in}}%
\pgfpathlineto{\pgfqpoint{2.060484in}{4.049382in}}%
\pgfpathlineto{\pgfqpoint{2.079263in}{4.038259in}}%
\pgfpathlineto{\pgfqpoint{2.097571in}{4.034726in}}%
\pgfpathlineto{\pgfqpoint{2.116350in}{4.036568in}}%
\pgfpathlineto{\pgfqpoint{2.137240in}{4.045804in}}%
\pgfpathlineto{\pgfqpoint{2.155080in}{4.069182in}}%
\pgfpathlineto{\pgfqpoint{2.172921in}{4.151510in}}%
\pgfpathlineto{\pgfqpoint{2.197097in}{4.390666in}}%
\pgfpathlineto{\pgfqpoint{2.210945in}{4.411441in}}%
\pgfpathlineto{\pgfqpoint{2.232541in}{4.338261in}}%
\pgfpathlineto{\pgfqpoint{2.250380in}{4.194965in}}%
\pgfpathlineto{\pgfqpoint{2.271507in}{4.078201in}}%
\pgfpathlineto{\pgfqpoint{2.290050in}{4.044893in}}%
\pgfpathlineto{\pgfqpoint{2.309766in}{4.036417in}}%
\pgfpathlineto{\pgfqpoint{2.328311in}{4.034805in}}%
\pgfpathlineto{\pgfqpoint{2.349201in}{4.036605in}}%
\pgfpathlineto{\pgfqpoint{2.367277in}{4.042886in}}%
\pgfpathlineto{\pgfqpoint{2.384880in}{4.063835in}}%
\pgfpathlineto{\pgfqpoint{2.403659in}{4.079059in}}%
\pgfpathlineto{\pgfqpoint{2.424315in}{4.120599in}}%
\pgfpathlineto{\pgfqpoint{2.441920in}{4.034988in}}%
\pgfpathlineto{\pgfqpoint{2.462576in}{4.036747in}}%
\pgfpathlineto{\pgfqpoint{2.480886in}{4.049505in}}%
\pgfpathlineto{\pgfqpoint{2.498490in}{4.103953in}}%
\pgfpathlineto{\pgfqpoint{2.520555in}{4.252770in}}%
\pgfpathlineto{\pgfqpoint{2.540975in}{4.416178in}}%
\pgfpathlineto{\pgfqpoint{2.559050in}{4.350817in}}%
\pgfpathlineto{\pgfqpoint{2.576419in}{4.173661in}}%
\pgfpathlineto{\pgfqpoint{2.597546in}{4.060890in}}%
\pgfpathlineto{\pgfqpoint{2.615619in}{4.046109in}}%
\pgfpathlineto{\pgfqpoint{2.636276in}{4.035934in}}%
\pgfpathlineto{\pgfqpoint{2.655054in}{4.034817in}}%
\pgfpathlineto{\pgfqpoint{2.675476in}{4.041156in}}%
\pgfpathlineto{\pgfqpoint{2.693550in}{4.058069in}}%
\pgfpathlineto{\pgfqpoint{2.711389in}{4.114017in}}%
\pgfpathlineto{\pgfqpoint{2.732750in}{4.295572in}}%
\pgfpathlineto{\pgfqpoint{2.750355in}{4.411759in}}%
\pgfpathlineto{\pgfqpoint{2.770072in}{4.372724in}}%
\pgfpathlineto{\pgfqpoint{2.790962in}{4.191981in}}%
\pgfpathlineto{\pgfqpoint{2.806219in}{4.099461in}}%
\pgfpathlineto{\pgfqpoint{2.828284in}{4.051006in}}%
\pgfpathlineto{\pgfqpoint{2.846125in}{4.038958in}}%
\pgfpathlineto{\pgfqpoint{2.864198in}{4.035062in}}%
\pgfpathlineto{\pgfqpoint{2.888375in}{4.037540in}}%
\pgfpathlineto{\pgfqpoint{2.902459in}{4.044590in}}%
\pgfpathlineto{\pgfqpoint{2.924524in}{4.078893in}}%
\pgfpathlineto{\pgfqpoint{2.941894in}{4.146182in}}%
\pgfpathlineto{\pgfqpoint{2.963019in}{4.357100in}}%
\pgfpathlineto{\pgfqpoint{2.983910in}{4.410573in}}%
\pgfpathlineto{\pgfqpoint{2.998463in}{4.350569in}}%
\pgfpathlineto{\pgfqpoint{3.019823in}{4.168819in}}%
\pgfpathlineto{\pgfqpoint{3.038367in}{4.078134in}}%
\pgfpathlineto{\pgfqpoint{3.062780in}{4.042547in}}%
\pgfpathlineto{\pgfqpoint{3.077097in}{4.038023in}}%
\pgfpathlineto{\pgfqpoint{3.098693in}{4.103363in}}%
\pgfpathlineto{\pgfqpoint{3.116532in}{4.053349in}}%
\pgfpathlineto{\pgfqpoint{3.130616in}{4.039677in}}%
\pgfpathlineto{\pgfqpoint{3.155967in}{4.035456in}}%
\pgfpathlineto{\pgfqpoint{3.173337in}{4.035913in}}%
\pgfpathlineto{\pgfqpoint{3.190942in}{4.042124in}}%
\pgfpathlineto{\pgfqpoint{3.212536in}{4.064360in}}%
\pgfpathlineto{\pgfqpoint{3.232489in}{4.135898in}}%
\pgfpathlineto{\pgfqpoint{3.250797in}{4.332966in}}%
\pgfpathlineto{\pgfqpoint{3.268402in}{4.420004in}}%
\pgfpathlineto{\pgfqpoint{3.288120in}{4.376125in}}%
\pgfpathlineto{\pgfqpoint{3.308071in}{4.273503in}}%
\pgfpathlineto{\pgfqpoint{3.325676in}{4.128796in}}%
\pgfpathlineto{\pgfqpoint{3.346566in}{4.060571in}}%
\pgfpathlineto{\pgfqpoint{3.364171in}{4.043140in}}%
\pgfpathlineto{\pgfqpoint{3.388349in}{4.035559in}}%
\pgfpathlineto{\pgfqpoint{3.403606in}{4.035449in}}%
\pgfpathlineto{\pgfqpoint{3.424966in}{4.041235in}}%
\pgfpathlineto{\pgfqpoint{3.443041in}{4.052797in}}%
\pgfpathlineto{\pgfqpoint{3.460880in}{4.079012in}}%
\pgfpathlineto{\pgfqpoint{3.484822in}{4.226944in}}%
\pgfpathlineto{\pgfqpoint{3.499610in}{4.389914in}}%
\pgfpathlineto{\pgfqpoint{3.517920in}{4.423301in}}%
\pgfpathlineto{\pgfqpoint{3.539045in}{4.342040in}}%
\pgfpathlineto{\pgfqpoint{3.556415in}{4.203880in}}%
\pgfpathlineto{\pgfqpoint{3.577540in}{4.083327in}}%
\pgfpathlineto{\pgfqpoint{3.597728in}{4.051212in}}%
\pgfpathlineto{\pgfqpoint{3.617679in}{4.038734in}}%
\pgfpathlineto{\pgfqpoint{3.633641in}{4.066768in}}%
\pgfpathlineto{\pgfqpoint{3.652888in}{4.043116in}}%
\pgfpathlineto{\pgfqpoint{3.673544in}{4.037411in}}%
\pgfpathlineto{\pgfqpoint{3.692089in}{4.035424in}}%
\pgfpathlineto{\pgfqpoint{3.713448in}{4.040385in}}%
\pgfpathlineto{\pgfqpoint{3.731053in}{4.052655in}}%
\pgfpathlineto{\pgfqpoint{3.748658in}{4.088673in}}%
\pgfpathlineto{\pgfqpoint{3.769314in}{4.215585in}}%
\pgfpathlineto{\pgfqpoint{3.789736in}{4.384157in}}%
\pgfpathlineto{\pgfqpoint{3.807106in}{4.435335in}}%
\pgfpathlineto{\pgfqpoint{3.828466in}{4.369609in}}%
\pgfpathlineto{\pgfqpoint{3.847010in}{4.246770in}}%
\pgfpathlineto{\pgfqpoint{3.866258in}{4.137547in}}%
\pgfpathlineto{\pgfqpoint{3.884097in}{4.071657in}}%
\pgfpathlineto{\pgfqpoint{3.903815in}{4.046143in}}%
\pgfpathlineto{\pgfqpoint{3.923297in}{4.036963in}}%
\pgfpathlineto{\pgfqpoint{3.941371in}{4.035998in}}%
\pgfpathlineto{\pgfqpoint{3.962732in}{4.041436in}}%
\pgfpathlineto{\pgfqpoint{3.979398in}{4.052895in}}%
\pgfpathlineto{\pgfqpoint{3.998411in}{4.072730in}}%
\pgfpathlineto{\pgfqpoint{4.017893in}{4.140246in}}%
\pgfpathlineto{\pgfqpoint{4.037375in}{4.091043in}}%
\pgfpathlineto{\pgfqpoint{4.058971in}{4.232958in}}%
\pgfpathlineto{\pgfqpoint{4.076105in}{4.406936in}}%
\pgfpathlineto{\pgfqpoint{4.094415in}{4.446531in}}%
\pgfpathlineto{\pgfqpoint{4.115540in}{4.395279in}}%
\pgfpathlineto{\pgfqpoint{4.134084in}{4.266940in}}%
\pgfpathlineto{\pgfqpoint{4.155444in}{4.107699in}}%
\pgfpathlineto{\pgfqpoint{4.173519in}{4.063638in}}%
\pgfpathlineto{\pgfqpoint{4.193939in}{4.043149in}}%
\pgfpathlineto{\pgfqpoint{4.212952in}{4.037100in}}%
\pgfpathlineto{\pgfqpoint{4.230557in}{4.037315in}}%
\pgfpathlineto{\pgfqpoint{4.251684in}{4.044255in}}%
\pgfpathlineto{\pgfqpoint{4.269289in}{4.059867in}}%
\pgfpathlineto{\pgfqpoint{4.286894in}{4.072682in}}%
\pgfpathlineto{\pgfqpoint{4.310365in}{4.162593in}}%
\pgfpathlineto{\pgfqpoint{4.324449in}{4.313771in}}%
\pgfpathlineto{\pgfqpoint{4.344166in}{4.444133in}}%
\pgfpathlineto{\pgfqpoint{4.365762in}{4.451549in}}%
\pgfpathlineto{\pgfqpoint{4.383601in}{4.422413in}}%
\pgfpathlineto{\pgfqpoint{4.400971in}{4.313122in}}%
\pgfpathlineto{\pgfqpoint{4.423036in}{4.139045in}}%
\pgfpathlineto{\pgfqpoint{4.440641in}{4.074394in}}%
\pgfpathlineto{\pgfqpoint{4.458949in}{4.050947in}}%
\pgfpathlineto{\pgfqpoint{4.480076in}{4.052923in}}%
\pgfpathlineto{\pgfqpoint{4.473737in}{4.063732in}}%
\pgfpathlineto{\pgfqpoint{4.455428in}{4.147667in}}%
\pgfpathlineto{\pgfqpoint{4.436415in}{4.348711in}}%
\pgfpathlineto{\pgfqpoint{4.418107in}{4.453897in}}%
\pgfpathlineto{\pgfqpoint{4.396746in}{4.398869in}}%
\pgfpathlineto{\pgfqpoint{4.378203in}{4.144507in}}%
\pgfpathlineto{\pgfqpoint{4.359424in}{4.062089in}}%
\pgfpathlineto{\pgfqpoint{4.341351in}{4.040617in}}%
\pgfpathlineto{\pgfqpoint{4.319286in}{4.036831in}}%
\pgfpathlineto{\pgfqpoint{4.299568in}{4.046700in}}%
\pgfpathlineto{\pgfqpoint{4.282197in}{4.078808in}}%
\pgfpathlineto{\pgfqpoint{4.261307in}{4.230117in}}%
\pgfpathlineto{\pgfqpoint{4.243233in}{4.405237in}}%
\pgfpathlineto{\pgfqpoint{4.224454in}{4.445234in}}%
\pgfpathlineto{\pgfqpoint{4.204972in}{4.252316in}}%
\pgfpathlineto{\pgfqpoint{4.187602in}{4.095604in}}%
\pgfpathlineto{\pgfqpoint{4.166711in}{4.046549in}}%
\pgfpathlineto{\pgfqpoint{4.145821in}{4.036303in}}%
\pgfpathlineto{\pgfqpoint{4.128919in}{4.038048in}}%
\pgfpathlineto{\pgfqpoint{4.107794in}{4.054675in}}%
\pgfpathlineto{\pgfqpoint{4.092537in}{4.106259in}}%
\pgfpathlineto{\pgfqpoint{4.070707in}{4.319529in}}%
\pgfpathlineto{\pgfqpoint{4.051928in}{4.429931in}}%
\pgfpathlineto{\pgfqpoint{4.032446in}{4.403414in}}%
\pgfpathlineto{\pgfqpoint{4.012493in}{4.135207in}}%
\pgfpathlineto{\pgfqpoint{3.995359in}{4.061095in}}%
\pgfpathlineto{\pgfqpoint{3.974467in}{4.039425in}}%
\pgfpathlineto{\pgfqpoint{3.950759in}{4.035582in}}%
\pgfpathlineto{\pgfqpoint{3.937146in}{4.040050in}}%
\pgfpathlineto{\pgfqpoint{3.915786in}{4.067150in}}%
\pgfpathlineto{\pgfqpoint{3.898650in}{4.163965in}}%
\pgfpathlineto{\pgfqpoint{3.878229in}{4.369968in}}%
\pgfpathlineto{\pgfqpoint{3.860389in}{4.432046in}}%
\pgfpathlineto{\pgfqpoint{3.840202in}{4.272807in}}%
\pgfpathlineto{\pgfqpoint{3.820017in}{4.087557in}}%
\pgfpathlineto{\pgfqpoint{3.802412in}{4.053879in}}%
\pgfpathlineto{\pgfqpoint{3.781519in}{4.037418in}}%
\pgfpathlineto{\pgfqpoint{3.763680in}{4.035417in}}%
\pgfpathlineto{\pgfqpoint{3.744198in}{4.039286in}}%
\pgfpathlineto{\pgfqpoint{3.726124in}{4.050895in}}%
\pgfpathlineto{\pgfqpoint{3.706172in}{4.107054in}}%
\pgfpathlineto{\pgfqpoint{3.685515in}{4.307242in}}%
\pgfpathlineto{\pgfqpoint{3.667676in}{4.393214in}}%
\pgfpathlineto{\pgfqpoint{3.647489in}{4.420181in}}%
\pgfpathlineto{\pgfqpoint{3.631058in}{4.255737in}}%
\pgfpathlineto{\pgfqpoint{3.610402in}{4.094105in}}%
\pgfpathlineto{\pgfqpoint{3.588808in}{4.043793in}}%
\pgfpathlineto{\pgfqpoint{3.571672in}{4.038244in}}%
\pgfpathlineto{\pgfqpoint{3.547730in}{4.035144in}}%
\pgfpathlineto{\pgfqpoint{3.532942in}{4.037967in}}%
\pgfpathlineto{\pgfqpoint{3.515338in}{4.048831in}}%
\pgfpathlineto{\pgfqpoint{3.495150in}{4.092701in}}%
\pgfpathlineto{\pgfqpoint{3.455952in}{4.400795in}}%
\pgfpathlineto{\pgfqpoint{3.438347in}{4.415147in}}%
\pgfpathlineto{\pgfqpoint{3.417691in}{4.191839in}}%
\pgfpathlineto{\pgfqpoint{3.397035in}{4.081151in}}%
\pgfpathlineto{\pgfqpoint{3.379430in}{4.046816in}}%
\pgfpathlineto{\pgfqpoint{3.358068in}{4.036254in}}%
\pgfpathlineto{\pgfqpoint{3.340933in}{4.035262in}}%
\pgfpathlineto{\pgfqpoint{3.319104in}{4.039052in}}%
\pgfpathlineto{\pgfqpoint{3.301499in}{4.052502in}}%
\pgfpathlineto{\pgfqpoint{3.283660in}{4.101187in}}%
\pgfpathlineto{\pgfqpoint{3.245399in}{4.353546in}}%
\pgfpathlineto{\pgfqpoint{3.224743in}{4.417191in}}%
\pgfpathlineto{\pgfqpoint{3.207842in}{4.336758in}}%
\pgfpathlineto{\pgfqpoint{3.186246in}{4.115095in}}%
\pgfpathlineto{\pgfqpoint{3.168407in}{4.061213in}}%
\pgfpathlineto{\pgfqpoint{3.147282in}{4.064182in}}%
\pgfpathlineto{\pgfqpoint{3.129911in}{4.052570in}}%
\pgfpathlineto{\pgfqpoint{3.111838in}{4.038752in}}%
\pgfpathlineto{\pgfqpoint{3.091885in}{4.034893in}}%
\pgfpathlineto{\pgfqpoint{3.070525in}{4.038286in}}%
\pgfpathlineto{\pgfqpoint{3.051278in}{4.053133in}}%
\pgfpathlineto{\pgfqpoint{3.032499in}{4.112596in}}%
\pgfpathlineto{\pgfqpoint{3.014425in}{4.261651in}}%
\pgfpathlineto{\pgfqpoint{2.995412in}{4.403762in}}%
\pgfpathlineto{\pgfqpoint{2.974521in}{4.381085in}}%
\pgfpathlineto{\pgfqpoint{2.958325in}{4.194918in}}%
\pgfpathlineto{\pgfqpoint{2.937669in}{4.073427in}}%
\pgfpathlineto{\pgfqpoint{2.918890in}{4.046820in}}%
\pgfpathlineto{\pgfqpoint{2.899642in}{4.036934in}}%
\pgfpathlineto{\pgfqpoint{2.877578in}{4.034812in}}%
\pgfpathlineto{\pgfqpoint{2.862321in}{4.037069in}}%
\pgfpathlineto{\pgfqpoint{2.841194in}{4.052141in}}%
\pgfpathlineto{\pgfqpoint{2.821712in}{4.069327in}}%
\pgfpathlineto{\pgfqpoint{2.802464in}{4.110755in}}%
\pgfpathlineto{\pgfqpoint{2.765846in}{4.392730in}}%
\pgfpathlineto{\pgfqpoint{2.743782in}{4.369160in}}%
\pgfpathlineto{\pgfqpoint{2.705286in}{4.097742in}}%
\pgfpathlineto{\pgfqpoint{2.686273in}{4.058819in}}%
\pgfpathlineto{\pgfqpoint{2.668199in}{4.040133in}}%
\pgfpathlineto{\pgfqpoint{2.649421in}{4.035320in}}%
\pgfpathlineto{\pgfqpoint{2.630407in}{4.035378in}}%
\pgfpathlineto{\pgfqpoint{2.615854in}{4.042235in}}%
\pgfpathlineto{\pgfqpoint{2.592146in}{4.065486in}}%
\pgfpathlineto{\pgfqpoint{2.573368in}{4.129975in}}%
\pgfpathlineto{\pgfqpoint{2.551069in}{4.309524in}}%
\pgfpathlineto{\pgfqpoint{2.536046in}{4.360094in}}%
\pgfpathlineto{\pgfqpoint{2.515156in}{4.402976in}}%
\pgfpathlineto{\pgfqpoint{2.494968in}{4.296641in}}%
\pgfpathlineto{\pgfqpoint{2.476660in}{4.157371in}}%
\pgfpathlineto{\pgfqpoint{2.454596in}{4.064556in}}%
\pgfpathlineto{\pgfqpoint{2.436051in}{4.042122in}}%
\pgfpathlineto{\pgfqpoint{2.417272in}{4.035639in}}%
\pgfpathlineto{\pgfqpoint{2.398964in}{4.034840in}}%
\pgfpathlineto{\pgfqpoint{2.380420in}{4.038205in}}%
\pgfpathlineto{\pgfqpoint{2.362581in}{4.050770in}}%
\pgfpathlineto{\pgfqpoint{2.337230in}{4.093116in}}%
\pgfpathlineto{\pgfqpoint{2.321503in}{4.186165in}}%
\pgfpathlineto{\pgfqpoint{2.302726in}{4.344675in}}%
\pgfpathlineto{\pgfqpoint{2.283947in}{4.408340in}}%
\pgfpathlineto{\pgfqpoint{2.262117in}{4.380686in}}%
\pgfpathlineto{\pgfqpoint{2.243338in}{4.176625in}}%
\pgfpathlineto{\pgfqpoint{2.227613in}{4.086814in}}%
\pgfpathlineto{\pgfqpoint{2.206251in}{4.048577in}}%
\pgfpathlineto{\pgfqpoint{2.187004in}{4.037893in}}%
\pgfpathlineto{\pgfqpoint{2.169164in}{4.034908in}}%
\pgfpathlineto{\pgfqpoint{2.151091in}{4.035825in}}%
\pgfpathlineto{\pgfqpoint{2.129729in}{4.042680in}}%
\pgfpathlineto{\pgfqpoint{2.110716in}{4.058095in}}%
\pgfpathlineto{\pgfqpoint{2.092174in}{4.104296in}}%
\pgfpathlineto{\pgfqpoint{2.073629in}{4.217711in}}%
\pgfpathlineto{\pgfqpoint{2.052739in}{4.365755in}}%
\pgfpathlineto{\pgfqpoint{2.033491in}{4.386529in}}%
\pgfpathlineto{\pgfqpoint{2.014243in}{4.084585in}}%
\pgfpathlineto{\pgfqpoint{1.996170in}{4.219029in}}%
\pgfpathlineto{\pgfqpoint{1.977391in}{4.313271in}}%
\pgfpathlineto{\pgfqpoint{1.955561in}{4.412396in}}%
\pgfpathlineto{\pgfqpoint{1.936313in}{4.368964in}}%
\pgfpathlineto{\pgfqpoint{1.918005in}{4.165198in}}%
\pgfpathlineto{\pgfqpoint{1.899695in}{4.068753in}}%
\pgfpathlineto{\pgfqpoint{1.881152in}{4.045086in}}%
\pgfpathlineto{\pgfqpoint{1.861199in}{4.036736in}}%
\pgfpathlineto{\pgfqpoint{1.841012in}{4.035141in}}%
\pgfpathlineto{\pgfqpoint{1.822470in}{4.039232in}}%
\pgfpathlineto{\pgfqpoint{1.803456in}{4.053197in}}%
\pgfpathlineto{\pgfqpoint{1.785383in}{4.101077in}}%
\pgfpathlineto{\pgfqpoint{1.747825in}{4.342140in}}%
\pgfpathlineto{\pgfqpoint{1.726229in}{4.406576in}}%
\pgfpathlineto{\pgfqpoint{1.707216in}{4.402488in}}%
\pgfpathlineto{\pgfqpoint{1.685857in}{4.190423in}}%
\pgfpathlineto{\pgfqpoint{1.664261in}{4.081432in}}%
\pgfpathlineto{\pgfqpoint{1.645718in}{4.048065in}}%
\pgfpathlineto{\pgfqpoint{1.629757in}{4.039241in}}%
\pgfpathlineto{\pgfqpoint{1.610743in}{4.035582in}}%
\pgfpathlineto{\pgfqpoint{1.593373in}{4.035941in}}%
\pgfpathlineto{\pgfqpoint{1.571779in}{4.038383in}}%
\pgfpathlineto{\pgfqpoint{1.553469in}{4.050322in}}%
\pgfpathlineto{\pgfqpoint{1.534456in}{4.089861in}}%
\pgfpathlineto{\pgfqpoint{1.512626in}{4.221481in}}%
\pgfpathlineto{\pgfqpoint{1.497369in}{4.346266in}}%
\pgfpathlineto{\pgfqpoint{1.476009in}{4.423820in}}%
\pgfpathlineto{\pgfqpoint{1.457934in}{4.399092in}}%
\pgfpathlineto{\pgfqpoint{1.438217in}{4.213987in}}%
\pgfpathlineto{\pgfqpoint{1.416153in}{4.085374in}}%
\pgfpathlineto{\pgfqpoint{1.401599in}{4.056100in}}%
\pgfpathlineto{\pgfqpoint{1.380240in}{4.046494in}}%
\pgfpathlineto{\pgfqpoint{1.361227in}{4.040165in}}%
\pgfpathlineto{\pgfqpoint{1.342917in}{4.035893in}}%
\pgfpathlineto{\pgfqpoint{1.324374in}{4.036894in}}%
\pgfpathlineto{\pgfqpoint{1.303013in}{4.044528in}}%
\pgfpathlineto{\pgfqpoint{1.284470in}{4.065546in}}%
\pgfpathlineto{\pgfqpoint{1.266160in}{4.117824in}}%
\pgfpathlineto{\pgfqpoint{1.246444in}{4.221473in}}%
\pgfpathlineto{\pgfqpoint{1.226022in}{4.370314in}}%
\pgfpathlineto{\pgfqpoint{1.207712in}{4.424558in}}%
\pgfpathlineto{\pgfqpoint{1.189170in}{4.432294in}}%
\pgfpathlineto{\pgfqpoint{1.169688in}{4.342553in}}%
\pgfpathlineto{\pgfqpoint{1.152083in}{4.214912in}}%
\pgfpathlineto{\pgfqpoint{1.130253in}{4.124561in}}%
\pgfpathlineto{\pgfqpoint{1.109831in}{4.061890in}}%
\pgfpathlineto{\pgfqpoint{1.092460in}{4.046654in}}%
\pgfpathlineto{\pgfqpoint{1.074387in}{4.042748in}}%
\pgfpathlineto{\pgfqpoint{1.052557in}{4.036451in}}%
\pgfpathlineto{\pgfqpoint{1.034014in}{4.037293in}}%
\pgfpathlineto{\pgfqpoint{1.015704in}{4.041914in}}%
\pgfpathlineto{\pgfqpoint{0.997631in}{4.056466in}}%
\pgfpathlineto{\pgfqpoint{0.975800in}{4.096860in}}%
\pgfpathlineto{\pgfqpoint{0.960073in}{4.196922in}}%
\pgfpathlineto{\pgfqpoint{0.938245in}{4.361544in}}%
\pgfpathlineto{\pgfqpoint{0.920169in}{4.393223in}}%
\pgfpathlineto{\pgfqpoint{0.901390in}{4.053151in}}%
\pgfpathlineto{\pgfqpoint{0.880265in}{4.107353in}}%
\pgfpathlineto{\pgfqpoint{0.862426in}{4.247749in}}%
\pgfpathlineto{\pgfqpoint{0.842239in}{4.397421in}}%
\pgfpathlineto{\pgfqpoint{0.823462in}{4.448900in}}%
\pgfpathlineto{\pgfqpoint{0.802806in}{4.382160in}}%
\pgfpathlineto{\pgfqpoint{0.784730in}{4.164944in}}%
\pgfpathlineto{\pgfqpoint{0.766656in}{4.084468in}}%
\pgfpathlineto{\pgfqpoint{0.744826in}{4.051662in}}%
\pgfpathlineto{\pgfqpoint{0.725579in}{4.039289in}}%
\pgfpathlineto{\pgfqpoint{0.707036in}{4.036573in}}%
\pgfpathlineto{\pgfqpoint{0.688960in}{4.038580in}}%
\pgfpathlineto{\pgfqpoint{0.669947in}{4.048688in}}%
\pgfpathlineto{\pgfqpoint{0.648353in}{4.074161in}}%
\pgfpathlineto{\pgfqpoint{0.648353in}{4.073396in}}%
\pgfpathlineto{\pgfqpoint{0.656802in}{4.053588in}}%
\pgfpathlineto{\pgfqpoint{0.677224in}{4.038172in}}%
\pgfpathlineto{\pgfqpoint{0.695768in}{4.038140in}}%
\pgfpathlineto{\pgfqpoint{0.711965in}{4.049593in}}%
\pgfpathlineto{\pgfqpoint{0.732855in}{4.097532in}}%
\pgfpathlineto{\pgfqpoint{0.753980in}{4.307558in}}%
\pgfpathlineto{\pgfqpoint{0.772994in}{4.448618in}}%
\pgfpathlineto{\pgfqpoint{0.793181in}{4.364603in}}%
\pgfpathlineto{\pgfqpoint{0.809143in}{4.183395in}}%
\pgfpathlineto{\pgfqpoint{0.829330in}{4.068035in}}%
\pgfpathlineto{\pgfqpoint{0.847169in}{4.041918in}}%
\pgfpathlineto{\pgfqpoint{0.868763in}{4.036179in}}%
\pgfpathlineto{\pgfqpoint{0.888482in}{4.043253in}}%
\pgfpathlineto{\pgfqpoint{0.907024in}{4.072213in}}%
\pgfpathlineto{\pgfqpoint{0.925803in}{4.182898in}}%
\pgfpathlineto{\pgfqpoint{0.943876in}{4.414329in}}%
\pgfpathlineto{\pgfqpoint{0.964298in}{4.415688in}}%
\pgfpathlineto{\pgfqpoint{0.983546in}{4.237001in}}%
\pgfpathlineto{\pgfqpoint{1.002559in}{4.087547in}}%
\pgfpathlineto{\pgfqpoint{1.022041in}{4.045837in}}%
\pgfpathlineto{\pgfqpoint{1.041055in}{4.036182in}}%
\pgfpathlineto{\pgfqpoint{1.059833in}{4.037846in}}%
\pgfpathlineto{\pgfqpoint{1.080021in}{4.052499in}}%
\pgfpathlineto{\pgfqpoint{1.098799in}{4.100419in}}%
\pgfpathlineto{\pgfqpoint{1.117108in}{4.283513in}}%
\pgfpathlineto{\pgfqpoint{1.137059in}{4.433649in}}%
\pgfpathlineto{\pgfqpoint{1.155603in}{4.340667in}}%
\pgfpathlineto{\pgfqpoint{1.177668in}{4.119031in}}%
\pgfpathlineto{\pgfqpoint{1.194333in}{4.058449in}}%
\pgfpathlineto{\pgfqpoint{1.212877in}{4.039421in}}%
\pgfpathlineto{\pgfqpoint{1.232828in}{4.035233in}}%
\pgfpathlineto{\pgfqpoint{1.252312in}{4.039868in}}%
\pgfpathlineto{\pgfqpoint{1.270855in}{4.058103in}}%
\pgfpathlineto{\pgfqpoint{1.290102in}{4.105050in}}%
\pgfpathlineto{\pgfqpoint{1.308881in}{4.278700in}}%
\pgfpathlineto{\pgfqpoint{1.330477in}{4.426284in}}%
\pgfpathlineto{\pgfqpoint{1.349725in}{4.335446in}}%
\pgfpathlineto{\pgfqpoint{1.366155in}{4.181798in}}%
\pgfpathlineto{\pgfqpoint{1.388454in}{4.066803in}}%
\pgfpathlineto{\pgfqpoint{1.406528in}{4.041181in}}%
\pgfpathlineto{\pgfqpoint{1.425072in}{4.035520in}}%
\pgfpathlineto{\pgfqpoint{1.444554in}{4.036880in}}%
\pgfpathlineto{\pgfqpoint{1.463568in}{4.035306in}}%
\pgfpathlineto{\pgfqpoint{1.482815in}{4.040583in}}%
\pgfpathlineto{\pgfqpoint{1.501594in}{4.056659in}}%
\pgfpathlineto{\pgfqpoint{1.524128in}{4.111314in}}%
\pgfpathlineto{\pgfqpoint{1.539150in}{4.268988in}}%
\pgfpathlineto{\pgfqpoint{1.561685in}{4.416597in}}%
\pgfpathlineto{\pgfqpoint{1.580462in}{4.370324in}}%
\pgfpathlineto{\pgfqpoint{1.599476in}{4.194623in}}%
\pgfpathlineto{\pgfqpoint{1.617551in}{4.075008in}}%
\pgfpathlineto{\pgfqpoint{1.636799in}{4.042782in}}%
\pgfpathlineto{\pgfqpoint{1.654638in}{4.035592in}}%
\pgfpathlineto{\pgfqpoint{1.673181in}{4.035636in}}%
\pgfpathlineto{\pgfqpoint{1.695716in}{4.043842in}}%
\pgfpathlineto{\pgfqpoint{1.714964in}{4.071511in}}%
\pgfpathlineto{\pgfqpoint{1.732334in}{4.155832in}}%
\pgfpathlineto{\pgfqpoint{1.753693in}{4.376282in}}%
\pgfpathlineto{\pgfqpoint{1.769890in}{4.417476in}}%
\pgfpathlineto{\pgfqpoint{1.790546in}{4.288367in}}%
\pgfpathlineto{\pgfqpoint{1.809325in}{4.117682in}}%
\pgfpathlineto{\pgfqpoint{1.828104in}{4.059147in}}%
\pgfpathlineto{\pgfqpoint{1.849932in}{4.038700in}}%
\pgfpathlineto{\pgfqpoint{1.869416in}{4.035095in}}%
\pgfpathlineto{\pgfqpoint{1.887724in}{4.035690in}}%
\pgfpathlineto{\pgfqpoint{1.905563in}{4.041663in}}%
\pgfpathlineto{\pgfqpoint{1.922228in}{4.058776in}}%
\pgfpathlineto{\pgfqpoint{1.940773in}{4.097204in}}%
\pgfpathlineto{\pgfqpoint{1.963072in}{4.253143in}}%
\pgfpathlineto{\pgfqpoint{1.983025in}{4.415477in}}%
\pgfpathlineto{\pgfqpoint{2.002741in}{4.394155in}}%
\pgfpathlineto{\pgfqpoint{2.020346in}{4.261926in}}%
\pgfpathlineto{\pgfqpoint{2.042879in}{4.082213in}}%
\pgfpathlineto{\pgfqpoint{2.057667in}{4.050938in}}%
\pgfpathlineto{\pgfqpoint{2.077855in}{4.037456in}}%
\pgfpathlineto{\pgfqpoint{2.095694in}{4.034918in}}%
\pgfpathlineto{\pgfqpoint{2.117055in}{4.036447in}}%
\pgfpathlineto{\pgfqpoint{2.137946in}{4.043564in}}%
\pgfpathlineto{\pgfqpoint{2.155316in}{4.068698in}}%
\pgfpathlineto{\pgfqpoint{2.173390in}{4.108945in}}%
\pgfpathlineto{\pgfqpoint{2.194046in}{4.216659in}}%
\pgfpathlineto{\pgfqpoint{2.213293in}{4.372572in}}%
\pgfpathlineto{\pgfqpoint{2.231133in}{4.407275in}}%
\pgfpathlineto{\pgfqpoint{2.251086in}{4.354355in}}%
\pgfpathlineto{\pgfqpoint{2.271976in}{4.158520in}}%
\pgfpathlineto{\pgfqpoint{2.293336in}{4.081595in}}%
\pgfpathlineto{\pgfqpoint{2.307655in}{4.048131in}}%
\pgfpathlineto{\pgfqpoint{2.329016in}{4.036575in}}%
\pgfpathlineto{\pgfqpoint{2.347324in}{4.034771in}}%
\pgfpathlineto{\pgfqpoint{2.368683in}{4.036802in}}%
\pgfpathlineto{\pgfqpoint{2.385819in}{4.045860in}}%
\pgfpathlineto{\pgfqpoint{2.404129in}{4.065223in}}%
\pgfpathlineto{\pgfqpoint{2.424315in}{4.157483in}}%
\pgfpathlineto{\pgfqpoint{2.443094in}{4.372722in}}%
\pgfpathlineto{\pgfqpoint{2.462341in}{4.405014in}}%
\pgfpathlineto{\pgfqpoint{2.482997in}{4.378211in}}%
\pgfpathlineto{\pgfqpoint{2.503888in}{4.198000in}}%
\pgfpathlineto{\pgfqpoint{2.521258in}{4.092788in}}%
\pgfpathlineto{\pgfqpoint{2.538160in}{4.170408in}}%
\pgfpathlineto{\pgfqpoint{2.561162in}{4.393763in}}%
\pgfpathlineto{\pgfqpoint{2.580410in}{4.413042in}}%
\pgfpathlineto{\pgfqpoint{2.598015in}{4.341900in}}%
\pgfpathlineto{\pgfqpoint{2.616793in}{4.160395in}}%
\pgfpathlineto{\pgfqpoint{2.635572in}{4.061457in}}%
\pgfpathlineto{\pgfqpoint{2.656228in}{4.039933in}}%
\pgfpathlineto{\pgfqpoint{2.675476in}{4.035297in}}%
\pgfpathlineto{\pgfqpoint{2.693550in}{4.036081in}}%
\pgfpathlineto{\pgfqpoint{2.711623in}{4.043139in}}%
\pgfpathlineto{\pgfqpoint{2.730402in}{4.066481in}}%
\pgfpathlineto{\pgfqpoint{2.751058in}{4.157962in}}%
\pgfpathlineto{\pgfqpoint{2.770306in}{4.376952in}}%
\pgfpathlineto{\pgfqpoint{2.789788in}{4.401167in}}%
\pgfpathlineto{\pgfqpoint{2.808333in}{4.271741in}}%
\pgfpathlineto{\pgfqpoint{2.829458in}{4.106411in}}%
\pgfpathlineto{\pgfqpoint{2.847062in}{4.053014in}}%
\pgfpathlineto{\pgfqpoint{2.865136in}{4.039847in}}%
\pgfpathlineto{\pgfqpoint{2.885558in}{4.035278in}}%
\pgfpathlineto{\pgfqpoint{2.903868in}{4.035924in}}%
\pgfpathlineto{\pgfqpoint{2.924758in}{4.044185in}}%
\pgfpathlineto{\pgfqpoint{2.943537in}{4.067745in}}%
\pgfpathlineto{\pgfqpoint{2.963488in}{4.166715in}}%
\pgfpathlineto{\pgfqpoint{2.981093in}{4.349765in}}%
\pgfpathlineto{\pgfqpoint{2.998698in}{4.413747in}}%
\pgfpathlineto{\pgfqpoint{3.020762in}{4.347372in}}%
\pgfpathlineto{\pgfqpoint{3.056675in}{4.096468in}}%
\pgfpathlineto{\pgfqpoint{3.081322in}{4.050056in}}%
\pgfpathlineto{\pgfqpoint{3.096815in}{4.039440in}}%
\pgfpathlineto{\pgfqpoint{3.117706in}{4.035059in}}%
\pgfpathlineto{\pgfqpoint{3.139065in}{4.037878in}}%
\pgfpathlineto{\pgfqpoint{3.153384in}{4.044346in}}%
\pgfpathlineto{\pgfqpoint{3.173103in}{4.068883in}}%
\pgfpathlineto{\pgfqpoint{3.197514in}{4.148687in}}%
\pgfpathlineto{\pgfqpoint{3.212301in}{4.243360in}}%
\pgfpathlineto{\pgfqpoint{3.229906in}{4.408558in}}%
\pgfpathlineto{\pgfqpoint{3.250797in}{4.386737in}}%
\pgfpathlineto{\pgfqpoint{3.269107in}{4.257498in}}%
\pgfpathlineto{\pgfqpoint{3.292109in}{4.109615in}}%
\pgfpathlineto{\pgfqpoint{3.309714in}{4.062170in}}%
\pgfpathlineto{\pgfqpoint{3.326145in}{4.045484in}}%
\pgfpathlineto{\pgfqpoint{3.347506in}{4.037209in}}%
\pgfpathlineto{\pgfqpoint{3.365111in}{4.035211in}}%
\pgfpathlineto{\pgfqpoint{3.386236in}{4.037399in}}%
\pgfpathlineto{\pgfqpoint{3.403841in}{4.041282in}}%
\pgfpathlineto{\pgfqpoint{3.423088in}{4.059029in}}%
\pgfpathlineto{\pgfqpoint{3.443744in}{4.112186in}}%
\pgfpathlineto{\pgfqpoint{3.479188in}{4.400644in}}%
\pgfpathlineto{\pgfqpoint{3.500550in}{4.413167in}}%
\pgfpathlineto{\pgfqpoint{3.518858in}{4.331349in}}%
\pgfpathlineto{\pgfqpoint{3.540922in}{4.136909in}}%
\pgfpathlineto{\pgfqpoint{3.558762in}{4.077769in}}%
\pgfpathlineto{\pgfqpoint{3.576601in}{4.155328in}}%
\pgfpathlineto{\pgfqpoint{3.597493in}{4.425042in}}%
\pgfpathlineto{\pgfqpoint{3.614393in}{4.371184in}}%
\pgfpathlineto{\pgfqpoint{3.635754in}{4.226044in}}%
\pgfpathlineto{\pgfqpoint{3.653123in}{4.097773in}}%
\pgfpathlineto{\pgfqpoint{3.671433in}{4.055062in}}%
\pgfpathlineto{\pgfqpoint{3.694201in}{4.038465in}}%
\pgfpathlineto{\pgfqpoint{3.711102in}{4.035477in}}%
\pgfpathlineto{\pgfqpoint{3.731758in}{4.039020in}}%
\pgfpathlineto{\pgfqpoint{3.749597in}{4.048120in}}%
\pgfpathlineto{\pgfqpoint{3.770722in}{4.082152in}}%
\pgfpathlineto{\pgfqpoint{3.789736in}{4.173865in}}%
\pgfpathlineto{\pgfqpoint{3.807809in}{4.342886in}}%
\pgfpathlineto{\pgfqpoint{3.826354in}{4.433965in}}%
\pgfpathlineto{\pgfqpoint{3.847479in}{4.394182in}}%
\pgfpathlineto{\pgfqpoint{3.864615in}{4.259917in}}%
\pgfpathlineto{\pgfqpoint{3.885740in}{4.113736in}}%
\pgfpathlineto{\pgfqpoint{3.902407in}{4.064720in}}%
\pgfpathlineto{\pgfqpoint{3.924001in}{4.043786in}}%
\pgfpathlineto{\pgfqpoint{3.942074in}{4.037472in}}%
\pgfpathlineto{\pgfqpoint{3.960619in}{4.035920in}}%
\pgfpathlineto{\pgfqpoint{3.981275in}{4.039322in}}%
\pgfpathlineto{\pgfqpoint{3.999348in}{4.050051in}}%
\pgfpathlineto{\pgfqpoint{4.019067in}{4.074616in}}%
\pgfpathlineto{\pgfqpoint{4.038549in}{4.147404in}}%
\pgfpathlineto{\pgfqpoint{4.059440in}{4.345001in}}%
\pgfpathlineto{\pgfqpoint{4.077044in}{4.441939in}}%
\pgfpathlineto{\pgfqpoint{4.096292in}{4.424851in}}%
\pgfpathlineto{\pgfqpoint{4.114837in}{4.332566in}}%
\pgfpathlineto{\pgfqpoint{4.135493in}{4.209876in}}%
\pgfpathlineto{\pgfqpoint{4.156383in}{4.093391in}}%
\pgfpathlineto{\pgfqpoint{4.173048in}{4.059682in}}%
\pgfpathlineto{\pgfqpoint{4.192296in}{4.047985in}}%
\pgfpathlineto{\pgfqpoint{4.212952in}{4.037418in}}%
\pgfpathlineto{\pgfqpoint{4.231028in}{4.036926in}}%
\pgfpathlineto{\pgfqpoint{4.249101in}{4.042880in}}%
\pgfpathlineto{\pgfqpoint{4.270697in}{4.055864in}}%
\pgfpathlineto{\pgfqpoint{4.287128in}{4.089728in}}%
\pgfpathlineto{\pgfqpoint{4.308722in}{4.184512in}}%
\pgfpathlineto{\pgfqpoint{4.326563in}{4.348319in}}%
\pgfpathlineto{\pgfqpoint{4.344871in}{4.439851in}}%
\pgfpathlineto{\pgfqpoint{4.365996in}{4.447487in}}%
\pgfpathlineto{\pgfqpoint{4.383835in}{4.372470in}}%
\pgfpathlineto{\pgfqpoint{4.405900in}{4.216931in}}%
\pgfpathlineto{\pgfqpoint{4.424444in}{4.115093in}}%
\pgfpathlineto{\pgfqpoint{4.442049in}{4.077882in}}%
\pgfpathlineto{\pgfqpoint{4.459888in}{4.050189in}}%
\pgfpathlineto{\pgfqpoint{4.481250in}{4.041069in}}%
\pgfpathlineto{\pgfqpoint{4.480779in}{4.041054in}}%
\pgfpathlineto{\pgfqpoint{4.474676in}{4.045208in}}%
\pgfpathlineto{\pgfqpoint{4.452143in}{4.088405in}}%
\pgfpathlineto{\pgfqpoint{4.433833in}{4.226805in}}%
\pgfpathlineto{\pgfqpoint{4.416933in}{4.408347in}}%
\pgfpathlineto{\pgfqpoint{4.393226in}{4.452717in}}%
\pgfpathlineto{\pgfqpoint{4.378438in}{4.316284in}}%
\pgfpathlineto{\pgfqpoint{4.357076in}{4.105883in}}%
\pgfpathlineto{\pgfqpoint{4.338768in}{4.055929in}}%
\pgfpathlineto{\pgfqpoint{4.322338in}{4.129855in}}%
\pgfpathlineto{\pgfqpoint{4.301445in}{4.357529in}}%
\pgfpathlineto{\pgfqpoint{4.282434in}{4.449880in}}%
\pgfpathlineto{\pgfqpoint{4.262246in}{4.349400in}}%
\pgfpathlineto{\pgfqpoint{4.243233in}{4.270812in}}%
\pgfpathlineto{\pgfqpoint{4.222577in}{4.087148in}}%
\pgfpathlineto{\pgfqpoint{4.206146in}{4.047997in}}%
\pgfpathlineto{\pgfqpoint{4.185256in}{4.036856in}}%
\pgfpathlineto{\pgfqpoint{4.165068in}{4.038855in}}%
\pgfpathlineto{\pgfqpoint{4.148403in}{4.052975in}}%
\pgfpathlineto{\pgfqpoint{4.126807in}{4.132780in}}%
\pgfpathlineto{\pgfqpoint{4.111080in}{4.310331in}}%
\pgfpathlineto{\pgfqpoint{4.089721in}{4.438851in}}%
\pgfpathlineto{\pgfqpoint{4.071881in}{4.390714in}}%
\pgfpathlineto{\pgfqpoint{4.051928in}{4.124076in}}%
\pgfpathlineto{\pgfqpoint{4.028221in}{4.049369in}}%
\pgfpathlineto{\pgfqpoint{4.010147in}{4.037753in}}%
\pgfpathlineto{\pgfqpoint{3.995123in}{4.035788in}}%
\pgfpathlineto{\pgfqpoint{3.974467in}{4.044574in}}%
\pgfpathlineto{\pgfqpoint{3.958741in}{4.073852in}}%
\pgfpathlineto{\pgfqpoint{3.935737in}{4.213719in}}%
\pgfpathlineto{\pgfqpoint{3.918132in}{4.377768in}}%
\pgfpathlineto{\pgfqpoint{3.897242in}{4.426960in}}%
\pgfpathlineto{\pgfqpoint{3.879637in}{4.224819in}}%
\pgfpathlineto{\pgfqpoint{3.860858in}{4.086336in}}%
\pgfpathlineto{\pgfqpoint{3.841611in}{4.047846in}}%
\pgfpathlineto{\pgfqpoint{3.821189in}{4.036639in}}%
\pgfpathlineto{\pgfqpoint{3.800298in}{4.035841in}}%
\pgfpathlineto{\pgfqpoint{3.782225in}{4.043012in}}%
\pgfpathlineto{\pgfqpoint{3.761803in}{4.079765in}}%
\pgfpathlineto{\pgfqpoint{3.746075in}{4.176700in}}%
\pgfpathlineto{\pgfqpoint{3.728236in}{4.378709in}}%
\pgfpathlineto{\pgfqpoint{3.703825in}{4.421481in}}%
\pgfpathlineto{\pgfqpoint{3.681995in}{4.187759in}}%
\pgfpathlineto{\pgfqpoint{3.666268in}{4.091328in}}%
\pgfpathlineto{\pgfqpoint{3.647725in}{4.049472in}}%
\pgfpathlineto{\pgfqpoint{3.628712in}{4.107961in}}%
\pgfpathlineto{\pgfqpoint{3.608994in}{4.056257in}}%
\pgfpathlineto{\pgfqpoint{3.592328in}{4.039735in}}%
\pgfpathlineto{\pgfqpoint{3.570498in}{4.035271in}}%
\pgfpathlineto{\pgfqpoint{3.549842in}{4.039612in}}%
\pgfpathlineto{\pgfqpoint{3.532472in}{4.051288in}}%
\pgfpathlineto{\pgfqpoint{3.512052in}{4.109922in}}%
\pgfpathlineto{\pgfqpoint{3.476137in}{4.384961in}}%
\pgfpathlineto{\pgfqpoint{3.455481in}{4.407890in}}%
\pgfpathlineto{\pgfqpoint{3.437642in}{4.222901in}}%
\pgfpathlineto{\pgfqpoint{3.417454in}{4.085746in}}%
\pgfpathlineto{\pgfqpoint{3.399615in}{4.048739in}}%
\pgfpathlineto{\pgfqpoint{3.379430in}{4.037545in}}%
\pgfpathlineto{\pgfqpoint{3.360651in}{4.035176in}}%
\pgfpathlineto{\pgfqpoint{3.339995in}{4.037691in}}%
\pgfpathlineto{\pgfqpoint{3.322390in}{4.048253in}}%
\pgfpathlineto{\pgfqpoint{3.304551in}{4.077957in}}%
\pgfpathlineto{\pgfqpoint{3.282015in}{4.193362in}}%
\pgfpathlineto{\pgfqpoint{3.263942in}{4.052995in}}%
\pgfpathlineto{\pgfqpoint{3.242112in}{4.126964in}}%
\pgfpathlineto{\pgfqpoint{3.225681in}{4.274201in}}%
\pgfpathlineto{\pgfqpoint{3.206433in}{4.406033in}}%
\pgfpathlineto{\pgfqpoint{3.186482in}{4.380494in}}%
\pgfpathlineto{\pgfqpoint{3.168172in}{4.178961in}}%
\pgfpathlineto{\pgfqpoint{3.147750in}{4.066180in}}%
\pgfpathlineto{\pgfqpoint{3.128974in}{4.043112in}}%
\pgfpathlineto{\pgfqpoint{3.110429in}{4.035917in}}%
\pgfpathlineto{\pgfqpoint{3.087896in}{4.036232in}}%
\pgfpathlineto{\pgfqpoint{3.072871in}{4.040867in}}%
\pgfpathlineto{\pgfqpoint{3.051043in}{4.055807in}}%
\pgfpathlineto{\pgfqpoint{3.034376in}{4.117310in}}%
\pgfpathlineto{\pgfqpoint{3.010669in}{4.325854in}}%
\pgfpathlineto{\pgfqpoint{2.994943in}{4.412905in}}%
\pgfpathlineto{\pgfqpoint{2.977102in}{4.400736in}}%
\pgfpathlineto{\pgfqpoint{2.957854in}{4.205461in}}%
\pgfpathlineto{\pgfqpoint{2.933443in}{4.068286in}}%
\pgfpathlineto{\pgfqpoint{2.917950in}{4.052087in}}%
\pgfpathlineto{\pgfqpoint{2.895651in}{4.040101in}}%
\pgfpathlineto{\pgfqpoint{2.880395in}{4.035760in}}%
\pgfpathlineto{\pgfqpoint{2.859504in}{4.035602in}}%
\pgfpathlineto{\pgfqpoint{2.839786in}{4.041915in}}%
\pgfpathlineto{\pgfqpoint{2.822181in}{4.060800in}}%
\pgfpathlineto{\pgfqpoint{2.804107in}{4.144787in}}%
\pgfpathlineto{\pgfqpoint{2.784860in}{4.291819in}}%
\pgfpathlineto{\pgfqpoint{2.765377in}{4.407626in}}%
\pgfpathlineto{\pgfqpoint{2.742608in}{4.333954in}}%
\pgfpathlineto{\pgfqpoint{2.726177in}{4.147452in}}%
\pgfpathlineto{\pgfqpoint{2.706226in}{4.064115in}}%
\pgfpathlineto{\pgfqpoint{2.687447in}{4.042393in}}%
\pgfpathlineto{\pgfqpoint{2.666557in}{4.035212in}}%
\pgfpathlineto{\pgfqpoint{2.648247in}{4.035311in}}%
\pgfpathlineto{\pgfqpoint{2.629468in}{4.039039in}}%
\pgfpathlineto{\pgfqpoint{2.610456in}{4.051349in}}%
\pgfpathlineto{\pgfqpoint{2.590972in}{4.079741in}}%
\pgfpathlineto{\pgfqpoint{2.572664in}{4.179938in}}%
\pgfpathlineto{\pgfqpoint{2.550834in}{4.350270in}}%
\pgfpathlineto{\pgfqpoint{2.532526in}{4.407834in}}%
\pgfpathlineto{\pgfqpoint{2.514687in}{4.385810in}}%
\pgfpathlineto{\pgfqpoint{2.494499in}{4.225364in}}%
\pgfpathlineto{\pgfqpoint{2.476190in}{4.103832in}}%
\pgfpathlineto{\pgfqpoint{2.457413in}{4.052386in}}%
\pgfpathlineto{\pgfqpoint{2.438165in}{4.038706in}}%
\pgfpathlineto{\pgfqpoint{2.417272in}{4.034961in}}%
\pgfpathlineto{\pgfqpoint{2.398496in}{4.035834in}}%
\pgfpathlineto{\pgfqpoint{2.380186in}{4.041292in}}%
\pgfpathlineto{\pgfqpoint{2.361409in}{4.059687in}}%
\pgfpathlineto{\pgfqpoint{2.342630in}{4.093573in}}%
\pgfpathlineto{\pgfqpoint{2.302492in}{4.306009in}}%
\pgfpathlineto{\pgfqpoint{2.284182in}{4.406977in}}%
\pgfpathlineto{\pgfqpoint{2.266108in}{4.359734in}}%
\pgfpathlineto{\pgfqpoint{2.242869in}{4.139146in}}%
\pgfpathlineto{\pgfqpoint{2.224090in}{4.066756in}}%
\pgfpathlineto{\pgfqpoint{2.206017in}{4.046346in}}%
\pgfpathlineto{\pgfqpoint{2.187709in}{4.038043in}}%
\pgfpathlineto{\pgfqpoint{2.169633in}{4.035071in}}%
\pgfpathlineto{\pgfqpoint{2.146865in}{4.037808in}}%
\pgfpathlineto{\pgfqpoint{2.129026in}{4.047017in}}%
\pgfpathlineto{\pgfqpoint{2.111187in}{4.060991in}}%
\pgfpathlineto{\pgfqpoint{2.092174in}{4.122048in}}%
\pgfpathlineto{\pgfqpoint{2.073629in}{4.236674in}}%
\pgfpathlineto{\pgfqpoint{2.052973in}{4.388465in}}%
\pgfpathlineto{\pgfqpoint{2.033491in}{4.402256in}}%
\pgfpathlineto{\pgfqpoint{2.014478in}{4.281102in}}%
\pgfpathlineto{\pgfqpoint{1.993118in}{4.125800in}}%
\pgfpathlineto{\pgfqpoint{1.974105in}{4.066226in}}%
\pgfpathlineto{\pgfqpoint{1.956500in}{4.043996in}}%
\pgfpathlineto{\pgfqpoint{1.938190in}{4.037475in}}%
\pgfpathlineto{\pgfqpoint{1.919648in}{4.035100in}}%
\pgfpathlineto{\pgfqpoint{1.897349in}{4.037741in}}%
\pgfpathlineto{\pgfqpoint{1.882795in}{4.041247in}}%
\pgfpathlineto{\pgfqpoint{1.860965in}{4.038244in}}%
\pgfpathlineto{\pgfqpoint{1.843126in}{4.049695in}}%
\pgfpathlineto{\pgfqpoint{1.819887in}{4.077367in}}%
\pgfpathlineto{\pgfqpoint{1.801814in}{4.187577in}}%
\pgfpathlineto{\pgfqpoint{1.785852in}{4.316652in}}%
\pgfpathlineto{\pgfqpoint{1.764961in}{4.416499in}}%
\pgfpathlineto{\pgfqpoint{1.744774in}{4.350100in}}%
\pgfpathlineto{\pgfqpoint{1.725292in}{4.196338in}}%
\pgfpathlineto{\pgfqpoint{1.707687in}{4.106327in}}%
\pgfpathlineto{\pgfqpoint{1.690316in}{4.065797in}}%
\pgfpathlineto{\pgfqpoint{1.668721in}{4.044895in}}%
\pgfpathlineto{\pgfqpoint{1.650178in}{4.037064in}}%
\pgfpathlineto{\pgfqpoint{1.632573in}{4.035372in}}%
\pgfpathlineto{\pgfqpoint{1.612855in}{4.036784in}}%
\pgfpathlineto{\pgfqpoint{1.588444in}{4.047451in}}%
\pgfpathlineto{\pgfqpoint{1.574125in}{4.057688in}}%
\pgfpathlineto{\pgfqpoint{1.554174in}{4.106350in}}%
\pgfpathlineto{\pgfqpoint{1.529762in}{4.241791in}}%
\pgfpathlineto{\pgfqpoint{1.514270in}{4.339633in}}%
\pgfpathlineto{\pgfqpoint{1.495021in}{4.421802in}}%
\pgfpathlineto{\pgfqpoint{1.476244in}{4.412995in}}%
\pgfpathlineto{\pgfqpoint{1.457934in}{4.248728in}}%
\pgfpathlineto{\pgfqpoint{1.437983in}{4.140372in}}%
\pgfpathlineto{\pgfqpoint{1.419907in}{4.074963in}}%
\pgfpathlineto{\pgfqpoint{1.398548in}{4.047837in}}%
\pgfpathlineto{\pgfqpoint{1.380004in}{4.038602in}}%
\pgfpathlineto{\pgfqpoint{1.360287in}{4.035960in}}%
\pgfpathlineto{\pgfqpoint{1.343387in}{4.035904in}}%
\pgfpathlineto{\pgfqpoint{1.321323in}{4.037646in}}%
\pgfpathlineto{\pgfqpoint{1.303013in}{4.047091in}}%
\pgfpathlineto{\pgfqpoint{1.284939in}{4.074110in}}%
\pgfpathlineto{\pgfqpoint{1.265692in}{4.138602in}}%
\pgfpathlineto{\pgfqpoint{1.247147in}{4.279744in}}%
\pgfpathlineto{\pgfqpoint{1.226257in}{4.382219in}}%
\pgfpathlineto{\pgfqpoint{1.207243in}{4.429874in}}%
\pgfpathlineto{\pgfqpoint{1.188230in}{4.425413in}}%
\pgfpathlineto{\pgfqpoint{1.169922in}{4.275929in}}%
\pgfpathlineto{\pgfqpoint{1.144806in}{4.142211in}}%
\pgfpathlineto{\pgfqpoint{1.129313in}{4.123549in}}%
\pgfpathlineto{\pgfqpoint{1.111005in}{4.065759in}}%
\pgfpathlineto{\pgfqpoint{1.089644in}{4.044418in}}%
\pgfpathlineto{\pgfqpoint{1.072039in}{4.037979in}}%
\pgfpathlineto{\pgfqpoint{1.052791in}{4.036281in}}%
\pgfpathlineto{\pgfqpoint{1.034717in}{4.039648in}}%
\pgfpathlineto{\pgfqpoint{1.016173in}{4.051074in}}%
\pgfpathlineto{\pgfqpoint{0.996691in}{4.072592in}}%
\pgfpathlineto{\pgfqpoint{0.957492in}{4.245376in}}%
\pgfpathlineto{\pgfqpoint{0.938479in}{4.273547in}}%
\pgfpathlineto{\pgfqpoint{0.919935in}{4.408392in}}%
\pgfpathlineto{\pgfqpoint{0.899513in}{4.444531in}}%
\pgfpathlineto{\pgfqpoint{0.880970in}{4.378918in}}%
\pgfpathlineto{\pgfqpoint{0.861957in}{4.235099in}}%
\pgfpathlineto{\pgfqpoint{0.842475in}{4.103987in}}%
\pgfpathlineto{\pgfqpoint{0.823931in}{4.062926in}}%
\pgfpathlineto{\pgfqpoint{0.803274in}{4.046213in}}%
\pgfpathlineto{\pgfqpoint{0.787313in}{4.040660in}}%
\pgfpathlineto{\pgfqpoint{0.766188in}{4.036468in}}%
\pgfpathlineto{\pgfqpoint{0.744592in}{4.040793in}}%
\pgfpathlineto{\pgfqpoint{0.726518in}{4.270093in}}%
\pgfpathlineto{\pgfqpoint{0.708679in}{4.127173in}}%
\pgfpathlineto{\pgfqpoint{0.687318in}{4.060123in}}%
\pgfpathlineto{\pgfqpoint{0.671356in}{4.042898in}}%
\pgfpathlineto{\pgfqpoint{0.652108in}{4.036913in}}%
\pgfpathlineto{\pgfqpoint{0.657273in}{4.038423in}}%
\pgfpathlineto{\pgfqpoint{0.675815in}{4.053006in}}%
\pgfpathlineto{\pgfqpoint{0.694125in}{4.102571in}}%
\pgfpathlineto{\pgfqpoint{0.714311in}{4.331587in}}%
\pgfpathlineto{\pgfqpoint{0.735907in}{4.447600in}}%
\pgfpathlineto{\pgfqpoint{0.750694in}{4.363992in}}%
\pgfpathlineto{\pgfqpoint{0.768299in}{4.156045in}}%
\pgfpathlineto{\pgfqpoint{0.791069in}{4.058628in}}%
\pgfpathlineto{\pgfqpoint{0.809143in}{4.039519in}}%
\pgfpathlineto{\pgfqpoint{0.829094in}{4.036557in}}%
\pgfpathlineto{\pgfqpoint{0.849515in}{4.048240in}}%
\pgfpathlineto{\pgfqpoint{0.868294in}{4.097271in}}%
\pgfpathlineto{\pgfqpoint{0.889185in}{4.272206in}}%
\pgfpathlineto{\pgfqpoint{0.907495in}{4.441223in}}%
\pgfpathlineto{\pgfqpoint{0.924160in}{4.381471in}}%
\pgfpathlineto{\pgfqpoint{0.943642in}{4.174208in}}%
\pgfpathlineto{\pgfqpoint{0.963595in}{4.061819in}}%
\pgfpathlineto{\pgfqpoint{0.984486in}{4.040087in}}%
\pgfpathlineto{\pgfqpoint{1.001856in}{4.035784in}}%
\pgfpathlineto{\pgfqpoint{1.021338in}{4.041034in}}%
\pgfpathlineto{\pgfqpoint{1.041760in}{4.067323in}}%
\pgfpathlineto{\pgfqpoint{1.060068in}{4.162499in}}%
\pgfpathlineto{\pgfqpoint{1.078612in}{4.399134in}}%
\pgfpathlineto{\pgfqpoint{1.101380in}{4.409228in}}%
\pgfpathlineto{\pgfqpoint{1.118047in}{4.266299in}}%
\pgfpathlineto{\pgfqpoint{1.136590in}{4.093964in}}%
\pgfpathlineto{\pgfqpoint{1.156072in}{4.047408in}}%
\pgfpathlineto{\pgfqpoint{1.173677in}{4.039057in}}%
\pgfpathlineto{\pgfqpoint{1.195507in}{4.088075in}}%
\pgfpathlineto{\pgfqpoint{1.216397in}{4.048613in}}%
\pgfpathlineto{\pgfqpoint{1.232594in}{4.038074in}}%
\pgfpathlineto{\pgfqpoint{1.252312in}{4.035769in}}%
\pgfpathlineto{\pgfqpoint{1.271089in}{4.042001in}}%
\pgfpathlineto{\pgfqpoint{1.293390in}{4.078948in}}%
\pgfpathlineto{\pgfqpoint{1.309350in}{4.182022in}}%
\pgfpathlineto{\pgfqpoint{1.329303in}{4.081722in}}%
\pgfpathlineto{\pgfqpoint{1.347142in}{4.224297in}}%
\pgfpathlineto{\pgfqpoint{1.368033in}{4.422889in}}%
\pgfpathlineto{\pgfqpoint{1.386811in}{4.368058in}}%
\pgfpathlineto{\pgfqpoint{1.405825in}{4.201307in}}%
\pgfpathlineto{\pgfqpoint{1.427420in}{4.073142in}}%
\pgfpathlineto{\pgfqpoint{1.445494in}{4.041942in}}%
\pgfpathlineto{\pgfqpoint{1.464037in}{4.035532in}}%
\pgfpathlineto{\pgfqpoint{1.482347in}{4.037048in}}%
\pgfpathlineto{\pgfqpoint{1.502063in}{4.048968in}}%
\pgfpathlineto{\pgfqpoint{1.521076in}{4.077065in}}%
\pgfpathlineto{\pgfqpoint{1.539855in}{4.134282in}}%
\pgfpathlineto{\pgfqpoint{1.558868in}{4.365859in}}%
\pgfpathlineto{\pgfqpoint{1.580699in}{4.406602in}}%
\pgfpathlineto{\pgfqpoint{1.596659in}{4.294845in}}%
\pgfpathlineto{\pgfqpoint{1.615203in}{4.119517in}}%
\pgfpathlineto{\pgfqpoint{1.636799in}{4.054019in}}%
\pgfpathlineto{\pgfqpoint{1.655341in}{4.038358in}}%
\pgfpathlineto{\pgfqpoint{1.677406in}{4.035116in}}%
\pgfpathlineto{\pgfqpoint{1.693133in}{4.037410in}}%
\pgfpathlineto{\pgfqpoint{1.714729in}{4.053068in}}%
\pgfpathlineto{\pgfqpoint{1.733037in}{4.098879in}}%
\pgfpathlineto{\pgfqpoint{1.752285in}{4.272977in}}%
\pgfpathlineto{\pgfqpoint{1.770829in}{4.418313in}}%
\pgfpathlineto{\pgfqpoint{1.790077in}{4.366322in}}%
\pgfpathlineto{\pgfqpoint{1.811202in}{4.192040in}}%
\pgfpathlineto{\pgfqpoint{1.827398in}{4.103148in}}%
\pgfpathlineto{\pgfqpoint{1.849229in}{4.048462in}}%
\pgfpathlineto{\pgfqpoint{1.868476in}{4.037881in}}%
\pgfpathlineto{\pgfqpoint{1.886550in}{4.034813in}}%
\pgfpathlineto{\pgfqpoint{1.903920in}{4.035993in}}%
\pgfpathlineto{\pgfqpoint{1.926454in}{4.047646in}}%
\pgfpathlineto{\pgfqpoint{1.945232in}{4.078538in}}%
\pgfpathlineto{\pgfqpoint{1.961663in}{4.179086in}}%
\pgfpathlineto{\pgfqpoint{1.981616in}{4.390067in}}%
\pgfpathlineto{\pgfqpoint{1.999924in}{4.413401in}}%
\pgfpathlineto{\pgfqpoint{2.021051in}{4.287366in}}%
\pgfpathlineto{\pgfqpoint{2.039359in}{4.115599in}}%
\pgfpathlineto{\pgfqpoint{2.060250in}{4.051445in}}%
\pgfpathlineto{\pgfqpoint{2.078794in}{4.039100in}}%
\pgfpathlineto{\pgfqpoint{2.098980in}{4.035983in}}%
\pgfpathlineto{\pgfqpoint{2.116821in}{4.034857in}}%
\pgfpathlineto{\pgfqpoint{2.133720in}{4.037519in}}%
\pgfpathlineto{\pgfqpoint{2.154376in}{4.050029in}}%
\pgfpathlineto{\pgfqpoint{2.175267in}{4.084272in}}%
\pgfpathlineto{\pgfqpoint{2.195689in}{4.226955in}}%
\pgfpathlineto{\pgfqpoint{2.212825in}{4.377866in}}%
\pgfpathlineto{\pgfqpoint{2.232541in}{4.407556in}}%
\pgfpathlineto{\pgfqpoint{2.251554in}{4.279456in}}%
\pgfpathlineto{\pgfqpoint{2.269628in}{4.363298in}}%
\pgfpathlineto{\pgfqpoint{2.290284in}{4.299853in}}%
\pgfpathlineto{\pgfqpoint{2.308360in}{4.168341in}}%
\pgfpathlineto{\pgfqpoint{2.325728in}{4.066489in}}%
\pgfpathlineto{\pgfqpoint{2.347324in}{4.040672in}}%
\pgfpathlineto{\pgfqpoint{2.368215in}{4.035067in}}%
\pgfpathlineto{\pgfqpoint{2.388636in}{4.036422in}}%
\pgfpathlineto{\pgfqpoint{2.405067in}{4.041883in}}%
\pgfpathlineto{\pgfqpoint{2.424785in}{4.066185in}}%
\pgfpathlineto{\pgfqpoint{2.442390in}{4.147264in}}%
\pgfpathlineto{\pgfqpoint{2.461167in}{4.350260in}}%
\pgfpathlineto{\pgfqpoint{2.480886in}{4.410964in}}%
\pgfpathlineto{\pgfqpoint{2.502245in}{4.321674in}}%
\pgfpathlineto{\pgfqpoint{2.520084in}{4.159609in}}%
\pgfpathlineto{\pgfqpoint{2.536986in}{4.082310in}}%
\pgfpathlineto{\pgfqpoint{2.555999in}{4.045225in}}%
\pgfpathlineto{\pgfqpoint{2.580175in}{4.036333in}}%
\pgfpathlineto{\pgfqpoint{2.598015in}{4.034888in}}%
\pgfpathlineto{\pgfqpoint{2.615619in}{4.037983in}}%
\pgfpathlineto{\pgfqpoint{2.636510in}{4.047934in}}%
\pgfpathlineto{\pgfqpoint{2.654349in}{4.064497in}}%
\pgfpathlineto{\pgfqpoint{2.676414in}{4.113214in}}%
\pgfpathlineto{\pgfqpoint{2.693784in}{4.259824in}}%
\pgfpathlineto{\pgfqpoint{2.712329in}{4.403359in}}%
\pgfpathlineto{\pgfqpoint{2.729699in}{4.387634in}}%
\pgfpathlineto{\pgfqpoint{2.771480in}{4.086671in}}%
\pgfpathlineto{\pgfqpoint{2.790728in}{4.050318in}}%
\pgfpathlineto{\pgfqpoint{2.807393in}{4.039136in}}%
\pgfpathlineto{\pgfqpoint{2.828520in}{4.035051in}}%
\pgfpathlineto{\pgfqpoint{2.846125in}{4.035181in}}%
\pgfpathlineto{\pgfqpoint{2.868424in}{4.041226in}}%
\pgfpathlineto{\pgfqpoint{2.886263in}{4.055080in}}%
\pgfpathlineto{\pgfqpoint{2.903868in}{4.085607in}}%
\pgfpathlineto{\pgfqpoint{2.921707in}{4.188191in}}%
\pgfpathlineto{\pgfqpoint{2.942598in}{4.389293in}}%
\pgfpathlineto{\pgfqpoint{2.960671in}{4.404967in}}%
\pgfpathlineto{\pgfqpoint{2.981327in}{4.333129in}}%
\pgfpathlineto{\pgfqpoint{2.999403in}{4.165249in}}%
\pgfpathlineto{\pgfqpoint{3.017711in}{4.077583in}}%
\pgfpathlineto{\pgfqpoint{3.038602in}{4.046227in}}%
\pgfpathlineto{\pgfqpoint{3.057849in}{4.038926in}}%
\pgfpathlineto{\pgfqpoint{3.077568in}{4.035093in}}%
\pgfpathlineto{\pgfqpoint{3.101978in}{4.037112in}}%
\pgfpathlineto{\pgfqpoint{3.117940in}{4.041401in}}%
\pgfpathlineto{\pgfqpoint{3.135076in}{4.051889in}}%
\pgfpathlineto{\pgfqpoint{3.155027in}{4.090417in}}%
\pgfpathlineto{\pgfqpoint{3.174040in}{4.173336in}}%
\pgfpathlineto{\pgfqpoint{3.192350in}{4.275168in}}%
\pgfpathlineto{\pgfqpoint{3.212536in}{4.415933in}}%
\pgfpathlineto{\pgfqpoint{3.231080in}{4.377692in}}%
\pgfpathlineto{\pgfqpoint{3.253379in}{4.253385in}}%
\pgfpathlineto{\pgfqpoint{3.268872in}{4.112615in}}%
\pgfpathlineto{\pgfqpoint{3.287649in}{4.060771in}}%
\pgfpathlineto{\pgfqpoint{3.309245in}{4.043326in}}%
\pgfpathlineto{\pgfqpoint{3.326381in}{4.139637in}}%
\pgfpathlineto{\pgfqpoint{3.345392in}{4.063060in}}%
\pgfpathlineto{\pgfqpoint{3.366988in}{4.042467in}}%
\pgfpathlineto{\pgfqpoint{3.384358in}{4.036392in}}%
\pgfpathlineto{\pgfqpoint{3.402901in}{4.035584in}}%
\pgfpathlineto{\pgfqpoint{3.424028in}{4.041685in}}%
\pgfpathlineto{\pgfqpoint{3.444215in}{4.055598in}}%
\pgfpathlineto{\pgfqpoint{3.463463in}{4.100083in}}%
\pgfpathlineto{\pgfqpoint{3.480831in}{4.209476in}}%
\pgfpathlineto{\pgfqpoint{3.501253in}{4.412421in}}%
\pgfpathlineto{\pgfqpoint{3.519797in}{4.412396in}}%
\pgfpathlineto{\pgfqpoint{3.539985in}{4.297677in}}%
\pgfpathlineto{\pgfqpoint{3.558762in}{4.166618in}}%
\pgfpathlineto{\pgfqpoint{3.576601in}{4.086092in}}%
\pgfpathlineto{\pgfqpoint{3.594440in}{4.051510in}}%
\pgfpathlineto{\pgfqpoint{3.616270in}{4.038667in}}%
\pgfpathlineto{\pgfqpoint{3.633875in}{4.035493in}}%
\pgfpathlineto{\pgfqpoint{3.654766in}{4.037054in}}%
\pgfpathlineto{\pgfqpoint{3.673076in}{4.041443in}}%
\pgfpathlineto{\pgfqpoint{3.692558in}{4.052967in}}%
\pgfpathlineto{\pgfqpoint{3.713448in}{4.084291in}}%
\pgfpathlineto{\pgfqpoint{3.730350in}{4.179077in}}%
\pgfpathlineto{\pgfqpoint{3.751006in}{4.366135in}}%
\pgfpathlineto{\pgfqpoint{3.768376in}{4.431275in}}%
\pgfpathlineto{\pgfqpoint{3.786919in}{4.414071in}}%
\pgfpathlineto{\pgfqpoint{3.807106in}{4.330214in}}%
\pgfpathlineto{\pgfqpoint{3.825414in}{4.198220in}}%
\pgfpathlineto{\pgfqpoint{3.847244in}{4.089936in}}%
\pgfpathlineto{\pgfqpoint{3.865318in}{4.059543in}}%
\pgfpathlineto{\pgfqpoint{3.885036in}{4.044510in}}%
\pgfpathlineto{\pgfqpoint{3.904519in}{4.037878in}}%
\pgfpathlineto{\pgfqpoint{3.923532in}{4.036204in}}%
\pgfpathlineto{\pgfqpoint{3.942074in}{4.039056in}}%
\pgfpathlineto{\pgfqpoint{3.959915in}{4.049992in}}%
\pgfpathlineto{\pgfqpoint{3.981980in}{4.074837in}}%
\pgfpathlineto{\pgfqpoint{3.998880in}{4.064657in}}%
\pgfpathlineto{\pgfqpoint{4.020241in}{4.128723in}}%
\pgfpathlineto{\pgfqpoint{4.037844in}{4.285576in}}%
\pgfpathlineto{\pgfqpoint{4.062257in}{4.440866in}}%
\pgfpathlineto{\pgfqpoint{4.074933in}{4.439038in}}%
\pgfpathlineto{\pgfqpoint{4.095589in}{4.350045in}}%
\pgfpathlineto{\pgfqpoint{4.116011in}{4.190929in}}%
\pgfpathlineto{\pgfqpoint{4.132207in}{4.101086in}}%
\pgfpathlineto{\pgfqpoint{4.152863in}{4.057377in}}%
\pgfpathlineto{\pgfqpoint{4.170937in}{4.042685in}}%
\pgfpathlineto{\pgfqpoint{4.192062in}{4.037630in}}%
\pgfpathlineto{\pgfqpoint{4.213423in}{4.037551in}}%
\pgfpathlineto{\pgfqpoint{4.227976in}{4.040330in}}%
\pgfpathlineto{\pgfqpoint{4.250744in}{4.052521in}}%
\pgfpathlineto{\pgfqpoint{4.269289in}{4.075273in}}%
\pgfpathlineto{\pgfqpoint{4.288536in}{4.133986in}}%
\pgfpathlineto{\pgfqpoint{4.306376in}{4.297941in}}%
\pgfpathlineto{\pgfqpoint{4.327969in}{4.439285in}}%
\pgfpathlineto{\pgfqpoint{4.346748in}{4.456856in}}%
\pgfpathlineto{\pgfqpoint{4.363650in}{4.422296in}}%
\pgfpathlineto{\pgfqpoint{4.384540in}{4.286796in}}%
\pgfpathlineto{\pgfqpoint{4.402380in}{4.155572in}}%
\pgfpathlineto{\pgfqpoint{4.421393in}{4.326503in}}%
\pgfpathlineto{\pgfqpoint{4.461062in}{4.102648in}}%
\pgfpathlineto{\pgfqpoint{4.481484in}{4.053534in}}%
\pgfpathlineto{\pgfqpoint{4.471859in}{4.069909in}}%
\pgfpathlineto{\pgfqpoint{4.454254in}{4.162280in}}%
\pgfpathlineto{\pgfqpoint{4.435946in}{4.360161in}}%
\pgfpathlineto{\pgfqpoint{4.417402in}{4.454330in}}%
\pgfpathlineto{\pgfqpoint{4.396980in}{4.408236in}}%
\pgfpathlineto{\pgfqpoint{4.379846in}{4.162540in}}%
\pgfpathlineto{\pgfqpoint{4.358954in}{4.062710in}}%
\pgfpathlineto{\pgfqpoint{4.338768in}{4.039889in}}%
\pgfpathlineto{\pgfqpoint{4.320929in}{4.036619in}}%
\pgfpathlineto{\pgfqpoint{4.303793in}{4.043456in}}%
\pgfpathlineto{\pgfqpoint{4.281494in}{4.089825in}}%
\pgfpathlineto{\pgfqpoint{4.263655in}{4.220916in}}%
\pgfpathlineto{\pgfqpoint{4.244407in}{4.417061in}}%
\pgfpathlineto{\pgfqpoint{4.224220in}{4.439120in}}%
\pgfpathlineto{\pgfqpoint{4.205441in}{4.233378in}}%
\pgfpathlineto{\pgfqpoint{4.185256in}{4.081430in}}%
\pgfpathlineto{\pgfqpoint{4.167651in}{4.047113in}}%
\pgfpathlineto{\pgfqpoint{4.147698in}{4.036275in}}%
\pgfpathlineto{\pgfqpoint{4.129624in}{4.038108in}}%
\pgfpathlineto{\pgfqpoint{4.106151in}{4.062576in}}%
\pgfpathlineto{\pgfqpoint{4.089250in}{4.142435in}}%
\pgfpathlineto{\pgfqpoint{4.071645in}{4.321265in}}%
\pgfpathlineto{\pgfqpoint{4.050754in}{4.438485in}}%
\pgfpathlineto{\pgfqpoint{4.030569in}{4.346033in}}%
\pgfpathlineto{\pgfqpoint{4.013433in}{4.124505in}}%
\pgfpathlineto{\pgfqpoint{3.993951in}{4.053764in}}%
\pgfpathlineto{\pgfqpoint{3.976815in}{4.040550in}}%
\pgfpathlineto{\pgfqpoint{3.954516in}{4.035412in}}%
\pgfpathlineto{\pgfqpoint{3.937380in}{4.039547in}}%
\pgfpathlineto{\pgfqpoint{3.916255in}{4.065241in}}%
\pgfpathlineto{\pgfqpoint{3.899119in}{4.156498in}}%
\pgfpathlineto{\pgfqpoint{3.878463in}{4.366193in}}%
\pgfpathlineto{\pgfqpoint{3.860858in}{4.432717in}}%
\pgfpathlineto{\pgfqpoint{3.839733in}{4.271470in}}%
\pgfpathlineto{\pgfqpoint{3.822128in}{4.120280in}}%
\pgfpathlineto{\pgfqpoint{3.801472in}{4.056778in}}%
\pgfpathlineto{\pgfqpoint{3.778939in}{4.037807in}}%
\pgfpathlineto{\pgfqpoint{3.763680in}{4.035197in}}%
\pgfpathlineto{\pgfqpoint{3.743495in}{4.039335in}}%
\pgfpathlineto{\pgfqpoint{3.725656in}{4.054391in}}%
\pgfpathlineto{\pgfqpoint{3.704763in}{4.136076in}}%
\pgfpathlineto{\pgfqpoint{3.684578in}{4.349209in}}%
\pgfpathlineto{\pgfqpoint{3.669085in}{4.411712in}}%
\pgfpathlineto{\pgfqpoint{3.646317in}{4.369028in}}%
\pgfpathlineto{\pgfqpoint{3.627772in}{4.139203in}}%
\pgfpathlineto{\pgfqpoint{3.607351in}{4.069618in}}%
\pgfpathlineto{\pgfqpoint{3.589511in}{4.043603in}}%
\pgfpathlineto{\pgfqpoint{3.572141in}{4.036844in}}%
\pgfpathlineto{\pgfqpoint{3.551016in}{4.035724in}}%
\pgfpathlineto{\pgfqpoint{3.533177in}{4.042834in}}%
\pgfpathlineto{\pgfqpoint{3.512052in}{4.064918in}}%
\pgfpathlineto{\pgfqpoint{3.494681in}{4.147271in}}%
\pgfpathlineto{\pgfqpoint{3.456186in}{4.391426in}}%
\pgfpathlineto{\pgfqpoint{3.435059in}{4.413486in}}%
\pgfpathlineto{\pgfqpoint{3.416751in}{4.196152in}}%
\pgfpathlineto{\pgfqpoint{3.396798in}{4.081554in}}%
\pgfpathlineto{\pgfqpoint{3.378256in}{4.224194in}}%
\pgfpathlineto{\pgfqpoint{3.358068in}{4.393356in}}%
\pgfpathlineto{\pgfqpoint{3.342812in}{4.408362in}}%
\pgfpathlineto{\pgfqpoint{3.317461in}{4.163551in}}%
\pgfpathlineto{\pgfqpoint{3.302672in}{4.072732in}}%
\pgfpathlineto{\pgfqpoint{3.283424in}{4.044098in}}%
\pgfpathlineto{\pgfqpoint{3.258544in}{4.035294in}}%
\pgfpathlineto{\pgfqpoint{3.240234in}{4.036090in}}%
\pgfpathlineto{\pgfqpoint{3.224978in}{4.043390in}}%
\pgfpathlineto{\pgfqpoint{3.206668in}{4.070407in}}%
\pgfpathlineto{\pgfqpoint{3.186717in}{4.193947in}}%
\pgfpathlineto{\pgfqpoint{3.167469in}{4.372350in}}%
\pgfpathlineto{\pgfqpoint{3.146578in}{4.417474in}}%
\pgfpathlineto{\pgfqpoint{3.129208in}{4.254025in}}%
\pgfpathlineto{\pgfqpoint{3.109021in}{4.097038in}}%
\pgfpathlineto{\pgfqpoint{3.090007in}{4.051115in}}%
\pgfpathlineto{\pgfqpoint{3.071699in}{4.037623in}}%
\pgfpathlineto{\pgfqpoint{3.053860in}{4.035164in}}%
\pgfpathlineto{\pgfqpoint{3.034376in}{4.036700in}}%
\pgfpathlineto{\pgfqpoint{3.012782in}{4.049108in}}%
\pgfpathlineto{\pgfqpoint{2.993535in}{4.083728in}}%
\pgfpathlineto{\pgfqpoint{2.975930in}{4.211289in}}%
\pgfpathlineto{\pgfqpoint{2.957385in}{4.373515in}}%
\pgfpathlineto{\pgfqpoint{2.938607in}{4.418450in}}%
\pgfpathlineto{\pgfqpoint{2.916778in}{4.342730in}}%
\pgfpathlineto{\pgfqpoint{2.897529in}{4.119460in}}%
\pgfpathlineto{\pgfqpoint{2.879926in}{4.060372in}}%
\pgfpathlineto{\pgfqpoint{2.861381in}{4.040948in}}%
\pgfpathlineto{\pgfqpoint{2.839082in}{4.034997in}}%
\pgfpathlineto{\pgfqpoint{2.822181in}{4.036183in}}%
\pgfpathlineto{\pgfqpoint{2.798239in}{4.048369in}}%
\pgfpathlineto{\pgfqpoint{2.784625in}{4.066602in}}%
\pgfpathlineto{\pgfqpoint{2.764203in}{4.130617in}}%
\pgfpathlineto{\pgfqpoint{2.746130in}{4.290204in}}%
\pgfpathlineto{\pgfqpoint{2.723360in}{4.413961in}}%
\pgfpathlineto{\pgfqpoint{2.707869in}{4.379124in}}%
\pgfpathlineto{\pgfqpoint{2.686507in}{4.156667in}}%
\pgfpathlineto{\pgfqpoint{2.667965in}{4.067365in}}%
\pgfpathlineto{\pgfqpoint{2.646135in}{4.041230in}}%
\pgfpathlineto{\pgfqpoint{2.628530in}{4.035494in}}%
\pgfpathlineto{\pgfqpoint{2.609048in}{4.035374in}}%
\pgfpathlineto{\pgfqpoint{2.590269in}{4.040533in}}%
\pgfpathlineto{\pgfqpoint{2.571959in}{4.060997in}}%
\pgfpathlineto{\pgfqpoint{2.549660in}{4.155995in}}%
\pgfpathlineto{\pgfqpoint{2.534872in}{4.278501in}}%
\pgfpathlineto{\pgfqpoint{2.516095in}{4.340225in}}%
\pgfpathlineto{\pgfqpoint{2.496846in}{4.409463in}}%
\pgfpathlineto{\pgfqpoint{2.475486in}{4.332984in}}%
\pgfpathlineto{\pgfqpoint{2.455770in}{4.145860in}}%
\pgfpathlineto{\pgfqpoint{2.436991in}{4.067802in}}%
\pgfpathlineto{\pgfqpoint{2.419855in}{4.042637in}}%
\pgfpathlineto{\pgfqpoint{2.398025in}{4.035544in}}%
\pgfpathlineto{\pgfqpoint{2.379717in}{4.035496in}}%
\pgfpathlineto{\pgfqpoint{2.359764in}{4.037574in}}%
\pgfpathlineto{\pgfqpoint{2.340752in}{4.046759in}}%
\pgfpathlineto{\pgfqpoint{2.322911in}{4.079970in}}%
\pgfpathlineto{\pgfqpoint{2.284887in}{4.327703in}}%
\pgfpathlineto{\pgfqpoint{2.263291in}{4.408796in}}%
\pgfpathlineto{\pgfqpoint{2.245686in}{4.341561in}}%
\pgfpathlineto{\pgfqpoint{2.225970in}{4.146790in}}%
\pgfpathlineto{\pgfqpoint{2.203669in}{4.065272in}}%
\pgfpathlineto{\pgfqpoint{2.187238in}{4.045867in}}%
\pgfpathlineto{\pgfqpoint{2.168461in}{4.037365in}}%
\pgfpathlineto{\pgfqpoint{2.150385in}{4.034880in}}%
\pgfpathlineto{\pgfqpoint{2.126912in}{4.038311in}}%
\pgfpathlineto{\pgfqpoint{2.108839in}{4.047889in}}%
\pgfpathlineto{\pgfqpoint{2.088651in}{4.085336in}}%
\pgfpathlineto{\pgfqpoint{2.072691in}{4.091371in}}%
\pgfpathlineto{\pgfqpoint{2.034899in}{4.346229in}}%
\pgfpathlineto{\pgfqpoint{2.016589in}{4.409935in}}%
\pgfpathlineto{\pgfqpoint{1.993587in}{4.379299in}}%
\pgfpathlineto{\pgfqpoint{1.975982in}{4.190139in}}%
\pgfpathlineto{\pgfqpoint{1.957672in}{4.084753in}}%
\pgfpathlineto{\pgfqpoint{1.938661in}{4.048131in}}%
\pgfpathlineto{\pgfqpoint{1.916596in}{4.038954in}}%
\pgfpathlineto{\pgfqpoint{1.898757in}{4.035125in}}%
\pgfpathlineto{\pgfqpoint{1.879744in}{4.035859in}}%
\pgfpathlineto{\pgfqpoint{1.861434in}{4.040350in}}%
\pgfpathlineto{\pgfqpoint{1.839604in}{4.057679in}}%
\pgfpathlineto{\pgfqpoint{1.820356in}{4.116016in}}%
\pgfpathlineto{\pgfqpoint{1.805099in}{4.241694in}}%
\pgfpathlineto{\pgfqpoint{1.783974in}{4.373685in}}%
\pgfpathlineto{\pgfqpoint{1.764961in}{4.415739in}}%
\pgfpathlineto{\pgfqpoint{1.746885in}{4.394747in}}%
\pgfpathlineto{\pgfqpoint{1.724586in}{4.178655in}}%
\pgfpathlineto{\pgfqpoint{1.707687in}{4.083473in}}%
\pgfpathlineto{\pgfqpoint{1.688674in}{4.055438in}}%
\pgfpathlineto{\pgfqpoint{1.669895in}{4.044011in}}%
\pgfpathlineto{\pgfqpoint{1.650881in}{4.036731in}}%
\pgfpathlineto{\pgfqpoint{1.629051in}{4.035406in}}%
\pgfpathlineto{\pgfqpoint{1.611212in}{4.038426in}}%
\pgfpathlineto{\pgfqpoint{1.593138in}{4.048842in}}%
\pgfpathlineto{\pgfqpoint{1.574125in}{4.079748in}}%
\pgfpathlineto{\pgfqpoint{1.552766in}{4.123571in}}%
\pgfpathlineto{\pgfqpoint{1.535161in}{4.239134in}}%
\pgfpathlineto{\pgfqpoint{1.515677in}{4.378335in}}%
\pgfpathlineto{\pgfqpoint{1.491501in}{4.422130in}}%
\pgfpathlineto{\pgfqpoint{1.476713in}{4.386608in}}%
\pgfpathlineto{\pgfqpoint{1.459577in}{4.035803in}}%
\pgfpathlineto{\pgfqpoint{1.440800in}{4.036235in}}%
\pgfpathlineto{\pgfqpoint{1.419673in}{4.047157in}}%
\pgfpathlineto{\pgfqpoint{1.397843in}{4.097016in}}%
\pgfpathlineto{\pgfqpoint{1.379769in}{4.195739in}}%
\pgfpathlineto{\pgfqpoint{1.361227in}{4.365453in}}%
\pgfpathlineto{\pgfqpoint{1.341979in}{4.429854in}}%
\pgfpathlineto{\pgfqpoint{1.321086in}{4.355095in}}%
\pgfpathlineto{\pgfqpoint{1.304187in}{4.139229in}}%
\pgfpathlineto{\pgfqpoint{1.287051in}{4.071643in}}%
\pgfpathlineto{\pgfqpoint{1.265692in}{4.043937in}}%
\pgfpathlineto{\pgfqpoint{1.244096in}{4.036270in}}%
\pgfpathlineto{\pgfqpoint{1.225082in}{4.036120in}}%
\pgfpathlineto{\pgfqpoint{1.209826in}{4.039030in}}%
\pgfpathlineto{\pgfqpoint{1.188230in}{4.052798in}}%
\pgfpathlineto{\pgfqpoint{1.169688in}{4.099580in}}%
\pgfpathlineto{\pgfqpoint{1.150674in}{4.232172in}}%
\pgfpathlineto{\pgfqpoint{1.132835in}{4.334965in}}%
\pgfpathlineto{\pgfqpoint{1.110300in}{4.434755in}}%
\pgfpathlineto{\pgfqpoint{1.091757in}{4.405809in}}%
\pgfpathlineto{\pgfqpoint{1.076030in}{4.204738in}}%
\pgfpathlineto{\pgfqpoint{1.055608in}{4.086135in}}%
\pgfpathlineto{\pgfqpoint{1.034014in}{4.049642in}}%
\pgfpathlineto{\pgfqpoint{1.015939in}{4.038890in}}%
\pgfpathlineto{\pgfqpoint{0.997160in}{4.035853in}}%
\pgfpathlineto{\pgfqpoint{0.976035in}{4.040111in}}%
\pgfpathlineto{\pgfqpoint{0.958664in}{4.050633in}}%
\pgfpathlineto{\pgfqpoint{0.939887in}{4.081640in}}%
\pgfpathlineto{\pgfqpoint{0.902095in}{4.304045in}}%
\pgfpathlineto{\pgfqpoint{0.880031in}{4.424118in}}%
\pgfpathlineto{\pgfqpoint{0.862426in}{4.447387in}}%
\pgfpathlineto{\pgfqpoint{0.842239in}{4.416639in}}%
\pgfpathlineto{\pgfqpoint{0.823931in}{4.214459in}}%
\pgfpathlineto{\pgfqpoint{0.805621in}{4.149151in}}%
\pgfpathlineto{\pgfqpoint{0.784261in}{4.066837in}}%
\pgfpathlineto{\pgfqpoint{0.765717in}{4.052484in}}%
\pgfpathlineto{\pgfqpoint{0.747174in}{4.041726in}}%
\pgfpathlineto{\pgfqpoint{0.729335in}{4.037523in}}%
\pgfpathlineto{\pgfqpoint{0.707270in}{4.038139in}}%
\pgfpathlineto{\pgfqpoint{0.689431in}{4.047661in}}%
\pgfpathlineto{\pgfqpoint{0.668070in}{4.068966in}}%
\pgfpathlineto{\pgfqpoint{0.650231in}{4.125431in}}%
\pgfpathlineto{\pgfqpoint{0.650700in}{4.120766in}}%
\pgfpathlineto{\pgfqpoint{0.657507in}{4.079324in}}%
\pgfpathlineto{\pgfqpoint{0.676755in}{4.044931in}}%
\pgfpathlineto{\pgfqpoint{0.696003in}{4.036507in}}%
\pgfpathlineto{\pgfqpoint{0.714076in}{4.039464in}}%
\pgfpathlineto{\pgfqpoint{0.731681in}{4.056668in}}%
\pgfpathlineto{\pgfqpoint{0.751165in}{4.122815in}}%
\pgfpathlineto{\pgfqpoint{0.773228in}{4.396430in}}%
\pgfpathlineto{\pgfqpoint{0.792007in}{4.444478in}}%
\pgfpathlineto{\pgfqpoint{0.811725in}{4.323767in}}%
\pgfpathlineto{\pgfqpoint{0.830738in}{4.131233in}}%
\pgfpathlineto{\pgfqpoint{0.850221in}{4.055970in}}%
\pgfpathlineto{\pgfqpoint{0.868529in}{4.035971in}}%
\pgfpathlineto{\pgfqpoint{0.888247in}{4.041590in}}%
\pgfpathlineto{\pgfqpoint{0.905381in}{4.060243in}}%
\pgfpathlineto{\pgfqpoint{0.926272in}{4.158980in}}%
\pgfpathlineto{\pgfqpoint{0.945756in}{4.409806in}}%
\pgfpathlineto{\pgfqpoint{0.967115in}{4.413599in}}%
\pgfpathlineto{\pgfqpoint{1.003264in}{4.102815in}}%
\pgfpathlineto{\pgfqpoint{1.022746in}{4.046856in}}%
\pgfpathlineto{\pgfqpoint{1.039646in}{4.036660in}}%
\pgfpathlineto{\pgfqpoint{1.058894in}{4.036776in}}%
\pgfpathlineto{\pgfqpoint{1.080958in}{4.050276in}}%
\pgfpathlineto{\pgfqpoint{1.097155in}{4.083366in}}%
\pgfpathlineto{\pgfqpoint{1.122036in}{4.339184in}}%
\pgfpathlineto{\pgfqpoint{1.138233in}{4.432806in}}%
\pgfpathlineto{\pgfqpoint{1.157246in}{4.364011in}}%
\pgfpathlineto{\pgfqpoint{1.176964in}{4.164860in}}%
\pgfpathlineto{\pgfqpoint{1.195272in}{4.062101in}}%
\pgfpathlineto{\pgfqpoint{1.214989in}{4.039747in}}%
\pgfpathlineto{\pgfqpoint{1.233533in}{4.035229in}}%
\pgfpathlineto{\pgfqpoint{1.254189in}{4.038762in}}%
\pgfpathlineto{\pgfqpoint{1.271560in}{4.053081in}}%
\pgfpathlineto{\pgfqpoint{1.291042in}{4.110525in}}%
\pgfpathlineto{\pgfqpoint{1.310055in}{4.347791in}}%
\pgfpathlineto{\pgfqpoint{1.328598in}{4.417602in}}%
\pgfpathlineto{\pgfqpoint{1.347847in}{4.409608in}}%
\pgfpathlineto{\pgfqpoint{1.368972in}{4.243713in}}%
\pgfpathlineto{\pgfqpoint{1.387751in}{4.087132in}}%
\pgfpathlineto{\pgfqpoint{1.405825in}{4.046501in}}%
\pgfpathlineto{\pgfqpoint{1.425541in}{4.036090in}}%
\pgfpathlineto{\pgfqpoint{1.446197in}{4.035422in}}%
\pgfpathlineto{\pgfqpoint{1.463568in}{4.039639in}}%
\pgfpathlineto{\pgfqpoint{1.483286in}{4.056472in}}%
\pgfpathlineto{\pgfqpoint{1.500420in}{4.119542in}}%
\pgfpathlineto{\pgfqpoint{1.523424in}{4.388132in}}%
\pgfpathlineto{\pgfqpoint{1.541733in}{4.419883in}}%
\pgfpathlineto{\pgfqpoint{1.560277in}{4.330878in}}%
\pgfpathlineto{\pgfqpoint{1.576239in}{4.161337in}}%
\pgfpathlineto{\pgfqpoint{1.598538in}{4.061878in}}%
\pgfpathlineto{\pgfqpoint{1.616612in}{4.041530in}}%
\pgfpathlineto{\pgfqpoint{1.637502in}{4.034898in}}%
\pgfpathlineto{\pgfqpoint{1.656750in}{4.037146in}}%
\pgfpathlineto{\pgfqpoint{1.673886in}{4.042735in}}%
\pgfpathlineto{\pgfqpoint{1.693837in}{4.064140in}}%
\pgfpathlineto{\pgfqpoint{1.713555in}{4.139424in}}%
\pgfpathlineto{\pgfqpoint{1.732334in}{4.360784in}}%
\pgfpathlineto{\pgfqpoint{1.756276in}{4.400124in}}%
\pgfpathlineto{\pgfqpoint{1.770829in}{4.322500in}}%
\pgfpathlineto{\pgfqpoint{1.789137in}{4.167612in}}%
\pgfpathlineto{\pgfqpoint{1.807211in}{4.073804in}}%
\pgfpathlineto{\pgfqpoint{1.828572in}{4.052605in}}%
\pgfpathlineto{\pgfqpoint{1.846646in}{4.040160in}}%
\pgfpathlineto{\pgfqpoint{1.867302in}{4.035007in}}%
\pgfpathlineto{\pgfqpoint{1.888898in}{4.036100in}}%
\pgfpathlineto{\pgfqpoint{1.906268in}{4.045488in}}%
\pgfpathlineto{\pgfqpoint{1.924342in}{4.065986in}}%
\pgfpathlineto{\pgfqpoint{1.944998in}{4.034739in}}%
\pgfpathlineto{\pgfqpoint{1.962603in}{4.037518in}}%
\pgfpathlineto{\pgfqpoint{1.981616in}{4.048630in}}%
\pgfpathlineto{\pgfqpoint{2.001567in}{4.084956in}}%
\pgfpathlineto{\pgfqpoint{2.019643in}{4.195301in}}%
\pgfpathlineto{\pgfqpoint{2.039828in}{4.412458in}}%
\pgfpathlineto{\pgfqpoint{2.059076in}{4.374145in}}%
\pgfpathlineto{\pgfqpoint{2.079497in}{4.178703in}}%
\pgfpathlineto{\pgfqpoint{2.097807in}{4.079209in}}%
\pgfpathlineto{\pgfqpoint{2.116115in}{4.050707in}}%
\pgfpathlineto{\pgfqpoint{2.139823in}{4.036748in}}%
\pgfpathlineto{\pgfqpoint{2.154845in}{4.034767in}}%
\pgfpathlineto{\pgfqpoint{2.176675in}{4.038602in}}%
\pgfpathlineto{\pgfqpoint{2.193341in}{4.050628in}}%
\pgfpathlineto{\pgfqpoint{2.213293in}{4.104441in}}%
\pgfpathlineto{\pgfqpoint{2.232307in}{4.263534in}}%
\pgfpathlineto{\pgfqpoint{2.250849in}{4.408275in}}%
\pgfpathlineto{\pgfqpoint{2.273150in}{4.389059in}}%
\pgfpathlineto{\pgfqpoint{2.288641in}{4.285628in}}%
\pgfpathlineto{\pgfqpoint{2.311646in}{4.111096in}}%
\pgfpathlineto{\pgfqpoint{2.328076in}{4.053403in}}%
\pgfpathlineto{\pgfqpoint{2.346150in}{4.038494in}}%
\pgfpathlineto{\pgfqpoint{2.365163in}{4.035182in}}%
\pgfpathlineto{\pgfqpoint{2.385819in}{4.036119in}}%
\pgfpathlineto{\pgfqpoint{2.405301in}{4.043437in}}%
\pgfpathlineto{\pgfqpoint{2.423611in}{4.066776in}}%
\pgfpathlineto{\pgfqpoint{2.443797in}{4.155909in}}%
\pgfpathlineto{\pgfqpoint{2.463281in}{4.299337in}}%
\pgfpathlineto{\pgfqpoint{2.480649in}{4.409104in}}%
\pgfpathlineto{\pgfqpoint{2.505767in}{4.345299in}}%
\pgfpathlineto{\pgfqpoint{2.518910in}{4.210612in}}%
\pgfpathlineto{\pgfqpoint{2.541446in}{4.073238in}}%
\pgfpathlineto{\pgfqpoint{2.556468in}{4.046618in}}%
\pgfpathlineto{\pgfqpoint{2.578767in}{4.036675in}}%
\pgfpathlineto{\pgfqpoint{2.597077in}{4.034611in}}%
\pgfpathlineto{\pgfqpoint{2.618202in}{4.036346in}}%
\pgfpathlineto{\pgfqpoint{2.637450in}{4.044682in}}%
\pgfpathlineto{\pgfqpoint{2.656228in}{4.058941in}}%
\pgfpathlineto{\pgfqpoint{2.674771in}{4.107318in}}%
\pgfpathlineto{\pgfqpoint{2.713503in}{4.407378in}}%
\pgfpathlineto{\pgfqpoint{2.731106in}{4.404660in}}%
\pgfpathlineto{\pgfqpoint{2.751998in}{4.313470in}}%
\pgfpathlineto{\pgfqpoint{2.771246in}{4.135915in}}%
\pgfpathlineto{\pgfqpoint{2.788850in}{4.064577in}}%
\pgfpathlineto{\pgfqpoint{2.809507in}{4.043919in}}%
\pgfpathlineto{\pgfqpoint{2.826877in}{4.036703in}}%
\pgfpathlineto{\pgfqpoint{2.848002in}{4.034810in}}%
\pgfpathlineto{\pgfqpoint{2.866076in}{4.037999in}}%
\pgfpathlineto{\pgfqpoint{2.883680in}{4.045042in}}%
\pgfpathlineto{\pgfqpoint{2.902225in}{4.065993in}}%
\pgfpathlineto{\pgfqpoint{2.923350in}{4.131819in}}%
\pgfpathlineto{\pgfqpoint{2.944946in}{4.318440in}}%
\pgfpathlineto{\pgfqpoint{2.962080in}{4.410051in}}%
\pgfpathlineto{\pgfqpoint{2.980390in}{4.383994in}}%
\pgfpathlineto{\pgfqpoint{3.001749in}{4.209446in}}%
\pgfpathlineto{\pgfqpoint{3.020997in}{4.086510in}}%
\pgfpathlineto{\pgfqpoint{3.037427in}{4.050746in}}%
\pgfpathlineto{\pgfqpoint{3.058320in}{4.038299in}}%
\pgfpathlineto{\pgfqpoint{3.076862in}{4.034804in}}%
\pgfpathlineto{\pgfqpoint{3.097050in}{4.035693in}}%
\pgfpathlineto{\pgfqpoint{3.114889in}{4.041911in}}%
\pgfpathlineto{\pgfqpoint{3.134842in}{4.057792in}}%
\pgfpathlineto{\pgfqpoint{3.154558in}{4.092493in}}%
\pgfpathlineto{\pgfqpoint{3.175918in}{4.206371in}}%
\pgfpathlineto{\pgfqpoint{3.193759in}{4.394453in}}%
\pgfpathlineto{\pgfqpoint{3.213710in}{4.411549in}}%
\pgfpathlineto{\pgfqpoint{3.232489in}{4.331112in}}%
\pgfpathlineto{\pgfqpoint{3.249859in}{4.180626in}}%
\pgfpathlineto{\pgfqpoint{3.271219in}{4.084971in}}%
\pgfpathlineto{\pgfqpoint{3.289763in}{4.054951in}}%
\pgfpathlineto{\pgfqpoint{3.309479in}{4.041739in}}%
\pgfpathlineto{\pgfqpoint{3.326381in}{4.036662in}}%
\pgfpathlineto{\pgfqpoint{3.347506in}{4.035390in}}%
\pgfpathlineto{\pgfqpoint{3.365580in}{4.037956in}}%
\pgfpathlineto{\pgfqpoint{3.386001in}{4.044696in}}%
\pgfpathlineto{\pgfqpoint{3.403841in}{4.058602in}}%
\pgfpathlineto{\pgfqpoint{3.425202in}{4.096993in}}%
\pgfpathlineto{\pgfqpoint{3.443041in}{4.162676in}}%
\pgfpathlineto{\pgfqpoint{3.461349in}{4.105266in}}%
\pgfpathlineto{\pgfqpoint{3.478954in}{4.056269in}}%
\pgfpathlineto{\pgfqpoint{3.500550in}{4.096279in}}%
\pgfpathlineto{\pgfqpoint{3.539748in}{4.425162in}}%
\pgfpathlineto{\pgfqpoint{3.557119in}{4.390746in}}%
\pgfpathlineto{\pgfqpoint{3.575663in}{4.267254in}}%
\pgfpathlineto{\pgfqpoint{3.597728in}{4.111583in}}%
\pgfpathlineto{\pgfqpoint{3.614862in}{4.058286in}}%
\pgfpathlineto{\pgfqpoint{3.635754in}{4.041139in}}%
\pgfpathlineto{\pgfqpoint{3.654766in}{4.036369in}}%
\pgfpathlineto{\pgfqpoint{3.676127in}{4.035683in}}%
\pgfpathlineto{\pgfqpoint{3.693263in}{4.040743in}}%
\pgfpathlineto{\pgfqpoint{3.710631in}{4.049892in}}%
\pgfpathlineto{\pgfqpoint{3.728707in}{4.074907in}}%
\pgfpathlineto{\pgfqpoint{3.749127in}{4.147315in}}%
\pgfpathlineto{\pgfqpoint{3.768611in}{4.340507in}}%
\pgfpathlineto{\pgfqpoint{3.786919in}{4.434753in}}%
\pgfpathlineto{\pgfqpoint{3.807809in}{4.399508in}}%
\pgfpathlineto{\pgfqpoint{3.827997in}{4.262515in}}%
\pgfpathlineto{\pgfqpoint{3.846070in}{4.169735in}}%
\pgfpathlineto{\pgfqpoint{3.864380in}{4.092729in}}%
\pgfpathlineto{\pgfqpoint{3.886210in}{4.051042in}}%
\pgfpathlineto{\pgfqpoint{3.902641in}{4.039736in}}%
\pgfpathlineto{\pgfqpoint{3.920246in}{4.035826in}}%
\pgfpathlineto{\pgfqpoint{3.944891in}{4.039102in}}%
\pgfpathlineto{\pgfqpoint{3.960619in}{4.046903in}}%
\pgfpathlineto{\pgfqpoint{3.982214in}{4.063092in}}%
\pgfpathlineto{\pgfqpoint{3.999585in}{4.082977in}}%
\pgfpathlineto{\pgfqpoint{4.020241in}{4.169556in}}%
\pgfpathlineto{\pgfqpoint{4.038080in}{4.368185in}}%
\pgfpathlineto{\pgfqpoint{4.055919in}{4.442017in}}%
\pgfpathlineto{\pgfqpoint{4.076810in}{4.427627in}}%
\pgfpathlineto{\pgfqpoint{4.095354in}{4.357812in}}%
\pgfpathlineto{\pgfqpoint{4.112723in}{4.271844in}}%
\pgfpathlineto{\pgfqpoint{4.134319in}{4.137403in}}%
\pgfpathlineto{\pgfqpoint{4.152392in}{4.071195in}}%
\pgfpathlineto{\pgfqpoint{4.173519in}{4.046083in}}%
\pgfpathlineto{\pgfqpoint{4.192296in}{4.037895in}}%
\pgfpathlineto{\pgfqpoint{4.215769in}{4.037657in}}%
\pgfpathlineto{\pgfqpoint{4.229149in}{4.040864in}}%
\pgfpathlineto{\pgfqpoint{4.248633in}{4.047954in}}%
\pgfpathlineto{\pgfqpoint{4.270226in}{4.072253in}}%
\pgfpathlineto{\pgfqpoint{4.290883in}{4.155290in}}%
\pgfpathlineto{\pgfqpoint{4.306610in}{4.202384in}}%
\pgfpathlineto{\pgfqpoint{4.327266in}{4.425881in}}%
\pgfpathlineto{\pgfqpoint{4.345340in}{4.457846in}}%
\pgfpathlineto{\pgfqpoint{4.367875in}{4.404801in}}%
\pgfpathlineto{\pgfqpoint{4.381958in}{4.364497in}}%
\pgfpathlineto{\pgfqpoint{4.423505in}{4.091838in}}%
\pgfpathlineto{\pgfqpoint{4.441109in}{4.056304in}}%
\pgfpathlineto{\pgfqpoint{4.461766in}{4.041437in}}%
\pgfpathlineto{\pgfqpoint{4.482656in}{4.036934in}}%
\pgfpathlineto{\pgfqpoint{4.473737in}{4.038708in}}%
\pgfpathlineto{\pgfqpoint{4.455428in}{4.054219in}}%
\pgfpathlineto{\pgfqpoint{4.435007in}{4.125202in}}%
\pgfpathlineto{\pgfqpoint{4.419516in}{4.316894in}}%
\pgfpathlineto{\pgfqpoint{4.397685in}{4.441528in}}%
\pgfpathlineto{\pgfqpoint{4.377967in}{4.432975in}}%
\pgfpathlineto{\pgfqpoint{4.357547in}{4.175612in}}%
\pgfpathlineto{\pgfqpoint{4.340177in}{4.071536in}}%
\pgfpathlineto{\pgfqpoint{4.319521in}{4.042859in}}%
\pgfpathlineto{\pgfqpoint{4.302150in}{4.036315in}}%
\pgfpathlineto{\pgfqpoint{4.283137in}{4.041787in}}%
\pgfpathlineto{\pgfqpoint{4.261307in}{4.126044in}}%
\pgfpathlineto{\pgfqpoint{4.244407in}{4.292976in}}%
\pgfpathlineto{\pgfqpoint{4.224923in}{4.440079in}}%
\pgfpathlineto{\pgfqpoint{4.204267in}{4.403148in}}%
\pgfpathlineto{\pgfqpoint{4.187367in}{4.164990in}}%
\pgfpathlineto{\pgfqpoint{4.165537in}{4.060730in}}%
\pgfpathlineto{\pgfqpoint{4.147698in}{4.040469in}}%
\pgfpathlineto{\pgfqpoint{4.127982in}{4.035744in}}%
\pgfpathlineto{\pgfqpoint{4.111785in}{4.040980in}}%
\pgfpathlineto{\pgfqpoint{4.090189in}{4.070814in}}%
\pgfpathlineto{\pgfqpoint{4.071645in}{4.168174in}}%
\pgfpathlineto{\pgfqpoint{4.051460in}{4.380303in}}%
\pgfpathlineto{\pgfqpoint{4.030803in}{4.436731in}}%
\pgfpathlineto{\pgfqpoint{3.996768in}{4.092448in}}%
\pgfpathlineto{\pgfqpoint{3.977049in}{4.047271in}}%
\pgfpathlineto{\pgfqpoint{3.955690in}{4.036249in}}%
\pgfpathlineto{\pgfqpoint{3.934563in}{4.036880in}}%
\pgfpathlineto{\pgfqpoint{3.915786in}{4.048530in}}%
\pgfpathlineto{\pgfqpoint{3.898181in}{4.092428in}}%
\pgfpathlineto{\pgfqpoint{3.874003in}{4.310045in}}%
\pgfpathlineto{\pgfqpoint{3.860389in}{4.409173in}}%
\pgfpathlineto{\pgfqpoint{3.839030in}{4.408681in}}%
\pgfpathlineto{\pgfqpoint{3.818843in}{4.167100in}}%
\pgfpathlineto{\pgfqpoint{3.800298in}{4.069886in}}%
\pgfpathlineto{\pgfqpoint{3.783633in}{4.043023in}}%
\pgfpathlineto{\pgfqpoint{3.762272in}{4.035624in}}%
\pgfpathlineto{\pgfqpoint{3.745607in}{4.036789in}}%
\pgfpathlineto{\pgfqpoint{3.724247in}{4.049275in}}%
\pgfpathlineto{\pgfqpoint{3.708051in}{4.093728in}}%
\pgfpathlineto{\pgfqpoint{3.686689in}{4.271361in}}%
\pgfpathlineto{\pgfqpoint{3.665799in}{4.410036in}}%
\pgfpathlineto{\pgfqpoint{3.648194in}{4.417992in}}%
\pgfpathlineto{\pgfqpoint{3.630589in}{4.269188in}}%
\pgfpathlineto{\pgfqpoint{3.610168in}{4.084036in}}%
\pgfpathlineto{\pgfqpoint{3.592797in}{4.056249in}}%
\pgfpathlineto{\pgfqpoint{3.570498in}{4.038871in}}%
\pgfpathlineto{\pgfqpoint{3.551250in}{4.035153in}}%
\pgfpathlineto{\pgfqpoint{3.532237in}{4.037218in}}%
\pgfpathlineto{\pgfqpoint{3.513224in}{4.047814in}}%
\pgfpathlineto{\pgfqpoint{3.494916in}{4.096111in}}%
\pgfpathlineto{\pgfqpoint{3.455481in}{4.392975in}}%
\pgfpathlineto{\pgfqpoint{3.435530in}{4.419355in}}%
\pgfpathlineto{\pgfqpoint{3.416751in}{4.239035in}}%
\pgfpathlineto{\pgfqpoint{3.399850in}{4.099028in}}%
\pgfpathlineto{\pgfqpoint{3.377551in}{4.047188in}}%
\pgfpathlineto{\pgfqpoint{3.359242in}{4.037205in}}%
\pgfpathlineto{\pgfqpoint{3.342577in}{4.034934in}}%
\pgfpathlineto{\pgfqpoint{3.318868in}{4.039709in}}%
\pgfpathlineto{\pgfqpoint{3.297274in}{4.057071in}}%
\pgfpathlineto{\pgfqpoint{3.282252in}{4.099494in}}%
\pgfpathlineto{\pgfqpoint{3.263239in}{4.257482in}}%
\pgfpathlineto{\pgfqpoint{3.245163in}{4.383112in}}%
\pgfpathlineto{\pgfqpoint{3.224038in}{4.416517in}}%
\pgfpathlineto{\pgfqpoint{3.205259in}{4.239400in}}%
\pgfpathlineto{\pgfqpoint{3.187654in}{4.124481in}}%
\pgfpathlineto{\pgfqpoint{3.168407in}{4.055700in}}%
\pgfpathlineto{\pgfqpoint{3.148925in}{4.040135in}}%
\pgfpathlineto{\pgfqpoint{3.127329in}{4.035032in}}%
\pgfpathlineto{\pgfqpoint{3.108552in}{4.036964in}}%
\pgfpathlineto{\pgfqpoint{3.090947in}{4.043268in}}%
\pgfpathlineto{\pgfqpoint{3.071229in}{4.066370in}}%
\pgfpathlineto{\pgfqpoint{3.051043in}{4.183834in}}%
\pgfpathlineto{\pgfqpoint{3.031090in}{4.281665in}}%
\pgfpathlineto{\pgfqpoint{3.015599in}{4.397128in}}%
\pgfpathlineto{\pgfqpoint{2.993064in}{4.391198in}}%
\pgfpathlineto{\pgfqpoint{2.975225in}{4.185593in}}%
\pgfpathlineto{\pgfqpoint{2.956682in}{4.091282in}}%
\pgfpathlineto{\pgfqpoint{2.937903in}{4.055735in}}%
\pgfpathlineto{\pgfqpoint{2.916542in}{4.039912in}}%
\pgfpathlineto{\pgfqpoint{2.897529in}{4.035279in}}%
\pgfpathlineto{\pgfqpoint{2.878047in}{4.035747in}}%
\pgfpathlineto{\pgfqpoint{2.859268in}{4.039833in}}%
\pgfpathlineto{\pgfqpoint{2.840960in}{4.060898in}}%
\pgfpathlineto{\pgfqpoint{2.820772in}{4.097235in}}%
\pgfpathlineto{\pgfqpoint{2.802230in}{4.231777in}}%
\pgfpathlineto{\pgfqpoint{2.782748in}{4.348266in}}%
\pgfpathlineto{\pgfqpoint{2.764203in}{4.409600in}}%
\pgfpathlineto{\pgfqpoint{2.744251in}{4.390709in}}%
\pgfpathlineto{\pgfqpoint{2.725474in}{4.219035in}}%
\pgfpathlineto{\pgfqpoint{2.706695in}{4.139940in}}%
\pgfpathlineto{\pgfqpoint{2.687682in}{4.073293in}}%
\pgfpathlineto{\pgfqpoint{2.667729in}{4.045095in}}%
\pgfpathlineto{\pgfqpoint{2.649421in}{4.036275in}}%
\pgfpathlineto{\pgfqpoint{2.630173in}{4.034979in}}%
\pgfpathlineto{\pgfqpoint{2.608108in}{4.040025in}}%
\pgfpathlineto{\pgfqpoint{2.572664in}{4.068230in}}%
\pgfpathlineto{\pgfqpoint{2.548722in}{4.057335in}}%
\pgfpathlineto{\pgfqpoint{2.534169in}{4.111158in}}%
\pgfpathlineto{\pgfqpoint{2.515390in}{4.236067in}}%
\pgfpathlineto{\pgfqpoint{2.494265in}{4.401848in}}%
\pgfpathlineto{\pgfqpoint{2.478069in}{4.396655in}}%
\pgfpathlineto{\pgfqpoint{2.456004in}{4.181070in}}%
\pgfpathlineto{\pgfqpoint{2.437225in}{4.080805in}}%
\pgfpathlineto{\pgfqpoint{2.418212in}{4.045977in}}%
\pgfpathlineto{\pgfqpoint{2.399433in}{4.037350in}}%
\pgfpathlineto{\pgfqpoint{2.379482in}{4.034766in}}%
\pgfpathlineto{\pgfqpoint{2.361172in}{4.038110in}}%
\pgfpathlineto{\pgfqpoint{2.342864in}{4.047435in}}%
\pgfpathlineto{\pgfqpoint{2.321268in}{4.053094in}}%
\pgfpathlineto{\pgfqpoint{2.301552in}{4.045876in}}%
\pgfpathlineto{\pgfqpoint{2.282304in}{4.086083in}}%
\pgfpathlineto{\pgfqpoint{2.264934in}{4.218135in}}%
\pgfpathlineto{\pgfqpoint{2.244512in}{4.349862in}}%
\pgfpathlineto{\pgfqpoint{2.225733in}{4.410952in}}%
\pgfpathlineto{\pgfqpoint{2.206486in}{4.272073in}}%
\pgfpathlineto{\pgfqpoint{2.186535in}{4.107165in}}%
\pgfpathlineto{\pgfqpoint{2.168930in}{4.056884in}}%
\pgfpathlineto{\pgfqpoint{2.149917in}{4.040281in}}%
\pgfpathlineto{\pgfqpoint{2.133251in}{4.035529in}}%
\pgfpathlineto{\pgfqpoint{2.110482in}{4.035603in}}%
\pgfpathlineto{\pgfqpoint{2.094285in}{4.039083in}}%
\pgfpathlineto{\pgfqpoint{2.073864in}{4.054771in}}%
\pgfpathlineto{\pgfqpoint{2.052504in}{4.082241in}}%
\pgfpathlineto{\pgfqpoint{2.033256in}{4.211954in}}%
\pgfpathlineto{\pgfqpoint{2.014712in}{4.353500in}}%
\pgfpathlineto{\pgfqpoint{1.995699in}{4.412194in}}%
\pgfpathlineto{\pgfqpoint{1.971757in}{4.254409in}}%
\pgfpathlineto{\pgfqpoint{1.955795in}{4.124005in}}%
\pgfpathlineto{\pgfqpoint{1.939364in}{4.069410in}}%
\pgfpathlineto{\pgfqpoint{1.917534in}{4.049746in}}%
\pgfpathlineto{\pgfqpoint{1.901338in}{4.041737in}}%
\pgfpathlineto{\pgfqpoint{1.878804in}{4.035969in}}%
\pgfpathlineto{\pgfqpoint{1.863548in}{4.035221in}}%
\pgfpathlineto{\pgfqpoint{1.841483in}{4.039540in}}%
\pgfpathlineto{\pgfqpoint{1.823407in}{4.052924in}}%
\pgfpathlineto{\pgfqpoint{1.802048in}{4.123319in}}%
\pgfpathlineto{\pgfqpoint{1.784912in}{4.232190in}}%
\pgfpathlineto{\pgfqpoint{1.766133in}{4.366795in}}%
\pgfpathlineto{\pgfqpoint{1.747591in}{4.417171in}}%
\pgfpathlineto{\pgfqpoint{1.727403in}{4.363832in}}%
\pgfpathlineto{\pgfqpoint{1.702287in}{4.132614in}}%
\pgfpathlineto{\pgfqpoint{1.687031in}{4.070738in}}%
\pgfpathlineto{\pgfqpoint{1.668721in}{4.046453in}}%
\pgfpathlineto{\pgfqpoint{1.650647in}{4.038209in}}%
\pgfpathlineto{\pgfqpoint{1.627879in}{4.035215in}}%
\pgfpathlineto{\pgfqpoint{1.612386in}{4.037027in}}%
\pgfpathlineto{\pgfqpoint{1.593138in}{4.045653in}}%
\pgfpathlineto{\pgfqpoint{1.574594in}{4.068271in}}%
\pgfpathlineto{\pgfqpoint{1.555583in}{4.152970in}}%
\pgfpathlineto{\pgfqpoint{1.529762in}{4.280591in}}%
\pgfpathlineto{\pgfqpoint{1.516851in}{4.375645in}}%
\pgfpathlineto{\pgfqpoint{1.492675in}{4.418219in}}%
\pgfpathlineto{\pgfqpoint{1.475773in}{4.424686in}}%
\pgfpathlineto{\pgfqpoint{1.454883in}{4.290345in}}%
\pgfpathlineto{\pgfqpoint{1.437278in}{4.144829in}}%
\pgfpathlineto{\pgfqpoint{1.418030in}{4.072856in}}%
\pgfpathlineto{\pgfqpoint{1.400896in}{4.050163in}}%
\pgfpathlineto{\pgfqpoint{1.381412in}{4.040540in}}%
\pgfpathlineto{\pgfqpoint{1.358879in}{4.035682in}}%
\pgfpathlineto{\pgfqpoint{1.343387in}{4.035967in}}%
\pgfpathlineto{\pgfqpoint{1.322026in}{4.038452in}}%
\pgfpathlineto{\pgfqpoint{1.300901in}{4.044758in}}%
\pgfpathlineto{\pgfqpoint{1.286348in}{4.054322in}}%
\pgfpathlineto{\pgfqpoint{1.264517in}{4.124313in}}%
\pgfpathlineto{\pgfqpoint{1.245739in}{4.258687in}}%
\pgfpathlineto{\pgfqpoint{1.227431in}{4.366037in}}%
\pgfpathlineto{\pgfqpoint{1.203489in}{4.433922in}}%
\pgfpathlineto{\pgfqpoint{1.187527in}{4.399040in}}%
\pgfpathlineto{\pgfqpoint{1.169451in}{4.203421in}}%
\pgfpathlineto{\pgfqpoint{1.150203in}{4.105981in}}%
\pgfpathlineto{\pgfqpoint{1.131661in}{4.069027in}}%
\pgfpathlineto{\pgfqpoint{1.112882in}{4.050145in}}%
\pgfpathlineto{\pgfqpoint{1.091757in}{4.040146in}}%
\pgfpathlineto{\pgfqpoint{1.073682in}{4.036504in}}%
\pgfpathlineto{\pgfqpoint{1.054434in}{4.037386in}}%
\pgfpathlineto{\pgfqpoint{1.033075in}{4.042346in}}%
\pgfpathlineto{\pgfqpoint{1.015001in}{4.051749in}}%
\pgfpathlineto{\pgfqpoint{0.996691in}{4.078233in}}%
\pgfpathlineto{\pgfqpoint{0.975566in}{4.178598in}}%
\pgfpathlineto{\pgfqpoint{0.957021in}{4.320280in}}%
\pgfpathlineto{\pgfqpoint{0.938245in}{4.340463in}}%
\pgfpathlineto{\pgfqpoint{0.919466in}{4.433253in}}%
\pgfpathlineto{\pgfqpoint{0.902564in}{4.445452in}}%
\pgfpathlineto{\pgfqpoint{0.880265in}{4.342308in}}%
\pgfpathlineto{\pgfqpoint{0.861017in}{4.308967in}}%
\pgfpathlineto{\pgfqpoint{0.842239in}{4.429568in}}%
\pgfpathlineto{\pgfqpoint{0.823462in}{4.442028in}}%
\pgfpathlineto{\pgfqpoint{0.802335in}{4.282946in}}%
\pgfpathlineto{\pgfqpoint{0.785201in}{4.135127in}}%
\pgfpathlineto{\pgfqpoint{0.766188in}{4.077445in}}%
\pgfpathlineto{\pgfqpoint{0.746704in}{4.048561in}}%
\pgfpathlineto{\pgfqpoint{0.727221in}{4.038366in}}%
\pgfpathlineto{\pgfqpoint{0.707036in}{4.036394in}}%
\pgfpathlineto{\pgfqpoint{0.688492in}{4.040491in}}%
\pgfpathlineto{\pgfqpoint{0.667601in}{4.055947in}}%
\pgfpathlineto{\pgfqpoint{0.649525in}{4.094502in}}%
\pgfpathlineto{\pgfqpoint{0.650700in}{4.091918in}}%
\pgfpathlineto{\pgfqpoint{0.657039in}{4.069480in}}%
\pgfpathlineto{\pgfqpoint{0.676052in}{4.041918in}}%
\pgfpathlineto{\pgfqpoint{0.696003in}{4.036468in}}%
\pgfpathlineto{\pgfqpoint{0.714076in}{4.044387in}}%
\pgfpathlineto{\pgfqpoint{0.733324in}{4.075319in}}%
\pgfpathlineto{\pgfqpoint{0.752103in}{4.196417in}}%
\pgfpathlineto{\pgfqpoint{0.773464in}{4.431077in}}%
\pgfpathlineto{\pgfqpoint{0.790598in}{4.431778in}}%
\pgfpathlineto{\pgfqpoint{0.828625in}{4.100729in}}%
\pgfpathlineto{\pgfqpoint{0.849047in}{4.047768in}}%
\pgfpathlineto{\pgfqpoint{0.868294in}{4.036854in}}%
\pgfpathlineto{\pgfqpoint{0.887776in}{4.038734in}}%
\pgfpathlineto{\pgfqpoint{0.907495in}{4.053861in}}%
\pgfpathlineto{\pgfqpoint{0.925803in}{4.110636in}}%
\pgfpathlineto{\pgfqpoint{0.945519in}{4.336861in}}%
\pgfpathlineto{\pgfqpoint{0.964064in}{4.440654in}}%
\pgfpathlineto{\pgfqpoint{0.983546in}{4.356132in}}%
\pgfpathlineto{\pgfqpoint{1.002559in}{4.144287in}}%
\pgfpathlineto{\pgfqpoint{1.021572in}{4.062045in}}%
\pgfpathlineto{\pgfqpoint{1.041994in}{4.039070in}}%
\pgfpathlineto{\pgfqpoint{1.061242in}{4.035774in}}%
\pgfpathlineto{\pgfqpoint{1.080255in}{4.042303in}}%
\pgfpathlineto{\pgfqpoint{1.098329in}{4.067610in}}%
\pgfpathlineto{\pgfqpoint{1.117342in}{4.168526in}}%
\pgfpathlineto{\pgfqpoint{1.135652in}{4.411415in}}%
\pgfpathlineto{\pgfqpoint{1.154900in}{4.418171in}}%
\pgfpathlineto{\pgfqpoint{1.192690in}{4.091997in}}%
\pgfpathlineto{\pgfqpoint{1.211703in}{4.048742in}}%
\pgfpathlineto{\pgfqpoint{1.234237in}{4.035804in}}%
\pgfpathlineto{\pgfqpoint{1.252312in}{4.036708in}}%
\pgfpathlineto{\pgfqpoint{1.275080in}{4.049043in}}%
\pgfpathlineto{\pgfqpoint{1.290102in}{4.079857in}}%
\pgfpathlineto{\pgfqpoint{1.309586in}{4.182089in}}%
\pgfpathlineto{\pgfqpoint{1.327660in}{4.402476in}}%
\pgfpathlineto{\pgfqpoint{1.346908in}{4.408793in}}%
\pgfpathlineto{\pgfqpoint{1.387280in}{4.105741in}}%
\pgfpathlineto{\pgfqpoint{1.406999in}{4.050069in}}%
\pgfpathlineto{\pgfqpoint{1.423898in}{4.039600in}}%
\pgfpathlineto{\pgfqpoint{1.444086in}{4.035159in}}%
\pgfpathlineto{\pgfqpoint{1.463333in}{4.038072in}}%
\pgfpathlineto{\pgfqpoint{1.482815in}{4.052826in}}%
\pgfpathlineto{\pgfqpoint{1.500889in}{4.107344in}}%
\pgfpathlineto{\pgfqpoint{1.520137in}{4.282812in}}%
\pgfpathlineto{\pgfqpoint{1.540793in}{4.422924in}}%
\pgfpathlineto{\pgfqpoint{1.560277in}{4.357479in}}%
\pgfpathlineto{\pgfqpoint{1.579525in}{4.165679in}}%
\pgfpathlineto{\pgfqpoint{1.598303in}{4.070372in}}%
\pgfpathlineto{\pgfqpoint{1.616377in}{4.042033in}}%
\pgfpathlineto{\pgfqpoint{1.640788in}{4.035172in}}%
\pgfpathlineto{\pgfqpoint{1.656281in}{4.035085in}}%
\pgfpathlineto{\pgfqpoint{1.675060in}{4.038267in}}%
\pgfpathlineto{\pgfqpoint{1.694307in}{4.051367in}}%
\pgfpathlineto{\pgfqpoint{1.712615in}{4.073104in}}%
\pgfpathlineto{\pgfqpoint{1.737028in}{4.224238in}}%
\pgfpathlineto{\pgfqpoint{1.752285in}{4.402707in}}%
\pgfpathlineto{\pgfqpoint{1.770124in}{4.394457in}}%
\pgfpathlineto{\pgfqpoint{1.788434in}{4.241683in}}%
\pgfpathlineto{\pgfqpoint{1.808854in}{4.161999in}}%
\pgfpathlineto{\pgfqpoint{1.830684in}{4.065534in}}%
\pgfpathlineto{\pgfqpoint{1.848289in}{4.043305in}}%
\pgfpathlineto{\pgfqpoint{1.867302in}{4.035619in}}%
\pgfpathlineto{\pgfqpoint{1.888193in}{4.035137in}}%
\pgfpathlineto{\pgfqpoint{1.906503in}{4.039328in}}%
\pgfpathlineto{\pgfqpoint{1.926219in}{4.057166in}}%
\pgfpathlineto{\pgfqpoint{1.947110in}{4.115660in}}%
\pgfpathlineto{\pgfqpoint{1.982319in}{4.414987in}}%
\pgfpathlineto{\pgfqpoint{2.002741in}{4.339014in}}%
\pgfpathlineto{\pgfqpoint{2.019643in}{4.183984in}}%
\pgfpathlineto{\pgfqpoint{2.040533in}{4.069275in}}%
\pgfpathlineto{\pgfqpoint{2.060955in}{4.042853in}}%
\pgfpathlineto{\pgfqpoint{2.078560in}{4.035820in}}%
\pgfpathlineto{\pgfqpoint{2.100154in}{4.035045in}}%
\pgfpathlineto{\pgfqpoint{2.117055in}{4.037348in}}%
\pgfpathlineto{\pgfqpoint{2.134660in}{4.037622in}}%
\pgfpathlineto{\pgfqpoint{2.156724in}{4.051977in}}%
\pgfpathlineto{\pgfqpoint{2.174329in}{4.103027in}}%
\pgfpathlineto{\pgfqpoint{2.192169in}{4.255826in}}%
\pgfpathlineto{\pgfqpoint{2.212354in}{4.408864in}}%
\pgfpathlineto{\pgfqpoint{2.233950in}{4.374568in}}%
\pgfpathlineto{\pgfqpoint{2.252258in}{4.281083in}}%
\pgfpathlineto{\pgfqpoint{2.270333in}{4.124122in}}%
\pgfpathlineto{\pgfqpoint{2.290755in}{4.055310in}}%
\pgfpathlineto{\pgfqpoint{2.309063in}{4.044561in}}%
\pgfpathlineto{\pgfqpoint{2.329954in}{4.036486in}}%
\pgfpathlineto{\pgfqpoint{2.347558in}{4.034600in}}%
\pgfpathlineto{\pgfqpoint{2.366806in}{4.036347in}}%
\pgfpathlineto{\pgfqpoint{2.386525in}{4.046691in}}%
\pgfpathlineto{\pgfqpoint{2.404364in}{4.081308in}}%
\pgfpathlineto{\pgfqpoint{2.443094in}{4.347042in}}%
\pgfpathlineto{\pgfqpoint{2.460230in}{4.092958in}}%
\pgfpathlineto{\pgfqpoint{2.482058in}{4.250152in}}%
\pgfpathlineto{\pgfqpoint{2.503185in}{4.412934in}}%
\pgfpathlineto{\pgfqpoint{2.519381in}{4.360324in}}%
\pgfpathlineto{\pgfqpoint{2.540272in}{4.190762in}}%
\pgfpathlineto{\pgfqpoint{2.557171in}{4.080612in}}%
\pgfpathlineto{\pgfqpoint{2.578064in}{4.043655in}}%
\pgfpathlineto{\pgfqpoint{2.595198in}{4.037030in}}%
\pgfpathlineto{\pgfqpoint{2.617028in}{4.034834in}}%
\pgfpathlineto{\pgfqpoint{2.634633in}{4.038333in}}%
\pgfpathlineto{\pgfqpoint{2.657401in}{4.052225in}}%
\pgfpathlineto{\pgfqpoint{2.675711in}{4.092257in}}%
\pgfpathlineto{\pgfqpoint{2.692610in}{4.201445in}}%
\pgfpathlineto{\pgfqpoint{2.713737in}{4.378333in}}%
\pgfpathlineto{\pgfqpoint{2.731811in}{4.409261in}}%
\pgfpathlineto{\pgfqpoint{2.752232in}{4.340800in}}%
\pgfpathlineto{\pgfqpoint{2.770775in}{4.178768in}}%
\pgfpathlineto{\pgfqpoint{2.788145in}{4.074895in}}%
\pgfpathlineto{\pgfqpoint{2.809741in}{4.044631in}}%
\pgfpathlineto{\pgfqpoint{2.827815in}{4.037081in}}%
\pgfpathlineto{\pgfqpoint{2.848471in}{4.034747in}}%
\pgfpathlineto{\pgfqpoint{2.884854in}{4.040043in}}%
\pgfpathlineto{\pgfqpoint{2.906214in}{4.061169in}}%
\pgfpathlineto{\pgfqpoint{2.923584in}{4.106421in}}%
\pgfpathlineto{\pgfqpoint{2.945883in}{4.288137in}}%
\pgfpathlineto{\pgfqpoint{2.961845in}{4.406187in}}%
\pgfpathlineto{\pgfqpoint{2.983441in}{4.369346in}}%
\pgfpathlineto{\pgfqpoint{3.000811in}{4.269557in}}%
\pgfpathlineto{\pgfqpoint{3.019354in}{4.120187in}}%
\pgfpathlineto{\pgfqpoint{3.041184in}{4.056678in}}%
\pgfpathlineto{\pgfqpoint{3.058554in}{4.040961in}}%
\pgfpathlineto{\pgfqpoint{3.076628in}{4.035611in}}%
\pgfpathlineto{\pgfqpoint{3.100570in}{4.035763in}}%
\pgfpathlineto{\pgfqpoint{3.116297in}{4.040103in}}%
\pgfpathlineto{\pgfqpoint{3.134371in}{4.047784in}}%
\pgfpathlineto{\pgfqpoint{3.155262in}{4.080534in}}%
\pgfpathlineto{\pgfqpoint{3.175918in}{4.183570in}}%
\pgfpathlineto{\pgfqpoint{3.193759in}{4.356160in}}%
\pgfpathlineto{\pgfqpoint{3.212067in}{4.415478in}}%
\pgfpathlineto{\pgfqpoint{3.232254in}{4.368759in}}%
\pgfpathlineto{\pgfqpoint{3.267933in}{4.117624in}}%
\pgfpathlineto{\pgfqpoint{3.289058in}{4.060274in}}%
\pgfpathlineto{\pgfqpoint{3.307368in}{4.042239in}}%
\pgfpathlineto{\pgfqpoint{3.327084in}{4.036974in}}%
\pgfpathlineto{\pgfqpoint{3.346801in}{4.034906in}}%
\pgfpathlineto{\pgfqpoint{3.365111in}{4.037123in}}%
\pgfpathlineto{\pgfqpoint{3.403606in}{4.052317in}}%
\pgfpathlineto{\pgfqpoint{3.421211in}{4.094781in}}%
\pgfpathlineto{\pgfqpoint{3.443276in}{4.209516in}}%
\pgfpathlineto{\pgfqpoint{3.462758in}{4.398261in}}%
\pgfpathlineto{\pgfqpoint{3.479894in}{4.423956in}}%
\pgfpathlineto{\pgfqpoint{3.501724in}{4.346017in}}%
\pgfpathlineto{\pgfqpoint{3.520501in}{4.211908in}}%
\pgfpathlineto{\pgfqpoint{3.537402in}{4.110767in}}%
\pgfpathlineto{\pgfqpoint{3.557353in}{4.058494in}}%
\pgfpathlineto{\pgfqpoint{3.578246in}{4.041748in}}%
\pgfpathlineto{\pgfqpoint{3.596554in}{4.036765in}}%
\pgfpathlineto{\pgfqpoint{3.615098in}{4.035446in}}%
\pgfpathlineto{\pgfqpoint{3.635518in}{4.037017in}}%
\pgfpathlineto{\pgfqpoint{3.653593in}{4.046281in}}%
\pgfpathlineto{\pgfqpoint{3.671667in}{4.059844in}}%
\pgfpathlineto{\pgfqpoint{3.690446in}{4.086009in}}%
\pgfpathlineto{\pgfqpoint{3.711337in}{4.186055in}}%
\pgfpathlineto{\pgfqpoint{3.732227in}{4.386649in}}%
\pgfpathlineto{\pgfqpoint{3.751006in}{4.432998in}}%
\pgfpathlineto{\pgfqpoint{3.769314in}{4.433916in}}%
\pgfpathlineto{\pgfqpoint{3.789736in}{4.377326in}}%
\pgfpathlineto{\pgfqpoint{3.806872in}{4.247697in}}%
\pgfpathlineto{\pgfqpoint{3.827762in}{4.104851in}}%
\pgfpathlineto{\pgfqpoint{3.846070in}{4.071313in}}%
\pgfpathlineto{\pgfqpoint{3.864146in}{4.047541in}}%
\pgfpathlineto{\pgfqpoint{3.885271in}{4.037330in}}%
\pgfpathlineto{\pgfqpoint{3.901936in}{4.035723in}}%
\pgfpathlineto{\pgfqpoint{3.922592in}{4.038352in}}%
\pgfpathlineto{\pgfqpoint{3.940902in}{4.048260in}}%
\pgfpathlineto{\pgfqpoint{3.960853in}{4.062593in}}%
\pgfpathlineto{\pgfqpoint{3.981980in}{4.110586in}}%
\pgfpathlineto{\pgfqpoint{3.999819in}{4.225778in}}%
\pgfpathlineto{\pgfqpoint{4.017893in}{4.332990in}}%
\pgfpathlineto{\pgfqpoint{4.041600in}{4.440003in}}%
\pgfpathlineto{\pgfqpoint{4.056857in}{4.438553in}}%
\pgfpathlineto{\pgfqpoint{4.077750in}{4.356898in}}%
\pgfpathlineto{\pgfqpoint{4.095589in}{4.273629in}}%
\pgfpathlineto{\pgfqpoint{4.116011in}{4.144124in}}%
\pgfpathlineto{\pgfqpoint{4.134787in}{4.066592in}}%
\pgfpathlineto{\pgfqpoint{4.153332in}{4.046680in}}%
\pgfpathlineto{\pgfqpoint{4.171171in}{4.042708in}}%
\pgfpathlineto{\pgfqpoint{4.192062in}{4.036243in}}%
\pgfpathlineto{\pgfqpoint{4.211544in}{4.037423in}}%
\pgfpathlineto{\pgfqpoint{4.228211in}{4.041456in}}%
\pgfpathlineto{\pgfqpoint{4.250041in}{4.059036in}}%
\pgfpathlineto{\pgfqpoint{4.270461in}{4.109723in}}%
\pgfpathlineto{\pgfqpoint{4.289005in}{4.227131in}}%
\pgfpathlineto{\pgfqpoint{4.306376in}{4.348878in}}%
\pgfpathlineto{\pgfqpoint{4.324918in}{4.453839in}}%
\pgfpathlineto{\pgfqpoint{4.346748in}{4.450044in}}%
\pgfpathlineto{\pgfqpoint{4.364588in}{4.431794in}}%
\pgfpathlineto{\pgfqpoint{4.385714in}{4.303876in}}%
\pgfpathlineto{\pgfqpoint{4.400032in}{4.146595in}}%
\pgfpathlineto{\pgfqpoint{4.421862in}{4.088890in}}%
\pgfpathlineto{\pgfqpoint{4.442049in}{4.051231in}}%
\pgfpathlineto{\pgfqpoint{4.461062in}{4.041362in}}%
\pgfpathlineto{\pgfqpoint{4.480544in}{4.037547in}}%
\pgfpathlineto{\pgfqpoint{4.475379in}{4.417174in}}%
\pgfpathlineto{\pgfqpoint{4.453786in}{4.135950in}}%
\pgfpathlineto{\pgfqpoint{4.436884in}{4.060644in}}%
\pgfpathlineto{\pgfqpoint{4.415993in}{4.039124in}}%
\pgfpathlineto{\pgfqpoint{4.400032in}{4.038472in}}%
\pgfpathlineto{\pgfqpoint{4.378203in}{4.048307in}}%
\pgfpathlineto{\pgfqpoint{4.358485in}{4.103543in}}%
\pgfpathlineto{\pgfqpoint{4.342289in}{4.261987in}}%
\pgfpathlineto{\pgfqpoint{4.320458in}{4.437569in}}%
\pgfpathlineto{\pgfqpoint{4.302619in}{4.433837in}}%
\pgfpathlineto{\pgfqpoint{4.281025in}{4.171292in}}%
\pgfpathlineto{\pgfqpoint{4.261778in}{4.065039in}}%
\pgfpathlineto{\pgfqpoint{4.245110in}{4.041279in}}%
\pgfpathlineto{\pgfqpoint{4.223751in}{4.036187in}}%
\pgfpathlineto{\pgfqpoint{4.207084in}{4.041638in}}%
\pgfpathlineto{\pgfqpoint{4.186664in}{4.069746in}}%
\pgfpathlineto{\pgfqpoint{4.163660in}{4.240392in}}%
\pgfpathlineto{\pgfqpoint{4.145350in}{4.405695in}}%
\pgfpathlineto{\pgfqpoint{4.127276in}{4.440085in}}%
\pgfpathlineto{\pgfqpoint{4.107560in}{4.232260in}}%
\pgfpathlineto{\pgfqpoint{4.089486in}{4.087374in}}%
\pgfpathlineto{\pgfqpoint{4.072585in}{4.047601in}}%
\pgfpathlineto{\pgfqpoint{4.051460in}{4.036329in}}%
\pgfpathlineto{\pgfqpoint{4.033620in}{4.036795in}}%
\pgfpathlineto{\pgfqpoint{4.010616in}{4.048580in}}%
\pgfpathlineto{\pgfqpoint{3.992777in}{4.093445in}}%
\pgfpathlineto{\pgfqpoint{3.973529in}{4.229210in}}%
\pgfpathlineto{\pgfqpoint{3.955690in}{4.404767in}}%
\pgfpathlineto{\pgfqpoint{3.937380in}{4.429897in}}%
\pgfpathlineto{\pgfqpoint{3.916490in}{4.246574in}}%
\pgfpathlineto{\pgfqpoint{3.895833in}{4.082236in}}%
\pgfpathlineto{\pgfqpoint{3.878229in}{4.045964in}}%
\pgfpathlineto{\pgfqpoint{3.860389in}{4.036668in}}%
\pgfpathlineto{\pgfqpoint{3.840202in}{4.036131in}}%
\pgfpathlineto{\pgfqpoint{3.822128in}{4.039504in}}%
\pgfpathlineto{\pgfqpoint{3.801941in}{4.061336in}}%
\pgfpathlineto{\pgfqpoint{3.781519in}{4.166799in}}%
\pgfpathlineto{\pgfqpoint{3.762977in}{4.334894in}}%
\pgfpathlineto{\pgfqpoint{3.744669in}{4.426494in}}%
\pgfpathlineto{\pgfqpoint{3.723542in}{4.403886in}}%
\pgfpathlineto{\pgfqpoint{3.705468in}{4.235501in}}%
\pgfpathlineto{\pgfqpoint{3.688332in}{4.112662in}}%
\pgfpathlineto{\pgfqpoint{3.666738in}{4.050842in}}%
\pgfpathlineto{\pgfqpoint{3.649134in}{4.038330in}}%
\pgfpathlineto{\pgfqpoint{3.628478in}{4.035291in}}%
\pgfpathlineto{\pgfqpoint{3.607351in}{4.038228in}}%
\pgfpathlineto{\pgfqpoint{3.589982in}{4.050854in}}%
\pgfpathlineto{\pgfqpoint{3.573081in}{4.078243in}}%
\pgfpathlineto{\pgfqpoint{3.551721in}{4.214536in}}%
\pgfpathlineto{\pgfqpoint{3.530594in}{4.270529in}}%
\pgfpathlineto{\pgfqpoint{3.515103in}{4.389992in}}%
\pgfpathlineto{\pgfqpoint{3.494916in}{4.416973in}}%
\pgfpathlineto{\pgfqpoint{3.477546in}{4.272243in}}%
\pgfpathlineto{\pgfqpoint{3.454778in}{4.117186in}}%
\pgfpathlineto{\pgfqpoint{3.437173in}{4.059320in}}%
\pgfpathlineto{\pgfqpoint{3.418863in}{4.127175in}}%
\pgfpathlineto{\pgfqpoint{3.398207in}{4.294422in}}%
\pgfpathlineto{\pgfqpoint{3.378725in}{4.416500in}}%
\pgfpathlineto{\pgfqpoint{3.361120in}{4.329456in}}%
\pgfpathlineto{\pgfqpoint{3.341638in}{4.142952in}}%
\pgfpathlineto{\pgfqpoint{3.319104in}{4.058559in}}%
\pgfpathlineto{\pgfqpoint{3.300794in}{4.040783in}}%
\pgfpathlineto{\pgfqpoint{3.283895in}{4.035269in}}%
\pgfpathlineto{\pgfqpoint{3.262299in}{4.037625in}}%
\pgfpathlineto{\pgfqpoint{3.244929in}{4.045946in}}%
\pgfpathlineto{\pgfqpoint{3.224038in}{4.078669in}}%
\pgfpathlineto{\pgfqpoint{3.204790in}{4.186260in}}%
\pgfpathlineto{\pgfqpoint{3.186482in}{4.364591in}}%
\pgfpathlineto{\pgfqpoint{3.169112in}{4.414243in}}%
\pgfpathlineto{\pgfqpoint{3.149159in}{4.236756in}}%
\pgfpathlineto{\pgfqpoint{3.131085in}{4.095077in}}%
\pgfpathlineto{\pgfqpoint{3.109490in}{4.048657in}}%
\pgfpathlineto{\pgfqpoint{3.089773in}{4.037582in}}%
\pgfpathlineto{\pgfqpoint{3.071699in}{4.034929in}}%
\pgfpathlineto{\pgfqpoint{3.051747in}{4.037095in}}%
\pgfpathlineto{\pgfqpoint{3.031325in}{4.050935in}}%
\pgfpathlineto{\pgfqpoint{3.015363in}{4.089796in}}%
\pgfpathlineto{\pgfqpoint{2.974756in}{4.400789in}}%
\pgfpathlineto{\pgfqpoint{2.957151in}{4.407416in}}%
\pgfpathlineto{\pgfqpoint{2.937434in}{4.202778in}}%
\pgfpathlineto{\pgfqpoint{2.916542in}{4.079855in}}%
\pgfpathlineto{\pgfqpoint{2.897294in}{4.046667in}}%
\pgfpathlineto{\pgfqpoint{2.880629in}{4.039591in}}%
\pgfpathlineto{\pgfqpoint{2.856922in}{4.034909in}}%
\pgfpathlineto{\pgfqpoint{2.840960in}{4.037039in}}%
\pgfpathlineto{\pgfqpoint{2.823120in}{4.045975in}}%
\pgfpathlineto{\pgfqpoint{2.801290in}{4.094319in}}%
\pgfpathlineto{\pgfqpoint{2.781808in}{4.155458in}}%
\pgfpathlineto{\pgfqpoint{2.744956in}{4.398042in}}%
\pgfpathlineto{\pgfqpoint{2.722891in}{4.401280in}}%
\pgfpathlineto{\pgfqpoint{2.707398in}{4.217368in}}%
\pgfpathlineto{\pgfqpoint{2.685804in}{4.074410in}}%
\pgfpathlineto{\pgfqpoint{2.667025in}{4.045204in}}%
\pgfpathlineto{\pgfqpoint{2.648481in}{4.036659in}}%
\pgfpathlineto{\pgfqpoint{2.627590in}{4.034861in}}%
\pgfpathlineto{\pgfqpoint{2.611628in}{4.036710in}}%
\pgfpathlineto{\pgfqpoint{2.590035in}{4.048365in}}%
\pgfpathlineto{\pgfqpoint{2.574307in}{4.075438in}}%
\pgfpathlineto{\pgfqpoint{2.552711in}{4.175790in}}%
\pgfpathlineto{\pgfqpoint{2.533934in}{4.312287in}}%
\pgfpathlineto{\pgfqpoint{2.514921in}{4.398030in}}%
\pgfpathlineto{\pgfqpoint{2.495203in}{4.372297in}}%
\pgfpathlineto{\pgfqpoint{2.476190in}{4.262440in}}%
\pgfpathlineto{\pgfqpoint{2.454830in}{4.096659in}}%
\pgfpathlineto{\pgfqpoint{2.438165in}{4.052992in}}%
\pgfpathlineto{\pgfqpoint{2.419386in}{4.039989in}}%
\pgfpathlineto{\pgfqpoint{2.400607in}{4.035174in}}%
\pgfpathlineto{\pgfqpoint{2.379482in}{4.035483in}}%
\pgfpathlineto{\pgfqpoint{2.360235in}{4.040851in}}%
\pgfpathlineto{\pgfqpoint{2.342159in}{4.061531in}}%
\pgfpathlineto{\pgfqpoint{2.324085in}{4.093964in}}%
\pgfpathlineto{\pgfqpoint{2.301317in}{4.245184in}}%
\pgfpathlineto{\pgfqpoint{2.283007in}{4.378570in}}%
\pgfpathlineto{\pgfqpoint{2.264934in}{4.398975in}}%
\pgfpathlineto{\pgfqpoint{2.245452in}{4.220824in}}%
\pgfpathlineto{\pgfqpoint{2.227142in}{4.092553in}}%
\pgfpathlineto{\pgfqpoint{2.209537in}{4.055191in}}%
\pgfpathlineto{\pgfqpoint{2.186066in}{4.040583in}}%
\pgfpathlineto{\pgfqpoint{2.168461in}{4.035543in}}%
\pgfpathlineto{\pgfqpoint{2.149213in}{4.035698in}}%
\pgfpathlineto{\pgfqpoint{2.129495in}{4.038682in}}%
\pgfpathlineto{\pgfqpoint{2.108604in}{4.051600in}}%
\pgfpathlineto{\pgfqpoint{2.091468in}{4.067928in}}%
\pgfpathlineto{\pgfqpoint{2.072455in}{4.161711in}}%
\pgfpathlineto{\pgfqpoint{2.054147in}{4.318421in}}%
\pgfpathlineto{\pgfqpoint{2.033022in}{4.411113in}}%
\pgfpathlineto{\pgfqpoint{2.014712in}{4.366187in}}%
\pgfpathlineto{\pgfqpoint{1.997107in}{4.187361in}}%
\pgfpathlineto{\pgfqpoint{1.978094in}{4.093501in}}%
\pgfpathlineto{\pgfqpoint{1.959317in}{4.054978in}}%
\pgfpathlineto{\pgfqpoint{1.936782in}{4.039516in}}%
\pgfpathlineto{\pgfqpoint{1.917768in}{4.035886in}}%
\pgfpathlineto{\pgfqpoint{1.899695in}{4.034974in}}%
\pgfpathlineto{\pgfqpoint{1.881387in}{4.037379in}}%
\pgfpathlineto{\pgfqpoint{1.859557in}{4.047411in}}%
\pgfpathlineto{\pgfqpoint{1.841483in}{4.074334in}}%
\pgfpathlineto{\pgfqpoint{1.822235in}{4.177043in}}%
\pgfpathlineto{\pgfqpoint{1.802282in}{4.327187in}}%
\pgfpathlineto{\pgfqpoint{1.779278in}{4.405118in}}%
\pgfpathlineto{\pgfqpoint{1.760736in}{4.404528in}}%
\pgfpathlineto{\pgfqpoint{1.745008in}{4.281206in}}%
\pgfpathlineto{\pgfqpoint{1.726229in}{4.178094in}}%
\pgfpathlineto{\pgfqpoint{1.707687in}{4.420057in}}%
\pgfpathlineto{\pgfqpoint{1.688674in}{4.342517in}}%
\pgfpathlineto{\pgfqpoint{1.670600in}{4.136905in}}%
\pgfpathlineto{\pgfqpoint{1.649004in}{4.059947in}}%
\pgfpathlineto{\pgfqpoint{1.633042in}{4.041550in}}%
\pgfpathlineto{\pgfqpoint{1.611683in}{4.036573in}}%
\pgfpathlineto{\pgfqpoint{1.590790in}{4.035464in}}%
\pgfpathlineto{\pgfqpoint{1.573422in}{4.038969in}}%
\pgfpathlineto{\pgfqpoint{1.552061in}{4.056551in}}%
\pgfpathlineto{\pgfqpoint{1.531873in}{4.094591in}}%
\pgfpathlineto{\pgfqpoint{1.513800in}{4.209723in}}%
\pgfpathlineto{\pgfqpoint{1.496429in}{4.309892in}}%
\pgfpathlineto{\pgfqpoint{1.476478in}{4.415010in}}%
\pgfpathlineto{\pgfqpoint{1.457934in}{4.411499in}}%
\pgfpathlineto{\pgfqpoint{1.417327in}{4.139794in}}%
\pgfpathlineto{\pgfqpoint{1.403242in}{4.078199in}}%
\pgfpathlineto{\pgfqpoint{1.381178in}{4.046909in}}%
\pgfpathlineto{\pgfqpoint{1.359347in}{4.038669in}}%
\pgfpathlineto{\pgfqpoint{1.343856in}{4.035974in}}%
\pgfpathlineto{\pgfqpoint{1.322026in}{4.036819in}}%
\pgfpathlineto{\pgfqpoint{1.303013in}{4.043186in}}%
\pgfpathlineto{\pgfqpoint{1.281888in}{4.067234in}}%
\pgfpathlineto{\pgfqpoint{1.264049in}{4.098822in}}%
\pgfpathlineto{\pgfqpoint{1.245035in}{4.196672in}}%
\pgfpathlineto{\pgfqpoint{1.226491in}{4.335710in}}%
\pgfpathlineto{\pgfqpoint{1.208417in}{4.410512in}}%
\pgfpathlineto{\pgfqpoint{1.186353in}{4.433496in}}%
\pgfpathlineto{\pgfqpoint{1.169451in}{4.360931in}}%
\pgfpathlineto{\pgfqpoint{1.149500in}{4.153158in}}%
\pgfpathlineto{\pgfqpoint{1.130018in}{4.081852in}}%
\pgfpathlineto{\pgfqpoint{1.112882in}{4.069553in}}%
\pgfpathlineto{\pgfqpoint{1.094103in}{4.051093in}}%
\pgfpathlineto{\pgfqpoint{1.073918in}{4.040774in}}%
\pgfpathlineto{\pgfqpoint{1.054670in}{4.035933in}}%
\pgfpathlineto{\pgfqpoint{1.034249in}{4.036558in}}%
\pgfpathlineto{\pgfqpoint{1.016409in}{4.041167in}}%
\pgfpathlineto{\pgfqpoint{0.995988in}{4.059345in}}%
\pgfpathlineto{\pgfqpoint{0.977912in}{4.094575in}}%
\pgfpathlineto{\pgfqpoint{0.957021in}{4.192866in}}%
\pgfpathlineto{\pgfqpoint{0.941296in}{4.302800in}}%
\pgfpathlineto{\pgfqpoint{0.919231in}{4.421828in}}%
\pgfpathlineto{\pgfqpoint{0.900218in}{4.443400in}}%
\pgfpathlineto{\pgfqpoint{0.882379in}{4.389713in}}%
\pgfpathlineto{\pgfqpoint{0.863600in}{4.213608in}}%
\pgfpathlineto{\pgfqpoint{0.842239in}{4.096806in}}%
\pgfpathlineto{\pgfqpoint{0.823696in}{4.064919in}}%
\pgfpathlineto{\pgfqpoint{0.801866in}{4.049287in}}%
\pgfpathlineto{\pgfqpoint{0.786844in}{4.040208in}}%
\pgfpathlineto{\pgfqpoint{0.765482in}{4.036641in}}%
\pgfpathlineto{\pgfqpoint{0.747174in}{4.037221in}}%
\pgfpathlineto{\pgfqpoint{0.726518in}{4.046149in}}%
\pgfpathlineto{\pgfqpoint{0.707505in}{4.060925in}}%
\pgfpathlineto{\pgfqpoint{0.688960in}{4.108829in}}%
\pgfpathlineto{\pgfqpoint{0.670652in}{4.180771in}}%
\pgfpathlineto{\pgfqpoint{0.652342in}{4.344164in}}%
\pgfpathlineto{\pgfqpoint{0.647414in}{4.355125in}}%
\pgfpathlineto{\pgfqpoint{0.654690in}{4.238739in}}%
\pgfpathlineto{\pgfqpoint{0.677224in}{4.081550in}}%
\pgfpathlineto{\pgfqpoint{0.696472in}{4.044705in}}%
\pgfpathlineto{\pgfqpoint{0.712199in}{4.036934in}}%
\pgfpathlineto{\pgfqpoint{0.735438in}{4.041216in}}%
\pgfpathlineto{\pgfqpoint{0.751869in}{4.056962in}}%
\pgfpathlineto{\pgfqpoint{0.772994in}{4.141659in}}%
\pgfpathlineto{\pgfqpoint{0.789895in}{4.347813in}}%
\pgfpathlineto{\pgfqpoint{0.807734in}{4.447087in}}%
\pgfpathlineto{\pgfqpoint{0.827922in}{4.356458in}}%
\pgfpathlineto{\pgfqpoint{0.850221in}{4.135152in}}%
\pgfpathlineto{\pgfqpoint{0.869703in}{4.056303in}}%
\pgfpathlineto{\pgfqpoint{0.885430in}{4.040382in}}%
\pgfpathlineto{\pgfqpoint{0.903738in}{4.035715in}}%
\pgfpathlineto{\pgfqpoint{0.926508in}{4.045710in}}%
\pgfpathlineto{\pgfqpoint{0.947633in}{4.080736in}}%
\pgfpathlineto{\pgfqpoint{0.964533in}{4.198395in}}%
\pgfpathlineto{\pgfqpoint{0.984486in}{4.424894in}}%
\pgfpathlineto{\pgfqpoint{1.006081in}{4.391548in}}%
\pgfpathlineto{\pgfqpoint{1.022981in}{4.228936in}}%
\pgfpathlineto{\pgfqpoint{1.041994in}{4.088012in}}%
\pgfpathlineto{\pgfqpoint{1.060068in}{4.045240in}}%
\pgfpathlineto{\pgfqpoint{1.080255in}{4.036207in}}%
\pgfpathlineto{\pgfqpoint{1.098329in}{4.037585in}}%
\pgfpathlineto{\pgfqpoint{1.117342in}{4.050814in}}%
\pgfpathlineto{\pgfqpoint{1.136590in}{4.094909in}}%
\pgfpathlineto{\pgfqpoint{1.156308in}{4.305748in}}%
\pgfpathlineto{\pgfqpoint{1.175556in}{4.430715in}}%
\pgfpathlineto{\pgfqpoint{1.194803in}{4.341676in}}%
\pgfpathlineto{\pgfqpoint{1.212172in}{4.170107in}}%
\pgfpathlineto{\pgfqpoint{1.231185in}{4.075850in}}%
\pgfpathlineto{\pgfqpoint{1.251841in}{4.041300in}}%
\pgfpathlineto{\pgfqpoint{1.272497in}{4.035404in}}%
\pgfpathlineto{\pgfqpoint{1.293390in}{4.039391in}}%
\pgfpathlineto{\pgfqpoint{1.309821in}{4.050026in}}%
\pgfpathlineto{\pgfqpoint{1.329537in}{4.107379in}}%
\pgfpathlineto{\pgfqpoint{1.348316in}{4.317159in}}%
\pgfpathlineto{\pgfqpoint{1.365450in}{4.423640in}}%
\pgfpathlineto{\pgfqpoint{1.386811in}{4.341503in}}%
\pgfpathlineto{\pgfqpoint{1.404182in}{4.161982in}}%
\pgfpathlineto{\pgfqpoint{1.427184in}{4.077292in}}%
\pgfpathlineto{\pgfqpoint{1.445729in}{4.045725in}}%
\pgfpathlineto{\pgfqpoint{1.464742in}{4.036535in}}%
\pgfpathlineto{\pgfqpoint{1.483050in}{4.035580in}}%
\pgfpathlineto{\pgfqpoint{1.501829in}{4.042706in}}%
\pgfpathlineto{\pgfqpoint{1.520608in}{4.065826in}}%
\pgfpathlineto{\pgfqpoint{1.541967in}{4.175731in}}%
\pgfpathlineto{\pgfqpoint{1.560277in}{4.384372in}}%
\pgfpathlineto{\pgfqpoint{1.579525in}{4.412982in}}%
\pgfpathlineto{\pgfqpoint{1.600415in}{4.272452in}}%
\pgfpathlineto{\pgfqpoint{1.617315in}{4.135574in}}%
\pgfpathlineto{\pgfqpoint{1.635156in}{4.075551in}}%
\pgfpathlineto{\pgfqpoint{1.657924in}{4.043273in}}%
\pgfpathlineto{\pgfqpoint{1.675529in}{4.035626in}}%
\pgfpathlineto{\pgfqpoint{1.694073in}{4.036059in}}%
\pgfpathlineto{\pgfqpoint{1.713790in}{4.043397in}}%
\pgfpathlineto{\pgfqpoint{1.734680in}{4.075178in}}%
\pgfpathlineto{\pgfqpoint{1.752519in}{4.093748in}}%
\pgfpathlineto{\pgfqpoint{1.775524in}{4.306772in}}%
\pgfpathlineto{\pgfqpoint{1.790780in}{4.413833in}}%
\pgfpathlineto{\pgfqpoint{1.808854in}{4.391544in}}%
\pgfpathlineto{\pgfqpoint{1.848054in}{4.092982in}}%
\pgfpathlineto{\pgfqpoint{1.867068in}{4.046804in}}%
\pgfpathlineto{\pgfqpoint{1.888663in}{4.037122in}}%
\pgfpathlineto{\pgfqpoint{1.906268in}{4.034856in}}%
\pgfpathlineto{\pgfqpoint{1.924342in}{4.037043in}}%
\pgfpathlineto{\pgfqpoint{1.943590in}{4.048224in}}%
\pgfpathlineto{\pgfqpoint{1.964480in}{4.103414in}}%
\pgfpathlineto{\pgfqpoint{2.000864in}{4.414483in}}%
\pgfpathlineto{\pgfqpoint{2.022223in}{4.053995in}}%
\pgfpathlineto{\pgfqpoint{2.039359in}{4.101427in}}%
\pgfpathlineto{\pgfqpoint{2.060719in}{4.305600in}}%
\pgfpathlineto{\pgfqpoint{2.076680in}{4.415293in}}%
\pgfpathlineto{\pgfqpoint{2.096633in}{4.332471in}}%
\pgfpathlineto{\pgfqpoint{2.117055in}{4.129133in}}%
\pgfpathlineto{\pgfqpoint{2.134894in}{4.062597in}}%
\pgfpathlineto{\pgfqpoint{2.154845in}{4.040331in}}%
\pgfpathlineto{\pgfqpoint{2.175736in}{4.034873in}}%
\pgfpathlineto{\pgfqpoint{2.192403in}{4.035563in}}%
\pgfpathlineto{\pgfqpoint{2.214702in}{4.043102in}}%
\pgfpathlineto{\pgfqpoint{2.231133in}{4.064476in}}%
\pgfpathlineto{\pgfqpoint{2.251554in}{4.159477in}}%
\pgfpathlineto{\pgfqpoint{2.271976in}{4.371879in}}%
\pgfpathlineto{\pgfqpoint{2.289110in}{4.418366in}}%
\pgfpathlineto{\pgfqpoint{2.305777in}{4.325827in}}%
\pgfpathlineto{\pgfqpoint{2.327371in}{4.161729in}}%
\pgfpathlineto{\pgfqpoint{2.348732in}{4.059093in}}%
\pgfpathlineto{\pgfqpoint{2.366103in}{4.040083in}}%
\pgfpathlineto{\pgfqpoint{2.389810in}{4.034907in}}%
\pgfpathlineto{\pgfqpoint{2.404129in}{4.035534in}}%
\pgfpathlineto{\pgfqpoint{2.425489in}{4.040928in}}%
\pgfpathlineto{\pgfqpoint{2.443562in}{4.056866in}}%
\pgfpathlineto{\pgfqpoint{2.464219in}{4.094149in}}%
\pgfpathlineto{\pgfqpoint{2.499663in}{4.341845in}}%
\pgfpathlineto{\pgfqpoint{2.519850in}{4.413543in}}%
\pgfpathlineto{\pgfqpoint{2.540037in}{4.289968in}}%
\pgfpathlineto{\pgfqpoint{2.560693in}{4.147068in}}%
\pgfpathlineto{\pgfqpoint{2.577827in}{4.061084in}}%
\pgfpathlineto{\pgfqpoint{2.599423in}{4.044294in}}%
\pgfpathlineto{\pgfqpoint{2.614680in}{4.037121in}}%
\pgfpathlineto{\pgfqpoint{2.635336in}{4.034718in}}%
\pgfpathlineto{\pgfqpoint{2.653646in}{4.036816in}}%
\pgfpathlineto{\pgfqpoint{2.675711in}{4.048588in}}%
\pgfpathlineto{\pgfqpoint{2.693550in}{4.085070in}}%
\pgfpathlineto{\pgfqpoint{2.713503in}{4.229668in}}%
\pgfpathlineto{\pgfqpoint{2.730402in}{4.392353in}}%
\pgfpathlineto{\pgfqpoint{2.750119in}{4.386604in}}%
\pgfpathlineto{\pgfqpoint{2.771011in}{4.215997in}}%
\pgfpathlineto{\pgfqpoint{2.794248in}{4.080716in}}%
\pgfpathlineto{\pgfqpoint{2.807627in}{4.053920in}}%
\pgfpathlineto{\pgfqpoint{2.827580in}{4.039310in}}%
\pgfpathlineto{\pgfqpoint{2.846125in}{4.035611in}}%
\pgfpathlineto{\pgfqpoint{2.867015in}{4.035545in}}%
\pgfpathlineto{\pgfqpoint{2.885558in}{4.040638in}}%
\pgfpathlineto{\pgfqpoint{2.906214in}{4.061751in}}%
\pgfpathlineto{\pgfqpoint{2.922410in}{4.095509in}}%
\pgfpathlineto{\pgfqpoint{2.943537in}{4.251253in}}%
\pgfpathlineto{\pgfqpoint{2.961611in}{4.398626in}}%
\pgfpathlineto{\pgfqpoint{2.980624in}{4.391425in}}%
\pgfpathlineto{\pgfqpoint{3.000811in}{4.284221in}}%
\pgfpathlineto{\pgfqpoint{3.020059in}{4.141787in}}%
\pgfpathlineto{\pgfqpoint{3.038602in}{4.088231in}}%
\pgfpathlineto{\pgfqpoint{3.059258in}{4.046178in}}%
\pgfpathlineto{\pgfqpoint{3.078505in}{4.037678in}}%
\pgfpathlineto{\pgfqpoint{3.098927in}{4.035030in}}%
\pgfpathlineto{\pgfqpoint{3.117237in}{4.037807in}}%
\pgfpathlineto{\pgfqpoint{3.137188in}{4.050252in}}%
\pgfpathlineto{\pgfqpoint{3.153384in}{4.070793in}}%
\pgfpathlineto{\pgfqpoint{3.173337in}{4.148724in}}%
\pgfpathlineto{\pgfqpoint{3.191176in}{4.324859in}}%
\pgfpathlineto{\pgfqpoint{3.211362in}{4.411678in}}%
\pgfpathlineto{\pgfqpoint{3.230141in}{4.408768in}}%
\pgfpathlineto{\pgfqpoint{3.251031in}{4.276463in}}%
\pgfpathlineto{\pgfqpoint{3.270750in}{4.123349in}}%
\pgfpathlineto{\pgfqpoint{3.287180in}{4.072097in}}%
\pgfpathlineto{\pgfqpoint{3.307602in}{4.047210in}}%
\pgfpathlineto{\pgfqpoint{3.325676in}{4.039234in}}%
\pgfpathlineto{\pgfqpoint{3.347037in}{4.035539in}}%
\pgfpathlineto{\pgfqpoint{3.365111in}{4.036251in}}%
\pgfpathlineto{\pgfqpoint{3.383890in}{4.038342in}}%
\pgfpathlineto{\pgfqpoint{3.404546in}{4.050221in}}%
\pgfpathlineto{\pgfqpoint{3.426610in}{4.088144in}}%
\pgfpathlineto{\pgfqpoint{3.441633in}{4.168352in}}%
\pgfpathlineto{\pgfqpoint{3.463463in}{4.384932in}}%
\pgfpathlineto{\pgfqpoint{3.482240in}{4.423583in}}%
\pgfpathlineto{\pgfqpoint{3.501253in}{4.360474in}}%
\pgfpathlineto{\pgfqpoint{3.518858in}{4.243586in}}%
\pgfpathlineto{\pgfqpoint{3.536697in}{4.113476in}}%
\pgfpathlineto{\pgfqpoint{3.558762in}{4.062521in}}%
\pgfpathlineto{\pgfqpoint{3.577072in}{4.050387in}}%
\pgfpathlineto{\pgfqpoint{3.597728in}{4.038776in}}%
\pgfpathlineto{\pgfqpoint{3.615333in}{4.035356in}}%
\pgfpathlineto{\pgfqpoint{3.633875in}{4.036780in}}%
\pgfpathlineto{\pgfqpoint{3.651951in}{4.041976in}}%
\pgfpathlineto{\pgfqpoint{3.673310in}{4.052001in}}%
\pgfpathlineto{\pgfqpoint{3.691618in}{4.058603in}}%
\pgfpathlineto{\pgfqpoint{3.711805in}{4.115533in}}%
\pgfpathlineto{\pgfqpoint{3.729645in}{4.188139in}}%
\pgfpathlineto{\pgfqpoint{3.750535in}{4.400349in}}%
\pgfpathlineto{\pgfqpoint{3.768611in}{4.433042in}}%
\pgfpathlineto{\pgfqpoint{3.789736in}{4.391897in}}%
\pgfpathlineto{\pgfqpoint{3.808046in}{4.267028in}}%
\pgfpathlineto{\pgfqpoint{3.825885in}{4.158720in}}%
\pgfpathlineto{\pgfqpoint{3.847479in}{4.072414in}}%
\pgfpathlineto{\pgfqpoint{3.865084in}{4.050450in}}%
\pgfpathlineto{\pgfqpoint{3.886445in}{4.040850in}}%
\pgfpathlineto{\pgfqpoint{3.906630in}{4.035729in}}%
\pgfpathlineto{\pgfqpoint{3.923063in}{4.036794in}}%
\pgfpathlineto{\pgfqpoint{3.940668in}{4.042972in}}%
\pgfpathlineto{\pgfqpoint{3.961324in}{4.058627in}}%
\pgfpathlineto{\pgfqpoint{3.979398in}{4.100161in}}%
\pgfpathlineto{\pgfqpoint{4.001228in}{4.225754in}}%
\pgfpathlineto{\pgfqpoint{4.018127in}{4.400194in}}%
\pgfpathlineto{\pgfqpoint{4.036672in}{4.435278in}}%
\pgfpathlineto{\pgfqpoint{4.058031in}{4.439681in}}%
\pgfpathlineto{\pgfqpoint{4.077044in}{4.249830in}}%
\pgfpathlineto{\pgfqpoint{4.093710in}{4.434630in}}%
\pgfpathlineto{\pgfqpoint{4.115540in}{4.431362in}}%
\pgfpathlineto{\pgfqpoint{4.133145in}{4.354216in}}%
\pgfpathlineto{\pgfqpoint{4.154035in}{4.179951in}}%
\pgfpathlineto{\pgfqpoint{4.174691in}{4.087490in}}%
\pgfpathlineto{\pgfqpoint{4.190653in}{4.058905in}}%
\pgfpathlineto{\pgfqpoint{4.211544in}{4.041625in}}%
\pgfpathlineto{\pgfqpoint{4.229619in}{4.037457in}}%
\pgfpathlineto{\pgfqpoint{4.251684in}{4.037668in}}%
\pgfpathlineto{\pgfqpoint{4.268584in}{4.044842in}}%
\pgfpathlineto{\pgfqpoint{4.287362in}{4.063886in}}%
\pgfpathlineto{\pgfqpoint{4.308019in}{4.135647in}}%
\pgfpathlineto{\pgfqpoint{4.326092in}{4.257902in}}%
\pgfpathlineto{\pgfqpoint{4.344402in}{4.437790in}}%
\pgfpathlineto{\pgfqpoint{4.364588in}{4.457282in}}%
\pgfpathlineto{\pgfqpoint{4.383132in}{4.427180in}}%
\pgfpathlineto{\pgfqpoint{4.422801in}{4.179348in}}%
\pgfpathlineto{\pgfqpoint{4.442752in}{4.083215in}}%
\pgfpathlineto{\pgfqpoint{4.461297in}{4.052579in}}%
\pgfpathlineto{\pgfqpoint{4.478667in}{4.041296in}}%
\pgfpathlineto{\pgfqpoint{4.473973in}{4.044789in}}%
\pgfpathlineto{\pgfqpoint{4.455663in}{4.071943in}}%
\pgfpathlineto{\pgfqpoint{4.436181in}{4.201436in}}%
\pgfpathlineto{\pgfqpoint{4.417636in}{4.394671in}}%
\pgfpathlineto{\pgfqpoint{4.397920in}{4.458569in}}%
\pgfpathlineto{\pgfqpoint{4.378438in}{4.320417in}}%
\pgfpathlineto{\pgfqpoint{4.358250in}{4.096606in}}%
\pgfpathlineto{\pgfqpoint{4.338768in}{4.051655in}}%
\pgfpathlineto{\pgfqpoint{4.321867in}{4.038526in}}%
\pgfpathlineto{\pgfqpoint{4.302619in}{4.037738in}}%
\pgfpathlineto{\pgfqpoint{4.280789in}{4.053416in}}%
\pgfpathlineto{\pgfqpoint{4.263655in}{4.107342in}}%
\pgfpathlineto{\pgfqpoint{4.244876in}{4.304763in}}%
\pgfpathlineto{\pgfqpoint{4.222811in}{4.440327in}}%
\pgfpathlineto{\pgfqpoint{4.204738in}{4.418384in}}%
\pgfpathlineto{\pgfqpoint{4.188071in}{4.178689in}}%
\pgfpathlineto{\pgfqpoint{4.162720in}{4.057131in}}%
\pgfpathlineto{\pgfqpoint{4.148872in}{4.042389in}}%
\pgfpathlineto{\pgfqpoint{4.129859in}{4.035900in}}%
\pgfpathlineto{\pgfqpoint{4.108263in}{4.041420in}}%
\pgfpathlineto{\pgfqpoint{4.090424in}{4.064867in}}%
\pgfpathlineto{\pgfqpoint{4.066247in}{4.218706in}}%
\pgfpathlineto{\pgfqpoint{4.053806in}{4.373936in}}%
\pgfpathlineto{\pgfqpoint{4.031741in}{4.438000in}}%
\pgfpathlineto{\pgfqpoint{4.011556in}{4.243618in}}%
\pgfpathlineto{\pgfqpoint{3.993951in}{4.092374in}}%
\pgfpathlineto{\pgfqpoint{3.976581in}{4.049690in}}%
\pgfpathlineto{\pgfqpoint{3.955690in}{4.036969in}}%
\pgfpathlineto{\pgfqpoint{3.936911in}{4.036505in}}%
\pgfpathlineto{\pgfqpoint{3.917898in}{4.047475in}}%
\pgfpathlineto{\pgfqpoint{3.898885in}{4.072601in}}%
\pgfpathlineto{\pgfqpoint{3.879168in}{4.111929in}}%
\pgfpathlineto{\pgfqpoint{3.860858in}{4.278302in}}%
\pgfpathlineto{\pgfqpoint{3.840202in}{4.420295in}}%
\pgfpathlineto{\pgfqpoint{3.819782in}{4.374456in}}%
\pgfpathlineto{\pgfqpoint{3.801707in}{4.149508in}}%
\pgfpathlineto{\pgfqpoint{3.783164in}{4.066193in}}%
\pgfpathlineto{\pgfqpoint{3.763680in}{4.041146in}}%
\pgfpathlineto{\pgfqpoint{3.740207in}{4.035178in}}%
\pgfpathlineto{\pgfqpoint{3.724716in}{4.036945in}}%
\pgfpathlineto{\pgfqpoint{3.704763in}{4.051442in}}%
\pgfpathlineto{\pgfqpoint{3.686689in}{4.105852in}}%
\pgfpathlineto{\pgfqpoint{3.668147in}{4.264146in}}%
\pgfpathlineto{\pgfqpoint{3.648428in}{4.410591in}}%
\pgfpathlineto{\pgfqpoint{3.626129in}{4.370286in}}%
\pgfpathlineto{\pgfqpoint{3.608290in}{4.162730in}}%
\pgfpathlineto{\pgfqpoint{3.590685in}{4.072961in}}%
\pgfpathlineto{\pgfqpoint{3.572612in}{4.046081in}}%
\pgfpathlineto{\pgfqpoint{3.552893in}{4.036061in}}%
\pgfpathlineto{\pgfqpoint{3.534116in}{4.035455in}}%
\pgfpathlineto{\pgfqpoint{3.512755in}{4.043745in}}%
\pgfpathlineto{\pgfqpoint{3.495385in}{4.073507in}}%
\pgfpathlineto{\pgfqpoint{3.473320in}{4.156310in}}%
\pgfpathlineto{\pgfqpoint{3.455012in}{4.313544in}}%
\pgfpathlineto{\pgfqpoint{3.436938in}{4.417035in}}%
\pgfpathlineto{\pgfqpoint{3.415108in}{4.406789in}}%
\pgfpathlineto{\pgfqpoint{3.395390in}{4.171574in}}%
\pgfpathlineto{\pgfqpoint{3.376613in}{4.069121in}}%
\pgfpathlineto{\pgfqpoint{3.359477in}{4.044861in}}%
\pgfpathlineto{\pgfqpoint{3.340229in}{4.036759in}}%
\pgfpathlineto{\pgfqpoint{3.321450in}{4.034857in}}%
\pgfpathlineto{\pgfqpoint{3.302672in}{4.038360in}}%
\pgfpathlineto{\pgfqpoint{3.285069in}{4.048246in}}%
\pgfpathlineto{\pgfqpoint{3.264881in}{4.097986in}}%
\pgfpathlineto{\pgfqpoint{3.242348in}{4.273188in}}%
\pgfpathlineto{\pgfqpoint{3.224743in}{4.384279in}}%
\pgfpathlineto{\pgfqpoint{3.206199in}{4.412832in}}%
\pgfpathlineto{\pgfqpoint{3.188828in}{4.243803in}}%
\pgfpathlineto{\pgfqpoint{3.166295in}{4.097714in}}%
\pgfpathlineto{\pgfqpoint{3.146813in}{4.050707in}}%
\pgfpathlineto{\pgfqpoint{3.128737in}{4.037870in}}%
\pgfpathlineto{\pgfqpoint{3.109960in}{4.034733in}}%
\pgfpathlineto{\pgfqpoint{3.091885in}{4.036696in}}%
\pgfpathlineto{\pgfqpoint{3.072403in}{4.040872in}}%
\pgfpathlineto{\pgfqpoint{3.051981in}{4.068886in}}%
\pgfpathlineto{\pgfqpoint{3.030151in}{4.162327in}}%
\pgfpathlineto{\pgfqpoint{3.013954in}{4.305953in}}%
\pgfpathlineto{\pgfqpoint{2.994707in}{4.412778in}}%
\pgfpathlineto{\pgfqpoint{2.977573in}{4.356444in}}%
\pgfpathlineto{\pgfqpoint{2.958091in}{4.139923in}}%
\pgfpathlineto{\pgfqpoint{2.936495in}{4.059112in}}%
\pgfpathlineto{\pgfqpoint{2.917247in}{4.041414in}}%
\pgfpathlineto{\pgfqpoint{2.898703in}{4.035435in}}%
\pgfpathlineto{\pgfqpoint{2.879926in}{4.034900in}}%
\pgfpathlineto{\pgfqpoint{2.858096in}{4.035740in}}%
\pgfpathlineto{\pgfqpoint{2.839786in}{4.039526in}}%
\pgfpathlineto{\pgfqpoint{2.821243in}{4.052449in}}%
\pgfpathlineto{\pgfqpoint{2.804107in}{4.092645in}}%
\pgfpathlineto{\pgfqpoint{2.784391in}{4.255058in}}%
\pgfpathlineto{\pgfqpoint{2.762795in}{4.404317in}}%
\pgfpathlineto{\pgfqpoint{2.743078in}{4.404247in}}%
\pgfpathlineto{\pgfqpoint{2.725003in}{4.211421in}}%
\pgfpathlineto{\pgfqpoint{2.705990in}{4.083295in}}%
\pgfpathlineto{\pgfqpoint{2.687682in}{4.048888in}}%
\pgfpathlineto{\pgfqpoint{2.667729in}{4.038635in}}%
\pgfpathlineto{\pgfqpoint{2.649655in}{4.034720in}}%
\pgfpathlineto{\pgfqpoint{2.628764in}{4.036884in}}%
\pgfpathlineto{\pgfqpoint{2.608108in}{4.046200in}}%
\pgfpathlineto{\pgfqpoint{2.589095in}{4.072722in}}%
\pgfpathlineto{\pgfqpoint{2.574307in}{4.134899in}}%
\pgfpathlineto{\pgfqpoint{2.552008in}{4.284300in}}%
\pgfpathlineto{\pgfqpoint{2.532760in}{4.362379in}}%
\pgfpathlineto{\pgfqpoint{2.515156in}{4.406307in}}%
\pgfpathlineto{\pgfqpoint{2.495439in}{4.351061in}}%
\pgfpathlineto{\pgfqpoint{2.474312in}{4.154741in}}%
\pgfpathlineto{\pgfqpoint{2.457178in}{4.111292in}}%
\pgfpathlineto{\pgfqpoint{2.438399in}{4.055003in}}%
\pgfpathlineto{\pgfqpoint{2.417038in}{4.038464in}}%
\pgfpathlineto{\pgfqpoint{2.398261in}{4.035367in}}%
\pgfpathlineto{\pgfqpoint{2.381125in}{4.035225in}}%
\pgfpathlineto{\pgfqpoint{2.357886in}{4.042683in}}%
\pgfpathlineto{\pgfqpoint{2.339813in}{4.060546in}}%
\pgfpathlineto{\pgfqpoint{2.323148in}{4.144788in}}%
\pgfpathlineto{\pgfqpoint{2.300378in}{4.330258in}}%
\pgfpathlineto{\pgfqpoint{2.282304in}{4.407017in}}%
\pgfpathlineto{\pgfqpoint{2.265168in}{4.386294in}}%
\pgfpathlineto{\pgfqpoint{2.241226in}{4.134268in}}%
\pgfpathlineto{\pgfqpoint{2.229021in}{4.097834in}}%
\pgfpathlineto{\pgfqpoint{2.208599in}{4.057554in}}%
\pgfpathlineto{\pgfqpoint{2.185361in}{4.039300in}}%
\pgfpathlineto{\pgfqpoint{2.170338in}{4.035515in}}%
\pgfpathlineto{\pgfqpoint{2.148274in}{4.036022in}}%
\pgfpathlineto{\pgfqpoint{2.129729in}{4.039910in}}%
\pgfpathlineto{\pgfqpoint{2.110247in}{4.050486in}}%
\pgfpathlineto{\pgfqpoint{2.090060in}{4.088612in}}%
\pgfpathlineto{\pgfqpoint{2.071517in}{4.205271in}}%
\pgfpathlineto{\pgfqpoint{2.052504in}{4.338462in}}%
\pgfpathlineto{\pgfqpoint{2.034665in}{4.408437in}}%
\pgfpathlineto{\pgfqpoint{2.016589in}{4.360346in}}%
\pgfpathlineto{\pgfqpoint{1.976686in}{4.109376in}}%
\pgfpathlineto{\pgfqpoint{1.956735in}{4.055219in}}%
\pgfpathlineto{\pgfqpoint{1.938190in}{4.041152in}}%
\pgfpathlineto{\pgfqpoint{1.918239in}{4.035841in}}%
\pgfpathlineto{\pgfqpoint{1.898052in}{4.035193in}}%
\pgfpathlineto{\pgfqpoint{1.879508in}{4.036578in}}%
\pgfpathlineto{\pgfqpoint{1.860731in}{4.042912in}}%
\pgfpathlineto{\pgfqpoint{1.839604in}{4.068642in}}%
\pgfpathlineto{\pgfqpoint{1.824113in}{4.134978in}}%
\pgfpathlineto{\pgfqpoint{1.802517in}{4.291805in}}%
\pgfpathlineto{\pgfqpoint{1.784209in}{4.403094in}}%
\pgfpathlineto{\pgfqpoint{1.764727in}{4.412704in}}%
\pgfpathlineto{\pgfqpoint{1.727874in}{4.142545in}}%
\pgfpathlineto{\pgfqpoint{1.706513in}{4.073882in}}%
\pgfpathlineto{\pgfqpoint{1.688205in}{4.047339in}}%
\pgfpathlineto{\pgfqpoint{1.668957in}{4.039668in}}%
\pgfpathlineto{\pgfqpoint{1.651587in}{4.035545in}}%
\pgfpathlineto{\pgfqpoint{1.629991in}{4.035649in}}%
\pgfpathlineto{\pgfqpoint{1.593373in}{4.044529in}}%
\pgfpathlineto{\pgfqpoint{1.571308in}{4.069314in}}%
\pgfpathlineto{\pgfqpoint{1.553938in}{4.141280in}}%
\pgfpathlineto{\pgfqpoint{1.534690in}{4.290173in}}%
\pgfpathlineto{\pgfqpoint{1.516382in}{4.385364in}}%
\pgfpathlineto{\pgfqpoint{1.495257in}{4.423206in}}%
\pgfpathlineto{\pgfqpoint{1.479530in}{4.375322in}}%
\pgfpathlineto{\pgfqpoint{1.457465in}{4.195796in}}%
\pgfpathlineto{\pgfqpoint{1.436104in}{4.097132in}}%
\pgfpathlineto{\pgfqpoint{1.421081in}{4.065766in}}%
\pgfpathlineto{\pgfqpoint{1.398782in}{4.046146in}}%
\pgfpathlineto{\pgfqpoint{1.381412in}{4.038081in}}%
\pgfpathlineto{\pgfqpoint{1.362635in}{4.035858in}}%
\pgfpathlineto{\pgfqpoint{1.340571in}{4.035908in}}%
\pgfpathlineto{\pgfqpoint{1.322729in}{4.040910in}}%
\pgfpathlineto{\pgfqpoint{1.306299in}{4.049145in}}%
\pgfpathlineto{\pgfqpoint{1.284470in}{4.088525in}}%
\pgfpathlineto{\pgfqpoint{1.266160in}{4.140315in}}%
\pgfpathlineto{\pgfqpoint{1.226725in}{4.394462in}}%
\pgfpathlineto{\pgfqpoint{1.208886in}{4.428155in}}%
\pgfpathlineto{\pgfqpoint{1.190344in}{4.424265in}}%
\pgfpathlineto{\pgfqpoint{1.168748in}{4.248027in}}%
\pgfpathlineto{\pgfqpoint{1.150909in}{4.140572in}}%
\pgfpathlineto{\pgfqpoint{1.130018in}{4.069973in}}%
\pgfpathlineto{\pgfqpoint{1.111708in}{4.047073in}}%
\pgfpathlineto{\pgfqpoint{1.074621in}{4.036403in}}%
\pgfpathlineto{\pgfqpoint{1.053025in}{4.037250in}}%
\pgfpathlineto{\pgfqpoint{1.035186in}{4.040080in}}%
\pgfpathlineto{\pgfqpoint{1.016173in}{4.052512in}}%
\pgfpathlineto{\pgfqpoint{0.995282in}{4.075383in}}%
\pgfpathlineto{\pgfqpoint{0.977912in}{4.139618in}}%
\pgfpathlineto{\pgfqpoint{0.958430in}{4.285096in}}%
\pgfpathlineto{\pgfqpoint{0.939182in}{4.369264in}}%
\pgfpathlineto{\pgfqpoint{0.914066in}{4.443148in}}%
\pgfpathlineto{\pgfqpoint{0.895993in}{4.425285in}}%
\pgfpathlineto{\pgfqpoint{0.880500in}{4.333795in}}%
\pgfpathlineto{\pgfqpoint{0.861486in}{4.159333in}}%
\pgfpathlineto{\pgfqpoint{0.843178in}{4.079638in}}%
\pgfpathlineto{\pgfqpoint{0.825573in}{4.056910in}}%
\pgfpathlineto{\pgfqpoint{0.806091in}{4.042535in}}%
\pgfpathlineto{\pgfqpoint{0.782618in}{4.153707in}}%
\pgfpathlineto{\pgfqpoint{0.767830in}{4.118699in}}%
\pgfpathlineto{\pgfqpoint{0.747878in}{4.061202in}}%
\pgfpathlineto{\pgfqpoint{0.729101in}{4.046443in}}%
\pgfpathlineto{\pgfqpoint{0.708679in}{4.037191in}}%
\pgfpathlineto{\pgfqpoint{0.687318in}{4.038358in}}%
\pgfpathlineto{\pgfqpoint{0.670418in}{4.044527in}}%
\pgfpathlineto{\pgfqpoint{0.649525in}{4.070901in}}%
\pgfpathlineto{\pgfqpoint{0.649996in}{4.067826in}}%
\pgfpathlineto{\pgfqpoint{0.657742in}{4.053540in}}%
\pgfpathlineto{\pgfqpoint{0.676990in}{4.038412in}}%
\pgfpathlineto{\pgfqpoint{0.696003in}{4.037781in}}%
\pgfpathlineto{\pgfqpoint{0.711730in}{4.047945in}}%
\pgfpathlineto{\pgfqpoint{0.734733in}{4.103006in}}%
\pgfpathlineto{\pgfqpoint{0.753511in}{4.278948in}}%
\pgfpathlineto{\pgfqpoint{0.773228in}{4.448179in}}%
\pgfpathlineto{\pgfqpoint{0.789661in}{4.083310in}}%
\pgfpathlineto{\pgfqpoint{0.808908in}{4.044836in}}%
\pgfpathlineto{\pgfqpoint{0.830268in}{4.035930in}}%
\pgfpathlineto{\pgfqpoint{0.850221in}{4.042588in}}%
\pgfpathlineto{\pgfqpoint{0.868060in}{4.063921in}}%
\pgfpathlineto{\pgfqpoint{0.887542in}{4.157410in}}%
\pgfpathlineto{\pgfqpoint{0.911015in}{4.426254in}}%
\pgfpathlineto{\pgfqpoint{0.925100in}{4.421669in}}%
\pgfpathlineto{\pgfqpoint{0.944347in}{4.280702in}}%
\pgfpathlineto{\pgfqpoint{0.962421in}{4.111748in}}%
\pgfpathlineto{\pgfqpoint{0.984954in}{4.046764in}}%
\pgfpathlineto{\pgfqpoint{1.001856in}{4.036412in}}%
\pgfpathlineto{\pgfqpoint{1.021104in}{4.037594in}}%
\pgfpathlineto{\pgfqpoint{1.040586in}{4.051209in}}%
\pgfpathlineto{\pgfqpoint{1.058894in}{4.093514in}}%
\pgfpathlineto{\pgfqpoint{1.078847in}{4.267356in}}%
\pgfpathlineto{\pgfqpoint{1.102085in}{4.432584in}}%
\pgfpathlineto{\pgfqpoint{1.118047in}{4.324785in}}%
\pgfpathlineto{\pgfqpoint{1.138467in}{4.153868in}}%
\pgfpathlineto{\pgfqpoint{1.154663in}{4.067764in}}%
\pgfpathlineto{\pgfqpoint{1.175319in}{4.039808in}}%
\pgfpathlineto{\pgfqpoint{1.196915in}{4.035290in}}%
\pgfpathlineto{\pgfqpoint{1.215694in}{4.039911in}}%
\pgfpathlineto{\pgfqpoint{1.234237in}{4.051316in}}%
\pgfpathlineto{\pgfqpoint{1.252547in}{4.104812in}}%
\pgfpathlineto{\pgfqpoint{1.271089in}{4.312334in}}%
\pgfpathlineto{\pgfqpoint{1.291981in}{4.427653in}}%
\pgfpathlineto{\pgfqpoint{1.314046in}{4.318421in}}%
\pgfpathlineto{\pgfqpoint{1.333294in}{4.149067in}}%
\pgfpathlineto{\pgfqpoint{1.349019in}{4.064928in}}%
\pgfpathlineto{\pgfqpoint{1.367329in}{4.042651in}}%
\pgfpathlineto{\pgfqpoint{1.386106in}{4.035643in}}%
\pgfpathlineto{\pgfqpoint{1.407936in}{4.036686in}}%
\pgfpathlineto{\pgfqpoint{1.426950in}{4.044739in}}%
\pgfpathlineto{\pgfqpoint{1.443851in}{4.064633in}}%
\pgfpathlineto{\pgfqpoint{1.464273in}{4.157911in}}%
\pgfpathlineto{\pgfqpoint{1.481641in}{4.345654in}}%
\pgfpathlineto{\pgfqpoint{1.501829in}{4.420628in}}%
\pgfpathlineto{\pgfqpoint{1.520608in}{4.343853in}}%
\pgfpathlineto{\pgfqpoint{1.539621in}{4.223252in}}%
\pgfpathlineto{\pgfqpoint{1.562623in}{4.068873in}}%
\pgfpathlineto{\pgfqpoint{1.579290in}{4.048730in}}%
\pgfpathlineto{\pgfqpoint{1.598772in}{4.037548in}}%
\pgfpathlineto{\pgfqpoint{1.619663in}{4.035156in}}%
\pgfpathlineto{\pgfqpoint{1.637737in}{4.038694in}}%
\pgfpathlineto{\pgfqpoint{1.656281in}{4.051364in}}%
\pgfpathlineto{\pgfqpoint{1.673651in}{4.095180in}}%
\pgfpathlineto{\pgfqpoint{1.691959in}{4.154034in}}%
\pgfpathlineto{\pgfqpoint{1.712381in}{4.361236in}}%
\pgfpathlineto{\pgfqpoint{1.731629in}{4.416082in}}%
\pgfpathlineto{\pgfqpoint{1.752990in}{4.294406in}}%
\pgfpathlineto{\pgfqpoint{1.771064in}{4.153231in}}%
\pgfpathlineto{\pgfqpoint{1.789137in}{4.068837in}}%
\pgfpathlineto{\pgfqpoint{1.809794in}{4.049758in}}%
\pgfpathlineto{\pgfqpoint{1.830215in}{4.036982in}}%
\pgfpathlineto{\pgfqpoint{1.848289in}{4.034900in}}%
\pgfpathlineto{\pgfqpoint{1.865659in}{4.037071in}}%
\pgfpathlineto{\pgfqpoint{1.887724in}{4.045083in}}%
\pgfpathlineto{\pgfqpoint{1.904155in}{4.069189in}}%
\pgfpathlineto{\pgfqpoint{1.922699in}{4.161587in}}%
\pgfpathlineto{\pgfqpoint{1.941712in}{4.157740in}}%
\pgfpathlineto{\pgfqpoint{1.964480in}{4.399520in}}%
\pgfpathlineto{\pgfqpoint{1.984197in}{4.401841in}}%
\pgfpathlineto{\pgfqpoint{1.999924in}{4.329805in}}%
\pgfpathlineto{\pgfqpoint{2.021286in}{4.131320in}}%
\pgfpathlineto{\pgfqpoint{2.042176in}{4.059110in}}%
\pgfpathlineto{\pgfqpoint{2.058372in}{4.040656in}}%
\pgfpathlineto{\pgfqpoint{2.079263in}{4.035232in}}%
\pgfpathlineto{\pgfqpoint{2.096633in}{4.035093in}}%
\pgfpathlineto{\pgfqpoint{2.117290in}{4.039076in}}%
\pgfpathlineto{\pgfqpoint{2.135129in}{4.051582in}}%
\pgfpathlineto{\pgfqpoint{2.155316in}{4.083626in}}%
\pgfpathlineto{\pgfqpoint{2.174329in}{4.201859in}}%
\pgfpathlineto{\pgfqpoint{2.195220in}{4.399917in}}%
\pgfpathlineto{\pgfqpoint{2.213059in}{4.405824in}}%
\pgfpathlineto{\pgfqpoint{2.231602in}{4.348530in}}%
\pgfpathlineto{\pgfqpoint{2.252258in}{4.153661in}}%
\pgfpathlineto{\pgfqpoint{2.269394in}{4.076653in}}%
\pgfpathlineto{\pgfqpoint{2.290989in}{4.041166in}}%
\pgfpathlineto{\pgfqpoint{2.311646in}{4.036495in}}%
\pgfpathlineto{\pgfqpoint{2.328545in}{4.034595in}}%
\pgfpathlineto{\pgfqpoint{2.348264in}{4.037817in}}%
\pgfpathlineto{\pgfqpoint{2.368449in}{4.052656in}}%
\pgfpathlineto{\pgfqpoint{2.387462in}{4.082156in}}%
\pgfpathlineto{\pgfqpoint{2.406710in}{4.166599in}}%
\pgfpathlineto{\pgfqpoint{2.425723in}{4.237589in}}%
\pgfpathlineto{\pgfqpoint{2.440511in}{4.392054in}}%
\pgfpathlineto{\pgfqpoint{2.462107in}{4.381511in}}%
\pgfpathlineto{\pgfqpoint{2.483937in}{4.238265in}}%
\pgfpathlineto{\pgfqpoint{2.500837in}{4.111222in}}%
\pgfpathlineto{\pgfqpoint{2.518676in}{4.054960in}}%
\pgfpathlineto{\pgfqpoint{2.539098in}{4.168880in}}%
\pgfpathlineto{\pgfqpoint{2.558816in}{4.085426in}}%
\pgfpathlineto{\pgfqpoint{2.576419in}{4.047959in}}%
\pgfpathlineto{\pgfqpoint{2.617497in}{4.035044in}}%
\pgfpathlineto{\pgfqpoint{2.635572in}{4.036095in}}%
\pgfpathlineto{\pgfqpoint{2.652941in}{4.044895in}}%
\pgfpathlineto{\pgfqpoint{2.675242in}{4.076761in}}%
\pgfpathlineto{\pgfqpoint{2.696132in}{4.184002in}}%
\pgfpathlineto{\pgfqpoint{2.710686in}{4.365533in}}%
\pgfpathlineto{\pgfqpoint{2.732514in}{4.399787in}}%
\pgfpathlineto{\pgfqpoint{2.749884in}{4.283495in}}%
\pgfpathlineto{\pgfqpoint{2.772420in}{4.116938in}}%
\pgfpathlineto{\pgfqpoint{2.788380in}{4.060916in}}%
\pgfpathlineto{\pgfqpoint{2.809507in}{4.043368in}}%
\pgfpathlineto{\pgfqpoint{2.828049in}{4.036662in}}%
\pgfpathlineto{\pgfqpoint{2.846828in}{4.034896in}}%
\pgfpathlineto{\pgfqpoint{2.867484in}{4.036260in}}%
\pgfpathlineto{\pgfqpoint{2.886497in}{4.043308in}}%
\pgfpathlineto{\pgfqpoint{2.903163in}{4.068688in}}%
\pgfpathlineto{\pgfqpoint{2.921473in}{4.138117in}}%
\pgfpathlineto{\pgfqpoint{2.941658in}{4.344416in}}%
\pgfpathlineto{\pgfqpoint{2.963254in}{4.411395in}}%
\pgfpathlineto{\pgfqpoint{2.981327in}{4.345110in}}%
\pgfpathlineto{\pgfqpoint{2.999637in}{4.229670in}}%
\pgfpathlineto{\pgfqpoint{3.021702in}{4.099476in}}%
\pgfpathlineto{\pgfqpoint{3.039541in}{4.122423in}}%
\pgfpathlineto{\pgfqpoint{3.057849in}{4.064289in}}%
\pgfpathlineto{\pgfqpoint{3.078505in}{4.041107in}}%
\pgfpathlineto{\pgfqpoint{3.094936in}{4.036187in}}%
\pgfpathlineto{\pgfqpoint{3.118646in}{4.036051in}}%
\pgfpathlineto{\pgfqpoint{3.134137in}{4.040551in}}%
\pgfpathlineto{\pgfqpoint{3.155498in}{4.057894in}}%
\pgfpathlineto{\pgfqpoint{3.173806in}{4.100586in}}%
\pgfpathlineto{\pgfqpoint{3.191176in}{4.208919in}}%
\pgfpathlineto{\pgfqpoint{3.213241in}{4.408817in}}%
\pgfpathlineto{\pgfqpoint{3.230141in}{4.406674in}}%
\pgfpathlineto{\pgfqpoint{3.248919in}{4.318787in}}%
\pgfpathlineto{\pgfqpoint{3.272861in}{4.123695in}}%
\pgfpathlineto{\pgfqpoint{3.291406in}{4.066371in}}%
\pgfpathlineto{\pgfqpoint{3.309011in}{4.045161in}}%
\pgfpathlineto{\pgfqpoint{3.327789in}{4.037519in}}%
\pgfpathlineto{\pgfqpoint{3.347975in}{4.035316in}}%
\pgfpathlineto{\pgfqpoint{3.388113in}{4.041225in}}%
\pgfpathlineto{\pgfqpoint{3.405015in}{4.054059in}}%
\pgfpathlineto{\pgfqpoint{3.423793in}{4.062971in}}%
\pgfpathlineto{\pgfqpoint{3.444215in}{4.088716in}}%
\pgfpathlineto{\pgfqpoint{3.462289in}{4.198253in}}%
\pgfpathlineto{\pgfqpoint{3.480128in}{4.397590in}}%
\pgfpathlineto{\pgfqpoint{3.501019in}{4.414057in}}%
\pgfpathlineto{\pgfqpoint{3.518858in}{4.319213in}}%
\pgfpathlineto{\pgfqpoint{3.537637in}{4.162591in}}%
\pgfpathlineto{\pgfqpoint{3.559701in}{4.070875in}}%
\pgfpathlineto{\pgfqpoint{3.579418in}{4.049659in}}%
\pgfpathlineto{\pgfqpoint{3.597728in}{4.039005in}}%
\pgfpathlineto{\pgfqpoint{3.615567in}{4.035690in}}%
\pgfpathlineto{\pgfqpoint{3.634346in}{4.036565in}}%
\pgfpathlineto{\pgfqpoint{3.652419in}{4.041850in}}%
\pgfpathlineto{\pgfqpoint{3.672841in}{4.053423in}}%
\pgfpathlineto{\pgfqpoint{3.691618in}{4.086856in}}%
\pgfpathlineto{\pgfqpoint{3.711571in}{4.188847in}}%
\pgfpathlineto{\pgfqpoint{3.729645in}{4.340557in}}%
\pgfpathlineto{\pgfqpoint{3.750301in}{4.433899in}}%
\pgfpathlineto{\pgfqpoint{3.770019in}{4.406447in}}%
\pgfpathlineto{\pgfqpoint{3.787153in}{4.308087in}}%
\pgfpathlineto{\pgfqpoint{3.808280in}{4.146554in}}%
\pgfpathlineto{\pgfqpoint{3.826588in}{4.085843in}}%
\pgfpathlineto{\pgfqpoint{3.847713in}{4.048755in}}%
\pgfpathlineto{\pgfqpoint{3.866963in}{4.044875in}}%
\pgfpathlineto{\pgfqpoint{3.885036in}{4.038342in}}%
\pgfpathlineto{\pgfqpoint{3.905458in}{4.035591in}}%
\pgfpathlineto{\pgfqpoint{3.922592in}{4.037906in}}%
\pgfpathlineto{\pgfqpoint{3.940902in}{4.041623in}}%
\pgfpathlineto{\pgfqpoint{3.962496in}{4.049441in}}%
\pgfpathlineto{\pgfqpoint{4.000991in}{4.086227in}}%
\pgfpathlineto{\pgfqpoint{4.018596in}{4.048418in}}%
\pgfpathlineto{\pgfqpoint{4.037375in}{4.039069in}}%
\pgfpathlineto{\pgfqpoint{4.058266in}{4.036278in}}%
\pgfpathlineto{\pgfqpoint{4.076810in}{4.042442in}}%
\pgfpathlineto{\pgfqpoint{4.095118in}{4.059893in}}%
\pgfpathlineto{\pgfqpoint{4.115774in}{4.129580in}}%
\pgfpathlineto{\pgfqpoint{4.134553in}{4.307324in}}%
\pgfpathlineto{\pgfqpoint{4.153801in}{4.442798in}}%
\pgfpathlineto{\pgfqpoint{4.170466in}{4.428777in}}%
\pgfpathlineto{\pgfqpoint{4.191827in}{4.299505in}}%
\pgfpathlineto{\pgfqpoint{4.209666in}{4.147468in}}%
\pgfpathlineto{\pgfqpoint{4.230793in}{4.071034in}}%
\pgfpathlineto{\pgfqpoint{4.249570in}{4.046608in}}%
\pgfpathlineto{\pgfqpoint{4.271400in}{4.036841in}}%
\pgfpathlineto{\pgfqpoint{4.289005in}{4.037115in}}%
\pgfpathlineto{\pgfqpoint{4.306141in}{4.043039in}}%
\pgfpathlineto{\pgfqpoint{4.327032in}{4.061350in}}%
\pgfpathlineto{\pgfqpoint{4.344871in}{4.114715in}}%
\pgfpathlineto{\pgfqpoint{4.363650in}{4.212360in}}%
\pgfpathlineto{\pgfqpoint{4.385244in}{4.434845in}}%
\pgfpathlineto{\pgfqpoint{4.403319in}{4.456283in}}%
\pgfpathlineto{\pgfqpoint{4.420922in}{4.397874in}}%
\pgfpathlineto{\pgfqpoint{4.440406in}{4.250746in}}%
\pgfpathlineto{\pgfqpoint{4.462000in}{4.133296in}}%
\pgfpathlineto{\pgfqpoint{4.481013in}{4.066784in}}%
\pgfpathlineto{\pgfqpoint{4.474676in}{4.091434in}}%
\pgfpathlineto{\pgfqpoint{4.455194in}{4.243504in}}%
\pgfpathlineto{\pgfqpoint{4.435712in}{4.187430in}}%
\pgfpathlineto{\pgfqpoint{4.417871in}{4.063991in}}%
\pgfpathlineto{\pgfqpoint{4.399094in}{4.039958in}}%
\pgfpathlineto{\pgfqpoint{4.377498in}{4.037389in}}%
\pgfpathlineto{\pgfqpoint{4.360128in}{4.046705in}}%
\pgfpathlineto{\pgfqpoint{4.342757in}{4.096232in}}%
\pgfpathlineto{\pgfqpoint{4.298628in}{4.442549in}}%
\pgfpathlineto{\pgfqpoint{4.281963in}{4.429934in}}%
\pgfpathlineto{\pgfqpoint{4.264593in}{4.156562in}}%
\pgfpathlineto{\pgfqpoint{4.244407in}{4.070136in}}%
\pgfpathlineto{\pgfqpoint{4.227271in}{4.043315in}}%
\pgfpathlineto{\pgfqpoint{4.205675in}{4.036016in}}%
\pgfpathlineto{\pgfqpoint{4.185490in}{4.040601in}}%
\pgfpathlineto{\pgfqpoint{4.168589in}{4.062767in}}%
\pgfpathlineto{\pgfqpoint{4.147698in}{4.176705in}}%
\pgfpathlineto{\pgfqpoint{4.127511in}{4.388519in}}%
\pgfpathlineto{\pgfqpoint{4.109437in}{4.441823in}}%
\pgfpathlineto{\pgfqpoint{4.091598in}{4.311722in}}%
\pgfpathlineto{\pgfqpoint{4.071645in}{4.094906in}}%
\pgfpathlineto{\pgfqpoint{4.050520in}{4.046998in}}%
\pgfpathlineto{\pgfqpoint{4.033855in}{4.036628in}}%
\pgfpathlineto{\pgfqpoint{4.015310in}{4.036577in}}%
\pgfpathlineto{\pgfqpoint{3.992777in}{4.048474in}}%
\pgfpathlineto{\pgfqpoint{3.974938in}{4.097206in}}%
\pgfpathlineto{\pgfqpoint{3.953342in}{4.267211in}}%
\pgfpathlineto{\pgfqpoint{3.936442in}{4.409251in}}%
\pgfpathlineto{\pgfqpoint{3.918838in}{4.426902in}}%
\pgfpathlineto{\pgfqpoint{3.897007in}{4.196673in}}%
\pgfpathlineto{\pgfqpoint{3.877994in}{4.073104in}}%
\pgfpathlineto{\pgfqpoint{3.861563in}{4.044294in}}%
\pgfpathlineto{\pgfqpoint{3.841142in}{4.036175in}}%
\pgfpathlineto{\pgfqpoint{3.819546in}{4.035529in}}%
\pgfpathlineto{\pgfqpoint{3.802646in}{4.040421in}}%
\pgfpathlineto{\pgfqpoint{3.781756in}{4.066066in}}%
\pgfpathlineto{\pgfqpoint{3.765089in}{4.130831in}}%
\pgfpathlineto{\pgfqpoint{3.743729in}{4.146039in}}%
\pgfpathlineto{\pgfqpoint{3.726593in}{4.320734in}}%
\pgfpathlineto{\pgfqpoint{3.704999in}{4.427595in}}%
\pgfpathlineto{\pgfqpoint{3.687863in}{4.360118in}}%
\pgfpathlineto{\pgfqpoint{3.666973in}{4.121459in}}%
\pgfpathlineto{\pgfqpoint{3.649603in}{4.058141in}}%
\pgfpathlineto{\pgfqpoint{3.628712in}{4.038757in}}%
\pgfpathlineto{\pgfqpoint{3.606882in}{4.035377in}}%
\pgfpathlineto{\pgfqpoint{3.589511in}{4.037455in}}%
\pgfpathlineto{\pgfqpoint{3.571203in}{4.047990in}}%
\pgfpathlineto{\pgfqpoint{3.553599in}{4.070444in}}%
\pgfpathlineto{\pgfqpoint{3.532237in}{4.205869in}}%
\pgfpathlineto{\pgfqpoint{3.510407in}{4.369278in}}%
\pgfpathlineto{\pgfqpoint{3.492333in}{4.422475in}}%
\pgfpathlineto{\pgfqpoint{3.473320in}{4.305510in}}%
\pgfpathlineto{\pgfqpoint{3.458063in}{4.126503in}}%
\pgfpathlineto{\pgfqpoint{3.436702in}{4.348585in}}%
\pgfpathlineto{\pgfqpoint{3.414874in}{4.417778in}}%
\pgfpathlineto{\pgfqpoint{3.378256in}{4.114581in}}%
\pgfpathlineto{\pgfqpoint{3.360417in}{4.054598in}}%
\pgfpathlineto{\pgfqpoint{3.341872in}{4.039335in}}%
\pgfpathlineto{\pgfqpoint{3.320747in}{4.034940in}}%
\pgfpathlineto{\pgfqpoint{3.301029in}{4.038147in}}%
\pgfpathlineto{\pgfqpoint{3.280607in}{4.051338in}}%
\pgfpathlineto{\pgfqpoint{3.262533in}{4.096829in}}%
\pgfpathlineto{\pgfqpoint{3.243991in}{4.248144in}}%
\pgfpathlineto{\pgfqpoint{3.225915in}{4.373055in}}%
\pgfpathlineto{\pgfqpoint{3.204556in}{4.414406in}}%
\pgfpathlineto{\pgfqpoint{3.166529in}{4.088520in}}%
\pgfpathlineto{\pgfqpoint{3.147750in}{4.049432in}}%
\pgfpathlineto{\pgfqpoint{3.128503in}{4.037531in}}%
\pgfpathlineto{\pgfqpoint{3.109960in}{4.034786in}}%
\pgfpathlineto{\pgfqpoint{3.091416in}{4.036586in}}%
\pgfpathlineto{\pgfqpoint{3.072871in}{4.042564in}}%
\pgfpathlineto{\pgfqpoint{3.050104in}{4.065334in}}%
\pgfpathlineto{\pgfqpoint{3.032733in}{4.167809in}}%
\pgfpathlineto{\pgfqpoint{3.013017in}{4.315350in}}%
\pgfpathlineto{\pgfqpoint{2.994472in}{4.416095in}}%
\pgfpathlineto{\pgfqpoint{2.955039in}{4.184871in}}%
\pgfpathlineto{\pgfqpoint{2.936495in}{4.091073in}}%
\pgfpathlineto{\pgfqpoint{2.917950in}{4.050454in}}%
\pgfpathlineto{\pgfqpoint{2.897060in}{4.037185in}}%
\pgfpathlineto{\pgfqpoint{2.880160in}{4.034971in}}%
\pgfpathlineto{\pgfqpoint{2.859973in}{4.037975in}}%
\pgfpathlineto{\pgfqpoint{2.841665in}{4.047584in}}%
\pgfpathlineto{\pgfqpoint{2.819130in}{4.078850in}}%
\pgfpathlineto{\pgfqpoint{2.802464in}{4.176156in}}%
\pgfpathlineto{\pgfqpoint{2.783685in}{4.342450in}}%
\pgfpathlineto{\pgfqpoint{2.766081in}{4.411223in}}%
\pgfpathlineto{\pgfqpoint{2.747302in}{4.305815in}}%
\pgfpathlineto{\pgfqpoint{2.724768in}{4.116628in}}%
\pgfpathlineto{\pgfqpoint{2.707634in}{4.055685in}}%
\pgfpathlineto{\pgfqpoint{2.687916in}{4.040906in}}%
\pgfpathlineto{\pgfqpoint{2.667494in}{4.035203in}}%
\pgfpathlineto{\pgfqpoint{2.647778in}{4.036055in}}%
\pgfpathlineto{\pgfqpoint{2.630407in}{4.041981in}}%
\pgfpathlineto{\pgfqpoint{2.611628in}{4.047141in}}%
\pgfpathlineto{\pgfqpoint{2.588392in}{4.066928in}}%
\pgfpathlineto{\pgfqpoint{2.568673in}{4.160721in}}%
\pgfpathlineto{\pgfqpoint{2.553885in}{4.297062in}}%
\pgfpathlineto{\pgfqpoint{2.534169in}{4.376106in}}%
\pgfpathlineto{\pgfqpoint{2.514687in}{4.407632in}}%
\pgfpathlineto{\pgfqpoint{2.494734in}{4.260407in}}%
\pgfpathlineto{\pgfqpoint{2.477364in}{4.108671in}}%
\pgfpathlineto{\pgfqpoint{2.457413in}{4.053160in}}%
\pgfpathlineto{\pgfqpoint{2.435582in}{4.038214in}}%
\pgfpathlineto{\pgfqpoint{2.417272in}{4.035393in}}%
\pgfpathlineto{\pgfqpoint{2.398496in}{4.034883in}}%
\pgfpathlineto{\pgfqpoint{2.379951in}{4.039949in}}%
\pgfpathlineto{\pgfqpoint{2.360938in}{4.035083in}}%
\pgfpathlineto{\pgfqpoint{2.342395in}{4.035476in}}%
\pgfpathlineto{\pgfqpoint{2.321034in}{4.043916in}}%
\pgfpathlineto{\pgfqpoint{2.302960in}{4.072952in}}%
\pgfpathlineto{\pgfqpoint{2.284650in}{4.161909in}}%
\pgfpathlineto{\pgfqpoint{2.265403in}{4.296959in}}%
\pgfpathlineto{\pgfqpoint{2.246626in}{4.400146in}}%
\pgfpathlineto{\pgfqpoint{2.225264in}{4.371709in}}%
\pgfpathlineto{\pgfqpoint{2.206017in}{4.168249in}}%
\pgfpathlineto{\pgfqpoint{2.188178in}{4.070569in}}%
\pgfpathlineto{\pgfqpoint{2.168695in}{4.044650in}}%
\pgfpathlineto{\pgfqpoint{2.148977in}{4.036163in}}%
\pgfpathlineto{\pgfqpoint{2.132312in}{4.035110in}}%
\pgfpathlineto{\pgfqpoint{2.107665in}{4.040453in}}%
\pgfpathlineto{\pgfqpoint{2.092642in}{4.045737in}}%
\pgfpathlineto{\pgfqpoint{2.072455in}{4.079427in}}%
\pgfpathlineto{\pgfqpoint{2.052035in}{4.212419in}}%
\pgfpathlineto{\pgfqpoint{2.033725in}{4.345751in}}%
\pgfpathlineto{\pgfqpoint{2.012366in}{4.410286in}}%
\pgfpathlineto{\pgfqpoint{1.997578in}{4.357873in}}%
\pgfpathlineto{\pgfqpoint{1.975277in}{4.183602in}}%
\pgfpathlineto{\pgfqpoint{1.956969in}{4.078365in}}%
\pgfpathlineto{\pgfqpoint{1.938425in}{4.052423in}}%
\pgfpathlineto{\pgfqpoint{1.919648in}{4.040056in}}%
\pgfpathlineto{\pgfqpoint{1.900634in}{4.035695in}}%
\pgfpathlineto{\pgfqpoint{1.882795in}{4.035641in}}%
\pgfpathlineto{\pgfqpoint{1.857679in}{4.042768in}}%
\pgfpathlineto{\pgfqpoint{1.841483in}{4.054756in}}%
\pgfpathlineto{\pgfqpoint{1.820590in}{4.128149in}}%
\pgfpathlineto{\pgfqpoint{1.802517in}{4.230496in}}%
\pgfpathlineto{\pgfqpoint{1.783738in}{4.360824in}}%
\pgfpathlineto{\pgfqpoint{1.764727in}{4.417362in}}%
\pgfpathlineto{\pgfqpoint{1.746182in}{4.379967in}}%
\pgfpathlineto{\pgfqpoint{1.727874in}{4.209947in}}%
\pgfpathlineto{\pgfqpoint{1.709330in}{4.116428in}}%
\pgfpathlineto{\pgfqpoint{1.687500in}{4.062110in}}%
\pgfpathlineto{\pgfqpoint{1.669191in}{4.042172in}}%
\pgfpathlineto{\pgfqpoint{1.650881in}{4.036758in}}%
\pgfpathlineto{\pgfqpoint{1.632105in}{4.035129in}}%
\pgfpathlineto{\pgfqpoint{1.610040in}{4.037502in}}%
\pgfpathlineto{\pgfqpoint{1.592670in}{4.042256in}}%
\pgfpathlineto{\pgfqpoint{1.573422in}{4.054393in}}%
\pgfpathlineto{\pgfqpoint{1.553235in}{4.107578in}}%
\pgfpathlineto{\pgfqpoint{1.533753in}{4.192541in}}%
\pgfpathlineto{\pgfqpoint{1.514739in}{4.339393in}}%
\pgfpathlineto{\pgfqpoint{1.496429in}{4.418919in}}%
\pgfpathlineto{\pgfqpoint{1.475773in}{4.399202in}}%
\pgfpathlineto{\pgfqpoint{1.458168in}{4.245005in}}%
\pgfpathlineto{\pgfqpoint{1.438217in}{4.111704in}}%
\pgfpathlineto{\pgfqpoint{1.416856in}{4.061029in}}%
\pgfpathlineto{\pgfqpoint{1.401130in}{4.047458in}}%
\pgfpathlineto{\pgfqpoint{1.379535in}{4.037982in}}%
\pgfpathlineto{\pgfqpoint{1.360990in}{4.219359in}}%
\pgfpathlineto{\pgfqpoint{1.342213in}{4.391168in}}%
\pgfpathlineto{\pgfqpoint{1.324138in}{4.425935in}}%
\pgfpathlineto{\pgfqpoint{1.305830in}{4.356286in}}%
\pgfpathlineto{\pgfqpoint{1.283531in}{4.118347in}}%
\pgfpathlineto{\pgfqpoint{1.265692in}{4.058632in}}%
\pgfpathlineto{\pgfqpoint{1.243627in}{4.040802in}}%
\pgfpathlineto{\pgfqpoint{1.226725in}{4.037603in}}%
\pgfpathlineto{\pgfqpoint{1.205132in}{4.035957in}}%
\pgfpathlineto{\pgfqpoint{1.186587in}{4.040816in}}%
\pgfpathlineto{\pgfqpoint{1.168748in}{4.059972in}}%
\pgfpathlineto{\pgfqpoint{1.151612in}{4.098419in}}%
\pgfpathlineto{\pgfqpoint{1.129547in}{4.268986in}}%
\pgfpathlineto{\pgfqpoint{1.107954in}{4.396052in}}%
\pgfpathlineto{\pgfqpoint{1.090349in}{4.437859in}}%
\pgfpathlineto{\pgfqpoint{1.074621in}{4.397367in}}%
\pgfpathlineto{\pgfqpoint{1.055842in}{4.225391in}}%
\pgfpathlineto{\pgfqpoint{1.034952in}{4.084734in}}%
\pgfpathlineto{\pgfqpoint{1.017113in}{4.050905in}}%
\pgfpathlineto{\pgfqpoint{0.995517in}{4.038723in}}%
\pgfpathlineto{\pgfqpoint{0.977443in}{4.036154in}}%
\pgfpathlineto{\pgfqpoint{0.957961in}{4.039077in}}%
\pgfpathlineto{\pgfqpoint{0.939887in}{4.036229in}}%
\pgfpathlineto{\pgfqpoint{0.918761in}{4.044045in}}%
\pgfpathlineto{\pgfqpoint{0.902095in}{4.064268in}}%
\pgfpathlineto{\pgfqpoint{0.880970in}{4.109840in}}%
\pgfpathlineto{\pgfqpoint{0.862895in}{4.231684in}}%
\pgfpathlineto{\pgfqpoint{0.844118in}{4.397261in}}%
\pgfpathlineto{\pgfqpoint{0.822053in}{4.448759in}}%
\pgfpathlineto{\pgfqpoint{0.799049in}{4.359960in}}%
\pgfpathlineto{\pgfqpoint{0.781679in}{4.179481in}}%
\pgfpathlineto{\pgfqpoint{0.765482in}{4.090242in}}%
\pgfpathlineto{\pgfqpoint{0.745061in}{4.053050in}}%
\pgfpathlineto{\pgfqpoint{0.725813in}{4.040838in}}%
\pgfpathlineto{\pgfqpoint{0.708913in}{4.036771in}}%
\pgfpathlineto{\pgfqpoint{0.689431in}{4.038800in}}%
\pgfpathlineto{\pgfqpoint{0.669478in}{4.050263in}}%
\pgfpathlineto{\pgfqpoint{0.650934in}{4.081261in}}%
\pgfpathlineto{\pgfqpoint{0.658447in}{4.056251in}}%
\pgfpathlineto{\pgfqpoint{0.677224in}{4.038602in}}%
\pgfpathlineto{\pgfqpoint{0.696237in}{4.037646in}}%
\pgfpathlineto{\pgfqpoint{0.715485in}{4.051241in}}%
\pgfpathlineto{\pgfqpoint{0.735438in}{4.097314in}}%
\pgfpathlineto{\pgfqpoint{0.756094in}{4.283908in}}%
\pgfpathlineto{\pgfqpoint{0.772994in}{4.446961in}}%
\pgfpathlineto{\pgfqpoint{0.791772in}{4.391912in}}%
\pgfpathlineto{\pgfqpoint{0.810551in}{4.193277in}}%
\pgfpathlineto{\pgfqpoint{0.830268in}{4.073696in}}%
\pgfpathlineto{\pgfqpoint{0.849515in}{4.043989in}}%
\pgfpathlineto{\pgfqpoint{0.869468in}{4.036098in}}%
\pgfpathlineto{\pgfqpoint{0.887073in}{4.041271in}}%
\pgfpathlineto{\pgfqpoint{0.906790in}{4.062873in}}%
\pgfpathlineto{\pgfqpoint{0.924863in}{4.148123in}}%
\pgfpathlineto{\pgfqpoint{0.944816in}{4.406319in}}%
\pgfpathlineto{\pgfqpoint{0.963361in}{4.410269in}}%
\pgfpathlineto{\pgfqpoint{0.984251in}{4.224066in}}%
\pgfpathlineto{\pgfqpoint{1.001856in}{4.087312in}}%
\pgfpathlineto{\pgfqpoint{1.019461in}{4.047833in}}%
\pgfpathlineto{\pgfqpoint{1.041525in}{4.036203in}}%
\pgfpathlineto{\pgfqpoint{1.060302in}{4.039060in}}%
\pgfpathlineto{\pgfqpoint{1.079081in}{4.053914in}}%
\pgfpathlineto{\pgfqpoint{1.097391in}{4.112281in}}%
\pgfpathlineto{\pgfqpoint{1.116873in}{4.329445in}}%
\pgfpathlineto{\pgfqpoint{1.135181in}{4.430561in}}%
\pgfpathlineto{\pgfqpoint{1.157951in}{4.289608in}}%
\pgfpathlineto{\pgfqpoint{1.176728in}{4.114652in}}%
\pgfpathlineto{\pgfqpoint{1.195507in}{4.052536in}}%
\pgfpathlineto{\pgfqpoint{1.214286in}{4.037757in}}%
\pgfpathlineto{\pgfqpoint{1.233533in}{4.035691in}}%
\pgfpathlineto{\pgfqpoint{1.251841in}{4.042693in}}%
\pgfpathlineto{\pgfqpoint{1.270386in}{4.062367in}}%
\pgfpathlineto{\pgfqpoint{1.292685in}{4.180977in}}%
\pgfpathlineto{\pgfqpoint{1.308178in}{4.393684in}}%
\pgfpathlineto{\pgfqpoint{1.329772in}{4.404917in}}%
\pgfpathlineto{\pgfqpoint{1.348785in}{4.262159in}}%
\pgfpathlineto{\pgfqpoint{1.366624in}{4.109136in}}%
\pgfpathlineto{\pgfqpoint{1.388220in}{4.053417in}}%
\pgfpathlineto{\pgfqpoint{1.406528in}{4.038647in}}%
\pgfpathlineto{\pgfqpoint{1.423664in}{4.035262in}}%
\pgfpathlineto{\pgfqpoint{1.445963in}{4.040742in}}%
\pgfpathlineto{\pgfqpoint{1.463333in}{4.058288in}}%
\pgfpathlineto{\pgfqpoint{1.480938in}{4.122583in}}%
\pgfpathlineto{\pgfqpoint{1.502298in}{4.306708in}}%
\pgfpathlineto{\pgfqpoint{1.519668in}{4.417817in}}%
\pgfpathlineto{\pgfqpoint{1.540793in}{4.371714in}}%
\pgfpathlineto{\pgfqpoint{1.558398in}{4.214353in}}%
\pgfpathlineto{\pgfqpoint{1.580933in}{4.073350in}}%
\pgfpathlineto{\pgfqpoint{1.598538in}{4.043508in}}%
\pgfpathlineto{\pgfqpoint{1.616377in}{4.036572in}}%
\pgfpathlineto{\pgfqpoint{1.636328in}{4.270387in}}%
\pgfpathlineto{\pgfqpoint{1.653933in}{4.129227in}}%
\pgfpathlineto{\pgfqpoint{1.676703in}{4.052602in}}%
\pgfpathlineto{\pgfqpoint{1.693133in}{4.038804in}}%
\pgfpathlineto{\pgfqpoint{1.718015in}{4.034959in}}%
\pgfpathlineto{\pgfqpoint{1.734211in}{4.038367in}}%
\pgfpathlineto{\pgfqpoint{1.751582in}{4.047110in}}%
\pgfpathlineto{\pgfqpoint{1.772001in}{4.087375in}}%
\pgfpathlineto{\pgfqpoint{1.790077in}{4.224237in}}%
\pgfpathlineto{\pgfqpoint{1.811907in}{4.406879in}}%
\pgfpathlineto{\pgfqpoint{1.827398in}{4.394260in}}%
\pgfpathlineto{\pgfqpoint{1.847820in}{4.206241in}}%
\pgfpathlineto{\pgfqpoint{1.868945in}{4.072949in}}%
\pgfpathlineto{\pgfqpoint{1.887021in}{4.043781in}}%
\pgfpathlineto{\pgfqpoint{1.906032in}{4.035758in}}%
\pgfpathlineto{\pgfqpoint{1.924811in}{4.034857in}}%
\pgfpathlineto{\pgfqpoint{1.945467in}{4.040877in}}%
\pgfpathlineto{\pgfqpoint{1.963306in}{4.058281in}}%
\pgfpathlineto{\pgfqpoint{1.981616in}{4.111773in}}%
\pgfpathlineto{\pgfqpoint{2.002272in}{4.249419in}}%
\pgfpathlineto{\pgfqpoint{2.020815in}{4.391298in}}%
\pgfpathlineto{\pgfqpoint{2.041471in}{4.390295in}}%
\pgfpathlineto{\pgfqpoint{2.059781in}{4.237783in}}%
\pgfpathlineto{\pgfqpoint{2.079732in}{4.082829in}}%
\pgfpathlineto{\pgfqpoint{2.098511in}{4.046103in}}%
\pgfpathlineto{\pgfqpoint{2.115881in}{4.037389in}}%
\pgfpathlineto{\pgfqpoint{2.137711in}{4.034924in}}%
\pgfpathlineto{\pgfqpoint{2.155316in}{4.038877in}}%
\pgfpathlineto{\pgfqpoint{2.172921in}{4.045191in}}%
\pgfpathlineto{\pgfqpoint{2.195454in}{4.088551in}}%
\pgfpathlineto{\pgfqpoint{2.212119in}{4.199240in}}%
\pgfpathlineto{\pgfqpoint{2.233715in}{4.402099in}}%
\pgfpathlineto{\pgfqpoint{2.251554in}{4.393014in}}%
\pgfpathlineto{\pgfqpoint{2.269159in}{4.296221in}}%
\pgfpathlineto{\pgfqpoint{2.290989in}{4.126289in}}%
\pgfpathlineto{\pgfqpoint{2.308360in}{4.056290in}}%
\pgfpathlineto{\pgfqpoint{2.329954in}{4.039208in}}%
\pgfpathlineto{\pgfqpoint{2.347558in}{4.035363in}}%
\pgfpathlineto{\pgfqpoint{2.365163in}{4.035619in}}%
\pgfpathlineto{\pgfqpoint{2.386288in}{4.041134in}}%
\pgfpathlineto{\pgfqpoint{2.404598in}{4.060243in}}%
\pgfpathlineto{\pgfqpoint{2.425254in}{4.055848in}}%
\pgfpathlineto{\pgfqpoint{2.439337in}{4.096052in}}%
\pgfpathlineto{\pgfqpoint{2.461402in}{4.201607in}}%
\pgfpathlineto{\pgfqpoint{2.482058in}{4.405927in}}%
\pgfpathlineto{\pgfqpoint{2.503419in}{4.360177in}}%
\pgfpathlineto{\pgfqpoint{2.521493in}{4.245235in}}%
\pgfpathlineto{\pgfqpoint{2.539566in}{4.105847in}}%
\pgfpathlineto{\pgfqpoint{2.563276in}{4.049241in}}%
\pgfpathlineto{\pgfqpoint{2.579236in}{4.038785in}}%
\pgfpathlineto{\pgfqpoint{2.596841in}{4.035001in}}%
\pgfpathlineto{\pgfqpoint{2.616325in}{4.036950in}}%
\pgfpathlineto{\pgfqpoint{2.637450in}{4.045735in}}%
\pgfpathlineto{\pgfqpoint{2.653177in}{4.070312in}}%
\pgfpathlineto{\pgfqpoint{2.674068in}{4.174376in}}%
\pgfpathlineto{\pgfqpoint{2.692610in}{4.349005in}}%
\pgfpathlineto{\pgfqpoint{2.713266in}{4.405991in}}%
\pgfpathlineto{\pgfqpoint{2.732045in}{4.349389in}}%
\pgfpathlineto{\pgfqpoint{2.750119in}{4.231822in}}%
\pgfpathlineto{\pgfqpoint{2.770306in}{4.131822in}}%
\pgfpathlineto{\pgfqpoint{2.788616in}{4.058233in}}%
\pgfpathlineto{\pgfqpoint{2.811150in}{4.040357in}}%
\pgfpathlineto{\pgfqpoint{2.827580in}{4.036464in}}%
\pgfpathlineto{\pgfqpoint{2.845420in}{4.034994in}}%
\pgfpathlineto{\pgfqpoint{2.866544in}{4.037344in}}%
\pgfpathlineto{\pgfqpoint{2.885323in}{4.046344in}}%
\pgfpathlineto{\pgfqpoint{2.903633in}{4.072322in}}%
\pgfpathlineto{\pgfqpoint{2.923819in}{4.168887in}}%
\pgfpathlineto{\pgfqpoint{2.942363in}{4.366186in}}%
\pgfpathlineto{\pgfqpoint{2.962550in}{4.411323in}}%
\pgfpathlineto{\pgfqpoint{2.980858in}{4.352940in}}%
\pgfpathlineto{\pgfqpoint{2.999167in}{4.201268in}}%
\pgfpathlineto{\pgfqpoint{3.020059in}{4.079580in}}%
\pgfpathlineto{\pgfqpoint{3.037664in}{4.052856in}}%
\pgfpathlineto{\pgfqpoint{3.057146in}{4.038829in}}%
\pgfpathlineto{\pgfqpoint{3.077333in}{4.035264in}}%
\pgfpathlineto{\pgfqpoint{3.096815in}{4.036376in}}%
\pgfpathlineto{\pgfqpoint{3.114420in}{4.042148in}}%
\pgfpathlineto{\pgfqpoint{3.134842in}{4.064618in}}%
\pgfpathlineto{\pgfqpoint{3.154324in}{4.096790in}}%
\pgfpathlineto{\pgfqpoint{3.174746in}{4.188662in}}%
\pgfpathlineto{\pgfqpoint{3.193523in}{4.383782in}}%
\pgfpathlineto{\pgfqpoint{3.215822in}{4.404554in}}%
\pgfpathlineto{\pgfqpoint{3.231080in}{4.341760in}}%
\pgfpathlineto{\pgfqpoint{3.251502in}{4.196588in}}%
\pgfpathlineto{\pgfqpoint{3.269107in}{4.085212in}}%
\pgfpathlineto{\pgfqpoint{3.288120in}{4.066077in}}%
\pgfpathlineto{\pgfqpoint{3.308540in}{4.042435in}}%
\pgfpathlineto{\pgfqpoint{3.328727in}{4.036327in}}%
\pgfpathlineto{\pgfqpoint{3.346566in}{4.035291in}}%
\pgfpathlineto{\pgfqpoint{3.363937in}{4.036749in}}%
\pgfpathlineto{\pgfqpoint{3.385298in}{4.045927in}}%
\pgfpathlineto{\pgfqpoint{3.407363in}{4.063918in}}%
\pgfpathlineto{\pgfqpoint{3.424262in}{4.104531in}}%
\pgfpathlineto{\pgfqpoint{3.442101in}{4.229679in}}%
\pgfpathlineto{\pgfqpoint{3.463697in}{4.395421in}}%
\pgfpathlineto{\pgfqpoint{3.480831in}{4.422773in}}%
\pgfpathlineto{\pgfqpoint{3.501724in}{4.337532in}}%
\pgfpathlineto{\pgfqpoint{3.519329in}{4.338465in}}%
\pgfpathlineto{\pgfqpoint{3.537637in}{4.195161in}}%
\pgfpathlineto{\pgfqpoint{3.558058in}{4.080328in}}%
\pgfpathlineto{\pgfqpoint{3.576366in}{4.052295in}}%
\pgfpathlineto{\pgfqpoint{3.594911in}{4.041578in}}%
\pgfpathlineto{\pgfqpoint{3.615801in}{4.035886in}}%
\pgfpathlineto{\pgfqpoint{3.633875in}{4.036370in}}%
\pgfpathlineto{\pgfqpoint{3.673544in}{4.050587in}}%
\pgfpathlineto{\pgfqpoint{3.691854in}{4.079720in}}%
\pgfpathlineto{\pgfqpoint{3.715796in}{4.190552in}}%
\pgfpathlineto{\pgfqpoint{3.731053in}{4.362710in}}%
\pgfpathlineto{\pgfqpoint{3.748189in}{4.432535in}}%
\pgfpathlineto{\pgfqpoint{3.770722in}{4.425635in}}%
\pgfpathlineto{\pgfqpoint{3.790910in}{4.321894in}}%
\pgfpathlineto{\pgfqpoint{3.809452in}{4.192202in}}%
\pgfpathlineto{\pgfqpoint{3.828231in}{4.112715in}}%
\pgfpathlineto{\pgfqpoint{3.845367in}{4.065445in}}%
\pgfpathlineto{\pgfqpoint{3.866726in}{4.045524in}}%
\pgfpathlineto{\pgfqpoint{3.884566in}{4.037559in}}%
\pgfpathlineto{\pgfqpoint{3.906161in}{4.035977in}}%
\pgfpathlineto{\pgfqpoint{3.920715in}{4.037177in}}%
\pgfpathlineto{\pgfqpoint{3.943014in}{4.046625in}}%
\pgfpathlineto{\pgfqpoint{3.961558in}{4.068932in}}%
\pgfpathlineto{\pgfqpoint{3.977755in}{4.092836in}}%
\pgfpathlineto{\pgfqpoint{4.000054in}{4.176321in}}%
\pgfpathlineto{\pgfqpoint{4.023996in}{4.323211in}}%
\pgfpathlineto{\pgfqpoint{4.038783in}{4.433413in}}%
\pgfpathlineto{\pgfqpoint{4.056388in}{4.057735in}}%
\pgfpathlineto{\pgfqpoint{4.077750in}{4.124211in}}%
\pgfpathlineto{\pgfqpoint{4.097701in}{4.278522in}}%
\pgfpathlineto{\pgfqpoint{4.114602in}{4.435411in}}%
\pgfpathlineto{\pgfqpoint{4.136667in}{4.429563in}}%
\pgfpathlineto{\pgfqpoint{4.153332in}{4.356754in}}%
\pgfpathlineto{\pgfqpoint{4.174222in}{4.169938in}}%
\pgfpathlineto{\pgfqpoint{4.194175in}{4.085847in}}%
\pgfpathlineto{\pgfqpoint{4.212483in}{4.051992in}}%
\pgfpathlineto{\pgfqpoint{4.231497in}{4.039942in}}%
\pgfpathlineto{\pgfqpoint{4.250510in}{4.036579in}}%
\pgfpathlineto{\pgfqpoint{4.269289in}{4.040155in}}%
\pgfpathlineto{\pgfqpoint{4.289709in}{4.052079in}}%
\pgfpathlineto{\pgfqpoint{4.306141in}{4.072130in}}%
\pgfpathlineto{\pgfqpoint{4.324918in}{4.128276in}}%
\pgfpathlineto{\pgfqpoint{4.365996in}{4.448749in}}%
\pgfpathlineto{\pgfqpoint{4.385009in}{4.453948in}}%
\pgfpathlineto{\pgfqpoint{4.404491in}{4.389806in}}%
\pgfpathlineto{\pgfqpoint{4.442518in}{4.123348in}}%
\pgfpathlineto{\pgfqpoint{4.461531in}{4.062858in}}%
\pgfpathlineto{\pgfqpoint{4.482187in}{4.045914in}}%
\pgfpathlineto{\pgfqpoint{4.476553in}{4.052332in}}%
\pgfpathlineto{\pgfqpoint{4.456368in}{4.105562in}}%
\pgfpathlineto{\pgfqpoint{4.435712in}{4.314261in}}%
\pgfpathlineto{\pgfqpoint{4.414819in}{4.449794in}}%
\pgfpathlineto{\pgfqpoint{4.398623in}{4.436595in}}%
\pgfpathlineto{\pgfqpoint{4.377029in}{4.179366in}}%
\pgfpathlineto{\pgfqpoint{4.359893in}{4.071537in}}%
\pgfpathlineto{\pgfqpoint{4.340177in}{4.044059in}}%
\pgfpathlineto{\pgfqpoint{4.318581in}{4.036540in}}%
\pgfpathlineto{\pgfqpoint{4.297222in}{4.045373in}}%
\pgfpathlineto{\pgfqpoint{4.283371in}{4.067851in}}%
\pgfpathlineto{\pgfqpoint{4.259898in}{4.223925in}}%
\pgfpathlineto{\pgfqpoint{4.242059in}{4.400183in}}%
\pgfpathlineto{\pgfqpoint{4.224689in}{4.447118in}}%
\pgfpathlineto{\pgfqpoint{4.203798in}{4.265981in}}%
\pgfpathlineto{\pgfqpoint{4.186193in}{4.101480in}}%
\pgfpathlineto{\pgfqpoint{4.165303in}{4.048325in}}%
\pgfpathlineto{\pgfqpoint{4.147932in}{4.036999in}}%
\pgfpathlineto{\pgfqpoint{4.127042in}{4.038235in}}%
\pgfpathlineto{\pgfqpoint{4.109437in}{4.050451in}}%
\pgfpathlineto{\pgfqpoint{4.091598in}{4.097200in}}%
\pgfpathlineto{\pgfqpoint{4.070942in}{4.282130in}}%
\pgfpathlineto{\pgfqpoint{4.052868in}{4.428578in}}%
\pgfpathlineto{\pgfqpoint{4.032915in}{4.413608in}}%
\pgfpathlineto{\pgfqpoint{4.012728in}{4.153597in}}%
\pgfpathlineto{\pgfqpoint{3.994889in}{4.066140in}}%
\pgfpathlineto{\pgfqpoint{3.975641in}{4.040987in}}%
\pgfpathlineto{\pgfqpoint{3.954282in}{4.035664in}}%
\pgfpathlineto{\pgfqpoint{3.935737in}{4.040056in}}%
\pgfpathlineto{\pgfqpoint{3.918838in}{4.059620in}}%
\pgfpathlineto{\pgfqpoint{3.900059in}{4.147131in}}%
\pgfpathlineto{\pgfqpoint{3.875177in}{4.384869in}}%
\pgfpathlineto{\pgfqpoint{3.860389in}{4.432088in}}%
\pgfpathlineto{\pgfqpoint{3.820720in}{4.091384in}}%
\pgfpathlineto{\pgfqpoint{3.802646in}{4.048805in}}%
\pgfpathlineto{\pgfqpoint{3.780582in}{4.036589in}}%
\pgfpathlineto{\pgfqpoint{3.762508in}{4.035626in}}%
\pgfpathlineto{\pgfqpoint{3.744198in}{4.040864in}}%
\pgfpathlineto{\pgfqpoint{3.725656in}{4.063828in}}%
\pgfpathlineto{\pgfqpoint{3.705234in}{4.149240in}}%
\pgfpathlineto{\pgfqpoint{3.686689in}{4.212827in}}%
\pgfpathlineto{\pgfqpoint{3.668147in}{4.373645in}}%
\pgfpathlineto{\pgfqpoint{3.649134in}{4.424983in}}%
\pgfpathlineto{\pgfqpoint{3.628712in}{4.266427in}}%
\pgfpathlineto{\pgfqpoint{3.612516in}{4.100490in}}%
\pgfpathlineto{\pgfqpoint{3.589982in}{4.047182in}}%
\pgfpathlineto{\pgfqpoint{3.570029in}{4.036580in}}%
\pgfpathlineto{\pgfqpoint{3.551016in}{4.035219in}}%
\pgfpathlineto{\pgfqpoint{3.531768in}{4.039324in}}%
\pgfpathlineto{\pgfqpoint{3.513929in}{4.061783in}}%
\pgfpathlineto{\pgfqpoint{3.492099in}{4.170268in}}%
\pgfpathlineto{\pgfqpoint{3.476842in}{4.301663in}}%
\pgfpathlineto{\pgfqpoint{3.455246in}{4.415877in}}%
\pgfpathlineto{\pgfqpoint{3.436702in}{4.387233in}}%
\pgfpathlineto{\pgfqpoint{3.414874in}{4.147589in}}%
\pgfpathlineto{\pgfqpoint{3.399146in}{4.083046in}}%
\pgfpathlineto{\pgfqpoint{3.377785in}{4.046211in}}%
\pgfpathlineto{\pgfqpoint{3.358537in}{4.036511in}}%
\pgfpathlineto{\pgfqpoint{3.339760in}{4.035347in}}%
\pgfpathlineto{\pgfqpoint{3.321450in}{4.037368in}}%
\pgfpathlineto{\pgfqpoint{3.301029in}{4.053843in}}%
\pgfpathlineto{\pgfqpoint{3.284598in}{4.117074in}}%
\pgfpathlineto{\pgfqpoint{3.262299in}{4.294339in}}%
\pgfpathlineto{\pgfqpoint{3.243991in}{4.158927in}}%
\pgfpathlineto{\pgfqpoint{3.207607in}{4.413177in}}%
\pgfpathlineto{\pgfqpoint{3.188360in}{4.384583in}}%
\pgfpathlineto{\pgfqpoint{3.164652in}{4.139346in}}%
\pgfpathlineto{\pgfqpoint{3.147516in}{4.068878in}}%
\pgfpathlineto{\pgfqpoint{3.129911in}{4.045341in}}%
\pgfpathlineto{\pgfqpoint{3.110898in}{4.036127in}}%
\pgfpathlineto{\pgfqpoint{3.089304in}{4.035655in}}%
\pgfpathlineto{\pgfqpoint{3.070291in}{4.040763in}}%
\pgfpathlineto{\pgfqpoint{3.052452in}{4.057100in}}%
\pgfpathlineto{\pgfqpoint{3.033438in}{4.116748in}}%
\pgfpathlineto{\pgfqpoint{3.013017in}{4.273520in}}%
\pgfpathlineto{\pgfqpoint{2.992595in}{4.403675in}}%
\pgfpathlineto{\pgfqpoint{2.973347in}{4.395101in}}%
\pgfpathlineto{\pgfqpoint{2.955743in}{4.234941in}}%
\pgfpathlineto{\pgfqpoint{2.937434in}{4.090222in}}%
\pgfpathlineto{\pgfqpoint{2.918421in}{4.059348in}}%
\pgfpathlineto{\pgfqpoint{2.900346in}{4.042532in}}%
\pgfpathlineto{\pgfqpoint{2.880864in}{4.036280in}}%
\pgfpathlineto{\pgfqpoint{2.859504in}{4.042836in}}%
\pgfpathlineto{\pgfqpoint{2.839082in}{4.035789in}}%
\pgfpathlineto{\pgfqpoint{2.820772in}{4.035739in}}%
\pgfpathlineto{\pgfqpoint{2.797770in}{4.045444in}}%
\pgfpathlineto{\pgfqpoint{2.781808in}{4.069884in}}%
\pgfpathlineto{\pgfqpoint{2.744721in}{4.319215in}}%
\pgfpathlineto{\pgfqpoint{2.725942in}{4.412662in}}%
\pgfpathlineto{\pgfqpoint{2.702469in}{4.362043in}}%
\pgfpathlineto{\pgfqpoint{2.688621in}{4.178870in}}%
\pgfpathlineto{\pgfqpoint{2.667260in}{4.070899in}}%
\pgfpathlineto{\pgfqpoint{2.648012in}{4.044650in}}%
\pgfpathlineto{\pgfqpoint{2.628530in}{4.035881in}}%
\pgfpathlineto{\pgfqpoint{2.605291in}{4.036146in}}%
\pgfpathlineto{\pgfqpoint{2.594963in}{4.041070in}}%
\pgfpathlineto{\pgfqpoint{2.570316in}{4.061286in}}%
\pgfpathlineto{\pgfqpoint{2.552711in}{4.131659in}}%
\pgfpathlineto{\pgfqpoint{2.533229in}{4.288805in}}%
\pgfpathlineto{\pgfqpoint{2.514687in}{4.403716in}}%
\pgfpathlineto{\pgfqpoint{2.496611in}{4.400021in}}%
\pgfpathlineto{\pgfqpoint{2.475486in}{4.180274in}}%
\pgfpathlineto{\pgfqpoint{2.455533in}{4.077005in}}%
\pgfpathlineto{\pgfqpoint{2.439808in}{4.056643in}}%
\pgfpathlineto{\pgfqpoint{2.420089in}{4.040927in}}%
\pgfpathlineto{\pgfqpoint{2.400138in}{4.035848in}}%
\pgfpathlineto{\pgfqpoint{2.378308in}{4.034965in}}%
\pgfpathlineto{\pgfqpoint{2.360235in}{4.036716in}}%
\pgfpathlineto{\pgfqpoint{2.341690in}{4.045393in}}%
\pgfpathlineto{\pgfqpoint{2.322442in}{4.074251in}}%
\pgfpathlineto{\pgfqpoint{2.301083in}{4.214695in}}%
\pgfpathlineto{\pgfqpoint{2.282539in}{4.364515in}}%
\pgfpathlineto{\pgfqpoint{2.263291in}{4.405641in}}%
\pgfpathlineto{\pgfqpoint{2.224561in}{4.130119in}}%
\pgfpathlineto{\pgfqpoint{2.205782in}{4.065727in}}%
\pgfpathlineto{\pgfqpoint{2.187004in}{4.042943in}}%
\pgfpathlineto{\pgfqpoint{2.168695in}{4.036842in}}%
\pgfpathlineto{\pgfqpoint{2.149682in}{4.035022in}}%
\pgfpathlineto{\pgfqpoint{2.126444in}{4.039838in}}%
\pgfpathlineto{\pgfqpoint{2.110482in}{4.046331in}}%
\pgfpathlineto{\pgfqpoint{2.091234in}{4.080177in}}%
\pgfpathlineto{\pgfqpoint{2.073395in}{4.187680in}}%
\pgfpathlineto{\pgfqpoint{2.051096in}{4.355625in}}%
\pgfpathlineto{\pgfqpoint{2.032786in}{4.409466in}}%
\pgfpathlineto{\pgfqpoint{2.014009in}{4.375261in}}%
\pgfpathlineto{\pgfqpoint{1.994996in}{4.198401in}}%
\pgfpathlineto{\pgfqpoint{1.977156in}{4.111892in}}%
\pgfpathlineto{\pgfqpoint{1.955561in}{4.054770in}}%
\pgfpathlineto{\pgfqpoint{1.936547in}{4.043042in}}%
\pgfpathlineto{\pgfqpoint{1.918943in}{4.036452in}}%
\pgfpathlineto{\pgfqpoint{1.899226in}{4.035095in}}%
\pgfpathlineto{\pgfqpoint{1.879039in}{4.038904in}}%
\pgfpathlineto{\pgfqpoint{1.860496in}{4.044584in}}%
\pgfpathlineto{\pgfqpoint{1.837961in}{4.079073in}}%
\pgfpathlineto{\pgfqpoint{1.822704in}{4.149808in}}%
\pgfpathlineto{\pgfqpoint{1.804630in}{4.039174in}}%
\pgfpathlineto{\pgfqpoint{1.783504in}{4.035321in}}%
\pgfpathlineto{\pgfqpoint{1.764256in}{4.038049in}}%
\pgfpathlineto{\pgfqpoint{1.745713in}{4.049657in}}%
\pgfpathlineto{\pgfqpoint{1.723178in}{4.089269in}}%
\pgfpathlineto{\pgfqpoint{1.704870in}{4.182392in}}%
\pgfpathlineto{\pgfqpoint{1.686326in}{4.285225in}}%
\pgfpathlineto{\pgfqpoint{1.667549in}{4.401361in}}%
\pgfpathlineto{\pgfqpoint{1.647596in}{4.398918in}}%
\pgfpathlineto{\pgfqpoint{1.631399in}{4.222946in}}%
\pgfpathlineto{\pgfqpoint{1.609804in}{4.079650in}}%
\pgfpathlineto{\pgfqpoint{1.591261in}{4.053846in}}%
\pgfpathlineto{\pgfqpoint{1.573422in}{4.040156in}}%
\pgfpathlineto{\pgfqpoint{1.553235in}{4.035269in}}%
\pgfpathlineto{\pgfqpoint{1.535630in}{4.037263in}}%
\pgfpathlineto{\pgfqpoint{1.518025in}{4.046517in}}%
\pgfpathlineto{\pgfqpoint{1.496195in}{4.080150in}}%
\pgfpathlineto{\pgfqpoint{1.478356in}{4.162765in}}%
\pgfpathlineto{\pgfqpoint{1.459577in}{4.312498in}}%
\pgfpathlineto{\pgfqpoint{1.437278in}{4.423568in}}%
\pgfpathlineto{\pgfqpoint{1.418970in}{4.398882in}}%
\pgfpathlineto{\pgfqpoint{1.402539in}{4.216873in}}%
\pgfpathlineto{\pgfqpoint{1.378831in}{4.092710in}}%
\pgfpathlineto{\pgfqpoint{1.361461in}{4.056004in}}%
\pgfpathlineto{\pgfqpoint{1.341979in}{4.040332in}}%
\pgfpathlineto{\pgfqpoint{1.323200in}{4.036674in}}%
\pgfpathlineto{\pgfqpoint{1.302073in}{4.036505in}}%
\pgfpathlineto{\pgfqpoint{1.284234in}{4.039680in}}%
\pgfpathlineto{\pgfqpoint{1.264986in}{4.046634in}}%
\pgfpathlineto{\pgfqpoint{1.244801in}{4.083037in}}%
\pgfpathlineto{\pgfqpoint{1.222266in}{4.199765in}}%
\pgfpathlineto{\pgfqpoint{1.206774in}{4.303353in}}%
\pgfpathlineto{\pgfqpoint{1.185179in}{4.409737in}}%
\pgfpathlineto{\pgfqpoint{1.169688in}{4.435430in}}%
\pgfpathlineto{\pgfqpoint{1.149031in}{4.370421in}}%
\pgfpathlineto{\pgfqpoint{1.126496in}{4.274259in}}%
\pgfpathlineto{\pgfqpoint{1.110770in}{4.142794in}}%
\pgfpathlineto{\pgfqpoint{1.089644in}{4.074657in}}%
\pgfpathlineto{\pgfqpoint{1.071804in}{4.050540in}}%
\pgfpathlineto{\pgfqpoint{1.054905in}{4.040112in}}%
\pgfpathlineto{\pgfqpoint{1.033778in}{4.036176in}}%
\pgfpathlineto{\pgfqpoint{1.014296in}{4.039926in}}%
\pgfpathlineto{\pgfqpoint{0.995282in}{4.049590in}}%
\pgfpathlineto{\pgfqpoint{0.976269in}{4.082306in}}%
\pgfpathlineto{\pgfqpoint{0.957727in}{4.146357in}}%
\pgfpathlineto{\pgfqpoint{0.941060in}{4.219820in}}%
\pgfpathlineto{\pgfqpoint{0.919935in}{4.354104in}}%
\pgfpathlineto{\pgfqpoint{0.899278in}{4.431555in}}%
\pgfpathlineto{\pgfqpoint{0.881908in}{4.443217in}}%
\pgfpathlineto{\pgfqpoint{0.860314in}{4.404527in}}%
\pgfpathlineto{\pgfqpoint{0.841535in}{4.226956in}}%
\pgfpathlineto{\pgfqpoint{0.823931in}{4.109175in}}%
\pgfpathlineto{\pgfqpoint{0.803978in}{4.060367in}}%
\pgfpathlineto{\pgfqpoint{0.783087in}{4.042582in}}%
\pgfpathlineto{\pgfqpoint{0.766891in}{4.043556in}}%
\pgfpathlineto{\pgfqpoint{0.745766in}{4.038873in}}%
\pgfpathlineto{\pgfqpoint{0.727456in}{4.036479in}}%
\pgfpathlineto{\pgfqpoint{0.707974in}{4.041830in}}%
\pgfpathlineto{\pgfqpoint{0.684971in}{4.065760in}}%
\pgfpathlineto{\pgfqpoint{0.669244in}{4.104263in}}%
\pgfpathlineto{\pgfqpoint{0.652108in}{4.185391in}}%
\pgfpathlineto{\pgfqpoint{0.652579in}{4.182284in}}%
\pgfpathlineto{\pgfqpoint{0.655865in}{4.153420in}}%
\pgfpathlineto{\pgfqpoint{0.675112in}{4.064491in}}%
\pgfpathlineto{\pgfqpoint{0.694594in}{4.041470in}}%
\pgfpathlineto{\pgfqpoint{0.714311in}{4.036539in}}%
\pgfpathlineto{\pgfqpoint{0.732855in}{4.046061in}}%
\pgfpathlineto{\pgfqpoint{0.751869in}{4.078642in}}%
\pgfpathlineto{\pgfqpoint{0.770413in}{4.201264in}}%
\pgfpathlineto{\pgfqpoint{0.792712in}{4.442196in}}%
\pgfpathlineto{\pgfqpoint{0.813837in}{4.393313in}}%
\pgfpathlineto{\pgfqpoint{0.829799in}{4.235773in}}%
\pgfpathlineto{\pgfqpoint{0.849750in}{4.089451in}}%
\pgfpathlineto{\pgfqpoint{0.868060in}{4.046763in}}%
\pgfpathlineto{\pgfqpoint{0.886602in}{4.036617in}}%
\pgfpathlineto{\pgfqpoint{0.905381in}{4.038937in}}%
\pgfpathlineto{\pgfqpoint{0.924394in}{4.055052in}}%
\pgfpathlineto{\pgfqpoint{0.943408in}{4.110663in}}%
\pgfpathlineto{\pgfqpoint{0.965472in}{4.380643in}}%
\pgfpathlineto{\pgfqpoint{0.984954in}{4.434793in}}%
\pgfpathlineto{\pgfqpoint{1.003264in}{4.320412in}}%
\pgfpathlineto{\pgfqpoint{1.022041in}{4.132783in}}%
\pgfpathlineto{\pgfqpoint{1.040351in}{4.057018in}}%
\pgfpathlineto{\pgfqpoint{1.060068in}{4.039110in}}%
\pgfpathlineto{\pgfqpoint{1.079550in}{4.035937in}}%
\pgfpathlineto{\pgfqpoint{1.096920in}{4.043088in}}%
\pgfpathlineto{\pgfqpoint{1.120393in}{4.076554in}}%
\pgfpathlineto{\pgfqpoint{1.136590in}{4.180026in}}%
\pgfpathlineto{\pgfqpoint{1.153491in}{4.407360in}}%
\pgfpathlineto{\pgfqpoint{1.176493in}{4.397335in}}%
\pgfpathlineto{\pgfqpoint{1.194569in}{4.212865in}}%
\pgfpathlineto{\pgfqpoint{1.214286in}{4.085736in}}%
\pgfpathlineto{\pgfqpoint{1.235176in}{4.043310in}}%
\pgfpathlineto{\pgfqpoint{1.250904in}{4.036335in}}%
\pgfpathlineto{\pgfqpoint{1.271794in}{4.037549in}}%
\pgfpathlineto{\pgfqpoint{1.291981in}{4.048237in}}%
\pgfpathlineto{\pgfqpoint{1.310055in}{4.084400in}}%
\pgfpathlineto{\pgfqpoint{1.329303in}{4.234551in}}%
\pgfpathlineto{\pgfqpoint{1.348785in}{4.424924in}}%
\pgfpathlineto{\pgfqpoint{1.368267in}{4.382303in}}%
\pgfpathlineto{\pgfqpoint{1.385637in}{4.228906in}}%
\pgfpathlineto{\pgfqpoint{1.406059in}{4.081968in}}%
\pgfpathlineto{\pgfqpoint{1.426481in}{4.043745in}}%
\pgfpathlineto{\pgfqpoint{1.443617in}{4.036125in}}%
\pgfpathlineto{\pgfqpoint{1.461690in}{4.035974in}}%
\pgfpathlineto{\pgfqpoint{1.481407in}{4.043628in}}%
\pgfpathlineto{\pgfqpoint{1.501829in}{4.071822in}}%
\pgfpathlineto{\pgfqpoint{1.522250in}{4.170201in}}%
\pgfpathlineto{\pgfqpoint{1.539855in}{4.380612in}}%
\pgfpathlineto{\pgfqpoint{1.559337in}{4.168465in}}%
\pgfpathlineto{\pgfqpoint{1.577177in}{4.400700in}}%
\pgfpathlineto{\pgfqpoint{1.598538in}{4.390031in}}%
\pgfpathlineto{\pgfqpoint{1.618960in}{4.228802in}}%
\pgfpathlineto{\pgfqpoint{1.640788in}{4.080119in}}%
\pgfpathlineto{\pgfqpoint{1.657455in}{4.046589in}}%
\pgfpathlineto{\pgfqpoint{1.673651in}{4.037497in}}%
\pgfpathlineto{\pgfqpoint{1.692899in}{4.034753in}}%
\pgfpathlineto{\pgfqpoint{1.710504in}{4.037232in}}%
\pgfpathlineto{\pgfqpoint{1.731863in}{4.052488in}}%
\pgfpathlineto{\pgfqpoint{1.753225in}{4.115861in}}%
\pgfpathlineto{\pgfqpoint{1.771767in}{4.306697in}}%
\pgfpathlineto{\pgfqpoint{1.792189in}{4.418422in}}%
\pgfpathlineto{\pgfqpoint{1.809794in}{4.384771in}}%
\pgfpathlineto{\pgfqpoint{1.829746in}{4.247630in}}%
\pgfpathlineto{\pgfqpoint{1.849932in}{4.077168in}}%
\pgfpathlineto{\pgfqpoint{1.869885in}{4.045243in}}%
\pgfpathlineto{\pgfqpoint{1.885376in}{4.037331in}}%
\pgfpathlineto{\pgfqpoint{1.905798in}{4.034919in}}%
\pgfpathlineto{\pgfqpoint{1.926688in}{4.039588in}}%
\pgfpathlineto{\pgfqpoint{1.944998in}{4.055251in}}%
\pgfpathlineto{\pgfqpoint{1.963541in}{4.103450in}}%
\pgfpathlineto{\pgfqpoint{1.983025in}{4.311655in}}%
\pgfpathlineto{\pgfqpoint{2.002038in}{4.412767in}}%
\pgfpathlineto{\pgfqpoint{2.022458in}{4.378443in}}%
\pgfpathlineto{\pgfqpoint{2.037716in}{4.262091in}}%
\pgfpathlineto{\pgfqpoint{2.058138in}{4.093608in}}%
\pgfpathlineto{\pgfqpoint{2.080437in}{4.046645in}}%
\pgfpathlineto{\pgfqpoint{2.097571in}{4.036436in}}%
\pgfpathlineto{\pgfqpoint{2.115881in}{4.034718in}}%
\pgfpathlineto{\pgfqpoint{2.136303in}{4.037818in}}%
\pgfpathlineto{\pgfqpoint{2.154611in}{4.049202in}}%
\pgfpathlineto{\pgfqpoint{2.172450in}{4.086628in}}%
\pgfpathlineto{\pgfqpoint{2.211885in}{4.401998in}}%
\pgfpathlineto{\pgfqpoint{2.232072in}{4.380275in}}%
\pgfpathlineto{\pgfqpoint{2.253197in}{4.219308in}}%
\pgfpathlineto{\pgfqpoint{2.268690in}{4.140516in}}%
\pgfpathlineto{\pgfqpoint{2.289110in}{4.064002in}}%
\pgfpathlineto{\pgfqpoint{2.310472in}{4.039736in}}%
\pgfpathlineto{\pgfqpoint{2.327842in}{4.035170in}}%
\pgfpathlineto{\pgfqpoint{2.346150in}{4.036721in}}%
\pgfpathlineto{\pgfqpoint{2.364929in}{4.034797in}}%
\pgfpathlineto{\pgfqpoint{2.386054in}{4.039324in}}%
\pgfpathlineto{\pgfqpoint{2.403893in}{4.056380in}}%
\pgfpathlineto{\pgfqpoint{2.425723in}{4.129453in}}%
\pgfpathlineto{\pgfqpoint{2.443328in}{4.115849in}}%
\pgfpathlineto{\pgfqpoint{2.463984in}{4.349601in}}%
\pgfpathlineto{\pgfqpoint{2.482294in}{4.392723in}}%
\pgfpathlineto{\pgfqpoint{2.499663in}{4.403415in}}%
\pgfpathlineto{\pgfqpoint{2.521024in}{4.244119in}}%
\pgfpathlineto{\pgfqpoint{2.539332in}{4.096114in}}%
\pgfpathlineto{\pgfqpoint{2.563979in}{4.044016in}}%
\pgfpathlineto{\pgfqpoint{2.578298in}{4.038919in}}%
\pgfpathlineto{\pgfqpoint{2.595903in}{4.036159in}}%
\pgfpathlineto{\pgfqpoint{2.617262in}{4.035429in}}%
\pgfpathlineto{\pgfqpoint{2.636510in}{4.041979in}}%
\pgfpathlineto{\pgfqpoint{2.653880in}{4.062512in}}%
\pgfpathlineto{\pgfqpoint{2.672659in}{4.110950in}}%
\pgfpathlineto{\pgfqpoint{2.692376in}{4.294875in}}%
\pgfpathlineto{\pgfqpoint{2.713266in}{4.394569in}}%
\pgfpathlineto{\pgfqpoint{2.731106in}{4.397858in}}%
\pgfpathlineto{\pgfqpoint{2.767020in}{4.122374in}}%
\pgfpathlineto{\pgfqpoint{2.789788in}{4.053458in}}%
\pgfpathlineto{\pgfqpoint{2.806219in}{4.040581in}}%
\pgfpathlineto{\pgfqpoint{2.827815in}{4.035874in}}%
\pgfpathlineto{\pgfqpoint{2.845888in}{4.034981in}}%
\pgfpathlineto{\pgfqpoint{2.867484in}{4.041066in}}%
\pgfpathlineto{\pgfqpoint{2.885792in}{4.054746in}}%
\pgfpathlineto{\pgfqpoint{2.905745in}{4.106741in}}%
\pgfpathlineto{\pgfqpoint{2.923584in}{4.240283in}}%
\pgfpathlineto{\pgfqpoint{2.942129in}{4.390557in}}%
\pgfpathlineto{\pgfqpoint{2.963019in}{4.399090in}}%
\pgfpathlineto{\pgfqpoint{2.981327in}{4.287633in}}%
\pgfpathlineto{\pgfqpoint{2.998229in}{4.147690in}}%
\pgfpathlineto{\pgfqpoint{3.020059in}{4.063371in}}%
\pgfpathlineto{\pgfqpoint{3.040479in}{4.045051in}}%
\pgfpathlineto{\pgfqpoint{3.056441in}{4.038481in}}%
\pgfpathlineto{\pgfqpoint{3.076628in}{4.035732in}}%
\pgfpathlineto{\pgfqpoint{3.098693in}{4.036002in}}%
\pgfpathlineto{\pgfqpoint{3.115829in}{4.041058in}}%
\pgfpathlineto{\pgfqpoint{3.134371in}{4.051973in}}%
\pgfpathlineto{\pgfqpoint{3.155027in}{4.101456in}}%
\pgfpathlineto{\pgfqpoint{3.172866in}{4.230028in}}%
\pgfpathlineto{\pgfqpoint{3.194228in}{4.395868in}}%
\pgfpathlineto{\pgfqpoint{3.212301in}{4.411905in}}%
\pgfpathlineto{\pgfqpoint{3.230375in}{4.323430in}}%
\pgfpathlineto{\pgfqpoint{3.254553in}{4.128889in}}%
\pgfpathlineto{\pgfqpoint{3.269810in}{4.077098in}}%
\pgfpathlineto{\pgfqpoint{3.287649in}{4.052767in}}%
\pgfpathlineto{\pgfqpoint{3.308305in}{4.039173in}}%
\pgfpathlineto{\pgfqpoint{3.326615in}{4.035915in}}%
\pgfpathlineto{\pgfqpoint{3.347975in}{4.035406in}}%
\pgfpathlineto{\pgfqpoint{3.366285in}{4.039634in}}%
\pgfpathlineto{\pgfqpoint{3.386236in}{4.046910in}}%
\pgfpathlineto{\pgfqpoint{3.405249in}{4.074114in}}%
\pgfpathlineto{\pgfqpoint{3.423088in}{4.117390in}}%
\pgfpathlineto{\pgfqpoint{3.444684in}{4.305588in}}%
\pgfpathlineto{\pgfqpoint{3.462054in}{4.397965in}}%
\pgfpathlineto{\pgfqpoint{3.483414in}{4.412258in}}%
\pgfpathlineto{\pgfqpoint{3.501958in}{4.345300in}}%
\pgfpathlineto{\pgfqpoint{3.519092in}{4.194292in}}%
\pgfpathlineto{\pgfqpoint{3.537168in}{4.089268in}}%
\pgfpathlineto{\pgfqpoint{3.558996in}{4.056711in}}%
\pgfpathlineto{\pgfqpoint{3.575898in}{4.044258in}}%
\pgfpathlineto{\pgfqpoint{3.597257in}{4.036488in}}%
\pgfpathlineto{\pgfqpoint{3.617444in}{4.065142in}}%
\pgfpathlineto{\pgfqpoint{3.634580in}{4.043300in}}%
\pgfpathlineto{\pgfqpoint{3.654531in}{4.036270in}}%
\pgfpathlineto{\pgfqpoint{3.673076in}{4.036533in}}%
\pgfpathlineto{\pgfqpoint{3.691618in}{4.042177in}}%
\pgfpathlineto{\pgfqpoint{3.712040in}{4.051680in}}%
\pgfpathlineto{\pgfqpoint{3.731287in}{4.086815in}}%
\pgfpathlineto{\pgfqpoint{3.751709in}{4.188728in}}%
\pgfpathlineto{\pgfqpoint{3.769080in}{4.388527in}}%
\pgfpathlineto{\pgfqpoint{3.786684in}{4.434488in}}%
\pgfpathlineto{\pgfqpoint{3.804524in}{4.389228in}}%
\pgfpathlineto{\pgfqpoint{3.825649in}{4.270020in}}%
\pgfpathlineto{\pgfqpoint{3.847479in}{4.139649in}}%
\pgfpathlineto{\pgfqpoint{3.866726in}{4.071288in}}%
\pgfpathlineto{\pgfqpoint{3.884566in}{4.044739in}}%
\pgfpathlineto{\pgfqpoint{3.903345in}{4.038082in}}%
\pgfpathlineto{\pgfqpoint{3.924940in}{4.035827in}}%
\pgfpathlineto{\pgfqpoint{3.941137in}{4.038239in}}%
\pgfpathlineto{\pgfqpoint{3.962027in}{4.050382in}}%
\pgfpathlineto{\pgfqpoint{3.981275in}{4.075711in}}%
\pgfpathlineto{\pgfqpoint{3.999114in}{4.144553in}}%
\pgfpathlineto{\pgfqpoint{4.019770in}{4.340431in}}%
\pgfpathlineto{\pgfqpoint{4.038549in}{4.408652in}}%
\pgfpathlineto{\pgfqpoint{4.058031in}{4.441846in}}%
\pgfpathlineto{\pgfqpoint{4.077279in}{4.431834in}}%
\pgfpathlineto{\pgfqpoint{4.096058in}{4.313644in}}%
\pgfpathlineto{\pgfqpoint{4.114131in}{4.174827in}}%
\pgfpathlineto{\pgfqpoint{4.135024in}{4.085998in}}%
\pgfpathlineto{\pgfqpoint{4.153566in}{4.052148in}}%
\pgfpathlineto{\pgfqpoint{4.173048in}{4.041336in}}%
\pgfpathlineto{\pgfqpoint{4.192062in}{4.036711in}}%
\pgfpathlineto{\pgfqpoint{4.211075in}{4.037200in}}%
\pgfpathlineto{\pgfqpoint{4.230323in}{4.040540in}}%
\pgfpathlineto{\pgfqpoint{4.250041in}{4.053340in}}%
\pgfpathlineto{\pgfqpoint{4.268115in}{4.090691in}}%
\pgfpathlineto{\pgfqpoint{4.287362in}{4.147384in}}%
\pgfpathlineto{\pgfqpoint{4.306610in}{4.342316in}}%
\pgfpathlineto{\pgfqpoint{4.326797in}{4.450223in}}%
\pgfpathlineto{\pgfqpoint{4.345340in}{4.446244in}}%
\pgfpathlineto{\pgfqpoint{4.365527in}{4.354061in}}%
\pgfpathlineto{\pgfqpoint{4.382663in}{4.244285in}}%
\pgfpathlineto{\pgfqpoint{4.402614in}{4.146811in}}%
\pgfpathlineto{\pgfqpoint{4.425147in}{4.070172in}}%
\pgfpathlineto{\pgfqpoint{4.440875in}{4.048701in}}%
\pgfpathlineto{\pgfqpoint{4.459419in}{4.042843in}}%
\pgfpathlineto{\pgfqpoint{4.482187in}{4.065530in}}%
\pgfpathlineto{\pgfqpoint{4.483361in}{4.063736in}}%
\pgfpathlineto{\pgfqpoint{4.473973in}{4.090790in}}%
\pgfpathlineto{\pgfqpoint{4.456368in}{4.227631in}}%
\pgfpathlineto{\pgfqpoint{4.435241in}{4.426153in}}%
\pgfpathlineto{\pgfqpoint{4.418342in}{4.456515in}}%
\pgfpathlineto{\pgfqpoint{4.378203in}{4.091182in}}%
\pgfpathlineto{\pgfqpoint{4.359190in}{4.047447in}}%
\pgfpathlineto{\pgfqpoint{4.341585in}{4.049254in}}%
\pgfpathlineto{\pgfqpoint{4.320929in}{4.110546in}}%
\pgfpathlineto{\pgfqpoint{4.300037in}{4.327738in}}%
\pgfpathlineto{\pgfqpoint{4.280320in}{4.448796in}}%
\pgfpathlineto{\pgfqpoint{4.262012in}{4.403847in}}%
\pgfpathlineto{\pgfqpoint{4.245345in}{4.161808in}}%
\pgfpathlineto{\pgfqpoint{4.224923in}{4.058689in}}%
\pgfpathlineto{\pgfqpoint{4.206850in}{4.039833in}}%
\pgfpathlineto{\pgfqpoint{4.185490in}{4.036369in}}%
\pgfpathlineto{\pgfqpoint{4.168589in}{4.044145in}}%
\pgfpathlineto{\pgfqpoint{4.146995in}{4.080005in}}%
\pgfpathlineto{\pgfqpoint{4.129154in}{4.216700in}}%
\pgfpathlineto{\pgfqpoint{4.107794in}{4.414056in}}%
\pgfpathlineto{\pgfqpoint{4.090424in}{4.431710in}}%
\pgfpathlineto{\pgfqpoint{4.052163in}{4.086689in}}%
\pgfpathlineto{\pgfqpoint{4.031976in}{4.044498in}}%
\pgfpathlineto{\pgfqpoint{4.013199in}{4.036127in}}%
\pgfpathlineto{\pgfqpoint{3.993951in}{4.037955in}}%
\pgfpathlineto{\pgfqpoint{3.976581in}{4.050369in}}%
\pgfpathlineto{\pgfqpoint{3.955690in}{4.122785in}}%
\pgfpathlineto{\pgfqpoint{3.936442in}{4.308111in}}%
\pgfpathlineto{\pgfqpoint{3.917195in}{4.428129in}}%
\pgfpathlineto{\pgfqpoint{3.898416in}{4.404560in}}%
\pgfpathlineto{\pgfqpoint{3.876820in}{4.147722in}}%
\pgfpathlineto{\pgfqpoint{3.859450in}{4.076469in}}%
\pgfpathlineto{\pgfqpoint{3.835742in}{4.040604in}}%
\pgfpathlineto{\pgfqpoint{3.820954in}{4.036088in}}%
\pgfpathlineto{\pgfqpoint{3.802881in}{4.035695in}}%
\pgfpathlineto{\pgfqpoint{3.781756in}{4.044788in}}%
\pgfpathlineto{\pgfqpoint{3.764151in}{4.077111in}}%
\pgfpathlineto{\pgfqpoint{3.743024in}{4.211559in}}%
\pgfpathlineto{\pgfqpoint{3.727298in}{4.364289in}}%
\pgfpathlineto{\pgfqpoint{3.706172in}{4.424306in}}%
\pgfpathlineto{\pgfqpoint{3.685281in}{4.231374in}}%
\pgfpathlineto{\pgfqpoint{3.665799in}{4.118183in}}%
\pgfpathlineto{\pgfqpoint{3.647489in}{4.079266in}}%
\pgfpathlineto{\pgfqpoint{3.625190in}{4.043248in}}%
\pgfpathlineto{\pgfqpoint{3.609933in}{4.036762in}}%
\pgfpathlineto{\pgfqpoint{3.590451in}{4.035444in}}%
\pgfpathlineto{\pgfqpoint{3.571907in}{4.040836in}}%
\pgfpathlineto{\pgfqpoint{3.551016in}{4.067840in}}%
\pgfpathlineto{\pgfqpoint{3.531768in}{4.130038in}}%
\pgfpathlineto{\pgfqpoint{3.512521in}{4.312577in}}%
\pgfpathlineto{\pgfqpoint{3.495150in}{4.416142in}}%
\pgfpathlineto{\pgfqpoint{3.475668in}{4.368164in}}%
\pgfpathlineto{\pgfqpoint{3.435530in}{4.076816in}}%
\pgfpathlineto{\pgfqpoint{3.416986in}{4.050149in}}%
\pgfpathlineto{\pgfqpoint{3.398207in}{4.039016in}}%
\pgfpathlineto{\pgfqpoint{3.379664in}{4.035106in}}%
\pgfpathlineto{\pgfqpoint{3.358068in}{4.038974in}}%
\pgfpathlineto{\pgfqpoint{3.339526in}{4.053334in}}%
\pgfpathlineto{\pgfqpoint{3.320042in}{4.096290in}}%
\pgfpathlineto{\pgfqpoint{3.304785in}{4.193924in}}%
\pgfpathlineto{\pgfqpoint{3.282721in}{4.345656in}}%
\pgfpathlineto{\pgfqpoint{3.263004in}{4.053304in}}%
\pgfpathlineto{\pgfqpoint{3.248451in}{4.084756in}}%
\pgfpathlineto{\pgfqpoint{3.225681in}{4.199651in}}%
\pgfpathlineto{\pgfqpoint{3.207842in}{4.375687in}}%
\pgfpathlineto{\pgfqpoint{3.184369in}{4.399952in}}%
\pgfpathlineto{\pgfqpoint{3.168407in}{4.207868in}}%
\pgfpathlineto{\pgfqpoint{3.150099in}{4.086324in}}%
\pgfpathlineto{\pgfqpoint{3.128503in}{4.045400in}}%
\pgfpathlineto{\pgfqpoint{3.109255in}{4.036658in}}%
\pgfpathlineto{\pgfqpoint{3.090947in}{4.035101in}}%
\pgfpathlineto{\pgfqpoint{3.071465in}{4.040369in}}%
\pgfpathlineto{\pgfqpoint{3.050809in}{4.062438in}}%
\pgfpathlineto{\pgfqpoint{3.031090in}{4.132320in}}%
\pgfpathlineto{\pgfqpoint{3.013486in}{4.259087in}}%
\pgfpathlineto{\pgfqpoint{2.994472in}{4.385737in}}%
\pgfpathlineto{\pgfqpoint{2.975930in}{4.403641in}}%
\pgfpathlineto{\pgfqpoint{2.957151in}{4.228437in}}%
\pgfpathlineto{\pgfqpoint{2.935790in}{4.081638in}}%
\pgfpathlineto{\pgfqpoint{2.917482in}{4.046712in}}%
\pgfpathlineto{\pgfqpoint{2.895417in}{4.036211in}}%
\pgfpathlineto{\pgfqpoint{2.881098in}{4.034853in}}%
\pgfpathlineto{\pgfqpoint{2.862085in}{4.035892in}}%
\pgfpathlineto{\pgfqpoint{2.840020in}{4.045499in}}%
\pgfpathlineto{\pgfqpoint{2.820069in}{4.073592in}}%
\pgfpathlineto{\pgfqpoint{2.802230in}{4.157225in}}%
\pgfpathlineto{\pgfqpoint{2.783217in}{4.323124in}}%
\pgfpathlineto{\pgfqpoint{2.764907in}{4.411101in}}%
\pgfpathlineto{\pgfqpoint{2.745895in}{4.363264in}}%
\pgfpathlineto{\pgfqpoint{2.724534in}{4.140008in}}%
\pgfpathlineto{\pgfqpoint{2.705990in}{4.073502in}}%
\pgfpathlineto{\pgfqpoint{2.687213in}{4.043539in}}%
\pgfpathlineto{\pgfqpoint{2.669137in}{4.036556in}}%
\pgfpathlineto{\pgfqpoint{2.649655in}{4.034644in}}%
\pgfpathlineto{\pgfqpoint{2.627825in}{4.039835in}}%
\pgfpathlineto{\pgfqpoint{2.609282in}{4.053999in}}%
\pgfpathlineto{\pgfqpoint{2.592615in}{4.069822in}}%
\pgfpathlineto{\pgfqpoint{2.573604in}{4.160333in}}%
\pgfpathlineto{\pgfqpoint{2.552711in}{4.296058in}}%
\pgfpathlineto{\pgfqpoint{2.532995in}{4.378895in}}%
\pgfpathlineto{\pgfqpoint{2.515390in}{4.399724in}}%
\pgfpathlineto{\pgfqpoint{2.496142in}{4.216659in}}%
\pgfpathlineto{\pgfqpoint{2.475252in}{4.092532in}}%
\pgfpathlineto{\pgfqpoint{2.456707in}{4.050889in}}%
\pgfpathlineto{\pgfqpoint{2.438399in}{4.037919in}}%
\pgfpathlineto{\pgfqpoint{2.415395in}{4.034654in}}%
\pgfpathlineto{\pgfqpoint{2.397790in}{4.035748in}}%
\pgfpathlineto{\pgfqpoint{2.380891in}{4.041535in}}%
\pgfpathlineto{\pgfqpoint{2.359061in}{4.057605in}}%
\pgfpathlineto{\pgfqpoint{2.340282in}{4.085537in}}%
\pgfpathlineto{\pgfqpoint{2.320331in}{4.220641in}}%
\pgfpathlineto{\pgfqpoint{2.301786in}{4.374560in}}%
\pgfpathlineto{\pgfqpoint{2.286530in}{4.407006in}}%
\pgfpathlineto{\pgfqpoint{2.264934in}{4.322510in}}%
\pgfpathlineto{\pgfqpoint{2.245921in}{4.140549in}}%
\pgfpathlineto{\pgfqpoint{2.225970in}{4.062843in}}%
\pgfpathlineto{\pgfqpoint{2.202731in}{4.043236in}}%
\pgfpathlineto{\pgfqpoint{2.187709in}{4.037084in}}%
\pgfpathlineto{\pgfqpoint{2.168461in}{4.092449in}}%
\pgfpathlineto{\pgfqpoint{2.150856in}{4.049938in}}%
\pgfpathlineto{\pgfqpoint{2.131843in}{4.037408in}}%
\pgfpathlineto{\pgfqpoint{2.110247in}{4.034847in}}%
\pgfpathlineto{\pgfqpoint{2.091234in}{4.038144in}}%
\pgfpathlineto{\pgfqpoint{2.073629in}{4.049510in}}%
\pgfpathlineto{\pgfqpoint{2.054381in}{4.090396in}}%
\pgfpathlineto{\pgfqpoint{2.013774in}{4.378306in}}%
\pgfpathlineto{\pgfqpoint{1.995464in}{4.411516in}}%
\pgfpathlineto{\pgfqpoint{1.976217in}{4.331713in}}%
\pgfpathlineto{\pgfqpoint{1.959317in}{4.135084in}}%
\pgfpathlineto{\pgfqpoint{1.937016in}{4.057379in}}%
\pgfpathlineto{\pgfqpoint{1.918708in}{4.041132in}}%
\pgfpathlineto{\pgfqpoint{1.900634in}{4.035924in}}%
\pgfpathlineto{\pgfqpoint{1.878335in}{4.035242in}}%
\pgfpathlineto{\pgfqpoint{1.860731in}{4.038754in}}%
\pgfpathlineto{\pgfqpoint{1.841717in}{4.045987in}}%
\pgfpathlineto{\pgfqpoint{1.820827in}{4.088951in}}%
\pgfpathlineto{\pgfqpoint{1.801108in}{4.194613in}}%
\pgfpathlineto{\pgfqpoint{1.786320in}{4.322175in}}%
\pgfpathlineto{\pgfqpoint{1.764490in}{4.414289in}}%
\pgfpathlineto{\pgfqpoint{1.745477in}{4.398031in}}%
\pgfpathlineto{\pgfqpoint{1.727874in}{4.354251in}}%
\pgfpathlineto{\pgfqpoint{1.708156in}{4.151340in}}%
\pgfpathlineto{\pgfqpoint{1.687265in}{4.064730in}}%
\pgfpathlineto{\pgfqpoint{1.667549in}{4.045754in}}%
\pgfpathlineto{\pgfqpoint{1.649239in}{4.038557in}}%
\pgfpathlineto{\pgfqpoint{1.630931in}{4.035785in}}%
\pgfpathlineto{\pgfqpoint{1.613326in}{4.035401in}}%
\pgfpathlineto{\pgfqpoint{1.588679in}{4.041098in}}%
\pgfpathlineto{\pgfqpoint{1.571543in}{4.062668in}}%
\pgfpathlineto{\pgfqpoint{1.555346in}{4.109279in}}%
\pgfpathlineto{\pgfqpoint{1.534690in}{4.217002in}}%
\pgfpathlineto{\pgfqpoint{1.514505in}{4.243269in}}%
\pgfpathlineto{\pgfqpoint{1.495726in}{4.382767in}}%
\pgfpathlineto{\pgfqpoint{1.477182in}{4.423883in}}%
\pgfpathlineto{\pgfqpoint{1.459342in}{4.393057in}}%
\pgfpathlineto{\pgfqpoint{1.437749in}{4.167165in}}%
\pgfpathlineto{\pgfqpoint{1.418501in}{4.098649in}}%
\pgfpathlineto{\pgfqpoint{1.401599in}{4.058062in}}%
\pgfpathlineto{\pgfqpoint{1.379769in}{4.041377in}}%
\pgfpathlineto{\pgfqpoint{1.357705in}{4.036049in}}%
\pgfpathlineto{\pgfqpoint{1.342213in}{4.035475in}}%
\pgfpathlineto{\pgfqpoint{1.323200in}{4.038415in}}%
\pgfpathlineto{\pgfqpoint{1.307238in}{4.051400in}}%
\pgfpathlineto{\pgfqpoint{1.285408in}{4.072993in}}%
\pgfpathlineto{\pgfqpoint{1.266395in}{4.121481in}}%
\pgfpathlineto{\pgfqpoint{1.246209in}{4.274144in}}%
\pgfpathlineto{\pgfqpoint{1.225788in}{4.405570in}}%
\pgfpathlineto{\pgfqpoint{1.205835in}{4.432730in}}%
\pgfpathlineto{\pgfqpoint{1.187527in}{4.433894in}}%
\pgfpathlineto{\pgfqpoint{1.169217in}{4.353603in}}%
\pgfpathlineto{\pgfqpoint{1.150909in}{4.186049in}}%
\pgfpathlineto{\pgfqpoint{1.129313in}{4.084521in}}%
\pgfpathlineto{\pgfqpoint{1.113822in}{4.063586in}}%
\pgfpathlineto{\pgfqpoint{1.092226in}{4.045334in}}%
\pgfpathlineto{\pgfqpoint{1.074152in}{4.427641in}}%
\pgfpathlineto{\pgfqpoint{1.058659in}{4.416642in}}%
\pgfpathlineto{\pgfqpoint{1.033778in}{4.221363in}}%
\pgfpathlineto{\pgfqpoint{1.015704in}{4.099551in}}%
\pgfpathlineto{\pgfqpoint{0.999039in}{4.057929in}}%
\pgfpathlineto{\pgfqpoint{0.977209in}{4.040571in}}%
\pgfpathlineto{\pgfqpoint{0.960309in}{4.036163in}}%
\pgfpathlineto{\pgfqpoint{0.939651in}{4.038609in}}%
\pgfpathlineto{\pgfqpoint{0.918526in}{4.047502in}}%
\pgfpathlineto{\pgfqpoint{0.901156in}{4.073945in}}%
\pgfpathlineto{\pgfqpoint{0.880970in}{4.162508in}}%
\pgfpathlineto{\pgfqpoint{0.863600in}{4.315957in}}%
\pgfpathlineto{\pgfqpoint{0.842944in}{4.434992in}}%
\pgfpathlineto{\pgfqpoint{0.822522in}{4.440193in}}%
\pgfpathlineto{\pgfqpoint{0.784730in}{4.135223in}}%
\pgfpathlineto{\pgfqpoint{0.767125in}{4.077156in}}%
\pgfpathlineto{\pgfqpoint{0.746940in}{4.048916in}}%
\pgfpathlineto{\pgfqpoint{0.726753in}{4.038481in}}%
\pgfpathlineto{\pgfqpoint{0.707036in}{4.036849in}}%
\pgfpathlineto{\pgfqpoint{0.691074in}{4.040403in}}%
\pgfpathlineto{\pgfqpoint{0.669244in}{4.059057in}}%
\pgfpathlineto{\pgfqpoint{0.651405in}{4.081904in}}%
\pgfpathlineto{\pgfqpoint{0.651874in}{4.078476in}}%
\pgfpathlineto{\pgfqpoint{0.655865in}{4.066998in}}%
\pgfpathlineto{\pgfqpoint{0.677460in}{4.040180in}}%
\pgfpathlineto{\pgfqpoint{0.695534in}{4.037084in}}%
\pgfpathlineto{\pgfqpoint{0.714547in}{4.047937in}}%
\pgfpathlineto{\pgfqpoint{0.731212in}{4.091783in}}%
\pgfpathlineto{\pgfqpoint{0.753277in}{4.214889in}}%
\pgfpathlineto{\pgfqpoint{0.770882in}{4.437771in}}%
\pgfpathlineto{\pgfqpoint{0.793650in}{4.400562in}}%
\pgfpathlineto{\pgfqpoint{0.811725in}{4.215359in}}%
\pgfpathlineto{\pgfqpoint{0.830033in}{4.083119in}}%
\pgfpathlineto{\pgfqpoint{0.850455in}{4.044866in}}%
\pgfpathlineto{\pgfqpoint{0.869937in}{4.036242in}}%
\pgfpathlineto{\pgfqpoint{0.890593in}{4.041299in}}%
\pgfpathlineto{\pgfqpoint{0.906555in}{4.058791in}}%
\pgfpathlineto{\pgfqpoint{0.924629in}{4.127636in}}%
\pgfpathlineto{\pgfqpoint{0.944347in}{4.375784in}}%
\pgfpathlineto{\pgfqpoint{0.965707in}{4.428676in}}%
\pgfpathlineto{\pgfqpoint{0.981903in}{4.312521in}}%
\pgfpathlineto{\pgfqpoint{1.003733in}{4.109491in}}%
\pgfpathlineto{\pgfqpoint{1.022278in}{4.051970in}}%
\pgfpathlineto{\pgfqpoint{1.041525in}{4.037425in}}%
\pgfpathlineto{\pgfqpoint{1.059599in}{4.036252in}}%
\pgfpathlineto{\pgfqpoint{1.079081in}{4.046127in}}%
\pgfpathlineto{\pgfqpoint{1.099737in}{4.093718in}}%
\pgfpathlineto{\pgfqpoint{1.116168in}{4.212192in}}%
\pgfpathlineto{\pgfqpoint{1.136355in}{4.426838in}}%
\pgfpathlineto{\pgfqpoint{1.157717in}{4.372727in}}%
\pgfpathlineto{\pgfqpoint{1.175790in}{4.174381in}}%
\pgfpathlineto{\pgfqpoint{1.194803in}{4.068945in}}%
\pgfpathlineto{\pgfqpoint{1.216632in}{4.040029in}}%
\pgfpathlineto{\pgfqpoint{1.231890in}{4.035342in}}%
\pgfpathlineto{\pgfqpoint{1.250198in}{4.037777in}}%
\pgfpathlineto{\pgfqpoint{1.273906in}{4.055505in}}%
\pgfpathlineto{\pgfqpoint{1.291745in}{4.113175in}}%
\pgfpathlineto{\pgfqpoint{1.310290in}{4.323871in}}%
\pgfpathlineto{\pgfqpoint{1.331180in}{4.421174in}}%
\pgfpathlineto{\pgfqpoint{1.346673in}{4.319673in}}%
\pgfpathlineto{\pgfqpoint{1.368503in}{4.141221in}}%
\pgfpathlineto{\pgfqpoint{1.384463in}{4.069536in}}%
\pgfpathlineto{\pgfqpoint{1.406059in}{4.040452in}}%
\pgfpathlineto{\pgfqpoint{1.424838in}{4.035414in}}%
\pgfpathlineto{\pgfqpoint{1.446903in}{4.036463in}}%
\pgfpathlineto{\pgfqpoint{1.464037in}{4.046560in}}%
\pgfpathlineto{\pgfqpoint{1.482347in}{4.076487in}}%
\pgfpathlineto{\pgfqpoint{1.500889in}{4.103209in}}%
\pgfpathlineto{\pgfqpoint{1.521545in}{4.325953in}}%
\pgfpathlineto{\pgfqpoint{1.540324in}{4.421904in}}%
\pgfpathlineto{\pgfqpoint{1.560980in}{4.311438in}}%
\pgfpathlineto{\pgfqpoint{1.579290in}{4.166998in}}%
\pgfpathlineto{\pgfqpoint{1.596659in}{4.081826in}}%
\pgfpathlineto{\pgfqpoint{1.618254in}{4.043157in}}%
\pgfpathlineto{\pgfqpoint{1.635625in}{4.036069in}}%
\pgfpathlineto{\pgfqpoint{1.655812in}{4.035424in}}%
\pgfpathlineto{\pgfqpoint{1.673417in}{4.039879in}}%
\pgfpathlineto{\pgfqpoint{1.694542in}{4.059886in}}%
\pgfpathlineto{\pgfqpoint{1.712381in}{4.119230in}}%
\pgfpathlineto{\pgfqpoint{1.734446in}{4.359768in}}%
\pgfpathlineto{\pgfqpoint{1.751582in}{4.414693in}}%
\pgfpathlineto{\pgfqpoint{1.772707in}{4.372436in}}%
\pgfpathlineto{\pgfqpoint{1.790077in}{4.322342in}}%
\pgfpathlineto{\pgfqpoint{1.811436in}{4.137394in}}%
\pgfpathlineto{\pgfqpoint{1.830215in}{4.063213in}}%
\pgfpathlineto{\pgfqpoint{1.847820in}{4.040597in}}%
\pgfpathlineto{\pgfqpoint{1.865894in}{4.035495in}}%
\pgfpathlineto{\pgfqpoint{1.887255in}{4.035486in}}%
\pgfpathlineto{\pgfqpoint{1.908146in}{4.042953in}}%
\pgfpathlineto{\pgfqpoint{1.925045in}{4.068982in}}%
\pgfpathlineto{\pgfqpoint{1.942650in}{4.071332in}}%
\pgfpathlineto{\pgfqpoint{1.968940in}{4.034784in}}%
\pgfpathlineto{\pgfqpoint{1.981382in}{4.038661in}}%
\pgfpathlineto{\pgfqpoint{2.003681in}{4.051544in}}%
\pgfpathlineto{\pgfqpoint{2.021051in}{4.099543in}}%
\pgfpathlineto{\pgfqpoint{2.038420in}{4.251068in}}%
\pgfpathlineto{\pgfqpoint{2.060015in}{4.416380in}}%
\pgfpathlineto{\pgfqpoint{2.078089in}{4.343603in}}%
\pgfpathlineto{\pgfqpoint{2.097337in}{4.215749in}}%
\pgfpathlineto{\pgfqpoint{2.114004in}{4.067663in}}%
\pgfpathlineto{\pgfqpoint{2.134424in}{4.043793in}}%
\pgfpathlineto{\pgfqpoint{2.157662in}{4.035313in}}%
\pgfpathlineto{\pgfqpoint{2.175736in}{4.034720in}}%
\pgfpathlineto{\pgfqpoint{2.193577in}{4.039751in}}%
\pgfpathlineto{\pgfqpoint{2.212825in}{4.057407in}}%
\pgfpathlineto{\pgfqpoint{2.230664in}{4.121085in}}%
\pgfpathlineto{\pgfqpoint{2.252728in}{4.244282in}}%
\pgfpathlineto{\pgfqpoint{2.269394in}{4.410889in}}%
\pgfpathlineto{\pgfqpoint{2.292163in}{4.369755in}}%
\pgfpathlineto{\pgfqpoint{2.309063in}{4.220296in}}%
\pgfpathlineto{\pgfqpoint{2.332536in}{4.069154in}}%
\pgfpathlineto{\pgfqpoint{2.346619in}{4.048443in}}%
\pgfpathlineto{\pgfqpoint{2.366103in}{4.037245in}}%
\pgfpathlineto{\pgfqpoint{2.386993in}{4.034573in}}%
\pgfpathlineto{\pgfqpoint{2.405067in}{4.034598in}}%
\pgfpathlineto{\pgfqpoint{2.422672in}{4.036967in}}%
\pgfpathlineto{\pgfqpoint{2.443797in}{4.051404in}}%
\pgfpathlineto{\pgfqpoint{2.461872in}{4.082627in}}%
\pgfpathlineto{\pgfqpoint{2.483466in}{4.241424in}}%
\pgfpathlineto{\pgfqpoint{2.501307in}{4.405393in}}%
\pgfpathlineto{\pgfqpoint{2.519381in}{4.380095in}}%
\pgfpathlineto{\pgfqpoint{2.539803in}{4.210441in}}%
\pgfpathlineto{\pgfqpoint{2.556702in}{4.084314in}}%
\pgfpathlineto{\pgfqpoint{2.579941in}{4.044965in}}%
\pgfpathlineto{\pgfqpoint{2.597311in}{4.036755in}}%
\pgfpathlineto{\pgfqpoint{2.613977in}{4.034534in}}%
\pgfpathlineto{\pgfqpoint{2.633459in}{4.035403in}}%
\pgfpathlineto{\pgfqpoint{2.654586in}{4.038000in}}%
\pgfpathlineto{\pgfqpoint{2.675242in}{4.055339in}}%
\pgfpathlineto{\pgfqpoint{2.693081in}{4.111725in}}%
\pgfpathlineto{\pgfqpoint{2.714206in}{4.293138in}}%
\pgfpathlineto{\pgfqpoint{2.731811in}{4.409848in}}%
\pgfpathlineto{\pgfqpoint{2.748947in}{4.351610in}}%
\pgfpathlineto{\pgfqpoint{2.772420in}{4.187262in}}%
\pgfpathlineto{\pgfqpoint{2.789554in}{4.082120in}}%
\pgfpathlineto{\pgfqpoint{2.808567in}{4.047073in}}%
\pgfpathlineto{\pgfqpoint{2.829223in}{4.036816in}}%
\pgfpathlineto{\pgfqpoint{2.848002in}{4.034765in}}%
\pgfpathlineto{\pgfqpoint{2.864198in}{4.035679in}}%
\pgfpathlineto{\pgfqpoint{2.884854in}{4.042952in}}%
\pgfpathlineto{\pgfqpoint{2.906919in}{4.062315in}}%
\pgfpathlineto{\pgfqpoint{2.923819in}{4.104809in}}%
\pgfpathlineto{\pgfqpoint{2.943772in}{4.237240in}}%
\pgfpathlineto{\pgfqpoint{2.962314in}{4.405819in}}%
\pgfpathlineto{\pgfqpoint{2.981798in}{4.387651in}}%
\pgfpathlineto{\pgfqpoint{3.003157in}{4.202572in}}%
\pgfpathlineto{\pgfqpoint{3.021936in}{4.106142in}}%
\pgfpathlineto{\pgfqpoint{3.041184in}{4.051914in}}%
\pgfpathlineto{\pgfqpoint{3.059023in}{4.041939in}}%
\pgfpathlineto{\pgfqpoint{3.076159in}{4.036271in}}%
\pgfpathlineto{\pgfqpoint{3.098224in}{4.034888in}}%
\pgfpathlineto{\pgfqpoint{3.115358in}{4.037124in}}%
\pgfpathlineto{\pgfqpoint{3.136719in}{4.046467in}}%
\pgfpathlineto{\pgfqpoint{3.155498in}{4.070864in}}%
\pgfpathlineto{\pgfqpoint{3.173572in}{4.138547in}}%
\pgfpathlineto{\pgfqpoint{3.193993in}{4.333763in}}%
\pgfpathlineto{\pgfqpoint{3.211127in}{4.415879in}}%
\pgfpathlineto{\pgfqpoint{3.230611in}{4.394486in}}%
\pgfpathlineto{\pgfqpoint{3.251031in}{4.333239in}}%
\pgfpathlineto{\pgfqpoint{3.268636in}{4.417231in}}%
\pgfpathlineto{\pgfqpoint{3.291640in}{4.364672in}}%
\pgfpathlineto{\pgfqpoint{3.307133in}{4.211147in}}%
\pgfpathlineto{\pgfqpoint{3.329196in}{4.083119in}}%
\pgfpathlineto{\pgfqpoint{3.346332in}{4.047782in}}%
\pgfpathlineto{\pgfqpoint{3.365580in}{4.037994in}}%
\pgfpathlineto{\pgfqpoint{3.384827in}{4.034900in}}%
\pgfpathlineto{\pgfqpoint{3.404309in}{4.037368in}}%
\pgfpathlineto{\pgfqpoint{3.421914in}{4.041534in}}%
\pgfpathlineto{\pgfqpoint{3.442807in}{4.060309in}}%
\pgfpathlineto{\pgfqpoint{3.460646in}{4.041497in}}%
\pgfpathlineto{\pgfqpoint{3.479188in}{4.035147in}}%
\pgfpathlineto{\pgfqpoint{3.499376in}{4.035751in}}%
\pgfpathlineto{\pgfqpoint{3.517920in}{4.042317in}}%
\pgfpathlineto{\pgfqpoint{3.540219in}{4.067555in}}%
\pgfpathlineto{\pgfqpoint{3.557353in}{4.148021in}}%
\pgfpathlineto{\pgfqpoint{3.578480in}{4.372810in}}%
\pgfpathlineto{\pgfqpoint{3.596319in}{4.426447in}}%
\pgfpathlineto{\pgfqpoint{3.615098in}{4.351873in}}%
\pgfpathlineto{\pgfqpoint{3.635283in}{4.169932in}}%
\pgfpathlineto{\pgfqpoint{3.654297in}{4.078605in}}%
\pgfpathlineto{\pgfqpoint{3.671902in}{4.056338in}}%
\pgfpathlineto{\pgfqpoint{3.696549in}{4.036908in}}%
\pgfpathlineto{\pgfqpoint{3.710397in}{4.035198in}}%
\pgfpathlineto{\pgfqpoint{3.733870in}{4.036117in}}%
\pgfpathlineto{\pgfqpoint{3.749363in}{4.041929in}}%
\pgfpathlineto{\pgfqpoint{3.767906in}{4.058143in}}%
\pgfpathlineto{\pgfqpoint{3.789032in}{4.126429in}}%
\pgfpathlineto{\pgfqpoint{3.807341in}{4.281153in}}%
\pgfpathlineto{\pgfqpoint{3.829640in}{4.431902in}}%
\pgfpathlineto{\pgfqpoint{3.846305in}{4.418754in}}%
\pgfpathlineto{\pgfqpoint{3.867900in}{4.292208in}}%
\pgfpathlineto{\pgfqpoint{3.885271in}{4.152569in}}%
\pgfpathlineto{\pgfqpoint{3.904284in}{4.071944in}}%
\pgfpathlineto{\pgfqpoint{3.922358in}{4.050425in}}%
\pgfpathlineto{\pgfqpoint{3.941840in}{4.038332in}}%
\pgfpathlineto{\pgfqpoint{3.960619in}{4.035471in}}%
\pgfpathlineto{\pgfqpoint{3.981980in}{4.039010in}}%
\pgfpathlineto{\pgfqpoint{4.000288in}{4.047748in}}%
\pgfpathlineto{\pgfqpoint{4.021649in}{4.073220in}}%
\pgfpathlineto{\pgfqpoint{4.038783in}{4.158066in}}%
\pgfpathlineto{\pgfqpoint{4.057093in}{4.324663in}}%
\pgfpathlineto{\pgfqpoint{4.076341in}{4.035618in}}%
\pgfpathlineto{\pgfqpoint{4.094649in}{4.037520in}}%
\pgfpathlineto{\pgfqpoint{4.116714in}{4.052359in}}%
\pgfpathlineto{\pgfqpoint{4.136196in}{4.097126in}}%
\pgfpathlineto{\pgfqpoint{4.158966in}{4.255980in}}%
\pgfpathlineto{\pgfqpoint{4.171640in}{4.418818in}}%
\pgfpathlineto{\pgfqpoint{4.193236in}{4.438203in}}%
\pgfpathlineto{\pgfqpoint{4.212249in}{4.342554in}}%
\pgfpathlineto{\pgfqpoint{4.231731in}{4.173877in}}%
\pgfpathlineto{\pgfqpoint{4.250275in}{4.093059in}}%
\pgfpathlineto{\pgfqpoint{4.270932in}{4.050634in}}%
\pgfpathlineto{\pgfqpoint{4.288066in}{4.039099in}}%
\pgfpathlineto{\pgfqpoint{4.308253in}{4.036155in}}%
\pgfpathlineto{\pgfqpoint{4.327501in}{4.039641in}}%
\pgfpathlineto{\pgfqpoint{4.345340in}{4.052810in}}%
\pgfpathlineto{\pgfqpoint{4.364822in}{4.095317in}}%
\pgfpathlineto{\pgfqpoint{4.384775in}{4.209480in}}%
\pgfpathlineto{\pgfqpoint{4.404257in}{4.428488in}}%
\pgfpathlineto{\pgfqpoint{4.423270in}{4.458587in}}%
\pgfpathlineto{\pgfqpoint{4.441580in}{4.413717in}}%
\pgfpathlineto{\pgfqpoint{4.480779in}{4.136335in}}%
\pgfpathlineto{\pgfqpoint{4.481953in}{4.133271in}}%
\pgfpathlineto{\pgfqpoint{4.476085in}{4.182625in}}%
\pgfpathlineto{\pgfqpoint{4.454723in}{4.395878in}}%
\pgfpathlineto{\pgfqpoint{4.431721in}{4.454281in}}%
\pgfpathlineto{\pgfqpoint{4.416464in}{4.323081in}}%
\pgfpathlineto{\pgfqpoint{4.397920in}{4.109079in}}%
\pgfpathlineto{\pgfqpoint{4.378438in}{4.035735in}}%
\pgfpathlineto{\pgfqpoint{4.359190in}{4.037526in}}%
\pgfpathlineto{\pgfqpoint{4.339237in}{4.057043in}}%
\pgfpathlineto{\pgfqpoint{4.321632in}{4.133850in}}%
\pgfpathlineto{\pgfqpoint{4.300037in}{4.356596in}}%
\pgfpathlineto{\pgfqpoint{4.280320in}{4.450066in}}%
\pgfpathlineto{\pgfqpoint{4.263420in}{4.369595in}}%
\pgfpathlineto{\pgfqpoint{4.245816in}{4.144112in}}%
\pgfpathlineto{\pgfqpoint{4.226097in}{4.054394in}}%
\pgfpathlineto{\pgfqpoint{4.201686in}{4.036271in}}%
\pgfpathlineto{\pgfqpoint{4.185725in}{4.035750in}}%
\pgfpathlineto{\pgfqpoint{4.167180in}{4.044303in}}%
\pgfpathlineto{\pgfqpoint{4.145821in}{4.096559in}}%
\pgfpathlineto{\pgfqpoint{4.128685in}{4.263465in}}%
\pgfpathlineto{\pgfqpoint{4.105917in}{4.437235in}}%
\pgfpathlineto{\pgfqpoint{4.088781in}{4.410455in}}%
\pgfpathlineto{\pgfqpoint{4.071645in}{4.199463in}}%
\pgfpathlineto{\pgfqpoint{4.050989in}{4.065551in}}%
\pgfpathlineto{\pgfqpoint{4.034558in}{4.041790in}}%
\pgfpathlineto{\pgfqpoint{4.012728in}{4.034915in}}%
\pgfpathlineto{\pgfqpoint{3.994889in}{4.037795in}}%
\pgfpathlineto{\pgfqpoint{3.974467in}{4.057330in}}%
\pgfpathlineto{\pgfqpoint{3.953108in}{4.163932in}}%
\pgfpathlineto{\pgfqpoint{3.935268in}{4.340910in}}%
\pgfpathlineto{\pgfqpoint{3.919072in}{4.412398in}}%
\pgfpathlineto{\pgfqpoint{3.895599in}{4.406837in}}%
\pgfpathlineto{\pgfqpoint{3.881045in}{4.200908in}}%
\pgfpathlineto{\pgfqpoint{3.860858in}{4.076836in}}%
\pgfpathlineto{\pgfqpoint{3.839499in}{4.043562in}}%
\pgfpathlineto{\pgfqpoint{3.820017in}{4.035567in}}%
\pgfpathlineto{\pgfqpoint{3.803584in}{4.035446in}}%
\pgfpathlineto{\pgfqpoint{3.782225in}{4.046182in}}%
\pgfpathlineto{\pgfqpoint{3.761568in}{4.086456in}}%
\pgfpathlineto{\pgfqpoint{3.724482in}{4.395476in}}%
\pgfpathlineto{\pgfqpoint{3.705468in}{4.423396in}}%
\pgfpathlineto{\pgfqpoint{3.681056in}{4.189345in}}%
\pgfpathlineto{\pgfqpoint{3.665799in}{4.083076in}}%
\pgfpathlineto{\pgfqpoint{3.646082in}{4.057907in}}%
\pgfpathlineto{\pgfqpoint{3.626598in}{4.039526in}}%
\pgfpathlineto{\pgfqpoint{3.609230in}{4.034842in}}%
\pgfpathlineto{\pgfqpoint{3.587634in}{4.035948in}}%
\pgfpathlineto{\pgfqpoint{3.569795in}{4.040915in}}%
\pgfpathlineto{\pgfqpoint{3.551721in}{4.065783in}}%
\pgfpathlineto{\pgfqpoint{3.530360in}{4.191034in}}%
\pgfpathlineto{\pgfqpoint{3.513224in}{4.350554in}}%
\pgfpathlineto{\pgfqpoint{3.494916in}{4.421905in}}%
\pgfpathlineto{\pgfqpoint{3.474494in}{4.378238in}}%
\pgfpathlineto{\pgfqpoint{3.455952in}{4.169333in}}%
\pgfpathlineto{\pgfqpoint{3.432479in}{4.075108in}}%
\pgfpathlineto{\pgfqpoint{3.416517in}{4.048374in}}%
\pgfpathlineto{\pgfqpoint{3.399850in}{4.037225in}}%
\pgfpathlineto{\pgfqpoint{3.380367in}{4.034549in}}%
\pgfpathlineto{\pgfqpoint{3.360651in}{4.037186in}}%
\pgfpathlineto{\pgfqpoint{3.339995in}{4.048551in}}%
\pgfpathlineto{\pgfqpoint{3.317930in}{4.099601in}}%
\pgfpathlineto{\pgfqpoint{3.302437in}{4.188048in}}%
\pgfpathlineto{\pgfqpoint{3.283895in}{4.331321in}}%
\pgfpathlineto{\pgfqpoint{3.261125in}{4.413525in}}%
\pgfpathlineto{\pgfqpoint{3.247511in}{4.322316in}}%
\pgfpathlineto{\pgfqpoint{3.223335in}{4.125175in}}%
\pgfpathlineto{\pgfqpoint{3.207138in}{4.063540in}}%
\pgfpathlineto{\pgfqpoint{3.181317in}{4.038507in}}%
\pgfpathlineto{\pgfqpoint{3.166060in}{4.035190in}}%
\pgfpathlineto{\pgfqpoint{3.145873in}{4.034383in}}%
\pgfpathlineto{\pgfqpoint{3.126860in}{4.035726in}}%
\pgfpathlineto{\pgfqpoint{3.107612in}{4.044579in}}%
\pgfpathlineto{\pgfqpoint{3.092121in}{4.068703in}}%
\pgfpathlineto{\pgfqpoint{3.072403in}{4.136994in}}%
\pgfpathlineto{\pgfqpoint{3.054095in}{4.247043in}}%
\pgfpathlineto{\pgfqpoint{3.035081in}{4.382700in}}%
\pgfpathlineto{\pgfqpoint{3.013720in}{4.388385in}}%
\pgfpathlineto{\pgfqpoint{2.994238in}{4.221850in}}%
\pgfpathlineto{\pgfqpoint{2.975225in}{4.089828in}}%
\pgfpathlineto{\pgfqpoint{2.956917in}{4.050511in}}%
\pgfpathlineto{\pgfqpoint{2.935555in}{4.038743in}}%
\pgfpathlineto{\pgfqpoint{2.916778in}{4.034543in}}%
\pgfpathlineto{\pgfqpoint{2.899877in}{4.035444in}}%
\pgfpathlineto{\pgfqpoint{2.877812in}{4.045739in}}%
\pgfpathlineto{\pgfqpoint{2.854574in}{4.083063in}}%
\pgfpathlineto{\pgfqpoint{2.840256in}{4.150460in}}%
\pgfpathlineto{\pgfqpoint{2.823355in}{4.265121in}}%
\pgfpathlineto{\pgfqpoint{2.801759in}{4.361904in}}%
\pgfpathlineto{\pgfqpoint{2.783217in}{4.407348in}}%
\pgfpathlineto{\pgfqpoint{2.764438in}{4.396695in}}%
\pgfpathlineto{\pgfqpoint{2.744956in}{4.200275in}}%
\pgfpathlineto{\pgfqpoint{2.721717in}{4.071662in}}%
\pgfpathlineto{\pgfqpoint{2.706226in}{4.049117in}}%
\pgfpathlineto{\pgfqpoint{2.686507in}{4.037233in}}%
\pgfpathlineto{\pgfqpoint{2.667494in}{4.034398in}}%
\pgfpathlineto{\pgfqpoint{2.646369in}{4.036907in}}%
\pgfpathlineto{\pgfqpoint{2.630173in}{4.043917in}}%
\pgfpathlineto{\pgfqpoint{2.612568in}{4.073057in}}%
\pgfpathlineto{\pgfqpoint{2.589800in}{4.206346in}}%
\pgfpathlineto{\pgfqpoint{2.572195in}{4.345582in}}%
\pgfpathlineto{\pgfqpoint{2.552243in}{4.404401in}}%
\pgfpathlineto{\pgfqpoint{2.533229in}{4.405989in}}%
\pgfpathlineto{\pgfqpoint{2.515156in}{4.281373in}}%
\pgfpathlineto{\pgfqpoint{2.496377in}{4.115107in}}%
\pgfpathlineto{\pgfqpoint{2.475252in}{4.052902in}}%
\pgfpathlineto{\pgfqpoint{2.456004in}{4.039141in}}%
\pgfpathlineto{\pgfqpoint{2.439103in}{4.034467in}}%
\pgfpathlineto{\pgfqpoint{2.417272in}{4.035074in}}%
\pgfpathlineto{\pgfqpoint{2.398496in}{4.041021in}}%
\pgfpathlineto{\pgfqpoint{2.380186in}{4.055665in}}%
\pgfpathlineto{\pgfqpoint{2.359529in}{4.111091in}}%
\pgfpathlineto{\pgfqpoint{2.343333in}{4.210485in}}%
\pgfpathlineto{\pgfqpoint{2.321503in}{4.377136in}}%
\pgfpathlineto{\pgfqpoint{2.304838in}{4.405188in}}%
\pgfpathlineto{\pgfqpoint{2.283478in}{4.300422in}}%
\pgfpathlineto{\pgfqpoint{2.263291in}{4.222161in}}%
\pgfpathlineto{\pgfqpoint{2.245217in}{4.123736in}}%
\pgfpathlineto{\pgfqpoint{2.227142in}{4.060496in}}%
\pgfpathlineto{\pgfqpoint{2.206251in}{4.040809in}}%
\pgfpathlineto{\pgfqpoint{2.188178in}{4.035276in}}%
\pgfpathlineto{\pgfqpoint{2.166816in}{4.034992in}}%
\pgfpathlineto{\pgfqpoint{2.148039in}{4.038786in}}%
\pgfpathlineto{\pgfqpoint{2.129026in}{4.049113in}}%
\pgfpathlineto{\pgfqpoint{2.109308in}{4.082636in}}%
\pgfpathlineto{\pgfqpoint{2.093111in}{4.095149in}}%
\pgfpathlineto{\pgfqpoint{2.051799in}{4.385147in}}%
\pgfpathlineto{\pgfqpoint{2.033256in}{4.402238in}}%
\pgfpathlineto{\pgfqpoint{1.996404in}{4.109066in}}%
\pgfpathlineto{\pgfqpoint{1.972931in}{4.051643in}}%
\pgfpathlineto{\pgfqpoint{1.956029in}{4.040026in}}%
\pgfpathlineto{\pgfqpoint{1.938661in}{4.414267in}}%
\pgfpathlineto{\pgfqpoint{1.918474in}{4.250377in}}%
\pgfpathlineto{\pgfqpoint{1.899929in}{4.110264in}}%
\pgfpathlineto{\pgfqpoint{1.878335in}{4.050450in}}%
\pgfpathlineto{\pgfqpoint{1.860260in}{4.039775in}}%
\pgfpathlineto{\pgfqpoint{1.840309in}{4.034880in}}%
\pgfpathlineto{\pgfqpoint{1.821061in}{4.034968in}}%
\pgfpathlineto{\pgfqpoint{1.800874in}{4.041763in}}%
\pgfpathlineto{\pgfqpoint{1.785383in}{4.058545in}}%
\pgfpathlineto{\pgfqpoint{1.765899in}{4.147323in}}%
\pgfpathlineto{\pgfqpoint{1.743365in}{4.337764in}}%
\pgfpathlineto{\pgfqpoint{1.726700in}{4.395498in}}%
\pgfpathlineto{\pgfqpoint{1.706278in}{4.411337in}}%
\pgfpathlineto{\pgfqpoint{1.689848in}{4.351832in}}%
\pgfpathlineto{\pgfqpoint{1.668252in}{4.134235in}}%
\pgfpathlineto{\pgfqpoint{1.649707in}{4.064529in}}%
\pgfpathlineto{\pgfqpoint{1.630460in}{4.042359in}}%
\pgfpathlineto{\pgfqpoint{1.612386in}{4.038164in}}%
\pgfpathlineto{\pgfqpoint{1.587505in}{4.034660in}}%
\pgfpathlineto{\pgfqpoint{1.572482in}{4.036396in}}%
\pgfpathlineto{\pgfqpoint{1.553938in}{4.042030in}}%
\pgfpathlineto{\pgfqpoint{1.534456in}{4.059943in}}%
\pgfpathlineto{\pgfqpoint{1.514505in}{4.103935in}}%
\pgfpathlineto{\pgfqpoint{1.495257in}{4.227813in}}%
\pgfpathlineto{\pgfqpoint{1.476713in}{4.382757in}}%
\pgfpathlineto{\pgfqpoint{1.457231in}{4.425191in}}%
\pgfpathlineto{\pgfqpoint{1.439860in}{4.374782in}}%
\pgfpathlineto{\pgfqpoint{1.419204in}{4.164996in}}%
\pgfpathlineto{\pgfqpoint{1.401130in}{4.074971in}}%
\pgfpathlineto{\pgfqpoint{1.378595in}{4.045639in}}%
\pgfpathlineto{\pgfqpoint{1.361461in}{4.038993in}}%
\pgfpathlineto{\pgfqpoint{1.340334in}{4.034935in}}%
\pgfpathlineto{\pgfqpoint{1.326486in}{4.035252in}}%
\pgfpathlineto{\pgfqpoint{1.302778in}{4.040149in}}%
\pgfpathlineto{\pgfqpoint{1.284939in}{4.050466in}}%
\pgfpathlineto{\pgfqpoint{1.264283in}{4.094287in}}%
\pgfpathlineto{\pgfqpoint{1.247616in}{4.187657in}}%
\pgfpathlineto{\pgfqpoint{1.226022in}{4.357407in}}%
\pgfpathlineto{\pgfqpoint{1.208886in}{4.423951in}}%
\pgfpathlineto{\pgfqpoint{1.189170in}{4.433096in}}%
\pgfpathlineto{\pgfqpoint{1.170625in}{4.431552in}}%
\pgfpathlineto{\pgfqpoint{1.152083in}{4.302508in}}%
\pgfpathlineto{\pgfqpoint{1.130721in}{4.123541in}}%
\pgfpathlineto{\pgfqpoint{1.111474in}{4.064397in}}%
\pgfpathlineto{\pgfqpoint{1.095043in}{4.196836in}}%
\pgfpathlineto{\pgfqpoint{1.071570in}{4.073682in}}%
\pgfpathlineto{\pgfqpoint{1.050914in}{4.047273in}}%
\pgfpathlineto{\pgfqpoint{1.034249in}{4.038703in}}%
\pgfpathlineto{\pgfqpoint{1.015704in}{4.034996in}}%
\pgfpathlineto{\pgfqpoint{0.998099in}{4.036810in}}%
\pgfpathlineto{\pgfqpoint{0.975800in}{4.047466in}}%
\pgfpathlineto{\pgfqpoint{0.959135in}{4.069010in}}%
\pgfpathlineto{\pgfqpoint{0.940825in}{4.146183in}}%
\pgfpathlineto{\pgfqpoint{0.917823in}{4.260056in}}%
\pgfpathlineto{\pgfqpoint{0.898810in}{4.395605in}}%
\pgfpathlineto{\pgfqpoint{0.880265in}{4.444407in}}%
\pgfpathlineto{\pgfqpoint{0.861017in}{4.417588in}}%
\pgfpathlineto{\pgfqpoint{0.843647in}{4.225533in}}%
\pgfpathlineto{\pgfqpoint{0.821583in}{4.103480in}}%
\pgfpathlineto{\pgfqpoint{0.803978in}{4.060944in}}%
\pgfpathlineto{\pgfqpoint{0.788018in}{4.046466in}}%
\pgfpathlineto{\pgfqpoint{0.766188in}{4.037370in}}%
\pgfpathlineto{\pgfqpoint{0.746704in}{4.035833in}}%
\pgfpathlineto{\pgfqpoint{0.729335in}{4.040143in}}%
\pgfpathlineto{\pgfqpoint{0.707739in}{4.047229in}}%
\pgfpathlineto{\pgfqpoint{0.689900in}{4.074033in}}%
\pgfpathlineto{\pgfqpoint{0.669478in}{4.155765in}}%
\pgfpathlineto{\pgfqpoint{0.649291in}{4.040588in}}%
\pgfpathlineto{\pgfqpoint{0.649996in}{4.041196in}}%
\pgfpathlineto{\pgfqpoint{0.656333in}{4.047547in}}%
\pgfpathlineto{\pgfqpoint{0.675815in}{4.090469in}}%
\pgfpathlineto{\pgfqpoint{0.695300in}{4.240859in}}%
\pgfpathlineto{\pgfqpoint{0.716190in}{4.449827in}}%
\pgfpathlineto{\pgfqpoint{0.734498in}{4.395662in}}%
\pgfpathlineto{\pgfqpoint{0.753043in}{4.189325in}}%
\pgfpathlineto{\pgfqpoint{0.774636in}{4.061643in}}%
\pgfpathlineto{\pgfqpoint{0.790129in}{4.040998in}}%
\pgfpathlineto{\pgfqpoint{0.810786in}{4.035131in}}%
\pgfpathlineto{\pgfqpoint{0.829799in}{4.041287in}}%
\pgfpathlineto{\pgfqpoint{0.849047in}{4.067242in}}%
\pgfpathlineto{\pgfqpoint{0.867825in}{4.177912in}}%
\pgfpathlineto{\pgfqpoint{0.886602in}{4.417671in}}%
\pgfpathlineto{\pgfqpoint{0.909607in}{4.407853in}}%
\pgfpathlineto{\pgfqpoint{0.925334in}{4.250312in}}%
\pgfpathlineto{\pgfqpoint{0.942939in}{4.092859in}}%
\pgfpathlineto{\pgfqpoint{0.964767in}{4.043344in}}%
\pgfpathlineto{\pgfqpoint{0.985894in}{4.034627in}}%
\pgfpathlineto{\pgfqpoint{1.003499in}{4.036998in}}%
\pgfpathlineto{\pgfqpoint{1.022041in}{4.052959in}}%
\pgfpathlineto{\pgfqpoint{1.041994in}{4.098619in}}%
\pgfpathlineto{\pgfqpoint{1.076969in}{4.432136in}}%
\pgfpathlineto{\pgfqpoint{1.095983in}{4.373342in}}%
\pgfpathlineto{\pgfqpoint{1.118516in}{4.135878in}}%
\pgfpathlineto{\pgfqpoint{1.137295in}{4.055911in}}%
\pgfpathlineto{\pgfqpoint{1.156543in}{4.037764in}}%
\pgfpathlineto{\pgfqpoint{1.174851in}{4.034361in}}%
\pgfpathlineto{\pgfqpoint{1.193864in}{4.038045in}}%
\pgfpathlineto{\pgfqpoint{1.212408in}{4.053991in}}%
\pgfpathlineto{\pgfqpoint{1.234237in}{4.140868in}}%
\pgfpathlineto{\pgfqpoint{1.250669in}{4.328631in}}%
\pgfpathlineto{\pgfqpoint{1.272263in}{4.428163in}}%
\pgfpathlineto{\pgfqpoint{1.291042in}{4.371359in}}%
\pgfpathlineto{\pgfqpoint{1.312167in}{4.160976in}}%
\pgfpathlineto{\pgfqpoint{1.328834in}{4.066339in}}%
\pgfpathlineto{\pgfqpoint{1.348082in}{4.043118in}}%
\pgfpathlineto{\pgfqpoint{1.369441in}{4.035111in}}%
\pgfpathlineto{\pgfqpoint{1.387751in}{4.034752in}}%
\pgfpathlineto{\pgfqpoint{1.404651in}{4.039504in}}%
\pgfpathlineto{\pgfqpoint{1.425072in}{4.066894in}}%
\pgfpathlineto{\pgfqpoint{1.446903in}{4.191234in}}%
\pgfpathlineto{\pgfqpoint{1.462159in}{4.342838in}}%
\pgfpathlineto{\pgfqpoint{1.480704in}{4.414742in}}%
\pgfpathlineto{\pgfqpoint{1.501594in}{4.413588in}}%
\pgfpathlineto{\pgfqpoint{1.540090in}{4.098400in}}%
\pgfpathlineto{\pgfqpoint{1.560980in}{4.047612in}}%
\pgfpathlineto{\pgfqpoint{1.579525in}{4.039139in}}%
\pgfpathlineto{\pgfqpoint{1.596424in}{4.034457in}}%
\pgfpathlineto{\pgfqpoint{1.618489in}{4.035161in}}%
\pgfpathlineto{\pgfqpoint{1.635390in}{4.042475in}}%
\pgfpathlineto{\pgfqpoint{1.656281in}{4.060226in}}%
\pgfpathlineto{\pgfqpoint{1.674823in}{4.133180in}}%
\pgfpathlineto{\pgfqpoint{1.695481in}{4.274440in}}%
\pgfpathlineto{\pgfqpoint{1.714258in}{4.415347in}}%
\pgfpathlineto{\pgfqpoint{1.732803in}{4.352370in}}%
\pgfpathlineto{\pgfqpoint{1.751345in}{4.209830in}}%
\pgfpathlineto{\pgfqpoint{1.771767in}{4.077256in}}%
\pgfpathlineto{\pgfqpoint{1.791015in}{4.043252in}}%
\pgfpathlineto{\pgfqpoint{1.811202in}{4.035094in}}%
\pgfpathlineto{\pgfqpoint{1.828338in}{4.033813in}}%
\pgfpathlineto{\pgfqpoint{1.847586in}{4.035617in}}%
\pgfpathlineto{\pgfqpoint{1.868945in}{4.040979in}}%
\pgfpathlineto{\pgfqpoint{1.887489in}{4.057307in}}%
\pgfpathlineto{\pgfqpoint{1.904389in}{4.117473in}}%
\pgfpathlineto{\pgfqpoint{1.926219in}{4.307144in}}%
\pgfpathlineto{\pgfqpoint{1.943121in}{4.405342in}}%
\pgfpathlineto{\pgfqpoint{1.961663in}{4.041029in}}%
\pgfpathlineto{\pgfqpoint{1.981851in}{4.071648in}}%
\pgfpathlineto{\pgfqpoint{2.001567in}{4.182440in}}%
\pgfpathlineto{\pgfqpoint{2.021051in}{4.403747in}}%
\pgfpathlineto{\pgfqpoint{2.038420in}{4.394972in}}%
\pgfpathlineto{\pgfqpoint{2.060484in}{4.222119in}}%
\pgfpathlineto{\pgfqpoint{2.079497in}{4.082505in}}%
\pgfpathlineto{\pgfqpoint{2.096633in}{4.046609in}}%
\pgfpathlineto{\pgfqpoint{2.117290in}{4.035659in}}%
\pgfpathlineto{\pgfqpoint{2.135832in}{4.033907in}}%
\pgfpathlineto{\pgfqpoint{2.156254in}{4.036492in}}%
\pgfpathlineto{\pgfqpoint{2.174798in}{4.044266in}}%
\pgfpathlineto{\pgfqpoint{2.195220in}{4.083084in}}%
\pgfpathlineto{\pgfqpoint{2.213293in}{4.149009in}}%
\pgfpathlineto{\pgfqpoint{2.236532in}{4.398904in}}%
\pgfpathlineto{\pgfqpoint{2.252023in}{4.411774in}}%
\pgfpathlineto{\pgfqpoint{2.269628in}{4.310166in}}%
\pgfpathlineto{\pgfqpoint{2.289581in}{4.117407in}}%
\pgfpathlineto{\pgfqpoint{2.312114in}{4.052999in}}%
\pgfpathlineto{\pgfqpoint{2.327842in}{4.039553in}}%
\pgfpathlineto{\pgfqpoint{2.345212in}{4.034637in}}%
\pgfpathlineto{\pgfqpoint{2.365398in}{4.034472in}}%
\pgfpathlineto{\pgfqpoint{2.386054in}{4.040695in}}%
\pgfpathlineto{\pgfqpoint{2.407181in}{4.058531in}}%
\pgfpathlineto{\pgfqpoint{2.421969in}{4.095673in}}%
\pgfpathlineto{\pgfqpoint{2.442390in}{4.224901in}}%
\pgfpathlineto{\pgfqpoint{2.466332in}{4.411289in}}%
\pgfpathlineto{\pgfqpoint{2.481355in}{4.373179in}}%
\pgfpathlineto{\pgfqpoint{2.502479in}{4.198798in}}%
\pgfpathlineto{\pgfqpoint{2.523136in}{4.084359in}}%
\pgfpathlineto{\pgfqpoint{2.538160in}{4.050018in}}%
\pgfpathlineto{\pgfqpoint{2.559285in}{4.036878in}}%
\pgfpathlineto{\pgfqpoint{2.598249in}{4.034400in}}%
\pgfpathlineto{\pgfqpoint{2.615385in}{4.038283in}}%
\pgfpathlineto{\pgfqpoint{2.637215in}{4.052008in}}%
\pgfpathlineto{\pgfqpoint{2.654115in}{4.096448in}}%
\pgfpathlineto{\pgfqpoint{2.672188in}{4.178476in}}%
\pgfpathlineto{\pgfqpoint{2.693081in}{4.367547in}}%
\pgfpathlineto{\pgfqpoint{2.711623in}{4.403685in}}%
\pgfpathlineto{\pgfqpoint{2.729933in}{4.301515in}}%
\pgfpathlineto{\pgfqpoint{2.749884in}{4.132260in}}%
\pgfpathlineto{\pgfqpoint{2.770775in}{4.055461in}}%
\pgfpathlineto{\pgfqpoint{2.789085in}{4.040579in}}%
\pgfpathlineto{\pgfqpoint{2.828520in}{4.034254in}}%
\pgfpathlineto{\pgfqpoint{2.846359in}{4.035082in}}%
\pgfpathlineto{\pgfqpoint{2.868189in}{4.038305in}}%
\pgfpathlineto{\pgfqpoint{2.885792in}{4.052315in}}%
\pgfpathlineto{\pgfqpoint{2.902694in}{4.075502in}}%
\pgfpathlineto{\pgfqpoint{2.924758in}{4.199302in}}%
\pgfpathlineto{\pgfqpoint{2.941658in}{4.350063in}}%
\pgfpathlineto{\pgfqpoint{2.963488in}{4.406271in}}%
\pgfpathlineto{\pgfqpoint{2.999637in}{4.184730in}}%
\pgfpathlineto{\pgfqpoint{3.020528in}{4.083621in}}%
\pgfpathlineto{\pgfqpoint{3.038602in}{4.047175in}}%
\pgfpathlineto{\pgfqpoint{3.060432in}{4.036327in}}%
\pgfpathlineto{\pgfqpoint{3.077568in}{4.034191in}}%
\pgfpathlineto{\pgfqpoint{3.096110in}{4.035927in}}%
\pgfpathlineto{\pgfqpoint{3.120288in}{4.042229in}}%
\pgfpathlineto{\pgfqpoint{3.135311in}{4.057494in}}%
\pgfpathlineto{\pgfqpoint{3.153384in}{4.104557in}}%
\pgfpathlineto{\pgfqpoint{3.174040in}{4.247284in}}%
\pgfpathlineto{\pgfqpoint{3.192819in}{4.396020in}}%
\pgfpathlineto{\pgfqpoint{3.209955in}{4.413713in}}%
\pgfpathlineto{\pgfqpoint{3.230375in}{4.333390in}}%
\pgfpathlineto{\pgfqpoint{3.255257in}{4.262957in}}%
\pgfpathlineto{\pgfqpoint{3.269107in}{4.128059in}}%
\pgfpathlineto{\pgfqpoint{3.288589in}{4.059054in}}%
\pgfpathlineto{\pgfqpoint{3.310888in}{4.039779in}}%
\pgfpathlineto{\pgfqpoint{3.331075in}{4.034736in}}%
\pgfpathlineto{\pgfqpoint{3.349149in}{4.034740in}}%
\pgfpathlineto{\pgfqpoint{3.366285in}{4.037966in}}%
\pgfpathlineto{\pgfqpoint{3.383890in}{4.045209in}}%
\pgfpathlineto{\pgfqpoint{3.406189in}{4.072821in}}%
\pgfpathlineto{\pgfqpoint{3.422619in}{4.141755in}}%
\pgfpathlineto{\pgfqpoint{3.462758in}{4.416772in}}%
\pgfpathlineto{\pgfqpoint{3.481771in}{4.416454in}}%
\pgfpathlineto{\pgfqpoint{3.502661in}{4.336018in}}%
\pgfpathlineto{\pgfqpoint{3.519329in}{4.190170in}}%
\pgfpathlineto{\pgfqpoint{3.540688in}{4.079914in}}%
\pgfpathlineto{\pgfqpoint{3.558293in}{4.052784in}}%
\pgfpathlineto{\pgfqpoint{3.575898in}{4.043505in}}%
\pgfpathlineto{\pgfqpoint{3.596788in}{4.035952in}}%
\pgfpathlineto{\pgfqpoint{3.614862in}{4.034975in}}%
\pgfpathlineto{\pgfqpoint{3.633172in}{4.037056in}}%
\pgfpathlineto{\pgfqpoint{3.655471in}{4.043434in}}%
\pgfpathlineto{\pgfqpoint{3.676361in}{4.065955in}}%
\pgfpathlineto{\pgfqpoint{3.693497in}{4.105698in}}%
\pgfpathlineto{\pgfqpoint{3.710868in}{4.183774in}}%
\pgfpathlineto{\pgfqpoint{3.728941in}{4.393046in}}%
\pgfpathlineto{\pgfqpoint{3.750066in}{4.431824in}}%
\pgfpathlineto{\pgfqpoint{3.771193in}{4.371162in}}%
\pgfpathlineto{\pgfqpoint{3.789267in}{4.236940in}}%
\pgfpathlineto{\pgfqpoint{3.810626in}{4.121638in}}%
\pgfpathlineto{\pgfqpoint{3.827997in}{4.100151in}}%
\pgfpathlineto{\pgfqpoint{3.845836in}{4.063487in}}%
\pgfpathlineto{\pgfqpoint{3.866963in}{4.130273in}}%
\pgfpathlineto{\pgfqpoint{3.885505in}{4.062475in}}%
\pgfpathlineto{\pgfqpoint{3.903579in}{4.042823in}}%
\pgfpathlineto{\pgfqpoint{3.922123in}{4.036386in}}%
\pgfpathlineto{\pgfqpoint{3.942545in}{4.035436in}}%
\pgfpathlineto{\pgfqpoint{3.963201in}{4.040497in}}%
\pgfpathlineto{\pgfqpoint{3.980806in}{4.050543in}}%
\pgfpathlineto{\pgfqpoint{3.998176in}{4.072718in}}%
\pgfpathlineto{\pgfqpoint{4.019067in}{4.146970in}}%
\pgfpathlineto{\pgfqpoint{4.037844in}{4.343521in}}%
\pgfpathlineto{\pgfqpoint{4.058971in}{4.435822in}}%
\pgfpathlineto{\pgfqpoint{4.077513in}{4.432432in}}%
\pgfpathlineto{\pgfqpoint{4.095354in}{4.357829in}}%
\pgfpathlineto{\pgfqpoint{4.116245in}{4.188975in}}%
\pgfpathlineto{\pgfqpoint{4.133615in}{4.122602in}}%
\pgfpathlineto{\pgfqpoint{4.152392in}{4.063465in}}%
\pgfpathlineto{\pgfqpoint{4.172814in}{4.041755in}}%
\pgfpathlineto{\pgfqpoint{4.192062in}{4.039571in}}%
\pgfpathlineto{\pgfqpoint{4.212952in}{4.035894in}}%
\pgfpathlineto{\pgfqpoint{4.230088in}{4.036750in}}%
\pgfpathlineto{\pgfqpoint{4.248396in}{4.042113in}}%
\pgfpathlineto{\pgfqpoint{4.271635in}{4.068129in}}%
\pgfpathlineto{\pgfqpoint{4.289474in}{4.122859in}}%
\pgfpathlineto{\pgfqpoint{4.325154in}{4.407826in}}%
\pgfpathlineto{\pgfqpoint{4.345811in}{4.456832in}}%
\pgfpathlineto{\pgfqpoint{4.365527in}{4.424648in}}%
\pgfpathlineto{\pgfqpoint{4.384306in}{4.340551in}}%
\pgfpathlineto{\pgfqpoint{4.405431in}{4.196112in}}%
\pgfpathlineto{\pgfqpoint{4.423505in}{4.108136in}}%
\pgfpathlineto{\pgfqpoint{4.442518in}{4.176989in}}%
\pgfpathlineto{\pgfqpoint{4.459888in}{4.096558in}}%
\pgfpathlineto{\pgfqpoint{4.481718in}{4.050528in}}%
\pgfpathlineto{\pgfqpoint{4.476553in}{4.058277in}}%
\pgfpathlineto{\pgfqpoint{4.456132in}{4.144963in}}%
\pgfpathlineto{\pgfqpoint{4.434303in}{4.380131in}}%
\pgfpathlineto{\pgfqpoint{4.416464in}{4.456506in}}%
\pgfpathlineto{\pgfqpoint{4.396277in}{4.393008in}}%
\pgfpathlineto{\pgfqpoint{4.379141in}{4.154152in}}%
\pgfpathlineto{\pgfqpoint{4.358250in}{4.060482in}}%
\pgfpathlineto{\pgfqpoint{4.335012in}{4.037400in}}%
\pgfpathlineto{\pgfqpoint{4.320458in}{4.036127in}}%
\pgfpathlineto{\pgfqpoint{4.299568in}{4.046610in}}%
\pgfpathlineto{\pgfqpoint{4.282903in}{4.084662in}}%
\pgfpathlineto{\pgfqpoint{4.262246in}{4.245546in}}%
\pgfpathlineto{\pgfqpoint{4.245110in}{4.411760in}}%
\pgfpathlineto{\pgfqpoint{4.224689in}{4.440880in}}%
\pgfpathlineto{\pgfqpoint{4.203798in}{4.223146in}}%
\pgfpathlineto{\pgfqpoint{4.186428in}{4.078831in}}%
\pgfpathlineto{\pgfqpoint{4.167180in}{4.043965in}}%
\pgfpathlineto{\pgfqpoint{4.148403in}{4.035738in}}%
\pgfpathlineto{\pgfqpoint{4.128216in}{4.038717in}}%
\pgfpathlineto{\pgfqpoint{4.110611in}{4.053921in}}%
\pgfpathlineto{\pgfqpoint{4.090189in}{4.132472in}}%
\pgfpathlineto{\pgfqpoint{4.069299in}{4.346850in}}%
\pgfpathlineto{\pgfqpoint{4.048643in}{4.439277in}}%
\pgfpathlineto{\pgfqpoint{4.031272in}{4.359225in}}%
\pgfpathlineto{\pgfqpoint{4.013902in}{4.127580in}}%
\pgfpathlineto{\pgfqpoint{3.997237in}{4.055281in}}%
\pgfpathlineto{\pgfqpoint{3.975875in}{4.039957in}}%
\pgfpathlineto{\pgfqpoint{3.958507in}{4.035097in}}%
\pgfpathlineto{\pgfqpoint{3.935268in}{4.039556in}}%
\pgfpathlineto{\pgfqpoint{3.918603in}{4.058091in}}%
\pgfpathlineto{\pgfqpoint{3.895833in}{4.136791in}}%
\pgfpathlineto{\pgfqpoint{3.878229in}{4.315275in}}%
\pgfpathlineto{\pgfqpoint{3.858512in}{4.430000in}}%
\pgfpathlineto{\pgfqpoint{3.839733in}{4.343074in}}%
\pgfpathlineto{\pgfqpoint{3.821660in}{4.129088in}}%
\pgfpathlineto{\pgfqpoint{3.801472in}{4.059412in}}%
\pgfpathlineto{\pgfqpoint{3.780582in}{4.040310in}}%
\pgfpathlineto{\pgfqpoint{3.763446in}{4.035236in}}%
\pgfpathlineto{\pgfqpoint{3.743258in}{4.037472in}}%
\pgfpathlineto{\pgfqpoint{3.724482in}{4.056778in}}%
\pgfpathlineto{\pgfqpoint{3.704529in}{4.112195in}}%
\pgfpathlineto{\pgfqpoint{3.686689in}{4.269304in}}%
\pgfpathlineto{\pgfqpoint{3.667676in}{4.412518in}}%
\pgfpathlineto{\pgfqpoint{3.649134in}{4.397651in}}%
\pgfpathlineto{\pgfqpoint{3.627538in}{4.148275in}}%
\pgfpathlineto{\pgfqpoint{3.609230in}{4.065070in}}%
\pgfpathlineto{\pgfqpoint{3.588337in}{4.040062in}}%
\pgfpathlineto{\pgfqpoint{3.570498in}{4.167643in}}%
\pgfpathlineto{\pgfqpoint{3.551956in}{4.090915in}}%
\pgfpathlineto{\pgfqpoint{3.534116in}{4.052699in}}%
\pgfpathlineto{\pgfqpoint{3.514867in}{4.037992in}}%
\pgfpathlineto{\pgfqpoint{3.494211in}{4.034669in}}%
\pgfpathlineto{\pgfqpoint{3.474729in}{4.037965in}}%
\pgfpathlineto{\pgfqpoint{3.456186in}{4.050531in}}%
\pgfpathlineto{\pgfqpoint{3.435764in}{4.039472in}}%
\pgfpathlineto{\pgfqpoint{3.418628in}{4.059235in}}%
\pgfpathlineto{\pgfqpoint{3.396095in}{4.151980in}}%
\pgfpathlineto{\pgfqpoint{3.378256in}{4.332669in}}%
\pgfpathlineto{\pgfqpoint{3.359946in}{4.417551in}}%
\pgfpathlineto{\pgfqpoint{3.341638in}{4.305824in}}%
\pgfpathlineto{\pgfqpoint{3.319808in}{4.118629in}}%
\pgfpathlineto{\pgfqpoint{3.301734in}{4.056818in}}%
\pgfpathlineto{\pgfqpoint{3.282721in}{4.039128in}}%
\pgfpathlineto{\pgfqpoint{3.264176in}{4.034709in}}%
\pgfpathlineto{\pgfqpoint{3.241643in}{4.038477in}}%
\pgfpathlineto{\pgfqpoint{3.224038in}{4.047266in}}%
\pgfpathlineto{\pgfqpoint{3.205259in}{4.092549in}}%
\pgfpathlineto{\pgfqpoint{3.186951in}{4.252083in}}%
\pgfpathlineto{\pgfqpoint{3.167469in}{4.380051in}}%
\pgfpathlineto{\pgfqpoint{3.148690in}{4.406787in}}%
\pgfpathlineto{\pgfqpoint{3.130382in}{4.210342in}}%
\pgfpathlineto{\pgfqpoint{3.108317in}{4.074227in}}%
\pgfpathlineto{\pgfqpoint{3.091650in}{4.059304in}}%
\pgfpathlineto{\pgfqpoint{3.072168in}{4.038906in}}%
\pgfpathlineto{\pgfqpoint{3.053389in}{4.034597in}}%
\pgfpathlineto{\pgfqpoint{3.028508in}{4.039135in}}%
\pgfpathlineto{\pgfqpoint{3.013017in}{4.049739in}}%
\pgfpathlineto{\pgfqpoint{2.994472in}{4.097098in}}%
\pgfpathlineto{\pgfqpoint{2.976164in}{4.256961in}}%
\pgfpathlineto{\pgfqpoint{2.954100in}{4.403468in}}%
\pgfpathlineto{\pgfqpoint{2.936026in}{4.408078in}}%
\pgfpathlineto{\pgfqpoint{2.918185in}{4.206676in}}%
\pgfpathlineto{\pgfqpoint{2.898234in}{4.082180in}}%
\pgfpathlineto{\pgfqpoint{2.877109in}{4.047904in}}%
\pgfpathlineto{\pgfqpoint{2.861616in}{4.038307in}}%
\pgfpathlineto{\pgfqpoint{2.843308in}{4.034645in}}%
\pgfpathlineto{\pgfqpoint{2.820304in}{4.036656in}}%
\pgfpathlineto{\pgfqpoint{2.800351in}{4.047782in}}%
\pgfpathlineto{\pgfqpoint{2.782277in}{4.071759in}}%
\pgfpathlineto{\pgfqpoint{2.764203in}{4.171605in}}%
\pgfpathlineto{\pgfqpoint{2.742608in}{4.375099in}}%
\pgfpathlineto{\pgfqpoint{2.725942in}{4.411135in}}%
\pgfpathlineto{\pgfqpoint{2.705286in}{4.224686in}}%
\pgfpathlineto{\pgfqpoint{2.686273in}{4.089945in}}%
\pgfpathlineto{\pgfqpoint{2.667494in}{4.051339in}}%
\pgfpathlineto{\pgfqpoint{2.649421in}{4.038547in}}%
\pgfpathlineto{\pgfqpoint{2.628296in}{4.034480in}}%
\pgfpathlineto{\pgfqpoint{2.610456in}{4.035242in}}%
\pgfpathlineto{\pgfqpoint{2.591209in}{4.041733in}}%
\pgfpathlineto{\pgfqpoint{2.569613in}{4.071185in}}%
\pgfpathlineto{\pgfqpoint{2.553885in}{4.134038in}}%
\pgfpathlineto{\pgfqpoint{2.532292in}{4.300221in}}%
\pgfpathlineto{\pgfqpoint{2.513278in}{4.407130in}}%
\pgfpathlineto{\pgfqpoint{2.494499in}{4.356014in}}%
\pgfpathlineto{\pgfqpoint{2.477834in}{4.173829in}}%
\pgfpathlineto{\pgfqpoint{2.451074in}{4.074731in}}%
\pgfpathlineto{\pgfqpoint{2.417038in}{4.039535in}}%
\pgfpathlineto{\pgfqpoint{2.398730in}{4.034855in}}%
\pgfpathlineto{\pgfqpoint{2.379717in}{4.039082in}}%
\pgfpathlineto{\pgfqpoint{2.360000in}{4.034482in}}%
\pgfpathlineto{\pgfqpoint{2.341925in}{4.035437in}}%
\pgfpathlineto{\pgfqpoint{2.323382in}{4.042195in}}%
\pgfpathlineto{\pgfqpoint{2.301552in}{4.068801in}}%
\pgfpathlineto{\pgfqpoint{2.281835in}{4.178833in}}%
\pgfpathlineto{\pgfqpoint{2.263994in}{4.306308in}}%
\pgfpathlineto{\pgfqpoint{2.243574in}{4.407234in}}%
\pgfpathlineto{\pgfqpoint{2.225499in}{4.346718in}}%
\pgfpathlineto{\pgfqpoint{2.206722in}{4.169937in}}%
\pgfpathlineto{\pgfqpoint{2.188881in}{4.073167in}}%
\pgfpathlineto{\pgfqpoint{2.167990in}{4.044396in}}%
\pgfpathlineto{\pgfqpoint{2.148743in}{4.036411in}}%
\pgfpathlineto{\pgfqpoint{2.129026in}{4.034787in}}%
\pgfpathlineto{\pgfqpoint{2.088888in}{4.043950in}}%
\pgfpathlineto{\pgfqpoint{2.071517in}{4.067971in}}%
\pgfpathlineto{\pgfqpoint{2.052504in}{4.147503in}}%
\pgfpathlineto{\pgfqpoint{2.035134in}{4.298241in}}%
\pgfpathlineto{\pgfqpoint{2.010721in}{4.409817in}}%
\pgfpathlineto{\pgfqpoint{1.998750in}{4.371079in}}%
\pgfpathlineto{\pgfqpoint{1.976217in}{4.186791in}}%
\pgfpathlineto{\pgfqpoint{1.958377in}{4.101644in}}%
\pgfpathlineto{\pgfqpoint{1.937487in}{4.052816in}}%
\pgfpathlineto{\pgfqpoint{1.919413in}{4.039835in}}%
\pgfpathlineto{\pgfqpoint{1.897583in}{4.035494in}}%
\pgfpathlineto{\pgfqpoint{1.879744in}{4.035047in}}%
\pgfpathlineto{\pgfqpoint{1.860965in}{4.039104in}}%
\pgfpathlineto{\pgfqpoint{1.842421in}{4.038623in}}%
\pgfpathlineto{\pgfqpoint{1.823878in}{4.034835in}}%
\pgfpathlineto{\pgfqpoint{1.803925in}{4.037045in}}%
\pgfpathlineto{\pgfqpoint{1.783738in}{4.047480in}}%
\pgfpathlineto{\pgfqpoint{1.763318in}{4.096621in}}%
\pgfpathlineto{\pgfqpoint{1.746182in}{4.194579in}}%
\pgfpathlineto{\pgfqpoint{1.727169in}{4.355136in}}%
\pgfpathlineto{\pgfqpoint{1.708861in}{4.417808in}}%
\pgfpathlineto{\pgfqpoint{1.688674in}{4.332830in}}%
\pgfpathlineto{\pgfqpoint{1.669191in}{4.154340in}}%
\pgfpathlineto{\pgfqpoint{1.650413in}{4.097783in}}%
\pgfpathlineto{\pgfqpoint{1.631399in}{4.053346in}}%
\pgfpathlineto{\pgfqpoint{1.609335in}{4.040191in}}%
\pgfpathlineto{\pgfqpoint{1.591964in}{4.035473in}}%
\pgfpathlineto{\pgfqpoint{1.572717in}{4.035848in}}%
\pgfpathlineto{\pgfqpoint{1.554878in}{4.041946in}}%
\pgfpathlineto{\pgfqpoint{1.532578in}{4.065586in}}%
\pgfpathlineto{\pgfqpoint{1.514505in}{4.136748in}}%
\pgfpathlineto{\pgfqpoint{1.496900in}{4.200711in}}%
\pgfpathlineto{\pgfqpoint{1.475070in}{4.385333in}}%
\pgfpathlineto{\pgfqpoint{1.458168in}{4.424285in}}%
\pgfpathlineto{\pgfqpoint{1.440565in}{4.385710in}}%
\pgfpathlineto{\pgfqpoint{1.418030in}{4.162448in}}%
\pgfpathlineto{\pgfqpoint{1.399956in}{4.087766in}}%
\pgfpathlineto{\pgfqpoint{1.381178in}{4.052725in}}%
\pgfpathlineto{\pgfqpoint{1.359582in}{4.038996in}}%
\pgfpathlineto{\pgfqpoint{1.341274in}{4.036675in}}%
\pgfpathlineto{\pgfqpoint{1.322495in}{4.035532in}}%
\pgfpathlineto{\pgfqpoint{1.304421in}{4.041585in}}%
\pgfpathlineto{\pgfqpoint{1.283062in}{4.035482in}}%
\pgfpathlineto{\pgfqpoint{1.265926in}{4.036756in}}%
\pgfpathlineto{\pgfqpoint{1.227665in}{4.053985in}}%
\pgfpathlineto{\pgfqpoint{1.205600in}{4.103853in}}%
\pgfpathlineto{\pgfqpoint{1.190344in}{4.190845in}}%
\pgfpathlineto{\pgfqpoint{1.171330in}{4.341079in}}%
\pgfpathlineto{\pgfqpoint{1.149969in}{4.428558in}}%
\pgfpathlineto{\pgfqpoint{1.131427in}{4.422472in}}%
\pgfpathlineto{\pgfqpoint{1.109362in}{4.212214in}}%
\pgfpathlineto{\pgfqpoint{1.087766in}{4.090401in}}%
\pgfpathlineto{\pgfqpoint{1.072744in}{4.060390in}}%
\pgfpathlineto{\pgfqpoint{1.053965in}{4.042674in}}%
\pgfpathlineto{\pgfqpoint{1.033309in}{4.036249in}}%
\pgfpathlineto{\pgfqpoint{1.016173in}{4.036685in}}%
\pgfpathlineto{\pgfqpoint{0.996222in}{4.043909in}}%
\pgfpathlineto{\pgfqpoint{0.977209in}{4.056598in}}%
\pgfpathlineto{\pgfqpoint{0.955847in}{4.095682in}}%
\pgfpathlineto{\pgfqpoint{0.938479in}{4.209405in}}%
\pgfpathlineto{\pgfqpoint{0.919700in}{4.350302in}}%
\pgfpathlineto{\pgfqpoint{0.898575in}{4.436426in}}%
\pgfpathlineto{\pgfqpoint{0.880970in}{4.441451in}}%
\pgfpathlineto{\pgfqpoint{0.863600in}{4.354348in}}%
\pgfpathlineto{\pgfqpoint{0.842475in}{4.197481in}}%
\pgfpathlineto{\pgfqpoint{0.823696in}{4.091580in}}%
\pgfpathlineto{\pgfqpoint{0.805621in}{4.061026in}}%
\pgfpathlineto{\pgfqpoint{0.784496in}{4.046250in}}%
\pgfpathlineto{\pgfqpoint{0.765014in}{4.037956in}}%
\pgfpathlineto{\pgfqpoint{0.747643in}{4.035948in}}%
\pgfpathlineto{\pgfqpoint{0.726987in}{4.039526in}}%
\pgfpathlineto{\pgfqpoint{0.709148in}{4.043876in}}%
\pgfpathlineto{\pgfqpoint{0.688023in}{4.036405in}}%
\pgfpathlineto{\pgfqpoint{0.667835in}{4.039682in}}%
\pgfpathlineto{\pgfqpoint{0.650231in}{4.049588in}}%
\pgfpathlineto{\pgfqpoint{0.650934in}{4.049135in}}%
\pgfpathlineto{\pgfqpoint{0.657742in}{4.042191in}}%
\pgfpathlineto{\pgfqpoint{0.677929in}{4.035799in}}%
\pgfpathlineto{\pgfqpoint{0.694594in}{4.042472in}}%
\pgfpathlineto{\pgfqpoint{0.715016in}{4.076362in}}%
\pgfpathlineto{\pgfqpoint{0.734969in}{4.194394in}}%
\pgfpathlineto{\pgfqpoint{0.753511in}{4.428735in}}%
\pgfpathlineto{\pgfqpoint{0.771351in}{4.435792in}}%
\pgfpathlineto{\pgfqpoint{0.810786in}{4.097022in}}%
\pgfpathlineto{\pgfqpoint{0.827685in}{4.053061in}}%
\pgfpathlineto{\pgfqpoint{0.849047in}{4.036668in}}%
\pgfpathlineto{\pgfqpoint{0.871814in}{4.039410in}}%
\pgfpathlineto{\pgfqpoint{0.888482in}{4.054328in}}%
\pgfpathlineto{\pgfqpoint{0.905147in}{4.098085in}}%
\pgfpathlineto{\pgfqpoint{0.927446in}{4.321150in}}%
\pgfpathlineto{\pgfqpoint{0.946225in}{4.439998in}}%
\pgfpathlineto{\pgfqpoint{0.964298in}{4.350136in}}%
\pgfpathlineto{\pgfqpoint{0.983077in}{4.143227in}}%
\pgfpathlineto{\pgfqpoint{1.000682in}{4.063349in}}%
\pgfpathlineto{\pgfqpoint{1.018521in}{4.037853in}}%
\pgfpathlineto{\pgfqpoint{1.040586in}{4.035506in}}%
\pgfpathlineto{\pgfqpoint{1.077907in}{4.036820in}}%
\pgfpathlineto{\pgfqpoint{1.100677in}{4.051447in}}%
\pgfpathlineto{\pgfqpoint{1.119690in}{4.099000in}}%
\pgfpathlineto{\pgfqpoint{1.137529in}{4.274558in}}%
\pgfpathlineto{\pgfqpoint{1.155134in}{4.428608in}}%
\pgfpathlineto{\pgfqpoint{1.173911in}{4.369449in}}%
\pgfpathlineto{\pgfqpoint{1.195507in}{4.149896in}}%
\pgfpathlineto{\pgfqpoint{1.214520in}{4.063439in}}%
\pgfpathlineto{\pgfqpoint{1.232594in}{4.040825in}}%
\pgfpathlineto{\pgfqpoint{1.257475in}{4.035279in}}%
\pgfpathlineto{\pgfqpoint{1.274611in}{4.039752in}}%
\pgfpathlineto{\pgfqpoint{1.294798in}{4.062773in}}%
\pgfpathlineto{\pgfqpoint{1.310758in}{4.124741in}}%
\pgfpathlineto{\pgfqpoint{1.330477in}{4.353662in}}%
\pgfpathlineto{\pgfqpoint{1.348551in}{4.422388in}}%
\pgfpathlineto{\pgfqpoint{1.367329in}{4.326060in}}%
\pgfpathlineto{\pgfqpoint{1.386343in}{4.147228in}}%
\pgfpathlineto{\pgfqpoint{1.404651in}{4.060158in}}%
\pgfpathlineto{\pgfqpoint{1.426950in}{4.038537in}}%
\pgfpathlineto{\pgfqpoint{1.445494in}{4.034887in}}%
\pgfpathlineto{\pgfqpoint{1.463802in}{4.038852in}}%
\pgfpathlineto{\pgfqpoint{1.482581in}{4.051219in}}%
\pgfpathlineto{\pgfqpoint{1.504411in}{4.115163in}}%
\pgfpathlineto{\pgfqpoint{1.520373in}{4.223846in}}%
\pgfpathlineto{\pgfqpoint{1.538681in}{4.414903in}}%
\pgfpathlineto{\pgfqpoint{1.557929in}{4.375785in}}%
\pgfpathlineto{\pgfqpoint{1.579290in}{4.200704in}}%
\pgfpathlineto{\pgfqpoint{1.598772in}{4.108856in}}%
\pgfpathlineto{\pgfqpoint{1.616846in}{4.052903in}}%
\pgfpathlineto{\pgfqpoint{1.635859in}{4.037774in}}%
\pgfpathlineto{\pgfqpoint{1.655576in}{4.034663in}}%
\pgfpathlineto{\pgfqpoint{1.674355in}{4.038976in}}%
\pgfpathlineto{\pgfqpoint{1.698533in}{4.058599in}}%
\pgfpathlineto{\pgfqpoint{1.713084in}{4.091410in}}%
\pgfpathlineto{\pgfqpoint{1.731394in}{4.223958in}}%
\pgfpathlineto{\pgfqpoint{1.751816in}{4.407873in}}%
\pgfpathlineto{\pgfqpoint{1.770124in}{4.403120in}}%
\pgfpathlineto{\pgfqpoint{1.791015in}{4.225918in}}%
\pgfpathlineto{\pgfqpoint{1.809325in}{4.103498in}}%
\pgfpathlineto{\pgfqpoint{1.830215in}{4.058669in}}%
\pgfpathlineto{\pgfqpoint{1.849229in}{4.040211in}}%
\pgfpathlineto{\pgfqpoint{1.866364in}{4.035142in}}%
\pgfpathlineto{\pgfqpoint{1.888663in}{4.036627in}}%
\pgfpathlineto{\pgfqpoint{1.905563in}{4.045021in}}%
\pgfpathlineto{\pgfqpoint{1.923168in}{4.073977in}}%
\pgfpathlineto{\pgfqpoint{1.944529in}{4.185212in}}%
\pgfpathlineto{\pgfqpoint{1.962603in}{4.329638in}}%
\pgfpathlineto{\pgfqpoint{1.986545in}{4.412755in}}%
\pgfpathlineto{\pgfqpoint{2.001098in}{4.356668in}}%
\pgfpathlineto{\pgfqpoint{2.021989in}{4.170799in}}%
\pgfpathlineto{\pgfqpoint{2.040299in}{4.069749in}}%
\pgfpathlineto{\pgfqpoint{2.059781in}{4.041929in}}%
\pgfpathlineto{\pgfqpoint{2.076915in}{4.035387in}}%
\pgfpathlineto{\pgfqpoint{2.098745in}{4.035190in}}%
\pgfpathlineto{\pgfqpoint{2.117993in}{4.040154in}}%
\pgfpathlineto{\pgfqpoint{2.135598in}{4.055700in}}%
\pgfpathlineto{\pgfqpoint{2.153673in}{4.110259in}}%
\pgfpathlineto{\pgfqpoint{2.174798in}{4.074157in}}%
\pgfpathlineto{\pgfqpoint{2.195454in}{4.044169in}}%
\pgfpathlineto{\pgfqpoint{2.213762in}{4.036018in}}%
\pgfpathlineto{\pgfqpoint{2.230898in}{4.035097in}}%
\pgfpathlineto{\pgfqpoint{2.253666in}{4.041219in}}%
\pgfpathlineto{\pgfqpoint{2.270333in}{4.060038in}}%
\pgfpathlineto{\pgfqpoint{2.287938in}{4.130591in}}%
\pgfpathlineto{\pgfqpoint{2.310003in}{4.375804in}}%
\pgfpathlineto{\pgfqpoint{2.327607in}{4.412180in}}%
\pgfpathlineto{\pgfqpoint{2.347558in}{4.266334in}}%
\pgfpathlineto{\pgfqpoint{2.365632in}{4.109331in}}%
\pgfpathlineto{\pgfqpoint{2.387462in}{4.048075in}}%
\pgfpathlineto{\pgfqpoint{2.405067in}{4.037010in}}%
\pgfpathlineto{\pgfqpoint{2.424315in}{4.034483in}}%
\pgfpathlineto{\pgfqpoint{2.443797in}{4.037322in}}%
\pgfpathlineto{\pgfqpoint{2.461402in}{4.042580in}}%
\pgfpathlineto{\pgfqpoint{2.483937in}{4.075601in}}%
\pgfpathlineto{\pgfqpoint{2.500602in}{4.175166in}}%
\pgfpathlineto{\pgfqpoint{2.518676in}{4.388548in}}%
\pgfpathlineto{\pgfqpoint{2.540740in}{4.400856in}}%
\pgfpathlineto{\pgfqpoint{2.558345in}{4.269341in}}%
\pgfpathlineto{\pgfqpoint{2.579236in}{4.101554in}}%
\pgfpathlineto{\pgfqpoint{2.597077in}{4.051284in}}%
\pgfpathlineto{\pgfqpoint{2.617968in}{4.039565in}}%
\pgfpathlineto{\pgfqpoint{2.636276in}{4.035365in}}%
\pgfpathlineto{\pgfqpoint{2.654115in}{4.034806in}}%
\pgfpathlineto{\pgfqpoint{2.671251in}{4.038434in}}%
\pgfpathlineto{\pgfqpoint{2.694019in}{4.052378in}}%
\pgfpathlineto{\pgfqpoint{2.713737in}{4.083942in}}%
\pgfpathlineto{\pgfqpoint{2.732514in}{4.128218in}}%
\pgfpathlineto{\pgfqpoint{2.750355in}{4.322900in}}%
\pgfpathlineto{\pgfqpoint{2.768663in}{4.411416in}}%
\pgfpathlineto{\pgfqpoint{2.789788in}{4.305548in}}%
\pgfpathlineto{\pgfqpoint{2.808098in}{4.130497in}}%
\pgfpathlineto{\pgfqpoint{2.828754in}{4.056210in}}%
\pgfpathlineto{\pgfqpoint{2.847768in}{4.042718in}}%
\pgfpathlineto{\pgfqpoint{2.864667in}{4.036022in}}%
\pgfpathlineto{\pgfqpoint{2.886029in}{4.035163in}}%
\pgfpathlineto{\pgfqpoint{2.904337in}{4.039192in}}%
\pgfpathlineto{\pgfqpoint{2.925227in}{4.056915in}}%
\pgfpathlineto{\pgfqpoint{2.942832in}{4.110918in}}%
\pgfpathlineto{\pgfqpoint{2.964193in}{4.289140in}}%
\pgfpathlineto{\pgfqpoint{2.982501in}{4.406581in}}%
\pgfpathlineto{\pgfqpoint{2.999872in}{4.388582in}}%
\pgfpathlineto{\pgfqpoint{3.020059in}{4.282554in}}%
\pgfpathlineto{\pgfqpoint{3.037193in}{4.129729in}}%
\pgfpathlineto{\pgfqpoint{3.056912in}{4.059623in}}%
\pgfpathlineto{\pgfqpoint{3.078036in}{4.040849in}}%
\pgfpathlineto{\pgfqpoint{3.098927in}{4.035245in}}%
\pgfpathlineto{\pgfqpoint{3.116532in}{4.035321in}}%
\pgfpathlineto{\pgfqpoint{3.135311in}{4.039267in}}%
\pgfpathlineto{\pgfqpoint{3.154324in}{4.050563in}}%
\pgfpathlineto{\pgfqpoint{3.175683in}{4.091192in}}%
\pgfpathlineto{\pgfqpoint{3.191880in}{4.196815in}}%
\pgfpathlineto{\pgfqpoint{3.213241in}{4.395866in}}%
\pgfpathlineto{\pgfqpoint{3.231549in}{4.412381in}}%
\pgfpathlineto{\pgfqpoint{3.248685in}{4.333063in}}%
\pgfpathlineto{\pgfqpoint{3.270044in}{4.189992in}}%
\pgfpathlineto{\pgfqpoint{3.288354in}{4.084001in}}%
\pgfpathlineto{\pgfqpoint{3.309948in}{4.065403in}}%
\pgfpathlineto{\pgfqpoint{3.328024in}{4.046613in}}%
\pgfpathlineto{\pgfqpoint{3.345392in}{4.037006in}}%
\pgfpathlineto{\pgfqpoint{3.366988in}{4.035215in}}%
\pgfpathlineto{\pgfqpoint{3.384593in}{4.038026in}}%
\pgfpathlineto{\pgfqpoint{3.405718in}{4.052465in}}%
\pgfpathlineto{\pgfqpoint{3.423088in}{4.077543in}}%
\pgfpathlineto{\pgfqpoint{3.441633in}{4.152650in}}%
\pgfpathlineto{\pgfqpoint{3.463226in}{4.378778in}}%
\pgfpathlineto{\pgfqpoint{3.479894in}{4.423521in}}%
\pgfpathlineto{\pgfqpoint{3.499141in}{4.384823in}}%
\pgfpathlineto{\pgfqpoint{3.516746in}{4.231565in}}%
\pgfpathlineto{\pgfqpoint{3.541391in}{4.110346in}}%
\pgfpathlineto{\pgfqpoint{3.558527in}{4.061765in}}%
\pgfpathlineto{\pgfqpoint{3.577072in}{4.043901in}}%
\pgfpathlineto{\pgfqpoint{3.594440in}{4.039430in}}%
\pgfpathlineto{\pgfqpoint{3.616975in}{4.036099in}}%
\pgfpathlineto{\pgfqpoint{3.634580in}{4.035332in}}%
\pgfpathlineto{\pgfqpoint{3.658053in}{4.043188in}}%
\pgfpathlineto{\pgfqpoint{3.671198in}{4.055555in}}%
\pgfpathlineto{\pgfqpoint{3.693732in}{4.108246in}}%
\pgfpathlineto{\pgfqpoint{3.712979in}{4.089869in}}%
\pgfpathlineto{\pgfqpoint{3.728941in}{4.191905in}}%
\pgfpathlineto{\pgfqpoint{3.751475in}{4.402094in}}%
\pgfpathlineto{\pgfqpoint{3.768845in}{4.432250in}}%
\pgfpathlineto{\pgfqpoint{3.787153in}{4.403605in}}%
\pgfpathlineto{\pgfqpoint{3.807341in}{4.300921in}}%
\pgfpathlineto{\pgfqpoint{3.825180in}{4.188535in}}%
\pgfpathlineto{\pgfqpoint{3.846776in}{4.081111in}}%
\pgfpathlineto{\pgfqpoint{3.864146in}{4.051426in}}%
\pgfpathlineto{\pgfqpoint{3.885036in}{4.038676in}}%
\pgfpathlineto{\pgfqpoint{3.903345in}{4.035307in}}%
\pgfpathlineto{\pgfqpoint{3.921654in}{4.109055in}}%
\pgfpathlineto{\pgfqpoint{3.939963in}{4.061706in}}%
\pgfpathlineto{\pgfqpoint{3.962027in}{4.040926in}}%
\pgfpathlineto{\pgfqpoint{3.982214in}{4.035479in}}%
\pgfpathlineto{\pgfqpoint{3.997237in}{4.037619in}}%
\pgfpathlineto{\pgfqpoint{4.017893in}{4.045526in}}%
\pgfpathlineto{\pgfqpoint{4.037844in}{4.075382in}}%
\pgfpathlineto{\pgfqpoint{4.037844in}{4.075382in}}%
\pgfusepath{stroke}%
\end{pgfscope}%
\begin{pgfscope}%
\pgfpathrectangle{\pgfqpoint{0.444748in}{4.012575in}}{\pgfqpoint{4.231419in}{0.467251in}}%
\pgfusepath{clip}%
\pgfsetbuttcap%
\pgfsetroundjoin%
\definecolor{currentfill}{rgb}{0.047059,0.364706,0.647059}%
\pgfsetfillcolor{currentfill}%
\pgfsetlinewidth{1.003750pt}%
\definecolor{currentstroke}{rgb}{0.047059,0.364706,0.647059}%
\pgfsetstrokecolor{currentstroke}%
\pgfsetdash{}{0pt}%
\pgfsys@defobject{currentmarker}{\pgfqpoint{-0.010417in}{-0.010417in}}{\pgfqpoint{0.010417in}{0.010417in}}{%
\pgfpathmoveto{\pgfqpoint{0.000000in}{-0.010417in}}%
\pgfpathcurveto{\pgfqpoint{0.002763in}{-0.010417in}}{\pgfqpoint{0.005412in}{-0.009319in}}{\pgfqpoint{0.007366in}{-0.007366in}}%
\pgfpathcurveto{\pgfqpoint{0.009319in}{-0.005412in}}{\pgfqpoint{0.010417in}{-0.002763in}}{\pgfqpoint{0.010417in}{0.000000in}}%
\pgfpathcurveto{\pgfqpoint{0.010417in}{0.002763in}}{\pgfqpoint{0.009319in}{0.005412in}}{\pgfqpoint{0.007366in}{0.007366in}}%
\pgfpathcurveto{\pgfqpoint{0.005412in}{0.009319in}}{\pgfqpoint{0.002763in}{0.010417in}}{\pgfqpoint{0.000000in}{0.010417in}}%
\pgfpathcurveto{\pgfqpoint{-0.002763in}{0.010417in}}{\pgfqpoint{-0.005412in}{0.009319in}}{\pgfqpoint{-0.007366in}{0.007366in}}%
\pgfpathcurveto{\pgfqpoint{-0.009319in}{0.005412in}}{\pgfqpoint{-0.010417in}{0.002763in}}{\pgfqpoint{-0.010417in}{0.000000in}}%
\pgfpathcurveto{\pgfqpoint{-0.010417in}{-0.002763in}}{\pgfqpoint{-0.009319in}{-0.005412in}}{\pgfqpoint{-0.007366in}{-0.007366in}}%
\pgfpathcurveto{\pgfqpoint{-0.005412in}{-0.009319in}}{\pgfqpoint{-0.002763in}{-0.010417in}}{\pgfqpoint{0.000000in}{-0.010417in}}%
\pgfpathlineto{\pgfqpoint{0.000000in}{-0.010417in}}%
\pgfpathclose%
\pgfusepath{stroke,fill}%
}%
\begin{pgfscope}%
\pgfsys@transformshift{0.637086in}{4.047001in}%
\pgfsys@useobject{currentmarker}{}%
\end{pgfscope}%
\begin{pgfscope}%
\pgfsys@transformshift{0.655396in}{4.037814in}%
\pgfsys@useobject{currentmarker}{}%
\end{pgfscope}%
\begin{pgfscope}%
\pgfsys@transformshift{0.676052in}{4.046150in}%
\pgfsys@useobject{currentmarker}{}%
\end{pgfscope}%
\begin{pgfscope}%
\pgfsys@transformshift{0.697177in}{4.085894in}%
\pgfsys@useobject{currentmarker}{}%
\end{pgfscope}%
\begin{pgfscope}%
\pgfsys@transformshift{0.718067in}{4.227070in}%
\pgfsys@useobject{currentmarker}{}%
\end{pgfscope}%
\begin{pgfscope}%
\pgfsys@transformshift{0.734029in}{4.423639in}%
\pgfsys@useobject{currentmarker}{}%
\end{pgfscope}%
\begin{pgfscope}%
\pgfsys@transformshift{0.754685in}{4.427741in}%
\pgfsys@useobject{currentmarker}{}%
\end{pgfscope}%
\begin{pgfscope}%
\pgfsys@transformshift{0.772994in}{4.276456in}%
\pgfsys@useobject{currentmarker}{}%
\end{pgfscope}%
\begin{pgfscope}%
\pgfsys@transformshift{0.791303in}{4.103346in}%
\pgfsys@useobject{currentmarker}{}%
\end{pgfscope}%
\begin{pgfscope}%
\pgfsys@transformshift{0.810317in}{4.055942in}%
\pgfsys@useobject{currentmarker}{}%
\end{pgfscope}%
\begin{pgfscope}%
\pgfsys@transformshift{0.829564in}{4.038485in}%
\pgfsys@useobject{currentmarker}{}%
\end{pgfscope}%
\begin{pgfscope}%
\pgfsys@transformshift{0.851629in}{4.042172in}%
\pgfsys@useobject{currentmarker}{}%
\end{pgfscope}%
\begin{pgfscope}%
\pgfsys@transformshift{0.868060in}{4.060973in}%
\pgfsys@useobject{currentmarker}{}%
\end{pgfscope}%
\begin{pgfscope}%
\pgfsys@transformshift{0.884725in}{4.115434in}%
\pgfsys@useobject{currentmarker}{}%
\end{pgfscope}%
\begin{pgfscope}%
\pgfsys@transformshift{0.905852in}{4.363447in}%
\pgfsys@useobject{currentmarker}{}%
\end{pgfscope}%
\begin{pgfscope}%
\pgfsys@transformshift{0.922283in}{4.440983in}%
\pgfsys@useobject{currentmarker}{}%
\end{pgfscope}%
\begin{pgfscope}%
\pgfsys@transformshift{0.945756in}{4.309659in}%
\pgfsys@useobject{currentmarker}{}%
\end{pgfscope}%
\begin{pgfscope}%
\pgfsys@transformshift{0.963829in}{4.120875in}%
\pgfsys@useobject{currentmarker}{}%
\end{pgfscope}%
\begin{pgfscope}%
\pgfsys@transformshift{0.981200in}{4.062183in}%
\pgfsys@useobject{currentmarker}{}%
\end{pgfscope}%
\begin{pgfscope}%
\pgfsys@transformshift{1.002559in}{4.039331in}%
\pgfsys@useobject{currentmarker}{}%
\end{pgfscope}%
\begin{pgfscope}%
\pgfsys@transformshift{1.021104in}{4.037443in}%
\pgfsys@useobject{currentmarker}{}%
\end{pgfscope}%
\begin{pgfscope}%
\pgfsys@transformshift{1.041994in}{4.051523in}%
\pgfsys@useobject{currentmarker}{}%
\end{pgfscope}%
\begin{pgfscope}%
\pgfsys@transformshift{1.058894in}{4.093891in}%
\pgfsys@useobject{currentmarker}{}%
\end{pgfscope}%
\begin{pgfscope}%
\pgfsys@transformshift{1.081664in}{4.274837in}%
\pgfsys@useobject{currentmarker}{}%
\end{pgfscope}%
\begin{pgfscope}%
\pgfsys@transformshift{1.097860in}{4.428299in}%
\pgfsys@useobject{currentmarker}{}%
\end{pgfscope}%
\begin{pgfscope}%
\pgfsys@transformshift{1.116873in}{4.392891in}%
\pgfsys@useobject{currentmarker}{}%
\end{pgfscope}%
\begin{pgfscope}%
\pgfsys@transformshift{1.136590in}{4.217478in}%
\pgfsys@useobject{currentmarker}{}%
\end{pgfscope}%
\begin{pgfscope}%
\pgfsys@transformshift{1.158420in}{4.072806in}%
\pgfsys@useobject{currentmarker}{}%
\end{pgfscope}%
\begin{pgfscope}%
\pgfsys@transformshift{1.174147in}{4.045768in}%
\pgfsys@useobject{currentmarker}{}%
\end{pgfscope}%
\begin{pgfscope}%
\pgfsys@transformshift{1.193864in}{4.036686in}%
\pgfsys@useobject{currentmarker}{}%
\end{pgfscope}%
\begin{pgfscope}%
\pgfsys@transformshift{1.213112in}{4.039233in}%
\pgfsys@useobject{currentmarker}{}%
\end{pgfscope}%
\begin{pgfscope}%
\pgfsys@transformshift{1.235411in}{4.062322in}%
\pgfsys@useobject{currentmarker}{}%
\end{pgfscope}%
\begin{pgfscope}%
\pgfsys@transformshift{1.251138in}{4.096982in}%
\pgfsys@useobject{currentmarker}{}%
\end{pgfscope}%
\begin{pgfscope}%
\pgfsys@transformshift{1.270386in}{4.244930in}%
\pgfsys@useobject{currentmarker}{}%
\end{pgfscope}%
\begin{pgfscope}%
\pgfsys@transformshift{1.292216in}{4.424644in}%
\pgfsys@useobject{currentmarker}{}%
\end{pgfscope}%
\begin{pgfscope}%
\pgfsys@transformshift{1.309821in}{4.395746in}%
\pgfsys@useobject{currentmarker}{}%
\end{pgfscope}%
\begin{pgfscope}%
\pgfsys@transformshift{1.329303in}{4.238839in}%
\pgfsys@useobject{currentmarker}{}%
\end{pgfscope}%
\begin{pgfscope}%
\pgfsys@transformshift{1.348316in}{4.130485in}%
\pgfsys@useobject{currentmarker}{}%
\end{pgfscope}%
\begin{pgfscope}%
\pgfsys@transformshift{1.368267in}{4.056144in}%
\pgfsys@useobject{currentmarker}{}%
\end{pgfscope}%
\begin{pgfscope}%
\pgfsys@transformshift{1.385403in}{4.269327in}%
\pgfsys@useobject{currentmarker}{}%
\end{pgfscope}%
\begin{pgfscope}%
\pgfsys@transformshift{1.406528in}{4.412665in}%
\pgfsys@useobject{currentmarker}{}%
\end{pgfscope}%
\begin{pgfscope}%
\pgfsys@transformshift{1.425776in}{4.274164in}%
\pgfsys@useobject{currentmarker}{}%
\end{pgfscope}%
\begin{pgfscope}%
\pgfsys@transformshift{1.444554in}{4.172358in}%
\pgfsys@useobject{currentmarker}{}%
\end{pgfscope}%
\begin{pgfscope}%
\pgfsys@transformshift{1.463333in}{4.069383in}%
\pgfsys@useobject{currentmarker}{}%
\end{pgfscope}%
\begin{pgfscope}%
\pgfsys@transformshift{1.483521in}{4.044237in}%
\pgfsys@useobject{currentmarker}{}%
\end{pgfscope}%
\begin{pgfscope}%
\pgfsys@transformshift{1.502298in}{4.036198in}%
\pgfsys@useobject{currentmarker}{}%
\end{pgfscope}%
\begin{pgfscope}%
\pgfsys@transformshift{1.521545in}{4.038209in}%
\pgfsys@useobject{currentmarker}{}%
\end{pgfscope}%
\begin{pgfscope}%
\pgfsys@transformshift{1.539621in}{4.050614in}%
\pgfsys@useobject{currentmarker}{}%
\end{pgfscope}%
\begin{pgfscope}%
\pgfsys@transformshift{1.559103in}{4.098175in}%
\pgfsys@useobject{currentmarker}{}%
\end{pgfscope}%
\begin{pgfscope}%
\pgfsys@transformshift{1.580228in}{4.304649in}%
\pgfsys@useobject{currentmarker}{}%
\end{pgfscope}%
\begin{pgfscope}%
\pgfsys@transformshift{1.597598in}{4.418135in}%
\pgfsys@useobject{currentmarker}{}%
\end{pgfscope}%
\begin{pgfscope}%
\pgfsys@transformshift{1.616377in}{4.362066in}%
\pgfsys@useobject{currentmarker}{}%
\end{pgfscope}%
\begin{pgfscope}%
\pgfsys@transformshift{1.634920in}{4.232475in}%
\pgfsys@useobject{currentmarker}{}%
\end{pgfscope}%
\begin{pgfscope}%
\pgfsys@transformshift{1.654638in}{4.086567in}%
\pgfsys@useobject{currentmarker}{}%
\end{pgfscope}%
\begin{pgfscope}%
\pgfsys@transformshift{1.676232in}{4.044236in}%
\pgfsys@useobject{currentmarker}{}%
\end{pgfscope}%
\begin{pgfscope}%
\pgfsys@transformshift{1.697124in}{4.036287in}%
\pgfsys@useobject{currentmarker}{}%
\end{pgfscope}%
\begin{pgfscope}%
\pgfsys@transformshift{1.712850in}{4.036728in}%
\pgfsys@useobject{currentmarker}{}%
\end{pgfscope}%
\begin{pgfscope}%
\pgfsys@transformshift{1.735149in}{4.044706in}%
\pgfsys@useobject{currentmarker}{}%
\end{pgfscope}%
\begin{pgfscope}%
\pgfsys@transformshift{1.750408in}{4.061442in}%
\pgfsys@useobject{currentmarker}{}%
\end{pgfscope}%
\begin{pgfscope}%
\pgfsys@transformshift{1.772707in}{4.165543in}%
\pgfsys@useobject{currentmarker}{}%
\end{pgfscope}%
\begin{pgfscope}%
\pgfsys@transformshift{1.790546in}{4.385294in}%
\pgfsys@useobject{currentmarker}{}%
\end{pgfscope}%
\begin{pgfscope}%
\pgfsys@transformshift{1.809559in}{4.411837in}%
\pgfsys@useobject{currentmarker}{}%
\end{pgfscope}%
\begin{pgfscope}%
\pgfsys@transformshift{1.828572in}{4.281163in}%
\pgfsys@useobject{currentmarker}{}%
\end{pgfscope}%
\begin{pgfscope}%
\pgfsys@transformshift{1.848523in}{4.116923in}%
\pgfsys@useobject{currentmarker}{}%
\end{pgfscope}%
\begin{pgfscope}%
\pgfsys@transformshift{1.866599in}{4.056482in}%
\pgfsys@useobject{currentmarker}{}%
\end{pgfscope}%
\begin{pgfscope}%
\pgfsys@transformshift{1.887958in}{4.237204in}%
\pgfsys@useobject{currentmarker}{}%
\end{pgfscope}%
\begin{pgfscope}%
\pgfsys@transformshift{1.903920in}{4.101526in}%
\pgfsys@useobject{currentmarker}{}%
\end{pgfscope}%
\begin{pgfscope}%
\pgfsys@transformshift{1.925750in}{4.047764in}%
\pgfsys@useobject{currentmarker}{}%
\end{pgfscope}%
\begin{pgfscope}%
\pgfsys@transformshift{1.941947in}{4.119876in}%
\pgfsys@useobject{currentmarker}{}%
\end{pgfscope}%
\begin{pgfscope}%
\pgfsys@transformshift{1.963306in}{4.052055in}%
\pgfsys@useobject{currentmarker}{}%
\end{pgfscope}%
\begin{pgfscope}%
\pgfsys@transformshift{1.982319in}{4.038675in}%
\pgfsys@useobject{currentmarker}{}%
\end{pgfscope}%
\begin{pgfscope}%
\pgfsys@transformshift{2.000629in}{4.035770in}%
\pgfsys@useobject{currentmarker}{}%
\end{pgfscope}%
\begin{pgfscope}%
\pgfsys@transformshift{2.020346in}{4.041267in}%
\pgfsys@useobject{currentmarker}{}%
\end{pgfscope}%
\begin{pgfscope}%
\pgfsys@transformshift{2.041002in}{4.059157in}%
\pgfsys@useobject{currentmarker}{}%
\end{pgfscope}%
\begin{pgfscope}%
\pgfsys@transformshift{2.059546in}{4.125416in}%
\pgfsys@useobject{currentmarker}{}%
\end{pgfscope}%
\begin{pgfscope}%
\pgfsys@transformshift{2.078560in}{4.279329in}%
\pgfsys@useobject{currentmarker}{}%
\end{pgfscope}%
\begin{pgfscope}%
\pgfsys@transformshift{2.097102in}{4.415770in}%
\pgfsys@useobject{currentmarker}{}%
\end{pgfscope}%
\begin{pgfscope}%
\pgfsys@transformshift{2.116350in}{4.369652in}%
\pgfsys@useobject{currentmarker}{}%
\end{pgfscope}%
\begin{pgfscope}%
\pgfsys@transformshift{2.135129in}{4.197432in}%
\pgfsys@useobject{currentmarker}{}%
\end{pgfscope}%
\begin{pgfscope}%
\pgfsys@transformshift{2.157193in}{4.083194in}%
\pgfsys@useobject{currentmarker}{}%
\end{pgfscope}%
\begin{pgfscope}%
\pgfsys@transformshift{2.171747in}{4.047704in}%
\pgfsys@useobject{currentmarker}{}%
\end{pgfscope}%
\begin{pgfscope}%
\pgfsys@transformshift{2.192637in}{4.037441in}%
\pgfsys@useobject{currentmarker}{}%
\end{pgfscope}%
\begin{pgfscope}%
\pgfsys@transformshift{2.214468in}{4.036609in}%
\pgfsys@useobject{currentmarker}{}%
\end{pgfscope}%
\begin{pgfscope}%
\pgfsys@transformshift{2.233010in}{4.042798in}%
\pgfsys@useobject{currentmarker}{}%
\end{pgfscope}%
\begin{pgfscope}%
\pgfsys@transformshift{2.250380in}{4.065205in}%
\pgfsys@useobject{currentmarker}{}%
\end{pgfscope}%
\begin{pgfscope}%
\pgfsys@transformshift{2.271976in}{4.159862in}%
\pgfsys@useobject{currentmarker}{}%
\end{pgfscope}%
\begin{pgfscope}%
\pgfsys@transformshift{2.289581in}{4.366870in}%
\pgfsys@useobject{currentmarker}{}%
\end{pgfscope}%
\begin{pgfscope}%
\pgfsys@transformshift{2.307889in}{4.418491in}%
\pgfsys@useobject{currentmarker}{}%
\end{pgfscope}%
\begin{pgfscope}%
\pgfsys@transformshift{2.328311in}{4.308736in}%
\pgfsys@useobject{currentmarker}{}%
\end{pgfscope}%
\begin{pgfscope}%
\pgfsys@transformshift{2.349436in}{4.148565in}%
\pgfsys@useobject{currentmarker}{}%
\end{pgfscope}%
\begin{pgfscope}%
\pgfsys@transformshift{2.365868in}{4.070987in}%
\pgfsys@useobject{currentmarker}{}%
\end{pgfscope}%
\begin{pgfscope}%
\pgfsys@transformshift{2.385819in}{4.043731in}%
\pgfsys@useobject{currentmarker}{}%
\end{pgfscope}%
\begin{pgfscope}%
\pgfsys@transformshift{2.402721in}{4.036478in}%
\pgfsys@useobject{currentmarker}{}%
\end{pgfscope}%
\begin{pgfscope}%
\pgfsys@transformshift{2.424080in}{4.036751in}%
\pgfsys@useobject{currentmarker}{}%
\end{pgfscope}%
\begin{pgfscope}%
\pgfsys@transformshift{2.445442in}{4.046354in}%
\pgfsys@useobject{currentmarker}{}%
\end{pgfscope}%
\begin{pgfscope}%
\pgfsys@transformshift{2.462576in}{4.071884in}%
\pgfsys@useobject{currentmarker}{}%
\end{pgfscope}%
\begin{pgfscope}%
\pgfsys@transformshift{2.480886in}{4.148324in}%
\pgfsys@useobject{currentmarker}{}%
\end{pgfscope}%
\begin{pgfscope}%
\pgfsys@transformshift{2.501542in}{4.368394in}%
\pgfsys@useobject{currentmarker}{}%
\end{pgfscope}%
\begin{pgfscope}%
\pgfsys@transformshift{2.522198in}{4.413163in}%
\pgfsys@useobject{currentmarker}{}%
\end{pgfscope}%
\begin{pgfscope}%
\pgfsys@transformshift{2.543557in}{4.293095in}%
\pgfsys@useobject{currentmarker}{}%
\end{pgfscope}%
\begin{pgfscope}%
\pgfsys@transformshift{2.558580in}{4.228119in}%
\pgfsys@useobject{currentmarker}{}%
\end{pgfscope}%
\begin{pgfscope}%
\pgfsys@transformshift{2.579472in}{4.084277in}%
\pgfsys@useobject{currentmarker}{}%
\end{pgfscope}%
\begin{pgfscope}%
\pgfsys@transformshift{2.598015in}{4.049070in}%
\pgfsys@useobject{currentmarker}{}%
\end{pgfscope}%
\begin{pgfscope}%
\pgfsys@transformshift{2.614916in}{4.038596in}%
\pgfsys@useobject{currentmarker}{}%
\end{pgfscope}%
\begin{pgfscope}%
\pgfsys@transformshift{2.636276in}{4.035641in}%
\pgfsys@useobject{currentmarker}{}%
\end{pgfscope}%
\begin{pgfscope}%
\pgfsys@transformshift{2.653646in}{4.039124in}%
\pgfsys@useobject{currentmarker}{}%
\end{pgfscope}%
\begin{pgfscope}%
\pgfsys@transformshift{2.675005in}{4.055232in}%
\pgfsys@useobject{currentmarker}{}%
\end{pgfscope}%
\begin{pgfscope}%
\pgfsys@transformshift{2.693784in}{4.084092in}%
\pgfsys@useobject{currentmarker}{}%
\end{pgfscope}%
\begin{pgfscope}%
\pgfsys@transformshift{2.713032in}{4.225559in}%
\pgfsys@useobject{currentmarker}{}%
\end{pgfscope}%
\begin{pgfscope}%
\pgfsys@transformshift{2.732280in}{4.403116in}%
\pgfsys@useobject{currentmarker}{}%
\end{pgfscope}%
\begin{pgfscope}%
\pgfsys@transformshift{2.749416in}{4.412505in}%
\pgfsys@useobject{currentmarker}{}%
\end{pgfscope}%
\begin{pgfscope}%
\pgfsys@transformshift{2.771246in}{4.329880in}%
\pgfsys@useobject{currentmarker}{}%
\end{pgfscope}%
\begin{pgfscope}%
\pgfsys@transformshift{2.788850in}{4.180553in}%
\pgfsys@useobject{currentmarker}{}%
\end{pgfscope}%
\begin{pgfscope}%
\pgfsys@transformshift{2.807393in}{4.113524in}%
\pgfsys@useobject{currentmarker}{}%
\end{pgfscope}%
\begin{pgfscope}%
\pgfsys@transformshift{2.828049in}{4.053084in}%
\pgfsys@useobject{currentmarker}{}%
\end{pgfscope}%
\begin{pgfscope}%
\pgfsys@transformshift{2.846594in}{4.041123in}%
\pgfsys@useobject{currentmarker}{}%
\end{pgfscope}%
\begin{pgfscope}%
\pgfsys@transformshift{2.866544in}{4.035735in}%
\pgfsys@useobject{currentmarker}{}%
\end{pgfscope}%
\begin{pgfscope}%
\pgfsys@transformshift{2.884854in}{4.038494in}%
\pgfsys@useobject{currentmarker}{}%
\end{pgfscope}%
\begin{pgfscope}%
\pgfsys@transformshift{2.902459in}{4.045188in}%
\pgfsys@useobject{currentmarker}{}%
\end{pgfscope}%
\begin{pgfscope}%
\pgfsys@transformshift{2.924289in}{4.075015in}%
\pgfsys@useobject{currentmarker}{}%
\end{pgfscope}%
\begin{pgfscope}%
\pgfsys@transformshift{2.941658in}{4.151431in}%
\pgfsys@useobject{currentmarker}{}%
\end{pgfscope}%
\begin{pgfscope}%
\pgfsys@transformshift{2.962785in}{4.328320in}%
\pgfsys@useobject{currentmarker}{}%
\end{pgfscope}%
\begin{pgfscope}%
\pgfsys@transformshift{2.981327in}{4.412210in}%
\pgfsys@useobject{currentmarker}{}%
\end{pgfscope}%
\begin{pgfscope}%
\pgfsys@transformshift{2.999872in}{4.364225in}%
\pgfsys@useobject{currentmarker}{}%
\end{pgfscope}%
\begin{pgfscope}%
\pgfsys@transformshift{3.020997in}{4.201348in}%
\pgfsys@useobject{currentmarker}{}%
\end{pgfscope}%
\begin{pgfscope}%
\pgfsys@transformshift{3.038133in}{4.094185in}%
\pgfsys@useobject{currentmarker}{}%
\end{pgfscope}%
\begin{pgfscope}%
\pgfsys@transformshift{3.056912in}{4.055908in}%
\pgfsys@useobject{currentmarker}{}%
\end{pgfscope}%
\begin{pgfscope}%
\pgfsys@transformshift{3.078036in}{4.040859in}%
\pgfsys@useobject{currentmarker}{}%
\end{pgfscope}%
\begin{pgfscope}%
\pgfsys@transformshift{3.095407in}{4.035862in}%
\pgfsys@useobject{currentmarker}{}%
\end{pgfscope}%
\begin{pgfscope}%
\pgfsys@transformshift{3.116766in}{4.036503in}%
\pgfsys@useobject{currentmarker}{}%
\end{pgfscope}%
\begin{pgfscope}%
\pgfsys@transformshift{3.134606in}{4.043355in}%
\pgfsys@useobject{currentmarker}{}%
\end{pgfscope}%
\begin{pgfscope}%
\pgfsys@transformshift{3.155498in}{4.071233in}%
\pgfsys@useobject{currentmarker}{}%
\end{pgfscope}%
\begin{pgfscope}%
\pgfsys@transformshift{3.173103in}{4.129380in}%
\pgfsys@useobject{currentmarker}{}%
\end{pgfscope}%
\begin{pgfscope}%
\pgfsys@transformshift{3.191645in}{4.269201in}%
\pgfsys@useobject{currentmarker}{}%
\end{pgfscope}%
\begin{pgfscope}%
\pgfsys@transformshift{3.213007in}{4.414789in}%
\pgfsys@useobject{currentmarker}{}%
\end{pgfscope}%
\begin{pgfscope}%
\pgfsys@transformshift{3.232254in}{4.396585in}%
\pgfsys@useobject{currentmarker}{}%
\end{pgfscope}%
\begin{pgfscope}%
\pgfsys@transformshift{3.249154in}{4.320499in}%
\pgfsys@useobject{currentmarker}{}%
\end{pgfscope}%
\begin{pgfscope}%
\pgfsys@transformshift{3.268636in}{4.177506in}%
\pgfsys@useobject{currentmarker}{}%
\end{pgfscope}%
\begin{pgfscope}%
\pgfsys@transformshift{3.289763in}{4.209958in}%
\pgfsys@useobject{currentmarker}{}%
\end{pgfscope}%
\begin{pgfscope}%
\pgfsys@transformshift{3.307368in}{4.098674in}%
\pgfsys@useobject{currentmarker}{}%
\end{pgfscope}%
\begin{pgfscope}%
\pgfsys@transformshift{3.324502in}{4.058548in}%
\pgfsys@useobject{currentmarker}{}%
\end{pgfscope}%
\begin{pgfscope}%
\pgfsys@transformshift{3.347272in}{4.041056in}%
\pgfsys@useobject{currentmarker}{}%
\end{pgfscope}%
\begin{pgfscope}%
\pgfsys@transformshift{3.364642in}{4.036392in}%
\pgfsys@useobject{currentmarker}{}%
\end{pgfscope}%
\begin{pgfscope}%
\pgfsys@transformshift{3.385767in}{4.036851in}%
\pgfsys@useobject{currentmarker}{}%
\end{pgfscope}%
\begin{pgfscope}%
\pgfsys@transformshift{3.403606in}{4.044259in}%
\pgfsys@useobject{currentmarker}{}%
\end{pgfscope}%
\begin{pgfscope}%
\pgfsys@transformshift{3.424497in}{4.069552in}%
\pgfsys@useobject{currentmarker}{}%
\end{pgfscope}%
\begin{pgfscope}%
\pgfsys@transformshift{3.442101in}{4.105886in}%
\pgfsys@useobject{currentmarker}{}%
\end{pgfscope}%
\begin{pgfscope}%
\pgfsys@transformshift{3.459941in}{4.209607in}%
\pgfsys@useobject{currentmarker}{}%
\end{pgfscope}%
\begin{pgfscope}%
\pgfsys@transformshift{3.481068in}{4.407970in}%
\pgfsys@useobject{currentmarker}{}%
\end{pgfscope}%
\begin{pgfscope}%
\pgfsys@transformshift{3.501958in}{4.417691in}%
\pgfsys@useobject{currentmarker}{}%
\end{pgfscope}%
\begin{pgfscope}%
\pgfsys@transformshift{3.519329in}{4.334392in}%
\pgfsys@useobject{currentmarker}{}%
\end{pgfscope}%
\begin{pgfscope}%
\pgfsys@transformshift{3.540688in}{4.193995in}%
\pgfsys@useobject{currentmarker}{}%
\end{pgfscope}%
\begin{pgfscope}%
\pgfsys@transformshift{3.559467in}{4.099497in}%
\pgfsys@useobject{currentmarker}{}%
\end{pgfscope}%
\begin{pgfscope}%
\pgfsys@transformshift{3.576837in}{4.063253in}%
\pgfsys@useobject{currentmarker}{}%
\end{pgfscope}%
\begin{pgfscope}%
\pgfsys@transformshift{3.596788in}{4.043423in}%
\pgfsys@useobject{currentmarker}{}%
\end{pgfscope}%
\begin{pgfscope}%
\pgfsys@transformshift{3.616270in}{4.038350in}%
\pgfsys@useobject{currentmarker}{}%
\end{pgfscope}%
\begin{pgfscope}%
\pgfsys@transformshift{3.636926in}{4.036630in}%
\pgfsys@useobject{currentmarker}{}%
\end{pgfscope}%
\begin{pgfscope}%
\pgfsys@transformshift{3.654062in}{4.040693in}%
\pgfsys@useobject{currentmarker}{}%
\end{pgfscope}%
\begin{pgfscope}%
\pgfsys@transformshift{3.674953in}{4.054982in}%
\pgfsys@useobject{currentmarker}{}%
\end{pgfscope}%
\begin{pgfscope}%
\pgfsys@transformshift{3.691854in}{4.080979in}%
\pgfsys@useobject{currentmarker}{}%
\end{pgfscope}%
\begin{pgfscope}%
\pgfsys@transformshift{3.715796in}{4.186227in}%
\pgfsys@useobject{currentmarker}{}%
\end{pgfscope}%
\begin{pgfscope}%
\pgfsys@transformshift{3.731053in}{4.303862in}%
\pgfsys@useobject{currentmarker}{}%
\end{pgfscope}%
\begin{pgfscope}%
\pgfsys@transformshift{3.750535in}{4.423607in}%
\pgfsys@useobject{currentmarker}{}%
\end{pgfscope}%
\begin{pgfscope}%
\pgfsys@transformshift{3.768845in}{4.422927in}%
\pgfsys@useobject{currentmarker}{}%
\end{pgfscope}%
\begin{pgfscope}%
\pgfsys@transformshift{3.793256in}{4.300610in}%
\pgfsys@useobject{currentmarker}{}%
\end{pgfscope}%
\begin{pgfscope}%
\pgfsys@transformshift{3.808280in}{4.207037in}%
\pgfsys@useobject{currentmarker}{}%
\end{pgfscope}%
\begin{pgfscope}%
\pgfsys@transformshift{3.826588in}{4.104869in}%
\pgfsys@useobject{currentmarker}{}%
\end{pgfscope}%
\begin{pgfscope}%
\pgfsys@transformshift{3.847950in}{4.060277in}%
\pgfsys@useobject{currentmarker}{}%
\end{pgfscope}%
\begin{pgfscope}%
\pgfsys@transformshift{3.864849in}{4.047718in}%
\pgfsys@useobject{currentmarker}{}%
\end{pgfscope}%
\begin{pgfscope}%
\pgfsys@transformshift{3.883394in}{4.040101in}%
\pgfsys@useobject{currentmarker}{}%
\end{pgfscope}%
\begin{pgfscope}%
\pgfsys@transformshift{3.905693in}{4.036934in}%
\pgfsys@useobject{currentmarker}{}%
\end{pgfscope}%
\begin{pgfscope}%
\pgfsys@transformshift{3.922827in}{4.037598in}%
\pgfsys@useobject{currentmarker}{}%
\end{pgfscope}%
\begin{pgfscope}%
\pgfsys@transformshift{3.943483in}{4.043324in}%
\pgfsys@useobject{currentmarker}{}%
\end{pgfscope}%
\begin{pgfscope}%
\pgfsys@transformshift{3.962262in}{4.060249in}%
\pgfsys@useobject{currentmarker}{}%
\end{pgfscope}%
\begin{pgfscope}%
\pgfsys@transformshift{3.980806in}{4.093052in}%
\pgfsys@useobject{currentmarker}{}%
\end{pgfscope}%
\begin{pgfscope}%
\pgfsys@transformshift{4.000991in}{4.170057in}%
\pgfsys@useobject{currentmarker}{}%
\end{pgfscope}%
\begin{pgfscope}%
\pgfsys@transformshift{4.019301in}{4.320166in}%
\pgfsys@useobject{currentmarker}{}%
\end{pgfscope}%
\begin{pgfscope}%
\pgfsys@transformshift{4.037844in}{4.434634in}%
\pgfsys@useobject{currentmarker}{}%
\end{pgfscope}%
\begin{pgfscope}%
\pgfsys@transformshift{4.057562in}{4.438979in}%
\pgfsys@useobject{currentmarker}{}%
\end{pgfscope}%
\begin{pgfscope}%
\pgfsys@transformshift{4.076341in}{4.375410in}%
\pgfsys@useobject{currentmarker}{}%
\end{pgfscope}%
\begin{pgfscope}%
\pgfsys@transformshift{4.097466in}{4.246353in}%
\pgfsys@useobject{currentmarker}{}%
\end{pgfscope}%
\begin{pgfscope}%
\pgfsys@transformshift{4.115540in}{4.139097in}%
\pgfsys@useobject{currentmarker}{}%
\end{pgfscope}%
\begin{pgfscope}%
\pgfsys@transformshift{4.132676in}{4.084415in}%
\pgfsys@useobject{currentmarker}{}%
\end{pgfscope}%
\begin{pgfscope}%
\pgfsys@transformshift{4.154740in}{4.052293in}%
\pgfsys@useobject{currentmarker}{}%
\end{pgfscope}%
\begin{pgfscope}%
\pgfsys@transformshift{4.172814in}{4.046232in}%
\pgfsys@useobject{currentmarker}{}%
\end{pgfscope}%
\begin{pgfscope}%
\pgfsys@transformshift{4.190184in}{4.040165in}%
\pgfsys@useobject{currentmarker}{}%
\end{pgfscope}%
\begin{pgfscope}%
\pgfsys@transformshift{4.212249in}{4.038000in}%
\pgfsys@useobject{currentmarker}{}%
\end{pgfscope}%
\begin{pgfscope}%
\pgfsys@transformshift{4.229619in}{4.044182in}%
\pgfsys@useobject{currentmarker}{}%
\end{pgfscope}%
\begin{pgfscope}%
\pgfsys@transformshift{4.248162in}{4.056370in}%
\pgfsys@useobject{currentmarker}{}%
\end{pgfscope}%
\begin{pgfscope}%
\pgfsys@transformshift{4.268818in}{4.095893in}%
\pgfsys@useobject{currentmarker}{}%
\end{pgfscope}%
\begin{pgfscope}%
\pgfsys@transformshift{4.287128in}{4.182930in}%
\pgfsys@useobject{currentmarker}{}%
\end{pgfscope}%
\begin{pgfscope}%
\pgfsys@transformshift{4.308253in}{4.356689in}%
\pgfsys@useobject{currentmarker}{}%
\end{pgfscope}%
\begin{pgfscope}%
\pgfsys@transformshift{4.325858in}{4.447516in}%
\pgfsys@useobject{currentmarker}{}%
\end{pgfscope}%
\begin{pgfscope}%
\pgfsys@transformshift{4.343697in}{4.453071in}%
\pgfsys@useobject{currentmarker}{}%
\end{pgfscope}%
\begin{pgfscope}%
\pgfsys@transformshift{4.365762in}{4.387631in}%
\pgfsys@useobject{currentmarker}{}%
\end{pgfscope}%
\begin{pgfscope}%
\pgfsys@transformshift{4.383835in}{4.251733in}%
\pgfsys@useobject{currentmarker}{}%
\end{pgfscope}%
\begin{pgfscope}%
\pgfsys@transformshift{4.404491in}{4.138202in}%
\pgfsys@useobject{currentmarker}{}%
\end{pgfscope}%
\begin{pgfscope}%
\pgfsys@transformshift{4.423270in}{4.089155in}%
\pgfsys@useobject{currentmarker}{}%
\end{pgfscope}%
\begin{pgfscope}%
\pgfsys@transformshift{4.440875in}{4.058666in}%
\pgfsys@useobject{currentmarker}{}%
\end{pgfscope}%
\begin{pgfscope}%
\pgfsys@transformshift{4.461766in}{4.044514in}%
\pgfsys@useobject{currentmarker}{}%
\end{pgfscope}%
\begin{pgfscope}%
\pgfsys@transformshift{4.479841in}{4.038533in}%
\pgfsys@useobject{currentmarker}{}%
\end{pgfscope}%
\begin{pgfscope}%
\pgfsys@transformshift{4.482187in}{4.038515in}%
\pgfsys@useobject{currentmarker}{}%
\end{pgfscope}%
\begin{pgfscope}%
\pgfsys@transformshift{4.475379in}{4.039896in}%
\pgfsys@useobject{currentmarker}{}%
\end{pgfscope}%
\begin{pgfscope}%
\pgfsys@transformshift{4.455194in}{4.057762in}%
\pgfsys@useobject{currentmarker}{}%
\end{pgfscope}%
\begin{pgfscope}%
\pgfsys@transformshift{4.434303in}{4.131518in}%
\pgfsys@useobject{currentmarker}{}%
\end{pgfscope}%
\begin{pgfscope}%
\pgfsys@transformshift{4.411534in}{4.335421in}%
\pgfsys@useobject{currentmarker}{}%
\end{pgfscope}%
\begin{pgfscope}%
\pgfsys@transformshift{4.396746in}{4.439247in}%
\pgfsys@useobject{currentmarker}{}%
\end{pgfscope}%
\begin{pgfscope}%
\pgfsys@transformshift{4.376795in}{4.440489in}%
\pgfsys@useobject{currentmarker}{}%
\end{pgfscope}%
\begin{pgfscope}%
\pgfsys@transformshift{4.359424in}{4.226292in}%
\pgfsys@useobject{currentmarker}{}%
\end{pgfscope}%
\begin{pgfscope}%
\pgfsys@transformshift{4.337360in}{4.072918in}%
\pgfsys@useobject{currentmarker}{}%
\end{pgfscope}%
\begin{pgfscope}%
\pgfsys@transformshift{4.319755in}{4.045382in}%
\pgfsys@useobject{currentmarker}{}%
\end{pgfscope}%
\begin{pgfscope}%
\pgfsys@transformshift{4.300273in}{4.037598in}%
\pgfsys@useobject{currentmarker}{}%
\end{pgfscope}%
\begin{pgfscope}%
\pgfsys@transformshift{4.280554in}{4.046811in}%
\pgfsys@useobject{currentmarker}{}%
\end{pgfscope}%
\begin{pgfscope}%
\pgfsys@transformshift{4.262715in}{4.079835in}%
\pgfsys@useobject{currentmarker}{}%
\end{pgfscope}%
\begin{pgfscope}%
\pgfsys@transformshift{4.242528in}{4.206440in}%
\pgfsys@useobject{currentmarker}{}%
\end{pgfscope}%
\begin{pgfscope}%
\pgfsys@transformshift{4.223517in}{4.390578in}%
\pgfsys@useobject{currentmarker}{}%
\end{pgfscope}%
\begin{pgfscope}%
\pgfsys@transformshift{4.205207in}{4.445990in}%
\pgfsys@useobject{currentmarker}{}%
\end{pgfscope}%
\begin{pgfscope}%
\pgfsys@transformshift{4.183847in}{4.276564in}%
\pgfsys@useobject{currentmarker}{}%
\end{pgfscope}%
\begin{pgfscope}%
\pgfsys@transformshift{4.165537in}{4.112593in}%
\pgfsys@useobject{currentmarker}{}%
\end{pgfscope}%
\begin{pgfscope}%
\pgfsys@transformshift{4.147698in}{4.056553in}%
\pgfsys@useobject{currentmarker}{}%
\end{pgfscope}%
\begin{pgfscope}%
\pgfsys@transformshift{4.129624in}{4.040644in}%
\pgfsys@useobject{currentmarker}{}%
\end{pgfscope}%
\begin{pgfscope}%
\pgfsys@transformshift{4.107089in}{4.038579in}%
\pgfsys@useobject{currentmarker}{}%
\end{pgfscope}%
\begin{pgfscope}%
\pgfsys@transformshift{4.090424in}{4.050974in}%
\pgfsys@useobject{currentmarker}{}%
\end{pgfscope}%
\begin{pgfscope}%
\pgfsys@transformshift{4.069533in}{4.103922in}%
\pgfsys@useobject{currentmarker}{}%
\end{pgfscope}%
\begin{pgfscope}%
\pgfsys@transformshift{4.051225in}{4.267852in}%
\pgfsys@useobject{currentmarker}{}%
\end{pgfscope}%
\begin{pgfscope}%
\pgfsys@transformshift{4.031038in}{4.420926in}%
\pgfsys@useobject{currentmarker}{}%
\end{pgfscope}%
\begin{pgfscope}%
\pgfsys@transformshift{4.013199in}{4.418776in}%
\pgfsys@useobject{currentmarker}{}%
\end{pgfscope}%
\begin{pgfscope}%
\pgfsys@transformshift{3.994654in}{4.210714in}%
\pgfsys@useobject{currentmarker}{}%
\end{pgfscope}%
\begin{pgfscope}%
\pgfsys@transformshift{3.974703in}{4.078408in}%
\pgfsys@useobject{currentmarker}{}%
\end{pgfscope}%
\begin{pgfscope}%
\pgfsys@transformshift{3.955924in}{4.046644in}%
\pgfsys@useobject{currentmarker}{}%
\end{pgfscope}%
\begin{pgfscope}%
\pgfsys@transformshift{3.935737in}{4.036662in}%
\pgfsys@useobject{currentmarker}{}%
\end{pgfscope}%
\begin{pgfscope}%
\pgfsys@transformshift{3.915315in}{4.041688in}%
\pgfsys@useobject{currentmarker}{}%
\end{pgfscope}%
\begin{pgfscope}%
\pgfsys@transformshift{3.897476in}{4.059329in}%
\pgfsys@useobject{currentmarker}{}%
\end{pgfscope}%
\begin{pgfscope}%
\pgfsys@transformshift{3.880577in}{4.156563in}%
\pgfsys@useobject{currentmarker}{}%
\end{pgfscope}%
\begin{pgfscope}%
\pgfsys@transformshift{3.856869in}{4.345680in}%
\pgfsys@useobject{currentmarker}{}%
\end{pgfscope}%
\begin{pgfscope}%
\pgfsys@transformshift{3.839030in}{4.427289in}%
\pgfsys@useobject{currentmarker}{}%
\end{pgfscope}%
\begin{pgfscope}%
\pgfsys@transformshift{3.819077in}{4.361433in}%
\pgfsys@useobject{currentmarker}{}%
\end{pgfscope}%
\begin{pgfscope}%
\pgfsys@transformshift{3.802646in}{4.365788in}%
\pgfsys@useobject{currentmarker}{}%
\end{pgfscope}%
\begin{pgfscope}%
\pgfsys@transformshift{3.784102in}{4.135751in}%
\pgfsys@useobject{currentmarker}{}%
\end{pgfscope}%
\begin{pgfscope}%
\pgfsys@transformshift{3.763211in}{4.059890in}%
\pgfsys@useobject{currentmarker}{}%
\end{pgfscope}%
\begin{pgfscope}%
\pgfsys@transformshift{3.742555in}{4.039927in}%
\pgfsys@useobject{currentmarker}{}%
\end{pgfscope}%
\begin{pgfscope}%
\pgfsys@transformshift{3.726593in}{4.036086in}%
\pgfsys@useobject{currentmarker}{}%
\end{pgfscope}%
\begin{pgfscope}%
\pgfsys@transformshift{3.704060in}{4.041476in}%
\pgfsys@useobject{currentmarker}{}%
\end{pgfscope}%
\begin{pgfscope}%
\pgfsys@transformshift{3.686689in}{4.059608in}%
\pgfsys@useobject{currentmarker}{}%
\end{pgfscope}%
\begin{pgfscope}%
\pgfsys@transformshift{3.670024in}{4.120137in}%
\pgfsys@useobject{currentmarker}{}%
\end{pgfscope}%
\begin{pgfscope}%
\pgfsys@transformshift{3.648663in}{4.291080in}%
\pgfsys@useobject{currentmarker}{}%
\end{pgfscope}%
\begin{pgfscope}%
\pgfsys@transformshift{3.629181in}{4.417290in}%
\pgfsys@useobject{currentmarker}{}%
\end{pgfscope}%
\begin{pgfscope}%
\pgfsys@transformshift{3.608525in}{4.360867in}%
\pgfsys@useobject{currentmarker}{}%
\end{pgfscope}%
\begin{pgfscope}%
\pgfsys@transformshift{3.590217in}{4.140543in}%
\pgfsys@useobject{currentmarker}{}%
\end{pgfscope}%
\begin{pgfscope}%
\pgfsys@transformshift{3.571203in}{4.061989in}%
\pgfsys@useobject{currentmarker}{}%
\end{pgfscope}%
\begin{pgfscope}%
\pgfsys@transformshift{3.552190in}{4.044433in}%
\pgfsys@useobject{currentmarker}{}%
\end{pgfscope}%
\begin{pgfscope}%
\pgfsys@transformshift{3.528951in}{4.036523in}%
\pgfsys@useobject{currentmarker}{}%
\end{pgfscope}%
\begin{pgfscope}%
\pgfsys@transformshift{3.514398in}{4.036077in}%
\pgfsys@useobject{currentmarker}{}%
\end{pgfscope}%
\begin{pgfscope}%
\pgfsys@transformshift{3.493507in}{4.044634in}%
\pgfsys@useobject{currentmarker}{}%
\end{pgfscope}%
\begin{pgfscope}%
\pgfsys@transformshift{3.473086in}{4.083183in}%
\pgfsys@useobject{currentmarker}{}%
\end{pgfscope}%
\begin{pgfscope}%
\pgfsys@transformshift{3.456655in}{4.146232in}%
\pgfsys@useobject{currentmarker}{}%
\end{pgfscope}%
\begin{pgfscope}%
\pgfsys@transformshift{3.436468in}{4.334575in}%
\pgfsys@useobject{currentmarker}{}%
\end{pgfscope}%
\begin{pgfscope}%
\pgfsys@transformshift{3.417925in}{4.418592in}%
\pgfsys@useobject{currentmarker}{}%
\end{pgfscope}%
\begin{pgfscope}%
\pgfsys@transformshift{3.397269in}{4.316843in}%
\pgfsys@useobject{currentmarker}{}%
\end{pgfscope}%
\begin{pgfscope}%
\pgfsys@transformshift{3.379430in}{4.138524in}%
\pgfsys@useobject{currentmarker}{}%
\end{pgfscope}%
\begin{pgfscope}%
\pgfsys@transformshift{3.361354in}{4.065370in}%
\pgfsys@useobject{currentmarker}{}%
\end{pgfscope}%
\begin{pgfscope}%
\pgfsys@transformshift{3.340933in}{4.044679in}%
\pgfsys@useobject{currentmarker}{}%
\end{pgfscope}%
\begin{pgfscope}%
\pgfsys@transformshift{3.319573in}{4.036050in}%
\pgfsys@useobject{currentmarker}{}%
\end{pgfscope}%
\begin{pgfscope}%
\pgfsys@transformshift{3.302672in}{4.038588in}%
\pgfsys@useobject{currentmarker}{}%
\end{pgfscope}%
\begin{pgfscope}%
\pgfsys@transformshift{3.284598in}{4.050208in}%
\pgfsys@useobject{currentmarker}{}%
\end{pgfscope}%
\begin{pgfscope}%
\pgfsys@transformshift{3.260890in}{4.045277in}%
\pgfsys@useobject{currentmarker}{}%
\end{pgfscope}%
\begin{pgfscope}%
\pgfsys@transformshift{3.243991in}{4.076092in}%
\pgfsys@useobject{currentmarker}{}%
\end{pgfscope}%
\begin{pgfscope}%
\pgfsys@transformshift{3.224978in}{4.160471in}%
\pgfsys@useobject{currentmarker}{}%
\end{pgfscope}%
\begin{pgfscope}%
\pgfsys@transformshift{3.204790in}{4.359561in}%
\pgfsys@useobject{currentmarker}{}%
\end{pgfscope}%
\begin{pgfscope}%
\pgfsys@transformshift{3.186717in}{4.419378in}%
\pgfsys@useobject{currentmarker}{}%
\end{pgfscope}%
\begin{pgfscope}%
\pgfsys@transformshift{3.169112in}{4.345018in}%
\pgfsys@useobject{currentmarker}{}%
\end{pgfscope}%
\begin{pgfscope}%
\pgfsys@transformshift{3.147985in}{4.130073in}%
\pgfsys@useobject{currentmarker}{}%
\end{pgfscope}%
\begin{pgfscope}%
\pgfsys@transformshift{3.130616in}{4.065659in}%
\pgfsys@useobject{currentmarker}{}%
\end{pgfscope}%
\begin{pgfscope}%
\pgfsys@transformshift{3.109255in}{4.041495in}%
\pgfsys@useobject{currentmarker}{}%
\end{pgfscope}%
\begin{pgfscope}%
\pgfsys@transformshift{3.090007in}{4.035659in}%
\pgfsys@useobject{currentmarker}{}%
\end{pgfscope}%
\begin{pgfscope}%
\pgfsys@transformshift{3.070760in}{4.037564in}%
\pgfsys@useobject{currentmarker}{}%
\end{pgfscope}%
\begin{pgfscope}%
\pgfsys@transformshift{3.053624in}{4.046162in}%
\pgfsys@useobject{currentmarker}{}%
\end{pgfscope}%
\begin{pgfscope}%
\pgfsys@transformshift{3.031796in}{4.081568in}%
\pgfsys@useobject{currentmarker}{}%
\end{pgfscope}%
\begin{pgfscope}%
\pgfsys@transformshift{3.013720in}{4.076075in}%
\pgfsys@useobject{currentmarker}{}%
\end{pgfscope}%
\begin{pgfscope}%
\pgfsys@transformshift{2.993064in}{4.092381in}%
\pgfsys@useobject{currentmarker}{}%
\end{pgfscope}%
\begin{pgfscope}%
\pgfsys@transformshift{2.975225in}{4.225966in}%
\pgfsys@useobject{currentmarker}{}%
\end{pgfscope}%
\begin{pgfscope}%
\pgfsys@transformshift{2.957620in}{4.079675in}%
\pgfsys@useobject{currentmarker}{}%
\end{pgfscope}%
\begin{pgfscope}%
\pgfsys@transformshift{2.939312in}{4.213680in}%
\pgfsys@useobject{currentmarker}{}%
\end{pgfscope}%
\begin{pgfscope}%
\pgfsys@transformshift{2.917950in}{4.393840in}%
\pgfsys@useobject{currentmarker}{}%
\end{pgfscope}%
\begin{pgfscope}%
\pgfsys@transformshift{2.897294in}{4.411580in}%
\pgfsys@useobject{currentmarker}{}%
\end{pgfscope}%
\begin{pgfscope}%
\pgfsys@transformshift{2.880629in}{4.258370in}%
\pgfsys@useobject{currentmarker}{}%
\end{pgfscope}%
\begin{pgfscope}%
\pgfsys@transformshift{2.857625in}{4.088066in}%
\pgfsys@useobject{currentmarker}{}%
\end{pgfscope}%
\begin{pgfscope}%
\pgfsys@transformshift{2.841665in}{4.052792in}%
\pgfsys@useobject{currentmarker}{}%
\end{pgfscope}%
\begin{pgfscope}%
\pgfsys@transformshift{2.823589in}{4.038584in}%
\pgfsys@useobject{currentmarker}{}%
\end{pgfscope}%
\begin{pgfscope}%
\pgfsys@transformshift{2.801525in}{4.035893in}%
\pgfsys@useobject{currentmarker}{}%
\end{pgfscope}%
\begin{pgfscope}%
\pgfsys@transformshift{2.782748in}{4.041309in}%
\pgfsys@useobject{currentmarker}{}%
\end{pgfscope}%
\begin{pgfscope}%
\pgfsys@transformshift{2.764438in}{4.055358in}%
\pgfsys@useobject{currentmarker}{}%
\end{pgfscope}%
\begin{pgfscope}%
\pgfsys@transformshift{2.745425in}{4.113621in}%
\pgfsys@useobject{currentmarker}{}%
\end{pgfscope}%
\begin{pgfscope}%
\pgfsys@transformshift{2.726882in}{4.274102in}%
\pgfsys@useobject{currentmarker}{}%
\end{pgfscope}%
\begin{pgfscope}%
\pgfsys@transformshift{2.707634in}{4.401230in}%
\pgfsys@useobject{currentmarker}{}%
\end{pgfscope}%
\begin{pgfscope}%
\pgfsys@transformshift{2.687213in}{4.382254in}%
\pgfsys@useobject{currentmarker}{}%
\end{pgfscope}%
\begin{pgfscope}%
\pgfsys@transformshift{2.670546in}{4.211789in}%
\pgfsys@useobject{currentmarker}{}%
\end{pgfscope}%
\begin{pgfscope}%
\pgfsys@transformshift{2.648952in}{4.110191in}%
\pgfsys@useobject{currentmarker}{}%
\end{pgfscope}%
\begin{pgfscope}%
\pgfsys@transformshift{2.630407in}{4.058083in}%
\pgfsys@useobject{currentmarker}{}%
\end{pgfscope}%
\begin{pgfscope}%
\pgfsys@transformshift{2.608343in}{4.038940in}%
\pgfsys@useobject{currentmarker}{}%
\end{pgfscope}%
\begin{pgfscope}%
\pgfsys@transformshift{2.589329in}{4.035416in}%
\pgfsys@useobject{currentmarker}{}%
\end{pgfscope}%
\begin{pgfscope}%
\pgfsys@transformshift{2.571256in}{4.038120in}%
\pgfsys@useobject{currentmarker}{}%
\end{pgfscope}%
\begin{pgfscope}%
\pgfsys@transformshift{2.551774in}{4.044918in}%
\pgfsys@useobject{currentmarker}{}%
\end{pgfscope}%
\begin{pgfscope}%
\pgfsys@transformshift{2.531352in}{4.073460in}%
\pgfsys@useobject{currentmarker}{}%
\end{pgfscope}%
\begin{pgfscope}%
\pgfsys@transformshift{2.514450in}{4.165569in}%
\pgfsys@useobject{currentmarker}{}%
\end{pgfscope}%
\begin{pgfscope}%
\pgfsys@transformshift{2.493091in}{4.367652in}%
\pgfsys@useobject{currentmarker}{}%
\end{pgfscope}%
\begin{pgfscope}%
\pgfsys@transformshift{2.477364in}{4.413313in}%
\pgfsys@useobject{currentmarker}{}%
\end{pgfscope}%
\begin{pgfscope}%
\pgfsys@transformshift{2.457413in}{4.334156in}%
\pgfsys@useobject{currentmarker}{}%
\end{pgfscope}%
\begin{pgfscope}%
\pgfsys@transformshift{2.438634in}{4.142352in}%
\pgfsys@useobject{currentmarker}{}%
\end{pgfscope}%
\begin{pgfscope}%
\pgfsys@transformshift{2.418446in}{4.065302in}%
\pgfsys@useobject{currentmarker}{}%
\end{pgfscope}%
\begin{pgfscope}%
\pgfsys@transformshift{2.399199in}{4.042240in}%
\pgfsys@useobject{currentmarker}{}%
\end{pgfscope}%
\begin{pgfscope}%
\pgfsys@transformshift{2.380186in}{4.036377in}%
\pgfsys@useobject{currentmarker}{}%
\end{pgfscope}%
\begin{pgfscope}%
\pgfsys@transformshift{2.361643in}{4.036310in}%
\pgfsys@useobject{currentmarker}{}%
\end{pgfscope}%
\begin{pgfscope}%
\pgfsys@transformshift{2.340282in}{4.041546in}%
\pgfsys@useobject{currentmarker}{}%
\end{pgfscope}%
\begin{pgfscope}%
\pgfsys@transformshift{2.321503in}{4.059753in}%
\pgfsys@useobject{currentmarker}{}%
\end{pgfscope}%
\begin{pgfscope}%
\pgfsys@transformshift{2.302726in}{4.084203in}%
\pgfsys@useobject{currentmarker}{}%
\end{pgfscope}%
\begin{pgfscope}%
\pgfsys@transformshift{2.284650in}{4.196286in}%
\pgfsys@useobject{currentmarker}{}%
\end{pgfscope}%
\begin{pgfscope}%
\pgfsys@transformshift{2.265639in}{4.346764in}%
\pgfsys@useobject{currentmarker}{}%
\end{pgfscope}%
\begin{pgfscope}%
\pgfsys@transformshift{2.246860in}{4.405021in}%
\pgfsys@useobject{currentmarker}{}%
\end{pgfscope}%
\begin{pgfscope}%
\pgfsys@transformshift{2.226438in}{4.341271in}%
\pgfsys@useobject{currentmarker}{}%
\end{pgfscope}%
\begin{pgfscope}%
\pgfsys@transformshift{2.207191in}{4.155338in}%
\pgfsys@useobject{currentmarker}{}%
\end{pgfscope}%
\begin{pgfscope}%
\pgfsys@transformshift{2.188178in}{4.082343in}%
\pgfsys@useobject{currentmarker}{}%
\end{pgfscope}%
\begin{pgfscope}%
\pgfsys@transformshift{2.166582in}{4.051611in}%
\pgfsys@useobject{currentmarker}{}%
\end{pgfscope}%
\begin{pgfscope}%
\pgfsys@transformshift{2.147805in}{4.038741in}%
\pgfsys@useobject{currentmarker}{}%
\end{pgfscope}%
\begin{pgfscope}%
\pgfsys@transformshift{2.129729in}{4.035703in}%
\pgfsys@useobject{currentmarker}{}%
\end{pgfscope}%
\begin{pgfscope}%
\pgfsys@transformshift{2.110716in}{4.038439in}%
\pgfsys@useobject{currentmarker}{}%
\end{pgfscope}%
\begin{pgfscope}%
\pgfsys@transformshift{2.089122in}{4.052332in}%
\pgfsys@useobject{currentmarker}{}%
\end{pgfscope}%
\begin{pgfscope}%
\pgfsys@transformshift{2.073395in}{4.071566in}%
\pgfsys@useobject{currentmarker}{}%
\end{pgfscope}%
\begin{pgfscope}%
\pgfsys@transformshift{2.052270in}{4.158185in}%
\pgfsys@useobject{currentmarker}{}%
\end{pgfscope}%
\begin{pgfscope}%
\pgfsys@transformshift{2.033725in}{4.313552in}%
\pgfsys@useobject{currentmarker}{}%
\end{pgfscope}%
\begin{pgfscope}%
\pgfsys@transformshift{2.016355in}{4.402958in}%
\pgfsys@useobject{currentmarker}{}%
\end{pgfscope}%
\begin{pgfscope}%
\pgfsys@transformshift{1.995933in}{4.397728in}%
\pgfsys@useobject{currentmarker}{}%
\end{pgfscope}%
\begin{pgfscope}%
\pgfsys@transformshift{1.976922in}{4.245695in}%
\pgfsys@useobject{currentmarker}{}%
\end{pgfscope}%
\begin{pgfscope}%
\pgfsys@transformshift{1.959081in}{4.110250in}%
\pgfsys@useobject{currentmarker}{}%
\end{pgfscope}%
\begin{pgfscope}%
\pgfsys@transformshift{1.937016in}{4.060086in}%
\pgfsys@useobject{currentmarker}{}%
\end{pgfscope}%
\begin{pgfscope}%
\pgfsys@transformshift{1.915422in}{4.042209in}%
\pgfsys@useobject{currentmarker}{}%
\end{pgfscope}%
\begin{pgfscope}%
\pgfsys@transformshift{1.899695in}{4.037682in}%
\pgfsys@useobject{currentmarker}{}%
\end{pgfscope}%
\begin{pgfscope}%
\pgfsys@transformshift{1.881621in}{4.035693in}%
\pgfsys@useobject{currentmarker}{}%
\end{pgfscope}%
\begin{pgfscope}%
\pgfsys@transformshift{1.863077in}{4.039263in}%
\pgfsys@useobject{currentmarker}{}%
\end{pgfscope}%
\begin{pgfscope}%
\pgfsys@transformshift{1.842186in}{4.054691in}%
\pgfsys@useobject{currentmarker}{}%
\end{pgfscope}%
\begin{pgfscope}%
\pgfsys@transformshift{1.822470in}{4.085355in}%
\pgfsys@useobject{currentmarker}{}%
\end{pgfscope}%
\begin{pgfscope}%
\pgfsys@transformshift{1.804394in}{4.176221in}%
\pgfsys@useobject{currentmarker}{}%
\end{pgfscope}%
\begin{pgfscope}%
\pgfsys@transformshift{1.781861in}{4.325672in}%
\pgfsys@useobject{currentmarker}{}%
\end{pgfscope}%
\begin{pgfscope}%
\pgfsys@transformshift{1.763318in}{4.412941in}%
\pgfsys@useobject{currentmarker}{}%
\end{pgfscope}%
\begin{pgfscope}%
\pgfsys@transformshift{1.745713in}{4.393741in}%
\pgfsys@useobject{currentmarker}{}%
\end{pgfscope}%
\begin{pgfscope}%
\pgfsys@transformshift{1.727403in}{4.332246in}%
\pgfsys@useobject{currentmarker}{}%
\end{pgfscope}%
\begin{pgfscope}%
\pgfsys@transformshift{1.707687in}{4.410352in}%
\pgfsys@useobject{currentmarker}{}%
\end{pgfscope}%
\begin{pgfscope}%
\pgfsys@transformshift{1.689613in}{4.396382in}%
\pgfsys@useobject{currentmarker}{}%
\end{pgfscope}%
\begin{pgfscope}%
\pgfsys@transformshift{1.667783in}{4.178424in}%
\pgfsys@useobject{currentmarker}{}%
\end{pgfscope}%
\begin{pgfscope}%
\pgfsys@transformshift{1.648770in}{4.083907in}%
\pgfsys@useobject{currentmarker}{}%
\end{pgfscope}%
\begin{pgfscope}%
\pgfsys@transformshift{1.629757in}{4.056415in}%
\pgfsys@useobject{currentmarker}{}%
\end{pgfscope}%
\begin{pgfscope}%
\pgfsys@transformshift{1.608632in}{4.039603in}%
\pgfsys@useobject{currentmarker}{}%
\end{pgfscope}%
\begin{pgfscope}%
\pgfsys@transformshift{1.590790in}{4.035997in}%
\pgfsys@useobject{currentmarker}{}%
\end{pgfscope}%
\begin{pgfscope}%
\pgfsys@transformshift{1.572248in}{4.038428in}%
\pgfsys@useobject{currentmarker}{}%
\end{pgfscope}%
\begin{pgfscope}%
\pgfsys@transformshift{1.553703in}{4.045264in}%
\pgfsys@useobject{currentmarker}{}%
\end{pgfscope}%
\begin{pgfscope}%
\pgfsys@transformshift{1.531639in}{4.072882in}%
\pgfsys@useobject{currentmarker}{}%
\end{pgfscope}%
\begin{pgfscope}%
\pgfsys@transformshift{1.516851in}{4.119404in}%
\pgfsys@useobject{currentmarker}{}%
\end{pgfscope}%
\begin{pgfscope}%
\pgfsys@transformshift{1.495021in}{4.279824in}%
\pgfsys@useobject{currentmarker}{}%
\end{pgfscope}%
\begin{pgfscope}%
\pgfsys@transformshift{1.479295in}{4.379226in}%
\pgfsys@useobject{currentmarker}{}%
\end{pgfscope}%
\begin{pgfscope}%
\pgfsys@transformshift{1.457231in}{4.425033in}%
\pgfsys@useobject{currentmarker}{}%
\end{pgfscope}%
\begin{pgfscope}%
\pgfsys@transformshift{1.439391in}{4.385134in}%
\pgfsys@useobject{currentmarker}{}%
\end{pgfscope}%
\begin{pgfscope}%
\pgfsys@transformshift{1.418030in}{4.190091in}%
\pgfsys@useobject{currentmarker}{}%
\end{pgfscope}%
\begin{pgfscope}%
\pgfsys@transformshift{1.399488in}{4.094424in}%
\pgfsys@useobject{currentmarker}{}%
\end{pgfscope}%
\begin{pgfscope}%
\pgfsys@transformshift{1.380004in}{4.056374in}%
\pgfsys@useobject{currentmarker}{}%
\end{pgfscope}%
\begin{pgfscope}%
\pgfsys@transformshift{1.360756in}{4.043595in}%
\pgfsys@useobject{currentmarker}{}%
\end{pgfscope}%
\begin{pgfscope}%
\pgfsys@transformshift{1.339865in}{4.037020in}%
\pgfsys@useobject{currentmarker}{}%
\end{pgfscope}%
\begin{pgfscope}%
\pgfsys@transformshift{1.325077in}{4.036994in}%
\pgfsys@useobject{currentmarker}{}%
\end{pgfscope}%
\begin{pgfscope}%
\pgfsys@transformshift{1.303718in}{4.040852in}%
\pgfsys@useobject{currentmarker}{}%
\end{pgfscope}%
\begin{pgfscope}%
\pgfsys@transformshift{1.284705in}{4.127541in}%
\pgfsys@useobject{currentmarker}{}%
\end{pgfscope}%
\begin{pgfscope}%
\pgfsys@transformshift{1.266160in}{4.061933in}%
\pgfsys@useobject{currentmarker}{}%
\end{pgfscope}%
\begin{pgfscope}%
\pgfsys@transformshift{1.247382in}{4.049353in}%
\pgfsys@useobject{currentmarker}{}%
\end{pgfscope}%
\begin{pgfscope}%
\pgfsys@transformshift{1.226960in}{4.037237in}%
\pgfsys@useobject{currentmarker}{}%
\end{pgfscope}%
\begin{pgfscope}%
\pgfsys@transformshift{1.209591in}{4.037434in}%
\pgfsys@useobject{currentmarker}{}%
\end{pgfscope}%
\begin{pgfscope}%
\pgfsys@transformshift{1.187996in}{4.043400in}%
\pgfsys@useobject{currentmarker}{}%
\end{pgfscope}%
\begin{pgfscope}%
\pgfsys@transformshift{1.167574in}{4.068278in}%
\pgfsys@useobject{currentmarker}{}%
\end{pgfscope}%
\begin{pgfscope}%
\pgfsys@transformshift{1.149031in}{4.137696in}%
\pgfsys@useobject{currentmarker}{}%
\end{pgfscope}%
\begin{pgfscope}%
\pgfsys@transformshift{1.131190in}{4.286713in}%
\pgfsys@useobject{currentmarker}{}%
\end{pgfscope}%
\begin{pgfscope}%
\pgfsys@transformshift{1.111239in}{4.413556in}%
\pgfsys@useobject{currentmarker}{}%
\end{pgfscope}%
\begin{pgfscope}%
\pgfsys@transformshift{1.089878in}{4.430227in}%
\pgfsys@useobject{currentmarker}{}%
\end{pgfscope}%
\begin{pgfscope}%
\pgfsys@transformshift{1.074152in}{4.288633in}%
\pgfsys@useobject{currentmarker}{}%
\end{pgfscope}%
\begin{pgfscope}%
\pgfsys@transformshift{1.052557in}{4.121210in}%
\pgfsys@useobject{currentmarker}{}%
\end{pgfscope}%
\begin{pgfscope}%
\pgfsys@transformshift{1.034014in}{4.074175in}%
\pgfsys@useobject{currentmarker}{}%
\end{pgfscope}%
\begin{pgfscope}%
\pgfsys@transformshift{1.015235in}{4.049566in}%
\pgfsys@useobject{currentmarker}{}%
\end{pgfscope}%
\begin{pgfscope}%
\pgfsys@transformshift{0.996925in}{4.039261in}%
\pgfsys@useobject{currentmarker}{}%
\end{pgfscope}%
\begin{pgfscope}%
\pgfsys@transformshift{0.973923in}{4.038650in}%
\pgfsys@useobject{currentmarker}{}%
\end{pgfscope}%
\begin{pgfscope}%
\pgfsys@transformshift{0.954910in}{4.046086in}%
\pgfsys@useobject{currentmarker}{}%
\end{pgfscope}%
\begin{pgfscope}%
\pgfsys@transformshift{0.937539in}{4.069871in}%
\pgfsys@useobject{currentmarker}{}%
\end{pgfscope}%
\begin{pgfscope}%
\pgfsys@transformshift{0.917118in}{4.134870in}%
\pgfsys@useobject{currentmarker}{}%
\end{pgfscope}%
\begin{pgfscope}%
\pgfsys@transformshift{0.898339in}{4.252415in}%
\pgfsys@useobject{currentmarker}{}%
\end{pgfscope}%
\begin{pgfscope}%
\pgfsys@transformshift{0.880031in}{4.364068in}%
\pgfsys@useobject{currentmarker}{}%
\end{pgfscope}%
\begin{pgfscope}%
\pgfsys@transformshift{0.861252in}{4.439117in}%
\pgfsys@useobject{currentmarker}{}%
\end{pgfscope}%
\begin{pgfscope}%
\pgfsys@transformshift{0.843178in}{4.433658in}%
\pgfsys@useobject{currentmarker}{}%
\end{pgfscope}%
\begin{pgfscope}%
\pgfsys@transformshift{0.824870in}{4.280367in}%
\pgfsys@useobject{currentmarker}{}%
\end{pgfscope}%
\begin{pgfscope}%
\pgfsys@transformshift{0.803040in}{4.122908in}%
\pgfsys@useobject{currentmarker}{}%
\end{pgfscope}%
\begin{pgfscope}%
\pgfsys@transformshift{0.784730in}{4.082102in}%
\pgfsys@useobject{currentmarker}{}%
\end{pgfscope}%
\begin{pgfscope}%
\pgfsys@transformshift{0.766188in}{4.053829in}%
\pgfsys@useobject{currentmarker}{}%
\end{pgfscope}%
\begin{pgfscope}%
\pgfsys@transformshift{0.747643in}{4.044997in}%
\pgfsys@useobject{currentmarker}{}%
\end{pgfscope}%
\begin{pgfscope}%
\pgfsys@transformshift{0.725579in}{4.037594in}%
\pgfsys@useobject{currentmarker}{}%
\end{pgfscope}%
\begin{pgfscope}%
\pgfsys@transformshift{0.708208in}{4.040004in}%
\pgfsys@useobject{currentmarker}{}%
\end{pgfscope}%
\begin{pgfscope}%
\pgfsys@transformshift{0.689431in}{4.048402in}%
\pgfsys@useobject{currentmarker}{}%
\end{pgfscope}%
\begin{pgfscope}%
\pgfsys@transformshift{0.671826in}{4.063942in}%
\pgfsys@useobject{currentmarker}{}%
\end{pgfscope}%
\begin{pgfscope}%
\pgfsys@transformshift{0.645536in}{4.144319in}%
\pgfsys@useobject{currentmarker}{}%
\end{pgfscope}%
\begin{pgfscope}%
\pgfsys@transformshift{0.645302in}{4.142270in}%
\pgfsys@useobject{currentmarker}{}%
\end{pgfscope}%
\begin{pgfscope}%
\pgfsys@transformshift{0.658211in}{4.079394in}%
\pgfsys@useobject{currentmarker}{}%
\end{pgfscope}%
\begin{pgfscope}%
\pgfsys@transformshift{0.676286in}{4.047515in}%
\pgfsys@useobject{currentmarker}{}%
\end{pgfscope}%
\begin{pgfscope}%
\pgfsys@transformshift{0.693420in}{4.037974in}%
\pgfsys@useobject{currentmarker}{}%
\end{pgfscope}%
\begin{pgfscope}%
\pgfsys@transformshift{0.712199in}{4.043657in}%
\pgfsys@useobject{currentmarker}{}%
\end{pgfscope}%
\begin{pgfscope}%
\pgfsys@transformshift{0.736141in}{4.081334in}%
\pgfsys@useobject{currentmarker}{}%
\end{pgfscope}%
\begin{pgfscope}%
\pgfsys@transformshift{0.754217in}{4.187446in}%
\pgfsys@useobject{currentmarker}{}%
\end{pgfscope}%
\begin{pgfscope}%
\pgfsys@transformshift{0.774636in}{4.430039in}%
\pgfsys@useobject{currentmarker}{}%
\end{pgfscope}%
\begin{pgfscope}%
\pgfsys@transformshift{0.790129in}{4.439660in}%
\pgfsys@useobject{currentmarker}{}%
\end{pgfscope}%
\begin{pgfscope}%
\pgfsys@transformshift{0.811020in}{4.296707in}%
\pgfsys@useobject{currentmarker}{}%
\end{pgfscope}%
\begin{pgfscope}%
\pgfsys@transformshift{0.829094in}{4.129642in}%
\pgfsys@useobject{currentmarker}{}%
\end{pgfscope}%
\begin{pgfscope}%
\pgfsys@transformshift{0.848107in}{4.057996in}%
\pgfsys@useobject{currentmarker}{}%
\end{pgfscope}%
\begin{pgfscope}%
\pgfsys@transformshift{0.868763in}{4.039375in}%
\pgfsys@useobject{currentmarker}{}%
\end{pgfscope}%
\begin{pgfscope}%
\pgfsys@transformshift{0.886133in}{4.038615in}%
\pgfsys@useobject{currentmarker}{}%
\end{pgfscope}%
\begin{pgfscope}%
\pgfsys@transformshift{0.907258in}{4.054117in}%
\pgfsys@useobject{currentmarker}{}%
\end{pgfscope}%
\begin{pgfscope}%
\pgfsys@transformshift{0.926037in}{4.099353in}%
\pgfsys@useobject{currentmarker}{}%
\end{pgfscope}%
\begin{pgfscope}%
\pgfsys@transformshift{0.946693in}{4.319222in}%
\pgfsys@useobject{currentmarker}{}%
\end{pgfscope}%
\begin{pgfscope}%
\pgfsys@transformshift{0.963361in}{4.438183in}%
\pgfsys@useobject{currentmarker}{}%
\end{pgfscope}%
\begin{pgfscope}%
\pgfsys@transformshift{0.984017in}{4.355569in}%
\pgfsys@useobject{currentmarker}{}%
\end{pgfscope}%
\begin{pgfscope}%
\pgfsys@transformshift{1.001385in}{4.168180in}%
\pgfsys@useobject{currentmarker}{}%
\end{pgfscope}%
\begin{pgfscope}%
\pgfsys@transformshift{1.020633in}{4.070178in}%
\pgfsys@useobject{currentmarker}{}%
\end{pgfscope}%
\begin{pgfscope}%
\pgfsys@transformshift{1.043871in}{4.040565in}%
\pgfsys@useobject{currentmarker}{}%
\end{pgfscope}%
\begin{pgfscope}%
\pgfsys@transformshift{1.060773in}{4.036661in}%
\pgfsys@useobject{currentmarker}{}%
\end{pgfscope}%
\begin{pgfscope}%
\pgfsys@transformshift{1.080255in}{4.046004in}%
\pgfsys@useobject{currentmarker}{}%
\end{pgfscope}%
\begin{pgfscope}%
\pgfsys@transformshift{1.099972in}{4.071048in}%
\pgfsys@useobject{currentmarker}{}%
\end{pgfscope}%
\begin{pgfscope}%
\pgfsys@transformshift{1.118282in}{4.174669in}%
\pgfsys@useobject{currentmarker}{}%
\end{pgfscope}%
\begin{pgfscope}%
\pgfsys@transformshift{1.139641in}{4.412157in}%
\pgfsys@useobject{currentmarker}{}%
\end{pgfscope}%
\begin{pgfscope}%
\pgfsys@transformshift{1.155134in}{4.419232in}%
\pgfsys@useobject{currentmarker}{}%
\end{pgfscope}%
\begin{pgfscope}%
\pgfsys@transformshift{1.176728in}{4.248937in}%
\pgfsys@useobject{currentmarker}{}%
\end{pgfscope}%
\begin{pgfscope}%
\pgfsys@transformshift{1.191987in}{4.125119in}%
\pgfsys@useobject{currentmarker}{}%
\end{pgfscope}%
\begin{pgfscope}%
\pgfsys@transformshift{1.214989in}{4.051418in}%
\pgfsys@useobject{currentmarker}{}%
\end{pgfscope}%
\begin{pgfscope}%
\pgfsys@transformshift{1.233768in}{4.038256in}%
\pgfsys@useobject{currentmarker}{}%
\end{pgfscope}%
\begin{pgfscope}%
\pgfsys@transformshift{1.252781in}{4.037071in}%
\pgfsys@useobject{currentmarker}{}%
\end{pgfscope}%
\begin{pgfscope}%
\pgfsys@transformshift{1.269681in}{4.045016in}%
\pgfsys@useobject{currentmarker}{}%
\end{pgfscope}%
\begin{pgfscope}%
\pgfsys@transformshift{1.288225in}{4.082117in}%
\pgfsys@useobject{currentmarker}{}%
\end{pgfscope}%
\begin{pgfscope}%
\pgfsys@transformshift{1.310290in}{4.217712in}%
\pgfsys@useobject{currentmarker}{}%
\end{pgfscope}%
\begin{pgfscope}%
\pgfsys@transformshift{1.326955in}{4.406161in}%
\pgfsys@useobject{currentmarker}{}%
\end{pgfscope}%
\begin{pgfscope}%
\pgfsys@transformshift{1.349725in}{4.388107in}%
\pgfsys@useobject{currentmarker}{}%
\end{pgfscope}%
\begin{pgfscope}%
\pgfsys@transformshift{1.369207in}{4.207379in}%
\pgfsys@useobject{currentmarker}{}%
\end{pgfscope}%
\begin{pgfscope}%
\pgfsys@transformshift{1.385872in}{4.084999in}%
\pgfsys@useobject{currentmarker}{}%
\end{pgfscope}%
\begin{pgfscope}%
\pgfsys@transformshift{1.405825in}{4.055582in}%
\pgfsys@useobject{currentmarker}{}%
\end{pgfscope}%
\begin{pgfscope}%
\pgfsys@transformshift{1.424604in}{4.038870in}%
\pgfsys@useobject{currentmarker}{}%
\end{pgfscope}%
\begin{pgfscope}%
\pgfsys@transformshift{1.443380in}{4.036045in}%
\pgfsys@useobject{currentmarker}{}%
\end{pgfscope}%
\begin{pgfscope}%
\pgfsys@transformshift{1.462864in}{4.043370in}%
\pgfsys@useobject{currentmarker}{}%
\end{pgfscope}%
\begin{pgfscope}%
\pgfsys@transformshift{1.481641in}{4.054650in}%
\pgfsys@useobject{currentmarker}{}%
\end{pgfscope}%
\begin{pgfscope}%
\pgfsys@transformshift{1.501594in}{4.111148in}%
\pgfsys@useobject{currentmarker}{}%
\end{pgfscope}%
\begin{pgfscope}%
\pgfsys@transformshift{1.519902in}{4.266435in}%
\pgfsys@useobject{currentmarker}{}%
\end{pgfscope}%
\begin{pgfscope}%
\pgfsys@transformshift{1.542201in}{4.421975in}%
\pgfsys@useobject{currentmarker}{}%
\end{pgfscope}%
\begin{pgfscope}%
\pgfsys@transformshift{1.561685in}{4.327102in}%
\pgfsys@useobject{currentmarker}{}%
\end{pgfscope}%
\begin{pgfscope}%
\pgfsys@transformshift{1.579759in}{4.203168in}%
\pgfsys@useobject{currentmarker}{}%
\end{pgfscope}%
\begin{pgfscope}%
\pgfsys@transformshift{1.599007in}{4.081492in}%
\pgfsys@useobject{currentmarker}{}%
\end{pgfscope}%
\begin{pgfscope}%
\pgfsys@transformshift{1.618723in}{4.048281in}%
\pgfsys@useobject{currentmarker}{}%
\end{pgfscope}%
\begin{pgfscope}%
\pgfsys@transformshift{1.636094in}{4.038310in}%
\pgfsys@useobject{currentmarker}{}%
\end{pgfscope}%
\begin{pgfscope}%
\pgfsys@transformshift{1.654404in}{4.404883in}%
\pgfsys@useobject{currentmarker}{}%
\end{pgfscope}%
\begin{pgfscope}%
\pgfsys@transformshift{1.677640in}{4.371987in}%
\pgfsys@useobject{currentmarker}{}%
\end{pgfscope}%
\begin{pgfscope}%
\pgfsys@transformshift{1.692428in}{4.255464in}%
\pgfsys@useobject{currentmarker}{}%
\end{pgfscope}%
\begin{pgfscope}%
\pgfsys@transformshift{1.715198in}{4.082529in}%
\pgfsys@useobject{currentmarker}{}%
\end{pgfscope}%
\begin{pgfscope}%
\pgfsys@transformshift{1.731160in}{4.048176in}%
\pgfsys@useobject{currentmarker}{}%
\end{pgfscope}%
\begin{pgfscope}%
\pgfsys@transformshift{1.751816in}{4.037083in}%
\pgfsys@useobject{currentmarker}{}%
\end{pgfscope}%
\begin{pgfscope}%
\pgfsys@transformshift{1.772001in}{4.037167in}%
\pgfsys@useobject{currentmarker}{}%
\end{pgfscope}%
\begin{pgfscope}%
\pgfsys@transformshift{1.792423in}{4.044779in}%
\pgfsys@useobject{currentmarker}{}%
\end{pgfscope}%
\begin{pgfscope}%
\pgfsys@transformshift{1.810499in}{4.075836in}%
\pgfsys@useobject{currentmarker}{}%
\end{pgfscope}%
\begin{pgfscope}%
\pgfsys@transformshift{1.829276in}{4.173873in}%
\pgfsys@useobject{currentmarker}{}%
\end{pgfscope}%
\begin{pgfscope}%
\pgfsys@transformshift{1.850403in}{4.390256in}%
\pgfsys@useobject{currentmarker}{}%
\end{pgfscope}%
\begin{pgfscope}%
\pgfsys@transformshift{1.865190in}{4.416515in}%
\pgfsys@useobject{currentmarker}{}%
\end{pgfscope}%
\begin{pgfscope}%
\pgfsys@transformshift{1.883967in}{4.348332in}%
\pgfsys@useobject{currentmarker}{}%
\end{pgfscope}%
\begin{pgfscope}%
\pgfsys@transformshift{1.903686in}{4.156489in}%
\pgfsys@useobject{currentmarker}{}%
\end{pgfscope}%
\begin{pgfscope}%
\pgfsys@transformshift{1.921525in}{4.063238in}%
\pgfsys@useobject{currentmarker}{}%
\end{pgfscope}%
\begin{pgfscope}%
\pgfsys@transformshift{1.944764in}{4.040033in}%
\pgfsys@useobject{currentmarker}{}%
\end{pgfscope}%
\begin{pgfscope}%
\pgfsys@transformshift{1.963072in}{4.035549in}%
\pgfsys@useobject{currentmarker}{}%
\end{pgfscope}%
\begin{pgfscope}%
\pgfsys@transformshift{1.980911in}{4.039333in}%
\pgfsys@useobject{currentmarker}{}%
\end{pgfscope}%
\begin{pgfscope}%
\pgfsys@transformshift{1.999455in}{4.055425in}%
\pgfsys@useobject{currentmarker}{}%
\end{pgfscope}%
\begin{pgfscope}%
\pgfsys@transformshift{2.022223in}{4.145587in}%
\pgfsys@useobject{currentmarker}{}%
\end{pgfscope}%
\begin{pgfscope}%
\pgfsys@transformshift{2.039125in}{4.104651in}%
\pgfsys@useobject{currentmarker}{}%
\end{pgfscope}%
\begin{pgfscope}%
\pgfsys@transformshift{2.059546in}{4.280183in}%
\pgfsys@useobject{currentmarker}{}%
\end{pgfscope}%
\begin{pgfscope}%
\pgfsys@transformshift{2.076915in}{4.417792in}%
\pgfsys@useobject{currentmarker}{}%
\end{pgfscope}%
\begin{pgfscope}%
\pgfsys@transformshift{2.096633in}{4.347446in}%
\pgfsys@useobject{currentmarker}{}%
\end{pgfscope}%
\begin{pgfscope}%
\pgfsys@transformshift{2.116115in}{4.166571in}%
\pgfsys@useobject{currentmarker}{}%
\end{pgfscope}%
\begin{pgfscope}%
\pgfsys@transformshift{2.135832in}{4.067651in}%
\pgfsys@useobject{currentmarker}{}%
\end{pgfscope}%
\begin{pgfscope}%
\pgfsys@transformshift{2.156019in}{4.042312in}%
\pgfsys@useobject{currentmarker}{}%
\end{pgfscope}%
\begin{pgfscope}%
\pgfsys@transformshift{2.173624in}{4.036010in}%
\pgfsys@useobject{currentmarker}{}%
\end{pgfscope}%
\begin{pgfscope}%
\pgfsys@transformshift{2.193106in}{4.035566in}%
\pgfsys@useobject{currentmarker}{}%
\end{pgfscope}%
\begin{pgfscope}%
\pgfsys@transformshift{2.214936in}{4.043560in}%
\pgfsys@useobject{currentmarker}{}%
\end{pgfscope}%
\begin{pgfscope}%
\pgfsys@transformshift{2.233246in}{4.045435in}%
\pgfsys@useobject{currentmarker}{}%
\end{pgfscope}%
\begin{pgfscope}%
\pgfsys@transformshift{2.251554in}{4.069259in}%
\pgfsys@useobject{currentmarker}{}%
\end{pgfscope}%
\begin{pgfscope}%
\pgfsys@transformshift{2.269394in}{4.159963in}%
\pgfsys@useobject{currentmarker}{}%
\end{pgfscope}%
\begin{pgfscope}%
\pgfsys@transformshift{2.286530in}{4.378489in}%
\pgfsys@useobject{currentmarker}{}%
\end{pgfscope}%
\begin{pgfscope}%
\pgfsys@transformshift{2.308594in}{4.412026in}%
\pgfsys@useobject{currentmarker}{}%
\end{pgfscope}%
\begin{pgfscope}%
\pgfsys@transformshift{2.327607in}{4.281649in}%
\pgfsys@useobject{currentmarker}{}%
\end{pgfscope}%
\begin{pgfscope}%
\pgfsys@transformshift{2.348498in}{4.100992in}%
\pgfsys@useobject{currentmarker}{}%
\end{pgfscope}%
\begin{pgfscope}%
\pgfsys@transformshift{2.366337in}{4.065538in}%
\pgfsys@useobject{currentmarker}{}%
\end{pgfscope}%
\begin{pgfscope}%
\pgfsys@transformshift{2.389105in}{4.043901in}%
\pgfsys@useobject{currentmarker}{}%
\end{pgfscope}%
\begin{pgfscope}%
\pgfsys@transformshift{2.408353in}{4.036128in}%
\pgfsys@useobject{currentmarker}{}%
\end{pgfscope}%
\begin{pgfscope}%
\pgfsys@transformshift{2.423611in}{4.035749in}%
\pgfsys@useobject{currentmarker}{}%
\end{pgfscope}%
\begin{pgfscope}%
\pgfsys@transformshift{2.443797in}{4.043302in}%
\pgfsys@useobject{currentmarker}{}%
\end{pgfscope}%
\begin{pgfscope}%
\pgfsys@transformshift{2.461872in}{4.067148in}%
\pgfsys@useobject{currentmarker}{}%
\end{pgfscope}%
\begin{pgfscope}%
\pgfsys@transformshift{2.483232in}{4.146092in}%
\pgfsys@useobject{currentmarker}{}%
\end{pgfscope}%
\begin{pgfscope}%
\pgfsys@transformshift{2.501307in}{4.329389in}%
\pgfsys@useobject{currentmarker}{}%
\end{pgfscope}%
\begin{pgfscope}%
\pgfsys@transformshift{2.519147in}{4.413851in}%
\pgfsys@useobject{currentmarker}{}%
\end{pgfscope}%
\begin{pgfscope}%
\pgfsys@transformshift{2.536986in}{4.396457in}%
\pgfsys@useobject{currentmarker}{}%
\end{pgfscope}%
\begin{pgfscope}%
\pgfsys@transformshift{2.558816in}{4.298548in}%
\pgfsys@useobject{currentmarker}{}%
\end{pgfscope}%
\begin{pgfscope}%
\pgfsys@transformshift{2.578533in}{4.112494in}%
\pgfsys@useobject{currentmarker}{}%
\end{pgfscope}%
\begin{pgfscope}%
\pgfsys@transformshift{2.598483in}{4.051544in}%
\pgfsys@useobject{currentmarker}{}%
\end{pgfscope}%
\begin{pgfscope}%
\pgfsys@transformshift{2.616793in}{4.039515in}%
\pgfsys@useobject{currentmarker}{}%
\end{pgfscope}%
\begin{pgfscope}%
\pgfsys@transformshift{2.638858in}{4.035488in}%
\pgfsys@useobject{currentmarker}{}%
\end{pgfscope}%
\begin{pgfscope}%
\pgfsys@transformshift{2.655054in}{4.039083in}%
\pgfsys@useobject{currentmarker}{}%
\end{pgfscope}%
\begin{pgfscope}%
\pgfsys@transformshift{2.673833in}{4.051574in}%
\pgfsys@useobject{currentmarker}{}%
\end{pgfscope}%
\begin{pgfscope}%
\pgfsys@transformshift{2.694253in}{4.091878in}%
\pgfsys@useobject{currentmarker}{}%
\end{pgfscope}%
\begin{pgfscope}%
\pgfsys@transformshift{2.712329in}{4.244261in}%
\pgfsys@useobject{currentmarker}{}%
\end{pgfscope}%
\begin{pgfscope}%
\pgfsys@transformshift{2.730637in}{4.401262in}%
\pgfsys@useobject{currentmarker}{}%
\end{pgfscope}%
\begin{pgfscope}%
\pgfsys@transformshift{2.751293in}{4.387065in}%
\pgfsys@useobject{currentmarker}{}%
\end{pgfscope}%
\begin{pgfscope}%
\pgfsys@transformshift{2.769603in}{4.238593in}%
\pgfsys@useobject{currentmarker}{}%
\end{pgfscope}%
\begin{pgfscope}%
\pgfsys@transformshift{2.790728in}{4.145028in}%
\pgfsys@useobject{currentmarker}{}%
\end{pgfscope}%
\begin{pgfscope}%
\pgfsys@transformshift{2.808098in}{4.333451in}%
\pgfsys@useobject{currentmarker}{}%
\end{pgfscope}%
\begin{pgfscope}%
\pgfsys@transformshift{2.826172in}{4.157887in}%
\pgfsys@useobject{currentmarker}{}%
\end{pgfscope}%
\begin{pgfscope}%
\pgfsys@transformshift{2.847062in}{4.069510in}%
\pgfsys@useobject{currentmarker}{}%
\end{pgfscope}%
\begin{pgfscope}%
\pgfsys@transformshift{2.865841in}{4.043897in}%
\pgfsys@useobject{currentmarker}{}%
\end{pgfscope}%
\begin{pgfscope}%
\pgfsys@transformshift{2.886497in}{4.037460in}%
\pgfsys@useobject{currentmarker}{}%
\end{pgfscope}%
\begin{pgfscope}%
\pgfsys@transformshift{2.903868in}{4.035546in}%
\pgfsys@useobject{currentmarker}{}%
\end{pgfscope}%
\begin{pgfscope}%
\pgfsys@transformshift{2.921473in}{4.039665in}%
\pgfsys@useobject{currentmarker}{}%
\end{pgfscope}%
\begin{pgfscope}%
\pgfsys@transformshift{2.942598in}{4.058152in}%
\pgfsys@useobject{currentmarker}{}%
\end{pgfscope}%
\begin{pgfscope}%
\pgfsys@transformshift{2.963959in}{4.122099in}%
\pgfsys@useobject{currentmarker}{}%
\end{pgfscope}%
\begin{pgfscope}%
\pgfsys@transformshift{2.982736in}{4.287212in}%
\pgfsys@useobject{currentmarker}{}%
\end{pgfscope}%
\begin{pgfscope}%
\pgfsys@transformshift{3.000811in}{4.405405in}%
\pgfsys@useobject{currentmarker}{}%
\end{pgfscope}%
\begin{pgfscope}%
\pgfsys@transformshift{3.020762in}{4.374006in}%
\pgfsys@useobject{currentmarker}{}%
\end{pgfscope}%
\begin{pgfscope}%
\pgfsys@transformshift{3.039072in}{4.262452in}%
\pgfsys@useobject{currentmarker}{}%
\end{pgfscope}%
\begin{pgfscope}%
\pgfsys@transformshift{3.061371in}{4.097552in}%
\pgfsys@useobject{currentmarker}{}%
\end{pgfscope}%
\begin{pgfscope}%
\pgfsys@transformshift{3.077802in}{4.057325in}%
\pgfsys@useobject{currentmarker}{}%
\end{pgfscope}%
\begin{pgfscope}%
\pgfsys@transformshift{3.096110in}{4.043307in}%
\pgfsys@useobject{currentmarker}{}%
\end{pgfscope}%
\begin{pgfscope}%
\pgfsys@transformshift{3.116297in}{4.036111in}%
\pgfsys@useobject{currentmarker}{}%
\end{pgfscope}%
\begin{pgfscope}%
\pgfsys@transformshift{3.138596in}{4.037941in}%
\pgfsys@useobject{currentmarker}{}%
\end{pgfscope}%
\begin{pgfscope}%
\pgfsys@transformshift{3.153150in}{4.044348in}%
\pgfsys@useobject{currentmarker}{}%
\end{pgfscope}%
\begin{pgfscope}%
\pgfsys@transformshift{3.174040in}{4.069157in}%
\pgfsys@useobject{currentmarker}{}%
\end{pgfscope}%
\begin{pgfscope}%
\pgfsys@transformshift{3.194697in}{4.135923in}%
\pgfsys@useobject{currentmarker}{}%
\end{pgfscope}%
\begin{pgfscope}%
\pgfsys@transformshift{3.210424in}{4.273891in}%
\pgfsys@useobject{currentmarker}{}%
\end{pgfscope}%
\begin{pgfscope}%
\pgfsys@transformshift{3.231549in}{4.417641in}%
\pgfsys@useobject{currentmarker}{}%
\end{pgfscope}%
\begin{pgfscope}%
\pgfsys@transformshift{3.249623in}{4.378616in}%
\pgfsys@useobject{currentmarker}{}%
\end{pgfscope}%
\begin{pgfscope}%
\pgfsys@transformshift{3.271219in}{4.267096in}%
\pgfsys@useobject{currentmarker}{}%
\end{pgfscope}%
\begin{pgfscope}%
\pgfsys@transformshift{3.289058in}{4.126634in}%
\pgfsys@useobject{currentmarker}{}%
\end{pgfscope}%
\begin{pgfscope}%
\pgfsys@transformshift{3.309948in}{4.059963in}%
\pgfsys@useobject{currentmarker}{}%
\end{pgfscope}%
\begin{pgfscope}%
\pgfsys@transformshift{3.329196in}{4.124063in}%
\pgfsys@useobject{currentmarker}{}%
\end{pgfscope}%
\begin{pgfscope}%
\pgfsys@transformshift{3.346098in}{4.064788in}%
\pgfsys@useobject{currentmarker}{}%
\end{pgfscope}%
\begin{pgfscope}%
\pgfsys@transformshift{3.363468in}{4.044061in}%
\pgfsys@useobject{currentmarker}{}%
\end{pgfscope}%
\begin{pgfscope}%
\pgfsys@transformshift{3.387175in}{4.036139in}%
\pgfsys@useobject{currentmarker}{}%
\end{pgfscope}%
\begin{pgfscope}%
\pgfsys@transformshift{3.402432in}{4.037442in}%
\pgfsys@useobject{currentmarker}{}%
\end{pgfscope}%
\begin{pgfscope}%
\pgfsys@transformshift{3.424497in}{4.045544in}%
\pgfsys@useobject{currentmarker}{}%
\end{pgfscope}%
\begin{pgfscope}%
\pgfsys@transformshift{3.441867in}{4.062579in}%
\pgfsys@useobject{currentmarker}{}%
\end{pgfscope}%
\begin{pgfscope}%
\pgfsys@transformshift{3.464166in}{4.111450in}%
\pgfsys@useobject{currentmarker}{}%
\end{pgfscope}%
\begin{pgfscope}%
\pgfsys@transformshift{3.480831in}{4.244330in}%
\pgfsys@useobject{currentmarker}{}%
\end{pgfscope}%
\begin{pgfscope}%
\pgfsys@transformshift{3.500784in}{4.415737in}%
\pgfsys@useobject{currentmarker}{}%
\end{pgfscope}%
\begin{pgfscope}%
\pgfsys@transformshift{3.520032in}{4.410122in}%
\pgfsys@useobject{currentmarker}{}%
\end{pgfscope}%
\begin{pgfscope}%
\pgfsys@transformshift{3.537402in}{4.306597in}%
\pgfsys@useobject{currentmarker}{}%
\end{pgfscope}%
\begin{pgfscope}%
\pgfsys@transformshift{3.558996in}{4.173101in}%
\pgfsys@useobject{currentmarker}{}%
\end{pgfscope}%
\begin{pgfscope}%
\pgfsys@transformshift{3.576837in}{4.082255in}%
\pgfsys@useobject{currentmarker}{}%
\end{pgfscope}%
\begin{pgfscope}%
\pgfsys@transformshift{3.597728in}{4.056574in}%
\pgfsys@useobject{currentmarker}{}%
\end{pgfscope}%
\begin{pgfscope}%
\pgfsys@transformshift{3.615801in}{4.041501in}%
\pgfsys@useobject{currentmarker}{}%
\end{pgfscope}%
\begin{pgfscope}%
\pgfsys@transformshift{3.634580in}{4.037471in}%
\pgfsys@useobject{currentmarker}{}%
\end{pgfscope}%
\begin{pgfscope}%
\pgfsys@transformshift{3.653828in}{4.036588in}%
\pgfsys@useobject{currentmarker}{}%
\end{pgfscope}%
\begin{pgfscope}%
\pgfsys@transformshift{3.676596in}{4.042512in}%
\pgfsys@useobject{currentmarker}{}%
\end{pgfscope}%
\begin{pgfscope}%
\pgfsys@transformshift{3.691149in}{4.054113in}%
\pgfsys@useobject{currentmarker}{}%
\end{pgfscope}%
\begin{pgfscope}%
\pgfsys@transformshift{3.715327in}{4.082556in}%
\pgfsys@useobject{currentmarker}{}%
\end{pgfscope}%
\begin{pgfscope}%
\pgfsys@transformshift{3.733401in}{4.151807in}%
\pgfsys@useobject{currentmarker}{}%
\end{pgfscope}%
\begin{pgfscope}%
\pgfsys@transformshift{3.750772in}{4.283675in}%
\pgfsys@useobject{currentmarker}{}%
\end{pgfscope}%
\begin{pgfscope}%
\pgfsys@transformshift{3.770722in}{4.426350in}%
\pgfsys@useobject{currentmarker}{}%
\end{pgfscope}%
\begin{pgfscope}%
\pgfsys@transformshift{3.786684in}{4.429874in}%
\pgfsys@useobject{currentmarker}{}%
\end{pgfscope}%
\begin{pgfscope}%
\pgfsys@transformshift{3.807341in}{4.332084in}%
\pgfsys@useobject{currentmarker}{}%
\end{pgfscope}%
\begin{pgfscope}%
\pgfsys@transformshift{3.826119in}{4.197596in}%
\pgfsys@useobject{currentmarker}{}%
\end{pgfscope}%
\begin{pgfscope}%
\pgfsys@transformshift{3.847010in}{4.093653in}%
\pgfsys@useobject{currentmarker}{}%
\end{pgfscope}%
\begin{pgfscope}%
\pgfsys@transformshift{3.866963in}{4.063963in}%
\pgfsys@useobject{currentmarker}{}%
\end{pgfscope}%
\begin{pgfscope}%
\pgfsys@transformshift{3.882923in}{4.046577in}%
\pgfsys@useobject{currentmarker}{}%
\end{pgfscope}%
\begin{pgfscope}%
\pgfsys@transformshift{3.903579in}{4.038905in}%
\pgfsys@useobject{currentmarker}{}%
\end{pgfscope}%
\begin{pgfscope}%
\pgfsys@transformshift{3.922358in}{4.036628in}%
\pgfsys@useobject{currentmarker}{}%
\end{pgfscope}%
\begin{pgfscope}%
\pgfsys@transformshift{3.946065in}{4.041913in}%
\pgfsys@useobject{currentmarker}{}%
\end{pgfscope}%
\begin{pgfscope}%
\pgfsys@transformshift{3.961324in}{4.049731in}%
\pgfsys@useobject{currentmarker}{}%
\end{pgfscope}%
\begin{pgfscope}%
\pgfsys@transformshift{3.979163in}{4.069250in}%
\pgfsys@useobject{currentmarker}{}%
\end{pgfscope}%
\begin{pgfscope}%
\pgfsys@transformshift{4.000288in}{4.124443in}%
\pgfsys@useobject{currentmarker}{}%
\end{pgfscope}%
\begin{pgfscope}%
\pgfsys@transformshift{4.018362in}{4.231108in}%
\pgfsys@useobject{currentmarker}{}%
\end{pgfscope}%
\begin{pgfscope}%
\pgfsys@transformshift{4.038783in}{4.425919in}%
\pgfsys@useobject{currentmarker}{}%
\end{pgfscope}%
\begin{pgfscope}%
\pgfsys@transformshift{4.056154in}{4.444742in}%
\pgfsys@useobject{currentmarker}{}%
\end{pgfscope}%
\begin{pgfscope}%
\pgfsys@transformshift{4.074462in}{4.403778in}%
\pgfsys@useobject{currentmarker}{}%
\end{pgfscope}%
\begin{pgfscope}%
\pgfsys@transformshift{4.096058in}{4.270305in}%
\pgfsys@useobject{currentmarker}{}%
\end{pgfscope}%
\begin{pgfscope}%
\pgfsys@transformshift{4.114366in}{4.192802in}%
\pgfsys@useobject{currentmarker}{}%
\end{pgfscope}%
\begin{pgfscope}%
\pgfsys@transformshift{4.136196in}{4.129124in}%
\pgfsys@useobject{currentmarker}{}%
\end{pgfscope}%
\begin{pgfscope}%
\pgfsys@transformshift{4.154975in}{4.069111in}%
\pgfsys@useobject{currentmarker}{}%
\end{pgfscope}%
\begin{pgfscope}%
\pgfsys@transformshift{4.171640in}{4.049252in}%
\pgfsys@useobject{currentmarker}{}%
\end{pgfscope}%
\begin{pgfscope}%
\pgfsys@transformshift{4.193001in}{4.038613in}%
\pgfsys@useobject{currentmarker}{}%
\end{pgfscope}%
\begin{pgfscope}%
\pgfsys@transformshift{4.210840in}{4.037813in}%
\pgfsys@useobject{currentmarker}{}%
\end{pgfscope}%
\begin{pgfscope}%
\pgfsys@transformshift{4.232905in}{4.045277in}%
\pgfsys@useobject{currentmarker}{}%
\end{pgfscope}%
\begin{pgfscope}%
\pgfsys@transformshift{4.250041in}{4.058529in}%
\pgfsys@useobject{currentmarker}{}%
\end{pgfscope}%
\begin{pgfscope}%
\pgfsys@transformshift{4.267644in}{4.094230in}%
\pgfsys@useobject{currentmarker}{}%
\end{pgfscope}%
\begin{pgfscope}%
\pgfsys@transformshift{4.288536in}{4.110860in}%
\pgfsys@useobject{currentmarker}{}%
\end{pgfscope}%
\begin{pgfscope}%
\pgfsys@transformshift{4.307079in}{4.222674in}%
\pgfsys@useobject{currentmarker}{}%
\end{pgfscope}%
\begin{pgfscope}%
\pgfsys@transformshift{4.327969in}{4.423479in}%
\pgfsys@useobject{currentmarker}{}%
\end{pgfscope}%
\begin{pgfscope}%
\pgfsys@transformshift{4.342054in}{4.450170in}%
\pgfsys@useobject{currentmarker}{}%
\end{pgfscope}%
\begin{pgfscope}%
\pgfsys@transformshift{4.365058in}{4.440284in}%
\pgfsys@useobject{currentmarker}{}%
\end{pgfscope}%
\begin{pgfscope}%
\pgfsys@transformshift{4.382427in}{4.375006in}%
\pgfsys@useobject{currentmarker}{}%
\end{pgfscope}%
\begin{pgfscope}%
\pgfsys@transformshift{4.404491in}{4.216280in}%
\pgfsys@useobject{currentmarker}{}%
\end{pgfscope}%
\begin{pgfscope}%
\pgfsys@transformshift{4.425147in}{4.102775in}%
\pgfsys@useobject{currentmarker}{}%
\end{pgfscope}%
\begin{pgfscope}%
\pgfsys@transformshift{4.441109in}{4.072198in}%
\pgfsys@useobject{currentmarker}{}%
\end{pgfscope}%
\begin{pgfscope}%
\pgfsys@transformshift{4.460123in}{4.048075in}%
\pgfsys@useobject{currentmarker}{}%
\end{pgfscope}%
\begin{pgfscope}%
\pgfsys@transformshift{4.481484in}{4.038592in}%
\pgfsys@useobject{currentmarker}{}%
\end{pgfscope}%
\begin{pgfscope}%
\pgfsys@transformshift{4.481013in}{4.042432in}%
\pgfsys@useobject{currentmarker}{}%
\end{pgfscope}%
\begin{pgfscope}%
\pgfsys@transformshift{4.471390in}{4.049912in}%
\pgfsys@useobject{currentmarker}{}%
\end{pgfscope}%
\begin{pgfscope}%
\pgfsys@transformshift{4.456837in}{4.079816in}%
\pgfsys@useobject{currentmarker}{}%
\end{pgfscope}%
\begin{pgfscope}%
\pgfsys@transformshift{4.436650in}{4.217371in}%
\pgfsys@useobject{currentmarker}{}%
\end{pgfscope}%
\begin{pgfscope}%
\pgfsys@transformshift{4.417636in}{4.399470in}%
\pgfsys@useobject{currentmarker}{}%
\end{pgfscope}%
\begin{pgfscope}%
\pgfsys@transformshift{4.398389in}{4.457929in}%
\pgfsys@useobject{currentmarker}{}%
\end{pgfscope}%
\begin{pgfscope}%
\pgfsys@transformshift{4.379375in}{4.039265in}%
\pgfsys@useobject{currentmarker}{}%
\end{pgfscope}%
\begin{pgfscope}%
\pgfsys@transformshift{4.357782in}{4.041109in}%
\pgfsys@useobject{currentmarker}{}%
\end{pgfscope}%
\begin{pgfscope}%
\pgfsys@transformshift{4.339472in}{4.062104in}%
\pgfsys@useobject{currentmarker}{}%
\end{pgfscope}%
\begin{pgfscope}%
\pgfsys@transformshift{4.320929in}{4.133763in}%
\pgfsys@useobject{currentmarker}{}%
\end{pgfscope}%
\begin{pgfscope}%
\pgfsys@transformshift{4.303090in}{4.328974in}%
\pgfsys@useobject{currentmarker}{}%
\end{pgfscope}%
\begin{pgfscope}%
\pgfsys@transformshift{4.283606in}{4.445046in}%
\pgfsys@useobject{currentmarker}{}%
\end{pgfscope}%
\begin{pgfscope}%
\pgfsys@transformshift{4.259430in}{4.345177in}%
\pgfsys@useobject{currentmarker}{}%
\end{pgfscope}%
\begin{pgfscope}%
\pgfsys@transformshift{4.239477in}{4.109459in}%
\pgfsys@useobject{currentmarker}{}%
\end{pgfscope}%
\begin{pgfscope}%
\pgfsys@transformshift{4.222108in}{4.061854in}%
\pgfsys@useobject{currentmarker}{}%
\end{pgfscope}%
\begin{pgfscope}%
\pgfsys@transformshift{4.203798in}{4.040316in}%
\pgfsys@useobject{currentmarker}{}%
\end{pgfscope}%
\begin{pgfscope}%
\pgfsys@transformshift{4.185490in}{4.037959in}%
\pgfsys@useobject{currentmarker}{}%
\end{pgfscope}%
\begin{pgfscope}%
\pgfsys@transformshift{4.166477in}{4.049681in}%
\pgfsys@useobject{currentmarker}{}%
\end{pgfscope}%
\begin{pgfscope}%
\pgfsys@transformshift{4.145116in}{4.105619in}%
\pgfsys@useobject{currentmarker}{}%
\end{pgfscope}%
\begin{pgfscope}%
\pgfsys@transformshift{4.129624in}{4.233942in}%
\pgfsys@useobject{currentmarker}{}%
\end{pgfscope}%
\begin{pgfscope}%
\pgfsys@transformshift{4.108734in}{4.420807in}%
\pgfsys@useobject{currentmarker}{}%
\end{pgfscope}%
\begin{pgfscope}%
\pgfsys@transformshift{4.089721in}{4.433001in}%
\pgfsys@useobject{currentmarker}{}%
\end{pgfscope}%
\begin{pgfscope}%
\pgfsys@transformshift{4.073290in}{4.271331in}%
\pgfsys@useobject{currentmarker}{}%
\end{pgfscope}%
\begin{pgfscope}%
\pgfsys@transformshift{4.050989in}{4.084442in}%
\pgfsys@useobject{currentmarker}{}%
\end{pgfscope}%
\begin{pgfscope}%
\pgfsys@transformshift{4.034793in}{4.048210in}%
\pgfsys@useobject{currentmarker}{}%
\end{pgfscope}%
\begin{pgfscope}%
\pgfsys@transformshift{4.013199in}{4.036658in}%
\pgfsys@useobject{currentmarker}{}%
\end{pgfscope}%
\begin{pgfscope}%
\pgfsys@transformshift{3.993951in}{4.040276in}%
\pgfsys@useobject{currentmarker}{}%
\end{pgfscope}%
\begin{pgfscope}%
\pgfsys@transformshift{3.975172in}{4.049967in}%
\pgfsys@useobject{currentmarker}{}%
\end{pgfscope}%
\begin{pgfscope}%
\pgfsys@transformshift{3.957802in}{4.093762in}%
\pgfsys@useobject{currentmarker}{}%
\end{pgfscope}%
\begin{pgfscope}%
\pgfsys@transformshift{3.937146in}{4.230630in}%
\pgfsys@useobject{currentmarker}{}%
\end{pgfscope}%
\begin{pgfscope}%
\pgfsys@transformshift{3.915786in}{4.409167in}%
\pgfsys@useobject{currentmarker}{}%
\end{pgfscope}%
\begin{pgfscope}%
\pgfsys@transformshift{3.896068in}{4.424908in}%
\pgfsys@useobject{currentmarker}{}%
\end{pgfscope}%
\begin{pgfscope}%
\pgfsys@transformshift{3.878697in}{4.252067in}%
\pgfsys@useobject{currentmarker}{}%
\end{pgfscope}%
\begin{pgfscope}%
\pgfsys@transformshift{3.858512in}{4.087108in}%
\pgfsys@useobject{currentmarker}{}%
\end{pgfscope}%
\begin{pgfscope}%
\pgfsys@transformshift{3.840673in}{4.049016in}%
\pgfsys@useobject{currentmarker}{}%
\end{pgfscope}%
\begin{pgfscope}%
\pgfsys@transformshift{3.820486in}{4.038838in}%
\pgfsys@useobject{currentmarker}{}%
\end{pgfscope}%
\begin{pgfscope}%
\pgfsys@transformshift{3.802412in}{4.036002in}%
\pgfsys@useobject{currentmarker}{}%
\end{pgfscope}%
\begin{pgfscope}%
\pgfsys@transformshift{3.782225in}{4.042025in}%
\pgfsys@useobject{currentmarker}{}%
\end{pgfscope}%
\begin{pgfscope}%
\pgfsys@transformshift{3.761803in}{4.067786in}%
\pgfsys@useobject{currentmarker}{}%
\end{pgfscope}%
\begin{pgfscope}%
\pgfsys@transformshift{3.744432in}{4.147045in}%
\pgfsys@useobject{currentmarker}{}%
\end{pgfscope}%
\begin{pgfscope}%
\pgfsys@transformshift{3.726828in}{4.314042in}%
\pgfsys@useobject{currentmarker}{}%
\end{pgfscope}%
\begin{pgfscope}%
\pgfsys@transformshift{3.704999in}{4.428262in}%
\pgfsys@useobject{currentmarker}{}%
\end{pgfscope}%
\begin{pgfscope}%
\pgfsys@transformshift{3.683169in}{4.377222in}%
\pgfsys@useobject{currentmarker}{}%
\end{pgfscope}%
\begin{pgfscope}%
\pgfsys@transformshift{3.666738in}{4.191143in}%
\pgfsys@useobject{currentmarker}{}%
\end{pgfscope}%
\begin{pgfscope}%
\pgfsys@transformshift{3.649603in}{4.094321in}%
\pgfsys@useobject{currentmarker}{}%
\end{pgfscope}%
\begin{pgfscope}%
\pgfsys@transformshift{3.629415in}{4.052723in}%
\pgfsys@useobject{currentmarker}{}%
\end{pgfscope}%
\begin{pgfscope}%
\pgfsys@transformshift{3.607351in}{4.037824in}%
\pgfsys@useobject{currentmarker}{}%
\end{pgfscope}%
\begin{pgfscope}%
\pgfsys@transformshift{3.589277in}{4.037815in}%
\pgfsys@useobject{currentmarker}{}%
\end{pgfscope}%
\begin{pgfscope}%
\pgfsys@transformshift{3.572377in}{4.036380in}%
\pgfsys@useobject{currentmarker}{}%
\end{pgfscope}%
\begin{pgfscope}%
\pgfsys@transformshift{3.552893in}{4.044667in}%
\pgfsys@useobject{currentmarker}{}%
\end{pgfscope}%
\begin{pgfscope}%
\pgfsys@transformshift{3.531300in}{4.071801in}%
\pgfsys@useobject{currentmarker}{}%
\end{pgfscope}%
\begin{pgfscope}%
\pgfsys@transformshift{3.513695in}{4.100491in}%
\pgfsys@useobject{currentmarker}{}%
\end{pgfscope}%
\begin{pgfscope}%
\pgfsys@transformshift{3.495385in}{4.249347in}%
\pgfsys@useobject{currentmarker}{}%
\end{pgfscope}%
\begin{pgfscope}%
\pgfsys@transformshift{3.476137in}{4.400433in}%
\pgfsys@useobject{currentmarker}{}%
\end{pgfscope}%
\begin{pgfscope}%
\pgfsys@transformshift{3.455715in}{4.405047in}%
\pgfsys@useobject{currentmarker}{}%
\end{pgfscope}%
\begin{pgfscope}%
\pgfsys@transformshift{3.436702in}{4.202141in}%
\pgfsys@useobject{currentmarker}{}%
\end{pgfscope}%
\begin{pgfscope}%
\pgfsys@transformshift{3.412995in}{4.069644in}%
\pgfsys@useobject{currentmarker}{}%
\end{pgfscope}%
\begin{pgfscope}%
\pgfsys@transformshift{3.397972in}{4.278888in}%
\pgfsys@useobject{currentmarker}{}%
\end{pgfscope}%
\begin{pgfscope}%
\pgfsys@transformshift{3.378959in}{4.104593in}%
\pgfsys@useobject{currentmarker}{}%
\end{pgfscope}%
\begin{pgfscope}%
\pgfsys@transformshift{3.360885in}{4.061092in}%
\pgfsys@useobject{currentmarker}{}%
\end{pgfscope}%
\begin{pgfscope}%
\pgfsys@transformshift{3.340933in}{4.040934in}%
\pgfsys@useobject{currentmarker}{}%
\end{pgfscope}%
\begin{pgfscope}%
\pgfsys@transformshift{3.320276in}{4.035740in}%
\pgfsys@useobject{currentmarker}{}%
\end{pgfscope}%
\begin{pgfscope}%
\pgfsys@transformshift{3.301029in}{4.040385in}%
\pgfsys@useobject{currentmarker}{}%
\end{pgfscope}%
\begin{pgfscope}%
\pgfsys@transformshift{3.283424in}{4.054392in}%
\pgfsys@useobject{currentmarker}{}%
\end{pgfscope}%
\begin{pgfscope}%
\pgfsys@transformshift{3.262064in}{4.114717in}%
\pgfsys@useobject{currentmarker}{}%
\end{pgfscope}%
\begin{pgfscope}%
\pgfsys@transformshift{3.245163in}{4.266853in}%
\pgfsys@useobject{currentmarker}{}%
\end{pgfscope}%
\begin{pgfscope}%
\pgfsys@transformshift{3.225212in}{4.386373in}%
\pgfsys@useobject{currentmarker}{}%
\end{pgfscope}%
\begin{pgfscope}%
\pgfsys@transformshift{3.204556in}{4.410910in}%
\pgfsys@useobject{currentmarker}{}%
\end{pgfscope}%
\begin{pgfscope}%
\pgfsys@transformshift{3.186951in}{4.232854in}%
\pgfsys@useobject{currentmarker}{}%
\end{pgfscope}%
\begin{pgfscope}%
\pgfsys@transformshift{3.162538in}{4.096234in}%
\pgfsys@useobject{currentmarker}{}%
\end{pgfscope}%
\begin{pgfscope}%
\pgfsys@transformshift{3.148221in}{4.059448in}%
\pgfsys@useobject{currentmarker}{}%
\end{pgfscope}%
\begin{pgfscope}%
\pgfsys@transformshift{3.129911in}{4.040694in}%
\pgfsys@useobject{currentmarker}{}%
\end{pgfscope}%
\begin{pgfscope}%
\pgfsys@transformshift{3.109021in}{4.201119in}%
\pgfsys@useobject{currentmarker}{}%
\end{pgfscope}%
\begin{pgfscope}%
\pgfsys@transformshift{3.089068in}{4.076175in}%
\pgfsys@useobject{currentmarker}{}%
\end{pgfscope}%
\begin{pgfscope}%
\pgfsys@transformshift{3.070525in}{4.044973in}%
\pgfsys@useobject{currentmarker}{}%
\end{pgfscope}%
\begin{pgfscope}%
\pgfsys@transformshift{3.052452in}{4.036731in}%
\pgfsys@useobject{currentmarker}{}%
\end{pgfscope}%
\begin{pgfscope}%
\pgfsys@transformshift{3.031559in}{4.037581in}%
\pgfsys@useobject{currentmarker}{}%
\end{pgfscope}%
\begin{pgfscope}%
\pgfsys@transformshift{3.013720in}{4.045860in}%
\pgfsys@useobject{currentmarker}{}%
\end{pgfscope}%
\begin{pgfscope}%
\pgfsys@transformshift{2.996115in}{4.073080in}%
\pgfsys@useobject{currentmarker}{}%
\end{pgfscope}%
\begin{pgfscope}%
\pgfsys@transformshift{2.975695in}{4.193597in}%
\pgfsys@useobject{currentmarker}{}%
\end{pgfscope}%
\begin{pgfscope}%
\pgfsys@transformshift{2.957151in}{4.342199in}%
\pgfsys@useobject{currentmarker}{}%
\end{pgfscope}%
\begin{pgfscope}%
\pgfsys@transformshift{2.935790in}{4.415939in}%
\pgfsys@useobject{currentmarker}{}%
\end{pgfscope}%
\begin{pgfscope}%
\pgfsys@transformshift{2.919593in}{4.301190in}%
\pgfsys@useobject{currentmarker}{}%
\end{pgfscope}%
\begin{pgfscope}%
\pgfsys@transformshift{2.901285in}{4.129963in}%
\pgfsys@useobject{currentmarker}{}%
\end{pgfscope}%
\begin{pgfscope}%
\pgfsys@transformshift{2.878281in}{4.055555in}%
\pgfsys@useobject{currentmarker}{}%
\end{pgfscope}%
\begin{pgfscope}%
\pgfsys@transformshift{2.860207in}{4.042824in}%
\pgfsys@useobject{currentmarker}{}%
\end{pgfscope}%
\begin{pgfscope}%
\pgfsys@transformshift{2.838377in}{4.035727in}%
\pgfsys@useobject{currentmarker}{}%
\end{pgfscope}%
\begin{pgfscope}%
\pgfsys@transformshift{2.822181in}{4.036291in}%
\pgfsys@useobject{currentmarker}{}%
\end{pgfscope}%
\begin{pgfscope}%
\pgfsys@transformshift{2.804342in}{4.039586in}%
\pgfsys@useobject{currentmarker}{}%
\end{pgfscope}%
\begin{pgfscope}%
\pgfsys@transformshift{2.782748in}{4.052586in}%
\pgfsys@useobject{currentmarker}{}%
\end{pgfscope}%
\begin{pgfscope}%
\pgfsys@transformshift{2.764203in}{4.105237in}%
\pgfsys@useobject{currentmarker}{}%
\end{pgfscope}%
\begin{pgfscope}%
\pgfsys@transformshift{2.745895in}{4.221337in}%
\pgfsys@useobject{currentmarker}{}%
\end{pgfscope}%
\begin{pgfscope}%
\pgfsys@transformshift{2.726177in}{4.392270in}%
\pgfsys@useobject{currentmarker}{}%
\end{pgfscope}%
\begin{pgfscope}%
\pgfsys@transformshift{2.706226in}{4.412718in}%
\pgfsys@useobject{currentmarker}{}%
\end{pgfscope}%
\begin{pgfscope}%
\pgfsys@transformshift{2.682518in}{4.292369in}%
\pgfsys@useobject{currentmarker}{}%
\end{pgfscope}%
\begin{pgfscope}%
\pgfsys@transformshift{2.667260in}{4.140600in}%
\pgfsys@useobject{currentmarker}{}%
\end{pgfscope}%
\begin{pgfscope}%
\pgfsys@transformshift{2.648247in}{4.067089in}%
\pgfsys@useobject{currentmarker}{}%
\end{pgfscope}%
\begin{pgfscope}%
\pgfsys@transformshift{2.628999in}{4.058925in}%
\pgfsys@useobject{currentmarker}{}%
\end{pgfscope}%
\begin{pgfscope}%
\pgfsys@transformshift{2.610456in}{4.040685in}%
\pgfsys@useobject{currentmarker}{}%
\end{pgfscope}%
\begin{pgfscope}%
\pgfsys@transformshift{2.592615in}{4.035654in}%
\pgfsys@useobject{currentmarker}{}%
\end{pgfscope}%
\begin{pgfscope}%
\pgfsys@transformshift{2.570787in}{4.039050in}%
\pgfsys@useobject{currentmarker}{}%
\end{pgfscope}%
\begin{pgfscope}%
\pgfsys@transformshift{2.552711in}{4.053064in}%
\pgfsys@useobject{currentmarker}{}%
\end{pgfscope}%
\begin{pgfscope}%
\pgfsys@transformshift{2.533464in}{4.102852in}%
\pgfsys@useobject{currentmarker}{}%
\end{pgfscope}%
\begin{pgfscope}%
\pgfsys@transformshift{2.513278in}{4.216506in}%
\pgfsys@useobject{currentmarker}{}%
\end{pgfscope}%
\begin{pgfscope}%
\pgfsys@transformshift{2.494265in}{4.372296in}%
\pgfsys@useobject{currentmarker}{}%
\end{pgfscope}%
\begin{pgfscope}%
\pgfsys@transformshift{2.474547in}{4.410068in}%
\pgfsys@useobject{currentmarker}{}%
\end{pgfscope}%
\begin{pgfscope}%
\pgfsys@transformshift{2.457413in}{4.266425in}%
\pgfsys@useobject{currentmarker}{}%
\end{pgfscope}%
\begin{pgfscope}%
\pgfsys@transformshift{2.437460in}{4.109251in}%
\pgfsys@useobject{currentmarker}{}%
\end{pgfscope}%
\begin{pgfscope}%
\pgfsys@transformshift{2.415630in}{4.052652in}%
\pgfsys@useobject{currentmarker}{}%
\end{pgfscope}%
\begin{pgfscope}%
\pgfsys@transformshift{2.400373in}{4.040733in}%
\pgfsys@useobject{currentmarker}{}%
\end{pgfscope}%
\begin{pgfscope}%
\pgfsys@transformshift{2.379011in}{4.035937in}%
\pgfsys@useobject{currentmarker}{}%
\end{pgfscope}%
\begin{pgfscope}%
\pgfsys@transformshift{2.360000in}{4.037294in}%
\pgfsys@useobject{currentmarker}{}%
\end{pgfscope}%
\begin{pgfscope}%
\pgfsys@transformshift{2.342395in}{4.042641in}%
\pgfsys@useobject{currentmarker}{}%
\end{pgfscope}%
\begin{pgfscope}%
\pgfsys@transformshift{2.323617in}{4.064537in}%
\pgfsys@useobject{currentmarker}{}%
\end{pgfscope}%
\begin{pgfscope}%
\pgfsys@transformshift{2.304603in}{4.140841in}%
\pgfsys@useobject{currentmarker}{}%
\end{pgfscope}%
\begin{pgfscope}%
\pgfsys@transformshift{2.282070in}{4.272598in}%
\pgfsys@useobject{currentmarker}{}%
\end{pgfscope}%
\begin{pgfscope}%
\pgfsys@transformshift{2.263994in}{4.397384in}%
\pgfsys@useobject{currentmarker}{}%
\end{pgfscope}%
\begin{pgfscope}%
\pgfsys@transformshift{2.245452in}{4.406571in}%
\pgfsys@useobject{currentmarker}{}%
\end{pgfscope}%
\begin{pgfscope}%
\pgfsys@transformshift{2.224561in}{4.241480in}%
\pgfsys@useobject{currentmarker}{}%
\end{pgfscope}%
\begin{pgfscope}%
\pgfsys@transformshift{2.206017in}{4.106977in}%
\pgfsys@useobject{currentmarker}{}%
\end{pgfscope}%
\begin{pgfscope}%
\pgfsys@transformshift{2.187238in}{4.056791in}%
\pgfsys@useobject{currentmarker}{}%
\end{pgfscope}%
\begin{pgfscope}%
\pgfsys@transformshift{2.169399in}{4.042382in}%
\pgfsys@useobject{currentmarker}{}%
\end{pgfscope}%
\begin{pgfscope}%
\pgfsys@transformshift{2.148743in}{4.036228in}%
\pgfsys@useobject{currentmarker}{}%
\end{pgfscope}%
\begin{pgfscope}%
\pgfsys@transformshift{2.129026in}{4.036941in}%
\pgfsys@useobject{currentmarker}{}%
\end{pgfscope}%
\begin{pgfscope}%
\pgfsys@transformshift{2.110013in}{4.042784in}%
\pgfsys@useobject{currentmarker}{}%
\end{pgfscope}%
\begin{pgfscope}%
\pgfsys@transformshift{2.090765in}{4.063065in}%
\pgfsys@useobject{currentmarker}{}%
\end{pgfscope}%
\begin{pgfscope}%
\pgfsys@transformshift{2.069875in}{4.124264in}%
\pgfsys@useobject{currentmarker}{}%
\end{pgfscope}%
\begin{pgfscope}%
\pgfsys@transformshift{2.049922in}{4.282479in}%
\pgfsys@useobject{currentmarker}{}%
\end{pgfscope}%
\begin{pgfscope}%
\pgfsys@transformshift{2.035839in}{4.390962in}%
\pgfsys@useobject{currentmarker}{}%
\end{pgfscope}%
\begin{pgfscope}%
\pgfsys@transformshift{2.015886in}{4.404688in}%
\pgfsys@useobject{currentmarker}{}%
\end{pgfscope}%
\begin{pgfscope}%
\pgfsys@transformshift{1.995699in}{4.311385in}%
\pgfsys@useobject{currentmarker}{}%
\end{pgfscope}%
\begin{pgfscope}%
\pgfsys@transformshift{1.974574in}{4.134276in}%
\pgfsys@useobject{currentmarker}{}%
\end{pgfscope}%
\begin{pgfscope}%
\pgfsys@transformshift{1.956969in}{4.066367in}%
\pgfsys@useobject{currentmarker}{}%
\end{pgfscope}%
\begin{pgfscope}%
\pgfsys@transformshift{1.940069in}{4.039090in}%
\pgfsys@useobject{currentmarker}{}%
\end{pgfscope}%
\begin{pgfscope}%
\pgfsys@transformshift{1.919882in}{4.048252in}%
\pgfsys@useobject{currentmarker}{}%
\end{pgfscope}%
\begin{pgfscope}%
\pgfsys@transformshift{1.900869in}{4.092408in}%
\pgfsys@useobject{currentmarker}{}%
\end{pgfscope}%
\begin{pgfscope}%
\pgfsys@transformshift{1.879508in}{4.228907in}%
\pgfsys@useobject{currentmarker}{}%
\end{pgfscope}%
\begin{pgfscope}%
\pgfsys@transformshift{1.861668in}{4.378286in}%
\pgfsys@useobject{currentmarker}{}%
\end{pgfscope}%
\begin{pgfscope}%
\pgfsys@transformshift{1.842186in}{4.410739in}%
\pgfsys@useobject{currentmarker}{}%
\end{pgfscope}%
\begin{pgfscope}%
\pgfsys@transformshift{1.820827in}{4.287310in}%
\pgfsys@useobject{currentmarker}{}%
\end{pgfscope}%
\begin{pgfscope}%
\pgfsys@transformshift{1.803925in}{4.113019in}%
\pgfsys@useobject{currentmarker}{}%
\end{pgfscope}%
\begin{pgfscope}%
\pgfsys@transformshift{1.786086in}{4.065684in}%
\pgfsys@useobject{currentmarker}{}%
\end{pgfscope}%
\begin{pgfscope}%
\pgfsys@transformshift{1.764727in}{4.045317in}%
\pgfsys@useobject{currentmarker}{}%
\end{pgfscope}%
\begin{pgfscope}%
\pgfsys@transformshift{1.746885in}{4.036136in}%
\pgfsys@useobject{currentmarker}{}%
\end{pgfscope}%
\begin{pgfscope}%
\pgfsys@transformshift{1.726700in}{4.037756in}%
\pgfsys@useobject{currentmarker}{}%
\end{pgfscope}%
\begin{pgfscope}%
\pgfsys@transformshift{1.707687in}{4.047610in}%
\pgfsys@useobject{currentmarker}{}%
\end{pgfscope}%
\begin{pgfscope}%
\pgfsys@transformshift{1.686326in}{4.099841in}%
\pgfsys@useobject{currentmarker}{}%
\end{pgfscope}%
\begin{pgfscope}%
\pgfsys@transformshift{1.664732in}{4.232366in}%
\pgfsys@useobject{currentmarker}{}%
\end{pgfscope}%
\begin{pgfscope}%
\pgfsys@transformshift{1.649707in}{4.370148in}%
\pgfsys@useobject{currentmarker}{}%
\end{pgfscope}%
\begin{pgfscope}%
\pgfsys@transformshift{1.628348in}{4.414188in}%
\pgfsys@useobject{currentmarker}{}%
\end{pgfscope}%
\begin{pgfscope}%
\pgfsys@transformshift{1.611212in}{4.391152in}%
\pgfsys@useobject{currentmarker}{}%
\end{pgfscope}%
\begin{pgfscope}%
\pgfsys@transformshift{1.592904in}{4.205638in}%
\pgfsys@useobject{currentmarker}{}%
\end{pgfscope}%
\begin{pgfscope}%
\pgfsys@transformshift{1.574830in}{4.083439in}%
\pgfsys@useobject{currentmarker}{}%
\end{pgfscope}%
\begin{pgfscope}%
\pgfsys@transformshift{1.554174in}{4.048226in}%
\pgfsys@useobject{currentmarker}{}%
\end{pgfscope}%
\begin{pgfscope}%
\pgfsys@transformshift{1.533047in}{4.038311in}%
\pgfsys@useobject{currentmarker}{}%
\end{pgfscope}%
\begin{pgfscope}%
\pgfsys@transformshift{1.516851in}{4.035960in}%
\pgfsys@useobject{currentmarker}{}%
\end{pgfscope}%
\begin{pgfscope}%
\pgfsys@transformshift{1.495726in}{4.042186in}%
\pgfsys@useobject{currentmarker}{}%
\end{pgfscope}%
\begin{pgfscope}%
\pgfsys@transformshift{1.476713in}{4.057584in}%
\pgfsys@useobject{currentmarker}{}%
\end{pgfscope}%
\begin{pgfscope}%
\pgfsys@transformshift{1.455353in}{4.126938in}%
\pgfsys@useobject{currentmarker}{}%
\end{pgfscope}%
\begin{pgfscope}%
\pgfsys@transformshift{1.439626in}{4.257343in}%
\pgfsys@useobject{currentmarker}{}%
\end{pgfscope}%
\begin{pgfscope}%
\pgfsys@transformshift{1.418501in}{4.396623in}%
\pgfsys@useobject{currentmarker}{}%
\end{pgfscope}%
\begin{pgfscope}%
\pgfsys@transformshift{1.402068in}{4.426168in}%
\pgfsys@useobject{currentmarker}{}%
\end{pgfscope}%
\begin{pgfscope}%
\pgfsys@transformshift{1.380709in}{4.362608in}%
\pgfsys@useobject{currentmarker}{}%
\end{pgfscope}%
\begin{pgfscope}%
\pgfsys@transformshift{1.360756in}{4.137168in}%
\pgfsys@useobject{currentmarker}{}%
\end{pgfscope}%
\begin{pgfscope}%
\pgfsys@transformshift{1.344560in}{4.070763in}%
\pgfsys@useobject{currentmarker}{}%
\end{pgfscope}%
\begin{pgfscope}%
\pgfsys@transformshift{1.324138in}{4.045266in}%
\pgfsys@useobject{currentmarker}{}%
\end{pgfscope}%
\begin{pgfscope}%
\pgfsys@transformshift{1.303952in}{4.038186in}%
\pgfsys@useobject{currentmarker}{}%
\end{pgfscope}%
\begin{pgfscope}%
\pgfsys@transformshift{1.284705in}{4.036995in}%
\pgfsys@useobject{currentmarker}{}%
\end{pgfscope}%
\begin{pgfscope}%
\pgfsys@transformshift{1.263343in}{4.046703in}%
\pgfsys@useobject{currentmarker}{}%
\end{pgfscope}%
\begin{pgfscope}%
\pgfsys@transformshift{1.248087in}{4.055092in}%
\pgfsys@useobject{currentmarker}{}%
\end{pgfscope}%
\begin{pgfscope}%
\pgfsys@transformshift{1.227665in}{4.103484in}%
\pgfsys@useobject{currentmarker}{}%
\end{pgfscope}%
\begin{pgfscope}%
\pgfsys@transformshift{1.207243in}{4.227172in}%
\pgfsys@useobject{currentmarker}{}%
\end{pgfscope}%
\begin{pgfscope}%
\pgfsys@transformshift{1.186353in}{4.356282in}%
\pgfsys@useobject{currentmarker}{}%
\end{pgfscope}%
\begin{pgfscope}%
\pgfsys@transformshift{1.167808in}{4.428949in}%
\pgfsys@useobject{currentmarker}{}%
\end{pgfscope}%
\begin{pgfscope}%
\pgfsys@transformshift{1.149266in}{4.424035in}%
\pgfsys@useobject{currentmarker}{}%
\end{pgfscope}%
\begin{pgfscope}%
\pgfsys@transformshift{1.131661in}{4.272281in}%
\pgfsys@useobject{currentmarker}{}%
\end{pgfscope}%
\begin{pgfscope}%
\pgfsys@transformshift{1.112413in}{4.134866in}%
\pgfsys@useobject{currentmarker}{}%
\end{pgfscope}%
\begin{pgfscope}%
\pgfsys@transformshift{1.093166in}{4.148902in}%
\pgfsys@useobject{currentmarker}{}%
\end{pgfscope}%
\begin{pgfscope}%
\pgfsys@transformshift{1.071570in}{4.066516in}%
\pgfsys@useobject{currentmarker}{}%
\end{pgfscope}%
\begin{pgfscope}%
\pgfsys@transformshift{1.053496in}{4.047210in}%
\pgfsys@useobject{currentmarker}{}%
\end{pgfscope}%
\begin{pgfscope}%
\pgfsys@transformshift{1.031900in}{4.037441in}%
\pgfsys@useobject{currentmarker}{}%
\end{pgfscope}%
\begin{pgfscope}%
\pgfsys@transformshift{1.012887in}{4.037758in}%
\pgfsys@useobject{currentmarker}{}%
\end{pgfscope}%
\begin{pgfscope}%
\pgfsys@transformshift{0.995282in}{4.044704in}%
\pgfsys@useobject{currentmarker}{}%
\end{pgfscope}%
\begin{pgfscope}%
\pgfsys@transformshift{0.976740in}{4.061727in}%
\pgfsys@useobject{currentmarker}{}%
\end{pgfscope}%
\begin{pgfscope}%
\pgfsys@transformshift{0.956084in}{4.109329in}%
\pgfsys@useobject{currentmarker}{}%
\end{pgfscope}%
\begin{pgfscope}%
\pgfsys@transformshift{0.937774in}{4.207885in}%
\pgfsys@useobject{currentmarker}{}%
\end{pgfscope}%
\begin{pgfscope}%
\pgfsys@transformshift{0.921812in}{4.308987in}%
\pgfsys@useobject{currentmarker}{}%
\end{pgfscope}%
\begin{pgfscope}%
\pgfsys@transformshift{0.902330in}{4.418215in}%
\pgfsys@useobject{currentmarker}{}%
\end{pgfscope}%
\begin{pgfscope}%
\pgfsys@transformshift{0.883317in}{4.443092in}%
\pgfsys@useobject{currentmarker}{}%
\end{pgfscope}%
\begin{pgfscope}%
\pgfsys@transformshift{0.861723in}{4.330267in}%
\pgfsys@useobject{currentmarker}{}%
\end{pgfscope}%
\begin{pgfscope}%
\pgfsys@transformshift{0.843647in}{4.205836in}%
\pgfsys@useobject{currentmarker}{}%
\end{pgfscope}%
\begin{pgfscope}%
\pgfsys@transformshift{0.824870in}{4.102047in}%
\pgfsys@useobject{currentmarker}{}%
\end{pgfscope}%
\begin{pgfscope}%
\pgfsys@transformshift{0.803978in}{4.059602in}%
\pgfsys@useobject{currentmarker}{}%
\end{pgfscope}%
\begin{pgfscope}%
\pgfsys@transformshift{0.785904in}{4.043635in}%
\pgfsys@useobject{currentmarker}{}%
\end{pgfscope}%
\begin{pgfscope}%
\pgfsys@transformshift{0.765248in}{4.038400in}%
\pgfsys@useobject{currentmarker}{}%
\end{pgfscope}%
\begin{pgfscope}%
\pgfsys@transformshift{0.746235in}{4.039168in}%
\pgfsys@useobject{currentmarker}{}%
\end{pgfscope}%
\begin{pgfscope}%
\pgfsys@transformshift{0.726753in}{4.049380in}%
\pgfsys@useobject{currentmarker}{}%
\end{pgfscope}%
\begin{pgfscope}%
\pgfsys@transformshift{0.708679in}{4.038533in}%
\pgfsys@useobject{currentmarker}{}%
\end{pgfscope}%
\begin{pgfscope}%
\pgfsys@transformshift{0.689666in}{4.047704in}%
\pgfsys@useobject{currentmarker}{}%
\end{pgfscope}%
\begin{pgfscope}%
\pgfsys@transformshift{0.667835in}{4.079419in}%
\pgfsys@useobject{currentmarker}{}%
\end{pgfscope}%
\begin{pgfscope}%
\pgfsys@transformshift{0.649762in}{4.144974in}%
\pgfsys@useobject{currentmarker}{}%
\end{pgfscope}%
\begin{pgfscope}%
\pgfsys@transformshift{0.649525in}{4.143592in}%
\pgfsys@useobject{currentmarker}{}%
\end{pgfscope}%
\begin{pgfscope}%
\pgfsys@transformshift{0.658916in}{4.094525in}%
\pgfsys@useobject{currentmarker}{}%
\end{pgfscope}%
\begin{pgfscope}%
\pgfsys@transformshift{0.676521in}{4.051102in}%
\pgfsys@useobject{currentmarker}{}%
\end{pgfscope}%
\begin{pgfscope}%
\pgfsys@transformshift{0.694829in}{4.038486in}%
\pgfsys@useobject{currentmarker}{}%
\end{pgfscope}%
\begin{pgfscope}%
\pgfsys@transformshift{0.715485in}{4.042804in}%
\pgfsys@useobject{currentmarker}{}%
\end{pgfscope}%
\begin{pgfscope}%
\pgfsys@transformshift{0.733795in}{4.063200in}%
\pgfsys@useobject{currentmarker}{}%
\end{pgfscope}%
\begin{pgfscope}%
\pgfsys@transformshift{0.751165in}{4.128509in}%
\pgfsys@useobject{currentmarker}{}%
\end{pgfscope}%
\begin{pgfscope}%
\pgfsys@transformshift{0.772759in}{4.374820in}%
\pgfsys@useobject{currentmarker}{}%
\end{pgfscope}%
\begin{pgfscope}%
\pgfsys@transformshift{0.791069in}{4.447928in}%
\pgfsys@useobject{currentmarker}{}%
\end{pgfscope}%
\begin{pgfscope}%
\pgfsys@transformshift{0.810786in}{4.362002in}%
\pgfsys@useobject{currentmarker}{}%
\end{pgfscope}%
\begin{pgfscope}%
\pgfsys@transformshift{0.830268in}{4.150862in}%
\pgfsys@useobject{currentmarker}{}%
\end{pgfscope}%
\begin{pgfscope}%
\pgfsys@transformshift{0.848578in}{4.066941in}%
\pgfsys@useobject{currentmarker}{}%
\end{pgfscope}%
\begin{pgfscope}%
\pgfsys@transformshift{0.870406in}{4.040928in}%
\pgfsys@useobject{currentmarker}{}%
\end{pgfscope}%
\begin{pgfscope}%
\pgfsys@transformshift{0.884256in}{4.037037in}%
\pgfsys@useobject{currentmarker}{}%
\end{pgfscope}%
\begin{pgfscope}%
\pgfsys@transformshift{0.906321in}{4.047851in}%
\pgfsys@useobject{currentmarker}{}%
\end{pgfscope}%
\begin{pgfscope}%
\pgfsys@transformshift{0.924629in}{4.079637in}%
\pgfsys@useobject{currentmarker}{}%
\end{pgfscope}%
\begin{pgfscope}%
\pgfsys@transformshift{0.945051in}{4.222260in}%
\pgfsys@useobject{currentmarker}{}%
\end{pgfscope}%
\begin{pgfscope}%
\pgfsys@transformshift{0.965472in}{4.432528in}%
\pgfsys@useobject{currentmarker}{}%
\end{pgfscope}%
\begin{pgfscope}%
\pgfsys@transformshift{0.981903in}{4.401968in}%
\pgfsys@useobject{currentmarker}{}%
\end{pgfscope}%
\begin{pgfscope}%
\pgfsys@transformshift{1.003030in}{4.192935in}%
\pgfsys@useobject{currentmarker}{}%
\end{pgfscope}%
\begin{pgfscope}%
\pgfsys@transformshift{1.021338in}{4.104107in}%
\pgfsys@useobject{currentmarker}{}%
\end{pgfscope}%
\begin{pgfscope}%
\pgfsys@transformshift{1.041760in}{4.051507in}%
\pgfsys@useobject{currentmarker}{}%
\end{pgfscope}%
\begin{pgfscope}%
\pgfsys@transformshift{1.061007in}{4.037754in}%
\pgfsys@useobject{currentmarker}{}%
\end{pgfscope}%
\begin{pgfscope}%
\pgfsys@transformshift{1.082838in}{4.039554in}%
\pgfsys@useobject{currentmarker}{}%
\end{pgfscope}%
\begin{pgfscope}%
\pgfsys@transformshift{1.097155in}{4.053181in}%
\pgfsys@useobject{currentmarker}{}%
\end{pgfscope}%
\begin{pgfscope}%
\pgfsys@transformshift{1.119219in}{4.114561in}%
\pgfsys@useobject{currentmarker}{}%
\end{pgfscope}%
\begin{pgfscope}%
\pgfsys@transformshift{1.135886in}{4.261556in}%
\pgfsys@useobject{currentmarker}{}%
\end{pgfscope}%
\begin{pgfscope}%
\pgfsys@transformshift{1.155837in}{4.430976in}%
\pgfsys@useobject{currentmarker}{}%
\end{pgfscope}%
\begin{pgfscope}%
\pgfsys@transformshift{1.176728in}{4.338384in}%
\pgfsys@useobject{currentmarker}{}%
\end{pgfscope}%
\begin{pgfscope}%
\pgfsys@transformshift{1.195741in}{4.147449in}%
\pgfsys@useobject{currentmarker}{}%
\end{pgfscope}%
\begin{pgfscope}%
\pgfsys@transformshift{1.213580in}{4.069107in}%
\pgfsys@useobject{currentmarker}{}%
\end{pgfscope}%
\begin{pgfscope}%
\pgfsys@transformshift{1.234707in}{4.040565in}%
\pgfsys@useobject{currentmarker}{}%
\end{pgfscope}%
\begin{pgfscope}%
\pgfsys@transformshift{1.254424in}{4.036346in}%
\pgfsys@useobject{currentmarker}{}%
\end{pgfscope}%
\begin{pgfscope}%
\pgfsys@transformshift{1.276254in}{4.044242in}%
\pgfsys@useobject{currentmarker}{}%
\end{pgfscope}%
\begin{pgfscope}%
\pgfsys@transformshift{1.291511in}{4.060943in}%
\pgfsys@useobject{currentmarker}{}%
\end{pgfscope}%
\begin{pgfscope}%
\pgfsys@transformshift{1.312638in}{4.139650in}%
\pgfsys@useobject{currentmarker}{}%
\end{pgfscope}%
\begin{pgfscope}%
\pgfsys@transformshift{1.327660in}{4.330767in}%
\pgfsys@useobject{currentmarker}{}%
\end{pgfscope}%
\begin{pgfscope}%
\pgfsys@transformshift{1.346439in}{4.425642in}%
\pgfsys@useobject{currentmarker}{}%
\end{pgfscope}%
\begin{pgfscope}%
\pgfsys@transformshift{1.365686in}{4.359130in}%
\pgfsys@useobject{currentmarker}{}%
\end{pgfscope}%
\begin{pgfscope}%
\pgfsys@transformshift{1.387517in}{4.153048in}%
\pgfsys@useobject{currentmarker}{}%
\end{pgfscope}%
\begin{pgfscope}%
\pgfsys@transformshift{1.407468in}{4.065972in}%
\pgfsys@useobject{currentmarker}{}%
\end{pgfscope}%
\begin{pgfscope}%
\pgfsys@transformshift{1.426246in}{4.041565in}%
\pgfsys@useobject{currentmarker}{}%
\end{pgfscope}%
\begin{pgfscope}%
\pgfsys@transformshift{1.445260in}{4.036073in}%
\pgfsys@useobject{currentmarker}{}%
\end{pgfscope}%
\begin{pgfscope}%
\pgfsys@transformshift{1.465681in}{4.040309in}%
\pgfsys@useobject{currentmarker}{}%
\end{pgfscope}%
\begin{pgfscope}%
\pgfsys@transformshift{1.483050in}{4.058843in}%
\pgfsys@useobject{currentmarker}{}%
\end{pgfscope}%
\begin{pgfscope}%
\pgfsys@transformshift{1.503706in}{4.120550in}%
\pgfsys@useobject{currentmarker}{}%
\end{pgfscope}%
\begin{pgfscope}%
\pgfsys@transformshift{1.521311in}{4.312609in}%
\pgfsys@useobject{currentmarker}{}%
\end{pgfscope}%
\begin{pgfscope}%
\pgfsys@transformshift{1.540793in}{4.422747in}%
\pgfsys@useobject{currentmarker}{}%
\end{pgfscope}%
\begin{pgfscope}%
\pgfsys@transformshift{1.559572in}{4.357981in}%
\pgfsys@useobject{currentmarker}{}%
\end{pgfscope}%
\begin{pgfscope}%
\pgfsys@transformshift{1.579993in}{4.181142in}%
\pgfsys@useobject{currentmarker}{}%
\end{pgfscope}%
\begin{pgfscope}%
\pgfsys@transformshift{1.600181in}{4.081489in}%
\pgfsys@useobject{currentmarker}{}%
\end{pgfscope}%
\begin{pgfscope}%
\pgfsys@transformshift{1.619663in}{4.046116in}%
\pgfsys@useobject{currentmarker}{}%
\end{pgfscope}%
\begin{pgfscope}%
\pgfsys@transformshift{1.634920in}{4.040897in}%
\pgfsys@useobject{currentmarker}{}%
\end{pgfscope}%
\begin{pgfscope}%
\pgfsys@transformshift{1.656046in}{4.035898in}%
\pgfsys@useobject{currentmarker}{}%
\end{pgfscope}%
\begin{pgfscope}%
\pgfsys@transformshift{1.675763in}{4.041117in}%
\pgfsys@useobject{currentmarker}{}%
\end{pgfscope}%
\begin{pgfscope}%
\pgfsys@transformshift{1.694542in}{4.055638in}%
\pgfsys@useobject{currentmarker}{}%
\end{pgfscope}%
\begin{pgfscope}%
\pgfsys@transformshift{1.713790in}{4.087216in}%
\pgfsys@useobject{currentmarker}{}%
\end{pgfscope}%
\begin{pgfscope}%
\pgfsys@transformshift{1.733740in}{4.162829in}%
\pgfsys@useobject{currentmarker}{}%
\end{pgfscope}%
\begin{pgfscope}%
\pgfsys@transformshift{1.751816in}{4.386513in}%
\pgfsys@useobject{currentmarker}{}%
\end{pgfscope}%
\begin{pgfscope}%
\pgfsys@transformshift{1.769655in}{4.415907in}%
\pgfsys@useobject{currentmarker}{}%
\end{pgfscope}%
\begin{pgfscope}%
\pgfsys@transformshift{1.793597in}{4.279890in}%
\pgfsys@useobject{currentmarker}{}%
\end{pgfscope}%
\begin{pgfscope}%
\pgfsys@transformshift{1.810028in}{4.117660in}%
\pgfsys@useobject{currentmarker}{}%
\end{pgfscope}%
\begin{pgfscope}%
\pgfsys@transformshift{1.829981in}{4.065540in}%
\pgfsys@useobject{currentmarker}{}%
\end{pgfscope}%
\begin{pgfscope}%
\pgfsys@transformshift{1.852983in}{4.041071in}%
\pgfsys@useobject{currentmarker}{}%
\end{pgfscope}%
\begin{pgfscope}%
\pgfsys@transformshift{1.868007in}{4.038929in}%
\pgfsys@useobject{currentmarker}{}%
\end{pgfscope}%
\begin{pgfscope}%
\pgfsys@transformshift{1.886315in}{4.035595in}%
\pgfsys@useobject{currentmarker}{}%
\end{pgfscope}%
\begin{pgfscope}%
\pgfsys@transformshift{1.908380in}{4.041478in}%
\pgfsys@useobject{currentmarker}{}%
\end{pgfscope}%
\begin{pgfscope}%
\pgfsys@transformshift{1.927159in}{4.059296in}%
\pgfsys@useobject{currentmarker}{}%
\end{pgfscope}%
\begin{pgfscope}%
\pgfsys@transformshift{1.944529in}{4.083613in}%
\pgfsys@useobject{currentmarker}{}%
\end{pgfscope}%
\begin{pgfscope}%
\pgfsys@transformshift{1.963541in}{4.106343in}%
\pgfsys@useobject{currentmarker}{}%
\end{pgfscope}%
\begin{pgfscope}%
\pgfsys@transformshift{1.981382in}{4.278562in}%
\pgfsys@useobject{currentmarker}{}%
\end{pgfscope}%
\begin{pgfscope}%
\pgfsys@transformshift{2.003446in}{4.415677in}%
\pgfsys@useobject{currentmarker}{}%
\end{pgfscope}%
\begin{pgfscope}%
\pgfsys@transformshift{2.019877in}{4.378784in}%
\pgfsys@useobject{currentmarker}{}%
\end{pgfscope}%
\begin{pgfscope}%
\pgfsys@transformshift{2.041471in}{4.200422in}%
\pgfsys@useobject{currentmarker}{}%
\end{pgfscope}%
\begin{pgfscope}%
\pgfsys@transformshift{2.060250in}{4.080269in}%
\pgfsys@useobject{currentmarker}{}%
\end{pgfscope}%
\begin{pgfscope}%
\pgfsys@transformshift{2.078794in}{4.410049in}%
\pgfsys@useobject{currentmarker}{}%
\end{pgfscope}%
\begin{pgfscope}%
\pgfsys@transformshift{2.097807in}{4.289223in}%
\pgfsys@useobject{currentmarker}{}%
\end{pgfscope}%
\begin{pgfscope}%
\pgfsys@transformshift{2.118698in}{4.113949in}%
\pgfsys@useobject{currentmarker}{}%
\end{pgfscope}%
\begin{pgfscope}%
\pgfsys@transformshift{2.133720in}{4.061423in}%
\pgfsys@useobject{currentmarker}{}%
\end{pgfscope}%
\begin{pgfscope}%
\pgfsys@transformshift{2.155316in}{4.040321in}%
\pgfsys@useobject{currentmarker}{}%
\end{pgfscope}%
\begin{pgfscope}%
\pgfsys@transformshift{2.173390in}{4.037285in}%
\pgfsys@useobject{currentmarker}{}%
\end{pgfscope}%
\begin{pgfscope}%
\pgfsys@transformshift{2.192403in}{4.036168in}%
\pgfsys@useobject{currentmarker}{}%
\end{pgfscope}%
\begin{pgfscope}%
\pgfsys@transformshift{2.216110in}{4.047956in}%
\pgfsys@useobject{currentmarker}{}%
\end{pgfscope}%
\begin{pgfscope}%
\pgfsys@transformshift{2.230898in}{4.071836in}%
\pgfsys@useobject{currentmarker}{}%
\end{pgfscope}%
\begin{pgfscope}%
\pgfsys@transformshift{2.252023in}{4.183569in}%
\pgfsys@useobject{currentmarker}{}%
\end{pgfscope}%
\begin{pgfscope}%
\pgfsys@transformshift{2.270333in}{4.367271in}%
\pgfsys@useobject{currentmarker}{}%
\end{pgfscope}%
\begin{pgfscope}%
\pgfsys@transformshift{2.291458in}{4.417560in}%
\pgfsys@useobject{currentmarker}{}%
\end{pgfscope}%
\begin{pgfscope}%
\pgfsys@transformshift{2.309063in}{4.361518in}%
\pgfsys@useobject{currentmarker}{}%
\end{pgfscope}%
\begin{pgfscope}%
\pgfsys@transformshift{2.326199in}{4.197001in}%
\pgfsys@useobject{currentmarker}{}%
\end{pgfscope}%
\begin{pgfscope}%
\pgfsys@transformshift{2.347324in}{4.074181in}%
\pgfsys@useobject{currentmarker}{}%
\end{pgfscope}%
\begin{pgfscope}%
\pgfsys@transformshift{2.363989in}{4.043777in}%
\pgfsys@useobject{currentmarker}{}%
\end{pgfscope}%
\begin{pgfscope}%
\pgfsys@transformshift{2.388636in}{4.036067in}%
\pgfsys@useobject{currentmarker}{}%
\end{pgfscope}%
\begin{pgfscope}%
\pgfsys@transformshift{2.405772in}{4.036662in}%
\pgfsys@useobject{currentmarker}{}%
\end{pgfscope}%
\begin{pgfscope}%
\pgfsys@transformshift{2.424549in}{4.043796in}%
\pgfsys@useobject{currentmarker}{}%
\end{pgfscope}%
\begin{pgfscope}%
\pgfsys@transformshift{2.441216in}{4.066582in}%
\pgfsys@useobject{currentmarker}{}%
\end{pgfscope}%
\begin{pgfscope}%
\pgfsys@transformshift{2.463984in}{4.125433in}%
\pgfsys@useobject{currentmarker}{}%
\end{pgfscope}%
\begin{pgfscope}%
\pgfsys@transformshift{2.480415in}{4.270451in}%
\pgfsys@useobject{currentmarker}{}%
\end{pgfscope}%
\begin{pgfscope}%
\pgfsys@transformshift{2.500602in}{4.409658in}%
\pgfsys@useobject{currentmarker}{}%
\end{pgfscope}%
\begin{pgfscope}%
\pgfsys@transformshift{2.519850in}{4.377409in}%
\pgfsys@useobject{currentmarker}{}%
\end{pgfscope}%
\begin{pgfscope}%
\pgfsys@transformshift{2.540037in}{4.200252in}%
\pgfsys@useobject{currentmarker}{}%
\end{pgfscope}%
\begin{pgfscope}%
\pgfsys@transformshift{2.558816in}{4.094385in}%
\pgfsys@useobject{currentmarker}{}%
\end{pgfscope}%
\begin{pgfscope}%
\pgfsys@transformshift{2.579472in}{4.047446in}%
\pgfsys@useobject{currentmarker}{}%
\end{pgfscope}%
\begin{pgfscope}%
\pgfsys@transformshift{2.597780in}{4.039029in}%
\pgfsys@useobject{currentmarker}{}%
\end{pgfscope}%
\begin{pgfscope}%
\pgfsys@transformshift{2.616088in}{4.035501in}%
\pgfsys@useobject{currentmarker}{}%
\end{pgfscope}%
\begin{pgfscope}%
\pgfsys@transformshift{2.635572in}{4.036773in}%
\pgfsys@useobject{currentmarker}{}%
\end{pgfscope}%
\begin{pgfscope}%
\pgfsys@transformshift{2.657401in}{4.049650in}%
\pgfsys@useobject{currentmarker}{}%
\end{pgfscope}%
\begin{pgfscope}%
\pgfsys@transformshift{2.673597in}{4.066689in}%
\pgfsys@useobject{currentmarker}{}%
\end{pgfscope}%
\begin{pgfscope}%
\pgfsys@transformshift{2.693550in}{4.157115in}%
\pgfsys@useobject{currentmarker}{}%
\end{pgfscope}%
\begin{pgfscope}%
\pgfsys@transformshift{2.713503in}{4.337437in}%
\pgfsys@useobject{currentmarker}{}%
\end{pgfscope}%
\begin{pgfscope}%
\pgfsys@transformshift{2.732045in}{4.412149in}%
\pgfsys@useobject{currentmarker}{}%
\end{pgfscope}%
\begin{pgfscope}%
\pgfsys@transformshift{2.750355in}{4.362200in}%
\pgfsys@useobject{currentmarker}{}%
\end{pgfscope}%
\begin{pgfscope}%
\pgfsys@transformshift{2.769837in}{4.303548in}%
\pgfsys@useobject{currentmarker}{}%
\end{pgfscope}%
\begin{pgfscope}%
\pgfsys@transformshift{2.789788in}{4.141583in}%
\pgfsys@useobject{currentmarker}{}%
\end{pgfscope}%
\begin{pgfscope}%
\pgfsys@transformshift{2.813732in}{4.053203in}%
\pgfsys@useobject{currentmarker}{}%
\end{pgfscope}%
\begin{pgfscope}%
\pgfsys@transformshift{2.828754in}{4.040726in}%
\pgfsys@useobject{currentmarker}{}%
\end{pgfscope}%
\begin{pgfscope}%
\pgfsys@transformshift{2.847768in}{4.036082in}%
\pgfsys@useobject{currentmarker}{}%
\end{pgfscope}%
\begin{pgfscope}%
\pgfsys@transformshift{2.865841in}{4.050336in}%
\pgfsys@useobject{currentmarker}{}%
\end{pgfscope}%
\begin{pgfscope}%
\pgfsys@transformshift{2.887201in}{4.038002in}%
\pgfsys@useobject{currentmarker}{}%
\end{pgfscope}%
\begin{pgfscope}%
\pgfsys@transformshift{2.906685in}{4.035653in}%
\pgfsys@useobject{currentmarker}{}%
\end{pgfscope}%
\begin{pgfscope}%
\pgfsys@transformshift{2.924524in}{4.039397in}%
\pgfsys@useobject{currentmarker}{}%
\end{pgfscope}%
\begin{pgfscope}%
\pgfsys@transformshift{2.941894in}{4.052698in}%
\pgfsys@useobject{currentmarker}{}%
\end{pgfscope}%
\begin{pgfscope}%
\pgfsys@transformshift{2.963254in}{4.098739in}%
\pgfsys@useobject{currentmarker}{}%
\end{pgfscope}%
\begin{pgfscope}%
\pgfsys@transformshift{2.984615in}{4.207865in}%
\pgfsys@useobject{currentmarker}{}%
\end{pgfscope}%
\begin{pgfscope}%
\pgfsys@transformshift{3.001983in}{4.391115in}%
\pgfsys@useobject{currentmarker}{}%
\end{pgfscope}%
\begin{pgfscope}%
\pgfsys@transformshift{3.019588in}{4.404598in}%
\pgfsys@useobject{currentmarker}{}%
\end{pgfscope}%
\begin{pgfscope}%
\pgfsys@transformshift{3.037898in}{4.289522in}%
\pgfsys@useobject{currentmarker}{}%
\end{pgfscope}%
\begin{pgfscope}%
\pgfsys@transformshift{3.059728in}{4.118935in}%
\pgfsys@useobject{currentmarker}{}%
\end{pgfscope}%
\begin{pgfscope}%
\pgfsys@transformshift{3.077333in}{4.074447in}%
\pgfsys@useobject{currentmarker}{}%
\end{pgfscope}%
\begin{pgfscope}%
\pgfsys@transformshift{3.095641in}{4.046784in}%
\pgfsys@useobject{currentmarker}{}%
\end{pgfscope}%
\begin{pgfscope}%
\pgfsys@transformshift{3.116063in}{4.189053in}%
\pgfsys@useobject{currentmarker}{}%
\end{pgfscope}%
\begin{pgfscope}%
\pgfsys@transformshift{3.134371in}{4.084917in}%
\pgfsys@useobject{currentmarker}{}%
\end{pgfscope}%
\begin{pgfscope}%
\pgfsys@transformshift{3.155967in}{4.045699in}%
\pgfsys@useobject{currentmarker}{}%
\end{pgfscope}%
\begin{pgfscope}%
\pgfsys@transformshift{3.174746in}{4.036752in}%
\pgfsys@useobject{currentmarker}{}%
\end{pgfscope}%
\begin{pgfscope}%
\pgfsys@transformshift{3.194228in}{4.036572in}%
\pgfsys@useobject{currentmarker}{}%
\end{pgfscope}%
\begin{pgfscope}%
\pgfsys@transformshift{3.211362in}{4.039358in}%
\pgfsys@useobject{currentmarker}{}%
\end{pgfscope}%
\begin{pgfscope}%
\pgfsys@transformshift{3.232489in}{4.057109in}%
\pgfsys@useobject{currentmarker}{}%
\end{pgfscope}%
\begin{pgfscope}%
\pgfsys@transformshift{3.251736in}{4.102632in}%
\pgfsys@useobject{currentmarker}{}%
\end{pgfscope}%
\begin{pgfscope}%
\pgfsys@transformshift{3.268402in}{4.247203in}%
\pgfsys@useobject{currentmarker}{}%
\end{pgfscope}%
\begin{pgfscope}%
\pgfsys@transformshift{3.289528in}{4.404155in}%
\pgfsys@useobject{currentmarker}{}%
\end{pgfscope}%
\begin{pgfscope}%
\pgfsys@transformshift{3.308305in}{4.406066in}%
\pgfsys@useobject{currentmarker}{}%
\end{pgfscope}%
\begin{pgfscope}%
\pgfsys@transformshift{3.325676in}{4.286565in}%
\pgfsys@useobject{currentmarker}{}%
\end{pgfscope}%
\begin{pgfscope}%
\pgfsys@transformshift{3.346566in}{4.124482in}%
\pgfsys@useobject{currentmarker}{}%
\end{pgfscope}%
\begin{pgfscope}%
\pgfsys@transformshift{3.364642in}{4.070513in}%
\pgfsys@useobject{currentmarker}{}%
\end{pgfscope}%
\begin{pgfscope}%
\pgfsys@transformshift{3.383653in}{4.044584in}%
\pgfsys@useobject{currentmarker}{}%
\end{pgfscope}%
\begin{pgfscope}%
\pgfsys@transformshift{3.403372in}{4.038152in}%
\pgfsys@useobject{currentmarker}{}%
\end{pgfscope}%
\begin{pgfscope}%
\pgfsys@transformshift{3.424966in}{4.036098in}%
\pgfsys@useobject{currentmarker}{}%
\end{pgfscope}%
\begin{pgfscope}%
\pgfsys@transformshift{3.443276in}{4.040569in}%
\pgfsys@useobject{currentmarker}{}%
\end{pgfscope}%
\begin{pgfscope}%
\pgfsys@transformshift{3.461818in}{4.054285in}%
\pgfsys@useobject{currentmarker}{}%
\end{pgfscope}%
\begin{pgfscope}%
\pgfsys@transformshift{3.480597in}{4.086567in}%
\pgfsys@useobject{currentmarker}{}%
\end{pgfscope}%
\begin{pgfscope}%
\pgfsys@transformshift{3.501958in}{4.218000in}%
\pgfsys@useobject{currentmarker}{}%
\end{pgfscope}%
\begin{pgfscope}%
\pgfsys@transformshift{3.518155in}{4.388392in}%
\pgfsys@useobject{currentmarker}{}%
\end{pgfscope}%
\begin{pgfscope}%
\pgfsys@transformshift{3.539280in}{4.422388in}%
\pgfsys@useobject{currentmarker}{}%
\end{pgfscope}%
\begin{pgfscope}%
\pgfsys@transformshift{3.558293in}{4.335721in}%
\pgfsys@useobject{currentmarker}{}%
\end{pgfscope}%
\begin{pgfscope}%
\pgfsys@transformshift{3.578009in}{4.178177in}%
\pgfsys@useobject{currentmarker}{}%
\end{pgfscope}%
\begin{pgfscope}%
\pgfsys@transformshift{3.596319in}{4.085623in}%
\pgfsys@useobject{currentmarker}{}%
\end{pgfscope}%
\begin{pgfscope}%
\pgfsys@transformshift{3.614159in}{4.058965in}%
\pgfsys@useobject{currentmarker}{}%
\end{pgfscope}%
\begin{pgfscope}%
\pgfsys@transformshift{3.635049in}{4.042116in}%
\pgfsys@useobject{currentmarker}{}%
\end{pgfscope}%
\begin{pgfscope}%
\pgfsys@transformshift{3.653123in}{4.037451in}%
\pgfsys@useobject{currentmarker}{}%
\end{pgfscope}%
\begin{pgfscope}%
\pgfsys@transformshift{3.674953in}{4.037429in}%
\pgfsys@useobject{currentmarker}{}%
\end{pgfscope}%
\begin{pgfscope}%
\pgfsys@transformshift{3.690446in}{4.044182in}%
\pgfsys@useobject{currentmarker}{}%
\end{pgfscope}%
\begin{pgfscope}%
\pgfsys@transformshift{3.713214in}{4.069119in}%
\pgfsys@useobject{currentmarker}{}%
\end{pgfscope}%
\begin{pgfscope}%
\pgfsys@transformshift{3.729645in}{4.108206in}%
\pgfsys@useobject{currentmarker}{}%
\end{pgfscope}%
\begin{pgfscope}%
\pgfsys@transformshift{3.749832in}{4.202575in}%
\pgfsys@useobject{currentmarker}{}%
\end{pgfscope}%
\begin{pgfscope}%
\pgfsys@transformshift{3.768140in}{4.396994in}%
\pgfsys@useobject{currentmarker}{}%
\end{pgfscope}%
\begin{pgfscope}%
\pgfsys@transformshift{3.789736in}{4.434267in}%
\pgfsys@useobject{currentmarker}{}%
\end{pgfscope}%
\begin{pgfscope}%
\pgfsys@transformshift{3.809923in}{4.410232in}%
\pgfsys@useobject{currentmarker}{}%
\end{pgfscope}%
\begin{pgfscope}%
\pgfsys@transformshift{3.825649in}{4.307411in}%
\pgfsys@useobject{currentmarker}{}%
\end{pgfscope}%
\begin{pgfscope}%
\pgfsys@transformshift{3.847950in}{4.152963in}%
\pgfsys@useobject{currentmarker}{}%
\end{pgfscope}%
\begin{pgfscope}%
\pgfsys@transformshift{3.865084in}{4.107589in}%
\pgfsys@useobject{currentmarker}{}%
\end{pgfscope}%
\begin{pgfscope}%
\pgfsys@transformshift{3.883862in}{4.067237in}%
\pgfsys@useobject{currentmarker}{}%
\end{pgfscope}%
\begin{pgfscope}%
\pgfsys@transformshift{3.904284in}{4.043375in}%
\pgfsys@useobject{currentmarker}{}%
\end{pgfscope}%
\begin{pgfscope}%
\pgfsys@transformshift{3.922123in}{4.037387in}%
\pgfsys@useobject{currentmarker}{}%
\end{pgfscope}%
\begin{pgfscope}%
\pgfsys@transformshift{3.942545in}{4.038024in}%
\pgfsys@useobject{currentmarker}{}%
\end{pgfscope}%
\begin{pgfscope}%
\pgfsys@transformshift{3.961324in}{4.044992in}%
\pgfsys@useobject{currentmarker}{}%
\end{pgfscope}%
\begin{pgfscope}%
\pgfsys@transformshift{3.979866in}{4.063126in}%
\pgfsys@useobject{currentmarker}{}%
\end{pgfscope}%
\begin{pgfscope}%
\pgfsys@transformshift{4.002400in}{4.037024in}%
\pgfsys@useobject{currentmarker}{}%
\end{pgfscope}%
\begin{pgfscope}%
\pgfsys@transformshift{4.017424in}{4.041898in}%
\pgfsys@useobject{currentmarker}{}%
\end{pgfscope}%
\begin{pgfscope}%
\pgfsys@transformshift{4.039489in}{4.053039in}%
\pgfsys@useobject{currentmarker}{}%
\end{pgfscope}%
\begin{pgfscope}%
\pgfsys@transformshift{4.058502in}{4.091213in}%
\pgfsys@useobject{currentmarker}{}%
\end{pgfscope}%
\begin{pgfscope}%
\pgfsys@transformshift{4.076810in}{4.186562in}%
\pgfsys@useobject{currentmarker}{}%
\end{pgfscope}%
\begin{pgfscope}%
\pgfsys@transformshift{4.095118in}{4.403557in}%
\pgfsys@useobject{currentmarker}{}%
\end{pgfscope}%
\begin{pgfscope}%
\pgfsys@transformshift{4.114602in}{4.447336in}%
\pgfsys@useobject{currentmarker}{}%
\end{pgfscope}%
\begin{pgfscope}%
\pgfsys@transformshift{4.135493in}{4.398784in}%
\pgfsys@useobject{currentmarker}{}%
\end{pgfscope}%
\begin{pgfscope}%
\pgfsys@transformshift{4.153097in}{4.279556in}%
\pgfsys@useobject{currentmarker}{}%
\end{pgfscope}%
\begin{pgfscope}%
\pgfsys@transformshift{4.171406in}{4.146322in}%
\pgfsys@useobject{currentmarker}{}%
\end{pgfscope}%
\begin{pgfscope}%
\pgfsys@transformshift{4.192062in}{4.078125in}%
\pgfsys@useobject{currentmarker}{}%
\end{pgfscope}%
\begin{pgfscope}%
\pgfsys@transformshift{4.210372in}{4.050823in}%
\pgfsys@useobject{currentmarker}{}%
\end{pgfscope}%
\begin{pgfscope}%
\pgfsys@transformshift{4.231965in}{4.040901in}%
\pgfsys@useobject{currentmarker}{}%
\end{pgfscope}%
\begin{pgfscope}%
\pgfsys@transformshift{4.250510in}{4.037554in}%
\pgfsys@useobject{currentmarker}{}%
\end{pgfscope}%
\begin{pgfscope}%
\pgfsys@transformshift{4.270932in}{4.042972in}%
\pgfsys@useobject{currentmarker}{}%
\end{pgfscope}%
\begin{pgfscope}%
\pgfsys@transformshift{4.289005in}{4.060032in}%
\pgfsys@useobject{currentmarker}{}%
\end{pgfscope}%
\begin{pgfscope}%
\pgfsys@transformshift{4.307550in}{4.085237in}%
\pgfsys@useobject{currentmarker}{}%
\end{pgfscope}%
\begin{pgfscope}%
\pgfsys@transformshift{4.327501in}{4.187863in}%
\pgfsys@useobject{currentmarker}{}%
\end{pgfscope}%
\begin{pgfscope}%
\pgfsys@transformshift{4.343228in}{4.313889in}%
\pgfsys@useobject{currentmarker}{}%
\end{pgfscope}%
\begin{pgfscope}%
\pgfsys@transformshift{4.364588in}{4.446704in}%
\pgfsys@useobject{currentmarker}{}%
\end{pgfscope}%
\begin{pgfscope}%
\pgfsys@transformshift{4.382427in}{4.451753in}%
\pgfsys@useobject{currentmarker}{}%
\end{pgfscope}%
\begin{pgfscope}%
\pgfsys@transformshift{4.404257in}{4.371033in}%
\pgfsys@useobject{currentmarker}{}%
\end{pgfscope}%
\begin{pgfscope}%
\pgfsys@transformshift{4.423270in}{4.238227in}%
\pgfsys@useobject{currentmarker}{}%
\end{pgfscope}%
\begin{pgfscope}%
\pgfsys@transformshift{4.441815in}{4.109622in}%
\pgfsys@useobject{currentmarker}{}%
\end{pgfscope}%
\begin{pgfscope}%
\pgfsys@transformshift{4.460828in}{4.069411in}%
\pgfsys@useobject{currentmarker}{}%
\end{pgfscope}%
\begin{pgfscope}%
\pgfsys@transformshift{4.479370in}{4.049000in}%
\pgfsys@useobject{currentmarker}{}%
\end{pgfscope}%
\begin{pgfscope}%
\pgfsys@transformshift{4.480544in}{4.047355in}%
\pgfsys@useobject{currentmarker}{}%
\end{pgfscope}%
\begin{pgfscope}%
\pgfsys@transformshift{4.473973in}{4.061407in}%
\pgfsys@useobject{currentmarker}{}%
\end{pgfscope}%
\begin{pgfscope}%
\pgfsys@transformshift{4.453551in}{4.129151in}%
\pgfsys@useobject{currentmarker}{}%
\end{pgfscope}%
\begin{pgfscope}%
\pgfsys@transformshift{4.436181in}{4.307252in}%
\pgfsys@useobject{currentmarker}{}%
\end{pgfscope}%
\begin{pgfscope}%
\pgfsys@transformshift{4.418107in}{4.440826in}%
\pgfsys@useobject{currentmarker}{}%
\end{pgfscope}%
\begin{pgfscope}%
\pgfsys@transformshift{4.395337in}{4.432275in}%
\pgfsys@useobject{currentmarker}{}%
\end{pgfscope}%
\begin{pgfscope}%
\pgfsys@transformshift{4.379610in}{4.228270in}%
\pgfsys@useobject{currentmarker}{}%
\end{pgfscope}%
\begin{pgfscope}%
\pgfsys@transformshift{4.358016in}{4.085886in}%
\pgfsys@useobject{currentmarker}{}%
\end{pgfscope}%
\begin{pgfscope}%
\pgfsys@transformshift{4.340177in}{4.046505in}%
\pgfsys@useobject{currentmarker}{}%
\end{pgfscope}%
\begin{pgfscope}%
\pgfsys@transformshift{4.322101in}{4.037669in}%
\pgfsys@useobject{currentmarker}{}%
\end{pgfscope}%
\begin{pgfscope}%
\pgfsys@transformshift{4.300037in}{4.045134in}%
\pgfsys@useobject{currentmarker}{}%
\end{pgfscope}%
\begin{pgfscope}%
\pgfsys@transformshift{4.281729in}{4.073545in}%
\pgfsys@useobject{currentmarker}{}%
\end{pgfscope}%
\begin{pgfscope}%
\pgfsys@transformshift{4.264358in}{4.181880in}%
\pgfsys@useobject{currentmarker}{}%
\end{pgfscope}%
\begin{pgfscope}%
\pgfsys@transformshift{4.242059in}{4.394010in}%
\pgfsys@useobject{currentmarker}{}%
\end{pgfscope}%
\begin{pgfscope}%
\pgfsys@transformshift{4.223517in}{4.447858in}%
\pgfsys@useobject{currentmarker}{}%
\end{pgfscope}%
\begin{pgfscope}%
\pgfsys@transformshift{4.205675in}{4.350664in}%
\pgfsys@useobject{currentmarker}{}%
\end{pgfscope}%
\begin{pgfscope}%
\pgfsys@transformshift{4.186428in}{4.123976in}%
\pgfsys@useobject{currentmarker}{}%
\end{pgfscope}%
\begin{pgfscope}%
\pgfsys@transformshift{4.166008in}{4.057099in}%
\pgfsys@useobject{currentmarker}{}%
\end{pgfscope}%
\begin{pgfscope}%
\pgfsys@transformshift{4.147229in}{4.039503in}%
\pgfsys@useobject{currentmarker}{}%
\end{pgfscope}%
\begin{pgfscope}%
\pgfsys@transformshift{4.129154in}{4.037878in}%
\pgfsys@useobject{currentmarker}{}%
\end{pgfscope}%
\begin{pgfscope}%
\pgfsys@transformshift{4.108734in}{4.050067in}%
\pgfsys@useobject{currentmarker}{}%
\end{pgfscope}%
\begin{pgfscope}%
\pgfsys@transformshift{4.087607in}{4.107849in}%
\pgfsys@useobject{currentmarker}{}%
\end{pgfscope}%
\begin{pgfscope}%
\pgfsys@transformshift{4.069768in}{4.265099in}%
\pgfsys@useobject{currentmarker}{}%
\end{pgfscope}%
\begin{pgfscope}%
\pgfsys@transformshift{4.054277in}{4.427694in}%
\pgfsys@useobject{currentmarker}{}%
\end{pgfscope}%
\begin{pgfscope}%
\pgfsys@transformshift{4.033620in}{4.429469in}%
\pgfsys@useobject{currentmarker}{}%
\end{pgfscope}%
\begin{pgfscope}%
\pgfsys@transformshift{4.011790in}{4.192512in}%
\pgfsys@useobject{currentmarker}{}%
\end{pgfscope}%
\begin{pgfscope}%
\pgfsys@transformshift{3.994420in}{4.081117in}%
\pgfsys@useobject{currentmarker}{}%
\end{pgfscope}%
\begin{pgfscope}%
\pgfsys@transformshift{3.974467in}{4.044335in}%
\pgfsys@useobject{currentmarker}{}%
\end{pgfscope}%
\begin{pgfscope}%
\pgfsys@transformshift{3.957333in}{4.037021in}%
\pgfsys@useobject{currentmarker}{}%
\end{pgfscope}%
\begin{pgfscope}%
\pgfsys@transformshift{3.936442in}{4.040142in}%
\pgfsys@useobject{currentmarker}{}%
\end{pgfscope}%
\begin{pgfscope}%
\pgfsys@transformshift{3.915315in}{4.061483in}%
\pgfsys@useobject{currentmarker}{}%
\end{pgfscope}%
\begin{pgfscope}%
\pgfsys@transformshift{3.899119in}{4.084423in}%
\pgfsys@useobject{currentmarker}{}%
\end{pgfscope}%
\begin{pgfscope}%
\pgfsys@transformshift{3.878934in}{4.241646in}%
\pgfsys@useobject{currentmarker}{}%
\end{pgfscope}%
\begin{pgfscope}%
\pgfsys@transformshift{3.862501in}{4.393330in}%
\pgfsys@useobject{currentmarker}{}%
\end{pgfscope}%
\begin{pgfscope}%
\pgfsys@transformshift{3.839264in}{4.423954in}%
\pgfsys@useobject{currentmarker}{}%
\end{pgfscope}%
\begin{pgfscope}%
\pgfsys@transformshift{3.823537in}{4.254001in}%
\pgfsys@useobject{currentmarker}{}%
\end{pgfscope}%
\begin{pgfscope}%
\pgfsys@transformshift{3.802412in}{4.087454in}%
\pgfsys@useobject{currentmarker}{}%
\end{pgfscope}%
\begin{pgfscope}%
\pgfsys@transformshift{3.781990in}{4.047415in}%
\pgfsys@useobject{currentmarker}{}%
\end{pgfscope}%
\begin{pgfscope}%
\pgfsys@transformshift{3.765089in}{4.037347in}%
\pgfsys@useobject{currentmarker}{}%
\end{pgfscope}%
\begin{pgfscope}%
\pgfsys@transformshift{3.744432in}{4.037765in}%
\pgfsys@useobject{currentmarker}{}%
\end{pgfscope}%
\begin{pgfscope}%
\pgfsys@transformshift{3.724247in}{4.049684in}%
\pgfsys@useobject{currentmarker}{}%
\end{pgfscope}%
\begin{pgfscope}%
\pgfsys@transformshift{3.704060in}{4.093940in}%
\pgfsys@useobject{currentmarker}{}%
\end{pgfscope}%
\begin{pgfscope}%
\pgfsys@transformshift{3.686924in}{4.224782in}%
\pgfsys@useobject{currentmarker}{}%
\end{pgfscope}%
\begin{pgfscope}%
\pgfsys@transformshift{3.666973in}{4.395649in}%
\pgfsys@useobject{currentmarker}{}%
\end{pgfscope}%
\begin{pgfscope}%
\pgfsys@transformshift{3.646551in}{4.416598in}%
\pgfsys@useobject{currentmarker}{}%
\end{pgfscope}%
\begin{pgfscope}%
\pgfsys@transformshift{3.628712in}{4.274832in}%
\pgfsys@useobject{currentmarker}{}%
\end{pgfscope}%
\begin{pgfscope}%
\pgfsys@transformshift{3.611576in}{4.121843in}%
\pgfsys@useobject{currentmarker}{}%
\end{pgfscope}%
\begin{pgfscope}%
\pgfsys@transformshift{3.589511in}{4.051996in}%
\pgfsys@useobject{currentmarker}{}%
\end{pgfscope}%
\begin{pgfscope}%
\pgfsys@transformshift{3.571672in}{4.040212in}%
\pgfsys@useobject{currentmarker}{}%
\end{pgfscope}%
\begin{pgfscope}%
\pgfsys@transformshift{3.551721in}{4.035812in}%
\pgfsys@useobject{currentmarker}{}%
\end{pgfscope}%
\begin{pgfscope}%
\pgfsys@transformshift{3.533411in}{4.040778in}%
\pgfsys@useobject{currentmarker}{}%
\end{pgfscope}%
\begin{pgfscope}%
\pgfsys@transformshift{3.512990in}{4.065574in}%
\pgfsys@useobject{currentmarker}{}%
\end{pgfscope}%
\begin{pgfscope}%
\pgfsys@transformshift{3.494681in}{4.123053in}%
\pgfsys@useobject{currentmarker}{}%
\end{pgfscope}%
\begin{pgfscope}%
\pgfsys@transformshift{3.474963in}{4.267916in}%
\pgfsys@useobject{currentmarker}{}%
\end{pgfscope}%
\begin{pgfscope}%
\pgfsys@transformshift{3.457358in}{4.355614in}%
\pgfsys@useobject{currentmarker}{}%
\end{pgfscope}%
\begin{pgfscope}%
\pgfsys@transformshift{3.436233in}{4.418938in}%
\pgfsys@useobject{currentmarker}{}%
\end{pgfscope}%
\begin{pgfscope}%
\pgfsys@transformshift{3.414637in}{4.249261in}%
\pgfsys@useobject{currentmarker}{}%
\end{pgfscope}%
\begin{pgfscope}%
\pgfsys@transformshift{3.397503in}{4.098506in}%
\pgfsys@useobject{currentmarker}{}%
\end{pgfscope}%
\begin{pgfscope}%
\pgfsys@transformshift{3.379193in}{4.061926in}%
\pgfsys@useobject{currentmarker}{}%
\end{pgfscope}%
\begin{pgfscope}%
\pgfsys@transformshift{3.357834in}{4.070041in}%
\pgfsys@useobject{currentmarker}{}%
\end{pgfscope}%
\begin{pgfscope}%
\pgfsys@transformshift{3.338586in}{4.044506in}%
\pgfsys@useobject{currentmarker}{}%
\end{pgfscope}%
\begin{pgfscope}%
\pgfsys@transformshift{3.321216in}{4.036476in}%
\pgfsys@useobject{currentmarker}{}%
\end{pgfscope}%
\begin{pgfscope}%
\pgfsys@transformshift{3.302437in}{4.037639in}%
\pgfsys@useobject{currentmarker}{}%
\end{pgfscope}%
\begin{pgfscope}%
\pgfsys@transformshift{3.281078in}{4.050946in}%
\pgfsys@useobject{currentmarker}{}%
\end{pgfscope}%
\begin{pgfscope}%
\pgfsys@transformshift{3.263707in}{4.091809in}%
\pgfsys@useobject{currentmarker}{}%
\end{pgfscope}%
\begin{pgfscope}%
\pgfsys@transformshift{3.243754in}{4.190328in}%
\pgfsys@useobject{currentmarker}{}%
\end{pgfscope}%
\begin{pgfscope}%
\pgfsys@transformshift{3.228498in}{4.352588in}%
\pgfsys@useobject{currentmarker}{}%
\end{pgfscope}%
\begin{pgfscope}%
\pgfsys@transformshift{3.207842in}{4.419421in}%
\pgfsys@useobject{currentmarker}{}%
\end{pgfscope}%
\begin{pgfscope}%
\pgfsys@transformshift{3.187654in}{4.377732in}%
\pgfsys@useobject{currentmarker}{}%
\end{pgfscope}%
\begin{pgfscope}%
\pgfsys@transformshift{3.167235in}{4.138366in}%
\pgfsys@useobject{currentmarker}{}%
\end{pgfscope}%
\begin{pgfscope}%
\pgfsys@transformshift{3.148925in}{4.065828in}%
\pgfsys@useobject{currentmarker}{}%
\end{pgfscope}%
\begin{pgfscope}%
\pgfsys@transformshift{3.127565in}{4.042750in}%
\pgfsys@useobject{currentmarker}{}%
\end{pgfscope}%
\begin{pgfscope}%
\pgfsys@transformshift{3.110664in}{4.035917in}%
\pgfsys@useobject{currentmarker}{}%
\end{pgfscope}%
\begin{pgfscope}%
\pgfsys@transformshift{3.091181in}{4.036610in}%
\pgfsys@useobject{currentmarker}{}%
\end{pgfscope}%
\begin{pgfscope}%
\pgfsys@transformshift{3.069586in}{4.045159in}%
\pgfsys@useobject{currentmarker}{}%
\end{pgfscope}%
\begin{pgfscope}%
\pgfsys@transformshift{3.051981in}{4.071713in}%
\pgfsys@useobject{currentmarker}{}%
\end{pgfscope}%
\begin{pgfscope}%
\pgfsys@transformshift{3.030622in}{4.190887in}%
\pgfsys@useobject{currentmarker}{}%
\end{pgfscope}%
\begin{pgfscope}%
\pgfsys@transformshift{3.015363in}{4.314805in}%
\pgfsys@useobject{currentmarker}{}%
\end{pgfscope}%
\begin{pgfscope}%
\pgfsys@transformshift{2.995412in}{4.414271in}%
\pgfsys@useobject{currentmarker}{}%
\end{pgfscope}%
\begin{pgfscope}%
\pgfsys@transformshift{2.973816in}{4.295915in}%
\pgfsys@useobject{currentmarker}{}%
\end{pgfscope}%
\begin{pgfscope}%
\pgfsys@transformshift{2.956211in}{4.132323in}%
\pgfsys@useobject{currentmarker}{}%
\end{pgfscope}%
\begin{pgfscope}%
\pgfsys@transformshift{2.938138in}{4.099042in}%
\pgfsys@useobject{currentmarker}{}%
\end{pgfscope}%
\begin{pgfscope}%
\pgfsys@transformshift{2.916308in}{4.050847in}%
\pgfsys@useobject{currentmarker}{}%
\end{pgfscope}%
\begin{pgfscope}%
\pgfsys@transformshift{2.897529in}{4.039009in}%
\pgfsys@useobject{currentmarker}{}%
\end{pgfscope}%
\begin{pgfscope}%
\pgfsys@transformshift{2.878281in}{4.035402in}%
\pgfsys@useobject{currentmarker}{}%
\end{pgfscope}%
\begin{pgfscope}%
\pgfsys@transformshift{2.859033in}{4.038282in}%
\pgfsys@useobject{currentmarker}{}%
\end{pgfscope}%
\begin{pgfscope}%
\pgfsys@transformshift{2.840020in}{4.043410in}%
\pgfsys@useobject{currentmarker}{}%
\end{pgfscope}%
\begin{pgfscope}%
\pgfsys@transformshift{2.821478in}{4.068638in}%
\pgfsys@useobject{currentmarker}{}%
\end{pgfscope}%
\begin{pgfscope}%
\pgfsys@transformshift{2.802699in}{4.159507in}%
\pgfsys@useobject{currentmarker}{}%
\end{pgfscope}%
\begin{pgfscope}%
\pgfsys@transformshift{2.782043in}{4.345061in}%
\pgfsys@useobject{currentmarker}{}%
\end{pgfscope}%
\begin{pgfscope}%
\pgfsys@transformshift{2.765143in}{4.410868in}%
\pgfsys@useobject{currentmarker}{}%
\end{pgfscope}%
\begin{pgfscope}%
\pgfsys@transformshift{2.743782in}{4.369644in}%
\pgfsys@useobject{currentmarker}{}%
\end{pgfscope}%
\begin{pgfscope}%
\pgfsys@transformshift{2.725708in}{4.165377in}%
\pgfsys@useobject{currentmarker}{}%
\end{pgfscope}%
\begin{pgfscope}%
\pgfsys@transformshift{2.706226in}{4.076846in}%
\pgfsys@useobject{currentmarker}{}%
\end{pgfscope}%
\begin{pgfscope}%
\pgfsys@transformshift{2.688621in}{4.189357in}%
\pgfsys@useobject{currentmarker}{}%
\end{pgfscope}%
\begin{pgfscope}%
\pgfsys@transformshift{2.668668in}{4.359012in}%
\pgfsys@useobject{currentmarker}{}%
\end{pgfscope}%
\begin{pgfscope}%
\pgfsys@transformshift{2.649889in}{4.418341in}%
\pgfsys@useobject{currentmarker}{}%
\end{pgfscope}%
\begin{pgfscope}%
\pgfsys@transformshift{2.625244in}{4.254190in}%
\pgfsys@useobject{currentmarker}{}%
\end{pgfscope}%
\begin{pgfscope}%
\pgfsys@transformshift{2.609986in}{4.160968in}%
\pgfsys@useobject{currentmarker}{}%
\end{pgfscope}%
\begin{pgfscope}%
\pgfsys@transformshift{2.591678in}{4.067306in}%
\pgfsys@useobject{currentmarker}{}%
\end{pgfscope}%
\begin{pgfscope}%
\pgfsys@transformshift{2.568439in}{4.045767in}%
\pgfsys@useobject{currentmarker}{}%
\end{pgfscope}%
\begin{pgfscope}%
\pgfsys@transformshift{2.554356in}{4.037958in}%
\pgfsys@useobject{currentmarker}{}%
\end{pgfscope}%
\begin{pgfscope}%
\pgfsys@transformshift{2.536515in}{4.035988in}%
\pgfsys@useobject{currentmarker}{}%
\end{pgfscope}%
\begin{pgfscope}%
\pgfsys@transformshift{2.516095in}{4.042204in}%
\pgfsys@useobject{currentmarker}{}%
\end{pgfscope}%
\begin{pgfscope}%
\pgfsys@transformshift{2.493560in}{4.074644in}%
\pgfsys@useobject{currentmarker}{}%
\end{pgfscope}%
\begin{pgfscope}%
\pgfsys@transformshift{2.475955in}{4.172691in}%
\pgfsys@useobject{currentmarker}{}%
\end{pgfscope}%
\begin{pgfscope}%
\pgfsys@transformshift{2.457413in}{4.355605in}%
\pgfsys@useobject{currentmarker}{}%
\end{pgfscope}%
\begin{pgfscope}%
\pgfsys@transformshift{2.439573in}{4.413970in}%
\pgfsys@useobject{currentmarker}{}%
\end{pgfscope}%
\begin{pgfscope}%
\pgfsys@transformshift{2.418681in}{4.289729in}%
\pgfsys@useobject{currentmarker}{}%
\end{pgfscope}%
\begin{pgfscope}%
\pgfsys@transformshift{2.397087in}{4.103023in}%
\pgfsys@useobject{currentmarker}{}%
\end{pgfscope}%
\begin{pgfscope}%
\pgfsys@transformshift{2.379011in}{4.054589in}%
\pgfsys@useobject{currentmarker}{}%
\end{pgfscope}%
\begin{pgfscope}%
\pgfsys@transformshift{2.359764in}{4.042358in}%
\pgfsys@useobject{currentmarker}{}%
\end{pgfscope}%
\begin{pgfscope}%
\pgfsys@transformshift{2.340752in}{4.037238in}%
\pgfsys@useobject{currentmarker}{}%
\end{pgfscope}%
\begin{pgfscope}%
\pgfsys@transformshift{2.322208in}{4.035748in}%
\pgfsys@useobject{currentmarker}{}%
\end{pgfscope}%
\begin{pgfscope}%
\pgfsys@transformshift{2.304134in}{4.040858in}%
\pgfsys@useobject{currentmarker}{}%
\end{pgfscope}%
\begin{pgfscope}%
\pgfsys@transformshift{2.282304in}{4.063251in}%
\pgfsys@useobject{currentmarker}{}%
\end{pgfscope}%
\begin{pgfscope}%
\pgfsys@transformshift{2.264465in}{4.137065in}%
\pgfsys@useobject{currentmarker}{}%
\end{pgfscope}%
\begin{pgfscope}%
\pgfsys@transformshift{2.245452in}{4.295195in}%
\pgfsys@useobject{currentmarker}{}%
\end{pgfscope}%
\begin{pgfscope}%
\pgfsys@transformshift{2.227847in}{4.402810in}%
\pgfsys@useobject{currentmarker}{}%
\end{pgfscope}%
\begin{pgfscope}%
\pgfsys@transformshift{2.205314in}{4.375277in}%
\pgfsys@useobject{currentmarker}{}%
\end{pgfscope}%
\begin{pgfscope}%
\pgfsys@transformshift{2.190760in}{4.255670in}%
\pgfsys@useobject{currentmarker}{}%
\end{pgfscope}%
\begin{pgfscope}%
\pgfsys@transformshift{2.167521in}{4.093098in}%
\pgfsys@useobject{currentmarker}{}%
\end{pgfscope}%
\begin{pgfscope}%
\pgfsys@transformshift{2.149448in}{4.052208in}%
\pgfsys@useobject{currentmarker}{}%
\end{pgfscope}%
\begin{pgfscope}%
\pgfsys@transformshift{2.131138in}{4.039342in}%
\pgfsys@useobject{currentmarker}{}%
\end{pgfscope}%
\begin{pgfscope}%
\pgfsys@transformshift{2.110013in}{4.036493in}%
\pgfsys@useobject{currentmarker}{}%
\end{pgfscope}%
\begin{pgfscope}%
\pgfsys@transformshift{2.089357in}{4.035897in}%
\pgfsys@useobject{currentmarker}{}%
\end{pgfscope}%
\begin{pgfscope}%
\pgfsys@transformshift{2.073629in}{4.038779in}%
\pgfsys@useobject{currentmarker}{}%
\end{pgfscope}%
\begin{pgfscope}%
\pgfsys@transformshift{2.051565in}{4.054201in}%
\pgfsys@useobject{currentmarker}{}%
\end{pgfscope}%
\begin{pgfscope}%
\pgfsys@transformshift{2.030440in}{4.113772in}%
\pgfsys@useobject{currentmarker}{}%
\end{pgfscope}%
\begin{pgfscope}%
\pgfsys@transformshift{2.016355in}{4.222980in}%
\pgfsys@useobject{currentmarker}{}%
\end{pgfscope}%
\begin{pgfscope}%
\pgfsys@transformshift{1.996404in}{4.327721in}%
\pgfsys@useobject{currentmarker}{}%
\end{pgfscope}%
\begin{pgfscope}%
\pgfsys@transformshift{1.976922in}{4.413566in}%
\pgfsys@useobject{currentmarker}{}%
\end{pgfscope}%
\begin{pgfscope}%
\pgfsys@transformshift{1.957672in}{4.312261in}%
\pgfsys@useobject{currentmarker}{}%
\end{pgfscope}%
\begin{pgfscope}%
\pgfsys@transformshift{1.937487in}{4.155147in}%
\pgfsys@useobject{currentmarker}{}%
\end{pgfscope}%
\begin{pgfscope}%
\pgfsys@transformshift{1.919413in}{4.074519in}%
\pgfsys@useobject{currentmarker}{}%
\end{pgfscope}%
\begin{pgfscope}%
\pgfsys@transformshift{1.899929in}{4.047282in}%
\pgfsys@useobject{currentmarker}{}%
\end{pgfscope}%
\begin{pgfscope}%
\pgfsys@transformshift{1.879508in}{4.037026in}%
\pgfsys@useobject{currentmarker}{}%
\end{pgfscope}%
\begin{pgfscope}%
\pgfsys@transformshift{1.862608in}{4.057671in}%
\pgfsys@useobject{currentmarker}{}%
\end{pgfscope}%
\begin{pgfscope}%
\pgfsys@transformshift{1.843829in}{4.041527in}%
\pgfsys@useobject{currentmarker}{}%
\end{pgfscope}%
\begin{pgfscope}%
\pgfsys@transformshift{1.820827in}{4.035997in}%
\pgfsys@useobject{currentmarker}{}%
\end{pgfscope}%
\begin{pgfscope}%
\pgfsys@transformshift{1.802517in}{4.038206in}%
\pgfsys@useobject{currentmarker}{}%
\end{pgfscope}%
\begin{pgfscope}%
\pgfsys@transformshift{1.783269in}{4.050930in}%
\pgfsys@useobject{currentmarker}{}%
\end{pgfscope}%
\begin{pgfscope}%
\pgfsys@transformshift{1.764021in}{4.070835in}%
\pgfsys@useobject{currentmarker}{}%
\end{pgfscope}%
\begin{pgfscope}%
\pgfsys@transformshift{1.747356in}{4.145150in}%
\pgfsys@useobject{currentmarker}{}%
\end{pgfscope}%
\begin{pgfscope}%
\pgfsys@transformshift{1.725760in}{4.333862in}%
\pgfsys@useobject{currentmarker}{}%
\end{pgfscope}%
\begin{pgfscope}%
\pgfsys@transformshift{1.709095in}{4.405182in}%
\pgfsys@useobject{currentmarker}{}%
\end{pgfscope}%
\begin{pgfscope}%
\pgfsys@transformshift{1.687265in}{4.401978in}%
\pgfsys@useobject{currentmarker}{}%
\end{pgfscope}%
\begin{pgfscope}%
\pgfsys@transformshift{1.670129in}{4.246008in}%
\pgfsys@useobject{currentmarker}{}%
\end{pgfscope}%
\begin{pgfscope}%
\pgfsys@transformshift{1.651116in}{4.112101in}%
\pgfsys@useobject{currentmarker}{}%
\end{pgfscope}%
\begin{pgfscope}%
\pgfsys@transformshift{1.628114in}{4.053644in}%
\pgfsys@useobject{currentmarker}{}%
\end{pgfscope}%
\begin{pgfscope}%
\pgfsys@transformshift{1.613326in}{4.041847in}%
\pgfsys@useobject{currentmarker}{}%
\end{pgfscope}%
\begin{pgfscope}%
\pgfsys@transformshift{1.591496in}{4.036568in}%
\pgfsys@useobject{currentmarker}{}%
\end{pgfscope}%
\begin{pgfscope}%
\pgfsys@transformshift{1.573891in}{4.036695in}%
\pgfsys@useobject{currentmarker}{}%
\end{pgfscope}%
\begin{pgfscope}%
\pgfsys@transformshift{1.553235in}{4.044071in}%
\pgfsys@useobject{currentmarker}{}%
\end{pgfscope}%
\begin{pgfscope}%
\pgfsys@transformshift{1.534927in}{4.066708in}%
\pgfsys@useobject{currentmarker}{}%
\end{pgfscope}%
\begin{pgfscope}%
\pgfsys@transformshift{1.513331in}{4.117381in}%
\pgfsys@useobject{currentmarker}{}%
\end{pgfscope}%
\begin{pgfscope}%
\pgfsys@transformshift{1.495492in}{4.231480in}%
\pgfsys@useobject{currentmarker}{}%
\end{pgfscope}%
\begin{pgfscope}%
\pgfsys@transformshift{1.476713in}{4.359959in}%
\pgfsys@useobject{currentmarker}{}%
\end{pgfscope}%
\begin{pgfscope}%
\pgfsys@transformshift{1.457934in}{4.424188in}%
\pgfsys@useobject{currentmarker}{}%
\end{pgfscope}%
\begin{pgfscope}%
\pgfsys@transformshift{1.439626in}{4.407076in}%
\pgfsys@useobject{currentmarker}{}%
\end{pgfscope}%
\begin{pgfscope}%
\pgfsys@transformshift{1.418265in}{4.224259in}%
\pgfsys@useobject{currentmarker}{}%
\end{pgfscope}%
\begin{pgfscope}%
\pgfsys@transformshift{1.401130in}{4.114177in}%
\pgfsys@useobject{currentmarker}{}%
\end{pgfscope}%
\begin{pgfscope}%
\pgfsys@transformshift{1.382820in}{4.067494in}%
\pgfsys@useobject{currentmarker}{}%
\end{pgfscope}%
\begin{pgfscope}%
\pgfsys@transformshift{1.362399in}{4.049049in}%
\pgfsys@useobject{currentmarker}{}%
\end{pgfscope}%
\begin{pgfscope}%
\pgfsys@transformshift{1.341039in}{4.038779in}%
\pgfsys@useobject{currentmarker}{}%
\end{pgfscope}%
\begin{pgfscope}%
\pgfsys@transformshift{1.322495in}{4.036390in}%
\pgfsys@useobject{currentmarker}{}%
\end{pgfscope}%
\begin{pgfscope}%
\pgfsys@transformshift{1.304187in}{4.041428in}%
\pgfsys@useobject{currentmarker}{}%
\end{pgfscope}%
\begin{pgfscope}%
\pgfsys@transformshift{1.282122in}{4.055629in}%
\pgfsys@useobject{currentmarker}{}%
\end{pgfscope}%
\begin{pgfscope}%
\pgfsys@transformshift{1.267569in}{4.082991in}%
\pgfsys@useobject{currentmarker}{}%
\end{pgfscope}%
\begin{pgfscope}%
\pgfsys@transformshift{1.245035in}{4.111133in}%
\pgfsys@useobject{currentmarker}{}%
\end{pgfscope}%
\begin{pgfscope}%
\pgfsys@transformshift{1.226725in}{4.238403in}%
\pgfsys@useobject{currentmarker}{}%
\end{pgfscope}%
\begin{pgfscope}%
\pgfsys@transformshift{1.208183in}{4.371788in}%
\pgfsys@useobject{currentmarker}{}%
\end{pgfscope}%
\begin{pgfscope}%
\pgfsys@transformshift{1.189170in}{4.430982in}%
\pgfsys@useobject{currentmarker}{}%
\end{pgfscope}%
\begin{pgfscope}%
\pgfsys@transformshift{1.167808in}{4.432752in}%
\pgfsys@useobject{currentmarker}{}%
\end{pgfscope}%
\begin{pgfscope}%
\pgfsys@transformshift{1.149500in}{4.316252in}%
\pgfsys@useobject{currentmarker}{}%
\end{pgfscope}%
\begin{pgfscope}%
\pgfsys@transformshift{1.130956in}{4.170133in}%
\pgfsys@useobject{currentmarker}{}%
\end{pgfscope}%
\begin{pgfscope}%
\pgfsys@transformshift{1.108891in}{4.268135in}%
\pgfsys@useobject{currentmarker}{}%
\end{pgfscope}%
\begin{pgfscope}%
\pgfsys@transformshift{1.091052in}{4.121293in}%
\pgfsys@useobject{currentmarker}{}%
\end{pgfscope}%
\begin{pgfscope}%
\pgfsys@transformshift{1.072039in}{4.067646in}%
\pgfsys@useobject{currentmarker}{}%
\end{pgfscope}%
\begin{pgfscope}%
\pgfsys@transformshift{1.054670in}{4.050429in}%
\pgfsys@useobject{currentmarker}{}%
\end{pgfscope}%
\begin{pgfscope}%
\pgfsys@transformshift{1.035421in}{4.038937in}%
\pgfsys@useobject{currentmarker}{}%
\end{pgfscope}%
\begin{pgfscope}%
\pgfsys@transformshift{1.014296in}{4.037797in}%
\pgfsys@useobject{currentmarker}{}%
\end{pgfscope}%
\begin{pgfscope}%
\pgfsys@transformshift{0.994345in}{4.046236in}%
\pgfsys@useobject{currentmarker}{}%
\end{pgfscope}%
\begin{pgfscope}%
\pgfsys@transformshift{0.977443in}{4.067261in}%
\pgfsys@useobject{currentmarker}{}%
\end{pgfscope}%
\begin{pgfscope}%
\pgfsys@transformshift{0.958196in}{4.103725in}%
\pgfsys@useobject{currentmarker}{}%
\end{pgfscope}%
\begin{pgfscope}%
\pgfsys@transformshift{0.936365in}{4.177443in}%
\pgfsys@useobject{currentmarker}{}%
\end{pgfscope}%
\begin{pgfscope}%
\pgfsys@transformshift{0.920874in}{4.297430in}%
\pgfsys@useobject{currentmarker}{}%
\end{pgfscope}%
\begin{pgfscope}%
\pgfsys@transformshift{0.899044in}{4.424578in}%
\pgfsys@useobject{currentmarker}{}%
\end{pgfscope}%
\begin{pgfscope}%
\pgfsys@transformshift{0.881908in}{4.445960in}%
\pgfsys@useobject{currentmarker}{}%
\end{pgfscope}%
\begin{pgfscope}%
\pgfsys@transformshift{0.862426in}{4.414637in}%
\pgfsys@useobject{currentmarker}{}%
\end{pgfscope}%
\begin{pgfscope}%
\pgfsys@transformshift{0.843882in}{4.206813in}%
\pgfsys@useobject{currentmarker}{}%
\end{pgfscope}%
\begin{pgfscope}%
\pgfsys@transformshift{0.822053in}{4.099555in}%
\pgfsys@useobject{currentmarker}{}%
\end{pgfscope}%
\begin{pgfscope}%
\pgfsys@transformshift{0.803509in}{4.060948in}%
\pgfsys@useobject{currentmarker}{}%
\end{pgfscope}%
\begin{pgfscope}%
\pgfsys@transformshift{0.784964in}{4.048247in}%
\pgfsys@useobject{currentmarker}{}%
\end{pgfscope}%
\begin{pgfscope}%
\pgfsys@transformshift{0.766422in}{4.038852in}%
\pgfsys@useobject{currentmarker}{}%
\end{pgfscope}%
\begin{pgfscope}%
\pgfsys@transformshift{0.744357in}{4.037511in}%
\pgfsys@useobject{currentmarker}{}%
\end{pgfscope}%
\begin{pgfscope}%
\pgfsys@transformshift{0.726518in}{4.043167in}%
\pgfsys@useobject{currentmarker}{}%
\end{pgfscope}%
\begin{pgfscope}%
\pgfsys@transformshift{0.708445in}{4.056957in}%
\pgfsys@useobject{currentmarker}{}%
\end{pgfscope}%
\begin{pgfscope}%
\pgfsys@transformshift{0.688960in}{4.085528in}%
\pgfsys@useobject{currentmarker}{}%
\end{pgfscope}%
\begin{pgfscope}%
\pgfsys@transformshift{0.669947in}{4.134390in}%
\pgfsys@useobject{currentmarker}{}%
\end{pgfscope}%
\begin{pgfscope}%
\pgfsys@transformshift{0.653048in}{4.254726in}%
\pgfsys@useobject{currentmarker}{}%
\end{pgfscope}%
\begin{pgfscope}%
\pgfsys@transformshift{0.653516in}{4.301977in}%
\pgfsys@useobject{currentmarker}{}%
\end{pgfscope}%
\begin{pgfscope}%
\pgfsys@transformshift{0.658681in}{4.037995in}%
\pgfsys@useobject{currentmarker}{}%
\end{pgfscope}%
\begin{pgfscope}%
\pgfsys@transformshift{0.676052in}{4.042638in}%
\pgfsys@useobject{currentmarker}{}%
\end{pgfscope}%
\begin{pgfscope}%
\pgfsys@transformshift{0.695768in}{4.066081in}%
\pgfsys@useobject{currentmarker}{}%
\end{pgfscope}%
\begin{pgfscope}%
\pgfsys@transformshift{0.715485in}{4.159491in}%
\pgfsys@useobject{currentmarker}{}%
\end{pgfscope}%
\begin{pgfscope}%
\pgfsys@transformshift{0.732855in}{4.389448in}%
\pgfsys@useobject{currentmarker}{}%
\end{pgfscope}%
\begin{pgfscope}%
\pgfsys@transformshift{0.751165in}{4.450271in}%
\pgfsys@useobject{currentmarker}{}%
\end{pgfscope}%
\begin{pgfscope}%
\pgfsys@transformshift{0.769004in}{4.345458in}%
\pgfsys@useobject{currentmarker}{}%
\end{pgfscope}%
\begin{pgfscope}%
\pgfsys@transformshift{0.789895in}{4.143641in}%
\pgfsys@useobject{currentmarker}{}%
\end{pgfscope}%
\begin{pgfscope}%
\pgfsys@transformshift{0.811254in}{4.059179in}%
\pgfsys@useobject{currentmarker}{}%
\end{pgfscope}%
\begin{pgfscope}%
\pgfsys@transformshift{0.829799in}{4.041218in}%
\pgfsys@useobject{currentmarker}{}%
\end{pgfscope}%
\begin{pgfscope}%
\pgfsys@transformshift{0.849047in}{4.037560in}%
\pgfsys@useobject{currentmarker}{}%
\end{pgfscope}%
\begin{pgfscope}%
\pgfsys@transformshift{0.865477in}{4.048903in}%
\pgfsys@useobject{currentmarker}{}%
\end{pgfscope}%
\begin{pgfscope}%
\pgfsys@transformshift{0.889890in}{4.100676in}%
\pgfsys@useobject{currentmarker}{}%
\end{pgfscope}%
\begin{pgfscope}%
\pgfsys@transformshift{0.904678in}{4.209585in}%
\pgfsys@useobject{currentmarker}{}%
\end{pgfscope}%
\begin{pgfscope}%
\pgfsys@transformshift{0.926037in}{4.435421in}%
\pgfsys@useobject{currentmarker}{}%
\end{pgfscope}%
\begin{pgfscope}%
\pgfsys@transformshift{0.944347in}{4.405584in}%
\pgfsys@useobject{currentmarker}{}%
\end{pgfscope}%
\begin{pgfscope}%
\pgfsys@transformshift{0.964064in}{4.420992in}%
\pgfsys@useobject{currentmarker}{}%
\end{pgfscope}%
\begin{pgfscope}%
\pgfsys@transformshift{0.981903in}{4.259064in}%
\pgfsys@useobject{currentmarker}{}%
\end{pgfscope}%
\begin{pgfscope}%
\pgfsys@transformshift{1.001856in}{4.095732in}%
\pgfsys@useobject{currentmarker}{}%
\end{pgfscope}%
\begin{pgfscope}%
\pgfsys@transformshift{1.023215in}{4.046197in}%
\pgfsys@useobject{currentmarker}{}%
\end{pgfscope}%
\begin{pgfscope}%
\pgfsys@transformshift{1.045985in}{4.036489in}%
\pgfsys@useobject{currentmarker}{}%
\end{pgfscope}%
\begin{pgfscope}%
\pgfsys@transformshift{1.060302in}{4.039991in}%
\pgfsys@useobject{currentmarker}{}%
\end{pgfscope}%
\begin{pgfscope}%
\pgfsys@transformshift{1.078847in}{4.057173in}%
\pgfsys@useobject{currentmarker}{}%
\end{pgfscope}%
\begin{pgfscope}%
\pgfsys@transformshift{1.099737in}{4.119048in}%
\pgfsys@useobject{currentmarker}{}%
\end{pgfscope}%
\begin{pgfscope}%
\pgfsys@transformshift{1.117342in}{4.340816in}%
\pgfsys@useobject{currentmarker}{}%
\end{pgfscope}%
\begin{pgfscope}%
\pgfsys@transformshift{1.138233in}{4.431077in}%
\pgfsys@useobject{currentmarker}{}%
\end{pgfscope}%
\begin{pgfscope}%
\pgfsys@transformshift{1.155837in}{4.320461in}%
\pgfsys@useobject{currentmarker}{}%
\end{pgfscope}%
\begin{pgfscope}%
\pgfsys@transformshift{1.174616in}{4.132005in}%
\pgfsys@useobject{currentmarker}{}%
\end{pgfscope}%
\begin{pgfscope}%
\pgfsys@transformshift{1.197384in}{4.053945in}%
\pgfsys@useobject{currentmarker}{}%
\end{pgfscope}%
\begin{pgfscope}%
\pgfsys@transformshift{1.214520in}{4.039785in}%
\pgfsys@useobject{currentmarker}{}%
\end{pgfscope}%
\begin{pgfscope}%
\pgfsys@transformshift{1.232359in}{4.036109in}%
\pgfsys@useobject{currentmarker}{}%
\end{pgfscope}%
\begin{pgfscope}%
\pgfsys@transformshift{1.251841in}{4.043703in}%
\pgfsys@useobject{currentmarker}{}%
\end{pgfscope}%
\begin{pgfscope}%
\pgfsys@transformshift{1.270855in}{4.069792in}%
\pgfsys@useobject{currentmarker}{}%
\end{pgfscope}%
\begin{pgfscope}%
\pgfsys@transformshift{1.290339in}{4.157199in}%
\pgfsys@useobject{currentmarker}{}%
\end{pgfscope}%
\begin{pgfscope}%
\pgfsys@transformshift{1.309116in}{4.371568in}%
\pgfsys@useobject{currentmarker}{}%
\end{pgfscope}%
\begin{pgfscope}%
\pgfsys@transformshift{1.332120in}{4.414187in}%
\pgfsys@useobject{currentmarker}{}%
\end{pgfscope}%
\begin{pgfscope}%
\pgfsys@transformshift{1.349725in}{4.270949in}%
\pgfsys@useobject{currentmarker}{}%
\end{pgfscope}%
\begin{pgfscope}%
\pgfsys@transformshift{1.365686in}{4.125159in}%
\pgfsys@useobject{currentmarker}{}%
\end{pgfscope}%
\begin{pgfscope}%
\pgfsys@transformshift{1.388454in}{4.052862in}%
\pgfsys@useobject{currentmarker}{}%
\end{pgfscope}%
\begin{pgfscope}%
\pgfsys@transformshift{1.406764in}{4.039553in}%
\pgfsys@useobject{currentmarker}{}%
\end{pgfscope}%
\begin{pgfscope}%
\pgfsys@transformshift{1.426012in}{4.035944in}%
\pgfsys@useobject{currentmarker}{}%
\end{pgfscope}%
\begin{pgfscope}%
\pgfsys@transformshift{1.445494in}{4.039917in}%
\pgfsys@useobject{currentmarker}{}%
\end{pgfscope}%
\begin{pgfscope}%
\pgfsys@transformshift{1.464742in}{4.056754in}%
\pgfsys@useobject{currentmarker}{}%
\end{pgfscope}%
\begin{pgfscope}%
\pgfsys@transformshift{1.483989in}{4.087609in}%
\pgfsys@useobject{currentmarker}{}%
\end{pgfscope}%
\begin{pgfscope}%
\pgfsys@transformshift{1.502768in}{4.221520in}%
\pgfsys@useobject{currentmarker}{}%
\end{pgfscope}%
\begin{pgfscope}%
\pgfsys@transformshift{1.522016in}{4.408731in}%
\pgfsys@useobject{currentmarker}{}%
\end{pgfscope}%
\begin{pgfscope}%
\pgfsys@transformshift{1.540558in}{4.406103in}%
\pgfsys@useobject{currentmarker}{}%
\end{pgfscope}%
\begin{pgfscope}%
\pgfsys@transformshift{1.559337in}{4.308517in}%
\pgfsys@useobject{currentmarker}{}%
\end{pgfscope}%
\begin{pgfscope}%
\pgfsys@transformshift{1.579054in}{4.138732in}%
\pgfsys@useobject{currentmarker}{}%
\end{pgfscope}%
\begin{pgfscope}%
\pgfsys@transformshift{1.597364in}{4.063508in}%
\pgfsys@useobject{currentmarker}{}%
\end{pgfscope}%
\begin{pgfscope}%
\pgfsys@transformshift{1.617080in}{4.042201in}%
\pgfsys@useobject{currentmarker}{}%
\end{pgfscope}%
\begin{pgfscope}%
\pgfsys@transformshift{1.636328in}{4.036029in}%
\pgfsys@useobject{currentmarker}{}%
\end{pgfscope}%
\begin{pgfscope}%
\pgfsys@transformshift{1.655107in}{4.040484in}%
\pgfsys@useobject{currentmarker}{}%
\end{pgfscope}%
\begin{pgfscope}%
\pgfsys@transformshift{1.676468in}{4.035736in}%
\pgfsys@useobject{currentmarker}{}%
\end{pgfscope}%
\begin{pgfscope}%
\pgfsys@transformshift{1.693837in}{4.039723in}%
\pgfsys@useobject{currentmarker}{}%
\end{pgfscope}%
\begin{pgfscope}%
\pgfsys@transformshift{1.712615in}{4.051773in}%
\pgfsys@useobject{currentmarker}{}%
\end{pgfscope}%
\begin{pgfscope}%
\pgfsys@transformshift{1.733740in}{4.110773in}%
\pgfsys@useobject{currentmarker}{}%
\end{pgfscope}%
\begin{pgfscope}%
\pgfsys@transformshift{1.749468in}{4.251008in}%
\pgfsys@useobject{currentmarker}{}%
\end{pgfscope}%
\begin{pgfscope}%
\pgfsys@transformshift{1.772001in}{4.419873in}%
\pgfsys@useobject{currentmarker}{}%
\end{pgfscope}%
\begin{pgfscope}%
\pgfsys@transformshift{1.791249in}{4.377716in}%
\pgfsys@useobject{currentmarker}{}%
\end{pgfscope}%
\begin{pgfscope}%
\pgfsys@transformshift{1.808854in}{4.224116in}%
\pgfsys@useobject{currentmarker}{}%
\end{pgfscope}%
\begin{pgfscope}%
\pgfsys@transformshift{1.828807in}{4.088808in}%
\pgfsys@useobject{currentmarker}{}%
\end{pgfscope}%
\begin{pgfscope}%
\pgfsys@transformshift{1.852280in}{4.045210in}%
\pgfsys@useobject{currentmarker}{}%
\end{pgfscope}%
\begin{pgfscope}%
\pgfsys@transformshift{1.868711in}{4.037044in}%
\pgfsys@useobject{currentmarker}{}%
\end{pgfscope}%
\begin{pgfscope}%
\pgfsys@transformshift{1.886550in}{4.036180in}%
\pgfsys@useobject{currentmarker}{}%
\end{pgfscope}%
\begin{pgfscope}%
\pgfsys@transformshift{1.906972in}{4.040642in}%
\pgfsys@useobject{currentmarker}{}%
\end{pgfscope}%
\begin{pgfscope}%
\pgfsys@transformshift{1.925280in}{4.055062in}%
\pgfsys@useobject{currentmarker}{}%
\end{pgfscope}%
\begin{pgfscope}%
\pgfsys@transformshift{1.943355in}{4.103815in}%
\pgfsys@useobject{currentmarker}{}%
\end{pgfscope}%
\begin{pgfscope}%
\pgfsys@transformshift{1.961898in}{4.241961in}%
\pgfsys@useobject{currentmarker}{}%
\end{pgfscope}%
\begin{pgfscope}%
\pgfsys@transformshift{1.981382in}{4.409792in}%
\pgfsys@useobject{currentmarker}{}%
\end{pgfscope}%
\begin{pgfscope}%
\pgfsys@transformshift{2.002272in}{4.384567in}%
\pgfsys@useobject{currentmarker}{}%
\end{pgfscope}%
\begin{pgfscope}%
\pgfsys@transformshift{2.022928in}{4.249972in}%
\pgfsys@useobject{currentmarker}{}%
\end{pgfscope}%
\begin{pgfscope}%
\pgfsys@transformshift{2.040768in}{4.103815in}%
\pgfsys@useobject{currentmarker}{}%
\end{pgfscope}%
\begin{pgfscope}%
\pgfsys@transformshift{2.059310in}{4.060253in}%
\pgfsys@useobject{currentmarker}{}%
\end{pgfscope}%
\begin{pgfscope}%
\pgfsys@transformshift{2.082080in}{4.041929in}%
\pgfsys@useobject{currentmarker}{}%
\end{pgfscope}%
\begin{pgfscope}%
\pgfsys@transformshift{2.098511in}{4.037224in}%
\pgfsys@useobject{currentmarker}{}%
\end{pgfscope}%
\begin{pgfscope}%
\pgfsys@transformshift{2.118227in}{4.036185in}%
\pgfsys@useobject{currentmarker}{}%
\end{pgfscope}%
\begin{pgfscope}%
\pgfsys@transformshift{2.137946in}{4.042043in}%
\pgfsys@useobject{currentmarker}{}%
\end{pgfscope}%
\begin{pgfscope}%
\pgfsys@transformshift{2.154845in}{4.059428in}%
\pgfsys@useobject{currentmarker}{}%
\end{pgfscope}%
\begin{pgfscope}%
\pgfsys@transformshift{2.175267in}{4.084038in}%
\pgfsys@useobject{currentmarker}{}%
\end{pgfscope}%
\begin{pgfscope}%
\pgfsys@transformshift{2.194046in}{4.147881in}%
\pgfsys@useobject{currentmarker}{}%
\end{pgfscope}%
\begin{pgfscope}%
\pgfsys@transformshift{2.216814in}{4.387935in}%
\pgfsys@useobject{currentmarker}{}%
\end{pgfscope}%
\begin{pgfscope}%
\pgfsys@transformshift{2.231602in}{4.412054in}%
\pgfsys@useobject{currentmarker}{}%
\end{pgfscope}%
\begin{pgfscope}%
\pgfsys@transformshift{2.253432in}{4.323602in}%
\pgfsys@useobject{currentmarker}{}%
\end{pgfscope}%
\begin{pgfscope}%
\pgfsys@transformshift{2.271507in}{4.210000in}%
\pgfsys@useobject{currentmarker}{}%
\end{pgfscope}%
\begin{pgfscope}%
\pgfsys@transformshift{2.291458in}{4.086132in}%
\pgfsys@useobject{currentmarker}{}%
\end{pgfscope}%
\begin{pgfscope}%
\pgfsys@transformshift{2.310003in}{4.050127in}%
\pgfsys@useobject{currentmarker}{}%
\end{pgfscope}%
\begin{pgfscope}%
\pgfsys@transformshift{2.327371in}{4.038894in}%
\pgfsys@useobject{currentmarker}{}%
\end{pgfscope}%
\begin{pgfscope}%
\pgfsys@transformshift{2.347793in}{4.035333in}%
\pgfsys@useobject{currentmarker}{}%
\end{pgfscope}%
\begin{pgfscope}%
\pgfsys@transformshift{2.367980in}{4.038333in}%
\pgfsys@useobject{currentmarker}{}%
\end{pgfscope}%
\begin{pgfscope}%
\pgfsys@transformshift{2.385116in}{4.048977in}%
\pgfsys@useobject{currentmarker}{}%
\end{pgfscope}%
\begin{pgfscope}%
\pgfsys@transformshift{2.406710in}{4.090436in}%
\pgfsys@useobject{currentmarker}{}%
\end{pgfscope}%
\begin{pgfscope}%
\pgfsys@transformshift{2.423141in}{4.214647in}%
\pgfsys@useobject{currentmarker}{}%
\end{pgfscope}%
\begin{pgfscope}%
\pgfsys@transformshift{2.444033in}{4.385089in}%
\pgfsys@useobject{currentmarker}{}%
\end{pgfscope}%
\begin{pgfscope}%
\pgfsys@transformshift{2.462576in}{4.409061in}%
\pgfsys@useobject{currentmarker}{}%
\end{pgfscope}%
\begin{pgfscope}%
\pgfsys@transformshift{2.482997in}{4.341122in}%
\pgfsys@useobject{currentmarker}{}%
\end{pgfscope}%
\begin{pgfscope}%
\pgfsys@transformshift{2.502479in}{4.260472in}%
\pgfsys@useobject{currentmarker}{}%
\end{pgfscope}%
\begin{pgfscope}%
\pgfsys@transformshift{2.520555in}{4.175071in}%
\pgfsys@useobject{currentmarker}{}%
\end{pgfscope}%
\begin{pgfscope}%
\pgfsys@transformshift{2.540740in}{4.070574in}%
\pgfsys@useobject{currentmarker}{}%
\end{pgfscope}%
\begin{pgfscope}%
\pgfsys@transformshift{2.558111in}{4.046388in}%
\pgfsys@useobject{currentmarker}{}%
\end{pgfscope}%
\begin{pgfscope}%
\pgfsys@transformshift{2.579472in}{4.037851in}%
\pgfsys@useobject{currentmarker}{}%
\end{pgfscope}%
\begin{pgfscope}%
\pgfsys@transformshift{2.597077in}{4.035571in}%
\pgfsys@useobject{currentmarker}{}%
\end{pgfscope}%
\begin{pgfscope}%
\pgfsys@transformshift{2.618905in}{4.041162in}%
\pgfsys@useobject{currentmarker}{}%
\end{pgfscope}%
\begin{pgfscope}%
\pgfsys@transformshift{2.636041in}{4.054229in}%
\pgfsys@useobject{currentmarker}{}%
\end{pgfscope}%
\begin{pgfscope}%
\pgfsys@transformshift{2.655054in}{4.092416in}%
\pgfsys@useobject{currentmarker}{}%
\end{pgfscope}%
\begin{pgfscope}%
\pgfsys@transformshift{2.677353in}{4.219683in}%
\pgfsys@useobject{currentmarker}{}%
\end{pgfscope}%
\begin{pgfscope}%
\pgfsys@transformshift{2.694019in}{4.388737in}%
\pgfsys@useobject{currentmarker}{}%
\end{pgfscope}%
\begin{pgfscope}%
\pgfsys@transformshift{2.711858in}{4.404229in}%
\pgfsys@useobject{currentmarker}{}%
\end{pgfscope}%
\begin{pgfscope}%
\pgfsys@transformshift{2.730871in}{4.288981in}%
\pgfsys@useobject{currentmarker}{}%
\end{pgfscope}%
\begin{pgfscope}%
\pgfsys@transformshift{2.749884in}{4.206815in}%
\pgfsys@useobject{currentmarker}{}%
\end{pgfscope}%
\begin{pgfscope}%
\pgfsys@transformshift{2.773826in}{4.067218in}%
\pgfsys@useobject{currentmarker}{}%
\end{pgfscope}%
\begin{pgfscope}%
\pgfsys@transformshift{2.789085in}{4.048596in}%
\pgfsys@useobject{currentmarker}{}%
\end{pgfscope}%
\begin{pgfscope}%
\pgfsys@transformshift{2.808801in}{4.037837in}%
\pgfsys@useobject{currentmarker}{}%
\end{pgfscope}%
\begin{pgfscope}%
\pgfsys@transformshift{2.830163in}{4.035654in}%
\pgfsys@useobject{currentmarker}{}%
\end{pgfscope}%
\begin{pgfscope}%
\pgfsys@transformshift{2.845185in}{4.038821in}%
\pgfsys@useobject{currentmarker}{}%
\end{pgfscope}%
\begin{pgfscope}%
\pgfsys@transformshift{2.866310in}{4.045330in}%
\pgfsys@useobject{currentmarker}{}%
\end{pgfscope}%
\begin{pgfscope}%
\pgfsys@transformshift{2.883680in}{4.058099in}%
\pgfsys@useobject{currentmarker}{}%
\end{pgfscope}%
\begin{pgfscope}%
\pgfsys@transformshift{2.906685in}{4.124848in}%
\pgfsys@useobject{currentmarker}{}%
\end{pgfscope}%
\begin{pgfscope}%
\pgfsys@transformshift{2.923350in}{4.214488in}%
\pgfsys@useobject{currentmarker}{}%
\end{pgfscope}%
\begin{pgfscope}%
\pgfsys@transformshift{2.947292in}{4.406534in}%
\pgfsys@useobject{currentmarker}{}%
\end{pgfscope}%
\begin{pgfscope}%
\pgfsys@transformshift{2.964897in}{4.392443in}%
\pgfsys@useobject{currentmarker}{}%
\end{pgfscope}%
\begin{pgfscope}%
\pgfsys@transformshift{2.981562in}{4.303155in}%
\pgfsys@useobject{currentmarker}{}%
\end{pgfscope}%
\begin{pgfscope}%
\pgfsys@transformshift{3.001749in}{4.127758in}%
\pgfsys@useobject{currentmarker}{}%
\end{pgfscope}%
\begin{pgfscope}%
\pgfsys@transformshift{3.021702in}{4.063283in}%
\pgfsys@useobject{currentmarker}{}%
\end{pgfscope}%
\begin{pgfscope}%
\pgfsys@transformshift{3.039776in}{4.045670in}%
\pgfsys@useobject{currentmarker}{}%
\end{pgfscope}%
\begin{pgfscope}%
\pgfsys@transformshift{3.058084in}{4.038002in}%
\pgfsys@useobject{currentmarker}{}%
\end{pgfscope}%
\begin{pgfscope}%
\pgfsys@transformshift{3.078505in}{4.036270in}%
\pgfsys@useobject{currentmarker}{}%
\end{pgfscope}%
\begin{pgfscope}%
\pgfsys@transformshift{3.097753in}{4.041257in}%
\pgfsys@useobject{currentmarker}{}%
\end{pgfscope}%
\begin{pgfscope}%
\pgfsys@transformshift{3.114420in}{4.047214in}%
\pgfsys@useobject{currentmarker}{}%
\end{pgfscope}%
\begin{pgfscope}%
\pgfsys@transformshift{3.136014in}{4.040852in}%
\pgfsys@useobject{currentmarker}{}%
\end{pgfscope}%
\begin{pgfscope}%
\pgfsys@transformshift{3.154324in}{4.050505in}%
\pgfsys@useobject{currentmarker}{}%
\end{pgfscope}%
\begin{pgfscope}%
\pgfsys@transformshift{3.174509in}{4.086430in}%
\pgfsys@useobject{currentmarker}{}%
\end{pgfscope}%
\begin{pgfscope}%
\pgfsys@transformshift{3.193523in}{4.179940in}%
\pgfsys@useobject{currentmarker}{}%
\end{pgfscope}%
\begin{pgfscope}%
\pgfsys@transformshift{3.214649in}{4.374432in}%
\pgfsys@useobject{currentmarker}{}%
\end{pgfscope}%
\begin{pgfscope}%
\pgfsys@transformshift{3.232254in}{4.417867in}%
\pgfsys@useobject{currentmarker}{}%
\end{pgfscope}%
\begin{pgfscope}%
\pgfsys@transformshift{3.249859in}{4.345888in}%
\pgfsys@useobject{currentmarker}{}%
\end{pgfscope}%
\begin{pgfscope}%
\pgfsys@transformshift{3.270515in}{4.170093in}%
\pgfsys@useobject{currentmarker}{}%
\end{pgfscope}%
\begin{pgfscope}%
\pgfsys@transformshift{3.289528in}{4.083868in}%
\pgfsys@useobject{currentmarker}{}%
\end{pgfscope}%
\begin{pgfscope}%
\pgfsys@transformshift{3.306663in}{4.362461in}%
\pgfsys@useobject{currentmarker}{}%
\end{pgfscope}%
\begin{pgfscope}%
\pgfsys@transformshift{3.329667in}{4.186887in}%
\pgfsys@useobject{currentmarker}{}%
\end{pgfscope}%
\begin{pgfscope}%
\pgfsys@transformshift{3.347975in}{4.088672in}%
\pgfsys@useobject{currentmarker}{}%
\end{pgfscope}%
\begin{pgfscope}%
\pgfsys@transformshift{3.366285in}{4.052458in}%
\pgfsys@useobject{currentmarker}{}%
\end{pgfscope}%
\begin{pgfscope}%
\pgfsys@transformshift{3.384827in}{4.038950in}%
\pgfsys@useobject{currentmarker}{}%
\end{pgfscope}%
\begin{pgfscope}%
\pgfsys@transformshift{3.402667in}{4.036196in}%
\pgfsys@useobject{currentmarker}{}%
\end{pgfscope}%
\begin{pgfscope}%
\pgfsys@transformshift{3.423793in}{4.039875in}%
\pgfsys@useobject{currentmarker}{}%
\end{pgfscope}%
\begin{pgfscope}%
\pgfsys@transformshift{3.442336in}{4.050717in}%
\pgfsys@useobject{currentmarker}{}%
\end{pgfscope}%
\begin{pgfscope}%
\pgfsys@transformshift{3.460411in}{4.072128in}%
\pgfsys@useobject{currentmarker}{}%
\end{pgfscope}%
\begin{pgfscope}%
\pgfsys@transformshift{3.481536in}{4.160944in}%
\pgfsys@useobject{currentmarker}{}%
\end{pgfscope}%
\begin{pgfscope}%
\pgfsys@transformshift{3.502193in}{4.371605in}%
\pgfsys@useobject{currentmarker}{}%
\end{pgfscope}%
\begin{pgfscope}%
\pgfsys@transformshift{3.516980in}{4.409889in}%
\pgfsys@useobject{currentmarker}{}%
\end{pgfscope}%
\begin{pgfscope}%
\pgfsys@transformshift{3.538105in}{4.410796in}%
\pgfsys@useobject{currentmarker}{}%
\end{pgfscope}%
\begin{pgfscope}%
\pgfsys@transformshift{3.559232in}{4.288009in}%
\pgfsys@useobject{currentmarker}{}%
\end{pgfscope}%
\begin{pgfscope}%
\pgfsys@transformshift{3.577540in}{4.162638in}%
\pgfsys@useobject{currentmarker}{}%
\end{pgfscope}%
\begin{pgfscope}%
\pgfsys@transformshift{3.596788in}{4.083933in}%
\pgfsys@useobject{currentmarker}{}%
\end{pgfscope}%
\begin{pgfscope}%
\pgfsys@transformshift{3.616270in}{4.049906in}%
\pgfsys@useobject{currentmarker}{}%
\end{pgfscope}%
\begin{pgfscope}%
\pgfsys@transformshift{3.634580in}{4.039912in}%
\pgfsys@useobject{currentmarker}{}%
\end{pgfscope}%
\begin{pgfscope}%
\pgfsys@transformshift{3.652888in}{4.036278in}%
\pgfsys@useobject{currentmarker}{}%
\end{pgfscope}%
\begin{pgfscope}%
\pgfsys@transformshift{3.672136in}{4.039294in}%
\pgfsys@useobject{currentmarker}{}%
\end{pgfscope}%
\begin{pgfscope}%
\pgfsys@transformshift{3.691854in}{4.049205in}%
\pgfsys@useobject{currentmarker}{}%
\end{pgfscope}%
\begin{pgfscope}%
\pgfsys@transformshift{3.712274in}{4.076682in}%
\pgfsys@useobject{currentmarker}{}%
\end{pgfscope}%
\begin{pgfscope}%
\pgfsys@transformshift{3.734575in}{4.177381in}%
\pgfsys@useobject{currentmarker}{}%
\end{pgfscope}%
\begin{pgfscope}%
\pgfsys@transformshift{3.749597in}{4.308416in}%
\pgfsys@useobject{currentmarker}{}%
\end{pgfscope}%
\begin{pgfscope}%
\pgfsys@transformshift{3.770957in}{4.413723in}%
\pgfsys@useobject{currentmarker}{}%
\end{pgfscope}%
\begin{pgfscope}%
\pgfsys@transformshift{3.788327in}{4.433319in}%
\pgfsys@useobject{currentmarker}{}%
\end{pgfscope}%
\begin{pgfscope}%
\pgfsys@transformshift{3.808983in}{4.351352in}%
\pgfsys@useobject{currentmarker}{}%
\end{pgfscope}%
\begin{pgfscope}%
\pgfsys@transformshift{3.826119in}{4.215173in}%
\pgfsys@useobject{currentmarker}{}%
\end{pgfscope}%
\begin{pgfscope}%
\pgfsys@transformshift{3.844662in}{4.135853in}%
\pgfsys@useobject{currentmarker}{}%
\end{pgfscope}%
\begin{pgfscope}%
\pgfsys@transformshift{3.865318in}{4.071200in}%
\pgfsys@useobject{currentmarker}{}%
\end{pgfscope}%
\begin{pgfscope}%
\pgfsys@transformshift{3.883159in}{4.053226in}%
\pgfsys@useobject{currentmarker}{}%
\end{pgfscope}%
\begin{pgfscope}%
\pgfsys@transformshift{3.902876in}{4.039342in}%
\pgfsys@useobject{currentmarker}{}%
\end{pgfscope}%
\begin{pgfscope}%
\pgfsys@transformshift{3.922592in}{4.036914in}%
\pgfsys@useobject{currentmarker}{}%
\end{pgfscope}%
\begin{pgfscope}%
\pgfsys@transformshift{3.939728in}{4.040254in}%
\pgfsys@useobject{currentmarker}{}%
\end{pgfscope}%
\begin{pgfscope}%
\pgfsys@transformshift{3.961558in}{4.053011in}%
\pgfsys@useobject{currentmarker}{}%
\end{pgfscope}%
\begin{pgfscope}%
\pgfsys@transformshift{3.983857in}{4.085858in}%
\pgfsys@useobject{currentmarker}{}%
\end{pgfscope}%
\begin{pgfscope}%
\pgfsys@transformshift{4.000991in}{4.143983in}%
\pgfsys@useobject{currentmarker}{}%
\end{pgfscope}%
\begin{pgfscope}%
\pgfsys@transformshift{4.021413in}{4.304006in}%
\pgfsys@useobject{currentmarker}{}%
\end{pgfscope}%
\begin{pgfscope}%
\pgfsys@transformshift{4.036672in}{4.395993in}%
\pgfsys@useobject{currentmarker}{}%
\end{pgfscope}%
\begin{pgfscope}%
\pgfsys@transformshift{4.057328in}{4.440849in}%
\pgfsys@useobject{currentmarker}{}%
\end{pgfscope}%
\begin{pgfscope}%
\pgfsys@transformshift{4.078922in}{4.416458in}%
\pgfsys@useobject{currentmarker}{}%
\end{pgfscope}%
\begin{pgfscope}%
\pgfsys@transformshift{4.096997in}{4.313776in}%
\pgfsys@useobject{currentmarker}{}%
\end{pgfscope}%
\begin{pgfscope}%
\pgfsys@transformshift{4.115071in}{4.174918in}%
\pgfsys@useobject{currentmarker}{}%
\end{pgfscope}%
\begin{pgfscope}%
\pgfsys@transformshift{4.133145in}{4.102619in}%
\pgfsys@useobject{currentmarker}{}%
\end{pgfscope}%
\begin{pgfscope}%
\pgfsys@transformshift{4.151455in}{4.063171in}%
\pgfsys@useobject{currentmarker}{}%
\end{pgfscope}%
\begin{pgfscope}%
\pgfsys@transformshift{4.171640in}{4.048232in}%
\pgfsys@useobject{currentmarker}{}%
\end{pgfscope}%
\begin{pgfscope}%
\pgfsys@transformshift{4.189716in}{4.040758in}%
\pgfsys@useobject{currentmarker}{}%
\end{pgfscope}%
\begin{pgfscope}%
\pgfsys@transformshift{4.213657in}{4.038931in}%
\pgfsys@useobject{currentmarker}{}%
\end{pgfscope}%
\begin{pgfscope}%
\pgfsys@transformshift{4.229385in}{4.044730in}%
\pgfsys@useobject{currentmarker}{}%
\end{pgfscope}%
\begin{pgfscope}%
\pgfsys@transformshift{4.250041in}{4.057876in}%
\pgfsys@useobject{currentmarker}{}%
\end{pgfscope}%
\begin{pgfscope}%
\pgfsys@transformshift{4.271400in}{4.101772in}%
\pgfsys@useobject{currentmarker}{}%
\end{pgfscope}%
\begin{pgfscope}%
\pgfsys@transformshift{4.289945in}{4.169190in}%
\pgfsys@useobject{currentmarker}{}%
\end{pgfscope}%
\begin{pgfscope}%
\pgfsys@transformshift{4.307550in}{4.292457in}%
\pgfsys@useobject{currentmarker}{}%
\end{pgfscope}%
\begin{pgfscope}%
\pgfsys@transformshift{4.324684in}{4.422107in}%
\pgfsys@useobject{currentmarker}{}%
\end{pgfscope}%
\begin{pgfscope}%
\pgfsys@transformshift{4.346279in}{4.455974in}%
\pgfsys@useobject{currentmarker}{}%
\end{pgfscope}%
\begin{pgfscope}%
\pgfsys@transformshift{4.364588in}{4.424942in}%
\pgfsys@useobject{currentmarker}{}%
\end{pgfscope}%
\begin{pgfscope}%
\pgfsys@transformshift{4.386183in}{4.315318in}%
\pgfsys@useobject{currentmarker}{}%
\end{pgfscope}%
\begin{pgfscope}%
\pgfsys@transformshift{4.404728in}{4.200775in}%
\pgfsys@useobject{currentmarker}{}%
\end{pgfscope}%
\begin{pgfscope}%
\pgfsys@transformshift{4.422096in}{4.124877in}%
\pgfsys@useobject{currentmarker}{}%
\end{pgfscope}%
\begin{pgfscope}%
\pgfsys@transformshift{4.444161in}{4.063872in}%
\pgfsys@useobject{currentmarker}{}%
\end{pgfscope}%
\begin{pgfscope}%
\pgfsys@transformshift{4.461062in}{4.049166in}%
\pgfsys@useobject{currentmarker}{}%
\end{pgfscope}%
\begin{pgfscope}%
\pgfsys@transformshift{4.481013in}{4.039655in}%
\pgfsys@useobject{currentmarker}{}%
\end{pgfscope}%
\begin{pgfscope}%
\pgfsys@transformshift{4.478902in}{4.040888in}%
\pgfsys@useobject{currentmarker}{}%
\end{pgfscope}%
\begin{pgfscope}%
\pgfsys@transformshift{4.475850in}{4.042843in}%
\pgfsys@useobject{currentmarker}{}%
\end{pgfscope}%
\begin{pgfscope}%
\pgfsys@transformshift{4.454254in}{4.070279in}%
\pgfsys@useobject{currentmarker}{}%
\end{pgfscope}%
\begin{pgfscope}%
\pgfsys@transformshift{4.435476in}{4.161760in}%
\pgfsys@useobject{currentmarker}{}%
\end{pgfscope}%
\begin{pgfscope}%
\pgfsys@transformshift{4.416933in}{4.357985in}%
\pgfsys@useobject{currentmarker}{}%
\end{pgfscope}%
\begin{pgfscope}%
\pgfsys@transformshift{4.398389in}{4.173361in}%
\pgfsys@useobject{currentmarker}{}%
\end{pgfscope}%
\begin{pgfscope}%
\pgfsys@transformshift{4.376090in}{4.066895in}%
\pgfsys@useobject{currentmarker}{}%
\end{pgfscope}%
\begin{pgfscope}%
\pgfsys@transformshift{4.358250in}{4.042830in}%
\pgfsys@useobject{currentmarker}{}%
\end{pgfscope}%
\begin{pgfscope}%
\pgfsys@transformshift{4.341351in}{4.037904in}%
\pgfsys@useobject{currentmarker}{}%
\end{pgfscope}%
\begin{pgfscope}%
\pgfsys@transformshift{4.321164in}{4.047927in}%
\pgfsys@useobject{currentmarker}{}%
\end{pgfscope}%
\begin{pgfscope}%
\pgfsys@transformshift{4.299802in}{4.100526in}%
\pgfsys@useobject{currentmarker}{}%
\end{pgfscope}%
\begin{pgfscope}%
\pgfsys@transformshift{4.280554in}{4.259581in}%
\pgfsys@useobject{currentmarker}{}%
\end{pgfscope}%
\begin{pgfscope}%
\pgfsys@transformshift{4.261072in}{4.433690in}%
\pgfsys@useobject{currentmarker}{}%
\end{pgfscope}%
\begin{pgfscope}%
\pgfsys@transformshift{4.244642in}{4.444603in}%
\pgfsys@useobject{currentmarker}{}%
\end{pgfscope}%
\begin{pgfscope}%
\pgfsys@transformshift{4.222811in}{4.242808in}%
\pgfsys@useobject{currentmarker}{}%
\end{pgfscope}%
\begin{pgfscope}%
\pgfsys@transformshift{4.204738in}{4.094751in}%
\pgfsys@useobject{currentmarker}{}%
\end{pgfscope}%
\begin{pgfscope}%
\pgfsys@transformshift{4.187367in}{4.051759in}%
\pgfsys@useobject{currentmarker}{}%
\end{pgfscope}%
\begin{pgfscope}%
\pgfsys@transformshift{4.165537in}{4.037602in}%
\pgfsys@useobject{currentmarker}{}%
\end{pgfscope}%
\begin{pgfscope}%
\pgfsys@transformshift{4.148167in}{4.039489in}%
\pgfsys@useobject{currentmarker}{}%
\end{pgfscope}%
\begin{pgfscope}%
\pgfsys@transformshift{4.128919in}{4.059672in}%
\pgfsys@useobject{currentmarker}{}%
\end{pgfscope}%
\begin{pgfscope}%
\pgfsys@transformshift{4.109906in}{4.121231in}%
\pgfsys@useobject{currentmarker}{}%
\end{pgfscope}%
\begin{pgfscope}%
\pgfsys@transformshift{4.085964in}{4.356605in}%
\pgfsys@useobject{currentmarker}{}%
\end{pgfscope}%
\begin{pgfscope}%
\pgfsys@transformshift{4.068830in}{4.440202in}%
\pgfsys@useobject{currentmarker}{}%
\end{pgfscope}%
\begin{pgfscope}%
\pgfsys@transformshift{4.053337in}{4.402083in}%
\pgfsys@useobject{currentmarker}{}%
\end{pgfscope}%
\begin{pgfscope}%
\pgfsys@transformshift{4.033150in}{4.194254in}%
\pgfsys@useobject{currentmarker}{}%
\end{pgfscope}%
\begin{pgfscope}%
\pgfsys@transformshift{4.010382in}{4.068719in}%
\pgfsys@useobject{currentmarker}{}%
\end{pgfscope}%
\begin{pgfscope}%
\pgfsys@transformshift{3.993246in}{4.043309in}%
\pgfsys@useobject{currentmarker}{}%
\end{pgfscope}%
\begin{pgfscope}%
\pgfsys@transformshift{3.976346in}{4.036424in}%
\pgfsys@useobject{currentmarker}{}%
\end{pgfscope}%
\begin{pgfscope}%
\pgfsys@transformshift{3.955219in}{4.040814in}%
\pgfsys@useobject{currentmarker}{}%
\end{pgfscope}%
\begin{pgfscope}%
\pgfsys@transformshift{3.934799in}{4.066534in}%
\pgfsys@useobject{currentmarker}{}%
\end{pgfscope}%
\begin{pgfscope}%
\pgfsys@transformshift{3.917195in}{4.142458in}%
\pgfsys@useobject{currentmarker}{}%
\end{pgfscope}%
\begin{pgfscope}%
\pgfsys@transformshift{3.898885in}{4.346570in}%
\pgfsys@useobject{currentmarker}{}%
\end{pgfscope}%
\begin{pgfscope}%
\pgfsys@transformshift{3.879403in}{4.432694in}%
\pgfsys@useobject{currentmarker}{}%
\end{pgfscope}%
\begin{pgfscope}%
\pgfsys@transformshift{3.858746in}{4.327857in}%
\pgfsys@useobject{currentmarker}{}%
\end{pgfscope}%
\begin{pgfscope}%
\pgfsys@transformshift{3.838559in}{4.109740in}%
\pgfsys@useobject{currentmarker}{}%
\end{pgfscope}%
\begin{pgfscope}%
\pgfsys@transformshift{3.822363in}{4.057135in}%
\pgfsys@useobject{currentmarker}{}%
\end{pgfscope}%
\begin{pgfscope}%
\pgfsys@transformshift{3.801238in}{4.038820in}%
\pgfsys@useobject{currentmarker}{}%
\end{pgfscope}%
\begin{pgfscope}%
\pgfsys@transformshift{3.782459in}{4.036199in}%
\pgfsys@useobject{currentmarker}{}%
\end{pgfscope}%
\begin{pgfscope}%
\pgfsys@transformshift{3.764854in}{4.041891in}%
\pgfsys@useobject{currentmarker}{}%
\end{pgfscope}%
\begin{pgfscope}%
\pgfsys@transformshift{3.745372in}{4.054789in}%
\pgfsys@useobject{currentmarker}{}%
\end{pgfscope}%
\begin{pgfscope}%
\pgfsys@transformshift{3.724716in}{4.133448in}%
\pgfsys@useobject{currentmarker}{}%
\end{pgfscope}%
\begin{pgfscope}%
\pgfsys@transformshift{3.704999in}{4.277161in}%
\pgfsys@useobject{currentmarker}{}%
\end{pgfscope}%
\begin{pgfscope}%
\pgfsys@transformshift{3.687629in}{4.405473in}%
\pgfsys@useobject{currentmarker}{}%
\end{pgfscope}%
\begin{pgfscope}%
\pgfsys@transformshift{3.666738in}{4.403025in}%
\pgfsys@useobject{currentmarker}{}%
\end{pgfscope}%
\begin{pgfscope}%
\pgfsys@transformshift{3.649603in}{4.248064in}%
\pgfsys@useobject{currentmarker}{}%
\end{pgfscope}%
\begin{pgfscope}%
\pgfsys@transformshift{3.628241in}{4.096193in}%
\pgfsys@useobject{currentmarker}{}%
\end{pgfscope}%
\begin{pgfscope}%
\pgfsys@transformshift{3.611342in}{4.057085in}%
\pgfsys@useobject{currentmarker}{}%
\end{pgfscope}%
\begin{pgfscope}%
\pgfsys@transformshift{3.589982in}{4.040004in}%
\pgfsys@useobject{currentmarker}{}%
\end{pgfscope}%
\begin{pgfscope}%
\pgfsys@transformshift{3.572141in}{4.036079in}%
\pgfsys@useobject{currentmarker}{}%
\end{pgfscope}%
\begin{pgfscope}%
\pgfsys@transformshift{3.552190in}{4.038196in}%
\pgfsys@useobject{currentmarker}{}%
\end{pgfscope}%
\begin{pgfscope}%
\pgfsys@transformshift{3.533880in}{4.049059in}%
\pgfsys@useobject{currentmarker}{}%
\end{pgfscope}%
\begin{pgfscope}%
\pgfsys@transformshift{3.513929in}{4.099562in}%
\pgfsys@useobject{currentmarker}{}%
\end{pgfscope}%
\begin{pgfscope}%
\pgfsys@transformshift{3.495619in}{4.215032in}%
\pgfsys@useobject{currentmarker}{}%
\end{pgfscope}%
\begin{pgfscope}%
\pgfsys@transformshift{3.475434in}{4.350557in}%
\pgfsys@useobject{currentmarker}{}%
\end{pgfscope}%
\begin{pgfscope}%
\pgfsys@transformshift{3.451726in}{4.416976in}%
\pgfsys@useobject{currentmarker}{}%
\end{pgfscope}%
\begin{pgfscope}%
\pgfsys@transformshift{3.436702in}{4.272234in}%
\pgfsys@useobject{currentmarker}{}%
\end{pgfscope}%
\begin{pgfscope}%
\pgfsys@transformshift{3.420271in}{4.135292in}%
\pgfsys@useobject{currentmarker}{}%
\end{pgfscope}%
\begin{pgfscope}%
\pgfsys@transformshift{3.398441in}{4.060996in}%
\pgfsys@useobject{currentmarker}{}%
\end{pgfscope}%
\begin{pgfscope}%
\pgfsys@transformshift{3.380367in}{4.042066in}%
\pgfsys@useobject{currentmarker}{}%
\end{pgfscope}%
\begin{pgfscope}%
\pgfsys@transformshift{3.356660in}{4.036368in}%
\pgfsys@useobject{currentmarker}{}%
\end{pgfscope}%
\begin{pgfscope}%
\pgfsys@transformshift{3.341872in}{4.036285in}%
\pgfsys@useobject{currentmarker}{}%
\end{pgfscope}%
\begin{pgfscope}%
\pgfsys@transformshift{3.320982in}{4.044071in}%
\pgfsys@useobject{currentmarker}{}%
\end{pgfscope}%
\begin{pgfscope}%
\pgfsys@transformshift{3.301968in}{4.073979in}%
\pgfsys@useobject{currentmarker}{}%
\end{pgfscope}%
\begin{pgfscope}%
\pgfsys@transformshift{3.283660in}{4.149549in}%
\pgfsys@useobject{currentmarker}{}%
\end{pgfscope}%
\begin{pgfscope}%
\pgfsys@transformshift{3.263942in}{4.310150in}%
\pgfsys@useobject{currentmarker}{}%
\end{pgfscope}%
\begin{pgfscope}%
\pgfsys@transformshift{3.247042in}{4.109652in}%
\pgfsys@useobject{currentmarker}{}%
\end{pgfscope}%
\begin{pgfscope}%
\pgfsys@transformshift{3.223804in}{4.233044in}%
\pgfsys@useobject{currentmarker}{}%
\end{pgfscope}%
\begin{pgfscope}%
\pgfsys@transformshift{3.206668in}{4.387238in}%
\pgfsys@useobject{currentmarker}{}%
\end{pgfscope}%
\begin{pgfscope}%
\pgfsys@transformshift{3.186011in}{4.407492in}%
\pgfsys@useobject{currentmarker}{}%
\end{pgfscope}%
\begin{pgfscope}%
\pgfsys@transformshift{3.169112in}{4.220961in}%
\pgfsys@useobject{currentmarker}{}%
\end{pgfscope}%
\begin{pgfscope}%
\pgfsys@transformshift{3.148456in}{4.081811in}%
\pgfsys@useobject{currentmarker}{}%
\end{pgfscope}%
\begin{pgfscope}%
\pgfsys@transformshift{3.130851in}{4.273543in}%
\pgfsys@useobject{currentmarker}{}%
\end{pgfscope}%
\begin{pgfscope}%
\pgfsys@transformshift{3.111132in}{4.096127in}%
\pgfsys@useobject{currentmarker}{}%
\end{pgfscope}%
\begin{pgfscope}%
\pgfsys@transformshift{3.090242in}{4.052130in}%
\pgfsys@useobject{currentmarker}{}%
\end{pgfscope}%
\begin{pgfscope}%
\pgfsys@transformshift{3.072637in}{4.038546in}%
\pgfsys@useobject{currentmarker}{}%
\end{pgfscope}%
\begin{pgfscope}%
\pgfsys@transformshift{3.051512in}{4.036078in}%
\pgfsys@useobject{currentmarker}{}%
\end{pgfscope}%
\begin{pgfscope}%
\pgfsys@transformshift{3.033204in}{4.041587in}%
\pgfsys@useobject{currentmarker}{}%
\end{pgfscope}%
\begin{pgfscope}%
\pgfsys@transformshift{3.013017in}{4.059485in}%
\pgfsys@useobject{currentmarker}{}%
\end{pgfscope}%
\begin{pgfscope}%
\pgfsys@transformshift{2.993535in}{4.129668in}%
\pgfsys@useobject{currentmarker}{}%
\end{pgfscope}%
\begin{pgfscope}%
\pgfsys@transformshift{2.978276in}{4.300458in}%
\pgfsys@useobject{currentmarker}{}%
\end{pgfscope}%
\begin{pgfscope}%
\pgfsys@transformshift{2.956682in}{4.391802in}%
\pgfsys@useobject{currentmarker}{}%
\end{pgfscope}%
\begin{pgfscope}%
\pgfsys@transformshift{2.937669in}{4.412970in}%
\pgfsys@useobject{currentmarker}{}%
\end{pgfscope}%
\begin{pgfscope}%
\pgfsys@transformshift{2.915839in}{4.210339in}%
\pgfsys@useobject{currentmarker}{}%
\end{pgfscope}%
\begin{pgfscope}%
\pgfsys@transformshift{2.897529in}{4.092158in}%
\pgfsys@useobject{currentmarker}{}%
\end{pgfscope}%
\begin{pgfscope}%
\pgfsys@transformshift{2.879221in}{4.050661in}%
\pgfsys@useobject{currentmarker}{}%
\end{pgfscope}%
\begin{pgfscope}%
\pgfsys@transformshift{2.860207in}{4.038739in}%
\pgfsys@useobject{currentmarker}{}%
\end{pgfscope}%
\begin{pgfscope}%
\pgfsys@transformshift{2.841899in}{4.035656in}%
\pgfsys@useobject{currentmarker}{}%
\end{pgfscope}%
\begin{pgfscope}%
\pgfsys@transformshift{2.823120in}{4.037047in}%
\pgfsys@useobject{currentmarker}{}%
\end{pgfscope}%
\begin{pgfscope}%
\pgfsys@transformshift{2.801056in}{4.048656in}%
\pgfsys@useobject{currentmarker}{}%
\end{pgfscope}%
\begin{pgfscope}%
\pgfsys@transformshift{2.783685in}{4.068976in}%
\pgfsys@useobject{currentmarker}{}%
\end{pgfscope}%
\begin{pgfscope}%
\pgfsys@transformshift{2.763969in}{4.168703in}%
\pgfsys@useobject{currentmarker}{}%
\end{pgfscope}%
\begin{pgfscope}%
\pgfsys@transformshift{2.744956in}{4.326510in}%
\pgfsys@useobject{currentmarker}{}%
\end{pgfscope}%
\begin{pgfscope}%
\pgfsys@transformshift{2.722891in}{4.412927in}%
\pgfsys@useobject{currentmarker}{}%
\end{pgfscope}%
\begin{pgfscope}%
\pgfsys@transformshift{2.706226in}{4.307905in}%
\pgfsys@useobject{currentmarker}{}%
\end{pgfscope}%
\begin{pgfscope}%
\pgfsys@transformshift{2.686273in}{4.126138in}%
\pgfsys@useobject{currentmarker}{}%
\end{pgfscope}%
\begin{pgfscope}%
\pgfsys@transformshift{2.668668in}{4.061935in}%
\pgfsys@useobject{currentmarker}{}%
\end{pgfscope}%
\begin{pgfscope}%
\pgfsys@transformshift{2.649889in}{4.044196in}%
\pgfsys@useobject{currentmarker}{}%
\end{pgfscope}%
\begin{pgfscope}%
\pgfsys@transformshift{2.629468in}{4.036028in}%
\pgfsys@useobject{currentmarker}{}%
\end{pgfscope}%
\begin{pgfscope}%
\pgfsys@transformshift{2.607639in}{4.036134in}%
\pgfsys@useobject{currentmarker}{}%
\end{pgfscope}%
\begin{pgfscope}%
\pgfsys@transformshift{2.593086in}{4.039597in}%
\pgfsys@useobject{currentmarker}{}%
\end{pgfscope}%
\begin{pgfscope}%
\pgfsys@transformshift{2.567736in}{4.055075in}%
\pgfsys@useobject{currentmarker}{}%
\end{pgfscope}%
\begin{pgfscope}%
\pgfsys@transformshift{2.552243in}{4.089197in}%
\pgfsys@useobject{currentmarker}{}%
\end{pgfscope}%
\begin{pgfscope}%
\pgfsys@transformshift{2.533698in}{4.194621in}%
\pgfsys@useobject{currentmarker}{}%
\end{pgfscope}%
\begin{pgfscope}%
\pgfsys@transformshift{2.514216in}{4.349307in}%
\pgfsys@useobject{currentmarker}{}%
\end{pgfscope}%
\begin{pgfscope}%
\pgfsys@transformshift{2.496377in}{4.409102in}%
\pgfsys@useobject{currentmarker}{}%
\end{pgfscope}%
\begin{pgfscope}%
\pgfsys@transformshift{2.477364in}{4.357898in}%
\pgfsys@useobject{currentmarker}{}%
\end{pgfscope}%
\begin{pgfscope}%
\pgfsys@transformshift{2.455770in}{4.163049in}%
\pgfsys@useobject{currentmarker}{}%
\end{pgfscope}%
\begin{pgfscope}%
\pgfsys@transformshift{2.436991in}{4.118689in}%
\pgfsys@useobject{currentmarker}{}%
\end{pgfscope}%
\begin{pgfscope}%
\pgfsys@transformshift{2.419152in}{4.059229in}%
\pgfsys@useobject{currentmarker}{}%
\end{pgfscope}%
\begin{pgfscope}%
\pgfsys@transformshift{2.397087in}{4.039119in}%
\pgfsys@useobject{currentmarker}{}%
\end{pgfscope}%
\begin{pgfscope}%
\pgfsys@transformshift{2.381828in}{4.035821in}%
\pgfsys@useobject{currentmarker}{}%
\end{pgfscope}%
\begin{pgfscope}%
\pgfsys@transformshift{2.359764in}{4.037284in}%
\pgfsys@useobject{currentmarker}{}%
\end{pgfscope}%
\begin{pgfscope}%
\pgfsys@transformshift{2.340987in}{4.044578in}%
\pgfsys@useobject{currentmarker}{}%
\end{pgfscope}%
\begin{pgfscope}%
\pgfsys@transformshift{2.322208in}{4.070582in}%
\pgfsys@useobject{currentmarker}{}%
\end{pgfscope}%
\begin{pgfscope}%
\pgfsys@transformshift{2.302960in}{4.149156in}%
\pgfsys@useobject{currentmarker}{}%
\end{pgfscope}%
\begin{pgfscope}%
\pgfsys@transformshift{2.284650in}{4.308424in}%
\pgfsys@useobject{currentmarker}{}%
\end{pgfscope}%
\begin{pgfscope}%
\pgfsys@transformshift{2.266108in}{4.403382in}%
\pgfsys@useobject{currentmarker}{}%
\end{pgfscope}%
\begin{pgfscope}%
\pgfsys@transformshift{2.244747in}{4.374004in}%
\pgfsys@useobject{currentmarker}{}%
\end{pgfscope}%
\begin{pgfscope}%
\pgfsys@transformshift{2.225499in}{4.191142in}%
\pgfsys@useobject{currentmarker}{}%
\end{pgfscope}%
\begin{pgfscope}%
\pgfsys@transformshift{2.206486in}{4.087379in}%
\pgfsys@useobject{currentmarker}{}%
\end{pgfscope}%
\begin{pgfscope}%
\pgfsys@transformshift{2.187943in}{4.051898in}%
\pgfsys@useobject{currentmarker}{}%
\end{pgfscope}%
\begin{pgfscope}%
\pgfsys@transformshift{2.168695in}{4.041251in}%
\pgfsys@useobject{currentmarker}{}%
\end{pgfscope}%
\begin{pgfscope}%
\pgfsys@transformshift{2.148274in}{4.035470in}%
\pgfsys@useobject{currentmarker}{}%
\end{pgfscope}%
\begin{pgfscope}%
\pgfsys@transformshift{2.128555in}{4.037990in}%
\pgfsys@useobject{currentmarker}{}%
\end{pgfscope}%
\begin{pgfscope}%
\pgfsys@transformshift{2.110716in}{4.047637in}%
\pgfsys@useobject{currentmarker}{}%
\end{pgfscope}%
\begin{pgfscope}%
\pgfsys@transformshift{2.093111in}{4.037527in}%
\pgfsys@useobject{currentmarker}{}%
\end{pgfscope}%
\begin{pgfscope}%
\pgfsys@transformshift{2.071517in}{4.035649in}%
\pgfsys@useobject{currentmarker}{}%
\end{pgfscope}%
\begin{pgfscope}%
\pgfsys@transformshift{2.054850in}{4.036535in}%
\pgfsys@useobject{currentmarker}{}%
\end{pgfscope}%
\begin{pgfscope}%
\pgfsys@transformshift{2.033725in}{4.047131in}%
\pgfsys@useobject{currentmarker}{}%
\end{pgfscope}%
\begin{pgfscope}%
\pgfsys@transformshift{2.014009in}{4.083818in}%
\pgfsys@useobject{currentmarker}{}%
\end{pgfscope}%
\begin{pgfscope}%
\pgfsys@transformshift{1.996170in}{4.214907in}%
\pgfsys@useobject{currentmarker}{}%
\end{pgfscope}%
\begin{pgfscope}%
\pgfsys@transformshift{1.977156in}{4.360288in}%
\pgfsys@useobject{currentmarker}{}%
\end{pgfscope}%
\begin{pgfscope}%
\pgfsys@transformshift{1.955326in}{4.414669in}%
\pgfsys@useobject{currentmarker}{}%
\end{pgfscope}%
\begin{pgfscope}%
\pgfsys@transformshift{1.936782in}{4.305831in}%
\pgfsys@useobject{currentmarker}{}%
\end{pgfscope}%
\begin{pgfscope}%
\pgfsys@transformshift{1.918005in}{4.143143in}%
\pgfsys@useobject{currentmarker}{}%
\end{pgfscope}%
\begin{pgfscope}%
\pgfsys@transformshift{1.900164in}{4.070341in}%
\pgfsys@useobject{currentmarker}{}%
\end{pgfscope}%
\begin{pgfscope}%
\pgfsys@transformshift{1.881621in}{4.046834in}%
\pgfsys@useobject{currentmarker}{}%
\end{pgfscope}%
\begin{pgfscope}%
\pgfsys@transformshift{1.863077in}{4.037633in}%
\pgfsys@useobject{currentmarker}{}%
\end{pgfscope}%
\begin{pgfscope}%
\pgfsys@transformshift{1.841247in}{4.036491in}%
\pgfsys@useobject{currentmarker}{}%
\end{pgfscope}%
\begin{pgfscope}%
\pgfsys@transformshift{1.822470in}{4.040641in}%
\pgfsys@useobject{currentmarker}{}%
\end{pgfscope}%
\begin{pgfscope}%
\pgfsys@transformshift{1.800874in}{4.061277in}%
\pgfsys@useobject{currentmarker}{}%
\end{pgfscope}%
\begin{pgfscope}%
\pgfsys@transformshift{1.781861in}{4.129509in}%
\pgfsys@useobject{currentmarker}{}%
\end{pgfscope}%
\begin{pgfscope}%
\pgfsys@transformshift{1.763553in}{4.262244in}%
\pgfsys@useobject{currentmarker}{}%
\end{pgfscope}%
\begin{pgfscope}%
\pgfsys@transformshift{1.745243in}{4.395935in}%
\pgfsys@useobject{currentmarker}{}%
\end{pgfscope}%
\begin{pgfscope}%
\pgfsys@transformshift{1.725760in}{4.414581in}%
\pgfsys@useobject{currentmarker}{}%
\end{pgfscope}%
\begin{pgfscope}%
\pgfsys@transformshift{1.708156in}{4.307339in}%
\pgfsys@useobject{currentmarker}{}%
\end{pgfscope}%
\begin{pgfscope}%
\pgfsys@transformshift{1.688908in}{4.163316in}%
\pgfsys@useobject{currentmarker}{}%
\end{pgfscope}%
\begin{pgfscope}%
\pgfsys@transformshift{1.668017in}{4.076939in}%
\pgfsys@useobject{currentmarker}{}%
\end{pgfscope}%
\begin{pgfscope}%
\pgfsys@transformshift{1.649239in}{4.050179in}%
\pgfsys@useobject{currentmarker}{}%
\end{pgfscope}%
\begin{pgfscope}%
\pgfsys@transformshift{1.630696in}{4.041914in}%
\pgfsys@useobject{currentmarker}{}%
\end{pgfscope}%
\begin{pgfscope}%
\pgfsys@transformshift{1.612152in}{4.036151in}%
\pgfsys@useobject{currentmarker}{}%
\end{pgfscope}%
\begin{pgfscope}%
\pgfsys@transformshift{1.590556in}{4.038465in}%
\pgfsys@useobject{currentmarker}{}%
\end{pgfscope}%
\begin{pgfscope}%
\pgfsys@transformshift{1.571779in}{4.047401in}%
\pgfsys@useobject{currentmarker}{}%
\end{pgfscope}%
\begin{pgfscope}%
\pgfsys@transformshift{1.550418in}{4.076402in}%
\pgfsys@useobject{currentmarker}{}%
\end{pgfscope}%
\begin{pgfscope}%
\pgfsys@transformshift{1.535161in}{4.130005in}%
\pgfsys@useobject{currentmarker}{}%
\end{pgfscope}%
\begin{pgfscope}%
\pgfsys@transformshift{1.513800in}{4.275324in}%
\pgfsys@useobject{currentmarker}{}%
\end{pgfscope}%
\begin{pgfscope}%
\pgfsys@transformshift{1.495492in}{4.312489in}%
\pgfsys@useobject{currentmarker}{}%
\end{pgfscope}%
\begin{pgfscope}%
\pgfsys@transformshift{1.476713in}{4.405756in}%
\pgfsys@useobject{currentmarker}{}%
\end{pgfscope}%
\begin{pgfscope}%
\pgfsys@transformshift{1.457699in}{4.422020in}%
\pgfsys@useobject{currentmarker}{}%
\end{pgfscope}%
\begin{pgfscope}%
\pgfsys@transformshift{1.439391in}{4.340329in}%
\pgfsys@useobject{currentmarker}{}%
\end{pgfscope}%
\begin{pgfscope}%
\pgfsys@transformshift{1.421081in}{4.153177in}%
\pgfsys@useobject{currentmarker}{}%
\end{pgfscope}%
\begin{pgfscope}%
\pgfsys@transformshift{1.395731in}{4.067459in}%
\pgfsys@useobject{currentmarker}{}%
\end{pgfscope}%
\begin{pgfscope}%
\pgfsys@transformshift{1.380474in}{4.049075in}%
\pgfsys@useobject{currentmarker}{}%
\end{pgfscope}%
\begin{pgfscope}%
\pgfsys@transformshift{1.361461in}{4.039263in}%
\pgfsys@useobject{currentmarker}{}%
\end{pgfscope}%
\begin{pgfscope}%
\pgfsys@transformshift{1.339396in}{4.036684in}%
\pgfsys@useobject{currentmarker}{}%
\end{pgfscope}%
\begin{pgfscope}%
\pgfsys@transformshift{1.322026in}{4.039674in}%
\pgfsys@useobject{currentmarker}{}%
\end{pgfscope}%
\begin{pgfscope}%
\pgfsys@transformshift{1.303718in}{4.045447in}%
\pgfsys@useobject{currentmarker}{}%
\end{pgfscope}%
\begin{pgfscope}%
\pgfsys@transformshift{1.284939in}{4.061512in}%
\pgfsys@useobject{currentmarker}{}%
\end{pgfscope}%
\begin{pgfscope}%
\pgfsys@transformshift{1.266395in}{4.115137in}%
\pgfsys@useobject{currentmarker}{}%
\end{pgfscope}%
\begin{pgfscope}%
\pgfsys@transformshift{1.247616in}{4.199623in}%
\pgfsys@useobject{currentmarker}{}%
\end{pgfscope}%
\begin{pgfscope}%
\pgfsys@transformshift{1.226022in}{4.363865in}%
\pgfsys@useobject{currentmarker}{}%
\end{pgfscope}%
\begin{pgfscope}%
\pgfsys@transformshift{1.204426in}{4.362669in}%
\pgfsys@useobject{currentmarker}{}%
\end{pgfscope}%
\begin{pgfscope}%
\pgfsys@transformshift{1.188935in}{4.423802in}%
\pgfsys@useobject{currentmarker}{}%
\end{pgfscope}%
\begin{pgfscope}%
\pgfsys@transformshift{1.167105in}{4.418694in}%
\pgfsys@useobject{currentmarker}{}%
\end{pgfscope}%
\begin{pgfscope}%
\pgfsys@transformshift{1.148561in}{4.284058in}%
\pgfsys@useobject{currentmarker}{}%
\end{pgfscope}%
\begin{pgfscope}%
\pgfsys@transformshift{1.130018in}{4.136289in}%
\pgfsys@useobject{currentmarker}{}%
\end{pgfscope}%
\begin{pgfscope}%
\pgfsys@transformshift{1.112648in}{4.080549in}%
\pgfsys@useobject{currentmarker}{}%
\end{pgfscope}%
\begin{pgfscope}%
\pgfsys@transformshift{1.090114in}{4.048768in}%
\pgfsys@useobject{currentmarker}{}%
\end{pgfscope}%
\begin{pgfscope}%
\pgfsys@transformshift{1.071335in}{4.040918in}%
\pgfsys@useobject{currentmarker}{}%
\end{pgfscope}%
\begin{pgfscope}%
\pgfsys@transformshift{1.053262in}{4.036935in}%
\pgfsys@useobject{currentmarker}{}%
\end{pgfscope}%
\begin{pgfscope}%
\pgfsys@transformshift{1.034483in}{4.037459in}%
\pgfsys@useobject{currentmarker}{}%
\end{pgfscope}%
\begin{pgfscope}%
\pgfsys@transformshift{1.015470in}{4.037295in}%
\pgfsys@useobject{currentmarker}{}%
\end{pgfscope}%
\begin{pgfscope}%
\pgfsys@transformshift{0.997865in}{4.040044in}%
\pgfsys@useobject{currentmarker}{}%
\end{pgfscope}%
\begin{pgfscope}%
\pgfsys@transformshift{0.975800in}{4.084953in}%
\pgfsys@useobject{currentmarker}{}%
\end{pgfscope}%
\begin{pgfscope}%
\pgfsys@transformshift{0.956787in}{4.069756in}%
\pgfsys@useobject{currentmarker}{}%
\end{pgfscope}%
\begin{pgfscope}%
\pgfsys@transformshift{0.938008in}{4.045705in}%
\pgfsys@useobject{currentmarker}{}%
\end{pgfscope}%
\begin{pgfscope}%
\pgfsys@transformshift{0.920403in}{4.037563in}%
\pgfsys@useobject{currentmarker}{}%
\end{pgfscope}%
\begin{pgfscope}%
\pgfsys@transformshift{0.898810in}{4.040982in}%
\pgfsys@useobject{currentmarker}{}%
\end{pgfscope}%
\begin{pgfscope}%
\pgfsys@transformshift{0.879796in}{4.054002in}%
\pgfsys@useobject{currentmarker}{}%
\end{pgfscope}%
\begin{pgfscope}%
\pgfsys@transformshift{0.861486in}{4.097538in}%
\pgfsys@useobject{currentmarker}{}%
\end{pgfscope}%
\begin{pgfscope}%
\pgfsys@transformshift{0.842944in}{4.212891in}%
\pgfsys@useobject{currentmarker}{}%
\end{pgfscope}%
\begin{pgfscope}%
\pgfsys@transformshift{0.822522in}{4.361772in}%
\pgfsys@useobject{currentmarker}{}%
\end{pgfscope}%
\begin{pgfscope}%
\pgfsys@transformshift{0.803040in}{4.440821in}%
\pgfsys@useobject{currentmarker}{}%
\end{pgfscope}%
\begin{pgfscope}%
\pgfsys@transformshift{0.784261in}{4.431890in}%
\pgfsys@useobject{currentmarker}{}%
\end{pgfscope}%
\begin{pgfscope}%
\pgfsys@transformshift{0.765717in}{4.264377in}%
\pgfsys@useobject{currentmarker}{}%
\end{pgfscope}%
\begin{pgfscope}%
\pgfsys@transformshift{0.748112in}{4.118530in}%
\pgfsys@useobject{currentmarker}{}%
\end{pgfscope}%
\begin{pgfscope}%
\pgfsys@transformshift{0.726047in}{4.069153in}%
\pgfsys@useobject{currentmarker}{}%
\end{pgfscope}%
\begin{pgfscope}%
\pgfsys@transformshift{0.708679in}{4.047498in}%
\pgfsys@useobject{currentmarker}{}%
\end{pgfscope}%
\begin{pgfscope}%
\pgfsys@transformshift{0.688492in}{4.038192in}%
\pgfsys@useobject{currentmarker}{}%
\end{pgfscope}%
\begin{pgfscope}%
\pgfsys@transformshift{0.670184in}{4.039928in}%
\pgfsys@useobject{currentmarker}{}%
\end{pgfscope}%
\begin{pgfscope}%
\pgfsys@transformshift{0.651874in}{4.049695in}%
\pgfsys@useobject{currentmarker}{}%
\end{pgfscope}%
\begin{pgfscope}%
\pgfsys@transformshift{0.651405in}{4.050027in}%
\pgfsys@useobject{currentmarker}{}%
\end{pgfscope}%
\begin{pgfscope}%
\pgfsys@transformshift{0.654690in}{4.046434in}%
\pgfsys@useobject{currentmarker}{}%
\end{pgfscope}%
\begin{pgfscope}%
\pgfsys@transformshift{0.677224in}{4.039286in}%
\pgfsys@useobject{currentmarker}{}%
\end{pgfscope}%
\begin{pgfscope}%
\pgfsys@transformshift{0.694125in}{4.052818in}%
\pgfsys@useobject{currentmarker}{}%
\end{pgfscope}%
\begin{pgfscope}%
\pgfsys@transformshift{0.714782in}{4.107633in}%
\pgfsys@useobject{currentmarker}{}%
\end{pgfscope}%
\begin{pgfscope}%
\pgfsys@transformshift{0.733560in}{4.275533in}%
\pgfsys@useobject{currentmarker}{}%
\end{pgfscope}%
\begin{pgfscope}%
\pgfsys@transformshift{0.751165in}{4.446717in}%
\pgfsys@useobject{currentmarker}{}%
\end{pgfscope}%
\begin{pgfscope}%
\pgfsys@transformshift{0.772290in}{4.390156in}%
\pgfsys@useobject{currentmarker}{}%
\end{pgfscope}%
\begin{pgfscope}%
\pgfsys@transformshift{0.791538in}{4.201969in}%
\pgfsys@useobject{currentmarker}{}%
\end{pgfscope}%
\begin{pgfscope}%
\pgfsys@transformshift{0.809377in}{4.081950in}%
\pgfsys@useobject{currentmarker}{}%
\end{pgfscope}%
\begin{pgfscope}%
\pgfsys@transformshift{0.834024in}{4.040116in}%
\pgfsys@useobject{currentmarker}{}%
\end{pgfscope}%
\begin{pgfscope}%
\pgfsys@transformshift{0.850221in}{4.036087in}%
\pgfsys@useobject{currentmarker}{}%
\end{pgfscope}%
\begin{pgfscope}%
\pgfsys@transformshift{0.867825in}{4.043922in}%
\pgfsys@useobject{currentmarker}{}%
\end{pgfscope}%
\begin{pgfscope}%
\pgfsys@transformshift{0.887073in}{4.070585in}%
\pgfsys@useobject{currentmarker}{}%
\end{pgfscope}%
\begin{pgfscope}%
\pgfsys@transformshift{0.908667in}{4.222814in}%
\pgfsys@useobject{currentmarker}{}%
\end{pgfscope}%
\begin{pgfscope}%
\pgfsys@transformshift{0.924863in}{4.424039in}%
\pgfsys@useobject{currentmarker}{}%
\end{pgfscope}%
\begin{pgfscope}%
\pgfsys@transformshift{0.945051in}{4.405352in}%
\pgfsys@useobject{currentmarker}{}%
\end{pgfscope}%
\begin{pgfscope}%
\pgfsys@transformshift{0.962890in}{4.217593in}%
\pgfsys@useobject{currentmarker}{}%
\end{pgfscope}%
\begin{pgfscope}%
\pgfsys@transformshift{0.982843in}{4.084712in}%
\pgfsys@useobject{currentmarker}{}%
\end{pgfscope}%
\begin{pgfscope}%
\pgfsys@transformshift{1.001385in}{4.044903in}%
\pgfsys@useobject{currentmarker}{}%
\end{pgfscope}%
\begin{pgfscope}%
\pgfsys@transformshift{1.023686in}{4.035536in}%
\pgfsys@useobject{currentmarker}{}%
\end{pgfscope}%
\begin{pgfscope}%
\pgfsys@transformshift{1.041994in}{4.038986in}%
\pgfsys@useobject{currentmarker}{}%
\end{pgfscope}%
\begin{pgfscope}%
\pgfsys@transformshift{1.061711in}{4.054733in}%
\pgfsys@useobject{currentmarker}{}%
\end{pgfscope}%
\begin{pgfscope}%
\pgfsys@transformshift{1.080255in}{4.119462in}%
\pgfsys@useobject{currentmarker}{}%
\end{pgfscope}%
\begin{pgfscope}%
\pgfsys@transformshift{1.099972in}{4.307051in}%
\pgfsys@useobject{currentmarker}{}%
\end{pgfscope}%
\begin{pgfscope}%
\pgfsys@transformshift{1.115699in}{4.432037in}%
\pgfsys@useobject{currentmarker}{}%
\end{pgfscope}%
\begin{pgfscope}%
\pgfsys@transformshift{1.134712in}{4.357428in}%
\pgfsys@useobject{currentmarker}{}%
\end{pgfscope}%
\begin{pgfscope}%
\pgfsys@transformshift{1.157717in}{4.138711in}%
\pgfsys@useobject{currentmarker}{}%
\end{pgfscope}%
\begin{pgfscope}%
\pgfsys@transformshift{1.177433in}{4.056208in}%
\pgfsys@useobject{currentmarker}{}%
\end{pgfscope}%
\begin{pgfscope}%
\pgfsys@transformshift{1.195741in}{4.037705in}%
\pgfsys@useobject{currentmarker}{}%
\end{pgfscope}%
\begin{pgfscope}%
\pgfsys@transformshift{1.214286in}{4.035209in}%
\pgfsys@useobject{currentmarker}{}%
\end{pgfscope}%
\begin{pgfscope}%
\pgfsys@transformshift{1.233768in}{4.041281in}%
\pgfsys@useobject{currentmarker}{}%
\end{pgfscope}%
\begin{pgfscope}%
\pgfsys@transformshift{1.252781in}{4.061215in}%
\pgfsys@useobject{currentmarker}{}%
\end{pgfscope}%
\begin{pgfscope}%
\pgfsys@transformshift{1.270620in}{4.129218in}%
\pgfsys@useobject{currentmarker}{}%
\end{pgfscope}%
\begin{pgfscope}%
\pgfsys@transformshift{1.292450in}{4.333412in}%
\pgfsys@useobject{currentmarker}{}%
\end{pgfscope}%
\begin{pgfscope}%
\pgfsys@transformshift{1.310993in}{4.425687in}%
\pgfsys@useobject{currentmarker}{}%
\end{pgfscope}%
\begin{pgfscope}%
\pgfsys@transformshift{1.330006in}{4.307293in}%
\pgfsys@useobject{currentmarker}{}%
\end{pgfscope}%
\begin{pgfscope}%
\pgfsys@transformshift{1.349959in}{4.122816in}%
\pgfsys@useobject{currentmarker}{}%
\end{pgfscope}%
\begin{pgfscope}%
\pgfsys@transformshift{1.368033in}{4.057894in}%
\pgfsys@useobject{currentmarker}{}%
\end{pgfscope}%
\begin{pgfscope}%
\pgfsys@transformshift{1.386577in}{4.038894in}%
\pgfsys@useobject{currentmarker}{}%
\end{pgfscope}%
\begin{pgfscope}%
\pgfsys@transformshift{1.405356in}{4.035276in}%
\pgfsys@useobject{currentmarker}{}%
\end{pgfscope}%
\begin{pgfscope}%
\pgfsys@transformshift{1.429532in}{4.038579in}%
\pgfsys@useobject{currentmarker}{}%
\end{pgfscope}%
\begin{pgfscope}%
\pgfsys@transformshift{1.444086in}{4.045986in}%
\pgfsys@useobject{currentmarker}{}%
\end{pgfscope}%
\begin{pgfscope}%
\pgfsys@transformshift{1.462394in}{4.079402in}%
\pgfsys@useobject{currentmarker}{}%
\end{pgfscope}%
\begin{pgfscope}%
\pgfsys@transformshift{1.484693in}{4.239664in}%
\pgfsys@useobject{currentmarker}{}%
\end{pgfscope}%
\begin{pgfscope}%
\pgfsys@transformshift{1.503706in}{4.419982in}%
\pgfsys@useobject{currentmarker}{}%
\end{pgfscope}%
\begin{pgfscope}%
\pgfsys@transformshift{1.523190in}{4.204423in}%
\pgfsys@useobject{currentmarker}{}%
\end{pgfscope}%
\begin{pgfscope}%
\pgfsys@transformshift{1.541498in}{4.409724in}%
\pgfsys@useobject{currentmarker}{}%
\end{pgfscope}%
\begin{pgfscope}%
\pgfsys@transformshift{1.563328in}{4.398999in}%
\pgfsys@useobject{currentmarker}{}%
\end{pgfscope}%
\begin{pgfscope}%
\pgfsys@transformshift{1.579054in}{4.262309in}%
\pgfsys@useobject{currentmarker}{}%
\end{pgfscope}%
\begin{pgfscope}%
\pgfsys@transformshift{1.598067in}{4.101207in}%
\pgfsys@useobject{currentmarker}{}%
\end{pgfscope}%
\begin{pgfscope}%
\pgfsys@transformshift{1.616377in}{4.050331in}%
\pgfsys@useobject{currentmarker}{}%
\end{pgfscope}%
\begin{pgfscope}%
\pgfsys@transformshift{1.638207in}{4.037094in}%
\pgfsys@useobject{currentmarker}{}%
\end{pgfscope}%
\begin{pgfscope}%
\pgfsys@transformshift{1.654638in}{4.035047in}%
\pgfsys@useobject{currentmarker}{}%
\end{pgfscope}%
\begin{pgfscope}%
\pgfsys@transformshift{1.676703in}{4.038411in}%
\pgfsys@useobject{currentmarker}{}%
\end{pgfscope}%
\begin{pgfscope}%
\pgfsys@transformshift{1.693837in}{4.049133in}%
\pgfsys@useobject{currentmarker}{}%
\end{pgfscope}%
\begin{pgfscope}%
\pgfsys@transformshift{1.714258in}{4.082035in}%
\pgfsys@useobject{currentmarker}{}%
\end{pgfscope}%
\begin{pgfscope}%
\pgfsys@transformshift{1.732568in}{4.212233in}%
\pgfsys@useobject{currentmarker}{}%
\end{pgfscope}%
\begin{pgfscope}%
\pgfsys@transformshift{1.751816in}{4.399336in}%
\pgfsys@useobject{currentmarker}{}%
\end{pgfscope}%
\begin{pgfscope}%
\pgfsys@transformshift{1.770359in}{4.410667in}%
\pgfsys@useobject{currentmarker}{}%
\end{pgfscope}%
\begin{pgfscope}%
\pgfsys@transformshift{1.789606in}{4.293762in}%
\pgfsys@useobject{currentmarker}{}%
\end{pgfscope}%
\begin{pgfscope}%
\pgfsys@transformshift{1.807682in}{4.116932in}%
\pgfsys@useobject{currentmarker}{}%
\end{pgfscope}%
\begin{pgfscope}%
\pgfsys@transformshift{1.832093in}{4.045982in}%
\pgfsys@useobject{currentmarker}{}%
\end{pgfscope}%
\begin{pgfscope}%
\pgfsys@transformshift{1.849697in}{4.038232in}%
\pgfsys@useobject{currentmarker}{}%
\end{pgfscope}%
\begin{pgfscope}%
\pgfsys@transformshift{1.868007in}{4.035235in}%
\pgfsys@useobject{currentmarker}{}%
\end{pgfscope}%
\begin{pgfscope}%
\pgfsys@transformshift{1.885612in}{4.035068in}%
\pgfsys@useobject{currentmarker}{}%
\end{pgfscope}%
\begin{pgfscope}%
\pgfsys@transformshift{1.907206in}{4.042126in}%
\pgfsys@useobject{currentmarker}{}%
\end{pgfscope}%
\begin{pgfscope}%
\pgfsys@transformshift{1.929741in}{4.074706in}%
\pgfsys@useobject{currentmarker}{}%
\end{pgfscope}%
\begin{pgfscope}%
\pgfsys@transformshift{1.945701in}{4.135074in}%
\pgfsys@useobject{currentmarker}{}%
\end{pgfscope}%
\begin{pgfscope}%
\pgfsys@transformshift{1.963777in}{4.332474in}%
\pgfsys@useobject{currentmarker}{}%
\end{pgfscope}%
\begin{pgfscope}%
\pgfsys@transformshift{1.983025in}{4.416375in}%
\pgfsys@useobject{currentmarker}{}%
\end{pgfscope}%
\begin{pgfscope}%
\pgfsys@transformshift{2.001098in}{4.331776in}%
\pgfsys@useobject{currentmarker}{}%
\end{pgfscope}%
\begin{pgfscope}%
\pgfsys@transformshift{2.020346in}{4.182242in}%
\pgfsys@useobject{currentmarker}{}%
\end{pgfscope}%
\begin{pgfscope}%
\pgfsys@transformshift{2.038420in}{4.104425in}%
\pgfsys@useobject{currentmarker}{}%
\end{pgfscope}%
\begin{pgfscope}%
\pgfsys@transformshift{2.060484in}{4.049382in}%
\pgfsys@useobject{currentmarker}{}%
\end{pgfscope}%
\begin{pgfscope}%
\pgfsys@transformshift{2.079263in}{4.038259in}%
\pgfsys@useobject{currentmarker}{}%
\end{pgfscope}%
\begin{pgfscope}%
\pgfsys@transformshift{2.097571in}{4.034726in}%
\pgfsys@useobject{currentmarker}{}%
\end{pgfscope}%
\begin{pgfscope}%
\pgfsys@transformshift{2.116350in}{4.036568in}%
\pgfsys@useobject{currentmarker}{}%
\end{pgfscope}%
\begin{pgfscope}%
\pgfsys@transformshift{2.137240in}{4.045804in}%
\pgfsys@useobject{currentmarker}{}%
\end{pgfscope}%
\begin{pgfscope}%
\pgfsys@transformshift{2.155080in}{4.069182in}%
\pgfsys@useobject{currentmarker}{}%
\end{pgfscope}%
\begin{pgfscope}%
\pgfsys@transformshift{2.172921in}{4.151510in}%
\pgfsys@useobject{currentmarker}{}%
\end{pgfscope}%
\begin{pgfscope}%
\pgfsys@transformshift{2.197097in}{4.390666in}%
\pgfsys@useobject{currentmarker}{}%
\end{pgfscope}%
\begin{pgfscope}%
\pgfsys@transformshift{2.210945in}{4.411441in}%
\pgfsys@useobject{currentmarker}{}%
\end{pgfscope}%
\begin{pgfscope}%
\pgfsys@transformshift{2.232541in}{4.338261in}%
\pgfsys@useobject{currentmarker}{}%
\end{pgfscope}%
\begin{pgfscope}%
\pgfsys@transformshift{2.250380in}{4.194965in}%
\pgfsys@useobject{currentmarker}{}%
\end{pgfscope}%
\begin{pgfscope}%
\pgfsys@transformshift{2.271507in}{4.078201in}%
\pgfsys@useobject{currentmarker}{}%
\end{pgfscope}%
\begin{pgfscope}%
\pgfsys@transformshift{2.290050in}{4.044893in}%
\pgfsys@useobject{currentmarker}{}%
\end{pgfscope}%
\begin{pgfscope}%
\pgfsys@transformshift{2.309766in}{4.036417in}%
\pgfsys@useobject{currentmarker}{}%
\end{pgfscope}%
\begin{pgfscope}%
\pgfsys@transformshift{2.328311in}{4.034805in}%
\pgfsys@useobject{currentmarker}{}%
\end{pgfscope}%
\begin{pgfscope}%
\pgfsys@transformshift{2.349201in}{4.036605in}%
\pgfsys@useobject{currentmarker}{}%
\end{pgfscope}%
\begin{pgfscope}%
\pgfsys@transformshift{2.367277in}{4.042886in}%
\pgfsys@useobject{currentmarker}{}%
\end{pgfscope}%
\begin{pgfscope}%
\pgfsys@transformshift{2.384880in}{4.063835in}%
\pgfsys@useobject{currentmarker}{}%
\end{pgfscope}%
\begin{pgfscope}%
\pgfsys@transformshift{2.403659in}{4.079059in}%
\pgfsys@useobject{currentmarker}{}%
\end{pgfscope}%
\begin{pgfscope}%
\pgfsys@transformshift{2.424315in}{4.120599in}%
\pgfsys@useobject{currentmarker}{}%
\end{pgfscope}%
\begin{pgfscope}%
\pgfsys@transformshift{2.441920in}{4.034988in}%
\pgfsys@useobject{currentmarker}{}%
\end{pgfscope}%
\begin{pgfscope}%
\pgfsys@transformshift{2.462576in}{4.036747in}%
\pgfsys@useobject{currentmarker}{}%
\end{pgfscope}%
\begin{pgfscope}%
\pgfsys@transformshift{2.480886in}{4.049505in}%
\pgfsys@useobject{currentmarker}{}%
\end{pgfscope}%
\begin{pgfscope}%
\pgfsys@transformshift{2.498490in}{4.103953in}%
\pgfsys@useobject{currentmarker}{}%
\end{pgfscope}%
\begin{pgfscope}%
\pgfsys@transformshift{2.520555in}{4.252770in}%
\pgfsys@useobject{currentmarker}{}%
\end{pgfscope}%
\begin{pgfscope}%
\pgfsys@transformshift{2.540975in}{4.416178in}%
\pgfsys@useobject{currentmarker}{}%
\end{pgfscope}%
\begin{pgfscope}%
\pgfsys@transformshift{2.559050in}{4.350817in}%
\pgfsys@useobject{currentmarker}{}%
\end{pgfscope}%
\begin{pgfscope}%
\pgfsys@transformshift{2.576419in}{4.173661in}%
\pgfsys@useobject{currentmarker}{}%
\end{pgfscope}%
\begin{pgfscope}%
\pgfsys@transformshift{2.597546in}{4.060890in}%
\pgfsys@useobject{currentmarker}{}%
\end{pgfscope}%
\begin{pgfscope}%
\pgfsys@transformshift{2.615619in}{4.046109in}%
\pgfsys@useobject{currentmarker}{}%
\end{pgfscope}%
\begin{pgfscope}%
\pgfsys@transformshift{2.636276in}{4.035934in}%
\pgfsys@useobject{currentmarker}{}%
\end{pgfscope}%
\begin{pgfscope}%
\pgfsys@transformshift{2.655054in}{4.034817in}%
\pgfsys@useobject{currentmarker}{}%
\end{pgfscope}%
\begin{pgfscope}%
\pgfsys@transformshift{2.675476in}{4.041156in}%
\pgfsys@useobject{currentmarker}{}%
\end{pgfscope}%
\begin{pgfscope}%
\pgfsys@transformshift{2.693550in}{4.058069in}%
\pgfsys@useobject{currentmarker}{}%
\end{pgfscope}%
\begin{pgfscope}%
\pgfsys@transformshift{2.711389in}{4.114017in}%
\pgfsys@useobject{currentmarker}{}%
\end{pgfscope}%
\begin{pgfscope}%
\pgfsys@transformshift{2.732750in}{4.295572in}%
\pgfsys@useobject{currentmarker}{}%
\end{pgfscope}%
\begin{pgfscope}%
\pgfsys@transformshift{2.750355in}{4.411759in}%
\pgfsys@useobject{currentmarker}{}%
\end{pgfscope}%
\begin{pgfscope}%
\pgfsys@transformshift{2.770072in}{4.372724in}%
\pgfsys@useobject{currentmarker}{}%
\end{pgfscope}%
\begin{pgfscope}%
\pgfsys@transformshift{2.790962in}{4.191981in}%
\pgfsys@useobject{currentmarker}{}%
\end{pgfscope}%
\begin{pgfscope}%
\pgfsys@transformshift{2.806219in}{4.099461in}%
\pgfsys@useobject{currentmarker}{}%
\end{pgfscope}%
\begin{pgfscope}%
\pgfsys@transformshift{2.828284in}{4.051006in}%
\pgfsys@useobject{currentmarker}{}%
\end{pgfscope}%
\begin{pgfscope}%
\pgfsys@transformshift{2.846125in}{4.038958in}%
\pgfsys@useobject{currentmarker}{}%
\end{pgfscope}%
\begin{pgfscope}%
\pgfsys@transformshift{2.864198in}{4.035062in}%
\pgfsys@useobject{currentmarker}{}%
\end{pgfscope}%
\begin{pgfscope}%
\pgfsys@transformshift{2.888375in}{4.037540in}%
\pgfsys@useobject{currentmarker}{}%
\end{pgfscope}%
\begin{pgfscope}%
\pgfsys@transformshift{2.902459in}{4.044590in}%
\pgfsys@useobject{currentmarker}{}%
\end{pgfscope}%
\begin{pgfscope}%
\pgfsys@transformshift{2.924524in}{4.078893in}%
\pgfsys@useobject{currentmarker}{}%
\end{pgfscope}%
\begin{pgfscope}%
\pgfsys@transformshift{2.941894in}{4.146182in}%
\pgfsys@useobject{currentmarker}{}%
\end{pgfscope}%
\begin{pgfscope}%
\pgfsys@transformshift{2.963019in}{4.357100in}%
\pgfsys@useobject{currentmarker}{}%
\end{pgfscope}%
\begin{pgfscope}%
\pgfsys@transformshift{2.983910in}{4.410573in}%
\pgfsys@useobject{currentmarker}{}%
\end{pgfscope}%
\begin{pgfscope}%
\pgfsys@transformshift{2.998463in}{4.350569in}%
\pgfsys@useobject{currentmarker}{}%
\end{pgfscope}%
\begin{pgfscope}%
\pgfsys@transformshift{3.019823in}{4.168819in}%
\pgfsys@useobject{currentmarker}{}%
\end{pgfscope}%
\begin{pgfscope}%
\pgfsys@transformshift{3.038367in}{4.078134in}%
\pgfsys@useobject{currentmarker}{}%
\end{pgfscope}%
\begin{pgfscope}%
\pgfsys@transformshift{3.062780in}{4.042547in}%
\pgfsys@useobject{currentmarker}{}%
\end{pgfscope}%
\begin{pgfscope}%
\pgfsys@transformshift{3.077097in}{4.038023in}%
\pgfsys@useobject{currentmarker}{}%
\end{pgfscope}%
\begin{pgfscope}%
\pgfsys@transformshift{3.098693in}{4.103363in}%
\pgfsys@useobject{currentmarker}{}%
\end{pgfscope}%
\begin{pgfscope}%
\pgfsys@transformshift{3.116532in}{4.053349in}%
\pgfsys@useobject{currentmarker}{}%
\end{pgfscope}%
\begin{pgfscope}%
\pgfsys@transformshift{3.130616in}{4.039677in}%
\pgfsys@useobject{currentmarker}{}%
\end{pgfscope}%
\begin{pgfscope}%
\pgfsys@transformshift{3.155967in}{4.035456in}%
\pgfsys@useobject{currentmarker}{}%
\end{pgfscope}%
\begin{pgfscope}%
\pgfsys@transformshift{3.173337in}{4.035913in}%
\pgfsys@useobject{currentmarker}{}%
\end{pgfscope}%
\begin{pgfscope}%
\pgfsys@transformshift{3.190942in}{4.042124in}%
\pgfsys@useobject{currentmarker}{}%
\end{pgfscope}%
\begin{pgfscope}%
\pgfsys@transformshift{3.212536in}{4.064360in}%
\pgfsys@useobject{currentmarker}{}%
\end{pgfscope}%
\begin{pgfscope}%
\pgfsys@transformshift{3.232489in}{4.135898in}%
\pgfsys@useobject{currentmarker}{}%
\end{pgfscope}%
\begin{pgfscope}%
\pgfsys@transformshift{3.250797in}{4.332966in}%
\pgfsys@useobject{currentmarker}{}%
\end{pgfscope}%
\begin{pgfscope}%
\pgfsys@transformshift{3.268402in}{4.420004in}%
\pgfsys@useobject{currentmarker}{}%
\end{pgfscope}%
\begin{pgfscope}%
\pgfsys@transformshift{3.288120in}{4.376125in}%
\pgfsys@useobject{currentmarker}{}%
\end{pgfscope}%
\begin{pgfscope}%
\pgfsys@transformshift{3.308071in}{4.273503in}%
\pgfsys@useobject{currentmarker}{}%
\end{pgfscope}%
\begin{pgfscope}%
\pgfsys@transformshift{3.325676in}{4.128796in}%
\pgfsys@useobject{currentmarker}{}%
\end{pgfscope}%
\begin{pgfscope}%
\pgfsys@transformshift{3.346566in}{4.060571in}%
\pgfsys@useobject{currentmarker}{}%
\end{pgfscope}%
\begin{pgfscope}%
\pgfsys@transformshift{3.364171in}{4.043140in}%
\pgfsys@useobject{currentmarker}{}%
\end{pgfscope}%
\begin{pgfscope}%
\pgfsys@transformshift{3.388349in}{4.035559in}%
\pgfsys@useobject{currentmarker}{}%
\end{pgfscope}%
\begin{pgfscope}%
\pgfsys@transformshift{3.403606in}{4.035449in}%
\pgfsys@useobject{currentmarker}{}%
\end{pgfscope}%
\begin{pgfscope}%
\pgfsys@transformshift{3.424966in}{4.041235in}%
\pgfsys@useobject{currentmarker}{}%
\end{pgfscope}%
\begin{pgfscope}%
\pgfsys@transformshift{3.443041in}{4.052797in}%
\pgfsys@useobject{currentmarker}{}%
\end{pgfscope}%
\begin{pgfscope}%
\pgfsys@transformshift{3.460880in}{4.079012in}%
\pgfsys@useobject{currentmarker}{}%
\end{pgfscope}%
\begin{pgfscope}%
\pgfsys@transformshift{3.484822in}{4.226944in}%
\pgfsys@useobject{currentmarker}{}%
\end{pgfscope}%
\begin{pgfscope}%
\pgfsys@transformshift{3.499610in}{4.389914in}%
\pgfsys@useobject{currentmarker}{}%
\end{pgfscope}%
\begin{pgfscope}%
\pgfsys@transformshift{3.517920in}{4.423301in}%
\pgfsys@useobject{currentmarker}{}%
\end{pgfscope}%
\begin{pgfscope}%
\pgfsys@transformshift{3.539045in}{4.342040in}%
\pgfsys@useobject{currentmarker}{}%
\end{pgfscope}%
\begin{pgfscope}%
\pgfsys@transformshift{3.556415in}{4.203880in}%
\pgfsys@useobject{currentmarker}{}%
\end{pgfscope}%
\begin{pgfscope}%
\pgfsys@transformshift{3.577540in}{4.083327in}%
\pgfsys@useobject{currentmarker}{}%
\end{pgfscope}%
\begin{pgfscope}%
\pgfsys@transformshift{3.597728in}{4.051212in}%
\pgfsys@useobject{currentmarker}{}%
\end{pgfscope}%
\begin{pgfscope}%
\pgfsys@transformshift{3.617679in}{4.038734in}%
\pgfsys@useobject{currentmarker}{}%
\end{pgfscope}%
\begin{pgfscope}%
\pgfsys@transformshift{3.633641in}{4.066768in}%
\pgfsys@useobject{currentmarker}{}%
\end{pgfscope}%
\begin{pgfscope}%
\pgfsys@transformshift{3.652888in}{4.043116in}%
\pgfsys@useobject{currentmarker}{}%
\end{pgfscope}%
\begin{pgfscope}%
\pgfsys@transformshift{3.673544in}{4.037411in}%
\pgfsys@useobject{currentmarker}{}%
\end{pgfscope}%
\begin{pgfscope}%
\pgfsys@transformshift{3.692089in}{4.035424in}%
\pgfsys@useobject{currentmarker}{}%
\end{pgfscope}%
\begin{pgfscope}%
\pgfsys@transformshift{3.713448in}{4.040385in}%
\pgfsys@useobject{currentmarker}{}%
\end{pgfscope}%
\begin{pgfscope}%
\pgfsys@transformshift{3.731053in}{4.052655in}%
\pgfsys@useobject{currentmarker}{}%
\end{pgfscope}%
\begin{pgfscope}%
\pgfsys@transformshift{3.748658in}{4.088673in}%
\pgfsys@useobject{currentmarker}{}%
\end{pgfscope}%
\begin{pgfscope}%
\pgfsys@transformshift{3.769314in}{4.215585in}%
\pgfsys@useobject{currentmarker}{}%
\end{pgfscope}%
\begin{pgfscope}%
\pgfsys@transformshift{3.789736in}{4.384157in}%
\pgfsys@useobject{currentmarker}{}%
\end{pgfscope}%
\begin{pgfscope}%
\pgfsys@transformshift{3.807106in}{4.435335in}%
\pgfsys@useobject{currentmarker}{}%
\end{pgfscope}%
\begin{pgfscope}%
\pgfsys@transformshift{3.828466in}{4.369609in}%
\pgfsys@useobject{currentmarker}{}%
\end{pgfscope}%
\begin{pgfscope}%
\pgfsys@transformshift{3.847010in}{4.246770in}%
\pgfsys@useobject{currentmarker}{}%
\end{pgfscope}%
\begin{pgfscope}%
\pgfsys@transformshift{3.866258in}{4.137547in}%
\pgfsys@useobject{currentmarker}{}%
\end{pgfscope}%
\begin{pgfscope}%
\pgfsys@transformshift{3.884097in}{4.071657in}%
\pgfsys@useobject{currentmarker}{}%
\end{pgfscope}%
\begin{pgfscope}%
\pgfsys@transformshift{3.903815in}{4.046143in}%
\pgfsys@useobject{currentmarker}{}%
\end{pgfscope}%
\begin{pgfscope}%
\pgfsys@transformshift{3.923297in}{4.036963in}%
\pgfsys@useobject{currentmarker}{}%
\end{pgfscope}%
\begin{pgfscope}%
\pgfsys@transformshift{3.941371in}{4.035998in}%
\pgfsys@useobject{currentmarker}{}%
\end{pgfscope}%
\begin{pgfscope}%
\pgfsys@transformshift{3.962732in}{4.041436in}%
\pgfsys@useobject{currentmarker}{}%
\end{pgfscope}%
\begin{pgfscope}%
\pgfsys@transformshift{3.979398in}{4.052895in}%
\pgfsys@useobject{currentmarker}{}%
\end{pgfscope}%
\begin{pgfscope}%
\pgfsys@transformshift{3.998411in}{4.072730in}%
\pgfsys@useobject{currentmarker}{}%
\end{pgfscope}%
\begin{pgfscope}%
\pgfsys@transformshift{4.017893in}{4.140246in}%
\pgfsys@useobject{currentmarker}{}%
\end{pgfscope}%
\begin{pgfscope}%
\pgfsys@transformshift{4.037375in}{4.091043in}%
\pgfsys@useobject{currentmarker}{}%
\end{pgfscope}%
\begin{pgfscope}%
\pgfsys@transformshift{4.058971in}{4.232958in}%
\pgfsys@useobject{currentmarker}{}%
\end{pgfscope}%
\begin{pgfscope}%
\pgfsys@transformshift{4.076105in}{4.406936in}%
\pgfsys@useobject{currentmarker}{}%
\end{pgfscope}%
\begin{pgfscope}%
\pgfsys@transformshift{4.094415in}{4.446531in}%
\pgfsys@useobject{currentmarker}{}%
\end{pgfscope}%
\begin{pgfscope}%
\pgfsys@transformshift{4.115540in}{4.395279in}%
\pgfsys@useobject{currentmarker}{}%
\end{pgfscope}%
\begin{pgfscope}%
\pgfsys@transformshift{4.134084in}{4.266940in}%
\pgfsys@useobject{currentmarker}{}%
\end{pgfscope}%
\begin{pgfscope}%
\pgfsys@transformshift{4.155444in}{4.107699in}%
\pgfsys@useobject{currentmarker}{}%
\end{pgfscope}%
\begin{pgfscope}%
\pgfsys@transformshift{4.173519in}{4.063638in}%
\pgfsys@useobject{currentmarker}{}%
\end{pgfscope}%
\begin{pgfscope}%
\pgfsys@transformshift{4.193939in}{4.043149in}%
\pgfsys@useobject{currentmarker}{}%
\end{pgfscope}%
\begin{pgfscope}%
\pgfsys@transformshift{4.212952in}{4.037100in}%
\pgfsys@useobject{currentmarker}{}%
\end{pgfscope}%
\begin{pgfscope}%
\pgfsys@transformshift{4.230557in}{4.037315in}%
\pgfsys@useobject{currentmarker}{}%
\end{pgfscope}%
\begin{pgfscope}%
\pgfsys@transformshift{4.251684in}{4.044255in}%
\pgfsys@useobject{currentmarker}{}%
\end{pgfscope}%
\begin{pgfscope}%
\pgfsys@transformshift{4.269289in}{4.059867in}%
\pgfsys@useobject{currentmarker}{}%
\end{pgfscope}%
\begin{pgfscope}%
\pgfsys@transformshift{4.286894in}{4.072682in}%
\pgfsys@useobject{currentmarker}{}%
\end{pgfscope}%
\begin{pgfscope}%
\pgfsys@transformshift{4.310365in}{4.162593in}%
\pgfsys@useobject{currentmarker}{}%
\end{pgfscope}%
\begin{pgfscope}%
\pgfsys@transformshift{4.324449in}{4.313771in}%
\pgfsys@useobject{currentmarker}{}%
\end{pgfscope}%
\begin{pgfscope}%
\pgfsys@transformshift{4.344166in}{4.444133in}%
\pgfsys@useobject{currentmarker}{}%
\end{pgfscope}%
\begin{pgfscope}%
\pgfsys@transformshift{4.365762in}{4.451549in}%
\pgfsys@useobject{currentmarker}{}%
\end{pgfscope}%
\begin{pgfscope}%
\pgfsys@transformshift{4.383601in}{4.422413in}%
\pgfsys@useobject{currentmarker}{}%
\end{pgfscope}%
\begin{pgfscope}%
\pgfsys@transformshift{4.400971in}{4.313122in}%
\pgfsys@useobject{currentmarker}{}%
\end{pgfscope}%
\begin{pgfscope}%
\pgfsys@transformshift{4.423036in}{4.139045in}%
\pgfsys@useobject{currentmarker}{}%
\end{pgfscope}%
\begin{pgfscope}%
\pgfsys@transformshift{4.440641in}{4.074394in}%
\pgfsys@useobject{currentmarker}{}%
\end{pgfscope}%
\begin{pgfscope}%
\pgfsys@transformshift{4.458949in}{4.050947in}%
\pgfsys@useobject{currentmarker}{}%
\end{pgfscope}%
\begin{pgfscope}%
\pgfsys@transformshift{4.479605in}{4.051851in}%
\pgfsys@useobject{currentmarker}{}%
\end{pgfscope}%
\begin{pgfscope}%
\pgfsys@transformshift{4.480076in}{4.052923in}%
\pgfsys@useobject{currentmarker}{}%
\end{pgfscope}%
\begin{pgfscope}%
\pgfsys@transformshift{4.473737in}{4.063732in}%
\pgfsys@useobject{currentmarker}{}%
\end{pgfscope}%
\begin{pgfscope}%
\pgfsys@transformshift{4.455428in}{4.147667in}%
\pgfsys@useobject{currentmarker}{}%
\end{pgfscope}%
\begin{pgfscope}%
\pgfsys@transformshift{4.436415in}{4.348711in}%
\pgfsys@useobject{currentmarker}{}%
\end{pgfscope}%
\begin{pgfscope}%
\pgfsys@transformshift{4.418107in}{4.453897in}%
\pgfsys@useobject{currentmarker}{}%
\end{pgfscope}%
\begin{pgfscope}%
\pgfsys@transformshift{4.396746in}{4.398869in}%
\pgfsys@useobject{currentmarker}{}%
\end{pgfscope}%
\begin{pgfscope}%
\pgfsys@transformshift{4.378203in}{4.144507in}%
\pgfsys@useobject{currentmarker}{}%
\end{pgfscope}%
\begin{pgfscope}%
\pgfsys@transformshift{4.359424in}{4.062089in}%
\pgfsys@useobject{currentmarker}{}%
\end{pgfscope}%
\begin{pgfscope}%
\pgfsys@transformshift{4.341351in}{4.040617in}%
\pgfsys@useobject{currentmarker}{}%
\end{pgfscope}%
\begin{pgfscope}%
\pgfsys@transformshift{4.319286in}{4.036831in}%
\pgfsys@useobject{currentmarker}{}%
\end{pgfscope}%
\begin{pgfscope}%
\pgfsys@transformshift{4.299568in}{4.046700in}%
\pgfsys@useobject{currentmarker}{}%
\end{pgfscope}%
\begin{pgfscope}%
\pgfsys@transformshift{4.282197in}{4.078808in}%
\pgfsys@useobject{currentmarker}{}%
\end{pgfscope}%
\begin{pgfscope}%
\pgfsys@transformshift{4.261307in}{4.230117in}%
\pgfsys@useobject{currentmarker}{}%
\end{pgfscope}%
\begin{pgfscope}%
\pgfsys@transformshift{4.243233in}{4.405237in}%
\pgfsys@useobject{currentmarker}{}%
\end{pgfscope}%
\begin{pgfscope}%
\pgfsys@transformshift{4.224454in}{4.445234in}%
\pgfsys@useobject{currentmarker}{}%
\end{pgfscope}%
\begin{pgfscope}%
\pgfsys@transformshift{4.204972in}{4.252316in}%
\pgfsys@useobject{currentmarker}{}%
\end{pgfscope}%
\begin{pgfscope}%
\pgfsys@transformshift{4.187602in}{4.095604in}%
\pgfsys@useobject{currentmarker}{}%
\end{pgfscope}%
\begin{pgfscope}%
\pgfsys@transformshift{4.166711in}{4.046549in}%
\pgfsys@useobject{currentmarker}{}%
\end{pgfscope}%
\begin{pgfscope}%
\pgfsys@transformshift{4.145821in}{4.036303in}%
\pgfsys@useobject{currentmarker}{}%
\end{pgfscope}%
\begin{pgfscope}%
\pgfsys@transformshift{4.128919in}{4.038048in}%
\pgfsys@useobject{currentmarker}{}%
\end{pgfscope}%
\begin{pgfscope}%
\pgfsys@transformshift{4.107794in}{4.054675in}%
\pgfsys@useobject{currentmarker}{}%
\end{pgfscope}%
\begin{pgfscope}%
\pgfsys@transformshift{4.092537in}{4.106259in}%
\pgfsys@useobject{currentmarker}{}%
\end{pgfscope}%
\begin{pgfscope}%
\pgfsys@transformshift{4.070707in}{4.319529in}%
\pgfsys@useobject{currentmarker}{}%
\end{pgfscope}%
\begin{pgfscope}%
\pgfsys@transformshift{4.051928in}{4.429931in}%
\pgfsys@useobject{currentmarker}{}%
\end{pgfscope}%
\begin{pgfscope}%
\pgfsys@transformshift{4.032446in}{4.403414in}%
\pgfsys@useobject{currentmarker}{}%
\end{pgfscope}%
\begin{pgfscope}%
\pgfsys@transformshift{4.012493in}{4.135207in}%
\pgfsys@useobject{currentmarker}{}%
\end{pgfscope}%
\begin{pgfscope}%
\pgfsys@transformshift{3.995359in}{4.061095in}%
\pgfsys@useobject{currentmarker}{}%
\end{pgfscope}%
\begin{pgfscope}%
\pgfsys@transformshift{3.974467in}{4.039425in}%
\pgfsys@useobject{currentmarker}{}%
\end{pgfscope}%
\begin{pgfscope}%
\pgfsys@transformshift{3.950759in}{4.035582in}%
\pgfsys@useobject{currentmarker}{}%
\end{pgfscope}%
\begin{pgfscope}%
\pgfsys@transformshift{3.937146in}{4.040050in}%
\pgfsys@useobject{currentmarker}{}%
\end{pgfscope}%
\begin{pgfscope}%
\pgfsys@transformshift{3.915786in}{4.067150in}%
\pgfsys@useobject{currentmarker}{}%
\end{pgfscope}%
\begin{pgfscope}%
\pgfsys@transformshift{3.898650in}{4.163965in}%
\pgfsys@useobject{currentmarker}{}%
\end{pgfscope}%
\begin{pgfscope}%
\pgfsys@transformshift{3.878229in}{4.369968in}%
\pgfsys@useobject{currentmarker}{}%
\end{pgfscope}%
\begin{pgfscope}%
\pgfsys@transformshift{3.860389in}{4.432046in}%
\pgfsys@useobject{currentmarker}{}%
\end{pgfscope}%
\begin{pgfscope}%
\pgfsys@transformshift{3.840202in}{4.272807in}%
\pgfsys@useobject{currentmarker}{}%
\end{pgfscope}%
\begin{pgfscope}%
\pgfsys@transformshift{3.820017in}{4.087557in}%
\pgfsys@useobject{currentmarker}{}%
\end{pgfscope}%
\begin{pgfscope}%
\pgfsys@transformshift{3.802412in}{4.053879in}%
\pgfsys@useobject{currentmarker}{}%
\end{pgfscope}%
\begin{pgfscope}%
\pgfsys@transformshift{3.781519in}{4.037418in}%
\pgfsys@useobject{currentmarker}{}%
\end{pgfscope}%
\begin{pgfscope}%
\pgfsys@transformshift{3.763680in}{4.035417in}%
\pgfsys@useobject{currentmarker}{}%
\end{pgfscope}%
\begin{pgfscope}%
\pgfsys@transformshift{3.744198in}{4.039286in}%
\pgfsys@useobject{currentmarker}{}%
\end{pgfscope}%
\begin{pgfscope}%
\pgfsys@transformshift{3.726124in}{4.050895in}%
\pgfsys@useobject{currentmarker}{}%
\end{pgfscope}%
\begin{pgfscope}%
\pgfsys@transformshift{3.706172in}{4.107054in}%
\pgfsys@useobject{currentmarker}{}%
\end{pgfscope}%
\begin{pgfscope}%
\pgfsys@transformshift{3.685515in}{4.307242in}%
\pgfsys@useobject{currentmarker}{}%
\end{pgfscope}%
\begin{pgfscope}%
\pgfsys@transformshift{3.667676in}{4.393214in}%
\pgfsys@useobject{currentmarker}{}%
\end{pgfscope}%
\begin{pgfscope}%
\pgfsys@transformshift{3.647489in}{4.420181in}%
\pgfsys@useobject{currentmarker}{}%
\end{pgfscope}%
\begin{pgfscope}%
\pgfsys@transformshift{3.631058in}{4.255737in}%
\pgfsys@useobject{currentmarker}{}%
\end{pgfscope}%
\begin{pgfscope}%
\pgfsys@transformshift{3.610402in}{4.094105in}%
\pgfsys@useobject{currentmarker}{}%
\end{pgfscope}%
\begin{pgfscope}%
\pgfsys@transformshift{3.588808in}{4.043793in}%
\pgfsys@useobject{currentmarker}{}%
\end{pgfscope}%
\begin{pgfscope}%
\pgfsys@transformshift{3.571672in}{4.038244in}%
\pgfsys@useobject{currentmarker}{}%
\end{pgfscope}%
\begin{pgfscope}%
\pgfsys@transformshift{3.547730in}{4.035144in}%
\pgfsys@useobject{currentmarker}{}%
\end{pgfscope}%
\begin{pgfscope}%
\pgfsys@transformshift{3.532942in}{4.037967in}%
\pgfsys@useobject{currentmarker}{}%
\end{pgfscope}%
\begin{pgfscope}%
\pgfsys@transformshift{3.515338in}{4.048831in}%
\pgfsys@useobject{currentmarker}{}%
\end{pgfscope}%
\begin{pgfscope}%
\pgfsys@transformshift{3.495150in}{4.092701in}%
\pgfsys@useobject{currentmarker}{}%
\end{pgfscope}%
\begin{pgfscope}%
\pgfsys@transformshift{3.477077in}{4.234106in}%
\pgfsys@useobject{currentmarker}{}%
\end{pgfscope}%
\begin{pgfscope}%
\pgfsys@transformshift{3.455952in}{4.400795in}%
\pgfsys@useobject{currentmarker}{}%
\end{pgfscope}%
\begin{pgfscope}%
\pgfsys@transformshift{3.438347in}{4.415147in}%
\pgfsys@useobject{currentmarker}{}%
\end{pgfscope}%
\begin{pgfscope}%
\pgfsys@transformshift{3.417691in}{4.191839in}%
\pgfsys@useobject{currentmarker}{}%
\end{pgfscope}%
\begin{pgfscope}%
\pgfsys@transformshift{3.397035in}{4.081151in}%
\pgfsys@useobject{currentmarker}{}%
\end{pgfscope}%
\begin{pgfscope}%
\pgfsys@transformshift{3.379430in}{4.046816in}%
\pgfsys@useobject{currentmarker}{}%
\end{pgfscope}%
\begin{pgfscope}%
\pgfsys@transformshift{3.358068in}{4.036254in}%
\pgfsys@useobject{currentmarker}{}%
\end{pgfscope}%
\begin{pgfscope}%
\pgfsys@transformshift{3.340933in}{4.035262in}%
\pgfsys@useobject{currentmarker}{}%
\end{pgfscope}%
\begin{pgfscope}%
\pgfsys@transformshift{3.319104in}{4.039052in}%
\pgfsys@useobject{currentmarker}{}%
\end{pgfscope}%
\begin{pgfscope}%
\pgfsys@transformshift{3.301499in}{4.052502in}%
\pgfsys@useobject{currentmarker}{}%
\end{pgfscope}%
\begin{pgfscope}%
\pgfsys@transformshift{3.283660in}{4.101187in}%
\pgfsys@useobject{currentmarker}{}%
\end{pgfscope}%
\begin{pgfscope}%
\pgfsys@transformshift{3.262299in}{4.238931in}%
\pgfsys@useobject{currentmarker}{}%
\end{pgfscope}%
\begin{pgfscope}%
\pgfsys@transformshift{3.245399in}{4.353546in}%
\pgfsys@useobject{currentmarker}{}%
\end{pgfscope}%
\begin{pgfscope}%
\pgfsys@transformshift{3.224743in}{4.417191in}%
\pgfsys@useobject{currentmarker}{}%
\end{pgfscope}%
\begin{pgfscope}%
\pgfsys@transformshift{3.207842in}{4.336758in}%
\pgfsys@useobject{currentmarker}{}%
\end{pgfscope}%
\begin{pgfscope}%
\pgfsys@transformshift{3.186246in}{4.115095in}%
\pgfsys@useobject{currentmarker}{}%
\end{pgfscope}%
\begin{pgfscope}%
\pgfsys@transformshift{3.168407in}{4.061213in}%
\pgfsys@useobject{currentmarker}{}%
\end{pgfscope}%
\begin{pgfscope}%
\pgfsys@transformshift{3.147282in}{4.064182in}%
\pgfsys@useobject{currentmarker}{}%
\end{pgfscope}%
\begin{pgfscope}%
\pgfsys@transformshift{3.129911in}{4.052570in}%
\pgfsys@useobject{currentmarker}{}%
\end{pgfscope}%
\begin{pgfscope}%
\pgfsys@transformshift{3.111838in}{4.038752in}%
\pgfsys@useobject{currentmarker}{}%
\end{pgfscope}%
\begin{pgfscope}%
\pgfsys@transformshift{3.091885in}{4.034893in}%
\pgfsys@useobject{currentmarker}{}%
\end{pgfscope}%
\begin{pgfscope}%
\pgfsys@transformshift{3.070525in}{4.038286in}%
\pgfsys@useobject{currentmarker}{}%
\end{pgfscope}%
\begin{pgfscope}%
\pgfsys@transformshift{3.051278in}{4.053133in}%
\pgfsys@useobject{currentmarker}{}%
\end{pgfscope}%
\begin{pgfscope}%
\pgfsys@transformshift{3.032499in}{4.112596in}%
\pgfsys@useobject{currentmarker}{}%
\end{pgfscope}%
\begin{pgfscope}%
\pgfsys@transformshift{3.014425in}{4.261651in}%
\pgfsys@useobject{currentmarker}{}%
\end{pgfscope}%
\begin{pgfscope}%
\pgfsys@transformshift{2.995412in}{4.403762in}%
\pgfsys@useobject{currentmarker}{}%
\end{pgfscope}%
\begin{pgfscope}%
\pgfsys@transformshift{2.974521in}{4.381085in}%
\pgfsys@useobject{currentmarker}{}%
\end{pgfscope}%
\begin{pgfscope}%
\pgfsys@transformshift{2.958325in}{4.194918in}%
\pgfsys@useobject{currentmarker}{}%
\end{pgfscope}%
\begin{pgfscope}%
\pgfsys@transformshift{2.937669in}{4.073427in}%
\pgfsys@useobject{currentmarker}{}%
\end{pgfscope}%
\begin{pgfscope}%
\pgfsys@transformshift{2.918890in}{4.046820in}%
\pgfsys@useobject{currentmarker}{}%
\end{pgfscope}%
\begin{pgfscope}%
\pgfsys@transformshift{2.899642in}{4.036934in}%
\pgfsys@useobject{currentmarker}{}%
\end{pgfscope}%
\begin{pgfscope}%
\pgfsys@transformshift{2.877578in}{4.034812in}%
\pgfsys@useobject{currentmarker}{}%
\end{pgfscope}%
\begin{pgfscope}%
\pgfsys@transformshift{2.862321in}{4.037069in}%
\pgfsys@useobject{currentmarker}{}%
\end{pgfscope}%
\begin{pgfscope}%
\pgfsys@transformshift{2.841194in}{4.052141in}%
\pgfsys@useobject{currentmarker}{}%
\end{pgfscope}%
\begin{pgfscope}%
\pgfsys@transformshift{2.821712in}{4.069327in}%
\pgfsys@useobject{currentmarker}{}%
\end{pgfscope}%
\begin{pgfscope}%
\pgfsys@transformshift{2.802464in}{4.110755in}%
\pgfsys@useobject{currentmarker}{}%
\end{pgfscope}%
\begin{pgfscope}%
\pgfsys@transformshift{2.784391in}{4.253862in}%
\pgfsys@useobject{currentmarker}{}%
\end{pgfscope}%
\begin{pgfscope}%
\pgfsys@transformshift{2.765846in}{4.392730in}%
\pgfsys@useobject{currentmarker}{}%
\end{pgfscope}%
\begin{pgfscope}%
\pgfsys@transformshift{2.743782in}{4.369160in}%
\pgfsys@useobject{currentmarker}{}%
\end{pgfscope}%
\begin{pgfscope}%
\pgfsys@transformshift{2.723594in}{4.223965in}%
\pgfsys@useobject{currentmarker}{}%
\end{pgfscope}%
\begin{pgfscope}%
\pgfsys@transformshift{2.705286in}{4.097742in}%
\pgfsys@useobject{currentmarker}{}%
\end{pgfscope}%
\begin{pgfscope}%
\pgfsys@transformshift{2.686273in}{4.058819in}%
\pgfsys@useobject{currentmarker}{}%
\end{pgfscope}%
\begin{pgfscope}%
\pgfsys@transformshift{2.668199in}{4.040133in}%
\pgfsys@useobject{currentmarker}{}%
\end{pgfscope}%
\begin{pgfscope}%
\pgfsys@transformshift{2.649421in}{4.035320in}%
\pgfsys@useobject{currentmarker}{}%
\end{pgfscope}%
\begin{pgfscope}%
\pgfsys@transformshift{2.630407in}{4.035378in}%
\pgfsys@useobject{currentmarker}{}%
\end{pgfscope}%
\begin{pgfscope}%
\pgfsys@transformshift{2.615854in}{4.042235in}%
\pgfsys@useobject{currentmarker}{}%
\end{pgfscope}%
\begin{pgfscope}%
\pgfsys@transformshift{2.592146in}{4.065486in}%
\pgfsys@useobject{currentmarker}{}%
\end{pgfscope}%
\begin{pgfscope}%
\pgfsys@transformshift{2.573368in}{4.129975in}%
\pgfsys@useobject{currentmarker}{}%
\end{pgfscope}%
\begin{pgfscope}%
\pgfsys@transformshift{2.551069in}{4.309524in}%
\pgfsys@useobject{currentmarker}{}%
\end{pgfscope}%
\begin{pgfscope}%
\pgfsys@transformshift{2.536046in}{4.360094in}%
\pgfsys@useobject{currentmarker}{}%
\end{pgfscope}%
\begin{pgfscope}%
\pgfsys@transformshift{2.515156in}{4.402976in}%
\pgfsys@useobject{currentmarker}{}%
\end{pgfscope}%
\begin{pgfscope}%
\pgfsys@transformshift{2.494968in}{4.296641in}%
\pgfsys@useobject{currentmarker}{}%
\end{pgfscope}%
\begin{pgfscope}%
\pgfsys@transformshift{2.476660in}{4.157371in}%
\pgfsys@useobject{currentmarker}{}%
\end{pgfscope}%
\begin{pgfscope}%
\pgfsys@transformshift{2.454596in}{4.064556in}%
\pgfsys@useobject{currentmarker}{}%
\end{pgfscope}%
\begin{pgfscope}%
\pgfsys@transformshift{2.436051in}{4.042122in}%
\pgfsys@useobject{currentmarker}{}%
\end{pgfscope}%
\begin{pgfscope}%
\pgfsys@transformshift{2.417272in}{4.035639in}%
\pgfsys@useobject{currentmarker}{}%
\end{pgfscope}%
\begin{pgfscope}%
\pgfsys@transformshift{2.398964in}{4.034840in}%
\pgfsys@useobject{currentmarker}{}%
\end{pgfscope}%
\begin{pgfscope}%
\pgfsys@transformshift{2.380420in}{4.038205in}%
\pgfsys@useobject{currentmarker}{}%
\end{pgfscope}%
\begin{pgfscope}%
\pgfsys@transformshift{2.362581in}{4.050770in}%
\pgfsys@useobject{currentmarker}{}%
\end{pgfscope}%
\begin{pgfscope}%
\pgfsys@transformshift{2.337230in}{4.093116in}%
\pgfsys@useobject{currentmarker}{}%
\end{pgfscope}%
\begin{pgfscope}%
\pgfsys@transformshift{2.321503in}{4.186165in}%
\pgfsys@useobject{currentmarker}{}%
\end{pgfscope}%
\begin{pgfscope}%
\pgfsys@transformshift{2.302726in}{4.344675in}%
\pgfsys@useobject{currentmarker}{}%
\end{pgfscope}%
\begin{pgfscope}%
\pgfsys@transformshift{2.283947in}{4.408340in}%
\pgfsys@useobject{currentmarker}{}%
\end{pgfscope}%
\begin{pgfscope}%
\pgfsys@transformshift{2.262117in}{4.380686in}%
\pgfsys@useobject{currentmarker}{}%
\end{pgfscope}%
\begin{pgfscope}%
\pgfsys@transformshift{2.243338in}{4.176625in}%
\pgfsys@useobject{currentmarker}{}%
\end{pgfscope}%
\begin{pgfscope}%
\pgfsys@transformshift{2.227613in}{4.086814in}%
\pgfsys@useobject{currentmarker}{}%
\end{pgfscope}%
\begin{pgfscope}%
\pgfsys@transformshift{2.206251in}{4.048577in}%
\pgfsys@useobject{currentmarker}{}%
\end{pgfscope}%
\begin{pgfscope}%
\pgfsys@transformshift{2.187004in}{4.037893in}%
\pgfsys@useobject{currentmarker}{}%
\end{pgfscope}%
\begin{pgfscope}%
\pgfsys@transformshift{2.169164in}{4.034908in}%
\pgfsys@useobject{currentmarker}{}%
\end{pgfscope}%
\begin{pgfscope}%
\pgfsys@transformshift{2.151091in}{4.035825in}%
\pgfsys@useobject{currentmarker}{}%
\end{pgfscope}%
\begin{pgfscope}%
\pgfsys@transformshift{2.129729in}{4.042680in}%
\pgfsys@useobject{currentmarker}{}%
\end{pgfscope}%
\begin{pgfscope}%
\pgfsys@transformshift{2.110716in}{4.058095in}%
\pgfsys@useobject{currentmarker}{}%
\end{pgfscope}%
\begin{pgfscope}%
\pgfsys@transformshift{2.092174in}{4.104296in}%
\pgfsys@useobject{currentmarker}{}%
\end{pgfscope}%
\begin{pgfscope}%
\pgfsys@transformshift{2.073629in}{4.217711in}%
\pgfsys@useobject{currentmarker}{}%
\end{pgfscope}%
\begin{pgfscope}%
\pgfsys@transformshift{2.052739in}{4.365755in}%
\pgfsys@useobject{currentmarker}{}%
\end{pgfscope}%
\begin{pgfscope}%
\pgfsys@transformshift{2.033491in}{4.386529in}%
\pgfsys@useobject{currentmarker}{}%
\end{pgfscope}%
\begin{pgfscope}%
\pgfsys@transformshift{2.014243in}{4.084585in}%
\pgfsys@useobject{currentmarker}{}%
\end{pgfscope}%
\begin{pgfscope}%
\pgfsys@transformshift{1.996170in}{4.219029in}%
\pgfsys@useobject{currentmarker}{}%
\end{pgfscope}%
\begin{pgfscope}%
\pgfsys@transformshift{1.977391in}{4.313271in}%
\pgfsys@useobject{currentmarker}{}%
\end{pgfscope}%
\begin{pgfscope}%
\pgfsys@transformshift{1.955561in}{4.412396in}%
\pgfsys@useobject{currentmarker}{}%
\end{pgfscope}%
\begin{pgfscope}%
\pgfsys@transformshift{1.936313in}{4.368964in}%
\pgfsys@useobject{currentmarker}{}%
\end{pgfscope}%
\begin{pgfscope}%
\pgfsys@transformshift{1.918005in}{4.165198in}%
\pgfsys@useobject{currentmarker}{}%
\end{pgfscope}%
\begin{pgfscope}%
\pgfsys@transformshift{1.899695in}{4.068753in}%
\pgfsys@useobject{currentmarker}{}%
\end{pgfscope}%
\begin{pgfscope}%
\pgfsys@transformshift{1.881152in}{4.045086in}%
\pgfsys@useobject{currentmarker}{}%
\end{pgfscope}%
\begin{pgfscope}%
\pgfsys@transformshift{1.861199in}{4.036736in}%
\pgfsys@useobject{currentmarker}{}%
\end{pgfscope}%
\begin{pgfscope}%
\pgfsys@transformshift{1.841012in}{4.035141in}%
\pgfsys@useobject{currentmarker}{}%
\end{pgfscope}%
\begin{pgfscope}%
\pgfsys@transformshift{1.822470in}{4.039232in}%
\pgfsys@useobject{currentmarker}{}%
\end{pgfscope}%
\begin{pgfscope}%
\pgfsys@transformshift{1.803456in}{4.053197in}%
\pgfsys@useobject{currentmarker}{}%
\end{pgfscope}%
\begin{pgfscope}%
\pgfsys@transformshift{1.785383in}{4.101077in}%
\pgfsys@useobject{currentmarker}{}%
\end{pgfscope}%
\begin{pgfscope}%
\pgfsys@transformshift{1.766133in}{4.228231in}%
\pgfsys@useobject{currentmarker}{}%
\end{pgfscope}%
\begin{pgfscope}%
\pgfsys@transformshift{1.747825in}{4.342140in}%
\pgfsys@useobject{currentmarker}{}%
\end{pgfscope}%
\begin{pgfscope}%
\pgfsys@transformshift{1.726229in}{4.406576in}%
\pgfsys@useobject{currentmarker}{}%
\end{pgfscope}%
\begin{pgfscope}%
\pgfsys@transformshift{1.707216in}{4.402488in}%
\pgfsys@useobject{currentmarker}{}%
\end{pgfscope}%
\begin{pgfscope}%
\pgfsys@transformshift{1.685857in}{4.190423in}%
\pgfsys@useobject{currentmarker}{}%
\end{pgfscope}%
\begin{pgfscope}%
\pgfsys@transformshift{1.664261in}{4.081432in}%
\pgfsys@useobject{currentmarker}{}%
\end{pgfscope}%
\begin{pgfscope}%
\pgfsys@transformshift{1.645718in}{4.048065in}%
\pgfsys@useobject{currentmarker}{}%
\end{pgfscope}%
\begin{pgfscope}%
\pgfsys@transformshift{1.629757in}{4.039241in}%
\pgfsys@useobject{currentmarker}{}%
\end{pgfscope}%
\begin{pgfscope}%
\pgfsys@transformshift{1.610743in}{4.035582in}%
\pgfsys@useobject{currentmarker}{}%
\end{pgfscope}%
\begin{pgfscope}%
\pgfsys@transformshift{1.593373in}{4.035941in}%
\pgfsys@useobject{currentmarker}{}%
\end{pgfscope}%
\begin{pgfscope}%
\pgfsys@transformshift{1.571779in}{4.038383in}%
\pgfsys@useobject{currentmarker}{}%
\end{pgfscope}%
\begin{pgfscope}%
\pgfsys@transformshift{1.553469in}{4.050322in}%
\pgfsys@useobject{currentmarker}{}%
\end{pgfscope}%
\begin{pgfscope}%
\pgfsys@transformshift{1.534456in}{4.089861in}%
\pgfsys@useobject{currentmarker}{}%
\end{pgfscope}%
\begin{pgfscope}%
\pgfsys@transformshift{1.512626in}{4.221481in}%
\pgfsys@useobject{currentmarker}{}%
\end{pgfscope}%
\begin{pgfscope}%
\pgfsys@transformshift{1.497369in}{4.346266in}%
\pgfsys@useobject{currentmarker}{}%
\end{pgfscope}%
\begin{pgfscope}%
\pgfsys@transformshift{1.476009in}{4.423820in}%
\pgfsys@useobject{currentmarker}{}%
\end{pgfscope}%
\begin{pgfscope}%
\pgfsys@transformshift{1.457934in}{4.399092in}%
\pgfsys@useobject{currentmarker}{}%
\end{pgfscope}%
\begin{pgfscope}%
\pgfsys@transformshift{1.438217in}{4.213987in}%
\pgfsys@useobject{currentmarker}{}%
\end{pgfscope}%
\begin{pgfscope}%
\pgfsys@transformshift{1.416153in}{4.085374in}%
\pgfsys@useobject{currentmarker}{}%
\end{pgfscope}%
\begin{pgfscope}%
\pgfsys@transformshift{1.401599in}{4.056100in}%
\pgfsys@useobject{currentmarker}{}%
\end{pgfscope}%
\begin{pgfscope}%
\pgfsys@transformshift{1.380240in}{4.046494in}%
\pgfsys@useobject{currentmarker}{}%
\end{pgfscope}%
\begin{pgfscope}%
\pgfsys@transformshift{1.361227in}{4.040165in}%
\pgfsys@useobject{currentmarker}{}%
\end{pgfscope}%
\begin{pgfscope}%
\pgfsys@transformshift{1.342917in}{4.035893in}%
\pgfsys@useobject{currentmarker}{}%
\end{pgfscope}%
\begin{pgfscope}%
\pgfsys@transformshift{1.324374in}{4.036894in}%
\pgfsys@useobject{currentmarker}{}%
\end{pgfscope}%
\begin{pgfscope}%
\pgfsys@transformshift{1.303013in}{4.044528in}%
\pgfsys@useobject{currentmarker}{}%
\end{pgfscope}%
\begin{pgfscope}%
\pgfsys@transformshift{1.284470in}{4.065546in}%
\pgfsys@useobject{currentmarker}{}%
\end{pgfscope}%
\begin{pgfscope}%
\pgfsys@transformshift{1.266160in}{4.117824in}%
\pgfsys@useobject{currentmarker}{}%
\end{pgfscope}%
\begin{pgfscope}%
\pgfsys@transformshift{1.246444in}{4.221473in}%
\pgfsys@useobject{currentmarker}{}%
\end{pgfscope}%
\begin{pgfscope}%
\pgfsys@transformshift{1.226022in}{4.370314in}%
\pgfsys@useobject{currentmarker}{}%
\end{pgfscope}%
\begin{pgfscope}%
\pgfsys@transformshift{1.207712in}{4.424558in}%
\pgfsys@useobject{currentmarker}{}%
\end{pgfscope}%
\begin{pgfscope}%
\pgfsys@transformshift{1.189170in}{4.432294in}%
\pgfsys@useobject{currentmarker}{}%
\end{pgfscope}%
\begin{pgfscope}%
\pgfsys@transformshift{1.169688in}{4.342553in}%
\pgfsys@useobject{currentmarker}{}%
\end{pgfscope}%
\begin{pgfscope}%
\pgfsys@transformshift{1.152083in}{4.214912in}%
\pgfsys@useobject{currentmarker}{}%
\end{pgfscope}%
\begin{pgfscope}%
\pgfsys@transformshift{1.130253in}{4.124561in}%
\pgfsys@useobject{currentmarker}{}%
\end{pgfscope}%
\begin{pgfscope}%
\pgfsys@transformshift{1.109831in}{4.061890in}%
\pgfsys@useobject{currentmarker}{}%
\end{pgfscope}%
\begin{pgfscope}%
\pgfsys@transformshift{1.092460in}{4.046654in}%
\pgfsys@useobject{currentmarker}{}%
\end{pgfscope}%
\begin{pgfscope}%
\pgfsys@transformshift{1.074387in}{4.042748in}%
\pgfsys@useobject{currentmarker}{}%
\end{pgfscope}%
\begin{pgfscope}%
\pgfsys@transformshift{1.052557in}{4.036451in}%
\pgfsys@useobject{currentmarker}{}%
\end{pgfscope}%
\begin{pgfscope}%
\pgfsys@transformshift{1.034014in}{4.037293in}%
\pgfsys@useobject{currentmarker}{}%
\end{pgfscope}%
\begin{pgfscope}%
\pgfsys@transformshift{1.015704in}{4.041914in}%
\pgfsys@useobject{currentmarker}{}%
\end{pgfscope}%
\begin{pgfscope}%
\pgfsys@transformshift{0.997631in}{4.056466in}%
\pgfsys@useobject{currentmarker}{}%
\end{pgfscope}%
\begin{pgfscope}%
\pgfsys@transformshift{0.975800in}{4.096860in}%
\pgfsys@useobject{currentmarker}{}%
\end{pgfscope}%
\begin{pgfscope}%
\pgfsys@transformshift{0.960073in}{4.196922in}%
\pgfsys@useobject{currentmarker}{}%
\end{pgfscope}%
\begin{pgfscope}%
\pgfsys@transformshift{0.938245in}{4.361544in}%
\pgfsys@useobject{currentmarker}{}%
\end{pgfscope}%
\begin{pgfscope}%
\pgfsys@transformshift{0.920169in}{4.393223in}%
\pgfsys@useobject{currentmarker}{}%
\end{pgfscope}%
\begin{pgfscope}%
\pgfsys@transformshift{0.901390in}{4.053151in}%
\pgfsys@useobject{currentmarker}{}%
\end{pgfscope}%
\begin{pgfscope}%
\pgfsys@transformshift{0.880265in}{4.107353in}%
\pgfsys@useobject{currentmarker}{}%
\end{pgfscope}%
\begin{pgfscope}%
\pgfsys@transformshift{0.862426in}{4.247749in}%
\pgfsys@useobject{currentmarker}{}%
\end{pgfscope}%
\begin{pgfscope}%
\pgfsys@transformshift{0.842239in}{4.397421in}%
\pgfsys@useobject{currentmarker}{}%
\end{pgfscope}%
\begin{pgfscope}%
\pgfsys@transformshift{0.823462in}{4.448900in}%
\pgfsys@useobject{currentmarker}{}%
\end{pgfscope}%
\begin{pgfscope}%
\pgfsys@transformshift{0.802806in}{4.382160in}%
\pgfsys@useobject{currentmarker}{}%
\end{pgfscope}%
\begin{pgfscope}%
\pgfsys@transformshift{0.784730in}{4.164944in}%
\pgfsys@useobject{currentmarker}{}%
\end{pgfscope}%
\begin{pgfscope}%
\pgfsys@transformshift{0.766656in}{4.084468in}%
\pgfsys@useobject{currentmarker}{}%
\end{pgfscope}%
\begin{pgfscope}%
\pgfsys@transformshift{0.744826in}{4.051662in}%
\pgfsys@useobject{currentmarker}{}%
\end{pgfscope}%
\begin{pgfscope}%
\pgfsys@transformshift{0.725579in}{4.039289in}%
\pgfsys@useobject{currentmarker}{}%
\end{pgfscope}%
\begin{pgfscope}%
\pgfsys@transformshift{0.707036in}{4.036573in}%
\pgfsys@useobject{currentmarker}{}%
\end{pgfscope}%
\begin{pgfscope}%
\pgfsys@transformshift{0.688960in}{4.038580in}%
\pgfsys@useobject{currentmarker}{}%
\end{pgfscope}%
\begin{pgfscope}%
\pgfsys@transformshift{0.669947in}{4.048688in}%
\pgfsys@useobject{currentmarker}{}%
\end{pgfscope}%
\begin{pgfscope}%
\pgfsys@transformshift{0.648353in}{4.074161in}%
\pgfsys@useobject{currentmarker}{}%
\end{pgfscope}%
\begin{pgfscope}%
\pgfsys@transformshift{0.648353in}{4.073396in}%
\pgfsys@useobject{currentmarker}{}%
\end{pgfscope}%
\begin{pgfscope}%
\pgfsys@transformshift{0.656802in}{4.053588in}%
\pgfsys@useobject{currentmarker}{}%
\end{pgfscope}%
\begin{pgfscope}%
\pgfsys@transformshift{0.677224in}{4.038172in}%
\pgfsys@useobject{currentmarker}{}%
\end{pgfscope}%
\begin{pgfscope}%
\pgfsys@transformshift{0.695768in}{4.038140in}%
\pgfsys@useobject{currentmarker}{}%
\end{pgfscope}%
\begin{pgfscope}%
\pgfsys@transformshift{0.711965in}{4.049593in}%
\pgfsys@useobject{currentmarker}{}%
\end{pgfscope}%
\begin{pgfscope}%
\pgfsys@transformshift{0.732855in}{4.097532in}%
\pgfsys@useobject{currentmarker}{}%
\end{pgfscope}%
\begin{pgfscope}%
\pgfsys@transformshift{0.753980in}{4.307558in}%
\pgfsys@useobject{currentmarker}{}%
\end{pgfscope}%
\begin{pgfscope}%
\pgfsys@transformshift{0.772994in}{4.448618in}%
\pgfsys@useobject{currentmarker}{}%
\end{pgfscope}%
\begin{pgfscope}%
\pgfsys@transformshift{0.793181in}{4.364603in}%
\pgfsys@useobject{currentmarker}{}%
\end{pgfscope}%
\begin{pgfscope}%
\pgfsys@transformshift{0.809143in}{4.183395in}%
\pgfsys@useobject{currentmarker}{}%
\end{pgfscope}%
\begin{pgfscope}%
\pgfsys@transformshift{0.829330in}{4.068035in}%
\pgfsys@useobject{currentmarker}{}%
\end{pgfscope}%
\begin{pgfscope}%
\pgfsys@transformshift{0.847169in}{4.041918in}%
\pgfsys@useobject{currentmarker}{}%
\end{pgfscope}%
\begin{pgfscope}%
\pgfsys@transformshift{0.868763in}{4.036179in}%
\pgfsys@useobject{currentmarker}{}%
\end{pgfscope}%
\begin{pgfscope}%
\pgfsys@transformshift{0.888482in}{4.043253in}%
\pgfsys@useobject{currentmarker}{}%
\end{pgfscope}%
\begin{pgfscope}%
\pgfsys@transformshift{0.907024in}{4.072213in}%
\pgfsys@useobject{currentmarker}{}%
\end{pgfscope}%
\begin{pgfscope}%
\pgfsys@transformshift{0.925803in}{4.182898in}%
\pgfsys@useobject{currentmarker}{}%
\end{pgfscope}%
\begin{pgfscope}%
\pgfsys@transformshift{0.943876in}{4.414329in}%
\pgfsys@useobject{currentmarker}{}%
\end{pgfscope}%
\begin{pgfscope}%
\pgfsys@transformshift{0.964298in}{4.415688in}%
\pgfsys@useobject{currentmarker}{}%
\end{pgfscope}%
\begin{pgfscope}%
\pgfsys@transformshift{0.983546in}{4.237001in}%
\pgfsys@useobject{currentmarker}{}%
\end{pgfscope}%
\begin{pgfscope}%
\pgfsys@transformshift{1.002559in}{4.087547in}%
\pgfsys@useobject{currentmarker}{}%
\end{pgfscope}%
\begin{pgfscope}%
\pgfsys@transformshift{1.022041in}{4.045837in}%
\pgfsys@useobject{currentmarker}{}%
\end{pgfscope}%
\begin{pgfscope}%
\pgfsys@transformshift{1.041055in}{4.036182in}%
\pgfsys@useobject{currentmarker}{}%
\end{pgfscope}%
\begin{pgfscope}%
\pgfsys@transformshift{1.059833in}{4.037846in}%
\pgfsys@useobject{currentmarker}{}%
\end{pgfscope}%
\begin{pgfscope}%
\pgfsys@transformshift{1.080021in}{4.052499in}%
\pgfsys@useobject{currentmarker}{}%
\end{pgfscope}%
\begin{pgfscope}%
\pgfsys@transformshift{1.098799in}{4.100419in}%
\pgfsys@useobject{currentmarker}{}%
\end{pgfscope}%
\begin{pgfscope}%
\pgfsys@transformshift{1.117108in}{4.283513in}%
\pgfsys@useobject{currentmarker}{}%
\end{pgfscope}%
\begin{pgfscope}%
\pgfsys@transformshift{1.137059in}{4.433649in}%
\pgfsys@useobject{currentmarker}{}%
\end{pgfscope}%
\begin{pgfscope}%
\pgfsys@transformshift{1.155603in}{4.340667in}%
\pgfsys@useobject{currentmarker}{}%
\end{pgfscope}%
\begin{pgfscope}%
\pgfsys@transformshift{1.177668in}{4.119031in}%
\pgfsys@useobject{currentmarker}{}%
\end{pgfscope}%
\begin{pgfscope}%
\pgfsys@transformshift{1.194333in}{4.058449in}%
\pgfsys@useobject{currentmarker}{}%
\end{pgfscope}%
\begin{pgfscope}%
\pgfsys@transformshift{1.212877in}{4.039421in}%
\pgfsys@useobject{currentmarker}{}%
\end{pgfscope}%
\begin{pgfscope}%
\pgfsys@transformshift{1.232828in}{4.035233in}%
\pgfsys@useobject{currentmarker}{}%
\end{pgfscope}%
\begin{pgfscope}%
\pgfsys@transformshift{1.252312in}{4.039868in}%
\pgfsys@useobject{currentmarker}{}%
\end{pgfscope}%
\begin{pgfscope}%
\pgfsys@transformshift{1.270855in}{4.058103in}%
\pgfsys@useobject{currentmarker}{}%
\end{pgfscope}%
\begin{pgfscope}%
\pgfsys@transformshift{1.290102in}{4.105050in}%
\pgfsys@useobject{currentmarker}{}%
\end{pgfscope}%
\begin{pgfscope}%
\pgfsys@transformshift{1.308881in}{4.278700in}%
\pgfsys@useobject{currentmarker}{}%
\end{pgfscope}%
\begin{pgfscope}%
\pgfsys@transformshift{1.330477in}{4.426284in}%
\pgfsys@useobject{currentmarker}{}%
\end{pgfscope}%
\begin{pgfscope}%
\pgfsys@transformshift{1.349725in}{4.335446in}%
\pgfsys@useobject{currentmarker}{}%
\end{pgfscope}%
\begin{pgfscope}%
\pgfsys@transformshift{1.366155in}{4.181798in}%
\pgfsys@useobject{currentmarker}{}%
\end{pgfscope}%
\begin{pgfscope}%
\pgfsys@transformshift{1.388454in}{4.066803in}%
\pgfsys@useobject{currentmarker}{}%
\end{pgfscope}%
\begin{pgfscope}%
\pgfsys@transformshift{1.406528in}{4.041181in}%
\pgfsys@useobject{currentmarker}{}%
\end{pgfscope}%
\begin{pgfscope}%
\pgfsys@transformshift{1.425072in}{4.035520in}%
\pgfsys@useobject{currentmarker}{}%
\end{pgfscope}%
\begin{pgfscope}%
\pgfsys@transformshift{1.444554in}{4.036880in}%
\pgfsys@useobject{currentmarker}{}%
\end{pgfscope}%
\begin{pgfscope}%
\pgfsys@transformshift{1.463568in}{4.035306in}%
\pgfsys@useobject{currentmarker}{}%
\end{pgfscope}%
\begin{pgfscope}%
\pgfsys@transformshift{1.482815in}{4.040583in}%
\pgfsys@useobject{currentmarker}{}%
\end{pgfscope}%
\begin{pgfscope}%
\pgfsys@transformshift{1.501594in}{4.056659in}%
\pgfsys@useobject{currentmarker}{}%
\end{pgfscope}%
\begin{pgfscope}%
\pgfsys@transformshift{1.524128in}{4.111314in}%
\pgfsys@useobject{currentmarker}{}%
\end{pgfscope}%
\begin{pgfscope}%
\pgfsys@transformshift{1.539150in}{4.268988in}%
\pgfsys@useobject{currentmarker}{}%
\end{pgfscope}%
\begin{pgfscope}%
\pgfsys@transformshift{1.561685in}{4.416597in}%
\pgfsys@useobject{currentmarker}{}%
\end{pgfscope}%
\begin{pgfscope}%
\pgfsys@transformshift{1.580462in}{4.370324in}%
\pgfsys@useobject{currentmarker}{}%
\end{pgfscope}%
\begin{pgfscope}%
\pgfsys@transformshift{1.599476in}{4.194623in}%
\pgfsys@useobject{currentmarker}{}%
\end{pgfscope}%
\begin{pgfscope}%
\pgfsys@transformshift{1.617551in}{4.075008in}%
\pgfsys@useobject{currentmarker}{}%
\end{pgfscope}%
\begin{pgfscope}%
\pgfsys@transformshift{1.636799in}{4.042782in}%
\pgfsys@useobject{currentmarker}{}%
\end{pgfscope}%
\begin{pgfscope}%
\pgfsys@transformshift{1.654638in}{4.035592in}%
\pgfsys@useobject{currentmarker}{}%
\end{pgfscope}%
\begin{pgfscope}%
\pgfsys@transformshift{1.673181in}{4.035636in}%
\pgfsys@useobject{currentmarker}{}%
\end{pgfscope}%
\begin{pgfscope}%
\pgfsys@transformshift{1.695716in}{4.043842in}%
\pgfsys@useobject{currentmarker}{}%
\end{pgfscope}%
\begin{pgfscope}%
\pgfsys@transformshift{1.714964in}{4.071511in}%
\pgfsys@useobject{currentmarker}{}%
\end{pgfscope}%
\begin{pgfscope}%
\pgfsys@transformshift{1.732334in}{4.155832in}%
\pgfsys@useobject{currentmarker}{}%
\end{pgfscope}%
\begin{pgfscope}%
\pgfsys@transformshift{1.753693in}{4.376282in}%
\pgfsys@useobject{currentmarker}{}%
\end{pgfscope}%
\begin{pgfscope}%
\pgfsys@transformshift{1.769890in}{4.417476in}%
\pgfsys@useobject{currentmarker}{}%
\end{pgfscope}%
\begin{pgfscope}%
\pgfsys@transformshift{1.790546in}{4.288367in}%
\pgfsys@useobject{currentmarker}{}%
\end{pgfscope}%
\begin{pgfscope}%
\pgfsys@transformshift{1.809325in}{4.117682in}%
\pgfsys@useobject{currentmarker}{}%
\end{pgfscope}%
\begin{pgfscope}%
\pgfsys@transformshift{1.828104in}{4.059147in}%
\pgfsys@useobject{currentmarker}{}%
\end{pgfscope}%
\begin{pgfscope}%
\pgfsys@transformshift{1.849932in}{4.038700in}%
\pgfsys@useobject{currentmarker}{}%
\end{pgfscope}%
\begin{pgfscope}%
\pgfsys@transformshift{1.869416in}{4.035095in}%
\pgfsys@useobject{currentmarker}{}%
\end{pgfscope}%
\begin{pgfscope}%
\pgfsys@transformshift{1.887724in}{4.035690in}%
\pgfsys@useobject{currentmarker}{}%
\end{pgfscope}%
\begin{pgfscope}%
\pgfsys@transformshift{1.905563in}{4.041663in}%
\pgfsys@useobject{currentmarker}{}%
\end{pgfscope}%
\begin{pgfscope}%
\pgfsys@transformshift{1.922228in}{4.058776in}%
\pgfsys@useobject{currentmarker}{}%
\end{pgfscope}%
\begin{pgfscope}%
\pgfsys@transformshift{1.940773in}{4.097204in}%
\pgfsys@useobject{currentmarker}{}%
\end{pgfscope}%
\begin{pgfscope}%
\pgfsys@transformshift{1.963072in}{4.253143in}%
\pgfsys@useobject{currentmarker}{}%
\end{pgfscope}%
\begin{pgfscope}%
\pgfsys@transformshift{1.983025in}{4.415477in}%
\pgfsys@useobject{currentmarker}{}%
\end{pgfscope}%
\begin{pgfscope}%
\pgfsys@transformshift{2.002741in}{4.394155in}%
\pgfsys@useobject{currentmarker}{}%
\end{pgfscope}%
\begin{pgfscope}%
\pgfsys@transformshift{2.020346in}{4.261926in}%
\pgfsys@useobject{currentmarker}{}%
\end{pgfscope}%
\begin{pgfscope}%
\pgfsys@transformshift{2.042879in}{4.082213in}%
\pgfsys@useobject{currentmarker}{}%
\end{pgfscope}%
\begin{pgfscope}%
\pgfsys@transformshift{2.057667in}{4.050938in}%
\pgfsys@useobject{currentmarker}{}%
\end{pgfscope}%
\begin{pgfscope}%
\pgfsys@transformshift{2.077855in}{4.037456in}%
\pgfsys@useobject{currentmarker}{}%
\end{pgfscope}%
\begin{pgfscope}%
\pgfsys@transformshift{2.095694in}{4.034918in}%
\pgfsys@useobject{currentmarker}{}%
\end{pgfscope}%
\begin{pgfscope}%
\pgfsys@transformshift{2.117055in}{4.036447in}%
\pgfsys@useobject{currentmarker}{}%
\end{pgfscope}%
\begin{pgfscope}%
\pgfsys@transformshift{2.137946in}{4.043564in}%
\pgfsys@useobject{currentmarker}{}%
\end{pgfscope}%
\begin{pgfscope}%
\pgfsys@transformshift{2.155316in}{4.068698in}%
\pgfsys@useobject{currentmarker}{}%
\end{pgfscope}%
\begin{pgfscope}%
\pgfsys@transformshift{2.173390in}{4.108945in}%
\pgfsys@useobject{currentmarker}{}%
\end{pgfscope}%
\begin{pgfscope}%
\pgfsys@transformshift{2.194046in}{4.216659in}%
\pgfsys@useobject{currentmarker}{}%
\end{pgfscope}%
\begin{pgfscope}%
\pgfsys@transformshift{2.213293in}{4.372572in}%
\pgfsys@useobject{currentmarker}{}%
\end{pgfscope}%
\begin{pgfscope}%
\pgfsys@transformshift{2.231133in}{4.407275in}%
\pgfsys@useobject{currentmarker}{}%
\end{pgfscope}%
\begin{pgfscope}%
\pgfsys@transformshift{2.251086in}{4.354355in}%
\pgfsys@useobject{currentmarker}{}%
\end{pgfscope}%
\begin{pgfscope}%
\pgfsys@transformshift{2.271976in}{4.158520in}%
\pgfsys@useobject{currentmarker}{}%
\end{pgfscope}%
\begin{pgfscope}%
\pgfsys@transformshift{2.293336in}{4.081595in}%
\pgfsys@useobject{currentmarker}{}%
\end{pgfscope}%
\begin{pgfscope}%
\pgfsys@transformshift{2.307655in}{4.048131in}%
\pgfsys@useobject{currentmarker}{}%
\end{pgfscope}%
\begin{pgfscope}%
\pgfsys@transformshift{2.329016in}{4.036575in}%
\pgfsys@useobject{currentmarker}{}%
\end{pgfscope}%
\begin{pgfscope}%
\pgfsys@transformshift{2.347324in}{4.034771in}%
\pgfsys@useobject{currentmarker}{}%
\end{pgfscope}%
\begin{pgfscope}%
\pgfsys@transformshift{2.368683in}{4.036802in}%
\pgfsys@useobject{currentmarker}{}%
\end{pgfscope}%
\begin{pgfscope}%
\pgfsys@transformshift{2.385819in}{4.045860in}%
\pgfsys@useobject{currentmarker}{}%
\end{pgfscope}%
\begin{pgfscope}%
\pgfsys@transformshift{2.404129in}{4.065223in}%
\pgfsys@useobject{currentmarker}{}%
\end{pgfscope}%
\begin{pgfscope}%
\pgfsys@transformshift{2.424315in}{4.157483in}%
\pgfsys@useobject{currentmarker}{}%
\end{pgfscope}%
\begin{pgfscope}%
\pgfsys@transformshift{2.443094in}{4.372722in}%
\pgfsys@useobject{currentmarker}{}%
\end{pgfscope}%
\begin{pgfscope}%
\pgfsys@transformshift{2.462341in}{4.405014in}%
\pgfsys@useobject{currentmarker}{}%
\end{pgfscope}%
\begin{pgfscope}%
\pgfsys@transformshift{2.482997in}{4.378211in}%
\pgfsys@useobject{currentmarker}{}%
\end{pgfscope}%
\begin{pgfscope}%
\pgfsys@transformshift{2.503888in}{4.198000in}%
\pgfsys@useobject{currentmarker}{}%
\end{pgfscope}%
\begin{pgfscope}%
\pgfsys@transformshift{2.521258in}{4.092788in}%
\pgfsys@useobject{currentmarker}{}%
\end{pgfscope}%
\begin{pgfscope}%
\pgfsys@transformshift{2.538160in}{4.170408in}%
\pgfsys@useobject{currentmarker}{}%
\end{pgfscope}%
\begin{pgfscope}%
\pgfsys@transformshift{2.561162in}{4.393763in}%
\pgfsys@useobject{currentmarker}{}%
\end{pgfscope}%
\begin{pgfscope}%
\pgfsys@transformshift{2.580410in}{4.413042in}%
\pgfsys@useobject{currentmarker}{}%
\end{pgfscope}%
\begin{pgfscope}%
\pgfsys@transformshift{2.598015in}{4.341900in}%
\pgfsys@useobject{currentmarker}{}%
\end{pgfscope}%
\begin{pgfscope}%
\pgfsys@transformshift{2.616793in}{4.160395in}%
\pgfsys@useobject{currentmarker}{}%
\end{pgfscope}%
\begin{pgfscope}%
\pgfsys@transformshift{2.635572in}{4.061457in}%
\pgfsys@useobject{currentmarker}{}%
\end{pgfscope}%
\begin{pgfscope}%
\pgfsys@transformshift{2.656228in}{4.039933in}%
\pgfsys@useobject{currentmarker}{}%
\end{pgfscope}%
\begin{pgfscope}%
\pgfsys@transformshift{2.675476in}{4.035297in}%
\pgfsys@useobject{currentmarker}{}%
\end{pgfscope}%
\begin{pgfscope}%
\pgfsys@transformshift{2.693550in}{4.036081in}%
\pgfsys@useobject{currentmarker}{}%
\end{pgfscope}%
\begin{pgfscope}%
\pgfsys@transformshift{2.711623in}{4.043139in}%
\pgfsys@useobject{currentmarker}{}%
\end{pgfscope}%
\begin{pgfscope}%
\pgfsys@transformshift{2.730402in}{4.066481in}%
\pgfsys@useobject{currentmarker}{}%
\end{pgfscope}%
\begin{pgfscope}%
\pgfsys@transformshift{2.751058in}{4.157962in}%
\pgfsys@useobject{currentmarker}{}%
\end{pgfscope}%
\begin{pgfscope}%
\pgfsys@transformshift{2.770306in}{4.376952in}%
\pgfsys@useobject{currentmarker}{}%
\end{pgfscope}%
\begin{pgfscope}%
\pgfsys@transformshift{2.789788in}{4.401167in}%
\pgfsys@useobject{currentmarker}{}%
\end{pgfscope}%
\begin{pgfscope}%
\pgfsys@transformshift{2.808333in}{4.271741in}%
\pgfsys@useobject{currentmarker}{}%
\end{pgfscope}%
\begin{pgfscope}%
\pgfsys@transformshift{2.829458in}{4.106411in}%
\pgfsys@useobject{currentmarker}{}%
\end{pgfscope}%
\begin{pgfscope}%
\pgfsys@transformshift{2.847062in}{4.053014in}%
\pgfsys@useobject{currentmarker}{}%
\end{pgfscope}%
\begin{pgfscope}%
\pgfsys@transformshift{2.865136in}{4.039847in}%
\pgfsys@useobject{currentmarker}{}%
\end{pgfscope}%
\begin{pgfscope}%
\pgfsys@transformshift{2.885558in}{4.035278in}%
\pgfsys@useobject{currentmarker}{}%
\end{pgfscope}%
\begin{pgfscope}%
\pgfsys@transformshift{2.903868in}{4.035924in}%
\pgfsys@useobject{currentmarker}{}%
\end{pgfscope}%
\begin{pgfscope}%
\pgfsys@transformshift{2.924758in}{4.044185in}%
\pgfsys@useobject{currentmarker}{}%
\end{pgfscope}%
\begin{pgfscope}%
\pgfsys@transformshift{2.943537in}{4.067745in}%
\pgfsys@useobject{currentmarker}{}%
\end{pgfscope}%
\begin{pgfscope}%
\pgfsys@transformshift{2.963488in}{4.166715in}%
\pgfsys@useobject{currentmarker}{}%
\end{pgfscope}%
\begin{pgfscope}%
\pgfsys@transformshift{2.981093in}{4.349765in}%
\pgfsys@useobject{currentmarker}{}%
\end{pgfscope}%
\begin{pgfscope}%
\pgfsys@transformshift{2.998698in}{4.413747in}%
\pgfsys@useobject{currentmarker}{}%
\end{pgfscope}%
\begin{pgfscope}%
\pgfsys@transformshift{3.020762in}{4.347372in}%
\pgfsys@useobject{currentmarker}{}%
\end{pgfscope}%
\begin{pgfscope}%
\pgfsys@transformshift{3.041887in}{4.197241in}%
\pgfsys@useobject{currentmarker}{}%
\end{pgfscope}%
\begin{pgfscope}%
\pgfsys@transformshift{3.056675in}{4.096468in}%
\pgfsys@useobject{currentmarker}{}%
\end{pgfscope}%
\begin{pgfscope}%
\pgfsys@transformshift{3.081322in}{4.050056in}%
\pgfsys@useobject{currentmarker}{}%
\end{pgfscope}%
\begin{pgfscope}%
\pgfsys@transformshift{3.096815in}{4.039440in}%
\pgfsys@useobject{currentmarker}{}%
\end{pgfscope}%
\begin{pgfscope}%
\pgfsys@transformshift{3.117706in}{4.035059in}%
\pgfsys@useobject{currentmarker}{}%
\end{pgfscope}%
\begin{pgfscope}%
\pgfsys@transformshift{3.139065in}{4.037878in}%
\pgfsys@useobject{currentmarker}{}%
\end{pgfscope}%
\begin{pgfscope}%
\pgfsys@transformshift{3.153384in}{4.044346in}%
\pgfsys@useobject{currentmarker}{}%
\end{pgfscope}%
\begin{pgfscope}%
\pgfsys@transformshift{3.173103in}{4.068883in}%
\pgfsys@useobject{currentmarker}{}%
\end{pgfscope}%
\begin{pgfscope}%
\pgfsys@transformshift{3.197514in}{4.148687in}%
\pgfsys@useobject{currentmarker}{}%
\end{pgfscope}%
\begin{pgfscope}%
\pgfsys@transformshift{3.212301in}{4.243360in}%
\pgfsys@useobject{currentmarker}{}%
\end{pgfscope}%
\begin{pgfscope}%
\pgfsys@transformshift{3.229906in}{4.408558in}%
\pgfsys@useobject{currentmarker}{}%
\end{pgfscope}%
\begin{pgfscope}%
\pgfsys@transformshift{3.250797in}{4.386737in}%
\pgfsys@useobject{currentmarker}{}%
\end{pgfscope}%
\begin{pgfscope}%
\pgfsys@transformshift{3.269107in}{4.257498in}%
\pgfsys@useobject{currentmarker}{}%
\end{pgfscope}%
\begin{pgfscope}%
\pgfsys@transformshift{3.292109in}{4.109615in}%
\pgfsys@useobject{currentmarker}{}%
\end{pgfscope}%
\begin{pgfscope}%
\pgfsys@transformshift{3.309714in}{4.062170in}%
\pgfsys@useobject{currentmarker}{}%
\end{pgfscope}%
\begin{pgfscope}%
\pgfsys@transformshift{3.326145in}{4.045484in}%
\pgfsys@useobject{currentmarker}{}%
\end{pgfscope}%
\begin{pgfscope}%
\pgfsys@transformshift{3.347506in}{4.037209in}%
\pgfsys@useobject{currentmarker}{}%
\end{pgfscope}%
\begin{pgfscope}%
\pgfsys@transformshift{3.365111in}{4.035211in}%
\pgfsys@useobject{currentmarker}{}%
\end{pgfscope}%
\begin{pgfscope}%
\pgfsys@transformshift{3.386236in}{4.037399in}%
\pgfsys@useobject{currentmarker}{}%
\end{pgfscope}%
\begin{pgfscope}%
\pgfsys@transformshift{3.403841in}{4.041282in}%
\pgfsys@useobject{currentmarker}{}%
\end{pgfscope}%
\begin{pgfscope}%
\pgfsys@transformshift{3.423088in}{4.059029in}%
\pgfsys@useobject{currentmarker}{}%
\end{pgfscope}%
\begin{pgfscope}%
\pgfsys@transformshift{3.443744in}{4.112186in}%
\pgfsys@useobject{currentmarker}{}%
\end{pgfscope}%
\begin{pgfscope}%
\pgfsys@transformshift{3.461349in}{4.256412in}%
\pgfsys@useobject{currentmarker}{}%
\end{pgfscope}%
\begin{pgfscope}%
\pgfsys@transformshift{3.479188in}{4.400644in}%
\pgfsys@useobject{currentmarker}{}%
\end{pgfscope}%
\begin{pgfscope}%
\pgfsys@transformshift{3.500550in}{4.413167in}%
\pgfsys@useobject{currentmarker}{}%
\end{pgfscope}%
\begin{pgfscope}%
\pgfsys@transformshift{3.518858in}{4.331349in}%
\pgfsys@useobject{currentmarker}{}%
\end{pgfscope}%
\begin{pgfscope}%
\pgfsys@transformshift{3.540922in}{4.136909in}%
\pgfsys@useobject{currentmarker}{}%
\end{pgfscope}%
\begin{pgfscope}%
\pgfsys@transformshift{3.558762in}{4.077769in}%
\pgfsys@useobject{currentmarker}{}%
\end{pgfscope}%
\begin{pgfscope}%
\pgfsys@transformshift{3.576601in}{4.155328in}%
\pgfsys@useobject{currentmarker}{}%
\end{pgfscope}%
\begin{pgfscope}%
\pgfsys@transformshift{3.597493in}{4.425042in}%
\pgfsys@useobject{currentmarker}{}%
\end{pgfscope}%
\begin{pgfscope}%
\pgfsys@transformshift{3.614393in}{4.371184in}%
\pgfsys@useobject{currentmarker}{}%
\end{pgfscope}%
\begin{pgfscope}%
\pgfsys@transformshift{3.635754in}{4.226044in}%
\pgfsys@useobject{currentmarker}{}%
\end{pgfscope}%
\begin{pgfscope}%
\pgfsys@transformshift{3.653123in}{4.097773in}%
\pgfsys@useobject{currentmarker}{}%
\end{pgfscope}%
\begin{pgfscope}%
\pgfsys@transformshift{3.671433in}{4.055062in}%
\pgfsys@useobject{currentmarker}{}%
\end{pgfscope}%
\begin{pgfscope}%
\pgfsys@transformshift{3.694201in}{4.038465in}%
\pgfsys@useobject{currentmarker}{}%
\end{pgfscope}%
\begin{pgfscope}%
\pgfsys@transformshift{3.711102in}{4.035477in}%
\pgfsys@useobject{currentmarker}{}%
\end{pgfscope}%
\begin{pgfscope}%
\pgfsys@transformshift{3.731758in}{4.039020in}%
\pgfsys@useobject{currentmarker}{}%
\end{pgfscope}%
\begin{pgfscope}%
\pgfsys@transformshift{3.749597in}{4.048120in}%
\pgfsys@useobject{currentmarker}{}%
\end{pgfscope}%
\begin{pgfscope}%
\pgfsys@transformshift{3.770722in}{4.082152in}%
\pgfsys@useobject{currentmarker}{}%
\end{pgfscope}%
\begin{pgfscope}%
\pgfsys@transformshift{3.789736in}{4.173865in}%
\pgfsys@useobject{currentmarker}{}%
\end{pgfscope}%
\begin{pgfscope}%
\pgfsys@transformshift{3.807809in}{4.342886in}%
\pgfsys@useobject{currentmarker}{}%
\end{pgfscope}%
\begin{pgfscope}%
\pgfsys@transformshift{3.826354in}{4.433965in}%
\pgfsys@useobject{currentmarker}{}%
\end{pgfscope}%
\begin{pgfscope}%
\pgfsys@transformshift{3.847479in}{4.394182in}%
\pgfsys@useobject{currentmarker}{}%
\end{pgfscope}%
\begin{pgfscope}%
\pgfsys@transformshift{3.864615in}{4.259917in}%
\pgfsys@useobject{currentmarker}{}%
\end{pgfscope}%
\begin{pgfscope}%
\pgfsys@transformshift{3.885740in}{4.113736in}%
\pgfsys@useobject{currentmarker}{}%
\end{pgfscope}%
\begin{pgfscope}%
\pgfsys@transformshift{3.902407in}{4.064720in}%
\pgfsys@useobject{currentmarker}{}%
\end{pgfscope}%
\begin{pgfscope}%
\pgfsys@transformshift{3.924001in}{4.043786in}%
\pgfsys@useobject{currentmarker}{}%
\end{pgfscope}%
\begin{pgfscope}%
\pgfsys@transformshift{3.942074in}{4.037472in}%
\pgfsys@useobject{currentmarker}{}%
\end{pgfscope}%
\begin{pgfscope}%
\pgfsys@transformshift{3.960619in}{4.035920in}%
\pgfsys@useobject{currentmarker}{}%
\end{pgfscope}%
\begin{pgfscope}%
\pgfsys@transformshift{3.981275in}{4.039322in}%
\pgfsys@useobject{currentmarker}{}%
\end{pgfscope}%
\begin{pgfscope}%
\pgfsys@transformshift{3.999348in}{4.050051in}%
\pgfsys@useobject{currentmarker}{}%
\end{pgfscope}%
\begin{pgfscope}%
\pgfsys@transformshift{4.019067in}{4.074616in}%
\pgfsys@useobject{currentmarker}{}%
\end{pgfscope}%
\begin{pgfscope}%
\pgfsys@transformshift{4.038549in}{4.147404in}%
\pgfsys@useobject{currentmarker}{}%
\end{pgfscope}%
\begin{pgfscope}%
\pgfsys@transformshift{4.059440in}{4.345001in}%
\pgfsys@useobject{currentmarker}{}%
\end{pgfscope}%
\begin{pgfscope}%
\pgfsys@transformshift{4.077044in}{4.441939in}%
\pgfsys@useobject{currentmarker}{}%
\end{pgfscope}%
\begin{pgfscope}%
\pgfsys@transformshift{4.096292in}{4.424851in}%
\pgfsys@useobject{currentmarker}{}%
\end{pgfscope}%
\begin{pgfscope}%
\pgfsys@transformshift{4.114837in}{4.332566in}%
\pgfsys@useobject{currentmarker}{}%
\end{pgfscope}%
\begin{pgfscope}%
\pgfsys@transformshift{4.135493in}{4.209876in}%
\pgfsys@useobject{currentmarker}{}%
\end{pgfscope}%
\begin{pgfscope}%
\pgfsys@transformshift{4.156383in}{4.093391in}%
\pgfsys@useobject{currentmarker}{}%
\end{pgfscope}%
\begin{pgfscope}%
\pgfsys@transformshift{4.173048in}{4.059682in}%
\pgfsys@useobject{currentmarker}{}%
\end{pgfscope}%
\begin{pgfscope}%
\pgfsys@transformshift{4.192296in}{4.047985in}%
\pgfsys@useobject{currentmarker}{}%
\end{pgfscope}%
\begin{pgfscope}%
\pgfsys@transformshift{4.212952in}{4.037418in}%
\pgfsys@useobject{currentmarker}{}%
\end{pgfscope}%
\begin{pgfscope}%
\pgfsys@transformshift{4.231028in}{4.036926in}%
\pgfsys@useobject{currentmarker}{}%
\end{pgfscope}%
\begin{pgfscope}%
\pgfsys@transformshift{4.249101in}{4.042880in}%
\pgfsys@useobject{currentmarker}{}%
\end{pgfscope}%
\begin{pgfscope}%
\pgfsys@transformshift{4.270697in}{4.055864in}%
\pgfsys@useobject{currentmarker}{}%
\end{pgfscope}%
\begin{pgfscope}%
\pgfsys@transformshift{4.287128in}{4.089728in}%
\pgfsys@useobject{currentmarker}{}%
\end{pgfscope}%
\begin{pgfscope}%
\pgfsys@transformshift{4.308722in}{4.184512in}%
\pgfsys@useobject{currentmarker}{}%
\end{pgfscope}%
\begin{pgfscope}%
\pgfsys@transformshift{4.326563in}{4.348319in}%
\pgfsys@useobject{currentmarker}{}%
\end{pgfscope}%
\begin{pgfscope}%
\pgfsys@transformshift{4.344871in}{4.439851in}%
\pgfsys@useobject{currentmarker}{}%
\end{pgfscope}%
\begin{pgfscope}%
\pgfsys@transformshift{4.365996in}{4.447487in}%
\pgfsys@useobject{currentmarker}{}%
\end{pgfscope}%
\begin{pgfscope}%
\pgfsys@transformshift{4.383835in}{4.372470in}%
\pgfsys@useobject{currentmarker}{}%
\end{pgfscope}%
\begin{pgfscope}%
\pgfsys@transformshift{4.405900in}{4.216931in}%
\pgfsys@useobject{currentmarker}{}%
\end{pgfscope}%
\begin{pgfscope}%
\pgfsys@transformshift{4.424444in}{4.115093in}%
\pgfsys@useobject{currentmarker}{}%
\end{pgfscope}%
\begin{pgfscope}%
\pgfsys@transformshift{4.442049in}{4.077882in}%
\pgfsys@useobject{currentmarker}{}%
\end{pgfscope}%
\begin{pgfscope}%
\pgfsys@transformshift{4.459888in}{4.050189in}%
\pgfsys@useobject{currentmarker}{}%
\end{pgfscope}%
\begin{pgfscope}%
\pgfsys@transformshift{4.481250in}{4.041069in}%
\pgfsys@useobject{currentmarker}{}%
\end{pgfscope}%
\begin{pgfscope}%
\pgfsys@transformshift{4.480779in}{4.041054in}%
\pgfsys@useobject{currentmarker}{}%
\end{pgfscope}%
\begin{pgfscope}%
\pgfsys@transformshift{4.474676in}{4.045208in}%
\pgfsys@useobject{currentmarker}{}%
\end{pgfscope}%
\begin{pgfscope}%
\pgfsys@transformshift{4.452143in}{4.088405in}%
\pgfsys@useobject{currentmarker}{}%
\end{pgfscope}%
\begin{pgfscope}%
\pgfsys@transformshift{4.433833in}{4.226805in}%
\pgfsys@useobject{currentmarker}{}%
\end{pgfscope}%
\begin{pgfscope}%
\pgfsys@transformshift{4.416933in}{4.408347in}%
\pgfsys@useobject{currentmarker}{}%
\end{pgfscope}%
\begin{pgfscope}%
\pgfsys@transformshift{4.393226in}{4.452717in}%
\pgfsys@useobject{currentmarker}{}%
\end{pgfscope}%
\begin{pgfscope}%
\pgfsys@transformshift{4.378438in}{4.316284in}%
\pgfsys@useobject{currentmarker}{}%
\end{pgfscope}%
\begin{pgfscope}%
\pgfsys@transformshift{4.357076in}{4.105883in}%
\pgfsys@useobject{currentmarker}{}%
\end{pgfscope}%
\begin{pgfscope}%
\pgfsys@transformshift{4.338768in}{4.055929in}%
\pgfsys@useobject{currentmarker}{}%
\end{pgfscope}%
\begin{pgfscope}%
\pgfsys@transformshift{4.322338in}{4.129855in}%
\pgfsys@useobject{currentmarker}{}%
\end{pgfscope}%
\begin{pgfscope}%
\pgfsys@transformshift{4.301445in}{4.357529in}%
\pgfsys@useobject{currentmarker}{}%
\end{pgfscope}%
\begin{pgfscope}%
\pgfsys@transformshift{4.282434in}{4.449880in}%
\pgfsys@useobject{currentmarker}{}%
\end{pgfscope}%
\begin{pgfscope}%
\pgfsys@transformshift{4.262246in}{4.349400in}%
\pgfsys@useobject{currentmarker}{}%
\end{pgfscope}%
\begin{pgfscope}%
\pgfsys@transformshift{4.243233in}{4.270812in}%
\pgfsys@useobject{currentmarker}{}%
\end{pgfscope}%
\begin{pgfscope}%
\pgfsys@transformshift{4.222577in}{4.087148in}%
\pgfsys@useobject{currentmarker}{}%
\end{pgfscope}%
\begin{pgfscope}%
\pgfsys@transformshift{4.206146in}{4.047997in}%
\pgfsys@useobject{currentmarker}{}%
\end{pgfscope}%
\begin{pgfscope}%
\pgfsys@transformshift{4.185256in}{4.036856in}%
\pgfsys@useobject{currentmarker}{}%
\end{pgfscope}%
\begin{pgfscope}%
\pgfsys@transformshift{4.165068in}{4.038855in}%
\pgfsys@useobject{currentmarker}{}%
\end{pgfscope}%
\begin{pgfscope}%
\pgfsys@transformshift{4.148403in}{4.052975in}%
\pgfsys@useobject{currentmarker}{}%
\end{pgfscope}%
\begin{pgfscope}%
\pgfsys@transformshift{4.126807in}{4.132780in}%
\pgfsys@useobject{currentmarker}{}%
\end{pgfscope}%
\begin{pgfscope}%
\pgfsys@transformshift{4.111080in}{4.310331in}%
\pgfsys@useobject{currentmarker}{}%
\end{pgfscope}%
\begin{pgfscope}%
\pgfsys@transformshift{4.089721in}{4.438851in}%
\pgfsys@useobject{currentmarker}{}%
\end{pgfscope}%
\begin{pgfscope}%
\pgfsys@transformshift{4.071881in}{4.390714in}%
\pgfsys@useobject{currentmarker}{}%
\end{pgfscope}%
\begin{pgfscope}%
\pgfsys@transformshift{4.051928in}{4.124076in}%
\pgfsys@useobject{currentmarker}{}%
\end{pgfscope}%
\begin{pgfscope}%
\pgfsys@transformshift{4.028221in}{4.049369in}%
\pgfsys@useobject{currentmarker}{}%
\end{pgfscope}%
\begin{pgfscope}%
\pgfsys@transformshift{4.010147in}{4.037753in}%
\pgfsys@useobject{currentmarker}{}%
\end{pgfscope}%
\begin{pgfscope}%
\pgfsys@transformshift{3.995123in}{4.035788in}%
\pgfsys@useobject{currentmarker}{}%
\end{pgfscope}%
\begin{pgfscope}%
\pgfsys@transformshift{3.974467in}{4.044574in}%
\pgfsys@useobject{currentmarker}{}%
\end{pgfscope}%
\begin{pgfscope}%
\pgfsys@transformshift{3.958741in}{4.073852in}%
\pgfsys@useobject{currentmarker}{}%
\end{pgfscope}%
\begin{pgfscope}%
\pgfsys@transformshift{3.935737in}{4.213719in}%
\pgfsys@useobject{currentmarker}{}%
\end{pgfscope}%
\begin{pgfscope}%
\pgfsys@transformshift{3.918132in}{4.377768in}%
\pgfsys@useobject{currentmarker}{}%
\end{pgfscope}%
\begin{pgfscope}%
\pgfsys@transformshift{3.897242in}{4.426960in}%
\pgfsys@useobject{currentmarker}{}%
\end{pgfscope}%
\begin{pgfscope}%
\pgfsys@transformshift{3.879637in}{4.224819in}%
\pgfsys@useobject{currentmarker}{}%
\end{pgfscope}%
\begin{pgfscope}%
\pgfsys@transformshift{3.860858in}{4.086336in}%
\pgfsys@useobject{currentmarker}{}%
\end{pgfscope}%
\begin{pgfscope}%
\pgfsys@transformshift{3.841611in}{4.047846in}%
\pgfsys@useobject{currentmarker}{}%
\end{pgfscope}%
\begin{pgfscope}%
\pgfsys@transformshift{3.821189in}{4.036639in}%
\pgfsys@useobject{currentmarker}{}%
\end{pgfscope}%
\begin{pgfscope}%
\pgfsys@transformshift{3.800298in}{4.035841in}%
\pgfsys@useobject{currentmarker}{}%
\end{pgfscope}%
\begin{pgfscope}%
\pgfsys@transformshift{3.782225in}{4.043012in}%
\pgfsys@useobject{currentmarker}{}%
\end{pgfscope}%
\begin{pgfscope}%
\pgfsys@transformshift{3.761803in}{4.079765in}%
\pgfsys@useobject{currentmarker}{}%
\end{pgfscope}%
\begin{pgfscope}%
\pgfsys@transformshift{3.746075in}{4.176700in}%
\pgfsys@useobject{currentmarker}{}%
\end{pgfscope}%
\begin{pgfscope}%
\pgfsys@transformshift{3.728236in}{4.378709in}%
\pgfsys@useobject{currentmarker}{}%
\end{pgfscope}%
\begin{pgfscope}%
\pgfsys@transformshift{3.703825in}{4.421481in}%
\pgfsys@useobject{currentmarker}{}%
\end{pgfscope}%
\begin{pgfscope}%
\pgfsys@transformshift{3.681995in}{4.187759in}%
\pgfsys@useobject{currentmarker}{}%
\end{pgfscope}%
\begin{pgfscope}%
\pgfsys@transformshift{3.666268in}{4.091328in}%
\pgfsys@useobject{currentmarker}{}%
\end{pgfscope}%
\begin{pgfscope}%
\pgfsys@transformshift{3.647725in}{4.049472in}%
\pgfsys@useobject{currentmarker}{}%
\end{pgfscope}%
\begin{pgfscope}%
\pgfsys@transformshift{3.628712in}{4.107961in}%
\pgfsys@useobject{currentmarker}{}%
\end{pgfscope}%
\begin{pgfscope}%
\pgfsys@transformshift{3.608994in}{4.056257in}%
\pgfsys@useobject{currentmarker}{}%
\end{pgfscope}%
\begin{pgfscope}%
\pgfsys@transformshift{3.592328in}{4.039735in}%
\pgfsys@useobject{currentmarker}{}%
\end{pgfscope}%
\begin{pgfscope}%
\pgfsys@transformshift{3.570498in}{4.035271in}%
\pgfsys@useobject{currentmarker}{}%
\end{pgfscope}%
\begin{pgfscope}%
\pgfsys@transformshift{3.549842in}{4.039612in}%
\pgfsys@useobject{currentmarker}{}%
\end{pgfscope}%
\begin{pgfscope}%
\pgfsys@transformshift{3.532472in}{4.051288in}%
\pgfsys@useobject{currentmarker}{}%
\end{pgfscope}%
\begin{pgfscope}%
\pgfsys@transformshift{3.512052in}{4.109922in}%
\pgfsys@useobject{currentmarker}{}%
\end{pgfscope}%
\begin{pgfscope}%
\pgfsys@transformshift{3.493742in}{4.247771in}%
\pgfsys@useobject{currentmarker}{}%
\end{pgfscope}%
\begin{pgfscope}%
\pgfsys@transformshift{3.476137in}{4.384961in}%
\pgfsys@useobject{currentmarker}{}%
\end{pgfscope}%
\begin{pgfscope}%
\pgfsys@transformshift{3.455481in}{4.407890in}%
\pgfsys@useobject{currentmarker}{}%
\end{pgfscope}%
\begin{pgfscope}%
\pgfsys@transformshift{3.437642in}{4.222901in}%
\pgfsys@useobject{currentmarker}{}%
\end{pgfscope}%
\begin{pgfscope}%
\pgfsys@transformshift{3.417454in}{4.085746in}%
\pgfsys@useobject{currentmarker}{}%
\end{pgfscope}%
\begin{pgfscope}%
\pgfsys@transformshift{3.399615in}{4.048739in}%
\pgfsys@useobject{currentmarker}{}%
\end{pgfscope}%
\begin{pgfscope}%
\pgfsys@transformshift{3.379430in}{4.037545in}%
\pgfsys@useobject{currentmarker}{}%
\end{pgfscope}%
\begin{pgfscope}%
\pgfsys@transformshift{3.360651in}{4.035176in}%
\pgfsys@useobject{currentmarker}{}%
\end{pgfscope}%
\begin{pgfscope}%
\pgfsys@transformshift{3.339995in}{4.037691in}%
\pgfsys@useobject{currentmarker}{}%
\end{pgfscope}%
\begin{pgfscope}%
\pgfsys@transformshift{3.322390in}{4.048253in}%
\pgfsys@useobject{currentmarker}{}%
\end{pgfscope}%
\begin{pgfscope}%
\pgfsys@transformshift{3.304551in}{4.077957in}%
\pgfsys@useobject{currentmarker}{}%
\end{pgfscope}%
\begin{pgfscope}%
\pgfsys@transformshift{3.282015in}{4.193362in}%
\pgfsys@useobject{currentmarker}{}%
\end{pgfscope}%
\begin{pgfscope}%
\pgfsys@transformshift{3.263942in}{4.052995in}%
\pgfsys@useobject{currentmarker}{}%
\end{pgfscope}%
\begin{pgfscope}%
\pgfsys@transformshift{3.242112in}{4.126964in}%
\pgfsys@useobject{currentmarker}{}%
\end{pgfscope}%
\begin{pgfscope}%
\pgfsys@transformshift{3.225681in}{4.274201in}%
\pgfsys@useobject{currentmarker}{}%
\end{pgfscope}%
\begin{pgfscope}%
\pgfsys@transformshift{3.206433in}{4.406033in}%
\pgfsys@useobject{currentmarker}{}%
\end{pgfscope}%
\begin{pgfscope}%
\pgfsys@transformshift{3.186482in}{4.380494in}%
\pgfsys@useobject{currentmarker}{}%
\end{pgfscope}%
\begin{pgfscope}%
\pgfsys@transformshift{3.168172in}{4.178961in}%
\pgfsys@useobject{currentmarker}{}%
\end{pgfscope}%
\begin{pgfscope}%
\pgfsys@transformshift{3.147750in}{4.066180in}%
\pgfsys@useobject{currentmarker}{}%
\end{pgfscope}%
\begin{pgfscope}%
\pgfsys@transformshift{3.128974in}{4.043112in}%
\pgfsys@useobject{currentmarker}{}%
\end{pgfscope}%
\begin{pgfscope}%
\pgfsys@transformshift{3.110429in}{4.035917in}%
\pgfsys@useobject{currentmarker}{}%
\end{pgfscope}%
\begin{pgfscope}%
\pgfsys@transformshift{3.087896in}{4.036232in}%
\pgfsys@useobject{currentmarker}{}%
\end{pgfscope}%
\begin{pgfscope}%
\pgfsys@transformshift{3.072871in}{4.040867in}%
\pgfsys@useobject{currentmarker}{}%
\end{pgfscope}%
\begin{pgfscope}%
\pgfsys@transformshift{3.051043in}{4.055807in}%
\pgfsys@useobject{currentmarker}{}%
\end{pgfscope}%
\begin{pgfscope}%
\pgfsys@transformshift{3.034376in}{4.117310in}%
\pgfsys@useobject{currentmarker}{}%
\end{pgfscope}%
\begin{pgfscope}%
\pgfsys@transformshift{3.010669in}{4.325854in}%
\pgfsys@useobject{currentmarker}{}%
\end{pgfscope}%
\begin{pgfscope}%
\pgfsys@transformshift{2.994943in}{4.412905in}%
\pgfsys@useobject{currentmarker}{}%
\end{pgfscope}%
\begin{pgfscope}%
\pgfsys@transformshift{2.977102in}{4.400736in}%
\pgfsys@useobject{currentmarker}{}%
\end{pgfscope}%
\begin{pgfscope}%
\pgfsys@transformshift{2.957854in}{4.205461in}%
\pgfsys@useobject{currentmarker}{}%
\end{pgfscope}%
\begin{pgfscope}%
\pgfsys@transformshift{2.933443in}{4.068286in}%
\pgfsys@useobject{currentmarker}{}%
\end{pgfscope}%
\begin{pgfscope}%
\pgfsys@transformshift{2.917950in}{4.052087in}%
\pgfsys@useobject{currentmarker}{}%
\end{pgfscope}%
\begin{pgfscope}%
\pgfsys@transformshift{2.895651in}{4.040101in}%
\pgfsys@useobject{currentmarker}{}%
\end{pgfscope}%
\begin{pgfscope}%
\pgfsys@transformshift{2.880395in}{4.035760in}%
\pgfsys@useobject{currentmarker}{}%
\end{pgfscope}%
\begin{pgfscope}%
\pgfsys@transformshift{2.859504in}{4.035602in}%
\pgfsys@useobject{currentmarker}{}%
\end{pgfscope}%
\begin{pgfscope}%
\pgfsys@transformshift{2.839786in}{4.041915in}%
\pgfsys@useobject{currentmarker}{}%
\end{pgfscope}%
\begin{pgfscope}%
\pgfsys@transformshift{2.822181in}{4.060800in}%
\pgfsys@useobject{currentmarker}{}%
\end{pgfscope}%
\begin{pgfscope}%
\pgfsys@transformshift{2.804107in}{4.144787in}%
\pgfsys@useobject{currentmarker}{}%
\end{pgfscope}%
\begin{pgfscope}%
\pgfsys@transformshift{2.784860in}{4.291819in}%
\pgfsys@useobject{currentmarker}{}%
\end{pgfscope}%
\begin{pgfscope}%
\pgfsys@transformshift{2.765377in}{4.407626in}%
\pgfsys@useobject{currentmarker}{}%
\end{pgfscope}%
\begin{pgfscope}%
\pgfsys@transformshift{2.742608in}{4.333954in}%
\pgfsys@useobject{currentmarker}{}%
\end{pgfscope}%
\begin{pgfscope}%
\pgfsys@transformshift{2.726177in}{4.147452in}%
\pgfsys@useobject{currentmarker}{}%
\end{pgfscope}%
\begin{pgfscope}%
\pgfsys@transformshift{2.706226in}{4.064115in}%
\pgfsys@useobject{currentmarker}{}%
\end{pgfscope}%
\begin{pgfscope}%
\pgfsys@transformshift{2.687447in}{4.042393in}%
\pgfsys@useobject{currentmarker}{}%
\end{pgfscope}%
\begin{pgfscope}%
\pgfsys@transformshift{2.666557in}{4.035212in}%
\pgfsys@useobject{currentmarker}{}%
\end{pgfscope}%
\begin{pgfscope}%
\pgfsys@transformshift{2.648247in}{4.035311in}%
\pgfsys@useobject{currentmarker}{}%
\end{pgfscope}%
\begin{pgfscope}%
\pgfsys@transformshift{2.629468in}{4.039039in}%
\pgfsys@useobject{currentmarker}{}%
\end{pgfscope}%
\begin{pgfscope}%
\pgfsys@transformshift{2.610456in}{4.051349in}%
\pgfsys@useobject{currentmarker}{}%
\end{pgfscope}%
\begin{pgfscope}%
\pgfsys@transformshift{2.590972in}{4.079741in}%
\pgfsys@useobject{currentmarker}{}%
\end{pgfscope}%
\begin{pgfscope}%
\pgfsys@transformshift{2.572664in}{4.179938in}%
\pgfsys@useobject{currentmarker}{}%
\end{pgfscope}%
\begin{pgfscope}%
\pgfsys@transformshift{2.550834in}{4.350270in}%
\pgfsys@useobject{currentmarker}{}%
\end{pgfscope}%
\begin{pgfscope}%
\pgfsys@transformshift{2.532526in}{4.407834in}%
\pgfsys@useobject{currentmarker}{}%
\end{pgfscope}%
\begin{pgfscope}%
\pgfsys@transformshift{2.514687in}{4.385810in}%
\pgfsys@useobject{currentmarker}{}%
\end{pgfscope}%
\begin{pgfscope}%
\pgfsys@transformshift{2.494499in}{4.225364in}%
\pgfsys@useobject{currentmarker}{}%
\end{pgfscope}%
\begin{pgfscope}%
\pgfsys@transformshift{2.476190in}{4.103832in}%
\pgfsys@useobject{currentmarker}{}%
\end{pgfscope}%
\begin{pgfscope}%
\pgfsys@transformshift{2.457413in}{4.052386in}%
\pgfsys@useobject{currentmarker}{}%
\end{pgfscope}%
\begin{pgfscope}%
\pgfsys@transformshift{2.438165in}{4.038706in}%
\pgfsys@useobject{currentmarker}{}%
\end{pgfscope}%
\begin{pgfscope}%
\pgfsys@transformshift{2.417272in}{4.034961in}%
\pgfsys@useobject{currentmarker}{}%
\end{pgfscope}%
\begin{pgfscope}%
\pgfsys@transformshift{2.398496in}{4.035834in}%
\pgfsys@useobject{currentmarker}{}%
\end{pgfscope}%
\begin{pgfscope}%
\pgfsys@transformshift{2.380186in}{4.041292in}%
\pgfsys@useobject{currentmarker}{}%
\end{pgfscope}%
\begin{pgfscope}%
\pgfsys@transformshift{2.361409in}{4.059687in}%
\pgfsys@useobject{currentmarker}{}%
\end{pgfscope}%
\begin{pgfscope}%
\pgfsys@transformshift{2.342630in}{4.093573in}%
\pgfsys@useobject{currentmarker}{}%
\end{pgfscope}%
\begin{pgfscope}%
\pgfsys@transformshift{2.321503in}{4.204186in}%
\pgfsys@useobject{currentmarker}{}%
\end{pgfscope}%
\begin{pgfscope}%
\pgfsys@transformshift{2.302492in}{4.306009in}%
\pgfsys@useobject{currentmarker}{}%
\end{pgfscope}%
\begin{pgfscope}%
\pgfsys@transformshift{2.284182in}{4.406977in}%
\pgfsys@useobject{currentmarker}{}%
\end{pgfscope}%
\begin{pgfscope}%
\pgfsys@transformshift{2.266108in}{4.359734in}%
\pgfsys@useobject{currentmarker}{}%
\end{pgfscope}%
\begin{pgfscope}%
\pgfsys@transformshift{2.242869in}{4.139146in}%
\pgfsys@useobject{currentmarker}{}%
\end{pgfscope}%
\begin{pgfscope}%
\pgfsys@transformshift{2.224090in}{4.066756in}%
\pgfsys@useobject{currentmarker}{}%
\end{pgfscope}%
\begin{pgfscope}%
\pgfsys@transformshift{2.206017in}{4.046346in}%
\pgfsys@useobject{currentmarker}{}%
\end{pgfscope}%
\begin{pgfscope}%
\pgfsys@transformshift{2.187709in}{4.038043in}%
\pgfsys@useobject{currentmarker}{}%
\end{pgfscope}%
\begin{pgfscope}%
\pgfsys@transformshift{2.169633in}{4.035071in}%
\pgfsys@useobject{currentmarker}{}%
\end{pgfscope}%
\begin{pgfscope}%
\pgfsys@transformshift{2.146865in}{4.037808in}%
\pgfsys@useobject{currentmarker}{}%
\end{pgfscope}%
\begin{pgfscope}%
\pgfsys@transformshift{2.129026in}{4.047017in}%
\pgfsys@useobject{currentmarker}{}%
\end{pgfscope}%
\begin{pgfscope}%
\pgfsys@transformshift{2.111187in}{4.060991in}%
\pgfsys@useobject{currentmarker}{}%
\end{pgfscope}%
\begin{pgfscope}%
\pgfsys@transformshift{2.092174in}{4.122048in}%
\pgfsys@useobject{currentmarker}{}%
\end{pgfscope}%
\begin{pgfscope}%
\pgfsys@transformshift{2.073629in}{4.236674in}%
\pgfsys@useobject{currentmarker}{}%
\end{pgfscope}%
\begin{pgfscope}%
\pgfsys@transformshift{2.052973in}{4.388465in}%
\pgfsys@useobject{currentmarker}{}%
\end{pgfscope}%
\begin{pgfscope}%
\pgfsys@transformshift{2.033491in}{4.402256in}%
\pgfsys@useobject{currentmarker}{}%
\end{pgfscope}%
\begin{pgfscope}%
\pgfsys@transformshift{2.014478in}{4.281102in}%
\pgfsys@useobject{currentmarker}{}%
\end{pgfscope}%
\begin{pgfscope}%
\pgfsys@transformshift{1.993118in}{4.125800in}%
\pgfsys@useobject{currentmarker}{}%
\end{pgfscope}%
\begin{pgfscope}%
\pgfsys@transformshift{1.974105in}{4.066226in}%
\pgfsys@useobject{currentmarker}{}%
\end{pgfscope}%
\begin{pgfscope}%
\pgfsys@transformshift{1.956500in}{4.043996in}%
\pgfsys@useobject{currentmarker}{}%
\end{pgfscope}%
\begin{pgfscope}%
\pgfsys@transformshift{1.938190in}{4.037475in}%
\pgfsys@useobject{currentmarker}{}%
\end{pgfscope}%
\begin{pgfscope}%
\pgfsys@transformshift{1.919648in}{4.035100in}%
\pgfsys@useobject{currentmarker}{}%
\end{pgfscope}%
\begin{pgfscope}%
\pgfsys@transformshift{1.897349in}{4.037741in}%
\pgfsys@useobject{currentmarker}{}%
\end{pgfscope}%
\begin{pgfscope}%
\pgfsys@transformshift{1.882795in}{4.041247in}%
\pgfsys@useobject{currentmarker}{}%
\end{pgfscope}%
\begin{pgfscope}%
\pgfsys@transformshift{1.860965in}{4.038244in}%
\pgfsys@useobject{currentmarker}{}%
\end{pgfscope}%
\begin{pgfscope}%
\pgfsys@transformshift{1.843126in}{4.049695in}%
\pgfsys@useobject{currentmarker}{}%
\end{pgfscope}%
\begin{pgfscope}%
\pgfsys@transformshift{1.819887in}{4.077367in}%
\pgfsys@useobject{currentmarker}{}%
\end{pgfscope}%
\begin{pgfscope}%
\pgfsys@transformshift{1.801814in}{4.187577in}%
\pgfsys@useobject{currentmarker}{}%
\end{pgfscope}%
\begin{pgfscope}%
\pgfsys@transformshift{1.785852in}{4.316652in}%
\pgfsys@useobject{currentmarker}{}%
\end{pgfscope}%
\begin{pgfscope}%
\pgfsys@transformshift{1.764961in}{4.416499in}%
\pgfsys@useobject{currentmarker}{}%
\end{pgfscope}%
\begin{pgfscope}%
\pgfsys@transformshift{1.744774in}{4.350100in}%
\pgfsys@useobject{currentmarker}{}%
\end{pgfscope}%
\begin{pgfscope}%
\pgfsys@transformshift{1.725292in}{4.196338in}%
\pgfsys@useobject{currentmarker}{}%
\end{pgfscope}%
\begin{pgfscope}%
\pgfsys@transformshift{1.707687in}{4.106327in}%
\pgfsys@useobject{currentmarker}{}%
\end{pgfscope}%
\begin{pgfscope}%
\pgfsys@transformshift{1.690316in}{4.065797in}%
\pgfsys@useobject{currentmarker}{}%
\end{pgfscope}%
\begin{pgfscope}%
\pgfsys@transformshift{1.668721in}{4.044895in}%
\pgfsys@useobject{currentmarker}{}%
\end{pgfscope}%
\begin{pgfscope}%
\pgfsys@transformshift{1.650178in}{4.037064in}%
\pgfsys@useobject{currentmarker}{}%
\end{pgfscope}%
\begin{pgfscope}%
\pgfsys@transformshift{1.632573in}{4.035372in}%
\pgfsys@useobject{currentmarker}{}%
\end{pgfscope}%
\begin{pgfscope}%
\pgfsys@transformshift{1.612855in}{4.036784in}%
\pgfsys@useobject{currentmarker}{}%
\end{pgfscope}%
\begin{pgfscope}%
\pgfsys@transformshift{1.588444in}{4.047451in}%
\pgfsys@useobject{currentmarker}{}%
\end{pgfscope}%
\begin{pgfscope}%
\pgfsys@transformshift{1.574125in}{4.057688in}%
\pgfsys@useobject{currentmarker}{}%
\end{pgfscope}%
\begin{pgfscope}%
\pgfsys@transformshift{1.554174in}{4.106350in}%
\pgfsys@useobject{currentmarker}{}%
\end{pgfscope}%
\begin{pgfscope}%
\pgfsys@transformshift{1.529762in}{4.241791in}%
\pgfsys@useobject{currentmarker}{}%
\end{pgfscope}%
\begin{pgfscope}%
\pgfsys@transformshift{1.514270in}{4.339633in}%
\pgfsys@useobject{currentmarker}{}%
\end{pgfscope}%
\begin{pgfscope}%
\pgfsys@transformshift{1.495021in}{4.421802in}%
\pgfsys@useobject{currentmarker}{}%
\end{pgfscope}%
\begin{pgfscope}%
\pgfsys@transformshift{1.476244in}{4.412995in}%
\pgfsys@useobject{currentmarker}{}%
\end{pgfscope}%
\begin{pgfscope}%
\pgfsys@transformshift{1.457934in}{4.248728in}%
\pgfsys@useobject{currentmarker}{}%
\end{pgfscope}%
\begin{pgfscope}%
\pgfsys@transformshift{1.437983in}{4.140372in}%
\pgfsys@useobject{currentmarker}{}%
\end{pgfscope}%
\begin{pgfscope}%
\pgfsys@transformshift{1.419907in}{4.074963in}%
\pgfsys@useobject{currentmarker}{}%
\end{pgfscope}%
\begin{pgfscope}%
\pgfsys@transformshift{1.398548in}{4.047837in}%
\pgfsys@useobject{currentmarker}{}%
\end{pgfscope}%
\begin{pgfscope}%
\pgfsys@transformshift{1.380004in}{4.038602in}%
\pgfsys@useobject{currentmarker}{}%
\end{pgfscope}%
\begin{pgfscope}%
\pgfsys@transformshift{1.360287in}{4.035960in}%
\pgfsys@useobject{currentmarker}{}%
\end{pgfscope}%
\begin{pgfscope}%
\pgfsys@transformshift{1.343387in}{4.035904in}%
\pgfsys@useobject{currentmarker}{}%
\end{pgfscope}%
\begin{pgfscope}%
\pgfsys@transformshift{1.321323in}{4.037646in}%
\pgfsys@useobject{currentmarker}{}%
\end{pgfscope}%
\begin{pgfscope}%
\pgfsys@transformshift{1.303013in}{4.047091in}%
\pgfsys@useobject{currentmarker}{}%
\end{pgfscope}%
\begin{pgfscope}%
\pgfsys@transformshift{1.284939in}{4.074110in}%
\pgfsys@useobject{currentmarker}{}%
\end{pgfscope}%
\begin{pgfscope}%
\pgfsys@transformshift{1.265692in}{4.138602in}%
\pgfsys@useobject{currentmarker}{}%
\end{pgfscope}%
\begin{pgfscope}%
\pgfsys@transformshift{1.247147in}{4.279744in}%
\pgfsys@useobject{currentmarker}{}%
\end{pgfscope}%
\begin{pgfscope}%
\pgfsys@transformshift{1.226257in}{4.382219in}%
\pgfsys@useobject{currentmarker}{}%
\end{pgfscope}%
\begin{pgfscope}%
\pgfsys@transformshift{1.207243in}{4.429874in}%
\pgfsys@useobject{currentmarker}{}%
\end{pgfscope}%
\begin{pgfscope}%
\pgfsys@transformshift{1.188230in}{4.425413in}%
\pgfsys@useobject{currentmarker}{}%
\end{pgfscope}%
\begin{pgfscope}%
\pgfsys@transformshift{1.169922in}{4.275929in}%
\pgfsys@useobject{currentmarker}{}%
\end{pgfscope}%
\begin{pgfscope}%
\pgfsys@transformshift{1.144806in}{4.142211in}%
\pgfsys@useobject{currentmarker}{}%
\end{pgfscope}%
\begin{pgfscope}%
\pgfsys@transformshift{1.129313in}{4.123549in}%
\pgfsys@useobject{currentmarker}{}%
\end{pgfscope}%
\begin{pgfscope}%
\pgfsys@transformshift{1.111005in}{4.065759in}%
\pgfsys@useobject{currentmarker}{}%
\end{pgfscope}%
\begin{pgfscope}%
\pgfsys@transformshift{1.089644in}{4.044418in}%
\pgfsys@useobject{currentmarker}{}%
\end{pgfscope}%
\begin{pgfscope}%
\pgfsys@transformshift{1.072039in}{4.037979in}%
\pgfsys@useobject{currentmarker}{}%
\end{pgfscope}%
\begin{pgfscope}%
\pgfsys@transformshift{1.052791in}{4.036281in}%
\pgfsys@useobject{currentmarker}{}%
\end{pgfscope}%
\begin{pgfscope}%
\pgfsys@transformshift{1.034717in}{4.039648in}%
\pgfsys@useobject{currentmarker}{}%
\end{pgfscope}%
\begin{pgfscope}%
\pgfsys@transformshift{1.016173in}{4.051074in}%
\pgfsys@useobject{currentmarker}{}%
\end{pgfscope}%
\begin{pgfscope}%
\pgfsys@transformshift{0.996691in}{4.072592in}%
\pgfsys@useobject{currentmarker}{}%
\end{pgfscope}%
\begin{pgfscope}%
\pgfsys@transformshift{0.976035in}{4.161847in}%
\pgfsys@useobject{currentmarker}{}%
\end{pgfscope}%
\begin{pgfscope}%
\pgfsys@transformshift{0.957492in}{4.245376in}%
\pgfsys@useobject{currentmarker}{}%
\end{pgfscope}%
\begin{pgfscope}%
\pgfsys@transformshift{0.938479in}{4.273547in}%
\pgfsys@useobject{currentmarker}{}%
\end{pgfscope}%
\begin{pgfscope}%
\pgfsys@transformshift{0.919935in}{4.408392in}%
\pgfsys@useobject{currentmarker}{}%
\end{pgfscope}%
\begin{pgfscope}%
\pgfsys@transformshift{0.899513in}{4.444531in}%
\pgfsys@useobject{currentmarker}{}%
\end{pgfscope}%
\begin{pgfscope}%
\pgfsys@transformshift{0.880970in}{4.378918in}%
\pgfsys@useobject{currentmarker}{}%
\end{pgfscope}%
\begin{pgfscope}%
\pgfsys@transformshift{0.861957in}{4.235099in}%
\pgfsys@useobject{currentmarker}{}%
\end{pgfscope}%
\begin{pgfscope}%
\pgfsys@transformshift{0.842475in}{4.103987in}%
\pgfsys@useobject{currentmarker}{}%
\end{pgfscope}%
\begin{pgfscope}%
\pgfsys@transformshift{0.823931in}{4.062926in}%
\pgfsys@useobject{currentmarker}{}%
\end{pgfscope}%
\begin{pgfscope}%
\pgfsys@transformshift{0.803274in}{4.046213in}%
\pgfsys@useobject{currentmarker}{}%
\end{pgfscope}%
\begin{pgfscope}%
\pgfsys@transformshift{0.787313in}{4.040660in}%
\pgfsys@useobject{currentmarker}{}%
\end{pgfscope}%
\begin{pgfscope}%
\pgfsys@transformshift{0.766188in}{4.036468in}%
\pgfsys@useobject{currentmarker}{}%
\end{pgfscope}%
\begin{pgfscope}%
\pgfsys@transformshift{0.744592in}{4.040793in}%
\pgfsys@useobject{currentmarker}{}%
\end{pgfscope}%
\begin{pgfscope}%
\pgfsys@transformshift{0.726518in}{4.270093in}%
\pgfsys@useobject{currentmarker}{}%
\end{pgfscope}%
\begin{pgfscope}%
\pgfsys@transformshift{0.708679in}{4.127173in}%
\pgfsys@useobject{currentmarker}{}%
\end{pgfscope}%
\begin{pgfscope}%
\pgfsys@transformshift{0.687318in}{4.060123in}%
\pgfsys@useobject{currentmarker}{}%
\end{pgfscope}%
\begin{pgfscope}%
\pgfsys@transformshift{0.671356in}{4.042898in}%
\pgfsys@useobject{currentmarker}{}%
\end{pgfscope}%
\begin{pgfscope}%
\pgfsys@transformshift{0.652108in}{4.036913in}%
\pgfsys@useobject{currentmarker}{}%
\end{pgfscope}%
\begin{pgfscope}%
\pgfsys@transformshift{0.653282in}{4.036809in}%
\pgfsys@useobject{currentmarker}{}%
\end{pgfscope}%
\begin{pgfscope}%
\pgfsys@transformshift{0.657273in}{4.038423in}%
\pgfsys@useobject{currentmarker}{}%
\end{pgfscope}%
\begin{pgfscope}%
\pgfsys@transformshift{0.675815in}{4.053006in}%
\pgfsys@useobject{currentmarker}{}%
\end{pgfscope}%
\begin{pgfscope}%
\pgfsys@transformshift{0.694125in}{4.102571in}%
\pgfsys@useobject{currentmarker}{}%
\end{pgfscope}%
\begin{pgfscope}%
\pgfsys@transformshift{0.714311in}{4.331587in}%
\pgfsys@useobject{currentmarker}{}%
\end{pgfscope}%
\begin{pgfscope}%
\pgfsys@transformshift{0.735907in}{4.447600in}%
\pgfsys@useobject{currentmarker}{}%
\end{pgfscope}%
\begin{pgfscope}%
\pgfsys@transformshift{0.750694in}{4.363992in}%
\pgfsys@useobject{currentmarker}{}%
\end{pgfscope}%
\begin{pgfscope}%
\pgfsys@transformshift{0.768299in}{4.156045in}%
\pgfsys@useobject{currentmarker}{}%
\end{pgfscope}%
\begin{pgfscope}%
\pgfsys@transformshift{0.791069in}{4.058628in}%
\pgfsys@useobject{currentmarker}{}%
\end{pgfscope}%
\begin{pgfscope}%
\pgfsys@transformshift{0.809143in}{4.039519in}%
\pgfsys@useobject{currentmarker}{}%
\end{pgfscope}%
\begin{pgfscope}%
\pgfsys@transformshift{0.829094in}{4.036557in}%
\pgfsys@useobject{currentmarker}{}%
\end{pgfscope}%
\begin{pgfscope}%
\pgfsys@transformshift{0.849515in}{4.048240in}%
\pgfsys@useobject{currentmarker}{}%
\end{pgfscope}%
\begin{pgfscope}%
\pgfsys@transformshift{0.868294in}{4.097271in}%
\pgfsys@useobject{currentmarker}{}%
\end{pgfscope}%
\begin{pgfscope}%
\pgfsys@transformshift{0.889185in}{4.272206in}%
\pgfsys@useobject{currentmarker}{}%
\end{pgfscope}%
\begin{pgfscope}%
\pgfsys@transformshift{0.907495in}{4.441223in}%
\pgfsys@useobject{currentmarker}{}%
\end{pgfscope}%
\begin{pgfscope}%
\pgfsys@transformshift{0.924160in}{4.381471in}%
\pgfsys@useobject{currentmarker}{}%
\end{pgfscope}%
\begin{pgfscope}%
\pgfsys@transformshift{0.943642in}{4.174208in}%
\pgfsys@useobject{currentmarker}{}%
\end{pgfscope}%
\begin{pgfscope}%
\pgfsys@transformshift{0.963595in}{4.061819in}%
\pgfsys@useobject{currentmarker}{}%
\end{pgfscope}%
\begin{pgfscope}%
\pgfsys@transformshift{0.984486in}{4.040087in}%
\pgfsys@useobject{currentmarker}{}%
\end{pgfscope}%
\begin{pgfscope}%
\pgfsys@transformshift{1.001856in}{4.035784in}%
\pgfsys@useobject{currentmarker}{}%
\end{pgfscope}%
\begin{pgfscope}%
\pgfsys@transformshift{1.021338in}{4.041034in}%
\pgfsys@useobject{currentmarker}{}%
\end{pgfscope}%
\begin{pgfscope}%
\pgfsys@transformshift{1.041760in}{4.067323in}%
\pgfsys@useobject{currentmarker}{}%
\end{pgfscope}%
\begin{pgfscope}%
\pgfsys@transformshift{1.060068in}{4.162499in}%
\pgfsys@useobject{currentmarker}{}%
\end{pgfscope}%
\begin{pgfscope}%
\pgfsys@transformshift{1.078612in}{4.399134in}%
\pgfsys@useobject{currentmarker}{}%
\end{pgfscope}%
\begin{pgfscope}%
\pgfsys@transformshift{1.101380in}{4.409228in}%
\pgfsys@useobject{currentmarker}{}%
\end{pgfscope}%
\begin{pgfscope}%
\pgfsys@transformshift{1.118047in}{4.266299in}%
\pgfsys@useobject{currentmarker}{}%
\end{pgfscope}%
\begin{pgfscope}%
\pgfsys@transformshift{1.136590in}{4.093964in}%
\pgfsys@useobject{currentmarker}{}%
\end{pgfscope}%
\begin{pgfscope}%
\pgfsys@transformshift{1.156072in}{4.047408in}%
\pgfsys@useobject{currentmarker}{}%
\end{pgfscope}%
\begin{pgfscope}%
\pgfsys@transformshift{1.173677in}{4.039057in}%
\pgfsys@useobject{currentmarker}{}%
\end{pgfscope}%
\begin{pgfscope}%
\pgfsys@transformshift{1.195507in}{4.088075in}%
\pgfsys@useobject{currentmarker}{}%
\end{pgfscope}%
\begin{pgfscope}%
\pgfsys@transformshift{1.216397in}{4.048613in}%
\pgfsys@useobject{currentmarker}{}%
\end{pgfscope}%
\begin{pgfscope}%
\pgfsys@transformshift{1.232594in}{4.038074in}%
\pgfsys@useobject{currentmarker}{}%
\end{pgfscope}%
\begin{pgfscope}%
\pgfsys@transformshift{1.252312in}{4.035769in}%
\pgfsys@useobject{currentmarker}{}%
\end{pgfscope}%
\begin{pgfscope}%
\pgfsys@transformshift{1.271089in}{4.042001in}%
\pgfsys@useobject{currentmarker}{}%
\end{pgfscope}%
\begin{pgfscope}%
\pgfsys@transformshift{1.293390in}{4.078948in}%
\pgfsys@useobject{currentmarker}{}%
\end{pgfscope}%
\begin{pgfscope}%
\pgfsys@transformshift{1.309350in}{4.182022in}%
\pgfsys@useobject{currentmarker}{}%
\end{pgfscope}%
\begin{pgfscope}%
\pgfsys@transformshift{1.329303in}{4.081722in}%
\pgfsys@useobject{currentmarker}{}%
\end{pgfscope}%
\begin{pgfscope}%
\pgfsys@transformshift{1.347142in}{4.224297in}%
\pgfsys@useobject{currentmarker}{}%
\end{pgfscope}%
\begin{pgfscope}%
\pgfsys@transformshift{1.368033in}{4.422889in}%
\pgfsys@useobject{currentmarker}{}%
\end{pgfscope}%
\begin{pgfscope}%
\pgfsys@transformshift{1.386811in}{4.368058in}%
\pgfsys@useobject{currentmarker}{}%
\end{pgfscope}%
\begin{pgfscope}%
\pgfsys@transformshift{1.405825in}{4.201307in}%
\pgfsys@useobject{currentmarker}{}%
\end{pgfscope}%
\begin{pgfscope}%
\pgfsys@transformshift{1.427420in}{4.073142in}%
\pgfsys@useobject{currentmarker}{}%
\end{pgfscope}%
\begin{pgfscope}%
\pgfsys@transformshift{1.445494in}{4.041942in}%
\pgfsys@useobject{currentmarker}{}%
\end{pgfscope}%
\begin{pgfscope}%
\pgfsys@transformshift{1.464037in}{4.035532in}%
\pgfsys@useobject{currentmarker}{}%
\end{pgfscope}%
\begin{pgfscope}%
\pgfsys@transformshift{1.482347in}{4.037048in}%
\pgfsys@useobject{currentmarker}{}%
\end{pgfscope}%
\begin{pgfscope}%
\pgfsys@transformshift{1.502063in}{4.048968in}%
\pgfsys@useobject{currentmarker}{}%
\end{pgfscope}%
\begin{pgfscope}%
\pgfsys@transformshift{1.521076in}{4.077065in}%
\pgfsys@useobject{currentmarker}{}%
\end{pgfscope}%
\begin{pgfscope}%
\pgfsys@transformshift{1.539855in}{4.134282in}%
\pgfsys@useobject{currentmarker}{}%
\end{pgfscope}%
\begin{pgfscope}%
\pgfsys@transformshift{1.558868in}{4.365859in}%
\pgfsys@useobject{currentmarker}{}%
\end{pgfscope}%
\begin{pgfscope}%
\pgfsys@transformshift{1.580699in}{4.406602in}%
\pgfsys@useobject{currentmarker}{}%
\end{pgfscope}%
\begin{pgfscope}%
\pgfsys@transformshift{1.596659in}{4.294845in}%
\pgfsys@useobject{currentmarker}{}%
\end{pgfscope}%
\begin{pgfscope}%
\pgfsys@transformshift{1.615203in}{4.119517in}%
\pgfsys@useobject{currentmarker}{}%
\end{pgfscope}%
\begin{pgfscope}%
\pgfsys@transformshift{1.636799in}{4.054019in}%
\pgfsys@useobject{currentmarker}{}%
\end{pgfscope}%
\begin{pgfscope}%
\pgfsys@transformshift{1.655341in}{4.038358in}%
\pgfsys@useobject{currentmarker}{}%
\end{pgfscope}%
\begin{pgfscope}%
\pgfsys@transformshift{1.677406in}{4.035116in}%
\pgfsys@useobject{currentmarker}{}%
\end{pgfscope}%
\begin{pgfscope}%
\pgfsys@transformshift{1.693133in}{4.037410in}%
\pgfsys@useobject{currentmarker}{}%
\end{pgfscope}%
\begin{pgfscope}%
\pgfsys@transformshift{1.714729in}{4.053068in}%
\pgfsys@useobject{currentmarker}{}%
\end{pgfscope}%
\begin{pgfscope}%
\pgfsys@transformshift{1.733037in}{4.098879in}%
\pgfsys@useobject{currentmarker}{}%
\end{pgfscope}%
\begin{pgfscope}%
\pgfsys@transformshift{1.752285in}{4.272977in}%
\pgfsys@useobject{currentmarker}{}%
\end{pgfscope}%
\begin{pgfscope}%
\pgfsys@transformshift{1.770829in}{4.418313in}%
\pgfsys@useobject{currentmarker}{}%
\end{pgfscope}%
\begin{pgfscope}%
\pgfsys@transformshift{1.790077in}{4.366322in}%
\pgfsys@useobject{currentmarker}{}%
\end{pgfscope}%
\begin{pgfscope}%
\pgfsys@transformshift{1.811202in}{4.192040in}%
\pgfsys@useobject{currentmarker}{}%
\end{pgfscope}%
\begin{pgfscope}%
\pgfsys@transformshift{1.827398in}{4.103148in}%
\pgfsys@useobject{currentmarker}{}%
\end{pgfscope}%
\begin{pgfscope}%
\pgfsys@transformshift{1.849229in}{4.048462in}%
\pgfsys@useobject{currentmarker}{}%
\end{pgfscope}%
\begin{pgfscope}%
\pgfsys@transformshift{1.868476in}{4.037881in}%
\pgfsys@useobject{currentmarker}{}%
\end{pgfscope}%
\begin{pgfscope}%
\pgfsys@transformshift{1.886550in}{4.034813in}%
\pgfsys@useobject{currentmarker}{}%
\end{pgfscope}%
\begin{pgfscope}%
\pgfsys@transformshift{1.903920in}{4.035993in}%
\pgfsys@useobject{currentmarker}{}%
\end{pgfscope}%
\begin{pgfscope}%
\pgfsys@transformshift{1.926454in}{4.047646in}%
\pgfsys@useobject{currentmarker}{}%
\end{pgfscope}%
\begin{pgfscope}%
\pgfsys@transformshift{1.945232in}{4.078538in}%
\pgfsys@useobject{currentmarker}{}%
\end{pgfscope}%
\begin{pgfscope}%
\pgfsys@transformshift{1.961663in}{4.179086in}%
\pgfsys@useobject{currentmarker}{}%
\end{pgfscope}%
\begin{pgfscope}%
\pgfsys@transformshift{1.981616in}{4.390067in}%
\pgfsys@useobject{currentmarker}{}%
\end{pgfscope}%
\begin{pgfscope}%
\pgfsys@transformshift{1.999924in}{4.413401in}%
\pgfsys@useobject{currentmarker}{}%
\end{pgfscope}%
\begin{pgfscope}%
\pgfsys@transformshift{2.021051in}{4.287366in}%
\pgfsys@useobject{currentmarker}{}%
\end{pgfscope}%
\begin{pgfscope}%
\pgfsys@transformshift{2.039359in}{4.115599in}%
\pgfsys@useobject{currentmarker}{}%
\end{pgfscope}%
\begin{pgfscope}%
\pgfsys@transformshift{2.060250in}{4.051445in}%
\pgfsys@useobject{currentmarker}{}%
\end{pgfscope}%
\begin{pgfscope}%
\pgfsys@transformshift{2.078794in}{4.039100in}%
\pgfsys@useobject{currentmarker}{}%
\end{pgfscope}%
\begin{pgfscope}%
\pgfsys@transformshift{2.098980in}{4.035983in}%
\pgfsys@useobject{currentmarker}{}%
\end{pgfscope}%
\begin{pgfscope}%
\pgfsys@transformshift{2.116821in}{4.034857in}%
\pgfsys@useobject{currentmarker}{}%
\end{pgfscope}%
\begin{pgfscope}%
\pgfsys@transformshift{2.133720in}{4.037519in}%
\pgfsys@useobject{currentmarker}{}%
\end{pgfscope}%
\begin{pgfscope}%
\pgfsys@transformshift{2.154376in}{4.050029in}%
\pgfsys@useobject{currentmarker}{}%
\end{pgfscope}%
\begin{pgfscope}%
\pgfsys@transformshift{2.175267in}{4.084272in}%
\pgfsys@useobject{currentmarker}{}%
\end{pgfscope}%
\begin{pgfscope}%
\pgfsys@transformshift{2.195689in}{4.226955in}%
\pgfsys@useobject{currentmarker}{}%
\end{pgfscope}%
\begin{pgfscope}%
\pgfsys@transformshift{2.212825in}{4.377866in}%
\pgfsys@useobject{currentmarker}{}%
\end{pgfscope}%
\begin{pgfscope}%
\pgfsys@transformshift{2.232541in}{4.407556in}%
\pgfsys@useobject{currentmarker}{}%
\end{pgfscope}%
\begin{pgfscope}%
\pgfsys@transformshift{2.251554in}{4.279456in}%
\pgfsys@useobject{currentmarker}{}%
\end{pgfscope}%
\begin{pgfscope}%
\pgfsys@transformshift{2.269628in}{4.363298in}%
\pgfsys@useobject{currentmarker}{}%
\end{pgfscope}%
\begin{pgfscope}%
\pgfsys@transformshift{2.290284in}{4.299853in}%
\pgfsys@useobject{currentmarker}{}%
\end{pgfscope}%
\begin{pgfscope}%
\pgfsys@transformshift{2.308360in}{4.168341in}%
\pgfsys@useobject{currentmarker}{}%
\end{pgfscope}%
\begin{pgfscope}%
\pgfsys@transformshift{2.325728in}{4.066489in}%
\pgfsys@useobject{currentmarker}{}%
\end{pgfscope}%
\begin{pgfscope}%
\pgfsys@transformshift{2.347324in}{4.040672in}%
\pgfsys@useobject{currentmarker}{}%
\end{pgfscope}%
\begin{pgfscope}%
\pgfsys@transformshift{2.368215in}{4.035067in}%
\pgfsys@useobject{currentmarker}{}%
\end{pgfscope}%
\begin{pgfscope}%
\pgfsys@transformshift{2.388636in}{4.036422in}%
\pgfsys@useobject{currentmarker}{}%
\end{pgfscope}%
\begin{pgfscope}%
\pgfsys@transformshift{2.405067in}{4.041883in}%
\pgfsys@useobject{currentmarker}{}%
\end{pgfscope}%
\begin{pgfscope}%
\pgfsys@transformshift{2.424785in}{4.066185in}%
\pgfsys@useobject{currentmarker}{}%
\end{pgfscope}%
\begin{pgfscope}%
\pgfsys@transformshift{2.442390in}{4.147264in}%
\pgfsys@useobject{currentmarker}{}%
\end{pgfscope}%
\begin{pgfscope}%
\pgfsys@transformshift{2.461167in}{4.350260in}%
\pgfsys@useobject{currentmarker}{}%
\end{pgfscope}%
\begin{pgfscope}%
\pgfsys@transformshift{2.480886in}{4.410964in}%
\pgfsys@useobject{currentmarker}{}%
\end{pgfscope}%
\begin{pgfscope}%
\pgfsys@transformshift{2.502245in}{4.321674in}%
\pgfsys@useobject{currentmarker}{}%
\end{pgfscope}%
\begin{pgfscope}%
\pgfsys@transformshift{2.520084in}{4.159609in}%
\pgfsys@useobject{currentmarker}{}%
\end{pgfscope}%
\begin{pgfscope}%
\pgfsys@transformshift{2.536986in}{4.082310in}%
\pgfsys@useobject{currentmarker}{}%
\end{pgfscope}%
\begin{pgfscope}%
\pgfsys@transformshift{2.555999in}{4.045225in}%
\pgfsys@useobject{currentmarker}{}%
\end{pgfscope}%
\begin{pgfscope}%
\pgfsys@transformshift{2.580175in}{4.036333in}%
\pgfsys@useobject{currentmarker}{}%
\end{pgfscope}%
\begin{pgfscope}%
\pgfsys@transformshift{2.598015in}{4.034888in}%
\pgfsys@useobject{currentmarker}{}%
\end{pgfscope}%
\begin{pgfscope}%
\pgfsys@transformshift{2.615619in}{4.037983in}%
\pgfsys@useobject{currentmarker}{}%
\end{pgfscope}%
\begin{pgfscope}%
\pgfsys@transformshift{2.636510in}{4.047934in}%
\pgfsys@useobject{currentmarker}{}%
\end{pgfscope}%
\begin{pgfscope}%
\pgfsys@transformshift{2.654349in}{4.064497in}%
\pgfsys@useobject{currentmarker}{}%
\end{pgfscope}%
\begin{pgfscope}%
\pgfsys@transformshift{2.676414in}{4.113214in}%
\pgfsys@useobject{currentmarker}{}%
\end{pgfscope}%
\begin{pgfscope}%
\pgfsys@transformshift{2.693784in}{4.259824in}%
\pgfsys@useobject{currentmarker}{}%
\end{pgfscope}%
\begin{pgfscope}%
\pgfsys@transformshift{2.712329in}{4.403359in}%
\pgfsys@useobject{currentmarker}{}%
\end{pgfscope}%
\begin{pgfscope}%
\pgfsys@transformshift{2.729699in}{4.387634in}%
\pgfsys@useobject{currentmarker}{}%
\end{pgfscope}%
\begin{pgfscope}%
\pgfsys@transformshift{2.750824in}{4.234116in}%
\pgfsys@useobject{currentmarker}{}%
\end{pgfscope}%
\begin{pgfscope}%
\pgfsys@transformshift{2.771480in}{4.086671in}%
\pgfsys@useobject{currentmarker}{}%
\end{pgfscope}%
\begin{pgfscope}%
\pgfsys@transformshift{2.790728in}{4.050318in}%
\pgfsys@useobject{currentmarker}{}%
\end{pgfscope}%
\begin{pgfscope}%
\pgfsys@transformshift{2.807393in}{4.039136in}%
\pgfsys@useobject{currentmarker}{}%
\end{pgfscope}%
\begin{pgfscope}%
\pgfsys@transformshift{2.828520in}{4.035051in}%
\pgfsys@useobject{currentmarker}{}%
\end{pgfscope}%
\begin{pgfscope}%
\pgfsys@transformshift{2.846125in}{4.035181in}%
\pgfsys@useobject{currentmarker}{}%
\end{pgfscope}%
\begin{pgfscope}%
\pgfsys@transformshift{2.868424in}{4.041226in}%
\pgfsys@useobject{currentmarker}{}%
\end{pgfscope}%
\begin{pgfscope}%
\pgfsys@transformshift{2.886263in}{4.055080in}%
\pgfsys@useobject{currentmarker}{}%
\end{pgfscope}%
\begin{pgfscope}%
\pgfsys@transformshift{2.903868in}{4.085607in}%
\pgfsys@useobject{currentmarker}{}%
\end{pgfscope}%
\begin{pgfscope}%
\pgfsys@transformshift{2.921707in}{4.188191in}%
\pgfsys@useobject{currentmarker}{}%
\end{pgfscope}%
\begin{pgfscope}%
\pgfsys@transformshift{2.942598in}{4.389293in}%
\pgfsys@useobject{currentmarker}{}%
\end{pgfscope}%
\begin{pgfscope}%
\pgfsys@transformshift{2.960671in}{4.404967in}%
\pgfsys@useobject{currentmarker}{}%
\end{pgfscope}%
\begin{pgfscope}%
\pgfsys@transformshift{2.981327in}{4.333129in}%
\pgfsys@useobject{currentmarker}{}%
\end{pgfscope}%
\begin{pgfscope}%
\pgfsys@transformshift{2.999403in}{4.165249in}%
\pgfsys@useobject{currentmarker}{}%
\end{pgfscope}%
\begin{pgfscope}%
\pgfsys@transformshift{3.017711in}{4.077583in}%
\pgfsys@useobject{currentmarker}{}%
\end{pgfscope}%
\begin{pgfscope}%
\pgfsys@transformshift{3.038602in}{4.046227in}%
\pgfsys@useobject{currentmarker}{}%
\end{pgfscope}%
\begin{pgfscope}%
\pgfsys@transformshift{3.057849in}{4.038926in}%
\pgfsys@useobject{currentmarker}{}%
\end{pgfscope}%
\begin{pgfscope}%
\pgfsys@transformshift{3.077568in}{4.035093in}%
\pgfsys@useobject{currentmarker}{}%
\end{pgfscope}%
\begin{pgfscope}%
\pgfsys@transformshift{3.101978in}{4.037112in}%
\pgfsys@useobject{currentmarker}{}%
\end{pgfscope}%
\begin{pgfscope}%
\pgfsys@transformshift{3.117940in}{4.041401in}%
\pgfsys@useobject{currentmarker}{}%
\end{pgfscope}%
\begin{pgfscope}%
\pgfsys@transformshift{3.135076in}{4.051889in}%
\pgfsys@useobject{currentmarker}{}%
\end{pgfscope}%
\begin{pgfscope}%
\pgfsys@transformshift{3.155027in}{4.090417in}%
\pgfsys@useobject{currentmarker}{}%
\end{pgfscope}%
\begin{pgfscope}%
\pgfsys@transformshift{3.174040in}{4.173336in}%
\pgfsys@useobject{currentmarker}{}%
\end{pgfscope}%
\begin{pgfscope}%
\pgfsys@transformshift{3.192350in}{4.275168in}%
\pgfsys@useobject{currentmarker}{}%
\end{pgfscope}%
\begin{pgfscope}%
\pgfsys@transformshift{3.212536in}{4.415933in}%
\pgfsys@useobject{currentmarker}{}%
\end{pgfscope}%
\begin{pgfscope}%
\pgfsys@transformshift{3.231080in}{4.377692in}%
\pgfsys@useobject{currentmarker}{}%
\end{pgfscope}%
\begin{pgfscope}%
\pgfsys@transformshift{3.253379in}{4.253385in}%
\pgfsys@useobject{currentmarker}{}%
\end{pgfscope}%
\begin{pgfscope}%
\pgfsys@transformshift{3.268872in}{4.112615in}%
\pgfsys@useobject{currentmarker}{}%
\end{pgfscope}%
\begin{pgfscope}%
\pgfsys@transformshift{3.287649in}{4.060771in}%
\pgfsys@useobject{currentmarker}{}%
\end{pgfscope}%
\begin{pgfscope}%
\pgfsys@transformshift{3.309245in}{4.043326in}%
\pgfsys@useobject{currentmarker}{}%
\end{pgfscope}%
\begin{pgfscope}%
\pgfsys@transformshift{3.326381in}{4.139637in}%
\pgfsys@useobject{currentmarker}{}%
\end{pgfscope}%
\begin{pgfscope}%
\pgfsys@transformshift{3.345392in}{4.063060in}%
\pgfsys@useobject{currentmarker}{}%
\end{pgfscope}%
\begin{pgfscope}%
\pgfsys@transformshift{3.366988in}{4.042467in}%
\pgfsys@useobject{currentmarker}{}%
\end{pgfscope}%
\begin{pgfscope}%
\pgfsys@transformshift{3.384358in}{4.036392in}%
\pgfsys@useobject{currentmarker}{}%
\end{pgfscope}%
\begin{pgfscope}%
\pgfsys@transformshift{3.402901in}{4.035584in}%
\pgfsys@useobject{currentmarker}{}%
\end{pgfscope}%
\begin{pgfscope}%
\pgfsys@transformshift{3.424028in}{4.041685in}%
\pgfsys@useobject{currentmarker}{}%
\end{pgfscope}%
\begin{pgfscope}%
\pgfsys@transformshift{3.444215in}{4.055598in}%
\pgfsys@useobject{currentmarker}{}%
\end{pgfscope}%
\begin{pgfscope}%
\pgfsys@transformshift{3.463463in}{4.100083in}%
\pgfsys@useobject{currentmarker}{}%
\end{pgfscope}%
\begin{pgfscope}%
\pgfsys@transformshift{3.480831in}{4.209476in}%
\pgfsys@useobject{currentmarker}{}%
\end{pgfscope}%
\begin{pgfscope}%
\pgfsys@transformshift{3.501253in}{4.412421in}%
\pgfsys@useobject{currentmarker}{}%
\end{pgfscope}%
\begin{pgfscope}%
\pgfsys@transformshift{3.519797in}{4.412396in}%
\pgfsys@useobject{currentmarker}{}%
\end{pgfscope}%
\begin{pgfscope}%
\pgfsys@transformshift{3.539985in}{4.297677in}%
\pgfsys@useobject{currentmarker}{}%
\end{pgfscope}%
\begin{pgfscope}%
\pgfsys@transformshift{3.558762in}{4.166618in}%
\pgfsys@useobject{currentmarker}{}%
\end{pgfscope}%
\begin{pgfscope}%
\pgfsys@transformshift{3.576601in}{4.086092in}%
\pgfsys@useobject{currentmarker}{}%
\end{pgfscope}%
\begin{pgfscope}%
\pgfsys@transformshift{3.594440in}{4.051510in}%
\pgfsys@useobject{currentmarker}{}%
\end{pgfscope}%
\begin{pgfscope}%
\pgfsys@transformshift{3.616270in}{4.038667in}%
\pgfsys@useobject{currentmarker}{}%
\end{pgfscope}%
\begin{pgfscope}%
\pgfsys@transformshift{3.633875in}{4.035493in}%
\pgfsys@useobject{currentmarker}{}%
\end{pgfscope}%
\begin{pgfscope}%
\pgfsys@transformshift{3.654766in}{4.037054in}%
\pgfsys@useobject{currentmarker}{}%
\end{pgfscope}%
\begin{pgfscope}%
\pgfsys@transformshift{3.673076in}{4.041443in}%
\pgfsys@useobject{currentmarker}{}%
\end{pgfscope}%
\begin{pgfscope}%
\pgfsys@transformshift{3.692558in}{4.052967in}%
\pgfsys@useobject{currentmarker}{}%
\end{pgfscope}%
\begin{pgfscope}%
\pgfsys@transformshift{3.713448in}{4.084291in}%
\pgfsys@useobject{currentmarker}{}%
\end{pgfscope}%
\begin{pgfscope}%
\pgfsys@transformshift{3.730350in}{4.179077in}%
\pgfsys@useobject{currentmarker}{}%
\end{pgfscope}%
\begin{pgfscope}%
\pgfsys@transformshift{3.751006in}{4.366135in}%
\pgfsys@useobject{currentmarker}{}%
\end{pgfscope}%
\begin{pgfscope}%
\pgfsys@transformshift{3.768376in}{4.431275in}%
\pgfsys@useobject{currentmarker}{}%
\end{pgfscope}%
\begin{pgfscope}%
\pgfsys@transformshift{3.786919in}{4.414071in}%
\pgfsys@useobject{currentmarker}{}%
\end{pgfscope}%
\begin{pgfscope}%
\pgfsys@transformshift{3.807106in}{4.330214in}%
\pgfsys@useobject{currentmarker}{}%
\end{pgfscope}%
\begin{pgfscope}%
\pgfsys@transformshift{3.825414in}{4.198220in}%
\pgfsys@useobject{currentmarker}{}%
\end{pgfscope}%
\begin{pgfscope}%
\pgfsys@transformshift{3.847244in}{4.089936in}%
\pgfsys@useobject{currentmarker}{}%
\end{pgfscope}%
\begin{pgfscope}%
\pgfsys@transformshift{3.865318in}{4.059543in}%
\pgfsys@useobject{currentmarker}{}%
\end{pgfscope}%
\begin{pgfscope}%
\pgfsys@transformshift{3.885036in}{4.044510in}%
\pgfsys@useobject{currentmarker}{}%
\end{pgfscope}%
\begin{pgfscope}%
\pgfsys@transformshift{3.904519in}{4.037878in}%
\pgfsys@useobject{currentmarker}{}%
\end{pgfscope}%
\begin{pgfscope}%
\pgfsys@transformshift{3.923532in}{4.036204in}%
\pgfsys@useobject{currentmarker}{}%
\end{pgfscope}%
\begin{pgfscope}%
\pgfsys@transformshift{3.942074in}{4.039056in}%
\pgfsys@useobject{currentmarker}{}%
\end{pgfscope}%
\begin{pgfscope}%
\pgfsys@transformshift{3.959915in}{4.049992in}%
\pgfsys@useobject{currentmarker}{}%
\end{pgfscope}%
\begin{pgfscope}%
\pgfsys@transformshift{3.981980in}{4.074837in}%
\pgfsys@useobject{currentmarker}{}%
\end{pgfscope}%
\begin{pgfscope}%
\pgfsys@transformshift{3.998880in}{4.064657in}%
\pgfsys@useobject{currentmarker}{}%
\end{pgfscope}%
\begin{pgfscope}%
\pgfsys@transformshift{4.020241in}{4.128723in}%
\pgfsys@useobject{currentmarker}{}%
\end{pgfscope}%
\begin{pgfscope}%
\pgfsys@transformshift{4.037844in}{4.285576in}%
\pgfsys@useobject{currentmarker}{}%
\end{pgfscope}%
\begin{pgfscope}%
\pgfsys@transformshift{4.062257in}{4.440866in}%
\pgfsys@useobject{currentmarker}{}%
\end{pgfscope}%
\begin{pgfscope}%
\pgfsys@transformshift{4.074933in}{4.439038in}%
\pgfsys@useobject{currentmarker}{}%
\end{pgfscope}%
\begin{pgfscope}%
\pgfsys@transformshift{4.095589in}{4.350045in}%
\pgfsys@useobject{currentmarker}{}%
\end{pgfscope}%
\begin{pgfscope}%
\pgfsys@transformshift{4.116011in}{4.190929in}%
\pgfsys@useobject{currentmarker}{}%
\end{pgfscope}%
\begin{pgfscope}%
\pgfsys@transformshift{4.132207in}{4.101086in}%
\pgfsys@useobject{currentmarker}{}%
\end{pgfscope}%
\begin{pgfscope}%
\pgfsys@transformshift{4.152863in}{4.057377in}%
\pgfsys@useobject{currentmarker}{}%
\end{pgfscope}%
\begin{pgfscope}%
\pgfsys@transformshift{4.170937in}{4.042685in}%
\pgfsys@useobject{currentmarker}{}%
\end{pgfscope}%
\begin{pgfscope}%
\pgfsys@transformshift{4.192062in}{4.037630in}%
\pgfsys@useobject{currentmarker}{}%
\end{pgfscope}%
\begin{pgfscope}%
\pgfsys@transformshift{4.213423in}{4.037551in}%
\pgfsys@useobject{currentmarker}{}%
\end{pgfscope}%
\begin{pgfscope}%
\pgfsys@transformshift{4.227976in}{4.040330in}%
\pgfsys@useobject{currentmarker}{}%
\end{pgfscope}%
\begin{pgfscope}%
\pgfsys@transformshift{4.250744in}{4.052521in}%
\pgfsys@useobject{currentmarker}{}%
\end{pgfscope}%
\begin{pgfscope}%
\pgfsys@transformshift{4.269289in}{4.075273in}%
\pgfsys@useobject{currentmarker}{}%
\end{pgfscope}%
\begin{pgfscope}%
\pgfsys@transformshift{4.288536in}{4.133986in}%
\pgfsys@useobject{currentmarker}{}%
\end{pgfscope}%
\begin{pgfscope}%
\pgfsys@transformshift{4.306376in}{4.297941in}%
\pgfsys@useobject{currentmarker}{}%
\end{pgfscope}%
\begin{pgfscope}%
\pgfsys@transformshift{4.327969in}{4.439285in}%
\pgfsys@useobject{currentmarker}{}%
\end{pgfscope}%
\begin{pgfscope}%
\pgfsys@transformshift{4.346748in}{4.456856in}%
\pgfsys@useobject{currentmarker}{}%
\end{pgfscope}%
\begin{pgfscope}%
\pgfsys@transformshift{4.363650in}{4.422296in}%
\pgfsys@useobject{currentmarker}{}%
\end{pgfscope}%
\begin{pgfscope}%
\pgfsys@transformshift{4.384540in}{4.286796in}%
\pgfsys@useobject{currentmarker}{}%
\end{pgfscope}%
\begin{pgfscope}%
\pgfsys@transformshift{4.402380in}{4.155572in}%
\pgfsys@useobject{currentmarker}{}%
\end{pgfscope}%
\begin{pgfscope}%
\pgfsys@transformshift{4.421393in}{4.326503in}%
\pgfsys@useobject{currentmarker}{}%
\end{pgfscope}%
\begin{pgfscope}%
\pgfsys@transformshift{4.441815in}{4.210576in}%
\pgfsys@useobject{currentmarker}{}%
\end{pgfscope}%
\begin{pgfscope}%
\pgfsys@transformshift{4.461062in}{4.102648in}%
\pgfsys@useobject{currentmarker}{}%
\end{pgfscope}%
\begin{pgfscope}%
\pgfsys@transformshift{4.481718in}{4.053923in}%
\pgfsys@useobject{currentmarker}{}%
\end{pgfscope}%
\begin{pgfscope}%
\pgfsys@transformshift{4.481484in}{4.053534in}%
\pgfsys@useobject{currentmarker}{}%
\end{pgfscope}%
\begin{pgfscope}%
\pgfsys@transformshift{4.471859in}{4.069909in}%
\pgfsys@useobject{currentmarker}{}%
\end{pgfscope}%
\begin{pgfscope}%
\pgfsys@transformshift{4.454254in}{4.162280in}%
\pgfsys@useobject{currentmarker}{}%
\end{pgfscope}%
\begin{pgfscope}%
\pgfsys@transformshift{4.435946in}{4.360161in}%
\pgfsys@useobject{currentmarker}{}%
\end{pgfscope}%
\begin{pgfscope}%
\pgfsys@transformshift{4.417402in}{4.454330in}%
\pgfsys@useobject{currentmarker}{}%
\end{pgfscope}%
\begin{pgfscope}%
\pgfsys@transformshift{4.396980in}{4.408236in}%
\pgfsys@useobject{currentmarker}{}%
\end{pgfscope}%
\begin{pgfscope}%
\pgfsys@transformshift{4.379846in}{4.162540in}%
\pgfsys@useobject{currentmarker}{}%
\end{pgfscope}%
\begin{pgfscope}%
\pgfsys@transformshift{4.358954in}{4.062710in}%
\pgfsys@useobject{currentmarker}{}%
\end{pgfscope}%
\begin{pgfscope}%
\pgfsys@transformshift{4.338768in}{4.039889in}%
\pgfsys@useobject{currentmarker}{}%
\end{pgfscope}%
\begin{pgfscope}%
\pgfsys@transformshift{4.320929in}{4.036619in}%
\pgfsys@useobject{currentmarker}{}%
\end{pgfscope}%
\begin{pgfscope}%
\pgfsys@transformshift{4.303793in}{4.043456in}%
\pgfsys@useobject{currentmarker}{}%
\end{pgfscope}%
\begin{pgfscope}%
\pgfsys@transformshift{4.281494in}{4.089825in}%
\pgfsys@useobject{currentmarker}{}%
\end{pgfscope}%
\begin{pgfscope}%
\pgfsys@transformshift{4.263655in}{4.220916in}%
\pgfsys@useobject{currentmarker}{}%
\end{pgfscope}%
\begin{pgfscope}%
\pgfsys@transformshift{4.244407in}{4.417061in}%
\pgfsys@useobject{currentmarker}{}%
\end{pgfscope}%
\begin{pgfscope}%
\pgfsys@transformshift{4.224220in}{4.439120in}%
\pgfsys@useobject{currentmarker}{}%
\end{pgfscope}%
\begin{pgfscope}%
\pgfsys@transformshift{4.205441in}{4.233378in}%
\pgfsys@useobject{currentmarker}{}%
\end{pgfscope}%
\begin{pgfscope}%
\pgfsys@transformshift{4.185256in}{4.081430in}%
\pgfsys@useobject{currentmarker}{}%
\end{pgfscope}%
\begin{pgfscope}%
\pgfsys@transformshift{4.167651in}{4.047113in}%
\pgfsys@useobject{currentmarker}{}%
\end{pgfscope}%
\begin{pgfscope}%
\pgfsys@transformshift{4.147698in}{4.036275in}%
\pgfsys@useobject{currentmarker}{}%
\end{pgfscope}%
\begin{pgfscope}%
\pgfsys@transformshift{4.129624in}{4.038108in}%
\pgfsys@useobject{currentmarker}{}%
\end{pgfscope}%
\begin{pgfscope}%
\pgfsys@transformshift{4.106151in}{4.062576in}%
\pgfsys@useobject{currentmarker}{}%
\end{pgfscope}%
\begin{pgfscope}%
\pgfsys@transformshift{4.089250in}{4.142435in}%
\pgfsys@useobject{currentmarker}{}%
\end{pgfscope}%
\begin{pgfscope}%
\pgfsys@transformshift{4.071645in}{4.321265in}%
\pgfsys@useobject{currentmarker}{}%
\end{pgfscope}%
\begin{pgfscope}%
\pgfsys@transformshift{4.050754in}{4.438485in}%
\pgfsys@useobject{currentmarker}{}%
\end{pgfscope}%
\begin{pgfscope}%
\pgfsys@transformshift{4.030569in}{4.346033in}%
\pgfsys@useobject{currentmarker}{}%
\end{pgfscope}%
\begin{pgfscope}%
\pgfsys@transformshift{4.013433in}{4.124505in}%
\pgfsys@useobject{currentmarker}{}%
\end{pgfscope}%
\begin{pgfscope}%
\pgfsys@transformshift{3.993951in}{4.053764in}%
\pgfsys@useobject{currentmarker}{}%
\end{pgfscope}%
\begin{pgfscope}%
\pgfsys@transformshift{3.976815in}{4.040550in}%
\pgfsys@useobject{currentmarker}{}%
\end{pgfscope}%
\begin{pgfscope}%
\pgfsys@transformshift{3.954516in}{4.035412in}%
\pgfsys@useobject{currentmarker}{}%
\end{pgfscope}%
\begin{pgfscope}%
\pgfsys@transformshift{3.937380in}{4.039547in}%
\pgfsys@useobject{currentmarker}{}%
\end{pgfscope}%
\begin{pgfscope}%
\pgfsys@transformshift{3.916255in}{4.065241in}%
\pgfsys@useobject{currentmarker}{}%
\end{pgfscope}%
\begin{pgfscope}%
\pgfsys@transformshift{3.899119in}{4.156498in}%
\pgfsys@useobject{currentmarker}{}%
\end{pgfscope}%
\begin{pgfscope}%
\pgfsys@transformshift{3.878463in}{4.366193in}%
\pgfsys@useobject{currentmarker}{}%
\end{pgfscope}%
\begin{pgfscope}%
\pgfsys@transformshift{3.860858in}{4.432717in}%
\pgfsys@useobject{currentmarker}{}%
\end{pgfscope}%
\begin{pgfscope}%
\pgfsys@transformshift{3.839733in}{4.271470in}%
\pgfsys@useobject{currentmarker}{}%
\end{pgfscope}%
\begin{pgfscope}%
\pgfsys@transformshift{3.822128in}{4.120280in}%
\pgfsys@useobject{currentmarker}{}%
\end{pgfscope}%
\begin{pgfscope}%
\pgfsys@transformshift{3.801472in}{4.056778in}%
\pgfsys@useobject{currentmarker}{}%
\end{pgfscope}%
\begin{pgfscope}%
\pgfsys@transformshift{3.778939in}{4.037807in}%
\pgfsys@useobject{currentmarker}{}%
\end{pgfscope}%
\begin{pgfscope}%
\pgfsys@transformshift{3.763680in}{4.035197in}%
\pgfsys@useobject{currentmarker}{}%
\end{pgfscope}%
\begin{pgfscope}%
\pgfsys@transformshift{3.743495in}{4.039335in}%
\pgfsys@useobject{currentmarker}{}%
\end{pgfscope}%
\begin{pgfscope}%
\pgfsys@transformshift{3.725656in}{4.054391in}%
\pgfsys@useobject{currentmarker}{}%
\end{pgfscope}%
\begin{pgfscope}%
\pgfsys@transformshift{3.704763in}{4.136076in}%
\pgfsys@useobject{currentmarker}{}%
\end{pgfscope}%
\begin{pgfscope}%
\pgfsys@transformshift{3.684578in}{4.349209in}%
\pgfsys@useobject{currentmarker}{}%
\end{pgfscope}%
\begin{pgfscope}%
\pgfsys@transformshift{3.669085in}{4.411712in}%
\pgfsys@useobject{currentmarker}{}%
\end{pgfscope}%
\begin{pgfscope}%
\pgfsys@transformshift{3.646317in}{4.369028in}%
\pgfsys@useobject{currentmarker}{}%
\end{pgfscope}%
\begin{pgfscope}%
\pgfsys@transformshift{3.627772in}{4.139203in}%
\pgfsys@useobject{currentmarker}{}%
\end{pgfscope}%
\begin{pgfscope}%
\pgfsys@transformshift{3.607351in}{4.069618in}%
\pgfsys@useobject{currentmarker}{}%
\end{pgfscope}%
\begin{pgfscope}%
\pgfsys@transformshift{3.589511in}{4.043603in}%
\pgfsys@useobject{currentmarker}{}%
\end{pgfscope}%
\begin{pgfscope}%
\pgfsys@transformshift{3.572141in}{4.036844in}%
\pgfsys@useobject{currentmarker}{}%
\end{pgfscope}%
\begin{pgfscope}%
\pgfsys@transformshift{3.551016in}{4.035724in}%
\pgfsys@useobject{currentmarker}{}%
\end{pgfscope}%
\begin{pgfscope}%
\pgfsys@transformshift{3.533177in}{4.042834in}%
\pgfsys@useobject{currentmarker}{}%
\end{pgfscope}%
\begin{pgfscope}%
\pgfsys@transformshift{3.512052in}{4.064918in}%
\pgfsys@useobject{currentmarker}{}%
\end{pgfscope}%
\begin{pgfscope}%
\pgfsys@transformshift{3.494681in}{4.147271in}%
\pgfsys@useobject{currentmarker}{}%
\end{pgfscope}%
\begin{pgfscope}%
\pgfsys@transformshift{3.474025in}{4.281678in}%
\pgfsys@useobject{currentmarker}{}%
\end{pgfscope}%
\begin{pgfscope}%
\pgfsys@transformshift{3.456186in}{4.391426in}%
\pgfsys@useobject{currentmarker}{}%
\end{pgfscope}%
\begin{pgfscope}%
\pgfsys@transformshift{3.435059in}{4.413486in}%
\pgfsys@useobject{currentmarker}{}%
\end{pgfscope}%
\begin{pgfscope}%
\pgfsys@transformshift{3.416751in}{4.196152in}%
\pgfsys@useobject{currentmarker}{}%
\end{pgfscope}%
\begin{pgfscope}%
\pgfsys@transformshift{3.396798in}{4.081554in}%
\pgfsys@useobject{currentmarker}{}%
\end{pgfscope}%
\begin{pgfscope}%
\pgfsys@transformshift{3.378256in}{4.224194in}%
\pgfsys@useobject{currentmarker}{}%
\end{pgfscope}%
\begin{pgfscope}%
\pgfsys@transformshift{3.358068in}{4.393356in}%
\pgfsys@useobject{currentmarker}{}%
\end{pgfscope}%
\begin{pgfscope}%
\pgfsys@transformshift{3.342812in}{4.408362in}%
\pgfsys@useobject{currentmarker}{}%
\end{pgfscope}%
\begin{pgfscope}%
\pgfsys@transformshift{3.317461in}{4.163551in}%
\pgfsys@useobject{currentmarker}{}%
\end{pgfscope}%
\begin{pgfscope}%
\pgfsys@transformshift{3.302672in}{4.072732in}%
\pgfsys@useobject{currentmarker}{}%
\end{pgfscope}%
\begin{pgfscope}%
\pgfsys@transformshift{3.283424in}{4.044098in}%
\pgfsys@useobject{currentmarker}{}%
\end{pgfscope}%
\begin{pgfscope}%
\pgfsys@transformshift{3.258544in}{4.035294in}%
\pgfsys@useobject{currentmarker}{}%
\end{pgfscope}%
\begin{pgfscope}%
\pgfsys@transformshift{3.240234in}{4.036090in}%
\pgfsys@useobject{currentmarker}{}%
\end{pgfscope}%
\begin{pgfscope}%
\pgfsys@transformshift{3.224978in}{4.043390in}%
\pgfsys@useobject{currentmarker}{}%
\end{pgfscope}%
\begin{pgfscope}%
\pgfsys@transformshift{3.206668in}{4.070407in}%
\pgfsys@useobject{currentmarker}{}%
\end{pgfscope}%
\begin{pgfscope}%
\pgfsys@transformshift{3.186717in}{4.193947in}%
\pgfsys@useobject{currentmarker}{}%
\end{pgfscope}%
\begin{pgfscope}%
\pgfsys@transformshift{3.167469in}{4.372350in}%
\pgfsys@useobject{currentmarker}{}%
\end{pgfscope}%
\begin{pgfscope}%
\pgfsys@transformshift{3.146578in}{4.417474in}%
\pgfsys@useobject{currentmarker}{}%
\end{pgfscope}%
\begin{pgfscope}%
\pgfsys@transformshift{3.129208in}{4.254025in}%
\pgfsys@useobject{currentmarker}{}%
\end{pgfscope}%
\begin{pgfscope}%
\pgfsys@transformshift{3.109021in}{4.097038in}%
\pgfsys@useobject{currentmarker}{}%
\end{pgfscope}%
\begin{pgfscope}%
\pgfsys@transformshift{3.090007in}{4.051115in}%
\pgfsys@useobject{currentmarker}{}%
\end{pgfscope}%
\begin{pgfscope}%
\pgfsys@transformshift{3.071699in}{4.037623in}%
\pgfsys@useobject{currentmarker}{}%
\end{pgfscope}%
\begin{pgfscope}%
\pgfsys@transformshift{3.053860in}{4.035164in}%
\pgfsys@useobject{currentmarker}{}%
\end{pgfscope}%
\begin{pgfscope}%
\pgfsys@transformshift{3.034376in}{4.036700in}%
\pgfsys@useobject{currentmarker}{}%
\end{pgfscope}%
\begin{pgfscope}%
\pgfsys@transformshift{3.012782in}{4.049108in}%
\pgfsys@useobject{currentmarker}{}%
\end{pgfscope}%
\begin{pgfscope}%
\pgfsys@transformshift{2.993535in}{4.083728in}%
\pgfsys@useobject{currentmarker}{}%
\end{pgfscope}%
\begin{pgfscope}%
\pgfsys@transformshift{2.975930in}{4.211289in}%
\pgfsys@useobject{currentmarker}{}%
\end{pgfscope}%
\begin{pgfscope}%
\pgfsys@transformshift{2.957385in}{4.373515in}%
\pgfsys@useobject{currentmarker}{}%
\end{pgfscope}%
\begin{pgfscope}%
\pgfsys@transformshift{2.938607in}{4.418450in}%
\pgfsys@useobject{currentmarker}{}%
\end{pgfscope}%
\begin{pgfscope}%
\pgfsys@transformshift{2.916778in}{4.342730in}%
\pgfsys@useobject{currentmarker}{}%
\end{pgfscope}%
\begin{pgfscope}%
\pgfsys@transformshift{2.897529in}{4.119460in}%
\pgfsys@useobject{currentmarker}{}%
\end{pgfscope}%
\begin{pgfscope}%
\pgfsys@transformshift{2.879926in}{4.060372in}%
\pgfsys@useobject{currentmarker}{}%
\end{pgfscope}%
\begin{pgfscope}%
\pgfsys@transformshift{2.861381in}{4.040948in}%
\pgfsys@useobject{currentmarker}{}%
\end{pgfscope}%
\begin{pgfscope}%
\pgfsys@transformshift{2.839082in}{4.034997in}%
\pgfsys@useobject{currentmarker}{}%
\end{pgfscope}%
\begin{pgfscope}%
\pgfsys@transformshift{2.822181in}{4.036183in}%
\pgfsys@useobject{currentmarker}{}%
\end{pgfscope}%
\begin{pgfscope}%
\pgfsys@transformshift{2.798239in}{4.048369in}%
\pgfsys@useobject{currentmarker}{}%
\end{pgfscope}%
\begin{pgfscope}%
\pgfsys@transformshift{2.784625in}{4.066602in}%
\pgfsys@useobject{currentmarker}{}%
\end{pgfscope}%
\begin{pgfscope}%
\pgfsys@transformshift{2.764203in}{4.130617in}%
\pgfsys@useobject{currentmarker}{}%
\end{pgfscope}%
\begin{pgfscope}%
\pgfsys@transformshift{2.746130in}{4.290204in}%
\pgfsys@useobject{currentmarker}{}%
\end{pgfscope}%
\begin{pgfscope}%
\pgfsys@transformshift{2.723360in}{4.413961in}%
\pgfsys@useobject{currentmarker}{}%
\end{pgfscope}%
\begin{pgfscope}%
\pgfsys@transformshift{2.707869in}{4.379124in}%
\pgfsys@useobject{currentmarker}{}%
\end{pgfscope}%
\begin{pgfscope}%
\pgfsys@transformshift{2.686507in}{4.156667in}%
\pgfsys@useobject{currentmarker}{}%
\end{pgfscope}%
\begin{pgfscope}%
\pgfsys@transformshift{2.667965in}{4.067365in}%
\pgfsys@useobject{currentmarker}{}%
\end{pgfscope}%
\begin{pgfscope}%
\pgfsys@transformshift{2.646135in}{4.041230in}%
\pgfsys@useobject{currentmarker}{}%
\end{pgfscope}%
\begin{pgfscope}%
\pgfsys@transformshift{2.628530in}{4.035494in}%
\pgfsys@useobject{currentmarker}{}%
\end{pgfscope}%
\begin{pgfscope}%
\pgfsys@transformshift{2.609048in}{4.035374in}%
\pgfsys@useobject{currentmarker}{}%
\end{pgfscope}%
\begin{pgfscope}%
\pgfsys@transformshift{2.590269in}{4.040533in}%
\pgfsys@useobject{currentmarker}{}%
\end{pgfscope}%
\begin{pgfscope}%
\pgfsys@transformshift{2.571959in}{4.060997in}%
\pgfsys@useobject{currentmarker}{}%
\end{pgfscope}%
\begin{pgfscope}%
\pgfsys@transformshift{2.549660in}{4.155995in}%
\pgfsys@useobject{currentmarker}{}%
\end{pgfscope}%
\begin{pgfscope}%
\pgfsys@transformshift{2.534872in}{4.278501in}%
\pgfsys@useobject{currentmarker}{}%
\end{pgfscope}%
\begin{pgfscope}%
\pgfsys@transformshift{2.516095in}{4.340225in}%
\pgfsys@useobject{currentmarker}{}%
\end{pgfscope}%
\begin{pgfscope}%
\pgfsys@transformshift{2.496846in}{4.409463in}%
\pgfsys@useobject{currentmarker}{}%
\end{pgfscope}%
\begin{pgfscope}%
\pgfsys@transformshift{2.475486in}{4.332984in}%
\pgfsys@useobject{currentmarker}{}%
\end{pgfscope}%
\begin{pgfscope}%
\pgfsys@transformshift{2.455770in}{4.145860in}%
\pgfsys@useobject{currentmarker}{}%
\end{pgfscope}%
\begin{pgfscope}%
\pgfsys@transformshift{2.436991in}{4.067802in}%
\pgfsys@useobject{currentmarker}{}%
\end{pgfscope}%
\begin{pgfscope}%
\pgfsys@transformshift{2.419855in}{4.042637in}%
\pgfsys@useobject{currentmarker}{}%
\end{pgfscope}%
\begin{pgfscope}%
\pgfsys@transformshift{2.398025in}{4.035544in}%
\pgfsys@useobject{currentmarker}{}%
\end{pgfscope}%
\begin{pgfscope}%
\pgfsys@transformshift{2.379717in}{4.035496in}%
\pgfsys@useobject{currentmarker}{}%
\end{pgfscope}%
\begin{pgfscope}%
\pgfsys@transformshift{2.359764in}{4.037574in}%
\pgfsys@useobject{currentmarker}{}%
\end{pgfscope}%
\begin{pgfscope}%
\pgfsys@transformshift{2.340752in}{4.046759in}%
\pgfsys@useobject{currentmarker}{}%
\end{pgfscope}%
\begin{pgfscope}%
\pgfsys@transformshift{2.322911in}{4.079970in}%
\pgfsys@useobject{currentmarker}{}%
\end{pgfscope}%
\begin{pgfscope}%
\pgfsys@transformshift{2.304369in}{4.200023in}%
\pgfsys@useobject{currentmarker}{}%
\end{pgfscope}%
\begin{pgfscope}%
\pgfsys@transformshift{2.284887in}{4.327703in}%
\pgfsys@useobject{currentmarker}{}%
\end{pgfscope}%
\begin{pgfscope}%
\pgfsys@transformshift{2.263291in}{4.408796in}%
\pgfsys@useobject{currentmarker}{}%
\end{pgfscope}%
\begin{pgfscope}%
\pgfsys@transformshift{2.245686in}{4.341561in}%
\pgfsys@useobject{currentmarker}{}%
\end{pgfscope}%
\begin{pgfscope}%
\pgfsys@transformshift{2.225970in}{4.146790in}%
\pgfsys@useobject{currentmarker}{}%
\end{pgfscope}%
\begin{pgfscope}%
\pgfsys@transformshift{2.203669in}{4.065272in}%
\pgfsys@useobject{currentmarker}{}%
\end{pgfscope}%
\begin{pgfscope}%
\pgfsys@transformshift{2.187238in}{4.045867in}%
\pgfsys@useobject{currentmarker}{}%
\end{pgfscope}%
\begin{pgfscope}%
\pgfsys@transformshift{2.168461in}{4.037365in}%
\pgfsys@useobject{currentmarker}{}%
\end{pgfscope}%
\begin{pgfscope}%
\pgfsys@transformshift{2.150385in}{4.034880in}%
\pgfsys@useobject{currentmarker}{}%
\end{pgfscope}%
\begin{pgfscope}%
\pgfsys@transformshift{2.126912in}{4.038311in}%
\pgfsys@useobject{currentmarker}{}%
\end{pgfscope}%
\begin{pgfscope}%
\pgfsys@transformshift{2.108839in}{4.047889in}%
\pgfsys@useobject{currentmarker}{}%
\end{pgfscope}%
\begin{pgfscope}%
\pgfsys@transformshift{2.088651in}{4.085336in}%
\pgfsys@useobject{currentmarker}{}%
\end{pgfscope}%
\begin{pgfscope}%
\pgfsys@transformshift{2.072691in}{4.091371in}%
\pgfsys@useobject{currentmarker}{}%
\end{pgfscope}%
\begin{pgfscope}%
\pgfsys@transformshift{2.052973in}{4.225525in}%
\pgfsys@useobject{currentmarker}{}%
\end{pgfscope}%
\begin{pgfscope}%
\pgfsys@transformshift{2.034899in}{4.346229in}%
\pgfsys@useobject{currentmarker}{}%
\end{pgfscope}%
\begin{pgfscope}%
\pgfsys@transformshift{2.016589in}{4.409935in}%
\pgfsys@useobject{currentmarker}{}%
\end{pgfscope}%
\begin{pgfscope}%
\pgfsys@transformshift{1.993587in}{4.379299in}%
\pgfsys@useobject{currentmarker}{}%
\end{pgfscope}%
\begin{pgfscope}%
\pgfsys@transformshift{1.975982in}{4.190139in}%
\pgfsys@useobject{currentmarker}{}%
\end{pgfscope}%
\begin{pgfscope}%
\pgfsys@transformshift{1.957672in}{4.084753in}%
\pgfsys@useobject{currentmarker}{}%
\end{pgfscope}%
\begin{pgfscope}%
\pgfsys@transformshift{1.938661in}{4.048131in}%
\pgfsys@useobject{currentmarker}{}%
\end{pgfscope}%
\begin{pgfscope}%
\pgfsys@transformshift{1.916596in}{4.038954in}%
\pgfsys@useobject{currentmarker}{}%
\end{pgfscope}%
\begin{pgfscope}%
\pgfsys@transformshift{1.898757in}{4.035125in}%
\pgfsys@useobject{currentmarker}{}%
\end{pgfscope}%
\begin{pgfscope}%
\pgfsys@transformshift{1.879744in}{4.035859in}%
\pgfsys@useobject{currentmarker}{}%
\end{pgfscope}%
\begin{pgfscope}%
\pgfsys@transformshift{1.861434in}{4.040350in}%
\pgfsys@useobject{currentmarker}{}%
\end{pgfscope}%
\begin{pgfscope}%
\pgfsys@transformshift{1.839604in}{4.057679in}%
\pgfsys@useobject{currentmarker}{}%
\end{pgfscope}%
\begin{pgfscope}%
\pgfsys@transformshift{1.820356in}{4.116016in}%
\pgfsys@useobject{currentmarker}{}%
\end{pgfscope}%
\begin{pgfscope}%
\pgfsys@transformshift{1.805099in}{4.241694in}%
\pgfsys@useobject{currentmarker}{}%
\end{pgfscope}%
\begin{pgfscope}%
\pgfsys@transformshift{1.783974in}{4.373685in}%
\pgfsys@useobject{currentmarker}{}%
\end{pgfscope}%
\begin{pgfscope}%
\pgfsys@transformshift{1.764961in}{4.415739in}%
\pgfsys@useobject{currentmarker}{}%
\end{pgfscope}%
\begin{pgfscope}%
\pgfsys@transformshift{1.746885in}{4.394747in}%
\pgfsys@useobject{currentmarker}{}%
\end{pgfscope}%
\begin{pgfscope}%
\pgfsys@transformshift{1.724586in}{4.178655in}%
\pgfsys@useobject{currentmarker}{}%
\end{pgfscope}%
\begin{pgfscope}%
\pgfsys@transformshift{1.707687in}{4.083473in}%
\pgfsys@useobject{currentmarker}{}%
\end{pgfscope}%
\begin{pgfscope}%
\pgfsys@transformshift{1.688674in}{4.055438in}%
\pgfsys@useobject{currentmarker}{}%
\end{pgfscope}%
\begin{pgfscope}%
\pgfsys@transformshift{1.669895in}{4.044011in}%
\pgfsys@useobject{currentmarker}{}%
\end{pgfscope}%
\begin{pgfscope}%
\pgfsys@transformshift{1.650881in}{4.036731in}%
\pgfsys@useobject{currentmarker}{}%
\end{pgfscope}%
\begin{pgfscope}%
\pgfsys@transformshift{1.629051in}{4.035406in}%
\pgfsys@useobject{currentmarker}{}%
\end{pgfscope}%
\begin{pgfscope}%
\pgfsys@transformshift{1.611212in}{4.038426in}%
\pgfsys@useobject{currentmarker}{}%
\end{pgfscope}%
\begin{pgfscope}%
\pgfsys@transformshift{1.593138in}{4.048842in}%
\pgfsys@useobject{currentmarker}{}%
\end{pgfscope}%
\begin{pgfscope}%
\pgfsys@transformshift{1.574125in}{4.079748in}%
\pgfsys@useobject{currentmarker}{}%
\end{pgfscope}%
\begin{pgfscope}%
\pgfsys@transformshift{1.552766in}{4.123571in}%
\pgfsys@useobject{currentmarker}{}%
\end{pgfscope}%
\begin{pgfscope}%
\pgfsys@transformshift{1.535161in}{4.239134in}%
\pgfsys@useobject{currentmarker}{}%
\end{pgfscope}%
\begin{pgfscope}%
\pgfsys@transformshift{1.515677in}{4.378335in}%
\pgfsys@useobject{currentmarker}{}%
\end{pgfscope}%
\begin{pgfscope}%
\pgfsys@transformshift{1.491501in}{4.422130in}%
\pgfsys@useobject{currentmarker}{}%
\end{pgfscope}%
\begin{pgfscope}%
\pgfsys@transformshift{1.476713in}{4.386608in}%
\pgfsys@useobject{currentmarker}{}%
\end{pgfscope}%
\begin{pgfscope}%
\pgfsys@transformshift{1.459577in}{4.035803in}%
\pgfsys@useobject{currentmarker}{}%
\end{pgfscope}%
\begin{pgfscope}%
\pgfsys@transformshift{1.440800in}{4.036235in}%
\pgfsys@useobject{currentmarker}{}%
\end{pgfscope}%
\begin{pgfscope}%
\pgfsys@transformshift{1.419673in}{4.047157in}%
\pgfsys@useobject{currentmarker}{}%
\end{pgfscope}%
\begin{pgfscope}%
\pgfsys@transformshift{1.397843in}{4.097016in}%
\pgfsys@useobject{currentmarker}{}%
\end{pgfscope}%
\begin{pgfscope}%
\pgfsys@transformshift{1.379769in}{4.195739in}%
\pgfsys@useobject{currentmarker}{}%
\end{pgfscope}%
\begin{pgfscope}%
\pgfsys@transformshift{1.361227in}{4.365453in}%
\pgfsys@useobject{currentmarker}{}%
\end{pgfscope}%
\begin{pgfscope}%
\pgfsys@transformshift{1.341979in}{4.429854in}%
\pgfsys@useobject{currentmarker}{}%
\end{pgfscope}%
\begin{pgfscope}%
\pgfsys@transformshift{1.321086in}{4.355095in}%
\pgfsys@useobject{currentmarker}{}%
\end{pgfscope}%
\begin{pgfscope}%
\pgfsys@transformshift{1.304187in}{4.139229in}%
\pgfsys@useobject{currentmarker}{}%
\end{pgfscope}%
\begin{pgfscope}%
\pgfsys@transformshift{1.287051in}{4.071643in}%
\pgfsys@useobject{currentmarker}{}%
\end{pgfscope}%
\begin{pgfscope}%
\pgfsys@transformshift{1.265692in}{4.043937in}%
\pgfsys@useobject{currentmarker}{}%
\end{pgfscope}%
\begin{pgfscope}%
\pgfsys@transformshift{1.244096in}{4.036270in}%
\pgfsys@useobject{currentmarker}{}%
\end{pgfscope}%
\begin{pgfscope}%
\pgfsys@transformshift{1.225082in}{4.036120in}%
\pgfsys@useobject{currentmarker}{}%
\end{pgfscope}%
\begin{pgfscope}%
\pgfsys@transformshift{1.209826in}{4.039030in}%
\pgfsys@useobject{currentmarker}{}%
\end{pgfscope}%
\begin{pgfscope}%
\pgfsys@transformshift{1.188230in}{4.052798in}%
\pgfsys@useobject{currentmarker}{}%
\end{pgfscope}%
\begin{pgfscope}%
\pgfsys@transformshift{1.169688in}{4.099580in}%
\pgfsys@useobject{currentmarker}{}%
\end{pgfscope}%
\begin{pgfscope}%
\pgfsys@transformshift{1.150674in}{4.232172in}%
\pgfsys@useobject{currentmarker}{}%
\end{pgfscope}%
\begin{pgfscope}%
\pgfsys@transformshift{1.132835in}{4.334965in}%
\pgfsys@useobject{currentmarker}{}%
\end{pgfscope}%
\begin{pgfscope}%
\pgfsys@transformshift{1.110300in}{4.434755in}%
\pgfsys@useobject{currentmarker}{}%
\end{pgfscope}%
\begin{pgfscope}%
\pgfsys@transformshift{1.091757in}{4.405809in}%
\pgfsys@useobject{currentmarker}{}%
\end{pgfscope}%
\begin{pgfscope}%
\pgfsys@transformshift{1.076030in}{4.204738in}%
\pgfsys@useobject{currentmarker}{}%
\end{pgfscope}%
\begin{pgfscope}%
\pgfsys@transformshift{1.055608in}{4.086135in}%
\pgfsys@useobject{currentmarker}{}%
\end{pgfscope}%
\begin{pgfscope}%
\pgfsys@transformshift{1.034014in}{4.049642in}%
\pgfsys@useobject{currentmarker}{}%
\end{pgfscope}%
\begin{pgfscope}%
\pgfsys@transformshift{1.015939in}{4.038890in}%
\pgfsys@useobject{currentmarker}{}%
\end{pgfscope}%
\begin{pgfscope}%
\pgfsys@transformshift{0.997160in}{4.035853in}%
\pgfsys@useobject{currentmarker}{}%
\end{pgfscope}%
\begin{pgfscope}%
\pgfsys@transformshift{0.976035in}{4.040111in}%
\pgfsys@useobject{currentmarker}{}%
\end{pgfscope}%
\begin{pgfscope}%
\pgfsys@transformshift{0.958664in}{4.050633in}%
\pgfsys@useobject{currentmarker}{}%
\end{pgfscope}%
\begin{pgfscope}%
\pgfsys@transformshift{0.939887in}{4.081640in}%
\pgfsys@useobject{currentmarker}{}%
\end{pgfscope}%
\begin{pgfscope}%
\pgfsys@transformshift{0.921109in}{4.193691in}%
\pgfsys@useobject{currentmarker}{}%
\end{pgfscope}%
\begin{pgfscope}%
\pgfsys@transformshift{0.902095in}{4.304045in}%
\pgfsys@useobject{currentmarker}{}%
\end{pgfscope}%
\begin{pgfscope}%
\pgfsys@transformshift{0.880031in}{4.424118in}%
\pgfsys@useobject{currentmarker}{}%
\end{pgfscope}%
\begin{pgfscope}%
\pgfsys@transformshift{0.862426in}{4.447387in}%
\pgfsys@useobject{currentmarker}{}%
\end{pgfscope}%
\begin{pgfscope}%
\pgfsys@transformshift{0.842239in}{4.416639in}%
\pgfsys@useobject{currentmarker}{}%
\end{pgfscope}%
\begin{pgfscope}%
\pgfsys@transformshift{0.823931in}{4.214459in}%
\pgfsys@useobject{currentmarker}{}%
\end{pgfscope}%
\begin{pgfscope}%
\pgfsys@transformshift{0.805621in}{4.149151in}%
\pgfsys@useobject{currentmarker}{}%
\end{pgfscope}%
\begin{pgfscope}%
\pgfsys@transformshift{0.784261in}{4.066837in}%
\pgfsys@useobject{currentmarker}{}%
\end{pgfscope}%
\begin{pgfscope}%
\pgfsys@transformshift{0.765717in}{4.052484in}%
\pgfsys@useobject{currentmarker}{}%
\end{pgfscope}%
\begin{pgfscope}%
\pgfsys@transformshift{0.747174in}{4.041726in}%
\pgfsys@useobject{currentmarker}{}%
\end{pgfscope}%
\begin{pgfscope}%
\pgfsys@transformshift{0.729335in}{4.037523in}%
\pgfsys@useobject{currentmarker}{}%
\end{pgfscope}%
\begin{pgfscope}%
\pgfsys@transformshift{0.707270in}{4.038139in}%
\pgfsys@useobject{currentmarker}{}%
\end{pgfscope}%
\begin{pgfscope}%
\pgfsys@transformshift{0.689431in}{4.047661in}%
\pgfsys@useobject{currentmarker}{}%
\end{pgfscope}%
\begin{pgfscope}%
\pgfsys@transformshift{0.668070in}{4.068966in}%
\pgfsys@useobject{currentmarker}{}%
\end{pgfscope}%
\begin{pgfscope}%
\pgfsys@transformshift{0.650231in}{4.125431in}%
\pgfsys@useobject{currentmarker}{}%
\end{pgfscope}%
\begin{pgfscope}%
\pgfsys@transformshift{0.650700in}{4.120766in}%
\pgfsys@useobject{currentmarker}{}%
\end{pgfscope}%
\begin{pgfscope}%
\pgfsys@transformshift{0.657507in}{4.079324in}%
\pgfsys@useobject{currentmarker}{}%
\end{pgfscope}%
\begin{pgfscope}%
\pgfsys@transformshift{0.676755in}{4.044931in}%
\pgfsys@useobject{currentmarker}{}%
\end{pgfscope}%
\begin{pgfscope}%
\pgfsys@transformshift{0.696003in}{4.036507in}%
\pgfsys@useobject{currentmarker}{}%
\end{pgfscope}%
\begin{pgfscope}%
\pgfsys@transformshift{0.714076in}{4.039464in}%
\pgfsys@useobject{currentmarker}{}%
\end{pgfscope}%
\begin{pgfscope}%
\pgfsys@transformshift{0.731681in}{4.056668in}%
\pgfsys@useobject{currentmarker}{}%
\end{pgfscope}%
\begin{pgfscope}%
\pgfsys@transformshift{0.751165in}{4.122815in}%
\pgfsys@useobject{currentmarker}{}%
\end{pgfscope}%
\begin{pgfscope}%
\pgfsys@transformshift{0.773228in}{4.396430in}%
\pgfsys@useobject{currentmarker}{}%
\end{pgfscope}%
\begin{pgfscope}%
\pgfsys@transformshift{0.792007in}{4.444478in}%
\pgfsys@useobject{currentmarker}{}%
\end{pgfscope}%
\begin{pgfscope}%
\pgfsys@transformshift{0.811725in}{4.323767in}%
\pgfsys@useobject{currentmarker}{}%
\end{pgfscope}%
\begin{pgfscope}%
\pgfsys@transformshift{0.830738in}{4.131233in}%
\pgfsys@useobject{currentmarker}{}%
\end{pgfscope}%
\begin{pgfscope}%
\pgfsys@transformshift{0.850221in}{4.055970in}%
\pgfsys@useobject{currentmarker}{}%
\end{pgfscope}%
\begin{pgfscope}%
\pgfsys@transformshift{0.868529in}{4.035971in}%
\pgfsys@useobject{currentmarker}{}%
\end{pgfscope}%
\begin{pgfscope}%
\pgfsys@transformshift{0.888247in}{4.041590in}%
\pgfsys@useobject{currentmarker}{}%
\end{pgfscope}%
\begin{pgfscope}%
\pgfsys@transformshift{0.905381in}{4.060243in}%
\pgfsys@useobject{currentmarker}{}%
\end{pgfscope}%
\begin{pgfscope}%
\pgfsys@transformshift{0.926272in}{4.158980in}%
\pgfsys@useobject{currentmarker}{}%
\end{pgfscope}%
\begin{pgfscope}%
\pgfsys@transformshift{0.945756in}{4.409806in}%
\pgfsys@useobject{currentmarker}{}%
\end{pgfscope}%
\begin{pgfscope}%
\pgfsys@transformshift{0.967115in}{4.413599in}%
\pgfsys@useobject{currentmarker}{}%
\end{pgfscope}%
\begin{pgfscope}%
\pgfsys@transformshift{0.983311in}{4.273827in}%
\pgfsys@useobject{currentmarker}{}%
\end{pgfscope}%
\begin{pgfscope}%
\pgfsys@transformshift{1.003264in}{4.102815in}%
\pgfsys@useobject{currentmarker}{}%
\end{pgfscope}%
\begin{pgfscope}%
\pgfsys@transformshift{1.022746in}{4.046856in}%
\pgfsys@useobject{currentmarker}{}%
\end{pgfscope}%
\begin{pgfscope}%
\pgfsys@transformshift{1.039646in}{4.036660in}%
\pgfsys@useobject{currentmarker}{}%
\end{pgfscope}%
\begin{pgfscope}%
\pgfsys@transformshift{1.058894in}{4.036776in}%
\pgfsys@useobject{currentmarker}{}%
\end{pgfscope}%
\begin{pgfscope}%
\pgfsys@transformshift{1.080958in}{4.050276in}%
\pgfsys@useobject{currentmarker}{}%
\end{pgfscope}%
\begin{pgfscope}%
\pgfsys@transformshift{1.097155in}{4.083366in}%
\pgfsys@useobject{currentmarker}{}%
\end{pgfscope}%
\begin{pgfscope}%
\pgfsys@transformshift{1.122036in}{4.339184in}%
\pgfsys@useobject{currentmarker}{}%
\end{pgfscope}%
\begin{pgfscope}%
\pgfsys@transformshift{1.138233in}{4.432806in}%
\pgfsys@useobject{currentmarker}{}%
\end{pgfscope}%
\begin{pgfscope}%
\pgfsys@transformshift{1.157246in}{4.364011in}%
\pgfsys@useobject{currentmarker}{}%
\end{pgfscope}%
\begin{pgfscope}%
\pgfsys@transformshift{1.176964in}{4.164860in}%
\pgfsys@useobject{currentmarker}{}%
\end{pgfscope}%
\begin{pgfscope}%
\pgfsys@transformshift{1.195272in}{4.062101in}%
\pgfsys@useobject{currentmarker}{}%
\end{pgfscope}%
\begin{pgfscope}%
\pgfsys@transformshift{1.214989in}{4.039747in}%
\pgfsys@useobject{currentmarker}{}%
\end{pgfscope}%
\begin{pgfscope}%
\pgfsys@transformshift{1.233533in}{4.035229in}%
\pgfsys@useobject{currentmarker}{}%
\end{pgfscope}%
\begin{pgfscope}%
\pgfsys@transformshift{1.254189in}{4.038762in}%
\pgfsys@useobject{currentmarker}{}%
\end{pgfscope}%
\begin{pgfscope}%
\pgfsys@transformshift{1.271560in}{4.053081in}%
\pgfsys@useobject{currentmarker}{}%
\end{pgfscope}%
\begin{pgfscope}%
\pgfsys@transformshift{1.291042in}{4.110525in}%
\pgfsys@useobject{currentmarker}{}%
\end{pgfscope}%
\begin{pgfscope}%
\pgfsys@transformshift{1.310055in}{4.347791in}%
\pgfsys@useobject{currentmarker}{}%
\end{pgfscope}%
\begin{pgfscope}%
\pgfsys@transformshift{1.328598in}{4.417602in}%
\pgfsys@useobject{currentmarker}{}%
\end{pgfscope}%
\begin{pgfscope}%
\pgfsys@transformshift{1.347847in}{4.409608in}%
\pgfsys@useobject{currentmarker}{}%
\end{pgfscope}%
\begin{pgfscope}%
\pgfsys@transformshift{1.368972in}{4.243713in}%
\pgfsys@useobject{currentmarker}{}%
\end{pgfscope}%
\begin{pgfscope}%
\pgfsys@transformshift{1.387751in}{4.087132in}%
\pgfsys@useobject{currentmarker}{}%
\end{pgfscope}%
\begin{pgfscope}%
\pgfsys@transformshift{1.405825in}{4.046501in}%
\pgfsys@useobject{currentmarker}{}%
\end{pgfscope}%
\begin{pgfscope}%
\pgfsys@transformshift{1.425541in}{4.036090in}%
\pgfsys@useobject{currentmarker}{}%
\end{pgfscope}%
\begin{pgfscope}%
\pgfsys@transformshift{1.446197in}{4.035422in}%
\pgfsys@useobject{currentmarker}{}%
\end{pgfscope}%
\begin{pgfscope}%
\pgfsys@transformshift{1.463568in}{4.039639in}%
\pgfsys@useobject{currentmarker}{}%
\end{pgfscope}%
\begin{pgfscope}%
\pgfsys@transformshift{1.483286in}{4.056472in}%
\pgfsys@useobject{currentmarker}{}%
\end{pgfscope}%
\begin{pgfscope}%
\pgfsys@transformshift{1.500420in}{4.119542in}%
\pgfsys@useobject{currentmarker}{}%
\end{pgfscope}%
\begin{pgfscope}%
\pgfsys@transformshift{1.523424in}{4.388132in}%
\pgfsys@useobject{currentmarker}{}%
\end{pgfscope}%
\begin{pgfscope}%
\pgfsys@transformshift{1.541733in}{4.419883in}%
\pgfsys@useobject{currentmarker}{}%
\end{pgfscope}%
\begin{pgfscope}%
\pgfsys@transformshift{1.560277in}{4.330878in}%
\pgfsys@useobject{currentmarker}{}%
\end{pgfscope}%
\begin{pgfscope}%
\pgfsys@transformshift{1.576239in}{4.161337in}%
\pgfsys@useobject{currentmarker}{}%
\end{pgfscope}%
\begin{pgfscope}%
\pgfsys@transformshift{1.598538in}{4.061878in}%
\pgfsys@useobject{currentmarker}{}%
\end{pgfscope}%
\begin{pgfscope}%
\pgfsys@transformshift{1.616612in}{4.041530in}%
\pgfsys@useobject{currentmarker}{}%
\end{pgfscope}%
\begin{pgfscope}%
\pgfsys@transformshift{1.637502in}{4.034898in}%
\pgfsys@useobject{currentmarker}{}%
\end{pgfscope}%
\begin{pgfscope}%
\pgfsys@transformshift{1.656750in}{4.037146in}%
\pgfsys@useobject{currentmarker}{}%
\end{pgfscope}%
\begin{pgfscope}%
\pgfsys@transformshift{1.673886in}{4.042735in}%
\pgfsys@useobject{currentmarker}{}%
\end{pgfscope}%
\begin{pgfscope}%
\pgfsys@transformshift{1.693837in}{4.064140in}%
\pgfsys@useobject{currentmarker}{}%
\end{pgfscope}%
\begin{pgfscope}%
\pgfsys@transformshift{1.713555in}{4.139424in}%
\pgfsys@useobject{currentmarker}{}%
\end{pgfscope}%
\begin{pgfscope}%
\pgfsys@transformshift{1.732334in}{4.360784in}%
\pgfsys@useobject{currentmarker}{}%
\end{pgfscope}%
\begin{pgfscope}%
\pgfsys@transformshift{1.756276in}{4.400124in}%
\pgfsys@useobject{currentmarker}{}%
\end{pgfscope}%
\begin{pgfscope}%
\pgfsys@transformshift{1.770829in}{4.322500in}%
\pgfsys@useobject{currentmarker}{}%
\end{pgfscope}%
\begin{pgfscope}%
\pgfsys@transformshift{1.789137in}{4.167612in}%
\pgfsys@useobject{currentmarker}{}%
\end{pgfscope}%
\begin{pgfscope}%
\pgfsys@transformshift{1.807211in}{4.073804in}%
\pgfsys@useobject{currentmarker}{}%
\end{pgfscope}%
\begin{pgfscope}%
\pgfsys@transformshift{1.828572in}{4.052605in}%
\pgfsys@useobject{currentmarker}{}%
\end{pgfscope}%
\begin{pgfscope}%
\pgfsys@transformshift{1.846646in}{4.040160in}%
\pgfsys@useobject{currentmarker}{}%
\end{pgfscope}%
\begin{pgfscope}%
\pgfsys@transformshift{1.867302in}{4.035007in}%
\pgfsys@useobject{currentmarker}{}%
\end{pgfscope}%
\begin{pgfscope}%
\pgfsys@transformshift{1.888898in}{4.036100in}%
\pgfsys@useobject{currentmarker}{}%
\end{pgfscope}%
\begin{pgfscope}%
\pgfsys@transformshift{1.906268in}{4.045488in}%
\pgfsys@useobject{currentmarker}{}%
\end{pgfscope}%
\begin{pgfscope}%
\pgfsys@transformshift{1.924342in}{4.065986in}%
\pgfsys@useobject{currentmarker}{}%
\end{pgfscope}%
\begin{pgfscope}%
\pgfsys@transformshift{1.944998in}{4.034739in}%
\pgfsys@useobject{currentmarker}{}%
\end{pgfscope}%
\begin{pgfscope}%
\pgfsys@transformshift{1.962603in}{4.037518in}%
\pgfsys@useobject{currentmarker}{}%
\end{pgfscope}%
\begin{pgfscope}%
\pgfsys@transformshift{1.981616in}{4.048630in}%
\pgfsys@useobject{currentmarker}{}%
\end{pgfscope}%
\begin{pgfscope}%
\pgfsys@transformshift{2.001567in}{4.084956in}%
\pgfsys@useobject{currentmarker}{}%
\end{pgfscope}%
\begin{pgfscope}%
\pgfsys@transformshift{2.019643in}{4.195301in}%
\pgfsys@useobject{currentmarker}{}%
\end{pgfscope}%
\begin{pgfscope}%
\pgfsys@transformshift{2.039828in}{4.412458in}%
\pgfsys@useobject{currentmarker}{}%
\end{pgfscope}%
\begin{pgfscope}%
\pgfsys@transformshift{2.059076in}{4.374145in}%
\pgfsys@useobject{currentmarker}{}%
\end{pgfscope}%
\begin{pgfscope}%
\pgfsys@transformshift{2.079497in}{4.178703in}%
\pgfsys@useobject{currentmarker}{}%
\end{pgfscope}%
\begin{pgfscope}%
\pgfsys@transformshift{2.097807in}{4.079209in}%
\pgfsys@useobject{currentmarker}{}%
\end{pgfscope}%
\begin{pgfscope}%
\pgfsys@transformshift{2.116115in}{4.050707in}%
\pgfsys@useobject{currentmarker}{}%
\end{pgfscope}%
\begin{pgfscope}%
\pgfsys@transformshift{2.139823in}{4.036748in}%
\pgfsys@useobject{currentmarker}{}%
\end{pgfscope}%
\begin{pgfscope}%
\pgfsys@transformshift{2.154845in}{4.034767in}%
\pgfsys@useobject{currentmarker}{}%
\end{pgfscope}%
\begin{pgfscope}%
\pgfsys@transformshift{2.176675in}{4.038602in}%
\pgfsys@useobject{currentmarker}{}%
\end{pgfscope}%
\begin{pgfscope}%
\pgfsys@transformshift{2.193341in}{4.050628in}%
\pgfsys@useobject{currentmarker}{}%
\end{pgfscope}%
\begin{pgfscope}%
\pgfsys@transformshift{2.213293in}{4.104441in}%
\pgfsys@useobject{currentmarker}{}%
\end{pgfscope}%
\begin{pgfscope}%
\pgfsys@transformshift{2.232307in}{4.263534in}%
\pgfsys@useobject{currentmarker}{}%
\end{pgfscope}%
\begin{pgfscope}%
\pgfsys@transformshift{2.250849in}{4.408275in}%
\pgfsys@useobject{currentmarker}{}%
\end{pgfscope}%
\begin{pgfscope}%
\pgfsys@transformshift{2.273150in}{4.389059in}%
\pgfsys@useobject{currentmarker}{}%
\end{pgfscope}%
\begin{pgfscope}%
\pgfsys@transformshift{2.288641in}{4.285628in}%
\pgfsys@useobject{currentmarker}{}%
\end{pgfscope}%
\begin{pgfscope}%
\pgfsys@transformshift{2.311646in}{4.111096in}%
\pgfsys@useobject{currentmarker}{}%
\end{pgfscope}%
\begin{pgfscope}%
\pgfsys@transformshift{2.328076in}{4.053403in}%
\pgfsys@useobject{currentmarker}{}%
\end{pgfscope}%
\begin{pgfscope}%
\pgfsys@transformshift{2.346150in}{4.038494in}%
\pgfsys@useobject{currentmarker}{}%
\end{pgfscope}%
\begin{pgfscope}%
\pgfsys@transformshift{2.365163in}{4.035182in}%
\pgfsys@useobject{currentmarker}{}%
\end{pgfscope}%
\begin{pgfscope}%
\pgfsys@transformshift{2.385819in}{4.036119in}%
\pgfsys@useobject{currentmarker}{}%
\end{pgfscope}%
\begin{pgfscope}%
\pgfsys@transformshift{2.405301in}{4.043437in}%
\pgfsys@useobject{currentmarker}{}%
\end{pgfscope}%
\begin{pgfscope}%
\pgfsys@transformshift{2.423611in}{4.066776in}%
\pgfsys@useobject{currentmarker}{}%
\end{pgfscope}%
\begin{pgfscope}%
\pgfsys@transformshift{2.443797in}{4.155909in}%
\pgfsys@useobject{currentmarker}{}%
\end{pgfscope}%
\begin{pgfscope}%
\pgfsys@transformshift{2.463281in}{4.299337in}%
\pgfsys@useobject{currentmarker}{}%
\end{pgfscope}%
\begin{pgfscope}%
\pgfsys@transformshift{2.480649in}{4.409104in}%
\pgfsys@useobject{currentmarker}{}%
\end{pgfscope}%
\begin{pgfscope}%
\pgfsys@transformshift{2.505767in}{4.345299in}%
\pgfsys@useobject{currentmarker}{}%
\end{pgfscope}%
\begin{pgfscope}%
\pgfsys@transformshift{2.518910in}{4.210612in}%
\pgfsys@useobject{currentmarker}{}%
\end{pgfscope}%
\begin{pgfscope}%
\pgfsys@transformshift{2.541446in}{4.073238in}%
\pgfsys@useobject{currentmarker}{}%
\end{pgfscope}%
\begin{pgfscope}%
\pgfsys@transformshift{2.556468in}{4.046618in}%
\pgfsys@useobject{currentmarker}{}%
\end{pgfscope}%
\begin{pgfscope}%
\pgfsys@transformshift{2.578767in}{4.036675in}%
\pgfsys@useobject{currentmarker}{}%
\end{pgfscope}%
\begin{pgfscope}%
\pgfsys@transformshift{2.597077in}{4.034611in}%
\pgfsys@useobject{currentmarker}{}%
\end{pgfscope}%
\begin{pgfscope}%
\pgfsys@transformshift{2.618202in}{4.036346in}%
\pgfsys@useobject{currentmarker}{}%
\end{pgfscope}%
\begin{pgfscope}%
\pgfsys@transformshift{2.637450in}{4.044682in}%
\pgfsys@useobject{currentmarker}{}%
\end{pgfscope}%
\begin{pgfscope}%
\pgfsys@transformshift{2.656228in}{4.058941in}%
\pgfsys@useobject{currentmarker}{}%
\end{pgfscope}%
\begin{pgfscope}%
\pgfsys@transformshift{2.674771in}{4.107318in}%
\pgfsys@useobject{currentmarker}{}%
\end{pgfscope}%
\begin{pgfscope}%
\pgfsys@transformshift{2.692141in}{4.244957in}%
\pgfsys@useobject{currentmarker}{}%
\end{pgfscope}%
\begin{pgfscope}%
\pgfsys@transformshift{2.713503in}{4.407378in}%
\pgfsys@useobject{currentmarker}{}%
\end{pgfscope}%
\begin{pgfscope}%
\pgfsys@transformshift{2.731106in}{4.404660in}%
\pgfsys@useobject{currentmarker}{}%
\end{pgfscope}%
\begin{pgfscope}%
\pgfsys@transformshift{2.751998in}{4.313470in}%
\pgfsys@useobject{currentmarker}{}%
\end{pgfscope}%
\begin{pgfscope}%
\pgfsys@transformshift{2.771246in}{4.135915in}%
\pgfsys@useobject{currentmarker}{}%
\end{pgfscope}%
\begin{pgfscope}%
\pgfsys@transformshift{2.788850in}{4.064577in}%
\pgfsys@useobject{currentmarker}{}%
\end{pgfscope}%
\begin{pgfscope}%
\pgfsys@transformshift{2.809507in}{4.043919in}%
\pgfsys@useobject{currentmarker}{}%
\end{pgfscope}%
\begin{pgfscope}%
\pgfsys@transformshift{2.826877in}{4.036703in}%
\pgfsys@useobject{currentmarker}{}%
\end{pgfscope}%
\begin{pgfscope}%
\pgfsys@transformshift{2.848002in}{4.034810in}%
\pgfsys@useobject{currentmarker}{}%
\end{pgfscope}%
\begin{pgfscope}%
\pgfsys@transformshift{2.866076in}{4.037999in}%
\pgfsys@useobject{currentmarker}{}%
\end{pgfscope}%
\begin{pgfscope}%
\pgfsys@transformshift{2.883680in}{4.045042in}%
\pgfsys@useobject{currentmarker}{}%
\end{pgfscope}%
\begin{pgfscope}%
\pgfsys@transformshift{2.902225in}{4.065993in}%
\pgfsys@useobject{currentmarker}{}%
\end{pgfscope}%
\begin{pgfscope}%
\pgfsys@transformshift{2.923350in}{4.131819in}%
\pgfsys@useobject{currentmarker}{}%
\end{pgfscope}%
\begin{pgfscope}%
\pgfsys@transformshift{2.944946in}{4.318440in}%
\pgfsys@useobject{currentmarker}{}%
\end{pgfscope}%
\begin{pgfscope}%
\pgfsys@transformshift{2.962080in}{4.410051in}%
\pgfsys@useobject{currentmarker}{}%
\end{pgfscope}%
\begin{pgfscope}%
\pgfsys@transformshift{2.980390in}{4.383994in}%
\pgfsys@useobject{currentmarker}{}%
\end{pgfscope}%
\begin{pgfscope}%
\pgfsys@transformshift{3.001749in}{4.209446in}%
\pgfsys@useobject{currentmarker}{}%
\end{pgfscope}%
\begin{pgfscope}%
\pgfsys@transformshift{3.020997in}{4.086510in}%
\pgfsys@useobject{currentmarker}{}%
\end{pgfscope}%
\begin{pgfscope}%
\pgfsys@transformshift{3.037427in}{4.050746in}%
\pgfsys@useobject{currentmarker}{}%
\end{pgfscope}%
\begin{pgfscope}%
\pgfsys@transformshift{3.058320in}{4.038299in}%
\pgfsys@useobject{currentmarker}{}%
\end{pgfscope}%
\begin{pgfscope}%
\pgfsys@transformshift{3.076862in}{4.034804in}%
\pgfsys@useobject{currentmarker}{}%
\end{pgfscope}%
\begin{pgfscope}%
\pgfsys@transformshift{3.097050in}{4.035693in}%
\pgfsys@useobject{currentmarker}{}%
\end{pgfscope}%
\begin{pgfscope}%
\pgfsys@transformshift{3.114889in}{4.041911in}%
\pgfsys@useobject{currentmarker}{}%
\end{pgfscope}%
\begin{pgfscope}%
\pgfsys@transformshift{3.134842in}{4.057792in}%
\pgfsys@useobject{currentmarker}{}%
\end{pgfscope}%
\begin{pgfscope}%
\pgfsys@transformshift{3.154558in}{4.092493in}%
\pgfsys@useobject{currentmarker}{}%
\end{pgfscope}%
\begin{pgfscope}%
\pgfsys@transformshift{3.175918in}{4.206371in}%
\pgfsys@useobject{currentmarker}{}%
\end{pgfscope}%
\begin{pgfscope}%
\pgfsys@transformshift{3.193759in}{4.394453in}%
\pgfsys@useobject{currentmarker}{}%
\end{pgfscope}%
\begin{pgfscope}%
\pgfsys@transformshift{3.213710in}{4.411549in}%
\pgfsys@useobject{currentmarker}{}%
\end{pgfscope}%
\begin{pgfscope}%
\pgfsys@transformshift{3.232489in}{4.331112in}%
\pgfsys@useobject{currentmarker}{}%
\end{pgfscope}%
\begin{pgfscope}%
\pgfsys@transformshift{3.249859in}{4.180626in}%
\pgfsys@useobject{currentmarker}{}%
\end{pgfscope}%
\begin{pgfscope}%
\pgfsys@transformshift{3.271219in}{4.084971in}%
\pgfsys@useobject{currentmarker}{}%
\end{pgfscope}%
\begin{pgfscope}%
\pgfsys@transformshift{3.289763in}{4.054951in}%
\pgfsys@useobject{currentmarker}{}%
\end{pgfscope}%
\begin{pgfscope}%
\pgfsys@transformshift{3.309479in}{4.041739in}%
\pgfsys@useobject{currentmarker}{}%
\end{pgfscope}%
\begin{pgfscope}%
\pgfsys@transformshift{3.326381in}{4.036662in}%
\pgfsys@useobject{currentmarker}{}%
\end{pgfscope}%
\begin{pgfscope}%
\pgfsys@transformshift{3.347506in}{4.035390in}%
\pgfsys@useobject{currentmarker}{}%
\end{pgfscope}%
\begin{pgfscope}%
\pgfsys@transformshift{3.365580in}{4.037956in}%
\pgfsys@useobject{currentmarker}{}%
\end{pgfscope}%
\begin{pgfscope}%
\pgfsys@transformshift{3.386001in}{4.044696in}%
\pgfsys@useobject{currentmarker}{}%
\end{pgfscope}%
\begin{pgfscope}%
\pgfsys@transformshift{3.403841in}{4.058602in}%
\pgfsys@useobject{currentmarker}{}%
\end{pgfscope}%
\begin{pgfscope}%
\pgfsys@transformshift{3.425202in}{4.096993in}%
\pgfsys@useobject{currentmarker}{}%
\end{pgfscope}%
\begin{pgfscope}%
\pgfsys@transformshift{3.443041in}{4.162676in}%
\pgfsys@useobject{currentmarker}{}%
\end{pgfscope}%
\begin{pgfscope}%
\pgfsys@transformshift{3.461349in}{4.105266in}%
\pgfsys@useobject{currentmarker}{}%
\end{pgfscope}%
\begin{pgfscope}%
\pgfsys@transformshift{3.478954in}{4.056269in}%
\pgfsys@useobject{currentmarker}{}%
\end{pgfscope}%
\begin{pgfscope}%
\pgfsys@transformshift{3.500550in}{4.096279in}%
\pgfsys@useobject{currentmarker}{}%
\end{pgfscope}%
\begin{pgfscope}%
\pgfsys@transformshift{3.518155in}{4.246713in}%
\pgfsys@useobject{currentmarker}{}%
\end{pgfscope}%
\begin{pgfscope}%
\pgfsys@transformshift{3.539748in}{4.425162in}%
\pgfsys@useobject{currentmarker}{}%
\end{pgfscope}%
\begin{pgfscope}%
\pgfsys@transformshift{3.557119in}{4.390746in}%
\pgfsys@useobject{currentmarker}{}%
\end{pgfscope}%
\begin{pgfscope}%
\pgfsys@transformshift{3.575663in}{4.267254in}%
\pgfsys@useobject{currentmarker}{}%
\end{pgfscope}%
\begin{pgfscope}%
\pgfsys@transformshift{3.597728in}{4.111583in}%
\pgfsys@useobject{currentmarker}{}%
\end{pgfscope}%
\begin{pgfscope}%
\pgfsys@transformshift{3.614862in}{4.058286in}%
\pgfsys@useobject{currentmarker}{}%
\end{pgfscope}%
\begin{pgfscope}%
\pgfsys@transformshift{3.635754in}{4.041139in}%
\pgfsys@useobject{currentmarker}{}%
\end{pgfscope}%
\begin{pgfscope}%
\pgfsys@transformshift{3.654766in}{4.036369in}%
\pgfsys@useobject{currentmarker}{}%
\end{pgfscope}%
\begin{pgfscope}%
\pgfsys@transformshift{3.676127in}{4.035683in}%
\pgfsys@useobject{currentmarker}{}%
\end{pgfscope}%
\begin{pgfscope}%
\pgfsys@transformshift{3.693263in}{4.040743in}%
\pgfsys@useobject{currentmarker}{}%
\end{pgfscope}%
\begin{pgfscope}%
\pgfsys@transformshift{3.710631in}{4.049892in}%
\pgfsys@useobject{currentmarker}{}%
\end{pgfscope}%
\begin{pgfscope}%
\pgfsys@transformshift{3.728707in}{4.074907in}%
\pgfsys@useobject{currentmarker}{}%
\end{pgfscope}%
\begin{pgfscope}%
\pgfsys@transformshift{3.749127in}{4.147315in}%
\pgfsys@useobject{currentmarker}{}%
\end{pgfscope}%
\begin{pgfscope}%
\pgfsys@transformshift{3.768611in}{4.340507in}%
\pgfsys@useobject{currentmarker}{}%
\end{pgfscope}%
\begin{pgfscope}%
\pgfsys@transformshift{3.786919in}{4.434753in}%
\pgfsys@useobject{currentmarker}{}%
\end{pgfscope}%
\begin{pgfscope}%
\pgfsys@transformshift{3.807809in}{4.399508in}%
\pgfsys@useobject{currentmarker}{}%
\end{pgfscope}%
\begin{pgfscope}%
\pgfsys@transformshift{3.827997in}{4.262515in}%
\pgfsys@useobject{currentmarker}{}%
\end{pgfscope}%
\begin{pgfscope}%
\pgfsys@transformshift{3.846070in}{4.169735in}%
\pgfsys@useobject{currentmarker}{}%
\end{pgfscope}%
\begin{pgfscope}%
\pgfsys@transformshift{3.864380in}{4.092729in}%
\pgfsys@useobject{currentmarker}{}%
\end{pgfscope}%
\begin{pgfscope}%
\pgfsys@transformshift{3.886210in}{4.051042in}%
\pgfsys@useobject{currentmarker}{}%
\end{pgfscope}%
\begin{pgfscope}%
\pgfsys@transformshift{3.902641in}{4.039736in}%
\pgfsys@useobject{currentmarker}{}%
\end{pgfscope}%
\begin{pgfscope}%
\pgfsys@transformshift{3.920246in}{4.035826in}%
\pgfsys@useobject{currentmarker}{}%
\end{pgfscope}%
\begin{pgfscope}%
\pgfsys@transformshift{3.944891in}{4.039102in}%
\pgfsys@useobject{currentmarker}{}%
\end{pgfscope}%
\begin{pgfscope}%
\pgfsys@transformshift{3.960619in}{4.046903in}%
\pgfsys@useobject{currentmarker}{}%
\end{pgfscope}%
\begin{pgfscope}%
\pgfsys@transformshift{3.982214in}{4.063092in}%
\pgfsys@useobject{currentmarker}{}%
\end{pgfscope}%
\begin{pgfscope}%
\pgfsys@transformshift{3.999585in}{4.082977in}%
\pgfsys@useobject{currentmarker}{}%
\end{pgfscope}%
\begin{pgfscope}%
\pgfsys@transformshift{4.020241in}{4.169556in}%
\pgfsys@useobject{currentmarker}{}%
\end{pgfscope}%
\begin{pgfscope}%
\pgfsys@transformshift{4.038080in}{4.368185in}%
\pgfsys@useobject{currentmarker}{}%
\end{pgfscope}%
\begin{pgfscope}%
\pgfsys@transformshift{4.055919in}{4.442017in}%
\pgfsys@useobject{currentmarker}{}%
\end{pgfscope}%
\begin{pgfscope}%
\pgfsys@transformshift{4.076810in}{4.427627in}%
\pgfsys@useobject{currentmarker}{}%
\end{pgfscope}%
\begin{pgfscope}%
\pgfsys@transformshift{4.095354in}{4.357812in}%
\pgfsys@useobject{currentmarker}{}%
\end{pgfscope}%
\begin{pgfscope}%
\pgfsys@transformshift{4.112723in}{4.271844in}%
\pgfsys@useobject{currentmarker}{}%
\end{pgfscope}%
\begin{pgfscope}%
\pgfsys@transformshift{4.134319in}{4.137403in}%
\pgfsys@useobject{currentmarker}{}%
\end{pgfscope}%
\begin{pgfscope}%
\pgfsys@transformshift{4.152392in}{4.071195in}%
\pgfsys@useobject{currentmarker}{}%
\end{pgfscope}%
\begin{pgfscope}%
\pgfsys@transformshift{4.173519in}{4.046083in}%
\pgfsys@useobject{currentmarker}{}%
\end{pgfscope}%
\begin{pgfscope}%
\pgfsys@transformshift{4.192296in}{4.037895in}%
\pgfsys@useobject{currentmarker}{}%
\end{pgfscope}%
\begin{pgfscope}%
\pgfsys@transformshift{4.215769in}{4.037657in}%
\pgfsys@useobject{currentmarker}{}%
\end{pgfscope}%
\begin{pgfscope}%
\pgfsys@transformshift{4.229149in}{4.040864in}%
\pgfsys@useobject{currentmarker}{}%
\end{pgfscope}%
\begin{pgfscope}%
\pgfsys@transformshift{4.248633in}{4.047954in}%
\pgfsys@useobject{currentmarker}{}%
\end{pgfscope}%
\begin{pgfscope}%
\pgfsys@transformshift{4.270226in}{4.072253in}%
\pgfsys@useobject{currentmarker}{}%
\end{pgfscope}%
\begin{pgfscope}%
\pgfsys@transformshift{4.290883in}{4.155290in}%
\pgfsys@useobject{currentmarker}{}%
\end{pgfscope}%
\begin{pgfscope}%
\pgfsys@transformshift{4.306610in}{4.202384in}%
\pgfsys@useobject{currentmarker}{}%
\end{pgfscope}%
\begin{pgfscope}%
\pgfsys@transformshift{4.327266in}{4.425881in}%
\pgfsys@useobject{currentmarker}{}%
\end{pgfscope}%
\begin{pgfscope}%
\pgfsys@transformshift{4.345340in}{4.457846in}%
\pgfsys@useobject{currentmarker}{}%
\end{pgfscope}%
\begin{pgfscope}%
\pgfsys@transformshift{4.367875in}{4.404801in}%
\pgfsys@useobject{currentmarker}{}%
\end{pgfscope}%
\begin{pgfscope}%
\pgfsys@transformshift{4.381958in}{4.364497in}%
\pgfsys@useobject{currentmarker}{}%
\end{pgfscope}%
\begin{pgfscope}%
\pgfsys@transformshift{4.402614in}{4.228077in}%
\pgfsys@useobject{currentmarker}{}%
\end{pgfscope}%
\begin{pgfscope}%
\pgfsys@transformshift{4.423505in}{4.091838in}%
\pgfsys@useobject{currentmarker}{}%
\end{pgfscope}%
\begin{pgfscope}%
\pgfsys@transformshift{4.441109in}{4.056304in}%
\pgfsys@useobject{currentmarker}{}%
\end{pgfscope}%
\begin{pgfscope}%
\pgfsys@transformshift{4.461766in}{4.041437in}%
\pgfsys@useobject{currentmarker}{}%
\end{pgfscope}%
\begin{pgfscope}%
\pgfsys@transformshift{4.481250in}{4.036865in}%
\pgfsys@useobject{currentmarker}{}%
\end{pgfscope}%
\begin{pgfscope}%
\pgfsys@transformshift{4.482656in}{4.036934in}%
\pgfsys@useobject{currentmarker}{}%
\end{pgfscope}%
\begin{pgfscope}%
\pgfsys@transformshift{4.473737in}{4.038708in}%
\pgfsys@useobject{currentmarker}{}%
\end{pgfscope}%
\begin{pgfscope}%
\pgfsys@transformshift{4.455428in}{4.054219in}%
\pgfsys@useobject{currentmarker}{}%
\end{pgfscope}%
\begin{pgfscope}%
\pgfsys@transformshift{4.435007in}{4.125202in}%
\pgfsys@useobject{currentmarker}{}%
\end{pgfscope}%
\begin{pgfscope}%
\pgfsys@transformshift{4.419516in}{4.316894in}%
\pgfsys@useobject{currentmarker}{}%
\end{pgfscope}%
\begin{pgfscope}%
\pgfsys@transformshift{4.397685in}{4.441528in}%
\pgfsys@useobject{currentmarker}{}%
\end{pgfscope}%
\begin{pgfscope}%
\pgfsys@transformshift{4.377967in}{4.432975in}%
\pgfsys@useobject{currentmarker}{}%
\end{pgfscope}%
\begin{pgfscope}%
\pgfsys@transformshift{4.357547in}{4.175612in}%
\pgfsys@useobject{currentmarker}{}%
\end{pgfscope}%
\begin{pgfscope}%
\pgfsys@transformshift{4.340177in}{4.071536in}%
\pgfsys@useobject{currentmarker}{}%
\end{pgfscope}%
\begin{pgfscope}%
\pgfsys@transformshift{4.319521in}{4.042859in}%
\pgfsys@useobject{currentmarker}{}%
\end{pgfscope}%
\begin{pgfscope}%
\pgfsys@transformshift{4.302150in}{4.036315in}%
\pgfsys@useobject{currentmarker}{}%
\end{pgfscope}%
\begin{pgfscope}%
\pgfsys@transformshift{4.283137in}{4.041787in}%
\pgfsys@useobject{currentmarker}{}%
\end{pgfscope}%
\begin{pgfscope}%
\pgfsys@transformshift{4.261307in}{4.126044in}%
\pgfsys@useobject{currentmarker}{}%
\end{pgfscope}%
\begin{pgfscope}%
\pgfsys@transformshift{4.244407in}{4.292976in}%
\pgfsys@useobject{currentmarker}{}%
\end{pgfscope}%
\begin{pgfscope}%
\pgfsys@transformshift{4.224923in}{4.440079in}%
\pgfsys@useobject{currentmarker}{}%
\end{pgfscope}%
\begin{pgfscope}%
\pgfsys@transformshift{4.204267in}{4.403148in}%
\pgfsys@useobject{currentmarker}{}%
\end{pgfscope}%
\begin{pgfscope}%
\pgfsys@transformshift{4.187367in}{4.164990in}%
\pgfsys@useobject{currentmarker}{}%
\end{pgfscope}%
\begin{pgfscope}%
\pgfsys@transformshift{4.165537in}{4.060730in}%
\pgfsys@useobject{currentmarker}{}%
\end{pgfscope}%
\begin{pgfscope}%
\pgfsys@transformshift{4.147698in}{4.040469in}%
\pgfsys@useobject{currentmarker}{}%
\end{pgfscope}%
\begin{pgfscope}%
\pgfsys@transformshift{4.127982in}{4.035744in}%
\pgfsys@useobject{currentmarker}{}%
\end{pgfscope}%
\begin{pgfscope}%
\pgfsys@transformshift{4.111785in}{4.040980in}%
\pgfsys@useobject{currentmarker}{}%
\end{pgfscope}%
\begin{pgfscope}%
\pgfsys@transformshift{4.090189in}{4.070814in}%
\pgfsys@useobject{currentmarker}{}%
\end{pgfscope}%
\begin{pgfscope}%
\pgfsys@transformshift{4.071645in}{4.168174in}%
\pgfsys@useobject{currentmarker}{}%
\end{pgfscope}%
\begin{pgfscope}%
\pgfsys@transformshift{4.051460in}{4.380303in}%
\pgfsys@useobject{currentmarker}{}%
\end{pgfscope}%
\begin{pgfscope}%
\pgfsys@transformshift{4.030803in}{4.436731in}%
\pgfsys@useobject{currentmarker}{}%
\end{pgfscope}%
\begin{pgfscope}%
\pgfsys@transformshift{4.013902in}{4.270165in}%
\pgfsys@useobject{currentmarker}{}%
\end{pgfscope}%
\begin{pgfscope}%
\pgfsys@transformshift{3.996768in}{4.092448in}%
\pgfsys@useobject{currentmarker}{}%
\end{pgfscope}%
\begin{pgfscope}%
\pgfsys@transformshift{3.977049in}{4.047271in}%
\pgfsys@useobject{currentmarker}{}%
\end{pgfscope}%
\begin{pgfscope}%
\pgfsys@transformshift{3.955690in}{4.036249in}%
\pgfsys@useobject{currentmarker}{}%
\end{pgfscope}%
\begin{pgfscope}%
\pgfsys@transformshift{3.934563in}{4.036880in}%
\pgfsys@useobject{currentmarker}{}%
\end{pgfscope}%
\begin{pgfscope}%
\pgfsys@transformshift{3.915786in}{4.048530in}%
\pgfsys@useobject{currentmarker}{}%
\end{pgfscope}%
\begin{pgfscope}%
\pgfsys@transformshift{3.898181in}{4.092428in}%
\pgfsys@useobject{currentmarker}{}%
\end{pgfscope}%
\begin{pgfscope}%
\pgfsys@transformshift{3.874003in}{4.310045in}%
\pgfsys@useobject{currentmarker}{}%
\end{pgfscope}%
\begin{pgfscope}%
\pgfsys@transformshift{3.860389in}{4.409173in}%
\pgfsys@useobject{currentmarker}{}%
\end{pgfscope}%
\begin{pgfscope}%
\pgfsys@transformshift{3.839030in}{4.408681in}%
\pgfsys@useobject{currentmarker}{}%
\end{pgfscope}%
\begin{pgfscope}%
\pgfsys@transformshift{3.818843in}{4.167100in}%
\pgfsys@useobject{currentmarker}{}%
\end{pgfscope}%
\begin{pgfscope}%
\pgfsys@transformshift{3.800298in}{4.069886in}%
\pgfsys@useobject{currentmarker}{}%
\end{pgfscope}%
\begin{pgfscope}%
\pgfsys@transformshift{3.783633in}{4.043023in}%
\pgfsys@useobject{currentmarker}{}%
\end{pgfscope}%
\begin{pgfscope}%
\pgfsys@transformshift{3.762272in}{4.035624in}%
\pgfsys@useobject{currentmarker}{}%
\end{pgfscope}%
\begin{pgfscope}%
\pgfsys@transformshift{3.745607in}{4.036789in}%
\pgfsys@useobject{currentmarker}{}%
\end{pgfscope}%
\begin{pgfscope}%
\pgfsys@transformshift{3.724247in}{4.049275in}%
\pgfsys@useobject{currentmarker}{}%
\end{pgfscope}%
\begin{pgfscope}%
\pgfsys@transformshift{3.708051in}{4.093728in}%
\pgfsys@useobject{currentmarker}{}%
\end{pgfscope}%
\begin{pgfscope}%
\pgfsys@transformshift{3.686689in}{4.271361in}%
\pgfsys@useobject{currentmarker}{}%
\end{pgfscope}%
\begin{pgfscope}%
\pgfsys@transformshift{3.665799in}{4.410036in}%
\pgfsys@useobject{currentmarker}{}%
\end{pgfscope}%
\begin{pgfscope}%
\pgfsys@transformshift{3.648194in}{4.417992in}%
\pgfsys@useobject{currentmarker}{}%
\end{pgfscope}%
\begin{pgfscope}%
\pgfsys@transformshift{3.630589in}{4.269188in}%
\pgfsys@useobject{currentmarker}{}%
\end{pgfscope}%
\begin{pgfscope}%
\pgfsys@transformshift{3.610168in}{4.084036in}%
\pgfsys@useobject{currentmarker}{}%
\end{pgfscope}%
\begin{pgfscope}%
\pgfsys@transformshift{3.592797in}{4.056249in}%
\pgfsys@useobject{currentmarker}{}%
\end{pgfscope}%
\begin{pgfscope}%
\pgfsys@transformshift{3.570498in}{4.038871in}%
\pgfsys@useobject{currentmarker}{}%
\end{pgfscope}%
\begin{pgfscope}%
\pgfsys@transformshift{3.551250in}{4.035153in}%
\pgfsys@useobject{currentmarker}{}%
\end{pgfscope}%
\begin{pgfscope}%
\pgfsys@transformshift{3.532237in}{4.037218in}%
\pgfsys@useobject{currentmarker}{}%
\end{pgfscope}%
\begin{pgfscope}%
\pgfsys@transformshift{3.513224in}{4.047814in}%
\pgfsys@useobject{currentmarker}{}%
\end{pgfscope}%
\begin{pgfscope}%
\pgfsys@transformshift{3.494916in}{4.096111in}%
\pgfsys@useobject{currentmarker}{}%
\end{pgfscope}%
\begin{pgfscope}%
\pgfsys@transformshift{3.476608in}{4.233430in}%
\pgfsys@useobject{currentmarker}{}%
\end{pgfscope}%
\begin{pgfscope}%
\pgfsys@transformshift{3.455481in}{4.392975in}%
\pgfsys@useobject{currentmarker}{}%
\end{pgfscope}%
\begin{pgfscope}%
\pgfsys@transformshift{3.435530in}{4.419355in}%
\pgfsys@useobject{currentmarker}{}%
\end{pgfscope}%
\begin{pgfscope}%
\pgfsys@transformshift{3.416751in}{4.239035in}%
\pgfsys@useobject{currentmarker}{}%
\end{pgfscope}%
\begin{pgfscope}%
\pgfsys@transformshift{3.399850in}{4.099028in}%
\pgfsys@useobject{currentmarker}{}%
\end{pgfscope}%
\begin{pgfscope}%
\pgfsys@transformshift{3.377551in}{4.047188in}%
\pgfsys@useobject{currentmarker}{}%
\end{pgfscope}%
\begin{pgfscope}%
\pgfsys@transformshift{3.359242in}{4.037205in}%
\pgfsys@useobject{currentmarker}{}%
\end{pgfscope}%
\begin{pgfscope}%
\pgfsys@transformshift{3.342577in}{4.034934in}%
\pgfsys@useobject{currentmarker}{}%
\end{pgfscope}%
\begin{pgfscope}%
\pgfsys@transformshift{3.318868in}{4.039709in}%
\pgfsys@useobject{currentmarker}{}%
\end{pgfscope}%
\begin{pgfscope}%
\pgfsys@transformshift{3.297274in}{4.057071in}%
\pgfsys@useobject{currentmarker}{}%
\end{pgfscope}%
\begin{pgfscope}%
\pgfsys@transformshift{3.282252in}{4.099494in}%
\pgfsys@useobject{currentmarker}{}%
\end{pgfscope}%
\begin{pgfscope}%
\pgfsys@transformshift{3.263239in}{4.257482in}%
\pgfsys@useobject{currentmarker}{}%
\end{pgfscope}%
\begin{pgfscope}%
\pgfsys@transformshift{3.245163in}{4.383112in}%
\pgfsys@useobject{currentmarker}{}%
\end{pgfscope}%
\begin{pgfscope}%
\pgfsys@transformshift{3.224038in}{4.416517in}%
\pgfsys@useobject{currentmarker}{}%
\end{pgfscope}%
\begin{pgfscope}%
\pgfsys@transformshift{3.205259in}{4.239400in}%
\pgfsys@useobject{currentmarker}{}%
\end{pgfscope}%
\begin{pgfscope}%
\pgfsys@transformshift{3.187654in}{4.124481in}%
\pgfsys@useobject{currentmarker}{}%
\end{pgfscope}%
\begin{pgfscope}%
\pgfsys@transformshift{3.168407in}{4.055700in}%
\pgfsys@useobject{currentmarker}{}%
\end{pgfscope}%
\begin{pgfscope}%
\pgfsys@transformshift{3.148925in}{4.040135in}%
\pgfsys@useobject{currentmarker}{}%
\end{pgfscope}%
\begin{pgfscope}%
\pgfsys@transformshift{3.127329in}{4.035032in}%
\pgfsys@useobject{currentmarker}{}%
\end{pgfscope}%
\begin{pgfscope}%
\pgfsys@transformshift{3.108552in}{4.036964in}%
\pgfsys@useobject{currentmarker}{}%
\end{pgfscope}%
\begin{pgfscope}%
\pgfsys@transformshift{3.090947in}{4.043268in}%
\pgfsys@useobject{currentmarker}{}%
\end{pgfscope}%
\begin{pgfscope}%
\pgfsys@transformshift{3.071229in}{4.066370in}%
\pgfsys@useobject{currentmarker}{}%
\end{pgfscope}%
\begin{pgfscope}%
\pgfsys@transformshift{3.051043in}{4.183834in}%
\pgfsys@useobject{currentmarker}{}%
\end{pgfscope}%
\begin{pgfscope}%
\pgfsys@transformshift{3.031090in}{4.281665in}%
\pgfsys@useobject{currentmarker}{}%
\end{pgfscope}%
\begin{pgfscope}%
\pgfsys@transformshift{3.015599in}{4.397128in}%
\pgfsys@useobject{currentmarker}{}%
\end{pgfscope}%
\begin{pgfscope}%
\pgfsys@transformshift{2.993064in}{4.391198in}%
\pgfsys@useobject{currentmarker}{}%
\end{pgfscope}%
\begin{pgfscope}%
\pgfsys@transformshift{2.975225in}{4.185593in}%
\pgfsys@useobject{currentmarker}{}%
\end{pgfscope}%
\begin{pgfscope}%
\pgfsys@transformshift{2.956682in}{4.091282in}%
\pgfsys@useobject{currentmarker}{}%
\end{pgfscope}%
\begin{pgfscope}%
\pgfsys@transformshift{2.937903in}{4.055735in}%
\pgfsys@useobject{currentmarker}{}%
\end{pgfscope}%
\begin{pgfscope}%
\pgfsys@transformshift{2.916542in}{4.039912in}%
\pgfsys@useobject{currentmarker}{}%
\end{pgfscope}%
\begin{pgfscope}%
\pgfsys@transformshift{2.897529in}{4.035279in}%
\pgfsys@useobject{currentmarker}{}%
\end{pgfscope}%
\begin{pgfscope}%
\pgfsys@transformshift{2.878047in}{4.035747in}%
\pgfsys@useobject{currentmarker}{}%
\end{pgfscope}%
\begin{pgfscope}%
\pgfsys@transformshift{2.859268in}{4.039833in}%
\pgfsys@useobject{currentmarker}{}%
\end{pgfscope}%
\begin{pgfscope}%
\pgfsys@transformshift{2.840960in}{4.060898in}%
\pgfsys@useobject{currentmarker}{}%
\end{pgfscope}%
\begin{pgfscope}%
\pgfsys@transformshift{2.820772in}{4.097235in}%
\pgfsys@useobject{currentmarker}{}%
\end{pgfscope}%
\begin{pgfscope}%
\pgfsys@transformshift{2.802230in}{4.231777in}%
\pgfsys@useobject{currentmarker}{}%
\end{pgfscope}%
\begin{pgfscope}%
\pgfsys@transformshift{2.782748in}{4.348266in}%
\pgfsys@useobject{currentmarker}{}%
\end{pgfscope}%
\begin{pgfscope}%
\pgfsys@transformshift{2.764203in}{4.409600in}%
\pgfsys@useobject{currentmarker}{}%
\end{pgfscope}%
\begin{pgfscope}%
\pgfsys@transformshift{2.744251in}{4.390709in}%
\pgfsys@useobject{currentmarker}{}%
\end{pgfscope}%
\begin{pgfscope}%
\pgfsys@transformshift{2.725474in}{4.219035in}%
\pgfsys@useobject{currentmarker}{}%
\end{pgfscope}%
\begin{pgfscope}%
\pgfsys@transformshift{2.706695in}{4.139940in}%
\pgfsys@useobject{currentmarker}{}%
\end{pgfscope}%
\begin{pgfscope}%
\pgfsys@transformshift{2.687682in}{4.073293in}%
\pgfsys@useobject{currentmarker}{}%
\end{pgfscope}%
\begin{pgfscope}%
\pgfsys@transformshift{2.667729in}{4.045095in}%
\pgfsys@useobject{currentmarker}{}%
\end{pgfscope}%
\begin{pgfscope}%
\pgfsys@transformshift{2.649421in}{4.036275in}%
\pgfsys@useobject{currentmarker}{}%
\end{pgfscope}%
\begin{pgfscope}%
\pgfsys@transformshift{2.630173in}{4.034979in}%
\pgfsys@useobject{currentmarker}{}%
\end{pgfscope}%
\begin{pgfscope}%
\pgfsys@transformshift{2.608108in}{4.040025in}%
\pgfsys@useobject{currentmarker}{}%
\end{pgfscope}%
\begin{pgfscope}%
\pgfsys@transformshift{2.590035in}{4.054707in}%
\pgfsys@useobject{currentmarker}{}%
\end{pgfscope}%
\begin{pgfscope}%
\pgfsys@transformshift{2.572664in}{4.068230in}%
\pgfsys@useobject{currentmarker}{}%
\end{pgfscope}%
\begin{pgfscope}%
\pgfsys@transformshift{2.548722in}{4.057335in}%
\pgfsys@useobject{currentmarker}{}%
\end{pgfscope}%
\begin{pgfscope}%
\pgfsys@transformshift{2.534169in}{4.111158in}%
\pgfsys@useobject{currentmarker}{}%
\end{pgfscope}%
\begin{pgfscope}%
\pgfsys@transformshift{2.515390in}{4.236067in}%
\pgfsys@useobject{currentmarker}{}%
\end{pgfscope}%
\begin{pgfscope}%
\pgfsys@transformshift{2.494265in}{4.401848in}%
\pgfsys@useobject{currentmarker}{}%
\end{pgfscope}%
\begin{pgfscope}%
\pgfsys@transformshift{2.478069in}{4.396655in}%
\pgfsys@useobject{currentmarker}{}%
\end{pgfscope}%
\begin{pgfscope}%
\pgfsys@transformshift{2.456004in}{4.181070in}%
\pgfsys@useobject{currentmarker}{}%
\end{pgfscope}%
\begin{pgfscope}%
\pgfsys@transformshift{2.437225in}{4.080805in}%
\pgfsys@useobject{currentmarker}{}%
\end{pgfscope}%
\begin{pgfscope}%
\pgfsys@transformshift{2.418212in}{4.045977in}%
\pgfsys@useobject{currentmarker}{}%
\end{pgfscope}%
\begin{pgfscope}%
\pgfsys@transformshift{2.399433in}{4.037350in}%
\pgfsys@useobject{currentmarker}{}%
\end{pgfscope}%
\begin{pgfscope}%
\pgfsys@transformshift{2.379482in}{4.034766in}%
\pgfsys@useobject{currentmarker}{}%
\end{pgfscope}%
\begin{pgfscope}%
\pgfsys@transformshift{2.361172in}{4.038110in}%
\pgfsys@useobject{currentmarker}{}%
\end{pgfscope}%
\begin{pgfscope}%
\pgfsys@transformshift{2.342864in}{4.047435in}%
\pgfsys@useobject{currentmarker}{}%
\end{pgfscope}%
\begin{pgfscope}%
\pgfsys@transformshift{2.321268in}{4.053094in}%
\pgfsys@useobject{currentmarker}{}%
\end{pgfscope}%
\begin{pgfscope}%
\pgfsys@transformshift{2.301552in}{4.045876in}%
\pgfsys@useobject{currentmarker}{}%
\end{pgfscope}%
\begin{pgfscope}%
\pgfsys@transformshift{2.282304in}{4.086083in}%
\pgfsys@useobject{currentmarker}{}%
\end{pgfscope}%
\begin{pgfscope}%
\pgfsys@transformshift{2.264934in}{4.218135in}%
\pgfsys@useobject{currentmarker}{}%
\end{pgfscope}%
\begin{pgfscope}%
\pgfsys@transformshift{2.244512in}{4.349862in}%
\pgfsys@useobject{currentmarker}{}%
\end{pgfscope}%
\begin{pgfscope}%
\pgfsys@transformshift{2.225733in}{4.410952in}%
\pgfsys@useobject{currentmarker}{}%
\end{pgfscope}%
\begin{pgfscope}%
\pgfsys@transformshift{2.206486in}{4.272073in}%
\pgfsys@useobject{currentmarker}{}%
\end{pgfscope}%
\begin{pgfscope}%
\pgfsys@transformshift{2.186535in}{4.107165in}%
\pgfsys@useobject{currentmarker}{}%
\end{pgfscope}%
\begin{pgfscope}%
\pgfsys@transformshift{2.168930in}{4.056884in}%
\pgfsys@useobject{currentmarker}{}%
\end{pgfscope}%
\begin{pgfscope}%
\pgfsys@transformshift{2.149917in}{4.040281in}%
\pgfsys@useobject{currentmarker}{}%
\end{pgfscope}%
\begin{pgfscope}%
\pgfsys@transformshift{2.133251in}{4.035529in}%
\pgfsys@useobject{currentmarker}{}%
\end{pgfscope}%
\begin{pgfscope}%
\pgfsys@transformshift{2.110482in}{4.035603in}%
\pgfsys@useobject{currentmarker}{}%
\end{pgfscope}%
\begin{pgfscope}%
\pgfsys@transformshift{2.094285in}{4.039083in}%
\pgfsys@useobject{currentmarker}{}%
\end{pgfscope}%
\begin{pgfscope}%
\pgfsys@transformshift{2.073864in}{4.054771in}%
\pgfsys@useobject{currentmarker}{}%
\end{pgfscope}%
\begin{pgfscope}%
\pgfsys@transformshift{2.052504in}{4.082241in}%
\pgfsys@useobject{currentmarker}{}%
\end{pgfscope}%
\begin{pgfscope}%
\pgfsys@transformshift{2.033256in}{4.211954in}%
\pgfsys@useobject{currentmarker}{}%
\end{pgfscope}%
\begin{pgfscope}%
\pgfsys@transformshift{2.014712in}{4.353500in}%
\pgfsys@useobject{currentmarker}{}%
\end{pgfscope}%
\begin{pgfscope}%
\pgfsys@transformshift{1.995699in}{4.412194in}%
\pgfsys@useobject{currentmarker}{}%
\end{pgfscope}%
\begin{pgfscope}%
\pgfsys@transformshift{1.971757in}{4.254409in}%
\pgfsys@useobject{currentmarker}{}%
\end{pgfscope}%
\begin{pgfscope}%
\pgfsys@transformshift{1.955795in}{4.124005in}%
\pgfsys@useobject{currentmarker}{}%
\end{pgfscope}%
\begin{pgfscope}%
\pgfsys@transformshift{1.939364in}{4.069410in}%
\pgfsys@useobject{currentmarker}{}%
\end{pgfscope}%
\begin{pgfscope}%
\pgfsys@transformshift{1.917534in}{4.049746in}%
\pgfsys@useobject{currentmarker}{}%
\end{pgfscope}%
\begin{pgfscope}%
\pgfsys@transformshift{1.901338in}{4.041737in}%
\pgfsys@useobject{currentmarker}{}%
\end{pgfscope}%
\begin{pgfscope}%
\pgfsys@transformshift{1.878804in}{4.035969in}%
\pgfsys@useobject{currentmarker}{}%
\end{pgfscope}%
\begin{pgfscope}%
\pgfsys@transformshift{1.863548in}{4.035221in}%
\pgfsys@useobject{currentmarker}{}%
\end{pgfscope}%
\begin{pgfscope}%
\pgfsys@transformshift{1.841483in}{4.039540in}%
\pgfsys@useobject{currentmarker}{}%
\end{pgfscope}%
\begin{pgfscope}%
\pgfsys@transformshift{1.823407in}{4.052924in}%
\pgfsys@useobject{currentmarker}{}%
\end{pgfscope}%
\begin{pgfscope}%
\pgfsys@transformshift{1.802048in}{4.123319in}%
\pgfsys@useobject{currentmarker}{}%
\end{pgfscope}%
\begin{pgfscope}%
\pgfsys@transformshift{1.784912in}{4.232190in}%
\pgfsys@useobject{currentmarker}{}%
\end{pgfscope}%
\begin{pgfscope}%
\pgfsys@transformshift{1.766133in}{4.366795in}%
\pgfsys@useobject{currentmarker}{}%
\end{pgfscope}%
\begin{pgfscope}%
\pgfsys@transformshift{1.747591in}{4.417171in}%
\pgfsys@useobject{currentmarker}{}%
\end{pgfscope}%
\begin{pgfscope}%
\pgfsys@transformshift{1.727403in}{4.363832in}%
\pgfsys@useobject{currentmarker}{}%
\end{pgfscope}%
\begin{pgfscope}%
\pgfsys@transformshift{1.702287in}{4.132614in}%
\pgfsys@useobject{currentmarker}{}%
\end{pgfscope}%
\begin{pgfscope}%
\pgfsys@transformshift{1.687031in}{4.070738in}%
\pgfsys@useobject{currentmarker}{}%
\end{pgfscope}%
\begin{pgfscope}%
\pgfsys@transformshift{1.668721in}{4.046453in}%
\pgfsys@useobject{currentmarker}{}%
\end{pgfscope}%
\begin{pgfscope}%
\pgfsys@transformshift{1.650647in}{4.038209in}%
\pgfsys@useobject{currentmarker}{}%
\end{pgfscope}%
\begin{pgfscope}%
\pgfsys@transformshift{1.627879in}{4.035215in}%
\pgfsys@useobject{currentmarker}{}%
\end{pgfscope}%
\begin{pgfscope}%
\pgfsys@transformshift{1.612386in}{4.037027in}%
\pgfsys@useobject{currentmarker}{}%
\end{pgfscope}%
\begin{pgfscope}%
\pgfsys@transformshift{1.593138in}{4.045653in}%
\pgfsys@useobject{currentmarker}{}%
\end{pgfscope}%
\begin{pgfscope}%
\pgfsys@transformshift{1.574594in}{4.068271in}%
\pgfsys@useobject{currentmarker}{}%
\end{pgfscope}%
\begin{pgfscope}%
\pgfsys@transformshift{1.555583in}{4.152970in}%
\pgfsys@useobject{currentmarker}{}%
\end{pgfscope}%
\begin{pgfscope}%
\pgfsys@transformshift{1.529762in}{4.280591in}%
\pgfsys@useobject{currentmarker}{}%
\end{pgfscope}%
\begin{pgfscope}%
\pgfsys@transformshift{1.516851in}{4.375645in}%
\pgfsys@useobject{currentmarker}{}%
\end{pgfscope}%
\begin{pgfscope}%
\pgfsys@transformshift{1.492675in}{4.418219in}%
\pgfsys@useobject{currentmarker}{}%
\end{pgfscope}%
\begin{pgfscope}%
\pgfsys@transformshift{1.475773in}{4.424686in}%
\pgfsys@useobject{currentmarker}{}%
\end{pgfscope}%
\begin{pgfscope}%
\pgfsys@transformshift{1.454883in}{4.290345in}%
\pgfsys@useobject{currentmarker}{}%
\end{pgfscope}%
\begin{pgfscope}%
\pgfsys@transformshift{1.437278in}{4.144829in}%
\pgfsys@useobject{currentmarker}{}%
\end{pgfscope}%
\begin{pgfscope}%
\pgfsys@transformshift{1.418030in}{4.072856in}%
\pgfsys@useobject{currentmarker}{}%
\end{pgfscope}%
\begin{pgfscope}%
\pgfsys@transformshift{1.400896in}{4.050163in}%
\pgfsys@useobject{currentmarker}{}%
\end{pgfscope}%
\begin{pgfscope}%
\pgfsys@transformshift{1.381412in}{4.040540in}%
\pgfsys@useobject{currentmarker}{}%
\end{pgfscope}%
\begin{pgfscope}%
\pgfsys@transformshift{1.358879in}{4.035682in}%
\pgfsys@useobject{currentmarker}{}%
\end{pgfscope}%
\begin{pgfscope}%
\pgfsys@transformshift{1.343387in}{4.035967in}%
\pgfsys@useobject{currentmarker}{}%
\end{pgfscope}%
\begin{pgfscope}%
\pgfsys@transformshift{1.322026in}{4.038452in}%
\pgfsys@useobject{currentmarker}{}%
\end{pgfscope}%
\begin{pgfscope}%
\pgfsys@transformshift{1.300901in}{4.044758in}%
\pgfsys@useobject{currentmarker}{}%
\end{pgfscope}%
\begin{pgfscope}%
\pgfsys@transformshift{1.286348in}{4.054322in}%
\pgfsys@useobject{currentmarker}{}%
\end{pgfscope}%
\begin{pgfscope}%
\pgfsys@transformshift{1.264517in}{4.124313in}%
\pgfsys@useobject{currentmarker}{}%
\end{pgfscope}%
\begin{pgfscope}%
\pgfsys@transformshift{1.245739in}{4.258687in}%
\pgfsys@useobject{currentmarker}{}%
\end{pgfscope}%
\begin{pgfscope}%
\pgfsys@transformshift{1.227431in}{4.366037in}%
\pgfsys@useobject{currentmarker}{}%
\end{pgfscope}%
\begin{pgfscope}%
\pgfsys@transformshift{1.203489in}{4.433922in}%
\pgfsys@useobject{currentmarker}{}%
\end{pgfscope}%
\begin{pgfscope}%
\pgfsys@transformshift{1.187527in}{4.399040in}%
\pgfsys@useobject{currentmarker}{}%
\end{pgfscope}%
\begin{pgfscope}%
\pgfsys@transformshift{1.169451in}{4.203421in}%
\pgfsys@useobject{currentmarker}{}%
\end{pgfscope}%
\begin{pgfscope}%
\pgfsys@transformshift{1.150203in}{4.105981in}%
\pgfsys@useobject{currentmarker}{}%
\end{pgfscope}%
\begin{pgfscope}%
\pgfsys@transformshift{1.131661in}{4.069027in}%
\pgfsys@useobject{currentmarker}{}%
\end{pgfscope}%
\begin{pgfscope}%
\pgfsys@transformshift{1.112882in}{4.050145in}%
\pgfsys@useobject{currentmarker}{}%
\end{pgfscope}%
\begin{pgfscope}%
\pgfsys@transformshift{1.091757in}{4.040146in}%
\pgfsys@useobject{currentmarker}{}%
\end{pgfscope}%
\begin{pgfscope}%
\pgfsys@transformshift{1.073682in}{4.036504in}%
\pgfsys@useobject{currentmarker}{}%
\end{pgfscope}%
\begin{pgfscope}%
\pgfsys@transformshift{1.054434in}{4.037386in}%
\pgfsys@useobject{currentmarker}{}%
\end{pgfscope}%
\begin{pgfscope}%
\pgfsys@transformshift{1.033075in}{4.042346in}%
\pgfsys@useobject{currentmarker}{}%
\end{pgfscope}%
\begin{pgfscope}%
\pgfsys@transformshift{1.015001in}{4.051749in}%
\pgfsys@useobject{currentmarker}{}%
\end{pgfscope}%
\begin{pgfscope}%
\pgfsys@transformshift{0.996691in}{4.078233in}%
\pgfsys@useobject{currentmarker}{}%
\end{pgfscope}%
\begin{pgfscope}%
\pgfsys@transformshift{0.975566in}{4.178598in}%
\pgfsys@useobject{currentmarker}{}%
\end{pgfscope}%
\begin{pgfscope}%
\pgfsys@transformshift{0.957021in}{4.320280in}%
\pgfsys@useobject{currentmarker}{}%
\end{pgfscope}%
\begin{pgfscope}%
\pgfsys@transformshift{0.938245in}{4.340463in}%
\pgfsys@useobject{currentmarker}{}%
\end{pgfscope}%
\begin{pgfscope}%
\pgfsys@transformshift{0.919466in}{4.433253in}%
\pgfsys@useobject{currentmarker}{}%
\end{pgfscope}%
\begin{pgfscope}%
\pgfsys@transformshift{0.902564in}{4.445452in}%
\pgfsys@useobject{currentmarker}{}%
\end{pgfscope}%
\begin{pgfscope}%
\pgfsys@transformshift{0.880265in}{4.342308in}%
\pgfsys@useobject{currentmarker}{}%
\end{pgfscope}%
\begin{pgfscope}%
\pgfsys@transformshift{0.861017in}{4.308967in}%
\pgfsys@useobject{currentmarker}{}%
\end{pgfscope}%
\begin{pgfscope}%
\pgfsys@transformshift{0.842239in}{4.429568in}%
\pgfsys@useobject{currentmarker}{}%
\end{pgfscope}%
\begin{pgfscope}%
\pgfsys@transformshift{0.823462in}{4.442028in}%
\pgfsys@useobject{currentmarker}{}%
\end{pgfscope}%
\begin{pgfscope}%
\pgfsys@transformshift{0.802335in}{4.282946in}%
\pgfsys@useobject{currentmarker}{}%
\end{pgfscope}%
\begin{pgfscope}%
\pgfsys@transformshift{0.785201in}{4.135127in}%
\pgfsys@useobject{currentmarker}{}%
\end{pgfscope}%
\begin{pgfscope}%
\pgfsys@transformshift{0.766188in}{4.077445in}%
\pgfsys@useobject{currentmarker}{}%
\end{pgfscope}%
\begin{pgfscope}%
\pgfsys@transformshift{0.746704in}{4.048561in}%
\pgfsys@useobject{currentmarker}{}%
\end{pgfscope}%
\begin{pgfscope}%
\pgfsys@transformshift{0.727221in}{4.038366in}%
\pgfsys@useobject{currentmarker}{}%
\end{pgfscope}%
\begin{pgfscope}%
\pgfsys@transformshift{0.707036in}{4.036394in}%
\pgfsys@useobject{currentmarker}{}%
\end{pgfscope}%
\begin{pgfscope}%
\pgfsys@transformshift{0.688492in}{4.040491in}%
\pgfsys@useobject{currentmarker}{}%
\end{pgfscope}%
\begin{pgfscope}%
\pgfsys@transformshift{0.667601in}{4.055947in}%
\pgfsys@useobject{currentmarker}{}%
\end{pgfscope}%
\begin{pgfscope}%
\pgfsys@transformshift{0.649525in}{4.094502in}%
\pgfsys@useobject{currentmarker}{}%
\end{pgfscope}%
\begin{pgfscope}%
\pgfsys@transformshift{0.650700in}{4.091918in}%
\pgfsys@useobject{currentmarker}{}%
\end{pgfscope}%
\begin{pgfscope}%
\pgfsys@transformshift{0.657039in}{4.069480in}%
\pgfsys@useobject{currentmarker}{}%
\end{pgfscope}%
\begin{pgfscope}%
\pgfsys@transformshift{0.676052in}{4.041918in}%
\pgfsys@useobject{currentmarker}{}%
\end{pgfscope}%
\begin{pgfscope}%
\pgfsys@transformshift{0.696003in}{4.036468in}%
\pgfsys@useobject{currentmarker}{}%
\end{pgfscope}%
\begin{pgfscope}%
\pgfsys@transformshift{0.714076in}{4.044387in}%
\pgfsys@useobject{currentmarker}{}%
\end{pgfscope}%
\begin{pgfscope}%
\pgfsys@transformshift{0.733324in}{4.075319in}%
\pgfsys@useobject{currentmarker}{}%
\end{pgfscope}%
\begin{pgfscope}%
\pgfsys@transformshift{0.752103in}{4.196417in}%
\pgfsys@useobject{currentmarker}{}%
\end{pgfscope}%
\begin{pgfscope}%
\pgfsys@transformshift{0.773464in}{4.431077in}%
\pgfsys@useobject{currentmarker}{}%
\end{pgfscope}%
\begin{pgfscope}%
\pgfsys@transformshift{0.790598in}{4.431778in}%
\pgfsys@useobject{currentmarker}{}%
\end{pgfscope}%
\begin{pgfscope}%
\pgfsys@transformshift{0.809612in}{4.268593in}%
\pgfsys@useobject{currentmarker}{}%
\end{pgfscope}%
\begin{pgfscope}%
\pgfsys@transformshift{0.828625in}{4.100729in}%
\pgfsys@useobject{currentmarker}{}%
\end{pgfscope}%
\begin{pgfscope}%
\pgfsys@transformshift{0.849047in}{4.047768in}%
\pgfsys@useobject{currentmarker}{}%
\end{pgfscope}%
\begin{pgfscope}%
\pgfsys@transformshift{0.868294in}{4.036854in}%
\pgfsys@useobject{currentmarker}{}%
\end{pgfscope}%
\begin{pgfscope}%
\pgfsys@transformshift{0.887776in}{4.038734in}%
\pgfsys@useobject{currentmarker}{}%
\end{pgfscope}%
\begin{pgfscope}%
\pgfsys@transformshift{0.907495in}{4.053861in}%
\pgfsys@useobject{currentmarker}{}%
\end{pgfscope}%
\begin{pgfscope}%
\pgfsys@transformshift{0.925803in}{4.110636in}%
\pgfsys@useobject{currentmarker}{}%
\end{pgfscope}%
\begin{pgfscope}%
\pgfsys@transformshift{0.945519in}{4.336861in}%
\pgfsys@useobject{currentmarker}{}%
\end{pgfscope}%
\begin{pgfscope}%
\pgfsys@transformshift{0.964064in}{4.440654in}%
\pgfsys@useobject{currentmarker}{}%
\end{pgfscope}%
\begin{pgfscope}%
\pgfsys@transformshift{0.983546in}{4.356132in}%
\pgfsys@useobject{currentmarker}{}%
\end{pgfscope}%
\begin{pgfscope}%
\pgfsys@transformshift{1.002559in}{4.144287in}%
\pgfsys@useobject{currentmarker}{}%
\end{pgfscope}%
\begin{pgfscope}%
\pgfsys@transformshift{1.021572in}{4.062045in}%
\pgfsys@useobject{currentmarker}{}%
\end{pgfscope}%
\begin{pgfscope}%
\pgfsys@transformshift{1.041994in}{4.039070in}%
\pgfsys@useobject{currentmarker}{}%
\end{pgfscope}%
\begin{pgfscope}%
\pgfsys@transformshift{1.061242in}{4.035774in}%
\pgfsys@useobject{currentmarker}{}%
\end{pgfscope}%
\begin{pgfscope}%
\pgfsys@transformshift{1.080255in}{4.042303in}%
\pgfsys@useobject{currentmarker}{}%
\end{pgfscope}%
\begin{pgfscope}%
\pgfsys@transformshift{1.098329in}{4.067610in}%
\pgfsys@useobject{currentmarker}{}%
\end{pgfscope}%
\begin{pgfscope}%
\pgfsys@transformshift{1.117342in}{4.168526in}%
\pgfsys@useobject{currentmarker}{}%
\end{pgfscope}%
\begin{pgfscope}%
\pgfsys@transformshift{1.135652in}{4.411415in}%
\pgfsys@useobject{currentmarker}{}%
\end{pgfscope}%
\begin{pgfscope}%
\pgfsys@transformshift{1.154900in}{4.418171in}%
\pgfsys@useobject{currentmarker}{}%
\end{pgfscope}%
\begin{pgfscope}%
\pgfsys@transformshift{1.173677in}{4.252454in}%
\pgfsys@useobject{currentmarker}{}%
\end{pgfscope}%
\begin{pgfscope}%
\pgfsys@transformshift{1.192690in}{4.091997in}%
\pgfsys@useobject{currentmarker}{}%
\end{pgfscope}%
\begin{pgfscope}%
\pgfsys@transformshift{1.211703in}{4.048742in}%
\pgfsys@useobject{currentmarker}{}%
\end{pgfscope}%
\begin{pgfscope}%
\pgfsys@transformshift{1.234237in}{4.035804in}%
\pgfsys@useobject{currentmarker}{}%
\end{pgfscope}%
\begin{pgfscope}%
\pgfsys@transformshift{1.252312in}{4.036708in}%
\pgfsys@useobject{currentmarker}{}%
\end{pgfscope}%
\begin{pgfscope}%
\pgfsys@transformshift{1.275080in}{4.049043in}%
\pgfsys@useobject{currentmarker}{}%
\end{pgfscope}%
\begin{pgfscope}%
\pgfsys@transformshift{1.290102in}{4.079857in}%
\pgfsys@useobject{currentmarker}{}%
\end{pgfscope}%
\begin{pgfscope}%
\pgfsys@transformshift{1.309586in}{4.182089in}%
\pgfsys@useobject{currentmarker}{}%
\end{pgfscope}%
\begin{pgfscope}%
\pgfsys@transformshift{1.327660in}{4.402476in}%
\pgfsys@useobject{currentmarker}{}%
\end{pgfscope}%
\begin{pgfscope}%
\pgfsys@transformshift{1.346908in}{4.408793in}%
\pgfsys@useobject{currentmarker}{}%
\end{pgfscope}%
\begin{pgfscope}%
\pgfsys@transformshift{1.369441in}{4.242691in}%
\pgfsys@useobject{currentmarker}{}%
\end{pgfscope}%
\begin{pgfscope}%
\pgfsys@transformshift{1.387280in}{4.105741in}%
\pgfsys@useobject{currentmarker}{}%
\end{pgfscope}%
\begin{pgfscope}%
\pgfsys@transformshift{1.406999in}{4.050069in}%
\pgfsys@useobject{currentmarker}{}%
\end{pgfscope}%
\begin{pgfscope}%
\pgfsys@transformshift{1.423898in}{4.039600in}%
\pgfsys@useobject{currentmarker}{}%
\end{pgfscope}%
\begin{pgfscope}%
\pgfsys@transformshift{1.444086in}{4.035159in}%
\pgfsys@useobject{currentmarker}{}%
\end{pgfscope}%
\begin{pgfscope}%
\pgfsys@transformshift{1.463333in}{4.038072in}%
\pgfsys@useobject{currentmarker}{}%
\end{pgfscope}%
\begin{pgfscope}%
\pgfsys@transformshift{1.482815in}{4.052826in}%
\pgfsys@useobject{currentmarker}{}%
\end{pgfscope}%
\begin{pgfscope}%
\pgfsys@transformshift{1.500889in}{4.107344in}%
\pgfsys@useobject{currentmarker}{}%
\end{pgfscope}%
\begin{pgfscope}%
\pgfsys@transformshift{1.520137in}{4.282812in}%
\pgfsys@useobject{currentmarker}{}%
\end{pgfscope}%
\begin{pgfscope}%
\pgfsys@transformshift{1.540793in}{4.422924in}%
\pgfsys@useobject{currentmarker}{}%
\end{pgfscope}%
\begin{pgfscope}%
\pgfsys@transformshift{1.560277in}{4.357479in}%
\pgfsys@useobject{currentmarker}{}%
\end{pgfscope}%
\begin{pgfscope}%
\pgfsys@transformshift{1.579525in}{4.165679in}%
\pgfsys@useobject{currentmarker}{}%
\end{pgfscope}%
\begin{pgfscope}%
\pgfsys@transformshift{1.598303in}{4.070372in}%
\pgfsys@useobject{currentmarker}{}%
\end{pgfscope}%
\begin{pgfscope}%
\pgfsys@transformshift{1.616377in}{4.042033in}%
\pgfsys@useobject{currentmarker}{}%
\end{pgfscope}%
\begin{pgfscope}%
\pgfsys@transformshift{1.640788in}{4.035172in}%
\pgfsys@useobject{currentmarker}{}%
\end{pgfscope}%
\begin{pgfscope}%
\pgfsys@transformshift{1.656281in}{4.035085in}%
\pgfsys@useobject{currentmarker}{}%
\end{pgfscope}%
\begin{pgfscope}%
\pgfsys@transformshift{1.675060in}{4.038267in}%
\pgfsys@useobject{currentmarker}{}%
\end{pgfscope}%
\begin{pgfscope}%
\pgfsys@transformshift{1.694307in}{4.051367in}%
\pgfsys@useobject{currentmarker}{}%
\end{pgfscope}%
\begin{pgfscope}%
\pgfsys@transformshift{1.712615in}{4.073104in}%
\pgfsys@useobject{currentmarker}{}%
\end{pgfscope}%
\begin{pgfscope}%
\pgfsys@transformshift{1.737028in}{4.224238in}%
\pgfsys@useobject{currentmarker}{}%
\end{pgfscope}%
\begin{pgfscope}%
\pgfsys@transformshift{1.752285in}{4.402707in}%
\pgfsys@useobject{currentmarker}{}%
\end{pgfscope}%
\begin{pgfscope}%
\pgfsys@transformshift{1.770124in}{4.394457in}%
\pgfsys@useobject{currentmarker}{}%
\end{pgfscope}%
\begin{pgfscope}%
\pgfsys@transformshift{1.788434in}{4.241683in}%
\pgfsys@useobject{currentmarker}{}%
\end{pgfscope}%
\begin{pgfscope}%
\pgfsys@transformshift{1.808854in}{4.161999in}%
\pgfsys@useobject{currentmarker}{}%
\end{pgfscope}%
\begin{pgfscope}%
\pgfsys@transformshift{1.830684in}{4.065534in}%
\pgfsys@useobject{currentmarker}{}%
\end{pgfscope}%
\begin{pgfscope}%
\pgfsys@transformshift{1.848289in}{4.043305in}%
\pgfsys@useobject{currentmarker}{}%
\end{pgfscope}%
\begin{pgfscope}%
\pgfsys@transformshift{1.867302in}{4.035619in}%
\pgfsys@useobject{currentmarker}{}%
\end{pgfscope}%
\begin{pgfscope}%
\pgfsys@transformshift{1.888193in}{4.035137in}%
\pgfsys@useobject{currentmarker}{}%
\end{pgfscope}%
\begin{pgfscope}%
\pgfsys@transformshift{1.906503in}{4.039328in}%
\pgfsys@useobject{currentmarker}{}%
\end{pgfscope}%
\begin{pgfscope}%
\pgfsys@transformshift{1.926219in}{4.057166in}%
\pgfsys@useobject{currentmarker}{}%
\end{pgfscope}%
\begin{pgfscope}%
\pgfsys@transformshift{1.947110in}{4.115660in}%
\pgfsys@useobject{currentmarker}{}%
\end{pgfscope}%
\begin{pgfscope}%
\pgfsys@transformshift{1.963777in}{4.257827in}%
\pgfsys@useobject{currentmarker}{}%
\end{pgfscope}%
\begin{pgfscope}%
\pgfsys@transformshift{1.982319in}{4.414987in}%
\pgfsys@useobject{currentmarker}{}%
\end{pgfscope}%
\begin{pgfscope}%
\pgfsys@transformshift{2.002741in}{4.339014in}%
\pgfsys@useobject{currentmarker}{}%
\end{pgfscope}%
\begin{pgfscope}%
\pgfsys@transformshift{2.019643in}{4.183984in}%
\pgfsys@useobject{currentmarker}{}%
\end{pgfscope}%
\begin{pgfscope}%
\pgfsys@transformshift{2.040533in}{4.069275in}%
\pgfsys@useobject{currentmarker}{}%
\end{pgfscope}%
\begin{pgfscope}%
\pgfsys@transformshift{2.060955in}{4.042853in}%
\pgfsys@useobject{currentmarker}{}%
\end{pgfscope}%
\begin{pgfscope}%
\pgfsys@transformshift{2.078560in}{4.035820in}%
\pgfsys@useobject{currentmarker}{}%
\end{pgfscope}%
\begin{pgfscope}%
\pgfsys@transformshift{2.100154in}{4.035045in}%
\pgfsys@useobject{currentmarker}{}%
\end{pgfscope}%
\begin{pgfscope}%
\pgfsys@transformshift{2.117055in}{4.037348in}%
\pgfsys@useobject{currentmarker}{}%
\end{pgfscope}%
\begin{pgfscope}%
\pgfsys@transformshift{2.134660in}{4.037622in}%
\pgfsys@useobject{currentmarker}{}%
\end{pgfscope}%
\begin{pgfscope}%
\pgfsys@transformshift{2.156724in}{4.051977in}%
\pgfsys@useobject{currentmarker}{}%
\end{pgfscope}%
\begin{pgfscope}%
\pgfsys@transformshift{2.174329in}{4.103027in}%
\pgfsys@useobject{currentmarker}{}%
\end{pgfscope}%
\begin{pgfscope}%
\pgfsys@transformshift{2.192169in}{4.255826in}%
\pgfsys@useobject{currentmarker}{}%
\end{pgfscope}%
\begin{pgfscope}%
\pgfsys@transformshift{2.212354in}{4.408864in}%
\pgfsys@useobject{currentmarker}{}%
\end{pgfscope}%
\begin{pgfscope}%
\pgfsys@transformshift{2.233950in}{4.374568in}%
\pgfsys@useobject{currentmarker}{}%
\end{pgfscope}%
\begin{pgfscope}%
\pgfsys@transformshift{2.252258in}{4.281083in}%
\pgfsys@useobject{currentmarker}{}%
\end{pgfscope}%
\begin{pgfscope}%
\pgfsys@transformshift{2.270333in}{4.124122in}%
\pgfsys@useobject{currentmarker}{}%
\end{pgfscope}%
\begin{pgfscope}%
\pgfsys@transformshift{2.290755in}{4.055310in}%
\pgfsys@useobject{currentmarker}{}%
\end{pgfscope}%
\begin{pgfscope}%
\pgfsys@transformshift{2.309063in}{4.044561in}%
\pgfsys@useobject{currentmarker}{}%
\end{pgfscope}%
\begin{pgfscope}%
\pgfsys@transformshift{2.329954in}{4.036486in}%
\pgfsys@useobject{currentmarker}{}%
\end{pgfscope}%
\begin{pgfscope}%
\pgfsys@transformshift{2.347558in}{4.034600in}%
\pgfsys@useobject{currentmarker}{}%
\end{pgfscope}%
\begin{pgfscope}%
\pgfsys@transformshift{2.366806in}{4.036347in}%
\pgfsys@useobject{currentmarker}{}%
\end{pgfscope}%
\begin{pgfscope}%
\pgfsys@transformshift{2.386525in}{4.046691in}%
\pgfsys@useobject{currentmarker}{}%
\end{pgfscope}%
\begin{pgfscope}%
\pgfsys@transformshift{2.404364in}{4.081308in}%
\pgfsys@useobject{currentmarker}{}%
\end{pgfscope}%
\begin{pgfscope}%
\pgfsys@transformshift{2.425254in}{4.221121in}%
\pgfsys@useobject{currentmarker}{}%
\end{pgfscope}%
\begin{pgfscope}%
\pgfsys@transformshift{2.443094in}{4.347042in}%
\pgfsys@useobject{currentmarker}{}%
\end{pgfscope}%
\begin{pgfscope}%
\pgfsys@transformshift{2.460230in}{4.092958in}%
\pgfsys@useobject{currentmarker}{}%
\end{pgfscope}%
\begin{pgfscope}%
\pgfsys@transformshift{2.482058in}{4.250152in}%
\pgfsys@useobject{currentmarker}{}%
\end{pgfscope}%
\begin{pgfscope}%
\pgfsys@transformshift{2.503185in}{4.412934in}%
\pgfsys@useobject{currentmarker}{}%
\end{pgfscope}%
\begin{pgfscope}%
\pgfsys@transformshift{2.519381in}{4.360324in}%
\pgfsys@useobject{currentmarker}{}%
\end{pgfscope}%
\begin{pgfscope}%
\pgfsys@transformshift{2.540272in}{4.190762in}%
\pgfsys@useobject{currentmarker}{}%
\end{pgfscope}%
\begin{pgfscope}%
\pgfsys@transformshift{2.557171in}{4.080612in}%
\pgfsys@useobject{currentmarker}{}%
\end{pgfscope}%
\begin{pgfscope}%
\pgfsys@transformshift{2.578064in}{4.043655in}%
\pgfsys@useobject{currentmarker}{}%
\end{pgfscope}%
\begin{pgfscope}%
\pgfsys@transformshift{2.595198in}{4.037030in}%
\pgfsys@useobject{currentmarker}{}%
\end{pgfscope}%
\begin{pgfscope}%
\pgfsys@transformshift{2.617028in}{4.034834in}%
\pgfsys@useobject{currentmarker}{}%
\end{pgfscope}%
\begin{pgfscope}%
\pgfsys@transformshift{2.634633in}{4.038333in}%
\pgfsys@useobject{currentmarker}{}%
\end{pgfscope}%
\begin{pgfscope}%
\pgfsys@transformshift{2.657401in}{4.052225in}%
\pgfsys@useobject{currentmarker}{}%
\end{pgfscope}%
\begin{pgfscope}%
\pgfsys@transformshift{2.675711in}{4.092257in}%
\pgfsys@useobject{currentmarker}{}%
\end{pgfscope}%
\begin{pgfscope}%
\pgfsys@transformshift{2.692610in}{4.201445in}%
\pgfsys@useobject{currentmarker}{}%
\end{pgfscope}%
\begin{pgfscope}%
\pgfsys@transformshift{2.713737in}{4.378333in}%
\pgfsys@useobject{currentmarker}{}%
\end{pgfscope}%
\begin{pgfscope}%
\pgfsys@transformshift{2.731811in}{4.409261in}%
\pgfsys@useobject{currentmarker}{}%
\end{pgfscope}%
\begin{pgfscope}%
\pgfsys@transformshift{2.752232in}{4.340800in}%
\pgfsys@useobject{currentmarker}{}%
\end{pgfscope}%
\begin{pgfscope}%
\pgfsys@transformshift{2.770775in}{4.178768in}%
\pgfsys@useobject{currentmarker}{}%
\end{pgfscope}%
\begin{pgfscope}%
\pgfsys@transformshift{2.788145in}{4.074895in}%
\pgfsys@useobject{currentmarker}{}%
\end{pgfscope}%
\begin{pgfscope}%
\pgfsys@transformshift{2.809741in}{4.044631in}%
\pgfsys@useobject{currentmarker}{}%
\end{pgfscope}%
\begin{pgfscope}%
\pgfsys@transformshift{2.827815in}{4.037081in}%
\pgfsys@useobject{currentmarker}{}%
\end{pgfscope}%
\begin{pgfscope}%
\pgfsys@transformshift{2.848471in}{4.034747in}%
\pgfsys@useobject{currentmarker}{}%
\end{pgfscope}%
\begin{pgfscope}%
\pgfsys@transformshift{2.866781in}{4.037700in}%
\pgfsys@useobject{currentmarker}{}%
\end{pgfscope}%
\begin{pgfscope}%
\pgfsys@transformshift{2.884854in}{4.040043in}%
\pgfsys@useobject{currentmarker}{}%
\end{pgfscope}%
\begin{pgfscope}%
\pgfsys@transformshift{2.906214in}{4.061169in}%
\pgfsys@useobject{currentmarker}{}%
\end{pgfscope}%
\begin{pgfscope}%
\pgfsys@transformshift{2.923584in}{4.106421in}%
\pgfsys@useobject{currentmarker}{}%
\end{pgfscope}%
\begin{pgfscope}%
\pgfsys@transformshift{2.945883in}{4.288137in}%
\pgfsys@useobject{currentmarker}{}%
\end{pgfscope}%
\begin{pgfscope}%
\pgfsys@transformshift{2.961845in}{4.406187in}%
\pgfsys@useobject{currentmarker}{}%
\end{pgfscope}%
\begin{pgfscope}%
\pgfsys@transformshift{2.983441in}{4.369346in}%
\pgfsys@useobject{currentmarker}{}%
\end{pgfscope}%
\begin{pgfscope}%
\pgfsys@transformshift{3.000811in}{4.269557in}%
\pgfsys@useobject{currentmarker}{}%
\end{pgfscope}%
\begin{pgfscope}%
\pgfsys@transformshift{3.019354in}{4.120187in}%
\pgfsys@useobject{currentmarker}{}%
\end{pgfscope}%
\begin{pgfscope}%
\pgfsys@transformshift{3.041184in}{4.056678in}%
\pgfsys@useobject{currentmarker}{}%
\end{pgfscope}%
\begin{pgfscope}%
\pgfsys@transformshift{3.058554in}{4.040961in}%
\pgfsys@useobject{currentmarker}{}%
\end{pgfscope}%
\begin{pgfscope}%
\pgfsys@transformshift{3.076628in}{4.035611in}%
\pgfsys@useobject{currentmarker}{}%
\end{pgfscope}%
\begin{pgfscope}%
\pgfsys@transformshift{3.100570in}{4.035763in}%
\pgfsys@useobject{currentmarker}{}%
\end{pgfscope}%
\begin{pgfscope}%
\pgfsys@transformshift{3.116297in}{4.040103in}%
\pgfsys@useobject{currentmarker}{}%
\end{pgfscope}%
\begin{pgfscope}%
\pgfsys@transformshift{3.134371in}{4.047784in}%
\pgfsys@useobject{currentmarker}{}%
\end{pgfscope}%
\begin{pgfscope}%
\pgfsys@transformshift{3.155262in}{4.080534in}%
\pgfsys@useobject{currentmarker}{}%
\end{pgfscope}%
\begin{pgfscope}%
\pgfsys@transformshift{3.175918in}{4.183570in}%
\pgfsys@useobject{currentmarker}{}%
\end{pgfscope}%
\begin{pgfscope}%
\pgfsys@transformshift{3.193759in}{4.356160in}%
\pgfsys@useobject{currentmarker}{}%
\end{pgfscope}%
\begin{pgfscope}%
\pgfsys@transformshift{3.212067in}{4.415478in}%
\pgfsys@useobject{currentmarker}{}%
\end{pgfscope}%
\begin{pgfscope}%
\pgfsys@transformshift{3.232254in}{4.368759in}%
\pgfsys@useobject{currentmarker}{}%
\end{pgfscope}%
\begin{pgfscope}%
\pgfsys@transformshift{3.250094in}{4.241450in}%
\pgfsys@useobject{currentmarker}{}%
\end{pgfscope}%
\begin{pgfscope}%
\pgfsys@transformshift{3.267933in}{4.117624in}%
\pgfsys@useobject{currentmarker}{}%
\end{pgfscope}%
\begin{pgfscope}%
\pgfsys@transformshift{3.289058in}{4.060274in}%
\pgfsys@useobject{currentmarker}{}%
\end{pgfscope}%
\begin{pgfscope}%
\pgfsys@transformshift{3.307368in}{4.042239in}%
\pgfsys@useobject{currentmarker}{}%
\end{pgfscope}%
\begin{pgfscope}%
\pgfsys@transformshift{3.327084in}{4.036974in}%
\pgfsys@useobject{currentmarker}{}%
\end{pgfscope}%
\begin{pgfscope}%
\pgfsys@transformshift{3.346801in}{4.034906in}%
\pgfsys@useobject{currentmarker}{}%
\end{pgfscope}%
\begin{pgfscope}%
\pgfsys@transformshift{3.365111in}{4.037123in}%
\pgfsys@useobject{currentmarker}{}%
\end{pgfscope}%
\begin{pgfscope}%
\pgfsys@transformshift{3.385298in}{4.045357in}%
\pgfsys@useobject{currentmarker}{}%
\end{pgfscope}%
\begin{pgfscope}%
\pgfsys@transformshift{3.403606in}{4.052317in}%
\pgfsys@useobject{currentmarker}{}%
\end{pgfscope}%
\begin{pgfscope}%
\pgfsys@transformshift{3.421211in}{4.094781in}%
\pgfsys@useobject{currentmarker}{}%
\end{pgfscope}%
\begin{pgfscope}%
\pgfsys@transformshift{3.443276in}{4.209516in}%
\pgfsys@useobject{currentmarker}{}%
\end{pgfscope}%
\begin{pgfscope}%
\pgfsys@transformshift{3.462758in}{4.398261in}%
\pgfsys@useobject{currentmarker}{}%
\end{pgfscope}%
\begin{pgfscope}%
\pgfsys@transformshift{3.479894in}{4.423956in}%
\pgfsys@useobject{currentmarker}{}%
\end{pgfscope}%
\begin{pgfscope}%
\pgfsys@transformshift{3.501724in}{4.346017in}%
\pgfsys@useobject{currentmarker}{}%
\end{pgfscope}%
\begin{pgfscope}%
\pgfsys@transformshift{3.520501in}{4.211908in}%
\pgfsys@useobject{currentmarker}{}%
\end{pgfscope}%
\begin{pgfscope}%
\pgfsys@transformshift{3.537402in}{4.110767in}%
\pgfsys@useobject{currentmarker}{}%
\end{pgfscope}%
\begin{pgfscope}%
\pgfsys@transformshift{3.557353in}{4.058494in}%
\pgfsys@useobject{currentmarker}{}%
\end{pgfscope}%
\begin{pgfscope}%
\pgfsys@transformshift{3.578246in}{4.041748in}%
\pgfsys@useobject{currentmarker}{}%
\end{pgfscope}%
\begin{pgfscope}%
\pgfsys@transformshift{3.596554in}{4.036765in}%
\pgfsys@useobject{currentmarker}{}%
\end{pgfscope}%
\begin{pgfscope}%
\pgfsys@transformshift{3.615098in}{4.035446in}%
\pgfsys@useobject{currentmarker}{}%
\end{pgfscope}%
\begin{pgfscope}%
\pgfsys@transformshift{3.635518in}{4.037017in}%
\pgfsys@useobject{currentmarker}{}%
\end{pgfscope}%
\begin{pgfscope}%
\pgfsys@transformshift{3.653593in}{4.046281in}%
\pgfsys@useobject{currentmarker}{}%
\end{pgfscope}%
\begin{pgfscope}%
\pgfsys@transformshift{3.671667in}{4.059844in}%
\pgfsys@useobject{currentmarker}{}%
\end{pgfscope}%
\begin{pgfscope}%
\pgfsys@transformshift{3.690446in}{4.086009in}%
\pgfsys@useobject{currentmarker}{}%
\end{pgfscope}%
\begin{pgfscope}%
\pgfsys@transformshift{3.711337in}{4.186055in}%
\pgfsys@useobject{currentmarker}{}%
\end{pgfscope}%
\begin{pgfscope}%
\pgfsys@transformshift{3.732227in}{4.386649in}%
\pgfsys@useobject{currentmarker}{}%
\end{pgfscope}%
\begin{pgfscope}%
\pgfsys@transformshift{3.751006in}{4.432998in}%
\pgfsys@useobject{currentmarker}{}%
\end{pgfscope}%
\begin{pgfscope}%
\pgfsys@transformshift{3.769314in}{4.433916in}%
\pgfsys@useobject{currentmarker}{}%
\end{pgfscope}%
\begin{pgfscope}%
\pgfsys@transformshift{3.789736in}{4.377326in}%
\pgfsys@useobject{currentmarker}{}%
\end{pgfscope}%
\begin{pgfscope}%
\pgfsys@transformshift{3.806872in}{4.247697in}%
\pgfsys@useobject{currentmarker}{}%
\end{pgfscope}%
\begin{pgfscope}%
\pgfsys@transformshift{3.827762in}{4.104851in}%
\pgfsys@useobject{currentmarker}{}%
\end{pgfscope}%
\begin{pgfscope}%
\pgfsys@transformshift{3.846070in}{4.071313in}%
\pgfsys@useobject{currentmarker}{}%
\end{pgfscope}%
\begin{pgfscope}%
\pgfsys@transformshift{3.864146in}{4.047541in}%
\pgfsys@useobject{currentmarker}{}%
\end{pgfscope}%
\begin{pgfscope}%
\pgfsys@transformshift{3.885271in}{4.037330in}%
\pgfsys@useobject{currentmarker}{}%
\end{pgfscope}%
\begin{pgfscope}%
\pgfsys@transformshift{3.901936in}{4.035723in}%
\pgfsys@useobject{currentmarker}{}%
\end{pgfscope}%
\begin{pgfscope}%
\pgfsys@transformshift{3.922592in}{4.038352in}%
\pgfsys@useobject{currentmarker}{}%
\end{pgfscope}%
\begin{pgfscope}%
\pgfsys@transformshift{3.940902in}{4.048260in}%
\pgfsys@useobject{currentmarker}{}%
\end{pgfscope}%
\begin{pgfscope}%
\pgfsys@transformshift{3.960853in}{4.062593in}%
\pgfsys@useobject{currentmarker}{}%
\end{pgfscope}%
\begin{pgfscope}%
\pgfsys@transformshift{3.981980in}{4.110586in}%
\pgfsys@useobject{currentmarker}{}%
\end{pgfscope}%
\begin{pgfscope}%
\pgfsys@transformshift{3.999819in}{4.225778in}%
\pgfsys@useobject{currentmarker}{}%
\end{pgfscope}%
\begin{pgfscope}%
\pgfsys@transformshift{4.017893in}{4.332990in}%
\pgfsys@useobject{currentmarker}{}%
\end{pgfscope}%
\begin{pgfscope}%
\pgfsys@transformshift{4.041600in}{4.440003in}%
\pgfsys@useobject{currentmarker}{}%
\end{pgfscope}%
\begin{pgfscope}%
\pgfsys@transformshift{4.056857in}{4.438553in}%
\pgfsys@useobject{currentmarker}{}%
\end{pgfscope}%
\begin{pgfscope}%
\pgfsys@transformshift{4.077750in}{4.356898in}%
\pgfsys@useobject{currentmarker}{}%
\end{pgfscope}%
\begin{pgfscope}%
\pgfsys@transformshift{4.095589in}{4.273629in}%
\pgfsys@useobject{currentmarker}{}%
\end{pgfscope}%
\begin{pgfscope}%
\pgfsys@transformshift{4.116011in}{4.144124in}%
\pgfsys@useobject{currentmarker}{}%
\end{pgfscope}%
\begin{pgfscope}%
\pgfsys@transformshift{4.134787in}{4.066592in}%
\pgfsys@useobject{currentmarker}{}%
\end{pgfscope}%
\begin{pgfscope}%
\pgfsys@transformshift{4.153332in}{4.046680in}%
\pgfsys@useobject{currentmarker}{}%
\end{pgfscope}%
\begin{pgfscope}%
\pgfsys@transformshift{4.171171in}{4.042708in}%
\pgfsys@useobject{currentmarker}{}%
\end{pgfscope}%
\begin{pgfscope}%
\pgfsys@transformshift{4.192062in}{4.036243in}%
\pgfsys@useobject{currentmarker}{}%
\end{pgfscope}%
\begin{pgfscope}%
\pgfsys@transformshift{4.211544in}{4.037423in}%
\pgfsys@useobject{currentmarker}{}%
\end{pgfscope}%
\begin{pgfscope}%
\pgfsys@transformshift{4.228211in}{4.041456in}%
\pgfsys@useobject{currentmarker}{}%
\end{pgfscope}%
\begin{pgfscope}%
\pgfsys@transformshift{4.250041in}{4.059036in}%
\pgfsys@useobject{currentmarker}{}%
\end{pgfscope}%
\begin{pgfscope}%
\pgfsys@transformshift{4.270461in}{4.109723in}%
\pgfsys@useobject{currentmarker}{}%
\end{pgfscope}%
\begin{pgfscope}%
\pgfsys@transformshift{4.289005in}{4.227131in}%
\pgfsys@useobject{currentmarker}{}%
\end{pgfscope}%
\begin{pgfscope}%
\pgfsys@transformshift{4.306376in}{4.348878in}%
\pgfsys@useobject{currentmarker}{}%
\end{pgfscope}%
\begin{pgfscope}%
\pgfsys@transformshift{4.324918in}{4.453839in}%
\pgfsys@useobject{currentmarker}{}%
\end{pgfscope}%
\begin{pgfscope}%
\pgfsys@transformshift{4.346748in}{4.450044in}%
\pgfsys@useobject{currentmarker}{}%
\end{pgfscope}%
\begin{pgfscope}%
\pgfsys@transformshift{4.364588in}{4.431794in}%
\pgfsys@useobject{currentmarker}{}%
\end{pgfscope}%
\begin{pgfscope}%
\pgfsys@transformshift{4.385714in}{4.303876in}%
\pgfsys@useobject{currentmarker}{}%
\end{pgfscope}%
\begin{pgfscope}%
\pgfsys@transformshift{4.400032in}{4.146595in}%
\pgfsys@useobject{currentmarker}{}%
\end{pgfscope}%
\begin{pgfscope}%
\pgfsys@transformshift{4.421862in}{4.088890in}%
\pgfsys@useobject{currentmarker}{}%
\end{pgfscope}%
\begin{pgfscope}%
\pgfsys@transformshift{4.442049in}{4.051231in}%
\pgfsys@useobject{currentmarker}{}%
\end{pgfscope}%
\begin{pgfscope}%
\pgfsys@transformshift{4.461062in}{4.041362in}%
\pgfsys@useobject{currentmarker}{}%
\end{pgfscope}%
\begin{pgfscope}%
\pgfsys@transformshift{4.479841in}{4.037847in}%
\pgfsys@useobject{currentmarker}{}%
\end{pgfscope}%
\begin{pgfscope}%
\pgfsys@transformshift{4.480544in}{4.037547in}%
\pgfsys@useobject{currentmarker}{}%
\end{pgfscope}%
\begin{pgfscope}%
\pgfsys@transformshift{4.475379in}{4.417174in}%
\pgfsys@useobject{currentmarker}{}%
\end{pgfscope}%
\begin{pgfscope}%
\pgfsys@transformshift{4.453786in}{4.135950in}%
\pgfsys@useobject{currentmarker}{}%
\end{pgfscope}%
\begin{pgfscope}%
\pgfsys@transformshift{4.436884in}{4.060644in}%
\pgfsys@useobject{currentmarker}{}%
\end{pgfscope}%
\begin{pgfscope}%
\pgfsys@transformshift{4.415993in}{4.039124in}%
\pgfsys@useobject{currentmarker}{}%
\end{pgfscope}%
\begin{pgfscope}%
\pgfsys@transformshift{4.400032in}{4.038472in}%
\pgfsys@useobject{currentmarker}{}%
\end{pgfscope}%
\begin{pgfscope}%
\pgfsys@transformshift{4.378203in}{4.048307in}%
\pgfsys@useobject{currentmarker}{}%
\end{pgfscope}%
\begin{pgfscope}%
\pgfsys@transformshift{4.358485in}{4.103543in}%
\pgfsys@useobject{currentmarker}{}%
\end{pgfscope}%
\begin{pgfscope}%
\pgfsys@transformshift{4.342289in}{4.261987in}%
\pgfsys@useobject{currentmarker}{}%
\end{pgfscope}%
\begin{pgfscope}%
\pgfsys@transformshift{4.320458in}{4.437569in}%
\pgfsys@useobject{currentmarker}{}%
\end{pgfscope}%
\begin{pgfscope}%
\pgfsys@transformshift{4.302619in}{4.433837in}%
\pgfsys@useobject{currentmarker}{}%
\end{pgfscope}%
\begin{pgfscope}%
\pgfsys@transformshift{4.281025in}{4.171292in}%
\pgfsys@useobject{currentmarker}{}%
\end{pgfscope}%
\begin{pgfscope}%
\pgfsys@transformshift{4.261778in}{4.065039in}%
\pgfsys@useobject{currentmarker}{}%
\end{pgfscope}%
\begin{pgfscope}%
\pgfsys@transformshift{4.245110in}{4.041279in}%
\pgfsys@useobject{currentmarker}{}%
\end{pgfscope}%
\begin{pgfscope}%
\pgfsys@transformshift{4.223751in}{4.036187in}%
\pgfsys@useobject{currentmarker}{}%
\end{pgfscope}%
\begin{pgfscope}%
\pgfsys@transformshift{4.207084in}{4.041638in}%
\pgfsys@useobject{currentmarker}{}%
\end{pgfscope}%
\begin{pgfscope}%
\pgfsys@transformshift{4.186664in}{4.069746in}%
\pgfsys@useobject{currentmarker}{}%
\end{pgfscope}%
\begin{pgfscope}%
\pgfsys@transformshift{4.163660in}{4.240392in}%
\pgfsys@useobject{currentmarker}{}%
\end{pgfscope}%
\begin{pgfscope}%
\pgfsys@transformshift{4.145350in}{4.405695in}%
\pgfsys@useobject{currentmarker}{}%
\end{pgfscope}%
\begin{pgfscope}%
\pgfsys@transformshift{4.127276in}{4.440085in}%
\pgfsys@useobject{currentmarker}{}%
\end{pgfscope}%
\begin{pgfscope}%
\pgfsys@transformshift{4.107560in}{4.232260in}%
\pgfsys@useobject{currentmarker}{}%
\end{pgfscope}%
\begin{pgfscope}%
\pgfsys@transformshift{4.089486in}{4.087374in}%
\pgfsys@useobject{currentmarker}{}%
\end{pgfscope}%
\begin{pgfscope}%
\pgfsys@transformshift{4.072585in}{4.047601in}%
\pgfsys@useobject{currentmarker}{}%
\end{pgfscope}%
\begin{pgfscope}%
\pgfsys@transformshift{4.051460in}{4.036329in}%
\pgfsys@useobject{currentmarker}{}%
\end{pgfscope}%
\begin{pgfscope}%
\pgfsys@transformshift{4.033620in}{4.036795in}%
\pgfsys@useobject{currentmarker}{}%
\end{pgfscope}%
\begin{pgfscope}%
\pgfsys@transformshift{4.010616in}{4.048580in}%
\pgfsys@useobject{currentmarker}{}%
\end{pgfscope}%
\begin{pgfscope}%
\pgfsys@transformshift{3.992777in}{4.093445in}%
\pgfsys@useobject{currentmarker}{}%
\end{pgfscope}%
\begin{pgfscope}%
\pgfsys@transformshift{3.973529in}{4.229210in}%
\pgfsys@useobject{currentmarker}{}%
\end{pgfscope}%
\begin{pgfscope}%
\pgfsys@transformshift{3.955690in}{4.404767in}%
\pgfsys@useobject{currentmarker}{}%
\end{pgfscope}%
\begin{pgfscope}%
\pgfsys@transformshift{3.937380in}{4.429897in}%
\pgfsys@useobject{currentmarker}{}%
\end{pgfscope}%
\begin{pgfscope}%
\pgfsys@transformshift{3.916490in}{4.246574in}%
\pgfsys@useobject{currentmarker}{}%
\end{pgfscope}%
\begin{pgfscope}%
\pgfsys@transformshift{3.895833in}{4.082236in}%
\pgfsys@useobject{currentmarker}{}%
\end{pgfscope}%
\begin{pgfscope}%
\pgfsys@transformshift{3.878229in}{4.045964in}%
\pgfsys@useobject{currentmarker}{}%
\end{pgfscope}%
\begin{pgfscope}%
\pgfsys@transformshift{3.860389in}{4.036668in}%
\pgfsys@useobject{currentmarker}{}%
\end{pgfscope}%
\begin{pgfscope}%
\pgfsys@transformshift{3.840202in}{4.036131in}%
\pgfsys@useobject{currentmarker}{}%
\end{pgfscope}%
\begin{pgfscope}%
\pgfsys@transformshift{3.822128in}{4.039504in}%
\pgfsys@useobject{currentmarker}{}%
\end{pgfscope}%
\begin{pgfscope}%
\pgfsys@transformshift{3.801941in}{4.061336in}%
\pgfsys@useobject{currentmarker}{}%
\end{pgfscope}%
\begin{pgfscope}%
\pgfsys@transformshift{3.781519in}{4.166799in}%
\pgfsys@useobject{currentmarker}{}%
\end{pgfscope}%
\begin{pgfscope}%
\pgfsys@transformshift{3.762977in}{4.334894in}%
\pgfsys@useobject{currentmarker}{}%
\end{pgfscope}%
\begin{pgfscope}%
\pgfsys@transformshift{3.744669in}{4.426494in}%
\pgfsys@useobject{currentmarker}{}%
\end{pgfscope}%
\begin{pgfscope}%
\pgfsys@transformshift{3.723542in}{4.403886in}%
\pgfsys@useobject{currentmarker}{}%
\end{pgfscope}%
\begin{pgfscope}%
\pgfsys@transformshift{3.705468in}{4.235501in}%
\pgfsys@useobject{currentmarker}{}%
\end{pgfscope}%
\begin{pgfscope}%
\pgfsys@transformshift{3.688332in}{4.112662in}%
\pgfsys@useobject{currentmarker}{}%
\end{pgfscope}%
\begin{pgfscope}%
\pgfsys@transformshift{3.666738in}{4.050842in}%
\pgfsys@useobject{currentmarker}{}%
\end{pgfscope}%
\begin{pgfscope}%
\pgfsys@transformshift{3.649134in}{4.038330in}%
\pgfsys@useobject{currentmarker}{}%
\end{pgfscope}%
\begin{pgfscope}%
\pgfsys@transformshift{3.628478in}{4.035291in}%
\pgfsys@useobject{currentmarker}{}%
\end{pgfscope}%
\begin{pgfscope}%
\pgfsys@transformshift{3.607351in}{4.038228in}%
\pgfsys@useobject{currentmarker}{}%
\end{pgfscope}%
\begin{pgfscope}%
\pgfsys@transformshift{3.589982in}{4.050854in}%
\pgfsys@useobject{currentmarker}{}%
\end{pgfscope}%
\begin{pgfscope}%
\pgfsys@transformshift{3.573081in}{4.078243in}%
\pgfsys@useobject{currentmarker}{}%
\end{pgfscope}%
\begin{pgfscope}%
\pgfsys@transformshift{3.551721in}{4.214536in}%
\pgfsys@useobject{currentmarker}{}%
\end{pgfscope}%
\begin{pgfscope}%
\pgfsys@transformshift{3.530594in}{4.270529in}%
\pgfsys@useobject{currentmarker}{}%
\end{pgfscope}%
\begin{pgfscope}%
\pgfsys@transformshift{3.515103in}{4.389992in}%
\pgfsys@useobject{currentmarker}{}%
\end{pgfscope}%
\begin{pgfscope}%
\pgfsys@transformshift{3.494916in}{4.416973in}%
\pgfsys@useobject{currentmarker}{}%
\end{pgfscope}%
\begin{pgfscope}%
\pgfsys@transformshift{3.477546in}{4.272243in}%
\pgfsys@useobject{currentmarker}{}%
\end{pgfscope}%
\begin{pgfscope}%
\pgfsys@transformshift{3.454778in}{4.117186in}%
\pgfsys@useobject{currentmarker}{}%
\end{pgfscope}%
\begin{pgfscope}%
\pgfsys@transformshift{3.437173in}{4.059320in}%
\pgfsys@useobject{currentmarker}{}%
\end{pgfscope}%
\begin{pgfscope}%
\pgfsys@transformshift{3.418863in}{4.127175in}%
\pgfsys@useobject{currentmarker}{}%
\end{pgfscope}%
\begin{pgfscope}%
\pgfsys@transformshift{3.398207in}{4.294422in}%
\pgfsys@useobject{currentmarker}{}%
\end{pgfscope}%
\begin{pgfscope}%
\pgfsys@transformshift{3.378725in}{4.416500in}%
\pgfsys@useobject{currentmarker}{}%
\end{pgfscope}%
\begin{pgfscope}%
\pgfsys@transformshift{3.361120in}{4.329456in}%
\pgfsys@useobject{currentmarker}{}%
\end{pgfscope}%
\begin{pgfscope}%
\pgfsys@transformshift{3.341638in}{4.142952in}%
\pgfsys@useobject{currentmarker}{}%
\end{pgfscope}%
\begin{pgfscope}%
\pgfsys@transformshift{3.319104in}{4.058559in}%
\pgfsys@useobject{currentmarker}{}%
\end{pgfscope}%
\begin{pgfscope}%
\pgfsys@transformshift{3.300794in}{4.040783in}%
\pgfsys@useobject{currentmarker}{}%
\end{pgfscope}%
\begin{pgfscope}%
\pgfsys@transformshift{3.283895in}{4.035269in}%
\pgfsys@useobject{currentmarker}{}%
\end{pgfscope}%
\begin{pgfscope}%
\pgfsys@transformshift{3.262299in}{4.037625in}%
\pgfsys@useobject{currentmarker}{}%
\end{pgfscope}%
\begin{pgfscope}%
\pgfsys@transformshift{3.244929in}{4.045946in}%
\pgfsys@useobject{currentmarker}{}%
\end{pgfscope}%
\begin{pgfscope}%
\pgfsys@transformshift{3.224038in}{4.078669in}%
\pgfsys@useobject{currentmarker}{}%
\end{pgfscope}%
\begin{pgfscope}%
\pgfsys@transformshift{3.204790in}{4.186260in}%
\pgfsys@useobject{currentmarker}{}%
\end{pgfscope}%
\begin{pgfscope}%
\pgfsys@transformshift{3.186482in}{4.364591in}%
\pgfsys@useobject{currentmarker}{}%
\end{pgfscope}%
\begin{pgfscope}%
\pgfsys@transformshift{3.169112in}{4.414243in}%
\pgfsys@useobject{currentmarker}{}%
\end{pgfscope}%
\begin{pgfscope}%
\pgfsys@transformshift{3.149159in}{4.236756in}%
\pgfsys@useobject{currentmarker}{}%
\end{pgfscope}%
\begin{pgfscope}%
\pgfsys@transformshift{3.131085in}{4.095077in}%
\pgfsys@useobject{currentmarker}{}%
\end{pgfscope}%
\begin{pgfscope}%
\pgfsys@transformshift{3.109490in}{4.048657in}%
\pgfsys@useobject{currentmarker}{}%
\end{pgfscope}%
\begin{pgfscope}%
\pgfsys@transformshift{3.089773in}{4.037582in}%
\pgfsys@useobject{currentmarker}{}%
\end{pgfscope}%
\begin{pgfscope}%
\pgfsys@transformshift{3.071699in}{4.034929in}%
\pgfsys@useobject{currentmarker}{}%
\end{pgfscope}%
\begin{pgfscope}%
\pgfsys@transformshift{3.051747in}{4.037095in}%
\pgfsys@useobject{currentmarker}{}%
\end{pgfscope}%
\begin{pgfscope}%
\pgfsys@transformshift{3.031325in}{4.050935in}%
\pgfsys@useobject{currentmarker}{}%
\end{pgfscope}%
\begin{pgfscope}%
\pgfsys@transformshift{3.015363in}{4.089796in}%
\pgfsys@useobject{currentmarker}{}%
\end{pgfscope}%
\begin{pgfscope}%
\pgfsys@transformshift{2.993535in}{4.261413in}%
\pgfsys@useobject{currentmarker}{}%
\end{pgfscope}%
\begin{pgfscope}%
\pgfsys@transformshift{2.974756in}{4.400789in}%
\pgfsys@useobject{currentmarker}{}%
\end{pgfscope}%
\begin{pgfscope}%
\pgfsys@transformshift{2.957151in}{4.407416in}%
\pgfsys@useobject{currentmarker}{}%
\end{pgfscope}%
\begin{pgfscope}%
\pgfsys@transformshift{2.937434in}{4.202778in}%
\pgfsys@useobject{currentmarker}{}%
\end{pgfscope}%
\begin{pgfscope}%
\pgfsys@transformshift{2.916542in}{4.079855in}%
\pgfsys@useobject{currentmarker}{}%
\end{pgfscope}%
\begin{pgfscope}%
\pgfsys@transformshift{2.897294in}{4.046667in}%
\pgfsys@useobject{currentmarker}{}%
\end{pgfscope}%
\begin{pgfscope}%
\pgfsys@transformshift{2.880629in}{4.039591in}%
\pgfsys@useobject{currentmarker}{}%
\end{pgfscope}%
\begin{pgfscope}%
\pgfsys@transformshift{2.856922in}{4.034909in}%
\pgfsys@useobject{currentmarker}{}%
\end{pgfscope}%
\begin{pgfscope}%
\pgfsys@transformshift{2.840960in}{4.037039in}%
\pgfsys@useobject{currentmarker}{}%
\end{pgfscope}%
\begin{pgfscope}%
\pgfsys@transformshift{2.823120in}{4.045975in}%
\pgfsys@useobject{currentmarker}{}%
\end{pgfscope}%
\begin{pgfscope}%
\pgfsys@transformshift{2.801290in}{4.094319in}%
\pgfsys@useobject{currentmarker}{}%
\end{pgfscope}%
\begin{pgfscope}%
\pgfsys@transformshift{2.781808in}{4.155458in}%
\pgfsys@useobject{currentmarker}{}%
\end{pgfscope}%
\begin{pgfscope}%
\pgfsys@transformshift{2.763735in}{4.277130in}%
\pgfsys@useobject{currentmarker}{}%
\end{pgfscope}%
\begin{pgfscope}%
\pgfsys@transformshift{2.744956in}{4.398042in}%
\pgfsys@useobject{currentmarker}{}%
\end{pgfscope}%
\begin{pgfscope}%
\pgfsys@transformshift{2.722891in}{4.401280in}%
\pgfsys@useobject{currentmarker}{}%
\end{pgfscope}%
\begin{pgfscope}%
\pgfsys@transformshift{2.707398in}{4.217368in}%
\pgfsys@useobject{currentmarker}{}%
\end{pgfscope}%
\begin{pgfscope}%
\pgfsys@transformshift{2.685804in}{4.074410in}%
\pgfsys@useobject{currentmarker}{}%
\end{pgfscope}%
\begin{pgfscope}%
\pgfsys@transformshift{2.667025in}{4.045204in}%
\pgfsys@useobject{currentmarker}{}%
\end{pgfscope}%
\begin{pgfscope}%
\pgfsys@transformshift{2.648481in}{4.036659in}%
\pgfsys@useobject{currentmarker}{}%
\end{pgfscope}%
\begin{pgfscope}%
\pgfsys@transformshift{2.627590in}{4.034861in}%
\pgfsys@useobject{currentmarker}{}%
\end{pgfscope}%
\begin{pgfscope}%
\pgfsys@transformshift{2.611628in}{4.036710in}%
\pgfsys@useobject{currentmarker}{}%
\end{pgfscope}%
\begin{pgfscope}%
\pgfsys@transformshift{2.590035in}{4.048365in}%
\pgfsys@useobject{currentmarker}{}%
\end{pgfscope}%
\begin{pgfscope}%
\pgfsys@transformshift{2.574307in}{4.075438in}%
\pgfsys@useobject{currentmarker}{}%
\end{pgfscope}%
\begin{pgfscope}%
\pgfsys@transformshift{2.552711in}{4.175790in}%
\pgfsys@useobject{currentmarker}{}%
\end{pgfscope}%
\begin{pgfscope}%
\pgfsys@transformshift{2.533934in}{4.312287in}%
\pgfsys@useobject{currentmarker}{}%
\end{pgfscope}%
\begin{pgfscope}%
\pgfsys@transformshift{2.514921in}{4.398030in}%
\pgfsys@useobject{currentmarker}{}%
\end{pgfscope}%
\begin{pgfscope}%
\pgfsys@transformshift{2.495203in}{4.372297in}%
\pgfsys@useobject{currentmarker}{}%
\end{pgfscope}%
\begin{pgfscope}%
\pgfsys@transformshift{2.476190in}{4.262440in}%
\pgfsys@useobject{currentmarker}{}%
\end{pgfscope}%
\begin{pgfscope}%
\pgfsys@transformshift{2.454830in}{4.096659in}%
\pgfsys@useobject{currentmarker}{}%
\end{pgfscope}%
\begin{pgfscope}%
\pgfsys@transformshift{2.438165in}{4.052992in}%
\pgfsys@useobject{currentmarker}{}%
\end{pgfscope}%
\begin{pgfscope}%
\pgfsys@transformshift{2.419386in}{4.039989in}%
\pgfsys@useobject{currentmarker}{}%
\end{pgfscope}%
\begin{pgfscope}%
\pgfsys@transformshift{2.400607in}{4.035174in}%
\pgfsys@useobject{currentmarker}{}%
\end{pgfscope}%
\begin{pgfscope}%
\pgfsys@transformshift{2.379482in}{4.035483in}%
\pgfsys@useobject{currentmarker}{}%
\end{pgfscope}%
\begin{pgfscope}%
\pgfsys@transformshift{2.360235in}{4.040851in}%
\pgfsys@useobject{currentmarker}{}%
\end{pgfscope}%
\begin{pgfscope}%
\pgfsys@transformshift{2.342159in}{4.061531in}%
\pgfsys@useobject{currentmarker}{}%
\end{pgfscope}%
\begin{pgfscope}%
\pgfsys@transformshift{2.324085in}{4.093964in}%
\pgfsys@useobject{currentmarker}{}%
\end{pgfscope}%
\begin{pgfscope}%
\pgfsys@transformshift{2.301317in}{4.245184in}%
\pgfsys@useobject{currentmarker}{}%
\end{pgfscope}%
\begin{pgfscope}%
\pgfsys@transformshift{2.283007in}{4.378570in}%
\pgfsys@useobject{currentmarker}{}%
\end{pgfscope}%
\begin{pgfscope}%
\pgfsys@transformshift{2.264934in}{4.398975in}%
\pgfsys@useobject{currentmarker}{}%
\end{pgfscope}%
\begin{pgfscope}%
\pgfsys@transformshift{2.245452in}{4.220824in}%
\pgfsys@useobject{currentmarker}{}%
\end{pgfscope}%
\begin{pgfscope}%
\pgfsys@transformshift{2.227142in}{4.092553in}%
\pgfsys@useobject{currentmarker}{}%
\end{pgfscope}%
\begin{pgfscope}%
\pgfsys@transformshift{2.209537in}{4.055191in}%
\pgfsys@useobject{currentmarker}{}%
\end{pgfscope}%
\begin{pgfscope}%
\pgfsys@transformshift{2.186066in}{4.040583in}%
\pgfsys@useobject{currentmarker}{}%
\end{pgfscope}%
\begin{pgfscope}%
\pgfsys@transformshift{2.168461in}{4.035543in}%
\pgfsys@useobject{currentmarker}{}%
\end{pgfscope}%
\begin{pgfscope}%
\pgfsys@transformshift{2.149213in}{4.035698in}%
\pgfsys@useobject{currentmarker}{}%
\end{pgfscope}%
\begin{pgfscope}%
\pgfsys@transformshift{2.129495in}{4.038682in}%
\pgfsys@useobject{currentmarker}{}%
\end{pgfscope}%
\begin{pgfscope}%
\pgfsys@transformshift{2.108604in}{4.051600in}%
\pgfsys@useobject{currentmarker}{}%
\end{pgfscope}%
\begin{pgfscope}%
\pgfsys@transformshift{2.091468in}{4.067928in}%
\pgfsys@useobject{currentmarker}{}%
\end{pgfscope}%
\begin{pgfscope}%
\pgfsys@transformshift{2.072455in}{4.161711in}%
\pgfsys@useobject{currentmarker}{}%
\end{pgfscope}%
\begin{pgfscope}%
\pgfsys@transformshift{2.054147in}{4.318421in}%
\pgfsys@useobject{currentmarker}{}%
\end{pgfscope}%
\begin{pgfscope}%
\pgfsys@transformshift{2.033022in}{4.411113in}%
\pgfsys@useobject{currentmarker}{}%
\end{pgfscope}%
\begin{pgfscope}%
\pgfsys@transformshift{2.014712in}{4.366187in}%
\pgfsys@useobject{currentmarker}{}%
\end{pgfscope}%
\begin{pgfscope}%
\pgfsys@transformshift{1.997107in}{4.187361in}%
\pgfsys@useobject{currentmarker}{}%
\end{pgfscope}%
\begin{pgfscope}%
\pgfsys@transformshift{1.978094in}{4.093501in}%
\pgfsys@useobject{currentmarker}{}%
\end{pgfscope}%
\begin{pgfscope}%
\pgfsys@transformshift{1.959317in}{4.054978in}%
\pgfsys@useobject{currentmarker}{}%
\end{pgfscope}%
\begin{pgfscope}%
\pgfsys@transformshift{1.936782in}{4.039516in}%
\pgfsys@useobject{currentmarker}{}%
\end{pgfscope}%
\begin{pgfscope}%
\pgfsys@transformshift{1.917768in}{4.035886in}%
\pgfsys@useobject{currentmarker}{}%
\end{pgfscope}%
\begin{pgfscope}%
\pgfsys@transformshift{1.899695in}{4.034974in}%
\pgfsys@useobject{currentmarker}{}%
\end{pgfscope}%
\begin{pgfscope}%
\pgfsys@transformshift{1.881387in}{4.037379in}%
\pgfsys@useobject{currentmarker}{}%
\end{pgfscope}%
\begin{pgfscope}%
\pgfsys@transformshift{1.859557in}{4.047411in}%
\pgfsys@useobject{currentmarker}{}%
\end{pgfscope}%
\begin{pgfscope}%
\pgfsys@transformshift{1.841483in}{4.074334in}%
\pgfsys@useobject{currentmarker}{}%
\end{pgfscope}%
\begin{pgfscope}%
\pgfsys@transformshift{1.822235in}{4.177043in}%
\pgfsys@useobject{currentmarker}{}%
\end{pgfscope}%
\begin{pgfscope}%
\pgfsys@transformshift{1.802282in}{4.327187in}%
\pgfsys@useobject{currentmarker}{}%
\end{pgfscope}%
\begin{pgfscope}%
\pgfsys@transformshift{1.779278in}{4.405118in}%
\pgfsys@useobject{currentmarker}{}%
\end{pgfscope}%
\begin{pgfscope}%
\pgfsys@transformshift{1.760736in}{4.404528in}%
\pgfsys@useobject{currentmarker}{}%
\end{pgfscope}%
\begin{pgfscope}%
\pgfsys@transformshift{1.745008in}{4.281206in}%
\pgfsys@useobject{currentmarker}{}%
\end{pgfscope}%
\begin{pgfscope}%
\pgfsys@transformshift{1.726229in}{4.178094in}%
\pgfsys@useobject{currentmarker}{}%
\end{pgfscope}%
\begin{pgfscope}%
\pgfsys@transformshift{1.707687in}{4.420057in}%
\pgfsys@useobject{currentmarker}{}%
\end{pgfscope}%
\begin{pgfscope}%
\pgfsys@transformshift{1.688674in}{4.342517in}%
\pgfsys@useobject{currentmarker}{}%
\end{pgfscope}%
\begin{pgfscope}%
\pgfsys@transformshift{1.670600in}{4.136905in}%
\pgfsys@useobject{currentmarker}{}%
\end{pgfscope}%
\begin{pgfscope}%
\pgfsys@transformshift{1.649004in}{4.059947in}%
\pgfsys@useobject{currentmarker}{}%
\end{pgfscope}%
\begin{pgfscope}%
\pgfsys@transformshift{1.633042in}{4.041550in}%
\pgfsys@useobject{currentmarker}{}%
\end{pgfscope}%
\begin{pgfscope}%
\pgfsys@transformshift{1.611683in}{4.036573in}%
\pgfsys@useobject{currentmarker}{}%
\end{pgfscope}%
\begin{pgfscope}%
\pgfsys@transformshift{1.590790in}{4.035464in}%
\pgfsys@useobject{currentmarker}{}%
\end{pgfscope}%
\begin{pgfscope}%
\pgfsys@transformshift{1.573422in}{4.038969in}%
\pgfsys@useobject{currentmarker}{}%
\end{pgfscope}%
\begin{pgfscope}%
\pgfsys@transformshift{1.552061in}{4.056551in}%
\pgfsys@useobject{currentmarker}{}%
\end{pgfscope}%
\begin{pgfscope}%
\pgfsys@transformshift{1.531873in}{4.094591in}%
\pgfsys@useobject{currentmarker}{}%
\end{pgfscope}%
\begin{pgfscope}%
\pgfsys@transformshift{1.513800in}{4.209723in}%
\pgfsys@useobject{currentmarker}{}%
\end{pgfscope}%
\begin{pgfscope}%
\pgfsys@transformshift{1.496429in}{4.309892in}%
\pgfsys@useobject{currentmarker}{}%
\end{pgfscope}%
\begin{pgfscope}%
\pgfsys@transformshift{1.476478in}{4.415010in}%
\pgfsys@useobject{currentmarker}{}%
\end{pgfscope}%
\begin{pgfscope}%
\pgfsys@transformshift{1.457934in}{4.411499in}%
\pgfsys@useobject{currentmarker}{}%
\end{pgfscope}%
\begin{pgfscope}%
\pgfsys@transformshift{1.436340in}{4.263577in}%
\pgfsys@useobject{currentmarker}{}%
\end{pgfscope}%
\begin{pgfscope}%
\pgfsys@transformshift{1.417327in}{4.139794in}%
\pgfsys@useobject{currentmarker}{}%
\end{pgfscope}%
\begin{pgfscope}%
\pgfsys@transformshift{1.403242in}{4.078199in}%
\pgfsys@useobject{currentmarker}{}%
\end{pgfscope}%
\begin{pgfscope}%
\pgfsys@transformshift{1.381178in}{4.046909in}%
\pgfsys@useobject{currentmarker}{}%
\end{pgfscope}%
\begin{pgfscope}%
\pgfsys@transformshift{1.359347in}{4.038669in}%
\pgfsys@useobject{currentmarker}{}%
\end{pgfscope}%
\begin{pgfscope}%
\pgfsys@transformshift{1.343856in}{4.035974in}%
\pgfsys@useobject{currentmarker}{}%
\end{pgfscope}%
\begin{pgfscope}%
\pgfsys@transformshift{1.322026in}{4.036819in}%
\pgfsys@useobject{currentmarker}{}%
\end{pgfscope}%
\begin{pgfscope}%
\pgfsys@transformshift{1.303013in}{4.043186in}%
\pgfsys@useobject{currentmarker}{}%
\end{pgfscope}%
\begin{pgfscope}%
\pgfsys@transformshift{1.281888in}{4.067234in}%
\pgfsys@useobject{currentmarker}{}%
\end{pgfscope}%
\begin{pgfscope}%
\pgfsys@transformshift{1.264049in}{4.098822in}%
\pgfsys@useobject{currentmarker}{}%
\end{pgfscope}%
\begin{pgfscope}%
\pgfsys@transformshift{1.245035in}{4.196672in}%
\pgfsys@useobject{currentmarker}{}%
\end{pgfscope}%
\begin{pgfscope}%
\pgfsys@transformshift{1.226491in}{4.335710in}%
\pgfsys@useobject{currentmarker}{}%
\end{pgfscope}%
\begin{pgfscope}%
\pgfsys@transformshift{1.208417in}{4.410512in}%
\pgfsys@useobject{currentmarker}{}%
\end{pgfscope}%
\begin{pgfscope}%
\pgfsys@transformshift{1.186353in}{4.433496in}%
\pgfsys@useobject{currentmarker}{}%
\end{pgfscope}%
\begin{pgfscope}%
\pgfsys@transformshift{1.169451in}{4.360931in}%
\pgfsys@useobject{currentmarker}{}%
\end{pgfscope}%
\begin{pgfscope}%
\pgfsys@transformshift{1.149500in}{4.153158in}%
\pgfsys@useobject{currentmarker}{}%
\end{pgfscope}%
\begin{pgfscope}%
\pgfsys@transformshift{1.130018in}{4.081852in}%
\pgfsys@useobject{currentmarker}{}%
\end{pgfscope}%
\begin{pgfscope}%
\pgfsys@transformshift{1.112882in}{4.069553in}%
\pgfsys@useobject{currentmarker}{}%
\end{pgfscope}%
\begin{pgfscope}%
\pgfsys@transformshift{1.094103in}{4.051093in}%
\pgfsys@useobject{currentmarker}{}%
\end{pgfscope}%
\begin{pgfscope}%
\pgfsys@transformshift{1.073918in}{4.040774in}%
\pgfsys@useobject{currentmarker}{}%
\end{pgfscope}%
\begin{pgfscope}%
\pgfsys@transformshift{1.054670in}{4.035933in}%
\pgfsys@useobject{currentmarker}{}%
\end{pgfscope}%
\begin{pgfscope}%
\pgfsys@transformshift{1.034249in}{4.036558in}%
\pgfsys@useobject{currentmarker}{}%
\end{pgfscope}%
\begin{pgfscope}%
\pgfsys@transformshift{1.016409in}{4.041167in}%
\pgfsys@useobject{currentmarker}{}%
\end{pgfscope}%
\begin{pgfscope}%
\pgfsys@transformshift{0.995988in}{4.059345in}%
\pgfsys@useobject{currentmarker}{}%
\end{pgfscope}%
\begin{pgfscope}%
\pgfsys@transformshift{0.977912in}{4.094575in}%
\pgfsys@useobject{currentmarker}{}%
\end{pgfscope}%
\begin{pgfscope}%
\pgfsys@transformshift{0.957021in}{4.192866in}%
\pgfsys@useobject{currentmarker}{}%
\end{pgfscope}%
\begin{pgfscope}%
\pgfsys@transformshift{0.941296in}{4.302800in}%
\pgfsys@useobject{currentmarker}{}%
\end{pgfscope}%
\begin{pgfscope}%
\pgfsys@transformshift{0.919231in}{4.421828in}%
\pgfsys@useobject{currentmarker}{}%
\end{pgfscope}%
\begin{pgfscope}%
\pgfsys@transformshift{0.900218in}{4.443400in}%
\pgfsys@useobject{currentmarker}{}%
\end{pgfscope}%
\begin{pgfscope}%
\pgfsys@transformshift{0.882379in}{4.389713in}%
\pgfsys@useobject{currentmarker}{}%
\end{pgfscope}%
\begin{pgfscope}%
\pgfsys@transformshift{0.863600in}{4.213608in}%
\pgfsys@useobject{currentmarker}{}%
\end{pgfscope}%
\begin{pgfscope}%
\pgfsys@transformshift{0.842239in}{4.096806in}%
\pgfsys@useobject{currentmarker}{}%
\end{pgfscope}%
\begin{pgfscope}%
\pgfsys@transformshift{0.823696in}{4.064919in}%
\pgfsys@useobject{currentmarker}{}%
\end{pgfscope}%
\begin{pgfscope}%
\pgfsys@transformshift{0.801866in}{4.049287in}%
\pgfsys@useobject{currentmarker}{}%
\end{pgfscope}%
\begin{pgfscope}%
\pgfsys@transformshift{0.786844in}{4.040208in}%
\pgfsys@useobject{currentmarker}{}%
\end{pgfscope}%
\begin{pgfscope}%
\pgfsys@transformshift{0.765482in}{4.036641in}%
\pgfsys@useobject{currentmarker}{}%
\end{pgfscope}%
\begin{pgfscope}%
\pgfsys@transformshift{0.747174in}{4.037221in}%
\pgfsys@useobject{currentmarker}{}%
\end{pgfscope}%
\begin{pgfscope}%
\pgfsys@transformshift{0.726518in}{4.046149in}%
\pgfsys@useobject{currentmarker}{}%
\end{pgfscope}%
\begin{pgfscope}%
\pgfsys@transformshift{0.707505in}{4.060925in}%
\pgfsys@useobject{currentmarker}{}%
\end{pgfscope}%
\begin{pgfscope}%
\pgfsys@transformshift{0.688960in}{4.108829in}%
\pgfsys@useobject{currentmarker}{}%
\end{pgfscope}%
\begin{pgfscope}%
\pgfsys@transformshift{0.670652in}{4.180771in}%
\pgfsys@useobject{currentmarker}{}%
\end{pgfscope}%
\begin{pgfscope}%
\pgfsys@transformshift{0.652342in}{4.344164in}%
\pgfsys@useobject{currentmarker}{}%
\end{pgfscope}%
\begin{pgfscope}%
\pgfsys@transformshift{0.647414in}{4.355125in}%
\pgfsys@useobject{currentmarker}{}%
\end{pgfscope}%
\begin{pgfscope}%
\pgfsys@transformshift{0.654690in}{4.238739in}%
\pgfsys@useobject{currentmarker}{}%
\end{pgfscope}%
\begin{pgfscope}%
\pgfsys@transformshift{0.677224in}{4.081550in}%
\pgfsys@useobject{currentmarker}{}%
\end{pgfscope}%
\begin{pgfscope}%
\pgfsys@transformshift{0.696472in}{4.044705in}%
\pgfsys@useobject{currentmarker}{}%
\end{pgfscope}%
\begin{pgfscope}%
\pgfsys@transformshift{0.712199in}{4.036934in}%
\pgfsys@useobject{currentmarker}{}%
\end{pgfscope}%
\begin{pgfscope}%
\pgfsys@transformshift{0.735438in}{4.041216in}%
\pgfsys@useobject{currentmarker}{}%
\end{pgfscope}%
\begin{pgfscope}%
\pgfsys@transformshift{0.751869in}{4.056962in}%
\pgfsys@useobject{currentmarker}{}%
\end{pgfscope}%
\begin{pgfscope}%
\pgfsys@transformshift{0.772994in}{4.141659in}%
\pgfsys@useobject{currentmarker}{}%
\end{pgfscope}%
\begin{pgfscope}%
\pgfsys@transformshift{0.789895in}{4.347813in}%
\pgfsys@useobject{currentmarker}{}%
\end{pgfscope}%
\begin{pgfscope}%
\pgfsys@transformshift{0.807734in}{4.447087in}%
\pgfsys@useobject{currentmarker}{}%
\end{pgfscope}%
\begin{pgfscope}%
\pgfsys@transformshift{0.827922in}{4.356458in}%
\pgfsys@useobject{currentmarker}{}%
\end{pgfscope}%
\begin{pgfscope}%
\pgfsys@transformshift{0.850221in}{4.135152in}%
\pgfsys@useobject{currentmarker}{}%
\end{pgfscope}%
\begin{pgfscope}%
\pgfsys@transformshift{0.869703in}{4.056303in}%
\pgfsys@useobject{currentmarker}{}%
\end{pgfscope}%
\begin{pgfscope}%
\pgfsys@transformshift{0.885430in}{4.040382in}%
\pgfsys@useobject{currentmarker}{}%
\end{pgfscope}%
\begin{pgfscope}%
\pgfsys@transformshift{0.903738in}{4.035715in}%
\pgfsys@useobject{currentmarker}{}%
\end{pgfscope}%
\begin{pgfscope}%
\pgfsys@transformshift{0.926508in}{4.045710in}%
\pgfsys@useobject{currentmarker}{}%
\end{pgfscope}%
\begin{pgfscope}%
\pgfsys@transformshift{0.947633in}{4.080736in}%
\pgfsys@useobject{currentmarker}{}%
\end{pgfscope}%
\begin{pgfscope}%
\pgfsys@transformshift{0.964533in}{4.198395in}%
\pgfsys@useobject{currentmarker}{}%
\end{pgfscope}%
\begin{pgfscope}%
\pgfsys@transformshift{0.984486in}{4.424894in}%
\pgfsys@useobject{currentmarker}{}%
\end{pgfscope}%
\begin{pgfscope}%
\pgfsys@transformshift{1.006081in}{4.391548in}%
\pgfsys@useobject{currentmarker}{}%
\end{pgfscope}%
\begin{pgfscope}%
\pgfsys@transformshift{1.022981in}{4.228936in}%
\pgfsys@useobject{currentmarker}{}%
\end{pgfscope}%
\begin{pgfscope}%
\pgfsys@transformshift{1.041994in}{4.088012in}%
\pgfsys@useobject{currentmarker}{}%
\end{pgfscope}%
\begin{pgfscope}%
\pgfsys@transformshift{1.060068in}{4.045240in}%
\pgfsys@useobject{currentmarker}{}%
\end{pgfscope}%
\begin{pgfscope}%
\pgfsys@transformshift{1.080255in}{4.036207in}%
\pgfsys@useobject{currentmarker}{}%
\end{pgfscope}%
\begin{pgfscope}%
\pgfsys@transformshift{1.098329in}{4.037585in}%
\pgfsys@useobject{currentmarker}{}%
\end{pgfscope}%
\begin{pgfscope}%
\pgfsys@transformshift{1.117342in}{4.050814in}%
\pgfsys@useobject{currentmarker}{}%
\end{pgfscope}%
\begin{pgfscope}%
\pgfsys@transformshift{1.136590in}{4.094909in}%
\pgfsys@useobject{currentmarker}{}%
\end{pgfscope}%
\begin{pgfscope}%
\pgfsys@transformshift{1.156308in}{4.305748in}%
\pgfsys@useobject{currentmarker}{}%
\end{pgfscope}%
\begin{pgfscope}%
\pgfsys@transformshift{1.175556in}{4.430715in}%
\pgfsys@useobject{currentmarker}{}%
\end{pgfscope}%
\begin{pgfscope}%
\pgfsys@transformshift{1.194803in}{4.341676in}%
\pgfsys@useobject{currentmarker}{}%
\end{pgfscope}%
\begin{pgfscope}%
\pgfsys@transformshift{1.212172in}{4.170107in}%
\pgfsys@useobject{currentmarker}{}%
\end{pgfscope}%
\begin{pgfscope}%
\pgfsys@transformshift{1.231185in}{4.075850in}%
\pgfsys@useobject{currentmarker}{}%
\end{pgfscope}%
\begin{pgfscope}%
\pgfsys@transformshift{1.251841in}{4.041300in}%
\pgfsys@useobject{currentmarker}{}%
\end{pgfscope}%
\begin{pgfscope}%
\pgfsys@transformshift{1.272497in}{4.035404in}%
\pgfsys@useobject{currentmarker}{}%
\end{pgfscope}%
\begin{pgfscope}%
\pgfsys@transformshift{1.293390in}{4.039391in}%
\pgfsys@useobject{currentmarker}{}%
\end{pgfscope}%
\begin{pgfscope}%
\pgfsys@transformshift{1.309821in}{4.050026in}%
\pgfsys@useobject{currentmarker}{}%
\end{pgfscope}%
\begin{pgfscope}%
\pgfsys@transformshift{1.329537in}{4.107379in}%
\pgfsys@useobject{currentmarker}{}%
\end{pgfscope}%
\begin{pgfscope}%
\pgfsys@transformshift{1.348316in}{4.317159in}%
\pgfsys@useobject{currentmarker}{}%
\end{pgfscope}%
\begin{pgfscope}%
\pgfsys@transformshift{1.365450in}{4.423640in}%
\pgfsys@useobject{currentmarker}{}%
\end{pgfscope}%
\begin{pgfscope}%
\pgfsys@transformshift{1.386811in}{4.341503in}%
\pgfsys@useobject{currentmarker}{}%
\end{pgfscope}%
\begin{pgfscope}%
\pgfsys@transformshift{1.404182in}{4.161982in}%
\pgfsys@useobject{currentmarker}{}%
\end{pgfscope}%
\begin{pgfscope}%
\pgfsys@transformshift{1.427184in}{4.077292in}%
\pgfsys@useobject{currentmarker}{}%
\end{pgfscope}%
\begin{pgfscope}%
\pgfsys@transformshift{1.445729in}{4.045725in}%
\pgfsys@useobject{currentmarker}{}%
\end{pgfscope}%
\begin{pgfscope}%
\pgfsys@transformshift{1.464742in}{4.036535in}%
\pgfsys@useobject{currentmarker}{}%
\end{pgfscope}%
\begin{pgfscope}%
\pgfsys@transformshift{1.483050in}{4.035580in}%
\pgfsys@useobject{currentmarker}{}%
\end{pgfscope}%
\begin{pgfscope}%
\pgfsys@transformshift{1.501829in}{4.042706in}%
\pgfsys@useobject{currentmarker}{}%
\end{pgfscope}%
\begin{pgfscope}%
\pgfsys@transformshift{1.520608in}{4.065826in}%
\pgfsys@useobject{currentmarker}{}%
\end{pgfscope}%
\begin{pgfscope}%
\pgfsys@transformshift{1.541967in}{4.175731in}%
\pgfsys@useobject{currentmarker}{}%
\end{pgfscope}%
\begin{pgfscope}%
\pgfsys@transformshift{1.560277in}{4.384372in}%
\pgfsys@useobject{currentmarker}{}%
\end{pgfscope}%
\begin{pgfscope}%
\pgfsys@transformshift{1.579525in}{4.412982in}%
\pgfsys@useobject{currentmarker}{}%
\end{pgfscope}%
\begin{pgfscope}%
\pgfsys@transformshift{1.600415in}{4.272452in}%
\pgfsys@useobject{currentmarker}{}%
\end{pgfscope}%
\begin{pgfscope}%
\pgfsys@transformshift{1.617315in}{4.135574in}%
\pgfsys@useobject{currentmarker}{}%
\end{pgfscope}%
\begin{pgfscope}%
\pgfsys@transformshift{1.635156in}{4.075551in}%
\pgfsys@useobject{currentmarker}{}%
\end{pgfscope}%
\begin{pgfscope}%
\pgfsys@transformshift{1.657924in}{4.043273in}%
\pgfsys@useobject{currentmarker}{}%
\end{pgfscope}%
\begin{pgfscope}%
\pgfsys@transformshift{1.675529in}{4.035626in}%
\pgfsys@useobject{currentmarker}{}%
\end{pgfscope}%
\begin{pgfscope}%
\pgfsys@transformshift{1.694073in}{4.036059in}%
\pgfsys@useobject{currentmarker}{}%
\end{pgfscope}%
\begin{pgfscope}%
\pgfsys@transformshift{1.713790in}{4.043397in}%
\pgfsys@useobject{currentmarker}{}%
\end{pgfscope}%
\begin{pgfscope}%
\pgfsys@transformshift{1.734680in}{4.075178in}%
\pgfsys@useobject{currentmarker}{}%
\end{pgfscope}%
\begin{pgfscope}%
\pgfsys@transformshift{1.752519in}{4.093748in}%
\pgfsys@useobject{currentmarker}{}%
\end{pgfscope}%
\begin{pgfscope}%
\pgfsys@transformshift{1.775524in}{4.306772in}%
\pgfsys@useobject{currentmarker}{}%
\end{pgfscope}%
\begin{pgfscope}%
\pgfsys@transformshift{1.790780in}{4.413833in}%
\pgfsys@useobject{currentmarker}{}%
\end{pgfscope}%
\begin{pgfscope}%
\pgfsys@transformshift{1.808854in}{4.391544in}%
\pgfsys@useobject{currentmarker}{}%
\end{pgfscope}%
\begin{pgfscope}%
\pgfsys@transformshift{1.827398in}{4.251757in}%
\pgfsys@useobject{currentmarker}{}%
\end{pgfscope}%
\begin{pgfscope}%
\pgfsys@transformshift{1.848054in}{4.092982in}%
\pgfsys@useobject{currentmarker}{}%
\end{pgfscope}%
\begin{pgfscope}%
\pgfsys@transformshift{1.867068in}{4.046804in}%
\pgfsys@useobject{currentmarker}{}%
\end{pgfscope}%
\begin{pgfscope}%
\pgfsys@transformshift{1.888663in}{4.037122in}%
\pgfsys@useobject{currentmarker}{}%
\end{pgfscope}%
\begin{pgfscope}%
\pgfsys@transformshift{1.906268in}{4.034856in}%
\pgfsys@useobject{currentmarker}{}%
\end{pgfscope}%
\begin{pgfscope}%
\pgfsys@transformshift{1.924342in}{4.037043in}%
\pgfsys@useobject{currentmarker}{}%
\end{pgfscope}%
\begin{pgfscope}%
\pgfsys@transformshift{1.943590in}{4.048224in}%
\pgfsys@useobject{currentmarker}{}%
\end{pgfscope}%
\begin{pgfscope}%
\pgfsys@transformshift{1.964480in}{4.103414in}%
\pgfsys@useobject{currentmarker}{}%
\end{pgfscope}%
\begin{pgfscope}%
\pgfsys@transformshift{1.979973in}{4.236126in}%
\pgfsys@useobject{currentmarker}{}%
\end{pgfscope}%
\begin{pgfscope}%
\pgfsys@transformshift{2.000864in}{4.414483in}%
\pgfsys@useobject{currentmarker}{}%
\end{pgfscope}%
\begin{pgfscope}%
\pgfsys@transformshift{2.022223in}{4.053995in}%
\pgfsys@useobject{currentmarker}{}%
\end{pgfscope}%
\begin{pgfscope}%
\pgfsys@transformshift{2.039359in}{4.101427in}%
\pgfsys@useobject{currentmarker}{}%
\end{pgfscope}%
\begin{pgfscope}%
\pgfsys@transformshift{2.060719in}{4.305600in}%
\pgfsys@useobject{currentmarker}{}%
\end{pgfscope}%
\begin{pgfscope}%
\pgfsys@transformshift{2.076680in}{4.415293in}%
\pgfsys@useobject{currentmarker}{}%
\end{pgfscope}%
\begin{pgfscope}%
\pgfsys@transformshift{2.096633in}{4.332471in}%
\pgfsys@useobject{currentmarker}{}%
\end{pgfscope}%
\begin{pgfscope}%
\pgfsys@transformshift{2.117055in}{4.129133in}%
\pgfsys@useobject{currentmarker}{}%
\end{pgfscope}%
\begin{pgfscope}%
\pgfsys@transformshift{2.134894in}{4.062597in}%
\pgfsys@useobject{currentmarker}{}%
\end{pgfscope}%
\begin{pgfscope}%
\pgfsys@transformshift{2.154845in}{4.040331in}%
\pgfsys@useobject{currentmarker}{}%
\end{pgfscope}%
\begin{pgfscope}%
\pgfsys@transformshift{2.175736in}{4.034873in}%
\pgfsys@useobject{currentmarker}{}%
\end{pgfscope}%
\begin{pgfscope}%
\pgfsys@transformshift{2.192403in}{4.035563in}%
\pgfsys@useobject{currentmarker}{}%
\end{pgfscope}%
\begin{pgfscope}%
\pgfsys@transformshift{2.214702in}{4.043102in}%
\pgfsys@useobject{currentmarker}{}%
\end{pgfscope}%
\begin{pgfscope}%
\pgfsys@transformshift{2.231133in}{4.064476in}%
\pgfsys@useobject{currentmarker}{}%
\end{pgfscope}%
\begin{pgfscope}%
\pgfsys@transformshift{2.251554in}{4.159477in}%
\pgfsys@useobject{currentmarker}{}%
\end{pgfscope}%
\begin{pgfscope}%
\pgfsys@transformshift{2.271976in}{4.371879in}%
\pgfsys@useobject{currentmarker}{}%
\end{pgfscope}%
\begin{pgfscope}%
\pgfsys@transformshift{2.289110in}{4.418366in}%
\pgfsys@useobject{currentmarker}{}%
\end{pgfscope}%
\begin{pgfscope}%
\pgfsys@transformshift{2.305777in}{4.325827in}%
\pgfsys@useobject{currentmarker}{}%
\end{pgfscope}%
\begin{pgfscope}%
\pgfsys@transformshift{2.327371in}{4.161729in}%
\pgfsys@useobject{currentmarker}{}%
\end{pgfscope}%
\begin{pgfscope}%
\pgfsys@transformshift{2.348732in}{4.059093in}%
\pgfsys@useobject{currentmarker}{}%
\end{pgfscope}%
\begin{pgfscope}%
\pgfsys@transformshift{2.366103in}{4.040083in}%
\pgfsys@useobject{currentmarker}{}%
\end{pgfscope}%
\begin{pgfscope}%
\pgfsys@transformshift{2.389810in}{4.034907in}%
\pgfsys@useobject{currentmarker}{}%
\end{pgfscope}%
\begin{pgfscope}%
\pgfsys@transformshift{2.404129in}{4.035534in}%
\pgfsys@useobject{currentmarker}{}%
\end{pgfscope}%
\begin{pgfscope}%
\pgfsys@transformshift{2.425489in}{4.040928in}%
\pgfsys@useobject{currentmarker}{}%
\end{pgfscope}%
\begin{pgfscope}%
\pgfsys@transformshift{2.443562in}{4.056866in}%
\pgfsys@useobject{currentmarker}{}%
\end{pgfscope}%
\begin{pgfscope}%
\pgfsys@transformshift{2.464219in}{4.094149in}%
\pgfsys@useobject{currentmarker}{}%
\end{pgfscope}%
\begin{pgfscope}%
\pgfsys@transformshift{2.482763in}{4.222564in}%
\pgfsys@useobject{currentmarker}{}%
\end{pgfscope}%
\begin{pgfscope}%
\pgfsys@transformshift{2.499663in}{4.341845in}%
\pgfsys@useobject{currentmarker}{}%
\end{pgfscope}%
\begin{pgfscope}%
\pgfsys@transformshift{2.519850in}{4.413543in}%
\pgfsys@useobject{currentmarker}{}%
\end{pgfscope}%
\begin{pgfscope}%
\pgfsys@transformshift{2.540037in}{4.289968in}%
\pgfsys@useobject{currentmarker}{}%
\end{pgfscope}%
\begin{pgfscope}%
\pgfsys@transformshift{2.560693in}{4.147068in}%
\pgfsys@useobject{currentmarker}{}%
\end{pgfscope}%
\begin{pgfscope}%
\pgfsys@transformshift{2.577827in}{4.061084in}%
\pgfsys@useobject{currentmarker}{}%
\end{pgfscope}%
\begin{pgfscope}%
\pgfsys@transformshift{2.599423in}{4.044294in}%
\pgfsys@useobject{currentmarker}{}%
\end{pgfscope}%
\begin{pgfscope}%
\pgfsys@transformshift{2.614680in}{4.037121in}%
\pgfsys@useobject{currentmarker}{}%
\end{pgfscope}%
\begin{pgfscope}%
\pgfsys@transformshift{2.635336in}{4.034718in}%
\pgfsys@useobject{currentmarker}{}%
\end{pgfscope}%
\begin{pgfscope}%
\pgfsys@transformshift{2.653646in}{4.036816in}%
\pgfsys@useobject{currentmarker}{}%
\end{pgfscope}%
\begin{pgfscope}%
\pgfsys@transformshift{2.675711in}{4.048588in}%
\pgfsys@useobject{currentmarker}{}%
\end{pgfscope}%
\begin{pgfscope}%
\pgfsys@transformshift{2.693550in}{4.085070in}%
\pgfsys@useobject{currentmarker}{}%
\end{pgfscope}%
\begin{pgfscope}%
\pgfsys@transformshift{2.713503in}{4.229668in}%
\pgfsys@useobject{currentmarker}{}%
\end{pgfscope}%
\begin{pgfscope}%
\pgfsys@transformshift{2.730402in}{4.392353in}%
\pgfsys@useobject{currentmarker}{}%
\end{pgfscope}%
\begin{pgfscope}%
\pgfsys@transformshift{2.750119in}{4.386604in}%
\pgfsys@useobject{currentmarker}{}%
\end{pgfscope}%
\begin{pgfscope}%
\pgfsys@transformshift{2.771011in}{4.215997in}%
\pgfsys@useobject{currentmarker}{}%
\end{pgfscope}%
\begin{pgfscope}%
\pgfsys@transformshift{2.794248in}{4.080716in}%
\pgfsys@useobject{currentmarker}{}%
\end{pgfscope}%
\begin{pgfscope}%
\pgfsys@transformshift{2.807627in}{4.053920in}%
\pgfsys@useobject{currentmarker}{}%
\end{pgfscope}%
\begin{pgfscope}%
\pgfsys@transformshift{2.827580in}{4.039310in}%
\pgfsys@useobject{currentmarker}{}%
\end{pgfscope}%
\begin{pgfscope}%
\pgfsys@transformshift{2.846125in}{4.035611in}%
\pgfsys@useobject{currentmarker}{}%
\end{pgfscope}%
\begin{pgfscope}%
\pgfsys@transformshift{2.867015in}{4.035545in}%
\pgfsys@useobject{currentmarker}{}%
\end{pgfscope}%
\begin{pgfscope}%
\pgfsys@transformshift{2.885558in}{4.040638in}%
\pgfsys@useobject{currentmarker}{}%
\end{pgfscope}%
\begin{pgfscope}%
\pgfsys@transformshift{2.906214in}{4.061751in}%
\pgfsys@useobject{currentmarker}{}%
\end{pgfscope}%
\begin{pgfscope}%
\pgfsys@transformshift{2.922410in}{4.095509in}%
\pgfsys@useobject{currentmarker}{}%
\end{pgfscope}%
\begin{pgfscope}%
\pgfsys@transformshift{2.943537in}{4.251253in}%
\pgfsys@useobject{currentmarker}{}%
\end{pgfscope}%
\begin{pgfscope}%
\pgfsys@transformshift{2.961611in}{4.398626in}%
\pgfsys@useobject{currentmarker}{}%
\end{pgfscope}%
\begin{pgfscope}%
\pgfsys@transformshift{2.980624in}{4.391425in}%
\pgfsys@useobject{currentmarker}{}%
\end{pgfscope}%
\begin{pgfscope}%
\pgfsys@transformshift{3.000811in}{4.284221in}%
\pgfsys@useobject{currentmarker}{}%
\end{pgfscope}%
\begin{pgfscope}%
\pgfsys@transformshift{3.020059in}{4.141787in}%
\pgfsys@useobject{currentmarker}{}%
\end{pgfscope}%
\begin{pgfscope}%
\pgfsys@transformshift{3.038602in}{4.088231in}%
\pgfsys@useobject{currentmarker}{}%
\end{pgfscope}%
\begin{pgfscope}%
\pgfsys@transformshift{3.059258in}{4.046178in}%
\pgfsys@useobject{currentmarker}{}%
\end{pgfscope}%
\begin{pgfscope}%
\pgfsys@transformshift{3.078505in}{4.037678in}%
\pgfsys@useobject{currentmarker}{}%
\end{pgfscope}%
\begin{pgfscope}%
\pgfsys@transformshift{3.098927in}{4.035030in}%
\pgfsys@useobject{currentmarker}{}%
\end{pgfscope}%
\begin{pgfscope}%
\pgfsys@transformshift{3.117237in}{4.037807in}%
\pgfsys@useobject{currentmarker}{}%
\end{pgfscope}%
\begin{pgfscope}%
\pgfsys@transformshift{3.137188in}{4.050252in}%
\pgfsys@useobject{currentmarker}{}%
\end{pgfscope}%
\begin{pgfscope}%
\pgfsys@transformshift{3.153384in}{4.070793in}%
\pgfsys@useobject{currentmarker}{}%
\end{pgfscope}%
\begin{pgfscope}%
\pgfsys@transformshift{3.173337in}{4.148724in}%
\pgfsys@useobject{currentmarker}{}%
\end{pgfscope}%
\begin{pgfscope}%
\pgfsys@transformshift{3.191176in}{4.324859in}%
\pgfsys@useobject{currentmarker}{}%
\end{pgfscope}%
\begin{pgfscope}%
\pgfsys@transformshift{3.211362in}{4.411678in}%
\pgfsys@useobject{currentmarker}{}%
\end{pgfscope}%
\begin{pgfscope}%
\pgfsys@transformshift{3.230141in}{4.408768in}%
\pgfsys@useobject{currentmarker}{}%
\end{pgfscope}%
\begin{pgfscope}%
\pgfsys@transformshift{3.251031in}{4.276463in}%
\pgfsys@useobject{currentmarker}{}%
\end{pgfscope}%
\begin{pgfscope}%
\pgfsys@transformshift{3.270750in}{4.123349in}%
\pgfsys@useobject{currentmarker}{}%
\end{pgfscope}%
\begin{pgfscope}%
\pgfsys@transformshift{3.287180in}{4.072097in}%
\pgfsys@useobject{currentmarker}{}%
\end{pgfscope}%
\begin{pgfscope}%
\pgfsys@transformshift{3.307602in}{4.047210in}%
\pgfsys@useobject{currentmarker}{}%
\end{pgfscope}%
\begin{pgfscope}%
\pgfsys@transformshift{3.325676in}{4.039234in}%
\pgfsys@useobject{currentmarker}{}%
\end{pgfscope}%
\begin{pgfscope}%
\pgfsys@transformshift{3.347037in}{4.035539in}%
\pgfsys@useobject{currentmarker}{}%
\end{pgfscope}%
\begin{pgfscope}%
\pgfsys@transformshift{3.365111in}{4.036251in}%
\pgfsys@useobject{currentmarker}{}%
\end{pgfscope}%
\begin{pgfscope}%
\pgfsys@transformshift{3.383890in}{4.038342in}%
\pgfsys@useobject{currentmarker}{}%
\end{pgfscope}%
\begin{pgfscope}%
\pgfsys@transformshift{3.404546in}{4.050221in}%
\pgfsys@useobject{currentmarker}{}%
\end{pgfscope}%
\begin{pgfscope}%
\pgfsys@transformshift{3.426610in}{4.088144in}%
\pgfsys@useobject{currentmarker}{}%
\end{pgfscope}%
\begin{pgfscope}%
\pgfsys@transformshift{3.441633in}{4.168352in}%
\pgfsys@useobject{currentmarker}{}%
\end{pgfscope}%
\begin{pgfscope}%
\pgfsys@transformshift{3.463463in}{4.384932in}%
\pgfsys@useobject{currentmarker}{}%
\end{pgfscope}%
\begin{pgfscope}%
\pgfsys@transformshift{3.482240in}{4.423583in}%
\pgfsys@useobject{currentmarker}{}%
\end{pgfscope}%
\begin{pgfscope}%
\pgfsys@transformshift{3.501253in}{4.360474in}%
\pgfsys@useobject{currentmarker}{}%
\end{pgfscope}%
\begin{pgfscope}%
\pgfsys@transformshift{3.518858in}{4.243586in}%
\pgfsys@useobject{currentmarker}{}%
\end{pgfscope}%
\begin{pgfscope}%
\pgfsys@transformshift{3.536697in}{4.113476in}%
\pgfsys@useobject{currentmarker}{}%
\end{pgfscope}%
\begin{pgfscope}%
\pgfsys@transformshift{3.558762in}{4.062521in}%
\pgfsys@useobject{currentmarker}{}%
\end{pgfscope}%
\begin{pgfscope}%
\pgfsys@transformshift{3.577072in}{4.050387in}%
\pgfsys@useobject{currentmarker}{}%
\end{pgfscope}%
\begin{pgfscope}%
\pgfsys@transformshift{3.597728in}{4.038776in}%
\pgfsys@useobject{currentmarker}{}%
\end{pgfscope}%
\begin{pgfscope}%
\pgfsys@transformshift{3.615333in}{4.035356in}%
\pgfsys@useobject{currentmarker}{}%
\end{pgfscope}%
\begin{pgfscope}%
\pgfsys@transformshift{3.633875in}{4.036780in}%
\pgfsys@useobject{currentmarker}{}%
\end{pgfscope}%
\begin{pgfscope}%
\pgfsys@transformshift{3.651951in}{4.041976in}%
\pgfsys@useobject{currentmarker}{}%
\end{pgfscope}%
\begin{pgfscope}%
\pgfsys@transformshift{3.673310in}{4.052001in}%
\pgfsys@useobject{currentmarker}{}%
\end{pgfscope}%
\begin{pgfscope}%
\pgfsys@transformshift{3.691618in}{4.058603in}%
\pgfsys@useobject{currentmarker}{}%
\end{pgfscope}%
\begin{pgfscope}%
\pgfsys@transformshift{3.711805in}{4.115533in}%
\pgfsys@useobject{currentmarker}{}%
\end{pgfscope}%
\begin{pgfscope}%
\pgfsys@transformshift{3.729645in}{4.188139in}%
\pgfsys@useobject{currentmarker}{}%
\end{pgfscope}%
\begin{pgfscope}%
\pgfsys@transformshift{3.750535in}{4.400349in}%
\pgfsys@useobject{currentmarker}{}%
\end{pgfscope}%
\begin{pgfscope}%
\pgfsys@transformshift{3.768611in}{4.433042in}%
\pgfsys@useobject{currentmarker}{}%
\end{pgfscope}%
\begin{pgfscope}%
\pgfsys@transformshift{3.789736in}{4.391897in}%
\pgfsys@useobject{currentmarker}{}%
\end{pgfscope}%
\begin{pgfscope}%
\pgfsys@transformshift{3.808046in}{4.267028in}%
\pgfsys@useobject{currentmarker}{}%
\end{pgfscope}%
\begin{pgfscope}%
\pgfsys@transformshift{3.825885in}{4.158720in}%
\pgfsys@useobject{currentmarker}{}%
\end{pgfscope}%
\begin{pgfscope}%
\pgfsys@transformshift{3.847479in}{4.072414in}%
\pgfsys@useobject{currentmarker}{}%
\end{pgfscope}%
\begin{pgfscope}%
\pgfsys@transformshift{3.865084in}{4.050450in}%
\pgfsys@useobject{currentmarker}{}%
\end{pgfscope}%
\begin{pgfscope}%
\pgfsys@transformshift{3.886445in}{4.040850in}%
\pgfsys@useobject{currentmarker}{}%
\end{pgfscope}%
\begin{pgfscope}%
\pgfsys@transformshift{3.906630in}{4.035729in}%
\pgfsys@useobject{currentmarker}{}%
\end{pgfscope}%
\begin{pgfscope}%
\pgfsys@transformshift{3.923063in}{4.036794in}%
\pgfsys@useobject{currentmarker}{}%
\end{pgfscope}%
\begin{pgfscope}%
\pgfsys@transformshift{3.940668in}{4.042972in}%
\pgfsys@useobject{currentmarker}{}%
\end{pgfscope}%
\begin{pgfscope}%
\pgfsys@transformshift{3.961324in}{4.058627in}%
\pgfsys@useobject{currentmarker}{}%
\end{pgfscope}%
\begin{pgfscope}%
\pgfsys@transformshift{3.979398in}{4.100161in}%
\pgfsys@useobject{currentmarker}{}%
\end{pgfscope}%
\begin{pgfscope}%
\pgfsys@transformshift{4.001228in}{4.225754in}%
\pgfsys@useobject{currentmarker}{}%
\end{pgfscope}%
\begin{pgfscope}%
\pgfsys@transformshift{4.018127in}{4.400194in}%
\pgfsys@useobject{currentmarker}{}%
\end{pgfscope}%
\begin{pgfscope}%
\pgfsys@transformshift{4.036672in}{4.435278in}%
\pgfsys@useobject{currentmarker}{}%
\end{pgfscope}%
\begin{pgfscope}%
\pgfsys@transformshift{4.058031in}{4.439681in}%
\pgfsys@useobject{currentmarker}{}%
\end{pgfscope}%
\begin{pgfscope}%
\pgfsys@transformshift{4.077044in}{4.249830in}%
\pgfsys@useobject{currentmarker}{}%
\end{pgfscope}%
\begin{pgfscope}%
\pgfsys@transformshift{4.093710in}{4.434630in}%
\pgfsys@useobject{currentmarker}{}%
\end{pgfscope}%
\begin{pgfscope}%
\pgfsys@transformshift{4.115540in}{4.431362in}%
\pgfsys@useobject{currentmarker}{}%
\end{pgfscope}%
\begin{pgfscope}%
\pgfsys@transformshift{4.133145in}{4.354216in}%
\pgfsys@useobject{currentmarker}{}%
\end{pgfscope}%
\begin{pgfscope}%
\pgfsys@transformshift{4.154035in}{4.179951in}%
\pgfsys@useobject{currentmarker}{}%
\end{pgfscope}%
\begin{pgfscope}%
\pgfsys@transformshift{4.174691in}{4.087490in}%
\pgfsys@useobject{currentmarker}{}%
\end{pgfscope}%
\begin{pgfscope}%
\pgfsys@transformshift{4.190653in}{4.058905in}%
\pgfsys@useobject{currentmarker}{}%
\end{pgfscope}%
\begin{pgfscope}%
\pgfsys@transformshift{4.211544in}{4.041625in}%
\pgfsys@useobject{currentmarker}{}%
\end{pgfscope}%
\begin{pgfscope}%
\pgfsys@transformshift{4.229619in}{4.037457in}%
\pgfsys@useobject{currentmarker}{}%
\end{pgfscope}%
\begin{pgfscope}%
\pgfsys@transformshift{4.251684in}{4.037668in}%
\pgfsys@useobject{currentmarker}{}%
\end{pgfscope}%
\begin{pgfscope}%
\pgfsys@transformshift{4.268584in}{4.044842in}%
\pgfsys@useobject{currentmarker}{}%
\end{pgfscope}%
\begin{pgfscope}%
\pgfsys@transformshift{4.287362in}{4.063886in}%
\pgfsys@useobject{currentmarker}{}%
\end{pgfscope}%
\begin{pgfscope}%
\pgfsys@transformshift{4.308019in}{4.135647in}%
\pgfsys@useobject{currentmarker}{}%
\end{pgfscope}%
\begin{pgfscope}%
\pgfsys@transformshift{4.326092in}{4.257902in}%
\pgfsys@useobject{currentmarker}{}%
\end{pgfscope}%
\begin{pgfscope}%
\pgfsys@transformshift{4.344402in}{4.437790in}%
\pgfsys@useobject{currentmarker}{}%
\end{pgfscope}%
\begin{pgfscope}%
\pgfsys@transformshift{4.364588in}{4.457282in}%
\pgfsys@useobject{currentmarker}{}%
\end{pgfscope}%
\begin{pgfscope}%
\pgfsys@transformshift{4.383132in}{4.427180in}%
\pgfsys@useobject{currentmarker}{}%
\end{pgfscope}%
\begin{pgfscope}%
\pgfsys@transformshift{4.403319in}{4.303993in}%
\pgfsys@useobject{currentmarker}{}%
\end{pgfscope}%
\begin{pgfscope}%
\pgfsys@transformshift{4.422801in}{4.179348in}%
\pgfsys@useobject{currentmarker}{}%
\end{pgfscope}%
\begin{pgfscope}%
\pgfsys@transformshift{4.442752in}{4.083215in}%
\pgfsys@useobject{currentmarker}{}%
\end{pgfscope}%
\begin{pgfscope}%
\pgfsys@transformshift{4.461297in}{4.052579in}%
\pgfsys@useobject{currentmarker}{}%
\end{pgfscope}%
\begin{pgfscope}%
\pgfsys@transformshift{4.477727in}{4.041748in}%
\pgfsys@useobject{currentmarker}{}%
\end{pgfscope}%
\begin{pgfscope}%
\pgfsys@transformshift{4.478667in}{4.041296in}%
\pgfsys@useobject{currentmarker}{}%
\end{pgfscope}%
\begin{pgfscope}%
\pgfsys@transformshift{4.473973in}{4.044789in}%
\pgfsys@useobject{currentmarker}{}%
\end{pgfscope}%
\begin{pgfscope}%
\pgfsys@transformshift{4.455663in}{4.071943in}%
\pgfsys@useobject{currentmarker}{}%
\end{pgfscope}%
\begin{pgfscope}%
\pgfsys@transformshift{4.436181in}{4.201436in}%
\pgfsys@useobject{currentmarker}{}%
\end{pgfscope}%
\begin{pgfscope}%
\pgfsys@transformshift{4.417636in}{4.394671in}%
\pgfsys@useobject{currentmarker}{}%
\end{pgfscope}%
\begin{pgfscope}%
\pgfsys@transformshift{4.397920in}{4.458569in}%
\pgfsys@useobject{currentmarker}{}%
\end{pgfscope}%
\begin{pgfscope}%
\pgfsys@transformshift{4.378438in}{4.320417in}%
\pgfsys@useobject{currentmarker}{}%
\end{pgfscope}%
\begin{pgfscope}%
\pgfsys@transformshift{4.358250in}{4.096606in}%
\pgfsys@useobject{currentmarker}{}%
\end{pgfscope}%
\begin{pgfscope}%
\pgfsys@transformshift{4.338768in}{4.051655in}%
\pgfsys@useobject{currentmarker}{}%
\end{pgfscope}%
\begin{pgfscope}%
\pgfsys@transformshift{4.321867in}{4.038526in}%
\pgfsys@useobject{currentmarker}{}%
\end{pgfscope}%
\begin{pgfscope}%
\pgfsys@transformshift{4.302619in}{4.037738in}%
\pgfsys@useobject{currentmarker}{}%
\end{pgfscope}%
\begin{pgfscope}%
\pgfsys@transformshift{4.280789in}{4.053416in}%
\pgfsys@useobject{currentmarker}{}%
\end{pgfscope}%
\begin{pgfscope}%
\pgfsys@transformshift{4.263655in}{4.107342in}%
\pgfsys@useobject{currentmarker}{}%
\end{pgfscope}%
\begin{pgfscope}%
\pgfsys@transformshift{4.244876in}{4.304763in}%
\pgfsys@useobject{currentmarker}{}%
\end{pgfscope}%
\begin{pgfscope}%
\pgfsys@transformshift{4.222811in}{4.440327in}%
\pgfsys@useobject{currentmarker}{}%
\end{pgfscope}%
\begin{pgfscope}%
\pgfsys@transformshift{4.204738in}{4.418384in}%
\pgfsys@useobject{currentmarker}{}%
\end{pgfscope}%
\begin{pgfscope}%
\pgfsys@transformshift{4.188071in}{4.178689in}%
\pgfsys@useobject{currentmarker}{}%
\end{pgfscope}%
\begin{pgfscope}%
\pgfsys@transformshift{4.162720in}{4.057131in}%
\pgfsys@useobject{currentmarker}{}%
\end{pgfscope}%
\begin{pgfscope}%
\pgfsys@transformshift{4.148872in}{4.042389in}%
\pgfsys@useobject{currentmarker}{}%
\end{pgfscope}%
\begin{pgfscope}%
\pgfsys@transformshift{4.129859in}{4.035900in}%
\pgfsys@useobject{currentmarker}{}%
\end{pgfscope}%
\begin{pgfscope}%
\pgfsys@transformshift{4.108263in}{4.041420in}%
\pgfsys@useobject{currentmarker}{}%
\end{pgfscope}%
\begin{pgfscope}%
\pgfsys@transformshift{4.090424in}{4.064867in}%
\pgfsys@useobject{currentmarker}{}%
\end{pgfscope}%
\begin{pgfscope}%
\pgfsys@transformshift{4.066247in}{4.218706in}%
\pgfsys@useobject{currentmarker}{}%
\end{pgfscope}%
\begin{pgfscope}%
\pgfsys@transformshift{4.053806in}{4.373936in}%
\pgfsys@useobject{currentmarker}{}%
\end{pgfscope}%
\begin{pgfscope}%
\pgfsys@transformshift{4.031741in}{4.438000in}%
\pgfsys@useobject{currentmarker}{}%
\end{pgfscope}%
\begin{pgfscope}%
\pgfsys@transformshift{4.011556in}{4.243618in}%
\pgfsys@useobject{currentmarker}{}%
\end{pgfscope}%
\begin{pgfscope}%
\pgfsys@transformshift{3.993951in}{4.092374in}%
\pgfsys@useobject{currentmarker}{}%
\end{pgfscope}%
\begin{pgfscope}%
\pgfsys@transformshift{3.976581in}{4.049690in}%
\pgfsys@useobject{currentmarker}{}%
\end{pgfscope}%
\begin{pgfscope}%
\pgfsys@transformshift{3.955690in}{4.036969in}%
\pgfsys@useobject{currentmarker}{}%
\end{pgfscope}%
\begin{pgfscope}%
\pgfsys@transformshift{3.936911in}{4.036505in}%
\pgfsys@useobject{currentmarker}{}%
\end{pgfscope}%
\begin{pgfscope}%
\pgfsys@transformshift{3.917898in}{4.047475in}%
\pgfsys@useobject{currentmarker}{}%
\end{pgfscope}%
\begin{pgfscope}%
\pgfsys@transformshift{3.898885in}{4.072601in}%
\pgfsys@useobject{currentmarker}{}%
\end{pgfscope}%
\begin{pgfscope}%
\pgfsys@transformshift{3.879168in}{4.111929in}%
\pgfsys@useobject{currentmarker}{}%
\end{pgfscope}%
\begin{pgfscope}%
\pgfsys@transformshift{3.860858in}{4.278302in}%
\pgfsys@useobject{currentmarker}{}%
\end{pgfscope}%
\begin{pgfscope}%
\pgfsys@transformshift{3.840202in}{4.420295in}%
\pgfsys@useobject{currentmarker}{}%
\end{pgfscope}%
\begin{pgfscope}%
\pgfsys@transformshift{3.819782in}{4.374456in}%
\pgfsys@useobject{currentmarker}{}%
\end{pgfscope}%
\begin{pgfscope}%
\pgfsys@transformshift{3.801707in}{4.149508in}%
\pgfsys@useobject{currentmarker}{}%
\end{pgfscope}%
\begin{pgfscope}%
\pgfsys@transformshift{3.783164in}{4.066193in}%
\pgfsys@useobject{currentmarker}{}%
\end{pgfscope}%
\begin{pgfscope}%
\pgfsys@transformshift{3.763680in}{4.041146in}%
\pgfsys@useobject{currentmarker}{}%
\end{pgfscope}%
\begin{pgfscope}%
\pgfsys@transformshift{3.740207in}{4.035178in}%
\pgfsys@useobject{currentmarker}{}%
\end{pgfscope}%
\begin{pgfscope}%
\pgfsys@transformshift{3.724716in}{4.036945in}%
\pgfsys@useobject{currentmarker}{}%
\end{pgfscope}%
\begin{pgfscope}%
\pgfsys@transformshift{3.704763in}{4.051442in}%
\pgfsys@useobject{currentmarker}{}%
\end{pgfscope}%
\begin{pgfscope}%
\pgfsys@transformshift{3.686689in}{4.105852in}%
\pgfsys@useobject{currentmarker}{}%
\end{pgfscope}%
\begin{pgfscope}%
\pgfsys@transformshift{3.668147in}{4.264146in}%
\pgfsys@useobject{currentmarker}{}%
\end{pgfscope}%
\begin{pgfscope}%
\pgfsys@transformshift{3.648428in}{4.410591in}%
\pgfsys@useobject{currentmarker}{}%
\end{pgfscope}%
\begin{pgfscope}%
\pgfsys@transformshift{3.626129in}{4.370286in}%
\pgfsys@useobject{currentmarker}{}%
\end{pgfscope}%
\begin{pgfscope}%
\pgfsys@transformshift{3.608290in}{4.162730in}%
\pgfsys@useobject{currentmarker}{}%
\end{pgfscope}%
\begin{pgfscope}%
\pgfsys@transformshift{3.590685in}{4.072961in}%
\pgfsys@useobject{currentmarker}{}%
\end{pgfscope}%
\begin{pgfscope}%
\pgfsys@transformshift{3.572612in}{4.046081in}%
\pgfsys@useobject{currentmarker}{}%
\end{pgfscope}%
\begin{pgfscope}%
\pgfsys@transformshift{3.552893in}{4.036061in}%
\pgfsys@useobject{currentmarker}{}%
\end{pgfscope}%
\begin{pgfscope}%
\pgfsys@transformshift{3.534116in}{4.035455in}%
\pgfsys@useobject{currentmarker}{}%
\end{pgfscope}%
\begin{pgfscope}%
\pgfsys@transformshift{3.512755in}{4.043745in}%
\pgfsys@useobject{currentmarker}{}%
\end{pgfscope}%
\begin{pgfscope}%
\pgfsys@transformshift{3.495385in}{4.073507in}%
\pgfsys@useobject{currentmarker}{}%
\end{pgfscope}%
\begin{pgfscope}%
\pgfsys@transformshift{3.473320in}{4.156310in}%
\pgfsys@useobject{currentmarker}{}%
\end{pgfscope}%
\begin{pgfscope}%
\pgfsys@transformshift{3.455012in}{4.313544in}%
\pgfsys@useobject{currentmarker}{}%
\end{pgfscope}%
\begin{pgfscope}%
\pgfsys@transformshift{3.436938in}{4.417035in}%
\pgfsys@useobject{currentmarker}{}%
\end{pgfscope}%
\begin{pgfscope}%
\pgfsys@transformshift{3.415108in}{4.406789in}%
\pgfsys@useobject{currentmarker}{}%
\end{pgfscope}%
\begin{pgfscope}%
\pgfsys@transformshift{3.395390in}{4.171574in}%
\pgfsys@useobject{currentmarker}{}%
\end{pgfscope}%
\begin{pgfscope}%
\pgfsys@transformshift{3.376613in}{4.069121in}%
\pgfsys@useobject{currentmarker}{}%
\end{pgfscope}%
\begin{pgfscope}%
\pgfsys@transformshift{3.359477in}{4.044861in}%
\pgfsys@useobject{currentmarker}{}%
\end{pgfscope}%
\begin{pgfscope}%
\pgfsys@transformshift{3.340229in}{4.036759in}%
\pgfsys@useobject{currentmarker}{}%
\end{pgfscope}%
\begin{pgfscope}%
\pgfsys@transformshift{3.321450in}{4.034857in}%
\pgfsys@useobject{currentmarker}{}%
\end{pgfscope}%
\begin{pgfscope}%
\pgfsys@transformshift{3.302672in}{4.038360in}%
\pgfsys@useobject{currentmarker}{}%
\end{pgfscope}%
\begin{pgfscope}%
\pgfsys@transformshift{3.285069in}{4.048246in}%
\pgfsys@useobject{currentmarker}{}%
\end{pgfscope}%
\begin{pgfscope}%
\pgfsys@transformshift{3.264881in}{4.097986in}%
\pgfsys@useobject{currentmarker}{}%
\end{pgfscope}%
\begin{pgfscope}%
\pgfsys@transformshift{3.242348in}{4.273188in}%
\pgfsys@useobject{currentmarker}{}%
\end{pgfscope}%
\begin{pgfscope}%
\pgfsys@transformshift{3.224743in}{4.384279in}%
\pgfsys@useobject{currentmarker}{}%
\end{pgfscope}%
\begin{pgfscope}%
\pgfsys@transformshift{3.206199in}{4.412832in}%
\pgfsys@useobject{currentmarker}{}%
\end{pgfscope}%
\begin{pgfscope}%
\pgfsys@transformshift{3.188828in}{4.243803in}%
\pgfsys@useobject{currentmarker}{}%
\end{pgfscope}%
\begin{pgfscope}%
\pgfsys@transformshift{3.166295in}{4.097714in}%
\pgfsys@useobject{currentmarker}{}%
\end{pgfscope}%
\begin{pgfscope}%
\pgfsys@transformshift{3.146813in}{4.050707in}%
\pgfsys@useobject{currentmarker}{}%
\end{pgfscope}%
\begin{pgfscope}%
\pgfsys@transformshift{3.128737in}{4.037870in}%
\pgfsys@useobject{currentmarker}{}%
\end{pgfscope}%
\begin{pgfscope}%
\pgfsys@transformshift{3.109960in}{4.034733in}%
\pgfsys@useobject{currentmarker}{}%
\end{pgfscope}%
\begin{pgfscope}%
\pgfsys@transformshift{3.091885in}{4.036696in}%
\pgfsys@useobject{currentmarker}{}%
\end{pgfscope}%
\begin{pgfscope}%
\pgfsys@transformshift{3.072403in}{4.040872in}%
\pgfsys@useobject{currentmarker}{}%
\end{pgfscope}%
\begin{pgfscope}%
\pgfsys@transformshift{3.051981in}{4.068886in}%
\pgfsys@useobject{currentmarker}{}%
\end{pgfscope}%
\begin{pgfscope}%
\pgfsys@transformshift{3.030151in}{4.162327in}%
\pgfsys@useobject{currentmarker}{}%
\end{pgfscope}%
\begin{pgfscope}%
\pgfsys@transformshift{3.013954in}{4.305953in}%
\pgfsys@useobject{currentmarker}{}%
\end{pgfscope}%
\begin{pgfscope}%
\pgfsys@transformshift{2.994707in}{4.412778in}%
\pgfsys@useobject{currentmarker}{}%
\end{pgfscope}%
\begin{pgfscope}%
\pgfsys@transformshift{2.977573in}{4.356444in}%
\pgfsys@useobject{currentmarker}{}%
\end{pgfscope}%
\begin{pgfscope}%
\pgfsys@transformshift{2.958091in}{4.139923in}%
\pgfsys@useobject{currentmarker}{}%
\end{pgfscope}%
\begin{pgfscope}%
\pgfsys@transformshift{2.936495in}{4.059112in}%
\pgfsys@useobject{currentmarker}{}%
\end{pgfscope}%
\begin{pgfscope}%
\pgfsys@transformshift{2.917247in}{4.041414in}%
\pgfsys@useobject{currentmarker}{}%
\end{pgfscope}%
\begin{pgfscope}%
\pgfsys@transformshift{2.898703in}{4.035435in}%
\pgfsys@useobject{currentmarker}{}%
\end{pgfscope}%
\begin{pgfscope}%
\pgfsys@transformshift{2.879926in}{4.034900in}%
\pgfsys@useobject{currentmarker}{}%
\end{pgfscope}%
\begin{pgfscope}%
\pgfsys@transformshift{2.858096in}{4.035740in}%
\pgfsys@useobject{currentmarker}{}%
\end{pgfscope}%
\begin{pgfscope}%
\pgfsys@transformshift{2.839786in}{4.039526in}%
\pgfsys@useobject{currentmarker}{}%
\end{pgfscope}%
\begin{pgfscope}%
\pgfsys@transformshift{2.821243in}{4.052449in}%
\pgfsys@useobject{currentmarker}{}%
\end{pgfscope}%
\begin{pgfscope}%
\pgfsys@transformshift{2.804107in}{4.092645in}%
\pgfsys@useobject{currentmarker}{}%
\end{pgfscope}%
\begin{pgfscope}%
\pgfsys@transformshift{2.784391in}{4.255058in}%
\pgfsys@useobject{currentmarker}{}%
\end{pgfscope}%
\begin{pgfscope}%
\pgfsys@transformshift{2.762795in}{4.404317in}%
\pgfsys@useobject{currentmarker}{}%
\end{pgfscope}%
\begin{pgfscope}%
\pgfsys@transformshift{2.743078in}{4.404247in}%
\pgfsys@useobject{currentmarker}{}%
\end{pgfscope}%
\begin{pgfscope}%
\pgfsys@transformshift{2.725003in}{4.211421in}%
\pgfsys@useobject{currentmarker}{}%
\end{pgfscope}%
\begin{pgfscope}%
\pgfsys@transformshift{2.705990in}{4.083295in}%
\pgfsys@useobject{currentmarker}{}%
\end{pgfscope}%
\begin{pgfscope}%
\pgfsys@transformshift{2.687682in}{4.048888in}%
\pgfsys@useobject{currentmarker}{}%
\end{pgfscope}%
\begin{pgfscope}%
\pgfsys@transformshift{2.667729in}{4.038635in}%
\pgfsys@useobject{currentmarker}{}%
\end{pgfscope}%
\begin{pgfscope}%
\pgfsys@transformshift{2.649655in}{4.034720in}%
\pgfsys@useobject{currentmarker}{}%
\end{pgfscope}%
\begin{pgfscope}%
\pgfsys@transformshift{2.628764in}{4.036884in}%
\pgfsys@useobject{currentmarker}{}%
\end{pgfscope}%
\begin{pgfscope}%
\pgfsys@transformshift{2.608108in}{4.046200in}%
\pgfsys@useobject{currentmarker}{}%
\end{pgfscope}%
\begin{pgfscope}%
\pgfsys@transformshift{2.589095in}{4.072722in}%
\pgfsys@useobject{currentmarker}{}%
\end{pgfscope}%
\begin{pgfscope}%
\pgfsys@transformshift{2.574307in}{4.134899in}%
\pgfsys@useobject{currentmarker}{}%
\end{pgfscope}%
\begin{pgfscope}%
\pgfsys@transformshift{2.552008in}{4.284300in}%
\pgfsys@useobject{currentmarker}{}%
\end{pgfscope}%
\begin{pgfscope}%
\pgfsys@transformshift{2.532760in}{4.362379in}%
\pgfsys@useobject{currentmarker}{}%
\end{pgfscope}%
\begin{pgfscope}%
\pgfsys@transformshift{2.515156in}{4.406307in}%
\pgfsys@useobject{currentmarker}{}%
\end{pgfscope}%
\begin{pgfscope}%
\pgfsys@transformshift{2.495439in}{4.351061in}%
\pgfsys@useobject{currentmarker}{}%
\end{pgfscope}%
\begin{pgfscope}%
\pgfsys@transformshift{2.474312in}{4.154741in}%
\pgfsys@useobject{currentmarker}{}%
\end{pgfscope}%
\begin{pgfscope}%
\pgfsys@transformshift{2.457178in}{4.111292in}%
\pgfsys@useobject{currentmarker}{}%
\end{pgfscope}%
\begin{pgfscope}%
\pgfsys@transformshift{2.438399in}{4.055003in}%
\pgfsys@useobject{currentmarker}{}%
\end{pgfscope}%
\begin{pgfscope}%
\pgfsys@transformshift{2.417038in}{4.038464in}%
\pgfsys@useobject{currentmarker}{}%
\end{pgfscope}%
\begin{pgfscope}%
\pgfsys@transformshift{2.398261in}{4.035367in}%
\pgfsys@useobject{currentmarker}{}%
\end{pgfscope}%
\begin{pgfscope}%
\pgfsys@transformshift{2.381125in}{4.035225in}%
\pgfsys@useobject{currentmarker}{}%
\end{pgfscope}%
\begin{pgfscope}%
\pgfsys@transformshift{2.357886in}{4.042683in}%
\pgfsys@useobject{currentmarker}{}%
\end{pgfscope}%
\begin{pgfscope}%
\pgfsys@transformshift{2.339813in}{4.060546in}%
\pgfsys@useobject{currentmarker}{}%
\end{pgfscope}%
\begin{pgfscope}%
\pgfsys@transformshift{2.323148in}{4.144788in}%
\pgfsys@useobject{currentmarker}{}%
\end{pgfscope}%
\begin{pgfscope}%
\pgfsys@transformshift{2.300378in}{4.330258in}%
\pgfsys@useobject{currentmarker}{}%
\end{pgfscope}%
\begin{pgfscope}%
\pgfsys@transformshift{2.282304in}{4.407017in}%
\pgfsys@useobject{currentmarker}{}%
\end{pgfscope}%
\begin{pgfscope}%
\pgfsys@transformshift{2.265168in}{4.386294in}%
\pgfsys@useobject{currentmarker}{}%
\end{pgfscope}%
\begin{pgfscope}%
\pgfsys@transformshift{2.241226in}{4.134268in}%
\pgfsys@useobject{currentmarker}{}%
\end{pgfscope}%
\begin{pgfscope}%
\pgfsys@transformshift{2.229021in}{4.097834in}%
\pgfsys@useobject{currentmarker}{}%
\end{pgfscope}%
\begin{pgfscope}%
\pgfsys@transformshift{2.208599in}{4.057554in}%
\pgfsys@useobject{currentmarker}{}%
\end{pgfscope}%
\begin{pgfscope}%
\pgfsys@transformshift{2.185361in}{4.039300in}%
\pgfsys@useobject{currentmarker}{}%
\end{pgfscope}%
\begin{pgfscope}%
\pgfsys@transformshift{2.170338in}{4.035515in}%
\pgfsys@useobject{currentmarker}{}%
\end{pgfscope}%
\begin{pgfscope}%
\pgfsys@transformshift{2.148274in}{4.036022in}%
\pgfsys@useobject{currentmarker}{}%
\end{pgfscope}%
\begin{pgfscope}%
\pgfsys@transformshift{2.129729in}{4.039910in}%
\pgfsys@useobject{currentmarker}{}%
\end{pgfscope}%
\begin{pgfscope}%
\pgfsys@transformshift{2.110247in}{4.050486in}%
\pgfsys@useobject{currentmarker}{}%
\end{pgfscope}%
\begin{pgfscope}%
\pgfsys@transformshift{2.090060in}{4.088612in}%
\pgfsys@useobject{currentmarker}{}%
\end{pgfscope}%
\begin{pgfscope}%
\pgfsys@transformshift{2.071517in}{4.205271in}%
\pgfsys@useobject{currentmarker}{}%
\end{pgfscope}%
\begin{pgfscope}%
\pgfsys@transformshift{2.052504in}{4.338462in}%
\pgfsys@useobject{currentmarker}{}%
\end{pgfscope}%
\begin{pgfscope}%
\pgfsys@transformshift{2.034665in}{4.408437in}%
\pgfsys@useobject{currentmarker}{}%
\end{pgfscope}%
\begin{pgfscope}%
\pgfsys@transformshift{2.016589in}{4.360346in}%
\pgfsys@useobject{currentmarker}{}%
\end{pgfscope}%
\begin{pgfscope}%
\pgfsys@transformshift{1.993587in}{4.211894in}%
\pgfsys@useobject{currentmarker}{}%
\end{pgfscope}%
\begin{pgfscope}%
\pgfsys@transformshift{1.976686in}{4.109376in}%
\pgfsys@useobject{currentmarker}{}%
\end{pgfscope}%
\begin{pgfscope}%
\pgfsys@transformshift{1.956735in}{4.055219in}%
\pgfsys@useobject{currentmarker}{}%
\end{pgfscope}%
\begin{pgfscope}%
\pgfsys@transformshift{1.938190in}{4.041152in}%
\pgfsys@useobject{currentmarker}{}%
\end{pgfscope}%
\begin{pgfscope}%
\pgfsys@transformshift{1.918239in}{4.035841in}%
\pgfsys@useobject{currentmarker}{}%
\end{pgfscope}%
\begin{pgfscope}%
\pgfsys@transformshift{1.898052in}{4.035193in}%
\pgfsys@useobject{currentmarker}{}%
\end{pgfscope}%
\begin{pgfscope}%
\pgfsys@transformshift{1.879508in}{4.036578in}%
\pgfsys@useobject{currentmarker}{}%
\end{pgfscope}%
\begin{pgfscope}%
\pgfsys@transformshift{1.860731in}{4.042912in}%
\pgfsys@useobject{currentmarker}{}%
\end{pgfscope}%
\begin{pgfscope}%
\pgfsys@transformshift{1.839604in}{4.068642in}%
\pgfsys@useobject{currentmarker}{}%
\end{pgfscope}%
\begin{pgfscope}%
\pgfsys@transformshift{1.824113in}{4.134978in}%
\pgfsys@useobject{currentmarker}{}%
\end{pgfscope}%
\begin{pgfscope}%
\pgfsys@transformshift{1.802517in}{4.291805in}%
\pgfsys@useobject{currentmarker}{}%
\end{pgfscope}%
\begin{pgfscope}%
\pgfsys@transformshift{1.784209in}{4.403094in}%
\pgfsys@useobject{currentmarker}{}%
\end{pgfscope}%
\begin{pgfscope}%
\pgfsys@transformshift{1.764727in}{4.412704in}%
\pgfsys@useobject{currentmarker}{}%
\end{pgfscope}%
\begin{pgfscope}%
\pgfsys@transformshift{1.743365in}{4.258278in}%
\pgfsys@useobject{currentmarker}{}%
\end{pgfscope}%
\begin{pgfscope}%
\pgfsys@transformshift{1.727874in}{4.142545in}%
\pgfsys@useobject{currentmarker}{}%
\end{pgfscope}%
\begin{pgfscope}%
\pgfsys@transformshift{1.706513in}{4.073882in}%
\pgfsys@useobject{currentmarker}{}%
\end{pgfscope}%
\begin{pgfscope}%
\pgfsys@transformshift{1.688205in}{4.047339in}%
\pgfsys@useobject{currentmarker}{}%
\end{pgfscope}%
\begin{pgfscope}%
\pgfsys@transformshift{1.668957in}{4.039668in}%
\pgfsys@useobject{currentmarker}{}%
\end{pgfscope}%
\begin{pgfscope}%
\pgfsys@transformshift{1.651587in}{4.035545in}%
\pgfsys@useobject{currentmarker}{}%
\end{pgfscope}%
\begin{pgfscope}%
\pgfsys@transformshift{1.629991in}{4.035649in}%
\pgfsys@useobject{currentmarker}{}%
\end{pgfscope}%
\begin{pgfscope}%
\pgfsys@transformshift{1.610978in}{4.040130in}%
\pgfsys@useobject{currentmarker}{}%
\end{pgfscope}%
\begin{pgfscope}%
\pgfsys@transformshift{1.593373in}{4.044529in}%
\pgfsys@useobject{currentmarker}{}%
\end{pgfscope}%
\begin{pgfscope}%
\pgfsys@transformshift{1.571308in}{4.069314in}%
\pgfsys@useobject{currentmarker}{}%
\end{pgfscope}%
\begin{pgfscope}%
\pgfsys@transformshift{1.553938in}{4.141280in}%
\pgfsys@useobject{currentmarker}{}%
\end{pgfscope}%
\begin{pgfscope}%
\pgfsys@transformshift{1.534690in}{4.290173in}%
\pgfsys@useobject{currentmarker}{}%
\end{pgfscope}%
\begin{pgfscope}%
\pgfsys@transformshift{1.516382in}{4.385364in}%
\pgfsys@useobject{currentmarker}{}%
\end{pgfscope}%
\begin{pgfscope}%
\pgfsys@transformshift{1.495257in}{4.423206in}%
\pgfsys@useobject{currentmarker}{}%
\end{pgfscope}%
\begin{pgfscope}%
\pgfsys@transformshift{1.479530in}{4.375322in}%
\pgfsys@useobject{currentmarker}{}%
\end{pgfscope}%
\begin{pgfscope}%
\pgfsys@transformshift{1.457465in}{4.195796in}%
\pgfsys@useobject{currentmarker}{}%
\end{pgfscope}%
\begin{pgfscope}%
\pgfsys@transformshift{1.436104in}{4.097132in}%
\pgfsys@useobject{currentmarker}{}%
\end{pgfscope}%
\begin{pgfscope}%
\pgfsys@transformshift{1.421081in}{4.065766in}%
\pgfsys@useobject{currentmarker}{}%
\end{pgfscope}%
\begin{pgfscope}%
\pgfsys@transformshift{1.398782in}{4.046146in}%
\pgfsys@useobject{currentmarker}{}%
\end{pgfscope}%
\begin{pgfscope}%
\pgfsys@transformshift{1.381412in}{4.038081in}%
\pgfsys@useobject{currentmarker}{}%
\end{pgfscope}%
\begin{pgfscope}%
\pgfsys@transformshift{1.362635in}{4.035858in}%
\pgfsys@useobject{currentmarker}{}%
\end{pgfscope}%
\begin{pgfscope}%
\pgfsys@transformshift{1.340571in}{4.035908in}%
\pgfsys@useobject{currentmarker}{}%
\end{pgfscope}%
\begin{pgfscope}%
\pgfsys@transformshift{1.322729in}{4.040910in}%
\pgfsys@useobject{currentmarker}{}%
\end{pgfscope}%
\begin{pgfscope}%
\pgfsys@transformshift{1.306299in}{4.049145in}%
\pgfsys@useobject{currentmarker}{}%
\end{pgfscope}%
\begin{pgfscope}%
\pgfsys@transformshift{1.284470in}{4.088525in}%
\pgfsys@useobject{currentmarker}{}%
\end{pgfscope}%
\begin{pgfscope}%
\pgfsys@transformshift{1.266160in}{4.140315in}%
\pgfsys@useobject{currentmarker}{}%
\end{pgfscope}%
\begin{pgfscope}%
\pgfsys@transformshift{1.248321in}{4.258489in}%
\pgfsys@useobject{currentmarker}{}%
\end{pgfscope}%
\begin{pgfscope}%
\pgfsys@transformshift{1.226725in}{4.394462in}%
\pgfsys@useobject{currentmarker}{}%
\end{pgfscope}%
\begin{pgfscope}%
\pgfsys@transformshift{1.208886in}{4.428155in}%
\pgfsys@useobject{currentmarker}{}%
\end{pgfscope}%
\begin{pgfscope}%
\pgfsys@transformshift{1.190344in}{4.424265in}%
\pgfsys@useobject{currentmarker}{}%
\end{pgfscope}%
\begin{pgfscope}%
\pgfsys@transformshift{1.168748in}{4.248027in}%
\pgfsys@useobject{currentmarker}{}%
\end{pgfscope}%
\begin{pgfscope}%
\pgfsys@transformshift{1.150909in}{4.140572in}%
\pgfsys@useobject{currentmarker}{}%
\end{pgfscope}%
\begin{pgfscope}%
\pgfsys@transformshift{1.130018in}{4.069973in}%
\pgfsys@useobject{currentmarker}{}%
\end{pgfscope}%
\begin{pgfscope}%
\pgfsys@transformshift{1.111708in}{4.047073in}%
\pgfsys@useobject{currentmarker}{}%
\end{pgfscope}%
\begin{pgfscope}%
\pgfsys@transformshift{1.087061in}{4.039347in}%
\pgfsys@useobject{currentmarker}{}%
\end{pgfscope}%
\begin{pgfscope}%
\pgfsys@transformshift{1.074621in}{4.036403in}%
\pgfsys@useobject{currentmarker}{}%
\end{pgfscope}%
\begin{pgfscope}%
\pgfsys@transformshift{1.053025in}{4.037250in}%
\pgfsys@useobject{currentmarker}{}%
\end{pgfscope}%
\begin{pgfscope}%
\pgfsys@transformshift{1.035186in}{4.040080in}%
\pgfsys@useobject{currentmarker}{}%
\end{pgfscope}%
\begin{pgfscope}%
\pgfsys@transformshift{1.016173in}{4.052512in}%
\pgfsys@useobject{currentmarker}{}%
\end{pgfscope}%
\begin{pgfscope}%
\pgfsys@transformshift{0.995282in}{4.075383in}%
\pgfsys@useobject{currentmarker}{}%
\end{pgfscope}%
\begin{pgfscope}%
\pgfsys@transformshift{0.977912in}{4.139618in}%
\pgfsys@useobject{currentmarker}{}%
\end{pgfscope}%
\begin{pgfscope}%
\pgfsys@transformshift{0.958430in}{4.285096in}%
\pgfsys@useobject{currentmarker}{}%
\end{pgfscope}%
\begin{pgfscope}%
\pgfsys@transformshift{0.939182in}{4.369264in}%
\pgfsys@useobject{currentmarker}{}%
\end{pgfscope}%
\begin{pgfscope}%
\pgfsys@transformshift{0.914066in}{4.443148in}%
\pgfsys@useobject{currentmarker}{}%
\end{pgfscope}%
\begin{pgfscope}%
\pgfsys@transformshift{0.895993in}{4.425285in}%
\pgfsys@useobject{currentmarker}{}%
\end{pgfscope}%
\begin{pgfscope}%
\pgfsys@transformshift{0.880500in}{4.333795in}%
\pgfsys@useobject{currentmarker}{}%
\end{pgfscope}%
\begin{pgfscope}%
\pgfsys@transformshift{0.861486in}{4.159333in}%
\pgfsys@useobject{currentmarker}{}%
\end{pgfscope}%
\begin{pgfscope}%
\pgfsys@transformshift{0.843178in}{4.079638in}%
\pgfsys@useobject{currentmarker}{}%
\end{pgfscope}%
\begin{pgfscope}%
\pgfsys@transformshift{0.825573in}{4.056910in}%
\pgfsys@useobject{currentmarker}{}%
\end{pgfscope}%
\begin{pgfscope}%
\pgfsys@transformshift{0.806091in}{4.042535in}%
\pgfsys@useobject{currentmarker}{}%
\end{pgfscope}%
\begin{pgfscope}%
\pgfsys@transformshift{0.782618in}{4.153707in}%
\pgfsys@useobject{currentmarker}{}%
\end{pgfscope}%
\begin{pgfscope}%
\pgfsys@transformshift{0.767830in}{4.118699in}%
\pgfsys@useobject{currentmarker}{}%
\end{pgfscope}%
\begin{pgfscope}%
\pgfsys@transformshift{0.747878in}{4.061202in}%
\pgfsys@useobject{currentmarker}{}%
\end{pgfscope}%
\begin{pgfscope}%
\pgfsys@transformshift{0.729101in}{4.046443in}%
\pgfsys@useobject{currentmarker}{}%
\end{pgfscope}%
\begin{pgfscope}%
\pgfsys@transformshift{0.708679in}{4.037191in}%
\pgfsys@useobject{currentmarker}{}%
\end{pgfscope}%
\begin{pgfscope}%
\pgfsys@transformshift{0.687318in}{4.038358in}%
\pgfsys@useobject{currentmarker}{}%
\end{pgfscope}%
\begin{pgfscope}%
\pgfsys@transformshift{0.670418in}{4.044527in}%
\pgfsys@useobject{currentmarker}{}%
\end{pgfscope}%
\begin{pgfscope}%
\pgfsys@transformshift{0.649525in}{4.070901in}%
\pgfsys@useobject{currentmarker}{}%
\end{pgfscope}%
\begin{pgfscope}%
\pgfsys@transformshift{0.649996in}{4.067826in}%
\pgfsys@useobject{currentmarker}{}%
\end{pgfscope}%
\begin{pgfscope}%
\pgfsys@transformshift{0.657742in}{4.053540in}%
\pgfsys@useobject{currentmarker}{}%
\end{pgfscope}%
\begin{pgfscope}%
\pgfsys@transformshift{0.676990in}{4.038412in}%
\pgfsys@useobject{currentmarker}{}%
\end{pgfscope}%
\begin{pgfscope}%
\pgfsys@transformshift{0.696003in}{4.037781in}%
\pgfsys@useobject{currentmarker}{}%
\end{pgfscope}%
\begin{pgfscope}%
\pgfsys@transformshift{0.711730in}{4.047945in}%
\pgfsys@useobject{currentmarker}{}%
\end{pgfscope}%
\begin{pgfscope}%
\pgfsys@transformshift{0.734733in}{4.103006in}%
\pgfsys@useobject{currentmarker}{}%
\end{pgfscope}%
\begin{pgfscope}%
\pgfsys@transformshift{0.753511in}{4.278948in}%
\pgfsys@useobject{currentmarker}{}%
\end{pgfscope}%
\begin{pgfscope}%
\pgfsys@transformshift{0.773228in}{4.448179in}%
\pgfsys@useobject{currentmarker}{}%
\end{pgfscope}%
\begin{pgfscope}%
\pgfsys@transformshift{0.789661in}{4.083310in}%
\pgfsys@useobject{currentmarker}{}%
\end{pgfscope}%
\begin{pgfscope}%
\pgfsys@transformshift{0.808908in}{4.044836in}%
\pgfsys@useobject{currentmarker}{}%
\end{pgfscope}%
\begin{pgfscope}%
\pgfsys@transformshift{0.830268in}{4.035930in}%
\pgfsys@useobject{currentmarker}{}%
\end{pgfscope}%
\begin{pgfscope}%
\pgfsys@transformshift{0.850221in}{4.042588in}%
\pgfsys@useobject{currentmarker}{}%
\end{pgfscope}%
\begin{pgfscope}%
\pgfsys@transformshift{0.868060in}{4.063921in}%
\pgfsys@useobject{currentmarker}{}%
\end{pgfscope}%
\begin{pgfscope}%
\pgfsys@transformshift{0.887542in}{4.157410in}%
\pgfsys@useobject{currentmarker}{}%
\end{pgfscope}%
\begin{pgfscope}%
\pgfsys@transformshift{0.911015in}{4.426254in}%
\pgfsys@useobject{currentmarker}{}%
\end{pgfscope}%
\begin{pgfscope}%
\pgfsys@transformshift{0.925100in}{4.421669in}%
\pgfsys@useobject{currentmarker}{}%
\end{pgfscope}%
\begin{pgfscope}%
\pgfsys@transformshift{0.944347in}{4.280702in}%
\pgfsys@useobject{currentmarker}{}%
\end{pgfscope}%
\begin{pgfscope}%
\pgfsys@transformshift{0.962421in}{4.111748in}%
\pgfsys@useobject{currentmarker}{}%
\end{pgfscope}%
\begin{pgfscope}%
\pgfsys@transformshift{0.984954in}{4.046764in}%
\pgfsys@useobject{currentmarker}{}%
\end{pgfscope}%
\begin{pgfscope}%
\pgfsys@transformshift{1.001856in}{4.036412in}%
\pgfsys@useobject{currentmarker}{}%
\end{pgfscope}%
\begin{pgfscope}%
\pgfsys@transformshift{1.021104in}{4.037594in}%
\pgfsys@useobject{currentmarker}{}%
\end{pgfscope}%
\begin{pgfscope}%
\pgfsys@transformshift{1.040586in}{4.051209in}%
\pgfsys@useobject{currentmarker}{}%
\end{pgfscope}%
\begin{pgfscope}%
\pgfsys@transformshift{1.058894in}{4.093514in}%
\pgfsys@useobject{currentmarker}{}%
\end{pgfscope}%
\begin{pgfscope}%
\pgfsys@transformshift{1.078847in}{4.267356in}%
\pgfsys@useobject{currentmarker}{}%
\end{pgfscope}%
\begin{pgfscope}%
\pgfsys@transformshift{1.102085in}{4.432584in}%
\pgfsys@useobject{currentmarker}{}%
\end{pgfscope}%
\begin{pgfscope}%
\pgfsys@transformshift{1.118047in}{4.324785in}%
\pgfsys@useobject{currentmarker}{}%
\end{pgfscope}%
\begin{pgfscope}%
\pgfsys@transformshift{1.138467in}{4.153868in}%
\pgfsys@useobject{currentmarker}{}%
\end{pgfscope}%
\begin{pgfscope}%
\pgfsys@transformshift{1.154663in}{4.067764in}%
\pgfsys@useobject{currentmarker}{}%
\end{pgfscope}%
\begin{pgfscope}%
\pgfsys@transformshift{1.175319in}{4.039808in}%
\pgfsys@useobject{currentmarker}{}%
\end{pgfscope}%
\begin{pgfscope}%
\pgfsys@transformshift{1.196915in}{4.035290in}%
\pgfsys@useobject{currentmarker}{}%
\end{pgfscope}%
\begin{pgfscope}%
\pgfsys@transformshift{1.215694in}{4.039911in}%
\pgfsys@useobject{currentmarker}{}%
\end{pgfscope}%
\begin{pgfscope}%
\pgfsys@transformshift{1.234237in}{4.051316in}%
\pgfsys@useobject{currentmarker}{}%
\end{pgfscope}%
\begin{pgfscope}%
\pgfsys@transformshift{1.252547in}{4.104812in}%
\pgfsys@useobject{currentmarker}{}%
\end{pgfscope}%
\begin{pgfscope}%
\pgfsys@transformshift{1.271089in}{4.312334in}%
\pgfsys@useobject{currentmarker}{}%
\end{pgfscope}%
\begin{pgfscope}%
\pgfsys@transformshift{1.291981in}{4.427653in}%
\pgfsys@useobject{currentmarker}{}%
\end{pgfscope}%
\begin{pgfscope}%
\pgfsys@transformshift{1.314046in}{4.318421in}%
\pgfsys@useobject{currentmarker}{}%
\end{pgfscope}%
\begin{pgfscope}%
\pgfsys@transformshift{1.333294in}{4.149067in}%
\pgfsys@useobject{currentmarker}{}%
\end{pgfscope}%
\begin{pgfscope}%
\pgfsys@transformshift{1.349019in}{4.064928in}%
\pgfsys@useobject{currentmarker}{}%
\end{pgfscope}%
\begin{pgfscope}%
\pgfsys@transformshift{1.367329in}{4.042651in}%
\pgfsys@useobject{currentmarker}{}%
\end{pgfscope}%
\begin{pgfscope}%
\pgfsys@transformshift{1.386106in}{4.035643in}%
\pgfsys@useobject{currentmarker}{}%
\end{pgfscope}%
\begin{pgfscope}%
\pgfsys@transformshift{1.407936in}{4.036686in}%
\pgfsys@useobject{currentmarker}{}%
\end{pgfscope}%
\begin{pgfscope}%
\pgfsys@transformshift{1.426950in}{4.044739in}%
\pgfsys@useobject{currentmarker}{}%
\end{pgfscope}%
\begin{pgfscope}%
\pgfsys@transformshift{1.443851in}{4.064633in}%
\pgfsys@useobject{currentmarker}{}%
\end{pgfscope}%
\begin{pgfscope}%
\pgfsys@transformshift{1.464273in}{4.157911in}%
\pgfsys@useobject{currentmarker}{}%
\end{pgfscope}%
\begin{pgfscope}%
\pgfsys@transformshift{1.481641in}{4.345654in}%
\pgfsys@useobject{currentmarker}{}%
\end{pgfscope}%
\begin{pgfscope}%
\pgfsys@transformshift{1.501829in}{4.420628in}%
\pgfsys@useobject{currentmarker}{}%
\end{pgfscope}%
\begin{pgfscope}%
\pgfsys@transformshift{1.520608in}{4.343853in}%
\pgfsys@useobject{currentmarker}{}%
\end{pgfscope}%
\begin{pgfscope}%
\pgfsys@transformshift{1.539621in}{4.223252in}%
\pgfsys@useobject{currentmarker}{}%
\end{pgfscope}%
\begin{pgfscope}%
\pgfsys@transformshift{1.562623in}{4.068873in}%
\pgfsys@useobject{currentmarker}{}%
\end{pgfscope}%
\begin{pgfscope}%
\pgfsys@transformshift{1.579290in}{4.048730in}%
\pgfsys@useobject{currentmarker}{}%
\end{pgfscope}%
\begin{pgfscope}%
\pgfsys@transformshift{1.598772in}{4.037548in}%
\pgfsys@useobject{currentmarker}{}%
\end{pgfscope}%
\begin{pgfscope}%
\pgfsys@transformshift{1.619663in}{4.035156in}%
\pgfsys@useobject{currentmarker}{}%
\end{pgfscope}%
\begin{pgfscope}%
\pgfsys@transformshift{1.637737in}{4.038694in}%
\pgfsys@useobject{currentmarker}{}%
\end{pgfscope}%
\begin{pgfscope}%
\pgfsys@transformshift{1.656281in}{4.051364in}%
\pgfsys@useobject{currentmarker}{}%
\end{pgfscope}%
\begin{pgfscope}%
\pgfsys@transformshift{1.673651in}{4.095180in}%
\pgfsys@useobject{currentmarker}{}%
\end{pgfscope}%
\begin{pgfscope}%
\pgfsys@transformshift{1.691959in}{4.154034in}%
\pgfsys@useobject{currentmarker}{}%
\end{pgfscope}%
\begin{pgfscope}%
\pgfsys@transformshift{1.712381in}{4.361236in}%
\pgfsys@useobject{currentmarker}{}%
\end{pgfscope}%
\begin{pgfscope}%
\pgfsys@transformshift{1.731629in}{4.416082in}%
\pgfsys@useobject{currentmarker}{}%
\end{pgfscope}%
\begin{pgfscope}%
\pgfsys@transformshift{1.752990in}{4.294406in}%
\pgfsys@useobject{currentmarker}{}%
\end{pgfscope}%
\begin{pgfscope}%
\pgfsys@transformshift{1.771064in}{4.153231in}%
\pgfsys@useobject{currentmarker}{}%
\end{pgfscope}%
\begin{pgfscope}%
\pgfsys@transformshift{1.789137in}{4.068837in}%
\pgfsys@useobject{currentmarker}{}%
\end{pgfscope}%
\begin{pgfscope}%
\pgfsys@transformshift{1.809794in}{4.049758in}%
\pgfsys@useobject{currentmarker}{}%
\end{pgfscope}%
\begin{pgfscope}%
\pgfsys@transformshift{1.830215in}{4.036982in}%
\pgfsys@useobject{currentmarker}{}%
\end{pgfscope}%
\begin{pgfscope}%
\pgfsys@transformshift{1.848289in}{4.034900in}%
\pgfsys@useobject{currentmarker}{}%
\end{pgfscope}%
\begin{pgfscope}%
\pgfsys@transformshift{1.865659in}{4.037071in}%
\pgfsys@useobject{currentmarker}{}%
\end{pgfscope}%
\begin{pgfscope}%
\pgfsys@transformshift{1.887724in}{4.045083in}%
\pgfsys@useobject{currentmarker}{}%
\end{pgfscope}%
\begin{pgfscope}%
\pgfsys@transformshift{1.904155in}{4.069189in}%
\pgfsys@useobject{currentmarker}{}%
\end{pgfscope}%
\begin{pgfscope}%
\pgfsys@transformshift{1.922699in}{4.161587in}%
\pgfsys@useobject{currentmarker}{}%
\end{pgfscope}%
\begin{pgfscope}%
\pgfsys@transformshift{1.941712in}{4.157740in}%
\pgfsys@useobject{currentmarker}{}%
\end{pgfscope}%
\begin{pgfscope}%
\pgfsys@transformshift{1.964480in}{4.399520in}%
\pgfsys@useobject{currentmarker}{}%
\end{pgfscope}%
\begin{pgfscope}%
\pgfsys@transformshift{1.984197in}{4.401841in}%
\pgfsys@useobject{currentmarker}{}%
\end{pgfscope}%
\begin{pgfscope}%
\pgfsys@transformshift{1.999924in}{4.329805in}%
\pgfsys@useobject{currentmarker}{}%
\end{pgfscope}%
\begin{pgfscope}%
\pgfsys@transformshift{2.021286in}{4.131320in}%
\pgfsys@useobject{currentmarker}{}%
\end{pgfscope}%
\begin{pgfscope}%
\pgfsys@transformshift{2.042176in}{4.059110in}%
\pgfsys@useobject{currentmarker}{}%
\end{pgfscope}%
\begin{pgfscope}%
\pgfsys@transformshift{2.058372in}{4.040656in}%
\pgfsys@useobject{currentmarker}{}%
\end{pgfscope}%
\begin{pgfscope}%
\pgfsys@transformshift{2.079263in}{4.035232in}%
\pgfsys@useobject{currentmarker}{}%
\end{pgfscope}%
\begin{pgfscope}%
\pgfsys@transformshift{2.096633in}{4.035093in}%
\pgfsys@useobject{currentmarker}{}%
\end{pgfscope}%
\begin{pgfscope}%
\pgfsys@transformshift{2.117290in}{4.039076in}%
\pgfsys@useobject{currentmarker}{}%
\end{pgfscope}%
\begin{pgfscope}%
\pgfsys@transformshift{2.135129in}{4.051582in}%
\pgfsys@useobject{currentmarker}{}%
\end{pgfscope}%
\begin{pgfscope}%
\pgfsys@transformshift{2.155316in}{4.083626in}%
\pgfsys@useobject{currentmarker}{}%
\end{pgfscope}%
\begin{pgfscope}%
\pgfsys@transformshift{2.174329in}{4.201859in}%
\pgfsys@useobject{currentmarker}{}%
\end{pgfscope}%
\begin{pgfscope}%
\pgfsys@transformshift{2.195220in}{4.399917in}%
\pgfsys@useobject{currentmarker}{}%
\end{pgfscope}%
\begin{pgfscope}%
\pgfsys@transformshift{2.213059in}{4.405824in}%
\pgfsys@useobject{currentmarker}{}%
\end{pgfscope}%
\begin{pgfscope}%
\pgfsys@transformshift{2.231602in}{4.348530in}%
\pgfsys@useobject{currentmarker}{}%
\end{pgfscope}%
\begin{pgfscope}%
\pgfsys@transformshift{2.252258in}{4.153661in}%
\pgfsys@useobject{currentmarker}{}%
\end{pgfscope}%
\begin{pgfscope}%
\pgfsys@transformshift{2.269394in}{4.076653in}%
\pgfsys@useobject{currentmarker}{}%
\end{pgfscope}%
\begin{pgfscope}%
\pgfsys@transformshift{2.290989in}{4.041166in}%
\pgfsys@useobject{currentmarker}{}%
\end{pgfscope}%
\begin{pgfscope}%
\pgfsys@transformshift{2.311646in}{4.036495in}%
\pgfsys@useobject{currentmarker}{}%
\end{pgfscope}%
\begin{pgfscope}%
\pgfsys@transformshift{2.328545in}{4.034595in}%
\pgfsys@useobject{currentmarker}{}%
\end{pgfscope}%
\begin{pgfscope}%
\pgfsys@transformshift{2.348264in}{4.037817in}%
\pgfsys@useobject{currentmarker}{}%
\end{pgfscope}%
\begin{pgfscope}%
\pgfsys@transformshift{2.368449in}{4.052656in}%
\pgfsys@useobject{currentmarker}{}%
\end{pgfscope}%
\begin{pgfscope}%
\pgfsys@transformshift{2.387462in}{4.082156in}%
\pgfsys@useobject{currentmarker}{}%
\end{pgfscope}%
\begin{pgfscope}%
\pgfsys@transformshift{2.406710in}{4.166599in}%
\pgfsys@useobject{currentmarker}{}%
\end{pgfscope}%
\begin{pgfscope}%
\pgfsys@transformshift{2.425723in}{4.237589in}%
\pgfsys@useobject{currentmarker}{}%
\end{pgfscope}%
\begin{pgfscope}%
\pgfsys@transformshift{2.440511in}{4.392054in}%
\pgfsys@useobject{currentmarker}{}%
\end{pgfscope}%
\begin{pgfscope}%
\pgfsys@transformshift{2.462107in}{4.381511in}%
\pgfsys@useobject{currentmarker}{}%
\end{pgfscope}%
\begin{pgfscope}%
\pgfsys@transformshift{2.483937in}{4.238265in}%
\pgfsys@useobject{currentmarker}{}%
\end{pgfscope}%
\begin{pgfscope}%
\pgfsys@transformshift{2.500837in}{4.111222in}%
\pgfsys@useobject{currentmarker}{}%
\end{pgfscope}%
\begin{pgfscope}%
\pgfsys@transformshift{2.518676in}{4.054960in}%
\pgfsys@useobject{currentmarker}{}%
\end{pgfscope}%
\begin{pgfscope}%
\pgfsys@transformshift{2.539098in}{4.168880in}%
\pgfsys@useobject{currentmarker}{}%
\end{pgfscope}%
\begin{pgfscope}%
\pgfsys@transformshift{2.558816in}{4.085426in}%
\pgfsys@useobject{currentmarker}{}%
\end{pgfscope}%
\begin{pgfscope}%
\pgfsys@transformshift{2.576419in}{4.047959in}%
\pgfsys@useobject{currentmarker}{}%
\end{pgfscope}%
\begin{pgfscope}%
\pgfsys@transformshift{2.597546in}{4.041151in}%
\pgfsys@useobject{currentmarker}{}%
\end{pgfscope}%
\begin{pgfscope}%
\pgfsys@transformshift{2.617497in}{4.035044in}%
\pgfsys@useobject{currentmarker}{}%
\end{pgfscope}%
\begin{pgfscope}%
\pgfsys@transformshift{2.635572in}{4.036095in}%
\pgfsys@useobject{currentmarker}{}%
\end{pgfscope}%
\begin{pgfscope}%
\pgfsys@transformshift{2.652941in}{4.044895in}%
\pgfsys@useobject{currentmarker}{}%
\end{pgfscope}%
\begin{pgfscope}%
\pgfsys@transformshift{2.675242in}{4.076761in}%
\pgfsys@useobject{currentmarker}{}%
\end{pgfscope}%
\begin{pgfscope}%
\pgfsys@transformshift{2.696132in}{4.184002in}%
\pgfsys@useobject{currentmarker}{}%
\end{pgfscope}%
\begin{pgfscope}%
\pgfsys@transformshift{2.710686in}{4.365533in}%
\pgfsys@useobject{currentmarker}{}%
\end{pgfscope}%
\begin{pgfscope}%
\pgfsys@transformshift{2.732514in}{4.399787in}%
\pgfsys@useobject{currentmarker}{}%
\end{pgfscope}%
\begin{pgfscope}%
\pgfsys@transformshift{2.749884in}{4.283495in}%
\pgfsys@useobject{currentmarker}{}%
\end{pgfscope}%
\begin{pgfscope}%
\pgfsys@transformshift{2.772420in}{4.116938in}%
\pgfsys@useobject{currentmarker}{}%
\end{pgfscope}%
\begin{pgfscope}%
\pgfsys@transformshift{2.788380in}{4.060916in}%
\pgfsys@useobject{currentmarker}{}%
\end{pgfscope}%
\begin{pgfscope}%
\pgfsys@transformshift{2.809507in}{4.043368in}%
\pgfsys@useobject{currentmarker}{}%
\end{pgfscope}%
\begin{pgfscope}%
\pgfsys@transformshift{2.828049in}{4.036662in}%
\pgfsys@useobject{currentmarker}{}%
\end{pgfscope}%
\begin{pgfscope}%
\pgfsys@transformshift{2.846828in}{4.034896in}%
\pgfsys@useobject{currentmarker}{}%
\end{pgfscope}%
\begin{pgfscope}%
\pgfsys@transformshift{2.867484in}{4.036260in}%
\pgfsys@useobject{currentmarker}{}%
\end{pgfscope}%
\begin{pgfscope}%
\pgfsys@transformshift{2.886497in}{4.043308in}%
\pgfsys@useobject{currentmarker}{}%
\end{pgfscope}%
\begin{pgfscope}%
\pgfsys@transformshift{2.903163in}{4.068688in}%
\pgfsys@useobject{currentmarker}{}%
\end{pgfscope}%
\begin{pgfscope}%
\pgfsys@transformshift{2.921473in}{4.138117in}%
\pgfsys@useobject{currentmarker}{}%
\end{pgfscope}%
\begin{pgfscope}%
\pgfsys@transformshift{2.941658in}{4.344416in}%
\pgfsys@useobject{currentmarker}{}%
\end{pgfscope}%
\begin{pgfscope}%
\pgfsys@transformshift{2.963254in}{4.411395in}%
\pgfsys@useobject{currentmarker}{}%
\end{pgfscope}%
\begin{pgfscope}%
\pgfsys@transformshift{2.981327in}{4.345110in}%
\pgfsys@useobject{currentmarker}{}%
\end{pgfscope}%
\begin{pgfscope}%
\pgfsys@transformshift{2.999637in}{4.229670in}%
\pgfsys@useobject{currentmarker}{}%
\end{pgfscope}%
\begin{pgfscope}%
\pgfsys@transformshift{3.021702in}{4.099476in}%
\pgfsys@useobject{currentmarker}{}%
\end{pgfscope}%
\begin{pgfscope}%
\pgfsys@transformshift{3.039541in}{4.122423in}%
\pgfsys@useobject{currentmarker}{}%
\end{pgfscope}%
\begin{pgfscope}%
\pgfsys@transformshift{3.057849in}{4.064289in}%
\pgfsys@useobject{currentmarker}{}%
\end{pgfscope}%
\begin{pgfscope}%
\pgfsys@transformshift{3.078505in}{4.041107in}%
\pgfsys@useobject{currentmarker}{}%
\end{pgfscope}%
\begin{pgfscope}%
\pgfsys@transformshift{3.094936in}{4.036187in}%
\pgfsys@useobject{currentmarker}{}%
\end{pgfscope}%
\begin{pgfscope}%
\pgfsys@transformshift{3.118646in}{4.036051in}%
\pgfsys@useobject{currentmarker}{}%
\end{pgfscope}%
\begin{pgfscope}%
\pgfsys@transformshift{3.134137in}{4.040551in}%
\pgfsys@useobject{currentmarker}{}%
\end{pgfscope}%
\begin{pgfscope}%
\pgfsys@transformshift{3.155498in}{4.057894in}%
\pgfsys@useobject{currentmarker}{}%
\end{pgfscope}%
\begin{pgfscope}%
\pgfsys@transformshift{3.173806in}{4.100586in}%
\pgfsys@useobject{currentmarker}{}%
\end{pgfscope}%
\begin{pgfscope}%
\pgfsys@transformshift{3.191176in}{4.208919in}%
\pgfsys@useobject{currentmarker}{}%
\end{pgfscope}%
\begin{pgfscope}%
\pgfsys@transformshift{3.213241in}{4.408817in}%
\pgfsys@useobject{currentmarker}{}%
\end{pgfscope}%
\begin{pgfscope}%
\pgfsys@transformshift{3.230141in}{4.406674in}%
\pgfsys@useobject{currentmarker}{}%
\end{pgfscope}%
\begin{pgfscope}%
\pgfsys@transformshift{3.248919in}{4.318787in}%
\pgfsys@useobject{currentmarker}{}%
\end{pgfscope}%
\begin{pgfscope}%
\pgfsys@transformshift{3.272861in}{4.123695in}%
\pgfsys@useobject{currentmarker}{}%
\end{pgfscope}%
\begin{pgfscope}%
\pgfsys@transformshift{3.291406in}{4.066371in}%
\pgfsys@useobject{currentmarker}{}%
\end{pgfscope}%
\begin{pgfscope}%
\pgfsys@transformshift{3.309011in}{4.045161in}%
\pgfsys@useobject{currentmarker}{}%
\end{pgfscope}%
\begin{pgfscope}%
\pgfsys@transformshift{3.327789in}{4.037519in}%
\pgfsys@useobject{currentmarker}{}%
\end{pgfscope}%
\begin{pgfscope}%
\pgfsys@transformshift{3.347975in}{4.035316in}%
\pgfsys@useobject{currentmarker}{}%
\end{pgfscope}%
\begin{pgfscope}%
\pgfsys@transformshift{3.362528in}{4.037215in}%
\pgfsys@useobject{currentmarker}{}%
\end{pgfscope}%
\begin{pgfscope}%
\pgfsys@transformshift{3.388113in}{4.041225in}%
\pgfsys@useobject{currentmarker}{}%
\end{pgfscope}%
\begin{pgfscope}%
\pgfsys@transformshift{3.405015in}{4.054059in}%
\pgfsys@useobject{currentmarker}{}%
\end{pgfscope}%
\begin{pgfscope}%
\pgfsys@transformshift{3.423793in}{4.062971in}%
\pgfsys@useobject{currentmarker}{}%
\end{pgfscope}%
\begin{pgfscope}%
\pgfsys@transformshift{3.444215in}{4.088716in}%
\pgfsys@useobject{currentmarker}{}%
\end{pgfscope}%
\begin{pgfscope}%
\pgfsys@transformshift{3.462289in}{4.198253in}%
\pgfsys@useobject{currentmarker}{}%
\end{pgfscope}%
\begin{pgfscope}%
\pgfsys@transformshift{3.480128in}{4.397590in}%
\pgfsys@useobject{currentmarker}{}%
\end{pgfscope}%
\begin{pgfscope}%
\pgfsys@transformshift{3.501019in}{4.414057in}%
\pgfsys@useobject{currentmarker}{}%
\end{pgfscope}%
\begin{pgfscope}%
\pgfsys@transformshift{3.518858in}{4.319213in}%
\pgfsys@useobject{currentmarker}{}%
\end{pgfscope}%
\begin{pgfscope}%
\pgfsys@transformshift{3.537637in}{4.162591in}%
\pgfsys@useobject{currentmarker}{}%
\end{pgfscope}%
\begin{pgfscope}%
\pgfsys@transformshift{3.559701in}{4.070875in}%
\pgfsys@useobject{currentmarker}{}%
\end{pgfscope}%
\begin{pgfscope}%
\pgfsys@transformshift{3.579418in}{4.049659in}%
\pgfsys@useobject{currentmarker}{}%
\end{pgfscope}%
\begin{pgfscope}%
\pgfsys@transformshift{3.597728in}{4.039005in}%
\pgfsys@useobject{currentmarker}{}%
\end{pgfscope}%
\begin{pgfscope}%
\pgfsys@transformshift{3.615567in}{4.035690in}%
\pgfsys@useobject{currentmarker}{}%
\end{pgfscope}%
\begin{pgfscope}%
\pgfsys@transformshift{3.634346in}{4.036565in}%
\pgfsys@useobject{currentmarker}{}%
\end{pgfscope}%
\begin{pgfscope}%
\pgfsys@transformshift{3.652419in}{4.041850in}%
\pgfsys@useobject{currentmarker}{}%
\end{pgfscope}%
\begin{pgfscope}%
\pgfsys@transformshift{3.672841in}{4.053423in}%
\pgfsys@useobject{currentmarker}{}%
\end{pgfscope}%
\begin{pgfscope}%
\pgfsys@transformshift{3.691618in}{4.086856in}%
\pgfsys@useobject{currentmarker}{}%
\end{pgfscope}%
\begin{pgfscope}%
\pgfsys@transformshift{3.711571in}{4.188847in}%
\pgfsys@useobject{currentmarker}{}%
\end{pgfscope}%
\begin{pgfscope}%
\pgfsys@transformshift{3.729645in}{4.340557in}%
\pgfsys@useobject{currentmarker}{}%
\end{pgfscope}%
\begin{pgfscope}%
\pgfsys@transformshift{3.750301in}{4.433899in}%
\pgfsys@useobject{currentmarker}{}%
\end{pgfscope}%
\begin{pgfscope}%
\pgfsys@transformshift{3.770019in}{4.406447in}%
\pgfsys@useobject{currentmarker}{}%
\end{pgfscope}%
\begin{pgfscope}%
\pgfsys@transformshift{3.787153in}{4.308087in}%
\pgfsys@useobject{currentmarker}{}%
\end{pgfscope}%
\begin{pgfscope}%
\pgfsys@transformshift{3.808280in}{4.146554in}%
\pgfsys@useobject{currentmarker}{}%
\end{pgfscope}%
\begin{pgfscope}%
\pgfsys@transformshift{3.826588in}{4.085843in}%
\pgfsys@useobject{currentmarker}{}%
\end{pgfscope}%
\begin{pgfscope}%
\pgfsys@transformshift{3.847713in}{4.048755in}%
\pgfsys@useobject{currentmarker}{}%
\end{pgfscope}%
\begin{pgfscope}%
\pgfsys@transformshift{3.866963in}{4.044875in}%
\pgfsys@useobject{currentmarker}{}%
\end{pgfscope}%
\begin{pgfscope}%
\pgfsys@transformshift{3.885036in}{4.038342in}%
\pgfsys@useobject{currentmarker}{}%
\end{pgfscope}%
\begin{pgfscope}%
\pgfsys@transformshift{3.905458in}{4.035591in}%
\pgfsys@useobject{currentmarker}{}%
\end{pgfscope}%
\begin{pgfscope}%
\pgfsys@transformshift{3.922592in}{4.037906in}%
\pgfsys@useobject{currentmarker}{}%
\end{pgfscope}%
\begin{pgfscope}%
\pgfsys@transformshift{3.940902in}{4.041623in}%
\pgfsys@useobject{currentmarker}{}%
\end{pgfscope}%
\begin{pgfscope}%
\pgfsys@transformshift{3.962496in}{4.049441in}%
\pgfsys@useobject{currentmarker}{}%
\end{pgfscope}%
\begin{pgfscope}%
\pgfsys@transformshift{3.979866in}{4.065450in}%
\pgfsys@useobject{currentmarker}{}%
\end{pgfscope}%
\begin{pgfscope}%
\pgfsys@transformshift{4.000991in}{4.086227in}%
\pgfsys@useobject{currentmarker}{}%
\end{pgfscope}%
\begin{pgfscope}%
\pgfsys@transformshift{4.018596in}{4.048418in}%
\pgfsys@useobject{currentmarker}{}%
\end{pgfscope}%
\begin{pgfscope}%
\pgfsys@transformshift{4.037375in}{4.039069in}%
\pgfsys@useobject{currentmarker}{}%
\end{pgfscope}%
\begin{pgfscope}%
\pgfsys@transformshift{4.058266in}{4.036278in}%
\pgfsys@useobject{currentmarker}{}%
\end{pgfscope}%
\begin{pgfscope}%
\pgfsys@transformshift{4.076810in}{4.042442in}%
\pgfsys@useobject{currentmarker}{}%
\end{pgfscope}%
\begin{pgfscope}%
\pgfsys@transformshift{4.095118in}{4.059893in}%
\pgfsys@useobject{currentmarker}{}%
\end{pgfscope}%
\begin{pgfscope}%
\pgfsys@transformshift{4.115774in}{4.129580in}%
\pgfsys@useobject{currentmarker}{}%
\end{pgfscope}%
\begin{pgfscope}%
\pgfsys@transformshift{4.134553in}{4.307324in}%
\pgfsys@useobject{currentmarker}{}%
\end{pgfscope}%
\begin{pgfscope}%
\pgfsys@transformshift{4.153801in}{4.442798in}%
\pgfsys@useobject{currentmarker}{}%
\end{pgfscope}%
\begin{pgfscope}%
\pgfsys@transformshift{4.170466in}{4.428777in}%
\pgfsys@useobject{currentmarker}{}%
\end{pgfscope}%
\begin{pgfscope}%
\pgfsys@transformshift{4.191827in}{4.299505in}%
\pgfsys@useobject{currentmarker}{}%
\end{pgfscope}%
\begin{pgfscope}%
\pgfsys@transformshift{4.209666in}{4.147468in}%
\pgfsys@useobject{currentmarker}{}%
\end{pgfscope}%
\begin{pgfscope}%
\pgfsys@transformshift{4.230793in}{4.071034in}%
\pgfsys@useobject{currentmarker}{}%
\end{pgfscope}%
\begin{pgfscope}%
\pgfsys@transformshift{4.249570in}{4.046608in}%
\pgfsys@useobject{currentmarker}{}%
\end{pgfscope}%
\begin{pgfscope}%
\pgfsys@transformshift{4.271400in}{4.036841in}%
\pgfsys@useobject{currentmarker}{}%
\end{pgfscope}%
\begin{pgfscope}%
\pgfsys@transformshift{4.289005in}{4.037115in}%
\pgfsys@useobject{currentmarker}{}%
\end{pgfscope}%
\begin{pgfscope}%
\pgfsys@transformshift{4.306141in}{4.043039in}%
\pgfsys@useobject{currentmarker}{}%
\end{pgfscope}%
\begin{pgfscope}%
\pgfsys@transformshift{4.327032in}{4.061350in}%
\pgfsys@useobject{currentmarker}{}%
\end{pgfscope}%
\begin{pgfscope}%
\pgfsys@transformshift{4.344871in}{4.114715in}%
\pgfsys@useobject{currentmarker}{}%
\end{pgfscope}%
\begin{pgfscope}%
\pgfsys@transformshift{4.363650in}{4.212360in}%
\pgfsys@useobject{currentmarker}{}%
\end{pgfscope}%
\begin{pgfscope}%
\pgfsys@transformshift{4.385244in}{4.434845in}%
\pgfsys@useobject{currentmarker}{}%
\end{pgfscope}%
\begin{pgfscope}%
\pgfsys@transformshift{4.403319in}{4.456283in}%
\pgfsys@useobject{currentmarker}{}%
\end{pgfscope}%
\begin{pgfscope}%
\pgfsys@transformshift{4.420922in}{4.397874in}%
\pgfsys@useobject{currentmarker}{}%
\end{pgfscope}%
\begin{pgfscope}%
\pgfsys@transformshift{4.440406in}{4.250746in}%
\pgfsys@useobject{currentmarker}{}%
\end{pgfscope}%
\begin{pgfscope}%
\pgfsys@transformshift{4.462000in}{4.133296in}%
\pgfsys@useobject{currentmarker}{}%
\end{pgfscope}%
\begin{pgfscope}%
\pgfsys@transformshift{4.480076in}{4.068091in}%
\pgfsys@useobject{currentmarker}{}%
\end{pgfscope}%
\begin{pgfscope}%
\pgfsys@transformshift{4.481013in}{4.066784in}%
\pgfsys@useobject{currentmarker}{}%
\end{pgfscope}%
\begin{pgfscope}%
\pgfsys@transformshift{4.474676in}{4.091434in}%
\pgfsys@useobject{currentmarker}{}%
\end{pgfscope}%
\begin{pgfscope}%
\pgfsys@transformshift{4.455194in}{4.243504in}%
\pgfsys@useobject{currentmarker}{}%
\end{pgfscope}%
\begin{pgfscope}%
\pgfsys@transformshift{4.435712in}{4.187430in}%
\pgfsys@useobject{currentmarker}{}%
\end{pgfscope}%
\begin{pgfscope}%
\pgfsys@transformshift{4.417871in}{4.063991in}%
\pgfsys@useobject{currentmarker}{}%
\end{pgfscope}%
\begin{pgfscope}%
\pgfsys@transformshift{4.399094in}{4.039958in}%
\pgfsys@useobject{currentmarker}{}%
\end{pgfscope}%
\begin{pgfscope}%
\pgfsys@transformshift{4.377498in}{4.037389in}%
\pgfsys@useobject{currentmarker}{}%
\end{pgfscope}%
\begin{pgfscope}%
\pgfsys@transformshift{4.360128in}{4.046705in}%
\pgfsys@useobject{currentmarker}{}%
\end{pgfscope}%
\begin{pgfscope}%
\pgfsys@transformshift{4.342757in}{4.096232in}%
\pgfsys@useobject{currentmarker}{}%
\end{pgfscope}%
\begin{pgfscope}%
\pgfsys@transformshift{4.319050in}{4.282045in}%
\pgfsys@useobject{currentmarker}{}%
\end{pgfscope}%
\begin{pgfscope}%
\pgfsys@transformshift{4.298628in}{4.442549in}%
\pgfsys@useobject{currentmarker}{}%
\end{pgfscope}%
\begin{pgfscope}%
\pgfsys@transformshift{4.281963in}{4.429934in}%
\pgfsys@useobject{currentmarker}{}%
\end{pgfscope}%
\begin{pgfscope}%
\pgfsys@transformshift{4.264593in}{4.156562in}%
\pgfsys@useobject{currentmarker}{}%
\end{pgfscope}%
\begin{pgfscope}%
\pgfsys@transformshift{4.244407in}{4.070136in}%
\pgfsys@useobject{currentmarker}{}%
\end{pgfscope}%
\begin{pgfscope}%
\pgfsys@transformshift{4.227271in}{4.043315in}%
\pgfsys@useobject{currentmarker}{}%
\end{pgfscope}%
\begin{pgfscope}%
\pgfsys@transformshift{4.205675in}{4.036016in}%
\pgfsys@useobject{currentmarker}{}%
\end{pgfscope}%
\begin{pgfscope}%
\pgfsys@transformshift{4.185490in}{4.040601in}%
\pgfsys@useobject{currentmarker}{}%
\end{pgfscope}%
\begin{pgfscope}%
\pgfsys@transformshift{4.168589in}{4.062767in}%
\pgfsys@useobject{currentmarker}{}%
\end{pgfscope}%
\begin{pgfscope}%
\pgfsys@transformshift{4.147698in}{4.176705in}%
\pgfsys@useobject{currentmarker}{}%
\end{pgfscope}%
\begin{pgfscope}%
\pgfsys@transformshift{4.127511in}{4.388519in}%
\pgfsys@useobject{currentmarker}{}%
\end{pgfscope}%
\begin{pgfscope}%
\pgfsys@transformshift{4.109437in}{4.441823in}%
\pgfsys@useobject{currentmarker}{}%
\end{pgfscope}%
\begin{pgfscope}%
\pgfsys@transformshift{4.091598in}{4.311722in}%
\pgfsys@useobject{currentmarker}{}%
\end{pgfscope}%
\begin{pgfscope}%
\pgfsys@transformshift{4.071645in}{4.094906in}%
\pgfsys@useobject{currentmarker}{}%
\end{pgfscope}%
\begin{pgfscope}%
\pgfsys@transformshift{4.050520in}{4.046998in}%
\pgfsys@useobject{currentmarker}{}%
\end{pgfscope}%
\begin{pgfscope}%
\pgfsys@transformshift{4.033855in}{4.036628in}%
\pgfsys@useobject{currentmarker}{}%
\end{pgfscope}%
\begin{pgfscope}%
\pgfsys@transformshift{4.015310in}{4.036577in}%
\pgfsys@useobject{currentmarker}{}%
\end{pgfscope}%
\begin{pgfscope}%
\pgfsys@transformshift{3.992777in}{4.048474in}%
\pgfsys@useobject{currentmarker}{}%
\end{pgfscope}%
\begin{pgfscope}%
\pgfsys@transformshift{3.974938in}{4.097206in}%
\pgfsys@useobject{currentmarker}{}%
\end{pgfscope}%
\begin{pgfscope}%
\pgfsys@transformshift{3.953342in}{4.267211in}%
\pgfsys@useobject{currentmarker}{}%
\end{pgfscope}%
\begin{pgfscope}%
\pgfsys@transformshift{3.936442in}{4.409251in}%
\pgfsys@useobject{currentmarker}{}%
\end{pgfscope}%
\begin{pgfscope}%
\pgfsys@transformshift{3.918838in}{4.426902in}%
\pgfsys@useobject{currentmarker}{}%
\end{pgfscope}%
\begin{pgfscope}%
\pgfsys@transformshift{3.897007in}{4.196673in}%
\pgfsys@useobject{currentmarker}{}%
\end{pgfscope}%
\begin{pgfscope}%
\pgfsys@transformshift{3.877994in}{4.073104in}%
\pgfsys@useobject{currentmarker}{}%
\end{pgfscope}%
\begin{pgfscope}%
\pgfsys@transformshift{3.861563in}{4.044294in}%
\pgfsys@useobject{currentmarker}{}%
\end{pgfscope}%
\begin{pgfscope}%
\pgfsys@transformshift{3.841142in}{4.036175in}%
\pgfsys@useobject{currentmarker}{}%
\end{pgfscope}%
\begin{pgfscope}%
\pgfsys@transformshift{3.819546in}{4.035529in}%
\pgfsys@useobject{currentmarker}{}%
\end{pgfscope}%
\begin{pgfscope}%
\pgfsys@transformshift{3.802646in}{4.040421in}%
\pgfsys@useobject{currentmarker}{}%
\end{pgfscope}%
\begin{pgfscope}%
\pgfsys@transformshift{3.781756in}{4.066066in}%
\pgfsys@useobject{currentmarker}{}%
\end{pgfscope}%
\begin{pgfscope}%
\pgfsys@transformshift{3.765089in}{4.130831in}%
\pgfsys@useobject{currentmarker}{}%
\end{pgfscope}%
\begin{pgfscope}%
\pgfsys@transformshift{3.743729in}{4.146039in}%
\pgfsys@useobject{currentmarker}{}%
\end{pgfscope}%
\begin{pgfscope}%
\pgfsys@transformshift{3.726593in}{4.320734in}%
\pgfsys@useobject{currentmarker}{}%
\end{pgfscope}%
\begin{pgfscope}%
\pgfsys@transformshift{3.704999in}{4.427595in}%
\pgfsys@useobject{currentmarker}{}%
\end{pgfscope}%
\begin{pgfscope}%
\pgfsys@transformshift{3.687863in}{4.360118in}%
\pgfsys@useobject{currentmarker}{}%
\end{pgfscope}%
\begin{pgfscope}%
\pgfsys@transformshift{3.666973in}{4.121459in}%
\pgfsys@useobject{currentmarker}{}%
\end{pgfscope}%
\begin{pgfscope}%
\pgfsys@transformshift{3.649603in}{4.058141in}%
\pgfsys@useobject{currentmarker}{}%
\end{pgfscope}%
\begin{pgfscope}%
\pgfsys@transformshift{3.628712in}{4.038757in}%
\pgfsys@useobject{currentmarker}{}%
\end{pgfscope}%
\begin{pgfscope}%
\pgfsys@transformshift{3.606882in}{4.035377in}%
\pgfsys@useobject{currentmarker}{}%
\end{pgfscope}%
\begin{pgfscope}%
\pgfsys@transformshift{3.589511in}{4.037455in}%
\pgfsys@useobject{currentmarker}{}%
\end{pgfscope}%
\begin{pgfscope}%
\pgfsys@transformshift{3.571203in}{4.047990in}%
\pgfsys@useobject{currentmarker}{}%
\end{pgfscope}%
\begin{pgfscope}%
\pgfsys@transformshift{3.553599in}{4.070444in}%
\pgfsys@useobject{currentmarker}{}%
\end{pgfscope}%
\begin{pgfscope}%
\pgfsys@transformshift{3.532237in}{4.205869in}%
\pgfsys@useobject{currentmarker}{}%
\end{pgfscope}%
\begin{pgfscope}%
\pgfsys@transformshift{3.510407in}{4.369278in}%
\pgfsys@useobject{currentmarker}{}%
\end{pgfscope}%
\begin{pgfscope}%
\pgfsys@transformshift{3.492333in}{4.422475in}%
\pgfsys@useobject{currentmarker}{}%
\end{pgfscope}%
\begin{pgfscope}%
\pgfsys@transformshift{3.473320in}{4.305510in}%
\pgfsys@useobject{currentmarker}{}%
\end{pgfscope}%
\begin{pgfscope}%
\pgfsys@transformshift{3.458063in}{4.126503in}%
\pgfsys@useobject{currentmarker}{}%
\end{pgfscope}%
\begin{pgfscope}%
\pgfsys@transformshift{3.436702in}{4.348585in}%
\pgfsys@useobject{currentmarker}{}%
\end{pgfscope}%
\begin{pgfscope}%
\pgfsys@transformshift{3.414874in}{4.417778in}%
\pgfsys@useobject{currentmarker}{}%
\end{pgfscope}%
\begin{pgfscope}%
\pgfsys@transformshift{3.399381in}{4.288281in}%
\pgfsys@useobject{currentmarker}{}%
\end{pgfscope}%
\begin{pgfscope}%
\pgfsys@transformshift{3.378256in}{4.114581in}%
\pgfsys@useobject{currentmarker}{}%
\end{pgfscope}%
\begin{pgfscope}%
\pgfsys@transformshift{3.360417in}{4.054598in}%
\pgfsys@useobject{currentmarker}{}%
\end{pgfscope}%
\begin{pgfscope}%
\pgfsys@transformshift{3.341872in}{4.039335in}%
\pgfsys@useobject{currentmarker}{}%
\end{pgfscope}%
\begin{pgfscope}%
\pgfsys@transformshift{3.320747in}{4.034940in}%
\pgfsys@useobject{currentmarker}{}%
\end{pgfscope}%
\begin{pgfscope}%
\pgfsys@transformshift{3.301029in}{4.038147in}%
\pgfsys@useobject{currentmarker}{}%
\end{pgfscope}%
\begin{pgfscope}%
\pgfsys@transformshift{3.280607in}{4.051338in}%
\pgfsys@useobject{currentmarker}{}%
\end{pgfscope}%
\begin{pgfscope}%
\pgfsys@transformshift{3.262533in}{4.096829in}%
\pgfsys@useobject{currentmarker}{}%
\end{pgfscope}%
\begin{pgfscope}%
\pgfsys@transformshift{3.243991in}{4.248144in}%
\pgfsys@useobject{currentmarker}{}%
\end{pgfscope}%
\begin{pgfscope}%
\pgfsys@transformshift{3.225915in}{4.373055in}%
\pgfsys@useobject{currentmarker}{}%
\end{pgfscope}%
\begin{pgfscope}%
\pgfsys@transformshift{3.204556in}{4.414406in}%
\pgfsys@useobject{currentmarker}{}%
\end{pgfscope}%
\begin{pgfscope}%
\pgfsys@transformshift{3.188360in}{4.274692in}%
\pgfsys@useobject{currentmarker}{}%
\end{pgfscope}%
\begin{pgfscope}%
\pgfsys@transformshift{3.166529in}{4.088520in}%
\pgfsys@useobject{currentmarker}{}%
\end{pgfscope}%
\begin{pgfscope}%
\pgfsys@transformshift{3.147750in}{4.049432in}%
\pgfsys@useobject{currentmarker}{}%
\end{pgfscope}%
\begin{pgfscope}%
\pgfsys@transformshift{3.128503in}{4.037531in}%
\pgfsys@useobject{currentmarker}{}%
\end{pgfscope}%
\begin{pgfscope}%
\pgfsys@transformshift{3.109960in}{4.034786in}%
\pgfsys@useobject{currentmarker}{}%
\end{pgfscope}%
\begin{pgfscope}%
\pgfsys@transformshift{3.091416in}{4.036586in}%
\pgfsys@useobject{currentmarker}{}%
\end{pgfscope}%
\begin{pgfscope}%
\pgfsys@transformshift{3.072871in}{4.042564in}%
\pgfsys@useobject{currentmarker}{}%
\end{pgfscope}%
\begin{pgfscope}%
\pgfsys@transformshift{3.050104in}{4.065334in}%
\pgfsys@useobject{currentmarker}{}%
\end{pgfscope}%
\begin{pgfscope}%
\pgfsys@transformshift{3.032733in}{4.167809in}%
\pgfsys@useobject{currentmarker}{}%
\end{pgfscope}%
\begin{pgfscope}%
\pgfsys@transformshift{3.013017in}{4.315350in}%
\pgfsys@useobject{currentmarker}{}%
\end{pgfscope}%
\begin{pgfscope}%
\pgfsys@transformshift{2.994472in}{4.416095in}%
\pgfsys@useobject{currentmarker}{}%
\end{pgfscope}%
\begin{pgfscope}%
\pgfsys@transformshift{2.973816in}{4.294440in}%
\pgfsys@useobject{currentmarker}{}%
\end{pgfscope}%
\begin{pgfscope}%
\pgfsys@transformshift{2.955039in}{4.184871in}%
\pgfsys@useobject{currentmarker}{}%
\end{pgfscope}%
\begin{pgfscope}%
\pgfsys@transformshift{2.936495in}{4.091073in}%
\pgfsys@useobject{currentmarker}{}%
\end{pgfscope}%
\begin{pgfscope}%
\pgfsys@transformshift{2.917950in}{4.050454in}%
\pgfsys@useobject{currentmarker}{}%
\end{pgfscope}%
\begin{pgfscope}%
\pgfsys@transformshift{2.897060in}{4.037185in}%
\pgfsys@useobject{currentmarker}{}%
\end{pgfscope}%
\begin{pgfscope}%
\pgfsys@transformshift{2.880160in}{4.034971in}%
\pgfsys@useobject{currentmarker}{}%
\end{pgfscope}%
\begin{pgfscope}%
\pgfsys@transformshift{2.859973in}{4.037975in}%
\pgfsys@useobject{currentmarker}{}%
\end{pgfscope}%
\begin{pgfscope}%
\pgfsys@transformshift{2.841665in}{4.047584in}%
\pgfsys@useobject{currentmarker}{}%
\end{pgfscope}%
\begin{pgfscope}%
\pgfsys@transformshift{2.819130in}{4.078850in}%
\pgfsys@useobject{currentmarker}{}%
\end{pgfscope}%
\begin{pgfscope}%
\pgfsys@transformshift{2.802464in}{4.176156in}%
\pgfsys@useobject{currentmarker}{}%
\end{pgfscope}%
\begin{pgfscope}%
\pgfsys@transformshift{2.783685in}{4.342450in}%
\pgfsys@useobject{currentmarker}{}%
\end{pgfscope}%
\begin{pgfscope}%
\pgfsys@transformshift{2.766081in}{4.411223in}%
\pgfsys@useobject{currentmarker}{}%
\end{pgfscope}%
\begin{pgfscope}%
\pgfsys@transformshift{2.747302in}{4.305815in}%
\pgfsys@useobject{currentmarker}{}%
\end{pgfscope}%
\begin{pgfscope}%
\pgfsys@transformshift{2.724768in}{4.116628in}%
\pgfsys@useobject{currentmarker}{}%
\end{pgfscope}%
\begin{pgfscope}%
\pgfsys@transformshift{2.707634in}{4.055685in}%
\pgfsys@useobject{currentmarker}{}%
\end{pgfscope}%
\begin{pgfscope}%
\pgfsys@transformshift{2.687916in}{4.040906in}%
\pgfsys@useobject{currentmarker}{}%
\end{pgfscope}%
\begin{pgfscope}%
\pgfsys@transformshift{2.667494in}{4.035203in}%
\pgfsys@useobject{currentmarker}{}%
\end{pgfscope}%
\begin{pgfscope}%
\pgfsys@transformshift{2.647778in}{4.036055in}%
\pgfsys@useobject{currentmarker}{}%
\end{pgfscope}%
\begin{pgfscope}%
\pgfsys@transformshift{2.630407in}{4.041981in}%
\pgfsys@useobject{currentmarker}{}%
\end{pgfscope}%
\begin{pgfscope}%
\pgfsys@transformshift{2.611628in}{4.047141in}%
\pgfsys@useobject{currentmarker}{}%
\end{pgfscope}%
\begin{pgfscope}%
\pgfsys@transformshift{2.588392in}{4.066928in}%
\pgfsys@useobject{currentmarker}{}%
\end{pgfscope}%
\begin{pgfscope}%
\pgfsys@transformshift{2.568673in}{4.160721in}%
\pgfsys@useobject{currentmarker}{}%
\end{pgfscope}%
\begin{pgfscope}%
\pgfsys@transformshift{2.553885in}{4.297062in}%
\pgfsys@useobject{currentmarker}{}%
\end{pgfscope}%
\begin{pgfscope}%
\pgfsys@transformshift{2.534169in}{4.376106in}%
\pgfsys@useobject{currentmarker}{}%
\end{pgfscope}%
\begin{pgfscope}%
\pgfsys@transformshift{2.514687in}{4.407632in}%
\pgfsys@useobject{currentmarker}{}%
\end{pgfscope}%
\begin{pgfscope}%
\pgfsys@transformshift{2.494734in}{4.260407in}%
\pgfsys@useobject{currentmarker}{}%
\end{pgfscope}%
\begin{pgfscope}%
\pgfsys@transformshift{2.477364in}{4.108671in}%
\pgfsys@useobject{currentmarker}{}%
\end{pgfscope}%
\begin{pgfscope}%
\pgfsys@transformshift{2.457413in}{4.053160in}%
\pgfsys@useobject{currentmarker}{}%
\end{pgfscope}%
\begin{pgfscope}%
\pgfsys@transformshift{2.435582in}{4.038214in}%
\pgfsys@useobject{currentmarker}{}%
\end{pgfscope}%
\begin{pgfscope}%
\pgfsys@transformshift{2.417272in}{4.035393in}%
\pgfsys@useobject{currentmarker}{}%
\end{pgfscope}%
\begin{pgfscope}%
\pgfsys@transformshift{2.398496in}{4.034883in}%
\pgfsys@useobject{currentmarker}{}%
\end{pgfscope}%
\begin{pgfscope}%
\pgfsys@transformshift{2.379951in}{4.039949in}%
\pgfsys@useobject{currentmarker}{}%
\end{pgfscope}%
\begin{pgfscope}%
\pgfsys@transformshift{2.360938in}{4.035083in}%
\pgfsys@useobject{currentmarker}{}%
\end{pgfscope}%
\begin{pgfscope}%
\pgfsys@transformshift{2.342395in}{4.035476in}%
\pgfsys@useobject{currentmarker}{}%
\end{pgfscope}%
\begin{pgfscope}%
\pgfsys@transformshift{2.321034in}{4.043916in}%
\pgfsys@useobject{currentmarker}{}%
\end{pgfscope}%
\begin{pgfscope}%
\pgfsys@transformshift{2.302960in}{4.072952in}%
\pgfsys@useobject{currentmarker}{}%
\end{pgfscope}%
\begin{pgfscope}%
\pgfsys@transformshift{2.284650in}{4.161909in}%
\pgfsys@useobject{currentmarker}{}%
\end{pgfscope}%
\begin{pgfscope}%
\pgfsys@transformshift{2.265403in}{4.296959in}%
\pgfsys@useobject{currentmarker}{}%
\end{pgfscope}%
\begin{pgfscope}%
\pgfsys@transformshift{2.246626in}{4.400146in}%
\pgfsys@useobject{currentmarker}{}%
\end{pgfscope}%
\begin{pgfscope}%
\pgfsys@transformshift{2.225264in}{4.371709in}%
\pgfsys@useobject{currentmarker}{}%
\end{pgfscope}%
\begin{pgfscope}%
\pgfsys@transformshift{2.206017in}{4.168249in}%
\pgfsys@useobject{currentmarker}{}%
\end{pgfscope}%
\begin{pgfscope}%
\pgfsys@transformshift{2.188178in}{4.070569in}%
\pgfsys@useobject{currentmarker}{}%
\end{pgfscope}%
\begin{pgfscope}%
\pgfsys@transformshift{2.168695in}{4.044650in}%
\pgfsys@useobject{currentmarker}{}%
\end{pgfscope}%
\begin{pgfscope}%
\pgfsys@transformshift{2.148977in}{4.036163in}%
\pgfsys@useobject{currentmarker}{}%
\end{pgfscope}%
\begin{pgfscope}%
\pgfsys@transformshift{2.132312in}{4.035110in}%
\pgfsys@useobject{currentmarker}{}%
\end{pgfscope}%
\begin{pgfscope}%
\pgfsys@transformshift{2.107665in}{4.040453in}%
\pgfsys@useobject{currentmarker}{}%
\end{pgfscope}%
\begin{pgfscope}%
\pgfsys@transformshift{2.092642in}{4.045737in}%
\pgfsys@useobject{currentmarker}{}%
\end{pgfscope}%
\begin{pgfscope}%
\pgfsys@transformshift{2.072455in}{4.079427in}%
\pgfsys@useobject{currentmarker}{}%
\end{pgfscope}%
\begin{pgfscope}%
\pgfsys@transformshift{2.052035in}{4.212419in}%
\pgfsys@useobject{currentmarker}{}%
\end{pgfscope}%
\begin{pgfscope}%
\pgfsys@transformshift{2.033725in}{4.345751in}%
\pgfsys@useobject{currentmarker}{}%
\end{pgfscope}%
\begin{pgfscope}%
\pgfsys@transformshift{2.012366in}{4.410286in}%
\pgfsys@useobject{currentmarker}{}%
\end{pgfscope}%
\begin{pgfscope}%
\pgfsys@transformshift{1.997578in}{4.357873in}%
\pgfsys@useobject{currentmarker}{}%
\end{pgfscope}%
\begin{pgfscope}%
\pgfsys@transformshift{1.975277in}{4.183602in}%
\pgfsys@useobject{currentmarker}{}%
\end{pgfscope}%
\begin{pgfscope}%
\pgfsys@transformshift{1.956969in}{4.078365in}%
\pgfsys@useobject{currentmarker}{}%
\end{pgfscope}%
\begin{pgfscope}%
\pgfsys@transformshift{1.938425in}{4.052423in}%
\pgfsys@useobject{currentmarker}{}%
\end{pgfscope}%
\begin{pgfscope}%
\pgfsys@transformshift{1.919648in}{4.040056in}%
\pgfsys@useobject{currentmarker}{}%
\end{pgfscope}%
\begin{pgfscope}%
\pgfsys@transformshift{1.900634in}{4.035695in}%
\pgfsys@useobject{currentmarker}{}%
\end{pgfscope}%
\begin{pgfscope}%
\pgfsys@transformshift{1.882795in}{4.035641in}%
\pgfsys@useobject{currentmarker}{}%
\end{pgfscope}%
\begin{pgfscope}%
\pgfsys@transformshift{1.857679in}{4.042768in}%
\pgfsys@useobject{currentmarker}{}%
\end{pgfscope}%
\begin{pgfscope}%
\pgfsys@transformshift{1.841483in}{4.054756in}%
\pgfsys@useobject{currentmarker}{}%
\end{pgfscope}%
\begin{pgfscope}%
\pgfsys@transformshift{1.820590in}{4.128149in}%
\pgfsys@useobject{currentmarker}{}%
\end{pgfscope}%
\begin{pgfscope}%
\pgfsys@transformshift{1.802517in}{4.230496in}%
\pgfsys@useobject{currentmarker}{}%
\end{pgfscope}%
\begin{pgfscope}%
\pgfsys@transformshift{1.783738in}{4.360824in}%
\pgfsys@useobject{currentmarker}{}%
\end{pgfscope}%
\begin{pgfscope}%
\pgfsys@transformshift{1.764727in}{4.417362in}%
\pgfsys@useobject{currentmarker}{}%
\end{pgfscope}%
\begin{pgfscope}%
\pgfsys@transformshift{1.746182in}{4.379967in}%
\pgfsys@useobject{currentmarker}{}%
\end{pgfscope}%
\begin{pgfscope}%
\pgfsys@transformshift{1.727874in}{4.209947in}%
\pgfsys@useobject{currentmarker}{}%
\end{pgfscope}%
\begin{pgfscope}%
\pgfsys@transformshift{1.709330in}{4.116428in}%
\pgfsys@useobject{currentmarker}{}%
\end{pgfscope}%
\begin{pgfscope}%
\pgfsys@transformshift{1.687500in}{4.062110in}%
\pgfsys@useobject{currentmarker}{}%
\end{pgfscope}%
\begin{pgfscope}%
\pgfsys@transformshift{1.669191in}{4.042172in}%
\pgfsys@useobject{currentmarker}{}%
\end{pgfscope}%
\begin{pgfscope}%
\pgfsys@transformshift{1.650881in}{4.036758in}%
\pgfsys@useobject{currentmarker}{}%
\end{pgfscope}%
\begin{pgfscope}%
\pgfsys@transformshift{1.632105in}{4.035129in}%
\pgfsys@useobject{currentmarker}{}%
\end{pgfscope}%
\begin{pgfscope}%
\pgfsys@transformshift{1.610040in}{4.037502in}%
\pgfsys@useobject{currentmarker}{}%
\end{pgfscope}%
\begin{pgfscope}%
\pgfsys@transformshift{1.592670in}{4.042256in}%
\pgfsys@useobject{currentmarker}{}%
\end{pgfscope}%
\begin{pgfscope}%
\pgfsys@transformshift{1.573422in}{4.054393in}%
\pgfsys@useobject{currentmarker}{}%
\end{pgfscope}%
\begin{pgfscope}%
\pgfsys@transformshift{1.553235in}{4.107578in}%
\pgfsys@useobject{currentmarker}{}%
\end{pgfscope}%
\begin{pgfscope}%
\pgfsys@transformshift{1.533753in}{4.192541in}%
\pgfsys@useobject{currentmarker}{}%
\end{pgfscope}%
\begin{pgfscope}%
\pgfsys@transformshift{1.514739in}{4.339393in}%
\pgfsys@useobject{currentmarker}{}%
\end{pgfscope}%
\begin{pgfscope}%
\pgfsys@transformshift{1.496429in}{4.418919in}%
\pgfsys@useobject{currentmarker}{}%
\end{pgfscope}%
\begin{pgfscope}%
\pgfsys@transformshift{1.475773in}{4.399202in}%
\pgfsys@useobject{currentmarker}{}%
\end{pgfscope}%
\begin{pgfscope}%
\pgfsys@transformshift{1.458168in}{4.245005in}%
\pgfsys@useobject{currentmarker}{}%
\end{pgfscope}%
\begin{pgfscope}%
\pgfsys@transformshift{1.438217in}{4.111704in}%
\pgfsys@useobject{currentmarker}{}%
\end{pgfscope}%
\begin{pgfscope}%
\pgfsys@transformshift{1.416856in}{4.061029in}%
\pgfsys@useobject{currentmarker}{}%
\end{pgfscope}%
\begin{pgfscope}%
\pgfsys@transformshift{1.401130in}{4.047458in}%
\pgfsys@useobject{currentmarker}{}%
\end{pgfscope}%
\begin{pgfscope}%
\pgfsys@transformshift{1.379535in}{4.037982in}%
\pgfsys@useobject{currentmarker}{}%
\end{pgfscope}%
\begin{pgfscope}%
\pgfsys@transformshift{1.360990in}{4.219359in}%
\pgfsys@useobject{currentmarker}{}%
\end{pgfscope}%
\begin{pgfscope}%
\pgfsys@transformshift{1.342213in}{4.391168in}%
\pgfsys@useobject{currentmarker}{}%
\end{pgfscope}%
\begin{pgfscope}%
\pgfsys@transformshift{1.324138in}{4.425935in}%
\pgfsys@useobject{currentmarker}{}%
\end{pgfscope}%
\begin{pgfscope}%
\pgfsys@transformshift{1.305830in}{4.356286in}%
\pgfsys@useobject{currentmarker}{}%
\end{pgfscope}%
\begin{pgfscope}%
\pgfsys@transformshift{1.283531in}{4.118347in}%
\pgfsys@useobject{currentmarker}{}%
\end{pgfscope}%
\begin{pgfscope}%
\pgfsys@transformshift{1.265692in}{4.058632in}%
\pgfsys@useobject{currentmarker}{}%
\end{pgfscope}%
\begin{pgfscope}%
\pgfsys@transformshift{1.243627in}{4.040802in}%
\pgfsys@useobject{currentmarker}{}%
\end{pgfscope}%
\begin{pgfscope}%
\pgfsys@transformshift{1.226725in}{4.037603in}%
\pgfsys@useobject{currentmarker}{}%
\end{pgfscope}%
\begin{pgfscope}%
\pgfsys@transformshift{1.205132in}{4.035957in}%
\pgfsys@useobject{currentmarker}{}%
\end{pgfscope}%
\begin{pgfscope}%
\pgfsys@transformshift{1.186587in}{4.040816in}%
\pgfsys@useobject{currentmarker}{}%
\end{pgfscope}%
\begin{pgfscope}%
\pgfsys@transformshift{1.168748in}{4.059972in}%
\pgfsys@useobject{currentmarker}{}%
\end{pgfscope}%
\begin{pgfscope}%
\pgfsys@transformshift{1.151612in}{4.098419in}%
\pgfsys@useobject{currentmarker}{}%
\end{pgfscope}%
\begin{pgfscope}%
\pgfsys@transformshift{1.129547in}{4.268986in}%
\pgfsys@useobject{currentmarker}{}%
\end{pgfscope}%
\begin{pgfscope}%
\pgfsys@transformshift{1.107954in}{4.396052in}%
\pgfsys@useobject{currentmarker}{}%
\end{pgfscope}%
\begin{pgfscope}%
\pgfsys@transformshift{1.090349in}{4.437859in}%
\pgfsys@useobject{currentmarker}{}%
\end{pgfscope}%
\begin{pgfscope}%
\pgfsys@transformshift{1.074621in}{4.397367in}%
\pgfsys@useobject{currentmarker}{}%
\end{pgfscope}%
\begin{pgfscope}%
\pgfsys@transformshift{1.055842in}{4.225391in}%
\pgfsys@useobject{currentmarker}{}%
\end{pgfscope}%
\begin{pgfscope}%
\pgfsys@transformshift{1.034952in}{4.084734in}%
\pgfsys@useobject{currentmarker}{}%
\end{pgfscope}%
\begin{pgfscope}%
\pgfsys@transformshift{1.017113in}{4.050905in}%
\pgfsys@useobject{currentmarker}{}%
\end{pgfscope}%
\begin{pgfscope}%
\pgfsys@transformshift{0.995517in}{4.038723in}%
\pgfsys@useobject{currentmarker}{}%
\end{pgfscope}%
\begin{pgfscope}%
\pgfsys@transformshift{0.977443in}{4.036154in}%
\pgfsys@useobject{currentmarker}{}%
\end{pgfscope}%
\begin{pgfscope}%
\pgfsys@transformshift{0.957961in}{4.039077in}%
\pgfsys@useobject{currentmarker}{}%
\end{pgfscope}%
\begin{pgfscope}%
\pgfsys@transformshift{0.939887in}{4.036229in}%
\pgfsys@useobject{currentmarker}{}%
\end{pgfscope}%
\begin{pgfscope}%
\pgfsys@transformshift{0.918761in}{4.044045in}%
\pgfsys@useobject{currentmarker}{}%
\end{pgfscope}%
\begin{pgfscope}%
\pgfsys@transformshift{0.902095in}{4.064268in}%
\pgfsys@useobject{currentmarker}{}%
\end{pgfscope}%
\begin{pgfscope}%
\pgfsys@transformshift{0.880970in}{4.109840in}%
\pgfsys@useobject{currentmarker}{}%
\end{pgfscope}%
\begin{pgfscope}%
\pgfsys@transformshift{0.862895in}{4.231684in}%
\pgfsys@useobject{currentmarker}{}%
\end{pgfscope}%
\begin{pgfscope}%
\pgfsys@transformshift{0.844118in}{4.397261in}%
\pgfsys@useobject{currentmarker}{}%
\end{pgfscope}%
\begin{pgfscope}%
\pgfsys@transformshift{0.822053in}{4.448759in}%
\pgfsys@useobject{currentmarker}{}%
\end{pgfscope}%
\begin{pgfscope}%
\pgfsys@transformshift{0.799049in}{4.359960in}%
\pgfsys@useobject{currentmarker}{}%
\end{pgfscope}%
\begin{pgfscope}%
\pgfsys@transformshift{0.781679in}{4.179481in}%
\pgfsys@useobject{currentmarker}{}%
\end{pgfscope}%
\begin{pgfscope}%
\pgfsys@transformshift{0.765482in}{4.090242in}%
\pgfsys@useobject{currentmarker}{}%
\end{pgfscope}%
\begin{pgfscope}%
\pgfsys@transformshift{0.745061in}{4.053050in}%
\pgfsys@useobject{currentmarker}{}%
\end{pgfscope}%
\begin{pgfscope}%
\pgfsys@transformshift{0.725813in}{4.040838in}%
\pgfsys@useobject{currentmarker}{}%
\end{pgfscope}%
\begin{pgfscope}%
\pgfsys@transformshift{0.708913in}{4.036771in}%
\pgfsys@useobject{currentmarker}{}%
\end{pgfscope}%
\begin{pgfscope}%
\pgfsys@transformshift{0.689431in}{4.038800in}%
\pgfsys@useobject{currentmarker}{}%
\end{pgfscope}%
\begin{pgfscope}%
\pgfsys@transformshift{0.669478in}{4.050263in}%
\pgfsys@useobject{currentmarker}{}%
\end{pgfscope}%
\begin{pgfscope}%
\pgfsys@transformshift{0.650934in}{4.081261in}%
\pgfsys@useobject{currentmarker}{}%
\end{pgfscope}%
\begin{pgfscope}%
\pgfsys@transformshift{0.652579in}{4.075788in}%
\pgfsys@useobject{currentmarker}{}%
\end{pgfscope}%
\begin{pgfscope}%
\pgfsys@transformshift{0.658447in}{4.056251in}%
\pgfsys@useobject{currentmarker}{}%
\end{pgfscope}%
\begin{pgfscope}%
\pgfsys@transformshift{0.677224in}{4.038602in}%
\pgfsys@useobject{currentmarker}{}%
\end{pgfscope}%
\begin{pgfscope}%
\pgfsys@transformshift{0.696237in}{4.037646in}%
\pgfsys@useobject{currentmarker}{}%
\end{pgfscope}%
\begin{pgfscope}%
\pgfsys@transformshift{0.715485in}{4.051241in}%
\pgfsys@useobject{currentmarker}{}%
\end{pgfscope}%
\begin{pgfscope}%
\pgfsys@transformshift{0.735438in}{4.097314in}%
\pgfsys@useobject{currentmarker}{}%
\end{pgfscope}%
\begin{pgfscope}%
\pgfsys@transformshift{0.756094in}{4.283908in}%
\pgfsys@useobject{currentmarker}{}%
\end{pgfscope}%
\begin{pgfscope}%
\pgfsys@transformshift{0.772994in}{4.446961in}%
\pgfsys@useobject{currentmarker}{}%
\end{pgfscope}%
\begin{pgfscope}%
\pgfsys@transformshift{0.791772in}{4.391912in}%
\pgfsys@useobject{currentmarker}{}%
\end{pgfscope}%
\begin{pgfscope}%
\pgfsys@transformshift{0.810551in}{4.193277in}%
\pgfsys@useobject{currentmarker}{}%
\end{pgfscope}%
\begin{pgfscope}%
\pgfsys@transformshift{0.830268in}{4.073696in}%
\pgfsys@useobject{currentmarker}{}%
\end{pgfscope}%
\begin{pgfscope}%
\pgfsys@transformshift{0.849515in}{4.043989in}%
\pgfsys@useobject{currentmarker}{}%
\end{pgfscope}%
\begin{pgfscope}%
\pgfsys@transformshift{0.869468in}{4.036098in}%
\pgfsys@useobject{currentmarker}{}%
\end{pgfscope}%
\begin{pgfscope}%
\pgfsys@transformshift{0.887073in}{4.041271in}%
\pgfsys@useobject{currentmarker}{}%
\end{pgfscope}%
\begin{pgfscope}%
\pgfsys@transformshift{0.906790in}{4.062873in}%
\pgfsys@useobject{currentmarker}{}%
\end{pgfscope}%
\begin{pgfscope}%
\pgfsys@transformshift{0.924863in}{4.148123in}%
\pgfsys@useobject{currentmarker}{}%
\end{pgfscope}%
\begin{pgfscope}%
\pgfsys@transformshift{0.944816in}{4.406319in}%
\pgfsys@useobject{currentmarker}{}%
\end{pgfscope}%
\begin{pgfscope}%
\pgfsys@transformshift{0.963361in}{4.410269in}%
\pgfsys@useobject{currentmarker}{}%
\end{pgfscope}%
\begin{pgfscope}%
\pgfsys@transformshift{0.984251in}{4.224066in}%
\pgfsys@useobject{currentmarker}{}%
\end{pgfscope}%
\begin{pgfscope}%
\pgfsys@transformshift{1.001856in}{4.087312in}%
\pgfsys@useobject{currentmarker}{}%
\end{pgfscope}%
\begin{pgfscope}%
\pgfsys@transformshift{1.019461in}{4.047833in}%
\pgfsys@useobject{currentmarker}{}%
\end{pgfscope}%
\begin{pgfscope}%
\pgfsys@transformshift{1.041525in}{4.036203in}%
\pgfsys@useobject{currentmarker}{}%
\end{pgfscope}%
\begin{pgfscope}%
\pgfsys@transformshift{1.060302in}{4.039060in}%
\pgfsys@useobject{currentmarker}{}%
\end{pgfscope}%
\begin{pgfscope}%
\pgfsys@transformshift{1.079081in}{4.053914in}%
\pgfsys@useobject{currentmarker}{}%
\end{pgfscope}%
\begin{pgfscope}%
\pgfsys@transformshift{1.097391in}{4.112281in}%
\pgfsys@useobject{currentmarker}{}%
\end{pgfscope}%
\begin{pgfscope}%
\pgfsys@transformshift{1.116873in}{4.329445in}%
\pgfsys@useobject{currentmarker}{}%
\end{pgfscope}%
\begin{pgfscope}%
\pgfsys@transformshift{1.135181in}{4.430561in}%
\pgfsys@useobject{currentmarker}{}%
\end{pgfscope}%
\begin{pgfscope}%
\pgfsys@transformshift{1.157951in}{4.289608in}%
\pgfsys@useobject{currentmarker}{}%
\end{pgfscope}%
\begin{pgfscope}%
\pgfsys@transformshift{1.176728in}{4.114652in}%
\pgfsys@useobject{currentmarker}{}%
\end{pgfscope}%
\begin{pgfscope}%
\pgfsys@transformshift{1.195507in}{4.052536in}%
\pgfsys@useobject{currentmarker}{}%
\end{pgfscope}%
\begin{pgfscope}%
\pgfsys@transformshift{1.214286in}{4.037757in}%
\pgfsys@useobject{currentmarker}{}%
\end{pgfscope}%
\begin{pgfscope}%
\pgfsys@transformshift{1.233533in}{4.035691in}%
\pgfsys@useobject{currentmarker}{}%
\end{pgfscope}%
\begin{pgfscope}%
\pgfsys@transformshift{1.251841in}{4.042693in}%
\pgfsys@useobject{currentmarker}{}%
\end{pgfscope}%
\begin{pgfscope}%
\pgfsys@transformshift{1.270386in}{4.062367in}%
\pgfsys@useobject{currentmarker}{}%
\end{pgfscope}%
\begin{pgfscope}%
\pgfsys@transformshift{1.292685in}{4.180977in}%
\pgfsys@useobject{currentmarker}{}%
\end{pgfscope}%
\begin{pgfscope}%
\pgfsys@transformshift{1.308178in}{4.393684in}%
\pgfsys@useobject{currentmarker}{}%
\end{pgfscope}%
\begin{pgfscope}%
\pgfsys@transformshift{1.329772in}{4.404917in}%
\pgfsys@useobject{currentmarker}{}%
\end{pgfscope}%
\begin{pgfscope}%
\pgfsys@transformshift{1.348785in}{4.262159in}%
\pgfsys@useobject{currentmarker}{}%
\end{pgfscope}%
\begin{pgfscope}%
\pgfsys@transformshift{1.366624in}{4.109136in}%
\pgfsys@useobject{currentmarker}{}%
\end{pgfscope}%
\begin{pgfscope}%
\pgfsys@transformshift{1.388220in}{4.053417in}%
\pgfsys@useobject{currentmarker}{}%
\end{pgfscope}%
\begin{pgfscope}%
\pgfsys@transformshift{1.406528in}{4.038647in}%
\pgfsys@useobject{currentmarker}{}%
\end{pgfscope}%
\begin{pgfscope}%
\pgfsys@transformshift{1.423664in}{4.035262in}%
\pgfsys@useobject{currentmarker}{}%
\end{pgfscope}%
\begin{pgfscope}%
\pgfsys@transformshift{1.445963in}{4.040742in}%
\pgfsys@useobject{currentmarker}{}%
\end{pgfscope}%
\begin{pgfscope}%
\pgfsys@transformshift{1.463333in}{4.058288in}%
\pgfsys@useobject{currentmarker}{}%
\end{pgfscope}%
\begin{pgfscope}%
\pgfsys@transformshift{1.480938in}{4.122583in}%
\pgfsys@useobject{currentmarker}{}%
\end{pgfscope}%
\begin{pgfscope}%
\pgfsys@transformshift{1.502298in}{4.306708in}%
\pgfsys@useobject{currentmarker}{}%
\end{pgfscope}%
\begin{pgfscope}%
\pgfsys@transformshift{1.519668in}{4.417817in}%
\pgfsys@useobject{currentmarker}{}%
\end{pgfscope}%
\begin{pgfscope}%
\pgfsys@transformshift{1.540793in}{4.371714in}%
\pgfsys@useobject{currentmarker}{}%
\end{pgfscope}%
\begin{pgfscope}%
\pgfsys@transformshift{1.558398in}{4.214353in}%
\pgfsys@useobject{currentmarker}{}%
\end{pgfscope}%
\begin{pgfscope}%
\pgfsys@transformshift{1.580933in}{4.073350in}%
\pgfsys@useobject{currentmarker}{}%
\end{pgfscope}%
\begin{pgfscope}%
\pgfsys@transformshift{1.598538in}{4.043508in}%
\pgfsys@useobject{currentmarker}{}%
\end{pgfscope}%
\begin{pgfscope}%
\pgfsys@transformshift{1.616377in}{4.036572in}%
\pgfsys@useobject{currentmarker}{}%
\end{pgfscope}%
\begin{pgfscope}%
\pgfsys@transformshift{1.636328in}{4.270387in}%
\pgfsys@useobject{currentmarker}{}%
\end{pgfscope}%
\begin{pgfscope}%
\pgfsys@transformshift{1.653933in}{4.129227in}%
\pgfsys@useobject{currentmarker}{}%
\end{pgfscope}%
\begin{pgfscope}%
\pgfsys@transformshift{1.676703in}{4.052602in}%
\pgfsys@useobject{currentmarker}{}%
\end{pgfscope}%
\begin{pgfscope}%
\pgfsys@transformshift{1.693133in}{4.038804in}%
\pgfsys@useobject{currentmarker}{}%
\end{pgfscope}%
\begin{pgfscope}%
\pgfsys@transformshift{1.718015in}{4.034959in}%
\pgfsys@useobject{currentmarker}{}%
\end{pgfscope}%
\begin{pgfscope}%
\pgfsys@transformshift{1.734211in}{4.038367in}%
\pgfsys@useobject{currentmarker}{}%
\end{pgfscope}%
\begin{pgfscope}%
\pgfsys@transformshift{1.751582in}{4.047110in}%
\pgfsys@useobject{currentmarker}{}%
\end{pgfscope}%
\begin{pgfscope}%
\pgfsys@transformshift{1.772001in}{4.087375in}%
\pgfsys@useobject{currentmarker}{}%
\end{pgfscope}%
\begin{pgfscope}%
\pgfsys@transformshift{1.790077in}{4.224237in}%
\pgfsys@useobject{currentmarker}{}%
\end{pgfscope}%
\begin{pgfscope}%
\pgfsys@transformshift{1.811907in}{4.406879in}%
\pgfsys@useobject{currentmarker}{}%
\end{pgfscope}%
\begin{pgfscope}%
\pgfsys@transformshift{1.827398in}{4.394260in}%
\pgfsys@useobject{currentmarker}{}%
\end{pgfscope}%
\begin{pgfscope}%
\pgfsys@transformshift{1.847820in}{4.206241in}%
\pgfsys@useobject{currentmarker}{}%
\end{pgfscope}%
\begin{pgfscope}%
\pgfsys@transformshift{1.868945in}{4.072949in}%
\pgfsys@useobject{currentmarker}{}%
\end{pgfscope}%
\begin{pgfscope}%
\pgfsys@transformshift{1.887021in}{4.043781in}%
\pgfsys@useobject{currentmarker}{}%
\end{pgfscope}%
\begin{pgfscope}%
\pgfsys@transformshift{1.906032in}{4.035758in}%
\pgfsys@useobject{currentmarker}{}%
\end{pgfscope}%
\begin{pgfscope}%
\pgfsys@transformshift{1.924811in}{4.034857in}%
\pgfsys@useobject{currentmarker}{}%
\end{pgfscope}%
\begin{pgfscope}%
\pgfsys@transformshift{1.945467in}{4.040877in}%
\pgfsys@useobject{currentmarker}{}%
\end{pgfscope}%
\begin{pgfscope}%
\pgfsys@transformshift{1.963306in}{4.058281in}%
\pgfsys@useobject{currentmarker}{}%
\end{pgfscope}%
\begin{pgfscope}%
\pgfsys@transformshift{1.981616in}{4.111773in}%
\pgfsys@useobject{currentmarker}{}%
\end{pgfscope}%
\begin{pgfscope}%
\pgfsys@transformshift{2.002272in}{4.249419in}%
\pgfsys@useobject{currentmarker}{}%
\end{pgfscope}%
\begin{pgfscope}%
\pgfsys@transformshift{2.020815in}{4.391298in}%
\pgfsys@useobject{currentmarker}{}%
\end{pgfscope}%
\begin{pgfscope}%
\pgfsys@transformshift{2.041471in}{4.390295in}%
\pgfsys@useobject{currentmarker}{}%
\end{pgfscope}%
\begin{pgfscope}%
\pgfsys@transformshift{2.059781in}{4.237783in}%
\pgfsys@useobject{currentmarker}{}%
\end{pgfscope}%
\begin{pgfscope}%
\pgfsys@transformshift{2.079732in}{4.082829in}%
\pgfsys@useobject{currentmarker}{}%
\end{pgfscope}%
\begin{pgfscope}%
\pgfsys@transformshift{2.098511in}{4.046103in}%
\pgfsys@useobject{currentmarker}{}%
\end{pgfscope}%
\begin{pgfscope}%
\pgfsys@transformshift{2.115881in}{4.037389in}%
\pgfsys@useobject{currentmarker}{}%
\end{pgfscope}%
\begin{pgfscope}%
\pgfsys@transformshift{2.137711in}{4.034924in}%
\pgfsys@useobject{currentmarker}{}%
\end{pgfscope}%
\begin{pgfscope}%
\pgfsys@transformshift{2.155316in}{4.038877in}%
\pgfsys@useobject{currentmarker}{}%
\end{pgfscope}%
\begin{pgfscope}%
\pgfsys@transformshift{2.172921in}{4.045191in}%
\pgfsys@useobject{currentmarker}{}%
\end{pgfscope}%
\begin{pgfscope}%
\pgfsys@transformshift{2.195454in}{4.088551in}%
\pgfsys@useobject{currentmarker}{}%
\end{pgfscope}%
\begin{pgfscope}%
\pgfsys@transformshift{2.212119in}{4.199240in}%
\pgfsys@useobject{currentmarker}{}%
\end{pgfscope}%
\begin{pgfscope}%
\pgfsys@transformshift{2.233715in}{4.402099in}%
\pgfsys@useobject{currentmarker}{}%
\end{pgfscope}%
\begin{pgfscope}%
\pgfsys@transformshift{2.251554in}{4.393014in}%
\pgfsys@useobject{currentmarker}{}%
\end{pgfscope}%
\begin{pgfscope}%
\pgfsys@transformshift{2.269159in}{4.296221in}%
\pgfsys@useobject{currentmarker}{}%
\end{pgfscope}%
\begin{pgfscope}%
\pgfsys@transformshift{2.290989in}{4.126289in}%
\pgfsys@useobject{currentmarker}{}%
\end{pgfscope}%
\begin{pgfscope}%
\pgfsys@transformshift{2.308360in}{4.056290in}%
\pgfsys@useobject{currentmarker}{}%
\end{pgfscope}%
\begin{pgfscope}%
\pgfsys@transformshift{2.329954in}{4.039208in}%
\pgfsys@useobject{currentmarker}{}%
\end{pgfscope}%
\begin{pgfscope}%
\pgfsys@transformshift{2.347558in}{4.035363in}%
\pgfsys@useobject{currentmarker}{}%
\end{pgfscope}%
\begin{pgfscope}%
\pgfsys@transformshift{2.365163in}{4.035619in}%
\pgfsys@useobject{currentmarker}{}%
\end{pgfscope}%
\begin{pgfscope}%
\pgfsys@transformshift{2.386288in}{4.041134in}%
\pgfsys@useobject{currentmarker}{}%
\end{pgfscope}%
\begin{pgfscope}%
\pgfsys@transformshift{2.404598in}{4.060243in}%
\pgfsys@useobject{currentmarker}{}%
\end{pgfscope}%
\begin{pgfscope}%
\pgfsys@transformshift{2.425254in}{4.055848in}%
\pgfsys@useobject{currentmarker}{}%
\end{pgfscope}%
\begin{pgfscope}%
\pgfsys@transformshift{2.439337in}{4.096052in}%
\pgfsys@useobject{currentmarker}{}%
\end{pgfscope}%
\begin{pgfscope}%
\pgfsys@transformshift{2.461402in}{4.201607in}%
\pgfsys@useobject{currentmarker}{}%
\end{pgfscope}%
\begin{pgfscope}%
\pgfsys@transformshift{2.482058in}{4.405927in}%
\pgfsys@useobject{currentmarker}{}%
\end{pgfscope}%
\begin{pgfscope}%
\pgfsys@transformshift{2.503419in}{4.360177in}%
\pgfsys@useobject{currentmarker}{}%
\end{pgfscope}%
\begin{pgfscope}%
\pgfsys@transformshift{2.521493in}{4.245235in}%
\pgfsys@useobject{currentmarker}{}%
\end{pgfscope}%
\begin{pgfscope}%
\pgfsys@transformshift{2.539566in}{4.105847in}%
\pgfsys@useobject{currentmarker}{}%
\end{pgfscope}%
\begin{pgfscope}%
\pgfsys@transformshift{2.563276in}{4.049241in}%
\pgfsys@useobject{currentmarker}{}%
\end{pgfscope}%
\begin{pgfscope}%
\pgfsys@transformshift{2.579236in}{4.038785in}%
\pgfsys@useobject{currentmarker}{}%
\end{pgfscope}%
\begin{pgfscope}%
\pgfsys@transformshift{2.596841in}{4.035001in}%
\pgfsys@useobject{currentmarker}{}%
\end{pgfscope}%
\begin{pgfscope}%
\pgfsys@transformshift{2.616325in}{4.036950in}%
\pgfsys@useobject{currentmarker}{}%
\end{pgfscope}%
\begin{pgfscope}%
\pgfsys@transformshift{2.637450in}{4.045735in}%
\pgfsys@useobject{currentmarker}{}%
\end{pgfscope}%
\begin{pgfscope}%
\pgfsys@transformshift{2.653177in}{4.070312in}%
\pgfsys@useobject{currentmarker}{}%
\end{pgfscope}%
\begin{pgfscope}%
\pgfsys@transformshift{2.674068in}{4.174376in}%
\pgfsys@useobject{currentmarker}{}%
\end{pgfscope}%
\begin{pgfscope}%
\pgfsys@transformshift{2.692610in}{4.349005in}%
\pgfsys@useobject{currentmarker}{}%
\end{pgfscope}%
\begin{pgfscope}%
\pgfsys@transformshift{2.713266in}{4.405991in}%
\pgfsys@useobject{currentmarker}{}%
\end{pgfscope}%
\begin{pgfscope}%
\pgfsys@transformshift{2.732045in}{4.349389in}%
\pgfsys@useobject{currentmarker}{}%
\end{pgfscope}%
\begin{pgfscope}%
\pgfsys@transformshift{2.750119in}{4.231822in}%
\pgfsys@useobject{currentmarker}{}%
\end{pgfscope}%
\begin{pgfscope}%
\pgfsys@transformshift{2.770306in}{4.131822in}%
\pgfsys@useobject{currentmarker}{}%
\end{pgfscope}%
\begin{pgfscope}%
\pgfsys@transformshift{2.788616in}{4.058233in}%
\pgfsys@useobject{currentmarker}{}%
\end{pgfscope}%
\begin{pgfscope}%
\pgfsys@transformshift{2.811150in}{4.040357in}%
\pgfsys@useobject{currentmarker}{}%
\end{pgfscope}%
\begin{pgfscope}%
\pgfsys@transformshift{2.827580in}{4.036464in}%
\pgfsys@useobject{currentmarker}{}%
\end{pgfscope}%
\begin{pgfscope}%
\pgfsys@transformshift{2.845420in}{4.034994in}%
\pgfsys@useobject{currentmarker}{}%
\end{pgfscope}%
\begin{pgfscope}%
\pgfsys@transformshift{2.866544in}{4.037344in}%
\pgfsys@useobject{currentmarker}{}%
\end{pgfscope}%
\begin{pgfscope}%
\pgfsys@transformshift{2.885323in}{4.046344in}%
\pgfsys@useobject{currentmarker}{}%
\end{pgfscope}%
\begin{pgfscope}%
\pgfsys@transformshift{2.903633in}{4.072322in}%
\pgfsys@useobject{currentmarker}{}%
\end{pgfscope}%
\begin{pgfscope}%
\pgfsys@transformshift{2.923819in}{4.168887in}%
\pgfsys@useobject{currentmarker}{}%
\end{pgfscope}%
\begin{pgfscope}%
\pgfsys@transformshift{2.942363in}{4.366186in}%
\pgfsys@useobject{currentmarker}{}%
\end{pgfscope}%
\begin{pgfscope}%
\pgfsys@transformshift{2.962550in}{4.411323in}%
\pgfsys@useobject{currentmarker}{}%
\end{pgfscope}%
\begin{pgfscope}%
\pgfsys@transformshift{2.980858in}{4.352940in}%
\pgfsys@useobject{currentmarker}{}%
\end{pgfscope}%
\begin{pgfscope}%
\pgfsys@transformshift{2.999167in}{4.201268in}%
\pgfsys@useobject{currentmarker}{}%
\end{pgfscope}%
\begin{pgfscope}%
\pgfsys@transformshift{3.020059in}{4.079580in}%
\pgfsys@useobject{currentmarker}{}%
\end{pgfscope}%
\begin{pgfscope}%
\pgfsys@transformshift{3.037664in}{4.052856in}%
\pgfsys@useobject{currentmarker}{}%
\end{pgfscope}%
\begin{pgfscope}%
\pgfsys@transformshift{3.057146in}{4.038829in}%
\pgfsys@useobject{currentmarker}{}%
\end{pgfscope}%
\begin{pgfscope}%
\pgfsys@transformshift{3.077333in}{4.035264in}%
\pgfsys@useobject{currentmarker}{}%
\end{pgfscope}%
\begin{pgfscope}%
\pgfsys@transformshift{3.096815in}{4.036376in}%
\pgfsys@useobject{currentmarker}{}%
\end{pgfscope}%
\begin{pgfscope}%
\pgfsys@transformshift{3.114420in}{4.042148in}%
\pgfsys@useobject{currentmarker}{}%
\end{pgfscope}%
\begin{pgfscope}%
\pgfsys@transformshift{3.134842in}{4.064618in}%
\pgfsys@useobject{currentmarker}{}%
\end{pgfscope}%
\begin{pgfscope}%
\pgfsys@transformshift{3.154324in}{4.096790in}%
\pgfsys@useobject{currentmarker}{}%
\end{pgfscope}%
\begin{pgfscope}%
\pgfsys@transformshift{3.174746in}{4.188662in}%
\pgfsys@useobject{currentmarker}{}%
\end{pgfscope}%
\begin{pgfscope}%
\pgfsys@transformshift{3.193523in}{4.383782in}%
\pgfsys@useobject{currentmarker}{}%
\end{pgfscope}%
\begin{pgfscope}%
\pgfsys@transformshift{3.215822in}{4.404554in}%
\pgfsys@useobject{currentmarker}{}%
\end{pgfscope}%
\begin{pgfscope}%
\pgfsys@transformshift{3.231080in}{4.341760in}%
\pgfsys@useobject{currentmarker}{}%
\end{pgfscope}%
\begin{pgfscope}%
\pgfsys@transformshift{3.251502in}{4.196588in}%
\pgfsys@useobject{currentmarker}{}%
\end{pgfscope}%
\begin{pgfscope}%
\pgfsys@transformshift{3.269107in}{4.085212in}%
\pgfsys@useobject{currentmarker}{}%
\end{pgfscope}%
\begin{pgfscope}%
\pgfsys@transformshift{3.288120in}{4.066077in}%
\pgfsys@useobject{currentmarker}{}%
\end{pgfscope}%
\begin{pgfscope}%
\pgfsys@transformshift{3.308540in}{4.042435in}%
\pgfsys@useobject{currentmarker}{}%
\end{pgfscope}%
\begin{pgfscope}%
\pgfsys@transformshift{3.328727in}{4.036327in}%
\pgfsys@useobject{currentmarker}{}%
\end{pgfscope}%
\begin{pgfscope}%
\pgfsys@transformshift{3.346566in}{4.035291in}%
\pgfsys@useobject{currentmarker}{}%
\end{pgfscope}%
\begin{pgfscope}%
\pgfsys@transformshift{3.363937in}{4.036749in}%
\pgfsys@useobject{currentmarker}{}%
\end{pgfscope}%
\begin{pgfscope}%
\pgfsys@transformshift{3.385298in}{4.045927in}%
\pgfsys@useobject{currentmarker}{}%
\end{pgfscope}%
\begin{pgfscope}%
\pgfsys@transformshift{3.407363in}{4.063918in}%
\pgfsys@useobject{currentmarker}{}%
\end{pgfscope}%
\begin{pgfscope}%
\pgfsys@transformshift{3.424262in}{4.104531in}%
\pgfsys@useobject{currentmarker}{}%
\end{pgfscope}%
\begin{pgfscope}%
\pgfsys@transformshift{3.442101in}{4.229679in}%
\pgfsys@useobject{currentmarker}{}%
\end{pgfscope}%
\begin{pgfscope}%
\pgfsys@transformshift{3.463697in}{4.395421in}%
\pgfsys@useobject{currentmarker}{}%
\end{pgfscope}%
\begin{pgfscope}%
\pgfsys@transformshift{3.480831in}{4.422773in}%
\pgfsys@useobject{currentmarker}{}%
\end{pgfscope}%
\begin{pgfscope}%
\pgfsys@transformshift{3.501724in}{4.337532in}%
\pgfsys@useobject{currentmarker}{}%
\end{pgfscope}%
\begin{pgfscope}%
\pgfsys@transformshift{3.519329in}{4.338465in}%
\pgfsys@useobject{currentmarker}{}%
\end{pgfscope}%
\begin{pgfscope}%
\pgfsys@transformshift{3.537637in}{4.195161in}%
\pgfsys@useobject{currentmarker}{}%
\end{pgfscope}%
\begin{pgfscope}%
\pgfsys@transformshift{3.558058in}{4.080328in}%
\pgfsys@useobject{currentmarker}{}%
\end{pgfscope}%
\begin{pgfscope}%
\pgfsys@transformshift{3.576366in}{4.052295in}%
\pgfsys@useobject{currentmarker}{}%
\end{pgfscope}%
\begin{pgfscope}%
\pgfsys@transformshift{3.594911in}{4.041578in}%
\pgfsys@useobject{currentmarker}{}%
\end{pgfscope}%
\begin{pgfscope}%
\pgfsys@transformshift{3.615801in}{4.035886in}%
\pgfsys@useobject{currentmarker}{}%
\end{pgfscope}%
\begin{pgfscope}%
\pgfsys@transformshift{3.633875in}{4.036370in}%
\pgfsys@useobject{currentmarker}{}%
\end{pgfscope}%
\begin{pgfscope}%
\pgfsys@transformshift{3.655236in}{4.043534in}%
\pgfsys@useobject{currentmarker}{}%
\end{pgfscope}%
\begin{pgfscope}%
\pgfsys@transformshift{3.673544in}{4.050587in}%
\pgfsys@useobject{currentmarker}{}%
\end{pgfscope}%
\begin{pgfscope}%
\pgfsys@transformshift{3.691854in}{4.079720in}%
\pgfsys@useobject{currentmarker}{}%
\end{pgfscope}%
\begin{pgfscope}%
\pgfsys@transformshift{3.715796in}{4.190552in}%
\pgfsys@useobject{currentmarker}{}%
\end{pgfscope}%
\begin{pgfscope}%
\pgfsys@transformshift{3.731053in}{4.362710in}%
\pgfsys@useobject{currentmarker}{}%
\end{pgfscope}%
\begin{pgfscope}%
\pgfsys@transformshift{3.748189in}{4.432535in}%
\pgfsys@useobject{currentmarker}{}%
\end{pgfscope}%
\begin{pgfscope}%
\pgfsys@transformshift{3.770722in}{4.425635in}%
\pgfsys@useobject{currentmarker}{}%
\end{pgfscope}%
\begin{pgfscope}%
\pgfsys@transformshift{3.790910in}{4.321894in}%
\pgfsys@useobject{currentmarker}{}%
\end{pgfscope}%
\begin{pgfscope}%
\pgfsys@transformshift{3.809452in}{4.192202in}%
\pgfsys@useobject{currentmarker}{}%
\end{pgfscope}%
\begin{pgfscope}%
\pgfsys@transformshift{3.828231in}{4.112715in}%
\pgfsys@useobject{currentmarker}{}%
\end{pgfscope}%
\begin{pgfscope}%
\pgfsys@transformshift{3.845367in}{4.065445in}%
\pgfsys@useobject{currentmarker}{}%
\end{pgfscope}%
\begin{pgfscope}%
\pgfsys@transformshift{3.866726in}{4.045524in}%
\pgfsys@useobject{currentmarker}{}%
\end{pgfscope}%
\begin{pgfscope}%
\pgfsys@transformshift{3.884566in}{4.037559in}%
\pgfsys@useobject{currentmarker}{}%
\end{pgfscope}%
\begin{pgfscope}%
\pgfsys@transformshift{3.906161in}{4.035977in}%
\pgfsys@useobject{currentmarker}{}%
\end{pgfscope}%
\begin{pgfscope}%
\pgfsys@transformshift{3.920715in}{4.037177in}%
\pgfsys@useobject{currentmarker}{}%
\end{pgfscope}%
\begin{pgfscope}%
\pgfsys@transformshift{3.943014in}{4.046625in}%
\pgfsys@useobject{currentmarker}{}%
\end{pgfscope}%
\begin{pgfscope}%
\pgfsys@transformshift{3.961558in}{4.068932in}%
\pgfsys@useobject{currentmarker}{}%
\end{pgfscope}%
\begin{pgfscope}%
\pgfsys@transformshift{3.977755in}{4.092836in}%
\pgfsys@useobject{currentmarker}{}%
\end{pgfscope}%
\begin{pgfscope}%
\pgfsys@transformshift{4.000054in}{4.176321in}%
\pgfsys@useobject{currentmarker}{}%
\end{pgfscope}%
\begin{pgfscope}%
\pgfsys@transformshift{4.023996in}{4.323211in}%
\pgfsys@useobject{currentmarker}{}%
\end{pgfscope}%
\begin{pgfscope}%
\pgfsys@transformshift{4.038783in}{4.433413in}%
\pgfsys@useobject{currentmarker}{}%
\end{pgfscope}%
\begin{pgfscope}%
\pgfsys@transformshift{4.056388in}{4.057735in}%
\pgfsys@useobject{currentmarker}{}%
\end{pgfscope}%
\begin{pgfscope}%
\pgfsys@transformshift{4.077750in}{4.124211in}%
\pgfsys@useobject{currentmarker}{}%
\end{pgfscope}%
\begin{pgfscope}%
\pgfsys@transformshift{4.097701in}{4.278522in}%
\pgfsys@useobject{currentmarker}{}%
\end{pgfscope}%
\begin{pgfscope}%
\pgfsys@transformshift{4.114602in}{4.435411in}%
\pgfsys@useobject{currentmarker}{}%
\end{pgfscope}%
\begin{pgfscope}%
\pgfsys@transformshift{4.136667in}{4.429563in}%
\pgfsys@useobject{currentmarker}{}%
\end{pgfscope}%
\begin{pgfscope}%
\pgfsys@transformshift{4.153332in}{4.356754in}%
\pgfsys@useobject{currentmarker}{}%
\end{pgfscope}%
\begin{pgfscope}%
\pgfsys@transformshift{4.174222in}{4.169938in}%
\pgfsys@useobject{currentmarker}{}%
\end{pgfscope}%
\begin{pgfscope}%
\pgfsys@transformshift{4.194175in}{4.085847in}%
\pgfsys@useobject{currentmarker}{}%
\end{pgfscope}%
\begin{pgfscope}%
\pgfsys@transformshift{4.212483in}{4.051992in}%
\pgfsys@useobject{currentmarker}{}%
\end{pgfscope}%
\begin{pgfscope}%
\pgfsys@transformshift{4.231497in}{4.039942in}%
\pgfsys@useobject{currentmarker}{}%
\end{pgfscope}%
\begin{pgfscope}%
\pgfsys@transformshift{4.250510in}{4.036579in}%
\pgfsys@useobject{currentmarker}{}%
\end{pgfscope}%
\begin{pgfscope}%
\pgfsys@transformshift{4.269289in}{4.040155in}%
\pgfsys@useobject{currentmarker}{}%
\end{pgfscope}%
\begin{pgfscope}%
\pgfsys@transformshift{4.289709in}{4.052079in}%
\pgfsys@useobject{currentmarker}{}%
\end{pgfscope}%
\begin{pgfscope}%
\pgfsys@transformshift{4.306141in}{4.072130in}%
\pgfsys@useobject{currentmarker}{}%
\end{pgfscope}%
\begin{pgfscope}%
\pgfsys@transformshift{4.324918in}{4.128276in}%
\pgfsys@useobject{currentmarker}{}%
\end{pgfscope}%
\begin{pgfscope}%
\pgfsys@transformshift{4.345340in}{4.289828in}%
\pgfsys@useobject{currentmarker}{}%
\end{pgfscope}%
\begin{pgfscope}%
\pgfsys@transformshift{4.365996in}{4.448749in}%
\pgfsys@useobject{currentmarker}{}%
\end{pgfscope}%
\begin{pgfscope}%
\pgfsys@transformshift{4.385009in}{4.453948in}%
\pgfsys@useobject{currentmarker}{}%
\end{pgfscope}%
\begin{pgfscope}%
\pgfsys@transformshift{4.404491in}{4.389806in}%
\pgfsys@useobject{currentmarker}{}%
\end{pgfscope}%
\begin{pgfscope}%
\pgfsys@transformshift{4.422331in}{4.267444in}%
\pgfsys@useobject{currentmarker}{}%
\end{pgfscope}%
\begin{pgfscope}%
\pgfsys@transformshift{4.442518in}{4.123348in}%
\pgfsys@useobject{currentmarker}{}%
\end{pgfscope}%
\begin{pgfscope}%
\pgfsys@transformshift{4.461531in}{4.062858in}%
\pgfsys@useobject{currentmarker}{}%
\end{pgfscope}%
\begin{pgfscope}%
\pgfsys@transformshift{4.480779in}{4.046211in}%
\pgfsys@useobject{currentmarker}{}%
\end{pgfscope}%
\begin{pgfscope}%
\pgfsys@transformshift{4.482187in}{4.045914in}%
\pgfsys@useobject{currentmarker}{}%
\end{pgfscope}%
\begin{pgfscope}%
\pgfsys@transformshift{4.476553in}{4.052332in}%
\pgfsys@useobject{currentmarker}{}%
\end{pgfscope}%
\begin{pgfscope}%
\pgfsys@transformshift{4.456368in}{4.105562in}%
\pgfsys@useobject{currentmarker}{}%
\end{pgfscope}%
\begin{pgfscope}%
\pgfsys@transformshift{4.435712in}{4.314261in}%
\pgfsys@useobject{currentmarker}{}%
\end{pgfscope}%
\begin{pgfscope}%
\pgfsys@transformshift{4.414819in}{4.449794in}%
\pgfsys@useobject{currentmarker}{}%
\end{pgfscope}%
\begin{pgfscope}%
\pgfsys@transformshift{4.398623in}{4.436595in}%
\pgfsys@useobject{currentmarker}{}%
\end{pgfscope}%
\begin{pgfscope}%
\pgfsys@transformshift{4.377029in}{4.179366in}%
\pgfsys@useobject{currentmarker}{}%
\end{pgfscope}%
\begin{pgfscope}%
\pgfsys@transformshift{4.359893in}{4.071537in}%
\pgfsys@useobject{currentmarker}{}%
\end{pgfscope}%
\begin{pgfscope}%
\pgfsys@transformshift{4.340177in}{4.044059in}%
\pgfsys@useobject{currentmarker}{}%
\end{pgfscope}%
\begin{pgfscope}%
\pgfsys@transformshift{4.318581in}{4.036540in}%
\pgfsys@useobject{currentmarker}{}%
\end{pgfscope}%
\begin{pgfscope}%
\pgfsys@transformshift{4.297222in}{4.045373in}%
\pgfsys@useobject{currentmarker}{}%
\end{pgfscope}%
\begin{pgfscope}%
\pgfsys@transformshift{4.283371in}{4.067851in}%
\pgfsys@useobject{currentmarker}{}%
\end{pgfscope}%
\begin{pgfscope}%
\pgfsys@transformshift{4.259898in}{4.223925in}%
\pgfsys@useobject{currentmarker}{}%
\end{pgfscope}%
\begin{pgfscope}%
\pgfsys@transformshift{4.242059in}{4.400183in}%
\pgfsys@useobject{currentmarker}{}%
\end{pgfscope}%
\begin{pgfscope}%
\pgfsys@transformshift{4.224689in}{4.447118in}%
\pgfsys@useobject{currentmarker}{}%
\end{pgfscope}%
\begin{pgfscope}%
\pgfsys@transformshift{4.203798in}{4.265981in}%
\pgfsys@useobject{currentmarker}{}%
\end{pgfscope}%
\begin{pgfscope}%
\pgfsys@transformshift{4.186193in}{4.101480in}%
\pgfsys@useobject{currentmarker}{}%
\end{pgfscope}%
\begin{pgfscope}%
\pgfsys@transformshift{4.165303in}{4.048325in}%
\pgfsys@useobject{currentmarker}{}%
\end{pgfscope}%
\begin{pgfscope}%
\pgfsys@transformshift{4.147932in}{4.036999in}%
\pgfsys@useobject{currentmarker}{}%
\end{pgfscope}%
\begin{pgfscope}%
\pgfsys@transformshift{4.127042in}{4.038235in}%
\pgfsys@useobject{currentmarker}{}%
\end{pgfscope}%
\begin{pgfscope}%
\pgfsys@transformshift{4.109437in}{4.050451in}%
\pgfsys@useobject{currentmarker}{}%
\end{pgfscope}%
\begin{pgfscope}%
\pgfsys@transformshift{4.091598in}{4.097200in}%
\pgfsys@useobject{currentmarker}{}%
\end{pgfscope}%
\begin{pgfscope}%
\pgfsys@transformshift{4.070942in}{4.282130in}%
\pgfsys@useobject{currentmarker}{}%
\end{pgfscope}%
\begin{pgfscope}%
\pgfsys@transformshift{4.052868in}{4.428578in}%
\pgfsys@useobject{currentmarker}{}%
\end{pgfscope}%
\begin{pgfscope}%
\pgfsys@transformshift{4.032915in}{4.413608in}%
\pgfsys@useobject{currentmarker}{}%
\end{pgfscope}%
\begin{pgfscope}%
\pgfsys@transformshift{4.012728in}{4.153597in}%
\pgfsys@useobject{currentmarker}{}%
\end{pgfscope}%
\begin{pgfscope}%
\pgfsys@transformshift{3.994889in}{4.066140in}%
\pgfsys@useobject{currentmarker}{}%
\end{pgfscope}%
\begin{pgfscope}%
\pgfsys@transformshift{3.975641in}{4.040987in}%
\pgfsys@useobject{currentmarker}{}%
\end{pgfscope}%
\begin{pgfscope}%
\pgfsys@transformshift{3.954282in}{4.035664in}%
\pgfsys@useobject{currentmarker}{}%
\end{pgfscope}%
\begin{pgfscope}%
\pgfsys@transformshift{3.935737in}{4.040056in}%
\pgfsys@useobject{currentmarker}{}%
\end{pgfscope}%
\begin{pgfscope}%
\pgfsys@transformshift{3.918838in}{4.059620in}%
\pgfsys@useobject{currentmarker}{}%
\end{pgfscope}%
\begin{pgfscope}%
\pgfsys@transformshift{3.900059in}{4.147131in}%
\pgfsys@useobject{currentmarker}{}%
\end{pgfscope}%
\begin{pgfscope}%
\pgfsys@transformshift{3.875177in}{4.384869in}%
\pgfsys@useobject{currentmarker}{}%
\end{pgfscope}%
\begin{pgfscope}%
\pgfsys@transformshift{3.860389in}{4.432088in}%
\pgfsys@useobject{currentmarker}{}%
\end{pgfscope}%
\begin{pgfscope}%
\pgfsys@transformshift{3.838794in}{4.250340in}%
\pgfsys@useobject{currentmarker}{}%
\end{pgfscope}%
\begin{pgfscope}%
\pgfsys@transformshift{3.820720in}{4.091384in}%
\pgfsys@useobject{currentmarker}{}%
\end{pgfscope}%
\begin{pgfscope}%
\pgfsys@transformshift{3.802646in}{4.048805in}%
\pgfsys@useobject{currentmarker}{}%
\end{pgfscope}%
\begin{pgfscope}%
\pgfsys@transformshift{3.780582in}{4.036589in}%
\pgfsys@useobject{currentmarker}{}%
\end{pgfscope}%
\begin{pgfscope}%
\pgfsys@transformshift{3.762508in}{4.035626in}%
\pgfsys@useobject{currentmarker}{}%
\end{pgfscope}%
\begin{pgfscope}%
\pgfsys@transformshift{3.744198in}{4.040864in}%
\pgfsys@useobject{currentmarker}{}%
\end{pgfscope}%
\begin{pgfscope}%
\pgfsys@transformshift{3.725656in}{4.063828in}%
\pgfsys@useobject{currentmarker}{}%
\end{pgfscope}%
\begin{pgfscope}%
\pgfsys@transformshift{3.705234in}{4.149240in}%
\pgfsys@useobject{currentmarker}{}%
\end{pgfscope}%
\begin{pgfscope}%
\pgfsys@transformshift{3.686689in}{4.212827in}%
\pgfsys@useobject{currentmarker}{}%
\end{pgfscope}%
\begin{pgfscope}%
\pgfsys@transformshift{3.668147in}{4.373645in}%
\pgfsys@useobject{currentmarker}{}%
\end{pgfscope}%
\begin{pgfscope}%
\pgfsys@transformshift{3.649134in}{4.424983in}%
\pgfsys@useobject{currentmarker}{}%
\end{pgfscope}%
\begin{pgfscope}%
\pgfsys@transformshift{3.628712in}{4.266427in}%
\pgfsys@useobject{currentmarker}{}%
\end{pgfscope}%
\begin{pgfscope}%
\pgfsys@transformshift{3.612516in}{4.100490in}%
\pgfsys@useobject{currentmarker}{}%
\end{pgfscope}%
\begin{pgfscope}%
\pgfsys@transformshift{3.589982in}{4.047182in}%
\pgfsys@useobject{currentmarker}{}%
\end{pgfscope}%
\begin{pgfscope}%
\pgfsys@transformshift{3.570029in}{4.036580in}%
\pgfsys@useobject{currentmarker}{}%
\end{pgfscope}%
\begin{pgfscope}%
\pgfsys@transformshift{3.551016in}{4.035219in}%
\pgfsys@useobject{currentmarker}{}%
\end{pgfscope}%
\begin{pgfscope}%
\pgfsys@transformshift{3.531768in}{4.039324in}%
\pgfsys@useobject{currentmarker}{}%
\end{pgfscope}%
\begin{pgfscope}%
\pgfsys@transformshift{3.513929in}{4.061783in}%
\pgfsys@useobject{currentmarker}{}%
\end{pgfscope}%
\begin{pgfscope}%
\pgfsys@transformshift{3.492099in}{4.170268in}%
\pgfsys@useobject{currentmarker}{}%
\end{pgfscope}%
\begin{pgfscope}%
\pgfsys@transformshift{3.476842in}{4.301663in}%
\pgfsys@useobject{currentmarker}{}%
\end{pgfscope}%
\begin{pgfscope}%
\pgfsys@transformshift{3.455246in}{4.415877in}%
\pgfsys@useobject{currentmarker}{}%
\end{pgfscope}%
\begin{pgfscope}%
\pgfsys@transformshift{3.436702in}{4.387233in}%
\pgfsys@useobject{currentmarker}{}%
\end{pgfscope}%
\begin{pgfscope}%
\pgfsys@transformshift{3.414874in}{4.147589in}%
\pgfsys@useobject{currentmarker}{}%
\end{pgfscope}%
\begin{pgfscope}%
\pgfsys@transformshift{3.399146in}{4.083046in}%
\pgfsys@useobject{currentmarker}{}%
\end{pgfscope}%
\begin{pgfscope}%
\pgfsys@transformshift{3.377785in}{4.046211in}%
\pgfsys@useobject{currentmarker}{}%
\end{pgfscope}%
\begin{pgfscope}%
\pgfsys@transformshift{3.358537in}{4.036511in}%
\pgfsys@useobject{currentmarker}{}%
\end{pgfscope}%
\begin{pgfscope}%
\pgfsys@transformshift{3.339760in}{4.035347in}%
\pgfsys@useobject{currentmarker}{}%
\end{pgfscope}%
\begin{pgfscope}%
\pgfsys@transformshift{3.321450in}{4.037368in}%
\pgfsys@useobject{currentmarker}{}%
\end{pgfscope}%
\begin{pgfscope}%
\pgfsys@transformshift{3.301029in}{4.053843in}%
\pgfsys@useobject{currentmarker}{}%
\end{pgfscope}%
\begin{pgfscope}%
\pgfsys@transformshift{3.284598in}{4.117074in}%
\pgfsys@useobject{currentmarker}{}%
\end{pgfscope}%
\begin{pgfscope}%
\pgfsys@transformshift{3.262299in}{4.294339in}%
\pgfsys@useobject{currentmarker}{}%
\end{pgfscope}%
\begin{pgfscope}%
\pgfsys@transformshift{3.243991in}{4.158927in}%
\pgfsys@useobject{currentmarker}{}%
\end{pgfscope}%
\begin{pgfscope}%
\pgfsys@transformshift{3.225681in}{4.289726in}%
\pgfsys@useobject{currentmarker}{}%
\end{pgfscope}%
\begin{pgfscope}%
\pgfsys@transformshift{3.207607in}{4.413177in}%
\pgfsys@useobject{currentmarker}{}%
\end{pgfscope}%
\begin{pgfscope}%
\pgfsys@transformshift{3.188360in}{4.384583in}%
\pgfsys@useobject{currentmarker}{}%
\end{pgfscope}%
\begin{pgfscope}%
\pgfsys@transformshift{3.164652in}{4.139346in}%
\pgfsys@useobject{currentmarker}{}%
\end{pgfscope}%
\begin{pgfscope}%
\pgfsys@transformshift{3.147516in}{4.068878in}%
\pgfsys@useobject{currentmarker}{}%
\end{pgfscope}%
\begin{pgfscope}%
\pgfsys@transformshift{3.129911in}{4.045341in}%
\pgfsys@useobject{currentmarker}{}%
\end{pgfscope}%
\begin{pgfscope}%
\pgfsys@transformshift{3.110898in}{4.036127in}%
\pgfsys@useobject{currentmarker}{}%
\end{pgfscope}%
\begin{pgfscope}%
\pgfsys@transformshift{3.089304in}{4.035655in}%
\pgfsys@useobject{currentmarker}{}%
\end{pgfscope}%
\begin{pgfscope}%
\pgfsys@transformshift{3.070291in}{4.040763in}%
\pgfsys@useobject{currentmarker}{}%
\end{pgfscope}%
\begin{pgfscope}%
\pgfsys@transformshift{3.052452in}{4.057100in}%
\pgfsys@useobject{currentmarker}{}%
\end{pgfscope}%
\begin{pgfscope}%
\pgfsys@transformshift{3.033438in}{4.116748in}%
\pgfsys@useobject{currentmarker}{}%
\end{pgfscope}%
\begin{pgfscope}%
\pgfsys@transformshift{3.013017in}{4.273520in}%
\pgfsys@useobject{currentmarker}{}%
\end{pgfscope}%
\begin{pgfscope}%
\pgfsys@transformshift{2.992595in}{4.403675in}%
\pgfsys@useobject{currentmarker}{}%
\end{pgfscope}%
\begin{pgfscope}%
\pgfsys@transformshift{2.973347in}{4.395101in}%
\pgfsys@useobject{currentmarker}{}%
\end{pgfscope}%
\begin{pgfscope}%
\pgfsys@transformshift{2.955743in}{4.234941in}%
\pgfsys@useobject{currentmarker}{}%
\end{pgfscope}%
\begin{pgfscope}%
\pgfsys@transformshift{2.937434in}{4.090222in}%
\pgfsys@useobject{currentmarker}{}%
\end{pgfscope}%
\begin{pgfscope}%
\pgfsys@transformshift{2.918421in}{4.059348in}%
\pgfsys@useobject{currentmarker}{}%
\end{pgfscope}%
\begin{pgfscope}%
\pgfsys@transformshift{2.900346in}{4.042532in}%
\pgfsys@useobject{currentmarker}{}%
\end{pgfscope}%
\begin{pgfscope}%
\pgfsys@transformshift{2.880864in}{4.036280in}%
\pgfsys@useobject{currentmarker}{}%
\end{pgfscope}%
\begin{pgfscope}%
\pgfsys@transformshift{2.859504in}{4.042836in}%
\pgfsys@useobject{currentmarker}{}%
\end{pgfscope}%
\begin{pgfscope}%
\pgfsys@transformshift{2.839082in}{4.035789in}%
\pgfsys@useobject{currentmarker}{}%
\end{pgfscope}%
\begin{pgfscope}%
\pgfsys@transformshift{2.820772in}{4.035739in}%
\pgfsys@useobject{currentmarker}{}%
\end{pgfscope}%
\begin{pgfscope}%
\pgfsys@transformshift{2.797770in}{4.045444in}%
\pgfsys@useobject{currentmarker}{}%
\end{pgfscope}%
\begin{pgfscope}%
\pgfsys@transformshift{2.781808in}{4.069884in}%
\pgfsys@useobject{currentmarker}{}%
\end{pgfscope}%
\begin{pgfscope}%
\pgfsys@transformshift{2.764907in}{4.182264in}%
\pgfsys@useobject{currentmarker}{}%
\end{pgfscope}%
\begin{pgfscope}%
\pgfsys@transformshift{2.744721in}{4.319215in}%
\pgfsys@useobject{currentmarker}{}%
\end{pgfscope}%
\begin{pgfscope}%
\pgfsys@transformshift{2.725942in}{4.412662in}%
\pgfsys@useobject{currentmarker}{}%
\end{pgfscope}%
\begin{pgfscope}%
\pgfsys@transformshift{2.702469in}{4.362043in}%
\pgfsys@useobject{currentmarker}{}%
\end{pgfscope}%
\begin{pgfscope}%
\pgfsys@transformshift{2.688621in}{4.178870in}%
\pgfsys@useobject{currentmarker}{}%
\end{pgfscope}%
\begin{pgfscope}%
\pgfsys@transformshift{2.667260in}{4.070899in}%
\pgfsys@useobject{currentmarker}{}%
\end{pgfscope}%
\begin{pgfscope}%
\pgfsys@transformshift{2.648012in}{4.044650in}%
\pgfsys@useobject{currentmarker}{}%
\end{pgfscope}%
\begin{pgfscope}%
\pgfsys@transformshift{2.628530in}{4.035881in}%
\pgfsys@useobject{currentmarker}{}%
\end{pgfscope}%
\begin{pgfscope}%
\pgfsys@transformshift{2.605291in}{4.036146in}%
\pgfsys@useobject{currentmarker}{}%
\end{pgfscope}%
\begin{pgfscope}%
\pgfsys@transformshift{2.594963in}{4.041070in}%
\pgfsys@useobject{currentmarker}{}%
\end{pgfscope}%
\begin{pgfscope}%
\pgfsys@transformshift{2.570316in}{4.061286in}%
\pgfsys@useobject{currentmarker}{}%
\end{pgfscope}%
\begin{pgfscope}%
\pgfsys@transformshift{2.552711in}{4.131659in}%
\pgfsys@useobject{currentmarker}{}%
\end{pgfscope}%
\begin{pgfscope}%
\pgfsys@transformshift{2.533229in}{4.288805in}%
\pgfsys@useobject{currentmarker}{}%
\end{pgfscope}%
\begin{pgfscope}%
\pgfsys@transformshift{2.514687in}{4.403716in}%
\pgfsys@useobject{currentmarker}{}%
\end{pgfscope}%
\begin{pgfscope}%
\pgfsys@transformshift{2.496611in}{4.400021in}%
\pgfsys@useobject{currentmarker}{}%
\end{pgfscope}%
\begin{pgfscope}%
\pgfsys@transformshift{2.475486in}{4.180274in}%
\pgfsys@useobject{currentmarker}{}%
\end{pgfscope}%
\begin{pgfscope}%
\pgfsys@transformshift{2.455533in}{4.077005in}%
\pgfsys@useobject{currentmarker}{}%
\end{pgfscope}%
\begin{pgfscope}%
\pgfsys@transformshift{2.439808in}{4.056643in}%
\pgfsys@useobject{currentmarker}{}%
\end{pgfscope}%
\begin{pgfscope}%
\pgfsys@transformshift{2.420089in}{4.040927in}%
\pgfsys@useobject{currentmarker}{}%
\end{pgfscope}%
\begin{pgfscope}%
\pgfsys@transformshift{2.400138in}{4.035848in}%
\pgfsys@useobject{currentmarker}{}%
\end{pgfscope}%
\begin{pgfscope}%
\pgfsys@transformshift{2.378308in}{4.034965in}%
\pgfsys@useobject{currentmarker}{}%
\end{pgfscope}%
\begin{pgfscope}%
\pgfsys@transformshift{2.360235in}{4.036716in}%
\pgfsys@useobject{currentmarker}{}%
\end{pgfscope}%
\begin{pgfscope}%
\pgfsys@transformshift{2.341690in}{4.045393in}%
\pgfsys@useobject{currentmarker}{}%
\end{pgfscope}%
\begin{pgfscope}%
\pgfsys@transformshift{2.322442in}{4.074251in}%
\pgfsys@useobject{currentmarker}{}%
\end{pgfscope}%
\begin{pgfscope}%
\pgfsys@transformshift{2.301083in}{4.214695in}%
\pgfsys@useobject{currentmarker}{}%
\end{pgfscope}%
\begin{pgfscope}%
\pgfsys@transformshift{2.282539in}{4.364515in}%
\pgfsys@useobject{currentmarker}{}%
\end{pgfscope}%
\begin{pgfscope}%
\pgfsys@transformshift{2.263291in}{4.405641in}%
\pgfsys@useobject{currentmarker}{}%
\end{pgfscope}%
\begin{pgfscope}%
\pgfsys@transformshift{2.244043in}{4.266295in}%
\pgfsys@useobject{currentmarker}{}%
\end{pgfscope}%
\begin{pgfscope}%
\pgfsys@transformshift{2.224561in}{4.130119in}%
\pgfsys@useobject{currentmarker}{}%
\end{pgfscope}%
\begin{pgfscope}%
\pgfsys@transformshift{2.205782in}{4.065727in}%
\pgfsys@useobject{currentmarker}{}%
\end{pgfscope}%
\begin{pgfscope}%
\pgfsys@transformshift{2.187004in}{4.042943in}%
\pgfsys@useobject{currentmarker}{}%
\end{pgfscope}%
\begin{pgfscope}%
\pgfsys@transformshift{2.168695in}{4.036842in}%
\pgfsys@useobject{currentmarker}{}%
\end{pgfscope}%
\begin{pgfscope}%
\pgfsys@transformshift{2.149682in}{4.035022in}%
\pgfsys@useobject{currentmarker}{}%
\end{pgfscope}%
\begin{pgfscope}%
\pgfsys@transformshift{2.126444in}{4.039838in}%
\pgfsys@useobject{currentmarker}{}%
\end{pgfscope}%
\begin{pgfscope}%
\pgfsys@transformshift{2.110482in}{4.046331in}%
\pgfsys@useobject{currentmarker}{}%
\end{pgfscope}%
\begin{pgfscope}%
\pgfsys@transformshift{2.091234in}{4.080177in}%
\pgfsys@useobject{currentmarker}{}%
\end{pgfscope}%
\begin{pgfscope}%
\pgfsys@transformshift{2.073395in}{4.187680in}%
\pgfsys@useobject{currentmarker}{}%
\end{pgfscope}%
\begin{pgfscope}%
\pgfsys@transformshift{2.051096in}{4.355625in}%
\pgfsys@useobject{currentmarker}{}%
\end{pgfscope}%
\begin{pgfscope}%
\pgfsys@transformshift{2.032786in}{4.409466in}%
\pgfsys@useobject{currentmarker}{}%
\end{pgfscope}%
\begin{pgfscope}%
\pgfsys@transformshift{2.014009in}{4.375261in}%
\pgfsys@useobject{currentmarker}{}%
\end{pgfscope}%
\begin{pgfscope}%
\pgfsys@transformshift{1.994996in}{4.198401in}%
\pgfsys@useobject{currentmarker}{}%
\end{pgfscope}%
\begin{pgfscope}%
\pgfsys@transformshift{1.977156in}{4.111892in}%
\pgfsys@useobject{currentmarker}{}%
\end{pgfscope}%
\begin{pgfscope}%
\pgfsys@transformshift{1.955561in}{4.054770in}%
\pgfsys@useobject{currentmarker}{}%
\end{pgfscope}%
\begin{pgfscope}%
\pgfsys@transformshift{1.936547in}{4.043042in}%
\pgfsys@useobject{currentmarker}{}%
\end{pgfscope}%
\begin{pgfscope}%
\pgfsys@transformshift{1.918943in}{4.036452in}%
\pgfsys@useobject{currentmarker}{}%
\end{pgfscope}%
\begin{pgfscope}%
\pgfsys@transformshift{1.899226in}{4.035095in}%
\pgfsys@useobject{currentmarker}{}%
\end{pgfscope}%
\begin{pgfscope}%
\pgfsys@transformshift{1.879039in}{4.038904in}%
\pgfsys@useobject{currentmarker}{}%
\end{pgfscope}%
\begin{pgfscope}%
\pgfsys@transformshift{1.860496in}{4.044584in}%
\pgfsys@useobject{currentmarker}{}%
\end{pgfscope}%
\begin{pgfscope}%
\pgfsys@transformshift{1.837961in}{4.079073in}%
\pgfsys@useobject{currentmarker}{}%
\end{pgfscope}%
\begin{pgfscope}%
\pgfsys@transformshift{1.822704in}{4.149808in}%
\pgfsys@useobject{currentmarker}{}%
\end{pgfscope}%
\begin{pgfscope}%
\pgfsys@transformshift{1.804630in}{4.039174in}%
\pgfsys@useobject{currentmarker}{}%
\end{pgfscope}%
\begin{pgfscope}%
\pgfsys@transformshift{1.783504in}{4.035321in}%
\pgfsys@useobject{currentmarker}{}%
\end{pgfscope}%
\begin{pgfscope}%
\pgfsys@transformshift{1.764256in}{4.038049in}%
\pgfsys@useobject{currentmarker}{}%
\end{pgfscope}%
\begin{pgfscope}%
\pgfsys@transformshift{1.745713in}{4.049657in}%
\pgfsys@useobject{currentmarker}{}%
\end{pgfscope}%
\begin{pgfscope}%
\pgfsys@transformshift{1.723178in}{4.089269in}%
\pgfsys@useobject{currentmarker}{}%
\end{pgfscope}%
\begin{pgfscope}%
\pgfsys@transformshift{1.704870in}{4.182392in}%
\pgfsys@useobject{currentmarker}{}%
\end{pgfscope}%
\begin{pgfscope}%
\pgfsys@transformshift{1.686326in}{4.285225in}%
\pgfsys@useobject{currentmarker}{}%
\end{pgfscope}%
\begin{pgfscope}%
\pgfsys@transformshift{1.667549in}{4.401361in}%
\pgfsys@useobject{currentmarker}{}%
\end{pgfscope}%
\begin{pgfscope}%
\pgfsys@transformshift{1.647596in}{4.398918in}%
\pgfsys@useobject{currentmarker}{}%
\end{pgfscope}%
\begin{pgfscope}%
\pgfsys@transformshift{1.631399in}{4.222946in}%
\pgfsys@useobject{currentmarker}{}%
\end{pgfscope}%
\begin{pgfscope}%
\pgfsys@transformshift{1.609804in}{4.079650in}%
\pgfsys@useobject{currentmarker}{}%
\end{pgfscope}%
\begin{pgfscope}%
\pgfsys@transformshift{1.591261in}{4.053846in}%
\pgfsys@useobject{currentmarker}{}%
\end{pgfscope}%
\begin{pgfscope}%
\pgfsys@transformshift{1.573422in}{4.040156in}%
\pgfsys@useobject{currentmarker}{}%
\end{pgfscope}%
\begin{pgfscope}%
\pgfsys@transformshift{1.553235in}{4.035269in}%
\pgfsys@useobject{currentmarker}{}%
\end{pgfscope}%
\begin{pgfscope}%
\pgfsys@transformshift{1.535630in}{4.037263in}%
\pgfsys@useobject{currentmarker}{}%
\end{pgfscope}%
\begin{pgfscope}%
\pgfsys@transformshift{1.518025in}{4.046517in}%
\pgfsys@useobject{currentmarker}{}%
\end{pgfscope}%
\begin{pgfscope}%
\pgfsys@transformshift{1.496195in}{4.080150in}%
\pgfsys@useobject{currentmarker}{}%
\end{pgfscope}%
\begin{pgfscope}%
\pgfsys@transformshift{1.478356in}{4.162765in}%
\pgfsys@useobject{currentmarker}{}%
\end{pgfscope}%
\begin{pgfscope}%
\pgfsys@transformshift{1.459577in}{4.312498in}%
\pgfsys@useobject{currentmarker}{}%
\end{pgfscope}%
\begin{pgfscope}%
\pgfsys@transformshift{1.437278in}{4.423568in}%
\pgfsys@useobject{currentmarker}{}%
\end{pgfscope}%
\begin{pgfscope}%
\pgfsys@transformshift{1.418970in}{4.398882in}%
\pgfsys@useobject{currentmarker}{}%
\end{pgfscope}%
\begin{pgfscope}%
\pgfsys@transformshift{1.402539in}{4.216873in}%
\pgfsys@useobject{currentmarker}{}%
\end{pgfscope}%
\begin{pgfscope}%
\pgfsys@transformshift{1.378831in}{4.092710in}%
\pgfsys@useobject{currentmarker}{}%
\end{pgfscope}%
\begin{pgfscope}%
\pgfsys@transformshift{1.361461in}{4.056004in}%
\pgfsys@useobject{currentmarker}{}%
\end{pgfscope}%
\begin{pgfscope}%
\pgfsys@transformshift{1.341979in}{4.040332in}%
\pgfsys@useobject{currentmarker}{}%
\end{pgfscope}%
\begin{pgfscope}%
\pgfsys@transformshift{1.323200in}{4.036674in}%
\pgfsys@useobject{currentmarker}{}%
\end{pgfscope}%
\begin{pgfscope}%
\pgfsys@transformshift{1.302073in}{4.036505in}%
\pgfsys@useobject{currentmarker}{}%
\end{pgfscope}%
\begin{pgfscope}%
\pgfsys@transformshift{1.284234in}{4.039680in}%
\pgfsys@useobject{currentmarker}{}%
\end{pgfscope}%
\begin{pgfscope}%
\pgfsys@transformshift{1.264986in}{4.046634in}%
\pgfsys@useobject{currentmarker}{}%
\end{pgfscope}%
\begin{pgfscope}%
\pgfsys@transformshift{1.244801in}{4.083037in}%
\pgfsys@useobject{currentmarker}{}%
\end{pgfscope}%
\begin{pgfscope}%
\pgfsys@transformshift{1.222266in}{4.199765in}%
\pgfsys@useobject{currentmarker}{}%
\end{pgfscope}%
\begin{pgfscope}%
\pgfsys@transformshift{1.206774in}{4.303353in}%
\pgfsys@useobject{currentmarker}{}%
\end{pgfscope}%
\begin{pgfscope}%
\pgfsys@transformshift{1.185179in}{4.409737in}%
\pgfsys@useobject{currentmarker}{}%
\end{pgfscope}%
\begin{pgfscope}%
\pgfsys@transformshift{1.169688in}{4.435430in}%
\pgfsys@useobject{currentmarker}{}%
\end{pgfscope}%
\begin{pgfscope}%
\pgfsys@transformshift{1.149031in}{4.370421in}%
\pgfsys@useobject{currentmarker}{}%
\end{pgfscope}%
\begin{pgfscope}%
\pgfsys@transformshift{1.126496in}{4.274259in}%
\pgfsys@useobject{currentmarker}{}%
\end{pgfscope}%
\begin{pgfscope}%
\pgfsys@transformshift{1.110770in}{4.142794in}%
\pgfsys@useobject{currentmarker}{}%
\end{pgfscope}%
\begin{pgfscope}%
\pgfsys@transformshift{1.089644in}{4.074657in}%
\pgfsys@useobject{currentmarker}{}%
\end{pgfscope}%
\begin{pgfscope}%
\pgfsys@transformshift{1.071804in}{4.050540in}%
\pgfsys@useobject{currentmarker}{}%
\end{pgfscope}%
\begin{pgfscope}%
\pgfsys@transformshift{1.054905in}{4.040112in}%
\pgfsys@useobject{currentmarker}{}%
\end{pgfscope}%
\begin{pgfscope}%
\pgfsys@transformshift{1.033778in}{4.036176in}%
\pgfsys@useobject{currentmarker}{}%
\end{pgfscope}%
\begin{pgfscope}%
\pgfsys@transformshift{1.014296in}{4.039926in}%
\pgfsys@useobject{currentmarker}{}%
\end{pgfscope}%
\begin{pgfscope}%
\pgfsys@transformshift{0.995282in}{4.049590in}%
\pgfsys@useobject{currentmarker}{}%
\end{pgfscope}%
\begin{pgfscope}%
\pgfsys@transformshift{0.976269in}{4.082306in}%
\pgfsys@useobject{currentmarker}{}%
\end{pgfscope}%
\begin{pgfscope}%
\pgfsys@transformshift{0.957727in}{4.146357in}%
\pgfsys@useobject{currentmarker}{}%
\end{pgfscope}%
\begin{pgfscope}%
\pgfsys@transformshift{0.941060in}{4.219820in}%
\pgfsys@useobject{currentmarker}{}%
\end{pgfscope}%
\begin{pgfscope}%
\pgfsys@transformshift{0.919935in}{4.354104in}%
\pgfsys@useobject{currentmarker}{}%
\end{pgfscope}%
\begin{pgfscope}%
\pgfsys@transformshift{0.899278in}{4.431555in}%
\pgfsys@useobject{currentmarker}{}%
\end{pgfscope}%
\begin{pgfscope}%
\pgfsys@transformshift{0.881908in}{4.443217in}%
\pgfsys@useobject{currentmarker}{}%
\end{pgfscope}%
\begin{pgfscope}%
\pgfsys@transformshift{0.860314in}{4.404527in}%
\pgfsys@useobject{currentmarker}{}%
\end{pgfscope}%
\begin{pgfscope}%
\pgfsys@transformshift{0.841535in}{4.226956in}%
\pgfsys@useobject{currentmarker}{}%
\end{pgfscope}%
\begin{pgfscope}%
\pgfsys@transformshift{0.823931in}{4.109175in}%
\pgfsys@useobject{currentmarker}{}%
\end{pgfscope}%
\begin{pgfscope}%
\pgfsys@transformshift{0.803978in}{4.060367in}%
\pgfsys@useobject{currentmarker}{}%
\end{pgfscope}%
\begin{pgfscope}%
\pgfsys@transformshift{0.783087in}{4.042582in}%
\pgfsys@useobject{currentmarker}{}%
\end{pgfscope}%
\begin{pgfscope}%
\pgfsys@transformshift{0.766891in}{4.043556in}%
\pgfsys@useobject{currentmarker}{}%
\end{pgfscope}%
\begin{pgfscope}%
\pgfsys@transformshift{0.745766in}{4.038873in}%
\pgfsys@useobject{currentmarker}{}%
\end{pgfscope}%
\begin{pgfscope}%
\pgfsys@transformshift{0.727456in}{4.036479in}%
\pgfsys@useobject{currentmarker}{}%
\end{pgfscope}%
\begin{pgfscope}%
\pgfsys@transformshift{0.707974in}{4.041830in}%
\pgfsys@useobject{currentmarker}{}%
\end{pgfscope}%
\begin{pgfscope}%
\pgfsys@transformshift{0.684971in}{4.065760in}%
\pgfsys@useobject{currentmarker}{}%
\end{pgfscope}%
\begin{pgfscope}%
\pgfsys@transformshift{0.669244in}{4.104263in}%
\pgfsys@useobject{currentmarker}{}%
\end{pgfscope}%
\begin{pgfscope}%
\pgfsys@transformshift{0.652108in}{4.185391in}%
\pgfsys@useobject{currentmarker}{}%
\end{pgfscope}%
\begin{pgfscope}%
\pgfsys@transformshift{0.652579in}{4.182284in}%
\pgfsys@useobject{currentmarker}{}%
\end{pgfscope}%
\begin{pgfscope}%
\pgfsys@transformshift{0.655865in}{4.153420in}%
\pgfsys@useobject{currentmarker}{}%
\end{pgfscope}%
\begin{pgfscope}%
\pgfsys@transformshift{0.675112in}{4.064491in}%
\pgfsys@useobject{currentmarker}{}%
\end{pgfscope}%
\begin{pgfscope}%
\pgfsys@transformshift{0.694594in}{4.041470in}%
\pgfsys@useobject{currentmarker}{}%
\end{pgfscope}%
\begin{pgfscope}%
\pgfsys@transformshift{0.714311in}{4.036539in}%
\pgfsys@useobject{currentmarker}{}%
\end{pgfscope}%
\begin{pgfscope}%
\pgfsys@transformshift{0.732855in}{4.046061in}%
\pgfsys@useobject{currentmarker}{}%
\end{pgfscope}%
\begin{pgfscope}%
\pgfsys@transformshift{0.751869in}{4.078642in}%
\pgfsys@useobject{currentmarker}{}%
\end{pgfscope}%
\begin{pgfscope}%
\pgfsys@transformshift{0.770413in}{4.201264in}%
\pgfsys@useobject{currentmarker}{}%
\end{pgfscope}%
\begin{pgfscope}%
\pgfsys@transformshift{0.792712in}{4.442196in}%
\pgfsys@useobject{currentmarker}{}%
\end{pgfscope}%
\begin{pgfscope}%
\pgfsys@transformshift{0.813837in}{4.393313in}%
\pgfsys@useobject{currentmarker}{}%
\end{pgfscope}%
\begin{pgfscope}%
\pgfsys@transformshift{0.829799in}{4.235773in}%
\pgfsys@useobject{currentmarker}{}%
\end{pgfscope}%
\begin{pgfscope}%
\pgfsys@transformshift{0.849750in}{4.089451in}%
\pgfsys@useobject{currentmarker}{}%
\end{pgfscope}%
\begin{pgfscope}%
\pgfsys@transformshift{0.868060in}{4.046763in}%
\pgfsys@useobject{currentmarker}{}%
\end{pgfscope}%
\begin{pgfscope}%
\pgfsys@transformshift{0.886602in}{4.036617in}%
\pgfsys@useobject{currentmarker}{}%
\end{pgfscope}%
\begin{pgfscope}%
\pgfsys@transformshift{0.905381in}{4.038937in}%
\pgfsys@useobject{currentmarker}{}%
\end{pgfscope}%
\begin{pgfscope}%
\pgfsys@transformshift{0.924394in}{4.055052in}%
\pgfsys@useobject{currentmarker}{}%
\end{pgfscope}%
\begin{pgfscope}%
\pgfsys@transformshift{0.943408in}{4.110663in}%
\pgfsys@useobject{currentmarker}{}%
\end{pgfscope}%
\begin{pgfscope}%
\pgfsys@transformshift{0.965472in}{4.380643in}%
\pgfsys@useobject{currentmarker}{}%
\end{pgfscope}%
\begin{pgfscope}%
\pgfsys@transformshift{0.984954in}{4.434793in}%
\pgfsys@useobject{currentmarker}{}%
\end{pgfscope}%
\begin{pgfscope}%
\pgfsys@transformshift{1.003264in}{4.320412in}%
\pgfsys@useobject{currentmarker}{}%
\end{pgfscope}%
\begin{pgfscope}%
\pgfsys@transformshift{1.022041in}{4.132783in}%
\pgfsys@useobject{currentmarker}{}%
\end{pgfscope}%
\begin{pgfscope}%
\pgfsys@transformshift{1.040351in}{4.057018in}%
\pgfsys@useobject{currentmarker}{}%
\end{pgfscope}%
\begin{pgfscope}%
\pgfsys@transformshift{1.060068in}{4.039110in}%
\pgfsys@useobject{currentmarker}{}%
\end{pgfscope}%
\begin{pgfscope}%
\pgfsys@transformshift{1.079550in}{4.035937in}%
\pgfsys@useobject{currentmarker}{}%
\end{pgfscope}%
\begin{pgfscope}%
\pgfsys@transformshift{1.096920in}{4.043088in}%
\pgfsys@useobject{currentmarker}{}%
\end{pgfscope}%
\begin{pgfscope}%
\pgfsys@transformshift{1.120393in}{4.076554in}%
\pgfsys@useobject{currentmarker}{}%
\end{pgfscope}%
\begin{pgfscope}%
\pgfsys@transformshift{1.136590in}{4.180026in}%
\pgfsys@useobject{currentmarker}{}%
\end{pgfscope}%
\begin{pgfscope}%
\pgfsys@transformshift{1.153491in}{4.407360in}%
\pgfsys@useobject{currentmarker}{}%
\end{pgfscope}%
\begin{pgfscope}%
\pgfsys@transformshift{1.176493in}{4.397335in}%
\pgfsys@useobject{currentmarker}{}%
\end{pgfscope}%
\begin{pgfscope}%
\pgfsys@transformshift{1.194569in}{4.212865in}%
\pgfsys@useobject{currentmarker}{}%
\end{pgfscope}%
\begin{pgfscope}%
\pgfsys@transformshift{1.214286in}{4.085736in}%
\pgfsys@useobject{currentmarker}{}%
\end{pgfscope}%
\begin{pgfscope}%
\pgfsys@transformshift{1.235176in}{4.043310in}%
\pgfsys@useobject{currentmarker}{}%
\end{pgfscope}%
\begin{pgfscope}%
\pgfsys@transformshift{1.250904in}{4.036335in}%
\pgfsys@useobject{currentmarker}{}%
\end{pgfscope}%
\begin{pgfscope}%
\pgfsys@transformshift{1.271794in}{4.037549in}%
\pgfsys@useobject{currentmarker}{}%
\end{pgfscope}%
\begin{pgfscope}%
\pgfsys@transformshift{1.291981in}{4.048237in}%
\pgfsys@useobject{currentmarker}{}%
\end{pgfscope}%
\begin{pgfscope}%
\pgfsys@transformshift{1.310055in}{4.084400in}%
\pgfsys@useobject{currentmarker}{}%
\end{pgfscope}%
\begin{pgfscope}%
\pgfsys@transformshift{1.329303in}{4.234551in}%
\pgfsys@useobject{currentmarker}{}%
\end{pgfscope}%
\begin{pgfscope}%
\pgfsys@transformshift{1.348785in}{4.424924in}%
\pgfsys@useobject{currentmarker}{}%
\end{pgfscope}%
\begin{pgfscope}%
\pgfsys@transformshift{1.368267in}{4.382303in}%
\pgfsys@useobject{currentmarker}{}%
\end{pgfscope}%
\begin{pgfscope}%
\pgfsys@transformshift{1.385637in}{4.228906in}%
\pgfsys@useobject{currentmarker}{}%
\end{pgfscope}%
\begin{pgfscope}%
\pgfsys@transformshift{1.406059in}{4.081968in}%
\pgfsys@useobject{currentmarker}{}%
\end{pgfscope}%
\begin{pgfscope}%
\pgfsys@transformshift{1.426481in}{4.043745in}%
\pgfsys@useobject{currentmarker}{}%
\end{pgfscope}%
\begin{pgfscope}%
\pgfsys@transformshift{1.443617in}{4.036125in}%
\pgfsys@useobject{currentmarker}{}%
\end{pgfscope}%
\begin{pgfscope}%
\pgfsys@transformshift{1.461690in}{4.035974in}%
\pgfsys@useobject{currentmarker}{}%
\end{pgfscope}%
\begin{pgfscope}%
\pgfsys@transformshift{1.481407in}{4.043628in}%
\pgfsys@useobject{currentmarker}{}%
\end{pgfscope}%
\begin{pgfscope}%
\pgfsys@transformshift{1.501829in}{4.071822in}%
\pgfsys@useobject{currentmarker}{}%
\end{pgfscope}%
\begin{pgfscope}%
\pgfsys@transformshift{1.522250in}{4.170201in}%
\pgfsys@useobject{currentmarker}{}%
\end{pgfscope}%
\begin{pgfscope}%
\pgfsys@transformshift{1.539855in}{4.380612in}%
\pgfsys@useobject{currentmarker}{}%
\end{pgfscope}%
\begin{pgfscope}%
\pgfsys@transformshift{1.559337in}{4.168465in}%
\pgfsys@useobject{currentmarker}{}%
\end{pgfscope}%
\begin{pgfscope}%
\pgfsys@transformshift{1.577177in}{4.400700in}%
\pgfsys@useobject{currentmarker}{}%
\end{pgfscope}%
\begin{pgfscope}%
\pgfsys@transformshift{1.598538in}{4.390031in}%
\pgfsys@useobject{currentmarker}{}%
\end{pgfscope}%
\begin{pgfscope}%
\pgfsys@transformshift{1.618960in}{4.228802in}%
\pgfsys@useobject{currentmarker}{}%
\end{pgfscope}%
\begin{pgfscope}%
\pgfsys@transformshift{1.640788in}{4.080119in}%
\pgfsys@useobject{currentmarker}{}%
\end{pgfscope}%
\begin{pgfscope}%
\pgfsys@transformshift{1.657455in}{4.046589in}%
\pgfsys@useobject{currentmarker}{}%
\end{pgfscope}%
\begin{pgfscope}%
\pgfsys@transformshift{1.673651in}{4.037497in}%
\pgfsys@useobject{currentmarker}{}%
\end{pgfscope}%
\begin{pgfscope}%
\pgfsys@transformshift{1.692899in}{4.034753in}%
\pgfsys@useobject{currentmarker}{}%
\end{pgfscope}%
\begin{pgfscope}%
\pgfsys@transformshift{1.710504in}{4.037232in}%
\pgfsys@useobject{currentmarker}{}%
\end{pgfscope}%
\begin{pgfscope}%
\pgfsys@transformshift{1.731863in}{4.052488in}%
\pgfsys@useobject{currentmarker}{}%
\end{pgfscope}%
\begin{pgfscope}%
\pgfsys@transformshift{1.753225in}{4.115861in}%
\pgfsys@useobject{currentmarker}{}%
\end{pgfscope}%
\begin{pgfscope}%
\pgfsys@transformshift{1.771767in}{4.306697in}%
\pgfsys@useobject{currentmarker}{}%
\end{pgfscope}%
\begin{pgfscope}%
\pgfsys@transformshift{1.792189in}{4.418422in}%
\pgfsys@useobject{currentmarker}{}%
\end{pgfscope}%
\begin{pgfscope}%
\pgfsys@transformshift{1.809794in}{4.384771in}%
\pgfsys@useobject{currentmarker}{}%
\end{pgfscope}%
\begin{pgfscope}%
\pgfsys@transformshift{1.829746in}{4.247630in}%
\pgfsys@useobject{currentmarker}{}%
\end{pgfscope}%
\begin{pgfscope}%
\pgfsys@transformshift{1.849932in}{4.077168in}%
\pgfsys@useobject{currentmarker}{}%
\end{pgfscope}%
\begin{pgfscope}%
\pgfsys@transformshift{1.869885in}{4.045243in}%
\pgfsys@useobject{currentmarker}{}%
\end{pgfscope}%
\begin{pgfscope}%
\pgfsys@transformshift{1.885376in}{4.037331in}%
\pgfsys@useobject{currentmarker}{}%
\end{pgfscope}%
\begin{pgfscope}%
\pgfsys@transformshift{1.905798in}{4.034919in}%
\pgfsys@useobject{currentmarker}{}%
\end{pgfscope}%
\begin{pgfscope}%
\pgfsys@transformshift{1.926688in}{4.039588in}%
\pgfsys@useobject{currentmarker}{}%
\end{pgfscope}%
\begin{pgfscope}%
\pgfsys@transformshift{1.944998in}{4.055251in}%
\pgfsys@useobject{currentmarker}{}%
\end{pgfscope}%
\begin{pgfscope}%
\pgfsys@transformshift{1.963541in}{4.103450in}%
\pgfsys@useobject{currentmarker}{}%
\end{pgfscope}%
\begin{pgfscope}%
\pgfsys@transformshift{1.983025in}{4.311655in}%
\pgfsys@useobject{currentmarker}{}%
\end{pgfscope}%
\begin{pgfscope}%
\pgfsys@transformshift{2.002038in}{4.412767in}%
\pgfsys@useobject{currentmarker}{}%
\end{pgfscope}%
\begin{pgfscope}%
\pgfsys@transformshift{2.022458in}{4.378443in}%
\pgfsys@useobject{currentmarker}{}%
\end{pgfscope}%
\begin{pgfscope}%
\pgfsys@transformshift{2.037716in}{4.262091in}%
\pgfsys@useobject{currentmarker}{}%
\end{pgfscope}%
\begin{pgfscope}%
\pgfsys@transformshift{2.058138in}{4.093608in}%
\pgfsys@useobject{currentmarker}{}%
\end{pgfscope}%
\begin{pgfscope}%
\pgfsys@transformshift{2.080437in}{4.046645in}%
\pgfsys@useobject{currentmarker}{}%
\end{pgfscope}%
\begin{pgfscope}%
\pgfsys@transformshift{2.097571in}{4.036436in}%
\pgfsys@useobject{currentmarker}{}%
\end{pgfscope}%
\begin{pgfscope}%
\pgfsys@transformshift{2.115881in}{4.034718in}%
\pgfsys@useobject{currentmarker}{}%
\end{pgfscope}%
\begin{pgfscope}%
\pgfsys@transformshift{2.136303in}{4.037818in}%
\pgfsys@useobject{currentmarker}{}%
\end{pgfscope}%
\begin{pgfscope}%
\pgfsys@transformshift{2.154611in}{4.049202in}%
\pgfsys@useobject{currentmarker}{}%
\end{pgfscope}%
\begin{pgfscope}%
\pgfsys@transformshift{2.172450in}{4.086628in}%
\pgfsys@useobject{currentmarker}{}%
\end{pgfscope}%
\begin{pgfscope}%
\pgfsys@transformshift{2.194046in}{4.259198in}%
\pgfsys@useobject{currentmarker}{}%
\end{pgfscope}%
\begin{pgfscope}%
\pgfsys@transformshift{2.211885in}{4.401998in}%
\pgfsys@useobject{currentmarker}{}%
\end{pgfscope}%
\begin{pgfscope}%
\pgfsys@transformshift{2.232072in}{4.380275in}%
\pgfsys@useobject{currentmarker}{}%
\end{pgfscope}%
\begin{pgfscope}%
\pgfsys@transformshift{2.253197in}{4.219308in}%
\pgfsys@useobject{currentmarker}{}%
\end{pgfscope}%
\begin{pgfscope}%
\pgfsys@transformshift{2.268690in}{4.140516in}%
\pgfsys@useobject{currentmarker}{}%
\end{pgfscope}%
\begin{pgfscope}%
\pgfsys@transformshift{2.289110in}{4.064002in}%
\pgfsys@useobject{currentmarker}{}%
\end{pgfscope}%
\begin{pgfscope}%
\pgfsys@transformshift{2.310472in}{4.039736in}%
\pgfsys@useobject{currentmarker}{}%
\end{pgfscope}%
\begin{pgfscope}%
\pgfsys@transformshift{2.327842in}{4.035170in}%
\pgfsys@useobject{currentmarker}{}%
\end{pgfscope}%
\begin{pgfscope}%
\pgfsys@transformshift{2.346150in}{4.036721in}%
\pgfsys@useobject{currentmarker}{}%
\end{pgfscope}%
\begin{pgfscope}%
\pgfsys@transformshift{2.364929in}{4.034797in}%
\pgfsys@useobject{currentmarker}{}%
\end{pgfscope}%
\begin{pgfscope}%
\pgfsys@transformshift{2.386054in}{4.039324in}%
\pgfsys@useobject{currentmarker}{}%
\end{pgfscope}%
\begin{pgfscope}%
\pgfsys@transformshift{2.403893in}{4.056380in}%
\pgfsys@useobject{currentmarker}{}%
\end{pgfscope}%
\begin{pgfscope}%
\pgfsys@transformshift{2.425723in}{4.129453in}%
\pgfsys@useobject{currentmarker}{}%
\end{pgfscope}%
\begin{pgfscope}%
\pgfsys@transformshift{2.443328in}{4.115849in}%
\pgfsys@useobject{currentmarker}{}%
\end{pgfscope}%
\begin{pgfscope}%
\pgfsys@transformshift{2.463984in}{4.349601in}%
\pgfsys@useobject{currentmarker}{}%
\end{pgfscope}%
\begin{pgfscope}%
\pgfsys@transformshift{2.482294in}{4.392723in}%
\pgfsys@useobject{currentmarker}{}%
\end{pgfscope}%
\begin{pgfscope}%
\pgfsys@transformshift{2.499663in}{4.403415in}%
\pgfsys@useobject{currentmarker}{}%
\end{pgfscope}%
\begin{pgfscope}%
\pgfsys@transformshift{2.521024in}{4.244119in}%
\pgfsys@useobject{currentmarker}{}%
\end{pgfscope}%
\begin{pgfscope}%
\pgfsys@transformshift{2.539332in}{4.096114in}%
\pgfsys@useobject{currentmarker}{}%
\end{pgfscope}%
\begin{pgfscope}%
\pgfsys@transformshift{2.563979in}{4.044016in}%
\pgfsys@useobject{currentmarker}{}%
\end{pgfscope}%
\begin{pgfscope}%
\pgfsys@transformshift{2.578298in}{4.038919in}%
\pgfsys@useobject{currentmarker}{}%
\end{pgfscope}%
\begin{pgfscope}%
\pgfsys@transformshift{2.595903in}{4.036159in}%
\pgfsys@useobject{currentmarker}{}%
\end{pgfscope}%
\begin{pgfscope}%
\pgfsys@transformshift{2.617262in}{4.035429in}%
\pgfsys@useobject{currentmarker}{}%
\end{pgfscope}%
\begin{pgfscope}%
\pgfsys@transformshift{2.636510in}{4.041979in}%
\pgfsys@useobject{currentmarker}{}%
\end{pgfscope}%
\begin{pgfscope}%
\pgfsys@transformshift{2.653880in}{4.062512in}%
\pgfsys@useobject{currentmarker}{}%
\end{pgfscope}%
\begin{pgfscope}%
\pgfsys@transformshift{2.672659in}{4.110950in}%
\pgfsys@useobject{currentmarker}{}%
\end{pgfscope}%
\begin{pgfscope}%
\pgfsys@transformshift{2.692376in}{4.294875in}%
\pgfsys@useobject{currentmarker}{}%
\end{pgfscope}%
\begin{pgfscope}%
\pgfsys@transformshift{2.713266in}{4.394569in}%
\pgfsys@useobject{currentmarker}{}%
\end{pgfscope}%
\begin{pgfscope}%
\pgfsys@transformshift{2.731106in}{4.397858in}%
\pgfsys@useobject{currentmarker}{}%
\end{pgfscope}%
\begin{pgfscope}%
\pgfsys@transformshift{2.755049in}{4.210413in}%
\pgfsys@useobject{currentmarker}{}%
\end{pgfscope}%
\begin{pgfscope}%
\pgfsys@transformshift{2.767020in}{4.122374in}%
\pgfsys@useobject{currentmarker}{}%
\end{pgfscope}%
\begin{pgfscope}%
\pgfsys@transformshift{2.789788in}{4.053458in}%
\pgfsys@useobject{currentmarker}{}%
\end{pgfscope}%
\begin{pgfscope}%
\pgfsys@transformshift{2.806219in}{4.040581in}%
\pgfsys@useobject{currentmarker}{}%
\end{pgfscope}%
\begin{pgfscope}%
\pgfsys@transformshift{2.827815in}{4.035874in}%
\pgfsys@useobject{currentmarker}{}%
\end{pgfscope}%
\begin{pgfscope}%
\pgfsys@transformshift{2.845888in}{4.034981in}%
\pgfsys@useobject{currentmarker}{}%
\end{pgfscope}%
\begin{pgfscope}%
\pgfsys@transformshift{2.867484in}{4.041066in}%
\pgfsys@useobject{currentmarker}{}%
\end{pgfscope}%
\begin{pgfscope}%
\pgfsys@transformshift{2.885792in}{4.054746in}%
\pgfsys@useobject{currentmarker}{}%
\end{pgfscope}%
\begin{pgfscope}%
\pgfsys@transformshift{2.905745in}{4.106741in}%
\pgfsys@useobject{currentmarker}{}%
\end{pgfscope}%
\begin{pgfscope}%
\pgfsys@transformshift{2.923584in}{4.240283in}%
\pgfsys@useobject{currentmarker}{}%
\end{pgfscope}%
\begin{pgfscope}%
\pgfsys@transformshift{2.942129in}{4.390557in}%
\pgfsys@useobject{currentmarker}{}%
\end{pgfscope}%
\begin{pgfscope}%
\pgfsys@transformshift{2.963019in}{4.399090in}%
\pgfsys@useobject{currentmarker}{}%
\end{pgfscope}%
\begin{pgfscope}%
\pgfsys@transformshift{2.981327in}{4.287633in}%
\pgfsys@useobject{currentmarker}{}%
\end{pgfscope}%
\begin{pgfscope}%
\pgfsys@transformshift{2.998229in}{4.147690in}%
\pgfsys@useobject{currentmarker}{}%
\end{pgfscope}%
\begin{pgfscope}%
\pgfsys@transformshift{3.020059in}{4.063371in}%
\pgfsys@useobject{currentmarker}{}%
\end{pgfscope}%
\begin{pgfscope}%
\pgfsys@transformshift{3.040479in}{4.045051in}%
\pgfsys@useobject{currentmarker}{}%
\end{pgfscope}%
\begin{pgfscope}%
\pgfsys@transformshift{3.056441in}{4.038481in}%
\pgfsys@useobject{currentmarker}{}%
\end{pgfscope}%
\begin{pgfscope}%
\pgfsys@transformshift{3.076628in}{4.035732in}%
\pgfsys@useobject{currentmarker}{}%
\end{pgfscope}%
\begin{pgfscope}%
\pgfsys@transformshift{3.098693in}{4.036002in}%
\pgfsys@useobject{currentmarker}{}%
\end{pgfscope}%
\begin{pgfscope}%
\pgfsys@transformshift{3.115829in}{4.041058in}%
\pgfsys@useobject{currentmarker}{}%
\end{pgfscope}%
\begin{pgfscope}%
\pgfsys@transformshift{3.134371in}{4.051973in}%
\pgfsys@useobject{currentmarker}{}%
\end{pgfscope}%
\begin{pgfscope}%
\pgfsys@transformshift{3.155027in}{4.101456in}%
\pgfsys@useobject{currentmarker}{}%
\end{pgfscope}%
\begin{pgfscope}%
\pgfsys@transformshift{3.172866in}{4.230028in}%
\pgfsys@useobject{currentmarker}{}%
\end{pgfscope}%
\begin{pgfscope}%
\pgfsys@transformshift{3.194228in}{4.395868in}%
\pgfsys@useobject{currentmarker}{}%
\end{pgfscope}%
\begin{pgfscope}%
\pgfsys@transformshift{3.212301in}{4.411905in}%
\pgfsys@useobject{currentmarker}{}%
\end{pgfscope}%
\begin{pgfscope}%
\pgfsys@transformshift{3.230375in}{4.323430in}%
\pgfsys@useobject{currentmarker}{}%
\end{pgfscope}%
\begin{pgfscope}%
\pgfsys@transformshift{3.254553in}{4.128889in}%
\pgfsys@useobject{currentmarker}{}%
\end{pgfscope}%
\begin{pgfscope}%
\pgfsys@transformshift{3.269810in}{4.077098in}%
\pgfsys@useobject{currentmarker}{}%
\end{pgfscope}%
\begin{pgfscope}%
\pgfsys@transformshift{3.287649in}{4.052767in}%
\pgfsys@useobject{currentmarker}{}%
\end{pgfscope}%
\begin{pgfscope}%
\pgfsys@transformshift{3.308305in}{4.039173in}%
\pgfsys@useobject{currentmarker}{}%
\end{pgfscope}%
\begin{pgfscope}%
\pgfsys@transformshift{3.326615in}{4.035915in}%
\pgfsys@useobject{currentmarker}{}%
\end{pgfscope}%
\begin{pgfscope}%
\pgfsys@transformshift{3.347975in}{4.035406in}%
\pgfsys@useobject{currentmarker}{}%
\end{pgfscope}%
\begin{pgfscope}%
\pgfsys@transformshift{3.366285in}{4.039634in}%
\pgfsys@useobject{currentmarker}{}%
\end{pgfscope}%
\begin{pgfscope}%
\pgfsys@transformshift{3.386236in}{4.046910in}%
\pgfsys@useobject{currentmarker}{}%
\end{pgfscope}%
\begin{pgfscope}%
\pgfsys@transformshift{3.405249in}{4.074114in}%
\pgfsys@useobject{currentmarker}{}%
\end{pgfscope}%
\begin{pgfscope}%
\pgfsys@transformshift{3.423088in}{4.117390in}%
\pgfsys@useobject{currentmarker}{}%
\end{pgfscope}%
\begin{pgfscope}%
\pgfsys@transformshift{3.444684in}{4.305588in}%
\pgfsys@useobject{currentmarker}{}%
\end{pgfscope}%
\begin{pgfscope}%
\pgfsys@transformshift{3.462054in}{4.397965in}%
\pgfsys@useobject{currentmarker}{}%
\end{pgfscope}%
\begin{pgfscope}%
\pgfsys@transformshift{3.483414in}{4.412258in}%
\pgfsys@useobject{currentmarker}{}%
\end{pgfscope}%
\begin{pgfscope}%
\pgfsys@transformshift{3.501958in}{4.345300in}%
\pgfsys@useobject{currentmarker}{}%
\end{pgfscope}%
\begin{pgfscope}%
\pgfsys@transformshift{3.519092in}{4.194292in}%
\pgfsys@useobject{currentmarker}{}%
\end{pgfscope}%
\begin{pgfscope}%
\pgfsys@transformshift{3.537168in}{4.089268in}%
\pgfsys@useobject{currentmarker}{}%
\end{pgfscope}%
\begin{pgfscope}%
\pgfsys@transformshift{3.558996in}{4.056711in}%
\pgfsys@useobject{currentmarker}{}%
\end{pgfscope}%
\begin{pgfscope}%
\pgfsys@transformshift{3.575898in}{4.044258in}%
\pgfsys@useobject{currentmarker}{}%
\end{pgfscope}%
\begin{pgfscope}%
\pgfsys@transformshift{3.597257in}{4.036488in}%
\pgfsys@useobject{currentmarker}{}%
\end{pgfscope}%
\begin{pgfscope}%
\pgfsys@transformshift{3.617444in}{4.065142in}%
\pgfsys@useobject{currentmarker}{}%
\end{pgfscope}%
\begin{pgfscope}%
\pgfsys@transformshift{3.634580in}{4.043300in}%
\pgfsys@useobject{currentmarker}{}%
\end{pgfscope}%
\begin{pgfscope}%
\pgfsys@transformshift{3.654531in}{4.036270in}%
\pgfsys@useobject{currentmarker}{}%
\end{pgfscope}%
\begin{pgfscope}%
\pgfsys@transformshift{3.673076in}{4.036533in}%
\pgfsys@useobject{currentmarker}{}%
\end{pgfscope}%
\begin{pgfscope}%
\pgfsys@transformshift{3.691618in}{4.042177in}%
\pgfsys@useobject{currentmarker}{}%
\end{pgfscope}%
\begin{pgfscope}%
\pgfsys@transformshift{3.712040in}{4.051680in}%
\pgfsys@useobject{currentmarker}{}%
\end{pgfscope}%
\begin{pgfscope}%
\pgfsys@transformshift{3.731287in}{4.086815in}%
\pgfsys@useobject{currentmarker}{}%
\end{pgfscope}%
\begin{pgfscope}%
\pgfsys@transformshift{3.751709in}{4.188728in}%
\pgfsys@useobject{currentmarker}{}%
\end{pgfscope}%
\begin{pgfscope}%
\pgfsys@transformshift{3.769080in}{4.388527in}%
\pgfsys@useobject{currentmarker}{}%
\end{pgfscope}%
\begin{pgfscope}%
\pgfsys@transformshift{3.786684in}{4.434488in}%
\pgfsys@useobject{currentmarker}{}%
\end{pgfscope}%
\begin{pgfscope}%
\pgfsys@transformshift{3.804524in}{4.389228in}%
\pgfsys@useobject{currentmarker}{}%
\end{pgfscope}%
\begin{pgfscope}%
\pgfsys@transformshift{3.825649in}{4.270020in}%
\pgfsys@useobject{currentmarker}{}%
\end{pgfscope}%
\begin{pgfscope}%
\pgfsys@transformshift{3.847479in}{4.139649in}%
\pgfsys@useobject{currentmarker}{}%
\end{pgfscope}%
\begin{pgfscope}%
\pgfsys@transformshift{3.866726in}{4.071288in}%
\pgfsys@useobject{currentmarker}{}%
\end{pgfscope}%
\begin{pgfscope}%
\pgfsys@transformshift{3.884566in}{4.044739in}%
\pgfsys@useobject{currentmarker}{}%
\end{pgfscope}%
\begin{pgfscope}%
\pgfsys@transformshift{3.903345in}{4.038082in}%
\pgfsys@useobject{currentmarker}{}%
\end{pgfscope}%
\begin{pgfscope}%
\pgfsys@transformshift{3.924940in}{4.035827in}%
\pgfsys@useobject{currentmarker}{}%
\end{pgfscope}%
\begin{pgfscope}%
\pgfsys@transformshift{3.941137in}{4.038239in}%
\pgfsys@useobject{currentmarker}{}%
\end{pgfscope}%
\begin{pgfscope}%
\pgfsys@transformshift{3.962027in}{4.050382in}%
\pgfsys@useobject{currentmarker}{}%
\end{pgfscope}%
\begin{pgfscope}%
\pgfsys@transformshift{3.981275in}{4.075711in}%
\pgfsys@useobject{currentmarker}{}%
\end{pgfscope}%
\begin{pgfscope}%
\pgfsys@transformshift{3.999114in}{4.144553in}%
\pgfsys@useobject{currentmarker}{}%
\end{pgfscope}%
\begin{pgfscope}%
\pgfsys@transformshift{4.019770in}{4.340431in}%
\pgfsys@useobject{currentmarker}{}%
\end{pgfscope}%
\begin{pgfscope}%
\pgfsys@transformshift{4.038549in}{4.408652in}%
\pgfsys@useobject{currentmarker}{}%
\end{pgfscope}%
\begin{pgfscope}%
\pgfsys@transformshift{4.058031in}{4.441846in}%
\pgfsys@useobject{currentmarker}{}%
\end{pgfscope}%
\begin{pgfscope}%
\pgfsys@transformshift{4.077279in}{4.431834in}%
\pgfsys@useobject{currentmarker}{}%
\end{pgfscope}%
\begin{pgfscope}%
\pgfsys@transformshift{4.096058in}{4.313644in}%
\pgfsys@useobject{currentmarker}{}%
\end{pgfscope}%
\begin{pgfscope}%
\pgfsys@transformshift{4.114131in}{4.174827in}%
\pgfsys@useobject{currentmarker}{}%
\end{pgfscope}%
\begin{pgfscope}%
\pgfsys@transformshift{4.135024in}{4.085998in}%
\pgfsys@useobject{currentmarker}{}%
\end{pgfscope}%
\begin{pgfscope}%
\pgfsys@transformshift{4.153566in}{4.052148in}%
\pgfsys@useobject{currentmarker}{}%
\end{pgfscope}%
\begin{pgfscope}%
\pgfsys@transformshift{4.173048in}{4.041336in}%
\pgfsys@useobject{currentmarker}{}%
\end{pgfscope}%
\begin{pgfscope}%
\pgfsys@transformshift{4.192062in}{4.036711in}%
\pgfsys@useobject{currentmarker}{}%
\end{pgfscope}%
\begin{pgfscope}%
\pgfsys@transformshift{4.211075in}{4.037200in}%
\pgfsys@useobject{currentmarker}{}%
\end{pgfscope}%
\begin{pgfscope}%
\pgfsys@transformshift{4.230323in}{4.040540in}%
\pgfsys@useobject{currentmarker}{}%
\end{pgfscope}%
\begin{pgfscope}%
\pgfsys@transformshift{4.250041in}{4.053340in}%
\pgfsys@useobject{currentmarker}{}%
\end{pgfscope}%
\begin{pgfscope}%
\pgfsys@transformshift{4.268115in}{4.090691in}%
\pgfsys@useobject{currentmarker}{}%
\end{pgfscope}%
\begin{pgfscope}%
\pgfsys@transformshift{4.287362in}{4.147384in}%
\pgfsys@useobject{currentmarker}{}%
\end{pgfscope}%
\begin{pgfscope}%
\pgfsys@transformshift{4.306610in}{4.342316in}%
\pgfsys@useobject{currentmarker}{}%
\end{pgfscope}%
\begin{pgfscope}%
\pgfsys@transformshift{4.326797in}{4.450223in}%
\pgfsys@useobject{currentmarker}{}%
\end{pgfscope}%
\begin{pgfscope}%
\pgfsys@transformshift{4.345340in}{4.446244in}%
\pgfsys@useobject{currentmarker}{}%
\end{pgfscope}%
\begin{pgfscope}%
\pgfsys@transformshift{4.365527in}{4.354061in}%
\pgfsys@useobject{currentmarker}{}%
\end{pgfscope}%
\begin{pgfscope}%
\pgfsys@transformshift{4.382663in}{4.244285in}%
\pgfsys@useobject{currentmarker}{}%
\end{pgfscope}%
\begin{pgfscope}%
\pgfsys@transformshift{4.402614in}{4.146811in}%
\pgfsys@useobject{currentmarker}{}%
\end{pgfscope}%
\begin{pgfscope}%
\pgfsys@transformshift{4.425147in}{4.070172in}%
\pgfsys@useobject{currentmarker}{}%
\end{pgfscope}%
\begin{pgfscope}%
\pgfsys@transformshift{4.440875in}{4.048701in}%
\pgfsys@useobject{currentmarker}{}%
\end{pgfscope}%
\begin{pgfscope}%
\pgfsys@transformshift{4.459419in}{4.042843in}%
\pgfsys@useobject{currentmarker}{}%
\end{pgfscope}%
\begin{pgfscope}%
\pgfsys@transformshift{4.482187in}{4.065530in}%
\pgfsys@useobject{currentmarker}{}%
\end{pgfscope}%
\begin{pgfscope}%
\pgfsys@transformshift{4.483361in}{4.063736in}%
\pgfsys@useobject{currentmarker}{}%
\end{pgfscope}%
\begin{pgfscope}%
\pgfsys@transformshift{4.473973in}{4.090790in}%
\pgfsys@useobject{currentmarker}{}%
\end{pgfscope}%
\begin{pgfscope}%
\pgfsys@transformshift{4.456368in}{4.227631in}%
\pgfsys@useobject{currentmarker}{}%
\end{pgfscope}%
\begin{pgfscope}%
\pgfsys@transformshift{4.435241in}{4.426153in}%
\pgfsys@useobject{currentmarker}{}%
\end{pgfscope}%
\begin{pgfscope}%
\pgfsys@transformshift{4.418342in}{4.456515in}%
\pgfsys@useobject{currentmarker}{}%
\end{pgfscope}%
\begin{pgfscope}%
\pgfsys@transformshift{4.397451in}{4.262871in}%
\pgfsys@useobject{currentmarker}{}%
\end{pgfscope}%
\begin{pgfscope}%
\pgfsys@transformshift{4.378203in}{4.091182in}%
\pgfsys@useobject{currentmarker}{}%
\end{pgfscope}%
\begin{pgfscope}%
\pgfsys@transformshift{4.359190in}{4.047447in}%
\pgfsys@useobject{currentmarker}{}%
\end{pgfscope}%
\begin{pgfscope}%
\pgfsys@transformshift{4.341585in}{4.049254in}%
\pgfsys@useobject{currentmarker}{}%
\end{pgfscope}%
\begin{pgfscope}%
\pgfsys@transformshift{4.320929in}{4.110546in}%
\pgfsys@useobject{currentmarker}{}%
\end{pgfscope}%
\begin{pgfscope}%
\pgfsys@transformshift{4.300037in}{4.327738in}%
\pgfsys@useobject{currentmarker}{}%
\end{pgfscope}%
\begin{pgfscope}%
\pgfsys@transformshift{4.280320in}{4.448796in}%
\pgfsys@useobject{currentmarker}{}%
\end{pgfscope}%
\begin{pgfscope}%
\pgfsys@transformshift{4.262012in}{4.403847in}%
\pgfsys@useobject{currentmarker}{}%
\end{pgfscope}%
\begin{pgfscope}%
\pgfsys@transformshift{4.245345in}{4.161808in}%
\pgfsys@useobject{currentmarker}{}%
\end{pgfscope}%
\begin{pgfscope}%
\pgfsys@transformshift{4.224923in}{4.058689in}%
\pgfsys@useobject{currentmarker}{}%
\end{pgfscope}%
\begin{pgfscope}%
\pgfsys@transformshift{4.206850in}{4.039833in}%
\pgfsys@useobject{currentmarker}{}%
\end{pgfscope}%
\begin{pgfscope}%
\pgfsys@transformshift{4.185490in}{4.036369in}%
\pgfsys@useobject{currentmarker}{}%
\end{pgfscope}%
\begin{pgfscope}%
\pgfsys@transformshift{4.168589in}{4.044145in}%
\pgfsys@useobject{currentmarker}{}%
\end{pgfscope}%
\begin{pgfscope}%
\pgfsys@transformshift{4.146995in}{4.080005in}%
\pgfsys@useobject{currentmarker}{}%
\end{pgfscope}%
\begin{pgfscope}%
\pgfsys@transformshift{4.129154in}{4.216700in}%
\pgfsys@useobject{currentmarker}{}%
\end{pgfscope}%
\begin{pgfscope}%
\pgfsys@transformshift{4.107794in}{4.414056in}%
\pgfsys@useobject{currentmarker}{}%
\end{pgfscope}%
\begin{pgfscope}%
\pgfsys@transformshift{4.090424in}{4.431710in}%
\pgfsys@useobject{currentmarker}{}%
\end{pgfscope}%
\begin{pgfscope}%
\pgfsys@transformshift{4.072116in}{4.270083in}%
\pgfsys@useobject{currentmarker}{}%
\end{pgfscope}%
\begin{pgfscope}%
\pgfsys@transformshift{4.052163in}{4.086689in}%
\pgfsys@useobject{currentmarker}{}%
\end{pgfscope}%
\begin{pgfscope}%
\pgfsys@transformshift{4.031976in}{4.044498in}%
\pgfsys@useobject{currentmarker}{}%
\end{pgfscope}%
\begin{pgfscope}%
\pgfsys@transformshift{4.013199in}{4.036127in}%
\pgfsys@useobject{currentmarker}{}%
\end{pgfscope}%
\begin{pgfscope}%
\pgfsys@transformshift{3.993951in}{4.037955in}%
\pgfsys@useobject{currentmarker}{}%
\end{pgfscope}%
\begin{pgfscope}%
\pgfsys@transformshift{3.976581in}{4.050369in}%
\pgfsys@useobject{currentmarker}{}%
\end{pgfscope}%
\begin{pgfscope}%
\pgfsys@transformshift{3.955690in}{4.122785in}%
\pgfsys@useobject{currentmarker}{}%
\end{pgfscope}%
\begin{pgfscope}%
\pgfsys@transformshift{3.936442in}{4.308111in}%
\pgfsys@useobject{currentmarker}{}%
\end{pgfscope}%
\begin{pgfscope}%
\pgfsys@transformshift{3.917195in}{4.428129in}%
\pgfsys@useobject{currentmarker}{}%
\end{pgfscope}%
\begin{pgfscope}%
\pgfsys@transformshift{3.898416in}{4.404560in}%
\pgfsys@useobject{currentmarker}{}%
\end{pgfscope}%
\begin{pgfscope}%
\pgfsys@transformshift{3.876820in}{4.147722in}%
\pgfsys@useobject{currentmarker}{}%
\end{pgfscope}%
\begin{pgfscope}%
\pgfsys@transformshift{3.859450in}{4.076469in}%
\pgfsys@useobject{currentmarker}{}%
\end{pgfscope}%
\begin{pgfscope}%
\pgfsys@transformshift{3.835742in}{4.040604in}%
\pgfsys@useobject{currentmarker}{}%
\end{pgfscope}%
\begin{pgfscope}%
\pgfsys@transformshift{3.820954in}{4.036088in}%
\pgfsys@useobject{currentmarker}{}%
\end{pgfscope}%
\begin{pgfscope}%
\pgfsys@transformshift{3.802881in}{4.035695in}%
\pgfsys@useobject{currentmarker}{}%
\end{pgfscope}%
\begin{pgfscope}%
\pgfsys@transformshift{3.781756in}{4.044788in}%
\pgfsys@useobject{currentmarker}{}%
\end{pgfscope}%
\begin{pgfscope}%
\pgfsys@transformshift{3.764151in}{4.077111in}%
\pgfsys@useobject{currentmarker}{}%
\end{pgfscope}%
\begin{pgfscope}%
\pgfsys@transformshift{3.743024in}{4.211559in}%
\pgfsys@useobject{currentmarker}{}%
\end{pgfscope}%
\begin{pgfscope}%
\pgfsys@transformshift{3.727298in}{4.364289in}%
\pgfsys@useobject{currentmarker}{}%
\end{pgfscope}%
\begin{pgfscope}%
\pgfsys@transformshift{3.706172in}{4.424306in}%
\pgfsys@useobject{currentmarker}{}%
\end{pgfscope}%
\begin{pgfscope}%
\pgfsys@transformshift{3.685281in}{4.231374in}%
\pgfsys@useobject{currentmarker}{}%
\end{pgfscope}%
\begin{pgfscope}%
\pgfsys@transformshift{3.665799in}{4.118183in}%
\pgfsys@useobject{currentmarker}{}%
\end{pgfscope}%
\begin{pgfscope}%
\pgfsys@transformshift{3.647489in}{4.079266in}%
\pgfsys@useobject{currentmarker}{}%
\end{pgfscope}%
\begin{pgfscope}%
\pgfsys@transformshift{3.625190in}{4.043248in}%
\pgfsys@useobject{currentmarker}{}%
\end{pgfscope}%
\begin{pgfscope}%
\pgfsys@transformshift{3.609933in}{4.036762in}%
\pgfsys@useobject{currentmarker}{}%
\end{pgfscope}%
\begin{pgfscope}%
\pgfsys@transformshift{3.590451in}{4.035444in}%
\pgfsys@useobject{currentmarker}{}%
\end{pgfscope}%
\begin{pgfscope}%
\pgfsys@transformshift{3.571907in}{4.040836in}%
\pgfsys@useobject{currentmarker}{}%
\end{pgfscope}%
\begin{pgfscope}%
\pgfsys@transformshift{3.551016in}{4.067840in}%
\pgfsys@useobject{currentmarker}{}%
\end{pgfscope}%
\begin{pgfscope}%
\pgfsys@transformshift{3.531768in}{4.130038in}%
\pgfsys@useobject{currentmarker}{}%
\end{pgfscope}%
\begin{pgfscope}%
\pgfsys@transformshift{3.512521in}{4.312577in}%
\pgfsys@useobject{currentmarker}{}%
\end{pgfscope}%
\begin{pgfscope}%
\pgfsys@transformshift{3.495150in}{4.416142in}%
\pgfsys@useobject{currentmarker}{}%
\end{pgfscope}%
\begin{pgfscope}%
\pgfsys@transformshift{3.475668in}{4.368164in}%
\pgfsys@useobject{currentmarker}{}%
\end{pgfscope}%
\begin{pgfscope}%
\pgfsys@transformshift{3.456655in}{4.229371in}%
\pgfsys@useobject{currentmarker}{}%
\end{pgfscope}%
\begin{pgfscope}%
\pgfsys@transformshift{3.435530in}{4.076816in}%
\pgfsys@useobject{currentmarker}{}%
\end{pgfscope}%
\begin{pgfscope}%
\pgfsys@transformshift{3.416986in}{4.050149in}%
\pgfsys@useobject{currentmarker}{}%
\end{pgfscope}%
\begin{pgfscope}%
\pgfsys@transformshift{3.398207in}{4.039016in}%
\pgfsys@useobject{currentmarker}{}%
\end{pgfscope}%
\begin{pgfscope}%
\pgfsys@transformshift{3.379664in}{4.035106in}%
\pgfsys@useobject{currentmarker}{}%
\end{pgfscope}%
\begin{pgfscope}%
\pgfsys@transformshift{3.358068in}{4.038974in}%
\pgfsys@useobject{currentmarker}{}%
\end{pgfscope}%
\begin{pgfscope}%
\pgfsys@transformshift{3.339526in}{4.053334in}%
\pgfsys@useobject{currentmarker}{}%
\end{pgfscope}%
\begin{pgfscope}%
\pgfsys@transformshift{3.320042in}{4.096290in}%
\pgfsys@useobject{currentmarker}{}%
\end{pgfscope}%
\begin{pgfscope}%
\pgfsys@transformshift{3.304785in}{4.193924in}%
\pgfsys@useobject{currentmarker}{}%
\end{pgfscope}%
\begin{pgfscope}%
\pgfsys@transformshift{3.282721in}{4.345656in}%
\pgfsys@useobject{currentmarker}{}%
\end{pgfscope}%
\begin{pgfscope}%
\pgfsys@transformshift{3.263004in}{4.053304in}%
\pgfsys@useobject{currentmarker}{}%
\end{pgfscope}%
\begin{pgfscope}%
\pgfsys@transformshift{3.248451in}{4.084756in}%
\pgfsys@useobject{currentmarker}{}%
\end{pgfscope}%
\begin{pgfscope}%
\pgfsys@transformshift{3.225681in}{4.199651in}%
\pgfsys@useobject{currentmarker}{}%
\end{pgfscope}%
\begin{pgfscope}%
\pgfsys@transformshift{3.207842in}{4.375687in}%
\pgfsys@useobject{currentmarker}{}%
\end{pgfscope}%
\begin{pgfscope}%
\pgfsys@transformshift{3.184369in}{4.399952in}%
\pgfsys@useobject{currentmarker}{}%
\end{pgfscope}%
\begin{pgfscope}%
\pgfsys@transformshift{3.168407in}{4.207868in}%
\pgfsys@useobject{currentmarker}{}%
\end{pgfscope}%
\begin{pgfscope}%
\pgfsys@transformshift{3.150099in}{4.086324in}%
\pgfsys@useobject{currentmarker}{}%
\end{pgfscope}%
\begin{pgfscope}%
\pgfsys@transformshift{3.128503in}{4.045400in}%
\pgfsys@useobject{currentmarker}{}%
\end{pgfscope}%
\begin{pgfscope}%
\pgfsys@transformshift{3.109255in}{4.036658in}%
\pgfsys@useobject{currentmarker}{}%
\end{pgfscope}%
\begin{pgfscope}%
\pgfsys@transformshift{3.090947in}{4.035101in}%
\pgfsys@useobject{currentmarker}{}%
\end{pgfscope}%
\begin{pgfscope}%
\pgfsys@transformshift{3.071465in}{4.040369in}%
\pgfsys@useobject{currentmarker}{}%
\end{pgfscope}%
\begin{pgfscope}%
\pgfsys@transformshift{3.050809in}{4.062438in}%
\pgfsys@useobject{currentmarker}{}%
\end{pgfscope}%
\begin{pgfscope}%
\pgfsys@transformshift{3.031090in}{4.132320in}%
\pgfsys@useobject{currentmarker}{}%
\end{pgfscope}%
\begin{pgfscope}%
\pgfsys@transformshift{3.013486in}{4.259087in}%
\pgfsys@useobject{currentmarker}{}%
\end{pgfscope}%
\begin{pgfscope}%
\pgfsys@transformshift{2.994472in}{4.385737in}%
\pgfsys@useobject{currentmarker}{}%
\end{pgfscope}%
\begin{pgfscope}%
\pgfsys@transformshift{2.975930in}{4.403641in}%
\pgfsys@useobject{currentmarker}{}%
\end{pgfscope}%
\begin{pgfscope}%
\pgfsys@transformshift{2.957151in}{4.228437in}%
\pgfsys@useobject{currentmarker}{}%
\end{pgfscope}%
\begin{pgfscope}%
\pgfsys@transformshift{2.935790in}{4.081638in}%
\pgfsys@useobject{currentmarker}{}%
\end{pgfscope}%
\begin{pgfscope}%
\pgfsys@transformshift{2.917482in}{4.046712in}%
\pgfsys@useobject{currentmarker}{}%
\end{pgfscope}%
\begin{pgfscope}%
\pgfsys@transformshift{2.895417in}{4.036211in}%
\pgfsys@useobject{currentmarker}{}%
\end{pgfscope}%
\begin{pgfscope}%
\pgfsys@transformshift{2.881098in}{4.034853in}%
\pgfsys@useobject{currentmarker}{}%
\end{pgfscope}%
\begin{pgfscope}%
\pgfsys@transformshift{2.862085in}{4.035892in}%
\pgfsys@useobject{currentmarker}{}%
\end{pgfscope}%
\begin{pgfscope}%
\pgfsys@transformshift{2.840020in}{4.045499in}%
\pgfsys@useobject{currentmarker}{}%
\end{pgfscope}%
\begin{pgfscope}%
\pgfsys@transformshift{2.820069in}{4.073592in}%
\pgfsys@useobject{currentmarker}{}%
\end{pgfscope}%
\begin{pgfscope}%
\pgfsys@transformshift{2.802230in}{4.157225in}%
\pgfsys@useobject{currentmarker}{}%
\end{pgfscope}%
\begin{pgfscope}%
\pgfsys@transformshift{2.783217in}{4.323124in}%
\pgfsys@useobject{currentmarker}{}%
\end{pgfscope}%
\begin{pgfscope}%
\pgfsys@transformshift{2.764907in}{4.411101in}%
\pgfsys@useobject{currentmarker}{}%
\end{pgfscope}%
\begin{pgfscope}%
\pgfsys@transformshift{2.745895in}{4.363264in}%
\pgfsys@useobject{currentmarker}{}%
\end{pgfscope}%
\begin{pgfscope}%
\pgfsys@transformshift{2.724534in}{4.140008in}%
\pgfsys@useobject{currentmarker}{}%
\end{pgfscope}%
\begin{pgfscope}%
\pgfsys@transformshift{2.705990in}{4.073502in}%
\pgfsys@useobject{currentmarker}{}%
\end{pgfscope}%
\begin{pgfscope}%
\pgfsys@transformshift{2.687213in}{4.043539in}%
\pgfsys@useobject{currentmarker}{}%
\end{pgfscope}%
\begin{pgfscope}%
\pgfsys@transformshift{2.669137in}{4.036556in}%
\pgfsys@useobject{currentmarker}{}%
\end{pgfscope}%
\begin{pgfscope}%
\pgfsys@transformshift{2.649655in}{4.034644in}%
\pgfsys@useobject{currentmarker}{}%
\end{pgfscope}%
\begin{pgfscope}%
\pgfsys@transformshift{2.627825in}{4.039835in}%
\pgfsys@useobject{currentmarker}{}%
\end{pgfscope}%
\begin{pgfscope}%
\pgfsys@transformshift{2.609282in}{4.053999in}%
\pgfsys@useobject{currentmarker}{}%
\end{pgfscope}%
\begin{pgfscope}%
\pgfsys@transformshift{2.592615in}{4.069822in}%
\pgfsys@useobject{currentmarker}{}%
\end{pgfscope}%
\begin{pgfscope}%
\pgfsys@transformshift{2.573604in}{4.160333in}%
\pgfsys@useobject{currentmarker}{}%
\end{pgfscope}%
\begin{pgfscope}%
\pgfsys@transformshift{2.552711in}{4.296058in}%
\pgfsys@useobject{currentmarker}{}%
\end{pgfscope}%
\begin{pgfscope}%
\pgfsys@transformshift{2.532995in}{4.378895in}%
\pgfsys@useobject{currentmarker}{}%
\end{pgfscope}%
\begin{pgfscope}%
\pgfsys@transformshift{2.515390in}{4.399724in}%
\pgfsys@useobject{currentmarker}{}%
\end{pgfscope}%
\begin{pgfscope}%
\pgfsys@transformshift{2.496142in}{4.216659in}%
\pgfsys@useobject{currentmarker}{}%
\end{pgfscope}%
\begin{pgfscope}%
\pgfsys@transformshift{2.475252in}{4.092532in}%
\pgfsys@useobject{currentmarker}{}%
\end{pgfscope}%
\begin{pgfscope}%
\pgfsys@transformshift{2.456707in}{4.050889in}%
\pgfsys@useobject{currentmarker}{}%
\end{pgfscope}%
\begin{pgfscope}%
\pgfsys@transformshift{2.438399in}{4.037919in}%
\pgfsys@useobject{currentmarker}{}%
\end{pgfscope}%
\begin{pgfscope}%
\pgfsys@transformshift{2.415395in}{4.034654in}%
\pgfsys@useobject{currentmarker}{}%
\end{pgfscope}%
\begin{pgfscope}%
\pgfsys@transformshift{2.397790in}{4.035748in}%
\pgfsys@useobject{currentmarker}{}%
\end{pgfscope}%
\begin{pgfscope}%
\pgfsys@transformshift{2.380891in}{4.041535in}%
\pgfsys@useobject{currentmarker}{}%
\end{pgfscope}%
\begin{pgfscope}%
\pgfsys@transformshift{2.359061in}{4.057605in}%
\pgfsys@useobject{currentmarker}{}%
\end{pgfscope}%
\begin{pgfscope}%
\pgfsys@transformshift{2.340282in}{4.085537in}%
\pgfsys@useobject{currentmarker}{}%
\end{pgfscope}%
\begin{pgfscope}%
\pgfsys@transformshift{2.320331in}{4.220641in}%
\pgfsys@useobject{currentmarker}{}%
\end{pgfscope}%
\begin{pgfscope}%
\pgfsys@transformshift{2.301786in}{4.374560in}%
\pgfsys@useobject{currentmarker}{}%
\end{pgfscope}%
\begin{pgfscope}%
\pgfsys@transformshift{2.286530in}{4.407006in}%
\pgfsys@useobject{currentmarker}{}%
\end{pgfscope}%
\begin{pgfscope}%
\pgfsys@transformshift{2.264934in}{4.322510in}%
\pgfsys@useobject{currentmarker}{}%
\end{pgfscope}%
\begin{pgfscope}%
\pgfsys@transformshift{2.245921in}{4.140549in}%
\pgfsys@useobject{currentmarker}{}%
\end{pgfscope}%
\begin{pgfscope}%
\pgfsys@transformshift{2.225970in}{4.062843in}%
\pgfsys@useobject{currentmarker}{}%
\end{pgfscope}%
\begin{pgfscope}%
\pgfsys@transformshift{2.202731in}{4.043236in}%
\pgfsys@useobject{currentmarker}{}%
\end{pgfscope}%
\begin{pgfscope}%
\pgfsys@transformshift{2.187709in}{4.037084in}%
\pgfsys@useobject{currentmarker}{}%
\end{pgfscope}%
\begin{pgfscope}%
\pgfsys@transformshift{2.168461in}{4.092449in}%
\pgfsys@useobject{currentmarker}{}%
\end{pgfscope}%
\begin{pgfscope}%
\pgfsys@transformshift{2.150856in}{4.049938in}%
\pgfsys@useobject{currentmarker}{}%
\end{pgfscope}%
\begin{pgfscope}%
\pgfsys@transformshift{2.131843in}{4.037408in}%
\pgfsys@useobject{currentmarker}{}%
\end{pgfscope}%
\begin{pgfscope}%
\pgfsys@transformshift{2.110247in}{4.034847in}%
\pgfsys@useobject{currentmarker}{}%
\end{pgfscope}%
\begin{pgfscope}%
\pgfsys@transformshift{2.091234in}{4.038144in}%
\pgfsys@useobject{currentmarker}{}%
\end{pgfscope}%
\begin{pgfscope}%
\pgfsys@transformshift{2.073629in}{4.049510in}%
\pgfsys@useobject{currentmarker}{}%
\end{pgfscope}%
\begin{pgfscope}%
\pgfsys@transformshift{2.054381in}{4.090396in}%
\pgfsys@useobject{currentmarker}{}%
\end{pgfscope}%
\begin{pgfscope}%
\pgfsys@transformshift{2.033491in}{4.238789in}%
\pgfsys@useobject{currentmarker}{}%
\end{pgfscope}%
\begin{pgfscope}%
\pgfsys@transformshift{2.013774in}{4.378306in}%
\pgfsys@useobject{currentmarker}{}%
\end{pgfscope}%
\begin{pgfscope}%
\pgfsys@transformshift{1.995464in}{4.411516in}%
\pgfsys@useobject{currentmarker}{}%
\end{pgfscope}%
\begin{pgfscope}%
\pgfsys@transformshift{1.976217in}{4.331713in}%
\pgfsys@useobject{currentmarker}{}%
\end{pgfscope}%
\begin{pgfscope}%
\pgfsys@transformshift{1.959317in}{4.135084in}%
\pgfsys@useobject{currentmarker}{}%
\end{pgfscope}%
\begin{pgfscope}%
\pgfsys@transformshift{1.937016in}{4.057379in}%
\pgfsys@useobject{currentmarker}{}%
\end{pgfscope}%
\begin{pgfscope}%
\pgfsys@transformshift{1.918708in}{4.041132in}%
\pgfsys@useobject{currentmarker}{}%
\end{pgfscope}%
\begin{pgfscope}%
\pgfsys@transformshift{1.900634in}{4.035924in}%
\pgfsys@useobject{currentmarker}{}%
\end{pgfscope}%
\begin{pgfscope}%
\pgfsys@transformshift{1.878335in}{4.035242in}%
\pgfsys@useobject{currentmarker}{}%
\end{pgfscope}%
\begin{pgfscope}%
\pgfsys@transformshift{1.860731in}{4.038754in}%
\pgfsys@useobject{currentmarker}{}%
\end{pgfscope}%
\begin{pgfscope}%
\pgfsys@transformshift{1.841717in}{4.045987in}%
\pgfsys@useobject{currentmarker}{}%
\end{pgfscope}%
\begin{pgfscope}%
\pgfsys@transformshift{1.820827in}{4.088951in}%
\pgfsys@useobject{currentmarker}{}%
\end{pgfscope}%
\begin{pgfscope}%
\pgfsys@transformshift{1.801108in}{4.194613in}%
\pgfsys@useobject{currentmarker}{}%
\end{pgfscope}%
\begin{pgfscope}%
\pgfsys@transformshift{1.786320in}{4.322175in}%
\pgfsys@useobject{currentmarker}{}%
\end{pgfscope}%
\begin{pgfscope}%
\pgfsys@transformshift{1.764490in}{4.414289in}%
\pgfsys@useobject{currentmarker}{}%
\end{pgfscope}%
\begin{pgfscope}%
\pgfsys@transformshift{1.745477in}{4.398031in}%
\pgfsys@useobject{currentmarker}{}%
\end{pgfscope}%
\begin{pgfscope}%
\pgfsys@transformshift{1.727874in}{4.354251in}%
\pgfsys@useobject{currentmarker}{}%
\end{pgfscope}%
\begin{pgfscope}%
\pgfsys@transformshift{1.708156in}{4.151340in}%
\pgfsys@useobject{currentmarker}{}%
\end{pgfscope}%
\begin{pgfscope}%
\pgfsys@transformshift{1.687265in}{4.064730in}%
\pgfsys@useobject{currentmarker}{}%
\end{pgfscope}%
\begin{pgfscope}%
\pgfsys@transformshift{1.667549in}{4.045754in}%
\pgfsys@useobject{currentmarker}{}%
\end{pgfscope}%
\begin{pgfscope}%
\pgfsys@transformshift{1.649239in}{4.038557in}%
\pgfsys@useobject{currentmarker}{}%
\end{pgfscope}%
\begin{pgfscope}%
\pgfsys@transformshift{1.630931in}{4.035785in}%
\pgfsys@useobject{currentmarker}{}%
\end{pgfscope}%
\begin{pgfscope}%
\pgfsys@transformshift{1.613326in}{4.035401in}%
\pgfsys@useobject{currentmarker}{}%
\end{pgfscope}%
\begin{pgfscope}%
\pgfsys@transformshift{1.588679in}{4.041098in}%
\pgfsys@useobject{currentmarker}{}%
\end{pgfscope}%
\begin{pgfscope}%
\pgfsys@transformshift{1.571543in}{4.062668in}%
\pgfsys@useobject{currentmarker}{}%
\end{pgfscope}%
\begin{pgfscope}%
\pgfsys@transformshift{1.555346in}{4.109279in}%
\pgfsys@useobject{currentmarker}{}%
\end{pgfscope}%
\begin{pgfscope}%
\pgfsys@transformshift{1.534690in}{4.217002in}%
\pgfsys@useobject{currentmarker}{}%
\end{pgfscope}%
\begin{pgfscope}%
\pgfsys@transformshift{1.514505in}{4.243269in}%
\pgfsys@useobject{currentmarker}{}%
\end{pgfscope}%
\begin{pgfscope}%
\pgfsys@transformshift{1.495726in}{4.382767in}%
\pgfsys@useobject{currentmarker}{}%
\end{pgfscope}%
\begin{pgfscope}%
\pgfsys@transformshift{1.477182in}{4.423883in}%
\pgfsys@useobject{currentmarker}{}%
\end{pgfscope}%
\begin{pgfscope}%
\pgfsys@transformshift{1.459342in}{4.393057in}%
\pgfsys@useobject{currentmarker}{}%
\end{pgfscope}%
\begin{pgfscope}%
\pgfsys@transformshift{1.437749in}{4.167165in}%
\pgfsys@useobject{currentmarker}{}%
\end{pgfscope}%
\begin{pgfscope}%
\pgfsys@transformshift{1.418501in}{4.098649in}%
\pgfsys@useobject{currentmarker}{}%
\end{pgfscope}%
\begin{pgfscope}%
\pgfsys@transformshift{1.401599in}{4.058062in}%
\pgfsys@useobject{currentmarker}{}%
\end{pgfscope}%
\begin{pgfscope}%
\pgfsys@transformshift{1.379769in}{4.041377in}%
\pgfsys@useobject{currentmarker}{}%
\end{pgfscope}%
\begin{pgfscope}%
\pgfsys@transformshift{1.357705in}{4.036049in}%
\pgfsys@useobject{currentmarker}{}%
\end{pgfscope}%
\begin{pgfscope}%
\pgfsys@transformshift{1.342213in}{4.035475in}%
\pgfsys@useobject{currentmarker}{}%
\end{pgfscope}%
\begin{pgfscope}%
\pgfsys@transformshift{1.323200in}{4.038415in}%
\pgfsys@useobject{currentmarker}{}%
\end{pgfscope}%
\begin{pgfscope}%
\pgfsys@transformshift{1.307238in}{4.051400in}%
\pgfsys@useobject{currentmarker}{}%
\end{pgfscope}%
\begin{pgfscope}%
\pgfsys@transformshift{1.285408in}{4.072993in}%
\pgfsys@useobject{currentmarker}{}%
\end{pgfscope}%
\begin{pgfscope}%
\pgfsys@transformshift{1.266395in}{4.121481in}%
\pgfsys@useobject{currentmarker}{}%
\end{pgfscope}%
\begin{pgfscope}%
\pgfsys@transformshift{1.246209in}{4.274144in}%
\pgfsys@useobject{currentmarker}{}%
\end{pgfscope}%
\begin{pgfscope}%
\pgfsys@transformshift{1.225788in}{4.405570in}%
\pgfsys@useobject{currentmarker}{}%
\end{pgfscope}%
\begin{pgfscope}%
\pgfsys@transformshift{1.205835in}{4.432730in}%
\pgfsys@useobject{currentmarker}{}%
\end{pgfscope}%
\begin{pgfscope}%
\pgfsys@transformshift{1.187527in}{4.433894in}%
\pgfsys@useobject{currentmarker}{}%
\end{pgfscope}%
\begin{pgfscope}%
\pgfsys@transformshift{1.169217in}{4.353603in}%
\pgfsys@useobject{currentmarker}{}%
\end{pgfscope}%
\begin{pgfscope}%
\pgfsys@transformshift{1.150909in}{4.186049in}%
\pgfsys@useobject{currentmarker}{}%
\end{pgfscope}%
\begin{pgfscope}%
\pgfsys@transformshift{1.129313in}{4.084521in}%
\pgfsys@useobject{currentmarker}{}%
\end{pgfscope}%
\begin{pgfscope}%
\pgfsys@transformshift{1.113822in}{4.063586in}%
\pgfsys@useobject{currentmarker}{}%
\end{pgfscope}%
\begin{pgfscope}%
\pgfsys@transformshift{1.092226in}{4.045334in}%
\pgfsys@useobject{currentmarker}{}%
\end{pgfscope}%
\begin{pgfscope}%
\pgfsys@transformshift{1.074152in}{4.427641in}%
\pgfsys@useobject{currentmarker}{}%
\end{pgfscope}%
\begin{pgfscope}%
\pgfsys@transformshift{1.058659in}{4.416642in}%
\pgfsys@useobject{currentmarker}{}%
\end{pgfscope}%
\begin{pgfscope}%
\pgfsys@transformshift{1.033778in}{4.221363in}%
\pgfsys@useobject{currentmarker}{}%
\end{pgfscope}%
\begin{pgfscope}%
\pgfsys@transformshift{1.015704in}{4.099551in}%
\pgfsys@useobject{currentmarker}{}%
\end{pgfscope}%
\begin{pgfscope}%
\pgfsys@transformshift{0.999039in}{4.057929in}%
\pgfsys@useobject{currentmarker}{}%
\end{pgfscope}%
\begin{pgfscope}%
\pgfsys@transformshift{0.977209in}{4.040571in}%
\pgfsys@useobject{currentmarker}{}%
\end{pgfscope}%
\begin{pgfscope}%
\pgfsys@transformshift{0.960309in}{4.036163in}%
\pgfsys@useobject{currentmarker}{}%
\end{pgfscope}%
\begin{pgfscope}%
\pgfsys@transformshift{0.939651in}{4.038609in}%
\pgfsys@useobject{currentmarker}{}%
\end{pgfscope}%
\begin{pgfscope}%
\pgfsys@transformshift{0.918526in}{4.047502in}%
\pgfsys@useobject{currentmarker}{}%
\end{pgfscope}%
\begin{pgfscope}%
\pgfsys@transformshift{0.901156in}{4.073945in}%
\pgfsys@useobject{currentmarker}{}%
\end{pgfscope}%
\begin{pgfscope}%
\pgfsys@transformshift{0.880970in}{4.162508in}%
\pgfsys@useobject{currentmarker}{}%
\end{pgfscope}%
\begin{pgfscope}%
\pgfsys@transformshift{0.863600in}{4.315957in}%
\pgfsys@useobject{currentmarker}{}%
\end{pgfscope}%
\begin{pgfscope}%
\pgfsys@transformshift{0.842944in}{4.434992in}%
\pgfsys@useobject{currentmarker}{}%
\end{pgfscope}%
\begin{pgfscope}%
\pgfsys@transformshift{0.822522in}{4.440193in}%
\pgfsys@useobject{currentmarker}{}%
\end{pgfscope}%
\begin{pgfscope}%
\pgfsys@transformshift{0.801866in}{4.270457in}%
\pgfsys@useobject{currentmarker}{}%
\end{pgfscope}%
\begin{pgfscope}%
\pgfsys@transformshift{0.784730in}{4.135223in}%
\pgfsys@useobject{currentmarker}{}%
\end{pgfscope}%
\begin{pgfscope}%
\pgfsys@transformshift{0.767125in}{4.077156in}%
\pgfsys@useobject{currentmarker}{}%
\end{pgfscope}%
\begin{pgfscope}%
\pgfsys@transformshift{0.746940in}{4.048916in}%
\pgfsys@useobject{currentmarker}{}%
\end{pgfscope}%
\begin{pgfscope}%
\pgfsys@transformshift{0.726753in}{4.038481in}%
\pgfsys@useobject{currentmarker}{}%
\end{pgfscope}%
\begin{pgfscope}%
\pgfsys@transformshift{0.707036in}{4.036849in}%
\pgfsys@useobject{currentmarker}{}%
\end{pgfscope}%
\begin{pgfscope}%
\pgfsys@transformshift{0.691074in}{4.040403in}%
\pgfsys@useobject{currentmarker}{}%
\end{pgfscope}%
\begin{pgfscope}%
\pgfsys@transformshift{0.669244in}{4.059057in}%
\pgfsys@useobject{currentmarker}{}%
\end{pgfscope}%
\begin{pgfscope}%
\pgfsys@transformshift{0.651405in}{4.081904in}%
\pgfsys@useobject{currentmarker}{}%
\end{pgfscope}%
\begin{pgfscope}%
\pgfsys@transformshift{0.651874in}{4.078476in}%
\pgfsys@useobject{currentmarker}{}%
\end{pgfscope}%
\begin{pgfscope}%
\pgfsys@transformshift{0.655865in}{4.066998in}%
\pgfsys@useobject{currentmarker}{}%
\end{pgfscope}%
\begin{pgfscope}%
\pgfsys@transformshift{0.677460in}{4.040180in}%
\pgfsys@useobject{currentmarker}{}%
\end{pgfscope}%
\begin{pgfscope}%
\pgfsys@transformshift{0.695534in}{4.037084in}%
\pgfsys@useobject{currentmarker}{}%
\end{pgfscope}%
\begin{pgfscope}%
\pgfsys@transformshift{0.714547in}{4.047937in}%
\pgfsys@useobject{currentmarker}{}%
\end{pgfscope}%
\begin{pgfscope}%
\pgfsys@transformshift{0.731212in}{4.091783in}%
\pgfsys@useobject{currentmarker}{}%
\end{pgfscope}%
\begin{pgfscope}%
\pgfsys@transformshift{0.753277in}{4.214889in}%
\pgfsys@useobject{currentmarker}{}%
\end{pgfscope}%
\begin{pgfscope}%
\pgfsys@transformshift{0.770882in}{4.437771in}%
\pgfsys@useobject{currentmarker}{}%
\end{pgfscope}%
\begin{pgfscope}%
\pgfsys@transformshift{0.793650in}{4.400562in}%
\pgfsys@useobject{currentmarker}{}%
\end{pgfscope}%
\begin{pgfscope}%
\pgfsys@transformshift{0.811725in}{4.215359in}%
\pgfsys@useobject{currentmarker}{}%
\end{pgfscope}%
\begin{pgfscope}%
\pgfsys@transformshift{0.830033in}{4.083119in}%
\pgfsys@useobject{currentmarker}{}%
\end{pgfscope}%
\begin{pgfscope}%
\pgfsys@transformshift{0.850455in}{4.044866in}%
\pgfsys@useobject{currentmarker}{}%
\end{pgfscope}%
\begin{pgfscope}%
\pgfsys@transformshift{0.869937in}{4.036242in}%
\pgfsys@useobject{currentmarker}{}%
\end{pgfscope}%
\begin{pgfscope}%
\pgfsys@transformshift{0.890593in}{4.041299in}%
\pgfsys@useobject{currentmarker}{}%
\end{pgfscope}%
\begin{pgfscope}%
\pgfsys@transformshift{0.906555in}{4.058791in}%
\pgfsys@useobject{currentmarker}{}%
\end{pgfscope}%
\begin{pgfscope}%
\pgfsys@transformshift{0.924629in}{4.127636in}%
\pgfsys@useobject{currentmarker}{}%
\end{pgfscope}%
\begin{pgfscope}%
\pgfsys@transformshift{0.944347in}{4.375784in}%
\pgfsys@useobject{currentmarker}{}%
\end{pgfscope}%
\begin{pgfscope}%
\pgfsys@transformshift{0.965707in}{4.428676in}%
\pgfsys@useobject{currentmarker}{}%
\end{pgfscope}%
\begin{pgfscope}%
\pgfsys@transformshift{0.981903in}{4.312521in}%
\pgfsys@useobject{currentmarker}{}%
\end{pgfscope}%
\begin{pgfscope}%
\pgfsys@transformshift{1.003733in}{4.109491in}%
\pgfsys@useobject{currentmarker}{}%
\end{pgfscope}%
\begin{pgfscope}%
\pgfsys@transformshift{1.022278in}{4.051970in}%
\pgfsys@useobject{currentmarker}{}%
\end{pgfscope}%
\begin{pgfscope}%
\pgfsys@transformshift{1.041525in}{4.037425in}%
\pgfsys@useobject{currentmarker}{}%
\end{pgfscope}%
\begin{pgfscope}%
\pgfsys@transformshift{1.059599in}{4.036252in}%
\pgfsys@useobject{currentmarker}{}%
\end{pgfscope}%
\begin{pgfscope}%
\pgfsys@transformshift{1.079081in}{4.046127in}%
\pgfsys@useobject{currentmarker}{}%
\end{pgfscope}%
\begin{pgfscope}%
\pgfsys@transformshift{1.099737in}{4.093718in}%
\pgfsys@useobject{currentmarker}{}%
\end{pgfscope}%
\begin{pgfscope}%
\pgfsys@transformshift{1.116168in}{4.212192in}%
\pgfsys@useobject{currentmarker}{}%
\end{pgfscope}%
\begin{pgfscope}%
\pgfsys@transformshift{1.136355in}{4.426838in}%
\pgfsys@useobject{currentmarker}{}%
\end{pgfscope}%
\begin{pgfscope}%
\pgfsys@transformshift{1.157717in}{4.372727in}%
\pgfsys@useobject{currentmarker}{}%
\end{pgfscope}%
\begin{pgfscope}%
\pgfsys@transformshift{1.175790in}{4.174381in}%
\pgfsys@useobject{currentmarker}{}%
\end{pgfscope}%
\begin{pgfscope}%
\pgfsys@transformshift{1.194803in}{4.068945in}%
\pgfsys@useobject{currentmarker}{}%
\end{pgfscope}%
\begin{pgfscope}%
\pgfsys@transformshift{1.216632in}{4.040029in}%
\pgfsys@useobject{currentmarker}{}%
\end{pgfscope}%
\begin{pgfscope}%
\pgfsys@transformshift{1.231890in}{4.035342in}%
\pgfsys@useobject{currentmarker}{}%
\end{pgfscope}%
\begin{pgfscope}%
\pgfsys@transformshift{1.250198in}{4.037777in}%
\pgfsys@useobject{currentmarker}{}%
\end{pgfscope}%
\begin{pgfscope}%
\pgfsys@transformshift{1.273906in}{4.055505in}%
\pgfsys@useobject{currentmarker}{}%
\end{pgfscope}%
\begin{pgfscope}%
\pgfsys@transformshift{1.291745in}{4.113175in}%
\pgfsys@useobject{currentmarker}{}%
\end{pgfscope}%
\begin{pgfscope}%
\pgfsys@transformshift{1.310290in}{4.323871in}%
\pgfsys@useobject{currentmarker}{}%
\end{pgfscope}%
\begin{pgfscope}%
\pgfsys@transformshift{1.331180in}{4.421174in}%
\pgfsys@useobject{currentmarker}{}%
\end{pgfscope}%
\begin{pgfscope}%
\pgfsys@transformshift{1.346673in}{4.319673in}%
\pgfsys@useobject{currentmarker}{}%
\end{pgfscope}%
\begin{pgfscope}%
\pgfsys@transformshift{1.368503in}{4.141221in}%
\pgfsys@useobject{currentmarker}{}%
\end{pgfscope}%
\begin{pgfscope}%
\pgfsys@transformshift{1.384463in}{4.069536in}%
\pgfsys@useobject{currentmarker}{}%
\end{pgfscope}%
\begin{pgfscope}%
\pgfsys@transformshift{1.406059in}{4.040452in}%
\pgfsys@useobject{currentmarker}{}%
\end{pgfscope}%
\begin{pgfscope}%
\pgfsys@transformshift{1.424838in}{4.035414in}%
\pgfsys@useobject{currentmarker}{}%
\end{pgfscope}%
\begin{pgfscope}%
\pgfsys@transformshift{1.446903in}{4.036463in}%
\pgfsys@useobject{currentmarker}{}%
\end{pgfscope}%
\begin{pgfscope}%
\pgfsys@transformshift{1.464037in}{4.046560in}%
\pgfsys@useobject{currentmarker}{}%
\end{pgfscope}%
\begin{pgfscope}%
\pgfsys@transformshift{1.482347in}{4.076487in}%
\pgfsys@useobject{currentmarker}{}%
\end{pgfscope}%
\begin{pgfscope}%
\pgfsys@transformshift{1.500889in}{4.103209in}%
\pgfsys@useobject{currentmarker}{}%
\end{pgfscope}%
\begin{pgfscope}%
\pgfsys@transformshift{1.521545in}{4.325953in}%
\pgfsys@useobject{currentmarker}{}%
\end{pgfscope}%
\begin{pgfscope}%
\pgfsys@transformshift{1.540324in}{4.421904in}%
\pgfsys@useobject{currentmarker}{}%
\end{pgfscope}%
\begin{pgfscope}%
\pgfsys@transformshift{1.560980in}{4.311438in}%
\pgfsys@useobject{currentmarker}{}%
\end{pgfscope}%
\begin{pgfscope}%
\pgfsys@transformshift{1.579290in}{4.166998in}%
\pgfsys@useobject{currentmarker}{}%
\end{pgfscope}%
\begin{pgfscope}%
\pgfsys@transformshift{1.596659in}{4.081826in}%
\pgfsys@useobject{currentmarker}{}%
\end{pgfscope}%
\begin{pgfscope}%
\pgfsys@transformshift{1.618254in}{4.043157in}%
\pgfsys@useobject{currentmarker}{}%
\end{pgfscope}%
\begin{pgfscope}%
\pgfsys@transformshift{1.635625in}{4.036069in}%
\pgfsys@useobject{currentmarker}{}%
\end{pgfscope}%
\begin{pgfscope}%
\pgfsys@transformshift{1.655812in}{4.035424in}%
\pgfsys@useobject{currentmarker}{}%
\end{pgfscope}%
\begin{pgfscope}%
\pgfsys@transformshift{1.673417in}{4.039879in}%
\pgfsys@useobject{currentmarker}{}%
\end{pgfscope}%
\begin{pgfscope}%
\pgfsys@transformshift{1.694542in}{4.059886in}%
\pgfsys@useobject{currentmarker}{}%
\end{pgfscope}%
\begin{pgfscope}%
\pgfsys@transformshift{1.712381in}{4.119230in}%
\pgfsys@useobject{currentmarker}{}%
\end{pgfscope}%
\begin{pgfscope}%
\pgfsys@transformshift{1.734446in}{4.359768in}%
\pgfsys@useobject{currentmarker}{}%
\end{pgfscope}%
\begin{pgfscope}%
\pgfsys@transformshift{1.751582in}{4.414693in}%
\pgfsys@useobject{currentmarker}{}%
\end{pgfscope}%
\begin{pgfscope}%
\pgfsys@transformshift{1.772707in}{4.372436in}%
\pgfsys@useobject{currentmarker}{}%
\end{pgfscope}%
\begin{pgfscope}%
\pgfsys@transformshift{1.790077in}{4.322342in}%
\pgfsys@useobject{currentmarker}{}%
\end{pgfscope}%
\begin{pgfscope}%
\pgfsys@transformshift{1.811436in}{4.137394in}%
\pgfsys@useobject{currentmarker}{}%
\end{pgfscope}%
\begin{pgfscope}%
\pgfsys@transformshift{1.830215in}{4.063213in}%
\pgfsys@useobject{currentmarker}{}%
\end{pgfscope}%
\begin{pgfscope}%
\pgfsys@transformshift{1.847820in}{4.040597in}%
\pgfsys@useobject{currentmarker}{}%
\end{pgfscope}%
\begin{pgfscope}%
\pgfsys@transformshift{1.865894in}{4.035495in}%
\pgfsys@useobject{currentmarker}{}%
\end{pgfscope}%
\begin{pgfscope}%
\pgfsys@transformshift{1.887255in}{4.035486in}%
\pgfsys@useobject{currentmarker}{}%
\end{pgfscope}%
\begin{pgfscope}%
\pgfsys@transformshift{1.908146in}{4.042953in}%
\pgfsys@useobject{currentmarker}{}%
\end{pgfscope}%
\begin{pgfscope}%
\pgfsys@transformshift{1.925045in}{4.068982in}%
\pgfsys@useobject{currentmarker}{}%
\end{pgfscope}%
\begin{pgfscope}%
\pgfsys@transformshift{1.942650in}{4.071332in}%
\pgfsys@useobject{currentmarker}{}%
\end{pgfscope}%
\begin{pgfscope}%
\pgfsys@transformshift{1.968940in}{4.034784in}%
\pgfsys@useobject{currentmarker}{}%
\end{pgfscope}%
\begin{pgfscope}%
\pgfsys@transformshift{1.981382in}{4.038661in}%
\pgfsys@useobject{currentmarker}{}%
\end{pgfscope}%
\begin{pgfscope}%
\pgfsys@transformshift{2.003681in}{4.051544in}%
\pgfsys@useobject{currentmarker}{}%
\end{pgfscope}%
\begin{pgfscope}%
\pgfsys@transformshift{2.021051in}{4.099543in}%
\pgfsys@useobject{currentmarker}{}%
\end{pgfscope}%
\begin{pgfscope}%
\pgfsys@transformshift{2.038420in}{4.251068in}%
\pgfsys@useobject{currentmarker}{}%
\end{pgfscope}%
\begin{pgfscope}%
\pgfsys@transformshift{2.060015in}{4.416380in}%
\pgfsys@useobject{currentmarker}{}%
\end{pgfscope}%
\begin{pgfscope}%
\pgfsys@transformshift{2.078089in}{4.343603in}%
\pgfsys@useobject{currentmarker}{}%
\end{pgfscope}%
\begin{pgfscope}%
\pgfsys@transformshift{2.097337in}{4.215749in}%
\pgfsys@useobject{currentmarker}{}%
\end{pgfscope}%
\begin{pgfscope}%
\pgfsys@transformshift{2.114004in}{4.067663in}%
\pgfsys@useobject{currentmarker}{}%
\end{pgfscope}%
\begin{pgfscope}%
\pgfsys@transformshift{2.134424in}{4.043793in}%
\pgfsys@useobject{currentmarker}{}%
\end{pgfscope}%
\begin{pgfscope}%
\pgfsys@transformshift{2.157662in}{4.035313in}%
\pgfsys@useobject{currentmarker}{}%
\end{pgfscope}%
\begin{pgfscope}%
\pgfsys@transformshift{2.175736in}{4.034720in}%
\pgfsys@useobject{currentmarker}{}%
\end{pgfscope}%
\begin{pgfscope}%
\pgfsys@transformshift{2.193577in}{4.039751in}%
\pgfsys@useobject{currentmarker}{}%
\end{pgfscope}%
\begin{pgfscope}%
\pgfsys@transformshift{2.212825in}{4.057407in}%
\pgfsys@useobject{currentmarker}{}%
\end{pgfscope}%
\begin{pgfscope}%
\pgfsys@transformshift{2.230664in}{4.121085in}%
\pgfsys@useobject{currentmarker}{}%
\end{pgfscope}%
\begin{pgfscope}%
\pgfsys@transformshift{2.252728in}{4.244282in}%
\pgfsys@useobject{currentmarker}{}%
\end{pgfscope}%
\begin{pgfscope}%
\pgfsys@transformshift{2.269394in}{4.410889in}%
\pgfsys@useobject{currentmarker}{}%
\end{pgfscope}%
\begin{pgfscope}%
\pgfsys@transformshift{2.292163in}{4.369755in}%
\pgfsys@useobject{currentmarker}{}%
\end{pgfscope}%
\begin{pgfscope}%
\pgfsys@transformshift{2.309063in}{4.220296in}%
\pgfsys@useobject{currentmarker}{}%
\end{pgfscope}%
\begin{pgfscope}%
\pgfsys@transformshift{2.332536in}{4.069154in}%
\pgfsys@useobject{currentmarker}{}%
\end{pgfscope}%
\begin{pgfscope}%
\pgfsys@transformshift{2.346619in}{4.048443in}%
\pgfsys@useobject{currentmarker}{}%
\end{pgfscope}%
\begin{pgfscope}%
\pgfsys@transformshift{2.366103in}{4.037245in}%
\pgfsys@useobject{currentmarker}{}%
\end{pgfscope}%
\begin{pgfscope}%
\pgfsys@transformshift{2.386993in}{4.034573in}%
\pgfsys@useobject{currentmarker}{}%
\end{pgfscope}%
\begin{pgfscope}%
\pgfsys@transformshift{2.405067in}{4.034598in}%
\pgfsys@useobject{currentmarker}{}%
\end{pgfscope}%
\begin{pgfscope}%
\pgfsys@transformshift{2.422672in}{4.036967in}%
\pgfsys@useobject{currentmarker}{}%
\end{pgfscope}%
\begin{pgfscope}%
\pgfsys@transformshift{2.443797in}{4.051404in}%
\pgfsys@useobject{currentmarker}{}%
\end{pgfscope}%
\begin{pgfscope}%
\pgfsys@transformshift{2.461872in}{4.082627in}%
\pgfsys@useobject{currentmarker}{}%
\end{pgfscope}%
\begin{pgfscope}%
\pgfsys@transformshift{2.483466in}{4.241424in}%
\pgfsys@useobject{currentmarker}{}%
\end{pgfscope}%
\begin{pgfscope}%
\pgfsys@transformshift{2.501307in}{4.405393in}%
\pgfsys@useobject{currentmarker}{}%
\end{pgfscope}%
\begin{pgfscope}%
\pgfsys@transformshift{2.519381in}{4.380095in}%
\pgfsys@useobject{currentmarker}{}%
\end{pgfscope}%
\begin{pgfscope}%
\pgfsys@transformshift{2.539803in}{4.210441in}%
\pgfsys@useobject{currentmarker}{}%
\end{pgfscope}%
\begin{pgfscope}%
\pgfsys@transformshift{2.556702in}{4.084314in}%
\pgfsys@useobject{currentmarker}{}%
\end{pgfscope}%
\begin{pgfscope}%
\pgfsys@transformshift{2.579941in}{4.044965in}%
\pgfsys@useobject{currentmarker}{}%
\end{pgfscope}%
\begin{pgfscope}%
\pgfsys@transformshift{2.597311in}{4.036755in}%
\pgfsys@useobject{currentmarker}{}%
\end{pgfscope}%
\begin{pgfscope}%
\pgfsys@transformshift{2.613977in}{4.034534in}%
\pgfsys@useobject{currentmarker}{}%
\end{pgfscope}%
\begin{pgfscope}%
\pgfsys@transformshift{2.633459in}{4.035403in}%
\pgfsys@useobject{currentmarker}{}%
\end{pgfscope}%
\begin{pgfscope}%
\pgfsys@transformshift{2.654586in}{4.038000in}%
\pgfsys@useobject{currentmarker}{}%
\end{pgfscope}%
\begin{pgfscope}%
\pgfsys@transformshift{2.675242in}{4.055339in}%
\pgfsys@useobject{currentmarker}{}%
\end{pgfscope}%
\begin{pgfscope}%
\pgfsys@transformshift{2.693081in}{4.111725in}%
\pgfsys@useobject{currentmarker}{}%
\end{pgfscope}%
\begin{pgfscope}%
\pgfsys@transformshift{2.714206in}{4.293138in}%
\pgfsys@useobject{currentmarker}{}%
\end{pgfscope}%
\begin{pgfscope}%
\pgfsys@transformshift{2.731811in}{4.409848in}%
\pgfsys@useobject{currentmarker}{}%
\end{pgfscope}%
\begin{pgfscope}%
\pgfsys@transformshift{2.748947in}{4.351610in}%
\pgfsys@useobject{currentmarker}{}%
\end{pgfscope}%
\begin{pgfscope}%
\pgfsys@transformshift{2.772420in}{4.187262in}%
\pgfsys@useobject{currentmarker}{}%
\end{pgfscope}%
\begin{pgfscope}%
\pgfsys@transformshift{2.789554in}{4.082120in}%
\pgfsys@useobject{currentmarker}{}%
\end{pgfscope}%
\begin{pgfscope}%
\pgfsys@transformshift{2.808567in}{4.047073in}%
\pgfsys@useobject{currentmarker}{}%
\end{pgfscope}%
\begin{pgfscope}%
\pgfsys@transformshift{2.829223in}{4.036816in}%
\pgfsys@useobject{currentmarker}{}%
\end{pgfscope}%
\begin{pgfscope}%
\pgfsys@transformshift{2.848002in}{4.034765in}%
\pgfsys@useobject{currentmarker}{}%
\end{pgfscope}%
\begin{pgfscope}%
\pgfsys@transformshift{2.864198in}{4.035679in}%
\pgfsys@useobject{currentmarker}{}%
\end{pgfscope}%
\begin{pgfscope}%
\pgfsys@transformshift{2.884854in}{4.042952in}%
\pgfsys@useobject{currentmarker}{}%
\end{pgfscope}%
\begin{pgfscope}%
\pgfsys@transformshift{2.906919in}{4.062315in}%
\pgfsys@useobject{currentmarker}{}%
\end{pgfscope}%
\begin{pgfscope}%
\pgfsys@transformshift{2.923819in}{4.104809in}%
\pgfsys@useobject{currentmarker}{}%
\end{pgfscope}%
\begin{pgfscope}%
\pgfsys@transformshift{2.943772in}{4.237240in}%
\pgfsys@useobject{currentmarker}{}%
\end{pgfscope}%
\begin{pgfscope}%
\pgfsys@transformshift{2.962314in}{4.405819in}%
\pgfsys@useobject{currentmarker}{}%
\end{pgfscope}%
\begin{pgfscope}%
\pgfsys@transformshift{2.981798in}{4.387651in}%
\pgfsys@useobject{currentmarker}{}%
\end{pgfscope}%
\begin{pgfscope}%
\pgfsys@transformshift{3.003157in}{4.202572in}%
\pgfsys@useobject{currentmarker}{}%
\end{pgfscope}%
\begin{pgfscope}%
\pgfsys@transformshift{3.021936in}{4.106142in}%
\pgfsys@useobject{currentmarker}{}%
\end{pgfscope}%
\begin{pgfscope}%
\pgfsys@transformshift{3.041184in}{4.051914in}%
\pgfsys@useobject{currentmarker}{}%
\end{pgfscope}%
\begin{pgfscope}%
\pgfsys@transformshift{3.059023in}{4.041939in}%
\pgfsys@useobject{currentmarker}{}%
\end{pgfscope}%
\begin{pgfscope}%
\pgfsys@transformshift{3.076159in}{4.036271in}%
\pgfsys@useobject{currentmarker}{}%
\end{pgfscope}%
\begin{pgfscope}%
\pgfsys@transformshift{3.098224in}{4.034888in}%
\pgfsys@useobject{currentmarker}{}%
\end{pgfscope}%
\begin{pgfscope}%
\pgfsys@transformshift{3.115358in}{4.037124in}%
\pgfsys@useobject{currentmarker}{}%
\end{pgfscope}%
\begin{pgfscope}%
\pgfsys@transformshift{3.136719in}{4.046467in}%
\pgfsys@useobject{currentmarker}{}%
\end{pgfscope}%
\begin{pgfscope}%
\pgfsys@transformshift{3.155498in}{4.070864in}%
\pgfsys@useobject{currentmarker}{}%
\end{pgfscope}%
\begin{pgfscope}%
\pgfsys@transformshift{3.173572in}{4.138547in}%
\pgfsys@useobject{currentmarker}{}%
\end{pgfscope}%
\begin{pgfscope}%
\pgfsys@transformshift{3.193993in}{4.333763in}%
\pgfsys@useobject{currentmarker}{}%
\end{pgfscope}%
\begin{pgfscope}%
\pgfsys@transformshift{3.211127in}{4.415879in}%
\pgfsys@useobject{currentmarker}{}%
\end{pgfscope}%
\begin{pgfscope}%
\pgfsys@transformshift{3.230611in}{4.394486in}%
\pgfsys@useobject{currentmarker}{}%
\end{pgfscope}%
\begin{pgfscope}%
\pgfsys@transformshift{3.251031in}{4.333239in}%
\pgfsys@useobject{currentmarker}{}%
\end{pgfscope}%
\begin{pgfscope}%
\pgfsys@transformshift{3.268636in}{4.417231in}%
\pgfsys@useobject{currentmarker}{}%
\end{pgfscope}%
\begin{pgfscope}%
\pgfsys@transformshift{3.291640in}{4.364672in}%
\pgfsys@useobject{currentmarker}{}%
\end{pgfscope}%
\begin{pgfscope}%
\pgfsys@transformshift{3.307133in}{4.211147in}%
\pgfsys@useobject{currentmarker}{}%
\end{pgfscope}%
\begin{pgfscope}%
\pgfsys@transformshift{3.329196in}{4.083119in}%
\pgfsys@useobject{currentmarker}{}%
\end{pgfscope}%
\begin{pgfscope}%
\pgfsys@transformshift{3.346332in}{4.047782in}%
\pgfsys@useobject{currentmarker}{}%
\end{pgfscope}%
\begin{pgfscope}%
\pgfsys@transformshift{3.365580in}{4.037994in}%
\pgfsys@useobject{currentmarker}{}%
\end{pgfscope}%
\begin{pgfscope}%
\pgfsys@transformshift{3.384827in}{4.034900in}%
\pgfsys@useobject{currentmarker}{}%
\end{pgfscope}%
\begin{pgfscope}%
\pgfsys@transformshift{3.404309in}{4.037368in}%
\pgfsys@useobject{currentmarker}{}%
\end{pgfscope}%
\begin{pgfscope}%
\pgfsys@transformshift{3.421914in}{4.041534in}%
\pgfsys@useobject{currentmarker}{}%
\end{pgfscope}%
\begin{pgfscope}%
\pgfsys@transformshift{3.442807in}{4.060309in}%
\pgfsys@useobject{currentmarker}{}%
\end{pgfscope}%
\begin{pgfscope}%
\pgfsys@transformshift{3.460646in}{4.041497in}%
\pgfsys@useobject{currentmarker}{}%
\end{pgfscope}%
\begin{pgfscope}%
\pgfsys@transformshift{3.479188in}{4.035147in}%
\pgfsys@useobject{currentmarker}{}%
\end{pgfscope}%
\begin{pgfscope}%
\pgfsys@transformshift{3.499376in}{4.035751in}%
\pgfsys@useobject{currentmarker}{}%
\end{pgfscope}%
\begin{pgfscope}%
\pgfsys@transformshift{3.517920in}{4.042317in}%
\pgfsys@useobject{currentmarker}{}%
\end{pgfscope}%
\begin{pgfscope}%
\pgfsys@transformshift{3.540219in}{4.067555in}%
\pgfsys@useobject{currentmarker}{}%
\end{pgfscope}%
\begin{pgfscope}%
\pgfsys@transformshift{3.557353in}{4.148021in}%
\pgfsys@useobject{currentmarker}{}%
\end{pgfscope}%
\begin{pgfscope}%
\pgfsys@transformshift{3.578480in}{4.372810in}%
\pgfsys@useobject{currentmarker}{}%
\end{pgfscope}%
\begin{pgfscope}%
\pgfsys@transformshift{3.596319in}{4.426447in}%
\pgfsys@useobject{currentmarker}{}%
\end{pgfscope}%
\begin{pgfscope}%
\pgfsys@transformshift{3.615098in}{4.351873in}%
\pgfsys@useobject{currentmarker}{}%
\end{pgfscope}%
\begin{pgfscope}%
\pgfsys@transformshift{3.635283in}{4.169932in}%
\pgfsys@useobject{currentmarker}{}%
\end{pgfscope}%
\begin{pgfscope}%
\pgfsys@transformshift{3.654297in}{4.078605in}%
\pgfsys@useobject{currentmarker}{}%
\end{pgfscope}%
\begin{pgfscope}%
\pgfsys@transformshift{3.671902in}{4.056338in}%
\pgfsys@useobject{currentmarker}{}%
\end{pgfscope}%
\begin{pgfscope}%
\pgfsys@transformshift{3.696549in}{4.036908in}%
\pgfsys@useobject{currentmarker}{}%
\end{pgfscope}%
\begin{pgfscope}%
\pgfsys@transformshift{3.710397in}{4.035198in}%
\pgfsys@useobject{currentmarker}{}%
\end{pgfscope}%
\begin{pgfscope}%
\pgfsys@transformshift{3.733870in}{4.036117in}%
\pgfsys@useobject{currentmarker}{}%
\end{pgfscope}%
\begin{pgfscope}%
\pgfsys@transformshift{3.749363in}{4.041929in}%
\pgfsys@useobject{currentmarker}{}%
\end{pgfscope}%
\begin{pgfscope}%
\pgfsys@transformshift{3.767906in}{4.058143in}%
\pgfsys@useobject{currentmarker}{}%
\end{pgfscope}%
\begin{pgfscope}%
\pgfsys@transformshift{3.789032in}{4.126429in}%
\pgfsys@useobject{currentmarker}{}%
\end{pgfscope}%
\begin{pgfscope}%
\pgfsys@transformshift{3.807341in}{4.281153in}%
\pgfsys@useobject{currentmarker}{}%
\end{pgfscope}%
\begin{pgfscope}%
\pgfsys@transformshift{3.829640in}{4.431902in}%
\pgfsys@useobject{currentmarker}{}%
\end{pgfscope}%
\begin{pgfscope}%
\pgfsys@transformshift{3.846305in}{4.418754in}%
\pgfsys@useobject{currentmarker}{}%
\end{pgfscope}%
\begin{pgfscope}%
\pgfsys@transformshift{3.867900in}{4.292208in}%
\pgfsys@useobject{currentmarker}{}%
\end{pgfscope}%
\begin{pgfscope}%
\pgfsys@transformshift{3.885271in}{4.152569in}%
\pgfsys@useobject{currentmarker}{}%
\end{pgfscope}%
\begin{pgfscope}%
\pgfsys@transformshift{3.904284in}{4.071944in}%
\pgfsys@useobject{currentmarker}{}%
\end{pgfscope}%
\begin{pgfscope}%
\pgfsys@transformshift{3.922358in}{4.050425in}%
\pgfsys@useobject{currentmarker}{}%
\end{pgfscope}%
\begin{pgfscope}%
\pgfsys@transformshift{3.941840in}{4.038332in}%
\pgfsys@useobject{currentmarker}{}%
\end{pgfscope}%
\begin{pgfscope}%
\pgfsys@transformshift{3.960619in}{4.035471in}%
\pgfsys@useobject{currentmarker}{}%
\end{pgfscope}%
\begin{pgfscope}%
\pgfsys@transformshift{3.981980in}{4.039010in}%
\pgfsys@useobject{currentmarker}{}%
\end{pgfscope}%
\begin{pgfscope}%
\pgfsys@transformshift{4.000288in}{4.047748in}%
\pgfsys@useobject{currentmarker}{}%
\end{pgfscope}%
\begin{pgfscope}%
\pgfsys@transformshift{4.021649in}{4.073220in}%
\pgfsys@useobject{currentmarker}{}%
\end{pgfscope}%
\begin{pgfscope}%
\pgfsys@transformshift{4.038783in}{4.158066in}%
\pgfsys@useobject{currentmarker}{}%
\end{pgfscope}%
\begin{pgfscope}%
\pgfsys@transformshift{4.057093in}{4.324663in}%
\pgfsys@useobject{currentmarker}{}%
\end{pgfscope}%
\begin{pgfscope}%
\pgfsys@transformshift{4.076341in}{4.035618in}%
\pgfsys@useobject{currentmarker}{}%
\end{pgfscope}%
\begin{pgfscope}%
\pgfsys@transformshift{4.094649in}{4.037520in}%
\pgfsys@useobject{currentmarker}{}%
\end{pgfscope}%
\begin{pgfscope}%
\pgfsys@transformshift{4.116714in}{4.052359in}%
\pgfsys@useobject{currentmarker}{}%
\end{pgfscope}%
\begin{pgfscope}%
\pgfsys@transformshift{4.136196in}{4.097126in}%
\pgfsys@useobject{currentmarker}{}%
\end{pgfscope}%
\begin{pgfscope}%
\pgfsys@transformshift{4.158966in}{4.255980in}%
\pgfsys@useobject{currentmarker}{}%
\end{pgfscope}%
\begin{pgfscope}%
\pgfsys@transformshift{4.171640in}{4.418818in}%
\pgfsys@useobject{currentmarker}{}%
\end{pgfscope}%
\begin{pgfscope}%
\pgfsys@transformshift{4.193236in}{4.438203in}%
\pgfsys@useobject{currentmarker}{}%
\end{pgfscope}%
\begin{pgfscope}%
\pgfsys@transformshift{4.212249in}{4.342554in}%
\pgfsys@useobject{currentmarker}{}%
\end{pgfscope}%
\begin{pgfscope}%
\pgfsys@transformshift{4.231731in}{4.173877in}%
\pgfsys@useobject{currentmarker}{}%
\end{pgfscope}%
\begin{pgfscope}%
\pgfsys@transformshift{4.250275in}{4.093059in}%
\pgfsys@useobject{currentmarker}{}%
\end{pgfscope}%
\begin{pgfscope}%
\pgfsys@transformshift{4.270932in}{4.050634in}%
\pgfsys@useobject{currentmarker}{}%
\end{pgfscope}%
\begin{pgfscope}%
\pgfsys@transformshift{4.288066in}{4.039099in}%
\pgfsys@useobject{currentmarker}{}%
\end{pgfscope}%
\begin{pgfscope}%
\pgfsys@transformshift{4.308253in}{4.036155in}%
\pgfsys@useobject{currentmarker}{}%
\end{pgfscope}%
\begin{pgfscope}%
\pgfsys@transformshift{4.327501in}{4.039641in}%
\pgfsys@useobject{currentmarker}{}%
\end{pgfscope}%
\begin{pgfscope}%
\pgfsys@transformshift{4.345340in}{4.052810in}%
\pgfsys@useobject{currentmarker}{}%
\end{pgfscope}%
\begin{pgfscope}%
\pgfsys@transformshift{4.364822in}{4.095317in}%
\pgfsys@useobject{currentmarker}{}%
\end{pgfscope}%
\begin{pgfscope}%
\pgfsys@transformshift{4.384775in}{4.209480in}%
\pgfsys@useobject{currentmarker}{}%
\end{pgfscope}%
\begin{pgfscope}%
\pgfsys@transformshift{4.404257in}{4.428488in}%
\pgfsys@useobject{currentmarker}{}%
\end{pgfscope}%
\begin{pgfscope}%
\pgfsys@transformshift{4.423270in}{4.458587in}%
\pgfsys@useobject{currentmarker}{}%
\end{pgfscope}%
\begin{pgfscope}%
\pgfsys@transformshift{4.441580in}{4.413717in}%
\pgfsys@useobject{currentmarker}{}%
\end{pgfscope}%
\begin{pgfscope}%
\pgfsys@transformshift{4.460592in}{4.275848in}%
\pgfsys@useobject{currentmarker}{}%
\end{pgfscope}%
\begin{pgfscope}%
\pgfsys@transformshift{4.480779in}{4.136335in}%
\pgfsys@useobject{currentmarker}{}%
\end{pgfscope}%
\begin{pgfscope}%
\pgfsys@transformshift{4.481953in}{4.133271in}%
\pgfsys@useobject{currentmarker}{}%
\end{pgfscope}%
\begin{pgfscope}%
\pgfsys@transformshift{4.476085in}{4.182625in}%
\pgfsys@useobject{currentmarker}{}%
\end{pgfscope}%
\begin{pgfscope}%
\pgfsys@transformshift{4.454723in}{4.395878in}%
\pgfsys@useobject{currentmarker}{}%
\end{pgfscope}%
\begin{pgfscope}%
\pgfsys@transformshift{4.431721in}{4.454281in}%
\pgfsys@useobject{currentmarker}{}%
\end{pgfscope}%
\begin{pgfscope}%
\pgfsys@transformshift{4.416464in}{4.323081in}%
\pgfsys@useobject{currentmarker}{}%
\end{pgfscope}%
\begin{pgfscope}%
\pgfsys@transformshift{4.397920in}{4.109079in}%
\pgfsys@useobject{currentmarker}{}%
\end{pgfscope}%
\begin{pgfscope}%
\pgfsys@transformshift{4.378438in}{4.035735in}%
\pgfsys@useobject{currentmarker}{}%
\end{pgfscope}%
\begin{pgfscope}%
\pgfsys@transformshift{4.359190in}{4.037526in}%
\pgfsys@useobject{currentmarker}{}%
\end{pgfscope}%
\begin{pgfscope}%
\pgfsys@transformshift{4.339237in}{4.057043in}%
\pgfsys@useobject{currentmarker}{}%
\end{pgfscope}%
\begin{pgfscope}%
\pgfsys@transformshift{4.321632in}{4.133850in}%
\pgfsys@useobject{currentmarker}{}%
\end{pgfscope}%
\begin{pgfscope}%
\pgfsys@transformshift{4.300037in}{4.356596in}%
\pgfsys@useobject{currentmarker}{}%
\end{pgfscope}%
\begin{pgfscope}%
\pgfsys@transformshift{4.280320in}{4.450066in}%
\pgfsys@useobject{currentmarker}{}%
\end{pgfscope}%
\begin{pgfscope}%
\pgfsys@transformshift{4.263420in}{4.369595in}%
\pgfsys@useobject{currentmarker}{}%
\end{pgfscope}%
\begin{pgfscope}%
\pgfsys@transformshift{4.245816in}{4.144112in}%
\pgfsys@useobject{currentmarker}{}%
\end{pgfscope}%
\begin{pgfscope}%
\pgfsys@transformshift{4.226097in}{4.054394in}%
\pgfsys@useobject{currentmarker}{}%
\end{pgfscope}%
\begin{pgfscope}%
\pgfsys@transformshift{4.201686in}{4.036271in}%
\pgfsys@useobject{currentmarker}{}%
\end{pgfscope}%
\begin{pgfscope}%
\pgfsys@transformshift{4.185725in}{4.035750in}%
\pgfsys@useobject{currentmarker}{}%
\end{pgfscope}%
\begin{pgfscope}%
\pgfsys@transformshift{4.167180in}{4.044303in}%
\pgfsys@useobject{currentmarker}{}%
\end{pgfscope}%
\begin{pgfscope}%
\pgfsys@transformshift{4.145821in}{4.096559in}%
\pgfsys@useobject{currentmarker}{}%
\end{pgfscope}%
\begin{pgfscope}%
\pgfsys@transformshift{4.128685in}{4.263465in}%
\pgfsys@useobject{currentmarker}{}%
\end{pgfscope}%
\begin{pgfscope}%
\pgfsys@transformshift{4.105917in}{4.437235in}%
\pgfsys@useobject{currentmarker}{}%
\end{pgfscope}%
\begin{pgfscope}%
\pgfsys@transformshift{4.088781in}{4.410455in}%
\pgfsys@useobject{currentmarker}{}%
\end{pgfscope}%
\begin{pgfscope}%
\pgfsys@transformshift{4.071645in}{4.199463in}%
\pgfsys@useobject{currentmarker}{}%
\end{pgfscope}%
\begin{pgfscope}%
\pgfsys@transformshift{4.050989in}{4.065551in}%
\pgfsys@useobject{currentmarker}{}%
\end{pgfscope}%
\begin{pgfscope}%
\pgfsys@transformshift{4.034558in}{4.041790in}%
\pgfsys@useobject{currentmarker}{}%
\end{pgfscope}%
\begin{pgfscope}%
\pgfsys@transformshift{4.012728in}{4.034915in}%
\pgfsys@useobject{currentmarker}{}%
\end{pgfscope}%
\begin{pgfscope}%
\pgfsys@transformshift{3.994889in}{4.037795in}%
\pgfsys@useobject{currentmarker}{}%
\end{pgfscope}%
\begin{pgfscope}%
\pgfsys@transformshift{3.974467in}{4.057330in}%
\pgfsys@useobject{currentmarker}{}%
\end{pgfscope}%
\begin{pgfscope}%
\pgfsys@transformshift{3.953108in}{4.163932in}%
\pgfsys@useobject{currentmarker}{}%
\end{pgfscope}%
\begin{pgfscope}%
\pgfsys@transformshift{3.935268in}{4.340910in}%
\pgfsys@useobject{currentmarker}{}%
\end{pgfscope}%
\begin{pgfscope}%
\pgfsys@transformshift{3.919072in}{4.412398in}%
\pgfsys@useobject{currentmarker}{}%
\end{pgfscope}%
\begin{pgfscope}%
\pgfsys@transformshift{3.895599in}{4.406837in}%
\pgfsys@useobject{currentmarker}{}%
\end{pgfscope}%
\begin{pgfscope}%
\pgfsys@transformshift{3.881045in}{4.200908in}%
\pgfsys@useobject{currentmarker}{}%
\end{pgfscope}%
\begin{pgfscope}%
\pgfsys@transformshift{3.860858in}{4.076836in}%
\pgfsys@useobject{currentmarker}{}%
\end{pgfscope}%
\begin{pgfscope}%
\pgfsys@transformshift{3.839499in}{4.043562in}%
\pgfsys@useobject{currentmarker}{}%
\end{pgfscope}%
\begin{pgfscope}%
\pgfsys@transformshift{3.820017in}{4.035567in}%
\pgfsys@useobject{currentmarker}{}%
\end{pgfscope}%
\begin{pgfscope}%
\pgfsys@transformshift{3.803584in}{4.035446in}%
\pgfsys@useobject{currentmarker}{}%
\end{pgfscope}%
\begin{pgfscope}%
\pgfsys@transformshift{3.782225in}{4.046182in}%
\pgfsys@useobject{currentmarker}{}%
\end{pgfscope}%
\begin{pgfscope}%
\pgfsys@transformshift{3.761568in}{4.086456in}%
\pgfsys@useobject{currentmarker}{}%
\end{pgfscope}%
\begin{pgfscope}%
\pgfsys@transformshift{3.743964in}{4.231263in}%
\pgfsys@useobject{currentmarker}{}%
\end{pgfscope}%
\begin{pgfscope}%
\pgfsys@transformshift{3.724482in}{4.395476in}%
\pgfsys@useobject{currentmarker}{}%
\end{pgfscope}%
\begin{pgfscope}%
\pgfsys@transformshift{3.705468in}{4.423396in}%
\pgfsys@useobject{currentmarker}{}%
\end{pgfscope}%
\begin{pgfscope}%
\pgfsys@transformshift{3.681056in}{4.189345in}%
\pgfsys@useobject{currentmarker}{}%
\end{pgfscope}%
\begin{pgfscope}%
\pgfsys@transformshift{3.665799in}{4.083076in}%
\pgfsys@useobject{currentmarker}{}%
\end{pgfscope}%
\begin{pgfscope}%
\pgfsys@transformshift{3.646082in}{4.057907in}%
\pgfsys@useobject{currentmarker}{}%
\end{pgfscope}%
\begin{pgfscope}%
\pgfsys@transformshift{3.626598in}{4.039526in}%
\pgfsys@useobject{currentmarker}{}%
\end{pgfscope}%
\begin{pgfscope}%
\pgfsys@transformshift{3.609230in}{4.034842in}%
\pgfsys@useobject{currentmarker}{}%
\end{pgfscope}%
\begin{pgfscope}%
\pgfsys@transformshift{3.587634in}{4.035948in}%
\pgfsys@useobject{currentmarker}{}%
\end{pgfscope}%
\begin{pgfscope}%
\pgfsys@transformshift{3.569795in}{4.040915in}%
\pgfsys@useobject{currentmarker}{}%
\end{pgfscope}%
\begin{pgfscope}%
\pgfsys@transformshift{3.551721in}{4.065783in}%
\pgfsys@useobject{currentmarker}{}%
\end{pgfscope}%
\begin{pgfscope}%
\pgfsys@transformshift{3.530360in}{4.191034in}%
\pgfsys@useobject{currentmarker}{}%
\end{pgfscope}%
\begin{pgfscope}%
\pgfsys@transformshift{3.513224in}{4.350554in}%
\pgfsys@useobject{currentmarker}{}%
\end{pgfscope}%
\begin{pgfscope}%
\pgfsys@transformshift{3.494916in}{4.421905in}%
\pgfsys@useobject{currentmarker}{}%
\end{pgfscope}%
\begin{pgfscope}%
\pgfsys@transformshift{3.474494in}{4.378238in}%
\pgfsys@useobject{currentmarker}{}%
\end{pgfscope}%
\begin{pgfscope}%
\pgfsys@transformshift{3.455952in}{4.169333in}%
\pgfsys@useobject{currentmarker}{}%
\end{pgfscope}%
\begin{pgfscope}%
\pgfsys@transformshift{3.432479in}{4.075108in}%
\pgfsys@useobject{currentmarker}{}%
\end{pgfscope}%
\begin{pgfscope}%
\pgfsys@transformshift{3.416517in}{4.048374in}%
\pgfsys@useobject{currentmarker}{}%
\end{pgfscope}%
\begin{pgfscope}%
\pgfsys@transformshift{3.399850in}{4.037225in}%
\pgfsys@useobject{currentmarker}{}%
\end{pgfscope}%
\begin{pgfscope}%
\pgfsys@transformshift{3.380367in}{4.034549in}%
\pgfsys@useobject{currentmarker}{}%
\end{pgfscope}%
\begin{pgfscope}%
\pgfsys@transformshift{3.360651in}{4.037186in}%
\pgfsys@useobject{currentmarker}{}%
\end{pgfscope}%
\begin{pgfscope}%
\pgfsys@transformshift{3.339995in}{4.048551in}%
\pgfsys@useobject{currentmarker}{}%
\end{pgfscope}%
\begin{pgfscope}%
\pgfsys@transformshift{3.317930in}{4.099601in}%
\pgfsys@useobject{currentmarker}{}%
\end{pgfscope}%
\begin{pgfscope}%
\pgfsys@transformshift{3.302437in}{4.188048in}%
\pgfsys@useobject{currentmarker}{}%
\end{pgfscope}%
\begin{pgfscope}%
\pgfsys@transformshift{3.283895in}{4.331321in}%
\pgfsys@useobject{currentmarker}{}%
\end{pgfscope}%
\begin{pgfscope}%
\pgfsys@transformshift{3.261125in}{4.413525in}%
\pgfsys@useobject{currentmarker}{}%
\end{pgfscope}%
\begin{pgfscope}%
\pgfsys@transformshift{3.247511in}{4.322316in}%
\pgfsys@useobject{currentmarker}{}%
\end{pgfscope}%
\begin{pgfscope}%
\pgfsys@transformshift{3.223335in}{4.125175in}%
\pgfsys@useobject{currentmarker}{}%
\end{pgfscope}%
\begin{pgfscope}%
\pgfsys@transformshift{3.207138in}{4.063540in}%
\pgfsys@useobject{currentmarker}{}%
\end{pgfscope}%
\begin{pgfscope}%
\pgfsys@transformshift{3.181317in}{4.038507in}%
\pgfsys@useobject{currentmarker}{}%
\end{pgfscope}%
\begin{pgfscope}%
\pgfsys@transformshift{3.166060in}{4.035190in}%
\pgfsys@useobject{currentmarker}{}%
\end{pgfscope}%
\begin{pgfscope}%
\pgfsys@transformshift{3.145873in}{4.034383in}%
\pgfsys@useobject{currentmarker}{}%
\end{pgfscope}%
\begin{pgfscope}%
\pgfsys@transformshift{3.126860in}{4.035726in}%
\pgfsys@useobject{currentmarker}{}%
\end{pgfscope}%
\begin{pgfscope}%
\pgfsys@transformshift{3.107612in}{4.044579in}%
\pgfsys@useobject{currentmarker}{}%
\end{pgfscope}%
\begin{pgfscope}%
\pgfsys@transformshift{3.092121in}{4.068703in}%
\pgfsys@useobject{currentmarker}{}%
\end{pgfscope}%
\begin{pgfscope}%
\pgfsys@transformshift{3.072403in}{4.136994in}%
\pgfsys@useobject{currentmarker}{}%
\end{pgfscope}%
\begin{pgfscope}%
\pgfsys@transformshift{3.054095in}{4.247043in}%
\pgfsys@useobject{currentmarker}{}%
\end{pgfscope}%
\begin{pgfscope}%
\pgfsys@transformshift{3.035081in}{4.382700in}%
\pgfsys@useobject{currentmarker}{}%
\end{pgfscope}%
\begin{pgfscope}%
\pgfsys@transformshift{3.013720in}{4.388385in}%
\pgfsys@useobject{currentmarker}{}%
\end{pgfscope}%
\begin{pgfscope}%
\pgfsys@transformshift{2.994238in}{4.221850in}%
\pgfsys@useobject{currentmarker}{}%
\end{pgfscope}%
\begin{pgfscope}%
\pgfsys@transformshift{2.975225in}{4.089828in}%
\pgfsys@useobject{currentmarker}{}%
\end{pgfscope}%
\begin{pgfscope}%
\pgfsys@transformshift{2.956917in}{4.050511in}%
\pgfsys@useobject{currentmarker}{}%
\end{pgfscope}%
\begin{pgfscope}%
\pgfsys@transformshift{2.935555in}{4.038743in}%
\pgfsys@useobject{currentmarker}{}%
\end{pgfscope}%
\begin{pgfscope}%
\pgfsys@transformshift{2.916778in}{4.034543in}%
\pgfsys@useobject{currentmarker}{}%
\end{pgfscope}%
\begin{pgfscope}%
\pgfsys@transformshift{2.899877in}{4.035444in}%
\pgfsys@useobject{currentmarker}{}%
\end{pgfscope}%
\begin{pgfscope}%
\pgfsys@transformshift{2.877812in}{4.045739in}%
\pgfsys@useobject{currentmarker}{}%
\end{pgfscope}%
\begin{pgfscope}%
\pgfsys@transformshift{2.854574in}{4.083063in}%
\pgfsys@useobject{currentmarker}{}%
\end{pgfscope}%
\begin{pgfscope}%
\pgfsys@transformshift{2.840256in}{4.150460in}%
\pgfsys@useobject{currentmarker}{}%
\end{pgfscope}%
\begin{pgfscope}%
\pgfsys@transformshift{2.823355in}{4.265121in}%
\pgfsys@useobject{currentmarker}{}%
\end{pgfscope}%
\begin{pgfscope}%
\pgfsys@transformshift{2.801759in}{4.361904in}%
\pgfsys@useobject{currentmarker}{}%
\end{pgfscope}%
\begin{pgfscope}%
\pgfsys@transformshift{2.783217in}{4.407348in}%
\pgfsys@useobject{currentmarker}{}%
\end{pgfscope}%
\begin{pgfscope}%
\pgfsys@transformshift{2.764438in}{4.396695in}%
\pgfsys@useobject{currentmarker}{}%
\end{pgfscope}%
\begin{pgfscope}%
\pgfsys@transformshift{2.744956in}{4.200275in}%
\pgfsys@useobject{currentmarker}{}%
\end{pgfscope}%
\begin{pgfscope}%
\pgfsys@transformshift{2.721717in}{4.071662in}%
\pgfsys@useobject{currentmarker}{}%
\end{pgfscope}%
\begin{pgfscope}%
\pgfsys@transformshift{2.706226in}{4.049117in}%
\pgfsys@useobject{currentmarker}{}%
\end{pgfscope}%
\begin{pgfscope}%
\pgfsys@transformshift{2.686507in}{4.037233in}%
\pgfsys@useobject{currentmarker}{}%
\end{pgfscope}%
\begin{pgfscope}%
\pgfsys@transformshift{2.667494in}{4.034398in}%
\pgfsys@useobject{currentmarker}{}%
\end{pgfscope}%
\begin{pgfscope}%
\pgfsys@transformshift{2.646369in}{4.036907in}%
\pgfsys@useobject{currentmarker}{}%
\end{pgfscope}%
\begin{pgfscope}%
\pgfsys@transformshift{2.630173in}{4.043917in}%
\pgfsys@useobject{currentmarker}{}%
\end{pgfscope}%
\begin{pgfscope}%
\pgfsys@transformshift{2.612568in}{4.073057in}%
\pgfsys@useobject{currentmarker}{}%
\end{pgfscope}%
\begin{pgfscope}%
\pgfsys@transformshift{2.589800in}{4.206346in}%
\pgfsys@useobject{currentmarker}{}%
\end{pgfscope}%
\begin{pgfscope}%
\pgfsys@transformshift{2.572195in}{4.345582in}%
\pgfsys@useobject{currentmarker}{}%
\end{pgfscope}%
\begin{pgfscope}%
\pgfsys@transformshift{2.552243in}{4.404401in}%
\pgfsys@useobject{currentmarker}{}%
\end{pgfscope}%
\begin{pgfscope}%
\pgfsys@transformshift{2.533229in}{4.405989in}%
\pgfsys@useobject{currentmarker}{}%
\end{pgfscope}%
\begin{pgfscope}%
\pgfsys@transformshift{2.515156in}{4.281373in}%
\pgfsys@useobject{currentmarker}{}%
\end{pgfscope}%
\begin{pgfscope}%
\pgfsys@transformshift{2.496377in}{4.115107in}%
\pgfsys@useobject{currentmarker}{}%
\end{pgfscope}%
\begin{pgfscope}%
\pgfsys@transformshift{2.475252in}{4.052902in}%
\pgfsys@useobject{currentmarker}{}%
\end{pgfscope}%
\begin{pgfscope}%
\pgfsys@transformshift{2.456004in}{4.039141in}%
\pgfsys@useobject{currentmarker}{}%
\end{pgfscope}%
\begin{pgfscope}%
\pgfsys@transformshift{2.439103in}{4.034467in}%
\pgfsys@useobject{currentmarker}{}%
\end{pgfscope}%
\begin{pgfscope}%
\pgfsys@transformshift{2.417272in}{4.035074in}%
\pgfsys@useobject{currentmarker}{}%
\end{pgfscope}%
\begin{pgfscope}%
\pgfsys@transformshift{2.398496in}{4.041021in}%
\pgfsys@useobject{currentmarker}{}%
\end{pgfscope}%
\begin{pgfscope}%
\pgfsys@transformshift{2.380186in}{4.055665in}%
\pgfsys@useobject{currentmarker}{}%
\end{pgfscope}%
\begin{pgfscope}%
\pgfsys@transformshift{2.359529in}{4.111091in}%
\pgfsys@useobject{currentmarker}{}%
\end{pgfscope}%
\begin{pgfscope}%
\pgfsys@transformshift{2.343333in}{4.210485in}%
\pgfsys@useobject{currentmarker}{}%
\end{pgfscope}%
\begin{pgfscope}%
\pgfsys@transformshift{2.321503in}{4.377136in}%
\pgfsys@useobject{currentmarker}{}%
\end{pgfscope}%
\begin{pgfscope}%
\pgfsys@transformshift{2.304838in}{4.405188in}%
\pgfsys@useobject{currentmarker}{}%
\end{pgfscope}%
\begin{pgfscope}%
\pgfsys@transformshift{2.283478in}{4.300422in}%
\pgfsys@useobject{currentmarker}{}%
\end{pgfscope}%
\begin{pgfscope}%
\pgfsys@transformshift{2.263291in}{4.222161in}%
\pgfsys@useobject{currentmarker}{}%
\end{pgfscope}%
\begin{pgfscope}%
\pgfsys@transformshift{2.245217in}{4.123736in}%
\pgfsys@useobject{currentmarker}{}%
\end{pgfscope}%
\begin{pgfscope}%
\pgfsys@transformshift{2.227142in}{4.060496in}%
\pgfsys@useobject{currentmarker}{}%
\end{pgfscope}%
\begin{pgfscope}%
\pgfsys@transformshift{2.206251in}{4.040809in}%
\pgfsys@useobject{currentmarker}{}%
\end{pgfscope}%
\begin{pgfscope}%
\pgfsys@transformshift{2.188178in}{4.035276in}%
\pgfsys@useobject{currentmarker}{}%
\end{pgfscope}%
\begin{pgfscope}%
\pgfsys@transformshift{2.166816in}{4.034992in}%
\pgfsys@useobject{currentmarker}{}%
\end{pgfscope}%
\begin{pgfscope}%
\pgfsys@transformshift{2.148039in}{4.038786in}%
\pgfsys@useobject{currentmarker}{}%
\end{pgfscope}%
\begin{pgfscope}%
\pgfsys@transformshift{2.129026in}{4.049113in}%
\pgfsys@useobject{currentmarker}{}%
\end{pgfscope}%
\begin{pgfscope}%
\pgfsys@transformshift{2.109308in}{4.082636in}%
\pgfsys@useobject{currentmarker}{}%
\end{pgfscope}%
\begin{pgfscope}%
\pgfsys@transformshift{2.093111in}{4.095149in}%
\pgfsys@useobject{currentmarker}{}%
\end{pgfscope}%
\begin{pgfscope}%
\pgfsys@transformshift{2.071047in}{4.247850in}%
\pgfsys@useobject{currentmarker}{}%
\end{pgfscope}%
\begin{pgfscope}%
\pgfsys@transformshift{2.051799in}{4.385147in}%
\pgfsys@useobject{currentmarker}{}%
\end{pgfscope}%
\begin{pgfscope}%
\pgfsys@transformshift{2.033256in}{4.402238in}%
\pgfsys@useobject{currentmarker}{}%
\end{pgfscope}%
\begin{pgfscope}%
\pgfsys@transformshift{2.014946in}{4.256425in}%
\pgfsys@useobject{currentmarker}{}%
\end{pgfscope}%
\begin{pgfscope}%
\pgfsys@transformshift{1.996404in}{4.109066in}%
\pgfsys@useobject{currentmarker}{}%
\end{pgfscope}%
\begin{pgfscope}%
\pgfsys@transformshift{1.972931in}{4.051643in}%
\pgfsys@useobject{currentmarker}{}%
\end{pgfscope}%
\begin{pgfscope}%
\pgfsys@transformshift{1.956029in}{4.040026in}%
\pgfsys@useobject{currentmarker}{}%
\end{pgfscope}%
\begin{pgfscope}%
\pgfsys@transformshift{1.938661in}{4.414267in}%
\pgfsys@useobject{currentmarker}{}%
\end{pgfscope}%
\begin{pgfscope}%
\pgfsys@transformshift{1.918474in}{4.250377in}%
\pgfsys@useobject{currentmarker}{}%
\end{pgfscope}%
\begin{pgfscope}%
\pgfsys@transformshift{1.899929in}{4.110264in}%
\pgfsys@useobject{currentmarker}{}%
\end{pgfscope}%
\begin{pgfscope}%
\pgfsys@transformshift{1.878335in}{4.050450in}%
\pgfsys@useobject{currentmarker}{}%
\end{pgfscope}%
\begin{pgfscope}%
\pgfsys@transformshift{1.860260in}{4.039775in}%
\pgfsys@useobject{currentmarker}{}%
\end{pgfscope}%
\begin{pgfscope}%
\pgfsys@transformshift{1.840309in}{4.034880in}%
\pgfsys@useobject{currentmarker}{}%
\end{pgfscope}%
\begin{pgfscope}%
\pgfsys@transformshift{1.821061in}{4.034968in}%
\pgfsys@useobject{currentmarker}{}%
\end{pgfscope}%
\begin{pgfscope}%
\pgfsys@transformshift{1.800874in}{4.041763in}%
\pgfsys@useobject{currentmarker}{}%
\end{pgfscope}%
\begin{pgfscope}%
\pgfsys@transformshift{1.785383in}{4.058545in}%
\pgfsys@useobject{currentmarker}{}%
\end{pgfscope}%
\begin{pgfscope}%
\pgfsys@transformshift{1.765899in}{4.147323in}%
\pgfsys@useobject{currentmarker}{}%
\end{pgfscope}%
\begin{pgfscope}%
\pgfsys@transformshift{1.743365in}{4.337764in}%
\pgfsys@useobject{currentmarker}{}%
\end{pgfscope}%
\begin{pgfscope}%
\pgfsys@transformshift{1.726700in}{4.395498in}%
\pgfsys@useobject{currentmarker}{}%
\end{pgfscope}%
\begin{pgfscope}%
\pgfsys@transformshift{1.706278in}{4.411337in}%
\pgfsys@useobject{currentmarker}{}%
\end{pgfscope}%
\begin{pgfscope}%
\pgfsys@transformshift{1.689848in}{4.351832in}%
\pgfsys@useobject{currentmarker}{}%
\end{pgfscope}%
\begin{pgfscope}%
\pgfsys@transformshift{1.668252in}{4.134235in}%
\pgfsys@useobject{currentmarker}{}%
\end{pgfscope}%
\begin{pgfscope}%
\pgfsys@transformshift{1.649707in}{4.064529in}%
\pgfsys@useobject{currentmarker}{}%
\end{pgfscope}%
\begin{pgfscope}%
\pgfsys@transformshift{1.630460in}{4.042359in}%
\pgfsys@useobject{currentmarker}{}%
\end{pgfscope}%
\begin{pgfscope}%
\pgfsys@transformshift{1.612386in}{4.038164in}%
\pgfsys@useobject{currentmarker}{}%
\end{pgfscope}%
\begin{pgfscope}%
\pgfsys@transformshift{1.587505in}{4.034660in}%
\pgfsys@useobject{currentmarker}{}%
\end{pgfscope}%
\begin{pgfscope}%
\pgfsys@transformshift{1.572482in}{4.036396in}%
\pgfsys@useobject{currentmarker}{}%
\end{pgfscope}%
\begin{pgfscope}%
\pgfsys@transformshift{1.553938in}{4.042030in}%
\pgfsys@useobject{currentmarker}{}%
\end{pgfscope}%
\begin{pgfscope}%
\pgfsys@transformshift{1.534456in}{4.059943in}%
\pgfsys@useobject{currentmarker}{}%
\end{pgfscope}%
\begin{pgfscope}%
\pgfsys@transformshift{1.514505in}{4.103935in}%
\pgfsys@useobject{currentmarker}{}%
\end{pgfscope}%
\begin{pgfscope}%
\pgfsys@transformshift{1.495257in}{4.227813in}%
\pgfsys@useobject{currentmarker}{}%
\end{pgfscope}%
\begin{pgfscope}%
\pgfsys@transformshift{1.476713in}{4.382757in}%
\pgfsys@useobject{currentmarker}{}%
\end{pgfscope}%
\begin{pgfscope}%
\pgfsys@transformshift{1.457231in}{4.425191in}%
\pgfsys@useobject{currentmarker}{}%
\end{pgfscope}%
\begin{pgfscope}%
\pgfsys@transformshift{1.439860in}{4.374782in}%
\pgfsys@useobject{currentmarker}{}%
\end{pgfscope}%
\begin{pgfscope}%
\pgfsys@transformshift{1.419204in}{4.164996in}%
\pgfsys@useobject{currentmarker}{}%
\end{pgfscope}%
\begin{pgfscope}%
\pgfsys@transformshift{1.401130in}{4.074971in}%
\pgfsys@useobject{currentmarker}{}%
\end{pgfscope}%
\begin{pgfscope}%
\pgfsys@transformshift{1.378595in}{4.045639in}%
\pgfsys@useobject{currentmarker}{}%
\end{pgfscope}%
\begin{pgfscope}%
\pgfsys@transformshift{1.361461in}{4.038993in}%
\pgfsys@useobject{currentmarker}{}%
\end{pgfscope}%
\begin{pgfscope}%
\pgfsys@transformshift{1.340334in}{4.034935in}%
\pgfsys@useobject{currentmarker}{}%
\end{pgfscope}%
\begin{pgfscope}%
\pgfsys@transformshift{1.326486in}{4.035252in}%
\pgfsys@useobject{currentmarker}{}%
\end{pgfscope}%
\begin{pgfscope}%
\pgfsys@transformshift{1.302778in}{4.040149in}%
\pgfsys@useobject{currentmarker}{}%
\end{pgfscope}%
\begin{pgfscope}%
\pgfsys@transformshift{1.284939in}{4.050466in}%
\pgfsys@useobject{currentmarker}{}%
\end{pgfscope}%
\begin{pgfscope}%
\pgfsys@transformshift{1.264283in}{4.094287in}%
\pgfsys@useobject{currentmarker}{}%
\end{pgfscope}%
\begin{pgfscope}%
\pgfsys@transformshift{1.247616in}{4.187657in}%
\pgfsys@useobject{currentmarker}{}%
\end{pgfscope}%
\begin{pgfscope}%
\pgfsys@transformshift{1.226022in}{4.357407in}%
\pgfsys@useobject{currentmarker}{}%
\end{pgfscope}%
\begin{pgfscope}%
\pgfsys@transformshift{1.208886in}{4.423951in}%
\pgfsys@useobject{currentmarker}{}%
\end{pgfscope}%
\begin{pgfscope}%
\pgfsys@transformshift{1.189170in}{4.433096in}%
\pgfsys@useobject{currentmarker}{}%
\end{pgfscope}%
\begin{pgfscope}%
\pgfsys@transformshift{1.170625in}{4.431552in}%
\pgfsys@useobject{currentmarker}{}%
\end{pgfscope}%
\begin{pgfscope}%
\pgfsys@transformshift{1.152083in}{4.302508in}%
\pgfsys@useobject{currentmarker}{}%
\end{pgfscope}%
\begin{pgfscope}%
\pgfsys@transformshift{1.130721in}{4.123541in}%
\pgfsys@useobject{currentmarker}{}%
\end{pgfscope}%
\begin{pgfscope}%
\pgfsys@transformshift{1.111474in}{4.064397in}%
\pgfsys@useobject{currentmarker}{}%
\end{pgfscope}%
\begin{pgfscope}%
\pgfsys@transformshift{1.095043in}{4.196836in}%
\pgfsys@useobject{currentmarker}{}%
\end{pgfscope}%
\begin{pgfscope}%
\pgfsys@transformshift{1.071570in}{4.073682in}%
\pgfsys@useobject{currentmarker}{}%
\end{pgfscope}%
\begin{pgfscope}%
\pgfsys@transformshift{1.050914in}{4.047273in}%
\pgfsys@useobject{currentmarker}{}%
\end{pgfscope}%
\begin{pgfscope}%
\pgfsys@transformshift{1.034249in}{4.038703in}%
\pgfsys@useobject{currentmarker}{}%
\end{pgfscope}%
\begin{pgfscope}%
\pgfsys@transformshift{1.015704in}{4.034996in}%
\pgfsys@useobject{currentmarker}{}%
\end{pgfscope}%
\begin{pgfscope}%
\pgfsys@transformshift{0.998099in}{4.036810in}%
\pgfsys@useobject{currentmarker}{}%
\end{pgfscope}%
\begin{pgfscope}%
\pgfsys@transformshift{0.975800in}{4.047466in}%
\pgfsys@useobject{currentmarker}{}%
\end{pgfscope}%
\begin{pgfscope}%
\pgfsys@transformshift{0.959135in}{4.069010in}%
\pgfsys@useobject{currentmarker}{}%
\end{pgfscope}%
\begin{pgfscope}%
\pgfsys@transformshift{0.940825in}{4.146183in}%
\pgfsys@useobject{currentmarker}{}%
\end{pgfscope}%
\begin{pgfscope}%
\pgfsys@transformshift{0.917823in}{4.260056in}%
\pgfsys@useobject{currentmarker}{}%
\end{pgfscope}%
\begin{pgfscope}%
\pgfsys@transformshift{0.898810in}{4.395605in}%
\pgfsys@useobject{currentmarker}{}%
\end{pgfscope}%
\begin{pgfscope}%
\pgfsys@transformshift{0.880265in}{4.444407in}%
\pgfsys@useobject{currentmarker}{}%
\end{pgfscope}%
\begin{pgfscope}%
\pgfsys@transformshift{0.861017in}{4.417588in}%
\pgfsys@useobject{currentmarker}{}%
\end{pgfscope}%
\begin{pgfscope}%
\pgfsys@transformshift{0.843647in}{4.225533in}%
\pgfsys@useobject{currentmarker}{}%
\end{pgfscope}%
\begin{pgfscope}%
\pgfsys@transformshift{0.821583in}{4.103480in}%
\pgfsys@useobject{currentmarker}{}%
\end{pgfscope}%
\begin{pgfscope}%
\pgfsys@transformshift{0.803978in}{4.060944in}%
\pgfsys@useobject{currentmarker}{}%
\end{pgfscope}%
\begin{pgfscope}%
\pgfsys@transformshift{0.788018in}{4.046466in}%
\pgfsys@useobject{currentmarker}{}%
\end{pgfscope}%
\begin{pgfscope}%
\pgfsys@transformshift{0.766188in}{4.037370in}%
\pgfsys@useobject{currentmarker}{}%
\end{pgfscope}%
\begin{pgfscope}%
\pgfsys@transformshift{0.746704in}{4.035833in}%
\pgfsys@useobject{currentmarker}{}%
\end{pgfscope}%
\begin{pgfscope}%
\pgfsys@transformshift{0.729335in}{4.040143in}%
\pgfsys@useobject{currentmarker}{}%
\end{pgfscope}%
\begin{pgfscope}%
\pgfsys@transformshift{0.707739in}{4.047229in}%
\pgfsys@useobject{currentmarker}{}%
\end{pgfscope}%
\begin{pgfscope}%
\pgfsys@transformshift{0.689900in}{4.074033in}%
\pgfsys@useobject{currentmarker}{}%
\end{pgfscope}%
\begin{pgfscope}%
\pgfsys@transformshift{0.669478in}{4.155765in}%
\pgfsys@useobject{currentmarker}{}%
\end{pgfscope}%
\begin{pgfscope}%
\pgfsys@transformshift{0.649291in}{4.040588in}%
\pgfsys@useobject{currentmarker}{}%
\end{pgfscope}%
\begin{pgfscope}%
\pgfsys@transformshift{0.649996in}{4.041196in}%
\pgfsys@useobject{currentmarker}{}%
\end{pgfscope}%
\begin{pgfscope}%
\pgfsys@transformshift{0.656333in}{4.047547in}%
\pgfsys@useobject{currentmarker}{}%
\end{pgfscope}%
\begin{pgfscope}%
\pgfsys@transformshift{0.675815in}{4.090469in}%
\pgfsys@useobject{currentmarker}{}%
\end{pgfscope}%
\begin{pgfscope}%
\pgfsys@transformshift{0.695300in}{4.240859in}%
\pgfsys@useobject{currentmarker}{}%
\end{pgfscope}%
\begin{pgfscope}%
\pgfsys@transformshift{0.716190in}{4.449827in}%
\pgfsys@useobject{currentmarker}{}%
\end{pgfscope}%
\begin{pgfscope}%
\pgfsys@transformshift{0.734498in}{4.395662in}%
\pgfsys@useobject{currentmarker}{}%
\end{pgfscope}%
\begin{pgfscope}%
\pgfsys@transformshift{0.753043in}{4.189325in}%
\pgfsys@useobject{currentmarker}{}%
\end{pgfscope}%
\begin{pgfscope}%
\pgfsys@transformshift{0.774636in}{4.061643in}%
\pgfsys@useobject{currentmarker}{}%
\end{pgfscope}%
\begin{pgfscope}%
\pgfsys@transformshift{0.790129in}{4.040998in}%
\pgfsys@useobject{currentmarker}{}%
\end{pgfscope}%
\begin{pgfscope}%
\pgfsys@transformshift{0.810786in}{4.035131in}%
\pgfsys@useobject{currentmarker}{}%
\end{pgfscope}%
\begin{pgfscope}%
\pgfsys@transformshift{0.829799in}{4.041287in}%
\pgfsys@useobject{currentmarker}{}%
\end{pgfscope}%
\begin{pgfscope}%
\pgfsys@transformshift{0.849047in}{4.067242in}%
\pgfsys@useobject{currentmarker}{}%
\end{pgfscope}%
\begin{pgfscope}%
\pgfsys@transformshift{0.867825in}{4.177912in}%
\pgfsys@useobject{currentmarker}{}%
\end{pgfscope}%
\begin{pgfscope}%
\pgfsys@transformshift{0.886602in}{4.417671in}%
\pgfsys@useobject{currentmarker}{}%
\end{pgfscope}%
\begin{pgfscope}%
\pgfsys@transformshift{0.909607in}{4.407853in}%
\pgfsys@useobject{currentmarker}{}%
\end{pgfscope}%
\begin{pgfscope}%
\pgfsys@transformshift{0.925334in}{4.250312in}%
\pgfsys@useobject{currentmarker}{}%
\end{pgfscope}%
\begin{pgfscope}%
\pgfsys@transformshift{0.942939in}{4.092859in}%
\pgfsys@useobject{currentmarker}{}%
\end{pgfscope}%
\begin{pgfscope}%
\pgfsys@transformshift{0.964767in}{4.043344in}%
\pgfsys@useobject{currentmarker}{}%
\end{pgfscope}%
\begin{pgfscope}%
\pgfsys@transformshift{0.985894in}{4.034627in}%
\pgfsys@useobject{currentmarker}{}%
\end{pgfscope}%
\begin{pgfscope}%
\pgfsys@transformshift{1.003499in}{4.036998in}%
\pgfsys@useobject{currentmarker}{}%
\end{pgfscope}%
\begin{pgfscope}%
\pgfsys@transformshift{1.022041in}{4.052959in}%
\pgfsys@useobject{currentmarker}{}%
\end{pgfscope}%
\begin{pgfscope}%
\pgfsys@transformshift{1.041994in}{4.098619in}%
\pgfsys@useobject{currentmarker}{}%
\end{pgfscope}%
\begin{pgfscope}%
\pgfsys@transformshift{1.061945in}{4.292742in}%
\pgfsys@useobject{currentmarker}{}%
\end{pgfscope}%
\begin{pgfscope}%
\pgfsys@transformshift{1.076969in}{4.432136in}%
\pgfsys@useobject{currentmarker}{}%
\end{pgfscope}%
\begin{pgfscope}%
\pgfsys@transformshift{1.095983in}{4.373342in}%
\pgfsys@useobject{currentmarker}{}%
\end{pgfscope}%
\begin{pgfscope}%
\pgfsys@transformshift{1.118516in}{4.135878in}%
\pgfsys@useobject{currentmarker}{}%
\end{pgfscope}%
\begin{pgfscope}%
\pgfsys@transformshift{1.137295in}{4.055911in}%
\pgfsys@useobject{currentmarker}{}%
\end{pgfscope}%
\begin{pgfscope}%
\pgfsys@transformshift{1.156543in}{4.037764in}%
\pgfsys@useobject{currentmarker}{}%
\end{pgfscope}%
\begin{pgfscope}%
\pgfsys@transformshift{1.174851in}{4.034361in}%
\pgfsys@useobject{currentmarker}{}%
\end{pgfscope}%
\begin{pgfscope}%
\pgfsys@transformshift{1.193864in}{4.038045in}%
\pgfsys@useobject{currentmarker}{}%
\end{pgfscope}%
\begin{pgfscope}%
\pgfsys@transformshift{1.212408in}{4.053991in}%
\pgfsys@useobject{currentmarker}{}%
\end{pgfscope}%
\begin{pgfscope}%
\pgfsys@transformshift{1.234237in}{4.140868in}%
\pgfsys@useobject{currentmarker}{}%
\end{pgfscope}%
\begin{pgfscope}%
\pgfsys@transformshift{1.250669in}{4.328631in}%
\pgfsys@useobject{currentmarker}{}%
\end{pgfscope}%
\begin{pgfscope}%
\pgfsys@transformshift{1.272263in}{4.428163in}%
\pgfsys@useobject{currentmarker}{}%
\end{pgfscope}%
\begin{pgfscope}%
\pgfsys@transformshift{1.291042in}{4.371359in}%
\pgfsys@useobject{currentmarker}{}%
\end{pgfscope}%
\begin{pgfscope}%
\pgfsys@transformshift{1.312167in}{4.160976in}%
\pgfsys@useobject{currentmarker}{}%
\end{pgfscope}%
\begin{pgfscope}%
\pgfsys@transformshift{1.328834in}{4.066339in}%
\pgfsys@useobject{currentmarker}{}%
\end{pgfscope}%
\begin{pgfscope}%
\pgfsys@transformshift{1.348082in}{4.043118in}%
\pgfsys@useobject{currentmarker}{}%
\end{pgfscope}%
\begin{pgfscope}%
\pgfsys@transformshift{1.369441in}{4.035111in}%
\pgfsys@useobject{currentmarker}{}%
\end{pgfscope}%
\begin{pgfscope}%
\pgfsys@transformshift{1.387751in}{4.034752in}%
\pgfsys@useobject{currentmarker}{}%
\end{pgfscope}%
\begin{pgfscope}%
\pgfsys@transformshift{1.404651in}{4.039504in}%
\pgfsys@useobject{currentmarker}{}%
\end{pgfscope}%
\begin{pgfscope}%
\pgfsys@transformshift{1.425072in}{4.066894in}%
\pgfsys@useobject{currentmarker}{}%
\end{pgfscope}%
\begin{pgfscope}%
\pgfsys@transformshift{1.446903in}{4.191234in}%
\pgfsys@useobject{currentmarker}{}%
\end{pgfscope}%
\begin{pgfscope}%
\pgfsys@transformshift{1.462159in}{4.342838in}%
\pgfsys@useobject{currentmarker}{}%
\end{pgfscope}%
\begin{pgfscope}%
\pgfsys@transformshift{1.480704in}{4.414742in}%
\pgfsys@useobject{currentmarker}{}%
\end{pgfscope}%
\begin{pgfscope}%
\pgfsys@transformshift{1.501594in}{4.413588in}%
\pgfsys@useobject{currentmarker}{}%
\end{pgfscope}%
\begin{pgfscope}%
\pgfsys@transformshift{1.521782in}{4.246718in}%
\pgfsys@useobject{currentmarker}{}%
\end{pgfscope}%
\begin{pgfscope}%
\pgfsys@transformshift{1.540090in}{4.098400in}%
\pgfsys@useobject{currentmarker}{}%
\end{pgfscope}%
\begin{pgfscope}%
\pgfsys@transformshift{1.560980in}{4.047612in}%
\pgfsys@useobject{currentmarker}{}%
\end{pgfscope}%
\begin{pgfscope}%
\pgfsys@transformshift{1.579525in}{4.039139in}%
\pgfsys@useobject{currentmarker}{}%
\end{pgfscope}%
\begin{pgfscope}%
\pgfsys@transformshift{1.596424in}{4.034457in}%
\pgfsys@useobject{currentmarker}{}%
\end{pgfscope}%
\begin{pgfscope}%
\pgfsys@transformshift{1.618489in}{4.035161in}%
\pgfsys@useobject{currentmarker}{}%
\end{pgfscope}%
\begin{pgfscope}%
\pgfsys@transformshift{1.635390in}{4.042475in}%
\pgfsys@useobject{currentmarker}{}%
\end{pgfscope}%
\begin{pgfscope}%
\pgfsys@transformshift{1.656281in}{4.060226in}%
\pgfsys@useobject{currentmarker}{}%
\end{pgfscope}%
\begin{pgfscope}%
\pgfsys@transformshift{1.674823in}{4.133180in}%
\pgfsys@useobject{currentmarker}{}%
\end{pgfscope}%
\begin{pgfscope}%
\pgfsys@transformshift{1.695481in}{4.274440in}%
\pgfsys@useobject{currentmarker}{}%
\end{pgfscope}%
\begin{pgfscope}%
\pgfsys@transformshift{1.714258in}{4.415347in}%
\pgfsys@useobject{currentmarker}{}%
\end{pgfscope}%
\begin{pgfscope}%
\pgfsys@transformshift{1.732803in}{4.352370in}%
\pgfsys@useobject{currentmarker}{}%
\end{pgfscope}%
\begin{pgfscope}%
\pgfsys@transformshift{1.751345in}{4.209830in}%
\pgfsys@useobject{currentmarker}{}%
\end{pgfscope}%
\begin{pgfscope}%
\pgfsys@transformshift{1.771767in}{4.077256in}%
\pgfsys@useobject{currentmarker}{}%
\end{pgfscope}%
\begin{pgfscope}%
\pgfsys@transformshift{1.791015in}{4.043252in}%
\pgfsys@useobject{currentmarker}{}%
\end{pgfscope}%
\begin{pgfscope}%
\pgfsys@transformshift{1.811202in}{4.035094in}%
\pgfsys@useobject{currentmarker}{}%
\end{pgfscope}%
\begin{pgfscope}%
\pgfsys@transformshift{1.828338in}{4.033813in}%
\pgfsys@useobject{currentmarker}{}%
\end{pgfscope}%
\begin{pgfscope}%
\pgfsys@transformshift{1.847586in}{4.035617in}%
\pgfsys@useobject{currentmarker}{}%
\end{pgfscope}%
\begin{pgfscope}%
\pgfsys@transformshift{1.868945in}{4.040979in}%
\pgfsys@useobject{currentmarker}{}%
\end{pgfscope}%
\begin{pgfscope}%
\pgfsys@transformshift{1.887489in}{4.057307in}%
\pgfsys@useobject{currentmarker}{}%
\end{pgfscope}%
\begin{pgfscope}%
\pgfsys@transformshift{1.904389in}{4.117473in}%
\pgfsys@useobject{currentmarker}{}%
\end{pgfscope}%
\begin{pgfscope}%
\pgfsys@transformshift{1.926219in}{4.307144in}%
\pgfsys@useobject{currentmarker}{}%
\end{pgfscope}%
\begin{pgfscope}%
\pgfsys@transformshift{1.943121in}{4.405342in}%
\pgfsys@useobject{currentmarker}{}%
\end{pgfscope}%
\begin{pgfscope}%
\pgfsys@transformshift{1.961663in}{4.041029in}%
\pgfsys@useobject{currentmarker}{}%
\end{pgfscope}%
\begin{pgfscope}%
\pgfsys@transformshift{1.981851in}{4.071648in}%
\pgfsys@useobject{currentmarker}{}%
\end{pgfscope}%
\begin{pgfscope}%
\pgfsys@transformshift{2.001567in}{4.182440in}%
\pgfsys@useobject{currentmarker}{}%
\end{pgfscope}%
\begin{pgfscope}%
\pgfsys@transformshift{2.021051in}{4.403747in}%
\pgfsys@useobject{currentmarker}{}%
\end{pgfscope}%
\begin{pgfscope}%
\pgfsys@transformshift{2.038420in}{4.394972in}%
\pgfsys@useobject{currentmarker}{}%
\end{pgfscope}%
\begin{pgfscope}%
\pgfsys@transformshift{2.060484in}{4.222119in}%
\pgfsys@useobject{currentmarker}{}%
\end{pgfscope}%
\begin{pgfscope}%
\pgfsys@transformshift{2.079497in}{4.082505in}%
\pgfsys@useobject{currentmarker}{}%
\end{pgfscope}%
\begin{pgfscope}%
\pgfsys@transformshift{2.096633in}{4.046609in}%
\pgfsys@useobject{currentmarker}{}%
\end{pgfscope}%
\begin{pgfscope}%
\pgfsys@transformshift{2.117290in}{4.035659in}%
\pgfsys@useobject{currentmarker}{}%
\end{pgfscope}%
\begin{pgfscope}%
\pgfsys@transformshift{2.135832in}{4.033907in}%
\pgfsys@useobject{currentmarker}{}%
\end{pgfscope}%
\begin{pgfscope}%
\pgfsys@transformshift{2.156254in}{4.036492in}%
\pgfsys@useobject{currentmarker}{}%
\end{pgfscope}%
\begin{pgfscope}%
\pgfsys@transformshift{2.174798in}{4.044266in}%
\pgfsys@useobject{currentmarker}{}%
\end{pgfscope}%
\begin{pgfscope}%
\pgfsys@transformshift{2.195220in}{4.083084in}%
\pgfsys@useobject{currentmarker}{}%
\end{pgfscope}%
\begin{pgfscope}%
\pgfsys@transformshift{2.213293in}{4.149009in}%
\pgfsys@useobject{currentmarker}{}%
\end{pgfscope}%
\begin{pgfscope}%
\pgfsys@transformshift{2.236532in}{4.398904in}%
\pgfsys@useobject{currentmarker}{}%
\end{pgfscope}%
\begin{pgfscope}%
\pgfsys@transformshift{2.252023in}{4.411774in}%
\pgfsys@useobject{currentmarker}{}%
\end{pgfscope}%
\begin{pgfscope}%
\pgfsys@transformshift{2.269628in}{4.310166in}%
\pgfsys@useobject{currentmarker}{}%
\end{pgfscope}%
\begin{pgfscope}%
\pgfsys@transformshift{2.289581in}{4.117407in}%
\pgfsys@useobject{currentmarker}{}%
\end{pgfscope}%
\begin{pgfscope}%
\pgfsys@transformshift{2.312114in}{4.052999in}%
\pgfsys@useobject{currentmarker}{}%
\end{pgfscope}%
\begin{pgfscope}%
\pgfsys@transformshift{2.327842in}{4.039553in}%
\pgfsys@useobject{currentmarker}{}%
\end{pgfscope}%
\begin{pgfscope}%
\pgfsys@transformshift{2.345212in}{4.034637in}%
\pgfsys@useobject{currentmarker}{}%
\end{pgfscope}%
\begin{pgfscope}%
\pgfsys@transformshift{2.365398in}{4.034472in}%
\pgfsys@useobject{currentmarker}{}%
\end{pgfscope}%
\begin{pgfscope}%
\pgfsys@transformshift{2.386054in}{4.040695in}%
\pgfsys@useobject{currentmarker}{}%
\end{pgfscope}%
\begin{pgfscope}%
\pgfsys@transformshift{2.407181in}{4.058531in}%
\pgfsys@useobject{currentmarker}{}%
\end{pgfscope}%
\begin{pgfscope}%
\pgfsys@transformshift{2.421969in}{4.095673in}%
\pgfsys@useobject{currentmarker}{}%
\end{pgfscope}%
\begin{pgfscope}%
\pgfsys@transformshift{2.442390in}{4.224901in}%
\pgfsys@useobject{currentmarker}{}%
\end{pgfscope}%
\begin{pgfscope}%
\pgfsys@transformshift{2.466332in}{4.411289in}%
\pgfsys@useobject{currentmarker}{}%
\end{pgfscope}%
\begin{pgfscope}%
\pgfsys@transformshift{2.481355in}{4.373179in}%
\pgfsys@useobject{currentmarker}{}%
\end{pgfscope}%
\begin{pgfscope}%
\pgfsys@transformshift{2.502479in}{4.198798in}%
\pgfsys@useobject{currentmarker}{}%
\end{pgfscope}%
\begin{pgfscope}%
\pgfsys@transformshift{2.523136in}{4.084359in}%
\pgfsys@useobject{currentmarker}{}%
\end{pgfscope}%
\begin{pgfscope}%
\pgfsys@transformshift{2.538160in}{4.050018in}%
\pgfsys@useobject{currentmarker}{}%
\end{pgfscope}%
\begin{pgfscope}%
\pgfsys@transformshift{2.559285in}{4.036878in}%
\pgfsys@useobject{currentmarker}{}%
\end{pgfscope}%
\begin{pgfscope}%
\pgfsys@transformshift{2.576890in}{4.035406in}%
\pgfsys@useobject{currentmarker}{}%
\end{pgfscope}%
\begin{pgfscope}%
\pgfsys@transformshift{2.598249in}{4.034400in}%
\pgfsys@useobject{currentmarker}{}%
\end{pgfscope}%
\begin{pgfscope}%
\pgfsys@transformshift{2.615385in}{4.038283in}%
\pgfsys@useobject{currentmarker}{}%
\end{pgfscope}%
\begin{pgfscope}%
\pgfsys@transformshift{2.637215in}{4.052008in}%
\pgfsys@useobject{currentmarker}{}%
\end{pgfscope}%
\begin{pgfscope}%
\pgfsys@transformshift{2.654115in}{4.096448in}%
\pgfsys@useobject{currentmarker}{}%
\end{pgfscope}%
\begin{pgfscope}%
\pgfsys@transformshift{2.672188in}{4.178476in}%
\pgfsys@useobject{currentmarker}{}%
\end{pgfscope}%
\begin{pgfscope}%
\pgfsys@transformshift{2.693081in}{4.367547in}%
\pgfsys@useobject{currentmarker}{}%
\end{pgfscope}%
\begin{pgfscope}%
\pgfsys@transformshift{2.711623in}{4.403685in}%
\pgfsys@useobject{currentmarker}{}%
\end{pgfscope}%
\begin{pgfscope}%
\pgfsys@transformshift{2.729933in}{4.301515in}%
\pgfsys@useobject{currentmarker}{}%
\end{pgfscope}%
\begin{pgfscope}%
\pgfsys@transformshift{2.749884in}{4.132260in}%
\pgfsys@useobject{currentmarker}{}%
\end{pgfscope}%
\begin{pgfscope}%
\pgfsys@transformshift{2.770775in}{4.055461in}%
\pgfsys@useobject{currentmarker}{}%
\end{pgfscope}%
\begin{pgfscope}%
\pgfsys@transformshift{2.789085in}{4.040579in}%
\pgfsys@useobject{currentmarker}{}%
\end{pgfscope}%
\begin{pgfscope}%
\pgfsys@transformshift{2.807393in}{4.037716in}%
\pgfsys@useobject{currentmarker}{}%
\end{pgfscope}%
\begin{pgfscope}%
\pgfsys@transformshift{2.828520in}{4.034254in}%
\pgfsys@useobject{currentmarker}{}%
\end{pgfscope}%
\begin{pgfscope}%
\pgfsys@transformshift{2.846359in}{4.035082in}%
\pgfsys@useobject{currentmarker}{}%
\end{pgfscope}%
\begin{pgfscope}%
\pgfsys@transformshift{2.868189in}{4.038305in}%
\pgfsys@useobject{currentmarker}{}%
\end{pgfscope}%
\begin{pgfscope}%
\pgfsys@transformshift{2.885792in}{4.052315in}%
\pgfsys@useobject{currentmarker}{}%
\end{pgfscope}%
\begin{pgfscope}%
\pgfsys@transformshift{2.902694in}{4.075502in}%
\pgfsys@useobject{currentmarker}{}%
\end{pgfscope}%
\begin{pgfscope}%
\pgfsys@transformshift{2.924758in}{4.199302in}%
\pgfsys@useobject{currentmarker}{}%
\end{pgfscope}%
\begin{pgfscope}%
\pgfsys@transformshift{2.941658in}{4.350063in}%
\pgfsys@useobject{currentmarker}{}%
\end{pgfscope}%
\begin{pgfscope}%
\pgfsys@transformshift{2.963488in}{4.406271in}%
\pgfsys@useobject{currentmarker}{}%
\end{pgfscope}%
\begin{pgfscope}%
\pgfsys@transformshift{2.982033in}{4.291045in}%
\pgfsys@useobject{currentmarker}{}%
\end{pgfscope}%
\begin{pgfscope}%
\pgfsys@transformshift{2.999637in}{4.184730in}%
\pgfsys@useobject{currentmarker}{}%
\end{pgfscope}%
\begin{pgfscope}%
\pgfsys@transformshift{3.020528in}{4.083621in}%
\pgfsys@useobject{currentmarker}{}%
\end{pgfscope}%
\begin{pgfscope}%
\pgfsys@transformshift{3.038602in}{4.047175in}%
\pgfsys@useobject{currentmarker}{}%
\end{pgfscope}%
\begin{pgfscope}%
\pgfsys@transformshift{3.060432in}{4.036327in}%
\pgfsys@useobject{currentmarker}{}%
\end{pgfscope}%
\begin{pgfscope}%
\pgfsys@transformshift{3.077568in}{4.034191in}%
\pgfsys@useobject{currentmarker}{}%
\end{pgfscope}%
\begin{pgfscope}%
\pgfsys@transformshift{3.096110in}{4.035927in}%
\pgfsys@useobject{currentmarker}{}%
\end{pgfscope}%
\begin{pgfscope}%
\pgfsys@transformshift{3.120288in}{4.042229in}%
\pgfsys@useobject{currentmarker}{}%
\end{pgfscope}%
\begin{pgfscope}%
\pgfsys@transformshift{3.135311in}{4.057494in}%
\pgfsys@useobject{currentmarker}{}%
\end{pgfscope}%
\begin{pgfscope}%
\pgfsys@transformshift{3.153384in}{4.104557in}%
\pgfsys@useobject{currentmarker}{}%
\end{pgfscope}%
\begin{pgfscope}%
\pgfsys@transformshift{3.174040in}{4.247284in}%
\pgfsys@useobject{currentmarker}{}%
\end{pgfscope}%
\begin{pgfscope}%
\pgfsys@transformshift{3.192819in}{4.396020in}%
\pgfsys@useobject{currentmarker}{}%
\end{pgfscope}%
\begin{pgfscope}%
\pgfsys@transformshift{3.209955in}{4.413713in}%
\pgfsys@useobject{currentmarker}{}%
\end{pgfscope}%
\begin{pgfscope}%
\pgfsys@transformshift{3.230375in}{4.333390in}%
\pgfsys@useobject{currentmarker}{}%
\end{pgfscope}%
\begin{pgfscope}%
\pgfsys@transformshift{3.255257in}{4.262957in}%
\pgfsys@useobject{currentmarker}{}%
\end{pgfscope}%
\begin{pgfscope}%
\pgfsys@transformshift{3.269107in}{4.128059in}%
\pgfsys@useobject{currentmarker}{}%
\end{pgfscope}%
\begin{pgfscope}%
\pgfsys@transformshift{3.288589in}{4.059054in}%
\pgfsys@useobject{currentmarker}{}%
\end{pgfscope}%
\begin{pgfscope}%
\pgfsys@transformshift{3.310888in}{4.039779in}%
\pgfsys@useobject{currentmarker}{}%
\end{pgfscope}%
\begin{pgfscope}%
\pgfsys@transformshift{3.331075in}{4.034736in}%
\pgfsys@useobject{currentmarker}{}%
\end{pgfscope}%
\begin{pgfscope}%
\pgfsys@transformshift{3.349149in}{4.034740in}%
\pgfsys@useobject{currentmarker}{}%
\end{pgfscope}%
\begin{pgfscope}%
\pgfsys@transformshift{3.366285in}{4.037966in}%
\pgfsys@useobject{currentmarker}{}%
\end{pgfscope}%
\begin{pgfscope}%
\pgfsys@transformshift{3.383890in}{4.045209in}%
\pgfsys@useobject{currentmarker}{}%
\end{pgfscope}%
\begin{pgfscope}%
\pgfsys@transformshift{3.406189in}{4.072821in}%
\pgfsys@useobject{currentmarker}{}%
\end{pgfscope}%
\begin{pgfscope}%
\pgfsys@transformshift{3.422619in}{4.141755in}%
\pgfsys@useobject{currentmarker}{}%
\end{pgfscope}%
\begin{pgfscope}%
\pgfsys@transformshift{3.444684in}{4.293579in}%
\pgfsys@useobject{currentmarker}{}%
\end{pgfscope}%
\begin{pgfscope}%
\pgfsys@transformshift{3.462758in}{4.416772in}%
\pgfsys@useobject{currentmarker}{}%
\end{pgfscope}%
\begin{pgfscope}%
\pgfsys@transformshift{3.481771in}{4.416454in}%
\pgfsys@useobject{currentmarker}{}%
\end{pgfscope}%
\begin{pgfscope}%
\pgfsys@transformshift{3.502661in}{4.336018in}%
\pgfsys@useobject{currentmarker}{}%
\end{pgfscope}%
\begin{pgfscope}%
\pgfsys@transformshift{3.519329in}{4.190170in}%
\pgfsys@useobject{currentmarker}{}%
\end{pgfscope}%
\begin{pgfscope}%
\pgfsys@transformshift{3.540688in}{4.079914in}%
\pgfsys@useobject{currentmarker}{}%
\end{pgfscope}%
\begin{pgfscope}%
\pgfsys@transformshift{3.558293in}{4.052784in}%
\pgfsys@useobject{currentmarker}{}%
\end{pgfscope}%
\begin{pgfscope}%
\pgfsys@transformshift{3.575898in}{4.043505in}%
\pgfsys@useobject{currentmarker}{}%
\end{pgfscope}%
\begin{pgfscope}%
\pgfsys@transformshift{3.596788in}{4.035952in}%
\pgfsys@useobject{currentmarker}{}%
\end{pgfscope}%
\begin{pgfscope}%
\pgfsys@transformshift{3.614862in}{4.034975in}%
\pgfsys@useobject{currentmarker}{}%
\end{pgfscope}%
\begin{pgfscope}%
\pgfsys@transformshift{3.633172in}{4.037056in}%
\pgfsys@useobject{currentmarker}{}%
\end{pgfscope}%
\begin{pgfscope}%
\pgfsys@transformshift{3.655471in}{4.043434in}%
\pgfsys@useobject{currentmarker}{}%
\end{pgfscope}%
\begin{pgfscope}%
\pgfsys@transformshift{3.676361in}{4.065955in}%
\pgfsys@useobject{currentmarker}{}%
\end{pgfscope}%
\begin{pgfscope}%
\pgfsys@transformshift{3.693497in}{4.105698in}%
\pgfsys@useobject{currentmarker}{}%
\end{pgfscope}%
\begin{pgfscope}%
\pgfsys@transformshift{3.710868in}{4.183774in}%
\pgfsys@useobject{currentmarker}{}%
\end{pgfscope}%
\begin{pgfscope}%
\pgfsys@transformshift{3.728941in}{4.393046in}%
\pgfsys@useobject{currentmarker}{}%
\end{pgfscope}%
\begin{pgfscope}%
\pgfsys@transformshift{3.750066in}{4.431824in}%
\pgfsys@useobject{currentmarker}{}%
\end{pgfscope}%
\begin{pgfscope}%
\pgfsys@transformshift{3.771193in}{4.371162in}%
\pgfsys@useobject{currentmarker}{}%
\end{pgfscope}%
\begin{pgfscope}%
\pgfsys@transformshift{3.789267in}{4.236940in}%
\pgfsys@useobject{currentmarker}{}%
\end{pgfscope}%
\begin{pgfscope}%
\pgfsys@transformshift{3.810626in}{4.121638in}%
\pgfsys@useobject{currentmarker}{}%
\end{pgfscope}%
\begin{pgfscope}%
\pgfsys@transformshift{3.827997in}{4.100151in}%
\pgfsys@useobject{currentmarker}{}%
\end{pgfscope}%
\begin{pgfscope}%
\pgfsys@transformshift{3.845836in}{4.063487in}%
\pgfsys@useobject{currentmarker}{}%
\end{pgfscope}%
\begin{pgfscope}%
\pgfsys@transformshift{3.866963in}{4.130273in}%
\pgfsys@useobject{currentmarker}{}%
\end{pgfscope}%
\begin{pgfscope}%
\pgfsys@transformshift{3.885505in}{4.062475in}%
\pgfsys@useobject{currentmarker}{}%
\end{pgfscope}%
\begin{pgfscope}%
\pgfsys@transformshift{3.903579in}{4.042823in}%
\pgfsys@useobject{currentmarker}{}%
\end{pgfscope}%
\begin{pgfscope}%
\pgfsys@transformshift{3.922123in}{4.036386in}%
\pgfsys@useobject{currentmarker}{}%
\end{pgfscope}%
\begin{pgfscope}%
\pgfsys@transformshift{3.942545in}{4.035436in}%
\pgfsys@useobject{currentmarker}{}%
\end{pgfscope}%
\begin{pgfscope}%
\pgfsys@transformshift{3.963201in}{4.040497in}%
\pgfsys@useobject{currentmarker}{}%
\end{pgfscope}%
\begin{pgfscope}%
\pgfsys@transformshift{3.980806in}{4.050543in}%
\pgfsys@useobject{currentmarker}{}%
\end{pgfscope}%
\begin{pgfscope}%
\pgfsys@transformshift{3.998176in}{4.072718in}%
\pgfsys@useobject{currentmarker}{}%
\end{pgfscope}%
\begin{pgfscope}%
\pgfsys@transformshift{4.019067in}{4.146970in}%
\pgfsys@useobject{currentmarker}{}%
\end{pgfscope}%
\begin{pgfscope}%
\pgfsys@transformshift{4.037844in}{4.343521in}%
\pgfsys@useobject{currentmarker}{}%
\end{pgfscope}%
\begin{pgfscope}%
\pgfsys@transformshift{4.058971in}{4.435822in}%
\pgfsys@useobject{currentmarker}{}%
\end{pgfscope}%
\begin{pgfscope}%
\pgfsys@transformshift{4.077513in}{4.432432in}%
\pgfsys@useobject{currentmarker}{}%
\end{pgfscope}%
\begin{pgfscope}%
\pgfsys@transformshift{4.095354in}{4.357829in}%
\pgfsys@useobject{currentmarker}{}%
\end{pgfscope}%
\begin{pgfscope}%
\pgfsys@transformshift{4.116245in}{4.188975in}%
\pgfsys@useobject{currentmarker}{}%
\end{pgfscope}%
\begin{pgfscope}%
\pgfsys@transformshift{4.133615in}{4.122602in}%
\pgfsys@useobject{currentmarker}{}%
\end{pgfscope}%
\begin{pgfscope}%
\pgfsys@transformshift{4.152392in}{4.063465in}%
\pgfsys@useobject{currentmarker}{}%
\end{pgfscope}%
\begin{pgfscope}%
\pgfsys@transformshift{4.172814in}{4.041755in}%
\pgfsys@useobject{currentmarker}{}%
\end{pgfscope}%
\begin{pgfscope}%
\pgfsys@transformshift{4.192062in}{4.039571in}%
\pgfsys@useobject{currentmarker}{}%
\end{pgfscope}%
\begin{pgfscope}%
\pgfsys@transformshift{4.212952in}{4.035894in}%
\pgfsys@useobject{currentmarker}{}%
\end{pgfscope}%
\begin{pgfscope}%
\pgfsys@transformshift{4.230088in}{4.036750in}%
\pgfsys@useobject{currentmarker}{}%
\end{pgfscope}%
\begin{pgfscope}%
\pgfsys@transformshift{4.248396in}{4.042113in}%
\pgfsys@useobject{currentmarker}{}%
\end{pgfscope}%
\begin{pgfscope}%
\pgfsys@transformshift{4.271635in}{4.068129in}%
\pgfsys@useobject{currentmarker}{}%
\end{pgfscope}%
\begin{pgfscope}%
\pgfsys@transformshift{4.289474in}{4.122859in}%
\pgfsys@useobject{currentmarker}{}%
\end{pgfscope}%
\begin{pgfscope}%
\pgfsys@transformshift{4.307784in}{4.270646in}%
\pgfsys@useobject{currentmarker}{}%
\end{pgfscope}%
\begin{pgfscope}%
\pgfsys@transformshift{4.325154in}{4.407826in}%
\pgfsys@useobject{currentmarker}{}%
\end{pgfscope}%
\begin{pgfscope}%
\pgfsys@transformshift{4.345811in}{4.456832in}%
\pgfsys@useobject{currentmarker}{}%
\end{pgfscope}%
\begin{pgfscope}%
\pgfsys@transformshift{4.365527in}{4.424648in}%
\pgfsys@useobject{currentmarker}{}%
\end{pgfscope}%
\begin{pgfscope}%
\pgfsys@transformshift{4.384306in}{4.340551in}%
\pgfsys@useobject{currentmarker}{}%
\end{pgfscope}%
\begin{pgfscope}%
\pgfsys@transformshift{4.405431in}{4.196112in}%
\pgfsys@useobject{currentmarker}{}%
\end{pgfscope}%
\begin{pgfscope}%
\pgfsys@transformshift{4.423505in}{4.108136in}%
\pgfsys@useobject{currentmarker}{}%
\end{pgfscope}%
\begin{pgfscope}%
\pgfsys@transformshift{4.442518in}{4.176989in}%
\pgfsys@useobject{currentmarker}{}%
\end{pgfscope}%
\begin{pgfscope}%
\pgfsys@transformshift{4.459888in}{4.096558in}%
\pgfsys@useobject{currentmarker}{}%
\end{pgfscope}%
\begin{pgfscope}%
\pgfsys@transformshift{4.479841in}{4.052789in}%
\pgfsys@useobject{currentmarker}{}%
\end{pgfscope}%
\begin{pgfscope}%
\pgfsys@transformshift{4.481718in}{4.050528in}%
\pgfsys@useobject{currentmarker}{}%
\end{pgfscope}%
\begin{pgfscope}%
\pgfsys@transformshift{4.476553in}{4.058277in}%
\pgfsys@useobject{currentmarker}{}%
\end{pgfscope}%
\begin{pgfscope}%
\pgfsys@transformshift{4.456132in}{4.144963in}%
\pgfsys@useobject{currentmarker}{}%
\end{pgfscope}%
\begin{pgfscope}%
\pgfsys@transformshift{4.434303in}{4.380131in}%
\pgfsys@useobject{currentmarker}{}%
\end{pgfscope}%
\begin{pgfscope}%
\pgfsys@transformshift{4.416464in}{4.456506in}%
\pgfsys@useobject{currentmarker}{}%
\end{pgfscope}%
\begin{pgfscope}%
\pgfsys@transformshift{4.396277in}{4.393008in}%
\pgfsys@useobject{currentmarker}{}%
\end{pgfscope}%
\begin{pgfscope}%
\pgfsys@transformshift{4.379141in}{4.154152in}%
\pgfsys@useobject{currentmarker}{}%
\end{pgfscope}%
\begin{pgfscope}%
\pgfsys@transformshift{4.358250in}{4.060482in}%
\pgfsys@useobject{currentmarker}{}%
\end{pgfscope}%
\begin{pgfscope}%
\pgfsys@transformshift{4.335012in}{4.037400in}%
\pgfsys@useobject{currentmarker}{}%
\end{pgfscope}%
\begin{pgfscope}%
\pgfsys@transformshift{4.320458in}{4.036127in}%
\pgfsys@useobject{currentmarker}{}%
\end{pgfscope}%
\begin{pgfscope}%
\pgfsys@transformshift{4.299568in}{4.046610in}%
\pgfsys@useobject{currentmarker}{}%
\end{pgfscope}%
\begin{pgfscope}%
\pgfsys@transformshift{4.282903in}{4.084662in}%
\pgfsys@useobject{currentmarker}{}%
\end{pgfscope}%
\begin{pgfscope}%
\pgfsys@transformshift{4.262246in}{4.245546in}%
\pgfsys@useobject{currentmarker}{}%
\end{pgfscope}%
\begin{pgfscope}%
\pgfsys@transformshift{4.245110in}{4.411760in}%
\pgfsys@useobject{currentmarker}{}%
\end{pgfscope}%
\begin{pgfscope}%
\pgfsys@transformshift{4.224689in}{4.440880in}%
\pgfsys@useobject{currentmarker}{}%
\end{pgfscope}%
\begin{pgfscope}%
\pgfsys@transformshift{4.203798in}{4.223146in}%
\pgfsys@useobject{currentmarker}{}%
\end{pgfscope}%
\begin{pgfscope}%
\pgfsys@transformshift{4.186428in}{4.078831in}%
\pgfsys@useobject{currentmarker}{}%
\end{pgfscope}%
\begin{pgfscope}%
\pgfsys@transformshift{4.167180in}{4.043965in}%
\pgfsys@useobject{currentmarker}{}%
\end{pgfscope}%
\begin{pgfscope}%
\pgfsys@transformshift{4.148403in}{4.035738in}%
\pgfsys@useobject{currentmarker}{}%
\end{pgfscope}%
\begin{pgfscope}%
\pgfsys@transformshift{4.128216in}{4.038717in}%
\pgfsys@useobject{currentmarker}{}%
\end{pgfscope}%
\begin{pgfscope}%
\pgfsys@transformshift{4.110611in}{4.053921in}%
\pgfsys@useobject{currentmarker}{}%
\end{pgfscope}%
\begin{pgfscope}%
\pgfsys@transformshift{4.090189in}{4.132472in}%
\pgfsys@useobject{currentmarker}{}%
\end{pgfscope}%
\begin{pgfscope}%
\pgfsys@transformshift{4.069299in}{4.346850in}%
\pgfsys@useobject{currentmarker}{}%
\end{pgfscope}%
\begin{pgfscope}%
\pgfsys@transformshift{4.048643in}{4.439277in}%
\pgfsys@useobject{currentmarker}{}%
\end{pgfscope}%
\begin{pgfscope}%
\pgfsys@transformshift{4.031272in}{4.359225in}%
\pgfsys@useobject{currentmarker}{}%
\end{pgfscope}%
\begin{pgfscope}%
\pgfsys@transformshift{4.013902in}{4.127580in}%
\pgfsys@useobject{currentmarker}{}%
\end{pgfscope}%
\begin{pgfscope}%
\pgfsys@transformshift{3.997237in}{4.055281in}%
\pgfsys@useobject{currentmarker}{}%
\end{pgfscope}%
\begin{pgfscope}%
\pgfsys@transformshift{3.975875in}{4.039957in}%
\pgfsys@useobject{currentmarker}{}%
\end{pgfscope}%
\begin{pgfscope}%
\pgfsys@transformshift{3.958507in}{4.035097in}%
\pgfsys@useobject{currentmarker}{}%
\end{pgfscope}%
\begin{pgfscope}%
\pgfsys@transformshift{3.935268in}{4.039556in}%
\pgfsys@useobject{currentmarker}{}%
\end{pgfscope}%
\begin{pgfscope}%
\pgfsys@transformshift{3.918603in}{4.058091in}%
\pgfsys@useobject{currentmarker}{}%
\end{pgfscope}%
\begin{pgfscope}%
\pgfsys@transformshift{3.895833in}{4.136791in}%
\pgfsys@useobject{currentmarker}{}%
\end{pgfscope}%
\begin{pgfscope}%
\pgfsys@transformshift{3.878229in}{4.315275in}%
\pgfsys@useobject{currentmarker}{}%
\end{pgfscope}%
\begin{pgfscope}%
\pgfsys@transformshift{3.858512in}{4.430000in}%
\pgfsys@useobject{currentmarker}{}%
\end{pgfscope}%
\begin{pgfscope}%
\pgfsys@transformshift{3.839733in}{4.343074in}%
\pgfsys@useobject{currentmarker}{}%
\end{pgfscope}%
\begin{pgfscope}%
\pgfsys@transformshift{3.821660in}{4.129088in}%
\pgfsys@useobject{currentmarker}{}%
\end{pgfscope}%
\begin{pgfscope}%
\pgfsys@transformshift{3.801472in}{4.059412in}%
\pgfsys@useobject{currentmarker}{}%
\end{pgfscope}%
\begin{pgfscope}%
\pgfsys@transformshift{3.780582in}{4.040310in}%
\pgfsys@useobject{currentmarker}{}%
\end{pgfscope}%
\begin{pgfscope}%
\pgfsys@transformshift{3.763446in}{4.035236in}%
\pgfsys@useobject{currentmarker}{}%
\end{pgfscope}%
\begin{pgfscope}%
\pgfsys@transformshift{3.743258in}{4.037472in}%
\pgfsys@useobject{currentmarker}{}%
\end{pgfscope}%
\begin{pgfscope}%
\pgfsys@transformshift{3.724482in}{4.056778in}%
\pgfsys@useobject{currentmarker}{}%
\end{pgfscope}%
\begin{pgfscope}%
\pgfsys@transformshift{3.704529in}{4.112195in}%
\pgfsys@useobject{currentmarker}{}%
\end{pgfscope}%
\begin{pgfscope}%
\pgfsys@transformshift{3.686689in}{4.269304in}%
\pgfsys@useobject{currentmarker}{}%
\end{pgfscope}%
\begin{pgfscope}%
\pgfsys@transformshift{3.667676in}{4.412518in}%
\pgfsys@useobject{currentmarker}{}%
\end{pgfscope}%
\begin{pgfscope}%
\pgfsys@transformshift{3.649134in}{4.397651in}%
\pgfsys@useobject{currentmarker}{}%
\end{pgfscope}%
\begin{pgfscope}%
\pgfsys@transformshift{3.627538in}{4.148275in}%
\pgfsys@useobject{currentmarker}{}%
\end{pgfscope}%
\begin{pgfscope}%
\pgfsys@transformshift{3.609230in}{4.065070in}%
\pgfsys@useobject{currentmarker}{}%
\end{pgfscope}%
\begin{pgfscope}%
\pgfsys@transformshift{3.588337in}{4.040062in}%
\pgfsys@useobject{currentmarker}{}%
\end{pgfscope}%
\begin{pgfscope}%
\pgfsys@transformshift{3.570498in}{4.167643in}%
\pgfsys@useobject{currentmarker}{}%
\end{pgfscope}%
\begin{pgfscope}%
\pgfsys@transformshift{3.551956in}{4.090915in}%
\pgfsys@useobject{currentmarker}{}%
\end{pgfscope}%
\begin{pgfscope}%
\pgfsys@transformshift{3.534116in}{4.052699in}%
\pgfsys@useobject{currentmarker}{}%
\end{pgfscope}%
\begin{pgfscope}%
\pgfsys@transformshift{3.514867in}{4.037992in}%
\pgfsys@useobject{currentmarker}{}%
\end{pgfscope}%
\begin{pgfscope}%
\pgfsys@transformshift{3.494211in}{4.034669in}%
\pgfsys@useobject{currentmarker}{}%
\end{pgfscope}%
\begin{pgfscope}%
\pgfsys@transformshift{3.474729in}{4.037965in}%
\pgfsys@useobject{currentmarker}{}%
\end{pgfscope}%
\begin{pgfscope}%
\pgfsys@transformshift{3.456186in}{4.050531in}%
\pgfsys@useobject{currentmarker}{}%
\end{pgfscope}%
\begin{pgfscope}%
\pgfsys@transformshift{3.435764in}{4.039472in}%
\pgfsys@useobject{currentmarker}{}%
\end{pgfscope}%
\begin{pgfscope}%
\pgfsys@transformshift{3.418628in}{4.059235in}%
\pgfsys@useobject{currentmarker}{}%
\end{pgfscope}%
\begin{pgfscope}%
\pgfsys@transformshift{3.396095in}{4.151980in}%
\pgfsys@useobject{currentmarker}{}%
\end{pgfscope}%
\begin{pgfscope}%
\pgfsys@transformshift{3.378256in}{4.332669in}%
\pgfsys@useobject{currentmarker}{}%
\end{pgfscope}%
\begin{pgfscope}%
\pgfsys@transformshift{3.359946in}{4.417551in}%
\pgfsys@useobject{currentmarker}{}%
\end{pgfscope}%
\begin{pgfscope}%
\pgfsys@transformshift{3.341638in}{4.305824in}%
\pgfsys@useobject{currentmarker}{}%
\end{pgfscope}%
\begin{pgfscope}%
\pgfsys@transformshift{3.319808in}{4.118629in}%
\pgfsys@useobject{currentmarker}{}%
\end{pgfscope}%
\begin{pgfscope}%
\pgfsys@transformshift{3.301734in}{4.056818in}%
\pgfsys@useobject{currentmarker}{}%
\end{pgfscope}%
\begin{pgfscope}%
\pgfsys@transformshift{3.282721in}{4.039128in}%
\pgfsys@useobject{currentmarker}{}%
\end{pgfscope}%
\begin{pgfscope}%
\pgfsys@transformshift{3.264176in}{4.034709in}%
\pgfsys@useobject{currentmarker}{}%
\end{pgfscope}%
\begin{pgfscope}%
\pgfsys@transformshift{3.241643in}{4.038477in}%
\pgfsys@useobject{currentmarker}{}%
\end{pgfscope}%
\begin{pgfscope}%
\pgfsys@transformshift{3.224038in}{4.047266in}%
\pgfsys@useobject{currentmarker}{}%
\end{pgfscope}%
\begin{pgfscope}%
\pgfsys@transformshift{3.205259in}{4.092549in}%
\pgfsys@useobject{currentmarker}{}%
\end{pgfscope}%
\begin{pgfscope}%
\pgfsys@transformshift{3.186951in}{4.252083in}%
\pgfsys@useobject{currentmarker}{}%
\end{pgfscope}%
\begin{pgfscope}%
\pgfsys@transformshift{3.167469in}{4.380051in}%
\pgfsys@useobject{currentmarker}{}%
\end{pgfscope}%
\begin{pgfscope}%
\pgfsys@transformshift{3.148690in}{4.406787in}%
\pgfsys@useobject{currentmarker}{}%
\end{pgfscope}%
\begin{pgfscope}%
\pgfsys@transformshift{3.130382in}{4.210342in}%
\pgfsys@useobject{currentmarker}{}%
\end{pgfscope}%
\begin{pgfscope}%
\pgfsys@transformshift{3.108317in}{4.074227in}%
\pgfsys@useobject{currentmarker}{}%
\end{pgfscope}%
\begin{pgfscope}%
\pgfsys@transformshift{3.091650in}{4.059304in}%
\pgfsys@useobject{currentmarker}{}%
\end{pgfscope}%
\begin{pgfscope}%
\pgfsys@transformshift{3.072168in}{4.038906in}%
\pgfsys@useobject{currentmarker}{}%
\end{pgfscope}%
\begin{pgfscope}%
\pgfsys@transformshift{3.053389in}{4.034597in}%
\pgfsys@useobject{currentmarker}{}%
\end{pgfscope}%
\begin{pgfscope}%
\pgfsys@transformshift{3.028508in}{4.039135in}%
\pgfsys@useobject{currentmarker}{}%
\end{pgfscope}%
\begin{pgfscope}%
\pgfsys@transformshift{3.013017in}{4.049739in}%
\pgfsys@useobject{currentmarker}{}%
\end{pgfscope}%
\begin{pgfscope}%
\pgfsys@transformshift{2.994472in}{4.097098in}%
\pgfsys@useobject{currentmarker}{}%
\end{pgfscope}%
\begin{pgfscope}%
\pgfsys@transformshift{2.976164in}{4.256961in}%
\pgfsys@useobject{currentmarker}{}%
\end{pgfscope}%
\begin{pgfscope}%
\pgfsys@transformshift{2.954100in}{4.403468in}%
\pgfsys@useobject{currentmarker}{}%
\end{pgfscope}%
\begin{pgfscope}%
\pgfsys@transformshift{2.936026in}{4.408078in}%
\pgfsys@useobject{currentmarker}{}%
\end{pgfscope}%
\begin{pgfscope}%
\pgfsys@transformshift{2.918185in}{4.206676in}%
\pgfsys@useobject{currentmarker}{}%
\end{pgfscope}%
\begin{pgfscope}%
\pgfsys@transformshift{2.898234in}{4.082180in}%
\pgfsys@useobject{currentmarker}{}%
\end{pgfscope}%
\begin{pgfscope}%
\pgfsys@transformshift{2.877109in}{4.047904in}%
\pgfsys@useobject{currentmarker}{}%
\end{pgfscope}%
\begin{pgfscope}%
\pgfsys@transformshift{2.861616in}{4.038307in}%
\pgfsys@useobject{currentmarker}{}%
\end{pgfscope}%
\begin{pgfscope}%
\pgfsys@transformshift{2.843308in}{4.034645in}%
\pgfsys@useobject{currentmarker}{}%
\end{pgfscope}%
\begin{pgfscope}%
\pgfsys@transformshift{2.820304in}{4.036656in}%
\pgfsys@useobject{currentmarker}{}%
\end{pgfscope}%
\begin{pgfscope}%
\pgfsys@transformshift{2.800351in}{4.047782in}%
\pgfsys@useobject{currentmarker}{}%
\end{pgfscope}%
\begin{pgfscope}%
\pgfsys@transformshift{2.782277in}{4.071759in}%
\pgfsys@useobject{currentmarker}{}%
\end{pgfscope}%
\begin{pgfscope}%
\pgfsys@transformshift{2.764203in}{4.171605in}%
\pgfsys@useobject{currentmarker}{}%
\end{pgfscope}%
\begin{pgfscope}%
\pgfsys@transformshift{2.742608in}{4.375099in}%
\pgfsys@useobject{currentmarker}{}%
\end{pgfscope}%
\begin{pgfscope}%
\pgfsys@transformshift{2.725942in}{4.411135in}%
\pgfsys@useobject{currentmarker}{}%
\end{pgfscope}%
\begin{pgfscope}%
\pgfsys@transformshift{2.705286in}{4.224686in}%
\pgfsys@useobject{currentmarker}{}%
\end{pgfscope}%
\begin{pgfscope}%
\pgfsys@transformshift{2.686273in}{4.089945in}%
\pgfsys@useobject{currentmarker}{}%
\end{pgfscope}%
\begin{pgfscope}%
\pgfsys@transformshift{2.667494in}{4.051339in}%
\pgfsys@useobject{currentmarker}{}%
\end{pgfscope}%
\begin{pgfscope}%
\pgfsys@transformshift{2.649421in}{4.038547in}%
\pgfsys@useobject{currentmarker}{}%
\end{pgfscope}%
\begin{pgfscope}%
\pgfsys@transformshift{2.628296in}{4.034480in}%
\pgfsys@useobject{currentmarker}{}%
\end{pgfscope}%
\begin{pgfscope}%
\pgfsys@transformshift{2.610456in}{4.035242in}%
\pgfsys@useobject{currentmarker}{}%
\end{pgfscope}%
\begin{pgfscope}%
\pgfsys@transformshift{2.591209in}{4.041733in}%
\pgfsys@useobject{currentmarker}{}%
\end{pgfscope}%
\begin{pgfscope}%
\pgfsys@transformshift{2.569613in}{4.071185in}%
\pgfsys@useobject{currentmarker}{}%
\end{pgfscope}%
\begin{pgfscope}%
\pgfsys@transformshift{2.553885in}{4.134038in}%
\pgfsys@useobject{currentmarker}{}%
\end{pgfscope}%
\begin{pgfscope}%
\pgfsys@transformshift{2.532292in}{4.300221in}%
\pgfsys@useobject{currentmarker}{}%
\end{pgfscope}%
\begin{pgfscope}%
\pgfsys@transformshift{2.513278in}{4.407130in}%
\pgfsys@useobject{currentmarker}{}%
\end{pgfscope}%
\begin{pgfscope}%
\pgfsys@transformshift{2.494499in}{4.356014in}%
\pgfsys@useobject{currentmarker}{}%
\end{pgfscope}%
\begin{pgfscope}%
\pgfsys@transformshift{2.477834in}{4.173829in}%
\pgfsys@useobject{currentmarker}{}%
\end{pgfscope}%
\begin{pgfscope}%
\pgfsys@transformshift{2.451074in}{4.074731in}%
\pgfsys@useobject{currentmarker}{}%
\end{pgfscope}%
\begin{pgfscope}%
\pgfsys@transformshift{2.438634in}{4.062285in}%
\pgfsys@useobject{currentmarker}{}%
\end{pgfscope}%
\begin{pgfscope}%
\pgfsys@transformshift{2.417038in}{4.039535in}%
\pgfsys@useobject{currentmarker}{}%
\end{pgfscope}%
\begin{pgfscope}%
\pgfsys@transformshift{2.398730in}{4.034855in}%
\pgfsys@useobject{currentmarker}{}%
\end{pgfscope}%
\begin{pgfscope}%
\pgfsys@transformshift{2.379717in}{4.039082in}%
\pgfsys@useobject{currentmarker}{}%
\end{pgfscope}%
\begin{pgfscope}%
\pgfsys@transformshift{2.360000in}{4.034482in}%
\pgfsys@useobject{currentmarker}{}%
\end{pgfscope}%
\begin{pgfscope}%
\pgfsys@transformshift{2.341925in}{4.035437in}%
\pgfsys@useobject{currentmarker}{}%
\end{pgfscope}%
\begin{pgfscope}%
\pgfsys@transformshift{2.323382in}{4.042195in}%
\pgfsys@useobject{currentmarker}{}%
\end{pgfscope}%
\begin{pgfscope}%
\pgfsys@transformshift{2.301552in}{4.068801in}%
\pgfsys@useobject{currentmarker}{}%
\end{pgfscope}%
\begin{pgfscope}%
\pgfsys@transformshift{2.281835in}{4.178833in}%
\pgfsys@useobject{currentmarker}{}%
\end{pgfscope}%
\begin{pgfscope}%
\pgfsys@transformshift{2.263994in}{4.306308in}%
\pgfsys@useobject{currentmarker}{}%
\end{pgfscope}%
\begin{pgfscope}%
\pgfsys@transformshift{2.243574in}{4.407234in}%
\pgfsys@useobject{currentmarker}{}%
\end{pgfscope}%
\begin{pgfscope}%
\pgfsys@transformshift{2.225499in}{4.346718in}%
\pgfsys@useobject{currentmarker}{}%
\end{pgfscope}%
\begin{pgfscope}%
\pgfsys@transformshift{2.206722in}{4.169937in}%
\pgfsys@useobject{currentmarker}{}%
\end{pgfscope}%
\begin{pgfscope}%
\pgfsys@transformshift{2.188881in}{4.073167in}%
\pgfsys@useobject{currentmarker}{}%
\end{pgfscope}%
\begin{pgfscope}%
\pgfsys@transformshift{2.167990in}{4.044396in}%
\pgfsys@useobject{currentmarker}{}%
\end{pgfscope}%
\begin{pgfscope}%
\pgfsys@transformshift{2.148743in}{4.036411in}%
\pgfsys@useobject{currentmarker}{}%
\end{pgfscope}%
\begin{pgfscope}%
\pgfsys@transformshift{2.129026in}{4.034787in}%
\pgfsys@useobject{currentmarker}{}%
\end{pgfscope}%
\begin{pgfscope}%
\pgfsys@transformshift{2.114004in}{4.038042in}%
\pgfsys@useobject{currentmarker}{}%
\end{pgfscope}%
\begin{pgfscope}%
\pgfsys@transformshift{2.088888in}{4.043950in}%
\pgfsys@useobject{currentmarker}{}%
\end{pgfscope}%
\begin{pgfscope}%
\pgfsys@transformshift{2.071517in}{4.067971in}%
\pgfsys@useobject{currentmarker}{}%
\end{pgfscope}%
\begin{pgfscope}%
\pgfsys@transformshift{2.052504in}{4.147503in}%
\pgfsys@useobject{currentmarker}{}%
\end{pgfscope}%
\begin{pgfscope}%
\pgfsys@transformshift{2.035134in}{4.298241in}%
\pgfsys@useobject{currentmarker}{}%
\end{pgfscope}%
\begin{pgfscope}%
\pgfsys@transformshift{2.010721in}{4.409817in}%
\pgfsys@useobject{currentmarker}{}%
\end{pgfscope}%
\begin{pgfscope}%
\pgfsys@transformshift{1.998750in}{4.371079in}%
\pgfsys@useobject{currentmarker}{}%
\end{pgfscope}%
\begin{pgfscope}%
\pgfsys@transformshift{1.976217in}{4.186791in}%
\pgfsys@useobject{currentmarker}{}%
\end{pgfscope}%
\begin{pgfscope}%
\pgfsys@transformshift{1.958377in}{4.101644in}%
\pgfsys@useobject{currentmarker}{}%
\end{pgfscope}%
\begin{pgfscope}%
\pgfsys@transformshift{1.937487in}{4.052816in}%
\pgfsys@useobject{currentmarker}{}%
\end{pgfscope}%
\begin{pgfscope}%
\pgfsys@transformshift{1.919413in}{4.039835in}%
\pgfsys@useobject{currentmarker}{}%
\end{pgfscope}%
\begin{pgfscope}%
\pgfsys@transformshift{1.897583in}{4.035494in}%
\pgfsys@useobject{currentmarker}{}%
\end{pgfscope}%
\begin{pgfscope}%
\pgfsys@transformshift{1.879744in}{4.035047in}%
\pgfsys@useobject{currentmarker}{}%
\end{pgfscope}%
\begin{pgfscope}%
\pgfsys@transformshift{1.860965in}{4.039104in}%
\pgfsys@useobject{currentmarker}{}%
\end{pgfscope}%
\begin{pgfscope}%
\pgfsys@transformshift{1.842421in}{4.038623in}%
\pgfsys@useobject{currentmarker}{}%
\end{pgfscope}%
\begin{pgfscope}%
\pgfsys@transformshift{1.823878in}{4.034835in}%
\pgfsys@useobject{currentmarker}{}%
\end{pgfscope}%
\begin{pgfscope}%
\pgfsys@transformshift{1.803925in}{4.037045in}%
\pgfsys@useobject{currentmarker}{}%
\end{pgfscope}%
\begin{pgfscope}%
\pgfsys@transformshift{1.783738in}{4.047480in}%
\pgfsys@useobject{currentmarker}{}%
\end{pgfscope}%
\begin{pgfscope}%
\pgfsys@transformshift{1.763318in}{4.096621in}%
\pgfsys@useobject{currentmarker}{}%
\end{pgfscope}%
\begin{pgfscope}%
\pgfsys@transformshift{1.746182in}{4.194579in}%
\pgfsys@useobject{currentmarker}{}%
\end{pgfscope}%
\begin{pgfscope}%
\pgfsys@transformshift{1.727169in}{4.355136in}%
\pgfsys@useobject{currentmarker}{}%
\end{pgfscope}%
\begin{pgfscope}%
\pgfsys@transformshift{1.708861in}{4.417808in}%
\pgfsys@useobject{currentmarker}{}%
\end{pgfscope}%
\begin{pgfscope}%
\pgfsys@transformshift{1.688674in}{4.332830in}%
\pgfsys@useobject{currentmarker}{}%
\end{pgfscope}%
\begin{pgfscope}%
\pgfsys@transformshift{1.669191in}{4.154340in}%
\pgfsys@useobject{currentmarker}{}%
\end{pgfscope}%
\begin{pgfscope}%
\pgfsys@transformshift{1.650413in}{4.097783in}%
\pgfsys@useobject{currentmarker}{}%
\end{pgfscope}%
\begin{pgfscope}%
\pgfsys@transformshift{1.631399in}{4.053346in}%
\pgfsys@useobject{currentmarker}{}%
\end{pgfscope}%
\begin{pgfscope}%
\pgfsys@transformshift{1.609335in}{4.040191in}%
\pgfsys@useobject{currentmarker}{}%
\end{pgfscope}%
\begin{pgfscope}%
\pgfsys@transformshift{1.591964in}{4.035473in}%
\pgfsys@useobject{currentmarker}{}%
\end{pgfscope}%
\begin{pgfscope}%
\pgfsys@transformshift{1.572717in}{4.035848in}%
\pgfsys@useobject{currentmarker}{}%
\end{pgfscope}%
\begin{pgfscope}%
\pgfsys@transformshift{1.554878in}{4.041946in}%
\pgfsys@useobject{currentmarker}{}%
\end{pgfscope}%
\begin{pgfscope}%
\pgfsys@transformshift{1.532578in}{4.065586in}%
\pgfsys@useobject{currentmarker}{}%
\end{pgfscope}%
\begin{pgfscope}%
\pgfsys@transformshift{1.514505in}{4.136748in}%
\pgfsys@useobject{currentmarker}{}%
\end{pgfscope}%
\begin{pgfscope}%
\pgfsys@transformshift{1.496900in}{4.200711in}%
\pgfsys@useobject{currentmarker}{}%
\end{pgfscope}%
\begin{pgfscope}%
\pgfsys@transformshift{1.475070in}{4.385333in}%
\pgfsys@useobject{currentmarker}{}%
\end{pgfscope}%
\begin{pgfscope}%
\pgfsys@transformshift{1.458168in}{4.424285in}%
\pgfsys@useobject{currentmarker}{}%
\end{pgfscope}%
\begin{pgfscope}%
\pgfsys@transformshift{1.440565in}{4.385710in}%
\pgfsys@useobject{currentmarker}{}%
\end{pgfscope}%
\begin{pgfscope}%
\pgfsys@transformshift{1.418030in}{4.162448in}%
\pgfsys@useobject{currentmarker}{}%
\end{pgfscope}%
\begin{pgfscope}%
\pgfsys@transformshift{1.399956in}{4.087766in}%
\pgfsys@useobject{currentmarker}{}%
\end{pgfscope}%
\begin{pgfscope}%
\pgfsys@transformshift{1.381178in}{4.052725in}%
\pgfsys@useobject{currentmarker}{}%
\end{pgfscope}%
\begin{pgfscope}%
\pgfsys@transformshift{1.359582in}{4.038996in}%
\pgfsys@useobject{currentmarker}{}%
\end{pgfscope}%
\begin{pgfscope}%
\pgfsys@transformshift{1.341274in}{4.036675in}%
\pgfsys@useobject{currentmarker}{}%
\end{pgfscope}%
\begin{pgfscope}%
\pgfsys@transformshift{1.322495in}{4.035532in}%
\pgfsys@useobject{currentmarker}{}%
\end{pgfscope}%
\begin{pgfscope}%
\pgfsys@transformshift{1.304421in}{4.041585in}%
\pgfsys@useobject{currentmarker}{}%
\end{pgfscope}%
\begin{pgfscope}%
\pgfsys@transformshift{1.283062in}{4.035482in}%
\pgfsys@useobject{currentmarker}{}%
\end{pgfscope}%
\begin{pgfscope}%
\pgfsys@transformshift{1.265926in}{4.036756in}%
\pgfsys@useobject{currentmarker}{}%
\end{pgfscope}%
\begin{pgfscope}%
\pgfsys@transformshift{1.243861in}{4.047204in}%
\pgfsys@useobject{currentmarker}{}%
\end{pgfscope}%
\begin{pgfscope}%
\pgfsys@transformshift{1.227665in}{4.053985in}%
\pgfsys@useobject{currentmarker}{}%
\end{pgfscope}%
\begin{pgfscope}%
\pgfsys@transformshift{1.205600in}{4.103853in}%
\pgfsys@useobject{currentmarker}{}%
\end{pgfscope}%
\begin{pgfscope}%
\pgfsys@transformshift{1.190344in}{4.190845in}%
\pgfsys@useobject{currentmarker}{}%
\end{pgfscope}%
\begin{pgfscope}%
\pgfsys@transformshift{1.171330in}{4.341079in}%
\pgfsys@useobject{currentmarker}{}%
\end{pgfscope}%
\begin{pgfscope}%
\pgfsys@transformshift{1.149969in}{4.428558in}%
\pgfsys@useobject{currentmarker}{}%
\end{pgfscope}%
\begin{pgfscope}%
\pgfsys@transformshift{1.131427in}{4.422472in}%
\pgfsys@useobject{currentmarker}{}%
\end{pgfscope}%
\begin{pgfscope}%
\pgfsys@transformshift{1.109362in}{4.212214in}%
\pgfsys@useobject{currentmarker}{}%
\end{pgfscope}%
\begin{pgfscope}%
\pgfsys@transformshift{1.087766in}{4.090401in}%
\pgfsys@useobject{currentmarker}{}%
\end{pgfscope}%
\begin{pgfscope}%
\pgfsys@transformshift{1.072744in}{4.060390in}%
\pgfsys@useobject{currentmarker}{}%
\end{pgfscope}%
\begin{pgfscope}%
\pgfsys@transformshift{1.053965in}{4.042674in}%
\pgfsys@useobject{currentmarker}{}%
\end{pgfscope}%
\begin{pgfscope}%
\pgfsys@transformshift{1.033309in}{4.036249in}%
\pgfsys@useobject{currentmarker}{}%
\end{pgfscope}%
\begin{pgfscope}%
\pgfsys@transformshift{1.016173in}{4.036685in}%
\pgfsys@useobject{currentmarker}{}%
\end{pgfscope}%
\begin{pgfscope}%
\pgfsys@transformshift{0.996222in}{4.043909in}%
\pgfsys@useobject{currentmarker}{}%
\end{pgfscope}%
\begin{pgfscope}%
\pgfsys@transformshift{0.977209in}{4.056598in}%
\pgfsys@useobject{currentmarker}{}%
\end{pgfscope}%
\begin{pgfscope}%
\pgfsys@transformshift{0.955847in}{4.095682in}%
\pgfsys@useobject{currentmarker}{}%
\end{pgfscope}%
\begin{pgfscope}%
\pgfsys@transformshift{0.938479in}{4.209405in}%
\pgfsys@useobject{currentmarker}{}%
\end{pgfscope}%
\begin{pgfscope}%
\pgfsys@transformshift{0.919700in}{4.350302in}%
\pgfsys@useobject{currentmarker}{}%
\end{pgfscope}%
\begin{pgfscope}%
\pgfsys@transformshift{0.898575in}{4.436426in}%
\pgfsys@useobject{currentmarker}{}%
\end{pgfscope}%
\begin{pgfscope}%
\pgfsys@transformshift{0.880970in}{4.441451in}%
\pgfsys@useobject{currentmarker}{}%
\end{pgfscope}%
\begin{pgfscope}%
\pgfsys@transformshift{0.863600in}{4.354348in}%
\pgfsys@useobject{currentmarker}{}%
\end{pgfscope}%
\begin{pgfscope}%
\pgfsys@transformshift{0.842475in}{4.197481in}%
\pgfsys@useobject{currentmarker}{}%
\end{pgfscope}%
\begin{pgfscope}%
\pgfsys@transformshift{0.823696in}{4.091580in}%
\pgfsys@useobject{currentmarker}{}%
\end{pgfscope}%
\begin{pgfscope}%
\pgfsys@transformshift{0.805621in}{4.061026in}%
\pgfsys@useobject{currentmarker}{}%
\end{pgfscope}%
\begin{pgfscope}%
\pgfsys@transformshift{0.784496in}{4.046250in}%
\pgfsys@useobject{currentmarker}{}%
\end{pgfscope}%
\begin{pgfscope}%
\pgfsys@transformshift{0.765014in}{4.037956in}%
\pgfsys@useobject{currentmarker}{}%
\end{pgfscope}%
\begin{pgfscope}%
\pgfsys@transformshift{0.747643in}{4.035948in}%
\pgfsys@useobject{currentmarker}{}%
\end{pgfscope}%
\begin{pgfscope}%
\pgfsys@transformshift{0.726987in}{4.039526in}%
\pgfsys@useobject{currentmarker}{}%
\end{pgfscope}%
\begin{pgfscope}%
\pgfsys@transformshift{0.709148in}{4.043876in}%
\pgfsys@useobject{currentmarker}{}%
\end{pgfscope}%
\begin{pgfscope}%
\pgfsys@transformshift{0.688023in}{4.036405in}%
\pgfsys@useobject{currentmarker}{}%
\end{pgfscope}%
\begin{pgfscope}%
\pgfsys@transformshift{0.667835in}{4.039682in}%
\pgfsys@useobject{currentmarker}{}%
\end{pgfscope}%
\begin{pgfscope}%
\pgfsys@transformshift{0.650231in}{4.049588in}%
\pgfsys@useobject{currentmarker}{}%
\end{pgfscope}%
\begin{pgfscope}%
\pgfsys@transformshift{0.650934in}{4.049135in}%
\pgfsys@useobject{currentmarker}{}%
\end{pgfscope}%
\begin{pgfscope}%
\pgfsys@transformshift{0.657742in}{4.042191in}%
\pgfsys@useobject{currentmarker}{}%
\end{pgfscope}%
\begin{pgfscope}%
\pgfsys@transformshift{0.677929in}{4.035799in}%
\pgfsys@useobject{currentmarker}{}%
\end{pgfscope}%
\begin{pgfscope}%
\pgfsys@transformshift{0.694594in}{4.042472in}%
\pgfsys@useobject{currentmarker}{}%
\end{pgfscope}%
\begin{pgfscope}%
\pgfsys@transformshift{0.715016in}{4.076362in}%
\pgfsys@useobject{currentmarker}{}%
\end{pgfscope}%
\begin{pgfscope}%
\pgfsys@transformshift{0.734969in}{4.194394in}%
\pgfsys@useobject{currentmarker}{}%
\end{pgfscope}%
\begin{pgfscope}%
\pgfsys@transformshift{0.753511in}{4.428735in}%
\pgfsys@useobject{currentmarker}{}%
\end{pgfscope}%
\begin{pgfscope}%
\pgfsys@transformshift{0.771351in}{4.435792in}%
\pgfsys@useobject{currentmarker}{}%
\end{pgfscope}%
\begin{pgfscope}%
\pgfsys@transformshift{0.792241in}{4.256612in}%
\pgfsys@useobject{currentmarker}{}%
\end{pgfscope}%
\begin{pgfscope}%
\pgfsys@transformshift{0.810786in}{4.097022in}%
\pgfsys@useobject{currentmarker}{}%
\end{pgfscope}%
\begin{pgfscope}%
\pgfsys@transformshift{0.827685in}{4.053061in}%
\pgfsys@useobject{currentmarker}{}%
\end{pgfscope}%
\begin{pgfscope}%
\pgfsys@transformshift{0.849047in}{4.036668in}%
\pgfsys@useobject{currentmarker}{}%
\end{pgfscope}%
\begin{pgfscope}%
\pgfsys@transformshift{0.871814in}{4.039410in}%
\pgfsys@useobject{currentmarker}{}%
\end{pgfscope}%
\begin{pgfscope}%
\pgfsys@transformshift{0.888482in}{4.054328in}%
\pgfsys@useobject{currentmarker}{}%
\end{pgfscope}%
\begin{pgfscope}%
\pgfsys@transformshift{0.905147in}{4.098085in}%
\pgfsys@useobject{currentmarker}{}%
\end{pgfscope}%
\begin{pgfscope}%
\pgfsys@transformshift{0.927446in}{4.321150in}%
\pgfsys@useobject{currentmarker}{}%
\end{pgfscope}%
\begin{pgfscope}%
\pgfsys@transformshift{0.946225in}{4.439998in}%
\pgfsys@useobject{currentmarker}{}%
\end{pgfscope}%
\begin{pgfscope}%
\pgfsys@transformshift{0.964298in}{4.350136in}%
\pgfsys@useobject{currentmarker}{}%
\end{pgfscope}%
\begin{pgfscope}%
\pgfsys@transformshift{0.983077in}{4.143227in}%
\pgfsys@useobject{currentmarker}{}%
\end{pgfscope}%
\begin{pgfscope}%
\pgfsys@transformshift{1.000682in}{4.063349in}%
\pgfsys@useobject{currentmarker}{}%
\end{pgfscope}%
\begin{pgfscope}%
\pgfsys@transformshift{1.018521in}{4.037853in}%
\pgfsys@useobject{currentmarker}{}%
\end{pgfscope}%
\begin{pgfscope}%
\pgfsys@transformshift{1.040586in}{4.035506in}%
\pgfsys@useobject{currentmarker}{}%
\end{pgfscope}%
\begin{pgfscope}%
\pgfsys@transformshift{1.059599in}{4.036684in}%
\pgfsys@useobject{currentmarker}{}%
\end{pgfscope}%
\begin{pgfscope}%
\pgfsys@transformshift{1.077907in}{4.036820in}%
\pgfsys@useobject{currentmarker}{}%
\end{pgfscope}%
\begin{pgfscope}%
\pgfsys@transformshift{1.100677in}{4.051447in}%
\pgfsys@useobject{currentmarker}{}%
\end{pgfscope}%
\begin{pgfscope}%
\pgfsys@transformshift{1.119690in}{4.099000in}%
\pgfsys@useobject{currentmarker}{}%
\end{pgfscope}%
\begin{pgfscope}%
\pgfsys@transformshift{1.137529in}{4.274558in}%
\pgfsys@useobject{currentmarker}{}%
\end{pgfscope}%
\begin{pgfscope}%
\pgfsys@transformshift{1.155134in}{4.428608in}%
\pgfsys@useobject{currentmarker}{}%
\end{pgfscope}%
\begin{pgfscope}%
\pgfsys@transformshift{1.173911in}{4.369449in}%
\pgfsys@useobject{currentmarker}{}%
\end{pgfscope}%
\begin{pgfscope}%
\pgfsys@transformshift{1.195507in}{4.149896in}%
\pgfsys@useobject{currentmarker}{}%
\end{pgfscope}%
\begin{pgfscope}%
\pgfsys@transformshift{1.214520in}{4.063439in}%
\pgfsys@useobject{currentmarker}{}%
\end{pgfscope}%
\begin{pgfscope}%
\pgfsys@transformshift{1.232594in}{4.040825in}%
\pgfsys@useobject{currentmarker}{}%
\end{pgfscope}%
\begin{pgfscope}%
\pgfsys@transformshift{1.257475in}{4.035279in}%
\pgfsys@useobject{currentmarker}{}%
\end{pgfscope}%
\begin{pgfscope}%
\pgfsys@transformshift{1.274611in}{4.039752in}%
\pgfsys@useobject{currentmarker}{}%
\end{pgfscope}%
\begin{pgfscope}%
\pgfsys@transformshift{1.294798in}{4.062773in}%
\pgfsys@useobject{currentmarker}{}%
\end{pgfscope}%
\begin{pgfscope}%
\pgfsys@transformshift{1.310758in}{4.124741in}%
\pgfsys@useobject{currentmarker}{}%
\end{pgfscope}%
\begin{pgfscope}%
\pgfsys@transformshift{1.330477in}{4.353662in}%
\pgfsys@useobject{currentmarker}{}%
\end{pgfscope}%
\begin{pgfscope}%
\pgfsys@transformshift{1.348551in}{4.422388in}%
\pgfsys@useobject{currentmarker}{}%
\end{pgfscope}%
\begin{pgfscope}%
\pgfsys@transformshift{1.367329in}{4.326060in}%
\pgfsys@useobject{currentmarker}{}%
\end{pgfscope}%
\begin{pgfscope}%
\pgfsys@transformshift{1.386343in}{4.147228in}%
\pgfsys@useobject{currentmarker}{}%
\end{pgfscope}%
\begin{pgfscope}%
\pgfsys@transformshift{1.404651in}{4.060158in}%
\pgfsys@useobject{currentmarker}{}%
\end{pgfscope}%
\begin{pgfscope}%
\pgfsys@transformshift{1.426950in}{4.038537in}%
\pgfsys@useobject{currentmarker}{}%
\end{pgfscope}%
\begin{pgfscope}%
\pgfsys@transformshift{1.445494in}{4.034887in}%
\pgfsys@useobject{currentmarker}{}%
\end{pgfscope}%
\begin{pgfscope}%
\pgfsys@transformshift{1.463802in}{4.038852in}%
\pgfsys@useobject{currentmarker}{}%
\end{pgfscope}%
\begin{pgfscope}%
\pgfsys@transformshift{1.482581in}{4.051219in}%
\pgfsys@useobject{currentmarker}{}%
\end{pgfscope}%
\begin{pgfscope}%
\pgfsys@transformshift{1.504411in}{4.115163in}%
\pgfsys@useobject{currentmarker}{}%
\end{pgfscope}%
\begin{pgfscope}%
\pgfsys@transformshift{1.520373in}{4.223846in}%
\pgfsys@useobject{currentmarker}{}%
\end{pgfscope}%
\begin{pgfscope}%
\pgfsys@transformshift{1.538681in}{4.414903in}%
\pgfsys@useobject{currentmarker}{}%
\end{pgfscope}%
\begin{pgfscope}%
\pgfsys@transformshift{1.557929in}{4.375785in}%
\pgfsys@useobject{currentmarker}{}%
\end{pgfscope}%
\begin{pgfscope}%
\pgfsys@transformshift{1.579290in}{4.200704in}%
\pgfsys@useobject{currentmarker}{}%
\end{pgfscope}%
\begin{pgfscope}%
\pgfsys@transformshift{1.598772in}{4.108856in}%
\pgfsys@useobject{currentmarker}{}%
\end{pgfscope}%
\begin{pgfscope}%
\pgfsys@transformshift{1.616846in}{4.052903in}%
\pgfsys@useobject{currentmarker}{}%
\end{pgfscope}%
\begin{pgfscope}%
\pgfsys@transformshift{1.635859in}{4.037774in}%
\pgfsys@useobject{currentmarker}{}%
\end{pgfscope}%
\begin{pgfscope}%
\pgfsys@transformshift{1.655576in}{4.034663in}%
\pgfsys@useobject{currentmarker}{}%
\end{pgfscope}%
\begin{pgfscope}%
\pgfsys@transformshift{1.674355in}{4.038976in}%
\pgfsys@useobject{currentmarker}{}%
\end{pgfscope}%
\begin{pgfscope}%
\pgfsys@transformshift{1.698533in}{4.058599in}%
\pgfsys@useobject{currentmarker}{}%
\end{pgfscope}%
\begin{pgfscope}%
\pgfsys@transformshift{1.713084in}{4.091410in}%
\pgfsys@useobject{currentmarker}{}%
\end{pgfscope}%
\begin{pgfscope}%
\pgfsys@transformshift{1.731394in}{4.223958in}%
\pgfsys@useobject{currentmarker}{}%
\end{pgfscope}%
\begin{pgfscope}%
\pgfsys@transformshift{1.751816in}{4.407873in}%
\pgfsys@useobject{currentmarker}{}%
\end{pgfscope}%
\begin{pgfscope}%
\pgfsys@transformshift{1.770124in}{4.403120in}%
\pgfsys@useobject{currentmarker}{}%
\end{pgfscope}%
\begin{pgfscope}%
\pgfsys@transformshift{1.791015in}{4.225918in}%
\pgfsys@useobject{currentmarker}{}%
\end{pgfscope}%
\begin{pgfscope}%
\pgfsys@transformshift{1.809325in}{4.103498in}%
\pgfsys@useobject{currentmarker}{}%
\end{pgfscope}%
\begin{pgfscope}%
\pgfsys@transformshift{1.830215in}{4.058669in}%
\pgfsys@useobject{currentmarker}{}%
\end{pgfscope}%
\begin{pgfscope}%
\pgfsys@transformshift{1.849229in}{4.040211in}%
\pgfsys@useobject{currentmarker}{}%
\end{pgfscope}%
\begin{pgfscope}%
\pgfsys@transformshift{1.866364in}{4.035142in}%
\pgfsys@useobject{currentmarker}{}%
\end{pgfscope}%
\begin{pgfscope}%
\pgfsys@transformshift{1.888663in}{4.036627in}%
\pgfsys@useobject{currentmarker}{}%
\end{pgfscope}%
\begin{pgfscope}%
\pgfsys@transformshift{1.905563in}{4.045021in}%
\pgfsys@useobject{currentmarker}{}%
\end{pgfscope}%
\begin{pgfscope}%
\pgfsys@transformshift{1.923168in}{4.073977in}%
\pgfsys@useobject{currentmarker}{}%
\end{pgfscope}%
\begin{pgfscope}%
\pgfsys@transformshift{1.944529in}{4.185212in}%
\pgfsys@useobject{currentmarker}{}%
\end{pgfscope}%
\begin{pgfscope}%
\pgfsys@transformshift{1.962603in}{4.329638in}%
\pgfsys@useobject{currentmarker}{}%
\end{pgfscope}%
\begin{pgfscope}%
\pgfsys@transformshift{1.986545in}{4.412755in}%
\pgfsys@useobject{currentmarker}{}%
\end{pgfscope}%
\begin{pgfscope}%
\pgfsys@transformshift{2.001098in}{4.356668in}%
\pgfsys@useobject{currentmarker}{}%
\end{pgfscope}%
\begin{pgfscope}%
\pgfsys@transformshift{2.021989in}{4.170799in}%
\pgfsys@useobject{currentmarker}{}%
\end{pgfscope}%
\begin{pgfscope}%
\pgfsys@transformshift{2.040299in}{4.069749in}%
\pgfsys@useobject{currentmarker}{}%
\end{pgfscope}%
\begin{pgfscope}%
\pgfsys@transformshift{2.059781in}{4.041929in}%
\pgfsys@useobject{currentmarker}{}%
\end{pgfscope}%
\begin{pgfscope}%
\pgfsys@transformshift{2.076915in}{4.035387in}%
\pgfsys@useobject{currentmarker}{}%
\end{pgfscope}%
\begin{pgfscope}%
\pgfsys@transformshift{2.098745in}{4.035190in}%
\pgfsys@useobject{currentmarker}{}%
\end{pgfscope}%
\begin{pgfscope}%
\pgfsys@transformshift{2.117993in}{4.040154in}%
\pgfsys@useobject{currentmarker}{}%
\end{pgfscope}%
\begin{pgfscope}%
\pgfsys@transformshift{2.135598in}{4.055700in}%
\pgfsys@useobject{currentmarker}{}%
\end{pgfscope}%
\begin{pgfscope}%
\pgfsys@transformshift{2.153673in}{4.110259in}%
\pgfsys@useobject{currentmarker}{}%
\end{pgfscope}%
\begin{pgfscope}%
\pgfsys@transformshift{2.174798in}{4.074157in}%
\pgfsys@useobject{currentmarker}{}%
\end{pgfscope}%
\begin{pgfscope}%
\pgfsys@transformshift{2.195454in}{4.044169in}%
\pgfsys@useobject{currentmarker}{}%
\end{pgfscope}%
\begin{pgfscope}%
\pgfsys@transformshift{2.213762in}{4.036018in}%
\pgfsys@useobject{currentmarker}{}%
\end{pgfscope}%
\begin{pgfscope}%
\pgfsys@transformshift{2.230898in}{4.035097in}%
\pgfsys@useobject{currentmarker}{}%
\end{pgfscope}%
\begin{pgfscope}%
\pgfsys@transformshift{2.253666in}{4.041219in}%
\pgfsys@useobject{currentmarker}{}%
\end{pgfscope}%
\begin{pgfscope}%
\pgfsys@transformshift{2.270333in}{4.060038in}%
\pgfsys@useobject{currentmarker}{}%
\end{pgfscope}%
\begin{pgfscope}%
\pgfsys@transformshift{2.287938in}{4.130591in}%
\pgfsys@useobject{currentmarker}{}%
\end{pgfscope}%
\begin{pgfscope}%
\pgfsys@transformshift{2.310003in}{4.375804in}%
\pgfsys@useobject{currentmarker}{}%
\end{pgfscope}%
\begin{pgfscope}%
\pgfsys@transformshift{2.327607in}{4.412180in}%
\pgfsys@useobject{currentmarker}{}%
\end{pgfscope}%
\begin{pgfscope}%
\pgfsys@transformshift{2.347558in}{4.266334in}%
\pgfsys@useobject{currentmarker}{}%
\end{pgfscope}%
\begin{pgfscope}%
\pgfsys@transformshift{2.365632in}{4.109331in}%
\pgfsys@useobject{currentmarker}{}%
\end{pgfscope}%
\begin{pgfscope}%
\pgfsys@transformshift{2.387462in}{4.048075in}%
\pgfsys@useobject{currentmarker}{}%
\end{pgfscope}%
\begin{pgfscope}%
\pgfsys@transformshift{2.405067in}{4.037010in}%
\pgfsys@useobject{currentmarker}{}%
\end{pgfscope}%
\begin{pgfscope}%
\pgfsys@transformshift{2.424315in}{4.034483in}%
\pgfsys@useobject{currentmarker}{}%
\end{pgfscope}%
\begin{pgfscope}%
\pgfsys@transformshift{2.443797in}{4.037322in}%
\pgfsys@useobject{currentmarker}{}%
\end{pgfscope}%
\begin{pgfscope}%
\pgfsys@transformshift{2.461402in}{4.042580in}%
\pgfsys@useobject{currentmarker}{}%
\end{pgfscope}%
\begin{pgfscope}%
\pgfsys@transformshift{2.483937in}{4.075601in}%
\pgfsys@useobject{currentmarker}{}%
\end{pgfscope}%
\begin{pgfscope}%
\pgfsys@transformshift{2.500602in}{4.175166in}%
\pgfsys@useobject{currentmarker}{}%
\end{pgfscope}%
\begin{pgfscope}%
\pgfsys@transformshift{2.518676in}{4.388548in}%
\pgfsys@useobject{currentmarker}{}%
\end{pgfscope}%
\begin{pgfscope}%
\pgfsys@transformshift{2.540740in}{4.400856in}%
\pgfsys@useobject{currentmarker}{}%
\end{pgfscope}%
\begin{pgfscope}%
\pgfsys@transformshift{2.558345in}{4.269341in}%
\pgfsys@useobject{currentmarker}{}%
\end{pgfscope}%
\begin{pgfscope}%
\pgfsys@transformshift{2.579236in}{4.101554in}%
\pgfsys@useobject{currentmarker}{}%
\end{pgfscope}%
\begin{pgfscope}%
\pgfsys@transformshift{2.597077in}{4.051284in}%
\pgfsys@useobject{currentmarker}{}%
\end{pgfscope}%
\begin{pgfscope}%
\pgfsys@transformshift{2.617968in}{4.039565in}%
\pgfsys@useobject{currentmarker}{}%
\end{pgfscope}%
\begin{pgfscope}%
\pgfsys@transformshift{2.636276in}{4.035365in}%
\pgfsys@useobject{currentmarker}{}%
\end{pgfscope}%
\begin{pgfscope}%
\pgfsys@transformshift{2.654115in}{4.034806in}%
\pgfsys@useobject{currentmarker}{}%
\end{pgfscope}%
\begin{pgfscope}%
\pgfsys@transformshift{2.671251in}{4.038434in}%
\pgfsys@useobject{currentmarker}{}%
\end{pgfscope}%
\begin{pgfscope}%
\pgfsys@transformshift{2.694019in}{4.052378in}%
\pgfsys@useobject{currentmarker}{}%
\end{pgfscope}%
\begin{pgfscope}%
\pgfsys@transformshift{2.713737in}{4.083942in}%
\pgfsys@useobject{currentmarker}{}%
\end{pgfscope}%
\begin{pgfscope}%
\pgfsys@transformshift{2.732514in}{4.128218in}%
\pgfsys@useobject{currentmarker}{}%
\end{pgfscope}%
\begin{pgfscope}%
\pgfsys@transformshift{2.750355in}{4.322900in}%
\pgfsys@useobject{currentmarker}{}%
\end{pgfscope}%
\begin{pgfscope}%
\pgfsys@transformshift{2.768663in}{4.411416in}%
\pgfsys@useobject{currentmarker}{}%
\end{pgfscope}%
\begin{pgfscope}%
\pgfsys@transformshift{2.789788in}{4.305548in}%
\pgfsys@useobject{currentmarker}{}%
\end{pgfscope}%
\begin{pgfscope}%
\pgfsys@transformshift{2.808098in}{4.130497in}%
\pgfsys@useobject{currentmarker}{}%
\end{pgfscope}%
\begin{pgfscope}%
\pgfsys@transformshift{2.828754in}{4.056210in}%
\pgfsys@useobject{currentmarker}{}%
\end{pgfscope}%
\begin{pgfscope}%
\pgfsys@transformshift{2.847768in}{4.042718in}%
\pgfsys@useobject{currentmarker}{}%
\end{pgfscope}%
\begin{pgfscope}%
\pgfsys@transformshift{2.864667in}{4.036022in}%
\pgfsys@useobject{currentmarker}{}%
\end{pgfscope}%
\begin{pgfscope}%
\pgfsys@transformshift{2.886029in}{4.035163in}%
\pgfsys@useobject{currentmarker}{}%
\end{pgfscope}%
\begin{pgfscope}%
\pgfsys@transformshift{2.904337in}{4.039192in}%
\pgfsys@useobject{currentmarker}{}%
\end{pgfscope}%
\begin{pgfscope}%
\pgfsys@transformshift{2.925227in}{4.056915in}%
\pgfsys@useobject{currentmarker}{}%
\end{pgfscope}%
\begin{pgfscope}%
\pgfsys@transformshift{2.942832in}{4.110918in}%
\pgfsys@useobject{currentmarker}{}%
\end{pgfscope}%
\begin{pgfscope}%
\pgfsys@transformshift{2.964193in}{4.289140in}%
\pgfsys@useobject{currentmarker}{}%
\end{pgfscope}%
\begin{pgfscope}%
\pgfsys@transformshift{2.982501in}{4.406581in}%
\pgfsys@useobject{currentmarker}{}%
\end{pgfscope}%
\begin{pgfscope}%
\pgfsys@transformshift{2.999872in}{4.388582in}%
\pgfsys@useobject{currentmarker}{}%
\end{pgfscope}%
\begin{pgfscope}%
\pgfsys@transformshift{3.020059in}{4.282554in}%
\pgfsys@useobject{currentmarker}{}%
\end{pgfscope}%
\begin{pgfscope}%
\pgfsys@transformshift{3.037193in}{4.129729in}%
\pgfsys@useobject{currentmarker}{}%
\end{pgfscope}%
\begin{pgfscope}%
\pgfsys@transformshift{3.056912in}{4.059623in}%
\pgfsys@useobject{currentmarker}{}%
\end{pgfscope}%
\begin{pgfscope}%
\pgfsys@transformshift{3.078036in}{4.040849in}%
\pgfsys@useobject{currentmarker}{}%
\end{pgfscope}%
\begin{pgfscope}%
\pgfsys@transformshift{3.098927in}{4.035245in}%
\pgfsys@useobject{currentmarker}{}%
\end{pgfscope}%
\begin{pgfscope}%
\pgfsys@transformshift{3.116532in}{4.035321in}%
\pgfsys@useobject{currentmarker}{}%
\end{pgfscope}%
\begin{pgfscope}%
\pgfsys@transformshift{3.135311in}{4.039267in}%
\pgfsys@useobject{currentmarker}{}%
\end{pgfscope}%
\begin{pgfscope}%
\pgfsys@transformshift{3.154324in}{4.050563in}%
\pgfsys@useobject{currentmarker}{}%
\end{pgfscope}%
\begin{pgfscope}%
\pgfsys@transformshift{3.175683in}{4.091192in}%
\pgfsys@useobject{currentmarker}{}%
\end{pgfscope}%
\begin{pgfscope}%
\pgfsys@transformshift{3.191880in}{4.196815in}%
\pgfsys@useobject{currentmarker}{}%
\end{pgfscope}%
\begin{pgfscope}%
\pgfsys@transformshift{3.213241in}{4.395866in}%
\pgfsys@useobject{currentmarker}{}%
\end{pgfscope}%
\begin{pgfscope}%
\pgfsys@transformshift{3.231549in}{4.412381in}%
\pgfsys@useobject{currentmarker}{}%
\end{pgfscope}%
\begin{pgfscope}%
\pgfsys@transformshift{3.248685in}{4.333063in}%
\pgfsys@useobject{currentmarker}{}%
\end{pgfscope}%
\begin{pgfscope}%
\pgfsys@transformshift{3.270044in}{4.189992in}%
\pgfsys@useobject{currentmarker}{}%
\end{pgfscope}%
\begin{pgfscope}%
\pgfsys@transformshift{3.288354in}{4.084001in}%
\pgfsys@useobject{currentmarker}{}%
\end{pgfscope}%
\begin{pgfscope}%
\pgfsys@transformshift{3.309948in}{4.065403in}%
\pgfsys@useobject{currentmarker}{}%
\end{pgfscope}%
\begin{pgfscope}%
\pgfsys@transformshift{3.328024in}{4.046613in}%
\pgfsys@useobject{currentmarker}{}%
\end{pgfscope}%
\begin{pgfscope}%
\pgfsys@transformshift{3.345392in}{4.037006in}%
\pgfsys@useobject{currentmarker}{}%
\end{pgfscope}%
\begin{pgfscope}%
\pgfsys@transformshift{3.366988in}{4.035215in}%
\pgfsys@useobject{currentmarker}{}%
\end{pgfscope}%
\begin{pgfscope}%
\pgfsys@transformshift{3.384593in}{4.038026in}%
\pgfsys@useobject{currentmarker}{}%
\end{pgfscope}%
\begin{pgfscope}%
\pgfsys@transformshift{3.405718in}{4.052465in}%
\pgfsys@useobject{currentmarker}{}%
\end{pgfscope}%
\begin{pgfscope}%
\pgfsys@transformshift{3.423088in}{4.077543in}%
\pgfsys@useobject{currentmarker}{}%
\end{pgfscope}%
\begin{pgfscope}%
\pgfsys@transformshift{3.441633in}{4.152650in}%
\pgfsys@useobject{currentmarker}{}%
\end{pgfscope}%
\begin{pgfscope}%
\pgfsys@transformshift{3.463226in}{4.378778in}%
\pgfsys@useobject{currentmarker}{}%
\end{pgfscope}%
\begin{pgfscope}%
\pgfsys@transformshift{3.479894in}{4.423521in}%
\pgfsys@useobject{currentmarker}{}%
\end{pgfscope}%
\begin{pgfscope}%
\pgfsys@transformshift{3.499141in}{4.384823in}%
\pgfsys@useobject{currentmarker}{}%
\end{pgfscope}%
\begin{pgfscope}%
\pgfsys@transformshift{3.516746in}{4.231565in}%
\pgfsys@useobject{currentmarker}{}%
\end{pgfscope}%
\begin{pgfscope}%
\pgfsys@transformshift{3.541391in}{4.110346in}%
\pgfsys@useobject{currentmarker}{}%
\end{pgfscope}%
\begin{pgfscope}%
\pgfsys@transformshift{3.558527in}{4.061765in}%
\pgfsys@useobject{currentmarker}{}%
\end{pgfscope}%
\begin{pgfscope}%
\pgfsys@transformshift{3.577072in}{4.043901in}%
\pgfsys@useobject{currentmarker}{}%
\end{pgfscope}%
\begin{pgfscope}%
\pgfsys@transformshift{3.594440in}{4.039430in}%
\pgfsys@useobject{currentmarker}{}%
\end{pgfscope}%
\begin{pgfscope}%
\pgfsys@transformshift{3.616975in}{4.036099in}%
\pgfsys@useobject{currentmarker}{}%
\end{pgfscope}%
\begin{pgfscope}%
\pgfsys@transformshift{3.634580in}{4.035332in}%
\pgfsys@useobject{currentmarker}{}%
\end{pgfscope}%
\begin{pgfscope}%
\pgfsys@transformshift{3.658053in}{4.043188in}%
\pgfsys@useobject{currentmarker}{}%
\end{pgfscope}%
\begin{pgfscope}%
\pgfsys@transformshift{3.671198in}{4.055555in}%
\pgfsys@useobject{currentmarker}{}%
\end{pgfscope}%
\begin{pgfscope}%
\pgfsys@transformshift{3.693732in}{4.108246in}%
\pgfsys@useobject{currentmarker}{}%
\end{pgfscope}%
\begin{pgfscope}%
\pgfsys@transformshift{3.712979in}{4.089869in}%
\pgfsys@useobject{currentmarker}{}%
\end{pgfscope}%
\begin{pgfscope}%
\pgfsys@transformshift{3.728941in}{4.191905in}%
\pgfsys@useobject{currentmarker}{}%
\end{pgfscope}%
\begin{pgfscope}%
\pgfsys@transformshift{3.751475in}{4.402094in}%
\pgfsys@useobject{currentmarker}{}%
\end{pgfscope}%
\begin{pgfscope}%
\pgfsys@transformshift{3.768845in}{4.432250in}%
\pgfsys@useobject{currentmarker}{}%
\end{pgfscope}%
\begin{pgfscope}%
\pgfsys@transformshift{3.787153in}{4.403605in}%
\pgfsys@useobject{currentmarker}{}%
\end{pgfscope}%
\begin{pgfscope}%
\pgfsys@transformshift{3.807341in}{4.300921in}%
\pgfsys@useobject{currentmarker}{}%
\end{pgfscope}%
\begin{pgfscope}%
\pgfsys@transformshift{3.825180in}{4.188535in}%
\pgfsys@useobject{currentmarker}{}%
\end{pgfscope}%
\begin{pgfscope}%
\pgfsys@transformshift{3.846776in}{4.081111in}%
\pgfsys@useobject{currentmarker}{}%
\end{pgfscope}%
\begin{pgfscope}%
\pgfsys@transformshift{3.864146in}{4.051426in}%
\pgfsys@useobject{currentmarker}{}%
\end{pgfscope}%
\begin{pgfscope}%
\pgfsys@transformshift{3.885036in}{4.038676in}%
\pgfsys@useobject{currentmarker}{}%
\end{pgfscope}%
\begin{pgfscope}%
\pgfsys@transformshift{3.903345in}{4.035307in}%
\pgfsys@useobject{currentmarker}{}%
\end{pgfscope}%
\begin{pgfscope}%
\pgfsys@transformshift{3.921654in}{4.109055in}%
\pgfsys@useobject{currentmarker}{}%
\end{pgfscope}%
\begin{pgfscope}%
\pgfsys@transformshift{3.939963in}{4.061706in}%
\pgfsys@useobject{currentmarker}{}%
\end{pgfscope}%
\begin{pgfscope}%
\pgfsys@transformshift{3.962027in}{4.040926in}%
\pgfsys@useobject{currentmarker}{}%
\end{pgfscope}%
\begin{pgfscope}%
\pgfsys@transformshift{3.982214in}{4.035479in}%
\pgfsys@useobject{currentmarker}{}%
\end{pgfscope}%
\begin{pgfscope}%
\pgfsys@transformshift{3.997237in}{4.037619in}%
\pgfsys@useobject{currentmarker}{}%
\end{pgfscope}%
\begin{pgfscope}%
\pgfsys@transformshift{4.017893in}{4.045526in}%
\pgfsys@useobject{currentmarker}{}%
\end{pgfscope}%
\begin{pgfscope}%
\pgfsys@transformshift{4.037844in}{4.075382in}%
\pgfsys@useobject{currentmarker}{}%
\end{pgfscope}%
\end{pgfscope}%
\begin{pgfscope}%
\pgfsetrectcap%
\pgfsetmiterjoin%
\pgfsetlinewidth{0.501875pt}%
\definecolor{currentstroke}{rgb}{0.000000,0.000000,0.000000}%
\pgfsetstrokecolor{currentstroke}%
\pgfsetdash{}{0pt}%
\pgfpathmoveto{\pgfqpoint{0.444748in}{4.012575in}}%
\pgfpathlineto{\pgfqpoint{0.444748in}{4.479825in}}%
\pgfusepath{stroke}%
\end{pgfscope}%
\begin{pgfscope}%
\pgfsetrectcap%
\pgfsetmiterjoin%
\pgfsetlinewidth{0.501875pt}%
\definecolor{currentstroke}{rgb}{0.000000,0.000000,0.000000}%
\pgfsetstrokecolor{currentstroke}%
\pgfsetdash{}{0pt}%
\pgfpathmoveto{\pgfqpoint{4.676167in}{4.012575in}}%
\pgfpathlineto{\pgfqpoint{4.676167in}{4.479825in}}%
\pgfusepath{stroke}%
\end{pgfscope}%
\begin{pgfscope}%
\pgfsetrectcap%
\pgfsetmiterjoin%
\pgfsetlinewidth{0.501875pt}%
\definecolor{currentstroke}{rgb}{0.000000,0.000000,0.000000}%
\pgfsetstrokecolor{currentstroke}%
\pgfsetdash{}{0pt}%
\pgfpathmoveto{\pgfqpoint{0.444748in}{4.012575in}}%
\pgfpathlineto{\pgfqpoint{4.676167in}{4.012575in}}%
\pgfusepath{stroke}%
\end{pgfscope}%
\begin{pgfscope}%
\pgfsetrectcap%
\pgfsetmiterjoin%
\pgfsetlinewidth{0.501875pt}%
\definecolor{currentstroke}{rgb}{0.000000,0.000000,0.000000}%
\pgfsetstrokecolor{currentstroke}%
\pgfsetdash{}{0pt}%
\pgfpathmoveto{\pgfqpoint{0.444748in}{4.479825in}}%
\pgfpathlineto{\pgfqpoint{4.676167in}{4.479825in}}%
\pgfusepath{stroke}%
\end{pgfscope}%
\begin{pgfscope}%
\definecolor{textcolor}{rgb}{0.000000,0.000000,0.000000}%
\pgfsetstrokecolor{textcolor}%
\pgfsetfillcolor{textcolor}%
\pgftext[x=2.560458in,y=4.563159in,,base]{\color{textcolor}\rmfamily\fontsize{12.000000}{14.400000}\selectfont T = \qty{2.8}{\kelvin}}%
\end{pgfscope}%
\begin{pgfscope}%
\pgfsetbuttcap%
\pgfsetmiterjoin%
\definecolor{currentfill}{rgb}{1.000000,1.000000,1.000000}%
\pgfsetfillcolor{currentfill}%
\pgfsetlinewidth{0.000000pt}%
\definecolor{currentstroke}{rgb}{0.000000,0.000000,0.000000}%
\pgfsetstrokecolor{currentstroke}%
\pgfsetstrokeopacity{0.000000}%
\pgfsetdash{}{0pt}%
\pgfpathmoveto{\pgfqpoint{0.444748in}{3.117349in}}%
\pgfpathlineto{\pgfqpoint{4.676167in}{3.117349in}}%
\pgfpathlineto{\pgfqpoint{4.676167in}{3.584600in}}%
\pgfpathlineto{\pgfqpoint{0.444748in}{3.584600in}}%
\pgfpathlineto{\pgfqpoint{0.444748in}{3.117349in}}%
\pgfpathclose%
\pgfusepath{fill}%
\end{pgfscope}%
\begin{pgfscope}%
\pgfsetbuttcap%
\pgfsetroundjoin%
\definecolor{currentfill}{rgb}{0.000000,0.000000,0.000000}%
\pgfsetfillcolor{currentfill}%
\pgfsetlinewidth{0.501875pt}%
\definecolor{currentstroke}{rgb}{0.000000,0.000000,0.000000}%
\pgfsetstrokecolor{currentstroke}%
\pgfsetdash{}{0pt}%
\pgfsys@defobject{currentmarker}{\pgfqpoint{0.000000in}{0.000000in}}{\pgfqpoint{0.000000in}{0.041667in}}{%
\pgfpathmoveto{\pgfqpoint{0.000000in}{0.000000in}}%
\pgfpathlineto{\pgfqpoint{0.000000in}{0.041667in}}%
\pgfusepath{stroke,fill}%
}%
\begin{pgfscope}%
\pgfsys@transformshift{0.643886in}{3.117349in}%
\pgfsys@useobject{currentmarker}{}%
\end{pgfscope}%
\end{pgfscope}%
\begin{pgfscope}%
\pgfsetbuttcap%
\pgfsetroundjoin%
\definecolor{currentfill}{rgb}{0.000000,0.000000,0.000000}%
\pgfsetfillcolor{currentfill}%
\pgfsetlinewidth{0.501875pt}%
\definecolor{currentstroke}{rgb}{0.000000,0.000000,0.000000}%
\pgfsetstrokecolor{currentstroke}%
\pgfsetdash{}{0pt}%
\pgfsys@defobject{currentmarker}{\pgfqpoint{0.000000in}{-0.041667in}}{\pgfqpoint{0.000000in}{0.000000in}}{%
\pgfpathmoveto{\pgfqpoint{0.000000in}{0.000000in}}%
\pgfpathlineto{\pgfqpoint{0.000000in}{-0.041667in}}%
\pgfusepath{stroke,fill}%
}%
\begin{pgfscope}%
\pgfsys@transformshift{0.643886in}{3.584600in}%
\pgfsys@useobject{currentmarker}{}%
\end{pgfscope}%
\end{pgfscope}%
\begin{pgfscope}%
\pgfsetbuttcap%
\pgfsetroundjoin%
\definecolor{currentfill}{rgb}{0.000000,0.000000,0.000000}%
\pgfsetfillcolor{currentfill}%
\pgfsetlinewidth{0.501875pt}%
\definecolor{currentstroke}{rgb}{0.000000,0.000000,0.000000}%
\pgfsetstrokecolor{currentstroke}%
\pgfsetdash{}{0pt}%
\pgfsys@defobject{currentmarker}{\pgfqpoint{0.000000in}{0.000000in}}{\pgfqpoint{0.000000in}{0.041667in}}{%
\pgfpathmoveto{\pgfqpoint{0.000000in}{0.000000in}}%
\pgfpathlineto{\pgfqpoint{0.000000in}{0.041667in}}%
\pgfusepath{stroke,fill}%
}%
\begin{pgfscope}%
\pgfsys@transformshift{1.124261in}{3.117349in}%
\pgfsys@useobject{currentmarker}{}%
\end{pgfscope}%
\end{pgfscope}%
\begin{pgfscope}%
\pgfsetbuttcap%
\pgfsetroundjoin%
\definecolor{currentfill}{rgb}{0.000000,0.000000,0.000000}%
\pgfsetfillcolor{currentfill}%
\pgfsetlinewidth{0.501875pt}%
\definecolor{currentstroke}{rgb}{0.000000,0.000000,0.000000}%
\pgfsetstrokecolor{currentstroke}%
\pgfsetdash{}{0pt}%
\pgfsys@defobject{currentmarker}{\pgfqpoint{0.000000in}{-0.041667in}}{\pgfqpoint{0.000000in}{0.000000in}}{%
\pgfpathmoveto{\pgfqpoint{0.000000in}{0.000000in}}%
\pgfpathlineto{\pgfqpoint{0.000000in}{-0.041667in}}%
\pgfusepath{stroke,fill}%
}%
\begin{pgfscope}%
\pgfsys@transformshift{1.124261in}{3.584600in}%
\pgfsys@useobject{currentmarker}{}%
\end{pgfscope}%
\end{pgfscope}%
\begin{pgfscope}%
\pgfsetbuttcap%
\pgfsetroundjoin%
\definecolor{currentfill}{rgb}{0.000000,0.000000,0.000000}%
\pgfsetfillcolor{currentfill}%
\pgfsetlinewidth{0.501875pt}%
\definecolor{currentstroke}{rgb}{0.000000,0.000000,0.000000}%
\pgfsetstrokecolor{currentstroke}%
\pgfsetdash{}{0pt}%
\pgfsys@defobject{currentmarker}{\pgfqpoint{0.000000in}{0.000000in}}{\pgfqpoint{0.000000in}{0.041667in}}{%
\pgfpathmoveto{\pgfqpoint{0.000000in}{0.000000in}}%
\pgfpathlineto{\pgfqpoint{0.000000in}{0.041667in}}%
\pgfusepath{stroke,fill}%
}%
\begin{pgfscope}%
\pgfsys@transformshift{1.604637in}{3.117349in}%
\pgfsys@useobject{currentmarker}{}%
\end{pgfscope}%
\end{pgfscope}%
\begin{pgfscope}%
\pgfsetbuttcap%
\pgfsetroundjoin%
\definecolor{currentfill}{rgb}{0.000000,0.000000,0.000000}%
\pgfsetfillcolor{currentfill}%
\pgfsetlinewidth{0.501875pt}%
\definecolor{currentstroke}{rgb}{0.000000,0.000000,0.000000}%
\pgfsetstrokecolor{currentstroke}%
\pgfsetdash{}{0pt}%
\pgfsys@defobject{currentmarker}{\pgfqpoint{0.000000in}{-0.041667in}}{\pgfqpoint{0.000000in}{0.000000in}}{%
\pgfpathmoveto{\pgfqpoint{0.000000in}{0.000000in}}%
\pgfpathlineto{\pgfqpoint{0.000000in}{-0.041667in}}%
\pgfusepath{stroke,fill}%
}%
\begin{pgfscope}%
\pgfsys@transformshift{1.604637in}{3.584600in}%
\pgfsys@useobject{currentmarker}{}%
\end{pgfscope}%
\end{pgfscope}%
\begin{pgfscope}%
\pgfsetbuttcap%
\pgfsetroundjoin%
\definecolor{currentfill}{rgb}{0.000000,0.000000,0.000000}%
\pgfsetfillcolor{currentfill}%
\pgfsetlinewidth{0.501875pt}%
\definecolor{currentstroke}{rgb}{0.000000,0.000000,0.000000}%
\pgfsetstrokecolor{currentstroke}%
\pgfsetdash{}{0pt}%
\pgfsys@defobject{currentmarker}{\pgfqpoint{0.000000in}{0.000000in}}{\pgfqpoint{0.000000in}{0.041667in}}{%
\pgfpathmoveto{\pgfqpoint{0.000000in}{0.000000in}}%
\pgfpathlineto{\pgfqpoint{0.000000in}{0.041667in}}%
\pgfusepath{stroke,fill}%
}%
\begin{pgfscope}%
\pgfsys@transformshift{2.085012in}{3.117349in}%
\pgfsys@useobject{currentmarker}{}%
\end{pgfscope}%
\end{pgfscope}%
\begin{pgfscope}%
\pgfsetbuttcap%
\pgfsetroundjoin%
\definecolor{currentfill}{rgb}{0.000000,0.000000,0.000000}%
\pgfsetfillcolor{currentfill}%
\pgfsetlinewidth{0.501875pt}%
\definecolor{currentstroke}{rgb}{0.000000,0.000000,0.000000}%
\pgfsetstrokecolor{currentstroke}%
\pgfsetdash{}{0pt}%
\pgfsys@defobject{currentmarker}{\pgfqpoint{0.000000in}{-0.041667in}}{\pgfqpoint{0.000000in}{0.000000in}}{%
\pgfpathmoveto{\pgfqpoint{0.000000in}{0.000000in}}%
\pgfpathlineto{\pgfqpoint{0.000000in}{-0.041667in}}%
\pgfusepath{stroke,fill}%
}%
\begin{pgfscope}%
\pgfsys@transformshift{2.085012in}{3.584600in}%
\pgfsys@useobject{currentmarker}{}%
\end{pgfscope}%
\end{pgfscope}%
\begin{pgfscope}%
\pgfsetbuttcap%
\pgfsetroundjoin%
\definecolor{currentfill}{rgb}{0.000000,0.000000,0.000000}%
\pgfsetfillcolor{currentfill}%
\pgfsetlinewidth{0.501875pt}%
\definecolor{currentstroke}{rgb}{0.000000,0.000000,0.000000}%
\pgfsetstrokecolor{currentstroke}%
\pgfsetdash{}{0pt}%
\pgfsys@defobject{currentmarker}{\pgfqpoint{0.000000in}{0.000000in}}{\pgfqpoint{0.000000in}{0.041667in}}{%
\pgfpathmoveto{\pgfqpoint{0.000000in}{0.000000in}}%
\pgfpathlineto{\pgfqpoint{0.000000in}{0.041667in}}%
\pgfusepath{stroke,fill}%
}%
\begin{pgfscope}%
\pgfsys@transformshift{2.565388in}{3.117349in}%
\pgfsys@useobject{currentmarker}{}%
\end{pgfscope}%
\end{pgfscope}%
\begin{pgfscope}%
\pgfsetbuttcap%
\pgfsetroundjoin%
\definecolor{currentfill}{rgb}{0.000000,0.000000,0.000000}%
\pgfsetfillcolor{currentfill}%
\pgfsetlinewidth{0.501875pt}%
\definecolor{currentstroke}{rgb}{0.000000,0.000000,0.000000}%
\pgfsetstrokecolor{currentstroke}%
\pgfsetdash{}{0pt}%
\pgfsys@defobject{currentmarker}{\pgfqpoint{0.000000in}{-0.041667in}}{\pgfqpoint{0.000000in}{0.000000in}}{%
\pgfpathmoveto{\pgfqpoint{0.000000in}{0.000000in}}%
\pgfpathlineto{\pgfqpoint{0.000000in}{-0.041667in}}%
\pgfusepath{stroke,fill}%
}%
\begin{pgfscope}%
\pgfsys@transformshift{2.565388in}{3.584600in}%
\pgfsys@useobject{currentmarker}{}%
\end{pgfscope}%
\end{pgfscope}%
\begin{pgfscope}%
\pgfsetbuttcap%
\pgfsetroundjoin%
\definecolor{currentfill}{rgb}{0.000000,0.000000,0.000000}%
\pgfsetfillcolor{currentfill}%
\pgfsetlinewidth{0.501875pt}%
\definecolor{currentstroke}{rgb}{0.000000,0.000000,0.000000}%
\pgfsetstrokecolor{currentstroke}%
\pgfsetdash{}{0pt}%
\pgfsys@defobject{currentmarker}{\pgfqpoint{0.000000in}{0.000000in}}{\pgfqpoint{0.000000in}{0.041667in}}{%
\pgfpathmoveto{\pgfqpoint{0.000000in}{0.000000in}}%
\pgfpathlineto{\pgfqpoint{0.000000in}{0.041667in}}%
\pgfusepath{stroke,fill}%
}%
\begin{pgfscope}%
\pgfsys@transformshift{3.045763in}{3.117349in}%
\pgfsys@useobject{currentmarker}{}%
\end{pgfscope}%
\end{pgfscope}%
\begin{pgfscope}%
\pgfsetbuttcap%
\pgfsetroundjoin%
\definecolor{currentfill}{rgb}{0.000000,0.000000,0.000000}%
\pgfsetfillcolor{currentfill}%
\pgfsetlinewidth{0.501875pt}%
\definecolor{currentstroke}{rgb}{0.000000,0.000000,0.000000}%
\pgfsetstrokecolor{currentstroke}%
\pgfsetdash{}{0pt}%
\pgfsys@defobject{currentmarker}{\pgfqpoint{0.000000in}{-0.041667in}}{\pgfqpoint{0.000000in}{0.000000in}}{%
\pgfpathmoveto{\pgfqpoint{0.000000in}{0.000000in}}%
\pgfpathlineto{\pgfqpoint{0.000000in}{-0.041667in}}%
\pgfusepath{stroke,fill}%
}%
\begin{pgfscope}%
\pgfsys@transformshift{3.045763in}{3.584600in}%
\pgfsys@useobject{currentmarker}{}%
\end{pgfscope}%
\end{pgfscope}%
\begin{pgfscope}%
\pgfsetbuttcap%
\pgfsetroundjoin%
\definecolor{currentfill}{rgb}{0.000000,0.000000,0.000000}%
\pgfsetfillcolor{currentfill}%
\pgfsetlinewidth{0.501875pt}%
\definecolor{currentstroke}{rgb}{0.000000,0.000000,0.000000}%
\pgfsetstrokecolor{currentstroke}%
\pgfsetdash{}{0pt}%
\pgfsys@defobject{currentmarker}{\pgfqpoint{0.000000in}{0.000000in}}{\pgfqpoint{0.000000in}{0.041667in}}{%
\pgfpathmoveto{\pgfqpoint{0.000000in}{0.000000in}}%
\pgfpathlineto{\pgfqpoint{0.000000in}{0.041667in}}%
\pgfusepath{stroke,fill}%
}%
\begin{pgfscope}%
\pgfsys@transformshift{3.526138in}{3.117349in}%
\pgfsys@useobject{currentmarker}{}%
\end{pgfscope}%
\end{pgfscope}%
\begin{pgfscope}%
\pgfsetbuttcap%
\pgfsetroundjoin%
\definecolor{currentfill}{rgb}{0.000000,0.000000,0.000000}%
\pgfsetfillcolor{currentfill}%
\pgfsetlinewidth{0.501875pt}%
\definecolor{currentstroke}{rgb}{0.000000,0.000000,0.000000}%
\pgfsetstrokecolor{currentstroke}%
\pgfsetdash{}{0pt}%
\pgfsys@defobject{currentmarker}{\pgfqpoint{0.000000in}{-0.041667in}}{\pgfqpoint{0.000000in}{0.000000in}}{%
\pgfpathmoveto{\pgfqpoint{0.000000in}{0.000000in}}%
\pgfpathlineto{\pgfqpoint{0.000000in}{-0.041667in}}%
\pgfusepath{stroke,fill}%
}%
\begin{pgfscope}%
\pgfsys@transformshift{3.526138in}{3.584600in}%
\pgfsys@useobject{currentmarker}{}%
\end{pgfscope}%
\end{pgfscope}%
\begin{pgfscope}%
\pgfsetbuttcap%
\pgfsetroundjoin%
\definecolor{currentfill}{rgb}{0.000000,0.000000,0.000000}%
\pgfsetfillcolor{currentfill}%
\pgfsetlinewidth{0.501875pt}%
\definecolor{currentstroke}{rgb}{0.000000,0.000000,0.000000}%
\pgfsetstrokecolor{currentstroke}%
\pgfsetdash{}{0pt}%
\pgfsys@defobject{currentmarker}{\pgfqpoint{0.000000in}{0.000000in}}{\pgfqpoint{0.000000in}{0.041667in}}{%
\pgfpathmoveto{\pgfqpoint{0.000000in}{0.000000in}}%
\pgfpathlineto{\pgfqpoint{0.000000in}{0.041667in}}%
\pgfusepath{stroke,fill}%
}%
\begin{pgfscope}%
\pgfsys@transformshift{4.006514in}{3.117349in}%
\pgfsys@useobject{currentmarker}{}%
\end{pgfscope}%
\end{pgfscope}%
\begin{pgfscope}%
\pgfsetbuttcap%
\pgfsetroundjoin%
\definecolor{currentfill}{rgb}{0.000000,0.000000,0.000000}%
\pgfsetfillcolor{currentfill}%
\pgfsetlinewidth{0.501875pt}%
\definecolor{currentstroke}{rgb}{0.000000,0.000000,0.000000}%
\pgfsetstrokecolor{currentstroke}%
\pgfsetdash{}{0pt}%
\pgfsys@defobject{currentmarker}{\pgfqpoint{0.000000in}{-0.041667in}}{\pgfqpoint{0.000000in}{0.000000in}}{%
\pgfpathmoveto{\pgfqpoint{0.000000in}{0.000000in}}%
\pgfpathlineto{\pgfqpoint{0.000000in}{-0.041667in}}%
\pgfusepath{stroke,fill}%
}%
\begin{pgfscope}%
\pgfsys@transformshift{4.006514in}{3.584600in}%
\pgfsys@useobject{currentmarker}{}%
\end{pgfscope}%
\end{pgfscope}%
\begin{pgfscope}%
\pgfsetbuttcap%
\pgfsetroundjoin%
\definecolor{currentfill}{rgb}{0.000000,0.000000,0.000000}%
\pgfsetfillcolor{currentfill}%
\pgfsetlinewidth{0.501875pt}%
\definecolor{currentstroke}{rgb}{0.000000,0.000000,0.000000}%
\pgfsetstrokecolor{currentstroke}%
\pgfsetdash{}{0pt}%
\pgfsys@defobject{currentmarker}{\pgfqpoint{0.000000in}{0.000000in}}{\pgfqpoint{0.000000in}{0.041667in}}{%
\pgfpathmoveto{\pgfqpoint{0.000000in}{0.000000in}}%
\pgfpathlineto{\pgfqpoint{0.000000in}{0.041667in}}%
\pgfusepath{stroke,fill}%
}%
\begin{pgfscope}%
\pgfsys@transformshift{4.486889in}{3.117349in}%
\pgfsys@useobject{currentmarker}{}%
\end{pgfscope}%
\end{pgfscope}%
\begin{pgfscope}%
\pgfsetbuttcap%
\pgfsetroundjoin%
\definecolor{currentfill}{rgb}{0.000000,0.000000,0.000000}%
\pgfsetfillcolor{currentfill}%
\pgfsetlinewidth{0.501875pt}%
\definecolor{currentstroke}{rgb}{0.000000,0.000000,0.000000}%
\pgfsetstrokecolor{currentstroke}%
\pgfsetdash{}{0pt}%
\pgfsys@defobject{currentmarker}{\pgfqpoint{0.000000in}{-0.041667in}}{\pgfqpoint{0.000000in}{0.000000in}}{%
\pgfpathmoveto{\pgfqpoint{0.000000in}{0.000000in}}%
\pgfpathlineto{\pgfqpoint{0.000000in}{-0.041667in}}%
\pgfusepath{stroke,fill}%
}%
\begin{pgfscope}%
\pgfsys@transformshift{4.486889in}{3.584600in}%
\pgfsys@useobject{currentmarker}{}%
\end{pgfscope}%
\end{pgfscope}%
\begin{pgfscope}%
\pgfsetbuttcap%
\pgfsetroundjoin%
\definecolor{currentfill}{rgb}{0.000000,0.000000,0.000000}%
\pgfsetfillcolor{currentfill}%
\pgfsetlinewidth{0.501875pt}%
\definecolor{currentstroke}{rgb}{0.000000,0.000000,0.000000}%
\pgfsetstrokecolor{currentstroke}%
\pgfsetdash{}{0pt}%
\pgfsys@defobject{currentmarker}{\pgfqpoint{0.000000in}{0.000000in}}{\pgfqpoint{0.000000in}{0.020833in}}{%
\pgfpathmoveto{\pgfqpoint{0.000000in}{0.000000in}}%
\pgfpathlineto{\pgfqpoint{0.000000in}{0.020833in}}%
\pgfusepath{stroke,fill}%
}%
\begin{pgfscope}%
\pgfsys@transformshift{0.451736in}{3.117349in}%
\pgfsys@useobject{currentmarker}{}%
\end{pgfscope}%
\end{pgfscope}%
\begin{pgfscope}%
\pgfsetbuttcap%
\pgfsetroundjoin%
\definecolor{currentfill}{rgb}{0.000000,0.000000,0.000000}%
\pgfsetfillcolor{currentfill}%
\pgfsetlinewidth{0.501875pt}%
\definecolor{currentstroke}{rgb}{0.000000,0.000000,0.000000}%
\pgfsetstrokecolor{currentstroke}%
\pgfsetdash{}{0pt}%
\pgfsys@defobject{currentmarker}{\pgfqpoint{0.000000in}{-0.020833in}}{\pgfqpoint{0.000000in}{0.000000in}}{%
\pgfpathmoveto{\pgfqpoint{0.000000in}{0.000000in}}%
\pgfpathlineto{\pgfqpoint{0.000000in}{-0.020833in}}%
\pgfusepath{stroke,fill}%
}%
\begin{pgfscope}%
\pgfsys@transformshift{0.451736in}{3.584600in}%
\pgfsys@useobject{currentmarker}{}%
\end{pgfscope}%
\end{pgfscope}%
\begin{pgfscope}%
\pgfsetbuttcap%
\pgfsetroundjoin%
\definecolor{currentfill}{rgb}{0.000000,0.000000,0.000000}%
\pgfsetfillcolor{currentfill}%
\pgfsetlinewidth{0.501875pt}%
\definecolor{currentstroke}{rgb}{0.000000,0.000000,0.000000}%
\pgfsetstrokecolor{currentstroke}%
\pgfsetdash{}{0pt}%
\pgfsys@defobject{currentmarker}{\pgfqpoint{0.000000in}{0.000000in}}{\pgfqpoint{0.000000in}{0.020833in}}{%
\pgfpathmoveto{\pgfqpoint{0.000000in}{0.000000in}}%
\pgfpathlineto{\pgfqpoint{0.000000in}{0.020833in}}%
\pgfusepath{stroke,fill}%
}%
\begin{pgfscope}%
\pgfsys@transformshift{0.547811in}{3.117349in}%
\pgfsys@useobject{currentmarker}{}%
\end{pgfscope}%
\end{pgfscope}%
\begin{pgfscope}%
\pgfsetbuttcap%
\pgfsetroundjoin%
\definecolor{currentfill}{rgb}{0.000000,0.000000,0.000000}%
\pgfsetfillcolor{currentfill}%
\pgfsetlinewidth{0.501875pt}%
\definecolor{currentstroke}{rgb}{0.000000,0.000000,0.000000}%
\pgfsetstrokecolor{currentstroke}%
\pgfsetdash{}{0pt}%
\pgfsys@defobject{currentmarker}{\pgfqpoint{0.000000in}{-0.020833in}}{\pgfqpoint{0.000000in}{0.000000in}}{%
\pgfpathmoveto{\pgfqpoint{0.000000in}{0.000000in}}%
\pgfpathlineto{\pgfqpoint{0.000000in}{-0.020833in}}%
\pgfusepath{stroke,fill}%
}%
\begin{pgfscope}%
\pgfsys@transformshift{0.547811in}{3.584600in}%
\pgfsys@useobject{currentmarker}{}%
\end{pgfscope}%
\end{pgfscope}%
\begin{pgfscope}%
\pgfsetbuttcap%
\pgfsetroundjoin%
\definecolor{currentfill}{rgb}{0.000000,0.000000,0.000000}%
\pgfsetfillcolor{currentfill}%
\pgfsetlinewidth{0.501875pt}%
\definecolor{currentstroke}{rgb}{0.000000,0.000000,0.000000}%
\pgfsetstrokecolor{currentstroke}%
\pgfsetdash{}{0pt}%
\pgfsys@defobject{currentmarker}{\pgfqpoint{0.000000in}{0.000000in}}{\pgfqpoint{0.000000in}{0.020833in}}{%
\pgfpathmoveto{\pgfqpoint{0.000000in}{0.000000in}}%
\pgfpathlineto{\pgfqpoint{0.000000in}{0.020833in}}%
\pgfusepath{stroke,fill}%
}%
\begin{pgfscope}%
\pgfsys@transformshift{0.739961in}{3.117349in}%
\pgfsys@useobject{currentmarker}{}%
\end{pgfscope}%
\end{pgfscope}%
\begin{pgfscope}%
\pgfsetbuttcap%
\pgfsetroundjoin%
\definecolor{currentfill}{rgb}{0.000000,0.000000,0.000000}%
\pgfsetfillcolor{currentfill}%
\pgfsetlinewidth{0.501875pt}%
\definecolor{currentstroke}{rgb}{0.000000,0.000000,0.000000}%
\pgfsetstrokecolor{currentstroke}%
\pgfsetdash{}{0pt}%
\pgfsys@defobject{currentmarker}{\pgfqpoint{0.000000in}{-0.020833in}}{\pgfqpoint{0.000000in}{0.000000in}}{%
\pgfpathmoveto{\pgfqpoint{0.000000in}{0.000000in}}%
\pgfpathlineto{\pgfqpoint{0.000000in}{-0.020833in}}%
\pgfusepath{stroke,fill}%
}%
\begin{pgfscope}%
\pgfsys@transformshift{0.739961in}{3.584600in}%
\pgfsys@useobject{currentmarker}{}%
\end{pgfscope}%
\end{pgfscope}%
\begin{pgfscope}%
\pgfsetbuttcap%
\pgfsetroundjoin%
\definecolor{currentfill}{rgb}{0.000000,0.000000,0.000000}%
\pgfsetfillcolor{currentfill}%
\pgfsetlinewidth{0.501875pt}%
\definecolor{currentstroke}{rgb}{0.000000,0.000000,0.000000}%
\pgfsetstrokecolor{currentstroke}%
\pgfsetdash{}{0pt}%
\pgfsys@defobject{currentmarker}{\pgfqpoint{0.000000in}{0.000000in}}{\pgfqpoint{0.000000in}{0.020833in}}{%
\pgfpathmoveto{\pgfqpoint{0.000000in}{0.000000in}}%
\pgfpathlineto{\pgfqpoint{0.000000in}{0.020833in}}%
\pgfusepath{stroke,fill}%
}%
\begin{pgfscope}%
\pgfsys@transformshift{0.836036in}{3.117349in}%
\pgfsys@useobject{currentmarker}{}%
\end{pgfscope}%
\end{pgfscope}%
\begin{pgfscope}%
\pgfsetbuttcap%
\pgfsetroundjoin%
\definecolor{currentfill}{rgb}{0.000000,0.000000,0.000000}%
\pgfsetfillcolor{currentfill}%
\pgfsetlinewidth{0.501875pt}%
\definecolor{currentstroke}{rgb}{0.000000,0.000000,0.000000}%
\pgfsetstrokecolor{currentstroke}%
\pgfsetdash{}{0pt}%
\pgfsys@defobject{currentmarker}{\pgfqpoint{0.000000in}{-0.020833in}}{\pgfqpoint{0.000000in}{0.000000in}}{%
\pgfpathmoveto{\pgfqpoint{0.000000in}{0.000000in}}%
\pgfpathlineto{\pgfqpoint{0.000000in}{-0.020833in}}%
\pgfusepath{stroke,fill}%
}%
\begin{pgfscope}%
\pgfsys@transformshift{0.836036in}{3.584600in}%
\pgfsys@useobject{currentmarker}{}%
\end{pgfscope}%
\end{pgfscope}%
\begin{pgfscope}%
\pgfsetbuttcap%
\pgfsetroundjoin%
\definecolor{currentfill}{rgb}{0.000000,0.000000,0.000000}%
\pgfsetfillcolor{currentfill}%
\pgfsetlinewidth{0.501875pt}%
\definecolor{currentstroke}{rgb}{0.000000,0.000000,0.000000}%
\pgfsetstrokecolor{currentstroke}%
\pgfsetdash{}{0pt}%
\pgfsys@defobject{currentmarker}{\pgfqpoint{0.000000in}{0.000000in}}{\pgfqpoint{0.000000in}{0.020833in}}{%
\pgfpathmoveto{\pgfqpoint{0.000000in}{0.000000in}}%
\pgfpathlineto{\pgfqpoint{0.000000in}{0.020833in}}%
\pgfusepath{stroke,fill}%
}%
\begin{pgfscope}%
\pgfsys@transformshift{0.932111in}{3.117349in}%
\pgfsys@useobject{currentmarker}{}%
\end{pgfscope}%
\end{pgfscope}%
\begin{pgfscope}%
\pgfsetbuttcap%
\pgfsetroundjoin%
\definecolor{currentfill}{rgb}{0.000000,0.000000,0.000000}%
\pgfsetfillcolor{currentfill}%
\pgfsetlinewidth{0.501875pt}%
\definecolor{currentstroke}{rgb}{0.000000,0.000000,0.000000}%
\pgfsetstrokecolor{currentstroke}%
\pgfsetdash{}{0pt}%
\pgfsys@defobject{currentmarker}{\pgfqpoint{0.000000in}{-0.020833in}}{\pgfqpoint{0.000000in}{0.000000in}}{%
\pgfpathmoveto{\pgfqpoint{0.000000in}{0.000000in}}%
\pgfpathlineto{\pgfqpoint{0.000000in}{-0.020833in}}%
\pgfusepath{stroke,fill}%
}%
\begin{pgfscope}%
\pgfsys@transformshift{0.932111in}{3.584600in}%
\pgfsys@useobject{currentmarker}{}%
\end{pgfscope}%
\end{pgfscope}%
\begin{pgfscope}%
\pgfsetbuttcap%
\pgfsetroundjoin%
\definecolor{currentfill}{rgb}{0.000000,0.000000,0.000000}%
\pgfsetfillcolor{currentfill}%
\pgfsetlinewidth{0.501875pt}%
\definecolor{currentstroke}{rgb}{0.000000,0.000000,0.000000}%
\pgfsetstrokecolor{currentstroke}%
\pgfsetdash{}{0pt}%
\pgfsys@defobject{currentmarker}{\pgfqpoint{0.000000in}{0.000000in}}{\pgfqpoint{0.000000in}{0.020833in}}{%
\pgfpathmoveto{\pgfqpoint{0.000000in}{0.000000in}}%
\pgfpathlineto{\pgfqpoint{0.000000in}{0.020833in}}%
\pgfusepath{stroke,fill}%
}%
\begin{pgfscope}%
\pgfsys@transformshift{1.028186in}{3.117349in}%
\pgfsys@useobject{currentmarker}{}%
\end{pgfscope}%
\end{pgfscope}%
\begin{pgfscope}%
\pgfsetbuttcap%
\pgfsetroundjoin%
\definecolor{currentfill}{rgb}{0.000000,0.000000,0.000000}%
\pgfsetfillcolor{currentfill}%
\pgfsetlinewidth{0.501875pt}%
\definecolor{currentstroke}{rgb}{0.000000,0.000000,0.000000}%
\pgfsetstrokecolor{currentstroke}%
\pgfsetdash{}{0pt}%
\pgfsys@defobject{currentmarker}{\pgfqpoint{0.000000in}{-0.020833in}}{\pgfqpoint{0.000000in}{0.000000in}}{%
\pgfpathmoveto{\pgfqpoint{0.000000in}{0.000000in}}%
\pgfpathlineto{\pgfqpoint{0.000000in}{-0.020833in}}%
\pgfusepath{stroke,fill}%
}%
\begin{pgfscope}%
\pgfsys@transformshift{1.028186in}{3.584600in}%
\pgfsys@useobject{currentmarker}{}%
\end{pgfscope}%
\end{pgfscope}%
\begin{pgfscope}%
\pgfsetbuttcap%
\pgfsetroundjoin%
\definecolor{currentfill}{rgb}{0.000000,0.000000,0.000000}%
\pgfsetfillcolor{currentfill}%
\pgfsetlinewidth{0.501875pt}%
\definecolor{currentstroke}{rgb}{0.000000,0.000000,0.000000}%
\pgfsetstrokecolor{currentstroke}%
\pgfsetdash{}{0pt}%
\pgfsys@defobject{currentmarker}{\pgfqpoint{0.000000in}{0.000000in}}{\pgfqpoint{0.000000in}{0.020833in}}{%
\pgfpathmoveto{\pgfqpoint{0.000000in}{0.000000in}}%
\pgfpathlineto{\pgfqpoint{0.000000in}{0.020833in}}%
\pgfusepath{stroke,fill}%
}%
\begin{pgfscope}%
\pgfsys@transformshift{1.220336in}{3.117349in}%
\pgfsys@useobject{currentmarker}{}%
\end{pgfscope}%
\end{pgfscope}%
\begin{pgfscope}%
\pgfsetbuttcap%
\pgfsetroundjoin%
\definecolor{currentfill}{rgb}{0.000000,0.000000,0.000000}%
\pgfsetfillcolor{currentfill}%
\pgfsetlinewidth{0.501875pt}%
\definecolor{currentstroke}{rgb}{0.000000,0.000000,0.000000}%
\pgfsetstrokecolor{currentstroke}%
\pgfsetdash{}{0pt}%
\pgfsys@defobject{currentmarker}{\pgfqpoint{0.000000in}{-0.020833in}}{\pgfqpoint{0.000000in}{0.000000in}}{%
\pgfpathmoveto{\pgfqpoint{0.000000in}{0.000000in}}%
\pgfpathlineto{\pgfqpoint{0.000000in}{-0.020833in}}%
\pgfusepath{stroke,fill}%
}%
\begin{pgfscope}%
\pgfsys@transformshift{1.220336in}{3.584600in}%
\pgfsys@useobject{currentmarker}{}%
\end{pgfscope}%
\end{pgfscope}%
\begin{pgfscope}%
\pgfsetbuttcap%
\pgfsetroundjoin%
\definecolor{currentfill}{rgb}{0.000000,0.000000,0.000000}%
\pgfsetfillcolor{currentfill}%
\pgfsetlinewidth{0.501875pt}%
\definecolor{currentstroke}{rgb}{0.000000,0.000000,0.000000}%
\pgfsetstrokecolor{currentstroke}%
\pgfsetdash{}{0pt}%
\pgfsys@defobject{currentmarker}{\pgfqpoint{0.000000in}{0.000000in}}{\pgfqpoint{0.000000in}{0.020833in}}{%
\pgfpathmoveto{\pgfqpoint{0.000000in}{0.000000in}}%
\pgfpathlineto{\pgfqpoint{0.000000in}{0.020833in}}%
\pgfusepath{stroke,fill}%
}%
\begin{pgfscope}%
\pgfsys@transformshift{1.316411in}{3.117349in}%
\pgfsys@useobject{currentmarker}{}%
\end{pgfscope}%
\end{pgfscope}%
\begin{pgfscope}%
\pgfsetbuttcap%
\pgfsetroundjoin%
\definecolor{currentfill}{rgb}{0.000000,0.000000,0.000000}%
\pgfsetfillcolor{currentfill}%
\pgfsetlinewidth{0.501875pt}%
\definecolor{currentstroke}{rgb}{0.000000,0.000000,0.000000}%
\pgfsetstrokecolor{currentstroke}%
\pgfsetdash{}{0pt}%
\pgfsys@defobject{currentmarker}{\pgfqpoint{0.000000in}{-0.020833in}}{\pgfqpoint{0.000000in}{0.000000in}}{%
\pgfpathmoveto{\pgfqpoint{0.000000in}{0.000000in}}%
\pgfpathlineto{\pgfqpoint{0.000000in}{-0.020833in}}%
\pgfusepath{stroke,fill}%
}%
\begin{pgfscope}%
\pgfsys@transformshift{1.316411in}{3.584600in}%
\pgfsys@useobject{currentmarker}{}%
\end{pgfscope}%
\end{pgfscope}%
\begin{pgfscope}%
\pgfsetbuttcap%
\pgfsetroundjoin%
\definecolor{currentfill}{rgb}{0.000000,0.000000,0.000000}%
\pgfsetfillcolor{currentfill}%
\pgfsetlinewidth{0.501875pt}%
\definecolor{currentstroke}{rgb}{0.000000,0.000000,0.000000}%
\pgfsetstrokecolor{currentstroke}%
\pgfsetdash{}{0pt}%
\pgfsys@defobject{currentmarker}{\pgfqpoint{0.000000in}{0.000000in}}{\pgfqpoint{0.000000in}{0.020833in}}{%
\pgfpathmoveto{\pgfqpoint{0.000000in}{0.000000in}}%
\pgfpathlineto{\pgfqpoint{0.000000in}{0.020833in}}%
\pgfusepath{stroke,fill}%
}%
\begin{pgfscope}%
\pgfsys@transformshift{1.412487in}{3.117349in}%
\pgfsys@useobject{currentmarker}{}%
\end{pgfscope}%
\end{pgfscope}%
\begin{pgfscope}%
\pgfsetbuttcap%
\pgfsetroundjoin%
\definecolor{currentfill}{rgb}{0.000000,0.000000,0.000000}%
\pgfsetfillcolor{currentfill}%
\pgfsetlinewidth{0.501875pt}%
\definecolor{currentstroke}{rgb}{0.000000,0.000000,0.000000}%
\pgfsetstrokecolor{currentstroke}%
\pgfsetdash{}{0pt}%
\pgfsys@defobject{currentmarker}{\pgfqpoint{0.000000in}{-0.020833in}}{\pgfqpoint{0.000000in}{0.000000in}}{%
\pgfpathmoveto{\pgfqpoint{0.000000in}{0.000000in}}%
\pgfpathlineto{\pgfqpoint{0.000000in}{-0.020833in}}%
\pgfusepath{stroke,fill}%
}%
\begin{pgfscope}%
\pgfsys@transformshift{1.412487in}{3.584600in}%
\pgfsys@useobject{currentmarker}{}%
\end{pgfscope}%
\end{pgfscope}%
\begin{pgfscope}%
\pgfsetbuttcap%
\pgfsetroundjoin%
\definecolor{currentfill}{rgb}{0.000000,0.000000,0.000000}%
\pgfsetfillcolor{currentfill}%
\pgfsetlinewidth{0.501875pt}%
\definecolor{currentstroke}{rgb}{0.000000,0.000000,0.000000}%
\pgfsetstrokecolor{currentstroke}%
\pgfsetdash{}{0pt}%
\pgfsys@defobject{currentmarker}{\pgfqpoint{0.000000in}{0.000000in}}{\pgfqpoint{0.000000in}{0.020833in}}{%
\pgfpathmoveto{\pgfqpoint{0.000000in}{0.000000in}}%
\pgfpathlineto{\pgfqpoint{0.000000in}{0.020833in}}%
\pgfusepath{stroke,fill}%
}%
\begin{pgfscope}%
\pgfsys@transformshift{1.508562in}{3.117349in}%
\pgfsys@useobject{currentmarker}{}%
\end{pgfscope}%
\end{pgfscope}%
\begin{pgfscope}%
\pgfsetbuttcap%
\pgfsetroundjoin%
\definecolor{currentfill}{rgb}{0.000000,0.000000,0.000000}%
\pgfsetfillcolor{currentfill}%
\pgfsetlinewidth{0.501875pt}%
\definecolor{currentstroke}{rgb}{0.000000,0.000000,0.000000}%
\pgfsetstrokecolor{currentstroke}%
\pgfsetdash{}{0pt}%
\pgfsys@defobject{currentmarker}{\pgfqpoint{0.000000in}{-0.020833in}}{\pgfqpoint{0.000000in}{0.000000in}}{%
\pgfpathmoveto{\pgfqpoint{0.000000in}{0.000000in}}%
\pgfpathlineto{\pgfqpoint{0.000000in}{-0.020833in}}%
\pgfusepath{stroke,fill}%
}%
\begin{pgfscope}%
\pgfsys@transformshift{1.508562in}{3.584600in}%
\pgfsys@useobject{currentmarker}{}%
\end{pgfscope}%
\end{pgfscope}%
\begin{pgfscope}%
\pgfsetbuttcap%
\pgfsetroundjoin%
\definecolor{currentfill}{rgb}{0.000000,0.000000,0.000000}%
\pgfsetfillcolor{currentfill}%
\pgfsetlinewidth{0.501875pt}%
\definecolor{currentstroke}{rgb}{0.000000,0.000000,0.000000}%
\pgfsetstrokecolor{currentstroke}%
\pgfsetdash{}{0pt}%
\pgfsys@defobject{currentmarker}{\pgfqpoint{0.000000in}{0.000000in}}{\pgfqpoint{0.000000in}{0.020833in}}{%
\pgfpathmoveto{\pgfqpoint{0.000000in}{0.000000in}}%
\pgfpathlineto{\pgfqpoint{0.000000in}{0.020833in}}%
\pgfusepath{stroke,fill}%
}%
\begin{pgfscope}%
\pgfsys@transformshift{1.700712in}{3.117349in}%
\pgfsys@useobject{currentmarker}{}%
\end{pgfscope}%
\end{pgfscope}%
\begin{pgfscope}%
\pgfsetbuttcap%
\pgfsetroundjoin%
\definecolor{currentfill}{rgb}{0.000000,0.000000,0.000000}%
\pgfsetfillcolor{currentfill}%
\pgfsetlinewidth{0.501875pt}%
\definecolor{currentstroke}{rgb}{0.000000,0.000000,0.000000}%
\pgfsetstrokecolor{currentstroke}%
\pgfsetdash{}{0pt}%
\pgfsys@defobject{currentmarker}{\pgfqpoint{0.000000in}{-0.020833in}}{\pgfqpoint{0.000000in}{0.000000in}}{%
\pgfpathmoveto{\pgfqpoint{0.000000in}{0.000000in}}%
\pgfpathlineto{\pgfqpoint{0.000000in}{-0.020833in}}%
\pgfusepath{stroke,fill}%
}%
\begin{pgfscope}%
\pgfsys@transformshift{1.700712in}{3.584600in}%
\pgfsys@useobject{currentmarker}{}%
\end{pgfscope}%
\end{pgfscope}%
\begin{pgfscope}%
\pgfsetbuttcap%
\pgfsetroundjoin%
\definecolor{currentfill}{rgb}{0.000000,0.000000,0.000000}%
\pgfsetfillcolor{currentfill}%
\pgfsetlinewidth{0.501875pt}%
\definecolor{currentstroke}{rgb}{0.000000,0.000000,0.000000}%
\pgfsetstrokecolor{currentstroke}%
\pgfsetdash{}{0pt}%
\pgfsys@defobject{currentmarker}{\pgfqpoint{0.000000in}{0.000000in}}{\pgfqpoint{0.000000in}{0.020833in}}{%
\pgfpathmoveto{\pgfqpoint{0.000000in}{0.000000in}}%
\pgfpathlineto{\pgfqpoint{0.000000in}{0.020833in}}%
\pgfusepath{stroke,fill}%
}%
\begin{pgfscope}%
\pgfsys@transformshift{1.796787in}{3.117349in}%
\pgfsys@useobject{currentmarker}{}%
\end{pgfscope}%
\end{pgfscope}%
\begin{pgfscope}%
\pgfsetbuttcap%
\pgfsetroundjoin%
\definecolor{currentfill}{rgb}{0.000000,0.000000,0.000000}%
\pgfsetfillcolor{currentfill}%
\pgfsetlinewidth{0.501875pt}%
\definecolor{currentstroke}{rgb}{0.000000,0.000000,0.000000}%
\pgfsetstrokecolor{currentstroke}%
\pgfsetdash{}{0pt}%
\pgfsys@defobject{currentmarker}{\pgfqpoint{0.000000in}{-0.020833in}}{\pgfqpoint{0.000000in}{0.000000in}}{%
\pgfpathmoveto{\pgfqpoint{0.000000in}{0.000000in}}%
\pgfpathlineto{\pgfqpoint{0.000000in}{-0.020833in}}%
\pgfusepath{stroke,fill}%
}%
\begin{pgfscope}%
\pgfsys@transformshift{1.796787in}{3.584600in}%
\pgfsys@useobject{currentmarker}{}%
\end{pgfscope}%
\end{pgfscope}%
\begin{pgfscope}%
\pgfsetbuttcap%
\pgfsetroundjoin%
\definecolor{currentfill}{rgb}{0.000000,0.000000,0.000000}%
\pgfsetfillcolor{currentfill}%
\pgfsetlinewidth{0.501875pt}%
\definecolor{currentstroke}{rgb}{0.000000,0.000000,0.000000}%
\pgfsetstrokecolor{currentstroke}%
\pgfsetdash{}{0pt}%
\pgfsys@defobject{currentmarker}{\pgfqpoint{0.000000in}{0.000000in}}{\pgfqpoint{0.000000in}{0.020833in}}{%
\pgfpathmoveto{\pgfqpoint{0.000000in}{0.000000in}}%
\pgfpathlineto{\pgfqpoint{0.000000in}{0.020833in}}%
\pgfusepath{stroke,fill}%
}%
\begin{pgfscope}%
\pgfsys@transformshift{1.892862in}{3.117349in}%
\pgfsys@useobject{currentmarker}{}%
\end{pgfscope}%
\end{pgfscope}%
\begin{pgfscope}%
\pgfsetbuttcap%
\pgfsetroundjoin%
\definecolor{currentfill}{rgb}{0.000000,0.000000,0.000000}%
\pgfsetfillcolor{currentfill}%
\pgfsetlinewidth{0.501875pt}%
\definecolor{currentstroke}{rgb}{0.000000,0.000000,0.000000}%
\pgfsetstrokecolor{currentstroke}%
\pgfsetdash{}{0pt}%
\pgfsys@defobject{currentmarker}{\pgfqpoint{0.000000in}{-0.020833in}}{\pgfqpoint{0.000000in}{0.000000in}}{%
\pgfpathmoveto{\pgfqpoint{0.000000in}{0.000000in}}%
\pgfpathlineto{\pgfqpoint{0.000000in}{-0.020833in}}%
\pgfusepath{stroke,fill}%
}%
\begin{pgfscope}%
\pgfsys@transformshift{1.892862in}{3.584600in}%
\pgfsys@useobject{currentmarker}{}%
\end{pgfscope}%
\end{pgfscope}%
\begin{pgfscope}%
\pgfsetbuttcap%
\pgfsetroundjoin%
\definecolor{currentfill}{rgb}{0.000000,0.000000,0.000000}%
\pgfsetfillcolor{currentfill}%
\pgfsetlinewidth{0.501875pt}%
\definecolor{currentstroke}{rgb}{0.000000,0.000000,0.000000}%
\pgfsetstrokecolor{currentstroke}%
\pgfsetdash{}{0pt}%
\pgfsys@defobject{currentmarker}{\pgfqpoint{0.000000in}{0.000000in}}{\pgfqpoint{0.000000in}{0.020833in}}{%
\pgfpathmoveto{\pgfqpoint{0.000000in}{0.000000in}}%
\pgfpathlineto{\pgfqpoint{0.000000in}{0.020833in}}%
\pgfusepath{stroke,fill}%
}%
\begin{pgfscope}%
\pgfsys@transformshift{1.988937in}{3.117349in}%
\pgfsys@useobject{currentmarker}{}%
\end{pgfscope}%
\end{pgfscope}%
\begin{pgfscope}%
\pgfsetbuttcap%
\pgfsetroundjoin%
\definecolor{currentfill}{rgb}{0.000000,0.000000,0.000000}%
\pgfsetfillcolor{currentfill}%
\pgfsetlinewidth{0.501875pt}%
\definecolor{currentstroke}{rgb}{0.000000,0.000000,0.000000}%
\pgfsetstrokecolor{currentstroke}%
\pgfsetdash{}{0pt}%
\pgfsys@defobject{currentmarker}{\pgfqpoint{0.000000in}{-0.020833in}}{\pgfqpoint{0.000000in}{0.000000in}}{%
\pgfpathmoveto{\pgfqpoint{0.000000in}{0.000000in}}%
\pgfpathlineto{\pgfqpoint{0.000000in}{-0.020833in}}%
\pgfusepath{stroke,fill}%
}%
\begin{pgfscope}%
\pgfsys@transformshift{1.988937in}{3.584600in}%
\pgfsys@useobject{currentmarker}{}%
\end{pgfscope}%
\end{pgfscope}%
\begin{pgfscope}%
\pgfsetbuttcap%
\pgfsetroundjoin%
\definecolor{currentfill}{rgb}{0.000000,0.000000,0.000000}%
\pgfsetfillcolor{currentfill}%
\pgfsetlinewidth{0.501875pt}%
\definecolor{currentstroke}{rgb}{0.000000,0.000000,0.000000}%
\pgfsetstrokecolor{currentstroke}%
\pgfsetdash{}{0pt}%
\pgfsys@defobject{currentmarker}{\pgfqpoint{0.000000in}{0.000000in}}{\pgfqpoint{0.000000in}{0.020833in}}{%
\pgfpathmoveto{\pgfqpoint{0.000000in}{0.000000in}}%
\pgfpathlineto{\pgfqpoint{0.000000in}{0.020833in}}%
\pgfusepath{stroke,fill}%
}%
\begin{pgfscope}%
\pgfsys@transformshift{2.181087in}{3.117349in}%
\pgfsys@useobject{currentmarker}{}%
\end{pgfscope}%
\end{pgfscope}%
\begin{pgfscope}%
\pgfsetbuttcap%
\pgfsetroundjoin%
\definecolor{currentfill}{rgb}{0.000000,0.000000,0.000000}%
\pgfsetfillcolor{currentfill}%
\pgfsetlinewidth{0.501875pt}%
\definecolor{currentstroke}{rgb}{0.000000,0.000000,0.000000}%
\pgfsetstrokecolor{currentstroke}%
\pgfsetdash{}{0pt}%
\pgfsys@defobject{currentmarker}{\pgfqpoint{0.000000in}{-0.020833in}}{\pgfqpoint{0.000000in}{0.000000in}}{%
\pgfpathmoveto{\pgfqpoint{0.000000in}{0.000000in}}%
\pgfpathlineto{\pgfqpoint{0.000000in}{-0.020833in}}%
\pgfusepath{stroke,fill}%
}%
\begin{pgfscope}%
\pgfsys@transformshift{2.181087in}{3.584600in}%
\pgfsys@useobject{currentmarker}{}%
\end{pgfscope}%
\end{pgfscope}%
\begin{pgfscope}%
\pgfsetbuttcap%
\pgfsetroundjoin%
\definecolor{currentfill}{rgb}{0.000000,0.000000,0.000000}%
\pgfsetfillcolor{currentfill}%
\pgfsetlinewidth{0.501875pt}%
\definecolor{currentstroke}{rgb}{0.000000,0.000000,0.000000}%
\pgfsetstrokecolor{currentstroke}%
\pgfsetdash{}{0pt}%
\pgfsys@defobject{currentmarker}{\pgfqpoint{0.000000in}{0.000000in}}{\pgfqpoint{0.000000in}{0.020833in}}{%
\pgfpathmoveto{\pgfqpoint{0.000000in}{0.000000in}}%
\pgfpathlineto{\pgfqpoint{0.000000in}{0.020833in}}%
\pgfusepath{stroke,fill}%
}%
\begin{pgfscope}%
\pgfsys@transformshift{2.277162in}{3.117349in}%
\pgfsys@useobject{currentmarker}{}%
\end{pgfscope}%
\end{pgfscope}%
\begin{pgfscope}%
\pgfsetbuttcap%
\pgfsetroundjoin%
\definecolor{currentfill}{rgb}{0.000000,0.000000,0.000000}%
\pgfsetfillcolor{currentfill}%
\pgfsetlinewidth{0.501875pt}%
\definecolor{currentstroke}{rgb}{0.000000,0.000000,0.000000}%
\pgfsetstrokecolor{currentstroke}%
\pgfsetdash{}{0pt}%
\pgfsys@defobject{currentmarker}{\pgfqpoint{0.000000in}{-0.020833in}}{\pgfqpoint{0.000000in}{0.000000in}}{%
\pgfpathmoveto{\pgfqpoint{0.000000in}{0.000000in}}%
\pgfpathlineto{\pgfqpoint{0.000000in}{-0.020833in}}%
\pgfusepath{stroke,fill}%
}%
\begin{pgfscope}%
\pgfsys@transformshift{2.277162in}{3.584600in}%
\pgfsys@useobject{currentmarker}{}%
\end{pgfscope}%
\end{pgfscope}%
\begin{pgfscope}%
\pgfsetbuttcap%
\pgfsetroundjoin%
\definecolor{currentfill}{rgb}{0.000000,0.000000,0.000000}%
\pgfsetfillcolor{currentfill}%
\pgfsetlinewidth{0.501875pt}%
\definecolor{currentstroke}{rgb}{0.000000,0.000000,0.000000}%
\pgfsetstrokecolor{currentstroke}%
\pgfsetdash{}{0pt}%
\pgfsys@defobject{currentmarker}{\pgfqpoint{0.000000in}{0.000000in}}{\pgfqpoint{0.000000in}{0.020833in}}{%
\pgfpathmoveto{\pgfqpoint{0.000000in}{0.000000in}}%
\pgfpathlineto{\pgfqpoint{0.000000in}{0.020833in}}%
\pgfusepath{stroke,fill}%
}%
\begin{pgfscope}%
\pgfsys@transformshift{2.373237in}{3.117349in}%
\pgfsys@useobject{currentmarker}{}%
\end{pgfscope}%
\end{pgfscope}%
\begin{pgfscope}%
\pgfsetbuttcap%
\pgfsetroundjoin%
\definecolor{currentfill}{rgb}{0.000000,0.000000,0.000000}%
\pgfsetfillcolor{currentfill}%
\pgfsetlinewidth{0.501875pt}%
\definecolor{currentstroke}{rgb}{0.000000,0.000000,0.000000}%
\pgfsetstrokecolor{currentstroke}%
\pgfsetdash{}{0pt}%
\pgfsys@defobject{currentmarker}{\pgfqpoint{0.000000in}{-0.020833in}}{\pgfqpoint{0.000000in}{0.000000in}}{%
\pgfpathmoveto{\pgfqpoint{0.000000in}{0.000000in}}%
\pgfpathlineto{\pgfqpoint{0.000000in}{-0.020833in}}%
\pgfusepath{stroke,fill}%
}%
\begin{pgfscope}%
\pgfsys@transformshift{2.373237in}{3.584600in}%
\pgfsys@useobject{currentmarker}{}%
\end{pgfscope}%
\end{pgfscope}%
\begin{pgfscope}%
\pgfsetbuttcap%
\pgfsetroundjoin%
\definecolor{currentfill}{rgb}{0.000000,0.000000,0.000000}%
\pgfsetfillcolor{currentfill}%
\pgfsetlinewidth{0.501875pt}%
\definecolor{currentstroke}{rgb}{0.000000,0.000000,0.000000}%
\pgfsetstrokecolor{currentstroke}%
\pgfsetdash{}{0pt}%
\pgfsys@defobject{currentmarker}{\pgfqpoint{0.000000in}{0.000000in}}{\pgfqpoint{0.000000in}{0.020833in}}{%
\pgfpathmoveto{\pgfqpoint{0.000000in}{0.000000in}}%
\pgfpathlineto{\pgfqpoint{0.000000in}{0.020833in}}%
\pgfusepath{stroke,fill}%
}%
\begin{pgfscope}%
\pgfsys@transformshift{2.469312in}{3.117349in}%
\pgfsys@useobject{currentmarker}{}%
\end{pgfscope}%
\end{pgfscope}%
\begin{pgfscope}%
\pgfsetbuttcap%
\pgfsetroundjoin%
\definecolor{currentfill}{rgb}{0.000000,0.000000,0.000000}%
\pgfsetfillcolor{currentfill}%
\pgfsetlinewidth{0.501875pt}%
\definecolor{currentstroke}{rgb}{0.000000,0.000000,0.000000}%
\pgfsetstrokecolor{currentstroke}%
\pgfsetdash{}{0pt}%
\pgfsys@defobject{currentmarker}{\pgfqpoint{0.000000in}{-0.020833in}}{\pgfqpoint{0.000000in}{0.000000in}}{%
\pgfpathmoveto{\pgfqpoint{0.000000in}{0.000000in}}%
\pgfpathlineto{\pgfqpoint{0.000000in}{-0.020833in}}%
\pgfusepath{stroke,fill}%
}%
\begin{pgfscope}%
\pgfsys@transformshift{2.469312in}{3.584600in}%
\pgfsys@useobject{currentmarker}{}%
\end{pgfscope}%
\end{pgfscope}%
\begin{pgfscope}%
\pgfsetbuttcap%
\pgfsetroundjoin%
\definecolor{currentfill}{rgb}{0.000000,0.000000,0.000000}%
\pgfsetfillcolor{currentfill}%
\pgfsetlinewidth{0.501875pt}%
\definecolor{currentstroke}{rgb}{0.000000,0.000000,0.000000}%
\pgfsetstrokecolor{currentstroke}%
\pgfsetdash{}{0pt}%
\pgfsys@defobject{currentmarker}{\pgfqpoint{0.000000in}{0.000000in}}{\pgfqpoint{0.000000in}{0.020833in}}{%
\pgfpathmoveto{\pgfqpoint{0.000000in}{0.000000in}}%
\pgfpathlineto{\pgfqpoint{0.000000in}{0.020833in}}%
\pgfusepath{stroke,fill}%
}%
\begin{pgfscope}%
\pgfsys@transformshift{2.661463in}{3.117349in}%
\pgfsys@useobject{currentmarker}{}%
\end{pgfscope}%
\end{pgfscope}%
\begin{pgfscope}%
\pgfsetbuttcap%
\pgfsetroundjoin%
\definecolor{currentfill}{rgb}{0.000000,0.000000,0.000000}%
\pgfsetfillcolor{currentfill}%
\pgfsetlinewidth{0.501875pt}%
\definecolor{currentstroke}{rgb}{0.000000,0.000000,0.000000}%
\pgfsetstrokecolor{currentstroke}%
\pgfsetdash{}{0pt}%
\pgfsys@defobject{currentmarker}{\pgfqpoint{0.000000in}{-0.020833in}}{\pgfqpoint{0.000000in}{0.000000in}}{%
\pgfpathmoveto{\pgfqpoint{0.000000in}{0.000000in}}%
\pgfpathlineto{\pgfqpoint{0.000000in}{-0.020833in}}%
\pgfusepath{stroke,fill}%
}%
\begin{pgfscope}%
\pgfsys@transformshift{2.661463in}{3.584600in}%
\pgfsys@useobject{currentmarker}{}%
\end{pgfscope}%
\end{pgfscope}%
\begin{pgfscope}%
\pgfsetbuttcap%
\pgfsetroundjoin%
\definecolor{currentfill}{rgb}{0.000000,0.000000,0.000000}%
\pgfsetfillcolor{currentfill}%
\pgfsetlinewidth{0.501875pt}%
\definecolor{currentstroke}{rgb}{0.000000,0.000000,0.000000}%
\pgfsetstrokecolor{currentstroke}%
\pgfsetdash{}{0pt}%
\pgfsys@defobject{currentmarker}{\pgfqpoint{0.000000in}{0.000000in}}{\pgfqpoint{0.000000in}{0.020833in}}{%
\pgfpathmoveto{\pgfqpoint{0.000000in}{0.000000in}}%
\pgfpathlineto{\pgfqpoint{0.000000in}{0.020833in}}%
\pgfusepath{stroke,fill}%
}%
\begin{pgfscope}%
\pgfsys@transformshift{2.757538in}{3.117349in}%
\pgfsys@useobject{currentmarker}{}%
\end{pgfscope}%
\end{pgfscope}%
\begin{pgfscope}%
\pgfsetbuttcap%
\pgfsetroundjoin%
\definecolor{currentfill}{rgb}{0.000000,0.000000,0.000000}%
\pgfsetfillcolor{currentfill}%
\pgfsetlinewidth{0.501875pt}%
\definecolor{currentstroke}{rgb}{0.000000,0.000000,0.000000}%
\pgfsetstrokecolor{currentstroke}%
\pgfsetdash{}{0pt}%
\pgfsys@defobject{currentmarker}{\pgfqpoint{0.000000in}{-0.020833in}}{\pgfqpoint{0.000000in}{0.000000in}}{%
\pgfpathmoveto{\pgfqpoint{0.000000in}{0.000000in}}%
\pgfpathlineto{\pgfqpoint{0.000000in}{-0.020833in}}%
\pgfusepath{stroke,fill}%
}%
\begin{pgfscope}%
\pgfsys@transformshift{2.757538in}{3.584600in}%
\pgfsys@useobject{currentmarker}{}%
\end{pgfscope}%
\end{pgfscope}%
\begin{pgfscope}%
\pgfsetbuttcap%
\pgfsetroundjoin%
\definecolor{currentfill}{rgb}{0.000000,0.000000,0.000000}%
\pgfsetfillcolor{currentfill}%
\pgfsetlinewidth{0.501875pt}%
\definecolor{currentstroke}{rgb}{0.000000,0.000000,0.000000}%
\pgfsetstrokecolor{currentstroke}%
\pgfsetdash{}{0pt}%
\pgfsys@defobject{currentmarker}{\pgfqpoint{0.000000in}{0.000000in}}{\pgfqpoint{0.000000in}{0.020833in}}{%
\pgfpathmoveto{\pgfqpoint{0.000000in}{0.000000in}}%
\pgfpathlineto{\pgfqpoint{0.000000in}{0.020833in}}%
\pgfusepath{stroke,fill}%
}%
\begin{pgfscope}%
\pgfsys@transformshift{2.853613in}{3.117349in}%
\pgfsys@useobject{currentmarker}{}%
\end{pgfscope}%
\end{pgfscope}%
\begin{pgfscope}%
\pgfsetbuttcap%
\pgfsetroundjoin%
\definecolor{currentfill}{rgb}{0.000000,0.000000,0.000000}%
\pgfsetfillcolor{currentfill}%
\pgfsetlinewidth{0.501875pt}%
\definecolor{currentstroke}{rgb}{0.000000,0.000000,0.000000}%
\pgfsetstrokecolor{currentstroke}%
\pgfsetdash{}{0pt}%
\pgfsys@defobject{currentmarker}{\pgfqpoint{0.000000in}{-0.020833in}}{\pgfqpoint{0.000000in}{0.000000in}}{%
\pgfpathmoveto{\pgfqpoint{0.000000in}{0.000000in}}%
\pgfpathlineto{\pgfqpoint{0.000000in}{-0.020833in}}%
\pgfusepath{stroke,fill}%
}%
\begin{pgfscope}%
\pgfsys@transformshift{2.853613in}{3.584600in}%
\pgfsys@useobject{currentmarker}{}%
\end{pgfscope}%
\end{pgfscope}%
\begin{pgfscope}%
\pgfsetbuttcap%
\pgfsetroundjoin%
\definecolor{currentfill}{rgb}{0.000000,0.000000,0.000000}%
\pgfsetfillcolor{currentfill}%
\pgfsetlinewidth{0.501875pt}%
\definecolor{currentstroke}{rgb}{0.000000,0.000000,0.000000}%
\pgfsetstrokecolor{currentstroke}%
\pgfsetdash{}{0pt}%
\pgfsys@defobject{currentmarker}{\pgfqpoint{0.000000in}{0.000000in}}{\pgfqpoint{0.000000in}{0.020833in}}{%
\pgfpathmoveto{\pgfqpoint{0.000000in}{0.000000in}}%
\pgfpathlineto{\pgfqpoint{0.000000in}{0.020833in}}%
\pgfusepath{stroke,fill}%
}%
\begin{pgfscope}%
\pgfsys@transformshift{2.949688in}{3.117349in}%
\pgfsys@useobject{currentmarker}{}%
\end{pgfscope}%
\end{pgfscope}%
\begin{pgfscope}%
\pgfsetbuttcap%
\pgfsetroundjoin%
\definecolor{currentfill}{rgb}{0.000000,0.000000,0.000000}%
\pgfsetfillcolor{currentfill}%
\pgfsetlinewidth{0.501875pt}%
\definecolor{currentstroke}{rgb}{0.000000,0.000000,0.000000}%
\pgfsetstrokecolor{currentstroke}%
\pgfsetdash{}{0pt}%
\pgfsys@defobject{currentmarker}{\pgfqpoint{0.000000in}{-0.020833in}}{\pgfqpoint{0.000000in}{0.000000in}}{%
\pgfpathmoveto{\pgfqpoint{0.000000in}{0.000000in}}%
\pgfpathlineto{\pgfqpoint{0.000000in}{-0.020833in}}%
\pgfusepath{stroke,fill}%
}%
\begin{pgfscope}%
\pgfsys@transformshift{2.949688in}{3.584600in}%
\pgfsys@useobject{currentmarker}{}%
\end{pgfscope}%
\end{pgfscope}%
\begin{pgfscope}%
\pgfsetbuttcap%
\pgfsetroundjoin%
\definecolor{currentfill}{rgb}{0.000000,0.000000,0.000000}%
\pgfsetfillcolor{currentfill}%
\pgfsetlinewidth{0.501875pt}%
\definecolor{currentstroke}{rgb}{0.000000,0.000000,0.000000}%
\pgfsetstrokecolor{currentstroke}%
\pgfsetdash{}{0pt}%
\pgfsys@defobject{currentmarker}{\pgfqpoint{0.000000in}{0.000000in}}{\pgfqpoint{0.000000in}{0.020833in}}{%
\pgfpathmoveto{\pgfqpoint{0.000000in}{0.000000in}}%
\pgfpathlineto{\pgfqpoint{0.000000in}{0.020833in}}%
\pgfusepath{stroke,fill}%
}%
\begin{pgfscope}%
\pgfsys@transformshift{3.141838in}{3.117349in}%
\pgfsys@useobject{currentmarker}{}%
\end{pgfscope}%
\end{pgfscope}%
\begin{pgfscope}%
\pgfsetbuttcap%
\pgfsetroundjoin%
\definecolor{currentfill}{rgb}{0.000000,0.000000,0.000000}%
\pgfsetfillcolor{currentfill}%
\pgfsetlinewidth{0.501875pt}%
\definecolor{currentstroke}{rgb}{0.000000,0.000000,0.000000}%
\pgfsetstrokecolor{currentstroke}%
\pgfsetdash{}{0pt}%
\pgfsys@defobject{currentmarker}{\pgfqpoint{0.000000in}{-0.020833in}}{\pgfqpoint{0.000000in}{0.000000in}}{%
\pgfpathmoveto{\pgfqpoint{0.000000in}{0.000000in}}%
\pgfpathlineto{\pgfqpoint{0.000000in}{-0.020833in}}%
\pgfusepath{stroke,fill}%
}%
\begin{pgfscope}%
\pgfsys@transformshift{3.141838in}{3.584600in}%
\pgfsys@useobject{currentmarker}{}%
\end{pgfscope}%
\end{pgfscope}%
\begin{pgfscope}%
\pgfsetbuttcap%
\pgfsetroundjoin%
\definecolor{currentfill}{rgb}{0.000000,0.000000,0.000000}%
\pgfsetfillcolor{currentfill}%
\pgfsetlinewidth{0.501875pt}%
\definecolor{currentstroke}{rgb}{0.000000,0.000000,0.000000}%
\pgfsetstrokecolor{currentstroke}%
\pgfsetdash{}{0pt}%
\pgfsys@defobject{currentmarker}{\pgfqpoint{0.000000in}{0.000000in}}{\pgfqpoint{0.000000in}{0.020833in}}{%
\pgfpathmoveto{\pgfqpoint{0.000000in}{0.000000in}}%
\pgfpathlineto{\pgfqpoint{0.000000in}{0.020833in}}%
\pgfusepath{stroke,fill}%
}%
\begin{pgfscope}%
\pgfsys@transformshift{3.237913in}{3.117349in}%
\pgfsys@useobject{currentmarker}{}%
\end{pgfscope}%
\end{pgfscope}%
\begin{pgfscope}%
\pgfsetbuttcap%
\pgfsetroundjoin%
\definecolor{currentfill}{rgb}{0.000000,0.000000,0.000000}%
\pgfsetfillcolor{currentfill}%
\pgfsetlinewidth{0.501875pt}%
\definecolor{currentstroke}{rgb}{0.000000,0.000000,0.000000}%
\pgfsetstrokecolor{currentstroke}%
\pgfsetdash{}{0pt}%
\pgfsys@defobject{currentmarker}{\pgfqpoint{0.000000in}{-0.020833in}}{\pgfqpoint{0.000000in}{0.000000in}}{%
\pgfpathmoveto{\pgfqpoint{0.000000in}{0.000000in}}%
\pgfpathlineto{\pgfqpoint{0.000000in}{-0.020833in}}%
\pgfusepath{stroke,fill}%
}%
\begin{pgfscope}%
\pgfsys@transformshift{3.237913in}{3.584600in}%
\pgfsys@useobject{currentmarker}{}%
\end{pgfscope}%
\end{pgfscope}%
\begin{pgfscope}%
\pgfsetbuttcap%
\pgfsetroundjoin%
\definecolor{currentfill}{rgb}{0.000000,0.000000,0.000000}%
\pgfsetfillcolor{currentfill}%
\pgfsetlinewidth{0.501875pt}%
\definecolor{currentstroke}{rgb}{0.000000,0.000000,0.000000}%
\pgfsetstrokecolor{currentstroke}%
\pgfsetdash{}{0pt}%
\pgfsys@defobject{currentmarker}{\pgfqpoint{0.000000in}{0.000000in}}{\pgfqpoint{0.000000in}{0.020833in}}{%
\pgfpathmoveto{\pgfqpoint{0.000000in}{0.000000in}}%
\pgfpathlineto{\pgfqpoint{0.000000in}{0.020833in}}%
\pgfusepath{stroke,fill}%
}%
\begin{pgfscope}%
\pgfsys@transformshift{3.333988in}{3.117349in}%
\pgfsys@useobject{currentmarker}{}%
\end{pgfscope}%
\end{pgfscope}%
\begin{pgfscope}%
\pgfsetbuttcap%
\pgfsetroundjoin%
\definecolor{currentfill}{rgb}{0.000000,0.000000,0.000000}%
\pgfsetfillcolor{currentfill}%
\pgfsetlinewidth{0.501875pt}%
\definecolor{currentstroke}{rgb}{0.000000,0.000000,0.000000}%
\pgfsetstrokecolor{currentstroke}%
\pgfsetdash{}{0pt}%
\pgfsys@defobject{currentmarker}{\pgfqpoint{0.000000in}{-0.020833in}}{\pgfqpoint{0.000000in}{0.000000in}}{%
\pgfpathmoveto{\pgfqpoint{0.000000in}{0.000000in}}%
\pgfpathlineto{\pgfqpoint{0.000000in}{-0.020833in}}%
\pgfusepath{stroke,fill}%
}%
\begin{pgfscope}%
\pgfsys@transformshift{3.333988in}{3.584600in}%
\pgfsys@useobject{currentmarker}{}%
\end{pgfscope}%
\end{pgfscope}%
\begin{pgfscope}%
\pgfsetbuttcap%
\pgfsetroundjoin%
\definecolor{currentfill}{rgb}{0.000000,0.000000,0.000000}%
\pgfsetfillcolor{currentfill}%
\pgfsetlinewidth{0.501875pt}%
\definecolor{currentstroke}{rgb}{0.000000,0.000000,0.000000}%
\pgfsetstrokecolor{currentstroke}%
\pgfsetdash{}{0pt}%
\pgfsys@defobject{currentmarker}{\pgfqpoint{0.000000in}{0.000000in}}{\pgfqpoint{0.000000in}{0.020833in}}{%
\pgfpathmoveto{\pgfqpoint{0.000000in}{0.000000in}}%
\pgfpathlineto{\pgfqpoint{0.000000in}{0.020833in}}%
\pgfusepath{stroke,fill}%
}%
\begin{pgfscope}%
\pgfsys@transformshift{3.430063in}{3.117349in}%
\pgfsys@useobject{currentmarker}{}%
\end{pgfscope}%
\end{pgfscope}%
\begin{pgfscope}%
\pgfsetbuttcap%
\pgfsetroundjoin%
\definecolor{currentfill}{rgb}{0.000000,0.000000,0.000000}%
\pgfsetfillcolor{currentfill}%
\pgfsetlinewidth{0.501875pt}%
\definecolor{currentstroke}{rgb}{0.000000,0.000000,0.000000}%
\pgfsetstrokecolor{currentstroke}%
\pgfsetdash{}{0pt}%
\pgfsys@defobject{currentmarker}{\pgfqpoint{0.000000in}{-0.020833in}}{\pgfqpoint{0.000000in}{0.000000in}}{%
\pgfpathmoveto{\pgfqpoint{0.000000in}{0.000000in}}%
\pgfpathlineto{\pgfqpoint{0.000000in}{-0.020833in}}%
\pgfusepath{stroke,fill}%
}%
\begin{pgfscope}%
\pgfsys@transformshift{3.430063in}{3.584600in}%
\pgfsys@useobject{currentmarker}{}%
\end{pgfscope}%
\end{pgfscope}%
\begin{pgfscope}%
\pgfsetbuttcap%
\pgfsetroundjoin%
\definecolor{currentfill}{rgb}{0.000000,0.000000,0.000000}%
\pgfsetfillcolor{currentfill}%
\pgfsetlinewidth{0.501875pt}%
\definecolor{currentstroke}{rgb}{0.000000,0.000000,0.000000}%
\pgfsetstrokecolor{currentstroke}%
\pgfsetdash{}{0pt}%
\pgfsys@defobject{currentmarker}{\pgfqpoint{0.000000in}{0.000000in}}{\pgfqpoint{0.000000in}{0.020833in}}{%
\pgfpathmoveto{\pgfqpoint{0.000000in}{0.000000in}}%
\pgfpathlineto{\pgfqpoint{0.000000in}{0.020833in}}%
\pgfusepath{stroke,fill}%
}%
\begin{pgfscope}%
\pgfsys@transformshift{3.622213in}{3.117349in}%
\pgfsys@useobject{currentmarker}{}%
\end{pgfscope}%
\end{pgfscope}%
\begin{pgfscope}%
\pgfsetbuttcap%
\pgfsetroundjoin%
\definecolor{currentfill}{rgb}{0.000000,0.000000,0.000000}%
\pgfsetfillcolor{currentfill}%
\pgfsetlinewidth{0.501875pt}%
\definecolor{currentstroke}{rgb}{0.000000,0.000000,0.000000}%
\pgfsetstrokecolor{currentstroke}%
\pgfsetdash{}{0pt}%
\pgfsys@defobject{currentmarker}{\pgfqpoint{0.000000in}{-0.020833in}}{\pgfqpoint{0.000000in}{0.000000in}}{%
\pgfpathmoveto{\pgfqpoint{0.000000in}{0.000000in}}%
\pgfpathlineto{\pgfqpoint{0.000000in}{-0.020833in}}%
\pgfusepath{stroke,fill}%
}%
\begin{pgfscope}%
\pgfsys@transformshift{3.622213in}{3.584600in}%
\pgfsys@useobject{currentmarker}{}%
\end{pgfscope}%
\end{pgfscope}%
\begin{pgfscope}%
\pgfsetbuttcap%
\pgfsetroundjoin%
\definecolor{currentfill}{rgb}{0.000000,0.000000,0.000000}%
\pgfsetfillcolor{currentfill}%
\pgfsetlinewidth{0.501875pt}%
\definecolor{currentstroke}{rgb}{0.000000,0.000000,0.000000}%
\pgfsetstrokecolor{currentstroke}%
\pgfsetdash{}{0pt}%
\pgfsys@defobject{currentmarker}{\pgfqpoint{0.000000in}{0.000000in}}{\pgfqpoint{0.000000in}{0.020833in}}{%
\pgfpathmoveto{\pgfqpoint{0.000000in}{0.000000in}}%
\pgfpathlineto{\pgfqpoint{0.000000in}{0.020833in}}%
\pgfusepath{stroke,fill}%
}%
\begin{pgfscope}%
\pgfsys@transformshift{3.718289in}{3.117349in}%
\pgfsys@useobject{currentmarker}{}%
\end{pgfscope}%
\end{pgfscope}%
\begin{pgfscope}%
\pgfsetbuttcap%
\pgfsetroundjoin%
\definecolor{currentfill}{rgb}{0.000000,0.000000,0.000000}%
\pgfsetfillcolor{currentfill}%
\pgfsetlinewidth{0.501875pt}%
\definecolor{currentstroke}{rgb}{0.000000,0.000000,0.000000}%
\pgfsetstrokecolor{currentstroke}%
\pgfsetdash{}{0pt}%
\pgfsys@defobject{currentmarker}{\pgfqpoint{0.000000in}{-0.020833in}}{\pgfqpoint{0.000000in}{0.000000in}}{%
\pgfpathmoveto{\pgfqpoint{0.000000in}{0.000000in}}%
\pgfpathlineto{\pgfqpoint{0.000000in}{-0.020833in}}%
\pgfusepath{stroke,fill}%
}%
\begin{pgfscope}%
\pgfsys@transformshift{3.718289in}{3.584600in}%
\pgfsys@useobject{currentmarker}{}%
\end{pgfscope}%
\end{pgfscope}%
\begin{pgfscope}%
\pgfsetbuttcap%
\pgfsetroundjoin%
\definecolor{currentfill}{rgb}{0.000000,0.000000,0.000000}%
\pgfsetfillcolor{currentfill}%
\pgfsetlinewidth{0.501875pt}%
\definecolor{currentstroke}{rgb}{0.000000,0.000000,0.000000}%
\pgfsetstrokecolor{currentstroke}%
\pgfsetdash{}{0pt}%
\pgfsys@defobject{currentmarker}{\pgfqpoint{0.000000in}{0.000000in}}{\pgfqpoint{0.000000in}{0.020833in}}{%
\pgfpathmoveto{\pgfqpoint{0.000000in}{0.000000in}}%
\pgfpathlineto{\pgfqpoint{0.000000in}{0.020833in}}%
\pgfusepath{stroke,fill}%
}%
\begin{pgfscope}%
\pgfsys@transformshift{3.814364in}{3.117349in}%
\pgfsys@useobject{currentmarker}{}%
\end{pgfscope}%
\end{pgfscope}%
\begin{pgfscope}%
\pgfsetbuttcap%
\pgfsetroundjoin%
\definecolor{currentfill}{rgb}{0.000000,0.000000,0.000000}%
\pgfsetfillcolor{currentfill}%
\pgfsetlinewidth{0.501875pt}%
\definecolor{currentstroke}{rgb}{0.000000,0.000000,0.000000}%
\pgfsetstrokecolor{currentstroke}%
\pgfsetdash{}{0pt}%
\pgfsys@defobject{currentmarker}{\pgfqpoint{0.000000in}{-0.020833in}}{\pgfqpoint{0.000000in}{0.000000in}}{%
\pgfpathmoveto{\pgfqpoint{0.000000in}{0.000000in}}%
\pgfpathlineto{\pgfqpoint{0.000000in}{-0.020833in}}%
\pgfusepath{stroke,fill}%
}%
\begin{pgfscope}%
\pgfsys@transformshift{3.814364in}{3.584600in}%
\pgfsys@useobject{currentmarker}{}%
\end{pgfscope}%
\end{pgfscope}%
\begin{pgfscope}%
\pgfsetbuttcap%
\pgfsetroundjoin%
\definecolor{currentfill}{rgb}{0.000000,0.000000,0.000000}%
\pgfsetfillcolor{currentfill}%
\pgfsetlinewidth{0.501875pt}%
\definecolor{currentstroke}{rgb}{0.000000,0.000000,0.000000}%
\pgfsetstrokecolor{currentstroke}%
\pgfsetdash{}{0pt}%
\pgfsys@defobject{currentmarker}{\pgfqpoint{0.000000in}{0.000000in}}{\pgfqpoint{0.000000in}{0.020833in}}{%
\pgfpathmoveto{\pgfqpoint{0.000000in}{0.000000in}}%
\pgfpathlineto{\pgfqpoint{0.000000in}{0.020833in}}%
\pgfusepath{stroke,fill}%
}%
\begin{pgfscope}%
\pgfsys@transformshift{3.910439in}{3.117349in}%
\pgfsys@useobject{currentmarker}{}%
\end{pgfscope}%
\end{pgfscope}%
\begin{pgfscope}%
\pgfsetbuttcap%
\pgfsetroundjoin%
\definecolor{currentfill}{rgb}{0.000000,0.000000,0.000000}%
\pgfsetfillcolor{currentfill}%
\pgfsetlinewidth{0.501875pt}%
\definecolor{currentstroke}{rgb}{0.000000,0.000000,0.000000}%
\pgfsetstrokecolor{currentstroke}%
\pgfsetdash{}{0pt}%
\pgfsys@defobject{currentmarker}{\pgfqpoint{0.000000in}{-0.020833in}}{\pgfqpoint{0.000000in}{0.000000in}}{%
\pgfpathmoveto{\pgfqpoint{0.000000in}{0.000000in}}%
\pgfpathlineto{\pgfqpoint{0.000000in}{-0.020833in}}%
\pgfusepath{stroke,fill}%
}%
\begin{pgfscope}%
\pgfsys@transformshift{3.910439in}{3.584600in}%
\pgfsys@useobject{currentmarker}{}%
\end{pgfscope}%
\end{pgfscope}%
\begin{pgfscope}%
\pgfsetbuttcap%
\pgfsetroundjoin%
\definecolor{currentfill}{rgb}{0.000000,0.000000,0.000000}%
\pgfsetfillcolor{currentfill}%
\pgfsetlinewidth{0.501875pt}%
\definecolor{currentstroke}{rgb}{0.000000,0.000000,0.000000}%
\pgfsetstrokecolor{currentstroke}%
\pgfsetdash{}{0pt}%
\pgfsys@defobject{currentmarker}{\pgfqpoint{0.000000in}{0.000000in}}{\pgfqpoint{0.000000in}{0.020833in}}{%
\pgfpathmoveto{\pgfqpoint{0.000000in}{0.000000in}}%
\pgfpathlineto{\pgfqpoint{0.000000in}{0.020833in}}%
\pgfusepath{stroke,fill}%
}%
\begin{pgfscope}%
\pgfsys@transformshift{4.102589in}{3.117349in}%
\pgfsys@useobject{currentmarker}{}%
\end{pgfscope}%
\end{pgfscope}%
\begin{pgfscope}%
\pgfsetbuttcap%
\pgfsetroundjoin%
\definecolor{currentfill}{rgb}{0.000000,0.000000,0.000000}%
\pgfsetfillcolor{currentfill}%
\pgfsetlinewidth{0.501875pt}%
\definecolor{currentstroke}{rgb}{0.000000,0.000000,0.000000}%
\pgfsetstrokecolor{currentstroke}%
\pgfsetdash{}{0pt}%
\pgfsys@defobject{currentmarker}{\pgfqpoint{0.000000in}{-0.020833in}}{\pgfqpoint{0.000000in}{0.000000in}}{%
\pgfpathmoveto{\pgfqpoint{0.000000in}{0.000000in}}%
\pgfpathlineto{\pgfqpoint{0.000000in}{-0.020833in}}%
\pgfusepath{stroke,fill}%
}%
\begin{pgfscope}%
\pgfsys@transformshift{4.102589in}{3.584600in}%
\pgfsys@useobject{currentmarker}{}%
\end{pgfscope}%
\end{pgfscope}%
\begin{pgfscope}%
\pgfsetbuttcap%
\pgfsetroundjoin%
\definecolor{currentfill}{rgb}{0.000000,0.000000,0.000000}%
\pgfsetfillcolor{currentfill}%
\pgfsetlinewidth{0.501875pt}%
\definecolor{currentstroke}{rgb}{0.000000,0.000000,0.000000}%
\pgfsetstrokecolor{currentstroke}%
\pgfsetdash{}{0pt}%
\pgfsys@defobject{currentmarker}{\pgfqpoint{0.000000in}{0.000000in}}{\pgfqpoint{0.000000in}{0.020833in}}{%
\pgfpathmoveto{\pgfqpoint{0.000000in}{0.000000in}}%
\pgfpathlineto{\pgfqpoint{0.000000in}{0.020833in}}%
\pgfusepath{stroke,fill}%
}%
\begin{pgfscope}%
\pgfsys@transformshift{4.198664in}{3.117349in}%
\pgfsys@useobject{currentmarker}{}%
\end{pgfscope}%
\end{pgfscope}%
\begin{pgfscope}%
\pgfsetbuttcap%
\pgfsetroundjoin%
\definecolor{currentfill}{rgb}{0.000000,0.000000,0.000000}%
\pgfsetfillcolor{currentfill}%
\pgfsetlinewidth{0.501875pt}%
\definecolor{currentstroke}{rgb}{0.000000,0.000000,0.000000}%
\pgfsetstrokecolor{currentstroke}%
\pgfsetdash{}{0pt}%
\pgfsys@defobject{currentmarker}{\pgfqpoint{0.000000in}{-0.020833in}}{\pgfqpoint{0.000000in}{0.000000in}}{%
\pgfpathmoveto{\pgfqpoint{0.000000in}{0.000000in}}%
\pgfpathlineto{\pgfqpoint{0.000000in}{-0.020833in}}%
\pgfusepath{stroke,fill}%
}%
\begin{pgfscope}%
\pgfsys@transformshift{4.198664in}{3.584600in}%
\pgfsys@useobject{currentmarker}{}%
\end{pgfscope}%
\end{pgfscope}%
\begin{pgfscope}%
\pgfsetbuttcap%
\pgfsetroundjoin%
\definecolor{currentfill}{rgb}{0.000000,0.000000,0.000000}%
\pgfsetfillcolor{currentfill}%
\pgfsetlinewidth{0.501875pt}%
\definecolor{currentstroke}{rgb}{0.000000,0.000000,0.000000}%
\pgfsetstrokecolor{currentstroke}%
\pgfsetdash{}{0pt}%
\pgfsys@defobject{currentmarker}{\pgfqpoint{0.000000in}{0.000000in}}{\pgfqpoint{0.000000in}{0.020833in}}{%
\pgfpathmoveto{\pgfqpoint{0.000000in}{0.000000in}}%
\pgfpathlineto{\pgfqpoint{0.000000in}{0.020833in}}%
\pgfusepath{stroke,fill}%
}%
\begin{pgfscope}%
\pgfsys@transformshift{4.294739in}{3.117349in}%
\pgfsys@useobject{currentmarker}{}%
\end{pgfscope}%
\end{pgfscope}%
\begin{pgfscope}%
\pgfsetbuttcap%
\pgfsetroundjoin%
\definecolor{currentfill}{rgb}{0.000000,0.000000,0.000000}%
\pgfsetfillcolor{currentfill}%
\pgfsetlinewidth{0.501875pt}%
\definecolor{currentstroke}{rgb}{0.000000,0.000000,0.000000}%
\pgfsetstrokecolor{currentstroke}%
\pgfsetdash{}{0pt}%
\pgfsys@defobject{currentmarker}{\pgfqpoint{0.000000in}{-0.020833in}}{\pgfqpoint{0.000000in}{0.000000in}}{%
\pgfpathmoveto{\pgfqpoint{0.000000in}{0.000000in}}%
\pgfpathlineto{\pgfqpoint{0.000000in}{-0.020833in}}%
\pgfusepath{stroke,fill}%
}%
\begin{pgfscope}%
\pgfsys@transformshift{4.294739in}{3.584600in}%
\pgfsys@useobject{currentmarker}{}%
\end{pgfscope}%
\end{pgfscope}%
\begin{pgfscope}%
\pgfsetbuttcap%
\pgfsetroundjoin%
\definecolor{currentfill}{rgb}{0.000000,0.000000,0.000000}%
\pgfsetfillcolor{currentfill}%
\pgfsetlinewidth{0.501875pt}%
\definecolor{currentstroke}{rgb}{0.000000,0.000000,0.000000}%
\pgfsetstrokecolor{currentstroke}%
\pgfsetdash{}{0pt}%
\pgfsys@defobject{currentmarker}{\pgfqpoint{0.000000in}{0.000000in}}{\pgfqpoint{0.000000in}{0.020833in}}{%
\pgfpathmoveto{\pgfqpoint{0.000000in}{0.000000in}}%
\pgfpathlineto{\pgfqpoint{0.000000in}{0.020833in}}%
\pgfusepath{stroke,fill}%
}%
\begin{pgfscope}%
\pgfsys@transformshift{4.390814in}{3.117349in}%
\pgfsys@useobject{currentmarker}{}%
\end{pgfscope}%
\end{pgfscope}%
\begin{pgfscope}%
\pgfsetbuttcap%
\pgfsetroundjoin%
\definecolor{currentfill}{rgb}{0.000000,0.000000,0.000000}%
\pgfsetfillcolor{currentfill}%
\pgfsetlinewidth{0.501875pt}%
\definecolor{currentstroke}{rgb}{0.000000,0.000000,0.000000}%
\pgfsetstrokecolor{currentstroke}%
\pgfsetdash{}{0pt}%
\pgfsys@defobject{currentmarker}{\pgfqpoint{0.000000in}{-0.020833in}}{\pgfqpoint{0.000000in}{0.000000in}}{%
\pgfpathmoveto{\pgfqpoint{0.000000in}{0.000000in}}%
\pgfpathlineto{\pgfqpoint{0.000000in}{-0.020833in}}%
\pgfusepath{stroke,fill}%
}%
\begin{pgfscope}%
\pgfsys@transformshift{4.390814in}{3.584600in}%
\pgfsys@useobject{currentmarker}{}%
\end{pgfscope}%
\end{pgfscope}%
\begin{pgfscope}%
\pgfsetbuttcap%
\pgfsetroundjoin%
\definecolor{currentfill}{rgb}{0.000000,0.000000,0.000000}%
\pgfsetfillcolor{currentfill}%
\pgfsetlinewidth{0.501875pt}%
\definecolor{currentstroke}{rgb}{0.000000,0.000000,0.000000}%
\pgfsetstrokecolor{currentstroke}%
\pgfsetdash{}{0pt}%
\pgfsys@defobject{currentmarker}{\pgfqpoint{0.000000in}{0.000000in}}{\pgfqpoint{0.000000in}{0.020833in}}{%
\pgfpathmoveto{\pgfqpoint{0.000000in}{0.000000in}}%
\pgfpathlineto{\pgfqpoint{0.000000in}{0.020833in}}%
\pgfusepath{stroke,fill}%
}%
\begin{pgfscope}%
\pgfsys@transformshift{4.582964in}{3.117349in}%
\pgfsys@useobject{currentmarker}{}%
\end{pgfscope}%
\end{pgfscope}%
\begin{pgfscope}%
\pgfsetbuttcap%
\pgfsetroundjoin%
\definecolor{currentfill}{rgb}{0.000000,0.000000,0.000000}%
\pgfsetfillcolor{currentfill}%
\pgfsetlinewidth{0.501875pt}%
\definecolor{currentstroke}{rgb}{0.000000,0.000000,0.000000}%
\pgfsetstrokecolor{currentstroke}%
\pgfsetdash{}{0pt}%
\pgfsys@defobject{currentmarker}{\pgfqpoint{0.000000in}{-0.020833in}}{\pgfqpoint{0.000000in}{0.000000in}}{%
\pgfpathmoveto{\pgfqpoint{0.000000in}{0.000000in}}%
\pgfpathlineto{\pgfqpoint{0.000000in}{-0.020833in}}%
\pgfusepath{stroke,fill}%
}%
\begin{pgfscope}%
\pgfsys@transformshift{4.582964in}{3.584600in}%
\pgfsys@useobject{currentmarker}{}%
\end{pgfscope}%
\end{pgfscope}%
\begin{pgfscope}%
\pgfsetbuttcap%
\pgfsetroundjoin%
\definecolor{currentfill}{rgb}{0.000000,0.000000,0.000000}%
\pgfsetfillcolor{currentfill}%
\pgfsetlinewidth{0.501875pt}%
\definecolor{currentstroke}{rgb}{0.000000,0.000000,0.000000}%
\pgfsetstrokecolor{currentstroke}%
\pgfsetdash{}{0pt}%
\pgfsys@defobject{currentmarker}{\pgfqpoint{0.000000in}{0.000000in}}{\pgfqpoint{0.041667in}{0.000000in}}{%
\pgfpathmoveto{\pgfqpoint{0.000000in}{0.000000in}}%
\pgfpathlineto{\pgfqpoint{0.041667in}{0.000000in}}%
\pgfusepath{stroke,fill}%
}%
\begin{pgfscope}%
\pgfsys@transformshift{0.444748in}{3.149046in}%
\pgfsys@useobject{currentmarker}{}%
\end{pgfscope}%
\end{pgfscope}%
\begin{pgfscope}%
\pgfsetbuttcap%
\pgfsetroundjoin%
\definecolor{currentfill}{rgb}{0.000000,0.000000,0.000000}%
\pgfsetfillcolor{currentfill}%
\pgfsetlinewidth{0.501875pt}%
\definecolor{currentstroke}{rgb}{0.000000,0.000000,0.000000}%
\pgfsetstrokecolor{currentstroke}%
\pgfsetdash{}{0pt}%
\pgfsys@defobject{currentmarker}{\pgfqpoint{-0.041667in}{0.000000in}}{\pgfqpoint{-0.000000in}{0.000000in}}{%
\pgfpathmoveto{\pgfqpoint{-0.000000in}{0.000000in}}%
\pgfpathlineto{\pgfqpoint{-0.041667in}{0.000000in}}%
\pgfusepath{stroke,fill}%
}%
\begin{pgfscope}%
\pgfsys@transformshift{4.676167in}{3.149046in}%
\pgfsys@useobject{currentmarker}{}%
\end{pgfscope}%
\end{pgfscope}%
\begin{pgfscope}%
\definecolor{textcolor}{rgb}{0.000000,0.000000,0.000000}%
\pgfsetstrokecolor{textcolor}%
\pgfsetfillcolor{textcolor}%
\pgftext[x=0.326693in, y=3.100828in, left, base]{\color{textcolor}\rmfamily\fontsize{10.000000}{12.000000}\selectfont \(\displaystyle {0}\)}%
\end{pgfscope}%
\begin{pgfscope}%
\pgfsetbuttcap%
\pgfsetroundjoin%
\definecolor{currentfill}{rgb}{0.000000,0.000000,0.000000}%
\pgfsetfillcolor{currentfill}%
\pgfsetlinewidth{0.501875pt}%
\definecolor{currentstroke}{rgb}{0.000000,0.000000,0.000000}%
\pgfsetstrokecolor{currentstroke}%
\pgfsetdash{}{0pt}%
\pgfsys@defobject{currentmarker}{\pgfqpoint{0.000000in}{0.000000in}}{\pgfqpoint{0.041667in}{0.000000in}}{%
\pgfpathmoveto{\pgfqpoint{0.000000in}{0.000000in}}%
\pgfpathlineto{\pgfqpoint{0.041667in}{0.000000in}}%
\pgfusepath{stroke,fill}%
}%
\begin{pgfscope}%
\pgfsys@transformshift{0.444748in}{3.435966in}%
\pgfsys@useobject{currentmarker}{}%
\end{pgfscope}%
\end{pgfscope}%
\begin{pgfscope}%
\pgfsetbuttcap%
\pgfsetroundjoin%
\definecolor{currentfill}{rgb}{0.000000,0.000000,0.000000}%
\pgfsetfillcolor{currentfill}%
\pgfsetlinewidth{0.501875pt}%
\definecolor{currentstroke}{rgb}{0.000000,0.000000,0.000000}%
\pgfsetstrokecolor{currentstroke}%
\pgfsetdash{}{0pt}%
\pgfsys@defobject{currentmarker}{\pgfqpoint{-0.041667in}{0.000000in}}{\pgfqpoint{-0.000000in}{0.000000in}}{%
\pgfpathmoveto{\pgfqpoint{-0.000000in}{0.000000in}}%
\pgfpathlineto{\pgfqpoint{-0.041667in}{0.000000in}}%
\pgfusepath{stroke,fill}%
}%
\begin{pgfscope}%
\pgfsys@transformshift{4.676167in}{3.435966in}%
\pgfsys@useobject{currentmarker}{}%
\end{pgfscope}%
\end{pgfscope}%
\begin{pgfscope}%
\definecolor{textcolor}{rgb}{0.000000,0.000000,0.000000}%
\pgfsetstrokecolor{textcolor}%
\pgfsetfillcolor{textcolor}%
\pgftext[x=0.257248in, y=3.387749in, left, base]{\color{textcolor}\rmfamily\fontsize{10.000000}{12.000000}\selectfont \(\displaystyle {25}\)}%
\end{pgfscope}%
\begin{pgfscope}%
\pgfsetbuttcap%
\pgfsetroundjoin%
\definecolor{currentfill}{rgb}{0.000000,0.000000,0.000000}%
\pgfsetfillcolor{currentfill}%
\pgfsetlinewidth{0.501875pt}%
\definecolor{currentstroke}{rgb}{0.000000,0.000000,0.000000}%
\pgfsetstrokecolor{currentstroke}%
\pgfsetdash{}{0pt}%
\pgfsys@defobject{currentmarker}{\pgfqpoint{0.000000in}{0.000000in}}{\pgfqpoint{0.020833in}{0.000000in}}{%
\pgfpathmoveto{\pgfqpoint{0.000000in}{0.000000in}}%
\pgfpathlineto{\pgfqpoint{0.020833in}{0.000000in}}%
\pgfusepath{stroke,fill}%
}%
\begin{pgfscope}%
\pgfsys@transformshift{0.444748in}{3.206430in}%
\pgfsys@useobject{currentmarker}{}%
\end{pgfscope}%
\end{pgfscope}%
\begin{pgfscope}%
\pgfsetbuttcap%
\pgfsetroundjoin%
\definecolor{currentfill}{rgb}{0.000000,0.000000,0.000000}%
\pgfsetfillcolor{currentfill}%
\pgfsetlinewidth{0.501875pt}%
\definecolor{currentstroke}{rgb}{0.000000,0.000000,0.000000}%
\pgfsetstrokecolor{currentstroke}%
\pgfsetdash{}{0pt}%
\pgfsys@defobject{currentmarker}{\pgfqpoint{-0.020833in}{0.000000in}}{\pgfqpoint{-0.000000in}{0.000000in}}{%
\pgfpathmoveto{\pgfqpoint{-0.000000in}{0.000000in}}%
\pgfpathlineto{\pgfqpoint{-0.020833in}{0.000000in}}%
\pgfusepath{stroke,fill}%
}%
\begin{pgfscope}%
\pgfsys@transformshift{4.676167in}{3.206430in}%
\pgfsys@useobject{currentmarker}{}%
\end{pgfscope}%
\end{pgfscope}%
\begin{pgfscope}%
\pgfsetbuttcap%
\pgfsetroundjoin%
\definecolor{currentfill}{rgb}{0.000000,0.000000,0.000000}%
\pgfsetfillcolor{currentfill}%
\pgfsetlinewidth{0.501875pt}%
\definecolor{currentstroke}{rgb}{0.000000,0.000000,0.000000}%
\pgfsetstrokecolor{currentstroke}%
\pgfsetdash{}{0pt}%
\pgfsys@defobject{currentmarker}{\pgfqpoint{0.000000in}{0.000000in}}{\pgfqpoint{0.020833in}{0.000000in}}{%
\pgfpathmoveto{\pgfqpoint{0.000000in}{0.000000in}}%
\pgfpathlineto{\pgfqpoint{0.020833in}{0.000000in}}%
\pgfusepath{stroke,fill}%
}%
\begin{pgfscope}%
\pgfsys@transformshift{0.444748in}{3.263814in}%
\pgfsys@useobject{currentmarker}{}%
\end{pgfscope}%
\end{pgfscope}%
\begin{pgfscope}%
\pgfsetbuttcap%
\pgfsetroundjoin%
\definecolor{currentfill}{rgb}{0.000000,0.000000,0.000000}%
\pgfsetfillcolor{currentfill}%
\pgfsetlinewidth{0.501875pt}%
\definecolor{currentstroke}{rgb}{0.000000,0.000000,0.000000}%
\pgfsetstrokecolor{currentstroke}%
\pgfsetdash{}{0pt}%
\pgfsys@defobject{currentmarker}{\pgfqpoint{-0.020833in}{0.000000in}}{\pgfqpoint{-0.000000in}{0.000000in}}{%
\pgfpathmoveto{\pgfqpoint{-0.000000in}{0.000000in}}%
\pgfpathlineto{\pgfqpoint{-0.020833in}{0.000000in}}%
\pgfusepath{stroke,fill}%
}%
\begin{pgfscope}%
\pgfsys@transformshift{4.676167in}{3.263814in}%
\pgfsys@useobject{currentmarker}{}%
\end{pgfscope}%
\end{pgfscope}%
\begin{pgfscope}%
\pgfsetbuttcap%
\pgfsetroundjoin%
\definecolor{currentfill}{rgb}{0.000000,0.000000,0.000000}%
\pgfsetfillcolor{currentfill}%
\pgfsetlinewidth{0.501875pt}%
\definecolor{currentstroke}{rgb}{0.000000,0.000000,0.000000}%
\pgfsetstrokecolor{currentstroke}%
\pgfsetdash{}{0pt}%
\pgfsys@defobject{currentmarker}{\pgfqpoint{0.000000in}{0.000000in}}{\pgfqpoint{0.020833in}{0.000000in}}{%
\pgfpathmoveto{\pgfqpoint{0.000000in}{0.000000in}}%
\pgfpathlineto{\pgfqpoint{0.020833in}{0.000000in}}%
\pgfusepath{stroke,fill}%
}%
\begin{pgfscope}%
\pgfsys@transformshift{0.444748in}{3.321198in}%
\pgfsys@useobject{currentmarker}{}%
\end{pgfscope}%
\end{pgfscope}%
\begin{pgfscope}%
\pgfsetbuttcap%
\pgfsetroundjoin%
\definecolor{currentfill}{rgb}{0.000000,0.000000,0.000000}%
\pgfsetfillcolor{currentfill}%
\pgfsetlinewidth{0.501875pt}%
\definecolor{currentstroke}{rgb}{0.000000,0.000000,0.000000}%
\pgfsetstrokecolor{currentstroke}%
\pgfsetdash{}{0pt}%
\pgfsys@defobject{currentmarker}{\pgfqpoint{-0.020833in}{0.000000in}}{\pgfqpoint{-0.000000in}{0.000000in}}{%
\pgfpathmoveto{\pgfqpoint{-0.000000in}{0.000000in}}%
\pgfpathlineto{\pgfqpoint{-0.020833in}{0.000000in}}%
\pgfusepath{stroke,fill}%
}%
\begin{pgfscope}%
\pgfsys@transformshift{4.676167in}{3.321198in}%
\pgfsys@useobject{currentmarker}{}%
\end{pgfscope}%
\end{pgfscope}%
\begin{pgfscope}%
\pgfsetbuttcap%
\pgfsetroundjoin%
\definecolor{currentfill}{rgb}{0.000000,0.000000,0.000000}%
\pgfsetfillcolor{currentfill}%
\pgfsetlinewidth{0.501875pt}%
\definecolor{currentstroke}{rgb}{0.000000,0.000000,0.000000}%
\pgfsetstrokecolor{currentstroke}%
\pgfsetdash{}{0pt}%
\pgfsys@defobject{currentmarker}{\pgfqpoint{0.000000in}{0.000000in}}{\pgfqpoint{0.020833in}{0.000000in}}{%
\pgfpathmoveto{\pgfqpoint{0.000000in}{0.000000in}}%
\pgfpathlineto{\pgfqpoint{0.020833in}{0.000000in}}%
\pgfusepath{stroke,fill}%
}%
\begin{pgfscope}%
\pgfsys@transformshift{0.444748in}{3.378582in}%
\pgfsys@useobject{currentmarker}{}%
\end{pgfscope}%
\end{pgfscope}%
\begin{pgfscope}%
\pgfsetbuttcap%
\pgfsetroundjoin%
\definecolor{currentfill}{rgb}{0.000000,0.000000,0.000000}%
\pgfsetfillcolor{currentfill}%
\pgfsetlinewidth{0.501875pt}%
\definecolor{currentstroke}{rgb}{0.000000,0.000000,0.000000}%
\pgfsetstrokecolor{currentstroke}%
\pgfsetdash{}{0pt}%
\pgfsys@defobject{currentmarker}{\pgfqpoint{-0.020833in}{0.000000in}}{\pgfqpoint{-0.000000in}{0.000000in}}{%
\pgfpathmoveto{\pgfqpoint{-0.000000in}{0.000000in}}%
\pgfpathlineto{\pgfqpoint{-0.020833in}{0.000000in}}%
\pgfusepath{stroke,fill}%
}%
\begin{pgfscope}%
\pgfsys@transformshift{4.676167in}{3.378582in}%
\pgfsys@useobject{currentmarker}{}%
\end{pgfscope}%
\end{pgfscope}%
\begin{pgfscope}%
\pgfsetbuttcap%
\pgfsetroundjoin%
\definecolor{currentfill}{rgb}{0.000000,0.000000,0.000000}%
\pgfsetfillcolor{currentfill}%
\pgfsetlinewidth{0.501875pt}%
\definecolor{currentstroke}{rgb}{0.000000,0.000000,0.000000}%
\pgfsetstrokecolor{currentstroke}%
\pgfsetdash{}{0pt}%
\pgfsys@defobject{currentmarker}{\pgfqpoint{0.000000in}{0.000000in}}{\pgfqpoint{0.020833in}{0.000000in}}{%
\pgfpathmoveto{\pgfqpoint{0.000000in}{0.000000in}}%
\pgfpathlineto{\pgfqpoint{0.020833in}{0.000000in}}%
\pgfusepath{stroke,fill}%
}%
\begin{pgfscope}%
\pgfsys@transformshift{0.444748in}{3.493351in}%
\pgfsys@useobject{currentmarker}{}%
\end{pgfscope}%
\end{pgfscope}%
\begin{pgfscope}%
\pgfsetbuttcap%
\pgfsetroundjoin%
\definecolor{currentfill}{rgb}{0.000000,0.000000,0.000000}%
\pgfsetfillcolor{currentfill}%
\pgfsetlinewidth{0.501875pt}%
\definecolor{currentstroke}{rgb}{0.000000,0.000000,0.000000}%
\pgfsetstrokecolor{currentstroke}%
\pgfsetdash{}{0pt}%
\pgfsys@defobject{currentmarker}{\pgfqpoint{-0.020833in}{0.000000in}}{\pgfqpoint{-0.000000in}{0.000000in}}{%
\pgfpathmoveto{\pgfqpoint{-0.000000in}{0.000000in}}%
\pgfpathlineto{\pgfqpoint{-0.020833in}{0.000000in}}%
\pgfusepath{stroke,fill}%
}%
\begin{pgfscope}%
\pgfsys@transformshift{4.676167in}{3.493351in}%
\pgfsys@useobject{currentmarker}{}%
\end{pgfscope}%
\end{pgfscope}%
\begin{pgfscope}%
\pgfsetbuttcap%
\pgfsetroundjoin%
\definecolor{currentfill}{rgb}{0.000000,0.000000,0.000000}%
\pgfsetfillcolor{currentfill}%
\pgfsetlinewidth{0.501875pt}%
\definecolor{currentstroke}{rgb}{0.000000,0.000000,0.000000}%
\pgfsetstrokecolor{currentstroke}%
\pgfsetdash{}{0pt}%
\pgfsys@defobject{currentmarker}{\pgfqpoint{0.000000in}{0.000000in}}{\pgfqpoint{0.020833in}{0.000000in}}{%
\pgfpathmoveto{\pgfqpoint{0.000000in}{0.000000in}}%
\pgfpathlineto{\pgfqpoint{0.020833in}{0.000000in}}%
\pgfusepath{stroke,fill}%
}%
\begin{pgfscope}%
\pgfsys@transformshift{0.444748in}{3.550735in}%
\pgfsys@useobject{currentmarker}{}%
\end{pgfscope}%
\end{pgfscope}%
\begin{pgfscope}%
\pgfsetbuttcap%
\pgfsetroundjoin%
\definecolor{currentfill}{rgb}{0.000000,0.000000,0.000000}%
\pgfsetfillcolor{currentfill}%
\pgfsetlinewidth{0.501875pt}%
\definecolor{currentstroke}{rgb}{0.000000,0.000000,0.000000}%
\pgfsetstrokecolor{currentstroke}%
\pgfsetdash{}{0pt}%
\pgfsys@defobject{currentmarker}{\pgfqpoint{-0.020833in}{0.000000in}}{\pgfqpoint{-0.000000in}{0.000000in}}{%
\pgfpathmoveto{\pgfqpoint{-0.000000in}{0.000000in}}%
\pgfpathlineto{\pgfqpoint{-0.020833in}{0.000000in}}%
\pgfusepath{stroke,fill}%
}%
\begin{pgfscope}%
\pgfsys@transformshift{4.676167in}{3.550735in}%
\pgfsys@useobject{currentmarker}{}%
\end{pgfscope}%
\end{pgfscope}%
\begin{pgfscope}%
\definecolor{textcolor}{rgb}{0.000000,0.000000,0.000000}%
\pgfsetstrokecolor{textcolor}%
\pgfsetfillcolor{textcolor}%
\pgftext[x=0.201692in,y=3.350974in,,bottom,rotate=90.000000]{\color{textcolor}\rmfamily\fontsize{12.000000}{14.400000}\selectfont \(\displaystyle V_s\) (\unit{\micro\volt})}%
\end{pgfscope}%
\begin{pgfscope}%
\pgfpathrectangle{\pgfqpoint{0.444748in}{3.117349in}}{\pgfqpoint{4.231419in}{0.467251in}}%
\pgfusepath{clip}%
\pgfsetbuttcap%
\pgfsetroundjoin%
\pgfsetlinewidth{1.003750pt}%
\definecolor{currentstroke}{rgb}{0.047059,0.364706,0.647059}%
\pgfsetstrokecolor{currentstroke}%
\pgfsetdash{{3.700000pt}{1.600000pt}}{0.000000pt}%
\pgfpathmoveto{\pgfqpoint{0.637791in}{3.358858in}}%
\pgfpathlineto{\pgfqpoint{0.656802in}{3.208523in}}%
\pgfpathlineto{\pgfqpoint{0.677224in}{3.160998in}}%
\pgfpathlineto{\pgfqpoint{0.694360in}{3.147068in}}%
\pgfpathlineto{\pgfqpoint{0.715250in}{3.182987in}}%
\pgfpathlineto{\pgfqpoint{0.732386in}{3.262648in}}%
\pgfpathlineto{\pgfqpoint{0.754217in}{3.483913in}}%
\pgfpathlineto{\pgfqpoint{0.772056in}{3.534292in}}%
\pgfpathlineto{\pgfqpoint{0.791303in}{3.396729in}}%
\pgfpathlineto{\pgfqpoint{0.808908in}{3.230300in}}%
\pgfpathlineto{\pgfqpoint{0.829094in}{3.166305in}}%
\pgfpathlineto{\pgfqpoint{0.849750in}{3.144184in}}%
\pgfpathlineto{\pgfqpoint{0.868763in}{3.145156in}}%
\pgfpathlineto{\pgfqpoint{0.887776in}{3.165414in}}%
\pgfpathlineto{\pgfqpoint{0.905852in}{3.222613in}}%
\pgfpathlineto{\pgfqpoint{0.925803in}{3.394154in}}%
\pgfpathlineto{\pgfqpoint{0.943876in}{3.525339in}}%
\pgfpathlineto{\pgfqpoint{0.965003in}{3.448474in}}%
\pgfpathlineto{\pgfqpoint{0.984720in}{3.259834in}}%
\pgfpathlineto{\pgfqpoint{1.001385in}{3.180450in}}%
\pgfpathlineto{\pgfqpoint{1.021104in}{3.147174in}}%
\pgfpathlineto{\pgfqpoint{1.041760in}{3.141188in}}%
\pgfpathlineto{\pgfqpoint{1.061007in}{3.151904in}}%
\pgfpathlineto{\pgfqpoint{1.079786in}{3.182568in}}%
\pgfpathlineto{\pgfqpoint{1.095983in}{3.203081in}}%
\pgfpathlineto{\pgfqpoint{1.118985in}{3.337288in}}%
\pgfpathlineto{\pgfqpoint{1.136824in}{3.500599in}}%
\pgfpathlineto{\pgfqpoint{1.156543in}{3.458174in}}%
\pgfpathlineto{\pgfqpoint{1.175319in}{3.294586in}}%
\pgfpathlineto{\pgfqpoint{1.194098in}{3.186050in}}%
\pgfpathlineto{\pgfqpoint{1.213112in}{3.149745in}}%
\pgfpathlineto{\pgfqpoint{1.235176in}{3.140828in}}%
\pgfpathlineto{\pgfqpoint{1.251138in}{3.145006in}}%
\pgfpathlineto{\pgfqpoint{1.270151in}{3.164123in}}%
\pgfpathlineto{\pgfqpoint{1.292919in}{3.241009in}}%
\pgfpathlineto{\pgfqpoint{1.311698in}{3.402717in}}%
\pgfpathlineto{\pgfqpoint{1.330006in}{3.497522in}}%
\pgfpathlineto{\pgfqpoint{1.349959in}{3.430815in}}%
\pgfpathlineto{\pgfqpoint{1.368738in}{3.284169in}}%
\pgfpathlineto{\pgfqpoint{1.387517in}{3.182669in}}%
\pgfpathlineto{\pgfqpoint{1.406294in}{3.150180in}}%
\pgfpathlineto{\pgfqpoint{1.425072in}{3.141940in}}%
\pgfpathlineto{\pgfqpoint{1.446903in}{3.144482in}}%
\pgfpathlineto{\pgfqpoint{1.463802in}{3.157369in}}%
\pgfpathlineto{\pgfqpoint{1.482815in}{3.196711in}}%
\pgfpathlineto{\pgfqpoint{1.500889in}{3.300855in}}%
\pgfpathlineto{\pgfqpoint{1.523424in}{3.492503in}}%
\pgfpathlineto{\pgfqpoint{1.538681in}{3.476091in}}%
\pgfpathlineto{\pgfqpoint{1.580933in}{3.212753in}}%
\pgfpathlineto{\pgfqpoint{1.596659in}{3.164916in}}%
\pgfpathlineto{\pgfqpoint{1.615908in}{3.145352in}}%
\pgfpathlineto{\pgfqpoint{1.639145in}{3.140289in}}%
\pgfpathlineto{\pgfqpoint{1.654638in}{3.144584in}}%
\pgfpathlineto{\pgfqpoint{1.676703in}{3.165698in}}%
\pgfpathlineto{\pgfqpoint{1.696185in}{3.214963in}}%
\pgfpathlineto{\pgfqpoint{1.713790in}{3.354844in}}%
\pgfpathlineto{\pgfqpoint{1.733740in}{3.463787in}}%
\pgfpathlineto{\pgfqpoint{1.749937in}{3.490407in}}%
\pgfpathlineto{\pgfqpoint{1.770829in}{3.400063in}}%
\pgfpathlineto{\pgfqpoint{1.792658in}{3.224413in}}%
\pgfpathlineto{\pgfqpoint{1.809559in}{3.169517in}}%
\pgfpathlineto{\pgfqpoint{1.825990in}{3.152709in}}%
\pgfpathlineto{\pgfqpoint{1.826459in}{3.143637in}}%
\pgfpathlineto{\pgfqpoint{1.848523in}{3.140353in}}%
\pgfpathlineto{\pgfqpoint{1.868476in}{3.141130in}}%
\pgfpathlineto{\pgfqpoint{1.907206in}{3.163228in}}%
\pgfpathlineto{\pgfqpoint{1.926454in}{3.223162in}}%
\pgfpathlineto{\pgfqpoint{1.946407in}{3.335926in}}%
\pgfpathlineto{\pgfqpoint{1.965185in}{3.476271in}}%
\pgfpathlineto{\pgfqpoint{1.986545in}{3.460383in}}%
\pgfpathlineto{\pgfqpoint{2.001333in}{3.357557in}}%
\pgfpathlineto{\pgfqpoint{2.022694in}{3.244484in}}%
\pgfpathlineto{\pgfqpoint{2.039594in}{3.170806in}}%
\pgfpathlineto{\pgfqpoint{2.061189in}{3.145518in}}%
\pgfpathlineto{\pgfqpoint{2.077151in}{3.140305in}}%
\pgfpathlineto{\pgfqpoint{2.114707in}{3.143546in}}%
\pgfpathlineto{\pgfqpoint{2.139120in}{3.161693in}}%
\pgfpathlineto{\pgfqpoint{2.154142in}{3.198712in}}%
\pgfpathlineto{\pgfqpoint{2.174798in}{3.324871in}}%
\pgfpathlineto{\pgfqpoint{2.192872in}{3.458713in}}%
\pgfpathlineto{\pgfqpoint{2.213762in}{3.482937in}}%
\pgfpathlineto{\pgfqpoint{2.232072in}{3.439718in}}%
\pgfpathlineto{\pgfqpoint{2.254371in}{3.288269in}}%
\pgfpathlineto{\pgfqpoint{2.271976in}{3.198817in}}%
\pgfpathlineto{\pgfqpoint{2.291693in}{3.161802in}}%
\pgfpathlineto{\pgfqpoint{2.311175in}{3.145267in}}%
\pgfpathlineto{\pgfqpoint{2.329719in}{3.139532in}}%
\pgfpathlineto{\pgfqpoint{2.348264in}{3.142305in}}%
\pgfpathlineto{\pgfqpoint{2.366806in}{3.146926in}}%
\pgfpathlineto{\pgfqpoint{2.384411in}{3.165829in}}%
\pgfpathlineto{\pgfqpoint{2.406007in}{3.221564in}}%
\pgfpathlineto{\pgfqpoint{2.422672in}{3.155509in}}%
\pgfpathlineto{\pgfqpoint{2.443328in}{3.188967in}}%
\pgfpathlineto{\pgfqpoint{2.462341in}{3.278010in}}%
\pgfpathlineto{\pgfqpoint{2.500368in}{3.484225in}}%
\pgfpathlineto{\pgfqpoint{2.521727in}{3.429345in}}%
\pgfpathlineto{\pgfqpoint{2.539566in}{3.276946in}}%
\pgfpathlineto{\pgfqpoint{2.557408in}{3.202741in}}%
\pgfpathlineto{\pgfqpoint{2.578767in}{3.154139in}}%
\pgfpathlineto{\pgfqpoint{2.596841in}{3.146755in}}%
\pgfpathlineto{\pgfqpoint{2.618202in}{3.139748in}}%
\pgfpathlineto{\pgfqpoint{2.634867in}{3.142749in}}%
\pgfpathlineto{\pgfqpoint{2.655523in}{3.153499in}}%
\pgfpathlineto{\pgfqpoint{2.674771in}{3.192247in}}%
\pgfpathlineto{\pgfqpoint{2.692610in}{3.281578in}}%
\pgfpathlineto{\pgfqpoint{2.712797in}{3.418697in}}%
\pgfpathlineto{\pgfqpoint{2.730871in}{3.485817in}}%
\pgfpathlineto{\pgfqpoint{2.748947in}{3.439223in}}%
\pgfpathlineto{\pgfqpoint{2.770306in}{3.291125in}}%
\pgfpathlineto{\pgfqpoint{2.788850in}{3.208172in}}%
\pgfpathlineto{\pgfqpoint{2.809975in}{3.158020in}}%
\pgfpathlineto{\pgfqpoint{2.827580in}{3.145282in}}%
\pgfpathlineto{\pgfqpoint{2.845420in}{3.140463in}}%
\pgfpathlineto{\pgfqpoint{2.867250in}{3.142216in}}%
\pgfpathlineto{\pgfqpoint{2.884386in}{3.153215in}}%
\pgfpathlineto{\pgfqpoint{2.903397in}{3.178614in}}%
\pgfpathlineto{\pgfqpoint{2.924053in}{3.256927in}}%
\pgfpathlineto{\pgfqpoint{2.942129in}{3.404050in}}%
\pgfpathlineto{\pgfqpoint{2.961142in}{3.480689in}}%
\pgfpathlineto{\pgfqpoint{2.980390in}{3.474066in}}%
\pgfpathlineto{\pgfqpoint{3.001046in}{3.342208in}}%
\pgfpathlineto{\pgfqpoint{3.018885in}{3.236304in}}%
\pgfpathlineto{\pgfqpoint{3.038367in}{3.175195in}}%
\pgfpathlineto{\pgfqpoint{3.058084in}{3.149860in}}%
\pgfpathlineto{\pgfqpoint{3.076628in}{3.143097in}}%
\pgfpathlineto{\pgfqpoint{3.094702in}{3.140823in}}%
\pgfpathlineto{\pgfqpoint{3.116766in}{3.146407in}}%
\pgfpathlineto{\pgfqpoint{3.134606in}{3.155362in}}%
\pgfpathlineto{\pgfqpoint{3.153150in}{3.182344in}}%
\pgfpathlineto{\pgfqpoint{3.173806in}{3.260469in}}%
\pgfpathlineto{\pgfqpoint{3.194228in}{3.407957in}}%
\pgfpathlineto{\pgfqpoint{3.212067in}{3.478552in}}%
\pgfpathlineto{\pgfqpoint{3.232018in}{3.496614in}}%
\pgfpathlineto{\pgfqpoint{3.250094in}{3.422020in}}%
\pgfpathlineto{\pgfqpoint{3.269576in}{3.326807in}}%
\pgfpathlineto{\pgfqpoint{3.290935in}{3.214043in}}%
\pgfpathlineto{\pgfqpoint{3.309011in}{3.170517in}}%
\pgfpathlineto{\pgfqpoint{3.326615in}{3.151404in}}%
\pgfpathlineto{\pgfqpoint{3.346566in}{3.141772in}}%
\pgfpathlineto{\pgfqpoint{3.364876in}{3.141925in}}%
\pgfpathlineto{\pgfqpoint{3.384593in}{3.147971in}}%
\pgfpathlineto{\pgfqpoint{3.405483in}{3.149448in}}%
\pgfpathlineto{\pgfqpoint{3.422619in}{3.142104in}}%
\pgfpathlineto{\pgfqpoint{3.440693in}{3.143317in}}%
\pgfpathlineto{\pgfqpoint{3.460646in}{3.148703in}}%
\pgfpathlineto{\pgfqpoint{3.481771in}{3.168952in}}%
\pgfpathlineto{\pgfqpoint{3.502661in}{3.226495in}}%
\pgfpathlineto{\pgfqpoint{3.517920in}{3.322471in}}%
\pgfpathlineto{\pgfqpoint{3.537168in}{3.486041in}}%
\pgfpathlineto{\pgfqpoint{3.559701in}{3.500524in}}%
\pgfpathlineto{\pgfqpoint{3.577775in}{3.423273in}}%
\pgfpathlineto{\pgfqpoint{3.595380in}{3.288186in}}%
\pgfpathlineto{\pgfqpoint{3.616505in}{3.198559in}}%
\pgfpathlineto{\pgfqpoint{3.635754in}{3.161851in}}%
\pgfpathlineto{\pgfqpoint{3.674015in}{3.142321in}}%
\pgfpathlineto{\pgfqpoint{3.691854in}{3.142736in}}%
\pgfpathlineto{\pgfqpoint{3.709928in}{3.147160in}}%
\pgfpathlineto{\pgfqpoint{3.730819in}{3.163831in}}%
\pgfpathlineto{\pgfqpoint{3.752414in}{3.202813in}}%
\pgfpathlineto{\pgfqpoint{3.768140in}{3.167239in}}%
\pgfpathlineto{\pgfqpoint{3.789267in}{3.227449in}}%
\pgfpathlineto{\pgfqpoint{3.826354in}{3.510828in}}%
\pgfpathlineto{\pgfqpoint{3.844427in}{3.515845in}}%
\pgfpathlineto{\pgfqpoint{3.866492in}{3.407651in}}%
\pgfpathlineto{\pgfqpoint{3.883862in}{3.284140in}}%
\pgfpathlineto{\pgfqpoint{3.904519in}{3.189580in}}%
\pgfpathlineto{\pgfqpoint{3.923532in}{3.160413in}}%
\pgfpathlineto{\pgfqpoint{3.941605in}{3.147105in}}%
\pgfpathlineto{\pgfqpoint{3.962496in}{3.142378in}}%
\pgfpathlineto{\pgfqpoint{3.980572in}{3.146221in}}%
\pgfpathlineto{\pgfqpoint{3.997706in}{3.154541in}}%
\pgfpathlineto{\pgfqpoint{4.019301in}{3.180682in}}%
\pgfpathlineto{\pgfqpoint{4.037844in}{3.234124in}}%
\pgfpathlineto{\pgfqpoint{4.060848in}{3.399044in}}%
\pgfpathlineto{\pgfqpoint{4.077513in}{3.495154in}}%
\pgfpathlineto{\pgfqpoint{4.097935in}{3.544921in}}%
\pgfpathlineto{\pgfqpoint{4.112723in}{3.501930in}}%
\pgfpathlineto{\pgfqpoint{4.133850in}{3.393537in}}%
\pgfpathlineto{\pgfqpoint{4.155914in}{3.261363in}}%
\pgfpathlineto{\pgfqpoint{4.173048in}{3.197774in}}%
\pgfpathlineto{\pgfqpoint{4.193939in}{3.162896in}}%
\pgfpathlineto{\pgfqpoint{4.212249in}{3.147535in}}%
\pgfpathlineto{\pgfqpoint{4.229854in}{3.143012in}}%
\pgfpathlineto{\pgfqpoint{4.250979in}{3.148973in}}%
\pgfpathlineto{\pgfqpoint{4.270697in}{3.161386in}}%
\pgfpathlineto{\pgfqpoint{4.287362in}{3.189897in}}%
\pgfpathlineto{\pgfqpoint{4.308487in}{3.245993in}}%
\pgfpathlineto{\pgfqpoint{4.326092in}{3.357425in}}%
\pgfpathlineto{\pgfqpoint{4.344637in}{3.491699in}}%
\pgfpathlineto{\pgfqpoint{4.365527in}{3.563361in}}%
\pgfpathlineto{\pgfqpoint{4.384775in}{3.516129in}}%
\pgfpathlineto{\pgfqpoint{4.401674in}{3.416170in}}%
\pgfpathlineto{\pgfqpoint{4.422331in}{3.279664in}}%
\pgfpathlineto{\pgfqpoint{4.441815in}{3.206355in}}%
\pgfpathlineto{\pgfqpoint{4.463879in}{3.164360in}}%
\pgfpathlineto{\pgfqpoint{4.480076in}{3.150582in}}%
\pgfpathlineto{\pgfqpoint{4.473268in}{3.157134in}}%
\pgfpathlineto{\pgfqpoint{4.454960in}{3.222958in}}%
\pgfpathlineto{\pgfqpoint{4.433364in}{3.406541in}}%
\pgfpathlineto{\pgfqpoint{4.417636in}{3.529728in}}%
\pgfpathlineto{\pgfqpoint{4.396746in}{3.530766in}}%
\pgfpathlineto{\pgfqpoint{4.379846in}{3.346099in}}%
\pgfpathlineto{\pgfqpoint{4.360598in}{3.166152in}}%
\pgfpathlineto{\pgfqpoint{4.338298in}{3.143610in}}%
\pgfpathlineto{\pgfqpoint{4.319989in}{3.145573in}}%
\pgfpathlineto{\pgfqpoint{4.299333in}{3.171112in}}%
\pgfpathlineto{\pgfqpoint{4.283606in}{3.222945in}}%
\pgfpathlineto{\pgfqpoint{4.262715in}{3.408414in}}%
\pgfpathlineto{\pgfqpoint{4.245110in}{3.534819in}}%
\pgfpathlineto{\pgfqpoint{4.223751in}{3.484532in}}%
\pgfpathlineto{\pgfqpoint{4.206146in}{3.289938in}}%
\pgfpathlineto{\pgfqpoint{4.186193in}{3.184681in}}%
\pgfpathlineto{\pgfqpoint{4.167885in}{3.151794in}}%
\pgfpathlineto{\pgfqpoint{4.147229in}{3.142020in}}%
\pgfpathlineto{\pgfqpoint{4.129154in}{3.147734in}}%
\pgfpathlineto{\pgfqpoint{4.108263in}{3.184102in}}%
\pgfpathlineto{\pgfqpoint{4.085495in}{3.317459in}}%
\pgfpathlineto{\pgfqpoint{4.071881in}{3.462730in}}%
\pgfpathlineto{\pgfqpoint{4.050989in}{3.529983in}}%
\pgfpathlineto{\pgfqpoint{4.033620in}{3.391096in}}%
\pgfpathlineto{\pgfqpoint{4.013902in}{3.214856in}}%
\pgfpathlineto{\pgfqpoint{3.993246in}{3.157918in}}%
\pgfpathlineto{\pgfqpoint{3.975407in}{3.144281in}}%
\pgfpathlineto{\pgfqpoint{3.956159in}{3.142345in}}%
\pgfpathlineto{\pgfqpoint{3.938320in}{3.155930in}}%
\pgfpathlineto{\pgfqpoint{3.917195in}{3.211217in}}%
\pgfpathlineto{\pgfqpoint{3.896773in}{3.373563in}}%
\pgfpathlineto{\pgfqpoint{3.878934in}{3.502319in}}%
\pgfpathlineto{\pgfqpoint{3.858041in}{3.470358in}}%
\pgfpathlineto{\pgfqpoint{3.840907in}{3.310677in}}%
\pgfpathlineto{\pgfqpoint{3.822597in}{3.187632in}}%
\pgfpathlineto{\pgfqpoint{3.802881in}{3.152509in}}%
\pgfpathlineto{\pgfqpoint{3.782459in}{3.140739in}}%
\pgfpathlineto{\pgfqpoint{3.763680in}{3.143394in}}%
\pgfpathlineto{\pgfqpoint{3.743024in}{3.163319in}}%
\pgfpathlineto{\pgfqpoint{3.725185in}{3.217343in}}%
\pgfpathlineto{\pgfqpoint{3.703825in}{3.406288in}}%
\pgfpathlineto{\pgfqpoint{3.685986in}{3.502899in}}%
\pgfpathlineto{\pgfqpoint{3.667676in}{3.464826in}}%
\pgfpathlineto{\pgfqpoint{3.650542in}{3.301089in}}%
\pgfpathlineto{\pgfqpoint{3.626598in}{3.184336in}}%
\pgfpathlineto{\pgfqpoint{3.609230in}{3.154128in}}%
\pgfpathlineto{\pgfqpoint{3.592328in}{3.142722in}}%
\pgfpathlineto{\pgfqpoint{3.572377in}{3.140365in}}%
\pgfpathlineto{\pgfqpoint{3.553599in}{3.150284in}}%
\pgfpathlineto{\pgfqpoint{3.532708in}{3.169457in}}%
\pgfpathlineto{\pgfqpoint{3.514398in}{3.171442in}}%
\pgfpathlineto{\pgfqpoint{3.493742in}{3.264894in}}%
\pgfpathlineto{\pgfqpoint{3.473791in}{3.432252in}}%
\pgfpathlineto{\pgfqpoint{3.455481in}{3.499565in}}%
\pgfpathlineto{\pgfqpoint{3.438347in}{3.468164in}}%
\pgfpathlineto{\pgfqpoint{3.415812in}{3.274732in}}%
\pgfpathlineto{\pgfqpoint{3.397503in}{3.182766in}}%
\pgfpathlineto{\pgfqpoint{3.379899in}{3.157092in}}%
\pgfpathlineto{\pgfqpoint{3.361589in}{3.143426in}}%
\pgfpathlineto{\pgfqpoint{3.339760in}{3.140063in}}%
\pgfpathlineto{\pgfqpoint{3.321685in}{3.146643in}}%
\pgfpathlineto{\pgfqpoint{3.300794in}{3.171495in}}%
\pgfpathlineto{\pgfqpoint{3.283189in}{3.208943in}}%
\pgfpathlineto{\pgfqpoint{3.262533in}{3.355992in}}%
\pgfpathlineto{\pgfqpoint{3.245399in}{3.461169in}}%
\pgfpathlineto{\pgfqpoint{3.226855in}{3.493042in}}%
\pgfpathlineto{\pgfqpoint{3.205495in}{3.361809in}}%
\pgfpathlineto{\pgfqpoint{3.188360in}{3.238602in}}%
\pgfpathlineto{\pgfqpoint{3.167703in}{3.164069in}}%
\pgfpathlineto{\pgfqpoint{3.146813in}{3.144769in}}%
\pgfpathlineto{\pgfqpoint{3.128974in}{3.142870in}}%
\pgfpathlineto{\pgfqpoint{3.110429in}{3.139645in}}%
\pgfpathlineto{\pgfqpoint{3.092590in}{3.144664in}}%
\pgfpathlineto{\pgfqpoint{3.071699in}{3.164346in}}%
\pgfpathlineto{\pgfqpoint{3.053624in}{3.222673in}}%
\pgfpathlineto{\pgfqpoint{3.031325in}{3.386563in}}%
\pgfpathlineto{\pgfqpoint{3.012077in}{3.480944in}}%
\pgfpathlineto{\pgfqpoint{2.994238in}{3.363768in}}%
\pgfpathlineto{\pgfqpoint{2.976164in}{3.484199in}}%
\pgfpathlineto{\pgfqpoint{2.958325in}{3.465807in}}%
\pgfpathlineto{\pgfqpoint{2.938607in}{3.302244in}}%
\pgfpathlineto{\pgfqpoint{2.913491in}{3.184015in}}%
\pgfpathlineto{\pgfqpoint{2.901051in}{3.168702in}}%
\pgfpathlineto{\pgfqpoint{2.879689in}{3.145507in}}%
\pgfpathlineto{\pgfqpoint{2.860913in}{3.154996in}}%
\pgfpathlineto{\pgfqpoint{2.837205in}{3.140369in}}%
\pgfpathlineto{\pgfqpoint{2.823120in}{3.140547in}}%
\pgfpathlineto{\pgfqpoint{2.801759in}{3.152019in}}%
\pgfpathlineto{\pgfqpoint{2.783685in}{3.182646in}}%
\pgfpathlineto{\pgfqpoint{2.763969in}{3.278289in}}%
\pgfpathlineto{\pgfqpoint{2.745425in}{3.438686in}}%
\pgfpathlineto{\pgfqpoint{2.723126in}{3.484445in}}%
\pgfpathlineto{\pgfqpoint{2.686273in}{3.252656in}}%
\pgfpathlineto{\pgfqpoint{2.669608in}{3.181445in}}%
\pgfpathlineto{\pgfqpoint{2.648481in}{3.152589in}}%
\pgfpathlineto{\pgfqpoint{2.630876in}{3.141558in}}%
\pgfpathlineto{\pgfqpoint{2.611628in}{3.140403in}}%
\pgfpathlineto{\pgfqpoint{2.590738in}{3.146284in}}%
\pgfpathlineto{\pgfqpoint{2.573133in}{3.162335in}}%
\pgfpathlineto{\pgfqpoint{2.551303in}{3.228706in}}%
\pgfpathlineto{\pgfqpoint{2.534872in}{3.361783in}}%
\pgfpathlineto{\pgfqpoint{2.516095in}{3.470342in}}%
\pgfpathlineto{\pgfqpoint{2.496846in}{3.474255in}}%
\pgfpathlineto{\pgfqpoint{2.475252in}{3.326128in}}%
\pgfpathlineto{\pgfqpoint{2.456942in}{3.215292in}}%
\pgfpathlineto{\pgfqpoint{2.438399in}{3.167234in}}%
\pgfpathlineto{\pgfqpoint{2.413283in}{3.145464in}}%
\pgfpathlineto{\pgfqpoint{2.398025in}{3.140440in}}%
\pgfpathlineto{\pgfqpoint{2.379011in}{3.142067in}}%
\pgfpathlineto{\pgfqpoint{2.360235in}{3.145885in}}%
\pgfpathlineto{\pgfqpoint{2.342395in}{3.146652in}}%
\pgfpathlineto{\pgfqpoint{2.323851in}{3.171003in}}%
\pgfpathlineto{\pgfqpoint{2.302021in}{3.261629in}}%
\pgfpathlineto{\pgfqpoint{2.283242in}{3.413301in}}%
\pgfpathlineto{\pgfqpoint{2.264465in}{3.487798in}}%
\pgfpathlineto{\pgfqpoint{2.241461in}{3.414616in}}%
\pgfpathlineto{\pgfqpoint{2.224325in}{3.289181in}}%
\pgfpathlineto{\pgfqpoint{2.205782in}{3.198806in}}%
\pgfpathlineto{\pgfqpoint{2.186535in}{3.164098in}}%
\pgfpathlineto{\pgfqpoint{2.168461in}{3.146100in}}%
\pgfpathlineto{\pgfqpoint{2.150385in}{3.140660in}}%
\pgfpathlineto{\pgfqpoint{2.128555in}{3.142736in}}%
\pgfpathlineto{\pgfqpoint{2.111421in}{3.149820in}}%
\pgfpathlineto{\pgfqpoint{2.092877in}{3.170965in}}%
\pgfpathlineto{\pgfqpoint{2.074100in}{3.214196in}}%
\pgfpathlineto{\pgfqpoint{2.051096in}{3.370699in}}%
\pgfpathlineto{\pgfqpoint{2.033022in}{3.479382in}}%
\pgfpathlineto{\pgfqpoint{2.013774in}{3.477476in}}%
\pgfpathlineto{\pgfqpoint{1.976686in}{3.235554in}}%
\pgfpathlineto{\pgfqpoint{1.957909in}{3.178611in}}%
\pgfpathlineto{\pgfqpoint{1.939130in}{3.155006in}}%
\pgfpathlineto{\pgfqpoint{1.918239in}{3.144264in}}%
\pgfpathlineto{\pgfqpoint{1.899460in}{3.140994in}}%
\pgfpathlineto{\pgfqpoint{1.881387in}{3.144228in}}%
\pgfpathlineto{\pgfqpoint{1.862139in}{3.152163in}}%
\pgfpathlineto{\pgfqpoint{1.842186in}{3.185892in}}%
\pgfpathlineto{\pgfqpoint{1.823173in}{3.234067in}}%
\pgfpathlineto{\pgfqpoint{1.803691in}{3.296311in}}%
\pgfpathlineto{\pgfqpoint{1.781392in}{3.380621in}}%
\pgfpathlineto{\pgfqpoint{1.766604in}{3.476366in}}%
\pgfpathlineto{\pgfqpoint{1.747356in}{3.493920in}}%
\pgfpathlineto{\pgfqpoint{1.725526in}{3.374794in}}%
\pgfpathlineto{\pgfqpoint{1.710269in}{3.280478in}}%
\pgfpathlineto{\pgfqpoint{1.688908in}{3.191514in}}%
\pgfpathlineto{\pgfqpoint{1.669895in}{3.159815in}}%
\pgfpathlineto{\pgfqpoint{1.645248in}{3.142775in}}%
\pgfpathlineto{\pgfqpoint{1.629288in}{3.140950in}}%
\pgfpathlineto{\pgfqpoint{1.611212in}{3.144063in}}%
\pgfpathlineto{\pgfqpoint{1.593138in}{3.269038in}}%
\pgfpathlineto{\pgfqpoint{1.570839in}{3.176447in}}%
\pgfpathlineto{\pgfqpoint{1.553000in}{3.150795in}}%
\pgfpathlineto{\pgfqpoint{1.533987in}{3.142259in}}%
\pgfpathlineto{\pgfqpoint{1.512862in}{3.141475in}}%
\pgfpathlineto{\pgfqpoint{1.492206in}{3.151359in}}%
\pgfpathlineto{\pgfqpoint{1.476478in}{3.175102in}}%
\pgfpathlineto{\pgfqpoint{1.456291in}{3.235332in}}%
\pgfpathlineto{\pgfqpoint{1.438217in}{3.368284in}}%
\pgfpathlineto{\pgfqpoint{1.419673in}{3.484903in}}%
\pgfpathlineto{\pgfqpoint{1.400660in}{3.510464in}}%
\pgfpathlineto{\pgfqpoint{1.383055in}{3.417761in}}%
\pgfpathlineto{\pgfqpoint{1.361461in}{3.254282in}}%
\pgfpathlineto{\pgfqpoint{1.341979in}{3.186165in}}%
\pgfpathlineto{\pgfqpoint{1.324138in}{3.161321in}}%
\pgfpathlineto{\pgfqpoint{1.302544in}{3.145514in}}%
\pgfpathlineto{\pgfqpoint{1.284470in}{3.141551in}}%
\pgfpathlineto{\pgfqpoint{1.266160in}{3.144694in}}%
\pgfpathlineto{\pgfqpoint{1.246913in}{3.156290in}}%
\pgfpathlineto{\pgfqpoint{1.228134in}{3.183542in}}%
\pgfpathlineto{\pgfqpoint{1.206304in}{3.273920in}}%
\pgfpathlineto{\pgfqpoint{1.187761in}{3.413693in}}%
\pgfpathlineto{\pgfqpoint{1.169451in}{3.498772in}}%
\pgfpathlineto{\pgfqpoint{1.151612in}{3.514326in}}%
\pgfpathlineto{\pgfqpoint{1.129313in}{3.402907in}}%
\pgfpathlineto{\pgfqpoint{1.110300in}{3.276516in}}%
\pgfpathlineto{\pgfqpoint{1.092695in}{3.197464in}}%
\pgfpathlineto{\pgfqpoint{1.073213in}{3.164357in}}%
\pgfpathlineto{\pgfqpoint{1.054670in}{3.149596in}}%
\pgfpathlineto{\pgfqpoint{1.034249in}{3.142250in}}%
\pgfpathlineto{\pgfqpoint{1.015001in}{3.143122in}}%
\pgfpathlineto{\pgfqpoint{0.996222in}{3.150799in}}%
\pgfpathlineto{\pgfqpoint{0.978383in}{3.170869in}}%
\pgfpathlineto{\pgfqpoint{0.957021in}{3.235229in}}%
\pgfpathlineto{\pgfqpoint{0.938245in}{3.355159in}}%
\pgfpathlineto{\pgfqpoint{0.919700in}{3.437150in}}%
\pgfpathlineto{\pgfqpoint{0.900687in}{3.485248in}}%
\pgfpathlineto{\pgfqpoint{0.878857in}{3.535703in}}%
\pgfpathlineto{\pgfqpoint{0.860549in}{3.500877in}}%
\pgfpathlineto{\pgfqpoint{0.842944in}{3.325083in}}%
\pgfpathlineto{\pgfqpoint{0.823696in}{3.444794in}}%
\pgfpathlineto{\pgfqpoint{0.805386in}{3.531628in}}%
\pgfpathlineto{\pgfqpoint{0.783087in}{3.485216in}}%
\pgfpathlineto{\pgfqpoint{0.766656in}{3.340892in}}%
\pgfpathlineto{\pgfqpoint{0.745295in}{3.214625in}}%
\pgfpathlineto{\pgfqpoint{0.723467in}{3.176282in}}%
\pgfpathlineto{\pgfqpoint{0.706331in}{3.156310in}}%
\pgfpathlineto{\pgfqpoint{0.687552in}{3.143871in}}%
\pgfpathlineto{\pgfqpoint{0.669713in}{3.142909in}}%
\pgfpathlineto{\pgfqpoint{0.651170in}{3.152508in}}%
\pgfpathlineto{\pgfqpoint{0.659619in}{3.145556in}}%
\pgfpathlineto{\pgfqpoint{0.674643in}{3.142609in}}%
\pgfpathlineto{\pgfqpoint{0.696942in}{3.159071in}}%
\pgfpathlineto{\pgfqpoint{0.713842in}{3.199819in}}%
\pgfpathlineto{\pgfqpoint{0.733090in}{3.311227in}}%
\pgfpathlineto{\pgfqpoint{0.754217in}{3.525823in}}%
\pgfpathlineto{\pgfqpoint{0.771821in}{3.505830in}}%
\pgfpathlineto{\pgfqpoint{0.791303in}{3.328378in}}%
\pgfpathlineto{\pgfqpoint{0.811254in}{3.186225in}}%
\pgfpathlineto{\pgfqpoint{0.829094in}{3.150822in}}%
\pgfpathlineto{\pgfqpoint{0.847873in}{3.141176in}}%
\pgfpathlineto{\pgfqpoint{0.866886in}{3.148174in}}%
\pgfpathlineto{\pgfqpoint{0.888950in}{3.184167in}}%
\pgfpathlineto{\pgfqpoint{0.908667in}{3.277770in}}%
\pgfpathlineto{\pgfqpoint{0.924160in}{3.438253in}}%
\pgfpathlineto{\pgfqpoint{0.943173in}{3.519886in}}%
\pgfpathlineto{\pgfqpoint{0.981200in}{3.285864in}}%
\pgfpathlineto{\pgfqpoint{1.004436in}{3.170382in}}%
\pgfpathlineto{\pgfqpoint{1.022746in}{3.144667in}}%
\pgfpathlineto{\pgfqpoint{1.042463in}{3.141017in}}%
\pgfpathlineto{\pgfqpoint{1.060537in}{3.150652in}}%
\pgfpathlineto{\pgfqpoint{1.080255in}{3.182415in}}%
\pgfpathlineto{\pgfqpoint{1.099503in}{3.290351in}}%
\pgfpathlineto{\pgfqpoint{1.116168in}{3.462959in}}%
\pgfpathlineto{\pgfqpoint{1.136824in}{3.491155in}}%
\pgfpathlineto{\pgfqpoint{1.156543in}{3.354695in}}%
\pgfpathlineto{\pgfqpoint{1.175556in}{3.198977in}}%
\pgfpathlineto{\pgfqpoint{1.194803in}{3.154318in}}%
\pgfpathlineto{\pgfqpoint{1.213580in}{3.141647in}}%
\pgfpathlineto{\pgfqpoint{1.233533in}{3.142658in}}%
\pgfpathlineto{\pgfqpoint{1.252078in}{3.156209in}}%
\pgfpathlineto{\pgfqpoint{1.270620in}{3.200181in}}%
\pgfpathlineto{\pgfqpoint{1.290339in}{3.336583in}}%
\pgfpathlineto{\pgfqpoint{1.308881in}{3.491784in}}%
\pgfpathlineto{\pgfqpoint{1.327426in}{3.470127in}}%
\pgfpathlineto{\pgfqpoint{1.348082in}{3.351211in}}%
\pgfpathlineto{\pgfqpoint{1.369441in}{3.214937in}}%
\pgfpathlineto{\pgfqpoint{1.388689in}{3.157864in}}%
\pgfpathlineto{\pgfqpoint{1.406999in}{3.142141in}}%
\pgfpathlineto{\pgfqpoint{1.426012in}{3.140268in}}%
\pgfpathlineto{\pgfqpoint{1.445260in}{3.147807in}}%
\pgfpathlineto{\pgfqpoint{1.467793in}{3.178687in}}%
\pgfpathlineto{\pgfqpoint{1.482112in}{3.239199in}}%
\pgfpathlineto{\pgfqpoint{1.503472in}{3.415095in}}%
\pgfpathlineto{\pgfqpoint{1.520842in}{3.489545in}}%
\pgfpathlineto{\pgfqpoint{1.538447in}{3.413508in}}%
\pgfpathlineto{\pgfqpoint{1.561451in}{3.274360in}}%
\pgfpathlineto{\pgfqpoint{1.580228in}{3.187386in}}%
\pgfpathlineto{\pgfqpoint{1.599007in}{3.151154in}}%
\pgfpathlineto{\pgfqpoint{1.618254in}{3.140954in}}%
\pgfpathlineto{\pgfqpoint{1.635859in}{3.141651in}}%
\pgfpathlineto{\pgfqpoint{1.655576in}{3.150166in}}%
\pgfpathlineto{\pgfqpoint{1.674823in}{3.175680in}}%
\pgfpathlineto{\pgfqpoint{1.692899in}{3.217823in}}%
\pgfpathlineto{\pgfqpoint{1.715198in}{3.371304in}}%
\pgfpathlineto{\pgfqpoint{1.733037in}{3.478700in}}%
\pgfpathlineto{\pgfqpoint{1.751816in}{3.479369in}}%
\pgfpathlineto{\pgfqpoint{1.770593in}{3.378912in}}%
\pgfpathlineto{\pgfqpoint{1.789606in}{3.236669in}}%
\pgfpathlineto{\pgfqpoint{1.808619in}{3.169065in}}%
\pgfpathlineto{\pgfqpoint{1.828338in}{3.148824in}}%
\pgfpathlineto{\pgfqpoint{1.848760in}{3.141344in}}%
\pgfpathlineto{\pgfqpoint{1.867537in}{3.140770in}}%
\pgfpathlineto{\pgfqpoint{1.885612in}{3.145498in}}%
\pgfpathlineto{\pgfqpoint{1.904623in}{3.165835in}}%
\pgfpathlineto{\pgfqpoint{1.923402in}{3.194808in}}%
\pgfpathlineto{\pgfqpoint{1.945938in}{3.328211in}}%
\pgfpathlineto{\pgfqpoint{1.963777in}{3.459083in}}%
\pgfpathlineto{\pgfqpoint{1.982085in}{3.481666in}}%
\pgfpathlineto{\pgfqpoint{1.999455in}{3.420302in}}%
\pgfpathlineto{\pgfqpoint{2.019406in}{3.277614in}}%
\pgfpathlineto{\pgfqpoint{2.041707in}{3.184523in}}%
\pgfpathlineto{\pgfqpoint{2.060484in}{3.153087in}}%
\pgfpathlineto{\pgfqpoint{2.078560in}{3.144604in}}%
\pgfpathlineto{\pgfqpoint{2.099216in}{3.140163in}}%
\pgfpathlineto{\pgfqpoint{2.118227in}{3.142219in}}%
\pgfpathlineto{\pgfqpoint{2.135598in}{3.151988in}}%
\pgfpathlineto{\pgfqpoint{2.153673in}{3.172896in}}%
\pgfpathlineto{\pgfqpoint{2.173859in}{3.253424in}}%
\pgfpathlineto{\pgfqpoint{2.194985in}{3.377160in}}%
\pgfpathlineto{\pgfqpoint{2.212825in}{3.474566in}}%
\pgfpathlineto{\pgfqpoint{2.230898in}{3.457533in}}%
\pgfpathlineto{\pgfqpoint{2.252258in}{3.317412in}}%
\pgfpathlineto{\pgfqpoint{2.269159in}{3.198668in}}%
\pgfpathlineto{\pgfqpoint{2.288172in}{3.343418in}}%
\pgfpathlineto{\pgfqpoint{2.309766in}{3.189939in}}%
\pgfpathlineto{\pgfqpoint{2.330893in}{3.156896in}}%
\pgfpathlineto{\pgfqpoint{2.347324in}{3.144377in}}%
\pgfpathlineto{\pgfqpoint{2.365868in}{3.139767in}}%
\pgfpathlineto{\pgfqpoint{2.386759in}{3.144275in}}%
\pgfpathlineto{\pgfqpoint{2.404364in}{3.158326in}}%
\pgfpathlineto{\pgfqpoint{2.428306in}{3.223962in}}%
\pgfpathlineto{\pgfqpoint{2.444736in}{3.320963in}}%
\pgfpathlineto{\pgfqpoint{2.461872in}{3.457319in}}%
\pgfpathlineto{\pgfqpoint{2.482529in}{3.458841in}}%
\pgfpathlineto{\pgfqpoint{2.503419in}{3.363003in}}%
\pgfpathlineto{\pgfqpoint{2.518207in}{3.239868in}}%
\pgfpathlineto{\pgfqpoint{2.540037in}{3.164940in}}%
\pgfpathlineto{\pgfqpoint{2.560693in}{3.144960in}}%
\pgfpathlineto{\pgfqpoint{2.578064in}{3.140910in}}%
\pgfpathlineto{\pgfqpoint{2.598249in}{3.141620in}}%
\pgfpathlineto{\pgfqpoint{2.617497in}{3.150086in}}%
\pgfpathlineto{\pgfqpoint{2.634398in}{3.171427in}}%
\pgfpathlineto{\pgfqpoint{2.656697in}{3.215957in}}%
\pgfpathlineto{\pgfqpoint{2.674068in}{3.314460in}}%
\pgfpathlineto{\pgfqpoint{2.692141in}{3.446668in}}%
\pgfpathlineto{\pgfqpoint{2.715849in}{3.460638in}}%
\pgfpathlineto{\pgfqpoint{2.731342in}{3.362943in}}%
\pgfpathlineto{\pgfqpoint{2.750119in}{3.295268in}}%
\pgfpathlineto{\pgfqpoint{2.771011in}{3.198229in}}%
\pgfpathlineto{\pgfqpoint{2.789085in}{3.158180in}}%
\pgfpathlineto{\pgfqpoint{2.808801in}{3.144646in}}%
\pgfpathlineto{\pgfqpoint{2.850819in}{3.140087in}}%
\pgfpathlineto{\pgfqpoint{2.865607in}{3.144062in}}%
\pgfpathlineto{\pgfqpoint{2.886966in}{3.163618in}}%
\pgfpathlineto{\pgfqpoint{2.904805in}{3.198699in}}%
\pgfpathlineto{\pgfqpoint{2.922176in}{3.264222in}}%
\pgfpathlineto{\pgfqpoint{2.943066in}{3.432289in}}%
\pgfpathlineto{\pgfqpoint{2.962785in}{3.460327in}}%
\pgfpathlineto{\pgfqpoint{2.984144in}{3.472236in}}%
\pgfpathlineto{\pgfqpoint{3.002454in}{3.365226in}}%
\pgfpathlineto{\pgfqpoint{3.020059in}{3.247927in}}%
\pgfpathlineto{\pgfqpoint{3.040950in}{3.168874in}}%
\pgfpathlineto{\pgfqpoint{3.059258in}{3.148598in}}%
\pgfpathlineto{\pgfqpoint{3.076159in}{3.143704in}}%
\pgfpathlineto{\pgfqpoint{3.097989in}{3.140111in}}%
\pgfpathlineto{\pgfqpoint{3.097519in}{3.142887in}}%
\pgfpathlineto{\pgfqpoint{3.115358in}{3.145583in}}%
\pgfpathlineto{\pgfqpoint{3.133902in}{3.156407in}}%
\pgfpathlineto{\pgfqpoint{3.154793in}{3.188423in}}%
\pgfpathlineto{\pgfqpoint{3.172866in}{3.261703in}}%
\pgfpathlineto{\pgfqpoint{3.194697in}{3.418942in}}%
\pgfpathlineto{\pgfqpoint{3.211833in}{3.485637in}}%
\pgfpathlineto{\pgfqpoint{3.229672in}{3.473200in}}%
\pgfpathlineto{\pgfqpoint{3.251031in}{3.379776in}}%
\pgfpathlineto{\pgfqpoint{3.271453in}{3.240385in}}%
\pgfpathlineto{\pgfqpoint{3.289058in}{3.181003in}}%
\pgfpathlineto{\pgfqpoint{3.306897in}{3.153613in}}%
\pgfpathlineto{\pgfqpoint{3.328727in}{3.143689in}}%
\pgfpathlineto{\pgfqpoint{3.346332in}{3.140600in}}%
\pgfpathlineto{\pgfqpoint{3.367222in}{3.143922in}}%
\pgfpathlineto{\pgfqpoint{3.385767in}{3.150375in}}%
\pgfpathlineto{\pgfqpoint{3.403372in}{3.165202in}}%
\pgfpathlineto{\pgfqpoint{3.422619in}{3.205884in}}%
\pgfpathlineto{\pgfqpoint{3.443979in}{3.300055in}}%
\pgfpathlineto{\pgfqpoint{3.461818in}{3.425597in}}%
\pgfpathlineto{\pgfqpoint{3.479188in}{3.500791in}}%
\pgfpathlineto{\pgfqpoint{3.501487in}{3.449126in}}%
\pgfpathlineto{\pgfqpoint{3.520501in}{3.380686in}}%
\pgfpathlineto{\pgfqpoint{3.538105in}{3.274851in}}%
\pgfpathlineto{\pgfqpoint{3.559467in}{3.180574in}}%
\pgfpathlineto{\pgfqpoint{3.577540in}{3.251653in}}%
\pgfpathlineto{\pgfqpoint{3.595614in}{3.372448in}}%
\pgfpathlineto{\pgfqpoint{3.616505in}{3.503642in}}%
\pgfpathlineto{\pgfqpoint{3.634580in}{3.500077in}}%
\pgfpathlineto{\pgfqpoint{3.651951in}{3.404899in}}%
\pgfpathlineto{\pgfqpoint{3.674718in}{3.270712in}}%
\pgfpathlineto{\pgfqpoint{3.692323in}{3.185802in}}%
\pgfpathlineto{\pgfqpoint{3.709928in}{3.157297in}}%
\pgfpathlineto{\pgfqpoint{3.730819in}{3.142859in}}%
\pgfpathlineto{\pgfqpoint{3.749363in}{3.142117in}}%
\pgfpathlineto{\pgfqpoint{3.768140in}{3.148935in}}%
\pgfpathlineto{\pgfqpoint{3.788796in}{3.176008in}}%
\pgfpathlineto{\pgfqpoint{3.808983in}{3.225924in}}%
\pgfpathlineto{\pgfqpoint{3.827057in}{3.316741in}}%
\pgfpathlineto{\pgfqpoint{3.844662in}{3.477211in}}%
\pgfpathlineto{\pgfqpoint{3.862737in}{3.521079in}}%
\pgfpathlineto{\pgfqpoint{3.886914in}{3.430425in}}%
\pgfpathlineto{\pgfqpoint{3.905927in}{3.286684in}}%
\pgfpathlineto{\pgfqpoint{3.923297in}{3.199943in}}%
\pgfpathlineto{\pgfqpoint{3.942311in}{3.162143in}}%
\pgfpathlineto{\pgfqpoint{3.958741in}{3.147208in}}%
\pgfpathlineto{\pgfqpoint{3.980335in}{3.141522in}}%
\pgfpathlineto{\pgfqpoint{4.000757in}{3.144735in}}%
\pgfpathlineto{\pgfqpoint{4.019536in}{3.156647in}}%
\pgfpathlineto{\pgfqpoint{4.037375in}{3.174639in}}%
\pgfpathlineto{\pgfqpoint{4.055214in}{3.220367in}}%
\pgfpathlineto{\pgfqpoint{4.076810in}{3.361488in}}%
\pgfpathlineto{\pgfqpoint{4.094884in}{3.510220in}}%
\pgfpathlineto{\pgfqpoint{4.115540in}{3.528761in}}%
\pgfpathlineto{\pgfqpoint{4.137136in}{3.461257in}}%
\pgfpathlineto{\pgfqpoint{4.153332in}{3.348248in}}%
\pgfpathlineto{\pgfqpoint{4.172814in}{3.234483in}}%
\pgfpathlineto{\pgfqpoint{4.191124in}{3.175917in}}%
\pgfpathlineto{\pgfqpoint{4.212718in}{3.153626in}}%
\pgfpathlineto{\pgfqpoint{4.230557in}{3.144092in}}%
\pgfpathlineto{\pgfqpoint{4.247927in}{3.142227in}}%
\pgfpathlineto{\pgfqpoint{4.265767in}{3.147533in}}%
\pgfpathlineto{\pgfqpoint{4.308253in}{3.181833in}}%
\pgfpathlineto{\pgfqpoint{4.325623in}{3.239632in}}%
\pgfpathlineto{\pgfqpoint{4.365996in}{3.535476in}}%
\pgfpathlineto{\pgfqpoint{4.383132in}{3.554347in}}%
\pgfpathlineto{\pgfqpoint{4.403788in}{3.458983in}}%
\pgfpathlineto{\pgfqpoint{4.419750in}{3.337675in}}%
\pgfpathlineto{\pgfqpoint{4.440875in}{3.251176in}}%
\pgfpathlineto{\pgfqpoint{4.462000in}{3.184133in}}%
\pgfpathlineto{\pgfqpoint{4.480076in}{3.157175in}}%
\pgfpathlineto{\pgfqpoint{4.483830in}{3.154988in}}%
\pgfpathlineto{\pgfqpoint{4.475379in}{3.168571in}}%
\pgfpathlineto{\pgfqpoint{4.455194in}{3.239177in}}%
\pgfpathlineto{\pgfqpoint{4.437120in}{3.403774in}}%
\pgfpathlineto{\pgfqpoint{4.418810in}{3.541258in}}%
\pgfpathlineto{\pgfqpoint{4.396511in}{3.485739in}}%
\pgfpathlineto{\pgfqpoint{4.377967in}{3.282080in}}%
\pgfpathlineto{\pgfqpoint{4.359659in}{3.186457in}}%
\pgfpathlineto{\pgfqpoint{4.338298in}{3.151428in}}%
\pgfpathlineto{\pgfqpoint{4.320224in}{3.141522in}}%
\pgfpathlineto{\pgfqpoint{4.301681in}{3.147979in}}%
\pgfpathlineto{\pgfqpoint{4.284545in}{3.173791in}}%
\pgfpathlineto{\pgfqpoint{4.263420in}{3.265437in}}%
\pgfpathlineto{\pgfqpoint{4.245816in}{3.437111in}}%
\pgfpathlineto{\pgfqpoint{4.225628in}{3.540014in}}%
\pgfpathlineto{\pgfqpoint{4.204503in}{3.404809in}}%
\pgfpathlineto{\pgfqpoint{4.186428in}{3.236068in}}%
\pgfpathlineto{\pgfqpoint{4.164600in}{3.163312in}}%
\pgfpathlineto{\pgfqpoint{4.147464in}{3.154170in}}%
\pgfpathlineto{\pgfqpoint{4.128919in}{3.194741in}}%
\pgfpathlineto{\pgfqpoint{4.108029in}{3.345756in}}%
\pgfpathlineto{\pgfqpoint{4.090189in}{3.492622in}}%
\pgfpathlineto{\pgfqpoint{4.070238in}{3.503109in}}%
\pgfpathlineto{\pgfqpoint{4.052868in}{3.322092in}}%
\pgfpathlineto{\pgfqpoint{4.032915in}{3.188789in}}%
\pgfpathlineto{\pgfqpoint{4.012493in}{3.151364in}}%
\pgfpathlineto{\pgfqpoint{3.994654in}{3.141233in}}%
\pgfpathlineto{\pgfqpoint{3.974233in}{3.145027in}}%
\pgfpathlineto{\pgfqpoint{3.953576in}{3.168320in}}%
\pgfpathlineto{\pgfqpoint{3.936442in}{3.234709in}}%
\pgfpathlineto{\pgfqpoint{3.918603in}{3.394281in}}%
\pgfpathlineto{\pgfqpoint{3.899590in}{3.508064in}}%
\pgfpathlineto{\pgfqpoint{3.880342in}{3.447422in}}%
\pgfpathlineto{\pgfqpoint{3.856869in}{3.234389in}}%
\pgfpathlineto{\pgfqpoint{3.838794in}{3.171669in}}%
\pgfpathlineto{\pgfqpoint{3.821189in}{3.147675in}}%
\pgfpathlineto{\pgfqpoint{3.801238in}{3.140615in}}%
\pgfpathlineto{\pgfqpoint{3.783164in}{3.145168in}}%
\pgfpathlineto{\pgfqpoint{3.761568in}{3.175887in}}%
\pgfpathlineto{\pgfqpoint{3.745841in}{3.232763in}}%
\pgfpathlineto{\pgfqpoint{3.722133in}{3.425141in}}%
\pgfpathlineto{\pgfqpoint{3.708051in}{3.495191in}}%
\pgfpathlineto{\pgfqpoint{3.685986in}{3.445673in}}%
\pgfpathlineto{\pgfqpoint{3.666268in}{3.260278in}}%
\pgfpathlineto{\pgfqpoint{3.648663in}{3.180876in}}%
\pgfpathlineto{\pgfqpoint{3.628478in}{3.157071in}}%
\pgfpathlineto{\pgfqpoint{3.610168in}{3.143062in}}%
\pgfpathlineto{\pgfqpoint{3.593737in}{3.140409in}}%
\pgfpathlineto{\pgfqpoint{3.570264in}{3.147010in}}%
\pgfpathlineto{\pgfqpoint{3.552425in}{3.162513in}}%
\pgfpathlineto{\pgfqpoint{3.532237in}{3.218162in}}%
\pgfpathlineto{\pgfqpoint{3.515572in}{3.333238in}}%
\pgfpathlineto{\pgfqpoint{3.495150in}{3.454613in}}%
\pgfpathlineto{\pgfqpoint{3.477780in}{3.491047in}}%
\pgfpathlineto{\pgfqpoint{3.455012in}{3.353834in}}%
\pgfpathlineto{\pgfqpoint{3.436468in}{3.216967in}}%
\pgfpathlineto{\pgfqpoint{3.418863in}{3.182025in}}%
\pgfpathlineto{\pgfqpoint{3.397738in}{3.153694in}}%
\pgfpathlineto{\pgfqpoint{3.378021in}{3.141922in}}%
\pgfpathlineto{\pgfqpoint{3.361354in}{3.140240in}}%
\pgfpathlineto{\pgfqpoint{3.340464in}{3.147570in}}%
\pgfpathlineto{\pgfqpoint{3.321921in}{3.172791in}}%
\pgfpathlineto{\pgfqpoint{3.301499in}{3.262973in}}%
\pgfpathlineto{\pgfqpoint{3.283424in}{3.399128in}}%
\pgfpathlineto{\pgfqpoint{3.263239in}{3.484847in}}%
\pgfpathlineto{\pgfqpoint{3.242582in}{3.409052in}}%
\pgfpathlineto{\pgfqpoint{3.224272in}{3.263145in}}%
\pgfpathlineto{\pgfqpoint{3.205259in}{3.407837in}}%
\pgfpathlineto{\pgfqpoint{3.187185in}{3.254347in}}%
\pgfpathlineto{\pgfqpoint{3.168877in}{3.178127in}}%
\pgfpathlineto{\pgfqpoint{3.169112in}{3.159333in}}%
\pgfpathlineto{\pgfqpoint{3.150567in}{3.153781in}}%
\pgfpathlineto{\pgfqpoint{3.130146in}{3.141940in}}%
\pgfpathlineto{\pgfqpoint{3.109021in}{3.140680in}}%
\pgfpathlineto{\pgfqpoint{3.091650in}{3.148774in}}%
\pgfpathlineto{\pgfqpoint{3.071229in}{3.170502in}}%
\pgfpathlineto{\pgfqpoint{3.053155in}{3.239545in}}%
\pgfpathlineto{\pgfqpoint{3.032264in}{3.406428in}}%
\pgfpathlineto{\pgfqpoint{3.013486in}{3.483047in}}%
\pgfpathlineto{\pgfqpoint{2.995412in}{3.439473in}}%
\pgfpathlineto{\pgfqpoint{2.973582in}{3.264742in}}%
\pgfpathlineto{\pgfqpoint{2.955039in}{3.182126in}}%
\pgfpathlineto{\pgfqpoint{2.936260in}{3.152565in}}%
\pgfpathlineto{\pgfqpoint{2.921002in}{3.145032in}}%
\pgfpathlineto{\pgfqpoint{2.898937in}{3.140262in}}%
\pgfpathlineto{\pgfqpoint{2.858096in}{3.139319in}}%
\pgfpathlineto{\pgfqpoint{2.840725in}{3.143884in}}%
\pgfpathlineto{\pgfqpoint{2.821243in}{3.160905in}}%
\pgfpathlineto{\pgfqpoint{2.805047in}{3.198801in}}%
\pgfpathlineto{\pgfqpoint{2.784625in}{3.347302in}}%
\pgfpathlineto{\pgfqpoint{2.763264in}{3.460286in}}%
\pgfpathlineto{\pgfqpoint{2.744721in}{3.461647in}}%
\pgfpathlineto{\pgfqpoint{2.725474in}{3.325179in}}%
\pgfpathlineto{\pgfqpoint{2.705521in}{3.219040in}}%
\pgfpathlineto{\pgfqpoint{2.686978in}{3.165738in}}%
\pgfpathlineto{\pgfqpoint{2.668668in}{3.146801in}}%
\pgfpathlineto{\pgfqpoint{2.650126in}{3.140167in}}%
\pgfpathlineto{\pgfqpoint{2.631581in}{3.141517in}}%
\pgfpathlineto{\pgfqpoint{2.610220in}{3.150656in}}%
\pgfpathlineto{\pgfqpoint{2.590503in}{3.175138in}}%
\pgfpathlineto{\pgfqpoint{2.589800in}{3.213382in}}%
\pgfpathlineto{\pgfqpoint{2.571959in}{3.248575in}}%
\pgfpathlineto{\pgfqpoint{2.552711in}{3.400754in}}%
\pgfpathlineto{\pgfqpoint{2.533229in}{3.474733in}}%
\pgfpathlineto{\pgfqpoint{2.512573in}{3.435525in}}%
\pgfpathlineto{\pgfqpoint{2.496142in}{3.285009in}}%
\pgfpathlineto{\pgfqpoint{2.476895in}{3.189213in}}%
\pgfpathlineto{\pgfqpoint{2.459524in}{3.164710in}}%
\pgfpathlineto{\pgfqpoint{2.436756in}{3.144624in}}%
\pgfpathlineto{\pgfqpoint{2.417038in}{3.139580in}}%
\pgfpathlineto{\pgfqpoint{2.397790in}{3.140050in}}%
\pgfpathlineto{\pgfqpoint{2.379482in}{3.145546in}}%
\pgfpathlineto{\pgfqpoint{2.360235in}{3.164475in}}%
\pgfpathlineto{\pgfqpoint{2.342159in}{3.223547in}}%
\pgfpathlineto{\pgfqpoint{2.319626in}{3.374933in}}%
\pgfpathlineto{\pgfqpoint{2.302492in}{3.464664in}}%
\pgfpathlineto{\pgfqpoint{2.285590in}{3.463790in}}%
\pgfpathlineto{\pgfqpoint{2.264231in}{3.357448in}}%
\pgfpathlineto{\pgfqpoint{2.244747in}{3.230878in}}%
\pgfpathlineto{\pgfqpoint{2.226907in}{3.176213in}}%
\pgfpathlineto{\pgfqpoint{2.205548in}{3.149825in}}%
\pgfpathlineto{\pgfqpoint{2.187004in}{3.143046in}}%
\pgfpathlineto{\pgfqpoint{2.168930in}{3.139538in}}%
\pgfpathlineto{\pgfqpoint{2.149213in}{3.143810in}}%
\pgfpathlineto{\pgfqpoint{2.127852in}{3.159514in}}%
\pgfpathlineto{\pgfqpoint{2.109308in}{3.195677in}}%
\pgfpathlineto{\pgfqpoint{2.090765in}{3.292754in}}%
\pgfpathlineto{\pgfqpoint{2.072455in}{3.399593in}}%
\pgfpathlineto{\pgfqpoint{2.053678in}{3.475195in}}%
\pgfpathlineto{\pgfqpoint{2.032082in}{3.452772in}}%
\pgfpathlineto{\pgfqpoint{2.013538in}{3.386270in}}%
\pgfpathlineto{\pgfqpoint{1.998281in}{3.255902in}}%
\pgfpathlineto{\pgfqpoint{1.975748in}{3.177656in}}%
\pgfpathlineto{\pgfqpoint{1.957672in}{3.155895in}}%
\pgfpathlineto{\pgfqpoint{1.935139in}{3.143363in}}%
\pgfpathlineto{\pgfqpoint{1.919882in}{3.140152in}}%
\pgfpathlineto{\pgfqpoint{1.898757in}{3.142105in}}%
\pgfpathlineto{\pgfqpoint{1.879978in}{3.147762in}}%
\pgfpathlineto{\pgfqpoint{1.859088in}{3.148724in}}%
\pgfpathlineto{\pgfqpoint{1.840309in}{3.167544in}}%
\pgfpathlineto{\pgfqpoint{1.822939in}{3.219734in}}%
\pgfpathlineto{\pgfqpoint{1.801814in}{3.362506in}}%
\pgfpathlineto{\pgfqpoint{1.784678in}{3.460805in}}%
\pgfpathlineto{\pgfqpoint{1.762613in}{3.483098in}}%
\pgfpathlineto{\pgfqpoint{1.747122in}{3.424095in}}%
\pgfpathlineto{\pgfqpoint{1.725292in}{3.282711in}}%
\pgfpathlineto{\pgfqpoint{1.706513in}{3.200363in}}%
\pgfpathlineto{\pgfqpoint{1.688439in}{3.161680in}}%
\pgfpathlineto{\pgfqpoint{1.669660in}{3.147167in}}%
\pgfpathlineto{\pgfqpoint{1.648301in}{3.140804in}}%
\pgfpathlineto{\pgfqpoint{1.629051in}{3.140765in}}%
\pgfpathlineto{\pgfqpoint{1.610743in}{3.143906in}}%
\pgfpathlineto{\pgfqpoint{1.591496in}{3.465906in}}%
\pgfpathlineto{\pgfqpoint{1.570134in}{3.287921in}}%
\pgfpathlineto{\pgfqpoint{1.548306in}{3.180337in}}%
\pgfpathlineto{\pgfqpoint{1.514974in}{3.148639in}}%
\pgfpathlineto{\pgfqpoint{1.496195in}{3.140158in}}%
\pgfpathlineto{\pgfqpoint{1.477416in}{3.142020in}}%
\pgfpathlineto{\pgfqpoint{1.456057in}{3.153776in}}%
\pgfpathlineto{\pgfqpoint{1.437043in}{3.186805in}}%
\pgfpathlineto{\pgfqpoint{1.418735in}{3.277186in}}%
\pgfpathlineto{\pgfqpoint{1.400191in}{3.429745in}}%
\pgfpathlineto{\pgfqpoint{1.382117in}{3.494040in}}%
\pgfpathlineto{\pgfqpoint{1.360053in}{3.440076in}}%
\pgfpathlineto{\pgfqpoint{1.341274in}{3.279551in}}%
\pgfpathlineto{\pgfqpoint{1.323435in}{3.197319in}}%
\pgfpathlineto{\pgfqpoint{1.301839in}{3.154727in}}%
\pgfpathlineto{\pgfqpoint{1.283296in}{3.145314in}}%
\pgfpathlineto{\pgfqpoint{1.246209in}{3.140464in}}%
\pgfpathlineto{\pgfqpoint{1.227431in}{3.142665in}}%
\pgfpathlineto{\pgfqpoint{1.205835in}{3.154652in}}%
\pgfpathlineto{\pgfqpoint{1.187761in}{3.185827in}}%
\pgfpathlineto{\pgfqpoint{1.168982in}{3.246050in}}%
\pgfpathlineto{\pgfqpoint{1.147623in}{3.411817in}}%
\pgfpathlineto{\pgfqpoint{1.130253in}{3.505464in}}%
\pgfpathlineto{\pgfqpoint{1.112648in}{3.496329in}}%
\pgfpathlineto{\pgfqpoint{1.091052in}{3.346625in}}%
\pgfpathlineto{\pgfqpoint{1.070161in}{3.215465in}}%
\pgfpathlineto{\pgfqpoint{1.054905in}{3.174164in}}%
\pgfpathlineto{\pgfqpoint{1.036126in}{3.152589in}}%
\pgfpathlineto{\pgfqpoint{1.017581in}{3.145070in}}%
\pgfpathlineto{\pgfqpoint{0.995988in}{3.141197in}}%
\pgfpathlineto{\pgfqpoint{0.976504in}{3.141737in}}%
\pgfpathlineto{\pgfqpoint{0.959135in}{3.150045in}}%
\pgfpathlineto{\pgfqpoint{0.937071in}{3.184906in}}%
\pgfpathlineto{\pgfqpoint{0.918057in}{3.259950in}}%
\pgfpathlineto{\pgfqpoint{0.900452in}{3.395324in}}%
\pgfpathlineto{\pgfqpoint{0.878388in}{3.316329in}}%
\pgfpathlineto{\pgfqpoint{0.860314in}{3.464023in}}%
\pgfpathlineto{\pgfqpoint{0.844587in}{3.523081in}}%
\pgfpathlineto{\pgfqpoint{0.823462in}{3.483812in}}%
\pgfpathlineto{\pgfqpoint{0.804448in}{3.314550in}}%
\pgfpathlineto{\pgfqpoint{0.784730in}{3.214840in}}%
\pgfpathlineto{\pgfqpoint{0.763839in}{3.173548in}}%
\pgfpathlineto{\pgfqpoint{0.745766in}{3.154510in}}%
\pgfpathlineto{\pgfqpoint{0.726753in}{3.143741in}}%
\pgfpathlineto{\pgfqpoint{0.709617in}{3.141492in}}%
\pgfpathlineto{\pgfqpoint{0.686849in}{3.147057in}}%
\pgfpathlineto{\pgfqpoint{0.669713in}{3.165872in}}%
\pgfpathlineto{\pgfqpoint{0.649762in}{3.209774in}}%
\pgfpathlineto{\pgfqpoint{0.649525in}{3.206262in}}%
\pgfpathlineto{\pgfqpoint{0.655396in}{3.184487in}}%
\pgfpathlineto{\pgfqpoint{0.676286in}{3.148336in}}%
\pgfpathlineto{\pgfqpoint{0.695300in}{3.141114in}}%
\pgfpathlineto{\pgfqpoint{0.712668in}{3.148184in}}%
\pgfpathlineto{\pgfqpoint{0.734264in}{3.177900in}}%
\pgfpathlineto{\pgfqpoint{0.751869in}{3.260016in}}%
\pgfpathlineto{\pgfqpoint{0.770413in}{3.445502in}}%
\pgfpathlineto{\pgfqpoint{0.790833in}{3.516949in}}%
\pgfpathlineto{\pgfqpoint{0.809143in}{3.397629in}}%
\pgfpathlineto{\pgfqpoint{0.830738in}{3.217187in}}%
\pgfpathlineto{\pgfqpoint{0.849047in}{3.156773in}}%
\pgfpathlineto{\pgfqpoint{0.868294in}{3.143026in}}%
\pgfpathlineto{\pgfqpoint{0.888011in}{3.143875in}}%
\pgfpathlineto{\pgfqpoint{0.906555in}{3.251291in}}%
\pgfpathlineto{\pgfqpoint{0.924629in}{3.441623in}}%
\pgfpathlineto{\pgfqpoint{0.943876in}{3.504466in}}%
\pgfpathlineto{\pgfqpoint{0.963361in}{3.361735in}}%
\pgfpathlineto{\pgfqpoint{0.984954in}{3.215920in}}%
\pgfpathlineto{\pgfqpoint{1.003499in}{3.156915in}}%
\pgfpathlineto{\pgfqpoint{1.022278in}{3.141883in}}%
\pgfpathlineto{\pgfqpoint{1.041994in}{3.141537in}}%
\pgfpathlineto{\pgfqpoint{1.060537in}{3.154102in}}%
\pgfpathlineto{\pgfqpoint{1.079786in}{3.188575in}}%
\pgfpathlineto{\pgfqpoint{1.098563in}{3.317674in}}%
\pgfpathlineto{\pgfqpoint{1.118516in}{3.487152in}}%
\pgfpathlineto{\pgfqpoint{1.134244in}{3.452068in}}%
\pgfpathlineto{\pgfqpoint{1.175790in}{3.175408in}}%
\pgfpathlineto{\pgfqpoint{1.192690in}{3.149555in}}%
\pgfpathlineto{\pgfqpoint{1.213817in}{3.139873in}}%
\pgfpathlineto{\pgfqpoint{1.233768in}{3.143070in}}%
\pgfpathlineto{\pgfqpoint{1.252781in}{3.159056in}}%
\pgfpathlineto{\pgfqpoint{1.272029in}{3.211468in}}%
\pgfpathlineto{\pgfqpoint{1.291042in}{3.357925in}}%
\pgfpathlineto{\pgfqpoint{1.309586in}{3.484867in}}%
\pgfpathlineto{\pgfqpoint{1.328834in}{3.432163in}}%
\pgfpathlineto{\pgfqpoint{1.347376in}{3.275849in}}%
\pgfpathlineto{\pgfqpoint{1.366155in}{3.173446in}}%
\pgfpathlineto{\pgfqpoint{1.385637in}{3.146222in}}%
\pgfpathlineto{\pgfqpoint{1.407936in}{3.139072in}}%
\pgfpathlineto{\pgfqpoint{1.423664in}{3.141187in}}%
\pgfpathlineto{\pgfqpoint{1.443851in}{3.155758in}}%
\pgfpathlineto{\pgfqpoint{1.463802in}{3.192393in}}%
\pgfpathlineto{\pgfqpoint{1.482112in}{3.304159in}}%
\pgfpathlineto{\pgfqpoint{1.502534in}{3.452343in}}%
\pgfpathlineto{\pgfqpoint{1.521311in}{3.471221in}}%
\pgfpathlineto{\pgfqpoint{1.541264in}{3.352906in}}%
\pgfpathlineto{\pgfqpoint{1.559103in}{3.231528in}}%
\pgfpathlineto{\pgfqpoint{1.579993in}{3.163130in}}%
\pgfpathlineto{\pgfqpoint{1.596424in}{3.145216in}}%
\pgfpathlineto{\pgfqpoint{1.620132in}{3.139084in}}%
\pgfpathlineto{\pgfqpoint{1.637502in}{3.140162in}}%
\pgfpathlineto{\pgfqpoint{1.656515in}{3.148954in}}%
\pgfpathlineto{\pgfqpoint{1.672477in}{3.173192in}}%
\pgfpathlineto{\pgfqpoint{1.694542in}{3.216719in}}%
\pgfpathlineto{\pgfqpoint{1.712850in}{3.375625in}}%
\pgfpathlineto{\pgfqpoint{1.731160in}{3.470812in}}%
\pgfpathlineto{\pgfqpoint{1.749702in}{3.444496in}}%
\pgfpathlineto{\pgfqpoint{1.772707in}{3.287178in}}%
\pgfpathlineto{\pgfqpoint{1.791954in}{3.185861in}}%
\pgfpathlineto{\pgfqpoint{1.810968in}{3.150245in}}%
\pgfpathlineto{\pgfqpoint{1.829041in}{3.141753in}}%
\pgfpathlineto{\pgfqpoint{1.848054in}{3.138752in}}%
\pgfpathlineto{\pgfqpoint{1.867302in}{3.140344in}}%
\pgfpathlineto{\pgfqpoint{1.884907in}{3.148475in}}%
\pgfpathlineto{\pgfqpoint{1.904155in}{3.174784in}}%
\pgfpathlineto{\pgfqpoint{1.925045in}{3.266312in}}%
\pgfpathlineto{\pgfqpoint{1.944293in}{3.420069in}}%
\pgfpathlineto{\pgfqpoint{1.962837in}{3.461719in}}%
\pgfpathlineto{\pgfqpoint{1.981851in}{3.376946in}}%
\pgfpathlineto{\pgfqpoint{2.003446in}{3.271920in}}%
\pgfpathlineto{\pgfqpoint{2.019406in}{3.182480in}}%
\pgfpathlineto{\pgfqpoint{2.040062in}{3.319495in}}%
\pgfpathlineto{\pgfqpoint{2.058841in}{3.195194in}}%
\pgfpathlineto{\pgfqpoint{2.077386in}{3.154068in}}%
\pgfpathlineto{\pgfqpoint{2.098276in}{3.140504in}}%
\pgfpathlineto{\pgfqpoint{2.116821in}{3.138889in}}%
\pgfpathlineto{\pgfqpoint{2.135129in}{3.143431in}}%
\pgfpathlineto{\pgfqpoint{2.156959in}{3.160163in}}%
\pgfpathlineto{\pgfqpoint{2.174564in}{3.176032in}}%
\pgfpathlineto{\pgfqpoint{2.191932in}{3.239641in}}%
\pgfpathlineto{\pgfqpoint{2.212590in}{3.406696in}}%
\pgfpathlineto{\pgfqpoint{2.235827in}{3.458347in}}%
\pgfpathlineto{\pgfqpoint{2.252728in}{3.409302in}}%
\pgfpathlineto{\pgfqpoint{2.270099in}{3.268679in}}%
\pgfpathlineto{\pgfqpoint{2.288876in}{3.248269in}}%
\pgfpathlineto{\pgfqpoint{2.309063in}{3.165088in}}%
\pgfpathlineto{\pgfqpoint{2.326902in}{3.160099in}}%
\pgfpathlineto{\pgfqpoint{2.348732in}{3.142982in}}%
\pgfpathlineto{\pgfqpoint{2.365868in}{3.138728in}}%
\pgfpathlineto{\pgfqpoint{2.386759in}{3.142279in}}%
\pgfpathlineto{\pgfqpoint{2.405067in}{3.153892in}}%
\pgfpathlineto{\pgfqpoint{2.425723in}{3.195556in}}%
\pgfpathlineto{\pgfqpoint{2.442625in}{3.296177in}}%
\pgfpathlineto{\pgfqpoint{2.460698in}{3.391442in}}%
\pgfpathlineto{\pgfqpoint{2.482294in}{3.457387in}}%
\pgfpathlineto{\pgfqpoint{2.499663in}{3.377432in}}%
\pgfpathlineto{\pgfqpoint{2.519147in}{3.228784in}}%
\pgfpathlineto{\pgfqpoint{2.539098in}{3.163103in}}%
\pgfpathlineto{\pgfqpoint{2.559519in}{3.146997in}}%
\pgfpathlineto{\pgfqpoint{2.577827in}{3.147774in}}%
\pgfpathlineto{\pgfqpoint{2.598015in}{3.139369in}}%
\pgfpathlineto{\pgfqpoint{2.615619in}{3.139469in}}%
\pgfpathlineto{\pgfqpoint{2.634164in}{3.146701in}}%
\pgfpathlineto{\pgfqpoint{2.655758in}{3.168345in}}%
\pgfpathlineto{\pgfqpoint{2.673128in}{3.200324in}}%
\pgfpathlineto{\pgfqpoint{2.691202in}{3.306937in}}%
\pgfpathlineto{\pgfqpoint{2.712094in}{3.447028in}}%
\pgfpathlineto{\pgfqpoint{2.733219in}{3.438061in}}%
\pgfpathlineto{\pgfqpoint{2.751058in}{3.333878in}}%
\pgfpathlineto{\pgfqpoint{2.770306in}{3.206344in}}%
\pgfpathlineto{\pgfqpoint{2.791197in}{3.167466in}}%
\pgfpathlineto{\pgfqpoint{2.808333in}{3.147369in}}%
\pgfpathlineto{\pgfqpoint{2.830163in}{3.139780in}}%
\pgfpathlineto{\pgfqpoint{2.848705in}{3.140312in}}%
\pgfpathlineto{\pgfqpoint{2.867015in}{3.144017in}}%
\pgfpathlineto{\pgfqpoint{2.887437in}{3.161052in}}%
\pgfpathlineto{\pgfqpoint{2.909265in}{3.214982in}}%
\pgfpathlineto{\pgfqpoint{2.923584in}{3.278016in}}%
\pgfpathlineto{\pgfqpoint{2.941658in}{3.417627in}}%
\pgfpathlineto{\pgfqpoint{2.962550in}{3.473370in}}%
\pgfpathlineto{\pgfqpoint{2.980155in}{3.417476in}}%
\pgfpathlineto{\pgfqpoint{3.001046in}{3.255287in}}%
\pgfpathlineto{\pgfqpoint{3.022405in}{3.171950in}}%
\pgfpathlineto{\pgfqpoint{3.040244in}{3.151550in}}%
\pgfpathlineto{\pgfqpoint{3.059023in}{3.143834in}}%
\pgfpathlineto{\pgfqpoint{3.078740in}{3.139783in}}%
\pgfpathlineto{\pgfqpoint{3.114655in}{3.139752in}}%
\pgfpathlineto{\pgfqpoint{3.136250in}{3.146950in}}%
\pgfpathlineto{\pgfqpoint{3.155498in}{3.170471in}}%
\pgfpathlineto{\pgfqpoint{3.172398in}{3.217383in}}%
\pgfpathlineto{\pgfqpoint{3.212301in}{3.461879in}}%
\pgfpathlineto{\pgfqpoint{3.235540in}{3.460963in}}%
\pgfpathlineto{\pgfqpoint{3.249859in}{3.431278in}}%
\pgfpathlineto{\pgfqpoint{3.268167in}{3.318510in}}%
\pgfpathlineto{\pgfqpoint{3.289528in}{3.196741in}}%
\pgfpathlineto{\pgfqpoint{3.308540in}{3.159187in}}%
\pgfpathlineto{\pgfqpoint{3.327319in}{3.143546in}}%
\pgfpathlineto{\pgfqpoint{3.347740in}{3.139707in}}%
\pgfpathlineto{\pgfqpoint{3.366048in}{3.141780in}}%
\pgfpathlineto{\pgfqpoint{3.385298in}{3.148200in}}%
\pgfpathlineto{\pgfqpoint{3.403841in}{3.161187in}}%
\pgfpathlineto{\pgfqpoint{3.422385in}{3.194569in}}%
\pgfpathlineto{\pgfqpoint{3.442101in}{3.279206in}}%
\pgfpathlineto{\pgfqpoint{3.460880in}{3.410965in}}%
\pgfpathlineto{\pgfqpoint{3.481302in}{3.492292in}}%
\pgfpathlineto{\pgfqpoint{3.499376in}{3.462412in}}%
\pgfpathlineto{\pgfqpoint{3.513695in}{3.338292in}}%
\pgfpathlineto{\pgfqpoint{3.537871in}{3.278386in}}%
\pgfpathlineto{\pgfqpoint{3.558293in}{3.190673in}}%
\pgfpathlineto{\pgfqpoint{3.575663in}{3.164847in}}%
\pgfpathlineto{\pgfqpoint{3.596788in}{3.147552in}}%
\pgfpathlineto{\pgfqpoint{3.612984in}{3.141184in}}%
\pgfpathlineto{\pgfqpoint{3.632701in}{3.158596in}}%
\pgfpathlineto{\pgfqpoint{3.654062in}{3.144591in}}%
\pgfpathlineto{\pgfqpoint{3.673310in}{3.141053in}}%
\pgfpathlineto{\pgfqpoint{3.691384in}{3.141888in}}%
\pgfpathlineto{\pgfqpoint{3.713214in}{3.154095in}}%
\pgfpathlineto{\pgfqpoint{3.730584in}{3.177896in}}%
\pgfpathlineto{\pgfqpoint{3.747955in}{3.222975in}}%
\pgfpathlineto{\pgfqpoint{3.789736in}{3.502014in}}%
\pgfpathlineto{\pgfqpoint{3.808983in}{3.493043in}}%
\pgfpathlineto{\pgfqpoint{3.826119in}{3.421083in}}%
\pgfpathlineto{\pgfqpoint{3.847010in}{3.276042in}}%
\pgfpathlineto{\pgfqpoint{3.865789in}{3.212729in}}%
\pgfpathlineto{\pgfqpoint{3.886445in}{3.163968in}}%
\pgfpathlineto{\pgfqpoint{3.902876in}{3.148388in}}%
\pgfpathlineto{\pgfqpoint{3.922123in}{3.141794in}}%
\pgfpathlineto{\pgfqpoint{3.942074in}{3.141778in}}%
\pgfpathlineto{\pgfqpoint{3.958976in}{3.147825in}}%
\pgfpathlineto{\pgfqpoint{3.979866in}{3.160899in}}%
\pgfpathlineto{\pgfqpoint{3.997706in}{3.182789in}}%
\pgfpathlineto{\pgfqpoint{4.018596in}{3.258053in}}%
\pgfpathlineto{\pgfqpoint{4.037141in}{3.386543in}}%
\pgfpathlineto{\pgfqpoint{4.058266in}{3.518643in}}%
\pgfpathlineto{\pgfqpoint{4.075636in}{3.162396in}}%
\pgfpathlineto{\pgfqpoint{4.093946in}{3.204504in}}%
\pgfpathlineto{\pgfqpoint{4.135024in}{3.491848in}}%
\pgfpathlineto{\pgfqpoint{4.153801in}{3.530904in}}%
\pgfpathlineto{\pgfqpoint{4.172345in}{3.471770in}}%
\pgfpathlineto{\pgfqpoint{4.193001in}{3.316078in}}%
\pgfpathlineto{\pgfqpoint{4.209198in}{3.218418in}}%
\pgfpathlineto{\pgfqpoint{4.229619in}{3.167611in}}%
\pgfpathlineto{\pgfqpoint{4.250744in}{3.153658in}}%
\pgfpathlineto{\pgfqpoint{4.268584in}{3.145388in}}%
\pgfpathlineto{\pgfqpoint{4.289474in}{3.141626in}}%
\pgfpathlineto{\pgfqpoint{4.306844in}{3.146611in}}%
\pgfpathlineto{\pgfqpoint{4.326327in}{3.158682in}}%
\pgfpathlineto{\pgfqpoint{4.346983in}{3.198704in}}%
\pgfpathlineto{\pgfqpoint{4.363650in}{3.277718in}}%
\pgfpathlineto{\pgfqpoint{4.384775in}{3.422955in}}%
\pgfpathlineto{\pgfqpoint{4.404023in}{3.543478in}}%
\pgfpathlineto{\pgfqpoint{4.422096in}{3.531686in}}%
\pgfpathlineto{\pgfqpoint{4.444161in}{3.529942in}}%
\pgfpathlineto{\pgfqpoint{4.462000in}{3.408686in}}%
\pgfpathlineto{\pgfqpoint{4.479841in}{3.269025in}}%
\pgfpathlineto{\pgfqpoint{4.479841in}{3.269550in}}%
\pgfpathlineto{\pgfqpoint{4.454489in}{3.506521in}}%
\pgfpathlineto{\pgfqpoint{4.434538in}{3.540579in}}%
\pgfpathlineto{\pgfqpoint{4.414351in}{3.335825in}}%
\pgfpathlineto{\pgfqpoint{4.396511in}{3.199941in}}%
\pgfpathlineto{\pgfqpoint{4.377029in}{3.156281in}}%
\pgfpathlineto{\pgfqpoint{4.359190in}{3.142279in}}%
\pgfpathlineto{\pgfqpoint{4.340177in}{3.144193in}}%
\pgfpathlineto{\pgfqpoint{4.318581in}{3.165720in}}%
\pgfpathlineto{\pgfqpoint{4.300507in}{3.229128in}}%
\pgfpathlineto{\pgfqpoint{4.281494in}{3.394694in}}%
\pgfpathlineto{\pgfqpoint{4.264829in}{3.519177in}}%
\pgfpathlineto{\pgfqpoint{4.244173in}{3.464980in}}%
\pgfpathlineto{\pgfqpoint{4.223280in}{3.253121in}}%
\pgfpathlineto{\pgfqpoint{4.205441in}{3.174923in}}%
\pgfpathlineto{\pgfqpoint{4.186899in}{3.147848in}}%
\pgfpathlineto{\pgfqpoint{4.167415in}{3.140675in}}%
\pgfpathlineto{\pgfqpoint{4.146995in}{3.149807in}}%
\pgfpathlineto{\pgfqpoint{4.131033in}{3.171404in}}%
\pgfpathlineto{\pgfqpoint{4.109203in}{3.308253in}}%
\pgfpathlineto{\pgfqpoint{4.090658in}{3.460848in}}%
\pgfpathlineto{\pgfqpoint{4.070002in}{3.508545in}}%
\pgfpathlineto{\pgfqpoint{4.049817in}{3.332078in}}%
\pgfpathlineto{\pgfqpoint{4.033620in}{3.202741in}}%
\pgfpathlineto{\pgfqpoint{4.012025in}{3.153851in}}%
\pgfpathlineto{\pgfqpoint{3.995123in}{3.141748in}}%
\pgfpathlineto{\pgfqpoint{3.974703in}{3.141991in}}%
\pgfpathlineto{\pgfqpoint{3.954282in}{3.143853in}}%
\pgfpathlineto{\pgfqpoint{3.934563in}{3.150772in}}%
\pgfpathlineto{\pgfqpoint{3.916255in}{3.184449in}}%
\pgfpathlineto{\pgfqpoint{3.896302in}{3.322825in}}%
\pgfpathlineto{\pgfqpoint{3.877289in}{3.467786in}}%
\pgfpathlineto{\pgfqpoint{3.858746in}{3.479259in}}%
\pgfpathlineto{\pgfqpoint{3.840673in}{3.328377in}}%
\pgfpathlineto{\pgfqpoint{3.821425in}{3.190513in}}%
\pgfpathlineto{\pgfqpoint{3.797481in}{3.148234in}}%
\pgfpathlineto{\pgfqpoint{3.783164in}{3.141173in}}%
\pgfpathlineto{\pgfqpoint{3.761334in}{3.141497in}}%
\pgfpathlineto{\pgfqpoint{3.744432in}{3.152114in}}%
\pgfpathlineto{\pgfqpoint{3.724011in}{3.197028in}}%
\pgfpathlineto{\pgfqpoint{3.706642in}{3.293247in}}%
\pgfpathlineto{\pgfqpoint{3.685986in}{3.438923in}}%
\pgfpathlineto{\pgfqpoint{3.666738in}{3.484555in}}%
\pgfpathlineto{\pgfqpoint{3.648428in}{3.355459in}}%
\pgfpathlineto{\pgfqpoint{3.627538in}{3.198530in}}%
\pgfpathlineto{\pgfqpoint{3.609699in}{3.157487in}}%
\pgfpathlineto{\pgfqpoint{3.592563in}{3.143233in}}%
\pgfpathlineto{\pgfqpoint{3.573549in}{3.139834in}}%
\pgfpathlineto{\pgfqpoint{3.551250in}{3.145579in}}%
\pgfpathlineto{\pgfqpoint{3.530829in}{3.162611in}}%
\pgfpathlineto{\pgfqpoint{3.512755in}{3.213059in}}%
\pgfpathlineto{\pgfqpoint{3.496559in}{3.337649in}}%
\pgfpathlineto{\pgfqpoint{3.474729in}{3.469190in}}%
\pgfpathlineto{\pgfqpoint{3.455012in}{3.464318in}}%
\pgfpathlineto{\pgfqpoint{3.437407in}{3.296196in}}%
\pgfpathlineto{\pgfqpoint{3.418863in}{3.198243in}}%
\pgfpathlineto{\pgfqpoint{3.398207in}{3.158359in}}%
\pgfpathlineto{\pgfqpoint{3.376142in}{3.142567in}}%
\pgfpathlineto{\pgfqpoint{3.359008in}{3.139380in}}%
\pgfpathlineto{\pgfqpoint{3.340698in}{3.141109in}}%
\pgfpathlineto{\pgfqpoint{3.320042in}{3.146407in}}%
\pgfpathlineto{\pgfqpoint{3.298683in}{3.175237in}}%
\pgfpathlineto{\pgfqpoint{3.283660in}{3.220264in}}%
\pgfpathlineto{\pgfqpoint{3.263707in}{3.388238in}}%
\pgfpathlineto{\pgfqpoint{3.244225in}{3.477005in}}%
\pgfpathlineto{\pgfqpoint{3.226386in}{3.441842in}}%
\pgfpathlineto{\pgfqpoint{3.207842in}{3.383561in}}%
\pgfpathlineto{\pgfqpoint{3.186246in}{3.230579in}}%
\pgfpathlineto{\pgfqpoint{3.165826in}{3.164959in}}%
\pgfpathlineto{\pgfqpoint{3.147516in}{3.145570in}}%
\pgfpathlineto{\pgfqpoint{3.130382in}{3.139607in}}%
\pgfpathlineto{\pgfqpoint{3.107143in}{3.141653in}}%
\pgfpathlineto{\pgfqpoint{3.090476in}{3.146385in}}%
\pgfpathlineto{\pgfqpoint{3.069820in}{3.169119in}}%
\pgfpathlineto{\pgfqpoint{3.051278in}{3.247172in}}%
\pgfpathlineto{\pgfqpoint{3.033907in}{3.380060in}}%
\pgfpathlineto{\pgfqpoint{3.012548in}{3.471856in}}%
\pgfpathlineto{\pgfqpoint{2.991892in}{3.395232in}}%
\pgfpathlineto{\pgfqpoint{2.973582in}{3.240151in}}%
\pgfpathlineto{\pgfqpoint{2.955977in}{3.175410in}}%
\pgfpathlineto{\pgfqpoint{2.933912in}{3.147137in}}%
\pgfpathlineto{\pgfqpoint{2.918185in}{3.141190in}}%
\pgfpathlineto{\pgfqpoint{2.897060in}{3.139911in}}%
\pgfpathlineto{\pgfqpoint{2.878752in}{3.145073in}}%
\pgfpathlineto{\pgfqpoint{2.859504in}{3.158567in}}%
\pgfpathlineto{\pgfqpoint{2.841194in}{3.206625in}}%
\pgfpathlineto{\pgfqpoint{2.823824in}{3.317600in}}%
\pgfpathlineto{\pgfqpoint{2.804107in}{3.427407in}}%
\pgfpathlineto{\pgfqpoint{2.781808in}{3.464424in}}%
\pgfpathlineto{\pgfqpoint{2.762795in}{3.394467in}}%
\pgfpathlineto{\pgfqpoint{2.743547in}{3.242688in}}%
\pgfpathlineto{\pgfqpoint{2.725474in}{3.176845in}}%
\pgfpathlineto{\pgfqpoint{2.706929in}{3.406206in}}%
\pgfpathlineto{\pgfqpoint{2.688621in}{3.392350in}}%
\pgfpathlineto{\pgfqpoint{2.666557in}{3.226724in}}%
\pgfpathlineto{\pgfqpoint{2.647543in}{3.164980in}}%
\pgfpathlineto{\pgfqpoint{2.629468in}{3.146797in}}%
\pgfpathlineto{\pgfqpoint{2.610456in}{3.141056in}}%
\pgfpathlineto{\pgfqpoint{2.590972in}{3.138842in}}%
\pgfpathlineto{\pgfqpoint{2.569144in}{3.145708in}}%
\pgfpathlineto{\pgfqpoint{2.551069in}{3.143454in}}%
\pgfpathlineto{\pgfqpoint{2.532526in}{3.138930in}}%
\pgfpathlineto{\pgfqpoint{2.515156in}{3.142557in}}%
\pgfpathlineto{\pgfqpoint{2.496846in}{3.157892in}}%
\pgfpathlineto{\pgfqpoint{2.476895in}{3.181524in}}%
\pgfpathlineto{\pgfqpoint{2.459055in}{3.281074in}}%
\pgfpathlineto{\pgfqpoint{2.436286in}{3.436554in}}%
\pgfpathlineto{\pgfqpoint{2.418917in}{3.454881in}}%
\pgfpathlineto{\pgfqpoint{2.399433in}{3.309889in}}%
\pgfpathlineto{\pgfqpoint{2.378543in}{3.189116in}}%
\pgfpathlineto{\pgfqpoint{2.359529in}{3.155416in}}%
\pgfpathlineto{\pgfqpoint{2.340047in}{3.141969in}}%
\pgfpathlineto{\pgfqpoint{2.321503in}{3.138657in}}%
\pgfpathlineto{\pgfqpoint{2.303664in}{3.141859in}}%
\pgfpathlineto{\pgfqpoint{2.285121in}{3.152611in}}%
\pgfpathlineto{\pgfqpoint{2.261414in}{3.211451in}}%
\pgfpathlineto{\pgfqpoint{2.244512in}{3.334902in}}%
\pgfpathlineto{\pgfqpoint{2.225264in}{3.451701in}}%
\pgfpathlineto{\pgfqpoint{2.207425in}{3.458379in}}%
\pgfpathlineto{\pgfqpoint{2.188178in}{3.327355in}}%
\pgfpathlineto{\pgfqpoint{2.166113in}{3.209233in}}%
\pgfpathlineto{\pgfqpoint{2.148274in}{3.163274in}}%
\pgfpathlineto{\pgfqpoint{2.129026in}{3.149502in}}%
\pgfpathlineto{\pgfqpoint{2.111187in}{3.141818in}}%
\pgfpathlineto{\pgfqpoint{2.092408in}{3.139302in}}%
\pgfpathlineto{\pgfqpoint{2.070812in}{3.143815in}}%
\pgfpathlineto{\pgfqpoint{2.052504in}{3.159912in}}%
\pgfpathlineto{\pgfqpoint{2.034194in}{3.210263in}}%
\pgfpathlineto{\pgfqpoint{2.014712in}{3.339054in}}%
\pgfpathlineto{\pgfqpoint{1.996638in}{3.449333in}}%
\pgfpathlineto{\pgfqpoint{1.974574in}{3.449849in}}%
\pgfpathlineto{\pgfqpoint{1.937721in}{3.208313in}}%
\pgfpathlineto{\pgfqpoint{1.918708in}{3.162961in}}%
\pgfpathlineto{\pgfqpoint{1.897818in}{3.144839in}}%
\pgfpathlineto{\pgfqpoint{1.878804in}{3.139680in}}%
\pgfpathlineto{\pgfqpoint{1.861903in}{3.140586in}}%
\pgfpathlineto{\pgfqpoint{1.843595in}{3.145837in}}%
\pgfpathlineto{\pgfqpoint{1.821530in}{3.164704in}}%
\pgfpathlineto{\pgfqpoint{1.802517in}{3.220388in}}%
\pgfpathlineto{\pgfqpoint{1.783269in}{3.344973in}}%
\pgfpathlineto{\pgfqpoint{1.765430in}{3.450390in}}%
\pgfpathlineto{\pgfqpoint{1.746417in}{3.478827in}}%
\pgfpathlineto{\pgfqpoint{1.727874in}{3.419525in}}%
\pgfpathlineto{\pgfqpoint{1.708625in}{3.288733in}}%
\pgfpathlineto{\pgfqpoint{1.687500in}{3.216886in}}%
\pgfpathlineto{\pgfqpoint{1.668486in}{3.165559in}}%
\pgfpathlineto{\pgfqpoint{1.650178in}{3.149404in}}%
\pgfpathlineto{\pgfqpoint{1.631634in}{3.140962in}}%
\pgfpathlineto{\pgfqpoint{1.607223in}{3.141117in}}%
\pgfpathlineto{\pgfqpoint{1.591261in}{3.146566in}}%
\pgfpathlineto{\pgfqpoint{1.574830in}{3.158967in}}%
\pgfpathlineto{\pgfqpoint{1.555112in}{3.196251in}}%
\pgfpathlineto{\pgfqpoint{1.535630in}{3.281959in}}%
\pgfpathlineto{\pgfqpoint{1.514739in}{3.430103in}}%
\pgfpathlineto{\pgfqpoint{1.496195in}{3.492108in}}%
\pgfpathlineto{\pgfqpoint{1.476244in}{3.444672in}}%
\pgfpathlineto{\pgfqpoint{1.455822in}{3.413867in}}%
\pgfpathlineto{\pgfqpoint{1.437749in}{3.271698in}}%
\pgfpathlineto{\pgfqpoint{1.418970in}{3.200270in}}%
\pgfpathlineto{\pgfqpoint{1.398314in}{3.162170in}}%
\pgfpathlineto{\pgfqpoint{1.379535in}{3.152159in}}%
\pgfpathlineto{\pgfqpoint{1.361461in}{3.151627in}}%
\pgfpathlineto{\pgfqpoint{1.342682in}{3.142089in}}%
\pgfpathlineto{\pgfqpoint{1.324374in}{3.140922in}}%
\pgfpathlineto{\pgfqpoint{1.299961in}{3.148859in}}%
\pgfpathlineto{\pgfqpoint{1.284939in}{3.163617in}}%
\pgfpathlineto{\pgfqpoint{1.265692in}{3.206745in}}%
\pgfpathlineto{\pgfqpoint{1.245973in}{3.175869in}}%
\pgfpathlineto{\pgfqpoint{1.228839in}{3.232807in}}%
\pgfpathlineto{\pgfqpoint{1.206304in}{3.386355in}}%
\pgfpathlineto{\pgfqpoint{1.188464in}{3.482262in}}%
\pgfpathlineto{\pgfqpoint{1.168279in}{3.500385in}}%
\pgfpathlineto{\pgfqpoint{1.150909in}{3.408639in}}%
\pgfpathlineto{\pgfqpoint{1.129784in}{3.250453in}}%
\pgfpathlineto{\pgfqpoint{1.108188in}{3.186374in}}%
\pgfpathlineto{\pgfqpoint{1.092460in}{3.162052in}}%
\pgfpathlineto{\pgfqpoint{1.074152in}{3.148296in}}%
\pgfpathlineto{\pgfqpoint{1.052322in}{3.141297in}}%
\pgfpathlineto{\pgfqpoint{1.035657in}{3.141823in}}%
\pgfpathlineto{\pgfqpoint{1.013122in}{3.151818in}}%
\pgfpathlineto{\pgfqpoint{0.995048in}{3.168184in}}%
\pgfpathlineto{\pgfqpoint{0.975331in}{3.140907in}}%
\pgfpathlineto{\pgfqpoint{0.958664in}{3.146567in}}%
\pgfpathlineto{\pgfqpoint{0.938948in}{3.163505in}}%
\pgfpathlineto{\pgfqpoint{0.920640in}{3.205359in}}%
\pgfpathlineto{\pgfqpoint{0.901627in}{3.300298in}}%
\pgfpathlineto{\pgfqpoint{0.880265in}{3.459953in}}%
\pgfpathlineto{\pgfqpoint{0.860549in}{3.514166in}}%
\pgfpathlineto{\pgfqpoint{0.842475in}{3.508969in}}%
\pgfpathlineto{\pgfqpoint{0.825339in}{3.387020in}}%
\pgfpathlineto{\pgfqpoint{0.802569in}{3.258639in}}%
\pgfpathlineto{\pgfqpoint{0.784730in}{3.190086in}}%
\pgfpathlineto{\pgfqpoint{0.766188in}{3.159410in}}%
\pgfpathlineto{\pgfqpoint{0.747174in}{3.146878in}}%
\pgfpathlineto{\pgfqpoint{0.723936in}{3.141999in}}%
\pgfpathlineto{\pgfqpoint{0.706800in}{3.144699in}}%
\pgfpathlineto{\pgfqpoint{0.688257in}{3.152105in}}%
\pgfpathlineto{\pgfqpoint{0.669947in}{3.175493in}}%
\pgfpathlineto{\pgfqpoint{0.651405in}{3.232618in}}%
\pgfpathlineto{\pgfqpoint{0.651170in}{3.151186in}}%
\pgfpathlineto{\pgfqpoint{0.656802in}{3.146510in}}%
\pgfpathlineto{\pgfqpoint{0.674173in}{3.142244in}}%
\pgfpathlineto{\pgfqpoint{0.697646in}{3.157408in}}%
\pgfpathlineto{\pgfqpoint{0.715719in}{3.200892in}}%
\pgfpathlineto{\pgfqpoint{0.732386in}{3.328054in}}%
\pgfpathlineto{\pgfqpoint{0.751400in}{3.516989in}}%
\pgfpathlineto{\pgfqpoint{0.770413in}{3.491893in}}%
\pgfpathlineto{\pgfqpoint{0.790833in}{3.315596in}}%
\pgfpathlineto{\pgfqpoint{0.811489in}{3.181570in}}%
\pgfpathlineto{\pgfqpoint{0.830268in}{3.151042in}}%
\pgfpathlineto{\pgfqpoint{0.850455in}{3.141196in}}%
\pgfpathlineto{\pgfqpoint{0.868763in}{3.145802in}}%
\pgfpathlineto{\pgfqpoint{0.889185in}{3.167210in}}%
\pgfpathlineto{\pgfqpoint{0.906086in}{3.241653in}}%
\pgfpathlineto{\pgfqpoint{0.925803in}{3.421919in}}%
\pgfpathlineto{\pgfqpoint{0.944582in}{3.509732in}}%
\pgfpathlineto{\pgfqpoint{0.964064in}{3.397273in}}%
\pgfpathlineto{\pgfqpoint{0.983311in}{3.224434in}}%
\pgfpathlineto{\pgfqpoint{1.002794in}{3.161673in}}%
\pgfpathlineto{\pgfqpoint{1.021572in}{3.143748in}}%
\pgfpathlineto{\pgfqpoint{1.041289in}{3.141082in}}%
\pgfpathlineto{\pgfqpoint{1.061711in}{3.153219in}}%
\pgfpathlineto{\pgfqpoint{1.080724in}{3.190299in}}%
\pgfpathlineto{\pgfqpoint{1.098799in}{3.296599in}}%
\pgfpathlineto{\pgfqpoint{1.117342in}{3.481414in}}%
\pgfpathlineto{\pgfqpoint{1.139407in}{3.455774in}}%
\pgfpathlineto{\pgfqpoint{1.156543in}{3.346605in}}%
\pgfpathlineto{\pgfqpoint{1.174616in}{3.212199in}}%
\pgfpathlineto{\pgfqpoint{1.193864in}{3.157239in}}%
\pgfpathlineto{\pgfqpoint{1.212643in}{3.142425in}}%
\pgfpathlineto{\pgfqpoint{1.231656in}{3.140736in}}%
\pgfpathlineto{\pgfqpoint{1.249730in}{3.148451in}}%
\pgfpathlineto{\pgfqpoint{1.269681in}{3.179242in}}%
\pgfpathlineto{\pgfqpoint{1.292685in}{3.290146in}}%
\pgfpathlineto{\pgfqpoint{1.310524in}{3.460515in}}%
\pgfpathlineto{\pgfqpoint{1.326251in}{3.485212in}}%
\pgfpathlineto{\pgfqpoint{1.348082in}{3.348990in}}%
\pgfpathlineto{\pgfqpoint{1.366624in}{3.216323in}}%
\pgfpathlineto{\pgfqpoint{1.388454in}{3.153943in}}%
\pgfpathlineto{\pgfqpoint{1.404885in}{3.142768in}}%
\pgfpathlineto{\pgfqpoint{1.424133in}{3.140076in}}%
\pgfpathlineto{\pgfqpoint{1.443380in}{3.146676in}}%
\pgfpathlineto{\pgfqpoint{1.464037in}{3.143888in}}%
\pgfpathlineto{\pgfqpoint{1.479999in}{3.154980in}}%
\pgfpathlineto{\pgfqpoint{1.502298in}{3.191762in}}%
\pgfpathlineto{\pgfqpoint{1.521545in}{3.309222in}}%
\pgfpathlineto{\pgfqpoint{1.539855in}{3.461872in}}%
\pgfpathlineto{\pgfqpoint{1.558163in}{3.466100in}}%
\pgfpathlineto{\pgfqpoint{1.602058in}{3.185846in}}%
\pgfpathlineto{\pgfqpoint{1.617786in}{3.159355in}}%
\pgfpathlineto{\pgfqpoint{1.633982in}{3.144833in}}%
\pgfpathlineto{\pgfqpoint{1.659801in}{3.139631in}}%
\pgfpathlineto{\pgfqpoint{1.674589in}{3.142304in}}%
\pgfpathlineto{\pgfqpoint{1.695481in}{3.160224in}}%
\pgfpathlineto{\pgfqpoint{1.714258in}{3.206744in}}%
\pgfpathlineto{\pgfqpoint{1.733037in}{3.316012in}}%
\pgfpathlineto{\pgfqpoint{1.752050in}{3.466149in}}%
\pgfpathlineto{\pgfqpoint{1.770593in}{3.442185in}}%
\pgfpathlineto{\pgfqpoint{1.790546in}{3.359836in}}%
\pgfpathlineto{\pgfqpoint{1.811671in}{3.201357in}}%
\pgfpathlineto{\pgfqpoint{1.829276in}{3.158000in}}%
\pgfpathlineto{\pgfqpoint{1.848523in}{3.145854in}}%
\pgfpathlineto{\pgfqpoint{1.866364in}{3.139629in}}%
\pgfpathlineto{\pgfqpoint{1.887021in}{3.140748in}}%
\pgfpathlineto{\pgfqpoint{1.904389in}{3.149200in}}%
\pgfpathlineto{\pgfqpoint{1.925750in}{3.181763in}}%
\pgfpathlineto{\pgfqpoint{1.943824in}{3.265502in}}%
\pgfpathlineto{\pgfqpoint{1.962837in}{3.427401in}}%
\pgfpathlineto{\pgfqpoint{1.984433in}{3.468055in}}%
\pgfpathlineto{\pgfqpoint{2.002272in}{3.445104in}}%
\pgfpathlineto{\pgfqpoint{2.021520in}{3.340379in}}%
\pgfpathlineto{\pgfqpoint{2.041471in}{3.199620in}}%
\pgfpathlineto{\pgfqpoint{2.059310in}{3.155543in}}%
\pgfpathlineto{\pgfqpoint{2.082080in}{3.141472in}}%
\pgfpathlineto{\pgfqpoint{2.098745in}{3.139244in}}%
\pgfpathlineto{\pgfqpoint{2.116584in}{3.142478in}}%
\pgfpathlineto{\pgfqpoint{2.136772in}{3.155559in}}%
\pgfpathlineto{\pgfqpoint{2.156019in}{3.189618in}}%
\pgfpathlineto{\pgfqpoint{2.174093in}{3.278453in}}%
\pgfpathlineto{\pgfqpoint{2.193577in}{3.444018in}}%
\pgfpathlineto{\pgfqpoint{2.211182in}{3.459530in}}%
\pgfpathlineto{\pgfqpoint{2.232541in}{3.382478in}}%
\pgfpathlineto{\pgfqpoint{2.250380in}{3.245825in}}%
\pgfpathlineto{\pgfqpoint{2.267985in}{3.172993in}}%
\pgfpathlineto{\pgfqpoint{2.289347in}{3.146411in}}%
\pgfpathlineto{\pgfqpoint{2.310003in}{3.143806in}}%
\pgfpathlineto{\pgfqpoint{2.328311in}{3.139854in}}%
\pgfpathlineto{\pgfqpoint{2.345916in}{3.143893in}}%
\pgfpathlineto{\pgfqpoint{2.367041in}{3.139010in}}%
\pgfpathlineto{\pgfqpoint{2.386288in}{3.141329in}}%
\pgfpathlineto{\pgfqpoint{2.406476in}{3.150500in}}%
\pgfpathlineto{\pgfqpoint{2.424080in}{3.172825in}}%
\pgfpathlineto{\pgfqpoint{2.448727in}{3.284115in}}%
\pgfpathlineto{\pgfqpoint{2.462107in}{3.409225in}}%
\pgfpathlineto{\pgfqpoint{2.480886in}{3.464823in}}%
\pgfpathlineto{\pgfqpoint{2.502245in}{3.424961in}}%
\pgfpathlineto{\pgfqpoint{2.519381in}{3.282032in}}%
\pgfpathlineto{\pgfqpoint{2.540037in}{3.183418in}}%
\pgfpathlineto{\pgfqpoint{2.558111in}{3.151571in}}%
\pgfpathlineto{\pgfqpoint{2.582523in}{3.140120in}}%
\pgfpathlineto{\pgfqpoint{2.597546in}{3.138697in}}%
\pgfpathlineto{\pgfqpoint{2.614916in}{3.142949in}}%
\pgfpathlineto{\pgfqpoint{2.636041in}{3.159606in}}%
\pgfpathlineto{\pgfqpoint{2.653880in}{3.200378in}}%
\pgfpathlineto{\pgfqpoint{2.677353in}{3.291127in}}%
\pgfpathlineto{\pgfqpoint{2.693081in}{3.426580in}}%
\pgfpathlineto{\pgfqpoint{2.713971in}{3.466755in}}%
\pgfpathlineto{\pgfqpoint{2.732280in}{3.400252in}}%
\pgfpathlineto{\pgfqpoint{2.749650in}{3.269539in}}%
\pgfpathlineto{\pgfqpoint{2.767724in}{3.202425in}}%
\pgfpathlineto{\pgfqpoint{2.787676in}{3.270781in}}%
\pgfpathlineto{\pgfqpoint{2.808333in}{3.456970in}}%
\pgfpathlineto{\pgfqpoint{2.829692in}{3.312727in}}%
\pgfpathlineto{\pgfqpoint{2.845888in}{3.213203in}}%
\pgfpathlineto{\pgfqpoint{2.868424in}{3.163073in}}%
\pgfpathlineto{\pgfqpoint{2.886029in}{3.149876in}}%
\pgfpathlineto{\pgfqpoint{2.905511in}{3.142770in}}%
\pgfpathlineto{\pgfqpoint{2.924758in}{3.139677in}}%
\pgfpathlineto{\pgfqpoint{2.942363in}{3.143947in}}%
\pgfpathlineto{\pgfqpoint{2.962080in}{3.164855in}}%
\pgfpathlineto{\pgfqpoint{2.982033in}{3.224024in}}%
\pgfpathlineto{\pgfqpoint{2.998229in}{3.298312in}}%
\pgfpathlineto{\pgfqpoint{3.022171in}{3.473728in}}%
\pgfpathlineto{\pgfqpoint{3.037664in}{3.467831in}}%
\pgfpathlineto{\pgfqpoint{3.059492in}{3.336601in}}%
\pgfpathlineto{\pgfqpoint{3.077333in}{3.216716in}}%
\pgfpathlineto{\pgfqpoint{3.097753in}{3.158587in}}%
\pgfpathlineto{\pgfqpoint{3.115829in}{3.144209in}}%
\pgfpathlineto{\pgfqpoint{3.134371in}{3.139590in}}%
\pgfpathlineto{\pgfqpoint{3.154558in}{3.143049in}}%
\pgfpathlineto{\pgfqpoint{3.172866in}{3.152651in}}%
\pgfpathlineto{\pgfqpoint{3.194462in}{3.186052in}}%
\pgfpathlineto{\pgfqpoint{3.213241in}{3.248040in}}%
\pgfpathlineto{\pgfqpoint{3.230141in}{3.403551in}}%
\pgfpathlineto{\pgfqpoint{3.251268in}{3.478036in}}%
\pgfpathlineto{\pgfqpoint{3.267228in}{3.470182in}}%
\pgfpathlineto{\pgfqpoint{3.285772in}{3.350901in}}%
\pgfpathlineto{\pgfqpoint{3.307837in}{3.221442in}}%
\pgfpathlineto{\pgfqpoint{3.325910in}{3.167007in}}%
\pgfpathlineto{\pgfqpoint{3.343749in}{3.147193in}}%
\pgfpathlineto{\pgfqpoint{3.368397in}{3.140374in}}%
\pgfpathlineto{\pgfqpoint{3.385767in}{3.141068in}}%
\pgfpathlineto{\pgfqpoint{3.404309in}{3.145387in}}%
\pgfpathlineto{\pgfqpoint{3.422619in}{3.157467in}}%
\pgfpathlineto{\pgfqpoint{3.443510in}{3.190815in}}%
\pgfpathlineto{\pgfqpoint{3.463932in}{3.261810in}}%
\pgfpathlineto{\pgfqpoint{3.502427in}{3.491366in}}%
\pgfpathlineto{\pgfqpoint{3.520971in}{3.461947in}}%
\pgfpathlineto{\pgfqpoint{3.539748in}{3.393445in}}%
\pgfpathlineto{\pgfqpoint{3.556650in}{3.268292in}}%
\pgfpathlineto{\pgfqpoint{3.577072in}{3.179823in}}%
\pgfpathlineto{\pgfqpoint{3.596319in}{3.152636in}}%
\pgfpathlineto{\pgfqpoint{3.616270in}{3.143504in}}%
\pgfpathlineto{\pgfqpoint{3.636458in}{3.140116in}}%
\pgfpathlineto{\pgfqpoint{3.655471in}{3.141872in}}%
\pgfpathlineto{\pgfqpoint{3.672607in}{3.149819in}}%
\pgfpathlineto{\pgfqpoint{3.690210in}{3.166429in}}%
\pgfpathlineto{\pgfqpoint{3.710868in}{3.208882in}}%
\pgfpathlineto{\pgfqpoint{3.731993in}{3.306105in}}%
\pgfpathlineto{\pgfqpoint{3.751006in}{3.445846in}}%
\pgfpathlineto{\pgfqpoint{3.770254in}{3.507308in}}%
\pgfpathlineto{\pgfqpoint{3.788327in}{3.468853in}}%
\pgfpathlineto{\pgfqpoint{3.805932in}{3.372190in}}%
\pgfpathlineto{\pgfqpoint{3.826354in}{3.241555in}}%
\pgfpathlineto{\pgfqpoint{3.846305in}{3.197245in}}%
\pgfpathlineto{\pgfqpoint{3.863206in}{3.166053in}}%
\pgfpathlineto{\pgfqpoint{3.884802in}{3.145929in}}%
\pgfpathlineto{\pgfqpoint{3.908744in}{3.140697in}}%
\pgfpathlineto{\pgfqpoint{3.924235in}{3.141936in}}%
\pgfpathlineto{\pgfqpoint{3.941840in}{3.147669in}}%
\pgfpathlineto{\pgfqpoint{3.962496in}{3.228292in}}%
\pgfpathlineto{\pgfqpoint{3.979866in}{3.167784in}}%
\pgfpathlineto{\pgfqpoint{4.001931in}{3.146747in}}%
\pgfpathlineto{\pgfqpoint{4.022587in}{3.141106in}}%
\pgfpathlineto{\pgfqpoint{4.037609in}{3.142912in}}%
\pgfpathlineto{\pgfqpoint{4.059205in}{3.153167in}}%
\pgfpathlineto{\pgfqpoint{4.077279in}{3.173098in}}%
\pgfpathlineto{\pgfqpoint{4.094649in}{3.218955in}}%
\pgfpathlineto{\pgfqpoint{4.119296in}{3.377838in}}%
\pgfpathlineto{\pgfqpoint{4.133615in}{3.493312in}}%
\pgfpathlineto{\pgfqpoint{4.153801in}{3.527851in}}%
\pgfpathlineto{\pgfqpoint{4.171640in}{3.525144in}}%
\pgfpathlineto{\pgfqpoint{4.196053in}{3.400889in}}%
\pgfpathlineto{\pgfqpoint{4.211075in}{3.272941in}}%
\pgfpathlineto{\pgfqpoint{4.231965in}{3.183885in}}%
\pgfpathlineto{\pgfqpoint{4.250979in}{3.155097in}}%
\pgfpathlineto{\pgfqpoint{4.269523in}{3.144827in}}%
\pgfpathlineto{\pgfqpoint{4.290179in}{3.142401in}}%
\pgfpathlineto{\pgfqpoint{4.306844in}{3.148348in}}%
\pgfpathlineto{\pgfqpoint{4.327266in}{3.167810in}}%
\pgfpathlineto{\pgfqpoint{4.346983in}{3.201982in}}%
\pgfpathlineto{\pgfqpoint{4.366936in}{3.307591in}}%
\pgfpathlineto{\pgfqpoint{4.385714in}{3.416106in}}%
\pgfpathlineto{\pgfqpoint{4.403319in}{3.541339in}}%
\pgfpathlineto{\pgfqpoint{4.419279in}{3.539280in}}%
\pgfpathlineto{\pgfqpoint{4.439701in}{3.454377in}}%
\pgfpathlineto{\pgfqpoint{4.462000in}{3.286837in}}%
\pgfpathlineto{\pgfqpoint{4.479136in}{3.202423in}}%
\pgfpathlineto{\pgfqpoint{4.475850in}{3.219080in}}%
\pgfpathlineto{\pgfqpoint{4.454254in}{3.410990in}}%
\pgfpathlineto{\pgfqpoint{4.435476in}{3.539158in}}%
\pgfpathlineto{\pgfqpoint{4.417871in}{3.498284in}}%
\pgfpathlineto{\pgfqpoint{4.398859in}{3.290656in}}%
\pgfpathlineto{\pgfqpoint{4.377029in}{3.180859in}}%
\pgfpathlineto{\pgfqpoint{4.358719in}{3.152360in}}%
\pgfpathlineto{\pgfqpoint{4.341351in}{3.141938in}}%
\pgfpathlineto{\pgfqpoint{4.319050in}{3.148941in}}%
\pgfpathlineto{\pgfqpoint{4.301211in}{3.175274in}}%
\pgfpathlineto{\pgfqpoint{4.283371in}{3.262148in}}%
\pgfpathlineto{\pgfqpoint{4.262950in}{3.459631in}}%
\pgfpathlineto{\pgfqpoint{4.246050in}{3.530724in}}%
\pgfpathlineto{\pgfqpoint{4.225394in}{3.407099in}}%
\pgfpathlineto{\pgfqpoint{4.205675in}{3.213632in}}%
\pgfpathlineto{\pgfqpoint{4.187133in}{3.161872in}}%
\pgfpathlineto{\pgfqpoint{4.166711in}{3.143795in}}%
\pgfpathlineto{\pgfqpoint{4.167415in}{3.141473in}}%
\pgfpathlineto{\pgfqpoint{4.149575in}{3.141734in}}%
\pgfpathlineto{\pgfqpoint{4.129154in}{3.156584in}}%
\pgfpathlineto{\pgfqpoint{4.110846in}{3.197051in}}%
\pgfpathlineto{\pgfqpoint{4.072350in}{3.494813in}}%
\pgfpathlineto{\pgfqpoint{4.051225in}{3.474636in}}%
\pgfpathlineto{\pgfqpoint{4.031507in}{3.256527in}}%
\pgfpathlineto{\pgfqpoint{4.013433in}{3.177614in}}%
\pgfpathlineto{\pgfqpoint{3.993951in}{3.148088in}}%
\pgfpathlineto{\pgfqpoint{3.974233in}{3.140658in}}%
\pgfpathlineto{\pgfqpoint{3.956393in}{3.145120in}}%
\pgfpathlineto{\pgfqpoint{3.935503in}{3.170428in}}%
\pgfpathlineto{\pgfqpoint{3.918603in}{3.245568in}}%
\pgfpathlineto{\pgfqpoint{3.899119in}{3.433656in}}%
\pgfpathlineto{\pgfqpoint{3.878463in}{3.501353in}}%
\pgfpathlineto{\pgfqpoint{3.838794in}{3.208713in}}%
\pgfpathlineto{\pgfqpoint{3.821189in}{3.159770in}}%
\pgfpathlineto{\pgfqpoint{3.803820in}{3.145405in}}%
\pgfpathlineto{\pgfqpoint{3.783164in}{3.140590in}}%
\pgfpathlineto{\pgfqpoint{3.762977in}{3.147716in}}%
\pgfpathlineto{\pgfqpoint{3.747484in}{3.183561in}}%
\pgfpathlineto{\pgfqpoint{3.725419in}{3.256643in}}%
\pgfpathlineto{\pgfqpoint{3.708520in}{3.415720in}}%
\pgfpathlineto{\pgfqpoint{3.686924in}{3.452068in}}%
\pgfpathlineto{\pgfqpoint{3.669319in}{3.488246in}}%
\pgfpathlineto{\pgfqpoint{3.647960in}{3.316669in}}%
\pgfpathlineto{\pgfqpoint{3.627538in}{3.188951in}}%
\pgfpathlineto{\pgfqpoint{3.609230in}{3.153129in}}%
\pgfpathlineto{\pgfqpoint{3.588337in}{3.143071in}}%
\pgfpathlineto{\pgfqpoint{3.572612in}{3.140056in}}%
\pgfpathlineto{\pgfqpoint{3.550313in}{3.146092in}}%
\pgfpathlineto{\pgfqpoint{3.532003in}{3.163699in}}%
\pgfpathlineto{\pgfqpoint{3.514867in}{3.217622in}}%
\pgfpathlineto{\pgfqpoint{3.473086in}{3.467277in}}%
\pgfpathlineto{\pgfqpoint{3.455952in}{3.471154in}}%
\pgfpathlineto{\pgfqpoint{3.438111in}{3.333291in}}%
\pgfpathlineto{\pgfqpoint{3.418863in}{3.200923in}}%
\pgfpathlineto{\pgfqpoint{3.393747in}{3.152451in}}%
\pgfpathlineto{\pgfqpoint{3.378021in}{3.142833in}}%
\pgfpathlineto{\pgfqpoint{3.360417in}{3.139737in}}%
\pgfpathlineto{\pgfqpoint{3.342577in}{3.142094in}}%
\pgfpathlineto{\pgfqpoint{3.320747in}{3.155986in}}%
\pgfpathlineto{\pgfqpoint{3.304551in}{3.187692in}}%
\pgfpathlineto{\pgfqpoint{3.303611in}{3.259328in}}%
\pgfpathlineto{\pgfqpoint{3.282486in}{3.209465in}}%
\pgfpathlineto{\pgfqpoint{3.264411in}{3.151397in}}%
\pgfpathlineto{\pgfqpoint{3.241643in}{3.200675in}}%
\pgfpathlineto{\pgfqpoint{3.222395in}{3.331890in}}%
\pgfpathlineto{\pgfqpoint{3.205259in}{3.457843in}}%
\pgfpathlineto{\pgfqpoint{3.186951in}{3.465507in}}%
\pgfpathlineto{\pgfqpoint{3.168407in}{3.300345in}}%
\pgfpathlineto{\pgfqpoint{3.147282in}{3.185075in}}%
\pgfpathlineto{\pgfqpoint{3.132025in}{3.158752in}}%
\pgfpathlineto{\pgfqpoint{3.109021in}{3.144241in}}%
\pgfpathlineto{\pgfqpoint{3.091181in}{3.139303in}}%
\pgfpathlineto{\pgfqpoint{3.070760in}{3.144929in}}%
\pgfpathlineto{\pgfqpoint{3.054563in}{3.152740in}}%
\pgfpathlineto{\pgfqpoint{3.032499in}{3.192786in}}%
\pgfpathlineto{\pgfqpoint{3.014894in}{3.289615in}}%
\pgfpathlineto{\pgfqpoint{2.994238in}{3.441680in}}%
\pgfpathlineto{\pgfqpoint{2.976164in}{3.468307in}}%
\pgfpathlineto{\pgfqpoint{2.954803in}{3.324456in}}%
\pgfpathlineto{\pgfqpoint{2.936260in}{3.235396in}}%
\pgfpathlineto{\pgfqpoint{2.918890in}{3.173173in}}%
\pgfpathlineto{\pgfqpoint{2.897765in}{3.146575in}}%
\pgfpathlineto{\pgfqpoint{2.878986in}{3.139684in}}%
\pgfpathlineto{\pgfqpoint{2.857625in}{3.140656in}}%
\pgfpathlineto{\pgfqpoint{2.839317in}{3.149661in}}%
\pgfpathlineto{\pgfqpoint{2.821712in}{3.173088in}}%
\pgfpathlineto{\pgfqpoint{2.802230in}{3.247778in}}%
\pgfpathlineto{\pgfqpoint{2.783451in}{3.378953in}}%
\pgfpathlineto{\pgfqpoint{2.764203in}{3.467471in}}%
\pgfpathlineto{\pgfqpoint{2.746130in}{3.436522in}}%
\pgfpathlineto{\pgfqpoint{2.726646in}{3.291678in}}%
\pgfpathlineto{\pgfqpoint{2.705052in}{3.210325in}}%
\pgfpathlineto{\pgfqpoint{2.686039in}{3.161354in}}%
\pgfpathlineto{\pgfqpoint{2.667494in}{3.144794in}}%
\pgfpathlineto{\pgfqpoint{2.649186in}{3.138923in}}%
\pgfpathlineto{\pgfqpoint{2.630407in}{3.139780in}}%
\pgfpathlineto{\pgfqpoint{2.611394in}{3.145794in}}%
\pgfpathlineto{\pgfqpoint{2.590503in}{3.153875in}}%
\pgfpathlineto{\pgfqpoint{2.571021in}{3.191922in}}%
\pgfpathlineto{\pgfqpoint{2.530883in}{3.434973in}}%
\pgfpathlineto{\pgfqpoint{2.514921in}{3.463309in}}%
\pgfpathlineto{\pgfqpoint{2.493560in}{3.405751in}}%
\pgfpathlineto{\pgfqpoint{2.474312in}{3.251860in}}%
\pgfpathlineto{\pgfqpoint{2.455770in}{3.294024in}}%
\pgfpathlineto{\pgfqpoint{2.437460in}{3.193247in}}%
\pgfpathlineto{\pgfqpoint{2.415866in}{3.152417in}}%
\pgfpathlineto{\pgfqpoint{2.397087in}{3.141064in}}%
\pgfpathlineto{\pgfqpoint{2.381125in}{3.138921in}}%
\pgfpathlineto{\pgfqpoint{2.359764in}{3.140416in}}%
\pgfpathlineto{\pgfqpoint{2.341221in}{3.147817in}}%
\pgfpathlineto{\pgfqpoint{2.322442in}{3.174354in}}%
\pgfpathlineto{\pgfqpoint{2.303195in}{3.248367in}}%
\pgfpathlineto{\pgfqpoint{2.284416in}{3.407658in}}%
\pgfpathlineto{\pgfqpoint{2.266108in}{3.465482in}}%
\pgfpathlineto{\pgfqpoint{2.243809in}{3.386457in}}%
\pgfpathlineto{\pgfqpoint{2.228787in}{3.242721in}}%
\pgfpathlineto{\pgfqpoint{2.207191in}{3.176039in}}%
\pgfpathlineto{\pgfqpoint{2.182544in}{3.149056in}}%
\pgfpathlineto{\pgfqpoint{2.169869in}{3.142419in}}%
\pgfpathlineto{\pgfqpoint{2.150856in}{3.138588in}}%
\pgfpathlineto{\pgfqpoint{2.130669in}{3.140440in}}%
\pgfpathlineto{\pgfqpoint{2.108135in}{3.151513in}}%
\pgfpathlineto{\pgfqpoint{2.094051in}{3.169378in}}%
\pgfpathlineto{\pgfqpoint{2.071283in}{3.248212in}}%
\pgfpathlineto{\pgfqpoint{2.051565in}{3.393115in}}%
\pgfpathlineto{\pgfqpoint{2.033022in}{3.462125in}}%
\pgfpathlineto{\pgfqpoint{2.014009in}{3.427360in}}%
\pgfpathlineto{\pgfqpoint{1.993822in}{3.370003in}}%
\pgfpathlineto{\pgfqpoint{1.974808in}{3.248082in}}%
\pgfpathlineto{\pgfqpoint{1.958846in}{3.187961in}}%
\pgfpathlineto{\pgfqpoint{1.936547in}{3.151589in}}%
\pgfpathlineto{\pgfqpoint{1.917768in}{3.142114in}}%
\pgfpathlineto{\pgfqpoint{1.901338in}{3.138923in}}%
\pgfpathlineto{\pgfqpoint{1.880447in}{3.141383in}}%
\pgfpathlineto{\pgfqpoint{1.860496in}{3.150148in}}%
\pgfpathlineto{\pgfqpoint{1.842186in}{3.163080in}}%
\pgfpathlineto{\pgfqpoint{1.820356in}{3.164070in}}%
\pgfpathlineto{\pgfqpoint{1.801343in}{3.215277in}}%
\pgfpathlineto{\pgfqpoint{1.782330in}{3.347850in}}%
\pgfpathlineto{\pgfqpoint{1.763787in}{3.454948in}}%
\pgfpathlineto{\pgfqpoint{1.744539in}{3.472473in}}%
\pgfpathlineto{\pgfqpoint{1.725760in}{3.356029in}}%
\pgfpathlineto{\pgfqpoint{1.704636in}{3.234926in}}%
\pgfpathlineto{\pgfqpoint{1.686560in}{3.178617in}}%
\pgfpathlineto{\pgfqpoint{1.671069in}{3.155609in}}%
\pgfpathlineto{\pgfqpoint{1.650178in}{3.143242in}}%
\pgfpathlineto{\pgfqpoint{1.629522in}{3.140196in}}%
\pgfpathlineto{\pgfqpoint{1.611917in}{3.139329in}}%
\pgfpathlineto{\pgfqpoint{1.593373in}{3.143685in}}%
\pgfpathlineto{\pgfqpoint{1.574594in}{3.153288in}}%
\pgfpathlineto{\pgfqpoint{1.574594in}{3.153288in}}%
\pgfusepath{stroke}%
\end{pgfscope}%
\begin{pgfscope}%
\pgfpathrectangle{\pgfqpoint{0.444748in}{3.117349in}}{\pgfqpoint{4.231419in}{0.467251in}}%
\pgfusepath{clip}%
\pgfsetbuttcap%
\pgfsetroundjoin%
\definecolor{currentfill}{rgb}{0.047059,0.364706,0.647059}%
\pgfsetfillcolor{currentfill}%
\pgfsetlinewidth{1.003750pt}%
\definecolor{currentstroke}{rgb}{0.047059,0.364706,0.647059}%
\pgfsetstrokecolor{currentstroke}%
\pgfsetdash{}{0pt}%
\pgfsys@defobject{currentmarker}{\pgfqpoint{-0.010417in}{-0.010417in}}{\pgfqpoint{0.010417in}{0.010417in}}{%
\pgfpathmoveto{\pgfqpoint{0.000000in}{-0.010417in}}%
\pgfpathcurveto{\pgfqpoint{0.002763in}{-0.010417in}}{\pgfqpoint{0.005412in}{-0.009319in}}{\pgfqpoint{0.007366in}{-0.007366in}}%
\pgfpathcurveto{\pgfqpoint{0.009319in}{-0.005412in}}{\pgfqpoint{0.010417in}{-0.002763in}}{\pgfqpoint{0.010417in}{0.000000in}}%
\pgfpathcurveto{\pgfqpoint{0.010417in}{0.002763in}}{\pgfqpoint{0.009319in}{0.005412in}}{\pgfqpoint{0.007366in}{0.007366in}}%
\pgfpathcurveto{\pgfqpoint{0.005412in}{0.009319in}}{\pgfqpoint{0.002763in}{0.010417in}}{\pgfqpoint{0.000000in}{0.010417in}}%
\pgfpathcurveto{\pgfqpoint{-0.002763in}{0.010417in}}{\pgfqpoint{-0.005412in}{0.009319in}}{\pgfqpoint{-0.007366in}{0.007366in}}%
\pgfpathcurveto{\pgfqpoint{-0.009319in}{0.005412in}}{\pgfqpoint{-0.010417in}{0.002763in}}{\pgfqpoint{-0.010417in}{0.000000in}}%
\pgfpathcurveto{\pgfqpoint{-0.010417in}{-0.002763in}}{\pgfqpoint{-0.009319in}{-0.005412in}}{\pgfqpoint{-0.007366in}{-0.007366in}}%
\pgfpathcurveto{\pgfqpoint{-0.005412in}{-0.009319in}}{\pgfqpoint{-0.002763in}{-0.010417in}}{\pgfqpoint{0.000000in}{-0.010417in}}%
\pgfpathlineto{\pgfqpoint{0.000000in}{-0.010417in}}%
\pgfpathclose%
\pgfusepath{stroke,fill}%
}%
\begin{pgfscope}%
\pgfsys@transformshift{0.637791in}{3.358858in}%
\pgfsys@useobject{currentmarker}{}%
\end{pgfscope}%
\begin{pgfscope}%
\pgfsys@transformshift{0.656802in}{3.208523in}%
\pgfsys@useobject{currentmarker}{}%
\end{pgfscope}%
\begin{pgfscope}%
\pgfsys@transformshift{0.677224in}{3.160998in}%
\pgfsys@useobject{currentmarker}{}%
\end{pgfscope}%
\begin{pgfscope}%
\pgfsys@transformshift{0.694360in}{3.147068in}%
\pgfsys@useobject{currentmarker}{}%
\end{pgfscope}%
\begin{pgfscope}%
\pgfsys@transformshift{0.715250in}{3.182987in}%
\pgfsys@useobject{currentmarker}{}%
\end{pgfscope}%
\begin{pgfscope}%
\pgfsys@transformshift{0.732386in}{3.262648in}%
\pgfsys@useobject{currentmarker}{}%
\end{pgfscope}%
\begin{pgfscope}%
\pgfsys@transformshift{0.754217in}{3.483913in}%
\pgfsys@useobject{currentmarker}{}%
\end{pgfscope}%
\begin{pgfscope}%
\pgfsys@transformshift{0.772056in}{3.534292in}%
\pgfsys@useobject{currentmarker}{}%
\end{pgfscope}%
\begin{pgfscope}%
\pgfsys@transformshift{0.791303in}{3.396729in}%
\pgfsys@useobject{currentmarker}{}%
\end{pgfscope}%
\begin{pgfscope}%
\pgfsys@transformshift{0.808908in}{3.230300in}%
\pgfsys@useobject{currentmarker}{}%
\end{pgfscope}%
\begin{pgfscope}%
\pgfsys@transformshift{0.829094in}{3.166305in}%
\pgfsys@useobject{currentmarker}{}%
\end{pgfscope}%
\begin{pgfscope}%
\pgfsys@transformshift{0.849750in}{3.144184in}%
\pgfsys@useobject{currentmarker}{}%
\end{pgfscope}%
\begin{pgfscope}%
\pgfsys@transformshift{0.868763in}{3.145156in}%
\pgfsys@useobject{currentmarker}{}%
\end{pgfscope}%
\begin{pgfscope}%
\pgfsys@transformshift{0.887776in}{3.165414in}%
\pgfsys@useobject{currentmarker}{}%
\end{pgfscope}%
\begin{pgfscope}%
\pgfsys@transformshift{0.905852in}{3.222613in}%
\pgfsys@useobject{currentmarker}{}%
\end{pgfscope}%
\begin{pgfscope}%
\pgfsys@transformshift{0.925803in}{3.394154in}%
\pgfsys@useobject{currentmarker}{}%
\end{pgfscope}%
\begin{pgfscope}%
\pgfsys@transformshift{0.943876in}{3.525339in}%
\pgfsys@useobject{currentmarker}{}%
\end{pgfscope}%
\begin{pgfscope}%
\pgfsys@transformshift{0.965003in}{3.448474in}%
\pgfsys@useobject{currentmarker}{}%
\end{pgfscope}%
\begin{pgfscope}%
\pgfsys@transformshift{0.984720in}{3.259834in}%
\pgfsys@useobject{currentmarker}{}%
\end{pgfscope}%
\begin{pgfscope}%
\pgfsys@transformshift{1.001385in}{3.180450in}%
\pgfsys@useobject{currentmarker}{}%
\end{pgfscope}%
\begin{pgfscope}%
\pgfsys@transformshift{1.021104in}{3.147174in}%
\pgfsys@useobject{currentmarker}{}%
\end{pgfscope}%
\begin{pgfscope}%
\pgfsys@transformshift{1.041760in}{3.141188in}%
\pgfsys@useobject{currentmarker}{}%
\end{pgfscope}%
\begin{pgfscope}%
\pgfsys@transformshift{1.061007in}{3.151904in}%
\pgfsys@useobject{currentmarker}{}%
\end{pgfscope}%
\begin{pgfscope}%
\pgfsys@transformshift{1.079786in}{3.182568in}%
\pgfsys@useobject{currentmarker}{}%
\end{pgfscope}%
\begin{pgfscope}%
\pgfsys@transformshift{1.095983in}{3.203081in}%
\pgfsys@useobject{currentmarker}{}%
\end{pgfscope}%
\begin{pgfscope}%
\pgfsys@transformshift{1.118985in}{3.337288in}%
\pgfsys@useobject{currentmarker}{}%
\end{pgfscope}%
\begin{pgfscope}%
\pgfsys@transformshift{1.136824in}{3.500599in}%
\pgfsys@useobject{currentmarker}{}%
\end{pgfscope}%
\begin{pgfscope}%
\pgfsys@transformshift{1.156543in}{3.458174in}%
\pgfsys@useobject{currentmarker}{}%
\end{pgfscope}%
\begin{pgfscope}%
\pgfsys@transformshift{1.175319in}{3.294586in}%
\pgfsys@useobject{currentmarker}{}%
\end{pgfscope}%
\begin{pgfscope}%
\pgfsys@transformshift{1.194098in}{3.186050in}%
\pgfsys@useobject{currentmarker}{}%
\end{pgfscope}%
\begin{pgfscope}%
\pgfsys@transformshift{1.213112in}{3.149745in}%
\pgfsys@useobject{currentmarker}{}%
\end{pgfscope}%
\begin{pgfscope}%
\pgfsys@transformshift{1.235176in}{3.140828in}%
\pgfsys@useobject{currentmarker}{}%
\end{pgfscope}%
\begin{pgfscope}%
\pgfsys@transformshift{1.251138in}{3.145006in}%
\pgfsys@useobject{currentmarker}{}%
\end{pgfscope}%
\begin{pgfscope}%
\pgfsys@transformshift{1.270151in}{3.164123in}%
\pgfsys@useobject{currentmarker}{}%
\end{pgfscope}%
\begin{pgfscope}%
\pgfsys@transformshift{1.292919in}{3.241009in}%
\pgfsys@useobject{currentmarker}{}%
\end{pgfscope}%
\begin{pgfscope}%
\pgfsys@transformshift{1.311698in}{3.402717in}%
\pgfsys@useobject{currentmarker}{}%
\end{pgfscope}%
\begin{pgfscope}%
\pgfsys@transformshift{1.330006in}{3.497522in}%
\pgfsys@useobject{currentmarker}{}%
\end{pgfscope}%
\begin{pgfscope}%
\pgfsys@transformshift{1.349959in}{3.430815in}%
\pgfsys@useobject{currentmarker}{}%
\end{pgfscope}%
\begin{pgfscope}%
\pgfsys@transformshift{1.368738in}{3.284169in}%
\pgfsys@useobject{currentmarker}{}%
\end{pgfscope}%
\begin{pgfscope}%
\pgfsys@transformshift{1.387517in}{3.182669in}%
\pgfsys@useobject{currentmarker}{}%
\end{pgfscope}%
\begin{pgfscope}%
\pgfsys@transformshift{1.406294in}{3.150180in}%
\pgfsys@useobject{currentmarker}{}%
\end{pgfscope}%
\begin{pgfscope}%
\pgfsys@transformshift{1.425072in}{3.141940in}%
\pgfsys@useobject{currentmarker}{}%
\end{pgfscope}%
\begin{pgfscope}%
\pgfsys@transformshift{1.446903in}{3.144482in}%
\pgfsys@useobject{currentmarker}{}%
\end{pgfscope}%
\begin{pgfscope}%
\pgfsys@transformshift{1.463802in}{3.157369in}%
\pgfsys@useobject{currentmarker}{}%
\end{pgfscope}%
\begin{pgfscope}%
\pgfsys@transformshift{1.482815in}{3.196711in}%
\pgfsys@useobject{currentmarker}{}%
\end{pgfscope}%
\begin{pgfscope}%
\pgfsys@transformshift{1.500889in}{3.300855in}%
\pgfsys@useobject{currentmarker}{}%
\end{pgfscope}%
\begin{pgfscope}%
\pgfsys@transformshift{1.523424in}{3.492503in}%
\pgfsys@useobject{currentmarker}{}%
\end{pgfscope}%
\begin{pgfscope}%
\pgfsys@transformshift{1.538681in}{3.476091in}%
\pgfsys@useobject{currentmarker}{}%
\end{pgfscope}%
\begin{pgfscope}%
\pgfsys@transformshift{1.564032in}{3.321453in}%
\pgfsys@useobject{currentmarker}{}%
\end{pgfscope}%
\begin{pgfscope}%
\pgfsys@transformshift{1.580933in}{3.212753in}%
\pgfsys@useobject{currentmarker}{}%
\end{pgfscope}%
\begin{pgfscope}%
\pgfsys@transformshift{1.596659in}{3.164916in}%
\pgfsys@useobject{currentmarker}{}%
\end{pgfscope}%
\begin{pgfscope}%
\pgfsys@transformshift{1.615908in}{3.145352in}%
\pgfsys@useobject{currentmarker}{}%
\end{pgfscope}%
\begin{pgfscope}%
\pgfsys@transformshift{1.639145in}{3.140289in}%
\pgfsys@useobject{currentmarker}{}%
\end{pgfscope}%
\begin{pgfscope}%
\pgfsys@transformshift{1.654638in}{3.144584in}%
\pgfsys@useobject{currentmarker}{}%
\end{pgfscope}%
\begin{pgfscope}%
\pgfsys@transformshift{1.676703in}{3.165698in}%
\pgfsys@useobject{currentmarker}{}%
\end{pgfscope}%
\begin{pgfscope}%
\pgfsys@transformshift{1.696185in}{3.214963in}%
\pgfsys@useobject{currentmarker}{}%
\end{pgfscope}%
\begin{pgfscope}%
\pgfsys@transformshift{1.713790in}{3.354844in}%
\pgfsys@useobject{currentmarker}{}%
\end{pgfscope}%
\begin{pgfscope}%
\pgfsys@transformshift{1.733740in}{3.463787in}%
\pgfsys@useobject{currentmarker}{}%
\end{pgfscope}%
\begin{pgfscope}%
\pgfsys@transformshift{1.749937in}{3.490407in}%
\pgfsys@useobject{currentmarker}{}%
\end{pgfscope}%
\begin{pgfscope}%
\pgfsys@transformshift{1.770829in}{3.400063in}%
\pgfsys@useobject{currentmarker}{}%
\end{pgfscope}%
\begin{pgfscope}%
\pgfsys@transformshift{1.792658in}{3.224413in}%
\pgfsys@useobject{currentmarker}{}%
\end{pgfscope}%
\begin{pgfscope}%
\pgfsys@transformshift{1.809559in}{3.169517in}%
\pgfsys@useobject{currentmarker}{}%
\end{pgfscope}%
\begin{pgfscope}%
\pgfsys@transformshift{1.825990in}{3.152709in}%
\pgfsys@useobject{currentmarker}{}%
\end{pgfscope}%
\begin{pgfscope}%
\pgfsys@transformshift{1.826459in}{3.143637in}%
\pgfsys@useobject{currentmarker}{}%
\end{pgfscope}%
\begin{pgfscope}%
\pgfsys@transformshift{1.848523in}{3.140353in}%
\pgfsys@useobject{currentmarker}{}%
\end{pgfscope}%
\begin{pgfscope}%
\pgfsys@transformshift{1.868476in}{3.141130in}%
\pgfsys@useobject{currentmarker}{}%
\end{pgfscope}%
\begin{pgfscope}%
\pgfsys@transformshift{1.885376in}{3.151321in}%
\pgfsys@useobject{currentmarker}{}%
\end{pgfscope}%
\begin{pgfscope}%
\pgfsys@transformshift{1.907206in}{3.163228in}%
\pgfsys@useobject{currentmarker}{}%
\end{pgfscope}%
\begin{pgfscope}%
\pgfsys@transformshift{1.926454in}{3.223162in}%
\pgfsys@useobject{currentmarker}{}%
\end{pgfscope}%
\begin{pgfscope}%
\pgfsys@transformshift{1.946407in}{3.335926in}%
\pgfsys@useobject{currentmarker}{}%
\end{pgfscope}%
\begin{pgfscope}%
\pgfsys@transformshift{1.965185in}{3.476271in}%
\pgfsys@useobject{currentmarker}{}%
\end{pgfscope}%
\begin{pgfscope}%
\pgfsys@transformshift{1.986545in}{3.460383in}%
\pgfsys@useobject{currentmarker}{}%
\end{pgfscope}%
\begin{pgfscope}%
\pgfsys@transformshift{2.001333in}{3.357557in}%
\pgfsys@useobject{currentmarker}{}%
\end{pgfscope}%
\begin{pgfscope}%
\pgfsys@transformshift{2.022694in}{3.244484in}%
\pgfsys@useobject{currentmarker}{}%
\end{pgfscope}%
\begin{pgfscope}%
\pgfsys@transformshift{2.039594in}{3.170806in}%
\pgfsys@useobject{currentmarker}{}%
\end{pgfscope}%
\begin{pgfscope}%
\pgfsys@transformshift{2.061189in}{3.145518in}%
\pgfsys@useobject{currentmarker}{}%
\end{pgfscope}%
\begin{pgfscope}%
\pgfsys@transformshift{2.077151in}{3.140305in}%
\pgfsys@useobject{currentmarker}{}%
\end{pgfscope}%
\begin{pgfscope}%
\pgfsys@transformshift{2.098511in}{3.142144in}%
\pgfsys@useobject{currentmarker}{}%
\end{pgfscope}%
\begin{pgfscope}%
\pgfsys@transformshift{2.114707in}{3.143546in}%
\pgfsys@useobject{currentmarker}{}%
\end{pgfscope}%
\begin{pgfscope}%
\pgfsys@transformshift{2.139120in}{3.161693in}%
\pgfsys@useobject{currentmarker}{}%
\end{pgfscope}%
\begin{pgfscope}%
\pgfsys@transformshift{2.154142in}{3.198712in}%
\pgfsys@useobject{currentmarker}{}%
\end{pgfscope}%
\begin{pgfscope}%
\pgfsys@transformshift{2.174798in}{3.324871in}%
\pgfsys@useobject{currentmarker}{}%
\end{pgfscope}%
\begin{pgfscope}%
\pgfsys@transformshift{2.192872in}{3.458713in}%
\pgfsys@useobject{currentmarker}{}%
\end{pgfscope}%
\begin{pgfscope}%
\pgfsys@transformshift{2.213762in}{3.482937in}%
\pgfsys@useobject{currentmarker}{}%
\end{pgfscope}%
\begin{pgfscope}%
\pgfsys@transformshift{2.232072in}{3.439718in}%
\pgfsys@useobject{currentmarker}{}%
\end{pgfscope}%
\begin{pgfscope}%
\pgfsys@transformshift{2.254371in}{3.288269in}%
\pgfsys@useobject{currentmarker}{}%
\end{pgfscope}%
\begin{pgfscope}%
\pgfsys@transformshift{2.271976in}{3.198817in}%
\pgfsys@useobject{currentmarker}{}%
\end{pgfscope}%
\begin{pgfscope}%
\pgfsys@transformshift{2.291693in}{3.161802in}%
\pgfsys@useobject{currentmarker}{}%
\end{pgfscope}%
\begin{pgfscope}%
\pgfsys@transformshift{2.311175in}{3.145267in}%
\pgfsys@useobject{currentmarker}{}%
\end{pgfscope}%
\begin{pgfscope}%
\pgfsys@transformshift{2.329719in}{3.139532in}%
\pgfsys@useobject{currentmarker}{}%
\end{pgfscope}%
\begin{pgfscope}%
\pgfsys@transformshift{2.348264in}{3.142305in}%
\pgfsys@useobject{currentmarker}{}%
\end{pgfscope}%
\begin{pgfscope}%
\pgfsys@transformshift{2.366806in}{3.146926in}%
\pgfsys@useobject{currentmarker}{}%
\end{pgfscope}%
\begin{pgfscope}%
\pgfsys@transformshift{2.384411in}{3.165829in}%
\pgfsys@useobject{currentmarker}{}%
\end{pgfscope}%
\begin{pgfscope}%
\pgfsys@transformshift{2.406007in}{3.221564in}%
\pgfsys@useobject{currentmarker}{}%
\end{pgfscope}%
\begin{pgfscope}%
\pgfsys@transformshift{2.422672in}{3.155509in}%
\pgfsys@useobject{currentmarker}{}%
\end{pgfscope}%
\begin{pgfscope}%
\pgfsys@transformshift{2.443328in}{3.188967in}%
\pgfsys@useobject{currentmarker}{}%
\end{pgfscope}%
\begin{pgfscope}%
\pgfsys@transformshift{2.462341in}{3.278010in}%
\pgfsys@useobject{currentmarker}{}%
\end{pgfscope}%
\begin{pgfscope}%
\pgfsys@transformshift{2.480415in}{3.376881in}%
\pgfsys@useobject{currentmarker}{}%
\end{pgfscope}%
\begin{pgfscope}%
\pgfsys@transformshift{2.500368in}{3.484225in}%
\pgfsys@useobject{currentmarker}{}%
\end{pgfscope}%
\begin{pgfscope}%
\pgfsys@transformshift{2.521727in}{3.429345in}%
\pgfsys@useobject{currentmarker}{}%
\end{pgfscope}%
\begin{pgfscope}%
\pgfsys@transformshift{2.539566in}{3.276946in}%
\pgfsys@useobject{currentmarker}{}%
\end{pgfscope}%
\begin{pgfscope}%
\pgfsys@transformshift{2.557408in}{3.202741in}%
\pgfsys@useobject{currentmarker}{}%
\end{pgfscope}%
\begin{pgfscope}%
\pgfsys@transformshift{2.578767in}{3.154139in}%
\pgfsys@useobject{currentmarker}{}%
\end{pgfscope}%
\begin{pgfscope}%
\pgfsys@transformshift{2.596841in}{3.146755in}%
\pgfsys@useobject{currentmarker}{}%
\end{pgfscope}%
\begin{pgfscope}%
\pgfsys@transformshift{2.618202in}{3.139748in}%
\pgfsys@useobject{currentmarker}{}%
\end{pgfscope}%
\begin{pgfscope}%
\pgfsys@transformshift{2.634867in}{3.142749in}%
\pgfsys@useobject{currentmarker}{}%
\end{pgfscope}%
\begin{pgfscope}%
\pgfsys@transformshift{2.655523in}{3.153499in}%
\pgfsys@useobject{currentmarker}{}%
\end{pgfscope}%
\begin{pgfscope}%
\pgfsys@transformshift{2.674771in}{3.192247in}%
\pgfsys@useobject{currentmarker}{}%
\end{pgfscope}%
\begin{pgfscope}%
\pgfsys@transformshift{2.692610in}{3.281578in}%
\pgfsys@useobject{currentmarker}{}%
\end{pgfscope}%
\begin{pgfscope}%
\pgfsys@transformshift{2.712797in}{3.418697in}%
\pgfsys@useobject{currentmarker}{}%
\end{pgfscope}%
\begin{pgfscope}%
\pgfsys@transformshift{2.730871in}{3.485817in}%
\pgfsys@useobject{currentmarker}{}%
\end{pgfscope}%
\begin{pgfscope}%
\pgfsys@transformshift{2.748947in}{3.439223in}%
\pgfsys@useobject{currentmarker}{}%
\end{pgfscope}%
\begin{pgfscope}%
\pgfsys@transformshift{2.770306in}{3.291125in}%
\pgfsys@useobject{currentmarker}{}%
\end{pgfscope}%
\begin{pgfscope}%
\pgfsys@transformshift{2.788850in}{3.208172in}%
\pgfsys@useobject{currentmarker}{}%
\end{pgfscope}%
\begin{pgfscope}%
\pgfsys@transformshift{2.809975in}{3.158020in}%
\pgfsys@useobject{currentmarker}{}%
\end{pgfscope}%
\begin{pgfscope}%
\pgfsys@transformshift{2.827580in}{3.145282in}%
\pgfsys@useobject{currentmarker}{}%
\end{pgfscope}%
\begin{pgfscope}%
\pgfsys@transformshift{2.845420in}{3.140463in}%
\pgfsys@useobject{currentmarker}{}%
\end{pgfscope}%
\begin{pgfscope}%
\pgfsys@transformshift{2.867250in}{3.142216in}%
\pgfsys@useobject{currentmarker}{}%
\end{pgfscope}%
\begin{pgfscope}%
\pgfsys@transformshift{2.884386in}{3.153215in}%
\pgfsys@useobject{currentmarker}{}%
\end{pgfscope}%
\begin{pgfscope}%
\pgfsys@transformshift{2.903397in}{3.178614in}%
\pgfsys@useobject{currentmarker}{}%
\end{pgfscope}%
\begin{pgfscope}%
\pgfsys@transformshift{2.924053in}{3.256927in}%
\pgfsys@useobject{currentmarker}{}%
\end{pgfscope}%
\begin{pgfscope}%
\pgfsys@transformshift{2.942129in}{3.404050in}%
\pgfsys@useobject{currentmarker}{}%
\end{pgfscope}%
\begin{pgfscope}%
\pgfsys@transformshift{2.961142in}{3.480689in}%
\pgfsys@useobject{currentmarker}{}%
\end{pgfscope}%
\begin{pgfscope}%
\pgfsys@transformshift{2.980390in}{3.474066in}%
\pgfsys@useobject{currentmarker}{}%
\end{pgfscope}%
\begin{pgfscope}%
\pgfsys@transformshift{3.001046in}{3.342208in}%
\pgfsys@useobject{currentmarker}{}%
\end{pgfscope}%
\begin{pgfscope}%
\pgfsys@transformshift{3.018885in}{3.236304in}%
\pgfsys@useobject{currentmarker}{}%
\end{pgfscope}%
\begin{pgfscope}%
\pgfsys@transformshift{3.038367in}{3.175195in}%
\pgfsys@useobject{currentmarker}{}%
\end{pgfscope}%
\begin{pgfscope}%
\pgfsys@transformshift{3.058084in}{3.149860in}%
\pgfsys@useobject{currentmarker}{}%
\end{pgfscope}%
\begin{pgfscope}%
\pgfsys@transformshift{3.076628in}{3.143097in}%
\pgfsys@useobject{currentmarker}{}%
\end{pgfscope}%
\begin{pgfscope}%
\pgfsys@transformshift{3.094702in}{3.140823in}%
\pgfsys@useobject{currentmarker}{}%
\end{pgfscope}%
\begin{pgfscope}%
\pgfsys@transformshift{3.116766in}{3.146407in}%
\pgfsys@useobject{currentmarker}{}%
\end{pgfscope}%
\begin{pgfscope}%
\pgfsys@transformshift{3.134606in}{3.155362in}%
\pgfsys@useobject{currentmarker}{}%
\end{pgfscope}%
\begin{pgfscope}%
\pgfsys@transformshift{3.153150in}{3.182344in}%
\pgfsys@useobject{currentmarker}{}%
\end{pgfscope}%
\begin{pgfscope}%
\pgfsys@transformshift{3.173806in}{3.260469in}%
\pgfsys@useobject{currentmarker}{}%
\end{pgfscope}%
\begin{pgfscope}%
\pgfsys@transformshift{3.194228in}{3.407957in}%
\pgfsys@useobject{currentmarker}{}%
\end{pgfscope}%
\begin{pgfscope}%
\pgfsys@transformshift{3.212067in}{3.478552in}%
\pgfsys@useobject{currentmarker}{}%
\end{pgfscope}%
\begin{pgfscope}%
\pgfsys@transformshift{3.232018in}{3.496614in}%
\pgfsys@useobject{currentmarker}{}%
\end{pgfscope}%
\begin{pgfscope}%
\pgfsys@transformshift{3.250094in}{3.422020in}%
\pgfsys@useobject{currentmarker}{}%
\end{pgfscope}%
\begin{pgfscope}%
\pgfsys@transformshift{3.269576in}{3.326807in}%
\pgfsys@useobject{currentmarker}{}%
\end{pgfscope}%
\begin{pgfscope}%
\pgfsys@transformshift{3.290935in}{3.214043in}%
\pgfsys@useobject{currentmarker}{}%
\end{pgfscope}%
\begin{pgfscope}%
\pgfsys@transformshift{3.309011in}{3.170517in}%
\pgfsys@useobject{currentmarker}{}%
\end{pgfscope}%
\begin{pgfscope}%
\pgfsys@transformshift{3.326615in}{3.151404in}%
\pgfsys@useobject{currentmarker}{}%
\end{pgfscope}%
\begin{pgfscope}%
\pgfsys@transformshift{3.346566in}{3.141772in}%
\pgfsys@useobject{currentmarker}{}%
\end{pgfscope}%
\begin{pgfscope}%
\pgfsys@transformshift{3.364876in}{3.141925in}%
\pgfsys@useobject{currentmarker}{}%
\end{pgfscope}%
\begin{pgfscope}%
\pgfsys@transformshift{3.384593in}{3.147971in}%
\pgfsys@useobject{currentmarker}{}%
\end{pgfscope}%
\begin{pgfscope}%
\pgfsys@transformshift{3.405483in}{3.149448in}%
\pgfsys@useobject{currentmarker}{}%
\end{pgfscope}%
\begin{pgfscope}%
\pgfsys@transformshift{3.422619in}{3.142104in}%
\pgfsys@useobject{currentmarker}{}%
\end{pgfscope}%
\begin{pgfscope}%
\pgfsys@transformshift{3.440693in}{3.143317in}%
\pgfsys@useobject{currentmarker}{}%
\end{pgfscope}%
\begin{pgfscope}%
\pgfsys@transformshift{3.460646in}{3.148703in}%
\pgfsys@useobject{currentmarker}{}%
\end{pgfscope}%
\begin{pgfscope}%
\pgfsys@transformshift{3.481771in}{3.168952in}%
\pgfsys@useobject{currentmarker}{}%
\end{pgfscope}%
\begin{pgfscope}%
\pgfsys@transformshift{3.502661in}{3.226495in}%
\pgfsys@useobject{currentmarker}{}%
\end{pgfscope}%
\begin{pgfscope}%
\pgfsys@transformshift{3.517920in}{3.322471in}%
\pgfsys@useobject{currentmarker}{}%
\end{pgfscope}%
\begin{pgfscope}%
\pgfsys@transformshift{3.537168in}{3.486041in}%
\pgfsys@useobject{currentmarker}{}%
\end{pgfscope}%
\begin{pgfscope}%
\pgfsys@transformshift{3.559701in}{3.500524in}%
\pgfsys@useobject{currentmarker}{}%
\end{pgfscope}%
\begin{pgfscope}%
\pgfsys@transformshift{3.577775in}{3.423273in}%
\pgfsys@useobject{currentmarker}{}%
\end{pgfscope}%
\begin{pgfscope}%
\pgfsys@transformshift{3.595380in}{3.288186in}%
\pgfsys@useobject{currentmarker}{}%
\end{pgfscope}%
\begin{pgfscope}%
\pgfsys@transformshift{3.616505in}{3.198559in}%
\pgfsys@useobject{currentmarker}{}%
\end{pgfscope}%
\begin{pgfscope}%
\pgfsys@transformshift{3.635754in}{3.161851in}%
\pgfsys@useobject{currentmarker}{}%
\end{pgfscope}%
\begin{pgfscope}%
\pgfsys@transformshift{3.652888in}{3.152547in}%
\pgfsys@useobject{currentmarker}{}%
\end{pgfscope}%
\begin{pgfscope}%
\pgfsys@transformshift{3.674015in}{3.142321in}%
\pgfsys@useobject{currentmarker}{}%
\end{pgfscope}%
\begin{pgfscope}%
\pgfsys@transformshift{3.691854in}{3.142736in}%
\pgfsys@useobject{currentmarker}{}%
\end{pgfscope}%
\begin{pgfscope}%
\pgfsys@transformshift{3.709928in}{3.147160in}%
\pgfsys@useobject{currentmarker}{}%
\end{pgfscope}%
\begin{pgfscope}%
\pgfsys@transformshift{3.730819in}{3.163831in}%
\pgfsys@useobject{currentmarker}{}%
\end{pgfscope}%
\begin{pgfscope}%
\pgfsys@transformshift{3.752414in}{3.202813in}%
\pgfsys@useobject{currentmarker}{}%
\end{pgfscope}%
\begin{pgfscope}%
\pgfsys@transformshift{3.768140in}{3.167239in}%
\pgfsys@useobject{currentmarker}{}%
\end{pgfscope}%
\begin{pgfscope}%
\pgfsys@transformshift{3.789267in}{3.227449in}%
\pgfsys@useobject{currentmarker}{}%
\end{pgfscope}%
\begin{pgfscope}%
\pgfsys@transformshift{3.808280in}{3.375550in}%
\pgfsys@useobject{currentmarker}{}%
\end{pgfscope}%
\begin{pgfscope}%
\pgfsys@transformshift{3.826354in}{3.510828in}%
\pgfsys@useobject{currentmarker}{}%
\end{pgfscope}%
\begin{pgfscope}%
\pgfsys@transformshift{3.844427in}{3.515845in}%
\pgfsys@useobject{currentmarker}{}%
\end{pgfscope}%
\begin{pgfscope}%
\pgfsys@transformshift{3.866492in}{3.407651in}%
\pgfsys@useobject{currentmarker}{}%
\end{pgfscope}%
\begin{pgfscope}%
\pgfsys@transformshift{3.883862in}{3.284140in}%
\pgfsys@useobject{currentmarker}{}%
\end{pgfscope}%
\begin{pgfscope}%
\pgfsys@transformshift{3.904519in}{3.189580in}%
\pgfsys@useobject{currentmarker}{}%
\end{pgfscope}%
\begin{pgfscope}%
\pgfsys@transformshift{3.923532in}{3.160413in}%
\pgfsys@useobject{currentmarker}{}%
\end{pgfscope}%
\begin{pgfscope}%
\pgfsys@transformshift{3.941605in}{3.147105in}%
\pgfsys@useobject{currentmarker}{}%
\end{pgfscope}%
\begin{pgfscope}%
\pgfsys@transformshift{3.962496in}{3.142378in}%
\pgfsys@useobject{currentmarker}{}%
\end{pgfscope}%
\begin{pgfscope}%
\pgfsys@transformshift{3.980572in}{3.146221in}%
\pgfsys@useobject{currentmarker}{}%
\end{pgfscope}%
\begin{pgfscope}%
\pgfsys@transformshift{3.997706in}{3.154541in}%
\pgfsys@useobject{currentmarker}{}%
\end{pgfscope}%
\begin{pgfscope}%
\pgfsys@transformshift{4.019301in}{3.180682in}%
\pgfsys@useobject{currentmarker}{}%
\end{pgfscope}%
\begin{pgfscope}%
\pgfsys@transformshift{4.037844in}{3.234124in}%
\pgfsys@useobject{currentmarker}{}%
\end{pgfscope}%
\begin{pgfscope}%
\pgfsys@transformshift{4.060848in}{3.399044in}%
\pgfsys@useobject{currentmarker}{}%
\end{pgfscope}%
\begin{pgfscope}%
\pgfsys@transformshift{4.077513in}{3.495154in}%
\pgfsys@useobject{currentmarker}{}%
\end{pgfscope}%
\begin{pgfscope}%
\pgfsys@transformshift{4.097935in}{3.544921in}%
\pgfsys@useobject{currentmarker}{}%
\end{pgfscope}%
\begin{pgfscope}%
\pgfsys@transformshift{4.112723in}{3.501930in}%
\pgfsys@useobject{currentmarker}{}%
\end{pgfscope}%
\begin{pgfscope}%
\pgfsys@transformshift{4.133850in}{3.393537in}%
\pgfsys@useobject{currentmarker}{}%
\end{pgfscope}%
\begin{pgfscope}%
\pgfsys@transformshift{4.155914in}{3.261363in}%
\pgfsys@useobject{currentmarker}{}%
\end{pgfscope}%
\begin{pgfscope}%
\pgfsys@transformshift{4.173048in}{3.197774in}%
\pgfsys@useobject{currentmarker}{}%
\end{pgfscope}%
\begin{pgfscope}%
\pgfsys@transformshift{4.193939in}{3.162896in}%
\pgfsys@useobject{currentmarker}{}%
\end{pgfscope}%
\begin{pgfscope}%
\pgfsys@transformshift{4.212249in}{3.147535in}%
\pgfsys@useobject{currentmarker}{}%
\end{pgfscope}%
\begin{pgfscope}%
\pgfsys@transformshift{4.229854in}{3.143012in}%
\pgfsys@useobject{currentmarker}{}%
\end{pgfscope}%
\begin{pgfscope}%
\pgfsys@transformshift{4.250979in}{3.148973in}%
\pgfsys@useobject{currentmarker}{}%
\end{pgfscope}%
\begin{pgfscope}%
\pgfsys@transformshift{4.270697in}{3.161386in}%
\pgfsys@useobject{currentmarker}{}%
\end{pgfscope}%
\begin{pgfscope}%
\pgfsys@transformshift{4.287362in}{3.189897in}%
\pgfsys@useobject{currentmarker}{}%
\end{pgfscope}%
\begin{pgfscope}%
\pgfsys@transformshift{4.308487in}{3.245993in}%
\pgfsys@useobject{currentmarker}{}%
\end{pgfscope}%
\begin{pgfscope}%
\pgfsys@transformshift{4.326092in}{3.357425in}%
\pgfsys@useobject{currentmarker}{}%
\end{pgfscope}%
\begin{pgfscope}%
\pgfsys@transformshift{4.344637in}{3.491699in}%
\pgfsys@useobject{currentmarker}{}%
\end{pgfscope}%
\begin{pgfscope}%
\pgfsys@transformshift{4.365527in}{3.563361in}%
\pgfsys@useobject{currentmarker}{}%
\end{pgfscope}%
\begin{pgfscope}%
\pgfsys@transformshift{4.384775in}{3.516129in}%
\pgfsys@useobject{currentmarker}{}%
\end{pgfscope}%
\begin{pgfscope}%
\pgfsys@transformshift{4.401674in}{3.416170in}%
\pgfsys@useobject{currentmarker}{}%
\end{pgfscope}%
\begin{pgfscope}%
\pgfsys@transformshift{4.422331in}{3.279664in}%
\pgfsys@useobject{currentmarker}{}%
\end{pgfscope}%
\begin{pgfscope}%
\pgfsys@transformshift{4.441815in}{3.206355in}%
\pgfsys@useobject{currentmarker}{}%
\end{pgfscope}%
\begin{pgfscope}%
\pgfsys@transformshift{4.463879in}{3.164360in}%
\pgfsys@useobject{currentmarker}{}%
\end{pgfscope}%
\begin{pgfscope}%
\pgfsys@transformshift{4.480076in}{3.150582in}%
\pgfsys@useobject{currentmarker}{}%
\end{pgfscope}%
\begin{pgfscope}%
\pgfsys@transformshift{4.479841in}{3.150869in}%
\pgfsys@useobject{currentmarker}{}%
\end{pgfscope}%
\begin{pgfscope}%
\pgfsys@transformshift{4.473268in}{3.157134in}%
\pgfsys@useobject{currentmarker}{}%
\end{pgfscope}%
\begin{pgfscope}%
\pgfsys@transformshift{4.454960in}{3.222958in}%
\pgfsys@useobject{currentmarker}{}%
\end{pgfscope}%
\begin{pgfscope}%
\pgfsys@transformshift{4.433364in}{3.406541in}%
\pgfsys@useobject{currentmarker}{}%
\end{pgfscope}%
\begin{pgfscope}%
\pgfsys@transformshift{4.417636in}{3.529728in}%
\pgfsys@useobject{currentmarker}{}%
\end{pgfscope}%
\begin{pgfscope}%
\pgfsys@transformshift{4.396746in}{3.530766in}%
\pgfsys@useobject{currentmarker}{}%
\end{pgfscope}%
\begin{pgfscope}%
\pgfsys@transformshift{4.379846in}{3.346099in}%
\pgfsys@useobject{currentmarker}{}%
\end{pgfscope}%
\begin{pgfscope}%
\pgfsys@transformshift{4.360598in}{3.166152in}%
\pgfsys@useobject{currentmarker}{}%
\end{pgfscope}%
\begin{pgfscope}%
\pgfsys@transformshift{4.338298in}{3.143610in}%
\pgfsys@useobject{currentmarker}{}%
\end{pgfscope}%
\begin{pgfscope}%
\pgfsys@transformshift{4.319989in}{3.145573in}%
\pgfsys@useobject{currentmarker}{}%
\end{pgfscope}%
\begin{pgfscope}%
\pgfsys@transformshift{4.299333in}{3.171112in}%
\pgfsys@useobject{currentmarker}{}%
\end{pgfscope}%
\begin{pgfscope}%
\pgfsys@transformshift{4.283606in}{3.222945in}%
\pgfsys@useobject{currentmarker}{}%
\end{pgfscope}%
\begin{pgfscope}%
\pgfsys@transformshift{4.262715in}{3.408414in}%
\pgfsys@useobject{currentmarker}{}%
\end{pgfscope}%
\begin{pgfscope}%
\pgfsys@transformshift{4.245110in}{3.534819in}%
\pgfsys@useobject{currentmarker}{}%
\end{pgfscope}%
\begin{pgfscope}%
\pgfsys@transformshift{4.223751in}{3.484532in}%
\pgfsys@useobject{currentmarker}{}%
\end{pgfscope}%
\begin{pgfscope}%
\pgfsys@transformshift{4.206146in}{3.289938in}%
\pgfsys@useobject{currentmarker}{}%
\end{pgfscope}%
\begin{pgfscope}%
\pgfsys@transformshift{4.186193in}{3.184681in}%
\pgfsys@useobject{currentmarker}{}%
\end{pgfscope}%
\begin{pgfscope}%
\pgfsys@transformshift{4.167885in}{3.151794in}%
\pgfsys@useobject{currentmarker}{}%
\end{pgfscope}%
\begin{pgfscope}%
\pgfsys@transformshift{4.147229in}{3.142020in}%
\pgfsys@useobject{currentmarker}{}%
\end{pgfscope}%
\begin{pgfscope}%
\pgfsys@transformshift{4.129154in}{3.147734in}%
\pgfsys@useobject{currentmarker}{}%
\end{pgfscope}%
\begin{pgfscope}%
\pgfsys@transformshift{4.108263in}{3.184102in}%
\pgfsys@useobject{currentmarker}{}%
\end{pgfscope}%
\begin{pgfscope}%
\pgfsys@transformshift{4.085495in}{3.317459in}%
\pgfsys@useobject{currentmarker}{}%
\end{pgfscope}%
\begin{pgfscope}%
\pgfsys@transformshift{4.071881in}{3.462730in}%
\pgfsys@useobject{currentmarker}{}%
\end{pgfscope}%
\begin{pgfscope}%
\pgfsys@transformshift{4.050989in}{3.529983in}%
\pgfsys@useobject{currentmarker}{}%
\end{pgfscope}%
\begin{pgfscope}%
\pgfsys@transformshift{4.033620in}{3.391096in}%
\pgfsys@useobject{currentmarker}{}%
\end{pgfscope}%
\begin{pgfscope}%
\pgfsys@transformshift{4.013902in}{3.214856in}%
\pgfsys@useobject{currentmarker}{}%
\end{pgfscope}%
\begin{pgfscope}%
\pgfsys@transformshift{3.993246in}{3.157918in}%
\pgfsys@useobject{currentmarker}{}%
\end{pgfscope}%
\begin{pgfscope}%
\pgfsys@transformshift{3.975407in}{3.144281in}%
\pgfsys@useobject{currentmarker}{}%
\end{pgfscope}%
\begin{pgfscope}%
\pgfsys@transformshift{3.956159in}{3.142345in}%
\pgfsys@useobject{currentmarker}{}%
\end{pgfscope}%
\begin{pgfscope}%
\pgfsys@transformshift{3.938320in}{3.155930in}%
\pgfsys@useobject{currentmarker}{}%
\end{pgfscope}%
\begin{pgfscope}%
\pgfsys@transformshift{3.917195in}{3.211217in}%
\pgfsys@useobject{currentmarker}{}%
\end{pgfscope}%
\begin{pgfscope}%
\pgfsys@transformshift{3.896773in}{3.373563in}%
\pgfsys@useobject{currentmarker}{}%
\end{pgfscope}%
\begin{pgfscope}%
\pgfsys@transformshift{3.878934in}{3.502319in}%
\pgfsys@useobject{currentmarker}{}%
\end{pgfscope}%
\begin{pgfscope}%
\pgfsys@transformshift{3.858041in}{3.470358in}%
\pgfsys@useobject{currentmarker}{}%
\end{pgfscope}%
\begin{pgfscope}%
\pgfsys@transformshift{3.840907in}{3.310677in}%
\pgfsys@useobject{currentmarker}{}%
\end{pgfscope}%
\begin{pgfscope}%
\pgfsys@transformshift{3.822597in}{3.187632in}%
\pgfsys@useobject{currentmarker}{}%
\end{pgfscope}%
\begin{pgfscope}%
\pgfsys@transformshift{3.802881in}{3.152509in}%
\pgfsys@useobject{currentmarker}{}%
\end{pgfscope}%
\begin{pgfscope}%
\pgfsys@transformshift{3.782459in}{3.140739in}%
\pgfsys@useobject{currentmarker}{}%
\end{pgfscope}%
\begin{pgfscope}%
\pgfsys@transformshift{3.763680in}{3.143394in}%
\pgfsys@useobject{currentmarker}{}%
\end{pgfscope}%
\begin{pgfscope}%
\pgfsys@transformshift{3.743024in}{3.163319in}%
\pgfsys@useobject{currentmarker}{}%
\end{pgfscope}%
\begin{pgfscope}%
\pgfsys@transformshift{3.725185in}{3.217343in}%
\pgfsys@useobject{currentmarker}{}%
\end{pgfscope}%
\begin{pgfscope}%
\pgfsys@transformshift{3.703825in}{3.406288in}%
\pgfsys@useobject{currentmarker}{}%
\end{pgfscope}%
\begin{pgfscope}%
\pgfsys@transformshift{3.685986in}{3.502899in}%
\pgfsys@useobject{currentmarker}{}%
\end{pgfscope}%
\begin{pgfscope}%
\pgfsys@transformshift{3.667676in}{3.464826in}%
\pgfsys@useobject{currentmarker}{}%
\end{pgfscope}%
\begin{pgfscope}%
\pgfsys@transformshift{3.650542in}{3.301089in}%
\pgfsys@useobject{currentmarker}{}%
\end{pgfscope}%
\begin{pgfscope}%
\pgfsys@transformshift{3.626598in}{3.184336in}%
\pgfsys@useobject{currentmarker}{}%
\end{pgfscope}%
\begin{pgfscope}%
\pgfsys@transformshift{3.609230in}{3.154128in}%
\pgfsys@useobject{currentmarker}{}%
\end{pgfscope}%
\begin{pgfscope}%
\pgfsys@transformshift{3.592328in}{3.142722in}%
\pgfsys@useobject{currentmarker}{}%
\end{pgfscope}%
\begin{pgfscope}%
\pgfsys@transformshift{3.572377in}{3.140365in}%
\pgfsys@useobject{currentmarker}{}%
\end{pgfscope}%
\begin{pgfscope}%
\pgfsys@transformshift{3.553599in}{3.150284in}%
\pgfsys@useobject{currentmarker}{}%
\end{pgfscope}%
\begin{pgfscope}%
\pgfsys@transformshift{3.532708in}{3.169457in}%
\pgfsys@useobject{currentmarker}{}%
\end{pgfscope}%
\begin{pgfscope}%
\pgfsys@transformshift{3.514398in}{3.171442in}%
\pgfsys@useobject{currentmarker}{}%
\end{pgfscope}%
\begin{pgfscope}%
\pgfsys@transformshift{3.493742in}{3.264894in}%
\pgfsys@useobject{currentmarker}{}%
\end{pgfscope}%
\begin{pgfscope}%
\pgfsys@transformshift{3.473791in}{3.432252in}%
\pgfsys@useobject{currentmarker}{}%
\end{pgfscope}%
\begin{pgfscope}%
\pgfsys@transformshift{3.455481in}{3.499565in}%
\pgfsys@useobject{currentmarker}{}%
\end{pgfscope}%
\begin{pgfscope}%
\pgfsys@transformshift{3.438347in}{3.468164in}%
\pgfsys@useobject{currentmarker}{}%
\end{pgfscope}%
\begin{pgfscope}%
\pgfsys@transformshift{3.415812in}{3.274732in}%
\pgfsys@useobject{currentmarker}{}%
\end{pgfscope}%
\begin{pgfscope}%
\pgfsys@transformshift{3.397503in}{3.182766in}%
\pgfsys@useobject{currentmarker}{}%
\end{pgfscope}%
\begin{pgfscope}%
\pgfsys@transformshift{3.379899in}{3.157092in}%
\pgfsys@useobject{currentmarker}{}%
\end{pgfscope}%
\begin{pgfscope}%
\pgfsys@transformshift{3.361589in}{3.143426in}%
\pgfsys@useobject{currentmarker}{}%
\end{pgfscope}%
\begin{pgfscope}%
\pgfsys@transformshift{3.339760in}{3.140063in}%
\pgfsys@useobject{currentmarker}{}%
\end{pgfscope}%
\begin{pgfscope}%
\pgfsys@transformshift{3.321685in}{3.146643in}%
\pgfsys@useobject{currentmarker}{}%
\end{pgfscope}%
\begin{pgfscope}%
\pgfsys@transformshift{3.300794in}{3.171495in}%
\pgfsys@useobject{currentmarker}{}%
\end{pgfscope}%
\begin{pgfscope}%
\pgfsys@transformshift{3.283189in}{3.208943in}%
\pgfsys@useobject{currentmarker}{}%
\end{pgfscope}%
\begin{pgfscope}%
\pgfsys@transformshift{3.262533in}{3.355992in}%
\pgfsys@useobject{currentmarker}{}%
\end{pgfscope}%
\begin{pgfscope}%
\pgfsys@transformshift{3.245399in}{3.461169in}%
\pgfsys@useobject{currentmarker}{}%
\end{pgfscope}%
\begin{pgfscope}%
\pgfsys@transformshift{3.226855in}{3.493042in}%
\pgfsys@useobject{currentmarker}{}%
\end{pgfscope}%
\begin{pgfscope}%
\pgfsys@transformshift{3.205495in}{3.361809in}%
\pgfsys@useobject{currentmarker}{}%
\end{pgfscope}%
\begin{pgfscope}%
\pgfsys@transformshift{3.188360in}{3.238602in}%
\pgfsys@useobject{currentmarker}{}%
\end{pgfscope}%
\begin{pgfscope}%
\pgfsys@transformshift{3.167703in}{3.164069in}%
\pgfsys@useobject{currentmarker}{}%
\end{pgfscope}%
\begin{pgfscope}%
\pgfsys@transformshift{3.146813in}{3.144769in}%
\pgfsys@useobject{currentmarker}{}%
\end{pgfscope}%
\begin{pgfscope}%
\pgfsys@transformshift{3.128974in}{3.142870in}%
\pgfsys@useobject{currentmarker}{}%
\end{pgfscope}%
\begin{pgfscope}%
\pgfsys@transformshift{3.110429in}{3.139645in}%
\pgfsys@useobject{currentmarker}{}%
\end{pgfscope}%
\begin{pgfscope}%
\pgfsys@transformshift{3.092590in}{3.144664in}%
\pgfsys@useobject{currentmarker}{}%
\end{pgfscope}%
\begin{pgfscope}%
\pgfsys@transformshift{3.071699in}{3.164346in}%
\pgfsys@useobject{currentmarker}{}%
\end{pgfscope}%
\begin{pgfscope}%
\pgfsys@transformshift{3.053624in}{3.222673in}%
\pgfsys@useobject{currentmarker}{}%
\end{pgfscope}%
\begin{pgfscope}%
\pgfsys@transformshift{3.031325in}{3.386563in}%
\pgfsys@useobject{currentmarker}{}%
\end{pgfscope}%
\begin{pgfscope}%
\pgfsys@transformshift{3.012077in}{3.480944in}%
\pgfsys@useobject{currentmarker}{}%
\end{pgfscope}%
\begin{pgfscope}%
\pgfsys@transformshift{2.994238in}{3.363768in}%
\pgfsys@useobject{currentmarker}{}%
\end{pgfscope}%
\begin{pgfscope}%
\pgfsys@transformshift{2.976164in}{3.484199in}%
\pgfsys@useobject{currentmarker}{}%
\end{pgfscope}%
\begin{pgfscope}%
\pgfsys@transformshift{2.958325in}{3.465807in}%
\pgfsys@useobject{currentmarker}{}%
\end{pgfscope}%
\begin{pgfscope}%
\pgfsys@transformshift{2.938607in}{3.302244in}%
\pgfsys@useobject{currentmarker}{}%
\end{pgfscope}%
\begin{pgfscope}%
\pgfsys@transformshift{2.913491in}{3.184015in}%
\pgfsys@useobject{currentmarker}{}%
\end{pgfscope}%
\begin{pgfscope}%
\pgfsys@transformshift{2.901051in}{3.168702in}%
\pgfsys@useobject{currentmarker}{}%
\end{pgfscope}%
\begin{pgfscope}%
\pgfsys@transformshift{2.879689in}{3.145507in}%
\pgfsys@useobject{currentmarker}{}%
\end{pgfscope}%
\begin{pgfscope}%
\pgfsys@transformshift{2.860913in}{3.154996in}%
\pgfsys@useobject{currentmarker}{}%
\end{pgfscope}%
\begin{pgfscope}%
\pgfsys@transformshift{2.837205in}{3.140369in}%
\pgfsys@useobject{currentmarker}{}%
\end{pgfscope}%
\begin{pgfscope}%
\pgfsys@transformshift{2.823120in}{3.140547in}%
\pgfsys@useobject{currentmarker}{}%
\end{pgfscope}%
\begin{pgfscope}%
\pgfsys@transformshift{2.801759in}{3.152019in}%
\pgfsys@useobject{currentmarker}{}%
\end{pgfscope}%
\begin{pgfscope}%
\pgfsys@transformshift{2.783685in}{3.182646in}%
\pgfsys@useobject{currentmarker}{}%
\end{pgfscope}%
\begin{pgfscope}%
\pgfsys@transformshift{2.763969in}{3.278289in}%
\pgfsys@useobject{currentmarker}{}%
\end{pgfscope}%
\begin{pgfscope}%
\pgfsys@transformshift{2.745425in}{3.438686in}%
\pgfsys@useobject{currentmarker}{}%
\end{pgfscope}%
\begin{pgfscope}%
\pgfsys@transformshift{2.723126in}{3.484445in}%
\pgfsys@useobject{currentmarker}{}%
\end{pgfscope}%
\begin{pgfscope}%
\pgfsys@transformshift{2.704817in}{3.370457in}%
\pgfsys@useobject{currentmarker}{}%
\end{pgfscope}%
\begin{pgfscope}%
\pgfsys@transformshift{2.686273in}{3.252656in}%
\pgfsys@useobject{currentmarker}{}%
\end{pgfscope}%
\begin{pgfscope}%
\pgfsys@transformshift{2.669608in}{3.181445in}%
\pgfsys@useobject{currentmarker}{}%
\end{pgfscope}%
\begin{pgfscope}%
\pgfsys@transformshift{2.648481in}{3.152589in}%
\pgfsys@useobject{currentmarker}{}%
\end{pgfscope}%
\begin{pgfscope}%
\pgfsys@transformshift{2.630876in}{3.141558in}%
\pgfsys@useobject{currentmarker}{}%
\end{pgfscope}%
\begin{pgfscope}%
\pgfsys@transformshift{2.611628in}{3.140403in}%
\pgfsys@useobject{currentmarker}{}%
\end{pgfscope}%
\begin{pgfscope}%
\pgfsys@transformshift{2.590738in}{3.146284in}%
\pgfsys@useobject{currentmarker}{}%
\end{pgfscope}%
\begin{pgfscope}%
\pgfsys@transformshift{2.573133in}{3.162335in}%
\pgfsys@useobject{currentmarker}{}%
\end{pgfscope}%
\begin{pgfscope}%
\pgfsys@transformshift{2.551303in}{3.228706in}%
\pgfsys@useobject{currentmarker}{}%
\end{pgfscope}%
\begin{pgfscope}%
\pgfsys@transformshift{2.534872in}{3.361783in}%
\pgfsys@useobject{currentmarker}{}%
\end{pgfscope}%
\begin{pgfscope}%
\pgfsys@transformshift{2.516095in}{3.470342in}%
\pgfsys@useobject{currentmarker}{}%
\end{pgfscope}%
\begin{pgfscope}%
\pgfsys@transformshift{2.496846in}{3.474255in}%
\pgfsys@useobject{currentmarker}{}%
\end{pgfscope}%
\begin{pgfscope}%
\pgfsys@transformshift{2.475252in}{3.326128in}%
\pgfsys@useobject{currentmarker}{}%
\end{pgfscope}%
\begin{pgfscope}%
\pgfsys@transformshift{2.456942in}{3.215292in}%
\pgfsys@useobject{currentmarker}{}%
\end{pgfscope}%
\begin{pgfscope}%
\pgfsys@transformshift{2.438399in}{3.167234in}%
\pgfsys@useobject{currentmarker}{}%
\end{pgfscope}%
\begin{pgfscope}%
\pgfsys@transformshift{2.413283in}{3.145464in}%
\pgfsys@useobject{currentmarker}{}%
\end{pgfscope}%
\begin{pgfscope}%
\pgfsys@transformshift{2.398025in}{3.140440in}%
\pgfsys@useobject{currentmarker}{}%
\end{pgfscope}%
\begin{pgfscope}%
\pgfsys@transformshift{2.379011in}{3.142067in}%
\pgfsys@useobject{currentmarker}{}%
\end{pgfscope}%
\begin{pgfscope}%
\pgfsys@transformshift{2.360235in}{3.145885in}%
\pgfsys@useobject{currentmarker}{}%
\end{pgfscope}%
\begin{pgfscope}%
\pgfsys@transformshift{2.342395in}{3.146652in}%
\pgfsys@useobject{currentmarker}{}%
\end{pgfscope}%
\begin{pgfscope}%
\pgfsys@transformshift{2.323851in}{3.171003in}%
\pgfsys@useobject{currentmarker}{}%
\end{pgfscope}%
\begin{pgfscope}%
\pgfsys@transformshift{2.302021in}{3.261629in}%
\pgfsys@useobject{currentmarker}{}%
\end{pgfscope}%
\begin{pgfscope}%
\pgfsys@transformshift{2.283242in}{3.413301in}%
\pgfsys@useobject{currentmarker}{}%
\end{pgfscope}%
\begin{pgfscope}%
\pgfsys@transformshift{2.264465in}{3.487798in}%
\pgfsys@useobject{currentmarker}{}%
\end{pgfscope}%
\begin{pgfscope}%
\pgfsys@transformshift{2.241461in}{3.414616in}%
\pgfsys@useobject{currentmarker}{}%
\end{pgfscope}%
\begin{pgfscope}%
\pgfsys@transformshift{2.224325in}{3.289181in}%
\pgfsys@useobject{currentmarker}{}%
\end{pgfscope}%
\begin{pgfscope}%
\pgfsys@transformshift{2.205782in}{3.198806in}%
\pgfsys@useobject{currentmarker}{}%
\end{pgfscope}%
\begin{pgfscope}%
\pgfsys@transformshift{2.186535in}{3.164098in}%
\pgfsys@useobject{currentmarker}{}%
\end{pgfscope}%
\begin{pgfscope}%
\pgfsys@transformshift{2.168461in}{3.146100in}%
\pgfsys@useobject{currentmarker}{}%
\end{pgfscope}%
\begin{pgfscope}%
\pgfsys@transformshift{2.150385in}{3.140660in}%
\pgfsys@useobject{currentmarker}{}%
\end{pgfscope}%
\begin{pgfscope}%
\pgfsys@transformshift{2.128555in}{3.142736in}%
\pgfsys@useobject{currentmarker}{}%
\end{pgfscope}%
\begin{pgfscope}%
\pgfsys@transformshift{2.111421in}{3.149820in}%
\pgfsys@useobject{currentmarker}{}%
\end{pgfscope}%
\begin{pgfscope}%
\pgfsys@transformshift{2.092877in}{3.170965in}%
\pgfsys@useobject{currentmarker}{}%
\end{pgfscope}%
\begin{pgfscope}%
\pgfsys@transformshift{2.074100in}{3.214196in}%
\pgfsys@useobject{currentmarker}{}%
\end{pgfscope}%
\begin{pgfscope}%
\pgfsys@transformshift{2.051096in}{3.370699in}%
\pgfsys@useobject{currentmarker}{}%
\end{pgfscope}%
\begin{pgfscope}%
\pgfsys@transformshift{2.033022in}{3.479382in}%
\pgfsys@useobject{currentmarker}{}%
\end{pgfscope}%
\begin{pgfscope}%
\pgfsys@transformshift{2.013774in}{3.477476in}%
\pgfsys@useobject{currentmarker}{}%
\end{pgfscope}%
\begin{pgfscope}%
\pgfsys@transformshift{1.994525in}{3.348007in}%
\pgfsys@useobject{currentmarker}{}%
\end{pgfscope}%
\begin{pgfscope}%
\pgfsys@transformshift{1.976686in}{3.235554in}%
\pgfsys@useobject{currentmarker}{}%
\end{pgfscope}%
\begin{pgfscope}%
\pgfsys@transformshift{1.957909in}{3.178611in}%
\pgfsys@useobject{currentmarker}{}%
\end{pgfscope}%
\begin{pgfscope}%
\pgfsys@transformshift{1.939130in}{3.155006in}%
\pgfsys@useobject{currentmarker}{}%
\end{pgfscope}%
\begin{pgfscope}%
\pgfsys@transformshift{1.918239in}{3.144264in}%
\pgfsys@useobject{currentmarker}{}%
\end{pgfscope}%
\begin{pgfscope}%
\pgfsys@transformshift{1.899460in}{3.140994in}%
\pgfsys@useobject{currentmarker}{}%
\end{pgfscope}%
\begin{pgfscope}%
\pgfsys@transformshift{1.881387in}{3.144228in}%
\pgfsys@useobject{currentmarker}{}%
\end{pgfscope}%
\begin{pgfscope}%
\pgfsys@transformshift{1.862139in}{3.152163in}%
\pgfsys@useobject{currentmarker}{}%
\end{pgfscope}%
\begin{pgfscope}%
\pgfsys@transformshift{1.842186in}{3.185892in}%
\pgfsys@useobject{currentmarker}{}%
\end{pgfscope}%
\begin{pgfscope}%
\pgfsys@transformshift{1.823173in}{3.234067in}%
\pgfsys@useobject{currentmarker}{}%
\end{pgfscope}%
\begin{pgfscope}%
\pgfsys@transformshift{1.803691in}{3.296311in}%
\pgfsys@useobject{currentmarker}{}%
\end{pgfscope}%
\begin{pgfscope}%
\pgfsys@transformshift{1.781392in}{3.380621in}%
\pgfsys@useobject{currentmarker}{}%
\end{pgfscope}%
\begin{pgfscope}%
\pgfsys@transformshift{1.766604in}{3.476366in}%
\pgfsys@useobject{currentmarker}{}%
\end{pgfscope}%
\begin{pgfscope}%
\pgfsys@transformshift{1.747356in}{3.493920in}%
\pgfsys@useobject{currentmarker}{}%
\end{pgfscope}%
\begin{pgfscope}%
\pgfsys@transformshift{1.725526in}{3.374794in}%
\pgfsys@useobject{currentmarker}{}%
\end{pgfscope}%
\begin{pgfscope}%
\pgfsys@transformshift{1.710269in}{3.280478in}%
\pgfsys@useobject{currentmarker}{}%
\end{pgfscope}%
\begin{pgfscope}%
\pgfsys@transformshift{1.688908in}{3.191514in}%
\pgfsys@useobject{currentmarker}{}%
\end{pgfscope}%
\begin{pgfscope}%
\pgfsys@transformshift{1.669895in}{3.159815in}%
\pgfsys@useobject{currentmarker}{}%
\end{pgfscope}%
\begin{pgfscope}%
\pgfsys@transformshift{1.645248in}{3.142775in}%
\pgfsys@useobject{currentmarker}{}%
\end{pgfscope}%
\begin{pgfscope}%
\pgfsys@transformshift{1.629288in}{3.140950in}%
\pgfsys@useobject{currentmarker}{}%
\end{pgfscope}%
\begin{pgfscope}%
\pgfsys@transformshift{1.611212in}{3.144063in}%
\pgfsys@useobject{currentmarker}{}%
\end{pgfscope}%
\begin{pgfscope}%
\pgfsys@transformshift{1.593138in}{3.269038in}%
\pgfsys@useobject{currentmarker}{}%
\end{pgfscope}%
\begin{pgfscope}%
\pgfsys@transformshift{1.570839in}{3.176447in}%
\pgfsys@useobject{currentmarker}{}%
\end{pgfscope}%
\begin{pgfscope}%
\pgfsys@transformshift{1.553000in}{3.150795in}%
\pgfsys@useobject{currentmarker}{}%
\end{pgfscope}%
\begin{pgfscope}%
\pgfsys@transformshift{1.533987in}{3.142259in}%
\pgfsys@useobject{currentmarker}{}%
\end{pgfscope}%
\begin{pgfscope}%
\pgfsys@transformshift{1.512862in}{3.141475in}%
\pgfsys@useobject{currentmarker}{}%
\end{pgfscope}%
\begin{pgfscope}%
\pgfsys@transformshift{1.492206in}{3.151359in}%
\pgfsys@useobject{currentmarker}{}%
\end{pgfscope}%
\begin{pgfscope}%
\pgfsys@transformshift{1.476478in}{3.175102in}%
\pgfsys@useobject{currentmarker}{}%
\end{pgfscope}%
\begin{pgfscope}%
\pgfsys@transformshift{1.456291in}{3.235332in}%
\pgfsys@useobject{currentmarker}{}%
\end{pgfscope}%
\begin{pgfscope}%
\pgfsys@transformshift{1.438217in}{3.368284in}%
\pgfsys@useobject{currentmarker}{}%
\end{pgfscope}%
\begin{pgfscope}%
\pgfsys@transformshift{1.419673in}{3.484903in}%
\pgfsys@useobject{currentmarker}{}%
\end{pgfscope}%
\begin{pgfscope}%
\pgfsys@transformshift{1.400660in}{3.510464in}%
\pgfsys@useobject{currentmarker}{}%
\end{pgfscope}%
\begin{pgfscope}%
\pgfsys@transformshift{1.383055in}{3.417761in}%
\pgfsys@useobject{currentmarker}{}%
\end{pgfscope}%
\begin{pgfscope}%
\pgfsys@transformshift{1.361461in}{3.254282in}%
\pgfsys@useobject{currentmarker}{}%
\end{pgfscope}%
\begin{pgfscope}%
\pgfsys@transformshift{1.341979in}{3.186165in}%
\pgfsys@useobject{currentmarker}{}%
\end{pgfscope}%
\begin{pgfscope}%
\pgfsys@transformshift{1.324138in}{3.161321in}%
\pgfsys@useobject{currentmarker}{}%
\end{pgfscope}%
\begin{pgfscope}%
\pgfsys@transformshift{1.302544in}{3.145514in}%
\pgfsys@useobject{currentmarker}{}%
\end{pgfscope}%
\begin{pgfscope}%
\pgfsys@transformshift{1.284470in}{3.141551in}%
\pgfsys@useobject{currentmarker}{}%
\end{pgfscope}%
\begin{pgfscope}%
\pgfsys@transformshift{1.266160in}{3.144694in}%
\pgfsys@useobject{currentmarker}{}%
\end{pgfscope}%
\begin{pgfscope}%
\pgfsys@transformshift{1.246913in}{3.156290in}%
\pgfsys@useobject{currentmarker}{}%
\end{pgfscope}%
\begin{pgfscope}%
\pgfsys@transformshift{1.228134in}{3.183542in}%
\pgfsys@useobject{currentmarker}{}%
\end{pgfscope}%
\begin{pgfscope}%
\pgfsys@transformshift{1.206304in}{3.273920in}%
\pgfsys@useobject{currentmarker}{}%
\end{pgfscope}%
\begin{pgfscope}%
\pgfsys@transformshift{1.187761in}{3.413693in}%
\pgfsys@useobject{currentmarker}{}%
\end{pgfscope}%
\begin{pgfscope}%
\pgfsys@transformshift{1.169451in}{3.498772in}%
\pgfsys@useobject{currentmarker}{}%
\end{pgfscope}%
\begin{pgfscope}%
\pgfsys@transformshift{1.151612in}{3.514326in}%
\pgfsys@useobject{currentmarker}{}%
\end{pgfscope}%
\begin{pgfscope}%
\pgfsys@transformshift{1.129313in}{3.402907in}%
\pgfsys@useobject{currentmarker}{}%
\end{pgfscope}%
\begin{pgfscope}%
\pgfsys@transformshift{1.110300in}{3.276516in}%
\pgfsys@useobject{currentmarker}{}%
\end{pgfscope}%
\begin{pgfscope}%
\pgfsys@transformshift{1.092695in}{3.197464in}%
\pgfsys@useobject{currentmarker}{}%
\end{pgfscope}%
\begin{pgfscope}%
\pgfsys@transformshift{1.073213in}{3.164357in}%
\pgfsys@useobject{currentmarker}{}%
\end{pgfscope}%
\begin{pgfscope}%
\pgfsys@transformshift{1.054670in}{3.149596in}%
\pgfsys@useobject{currentmarker}{}%
\end{pgfscope}%
\begin{pgfscope}%
\pgfsys@transformshift{1.034249in}{3.142250in}%
\pgfsys@useobject{currentmarker}{}%
\end{pgfscope}%
\begin{pgfscope}%
\pgfsys@transformshift{1.015001in}{3.143122in}%
\pgfsys@useobject{currentmarker}{}%
\end{pgfscope}%
\begin{pgfscope}%
\pgfsys@transformshift{0.996222in}{3.150799in}%
\pgfsys@useobject{currentmarker}{}%
\end{pgfscope}%
\begin{pgfscope}%
\pgfsys@transformshift{0.978383in}{3.170869in}%
\pgfsys@useobject{currentmarker}{}%
\end{pgfscope}%
\begin{pgfscope}%
\pgfsys@transformshift{0.957021in}{3.235229in}%
\pgfsys@useobject{currentmarker}{}%
\end{pgfscope}%
\begin{pgfscope}%
\pgfsys@transformshift{0.938245in}{3.355159in}%
\pgfsys@useobject{currentmarker}{}%
\end{pgfscope}%
\begin{pgfscope}%
\pgfsys@transformshift{0.919700in}{3.437150in}%
\pgfsys@useobject{currentmarker}{}%
\end{pgfscope}%
\begin{pgfscope}%
\pgfsys@transformshift{0.900687in}{3.485248in}%
\pgfsys@useobject{currentmarker}{}%
\end{pgfscope}%
\begin{pgfscope}%
\pgfsys@transformshift{0.878857in}{3.535703in}%
\pgfsys@useobject{currentmarker}{}%
\end{pgfscope}%
\begin{pgfscope}%
\pgfsys@transformshift{0.860549in}{3.500877in}%
\pgfsys@useobject{currentmarker}{}%
\end{pgfscope}%
\begin{pgfscope}%
\pgfsys@transformshift{0.842944in}{3.325083in}%
\pgfsys@useobject{currentmarker}{}%
\end{pgfscope}%
\begin{pgfscope}%
\pgfsys@transformshift{0.823696in}{3.444794in}%
\pgfsys@useobject{currentmarker}{}%
\end{pgfscope}%
\begin{pgfscope}%
\pgfsys@transformshift{0.805386in}{3.531628in}%
\pgfsys@useobject{currentmarker}{}%
\end{pgfscope}%
\begin{pgfscope}%
\pgfsys@transformshift{0.783087in}{3.485216in}%
\pgfsys@useobject{currentmarker}{}%
\end{pgfscope}%
\begin{pgfscope}%
\pgfsys@transformshift{0.766656in}{3.340892in}%
\pgfsys@useobject{currentmarker}{}%
\end{pgfscope}%
\begin{pgfscope}%
\pgfsys@transformshift{0.745295in}{3.214625in}%
\pgfsys@useobject{currentmarker}{}%
\end{pgfscope}%
\begin{pgfscope}%
\pgfsys@transformshift{0.723467in}{3.176282in}%
\pgfsys@useobject{currentmarker}{}%
\end{pgfscope}%
\begin{pgfscope}%
\pgfsys@transformshift{0.706331in}{3.156310in}%
\pgfsys@useobject{currentmarker}{}%
\end{pgfscope}%
\begin{pgfscope}%
\pgfsys@transformshift{0.687552in}{3.143871in}%
\pgfsys@useobject{currentmarker}{}%
\end{pgfscope}%
\begin{pgfscope}%
\pgfsys@transformshift{0.669713in}{3.142909in}%
\pgfsys@useobject{currentmarker}{}%
\end{pgfscope}%
\begin{pgfscope}%
\pgfsys@transformshift{0.651639in}{3.151450in}%
\pgfsys@useobject{currentmarker}{}%
\end{pgfscope}%
\begin{pgfscope}%
\pgfsys@transformshift{0.651170in}{3.152508in}%
\pgfsys@useobject{currentmarker}{}%
\end{pgfscope}%
\begin{pgfscope}%
\pgfsys@transformshift{0.659619in}{3.145556in}%
\pgfsys@useobject{currentmarker}{}%
\end{pgfscope}%
\begin{pgfscope}%
\pgfsys@transformshift{0.674643in}{3.142609in}%
\pgfsys@useobject{currentmarker}{}%
\end{pgfscope}%
\begin{pgfscope}%
\pgfsys@transformshift{0.696942in}{3.159071in}%
\pgfsys@useobject{currentmarker}{}%
\end{pgfscope}%
\begin{pgfscope}%
\pgfsys@transformshift{0.713842in}{3.199819in}%
\pgfsys@useobject{currentmarker}{}%
\end{pgfscope}%
\begin{pgfscope}%
\pgfsys@transformshift{0.733090in}{3.311227in}%
\pgfsys@useobject{currentmarker}{}%
\end{pgfscope}%
\begin{pgfscope}%
\pgfsys@transformshift{0.754217in}{3.525823in}%
\pgfsys@useobject{currentmarker}{}%
\end{pgfscope}%
\begin{pgfscope}%
\pgfsys@transformshift{0.771821in}{3.505830in}%
\pgfsys@useobject{currentmarker}{}%
\end{pgfscope}%
\begin{pgfscope}%
\pgfsys@transformshift{0.791303in}{3.328378in}%
\pgfsys@useobject{currentmarker}{}%
\end{pgfscope}%
\begin{pgfscope}%
\pgfsys@transformshift{0.811254in}{3.186225in}%
\pgfsys@useobject{currentmarker}{}%
\end{pgfscope}%
\begin{pgfscope}%
\pgfsys@transformshift{0.829094in}{3.150822in}%
\pgfsys@useobject{currentmarker}{}%
\end{pgfscope}%
\begin{pgfscope}%
\pgfsys@transformshift{0.847873in}{3.141176in}%
\pgfsys@useobject{currentmarker}{}%
\end{pgfscope}%
\begin{pgfscope}%
\pgfsys@transformshift{0.866886in}{3.148174in}%
\pgfsys@useobject{currentmarker}{}%
\end{pgfscope}%
\begin{pgfscope}%
\pgfsys@transformshift{0.888950in}{3.184167in}%
\pgfsys@useobject{currentmarker}{}%
\end{pgfscope}%
\begin{pgfscope}%
\pgfsys@transformshift{0.908667in}{3.277770in}%
\pgfsys@useobject{currentmarker}{}%
\end{pgfscope}%
\begin{pgfscope}%
\pgfsys@transformshift{0.924160in}{3.438253in}%
\pgfsys@useobject{currentmarker}{}%
\end{pgfscope}%
\begin{pgfscope}%
\pgfsys@transformshift{0.943173in}{3.519886in}%
\pgfsys@useobject{currentmarker}{}%
\end{pgfscope}%
\begin{pgfscope}%
\pgfsys@transformshift{0.962655in}{3.400003in}%
\pgfsys@useobject{currentmarker}{}%
\end{pgfscope}%
\begin{pgfscope}%
\pgfsys@transformshift{0.981200in}{3.285864in}%
\pgfsys@useobject{currentmarker}{}%
\end{pgfscope}%
\begin{pgfscope}%
\pgfsys@transformshift{1.004436in}{3.170382in}%
\pgfsys@useobject{currentmarker}{}%
\end{pgfscope}%
\begin{pgfscope}%
\pgfsys@transformshift{1.022746in}{3.144667in}%
\pgfsys@useobject{currentmarker}{}%
\end{pgfscope}%
\begin{pgfscope}%
\pgfsys@transformshift{1.042463in}{3.141017in}%
\pgfsys@useobject{currentmarker}{}%
\end{pgfscope}%
\begin{pgfscope}%
\pgfsys@transformshift{1.060537in}{3.150652in}%
\pgfsys@useobject{currentmarker}{}%
\end{pgfscope}%
\begin{pgfscope}%
\pgfsys@transformshift{1.080255in}{3.182415in}%
\pgfsys@useobject{currentmarker}{}%
\end{pgfscope}%
\begin{pgfscope}%
\pgfsys@transformshift{1.099503in}{3.290351in}%
\pgfsys@useobject{currentmarker}{}%
\end{pgfscope}%
\begin{pgfscope}%
\pgfsys@transformshift{1.116168in}{3.462959in}%
\pgfsys@useobject{currentmarker}{}%
\end{pgfscope}%
\begin{pgfscope}%
\pgfsys@transformshift{1.136824in}{3.491155in}%
\pgfsys@useobject{currentmarker}{}%
\end{pgfscope}%
\begin{pgfscope}%
\pgfsys@transformshift{1.156543in}{3.354695in}%
\pgfsys@useobject{currentmarker}{}%
\end{pgfscope}%
\begin{pgfscope}%
\pgfsys@transformshift{1.175556in}{3.198977in}%
\pgfsys@useobject{currentmarker}{}%
\end{pgfscope}%
\begin{pgfscope}%
\pgfsys@transformshift{1.194803in}{3.154318in}%
\pgfsys@useobject{currentmarker}{}%
\end{pgfscope}%
\begin{pgfscope}%
\pgfsys@transformshift{1.213580in}{3.141647in}%
\pgfsys@useobject{currentmarker}{}%
\end{pgfscope}%
\begin{pgfscope}%
\pgfsys@transformshift{1.233533in}{3.142658in}%
\pgfsys@useobject{currentmarker}{}%
\end{pgfscope}%
\begin{pgfscope}%
\pgfsys@transformshift{1.252078in}{3.156209in}%
\pgfsys@useobject{currentmarker}{}%
\end{pgfscope}%
\begin{pgfscope}%
\pgfsys@transformshift{1.270620in}{3.200181in}%
\pgfsys@useobject{currentmarker}{}%
\end{pgfscope}%
\begin{pgfscope}%
\pgfsys@transformshift{1.290339in}{3.336583in}%
\pgfsys@useobject{currentmarker}{}%
\end{pgfscope}%
\begin{pgfscope}%
\pgfsys@transformshift{1.308881in}{3.491784in}%
\pgfsys@useobject{currentmarker}{}%
\end{pgfscope}%
\begin{pgfscope}%
\pgfsys@transformshift{1.327426in}{3.470127in}%
\pgfsys@useobject{currentmarker}{}%
\end{pgfscope}%
\begin{pgfscope}%
\pgfsys@transformshift{1.348082in}{3.351211in}%
\pgfsys@useobject{currentmarker}{}%
\end{pgfscope}%
\begin{pgfscope}%
\pgfsys@transformshift{1.369441in}{3.214937in}%
\pgfsys@useobject{currentmarker}{}%
\end{pgfscope}%
\begin{pgfscope}%
\pgfsys@transformshift{1.388689in}{3.157864in}%
\pgfsys@useobject{currentmarker}{}%
\end{pgfscope}%
\begin{pgfscope}%
\pgfsys@transformshift{1.406999in}{3.142141in}%
\pgfsys@useobject{currentmarker}{}%
\end{pgfscope}%
\begin{pgfscope}%
\pgfsys@transformshift{1.426012in}{3.140268in}%
\pgfsys@useobject{currentmarker}{}%
\end{pgfscope}%
\begin{pgfscope}%
\pgfsys@transformshift{1.445260in}{3.147807in}%
\pgfsys@useobject{currentmarker}{}%
\end{pgfscope}%
\begin{pgfscope}%
\pgfsys@transformshift{1.467793in}{3.178687in}%
\pgfsys@useobject{currentmarker}{}%
\end{pgfscope}%
\begin{pgfscope}%
\pgfsys@transformshift{1.482112in}{3.239199in}%
\pgfsys@useobject{currentmarker}{}%
\end{pgfscope}%
\begin{pgfscope}%
\pgfsys@transformshift{1.503472in}{3.415095in}%
\pgfsys@useobject{currentmarker}{}%
\end{pgfscope}%
\begin{pgfscope}%
\pgfsys@transformshift{1.520842in}{3.489545in}%
\pgfsys@useobject{currentmarker}{}%
\end{pgfscope}%
\begin{pgfscope}%
\pgfsys@transformshift{1.538447in}{3.413508in}%
\pgfsys@useobject{currentmarker}{}%
\end{pgfscope}%
\begin{pgfscope}%
\pgfsys@transformshift{1.561451in}{3.274360in}%
\pgfsys@useobject{currentmarker}{}%
\end{pgfscope}%
\begin{pgfscope}%
\pgfsys@transformshift{1.580228in}{3.187386in}%
\pgfsys@useobject{currentmarker}{}%
\end{pgfscope}%
\begin{pgfscope}%
\pgfsys@transformshift{1.599007in}{3.151154in}%
\pgfsys@useobject{currentmarker}{}%
\end{pgfscope}%
\begin{pgfscope}%
\pgfsys@transformshift{1.618254in}{3.140954in}%
\pgfsys@useobject{currentmarker}{}%
\end{pgfscope}%
\begin{pgfscope}%
\pgfsys@transformshift{1.635859in}{3.141651in}%
\pgfsys@useobject{currentmarker}{}%
\end{pgfscope}%
\begin{pgfscope}%
\pgfsys@transformshift{1.655576in}{3.150166in}%
\pgfsys@useobject{currentmarker}{}%
\end{pgfscope}%
\begin{pgfscope}%
\pgfsys@transformshift{1.674823in}{3.175680in}%
\pgfsys@useobject{currentmarker}{}%
\end{pgfscope}%
\begin{pgfscope}%
\pgfsys@transformshift{1.692899in}{3.217823in}%
\pgfsys@useobject{currentmarker}{}%
\end{pgfscope}%
\begin{pgfscope}%
\pgfsys@transformshift{1.715198in}{3.371304in}%
\pgfsys@useobject{currentmarker}{}%
\end{pgfscope}%
\begin{pgfscope}%
\pgfsys@transformshift{1.733037in}{3.478700in}%
\pgfsys@useobject{currentmarker}{}%
\end{pgfscope}%
\begin{pgfscope}%
\pgfsys@transformshift{1.751816in}{3.479369in}%
\pgfsys@useobject{currentmarker}{}%
\end{pgfscope}%
\begin{pgfscope}%
\pgfsys@transformshift{1.770593in}{3.378912in}%
\pgfsys@useobject{currentmarker}{}%
\end{pgfscope}%
\begin{pgfscope}%
\pgfsys@transformshift{1.789606in}{3.236669in}%
\pgfsys@useobject{currentmarker}{}%
\end{pgfscope}%
\begin{pgfscope}%
\pgfsys@transformshift{1.808619in}{3.169065in}%
\pgfsys@useobject{currentmarker}{}%
\end{pgfscope}%
\begin{pgfscope}%
\pgfsys@transformshift{1.828338in}{3.148824in}%
\pgfsys@useobject{currentmarker}{}%
\end{pgfscope}%
\begin{pgfscope}%
\pgfsys@transformshift{1.848760in}{3.141344in}%
\pgfsys@useobject{currentmarker}{}%
\end{pgfscope}%
\begin{pgfscope}%
\pgfsys@transformshift{1.867537in}{3.140770in}%
\pgfsys@useobject{currentmarker}{}%
\end{pgfscope}%
\begin{pgfscope}%
\pgfsys@transformshift{1.885612in}{3.145498in}%
\pgfsys@useobject{currentmarker}{}%
\end{pgfscope}%
\begin{pgfscope}%
\pgfsys@transformshift{1.904623in}{3.165835in}%
\pgfsys@useobject{currentmarker}{}%
\end{pgfscope}%
\begin{pgfscope}%
\pgfsys@transformshift{1.923402in}{3.194808in}%
\pgfsys@useobject{currentmarker}{}%
\end{pgfscope}%
\begin{pgfscope}%
\pgfsys@transformshift{1.945938in}{3.328211in}%
\pgfsys@useobject{currentmarker}{}%
\end{pgfscope}%
\begin{pgfscope}%
\pgfsys@transformshift{1.963777in}{3.459083in}%
\pgfsys@useobject{currentmarker}{}%
\end{pgfscope}%
\begin{pgfscope}%
\pgfsys@transformshift{1.982085in}{3.481666in}%
\pgfsys@useobject{currentmarker}{}%
\end{pgfscope}%
\begin{pgfscope}%
\pgfsys@transformshift{1.999455in}{3.420302in}%
\pgfsys@useobject{currentmarker}{}%
\end{pgfscope}%
\begin{pgfscope}%
\pgfsys@transformshift{2.019406in}{3.277614in}%
\pgfsys@useobject{currentmarker}{}%
\end{pgfscope}%
\begin{pgfscope}%
\pgfsys@transformshift{2.041707in}{3.184523in}%
\pgfsys@useobject{currentmarker}{}%
\end{pgfscope}%
\begin{pgfscope}%
\pgfsys@transformshift{2.060484in}{3.153087in}%
\pgfsys@useobject{currentmarker}{}%
\end{pgfscope}%
\begin{pgfscope}%
\pgfsys@transformshift{2.078560in}{3.144604in}%
\pgfsys@useobject{currentmarker}{}%
\end{pgfscope}%
\begin{pgfscope}%
\pgfsys@transformshift{2.099216in}{3.140163in}%
\pgfsys@useobject{currentmarker}{}%
\end{pgfscope}%
\begin{pgfscope}%
\pgfsys@transformshift{2.118227in}{3.142219in}%
\pgfsys@useobject{currentmarker}{}%
\end{pgfscope}%
\begin{pgfscope}%
\pgfsys@transformshift{2.135598in}{3.151988in}%
\pgfsys@useobject{currentmarker}{}%
\end{pgfscope}%
\begin{pgfscope}%
\pgfsys@transformshift{2.153673in}{3.172896in}%
\pgfsys@useobject{currentmarker}{}%
\end{pgfscope}%
\begin{pgfscope}%
\pgfsys@transformshift{2.173859in}{3.253424in}%
\pgfsys@useobject{currentmarker}{}%
\end{pgfscope}%
\begin{pgfscope}%
\pgfsys@transformshift{2.194985in}{3.377160in}%
\pgfsys@useobject{currentmarker}{}%
\end{pgfscope}%
\begin{pgfscope}%
\pgfsys@transformshift{2.212825in}{3.474566in}%
\pgfsys@useobject{currentmarker}{}%
\end{pgfscope}%
\begin{pgfscope}%
\pgfsys@transformshift{2.230898in}{3.457533in}%
\pgfsys@useobject{currentmarker}{}%
\end{pgfscope}%
\begin{pgfscope}%
\pgfsys@transformshift{2.252258in}{3.317412in}%
\pgfsys@useobject{currentmarker}{}%
\end{pgfscope}%
\begin{pgfscope}%
\pgfsys@transformshift{2.269159in}{3.198668in}%
\pgfsys@useobject{currentmarker}{}%
\end{pgfscope}%
\begin{pgfscope}%
\pgfsys@transformshift{2.288172in}{3.343418in}%
\pgfsys@useobject{currentmarker}{}%
\end{pgfscope}%
\begin{pgfscope}%
\pgfsys@transformshift{2.309766in}{3.189939in}%
\pgfsys@useobject{currentmarker}{}%
\end{pgfscope}%
\begin{pgfscope}%
\pgfsys@transformshift{2.330893in}{3.156896in}%
\pgfsys@useobject{currentmarker}{}%
\end{pgfscope}%
\begin{pgfscope}%
\pgfsys@transformshift{2.347324in}{3.144377in}%
\pgfsys@useobject{currentmarker}{}%
\end{pgfscope}%
\begin{pgfscope}%
\pgfsys@transformshift{2.365868in}{3.139767in}%
\pgfsys@useobject{currentmarker}{}%
\end{pgfscope}%
\begin{pgfscope}%
\pgfsys@transformshift{2.386759in}{3.144275in}%
\pgfsys@useobject{currentmarker}{}%
\end{pgfscope}%
\begin{pgfscope}%
\pgfsys@transformshift{2.404364in}{3.158326in}%
\pgfsys@useobject{currentmarker}{}%
\end{pgfscope}%
\begin{pgfscope}%
\pgfsys@transformshift{2.428306in}{3.223962in}%
\pgfsys@useobject{currentmarker}{}%
\end{pgfscope}%
\begin{pgfscope}%
\pgfsys@transformshift{2.444736in}{3.320963in}%
\pgfsys@useobject{currentmarker}{}%
\end{pgfscope}%
\begin{pgfscope}%
\pgfsys@transformshift{2.461872in}{3.457319in}%
\pgfsys@useobject{currentmarker}{}%
\end{pgfscope}%
\begin{pgfscope}%
\pgfsys@transformshift{2.482529in}{3.458841in}%
\pgfsys@useobject{currentmarker}{}%
\end{pgfscope}%
\begin{pgfscope}%
\pgfsys@transformshift{2.503419in}{3.363003in}%
\pgfsys@useobject{currentmarker}{}%
\end{pgfscope}%
\begin{pgfscope}%
\pgfsys@transformshift{2.518207in}{3.239868in}%
\pgfsys@useobject{currentmarker}{}%
\end{pgfscope}%
\begin{pgfscope}%
\pgfsys@transformshift{2.540037in}{3.164940in}%
\pgfsys@useobject{currentmarker}{}%
\end{pgfscope}%
\begin{pgfscope}%
\pgfsys@transformshift{2.560693in}{3.144960in}%
\pgfsys@useobject{currentmarker}{}%
\end{pgfscope}%
\begin{pgfscope}%
\pgfsys@transformshift{2.578064in}{3.140910in}%
\pgfsys@useobject{currentmarker}{}%
\end{pgfscope}%
\begin{pgfscope}%
\pgfsys@transformshift{2.598249in}{3.141620in}%
\pgfsys@useobject{currentmarker}{}%
\end{pgfscope}%
\begin{pgfscope}%
\pgfsys@transformshift{2.617497in}{3.150086in}%
\pgfsys@useobject{currentmarker}{}%
\end{pgfscope}%
\begin{pgfscope}%
\pgfsys@transformshift{2.634398in}{3.171427in}%
\pgfsys@useobject{currentmarker}{}%
\end{pgfscope}%
\begin{pgfscope}%
\pgfsys@transformshift{2.656697in}{3.215957in}%
\pgfsys@useobject{currentmarker}{}%
\end{pgfscope}%
\begin{pgfscope}%
\pgfsys@transformshift{2.674068in}{3.314460in}%
\pgfsys@useobject{currentmarker}{}%
\end{pgfscope}%
\begin{pgfscope}%
\pgfsys@transformshift{2.692141in}{3.446668in}%
\pgfsys@useobject{currentmarker}{}%
\end{pgfscope}%
\begin{pgfscope}%
\pgfsys@transformshift{2.715849in}{3.460638in}%
\pgfsys@useobject{currentmarker}{}%
\end{pgfscope}%
\begin{pgfscope}%
\pgfsys@transformshift{2.731342in}{3.362943in}%
\pgfsys@useobject{currentmarker}{}%
\end{pgfscope}%
\begin{pgfscope}%
\pgfsys@transformshift{2.750119in}{3.295268in}%
\pgfsys@useobject{currentmarker}{}%
\end{pgfscope}%
\begin{pgfscope}%
\pgfsys@transformshift{2.771011in}{3.198229in}%
\pgfsys@useobject{currentmarker}{}%
\end{pgfscope}%
\begin{pgfscope}%
\pgfsys@transformshift{2.789085in}{3.158180in}%
\pgfsys@useobject{currentmarker}{}%
\end{pgfscope}%
\begin{pgfscope}%
\pgfsys@transformshift{2.808801in}{3.144646in}%
\pgfsys@useobject{currentmarker}{}%
\end{pgfscope}%
\begin{pgfscope}%
\pgfsys@transformshift{2.826172in}{3.142696in}%
\pgfsys@useobject{currentmarker}{}%
\end{pgfscope}%
\begin{pgfscope}%
\pgfsys@transformshift{2.850819in}{3.140087in}%
\pgfsys@useobject{currentmarker}{}%
\end{pgfscope}%
\begin{pgfscope}%
\pgfsys@transformshift{2.865607in}{3.144062in}%
\pgfsys@useobject{currentmarker}{}%
\end{pgfscope}%
\begin{pgfscope}%
\pgfsys@transformshift{2.886966in}{3.163618in}%
\pgfsys@useobject{currentmarker}{}%
\end{pgfscope}%
\begin{pgfscope}%
\pgfsys@transformshift{2.904805in}{3.198699in}%
\pgfsys@useobject{currentmarker}{}%
\end{pgfscope}%
\begin{pgfscope}%
\pgfsys@transformshift{2.922176in}{3.264222in}%
\pgfsys@useobject{currentmarker}{}%
\end{pgfscope}%
\begin{pgfscope}%
\pgfsys@transformshift{2.943066in}{3.432289in}%
\pgfsys@useobject{currentmarker}{}%
\end{pgfscope}%
\begin{pgfscope}%
\pgfsys@transformshift{2.962785in}{3.460327in}%
\pgfsys@useobject{currentmarker}{}%
\end{pgfscope}%
\begin{pgfscope}%
\pgfsys@transformshift{2.984144in}{3.472236in}%
\pgfsys@useobject{currentmarker}{}%
\end{pgfscope}%
\begin{pgfscope}%
\pgfsys@transformshift{3.002454in}{3.365226in}%
\pgfsys@useobject{currentmarker}{}%
\end{pgfscope}%
\begin{pgfscope}%
\pgfsys@transformshift{3.020059in}{3.247927in}%
\pgfsys@useobject{currentmarker}{}%
\end{pgfscope}%
\begin{pgfscope}%
\pgfsys@transformshift{3.040950in}{3.168874in}%
\pgfsys@useobject{currentmarker}{}%
\end{pgfscope}%
\begin{pgfscope}%
\pgfsys@transformshift{3.059258in}{3.148598in}%
\pgfsys@useobject{currentmarker}{}%
\end{pgfscope}%
\begin{pgfscope}%
\pgfsys@transformshift{3.076159in}{3.143704in}%
\pgfsys@useobject{currentmarker}{}%
\end{pgfscope}%
\begin{pgfscope}%
\pgfsys@transformshift{3.097989in}{3.140111in}%
\pgfsys@useobject{currentmarker}{}%
\end{pgfscope}%
\begin{pgfscope}%
\pgfsys@transformshift{3.097519in}{3.142887in}%
\pgfsys@useobject{currentmarker}{}%
\end{pgfscope}%
\begin{pgfscope}%
\pgfsys@transformshift{3.115358in}{3.145583in}%
\pgfsys@useobject{currentmarker}{}%
\end{pgfscope}%
\begin{pgfscope}%
\pgfsys@transformshift{3.133902in}{3.156407in}%
\pgfsys@useobject{currentmarker}{}%
\end{pgfscope}%
\begin{pgfscope}%
\pgfsys@transformshift{3.154793in}{3.188423in}%
\pgfsys@useobject{currentmarker}{}%
\end{pgfscope}%
\begin{pgfscope}%
\pgfsys@transformshift{3.172866in}{3.261703in}%
\pgfsys@useobject{currentmarker}{}%
\end{pgfscope}%
\begin{pgfscope}%
\pgfsys@transformshift{3.194697in}{3.418942in}%
\pgfsys@useobject{currentmarker}{}%
\end{pgfscope}%
\begin{pgfscope}%
\pgfsys@transformshift{3.211833in}{3.485637in}%
\pgfsys@useobject{currentmarker}{}%
\end{pgfscope}%
\begin{pgfscope}%
\pgfsys@transformshift{3.229672in}{3.473200in}%
\pgfsys@useobject{currentmarker}{}%
\end{pgfscope}%
\begin{pgfscope}%
\pgfsys@transformshift{3.251031in}{3.379776in}%
\pgfsys@useobject{currentmarker}{}%
\end{pgfscope}%
\begin{pgfscope}%
\pgfsys@transformshift{3.271453in}{3.240385in}%
\pgfsys@useobject{currentmarker}{}%
\end{pgfscope}%
\begin{pgfscope}%
\pgfsys@transformshift{3.289058in}{3.181003in}%
\pgfsys@useobject{currentmarker}{}%
\end{pgfscope}%
\begin{pgfscope}%
\pgfsys@transformshift{3.306897in}{3.153613in}%
\pgfsys@useobject{currentmarker}{}%
\end{pgfscope}%
\begin{pgfscope}%
\pgfsys@transformshift{3.328727in}{3.143689in}%
\pgfsys@useobject{currentmarker}{}%
\end{pgfscope}%
\begin{pgfscope}%
\pgfsys@transformshift{3.346332in}{3.140600in}%
\pgfsys@useobject{currentmarker}{}%
\end{pgfscope}%
\begin{pgfscope}%
\pgfsys@transformshift{3.367222in}{3.143922in}%
\pgfsys@useobject{currentmarker}{}%
\end{pgfscope}%
\begin{pgfscope}%
\pgfsys@transformshift{3.385767in}{3.150375in}%
\pgfsys@useobject{currentmarker}{}%
\end{pgfscope}%
\begin{pgfscope}%
\pgfsys@transformshift{3.403372in}{3.165202in}%
\pgfsys@useobject{currentmarker}{}%
\end{pgfscope}%
\begin{pgfscope}%
\pgfsys@transformshift{3.422619in}{3.205884in}%
\pgfsys@useobject{currentmarker}{}%
\end{pgfscope}%
\begin{pgfscope}%
\pgfsys@transformshift{3.443979in}{3.300055in}%
\pgfsys@useobject{currentmarker}{}%
\end{pgfscope}%
\begin{pgfscope}%
\pgfsys@transformshift{3.461818in}{3.425597in}%
\pgfsys@useobject{currentmarker}{}%
\end{pgfscope}%
\begin{pgfscope}%
\pgfsys@transformshift{3.479188in}{3.500791in}%
\pgfsys@useobject{currentmarker}{}%
\end{pgfscope}%
\begin{pgfscope}%
\pgfsys@transformshift{3.501487in}{3.449126in}%
\pgfsys@useobject{currentmarker}{}%
\end{pgfscope}%
\begin{pgfscope}%
\pgfsys@transformshift{3.520501in}{3.380686in}%
\pgfsys@useobject{currentmarker}{}%
\end{pgfscope}%
\begin{pgfscope}%
\pgfsys@transformshift{3.538105in}{3.274851in}%
\pgfsys@useobject{currentmarker}{}%
\end{pgfscope}%
\begin{pgfscope}%
\pgfsys@transformshift{3.559467in}{3.180574in}%
\pgfsys@useobject{currentmarker}{}%
\end{pgfscope}%
\begin{pgfscope}%
\pgfsys@transformshift{3.577540in}{3.251653in}%
\pgfsys@useobject{currentmarker}{}%
\end{pgfscope}%
\begin{pgfscope}%
\pgfsys@transformshift{3.595614in}{3.372448in}%
\pgfsys@useobject{currentmarker}{}%
\end{pgfscope}%
\begin{pgfscope}%
\pgfsys@transformshift{3.616505in}{3.503642in}%
\pgfsys@useobject{currentmarker}{}%
\end{pgfscope}%
\begin{pgfscope}%
\pgfsys@transformshift{3.634580in}{3.500077in}%
\pgfsys@useobject{currentmarker}{}%
\end{pgfscope}%
\begin{pgfscope}%
\pgfsys@transformshift{3.651951in}{3.404899in}%
\pgfsys@useobject{currentmarker}{}%
\end{pgfscope}%
\begin{pgfscope}%
\pgfsys@transformshift{3.674718in}{3.270712in}%
\pgfsys@useobject{currentmarker}{}%
\end{pgfscope}%
\begin{pgfscope}%
\pgfsys@transformshift{3.692323in}{3.185802in}%
\pgfsys@useobject{currentmarker}{}%
\end{pgfscope}%
\begin{pgfscope}%
\pgfsys@transformshift{3.709928in}{3.157297in}%
\pgfsys@useobject{currentmarker}{}%
\end{pgfscope}%
\begin{pgfscope}%
\pgfsys@transformshift{3.730819in}{3.142859in}%
\pgfsys@useobject{currentmarker}{}%
\end{pgfscope}%
\begin{pgfscope}%
\pgfsys@transformshift{3.749363in}{3.142117in}%
\pgfsys@useobject{currentmarker}{}%
\end{pgfscope}%
\begin{pgfscope}%
\pgfsys@transformshift{3.768140in}{3.148935in}%
\pgfsys@useobject{currentmarker}{}%
\end{pgfscope}%
\begin{pgfscope}%
\pgfsys@transformshift{3.788796in}{3.176008in}%
\pgfsys@useobject{currentmarker}{}%
\end{pgfscope}%
\begin{pgfscope}%
\pgfsys@transformshift{3.808983in}{3.225924in}%
\pgfsys@useobject{currentmarker}{}%
\end{pgfscope}%
\begin{pgfscope}%
\pgfsys@transformshift{3.827057in}{3.316741in}%
\pgfsys@useobject{currentmarker}{}%
\end{pgfscope}%
\begin{pgfscope}%
\pgfsys@transformshift{3.844662in}{3.477211in}%
\pgfsys@useobject{currentmarker}{}%
\end{pgfscope}%
\begin{pgfscope}%
\pgfsys@transformshift{3.862737in}{3.521079in}%
\pgfsys@useobject{currentmarker}{}%
\end{pgfscope}%
\begin{pgfscope}%
\pgfsys@transformshift{3.886914in}{3.430425in}%
\pgfsys@useobject{currentmarker}{}%
\end{pgfscope}%
\begin{pgfscope}%
\pgfsys@transformshift{3.905927in}{3.286684in}%
\pgfsys@useobject{currentmarker}{}%
\end{pgfscope}%
\begin{pgfscope}%
\pgfsys@transformshift{3.923297in}{3.199943in}%
\pgfsys@useobject{currentmarker}{}%
\end{pgfscope}%
\begin{pgfscope}%
\pgfsys@transformshift{3.942311in}{3.162143in}%
\pgfsys@useobject{currentmarker}{}%
\end{pgfscope}%
\begin{pgfscope}%
\pgfsys@transformshift{3.958741in}{3.147208in}%
\pgfsys@useobject{currentmarker}{}%
\end{pgfscope}%
\begin{pgfscope}%
\pgfsys@transformshift{3.980335in}{3.141522in}%
\pgfsys@useobject{currentmarker}{}%
\end{pgfscope}%
\begin{pgfscope}%
\pgfsys@transformshift{4.000757in}{3.144735in}%
\pgfsys@useobject{currentmarker}{}%
\end{pgfscope}%
\begin{pgfscope}%
\pgfsys@transformshift{4.019536in}{3.156647in}%
\pgfsys@useobject{currentmarker}{}%
\end{pgfscope}%
\begin{pgfscope}%
\pgfsys@transformshift{4.037375in}{3.174639in}%
\pgfsys@useobject{currentmarker}{}%
\end{pgfscope}%
\begin{pgfscope}%
\pgfsys@transformshift{4.055214in}{3.220367in}%
\pgfsys@useobject{currentmarker}{}%
\end{pgfscope}%
\begin{pgfscope}%
\pgfsys@transformshift{4.076810in}{3.361488in}%
\pgfsys@useobject{currentmarker}{}%
\end{pgfscope}%
\begin{pgfscope}%
\pgfsys@transformshift{4.094884in}{3.510220in}%
\pgfsys@useobject{currentmarker}{}%
\end{pgfscope}%
\begin{pgfscope}%
\pgfsys@transformshift{4.115540in}{3.528761in}%
\pgfsys@useobject{currentmarker}{}%
\end{pgfscope}%
\begin{pgfscope}%
\pgfsys@transformshift{4.137136in}{3.461257in}%
\pgfsys@useobject{currentmarker}{}%
\end{pgfscope}%
\begin{pgfscope}%
\pgfsys@transformshift{4.153332in}{3.348248in}%
\pgfsys@useobject{currentmarker}{}%
\end{pgfscope}%
\begin{pgfscope}%
\pgfsys@transformshift{4.172814in}{3.234483in}%
\pgfsys@useobject{currentmarker}{}%
\end{pgfscope}%
\begin{pgfscope}%
\pgfsys@transformshift{4.191124in}{3.175917in}%
\pgfsys@useobject{currentmarker}{}%
\end{pgfscope}%
\begin{pgfscope}%
\pgfsys@transformshift{4.212718in}{3.153626in}%
\pgfsys@useobject{currentmarker}{}%
\end{pgfscope}%
\begin{pgfscope}%
\pgfsys@transformshift{4.230557in}{3.144092in}%
\pgfsys@useobject{currentmarker}{}%
\end{pgfscope}%
\begin{pgfscope}%
\pgfsys@transformshift{4.247927in}{3.142227in}%
\pgfsys@useobject{currentmarker}{}%
\end{pgfscope}%
\begin{pgfscope}%
\pgfsys@transformshift{4.265767in}{3.147533in}%
\pgfsys@useobject{currentmarker}{}%
\end{pgfscope}%
\begin{pgfscope}%
\pgfsys@transformshift{4.289709in}{3.167475in}%
\pgfsys@useobject{currentmarker}{}%
\end{pgfscope}%
\begin{pgfscope}%
\pgfsys@transformshift{4.308253in}{3.181833in}%
\pgfsys@useobject{currentmarker}{}%
\end{pgfscope}%
\begin{pgfscope}%
\pgfsys@transformshift{4.325623in}{3.239632in}%
\pgfsys@useobject{currentmarker}{}%
\end{pgfscope}%
\begin{pgfscope}%
\pgfsys@transformshift{4.346983in}{3.397977in}%
\pgfsys@useobject{currentmarker}{}%
\end{pgfscope}%
\begin{pgfscope}%
\pgfsys@transformshift{4.365996in}{3.535476in}%
\pgfsys@useobject{currentmarker}{}%
\end{pgfscope}%
\begin{pgfscope}%
\pgfsys@transformshift{4.383132in}{3.554347in}%
\pgfsys@useobject{currentmarker}{}%
\end{pgfscope}%
\begin{pgfscope}%
\pgfsys@transformshift{4.403788in}{3.458983in}%
\pgfsys@useobject{currentmarker}{}%
\end{pgfscope}%
\begin{pgfscope}%
\pgfsys@transformshift{4.419750in}{3.337675in}%
\pgfsys@useobject{currentmarker}{}%
\end{pgfscope}%
\begin{pgfscope}%
\pgfsys@transformshift{4.440875in}{3.251176in}%
\pgfsys@useobject{currentmarker}{}%
\end{pgfscope}%
\begin{pgfscope}%
\pgfsys@transformshift{4.462000in}{3.184133in}%
\pgfsys@useobject{currentmarker}{}%
\end{pgfscope}%
\begin{pgfscope}%
\pgfsys@transformshift{4.480076in}{3.157175in}%
\pgfsys@useobject{currentmarker}{}%
\end{pgfscope}%
\begin{pgfscope}%
\pgfsys@transformshift{4.483830in}{3.154988in}%
\pgfsys@useobject{currentmarker}{}%
\end{pgfscope}%
\begin{pgfscope}%
\pgfsys@transformshift{4.475379in}{3.168571in}%
\pgfsys@useobject{currentmarker}{}%
\end{pgfscope}%
\begin{pgfscope}%
\pgfsys@transformshift{4.455194in}{3.239177in}%
\pgfsys@useobject{currentmarker}{}%
\end{pgfscope}%
\begin{pgfscope}%
\pgfsys@transformshift{4.437120in}{3.403774in}%
\pgfsys@useobject{currentmarker}{}%
\end{pgfscope}%
\begin{pgfscope}%
\pgfsys@transformshift{4.418810in}{3.541258in}%
\pgfsys@useobject{currentmarker}{}%
\end{pgfscope}%
\begin{pgfscope}%
\pgfsys@transformshift{4.396511in}{3.485739in}%
\pgfsys@useobject{currentmarker}{}%
\end{pgfscope}%
\begin{pgfscope}%
\pgfsys@transformshift{4.377967in}{3.282080in}%
\pgfsys@useobject{currentmarker}{}%
\end{pgfscope}%
\begin{pgfscope}%
\pgfsys@transformshift{4.359659in}{3.186457in}%
\pgfsys@useobject{currentmarker}{}%
\end{pgfscope}%
\begin{pgfscope}%
\pgfsys@transformshift{4.338298in}{3.151428in}%
\pgfsys@useobject{currentmarker}{}%
\end{pgfscope}%
\begin{pgfscope}%
\pgfsys@transformshift{4.320224in}{3.141522in}%
\pgfsys@useobject{currentmarker}{}%
\end{pgfscope}%
\begin{pgfscope}%
\pgfsys@transformshift{4.301681in}{3.147979in}%
\pgfsys@useobject{currentmarker}{}%
\end{pgfscope}%
\begin{pgfscope}%
\pgfsys@transformshift{4.284545in}{3.173791in}%
\pgfsys@useobject{currentmarker}{}%
\end{pgfscope}%
\begin{pgfscope}%
\pgfsys@transformshift{4.263420in}{3.265437in}%
\pgfsys@useobject{currentmarker}{}%
\end{pgfscope}%
\begin{pgfscope}%
\pgfsys@transformshift{4.245816in}{3.437111in}%
\pgfsys@useobject{currentmarker}{}%
\end{pgfscope}%
\begin{pgfscope}%
\pgfsys@transformshift{4.225628in}{3.540014in}%
\pgfsys@useobject{currentmarker}{}%
\end{pgfscope}%
\begin{pgfscope}%
\pgfsys@transformshift{4.204503in}{3.404809in}%
\pgfsys@useobject{currentmarker}{}%
\end{pgfscope}%
\begin{pgfscope}%
\pgfsys@transformshift{4.186428in}{3.236068in}%
\pgfsys@useobject{currentmarker}{}%
\end{pgfscope}%
\begin{pgfscope}%
\pgfsys@transformshift{4.164600in}{3.163312in}%
\pgfsys@useobject{currentmarker}{}%
\end{pgfscope}%
\begin{pgfscope}%
\pgfsys@transformshift{4.147464in}{3.154170in}%
\pgfsys@useobject{currentmarker}{}%
\end{pgfscope}%
\begin{pgfscope}%
\pgfsys@transformshift{4.128919in}{3.194741in}%
\pgfsys@useobject{currentmarker}{}%
\end{pgfscope}%
\begin{pgfscope}%
\pgfsys@transformshift{4.108029in}{3.345756in}%
\pgfsys@useobject{currentmarker}{}%
\end{pgfscope}%
\begin{pgfscope}%
\pgfsys@transformshift{4.090189in}{3.492622in}%
\pgfsys@useobject{currentmarker}{}%
\end{pgfscope}%
\begin{pgfscope}%
\pgfsys@transformshift{4.070238in}{3.503109in}%
\pgfsys@useobject{currentmarker}{}%
\end{pgfscope}%
\begin{pgfscope}%
\pgfsys@transformshift{4.052868in}{3.322092in}%
\pgfsys@useobject{currentmarker}{}%
\end{pgfscope}%
\begin{pgfscope}%
\pgfsys@transformshift{4.032915in}{3.188789in}%
\pgfsys@useobject{currentmarker}{}%
\end{pgfscope}%
\begin{pgfscope}%
\pgfsys@transformshift{4.012493in}{3.151364in}%
\pgfsys@useobject{currentmarker}{}%
\end{pgfscope}%
\begin{pgfscope}%
\pgfsys@transformshift{3.994654in}{3.141233in}%
\pgfsys@useobject{currentmarker}{}%
\end{pgfscope}%
\begin{pgfscope}%
\pgfsys@transformshift{3.974233in}{3.145027in}%
\pgfsys@useobject{currentmarker}{}%
\end{pgfscope}%
\begin{pgfscope}%
\pgfsys@transformshift{3.953576in}{3.168320in}%
\pgfsys@useobject{currentmarker}{}%
\end{pgfscope}%
\begin{pgfscope}%
\pgfsys@transformshift{3.936442in}{3.234709in}%
\pgfsys@useobject{currentmarker}{}%
\end{pgfscope}%
\begin{pgfscope}%
\pgfsys@transformshift{3.918603in}{3.394281in}%
\pgfsys@useobject{currentmarker}{}%
\end{pgfscope}%
\begin{pgfscope}%
\pgfsys@transformshift{3.899590in}{3.508064in}%
\pgfsys@useobject{currentmarker}{}%
\end{pgfscope}%
\begin{pgfscope}%
\pgfsys@transformshift{3.880342in}{3.447422in}%
\pgfsys@useobject{currentmarker}{}%
\end{pgfscope}%
\begin{pgfscope}%
\pgfsys@transformshift{3.856869in}{3.234389in}%
\pgfsys@useobject{currentmarker}{}%
\end{pgfscope}%
\begin{pgfscope}%
\pgfsys@transformshift{3.838794in}{3.171669in}%
\pgfsys@useobject{currentmarker}{}%
\end{pgfscope}%
\begin{pgfscope}%
\pgfsys@transformshift{3.821189in}{3.147675in}%
\pgfsys@useobject{currentmarker}{}%
\end{pgfscope}%
\begin{pgfscope}%
\pgfsys@transformshift{3.801238in}{3.140615in}%
\pgfsys@useobject{currentmarker}{}%
\end{pgfscope}%
\begin{pgfscope}%
\pgfsys@transformshift{3.783164in}{3.145168in}%
\pgfsys@useobject{currentmarker}{}%
\end{pgfscope}%
\begin{pgfscope}%
\pgfsys@transformshift{3.761568in}{3.175887in}%
\pgfsys@useobject{currentmarker}{}%
\end{pgfscope}%
\begin{pgfscope}%
\pgfsys@transformshift{3.745841in}{3.232763in}%
\pgfsys@useobject{currentmarker}{}%
\end{pgfscope}%
\begin{pgfscope}%
\pgfsys@transformshift{3.722133in}{3.425141in}%
\pgfsys@useobject{currentmarker}{}%
\end{pgfscope}%
\begin{pgfscope}%
\pgfsys@transformshift{3.708051in}{3.495191in}%
\pgfsys@useobject{currentmarker}{}%
\end{pgfscope}%
\begin{pgfscope}%
\pgfsys@transformshift{3.685986in}{3.445673in}%
\pgfsys@useobject{currentmarker}{}%
\end{pgfscope}%
\begin{pgfscope}%
\pgfsys@transformshift{3.666268in}{3.260278in}%
\pgfsys@useobject{currentmarker}{}%
\end{pgfscope}%
\begin{pgfscope}%
\pgfsys@transformshift{3.648663in}{3.180876in}%
\pgfsys@useobject{currentmarker}{}%
\end{pgfscope}%
\begin{pgfscope}%
\pgfsys@transformshift{3.628478in}{3.157071in}%
\pgfsys@useobject{currentmarker}{}%
\end{pgfscope}%
\begin{pgfscope}%
\pgfsys@transformshift{3.610168in}{3.143062in}%
\pgfsys@useobject{currentmarker}{}%
\end{pgfscope}%
\begin{pgfscope}%
\pgfsys@transformshift{3.593737in}{3.140409in}%
\pgfsys@useobject{currentmarker}{}%
\end{pgfscope}%
\begin{pgfscope}%
\pgfsys@transformshift{3.570264in}{3.147010in}%
\pgfsys@useobject{currentmarker}{}%
\end{pgfscope}%
\begin{pgfscope}%
\pgfsys@transformshift{3.552425in}{3.162513in}%
\pgfsys@useobject{currentmarker}{}%
\end{pgfscope}%
\begin{pgfscope}%
\pgfsys@transformshift{3.532237in}{3.218162in}%
\pgfsys@useobject{currentmarker}{}%
\end{pgfscope}%
\begin{pgfscope}%
\pgfsys@transformshift{3.515572in}{3.333238in}%
\pgfsys@useobject{currentmarker}{}%
\end{pgfscope}%
\begin{pgfscope}%
\pgfsys@transformshift{3.495150in}{3.454613in}%
\pgfsys@useobject{currentmarker}{}%
\end{pgfscope}%
\begin{pgfscope}%
\pgfsys@transformshift{3.477780in}{3.491047in}%
\pgfsys@useobject{currentmarker}{}%
\end{pgfscope}%
\begin{pgfscope}%
\pgfsys@transformshift{3.455012in}{3.353834in}%
\pgfsys@useobject{currentmarker}{}%
\end{pgfscope}%
\begin{pgfscope}%
\pgfsys@transformshift{3.436468in}{3.216967in}%
\pgfsys@useobject{currentmarker}{}%
\end{pgfscope}%
\begin{pgfscope}%
\pgfsys@transformshift{3.418863in}{3.182025in}%
\pgfsys@useobject{currentmarker}{}%
\end{pgfscope}%
\begin{pgfscope}%
\pgfsys@transformshift{3.397738in}{3.153694in}%
\pgfsys@useobject{currentmarker}{}%
\end{pgfscope}%
\begin{pgfscope}%
\pgfsys@transformshift{3.378021in}{3.141922in}%
\pgfsys@useobject{currentmarker}{}%
\end{pgfscope}%
\begin{pgfscope}%
\pgfsys@transformshift{3.361354in}{3.140240in}%
\pgfsys@useobject{currentmarker}{}%
\end{pgfscope}%
\begin{pgfscope}%
\pgfsys@transformshift{3.340464in}{3.147570in}%
\pgfsys@useobject{currentmarker}{}%
\end{pgfscope}%
\begin{pgfscope}%
\pgfsys@transformshift{3.321921in}{3.172791in}%
\pgfsys@useobject{currentmarker}{}%
\end{pgfscope}%
\begin{pgfscope}%
\pgfsys@transformshift{3.301499in}{3.262973in}%
\pgfsys@useobject{currentmarker}{}%
\end{pgfscope}%
\begin{pgfscope}%
\pgfsys@transformshift{3.283424in}{3.399128in}%
\pgfsys@useobject{currentmarker}{}%
\end{pgfscope}%
\begin{pgfscope}%
\pgfsys@transformshift{3.263239in}{3.484847in}%
\pgfsys@useobject{currentmarker}{}%
\end{pgfscope}%
\begin{pgfscope}%
\pgfsys@transformshift{3.242582in}{3.409052in}%
\pgfsys@useobject{currentmarker}{}%
\end{pgfscope}%
\begin{pgfscope}%
\pgfsys@transformshift{3.224272in}{3.263145in}%
\pgfsys@useobject{currentmarker}{}%
\end{pgfscope}%
\begin{pgfscope}%
\pgfsys@transformshift{3.205259in}{3.407837in}%
\pgfsys@useobject{currentmarker}{}%
\end{pgfscope}%
\begin{pgfscope}%
\pgfsys@transformshift{3.187185in}{3.254347in}%
\pgfsys@useobject{currentmarker}{}%
\end{pgfscope}%
\begin{pgfscope}%
\pgfsys@transformshift{3.168877in}{3.178127in}%
\pgfsys@useobject{currentmarker}{}%
\end{pgfscope}%
\begin{pgfscope}%
\pgfsys@transformshift{3.169112in}{3.159333in}%
\pgfsys@useobject{currentmarker}{}%
\end{pgfscope}%
\begin{pgfscope}%
\pgfsys@transformshift{3.150567in}{3.153781in}%
\pgfsys@useobject{currentmarker}{}%
\end{pgfscope}%
\begin{pgfscope}%
\pgfsys@transformshift{3.130146in}{3.141940in}%
\pgfsys@useobject{currentmarker}{}%
\end{pgfscope}%
\begin{pgfscope}%
\pgfsys@transformshift{3.109021in}{3.140680in}%
\pgfsys@useobject{currentmarker}{}%
\end{pgfscope}%
\begin{pgfscope}%
\pgfsys@transformshift{3.091650in}{3.148774in}%
\pgfsys@useobject{currentmarker}{}%
\end{pgfscope}%
\begin{pgfscope}%
\pgfsys@transformshift{3.071229in}{3.170502in}%
\pgfsys@useobject{currentmarker}{}%
\end{pgfscope}%
\begin{pgfscope}%
\pgfsys@transformshift{3.053155in}{3.239545in}%
\pgfsys@useobject{currentmarker}{}%
\end{pgfscope}%
\begin{pgfscope}%
\pgfsys@transformshift{3.032264in}{3.406428in}%
\pgfsys@useobject{currentmarker}{}%
\end{pgfscope}%
\begin{pgfscope}%
\pgfsys@transformshift{3.013486in}{3.483047in}%
\pgfsys@useobject{currentmarker}{}%
\end{pgfscope}%
\begin{pgfscope}%
\pgfsys@transformshift{2.995412in}{3.439473in}%
\pgfsys@useobject{currentmarker}{}%
\end{pgfscope}%
\begin{pgfscope}%
\pgfsys@transformshift{2.973582in}{3.264742in}%
\pgfsys@useobject{currentmarker}{}%
\end{pgfscope}%
\begin{pgfscope}%
\pgfsys@transformshift{2.955039in}{3.182126in}%
\pgfsys@useobject{currentmarker}{}%
\end{pgfscope}%
\begin{pgfscope}%
\pgfsys@transformshift{2.936260in}{3.152565in}%
\pgfsys@useobject{currentmarker}{}%
\end{pgfscope}%
\begin{pgfscope}%
\pgfsys@transformshift{2.921002in}{3.145032in}%
\pgfsys@useobject{currentmarker}{}%
\end{pgfscope}%
\begin{pgfscope}%
\pgfsys@transformshift{2.898937in}{3.140262in}%
\pgfsys@useobject{currentmarker}{}%
\end{pgfscope}%
\begin{pgfscope}%
\pgfsys@transformshift{2.880629in}{3.139557in}%
\pgfsys@useobject{currentmarker}{}%
\end{pgfscope}%
\begin{pgfscope}%
\pgfsys@transformshift{2.858096in}{3.139319in}%
\pgfsys@useobject{currentmarker}{}%
\end{pgfscope}%
\begin{pgfscope}%
\pgfsys@transformshift{2.840725in}{3.143884in}%
\pgfsys@useobject{currentmarker}{}%
\end{pgfscope}%
\begin{pgfscope}%
\pgfsys@transformshift{2.821243in}{3.160905in}%
\pgfsys@useobject{currentmarker}{}%
\end{pgfscope}%
\begin{pgfscope}%
\pgfsys@transformshift{2.805047in}{3.198801in}%
\pgfsys@useobject{currentmarker}{}%
\end{pgfscope}%
\begin{pgfscope}%
\pgfsys@transformshift{2.784625in}{3.347302in}%
\pgfsys@useobject{currentmarker}{}%
\end{pgfscope}%
\begin{pgfscope}%
\pgfsys@transformshift{2.763264in}{3.460286in}%
\pgfsys@useobject{currentmarker}{}%
\end{pgfscope}%
\begin{pgfscope}%
\pgfsys@transformshift{2.744721in}{3.461647in}%
\pgfsys@useobject{currentmarker}{}%
\end{pgfscope}%
\begin{pgfscope}%
\pgfsys@transformshift{2.725474in}{3.325179in}%
\pgfsys@useobject{currentmarker}{}%
\end{pgfscope}%
\begin{pgfscope}%
\pgfsys@transformshift{2.705521in}{3.219040in}%
\pgfsys@useobject{currentmarker}{}%
\end{pgfscope}%
\begin{pgfscope}%
\pgfsys@transformshift{2.686978in}{3.165738in}%
\pgfsys@useobject{currentmarker}{}%
\end{pgfscope}%
\begin{pgfscope}%
\pgfsys@transformshift{2.668668in}{3.146801in}%
\pgfsys@useobject{currentmarker}{}%
\end{pgfscope}%
\begin{pgfscope}%
\pgfsys@transformshift{2.650126in}{3.140167in}%
\pgfsys@useobject{currentmarker}{}%
\end{pgfscope}%
\begin{pgfscope}%
\pgfsys@transformshift{2.631581in}{3.141517in}%
\pgfsys@useobject{currentmarker}{}%
\end{pgfscope}%
\begin{pgfscope}%
\pgfsys@transformshift{2.610220in}{3.150656in}%
\pgfsys@useobject{currentmarker}{}%
\end{pgfscope}%
\begin{pgfscope}%
\pgfsys@transformshift{2.590503in}{3.175138in}%
\pgfsys@useobject{currentmarker}{}%
\end{pgfscope}%
\begin{pgfscope}%
\pgfsys@transformshift{2.589800in}{3.213382in}%
\pgfsys@useobject{currentmarker}{}%
\end{pgfscope}%
\begin{pgfscope}%
\pgfsys@transformshift{2.571959in}{3.248575in}%
\pgfsys@useobject{currentmarker}{}%
\end{pgfscope}%
\begin{pgfscope}%
\pgfsys@transformshift{2.552711in}{3.400754in}%
\pgfsys@useobject{currentmarker}{}%
\end{pgfscope}%
\begin{pgfscope}%
\pgfsys@transformshift{2.533229in}{3.474733in}%
\pgfsys@useobject{currentmarker}{}%
\end{pgfscope}%
\begin{pgfscope}%
\pgfsys@transformshift{2.512573in}{3.435525in}%
\pgfsys@useobject{currentmarker}{}%
\end{pgfscope}%
\begin{pgfscope}%
\pgfsys@transformshift{2.496142in}{3.285009in}%
\pgfsys@useobject{currentmarker}{}%
\end{pgfscope}%
\begin{pgfscope}%
\pgfsys@transformshift{2.476895in}{3.189213in}%
\pgfsys@useobject{currentmarker}{}%
\end{pgfscope}%
\begin{pgfscope}%
\pgfsys@transformshift{2.459524in}{3.164710in}%
\pgfsys@useobject{currentmarker}{}%
\end{pgfscope}%
\begin{pgfscope}%
\pgfsys@transformshift{2.436756in}{3.144624in}%
\pgfsys@useobject{currentmarker}{}%
\end{pgfscope}%
\begin{pgfscope}%
\pgfsys@transformshift{2.417038in}{3.139580in}%
\pgfsys@useobject{currentmarker}{}%
\end{pgfscope}%
\begin{pgfscope}%
\pgfsys@transformshift{2.397790in}{3.140050in}%
\pgfsys@useobject{currentmarker}{}%
\end{pgfscope}%
\begin{pgfscope}%
\pgfsys@transformshift{2.379482in}{3.145546in}%
\pgfsys@useobject{currentmarker}{}%
\end{pgfscope}%
\begin{pgfscope}%
\pgfsys@transformshift{2.360235in}{3.164475in}%
\pgfsys@useobject{currentmarker}{}%
\end{pgfscope}%
\begin{pgfscope}%
\pgfsys@transformshift{2.342159in}{3.223547in}%
\pgfsys@useobject{currentmarker}{}%
\end{pgfscope}%
\begin{pgfscope}%
\pgfsys@transformshift{2.319626in}{3.374933in}%
\pgfsys@useobject{currentmarker}{}%
\end{pgfscope}%
\begin{pgfscope}%
\pgfsys@transformshift{2.302492in}{3.464664in}%
\pgfsys@useobject{currentmarker}{}%
\end{pgfscope}%
\begin{pgfscope}%
\pgfsys@transformshift{2.285590in}{3.463790in}%
\pgfsys@useobject{currentmarker}{}%
\end{pgfscope}%
\begin{pgfscope}%
\pgfsys@transformshift{2.264231in}{3.357448in}%
\pgfsys@useobject{currentmarker}{}%
\end{pgfscope}%
\begin{pgfscope}%
\pgfsys@transformshift{2.244747in}{3.230878in}%
\pgfsys@useobject{currentmarker}{}%
\end{pgfscope}%
\begin{pgfscope}%
\pgfsys@transformshift{2.226907in}{3.176213in}%
\pgfsys@useobject{currentmarker}{}%
\end{pgfscope}%
\begin{pgfscope}%
\pgfsys@transformshift{2.205548in}{3.149825in}%
\pgfsys@useobject{currentmarker}{}%
\end{pgfscope}%
\begin{pgfscope}%
\pgfsys@transformshift{2.187004in}{3.143046in}%
\pgfsys@useobject{currentmarker}{}%
\end{pgfscope}%
\begin{pgfscope}%
\pgfsys@transformshift{2.168930in}{3.139538in}%
\pgfsys@useobject{currentmarker}{}%
\end{pgfscope}%
\begin{pgfscope}%
\pgfsys@transformshift{2.149213in}{3.143810in}%
\pgfsys@useobject{currentmarker}{}%
\end{pgfscope}%
\begin{pgfscope}%
\pgfsys@transformshift{2.127852in}{3.159514in}%
\pgfsys@useobject{currentmarker}{}%
\end{pgfscope}%
\begin{pgfscope}%
\pgfsys@transformshift{2.109308in}{3.195677in}%
\pgfsys@useobject{currentmarker}{}%
\end{pgfscope}%
\begin{pgfscope}%
\pgfsys@transformshift{2.090765in}{3.292754in}%
\pgfsys@useobject{currentmarker}{}%
\end{pgfscope}%
\begin{pgfscope}%
\pgfsys@transformshift{2.072455in}{3.399593in}%
\pgfsys@useobject{currentmarker}{}%
\end{pgfscope}%
\begin{pgfscope}%
\pgfsys@transformshift{2.053678in}{3.475195in}%
\pgfsys@useobject{currentmarker}{}%
\end{pgfscope}%
\begin{pgfscope}%
\pgfsys@transformshift{2.032082in}{3.452772in}%
\pgfsys@useobject{currentmarker}{}%
\end{pgfscope}%
\begin{pgfscope}%
\pgfsys@transformshift{2.013538in}{3.386270in}%
\pgfsys@useobject{currentmarker}{}%
\end{pgfscope}%
\begin{pgfscope}%
\pgfsys@transformshift{1.998281in}{3.255902in}%
\pgfsys@useobject{currentmarker}{}%
\end{pgfscope}%
\begin{pgfscope}%
\pgfsys@transformshift{1.975748in}{3.177656in}%
\pgfsys@useobject{currentmarker}{}%
\end{pgfscope}%
\begin{pgfscope}%
\pgfsys@transformshift{1.957672in}{3.155895in}%
\pgfsys@useobject{currentmarker}{}%
\end{pgfscope}%
\begin{pgfscope}%
\pgfsys@transformshift{1.935139in}{3.143363in}%
\pgfsys@useobject{currentmarker}{}%
\end{pgfscope}%
\begin{pgfscope}%
\pgfsys@transformshift{1.919882in}{3.140152in}%
\pgfsys@useobject{currentmarker}{}%
\end{pgfscope}%
\begin{pgfscope}%
\pgfsys@transformshift{1.898757in}{3.142105in}%
\pgfsys@useobject{currentmarker}{}%
\end{pgfscope}%
\begin{pgfscope}%
\pgfsys@transformshift{1.879978in}{3.147762in}%
\pgfsys@useobject{currentmarker}{}%
\end{pgfscope}%
\begin{pgfscope}%
\pgfsys@transformshift{1.859088in}{3.148724in}%
\pgfsys@useobject{currentmarker}{}%
\end{pgfscope}%
\begin{pgfscope}%
\pgfsys@transformshift{1.840309in}{3.167544in}%
\pgfsys@useobject{currentmarker}{}%
\end{pgfscope}%
\begin{pgfscope}%
\pgfsys@transformshift{1.822939in}{3.219734in}%
\pgfsys@useobject{currentmarker}{}%
\end{pgfscope}%
\begin{pgfscope}%
\pgfsys@transformshift{1.801814in}{3.362506in}%
\pgfsys@useobject{currentmarker}{}%
\end{pgfscope}%
\begin{pgfscope}%
\pgfsys@transformshift{1.784678in}{3.460805in}%
\pgfsys@useobject{currentmarker}{}%
\end{pgfscope}%
\begin{pgfscope}%
\pgfsys@transformshift{1.762613in}{3.483098in}%
\pgfsys@useobject{currentmarker}{}%
\end{pgfscope}%
\begin{pgfscope}%
\pgfsys@transformshift{1.747122in}{3.424095in}%
\pgfsys@useobject{currentmarker}{}%
\end{pgfscope}%
\begin{pgfscope}%
\pgfsys@transformshift{1.725292in}{3.282711in}%
\pgfsys@useobject{currentmarker}{}%
\end{pgfscope}%
\begin{pgfscope}%
\pgfsys@transformshift{1.706513in}{3.200363in}%
\pgfsys@useobject{currentmarker}{}%
\end{pgfscope}%
\begin{pgfscope}%
\pgfsys@transformshift{1.688439in}{3.161680in}%
\pgfsys@useobject{currentmarker}{}%
\end{pgfscope}%
\begin{pgfscope}%
\pgfsys@transformshift{1.669660in}{3.147167in}%
\pgfsys@useobject{currentmarker}{}%
\end{pgfscope}%
\begin{pgfscope}%
\pgfsys@transformshift{1.648301in}{3.140804in}%
\pgfsys@useobject{currentmarker}{}%
\end{pgfscope}%
\begin{pgfscope}%
\pgfsys@transformshift{1.629051in}{3.140765in}%
\pgfsys@useobject{currentmarker}{}%
\end{pgfscope}%
\begin{pgfscope}%
\pgfsys@transformshift{1.610743in}{3.143906in}%
\pgfsys@useobject{currentmarker}{}%
\end{pgfscope}%
\begin{pgfscope}%
\pgfsys@transformshift{1.591496in}{3.465906in}%
\pgfsys@useobject{currentmarker}{}%
\end{pgfscope}%
\begin{pgfscope}%
\pgfsys@transformshift{1.570134in}{3.287921in}%
\pgfsys@useobject{currentmarker}{}%
\end{pgfscope}%
\begin{pgfscope}%
\pgfsys@transformshift{1.548306in}{3.180337in}%
\pgfsys@useobject{currentmarker}{}%
\end{pgfscope}%
\begin{pgfscope}%
\pgfsys@transformshift{1.534456in}{3.166898in}%
\pgfsys@useobject{currentmarker}{}%
\end{pgfscope}%
\begin{pgfscope}%
\pgfsys@transformshift{1.514974in}{3.148639in}%
\pgfsys@useobject{currentmarker}{}%
\end{pgfscope}%
\begin{pgfscope}%
\pgfsys@transformshift{1.496195in}{3.140158in}%
\pgfsys@useobject{currentmarker}{}%
\end{pgfscope}%
\begin{pgfscope}%
\pgfsys@transformshift{1.477416in}{3.142020in}%
\pgfsys@useobject{currentmarker}{}%
\end{pgfscope}%
\begin{pgfscope}%
\pgfsys@transformshift{1.456057in}{3.153776in}%
\pgfsys@useobject{currentmarker}{}%
\end{pgfscope}%
\begin{pgfscope}%
\pgfsys@transformshift{1.437043in}{3.186805in}%
\pgfsys@useobject{currentmarker}{}%
\end{pgfscope}%
\begin{pgfscope}%
\pgfsys@transformshift{1.418735in}{3.277186in}%
\pgfsys@useobject{currentmarker}{}%
\end{pgfscope}%
\begin{pgfscope}%
\pgfsys@transformshift{1.400191in}{3.429745in}%
\pgfsys@useobject{currentmarker}{}%
\end{pgfscope}%
\begin{pgfscope}%
\pgfsys@transformshift{1.382117in}{3.494040in}%
\pgfsys@useobject{currentmarker}{}%
\end{pgfscope}%
\begin{pgfscope}%
\pgfsys@transformshift{1.360053in}{3.440076in}%
\pgfsys@useobject{currentmarker}{}%
\end{pgfscope}%
\begin{pgfscope}%
\pgfsys@transformshift{1.341274in}{3.279551in}%
\pgfsys@useobject{currentmarker}{}%
\end{pgfscope}%
\begin{pgfscope}%
\pgfsys@transformshift{1.323435in}{3.197319in}%
\pgfsys@useobject{currentmarker}{}%
\end{pgfscope}%
\begin{pgfscope}%
\pgfsys@transformshift{1.301839in}{3.154727in}%
\pgfsys@useobject{currentmarker}{}%
\end{pgfscope}%
\begin{pgfscope}%
\pgfsys@transformshift{1.283296in}{3.145314in}%
\pgfsys@useobject{currentmarker}{}%
\end{pgfscope}%
\begin{pgfscope}%
\pgfsys@transformshift{1.264517in}{3.142744in}%
\pgfsys@useobject{currentmarker}{}%
\end{pgfscope}%
\begin{pgfscope}%
\pgfsys@transformshift{1.246209in}{3.140464in}%
\pgfsys@useobject{currentmarker}{}%
\end{pgfscope}%
\begin{pgfscope}%
\pgfsys@transformshift{1.227431in}{3.142665in}%
\pgfsys@useobject{currentmarker}{}%
\end{pgfscope}%
\begin{pgfscope}%
\pgfsys@transformshift{1.205835in}{3.154652in}%
\pgfsys@useobject{currentmarker}{}%
\end{pgfscope}%
\begin{pgfscope}%
\pgfsys@transformshift{1.187761in}{3.185827in}%
\pgfsys@useobject{currentmarker}{}%
\end{pgfscope}%
\begin{pgfscope}%
\pgfsys@transformshift{1.168982in}{3.246050in}%
\pgfsys@useobject{currentmarker}{}%
\end{pgfscope}%
\begin{pgfscope}%
\pgfsys@transformshift{1.147623in}{3.411817in}%
\pgfsys@useobject{currentmarker}{}%
\end{pgfscope}%
\begin{pgfscope}%
\pgfsys@transformshift{1.130253in}{3.505464in}%
\pgfsys@useobject{currentmarker}{}%
\end{pgfscope}%
\begin{pgfscope}%
\pgfsys@transformshift{1.112648in}{3.496329in}%
\pgfsys@useobject{currentmarker}{}%
\end{pgfscope}%
\begin{pgfscope}%
\pgfsys@transformshift{1.091052in}{3.346625in}%
\pgfsys@useobject{currentmarker}{}%
\end{pgfscope}%
\begin{pgfscope}%
\pgfsys@transformshift{1.070161in}{3.215465in}%
\pgfsys@useobject{currentmarker}{}%
\end{pgfscope}%
\begin{pgfscope}%
\pgfsys@transformshift{1.054905in}{3.174164in}%
\pgfsys@useobject{currentmarker}{}%
\end{pgfscope}%
\begin{pgfscope}%
\pgfsys@transformshift{1.036126in}{3.152589in}%
\pgfsys@useobject{currentmarker}{}%
\end{pgfscope}%
\begin{pgfscope}%
\pgfsys@transformshift{1.017581in}{3.145070in}%
\pgfsys@useobject{currentmarker}{}%
\end{pgfscope}%
\begin{pgfscope}%
\pgfsys@transformshift{0.995988in}{3.141197in}%
\pgfsys@useobject{currentmarker}{}%
\end{pgfscope}%
\begin{pgfscope}%
\pgfsys@transformshift{0.976504in}{3.141737in}%
\pgfsys@useobject{currentmarker}{}%
\end{pgfscope}%
\begin{pgfscope}%
\pgfsys@transformshift{0.959135in}{3.150045in}%
\pgfsys@useobject{currentmarker}{}%
\end{pgfscope}%
\begin{pgfscope}%
\pgfsys@transformshift{0.937071in}{3.184906in}%
\pgfsys@useobject{currentmarker}{}%
\end{pgfscope}%
\begin{pgfscope}%
\pgfsys@transformshift{0.918057in}{3.259950in}%
\pgfsys@useobject{currentmarker}{}%
\end{pgfscope}%
\begin{pgfscope}%
\pgfsys@transformshift{0.900452in}{3.395324in}%
\pgfsys@useobject{currentmarker}{}%
\end{pgfscope}%
\begin{pgfscope}%
\pgfsys@transformshift{0.878388in}{3.316329in}%
\pgfsys@useobject{currentmarker}{}%
\end{pgfscope}%
\begin{pgfscope}%
\pgfsys@transformshift{0.860314in}{3.464023in}%
\pgfsys@useobject{currentmarker}{}%
\end{pgfscope}%
\begin{pgfscope}%
\pgfsys@transformshift{0.844587in}{3.523081in}%
\pgfsys@useobject{currentmarker}{}%
\end{pgfscope}%
\begin{pgfscope}%
\pgfsys@transformshift{0.823462in}{3.483812in}%
\pgfsys@useobject{currentmarker}{}%
\end{pgfscope}%
\begin{pgfscope}%
\pgfsys@transformshift{0.804448in}{3.314550in}%
\pgfsys@useobject{currentmarker}{}%
\end{pgfscope}%
\begin{pgfscope}%
\pgfsys@transformshift{0.784730in}{3.214840in}%
\pgfsys@useobject{currentmarker}{}%
\end{pgfscope}%
\begin{pgfscope}%
\pgfsys@transformshift{0.763839in}{3.173548in}%
\pgfsys@useobject{currentmarker}{}%
\end{pgfscope}%
\begin{pgfscope}%
\pgfsys@transformshift{0.745766in}{3.154510in}%
\pgfsys@useobject{currentmarker}{}%
\end{pgfscope}%
\begin{pgfscope}%
\pgfsys@transformshift{0.726753in}{3.143741in}%
\pgfsys@useobject{currentmarker}{}%
\end{pgfscope}%
\begin{pgfscope}%
\pgfsys@transformshift{0.709617in}{3.141492in}%
\pgfsys@useobject{currentmarker}{}%
\end{pgfscope}%
\begin{pgfscope}%
\pgfsys@transformshift{0.686849in}{3.147057in}%
\pgfsys@useobject{currentmarker}{}%
\end{pgfscope}%
\begin{pgfscope}%
\pgfsys@transformshift{0.669713in}{3.165872in}%
\pgfsys@useobject{currentmarker}{}%
\end{pgfscope}%
\begin{pgfscope}%
\pgfsys@transformshift{0.649762in}{3.209774in}%
\pgfsys@useobject{currentmarker}{}%
\end{pgfscope}%
\begin{pgfscope}%
\pgfsys@transformshift{0.649525in}{3.206262in}%
\pgfsys@useobject{currentmarker}{}%
\end{pgfscope}%
\begin{pgfscope}%
\pgfsys@transformshift{0.655396in}{3.184487in}%
\pgfsys@useobject{currentmarker}{}%
\end{pgfscope}%
\begin{pgfscope}%
\pgfsys@transformshift{0.676286in}{3.148336in}%
\pgfsys@useobject{currentmarker}{}%
\end{pgfscope}%
\begin{pgfscope}%
\pgfsys@transformshift{0.695300in}{3.141114in}%
\pgfsys@useobject{currentmarker}{}%
\end{pgfscope}%
\begin{pgfscope}%
\pgfsys@transformshift{0.712668in}{3.148184in}%
\pgfsys@useobject{currentmarker}{}%
\end{pgfscope}%
\begin{pgfscope}%
\pgfsys@transformshift{0.734264in}{3.177900in}%
\pgfsys@useobject{currentmarker}{}%
\end{pgfscope}%
\begin{pgfscope}%
\pgfsys@transformshift{0.751869in}{3.260016in}%
\pgfsys@useobject{currentmarker}{}%
\end{pgfscope}%
\begin{pgfscope}%
\pgfsys@transformshift{0.770413in}{3.445502in}%
\pgfsys@useobject{currentmarker}{}%
\end{pgfscope}%
\begin{pgfscope}%
\pgfsys@transformshift{0.790833in}{3.516949in}%
\pgfsys@useobject{currentmarker}{}%
\end{pgfscope}%
\begin{pgfscope}%
\pgfsys@transformshift{0.809143in}{3.397629in}%
\pgfsys@useobject{currentmarker}{}%
\end{pgfscope}%
\begin{pgfscope}%
\pgfsys@transformshift{0.830738in}{3.217187in}%
\pgfsys@useobject{currentmarker}{}%
\end{pgfscope}%
\begin{pgfscope}%
\pgfsys@transformshift{0.849047in}{3.156773in}%
\pgfsys@useobject{currentmarker}{}%
\end{pgfscope}%
\begin{pgfscope}%
\pgfsys@transformshift{0.868294in}{3.143026in}%
\pgfsys@useobject{currentmarker}{}%
\end{pgfscope}%
\begin{pgfscope}%
\pgfsys@transformshift{0.888011in}{3.143875in}%
\pgfsys@useobject{currentmarker}{}%
\end{pgfscope}%
\begin{pgfscope}%
\pgfsys@transformshift{0.906555in}{3.251291in}%
\pgfsys@useobject{currentmarker}{}%
\end{pgfscope}%
\begin{pgfscope}%
\pgfsys@transformshift{0.924629in}{3.441623in}%
\pgfsys@useobject{currentmarker}{}%
\end{pgfscope}%
\begin{pgfscope}%
\pgfsys@transformshift{0.943876in}{3.504466in}%
\pgfsys@useobject{currentmarker}{}%
\end{pgfscope}%
\begin{pgfscope}%
\pgfsys@transformshift{0.963361in}{3.361735in}%
\pgfsys@useobject{currentmarker}{}%
\end{pgfscope}%
\begin{pgfscope}%
\pgfsys@transformshift{0.984954in}{3.215920in}%
\pgfsys@useobject{currentmarker}{}%
\end{pgfscope}%
\begin{pgfscope}%
\pgfsys@transformshift{1.003499in}{3.156915in}%
\pgfsys@useobject{currentmarker}{}%
\end{pgfscope}%
\begin{pgfscope}%
\pgfsys@transformshift{1.022278in}{3.141883in}%
\pgfsys@useobject{currentmarker}{}%
\end{pgfscope}%
\begin{pgfscope}%
\pgfsys@transformshift{1.041994in}{3.141537in}%
\pgfsys@useobject{currentmarker}{}%
\end{pgfscope}%
\begin{pgfscope}%
\pgfsys@transformshift{1.060537in}{3.154102in}%
\pgfsys@useobject{currentmarker}{}%
\end{pgfscope}%
\begin{pgfscope}%
\pgfsys@transformshift{1.079786in}{3.188575in}%
\pgfsys@useobject{currentmarker}{}%
\end{pgfscope}%
\begin{pgfscope}%
\pgfsys@transformshift{1.098563in}{3.317674in}%
\pgfsys@useobject{currentmarker}{}%
\end{pgfscope}%
\begin{pgfscope}%
\pgfsys@transformshift{1.118516in}{3.487152in}%
\pgfsys@useobject{currentmarker}{}%
\end{pgfscope}%
\begin{pgfscope}%
\pgfsys@transformshift{1.134244in}{3.452068in}%
\pgfsys@useobject{currentmarker}{}%
\end{pgfscope}%
\begin{pgfscope}%
\pgfsys@transformshift{1.154194in}{3.319610in}%
\pgfsys@useobject{currentmarker}{}%
\end{pgfscope}%
\begin{pgfscope}%
\pgfsys@transformshift{1.175790in}{3.175408in}%
\pgfsys@useobject{currentmarker}{}%
\end{pgfscope}%
\begin{pgfscope}%
\pgfsys@transformshift{1.192690in}{3.149555in}%
\pgfsys@useobject{currentmarker}{}%
\end{pgfscope}%
\begin{pgfscope}%
\pgfsys@transformshift{1.213817in}{3.139873in}%
\pgfsys@useobject{currentmarker}{}%
\end{pgfscope}%
\begin{pgfscope}%
\pgfsys@transformshift{1.233768in}{3.143070in}%
\pgfsys@useobject{currentmarker}{}%
\end{pgfscope}%
\begin{pgfscope}%
\pgfsys@transformshift{1.252781in}{3.159056in}%
\pgfsys@useobject{currentmarker}{}%
\end{pgfscope}%
\begin{pgfscope}%
\pgfsys@transformshift{1.272029in}{3.211468in}%
\pgfsys@useobject{currentmarker}{}%
\end{pgfscope}%
\begin{pgfscope}%
\pgfsys@transformshift{1.291042in}{3.357925in}%
\pgfsys@useobject{currentmarker}{}%
\end{pgfscope}%
\begin{pgfscope}%
\pgfsys@transformshift{1.309586in}{3.484867in}%
\pgfsys@useobject{currentmarker}{}%
\end{pgfscope}%
\begin{pgfscope}%
\pgfsys@transformshift{1.328834in}{3.432163in}%
\pgfsys@useobject{currentmarker}{}%
\end{pgfscope}%
\begin{pgfscope}%
\pgfsys@transformshift{1.347376in}{3.275849in}%
\pgfsys@useobject{currentmarker}{}%
\end{pgfscope}%
\begin{pgfscope}%
\pgfsys@transformshift{1.366155in}{3.173446in}%
\pgfsys@useobject{currentmarker}{}%
\end{pgfscope}%
\begin{pgfscope}%
\pgfsys@transformshift{1.385637in}{3.146222in}%
\pgfsys@useobject{currentmarker}{}%
\end{pgfscope}%
\begin{pgfscope}%
\pgfsys@transformshift{1.407936in}{3.139072in}%
\pgfsys@useobject{currentmarker}{}%
\end{pgfscope}%
\begin{pgfscope}%
\pgfsys@transformshift{1.423664in}{3.141187in}%
\pgfsys@useobject{currentmarker}{}%
\end{pgfscope}%
\begin{pgfscope}%
\pgfsys@transformshift{1.443851in}{3.155758in}%
\pgfsys@useobject{currentmarker}{}%
\end{pgfscope}%
\begin{pgfscope}%
\pgfsys@transformshift{1.463802in}{3.192393in}%
\pgfsys@useobject{currentmarker}{}%
\end{pgfscope}%
\begin{pgfscope}%
\pgfsys@transformshift{1.482112in}{3.304159in}%
\pgfsys@useobject{currentmarker}{}%
\end{pgfscope}%
\begin{pgfscope}%
\pgfsys@transformshift{1.502534in}{3.452343in}%
\pgfsys@useobject{currentmarker}{}%
\end{pgfscope}%
\begin{pgfscope}%
\pgfsys@transformshift{1.521311in}{3.471221in}%
\pgfsys@useobject{currentmarker}{}%
\end{pgfscope}%
\begin{pgfscope}%
\pgfsys@transformshift{1.541264in}{3.352906in}%
\pgfsys@useobject{currentmarker}{}%
\end{pgfscope}%
\begin{pgfscope}%
\pgfsys@transformshift{1.559103in}{3.231528in}%
\pgfsys@useobject{currentmarker}{}%
\end{pgfscope}%
\begin{pgfscope}%
\pgfsys@transformshift{1.579993in}{3.163130in}%
\pgfsys@useobject{currentmarker}{}%
\end{pgfscope}%
\begin{pgfscope}%
\pgfsys@transformshift{1.596424in}{3.145216in}%
\pgfsys@useobject{currentmarker}{}%
\end{pgfscope}%
\begin{pgfscope}%
\pgfsys@transformshift{1.620132in}{3.139084in}%
\pgfsys@useobject{currentmarker}{}%
\end{pgfscope}%
\begin{pgfscope}%
\pgfsys@transformshift{1.637502in}{3.140162in}%
\pgfsys@useobject{currentmarker}{}%
\end{pgfscope}%
\begin{pgfscope}%
\pgfsys@transformshift{1.656515in}{3.148954in}%
\pgfsys@useobject{currentmarker}{}%
\end{pgfscope}%
\begin{pgfscope}%
\pgfsys@transformshift{1.672477in}{3.173192in}%
\pgfsys@useobject{currentmarker}{}%
\end{pgfscope}%
\begin{pgfscope}%
\pgfsys@transformshift{1.694542in}{3.216719in}%
\pgfsys@useobject{currentmarker}{}%
\end{pgfscope}%
\begin{pgfscope}%
\pgfsys@transformshift{1.712850in}{3.375625in}%
\pgfsys@useobject{currentmarker}{}%
\end{pgfscope}%
\begin{pgfscope}%
\pgfsys@transformshift{1.731160in}{3.470812in}%
\pgfsys@useobject{currentmarker}{}%
\end{pgfscope}%
\begin{pgfscope}%
\pgfsys@transformshift{1.749702in}{3.444496in}%
\pgfsys@useobject{currentmarker}{}%
\end{pgfscope}%
\begin{pgfscope}%
\pgfsys@transformshift{1.772707in}{3.287178in}%
\pgfsys@useobject{currentmarker}{}%
\end{pgfscope}%
\begin{pgfscope}%
\pgfsys@transformshift{1.791954in}{3.185861in}%
\pgfsys@useobject{currentmarker}{}%
\end{pgfscope}%
\begin{pgfscope}%
\pgfsys@transformshift{1.810968in}{3.150245in}%
\pgfsys@useobject{currentmarker}{}%
\end{pgfscope}%
\begin{pgfscope}%
\pgfsys@transformshift{1.829041in}{3.141753in}%
\pgfsys@useobject{currentmarker}{}%
\end{pgfscope}%
\begin{pgfscope}%
\pgfsys@transformshift{1.848054in}{3.138752in}%
\pgfsys@useobject{currentmarker}{}%
\end{pgfscope}%
\begin{pgfscope}%
\pgfsys@transformshift{1.867302in}{3.140344in}%
\pgfsys@useobject{currentmarker}{}%
\end{pgfscope}%
\begin{pgfscope}%
\pgfsys@transformshift{1.884907in}{3.148475in}%
\pgfsys@useobject{currentmarker}{}%
\end{pgfscope}%
\begin{pgfscope}%
\pgfsys@transformshift{1.904155in}{3.174784in}%
\pgfsys@useobject{currentmarker}{}%
\end{pgfscope}%
\begin{pgfscope}%
\pgfsys@transformshift{1.925045in}{3.266312in}%
\pgfsys@useobject{currentmarker}{}%
\end{pgfscope}%
\begin{pgfscope}%
\pgfsys@transformshift{1.944293in}{3.420069in}%
\pgfsys@useobject{currentmarker}{}%
\end{pgfscope}%
\begin{pgfscope}%
\pgfsys@transformshift{1.962837in}{3.461719in}%
\pgfsys@useobject{currentmarker}{}%
\end{pgfscope}%
\begin{pgfscope}%
\pgfsys@transformshift{1.981851in}{3.376946in}%
\pgfsys@useobject{currentmarker}{}%
\end{pgfscope}%
\begin{pgfscope}%
\pgfsys@transformshift{2.003446in}{3.271920in}%
\pgfsys@useobject{currentmarker}{}%
\end{pgfscope}%
\begin{pgfscope}%
\pgfsys@transformshift{2.019406in}{3.182480in}%
\pgfsys@useobject{currentmarker}{}%
\end{pgfscope}%
\begin{pgfscope}%
\pgfsys@transformshift{2.040062in}{3.319495in}%
\pgfsys@useobject{currentmarker}{}%
\end{pgfscope}%
\begin{pgfscope}%
\pgfsys@transformshift{2.058841in}{3.195194in}%
\pgfsys@useobject{currentmarker}{}%
\end{pgfscope}%
\begin{pgfscope}%
\pgfsys@transformshift{2.077386in}{3.154068in}%
\pgfsys@useobject{currentmarker}{}%
\end{pgfscope}%
\begin{pgfscope}%
\pgfsys@transformshift{2.098276in}{3.140504in}%
\pgfsys@useobject{currentmarker}{}%
\end{pgfscope}%
\begin{pgfscope}%
\pgfsys@transformshift{2.116821in}{3.138889in}%
\pgfsys@useobject{currentmarker}{}%
\end{pgfscope}%
\begin{pgfscope}%
\pgfsys@transformshift{2.135129in}{3.143431in}%
\pgfsys@useobject{currentmarker}{}%
\end{pgfscope}%
\begin{pgfscope}%
\pgfsys@transformshift{2.156959in}{3.160163in}%
\pgfsys@useobject{currentmarker}{}%
\end{pgfscope}%
\begin{pgfscope}%
\pgfsys@transformshift{2.174564in}{3.176032in}%
\pgfsys@useobject{currentmarker}{}%
\end{pgfscope}%
\begin{pgfscope}%
\pgfsys@transformshift{2.191932in}{3.239641in}%
\pgfsys@useobject{currentmarker}{}%
\end{pgfscope}%
\begin{pgfscope}%
\pgfsys@transformshift{2.212590in}{3.406696in}%
\pgfsys@useobject{currentmarker}{}%
\end{pgfscope}%
\begin{pgfscope}%
\pgfsys@transformshift{2.235827in}{3.458347in}%
\pgfsys@useobject{currentmarker}{}%
\end{pgfscope}%
\begin{pgfscope}%
\pgfsys@transformshift{2.252728in}{3.409302in}%
\pgfsys@useobject{currentmarker}{}%
\end{pgfscope}%
\begin{pgfscope}%
\pgfsys@transformshift{2.270099in}{3.268679in}%
\pgfsys@useobject{currentmarker}{}%
\end{pgfscope}%
\begin{pgfscope}%
\pgfsys@transformshift{2.288876in}{3.248269in}%
\pgfsys@useobject{currentmarker}{}%
\end{pgfscope}%
\begin{pgfscope}%
\pgfsys@transformshift{2.309063in}{3.165088in}%
\pgfsys@useobject{currentmarker}{}%
\end{pgfscope}%
\begin{pgfscope}%
\pgfsys@transformshift{2.326902in}{3.160099in}%
\pgfsys@useobject{currentmarker}{}%
\end{pgfscope}%
\begin{pgfscope}%
\pgfsys@transformshift{2.348732in}{3.142982in}%
\pgfsys@useobject{currentmarker}{}%
\end{pgfscope}%
\begin{pgfscope}%
\pgfsys@transformshift{2.365868in}{3.138728in}%
\pgfsys@useobject{currentmarker}{}%
\end{pgfscope}%
\begin{pgfscope}%
\pgfsys@transformshift{2.386759in}{3.142279in}%
\pgfsys@useobject{currentmarker}{}%
\end{pgfscope}%
\begin{pgfscope}%
\pgfsys@transformshift{2.405067in}{3.153892in}%
\pgfsys@useobject{currentmarker}{}%
\end{pgfscope}%
\begin{pgfscope}%
\pgfsys@transformshift{2.425723in}{3.195556in}%
\pgfsys@useobject{currentmarker}{}%
\end{pgfscope}%
\begin{pgfscope}%
\pgfsys@transformshift{2.442625in}{3.296177in}%
\pgfsys@useobject{currentmarker}{}%
\end{pgfscope}%
\begin{pgfscope}%
\pgfsys@transformshift{2.460698in}{3.391442in}%
\pgfsys@useobject{currentmarker}{}%
\end{pgfscope}%
\begin{pgfscope}%
\pgfsys@transformshift{2.482294in}{3.457387in}%
\pgfsys@useobject{currentmarker}{}%
\end{pgfscope}%
\begin{pgfscope}%
\pgfsys@transformshift{2.499663in}{3.377432in}%
\pgfsys@useobject{currentmarker}{}%
\end{pgfscope}%
\begin{pgfscope}%
\pgfsys@transformshift{2.519147in}{3.228784in}%
\pgfsys@useobject{currentmarker}{}%
\end{pgfscope}%
\begin{pgfscope}%
\pgfsys@transformshift{2.539098in}{3.163103in}%
\pgfsys@useobject{currentmarker}{}%
\end{pgfscope}%
\begin{pgfscope}%
\pgfsys@transformshift{2.559519in}{3.146997in}%
\pgfsys@useobject{currentmarker}{}%
\end{pgfscope}%
\begin{pgfscope}%
\pgfsys@transformshift{2.577827in}{3.147774in}%
\pgfsys@useobject{currentmarker}{}%
\end{pgfscope}%
\begin{pgfscope}%
\pgfsys@transformshift{2.598015in}{3.139369in}%
\pgfsys@useobject{currentmarker}{}%
\end{pgfscope}%
\begin{pgfscope}%
\pgfsys@transformshift{2.615619in}{3.139469in}%
\pgfsys@useobject{currentmarker}{}%
\end{pgfscope}%
\begin{pgfscope}%
\pgfsys@transformshift{2.634164in}{3.146701in}%
\pgfsys@useobject{currentmarker}{}%
\end{pgfscope}%
\begin{pgfscope}%
\pgfsys@transformshift{2.655758in}{3.168345in}%
\pgfsys@useobject{currentmarker}{}%
\end{pgfscope}%
\begin{pgfscope}%
\pgfsys@transformshift{2.673128in}{3.200324in}%
\pgfsys@useobject{currentmarker}{}%
\end{pgfscope}%
\begin{pgfscope}%
\pgfsys@transformshift{2.691202in}{3.306937in}%
\pgfsys@useobject{currentmarker}{}%
\end{pgfscope}%
\begin{pgfscope}%
\pgfsys@transformshift{2.712094in}{3.447028in}%
\pgfsys@useobject{currentmarker}{}%
\end{pgfscope}%
\begin{pgfscope}%
\pgfsys@transformshift{2.733219in}{3.438061in}%
\pgfsys@useobject{currentmarker}{}%
\end{pgfscope}%
\begin{pgfscope}%
\pgfsys@transformshift{2.751058in}{3.333878in}%
\pgfsys@useobject{currentmarker}{}%
\end{pgfscope}%
\begin{pgfscope}%
\pgfsys@transformshift{2.770306in}{3.206344in}%
\pgfsys@useobject{currentmarker}{}%
\end{pgfscope}%
\begin{pgfscope}%
\pgfsys@transformshift{2.791197in}{3.167466in}%
\pgfsys@useobject{currentmarker}{}%
\end{pgfscope}%
\begin{pgfscope}%
\pgfsys@transformshift{2.808333in}{3.147369in}%
\pgfsys@useobject{currentmarker}{}%
\end{pgfscope}%
\begin{pgfscope}%
\pgfsys@transformshift{2.830163in}{3.139780in}%
\pgfsys@useobject{currentmarker}{}%
\end{pgfscope}%
\begin{pgfscope}%
\pgfsys@transformshift{2.848705in}{3.140312in}%
\pgfsys@useobject{currentmarker}{}%
\end{pgfscope}%
\begin{pgfscope}%
\pgfsys@transformshift{2.867015in}{3.144017in}%
\pgfsys@useobject{currentmarker}{}%
\end{pgfscope}%
\begin{pgfscope}%
\pgfsys@transformshift{2.887437in}{3.161052in}%
\pgfsys@useobject{currentmarker}{}%
\end{pgfscope}%
\begin{pgfscope}%
\pgfsys@transformshift{2.909265in}{3.214982in}%
\pgfsys@useobject{currentmarker}{}%
\end{pgfscope}%
\begin{pgfscope}%
\pgfsys@transformshift{2.923584in}{3.278016in}%
\pgfsys@useobject{currentmarker}{}%
\end{pgfscope}%
\begin{pgfscope}%
\pgfsys@transformshift{2.941658in}{3.417627in}%
\pgfsys@useobject{currentmarker}{}%
\end{pgfscope}%
\begin{pgfscope}%
\pgfsys@transformshift{2.962550in}{3.473370in}%
\pgfsys@useobject{currentmarker}{}%
\end{pgfscope}%
\begin{pgfscope}%
\pgfsys@transformshift{2.980155in}{3.417476in}%
\pgfsys@useobject{currentmarker}{}%
\end{pgfscope}%
\begin{pgfscope}%
\pgfsys@transformshift{3.001046in}{3.255287in}%
\pgfsys@useobject{currentmarker}{}%
\end{pgfscope}%
\begin{pgfscope}%
\pgfsys@transformshift{3.022405in}{3.171950in}%
\pgfsys@useobject{currentmarker}{}%
\end{pgfscope}%
\begin{pgfscope}%
\pgfsys@transformshift{3.040244in}{3.151550in}%
\pgfsys@useobject{currentmarker}{}%
\end{pgfscope}%
\begin{pgfscope}%
\pgfsys@transformshift{3.059023in}{3.143834in}%
\pgfsys@useobject{currentmarker}{}%
\end{pgfscope}%
\begin{pgfscope}%
\pgfsys@transformshift{3.078740in}{3.139783in}%
\pgfsys@useobject{currentmarker}{}%
\end{pgfscope}%
\begin{pgfscope}%
\pgfsys@transformshift{3.097519in}{3.139893in}%
\pgfsys@useobject{currentmarker}{}%
\end{pgfscope}%
\begin{pgfscope}%
\pgfsys@transformshift{3.097284in}{3.139814in}%
\pgfsys@useobject{currentmarker}{}%
\end{pgfscope}%
\begin{pgfscope}%
\pgfsys@transformshift{3.114655in}{3.139752in}%
\pgfsys@useobject{currentmarker}{}%
\end{pgfscope}%
\begin{pgfscope}%
\pgfsys@transformshift{3.136250in}{3.146950in}%
\pgfsys@useobject{currentmarker}{}%
\end{pgfscope}%
\begin{pgfscope}%
\pgfsys@transformshift{3.155498in}{3.170471in}%
\pgfsys@useobject{currentmarker}{}%
\end{pgfscope}%
\begin{pgfscope}%
\pgfsys@transformshift{3.172398in}{3.217383in}%
\pgfsys@useobject{currentmarker}{}%
\end{pgfscope}%
\begin{pgfscope}%
\pgfsys@transformshift{3.194228in}{3.351165in}%
\pgfsys@useobject{currentmarker}{}%
\end{pgfscope}%
\begin{pgfscope}%
\pgfsys@transformshift{3.212301in}{3.461879in}%
\pgfsys@useobject{currentmarker}{}%
\end{pgfscope}%
\begin{pgfscope}%
\pgfsys@transformshift{3.235540in}{3.460963in}%
\pgfsys@useobject{currentmarker}{}%
\end{pgfscope}%
\begin{pgfscope}%
\pgfsys@transformshift{3.249859in}{3.431278in}%
\pgfsys@useobject{currentmarker}{}%
\end{pgfscope}%
\begin{pgfscope}%
\pgfsys@transformshift{3.268167in}{3.318510in}%
\pgfsys@useobject{currentmarker}{}%
\end{pgfscope}%
\begin{pgfscope}%
\pgfsys@transformshift{3.289528in}{3.196741in}%
\pgfsys@useobject{currentmarker}{}%
\end{pgfscope}%
\begin{pgfscope}%
\pgfsys@transformshift{3.308540in}{3.159187in}%
\pgfsys@useobject{currentmarker}{}%
\end{pgfscope}%
\begin{pgfscope}%
\pgfsys@transformshift{3.327319in}{3.143546in}%
\pgfsys@useobject{currentmarker}{}%
\end{pgfscope}%
\begin{pgfscope}%
\pgfsys@transformshift{3.347740in}{3.139707in}%
\pgfsys@useobject{currentmarker}{}%
\end{pgfscope}%
\begin{pgfscope}%
\pgfsys@transformshift{3.366048in}{3.141780in}%
\pgfsys@useobject{currentmarker}{}%
\end{pgfscope}%
\begin{pgfscope}%
\pgfsys@transformshift{3.385298in}{3.148200in}%
\pgfsys@useobject{currentmarker}{}%
\end{pgfscope}%
\begin{pgfscope}%
\pgfsys@transformshift{3.403841in}{3.161187in}%
\pgfsys@useobject{currentmarker}{}%
\end{pgfscope}%
\begin{pgfscope}%
\pgfsys@transformshift{3.422385in}{3.194569in}%
\pgfsys@useobject{currentmarker}{}%
\end{pgfscope}%
\begin{pgfscope}%
\pgfsys@transformshift{3.442101in}{3.279206in}%
\pgfsys@useobject{currentmarker}{}%
\end{pgfscope}%
\begin{pgfscope}%
\pgfsys@transformshift{3.460880in}{3.410965in}%
\pgfsys@useobject{currentmarker}{}%
\end{pgfscope}%
\begin{pgfscope}%
\pgfsys@transformshift{3.481302in}{3.492292in}%
\pgfsys@useobject{currentmarker}{}%
\end{pgfscope}%
\begin{pgfscope}%
\pgfsys@transformshift{3.499376in}{3.462412in}%
\pgfsys@useobject{currentmarker}{}%
\end{pgfscope}%
\begin{pgfscope}%
\pgfsys@transformshift{3.513695in}{3.338292in}%
\pgfsys@useobject{currentmarker}{}%
\end{pgfscope}%
\begin{pgfscope}%
\pgfsys@transformshift{3.537871in}{3.278386in}%
\pgfsys@useobject{currentmarker}{}%
\end{pgfscope}%
\begin{pgfscope}%
\pgfsys@transformshift{3.558293in}{3.190673in}%
\pgfsys@useobject{currentmarker}{}%
\end{pgfscope}%
\begin{pgfscope}%
\pgfsys@transformshift{3.575663in}{3.164847in}%
\pgfsys@useobject{currentmarker}{}%
\end{pgfscope}%
\begin{pgfscope}%
\pgfsys@transformshift{3.596788in}{3.147552in}%
\pgfsys@useobject{currentmarker}{}%
\end{pgfscope}%
\begin{pgfscope}%
\pgfsys@transformshift{3.612984in}{3.141184in}%
\pgfsys@useobject{currentmarker}{}%
\end{pgfscope}%
\begin{pgfscope}%
\pgfsys@transformshift{3.632701in}{3.158596in}%
\pgfsys@useobject{currentmarker}{}%
\end{pgfscope}%
\begin{pgfscope}%
\pgfsys@transformshift{3.654062in}{3.144591in}%
\pgfsys@useobject{currentmarker}{}%
\end{pgfscope}%
\begin{pgfscope}%
\pgfsys@transformshift{3.673310in}{3.141053in}%
\pgfsys@useobject{currentmarker}{}%
\end{pgfscope}%
\begin{pgfscope}%
\pgfsys@transformshift{3.691384in}{3.141888in}%
\pgfsys@useobject{currentmarker}{}%
\end{pgfscope}%
\begin{pgfscope}%
\pgfsys@transformshift{3.713214in}{3.154095in}%
\pgfsys@useobject{currentmarker}{}%
\end{pgfscope}%
\begin{pgfscope}%
\pgfsys@transformshift{3.730584in}{3.177896in}%
\pgfsys@useobject{currentmarker}{}%
\end{pgfscope}%
\begin{pgfscope}%
\pgfsys@transformshift{3.747955in}{3.222975in}%
\pgfsys@useobject{currentmarker}{}%
\end{pgfscope}%
\begin{pgfscope}%
\pgfsys@transformshift{3.769080in}{3.367034in}%
\pgfsys@useobject{currentmarker}{}%
\end{pgfscope}%
\begin{pgfscope}%
\pgfsys@transformshift{3.789736in}{3.502014in}%
\pgfsys@useobject{currentmarker}{}%
\end{pgfscope}%
\begin{pgfscope}%
\pgfsys@transformshift{3.808983in}{3.493043in}%
\pgfsys@useobject{currentmarker}{}%
\end{pgfscope}%
\begin{pgfscope}%
\pgfsys@transformshift{3.826119in}{3.421083in}%
\pgfsys@useobject{currentmarker}{}%
\end{pgfscope}%
\begin{pgfscope}%
\pgfsys@transformshift{3.847010in}{3.276042in}%
\pgfsys@useobject{currentmarker}{}%
\end{pgfscope}%
\begin{pgfscope}%
\pgfsys@transformshift{3.865789in}{3.212729in}%
\pgfsys@useobject{currentmarker}{}%
\end{pgfscope}%
\begin{pgfscope}%
\pgfsys@transformshift{3.886445in}{3.163968in}%
\pgfsys@useobject{currentmarker}{}%
\end{pgfscope}%
\begin{pgfscope}%
\pgfsys@transformshift{3.902876in}{3.148388in}%
\pgfsys@useobject{currentmarker}{}%
\end{pgfscope}%
\begin{pgfscope}%
\pgfsys@transformshift{3.922123in}{3.141794in}%
\pgfsys@useobject{currentmarker}{}%
\end{pgfscope}%
\begin{pgfscope}%
\pgfsys@transformshift{3.942074in}{3.141778in}%
\pgfsys@useobject{currentmarker}{}%
\end{pgfscope}%
\begin{pgfscope}%
\pgfsys@transformshift{3.958976in}{3.147825in}%
\pgfsys@useobject{currentmarker}{}%
\end{pgfscope}%
\begin{pgfscope}%
\pgfsys@transformshift{3.979866in}{3.160899in}%
\pgfsys@useobject{currentmarker}{}%
\end{pgfscope}%
\begin{pgfscope}%
\pgfsys@transformshift{3.997706in}{3.182789in}%
\pgfsys@useobject{currentmarker}{}%
\end{pgfscope}%
\begin{pgfscope}%
\pgfsys@transformshift{4.018596in}{3.258053in}%
\pgfsys@useobject{currentmarker}{}%
\end{pgfscope}%
\begin{pgfscope}%
\pgfsys@transformshift{4.037141in}{3.386543in}%
\pgfsys@useobject{currentmarker}{}%
\end{pgfscope}%
\begin{pgfscope}%
\pgfsys@transformshift{4.058266in}{3.518643in}%
\pgfsys@useobject{currentmarker}{}%
\end{pgfscope}%
\begin{pgfscope}%
\pgfsys@transformshift{4.075636in}{3.162396in}%
\pgfsys@useobject{currentmarker}{}%
\end{pgfscope}%
\begin{pgfscope}%
\pgfsys@transformshift{4.093946in}{3.204504in}%
\pgfsys@useobject{currentmarker}{}%
\end{pgfscope}%
\begin{pgfscope}%
\pgfsys@transformshift{4.114131in}{3.345595in}%
\pgfsys@useobject{currentmarker}{}%
\end{pgfscope}%
\begin{pgfscope}%
\pgfsys@transformshift{4.135024in}{3.491848in}%
\pgfsys@useobject{currentmarker}{}%
\end{pgfscope}%
\begin{pgfscope}%
\pgfsys@transformshift{4.153801in}{3.530904in}%
\pgfsys@useobject{currentmarker}{}%
\end{pgfscope}%
\begin{pgfscope}%
\pgfsys@transformshift{4.172345in}{3.471770in}%
\pgfsys@useobject{currentmarker}{}%
\end{pgfscope}%
\begin{pgfscope}%
\pgfsys@transformshift{4.193001in}{3.316078in}%
\pgfsys@useobject{currentmarker}{}%
\end{pgfscope}%
\begin{pgfscope}%
\pgfsys@transformshift{4.209198in}{3.218418in}%
\pgfsys@useobject{currentmarker}{}%
\end{pgfscope}%
\begin{pgfscope}%
\pgfsys@transformshift{4.229619in}{3.167611in}%
\pgfsys@useobject{currentmarker}{}%
\end{pgfscope}%
\begin{pgfscope}%
\pgfsys@transformshift{4.250744in}{3.153658in}%
\pgfsys@useobject{currentmarker}{}%
\end{pgfscope}%
\begin{pgfscope}%
\pgfsys@transformshift{4.268584in}{3.145388in}%
\pgfsys@useobject{currentmarker}{}%
\end{pgfscope}%
\begin{pgfscope}%
\pgfsys@transformshift{4.289474in}{3.141626in}%
\pgfsys@useobject{currentmarker}{}%
\end{pgfscope}%
\begin{pgfscope}%
\pgfsys@transformshift{4.306844in}{3.146611in}%
\pgfsys@useobject{currentmarker}{}%
\end{pgfscope}%
\begin{pgfscope}%
\pgfsys@transformshift{4.326327in}{3.158682in}%
\pgfsys@useobject{currentmarker}{}%
\end{pgfscope}%
\begin{pgfscope}%
\pgfsys@transformshift{4.346983in}{3.198704in}%
\pgfsys@useobject{currentmarker}{}%
\end{pgfscope}%
\begin{pgfscope}%
\pgfsys@transformshift{4.363650in}{3.277718in}%
\pgfsys@useobject{currentmarker}{}%
\end{pgfscope}%
\begin{pgfscope}%
\pgfsys@transformshift{4.384775in}{3.422955in}%
\pgfsys@useobject{currentmarker}{}%
\end{pgfscope}%
\begin{pgfscope}%
\pgfsys@transformshift{4.404023in}{3.543478in}%
\pgfsys@useobject{currentmarker}{}%
\end{pgfscope}%
\begin{pgfscope}%
\pgfsys@transformshift{4.422096in}{3.531686in}%
\pgfsys@useobject{currentmarker}{}%
\end{pgfscope}%
\begin{pgfscope}%
\pgfsys@transformshift{4.444161in}{3.529942in}%
\pgfsys@useobject{currentmarker}{}%
\end{pgfscope}%
\begin{pgfscope}%
\pgfsys@transformshift{4.462000in}{3.408686in}%
\pgfsys@useobject{currentmarker}{}%
\end{pgfscope}%
\begin{pgfscope}%
\pgfsys@transformshift{4.479841in}{3.269025in}%
\pgfsys@useobject{currentmarker}{}%
\end{pgfscope}%
\begin{pgfscope}%
\pgfsys@transformshift{4.479841in}{3.269550in}%
\pgfsys@useobject{currentmarker}{}%
\end{pgfscope}%
\begin{pgfscope}%
\pgfsys@transformshift{4.470685in}{3.358129in}%
\pgfsys@useobject{currentmarker}{}%
\end{pgfscope}%
\begin{pgfscope}%
\pgfsys@transformshift{4.454489in}{3.506521in}%
\pgfsys@useobject{currentmarker}{}%
\end{pgfscope}%
\begin{pgfscope}%
\pgfsys@transformshift{4.434538in}{3.540579in}%
\pgfsys@useobject{currentmarker}{}%
\end{pgfscope}%
\begin{pgfscope}%
\pgfsys@transformshift{4.414351in}{3.335825in}%
\pgfsys@useobject{currentmarker}{}%
\end{pgfscope}%
\begin{pgfscope}%
\pgfsys@transformshift{4.396511in}{3.199941in}%
\pgfsys@useobject{currentmarker}{}%
\end{pgfscope}%
\begin{pgfscope}%
\pgfsys@transformshift{4.377029in}{3.156281in}%
\pgfsys@useobject{currentmarker}{}%
\end{pgfscope}%
\begin{pgfscope}%
\pgfsys@transformshift{4.359190in}{3.142279in}%
\pgfsys@useobject{currentmarker}{}%
\end{pgfscope}%
\begin{pgfscope}%
\pgfsys@transformshift{4.340177in}{3.144193in}%
\pgfsys@useobject{currentmarker}{}%
\end{pgfscope}%
\begin{pgfscope}%
\pgfsys@transformshift{4.318581in}{3.165720in}%
\pgfsys@useobject{currentmarker}{}%
\end{pgfscope}%
\begin{pgfscope}%
\pgfsys@transformshift{4.300507in}{3.229128in}%
\pgfsys@useobject{currentmarker}{}%
\end{pgfscope}%
\begin{pgfscope}%
\pgfsys@transformshift{4.281494in}{3.394694in}%
\pgfsys@useobject{currentmarker}{}%
\end{pgfscope}%
\begin{pgfscope}%
\pgfsys@transformshift{4.264829in}{3.519177in}%
\pgfsys@useobject{currentmarker}{}%
\end{pgfscope}%
\begin{pgfscope}%
\pgfsys@transformshift{4.244173in}{3.464980in}%
\pgfsys@useobject{currentmarker}{}%
\end{pgfscope}%
\begin{pgfscope}%
\pgfsys@transformshift{4.223280in}{3.253121in}%
\pgfsys@useobject{currentmarker}{}%
\end{pgfscope}%
\begin{pgfscope}%
\pgfsys@transformshift{4.205441in}{3.174923in}%
\pgfsys@useobject{currentmarker}{}%
\end{pgfscope}%
\begin{pgfscope}%
\pgfsys@transformshift{4.186899in}{3.147848in}%
\pgfsys@useobject{currentmarker}{}%
\end{pgfscope}%
\begin{pgfscope}%
\pgfsys@transformshift{4.167415in}{3.140675in}%
\pgfsys@useobject{currentmarker}{}%
\end{pgfscope}%
\begin{pgfscope}%
\pgfsys@transformshift{4.146995in}{3.149807in}%
\pgfsys@useobject{currentmarker}{}%
\end{pgfscope}%
\begin{pgfscope}%
\pgfsys@transformshift{4.131033in}{3.171404in}%
\pgfsys@useobject{currentmarker}{}%
\end{pgfscope}%
\begin{pgfscope}%
\pgfsys@transformshift{4.109203in}{3.308253in}%
\pgfsys@useobject{currentmarker}{}%
\end{pgfscope}%
\begin{pgfscope}%
\pgfsys@transformshift{4.090658in}{3.460848in}%
\pgfsys@useobject{currentmarker}{}%
\end{pgfscope}%
\begin{pgfscope}%
\pgfsys@transformshift{4.070002in}{3.508545in}%
\pgfsys@useobject{currentmarker}{}%
\end{pgfscope}%
\begin{pgfscope}%
\pgfsys@transformshift{4.049817in}{3.332078in}%
\pgfsys@useobject{currentmarker}{}%
\end{pgfscope}%
\begin{pgfscope}%
\pgfsys@transformshift{4.033620in}{3.202741in}%
\pgfsys@useobject{currentmarker}{}%
\end{pgfscope}%
\begin{pgfscope}%
\pgfsys@transformshift{4.012025in}{3.153851in}%
\pgfsys@useobject{currentmarker}{}%
\end{pgfscope}%
\begin{pgfscope}%
\pgfsys@transformshift{3.995123in}{3.141748in}%
\pgfsys@useobject{currentmarker}{}%
\end{pgfscope}%
\begin{pgfscope}%
\pgfsys@transformshift{3.974703in}{3.141991in}%
\pgfsys@useobject{currentmarker}{}%
\end{pgfscope}%
\begin{pgfscope}%
\pgfsys@transformshift{3.954282in}{3.143853in}%
\pgfsys@useobject{currentmarker}{}%
\end{pgfscope}%
\begin{pgfscope}%
\pgfsys@transformshift{3.934563in}{3.150772in}%
\pgfsys@useobject{currentmarker}{}%
\end{pgfscope}%
\begin{pgfscope}%
\pgfsys@transformshift{3.916255in}{3.184449in}%
\pgfsys@useobject{currentmarker}{}%
\end{pgfscope}%
\begin{pgfscope}%
\pgfsys@transformshift{3.896302in}{3.322825in}%
\pgfsys@useobject{currentmarker}{}%
\end{pgfscope}%
\begin{pgfscope}%
\pgfsys@transformshift{3.877289in}{3.467786in}%
\pgfsys@useobject{currentmarker}{}%
\end{pgfscope}%
\begin{pgfscope}%
\pgfsys@transformshift{3.858746in}{3.479259in}%
\pgfsys@useobject{currentmarker}{}%
\end{pgfscope}%
\begin{pgfscope}%
\pgfsys@transformshift{3.840673in}{3.328377in}%
\pgfsys@useobject{currentmarker}{}%
\end{pgfscope}%
\begin{pgfscope}%
\pgfsys@transformshift{3.821425in}{3.190513in}%
\pgfsys@useobject{currentmarker}{}%
\end{pgfscope}%
\begin{pgfscope}%
\pgfsys@transformshift{3.797481in}{3.148234in}%
\pgfsys@useobject{currentmarker}{}%
\end{pgfscope}%
\begin{pgfscope}%
\pgfsys@transformshift{3.783164in}{3.141173in}%
\pgfsys@useobject{currentmarker}{}%
\end{pgfscope}%
\begin{pgfscope}%
\pgfsys@transformshift{3.761334in}{3.141497in}%
\pgfsys@useobject{currentmarker}{}%
\end{pgfscope}%
\begin{pgfscope}%
\pgfsys@transformshift{3.744432in}{3.152114in}%
\pgfsys@useobject{currentmarker}{}%
\end{pgfscope}%
\begin{pgfscope}%
\pgfsys@transformshift{3.724011in}{3.197028in}%
\pgfsys@useobject{currentmarker}{}%
\end{pgfscope}%
\begin{pgfscope}%
\pgfsys@transformshift{3.706642in}{3.293247in}%
\pgfsys@useobject{currentmarker}{}%
\end{pgfscope}%
\begin{pgfscope}%
\pgfsys@transformshift{3.685986in}{3.438923in}%
\pgfsys@useobject{currentmarker}{}%
\end{pgfscope}%
\begin{pgfscope}%
\pgfsys@transformshift{3.666738in}{3.484555in}%
\pgfsys@useobject{currentmarker}{}%
\end{pgfscope}%
\begin{pgfscope}%
\pgfsys@transformshift{3.648428in}{3.355459in}%
\pgfsys@useobject{currentmarker}{}%
\end{pgfscope}%
\begin{pgfscope}%
\pgfsys@transformshift{3.627538in}{3.198530in}%
\pgfsys@useobject{currentmarker}{}%
\end{pgfscope}%
\begin{pgfscope}%
\pgfsys@transformshift{3.609699in}{3.157487in}%
\pgfsys@useobject{currentmarker}{}%
\end{pgfscope}%
\begin{pgfscope}%
\pgfsys@transformshift{3.592563in}{3.143233in}%
\pgfsys@useobject{currentmarker}{}%
\end{pgfscope}%
\begin{pgfscope}%
\pgfsys@transformshift{3.573549in}{3.139834in}%
\pgfsys@useobject{currentmarker}{}%
\end{pgfscope}%
\begin{pgfscope}%
\pgfsys@transformshift{3.551250in}{3.145579in}%
\pgfsys@useobject{currentmarker}{}%
\end{pgfscope}%
\begin{pgfscope}%
\pgfsys@transformshift{3.530829in}{3.162611in}%
\pgfsys@useobject{currentmarker}{}%
\end{pgfscope}%
\begin{pgfscope}%
\pgfsys@transformshift{3.512755in}{3.213059in}%
\pgfsys@useobject{currentmarker}{}%
\end{pgfscope}%
\begin{pgfscope}%
\pgfsys@transformshift{3.496559in}{3.337649in}%
\pgfsys@useobject{currentmarker}{}%
\end{pgfscope}%
\begin{pgfscope}%
\pgfsys@transformshift{3.474729in}{3.469190in}%
\pgfsys@useobject{currentmarker}{}%
\end{pgfscope}%
\begin{pgfscope}%
\pgfsys@transformshift{3.455012in}{3.464318in}%
\pgfsys@useobject{currentmarker}{}%
\end{pgfscope}%
\begin{pgfscope}%
\pgfsys@transformshift{3.437407in}{3.296196in}%
\pgfsys@useobject{currentmarker}{}%
\end{pgfscope}%
\begin{pgfscope}%
\pgfsys@transformshift{3.418863in}{3.198243in}%
\pgfsys@useobject{currentmarker}{}%
\end{pgfscope}%
\begin{pgfscope}%
\pgfsys@transformshift{3.398207in}{3.158359in}%
\pgfsys@useobject{currentmarker}{}%
\end{pgfscope}%
\begin{pgfscope}%
\pgfsys@transformshift{3.376142in}{3.142567in}%
\pgfsys@useobject{currentmarker}{}%
\end{pgfscope}%
\begin{pgfscope}%
\pgfsys@transformshift{3.359008in}{3.139380in}%
\pgfsys@useobject{currentmarker}{}%
\end{pgfscope}%
\begin{pgfscope}%
\pgfsys@transformshift{3.340698in}{3.141109in}%
\pgfsys@useobject{currentmarker}{}%
\end{pgfscope}%
\begin{pgfscope}%
\pgfsys@transformshift{3.320042in}{3.146407in}%
\pgfsys@useobject{currentmarker}{}%
\end{pgfscope}%
\begin{pgfscope}%
\pgfsys@transformshift{3.298683in}{3.175237in}%
\pgfsys@useobject{currentmarker}{}%
\end{pgfscope}%
\begin{pgfscope}%
\pgfsys@transformshift{3.283660in}{3.220264in}%
\pgfsys@useobject{currentmarker}{}%
\end{pgfscope}%
\begin{pgfscope}%
\pgfsys@transformshift{3.263707in}{3.388238in}%
\pgfsys@useobject{currentmarker}{}%
\end{pgfscope}%
\begin{pgfscope}%
\pgfsys@transformshift{3.244225in}{3.477005in}%
\pgfsys@useobject{currentmarker}{}%
\end{pgfscope}%
\begin{pgfscope}%
\pgfsys@transformshift{3.226386in}{3.441842in}%
\pgfsys@useobject{currentmarker}{}%
\end{pgfscope}%
\begin{pgfscope}%
\pgfsys@transformshift{3.207842in}{3.383561in}%
\pgfsys@useobject{currentmarker}{}%
\end{pgfscope}%
\begin{pgfscope}%
\pgfsys@transformshift{3.186246in}{3.230579in}%
\pgfsys@useobject{currentmarker}{}%
\end{pgfscope}%
\begin{pgfscope}%
\pgfsys@transformshift{3.165826in}{3.164959in}%
\pgfsys@useobject{currentmarker}{}%
\end{pgfscope}%
\begin{pgfscope}%
\pgfsys@transformshift{3.147516in}{3.145570in}%
\pgfsys@useobject{currentmarker}{}%
\end{pgfscope}%
\begin{pgfscope}%
\pgfsys@transformshift{3.130382in}{3.139607in}%
\pgfsys@useobject{currentmarker}{}%
\end{pgfscope}%
\begin{pgfscope}%
\pgfsys@transformshift{3.107143in}{3.141653in}%
\pgfsys@useobject{currentmarker}{}%
\end{pgfscope}%
\begin{pgfscope}%
\pgfsys@transformshift{3.090476in}{3.146385in}%
\pgfsys@useobject{currentmarker}{}%
\end{pgfscope}%
\begin{pgfscope}%
\pgfsys@transformshift{3.069820in}{3.169119in}%
\pgfsys@useobject{currentmarker}{}%
\end{pgfscope}%
\begin{pgfscope}%
\pgfsys@transformshift{3.051278in}{3.247172in}%
\pgfsys@useobject{currentmarker}{}%
\end{pgfscope}%
\begin{pgfscope}%
\pgfsys@transformshift{3.033907in}{3.380060in}%
\pgfsys@useobject{currentmarker}{}%
\end{pgfscope}%
\begin{pgfscope}%
\pgfsys@transformshift{3.012548in}{3.471856in}%
\pgfsys@useobject{currentmarker}{}%
\end{pgfscope}%
\begin{pgfscope}%
\pgfsys@transformshift{2.991892in}{3.395232in}%
\pgfsys@useobject{currentmarker}{}%
\end{pgfscope}%
\begin{pgfscope}%
\pgfsys@transformshift{2.973582in}{3.240151in}%
\pgfsys@useobject{currentmarker}{}%
\end{pgfscope}%
\begin{pgfscope}%
\pgfsys@transformshift{2.955977in}{3.175410in}%
\pgfsys@useobject{currentmarker}{}%
\end{pgfscope}%
\begin{pgfscope}%
\pgfsys@transformshift{2.933912in}{3.147137in}%
\pgfsys@useobject{currentmarker}{}%
\end{pgfscope}%
\begin{pgfscope}%
\pgfsys@transformshift{2.918185in}{3.141190in}%
\pgfsys@useobject{currentmarker}{}%
\end{pgfscope}%
\begin{pgfscope}%
\pgfsys@transformshift{2.897060in}{3.139911in}%
\pgfsys@useobject{currentmarker}{}%
\end{pgfscope}%
\begin{pgfscope}%
\pgfsys@transformshift{2.878752in}{3.145073in}%
\pgfsys@useobject{currentmarker}{}%
\end{pgfscope}%
\begin{pgfscope}%
\pgfsys@transformshift{2.859504in}{3.158567in}%
\pgfsys@useobject{currentmarker}{}%
\end{pgfscope}%
\begin{pgfscope}%
\pgfsys@transformshift{2.841194in}{3.206625in}%
\pgfsys@useobject{currentmarker}{}%
\end{pgfscope}%
\begin{pgfscope}%
\pgfsys@transformshift{2.823824in}{3.317600in}%
\pgfsys@useobject{currentmarker}{}%
\end{pgfscope}%
\begin{pgfscope}%
\pgfsys@transformshift{2.804107in}{3.427407in}%
\pgfsys@useobject{currentmarker}{}%
\end{pgfscope}%
\begin{pgfscope}%
\pgfsys@transformshift{2.781808in}{3.464424in}%
\pgfsys@useobject{currentmarker}{}%
\end{pgfscope}%
\begin{pgfscope}%
\pgfsys@transformshift{2.762795in}{3.394467in}%
\pgfsys@useobject{currentmarker}{}%
\end{pgfscope}%
\begin{pgfscope}%
\pgfsys@transformshift{2.743547in}{3.242688in}%
\pgfsys@useobject{currentmarker}{}%
\end{pgfscope}%
\begin{pgfscope}%
\pgfsys@transformshift{2.725474in}{3.176845in}%
\pgfsys@useobject{currentmarker}{}%
\end{pgfscope}%
\begin{pgfscope}%
\pgfsys@transformshift{2.706929in}{3.406206in}%
\pgfsys@useobject{currentmarker}{}%
\end{pgfscope}%
\begin{pgfscope}%
\pgfsys@transformshift{2.688621in}{3.392350in}%
\pgfsys@useobject{currentmarker}{}%
\end{pgfscope}%
\begin{pgfscope}%
\pgfsys@transformshift{2.666557in}{3.226724in}%
\pgfsys@useobject{currentmarker}{}%
\end{pgfscope}%
\begin{pgfscope}%
\pgfsys@transformshift{2.647543in}{3.164980in}%
\pgfsys@useobject{currentmarker}{}%
\end{pgfscope}%
\begin{pgfscope}%
\pgfsys@transformshift{2.629468in}{3.146797in}%
\pgfsys@useobject{currentmarker}{}%
\end{pgfscope}%
\begin{pgfscope}%
\pgfsys@transformshift{2.610456in}{3.141056in}%
\pgfsys@useobject{currentmarker}{}%
\end{pgfscope}%
\begin{pgfscope}%
\pgfsys@transformshift{2.590972in}{3.138842in}%
\pgfsys@useobject{currentmarker}{}%
\end{pgfscope}%
\begin{pgfscope}%
\pgfsys@transformshift{2.569144in}{3.145708in}%
\pgfsys@useobject{currentmarker}{}%
\end{pgfscope}%
\begin{pgfscope}%
\pgfsys@transformshift{2.551069in}{3.143454in}%
\pgfsys@useobject{currentmarker}{}%
\end{pgfscope}%
\begin{pgfscope}%
\pgfsys@transformshift{2.532526in}{3.138930in}%
\pgfsys@useobject{currentmarker}{}%
\end{pgfscope}%
\begin{pgfscope}%
\pgfsys@transformshift{2.515156in}{3.142557in}%
\pgfsys@useobject{currentmarker}{}%
\end{pgfscope}%
\begin{pgfscope}%
\pgfsys@transformshift{2.496846in}{3.157892in}%
\pgfsys@useobject{currentmarker}{}%
\end{pgfscope}%
\begin{pgfscope}%
\pgfsys@transformshift{2.476895in}{3.181524in}%
\pgfsys@useobject{currentmarker}{}%
\end{pgfscope}%
\begin{pgfscope}%
\pgfsys@transformshift{2.459055in}{3.281074in}%
\pgfsys@useobject{currentmarker}{}%
\end{pgfscope}%
\begin{pgfscope}%
\pgfsys@transformshift{2.436286in}{3.436554in}%
\pgfsys@useobject{currentmarker}{}%
\end{pgfscope}%
\begin{pgfscope}%
\pgfsys@transformshift{2.418917in}{3.454881in}%
\pgfsys@useobject{currentmarker}{}%
\end{pgfscope}%
\begin{pgfscope}%
\pgfsys@transformshift{2.399433in}{3.309889in}%
\pgfsys@useobject{currentmarker}{}%
\end{pgfscope}%
\begin{pgfscope}%
\pgfsys@transformshift{2.378543in}{3.189116in}%
\pgfsys@useobject{currentmarker}{}%
\end{pgfscope}%
\begin{pgfscope}%
\pgfsys@transformshift{2.359529in}{3.155416in}%
\pgfsys@useobject{currentmarker}{}%
\end{pgfscope}%
\begin{pgfscope}%
\pgfsys@transformshift{2.340047in}{3.141969in}%
\pgfsys@useobject{currentmarker}{}%
\end{pgfscope}%
\begin{pgfscope}%
\pgfsys@transformshift{2.321503in}{3.138657in}%
\pgfsys@useobject{currentmarker}{}%
\end{pgfscope}%
\begin{pgfscope}%
\pgfsys@transformshift{2.303664in}{3.141859in}%
\pgfsys@useobject{currentmarker}{}%
\end{pgfscope}%
\begin{pgfscope}%
\pgfsys@transformshift{2.285121in}{3.152611in}%
\pgfsys@useobject{currentmarker}{}%
\end{pgfscope}%
\begin{pgfscope}%
\pgfsys@transformshift{2.261414in}{3.211451in}%
\pgfsys@useobject{currentmarker}{}%
\end{pgfscope}%
\begin{pgfscope}%
\pgfsys@transformshift{2.244512in}{3.334902in}%
\pgfsys@useobject{currentmarker}{}%
\end{pgfscope}%
\begin{pgfscope}%
\pgfsys@transformshift{2.225264in}{3.451701in}%
\pgfsys@useobject{currentmarker}{}%
\end{pgfscope}%
\begin{pgfscope}%
\pgfsys@transformshift{2.207425in}{3.458379in}%
\pgfsys@useobject{currentmarker}{}%
\end{pgfscope}%
\begin{pgfscope}%
\pgfsys@transformshift{2.188178in}{3.327355in}%
\pgfsys@useobject{currentmarker}{}%
\end{pgfscope}%
\begin{pgfscope}%
\pgfsys@transformshift{2.166113in}{3.209233in}%
\pgfsys@useobject{currentmarker}{}%
\end{pgfscope}%
\begin{pgfscope}%
\pgfsys@transformshift{2.148274in}{3.163274in}%
\pgfsys@useobject{currentmarker}{}%
\end{pgfscope}%
\begin{pgfscope}%
\pgfsys@transformshift{2.129026in}{3.149502in}%
\pgfsys@useobject{currentmarker}{}%
\end{pgfscope}%
\begin{pgfscope}%
\pgfsys@transformshift{2.111187in}{3.141818in}%
\pgfsys@useobject{currentmarker}{}%
\end{pgfscope}%
\begin{pgfscope}%
\pgfsys@transformshift{2.092408in}{3.139302in}%
\pgfsys@useobject{currentmarker}{}%
\end{pgfscope}%
\begin{pgfscope}%
\pgfsys@transformshift{2.070812in}{3.143815in}%
\pgfsys@useobject{currentmarker}{}%
\end{pgfscope}%
\begin{pgfscope}%
\pgfsys@transformshift{2.052504in}{3.159912in}%
\pgfsys@useobject{currentmarker}{}%
\end{pgfscope}%
\begin{pgfscope}%
\pgfsys@transformshift{2.034194in}{3.210263in}%
\pgfsys@useobject{currentmarker}{}%
\end{pgfscope}%
\begin{pgfscope}%
\pgfsys@transformshift{2.014712in}{3.339054in}%
\pgfsys@useobject{currentmarker}{}%
\end{pgfscope}%
\begin{pgfscope}%
\pgfsys@transformshift{1.996638in}{3.449333in}%
\pgfsys@useobject{currentmarker}{}%
\end{pgfscope}%
\begin{pgfscope}%
\pgfsys@transformshift{1.974574in}{3.449849in}%
\pgfsys@useobject{currentmarker}{}%
\end{pgfscope}%
\begin{pgfscope}%
\pgfsys@transformshift{1.956029in}{3.330419in}%
\pgfsys@useobject{currentmarker}{}%
\end{pgfscope}%
\begin{pgfscope}%
\pgfsys@transformshift{1.937721in}{3.208313in}%
\pgfsys@useobject{currentmarker}{}%
\end{pgfscope}%
\begin{pgfscope}%
\pgfsys@transformshift{1.918708in}{3.162961in}%
\pgfsys@useobject{currentmarker}{}%
\end{pgfscope}%
\begin{pgfscope}%
\pgfsys@transformshift{1.897818in}{3.144839in}%
\pgfsys@useobject{currentmarker}{}%
\end{pgfscope}%
\begin{pgfscope}%
\pgfsys@transformshift{1.878804in}{3.139680in}%
\pgfsys@useobject{currentmarker}{}%
\end{pgfscope}%
\begin{pgfscope}%
\pgfsys@transformshift{1.861903in}{3.140586in}%
\pgfsys@useobject{currentmarker}{}%
\end{pgfscope}%
\begin{pgfscope}%
\pgfsys@transformshift{1.843595in}{3.145837in}%
\pgfsys@useobject{currentmarker}{}%
\end{pgfscope}%
\begin{pgfscope}%
\pgfsys@transformshift{1.821530in}{3.164704in}%
\pgfsys@useobject{currentmarker}{}%
\end{pgfscope}%
\begin{pgfscope}%
\pgfsys@transformshift{1.802517in}{3.220388in}%
\pgfsys@useobject{currentmarker}{}%
\end{pgfscope}%
\begin{pgfscope}%
\pgfsys@transformshift{1.783269in}{3.344973in}%
\pgfsys@useobject{currentmarker}{}%
\end{pgfscope}%
\begin{pgfscope}%
\pgfsys@transformshift{1.765430in}{3.450390in}%
\pgfsys@useobject{currentmarker}{}%
\end{pgfscope}%
\begin{pgfscope}%
\pgfsys@transformshift{1.746417in}{3.478827in}%
\pgfsys@useobject{currentmarker}{}%
\end{pgfscope}%
\begin{pgfscope}%
\pgfsys@transformshift{1.727874in}{3.419525in}%
\pgfsys@useobject{currentmarker}{}%
\end{pgfscope}%
\begin{pgfscope}%
\pgfsys@transformshift{1.708625in}{3.288733in}%
\pgfsys@useobject{currentmarker}{}%
\end{pgfscope}%
\begin{pgfscope}%
\pgfsys@transformshift{1.687500in}{3.216886in}%
\pgfsys@useobject{currentmarker}{}%
\end{pgfscope}%
\begin{pgfscope}%
\pgfsys@transformshift{1.668486in}{3.165559in}%
\pgfsys@useobject{currentmarker}{}%
\end{pgfscope}%
\begin{pgfscope}%
\pgfsys@transformshift{1.650178in}{3.149404in}%
\pgfsys@useobject{currentmarker}{}%
\end{pgfscope}%
\begin{pgfscope}%
\pgfsys@transformshift{1.631634in}{3.140962in}%
\pgfsys@useobject{currentmarker}{}%
\end{pgfscope}%
\begin{pgfscope}%
\pgfsys@transformshift{1.607223in}{3.141117in}%
\pgfsys@useobject{currentmarker}{}%
\end{pgfscope}%
\begin{pgfscope}%
\pgfsys@transformshift{1.591261in}{3.146566in}%
\pgfsys@useobject{currentmarker}{}%
\end{pgfscope}%
\begin{pgfscope}%
\pgfsys@transformshift{1.574830in}{3.158967in}%
\pgfsys@useobject{currentmarker}{}%
\end{pgfscope}%
\begin{pgfscope}%
\pgfsys@transformshift{1.555112in}{3.196251in}%
\pgfsys@useobject{currentmarker}{}%
\end{pgfscope}%
\begin{pgfscope}%
\pgfsys@transformshift{1.535630in}{3.281959in}%
\pgfsys@useobject{currentmarker}{}%
\end{pgfscope}%
\begin{pgfscope}%
\pgfsys@transformshift{1.514739in}{3.430103in}%
\pgfsys@useobject{currentmarker}{}%
\end{pgfscope}%
\begin{pgfscope}%
\pgfsys@transformshift{1.496195in}{3.492108in}%
\pgfsys@useobject{currentmarker}{}%
\end{pgfscope}%
\begin{pgfscope}%
\pgfsys@transformshift{1.476244in}{3.444672in}%
\pgfsys@useobject{currentmarker}{}%
\end{pgfscope}%
\begin{pgfscope}%
\pgfsys@transformshift{1.455822in}{3.413867in}%
\pgfsys@useobject{currentmarker}{}%
\end{pgfscope}%
\begin{pgfscope}%
\pgfsys@transformshift{1.437749in}{3.271698in}%
\pgfsys@useobject{currentmarker}{}%
\end{pgfscope}%
\begin{pgfscope}%
\pgfsys@transformshift{1.418970in}{3.200270in}%
\pgfsys@useobject{currentmarker}{}%
\end{pgfscope}%
\begin{pgfscope}%
\pgfsys@transformshift{1.398314in}{3.162170in}%
\pgfsys@useobject{currentmarker}{}%
\end{pgfscope}%
\begin{pgfscope}%
\pgfsys@transformshift{1.379535in}{3.152159in}%
\pgfsys@useobject{currentmarker}{}%
\end{pgfscope}%
\begin{pgfscope}%
\pgfsys@transformshift{1.361461in}{3.151627in}%
\pgfsys@useobject{currentmarker}{}%
\end{pgfscope}%
\begin{pgfscope}%
\pgfsys@transformshift{1.342682in}{3.142089in}%
\pgfsys@useobject{currentmarker}{}%
\end{pgfscope}%
\begin{pgfscope}%
\pgfsys@transformshift{1.324374in}{3.140922in}%
\pgfsys@useobject{currentmarker}{}%
\end{pgfscope}%
\begin{pgfscope}%
\pgfsys@transformshift{1.299961in}{3.148859in}%
\pgfsys@useobject{currentmarker}{}%
\end{pgfscope}%
\begin{pgfscope}%
\pgfsys@transformshift{1.284939in}{3.163617in}%
\pgfsys@useobject{currentmarker}{}%
\end{pgfscope}%
\begin{pgfscope}%
\pgfsys@transformshift{1.265692in}{3.206745in}%
\pgfsys@useobject{currentmarker}{}%
\end{pgfscope}%
\begin{pgfscope}%
\pgfsys@transformshift{1.245973in}{3.175869in}%
\pgfsys@useobject{currentmarker}{}%
\end{pgfscope}%
\begin{pgfscope}%
\pgfsys@transformshift{1.228839in}{3.232807in}%
\pgfsys@useobject{currentmarker}{}%
\end{pgfscope}%
\begin{pgfscope}%
\pgfsys@transformshift{1.206304in}{3.386355in}%
\pgfsys@useobject{currentmarker}{}%
\end{pgfscope}%
\begin{pgfscope}%
\pgfsys@transformshift{1.188464in}{3.482262in}%
\pgfsys@useobject{currentmarker}{}%
\end{pgfscope}%
\begin{pgfscope}%
\pgfsys@transformshift{1.168279in}{3.500385in}%
\pgfsys@useobject{currentmarker}{}%
\end{pgfscope}%
\begin{pgfscope}%
\pgfsys@transformshift{1.150909in}{3.408639in}%
\pgfsys@useobject{currentmarker}{}%
\end{pgfscope}%
\begin{pgfscope}%
\pgfsys@transformshift{1.129784in}{3.250453in}%
\pgfsys@useobject{currentmarker}{}%
\end{pgfscope}%
\begin{pgfscope}%
\pgfsys@transformshift{1.108188in}{3.186374in}%
\pgfsys@useobject{currentmarker}{}%
\end{pgfscope}%
\begin{pgfscope}%
\pgfsys@transformshift{1.092460in}{3.162052in}%
\pgfsys@useobject{currentmarker}{}%
\end{pgfscope}%
\begin{pgfscope}%
\pgfsys@transformshift{1.074152in}{3.148296in}%
\pgfsys@useobject{currentmarker}{}%
\end{pgfscope}%
\begin{pgfscope}%
\pgfsys@transformshift{1.052322in}{3.141297in}%
\pgfsys@useobject{currentmarker}{}%
\end{pgfscope}%
\begin{pgfscope}%
\pgfsys@transformshift{1.035657in}{3.141823in}%
\pgfsys@useobject{currentmarker}{}%
\end{pgfscope}%
\begin{pgfscope}%
\pgfsys@transformshift{1.013122in}{3.151818in}%
\pgfsys@useobject{currentmarker}{}%
\end{pgfscope}%
\begin{pgfscope}%
\pgfsys@transformshift{0.995048in}{3.168184in}%
\pgfsys@useobject{currentmarker}{}%
\end{pgfscope}%
\begin{pgfscope}%
\pgfsys@transformshift{0.975331in}{3.140907in}%
\pgfsys@useobject{currentmarker}{}%
\end{pgfscope}%
\begin{pgfscope}%
\pgfsys@transformshift{0.958664in}{3.146567in}%
\pgfsys@useobject{currentmarker}{}%
\end{pgfscope}%
\begin{pgfscope}%
\pgfsys@transformshift{0.938948in}{3.163505in}%
\pgfsys@useobject{currentmarker}{}%
\end{pgfscope}%
\begin{pgfscope}%
\pgfsys@transformshift{0.920640in}{3.205359in}%
\pgfsys@useobject{currentmarker}{}%
\end{pgfscope}%
\begin{pgfscope}%
\pgfsys@transformshift{0.901627in}{3.300298in}%
\pgfsys@useobject{currentmarker}{}%
\end{pgfscope}%
\begin{pgfscope}%
\pgfsys@transformshift{0.880265in}{3.459953in}%
\pgfsys@useobject{currentmarker}{}%
\end{pgfscope}%
\begin{pgfscope}%
\pgfsys@transformshift{0.860549in}{3.514166in}%
\pgfsys@useobject{currentmarker}{}%
\end{pgfscope}%
\begin{pgfscope}%
\pgfsys@transformshift{0.842475in}{3.508969in}%
\pgfsys@useobject{currentmarker}{}%
\end{pgfscope}%
\begin{pgfscope}%
\pgfsys@transformshift{0.825339in}{3.387020in}%
\pgfsys@useobject{currentmarker}{}%
\end{pgfscope}%
\begin{pgfscope}%
\pgfsys@transformshift{0.802569in}{3.258639in}%
\pgfsys@useobject{currentmarker}{}%
\end{pgfscope}%
\begin{pgfscope}%
\pgfsys@transformshift{0.784730in}{3.190086in}%
\pgfsys@useobject{currentmarker}{}%
\end{pgfscope}%
\begin{pgfscope}%
\pgfsys@transformshift{0.766188in}{3.159410in}%
\pgfsys@useobject{currentmarker}{}%
\end{pgfscope}%
\begin{pgfscope}%
\pgfsys@transformshift{0.747174in}{3.146878in}%
\pgfsys@useobject{currentmarker}{}%
\end{pgfscope}%
\begin{pgfscope}%
\pgfsys@transformshift{0.723936in}{3.141999in}%
\pgfsys@useobject{currentmarker}{}%
\end{pgfscope}%
\begin{pgfscope}%
\pgfsys@transformshift{0.706800in}{3.144699in}%
\pgfsys@useobject{currentmarker}{}%
\end{pgfscope}%
\begin{pgfscope}%
\pgfsys@transformshift{0.688257in}{3.152105in}%
\pgfsys@useobject{currentmarker}{}%
\end{pgfscope}%
\begin{pgfscope}%
\pgfsys@transformshift{0.669947in}{3.175493in}%
\pgfsys@useobject{currentmarker}{}%
\end{pgfscope}%
\begin{pgfscope}%
\pgfsys@transformshift{0.651405in}{3.232618in}%
\pgfsys@useobject{currentmarker}{}%
\end{pgfscope}%
\begin{pgfscope}%
\pgfsys@transformshift{0.651170in}{3.151186in}%
\pgfsys@useobject{currentmarker}{}%
\end{pgfscope}%
\begin{pgfscope}%
\pgfsys@transformshift{0.656802in}{3.146510in}%
\pgfsys@useobject{currentmarker}{}%
\end{pgfscope}%
\begin{pgfscope}%
\pgfsys@transformshift{0.674173in}{3.142244in}%
\pgfsys@useobject{currentmarker}{}%
\end{pgfscope}%
\begin{pgfscope}%
\pgfsys@transformshift{0.697646in}{3.157408in}%
\pgfsys@useobject{currentmarker}{}%
\end{pgfscope}%
\begin{pgfscope}%
\pgfsys@transformshift{0.715719in}{3.200892in}%
\pgfsys@useobject{currentmarker}{}%
\end{pgfscope}%
\begin{pgfscope}%
\pgfsys@transformshift{0.732386in}{3.328054in}%
\pgfsys@useobject{currentmarker}{}%
\end{pgfscope}%
\begin{pgfscope}%
\pgfsys@transformshift{0.751400in}{3.516989in}%
\pgfsys@useobject{currentmarker}{}%
\end{pgfscope}%
\begin{pgfscope}%
\pgfsys@transformshift{0.770413in}{3.491893in}%
\pgfsys@useobject{currentmarker}{}%
\end{pgfscope}%
\begin{pgfscope}%
\pgfsys@transformshift{0.790833in}{3.315596in}%
\pgfsys@useobject{currentmarker}{}%
\end{pgfscope}%
\begin{pgfscope}%
\pgfsys@transformshift{0.811489in}{3.181570in}%
\pgfsys@useobject{currentmarker}{}%
\end{pgfscope}%
\begin{pgfscope}%
\pgfsys@transformshift{0.830268in}{3.151042in}%
\pgfsys@useobject{currentmarker}{}%
\end{pgfscope}%
\begin{pgfscope}%
\pgfsys@transformshift{0.850455in}{3.141196in}%
\pgfsys@useobject{currentmarker}{}%
\end{pgfscope}%
\begin{pgfscope}%
\pgfsys@transformshift{0.868763in}{3.145802in}%
\pgfsys@useobject{currentmarker}{}%
\end{pgfscope}%
\begin{pgfscope}%
\pgfsys@transformshift{0.889185in}{3.167210in}%
\pgfsys@useobject{currentmarker}{}%
\end{pgfscope}%
\begin{pgfscope}%
\pgfsys@transformshift{0.906086in}{3.241653in}%
\pgfsys@useobject{currentmarker}{}%
\end{pgfscope}%
\begin{pgfscope}%
\pgfsys@transformshift{0.925803in}{3.421919in}%
\pgfsys@useobject{currentmarker}{}%
\end{pgfscope}%
\begin{pgfscope}%
\pgfsys@transformshift{0.944582in}{3.509732in}%
\pgfsys@useobject{currentmarker}{}%
\end{pgfscope}%
\begin{pgfscope}%
\pgfsys@transformshift{0.964064in}{3.397273in}%
\pgfsys@useobject{currentmarker}{}%
\end{pgfscope}%
\begin{pgfscope}%
\pgfsys@transformshift{0.983311in}{3.224434in}%
\pgfsys@useobject{currentmarker}{}%
\end{pgfscope}%
\begin{pgfscope}%
\pgfsys@transformshift{1.002794in}{3.161673in}%
\pgfsys@useobject{currentmarker}{}%
\end{pgfscope}%
\begin{pgfscope}%
\pgfsys@transformshift{1.021572in}{3.143748in}%
\pgfsys@useobject{currentmarker}{}%
\end{pgfscope}%
\begin{pgfscope}%
\pgfsys@transformshift{1.041289in}{3.141082in}%
\pgfsys@useobject{currentmarker}{}%
\end{pgfscope}%
\begin{pgfscope}%
\pgfsys@transformshift{1.061711in}{3.153219in}%
\pgfsys@useobject{currentmarker}{}%
\end{pgfscope}%
\begin{pgfscope}%
\pgfsys@transformshift{1.080724in}{3.190299in}%
\pgfsys@useobject{currentmarker}{}%
\end{pgfscope}%
\begin{pgfscope}%
\pgfsys@transformshift{1.098799in}{3.296599in}%
\pgfsys@useobject{currentmarker}{}%
\end{pgfscope}%
\begin{pgfscope}%
\pgfsys@transformshift{1.117342in}{3.481414in}%
\pgfsys@useobject{currentmarker}{}%
\end{pgfscope}%
\begin{pgfscope}%
\pgfsys@transformshift{1.139407in}{3.455774in}%
\pgfsys@useobject{currentmarker}{}%
\end{pgfscope}%
\begin{pgfscope}%
\pgfsys@transformshift{1.156543in}{3.346605in}%
\pgfsys@useobject{currentmarker}{}%
\end{pgfscope}%
\begin{pgfscope}%
\pgfsys@transformshift{1.174616in}{3.212199in}%
\pgfsys@useobject{currentmarker}{}%
\end{pgfscope}%
\begin{pgfscope}%
\pgfsys@transformshift{1.193864in}{3.157239in}%
\pgfsys@useobject{currentmarker}{}%
\end{pgfscope}%
\begin{pgfscope}%
\pgfsys@transformshift{1.212643in}{3.142425in}%
\pgfsys@useobject{currentmarker}{}%
\end{pgfscope}%
\begin{pgfscope}%
\pgfsys@transformshift{1.231656in}{3.140736in}%
\pgfsys@useobject{currentmarker}{}%
\end{pgfscope}%
\begin{pgfscope}%
\pgfsys@transformshift{1.249730in}{3.148451in}%
\pgfsys@useobject{currentmarker}{}%
\end{pgfscope}%
\begin{pgfscope}%
\pgfsys@transformshift{1.269681in}{3.179242in}%
\pgfsys@useobject{currentmarker}{}%
\end{pgfscope}%
\begin{pgfscope}%
\pgfsys@transformshift{1.292685in}{3.290146in}%
\pgfsys@useobject{currentmarker}{}%
\end{pgfscope}%
\begin{pgfscope}%
\pgfsys@transformshift{1.310524in}{3.460515in}%
\pgfsys@useobject{currentmarker}{}%
\end{pgfscope}%
\begin{pgfscope}%
\pgfsys@transformshift{1.326251in}{3.485212in}%
\pgfsys@useobject{currentmarker}{}%
\end{pgfscope}%
\begin{pgfscope}%
\pgfsys@transformshift{1.348082in}{3.348990in}%
\pgfsys@useobject{currentmarker}{}%
\end{pgfscope}%
\begin{pgfscope}%
\pgfsys@transformshift{1.366624in}{3.216323in}%
\pgfsys@useobject{currentmarker}{}%
\end{pgfscope}%
\begin{pgfscope}%
\pgfsys@transformshift{1.388454in}{3.153943in}%
\pgfsys@useobject{currentmarker}{}%
\end{pgfscope}%
\begin{pgfscope}%
\pgfsys@transformshift{1.404885in}{3.142768in}%
\pgfsys@useobject{currentmarker}{}%
\end{pgfscope}%
\begin{pgfscope}%
\pgfsys@transformshift{1.424133in}{3.140076in}%
\pgfsys@useobject{currentmarker}{}%
\end{pgfscope}%
\begin{pgfscope}%
\pgfsys@transformshift{1.443380in}{3.146676in}%
\pgfsys@useobject{currentmarker}{}%
\end{pgfscope}%
\begin{pgfscope}%
\pgfsys@transformshift{1.464037in}{3.143888in}%
\pgfsys@useobject{currentmarker}{}%
\end{pgfscope}%
\begin{pgfscope}%
\pgfsys@transformshift{1.479999in}{3.154980in}%
\pgfsys@useobject{currentmarker}{}%
\end{pgfscope}%
\begin{pgfscope}%
\pgfsys@transformshift{1.502298in}{3.191762in}%
\pgfsys@useobject{currentmarker}{}%
\end{pgfscope}%
\begin{pgfscope}%
\pgfsys@transformshift{1.521545in}{3.309222in}%
\pgfsys@useobject{currentmarker}{}%
\end{pgfscope}%
\begin{pgfscope}%
\pgfsys@transformshift{1.539855in}{3.461872in}%
\pgfsys@useobject{currentmarker}{}%
\end{pgfscope}%
\begin{pgfscope}%
\pgfsys@transformshift{1.558163in}{3.466100in}%
\pgfsys@useobject{currentmarker}{}%
\end{pgfscope}%
\begin{pgfscope}%
\pgfsys@transformshift{1.580462in}{3.322211in}%
\pgfsys@useobject{currentmarker}{}%
\end{pgfscope}%
\begin{pgfscope}%
\pgfsys@transformshift{1.602058in}{3.185846in}%
\pgfsys@useobject{currentmarker}{}%
\end{pgfscope}%
\begin{pgfscope}%
\pgfsys@transformshift{1.617786in}{3.159355in}%
\pgfsys@useobject{currentmarker}{}%
\end{pgfscope}%
\begin{pgfscope}%
\pgfsys@transformshift{1.633982in}{3.144833in}%
\pgfsys@useobject{currentmarker}{}%
\end{pgfscope}%
\begin{pgfscope}%
\pgfsys@transformshift{1.659801in}{3.139631in}%
\pgfsys@useobject{currentmarker}{}%
\end{pgfscope}%
\begin{pgfscope}%
\pgfsys@transformshift{1.674589in}{3.142304in}%
\pgfsys@useobject{currentmarker}{}%
\end{pgfscope}%
\begin{pgfscope}%
\pgfsys@transformshift{1.695481in}{3.160224in}%
\pgfsys@useobject{currentmarker}{}%
\end{pgfscope}%
\begin{pgfscope}%
\pgfsys@transformshift{1.714258in}{3.206744in}%
\pgfsys@useobject{currentmarker}{}%
\end{pgfscope}%
\begin{pgfscope}%
\pgfsys@transformshift{1.733037in}{3.316012in}%
\pgfsys@useobject{currentmarker}{}%
\end{pgfscope}%
\begin{pgfscope}%
\pgfsys@transformshift{1.752050in}{3.466149in}%
\pgfsys@useobject{currentmarker}{}%
\end{pgfscope}%
\begin{pgfscope}%
\pgfsys@transformshift{1.770593in}{3.442185in}%
\pgfsys@useobject{currentmarker}{}%
\end{pgfscope}%
\begin{pgfscope}%
\pgfsys@transformshift{1.790546in}{3.359836in}%
\pgfsys@useobject{currentmarker}{}%
\end{pgfscope}%
\begin{pgfscope}%
\pgfsys@transformshift{1.811671in}{3.201357in}%
\pgfsys@useobject{currentmarker}{}%
\end{pgfscope}%
\begin{pgfscope}%
\pgfsys@transformshift{1.829276in}{3.158000in}%
\pgfsys@useobject{currentmarker}{}%
\end{pgfscope}%
\begin{pgfscope}%
\pgfsys@transformshift{1.848523in}{3.145854in}%
\pgfsys@useobject{currentmarker}{}%
\end{pgfscope}%
\begin{pgfscope}%
\pgfsys@transformshift{1.866364in}{3.139629in}%
\pgfsys@useobject{currentmarker}{}%
\end{pgfscope}%
\begin{pgfscope}%
\pgfsys@transformshift{1.887021in}{3.140748in}%
\pgfsys@useobject{currentmarker}{}%
\end{pgfscope}%
\begin{pgfscope}%
\pgfsys@transformshift{1.904389in}{3.149200in}%
\pgfsys@useobject{currentmarker}{}%
\end{pgfscope}%
\begin{pgfscope}%
\pgfsys@transformshift{1.925750in}{3.181763in}%
\pgfsys@useobject{currentmarker}{}%
\end{pgfscope}%
\begin{pgfscope}%
\pgfsys@transformshift{1.943824in}{3.265502in}%
\pgfsys@useobject{currentmarker}{}%
\end{pgfscope}%
\begin{pgfscope}%
\pgfsys@transformshift{1.962837in}{3.427401in}%
\pgfsys@useobject{currentmarker}{}%
\end{pgfscope}%
\begin{pgfscope}%
\pgfsys@transformshift{1.984433in}{3.468055in}%
\pgfsys@useobject{currentmarker}{}%
\end{pgfscope}%
\begin{pgfscope}%
\pgfsys@transformshift{2.002272in}{3.445104in}%
\pgfsys@useobject{currentmarker}{}%
\end{pgfscope}%
\begin{pgfscope}%
\pgfsys@transformshift{2.021520in}{3.340379in}%
\pgfsys@useobject{currentmarker}{}%
\end{pgfscope}%
\begin{pgfscope}%
\pgfsys@transformshift{2.041471in}{3.199620in}%
\pgfsys@useobject{currentmarker}{}%
\end{pgfscope}%
\begin{pgfscope}%
\pgfsys@transformshift{2.059310in}{3.155543in}%
\pgfsys@useobject{currentmarker}{}%
\end{pgfscope}%
\begin{pgfscope}%
\pgfsys@transformshift{2.082080in}{3.141472in}%
\pgfsys@useobject{currentmarker}{}%
\end{pgfscope}%
\begin{pgfscope}%
\pgfsys@transformshift{2.098745in}{3.139244in}%
\pgfsys@useobject{currentmarker}{}%
\end{pgfscope}%
\begin{pgfscope}%
\pgfsys@transformshift{2.116584in}{3.142478in}%
\pgfsys@useobject{currentmarker}{}%
\end{pgfscope}%
\begin{pgfscope}%
\pgfsys@transformshift{2.136772in}{3.155559in}%
\pgfsys@useobject{currentmarker}{}%
\end{pgfscope}%
\begin{pgfscope}%
\pgfsys@transformshift{2.156019in}{3.189618in}%
\pgfsys@useobject{currentmarker}{}%
\end{pgfscope}%
\begin{pgfscope}%
\pgfsys@transformshift{2.174093in}{3.278453in}%
\pgfsys@useobject{currentmarker}{}%
\end{pgfscope}%
\begin{pgfscope}%
\pgfsys@transformshift{2.193577in}{3.444018in}%
\pgfsys@useobject{currentmarker}{}%
\end{pgfscope}%
\begin{pgfscope}%
\pgfsys@transformshift{2.211182in}{3.459530in}%
\pgfsys@useobject{currentmarker}{}%
\end{pgfscope}%
\begin{pgfscope}%
\pgfsys@transformshift{2.232541in}{3.382478in}%
\pgfsys@useobject{currentmarker}{}%
\end{pgfscope}%
\begin{pgfscope}%
\pgfsys@transformshift{2.250380in}{3.245825in}%
\pgfsys@useobject{currentmarker}{}%
\end{pgfscope}%
\begin{pgfscope}%
\pgfsys@transformshift{2.267985in}{3.172993in}%
\pgfsys@useobject{currentmarker}{}%
\end{pgfscope}%
\begin{pgfscope}%
\pgfsys@transformshift{2.289347in}{3.146411in}%
\pgfsys@useobject{currentmarker}{}%
\end{pgfscope}%
\begin{pgfscope}%
\pgfsys@transformshift{2.310003in}{3.143806in}%
\pgfsys@useobject{currentmarker}{}%
\end{pgfscope}%
\begin{pgfscope}%
\pgfsys@transformshift{2.328311in}{3.139854in}%
\pgfsys@useobject{currentmarker}{}%
\end{pgfscope}%
\begin{pgfscope}%
\pgfsys@transformshift{2.345916in}{3.143893in}%
\pgfsys@useobject{currentmarker}{}%
\end{pgfscope}%
\begin{pgfscope}%
\pgfsys@transformshift{2.367041in}{3.139010in}%
\pgfsys@useobject{currentmarker}{}%
\end{pgfscope}%
\begin{pgfscope}%
\pgfsys@transformshift{2.386288in}{3.141329in}%
\pgfsys@useobject{currentmarker}{}%
\end{pgfscope}%
\begin{pgfscope}%
\pgfsys@transformshift{2.406476in}{3.150500in}%
\pgfsys@useobject{currentmarker}{}%
\end{pgfscope}%
\begin{pgfscope}%
\pgfsys@transformshift{2.424080in}{3.172825in}%
\pgfsys@useobject{currentmarker}{}%
\end{pgfscope}%
\begin{pgfscope}%
\pgfsys@transformshift{2.448727in}{3.284115in}%
\pgfsys@useobject{currentmarker}{}%
\end{pgfscope}%
\begin{pgfscope}%
\pgfsys@transformshift{2.462107in}{3.409225in}%
\pgfsys@useobject{currentmarker}{}%
\end{pgfscope}%
\begin{pgfscope}%
\pgfsys@transformshift{2.480886in}{3.464823in}%
\pgfsys@useobject{currentmarker}{}%
\end{pgfscope}%
\begin{pgfscope}%
\pgfsys@transformshift{2.502245in}{3.424961in}%
\pgfsys@useobject{currentmarker}{}%
\end{pgfscope}%
\begin{pgfscope}%
\pgfsys@transformshift{2.519381in}{3.282032in}%
\pgfsys@useobject{currentmarker}{}%
\end{pgfscope}%
\begin{pgfscope}%
\pgfsys@transformshift{2.540037in}{3.183418in}%
\pgfsys@useobject{currentmarker}{}%
\end{pgfscope}%
\begin{pgfscope}%
\pgfsys@transformshift{2.558111in}{3.151571in}%
\pgfsys@useobject{currentmarker}{}%
\end{pgfscope}%
\begin{pgfscope}%
\pgfsys@transformshift{2.582523in}{3.140120in}%
\pgfsys@useobject{currentmarker}{}%
\end{pgfscope}%
\begin{pgfscope}%
\pgfsys@transformshift{2.597546in}{3.138697in}%
\pgfsys@useobject{currentmarker}{}%
\end{pgfscope}%
\begin{pgfscope}%
\pgfsys@transformshift{2.614916in}{3.142949in}%
\pgfsys@useobject{currentmarker}{}%
\end{pgfscope}%
\begin{pgfscope}%
\pgfsys@transformshift{2.636041in}{3.159606in}%
\pgfsys@useobject{currentmarker}{}%
\end{pgfscope}%
\begin{pgfscope}%
\pgfsys@transformshift{2.653880in}{3.200378in}%
\pgfsys@useobject{currentmarker}{}%
\end{pgfscope}%
\begin{pgfscope}%
\pgfsys@transformshift{2.677353in}{3.291127in}%
\pgfsys@useobject{currentmarker}{}%
\end{pgfscope}%
\begin{pgfscope}%
\pgfsys@transformshift{2.693081in}{3.426580in}%
\pgfsys@useobject{currentmarker}{}%
\end{pgfscope}%
\begin{pgfscope}%
\pgfsys@transformshift{2.713971in}{3.466755in}%
\pgfsys@useobject{currentmarker}{}%
\end{pgfscope}%
\begin{pgfscope}%
\pgfsys@transformshift{2.732280in}{3.400252in}%
\pgfsys@useobject{currentmarker}{}%
\end{pgfscope}%
\begin{pgfscope}%
\pgfsys@transformshift{2.749650in}{3.269539in}%
\pgfsys@useobject{currentmarker}{}%
\end{pgfscope}%
\begin{pgfscope}%
\pgfsys@transformshift{2.767724in}{3.202425in}%
\pgfsys@useobject{currentmarker}{}%
\end{pgfscope}%
\begin{pgfscope}%
\pgfsys@transformshift{2.787676in}{3.270781in}%
\pgfsys@useobject{currentmarker}{}%
\end{pgfscope}%
\begin{pgfscope}%
\pgfsys@transformshift{2.808333in}{3.456970in}%
\pgfsys@useobject{currentmarker}{}%
\end{pgfscope}%
\begin{pgfscope}%
\pgfsys@transformshift{2.829692in}{3.312727in}%
\pgfsys@useobject{currentmarker}{}%
\end{pgfscope}%
\begin{pgfscope}%
\pgfsys@transformshift{2.845888in}{3.213203in}%
\pgfsys@useobject{currentmarker}{}%
\end{pgfscope}%
\begin{pgfscope}%
\pgfsys@transformshift{2.868424in}{3.163073in}%
\pgfsys@useobject{currentmarker}{}%
\end{pgfscope}%
\begin{pgfscope}%
\pgfsys@transformshift{2.886029in}{3.149876in}%
\pgfsys@useobject{currentmarker}{}%
\end{pgfscope}%
\begin{pgfscope}%
\pgfsys@transformshift{2.905511in}{3.142770in}%
\pgfsys@useobject{currentmarker}{}%
\end{pgfscope}%
\begin{pgfscope}%
\pgfsys@transformshift{2.924758in}{3.139677in}%
\pgfsys@useobject{currentmarker}{}%
\end{pgfscope}%
\begin{pgfscope}%
\pgfsys@transformshift{2.942363in}{3.143947in}%
\pgfsys@useobject{currentmarker}{}%
\end{pgfscope}%
\begin{pgfscope}%
\pgfsys@transformshift{2.962080in}{3.164855in}%
\pgfsys@useobject{currentmarker}{}%
\end{pgfscope}%
\begin{pgfscope}%
\pgfsys@transformshift{2.982033in}{3.224024in}%
\pgfsys@useobject{currentmarker}{}%
\end{pgfscope}%
\begin{pgfscope}%
\pgfsys@transformshift{2.998229in}{3.298312in}%
\pgfsys@useobject{currentmarker}{}%
\end{pgfscope}%
\begin{pgfscope}%
\pgfsys@transformshift{3.022171in}{3.473728in}%
\pgfsys@useobject{currentmarker}{}%
\end{pgfscope}%
\begin{pgfscope}%
\pgfsys@transformshift{3.037664in}{3.467831in}%
\pgfsys@useobject{currentmarker}{}%
\end{pgfscope}%
\begin{pgfscope}%
\pgfsys@transformshift{3.059492in}{3.336601in}%
\pgfsys@useobject{currentmarker}{}%
\end{pgfscope}%
\begin{pgfscope}%
\pgfsys@transformshift{3.077333in}{3.216716in}%
\pgfsys@useobject{currentmarker}{}%
\end{pgfscope}%
\begin{pgfscope}%
\pgfsys@transformshift{3.097753in}{3.158587in}%
\pgfsys@useobject{currentmarker}{}%
\end{pgfscope}%
\begin{pgfscope}%
\pgfsys@transformshift{3.115829in}{3.144209in}%
\pgfsys@useobject{currentmarker}{}%
\end{pgfscope}%
\begin{pgfscope}%
\pgfsys@transformshift{3.134371in}{3.139590in}%
\pgfsys@useobject{currentmarker}{}%
\end{pgfscope}%
\begin{pgfscope}%
\pgfsys@transformshift{3.154558in}{3.143049in}%
\pgfsys@useobject{currentmarker}{}%
\end{pgfscope}%
\begin{pgfscope}%
\pgfsys@transformshift{3.172866in}{3.152651in}%
\pgfsys@useobject{currentmarker}{}%
\end{pgfscope}%
\begin{pgfscope}%
\pgfsys@transformshift{3.194462in}{3.186052in}%
\pgfsys@useobject{currentmarker}{}%
\end{pgfscope}%
\begin{pgfscope}%
\pgfsys@transformshift{3.213241in}{3.248040in}%
\pgfsys@useobject{currentmarker}{}%
\end{pgfscope}%
\begin{pgfscope}%
\pgfsys@transformshift{3.230141in}{3.403551in}%
\pgfsys@useobject{currentmarker}{}%
\end{pgfscope}%
\begin{pgfscope}%
\pgfsys@transformshift{3.251268in}{3.478036in}%
\pgfsys@useobject{currentmarker}{}%
\end{pgfscope}%
\begin{pgfscope}%
\pgfsys@transformshift{3.267228in}{3.470182in}%
\pgfsys@useobject{currentmarker}{}%
\end{pgfscope}%
\begin{pgfscope}%
\pgfsys@transformshift{3.285772in}{3.350901in}%
\pgfsys@useobject{currentmarker}{}%
\end{pgfscope}%
\begin{pgfscope}%
\pgfsys@transformshift{3.307837in}{3.221442in}%
\pgfsys@useobject{currentmarker}{}%
\end{pgfscope}%
\begin{pgfscope}%
\pgfsys@transformshift{3.325910in}{3.167007in}%
\pgfsys@useobject{currentmarker}{}%
\end{pgfscope}%
\begin{pgfscope}%
\pgfsys@transformshift{3.343749in}{3.147193in}%
\pgfsys@useobject{currentmarker}{}%
\end{pgfscope}%
\begin{pgfscope}%
\pgfsys@transformshift{3.368397in}{3.140374in}%
\pgfsys@useobject{currentmarker}{}%
\end{pgfscope}%
\begin{pgfscope}%
\pgfsys@transformshift{3.385767in}{3.141068in}%
\pgfsys@useobject{currentmarker}{}%
\end{pgfscope}%
\begin{pgfscope}%
\pgfsys@transformshift{3.404309in}{3.145387in}%
\pgfsys@useobject{currentmarker}{}%
\end{pgfscope}%
\begin{pgfscope}%
\pgfsys@transformshift{3.422619in}{3.157467in}%
\pgfsys@useobject{currentmarker}{}%
\end{pgfscope}%
\begin{pgfscope}%
\pgfsys@transformshift{3.443510in}{3.190815in}%
\pgfsys@useobject{currentmarker}{}%
\end{pgfscope}%
\begin{pgfscope}%
\pgfsys@transformshift{3.463932in}{3.261810in}%
\pgfsys@useobject{currentmarker}{}%
\end{pgfscope}%
\begin{pgfscope}%
\pgfsys@transformshift{3.479188in}{3.351273in}%
\pgfsys@useobject{currentmarker}{}%
\end{pgfscope}%
\begin{pgfscope}%
\pgfsys@transformshift{3.502427in}{3.491366in}%
\pgfsys@useobject{currentmarker}{}%
\end{pgfscope}%
\begin{pgfscope}%
\pgfsys@transformshift{3.520971in}{3.461947in}%
\pgfsys@useobject{currentmarker}{}%
\end{pgfscope}%
\begin{pgfscope}%
\pgfsys@transformshift{3.539748in}{3.393445in}%
\pgfsys@useobject{currentmarker}{}%
\end{pgfscope}%
\begin{pgfscope}%
\pgfsys@transformshift{3.556650in}{3.268292in}%
\pgfsys@useobject{currentmarker}{}%
\end{pgfscope}%
\begin{pgfscope}%
\pgfsys@transformshift{3.577072in}{3.179823in}%
\pgfsys@useobject{currentmarker}{}%
\end{pgfscope}%
\begin{pgfscope}%
\pgfsys@transformshift{3.596319in}{3.152636in}%
\pgfsys@useobject{currentmarker}{}%
\end{pgfscope}%
\begin{pgfscope}%
\pgfsys@transformshift{3.616270in}{3.143504in}%
\pgfsys@useobject{currentmarker}{}%
\end{pgfscope}%
\begin{pgfscope}%
\pgfsys@transformshift{3.636458in}{3.140116in}%
\pgfsys@useobject{currentmarker}{}%
\end{pgfscope}%
\begin{pgfscope}%
\pgfsys@transformshift{3.655471in}{3.141872in}%
\pgfsys@useobject{currentmarker}{}%
\end{pgfscope}%
\begin{pgfscope}%
\pgfsys@transformshift{3.672607in}{3.149819in}%
\pgfsys@useobject{currentmarker}{}%
\end{pgfscope}%
\begin{pgfscope}%
\pgfsys@transformshift{3.690210in}{3.166429in}%
\pgfsys@useobject{currentmarker}{}%
\end{pgfscope}%
\begin{pgfscope}%
\pgfsys@transformshift{3.710868in}{3.208882in}%
\pgfsys@useobject{currentmarker}{}%
\end{pgfscope}%
\begin{pgfscope}%
\pgfsys@transformshift{3.731993in}{3.306105in}%
\pgfsys@useobject{currentmarker}{}%
\end{pgfscope}%
\begin{pgfscope}%
\pgfsys@transformshift{3.751006in}{3.445846in}%
\pgfsys@useobject{currentmarker}{}%
\end{pgfscope}%
\begin{pgfscope}%
\pgfsys@transformshift{3.770254in}{3.507308in}%
\pgfsys@useobject{currentmarker}{}%
\end{pgfscope}%
\begin{pgfscope}%
\pgfsys@transformshift{3.788327in}{3.468853in}%
\pgfsys@useobject{currentmarker}{}%
\end{pgfscope}%
\begin{pgfscope}%
\pgfsys@transformshift{3.805932in}{3.372190in}%
\pgfsys@useobject{currentmarker}{}%
\end{pgfscope}%
\begin{pgfscope}%
\pgfsys@transformshift{3.826354in}{3.241555in}%
\pgfsys@useobject{currentmarker}{}%
\end{pgfscope}%
\begin{pgfscope}%
\pgfsys@transformshift{3.846305in}{3.197245in}%
\pgfsys@useobject{currentmarker}{}%
\end{pgfscope}%
\begin{pgfscope}%
\pgfsys@transformshift{3.863206in}{3.166053in}%
\pgfsys@useobject{currentmarker}{}%
\end{pgfscope}%
\begin{pgfscope}%
\pgfsys@transformshift{3.884802in}{3.145929in}%
\pgfsys@useobject{currentmarker}{}%
\end{pgfscope}%
\begin{pgfscope}%
\pgfsys@transformshift{3.908744in}{3.140697in}%
\pgfsys@useobject{currentmarker}{}%
\end{pgfscope}%
\begin{pgfscope}%
\pgfsys@transformshift{3.924235in}{3.141936in}%
\pgfsys@useobject{currentmarker}{}%
\end{pgfscope}%
\begin{pgfscope}%
\pgfsys@transformshift{3.941840in}{3.147669in}%
\pgfsys@useobject{currentmarker}{}%
\end{pgfscope}%
\begin{pgfscope}%
\pgfsys@transformshift{3.962496in}{3.228292in}%
\pgfsys@useobject{currentmarker}{}%
\end{pgfscope}%
\begin{pgfscope}%
\pgfsys@transformshift{3.979866in}{3.167784in}%
\pgfsys@useobject{currentmarker}{}%
\end{pgfscope}%
\begin{pgfscope}%
\pgfsys@transformshift{4.001931in}{3.146747in}%
\pgfsys@useobject{currentmarker}{}%
\end{pgfscope}%
\begin{pgfscope}%
\pgfsys@transformshift{4.022587in}{3.141106in}%
\pgfsys@useobject{currentmarker}{}%
\end{pgfscope}%
\begin{pgfscope}%
\pgfsys@transformshift{4.037609in}{3.142912in}%
\pgfsys@useobject{currentmarker}{}%
\end{pgfscope}%
\begin{pgfscope}%
\pgfsys@transformshift{4.059205in}{3.153167in}%
\pgfsys@useobject{currentmarker}{}%
\end{pgfscope}%
\begin{pgfscope}%
\pgfsys@transformshift{4.077279in}{3.173098in}%
\pgfsys@useobject{currentmarker}{}%
\end{pgfscope}%
\begin{pgfscope}%
\pgfsys@transformshift{4.094649in}{3.218955in}%
\pgfsys@useobject{currentmarker}{}%
\end{pgfscope}%
\begin{pgfscope}%
\pgfsys@transformshift{4.119296in}{3.377838in}%
\pgfsys@useobject{currentmarker}{}%
\end{pgfscope}%
\begin{pgfscope}%
\pgfsys@transformshift{4.133615in}{3.493312in}%
\pgfsys@useobject{currentmarker}{}%
\end{pgfscope}%
\begin{pgfscope}%
\pgfsys@transformshift{4.153801in}{3.527851in}%
\pgfsys@useobject{currentmarker}{}%
\end{pgfscope}%
\begin{pgfscope}%
\pgfsys@transformshift{4.171640in}{3.525144in}%
\pgfsys@useobject{currentmarker}{}%
\end{pgfscope}%
\begin{pgfscope}%
\pgfsys@transformshift{4.196053in}{3.400889in}%
\pgfsys@useobject{currentmarker}{}%
\end{pgfscope}%
\begin{pgfscope}%
\pgfsys@transformshift{4.211075in}{3.272941in}%
\pgfsys@useobject{currentmarker}{}%
\end{pgfscope}%
\begin{pgfscope}%
\pgfsys@transformshift{4.231965in}{3.183885in}%
\pgfsys@useobject{currentmarker}{}%
\end{pgfscope}%
\begin{pgfscope}%
\pgfsys@transformshift{4.250979in}{3.155097in}%
\pgfsys@useobject{currentmarker}{}%
\end{pgfscope}%
\begin{pgfscope}%
\pgfsys@transformshift{4.269523in}{3.144827in}%
\pgfsys@useobject{currentmarker}{}%
\end{pgfscope}%
\begin{pgfscope}%
\pgfsys@transformshift{4.290179in}{3.142401in}%
\pgfsys@useobject{currentmarker}{}%
\end{pgfscope}%
\begin{pgfscope}%
\pgfsys@transformshift{4.306844in}{3.148348in}%
\pgfsys@useobject{currentmarker}{}%
\end{pgfscope}%
\begin{pgfscope}%
\pgfsys@transformshift{4.327266in}{3.167810in}%
\pgfsys@useobject{currentmarker}{}%
\end{pgfscope}%
\begin{pgfscope}%
\pgfsys@transformshift{4.346983in}{3.201982in}%
\pgfsys@useobject{currentmarker}{}%
\end{pgfscope}%
\begin{pgfscope}%
\pgfsys@transformshift{4.366936in}{3.307591in}%
\pgfsys@useobject{currentmarker}{}%
\end{pgfscope}%
\begin{pgfscope}%
\pgfsys@transformshift{4.385714in}{3.416106in}%
\pgfsys@useobject{currentmarker}{}%
\end{pgfscope}%
\begin{pgfscope}%
\pgfsys@transformshift{4.403319in}{3.541339in}%
\pgfsys@useobject{currentmarker}{}%
\end{pgfscope}%
\begin{pgfscope}%
\pgfsys@transformshift{4.419279in}{3.539280in}%
\pgfsys@useobject{currentmarker}{}%
\end{pgfscope}%
\begin{pgfscope}%
\pgfsys@transformshift{4.439701in}{3.454377in}%
\pgfsys@useobject{currentmarker}{}%
\end{pgfscope}%
\begin{pgfscope}%
\pgfsys@transformshift{4.462000in}{3.286837in}%
\pgfsys@useobject{currentmarker}{}%
\end{pgfscope}%
\begin{pgfscope}%
\pgfsys@transformshift{4.479136in}{3.202423in}%
\pgfsys@useobject{currentmarker}{}%
\end{pgfscope}%
\begin{pgfscope}%
\pgfsys@transformshift{4.478902in}{3.203565in}%
\pgfsys@useobject{currentmarker}{}%
\end{pgfscope}%
\begin{pgfscope}%
\pgfsys@transformshift{4.475850in}{3.219080in}%
\pgfsys@useobject{currentmarker}{}%
\end{pgfscope}%
\begin{pgfscope}%
\pgfsys@transformshift{4.454254in}{3.410990in}%
\pgfsys@useobject{currentmarker}{}%
\end{pgfscope}%
\begin{pgfscope}%
\pgfsys@transformshift{4.435476in}{3.539158in}%
\pgfsys@useobject{currentmarker}{}%
\end{pgfscope}%
\begin{pgfscope}%
\pgfsys@transformshift{4.417871in}{3.498284in}%
\pgfsys@useobject{currentmarker}{}%
\end{pgfscope}%
\begin{pgfscope}%
\pgfsys@transformshift{4.398859in}{3.290656in}%
\pgfsys@useobject{currentmarker}{}%
\end{pgfscope}%
\begin{pgfscope}%
\pgfsys@transformshift{4.377029in}{3.180859in}%
\pgfsys@useobject{currentmarker}{}%
\end{pgfscope}%
\begin{pgfscope}%
\pgfsys@transformshift{4.358719in}{3.152360in}%
\pgfsys@useobject{currentmarker}{}%
\end{pgfscope}%
\begin{pgfscope}%
\pgfsys@transformshift{4.341351in}{3.141938in}%
\pgfsys@useobject{currentmarker}{}%
\end{pgfscope}%
\begin{pgfscope}%
\pgfsys@transformshift{4.319050in}{3.148941in}%
\pgfsys@useobject{currentmarker}{}%
\end{pgfscope}%
\begin{pgfscope}%
\pgfsys@transformshift{4.301211in}{3.175274in}%
\pgfsys@useobject{currentmarker}{}%
\end{pgfscope}%
\begin{pgfscope}%
\pgfsys@transformshift{4.283371in}{3.262148in}%
\pgfsys@useobject{currentmarker}{}%
\end{pgfscope}%
\begin{pgfscope}%
\pgfsys@transformshift{4.262950in}{3.459631in}%
\pgfsys@useobject{currentmarker}{}%
\end{pgfscope}%
\begin{pgfscope}%
\pgfsys@transformshift{4.246050in}{3.530724in}%
\pgfsys@useobject{currentmarker}{}%
\end{pgfscope}%
\begin{pgfscope}%
\pgfsys@transformshift{4.225394in}{3.407099in}%
\pgfsys@useobject{currentmarker}{}%
\end{pgfscope}%
\begin{pgfscope}%
\pgfsys@transformshift{4.205675in}{3.213632in}%
\pgfsys@useobject{currentmarker}{}%
\end{pgfscope}%
\begin{pgfscope}%
\pgfsys@transformshift{4.187133in}{3.161872in}%
\pgfsys@useobject{currentmarker}{}%
\end{pgfscope}%
\begin{pgfscope}%
\pgfsys@transformshift{4.166711in}{3.143795in}%
\pgfsys@useobject{currentmarker}{}%
\end{pgfscope}%
\begin{pgfscope}%
\pgfsys@transformshift{4.167415in}{3.141473in}%
\pgfsys@useobject{currentmarker}{}%
\end{pgfscope}%
\begin{pgfscope}%
\pgfsys@transformshift{4.149575in}{3.141734in}%
\pgfsys@useobject{currentmarker}{}%
\end{pgfscope}%
\begin{pgfscope}%
\pgfsys@transformshift{4.129154in}{3.156584in}%
\pgfsys@useobject{currentmarker}{}%
\end{pgfscope}%
\begin{pgfscope}%
\pgfsys@transformshift{4.110846in}{3.197051in}%
\pgfsys@useobject{currentmarker}{}%
\end{pgfscope}%
\begin{pgfscope}%
\pgfsys@transformshift{4.090424in}{3.358716in}%
\pgfsys@useobject{currentmarker}{}%
\end{pgfscope}%
\begin{pgfscope}%
\pgfsys@transformshift{4.072350in}{3.494813in}%
\pgfsys@useobject{currentmarker}{}%
\end{pgfscope}%
\begin{pgfscope}%
\pgfsys@transformshift{4.051225in}{3.474636in}%
\pgfsys@useobject{currentmarker}{}%
\end{pgfscope}%
\begin{pgfscope}%
\pgfsys@transformshift{4.031507in}{3.256527in}%
\pgfsys@useobject{currentmarker}{}%
\end{pgfscope}%
\begin{pgfscope}%
\pgfsys@transformshift{4.013433in}{3.177614in}%
\pgfsys@useobject{currentmarker}{}%
\end{pgfscope}%
\begin{pgfscope}%
\pgfsys@transformshift{3.993951in}{3.148088in}%
\pgfsys@useobject{currentmarker}{}%
\end{pgfscope}%
\begin{pgfscope}%
\pgfsys@transformshift{3.974233in}{3.140658in}%
\pgfsys@useobject{currentmarker}{}%
\end{pgfscope}%
\begin{pgfscope}%
\pgfsys@transformshift{3.956393in}{3.145120in}%
\pgfsys@useobject{currentmarker}{}%
\end{pgfscope}%
\begin{pgfscope}%
\pgfsys@transformshift{3.935503in}{3.170428in}%
\pgfsys@useobject{currentmarker}{}%
\end{pgfscope}%
\begin{pgfscope}%
\pgfsys@transformshift{3.918603in}{3.245568in}%
\pgfsys@useobject{currentmarker}{}%
\end{pgfscope}%
\begin{pgfscope}%
\pgfsys@transformshift{3.899119in}{3.433656in}%
\pgfsys@useobject{currentmarker}{}%
\end{pgfscope}%
\begin{pgfscope}%
\pgfsys@transformshift{3.878463in}{3.501353in}%
\pgfsys@useobject{currentmarker}{}%
\end{pgfscope}%
\begin{pgfscope}%
\pgfsys@transformshift{3.861329in}{3.377843in}%
\pgfsys@useobject{currentmarker}{}%
\end{pgfscope}%
\begin{pgfscope}%
\pgfsys@transformshift{3.838794in}{3.208713in}%
\pgfsys@useobject{currentmarker}{}%
\end{pgfscope}%
\begin{pgfscope}%
\pgfsys@transformshift{3.821189in}{3.159770in}%
\pgfsys@useobject{currentmarker}{}%
\end{pgfscope}%
\begin{pgfscope}%
\pgfsys@transformshift{3.803820in}{3.145405in}%
\pgfsys@useobject{currentmarker}{}%
\end{pgfscope}%
\begin{pgfscope}%
\pgfsys@transformshift{3.783164in}{3.140590in}%
\pgfsys@useobject{currentmarker}{}%
\end{pgfscope}%
\begin{pgfscope}%
\pgfsys@transformshift{3.762977in}{3.147716in}%
\pgfsys@useobject{currentmarker}{}%
\end{pgfscope}%
\begin{pgfscope}%
\pgfsys@transformshift{3.747484in}{3.183561in}%
\pgfsys@useobject{currentmarker}{}%
\end{pgfscope}%
\begin{pgfscope}%
\pgfsys@transformshift{3.725419in}{3.256643in}%
\pgfsys@useobject{currentmarker}{}%
\end{pgfscope}%
\begin{pgfscope}%
\pgfsys@transformshift{3.708520in}{3.415720in}%
\pgfsys@useobject{currentmarker}{}%
\end{pgfscope}%
\begin{pgfscope}%
\pgfsys@transformshift{3.686924in}{3.452068in}%
\pgfsys@useobject{currentmarker}{}%
\end{pgfscope}%
\begin{pgfscope}%
\pgfsys@transformshift{3.669319in}{3.488246in}%
\pgfsys@useobject{currentmarker}{}%
\end{pgfscope}%
\begin{pgfscope}%
\pgfsys@transformshift{3.647960in}{3.316669in}%
\pgfsys@useobject{currentmarker}{}%
\end{pgfscope}%
\begin{pgfscope}%
\pgfsys@transformshift{3.627538in}{3.188951in}%
\pgfsys@useobject{currentmarker}{}%
\end{pgfscope}%
\begin{pgfscope}%
\pgfsys@transformshift{3.609230in}{3.153129in}%
\pgfsys@useobject{currentmarker}{}%
\end{pgfscope}%
\begin{pgfscope}%
\pgfsys@transformshift{3.588337in}{3.143071in}%
\pgfsys@useobject{currentmarker}{}%
\end{pgfscope}%
\begin{pgfscope}%
\pgfsys@transformshift{3.572612in}{3.140056in}%
\pgfsys@useobject{currentmarker}{}%
\end{pgfscope}%
\begin{pgfscope}%
\pgfsys@transformshift{3.550313in}{3.146092in}%
\pgfsys@useobject{currentmarker}{}%
\end{pgfscope}%
\begin{pgfscope}%
\pgfsys@transformshift{3.532003in}{3.163699in}%
\pgfsys@useobject{currentmarker}{}%
\end{pgfscope}%
\begin{pgfscope}%
\pgfsys@transformshift{3.514867in}{3.217622in}%
\pgfsys@useobject{currentmarker}{}%
\end{pgfscope}%
\begin{pgfscope}%
\pgfsys@transformshift{3.494447in}{3.340561in}%
\pgfsys@useobject{currentmarker}{}%
\end{pgfscope}%
\begin{pgfscope}%
\pgfsys@transformshift{3.473086in}{3.467277in}%
\pgfsys@useobject{currentmarker}{}%
\end{pgfscope}%
\begin{pgfscope}%
\pgfsys@transformshift{3.455952in}{3.471154in}%
\pgfsys@useobject{currentmarker}{}%
\end{pgfscope}%
\begin{pgfscope}%
\pgfsys@transformshift{3.438111in}{3.333291in}%
\pgfsys@useobject{currentmarker}{}%
\end{pgfscope}%
\begin{pgfscope}%
\pgfsys@transformshift{3.418863in}{3.200923in}%
\pgfsys@useobject{currentmarker}{}%
\end{pgfscope}%
\begin{pgfscope}%
\pgfsys@transformshift{3.393747in}{3.152451in}%
\pgfsys@useobject{currentmarker}{}%
\end{pgfscope}%
\begin{pgfscope}%
\pgfsys@transformshift{3.378021in}{3.142833in}%
\pgfsys@useobject{currentmarker}{}%
\end{pgfscope}%
\begin{pgfscope}%
\pgfsys@transformshift{3.360417in}{3.139737in}%
\pgfsys@useobject{currentmarker}{}%
\end{pgfscope}%
\begin{pgfscope}%
\pgfsys@transformshift{3.342577in}{3.142094in}%
\pgfsys@useobject{currentmarker}{}%
\end{pgfscope}%
\begin{pgfscope}%
\pgfsys@transformshift{3.320747in}{3.155986in}%
\pgfsys@useobject{currentmarker}{}%
\end{pgfscope}%
\begin{pgfscope}%
\pgfsys@transformshift{3.304551in}{3.187692in}%
\pgfsys@useobject{currentmarker}{}%
\end{pgfscope}%
\begin{pgfscope}%
\pgfsys@transformshift{3.303611in}{3.259328in}%
\pgfsys@useobject{currentmarker}{}%
\end{pgfscope}%
\begin{pgfscope}%
\pgfsys@transformshift{3.282486in}{3.209465in}%
\pgfsys@useobject{currentmarker}{}%
\end{pgfscope}%
\begin{pgfscope}%
\pgfsys@transformshift{3.264411in}{3.151397in}%
\pgfsys@useobject{currentmarker}{}%
\end{pgfscope}%
\begin{pgfscope}%
\pgfsys@transformshift{3.241643in}{3.200675in}%
\pgfsys@useobject{currentmarker}{}%
\end{pgfscope}%
\begin{pgfscope}%
\pgfsys@transformshift{3.222395in}{3.331890in}%
\pgfsys@useobject{currentmarker}{}%
\end{pgfscope}%
\begin{pgfscope}%
\pgfsys@transformshift{3.205259in}{3.457843in}%
\pgfsys@useobject{currentmarker}{}%
\end{pgfscope}%
\begin{pgfscope}%
\pgfsys@transformshift{3.186951in}{3.465507in}%
\pgfsys@useobject{currentmarker}{}%
\end{pgfscope}%
\begin{pgfscope}%
\pgfsys@transformshift{3.168407in}{3.300345in}%
\pgfsys@useobject{currentmarker}{}%
\end{pgfscope}%
\begin{pgfscope}%
\pgfsys@transformshift{3.147282in}{3.185075in}%
\pgfsys@useobject{currentmarker}{}%
\end{pgfscope}%
\begin{pgfscope}%
\pgfsys@transformshift{3.132025in}{3.158752in}%
\pgfsys@useobject{currentmarker}{}%
\end{pgfscope}%
\begin{pgfscope}%
\pgfsys@transformshift{3.109021in}{3.144241in}%
\pgfsys@useobject{currentmarker}{}%
\end{pgfscope}%
\begin{pgfscope}%
\pgfsys@transformshift{3.091181in}{3.139303in}%
\pgfsys@useobject{currentmarker}{}%
\end{pgfscope}%
\begin{pgfscope}%
\pgfsys@transformshift{3.070760in}{3.144929in}%
\pgfsys@useobject{currentmarker}{}%
\end{pgfscope}%
\begin{pgfscope}%
\pgfsys@transformshift{3.054563in}{3.152740in}%
\pgfsys@useobject{currentmarker}{}%
\end{pgfscope}%
\begin{pgfscope}%
\pgfsys@transformshift{3.032499in}{3.192786in}%
\pgfsys@useobject{currentmarker}{}%
\end{pgfscope}%
\begin{pgfscope}%
\pgfsys@transformshift{3.014894in}{3.289615in}%
\pgfsys@useobject{currentmarker}{}%
\end{pgfscope}%
\begin{pgfscope}%
\pgfsys@transformshift{2.994238in}{3.441680in}%
\pgfsys@useobject{currentmarker}{}%
\end{pgfscope}%
\begin{pgfscope}%
\pgfsys@transformshift{2.976164in}{3.468307in}%
\pgfsys@useobject{currentmarker}{}%
\end{pgfscope}%
\begin{pgfscope}%
\pgfsys@transformshift{2.954803in}{3.324456in}%
\pgfsys@useobject{currentmarker}{}%
\end{pgfscope}%
\begin{pgfscope}%
\pgfsys@transformshift{2.936260in}{3.235396in}%
\pgfsys@useobject{currentmarker}{}%
\end{pgfscope}%
\begin{pgfscope}%
\pgfsys@transformshift{2.918890in}{3.173173in}%
\pgfsys@useobject{currentmarker}{}%
\end{pgfscope}%
\begin{pgfscope}%
\pgfsys@transformshift{2.897765in}{3.146575in}%
\pgfsys@useobject{currentmarker}{}%
\end{pgfscope}%
\begin{pgfscope}%
\pgfsys@transformshift{2.878986in}{3.139684in}%
\pgfsys@useobject{currentmarker}{}%
\end{pgfscope}%
\begin{pgfscope}%
\pgfsys@transformshift{2.857625in}{3.140656in}%
\pgfsys@useobject{currentmarker}{}%
\end{pgfscope}%
\begin{pgfscope}%
\pgfsys@transformshift{2.839317in}{3.149661in}%
\pgfsys@useobject{currentmarker}{}%
\end{pgfscope}%
\begin{pgfscope}%
\pgfsys@transformshift{2.821712in}{3.173088in}%
\pgfsys@useobject{currentmarker}{}%
\end{pgfscope}%
\begin{pgfscope}%
\pgfsys@transformshift{2.802230in}{3.247778in}%
\pgfsys@useobject{currentmarker}{}%
\end{pgfscope}%
\begin{pgfscope}%
\pgfsys@transformshift{2.783451in}{3.378953in}%
\pgfsys@useobject{currentmarker}{}%
\end{pgfscope}%
\begin{pgfscope}%
\pgfsys@transformshift{2.764203in}{3.467471in}%
\pgfsys@useobject{currentmarker}{}%
\end{pgfscope}%
\begin{pgfscope}%
\pgfsys@transformshift{2.746130in}{3.436522in}%
\pgfsys@useobject{currentmarker}{}%
\end{pgfscope}%
\begin{pgfscope}%
\pgfsys@transformshift{2.726646in}{3.291678in}%
\pgfsys@useobject{currentmarker}{}%
\end{pgfscope}%
\begin{pgfscope}%
\pgfsys@transformshift{2.705052in}{3.210325in}%
\pgfsys@useobject{currentmarker}{}%
\end{pgfscope}%
\begin{pgfscope}%
\pgfsys@transformshift{2.686039in}{3.161354in}%
\pgfsys@useobject{currentmarker}{}%
\end{pgfscope}%
\begin{pgfscope}%
\pgfsys@transformshift{2.667494in}{3.144794in}%
\pgfsys@useobject{currentmarker}{}%
\end{pgfscope}%
\begin{pgfscope}%
\pgfsys@transformshift{2.649186in}{3.138923in}%
\pgfsys@useobject{currentmarker}{}%
\end{pgfscope}%
\begin{pgfscope}%
\pgfsys@transformshift{2.630407in}{3.139780in}%
\pgfsys@useobject{currentmarker}{}%
\end{pgfscope}%
\begin{pgfscope}%
\pgfsys@transformshift{2.611394in}{3.145794in}%
\pgfsys@useobject{currentmarker}{}%
\end{pgfscope}%
\begin{pgfscope}%
\pgfsys@transformshift{2.590503in}{3.153875in}%
\pgfsys@useobject{currentmarker}{}%
\end{pgfscope}%
\begin{pgfscope}%
\pgfsys@transformshift{2.571021in}{3.191922in}%
\pgfsys@useobject{currentmarker}{}%
\end{pgfscope}%
\begin{pgfscope}%
\pgfsys@transformshift{2.554120in}{3.291762in}%
\pgfsys@useobject{currentmarker}{}%
\end{pgfscope}%
\begin{pgfscope}%
\pgfsys@transformshift{2.530883in}{3.434973in}%
\pgfsys@useobject{currentmarker}{}%
\end{pgfscope}%
\begin{pgfscope}%
\pgfsys@transformshift{2.514921in}{3.463309in}%
\pgfsys@useobject{currentmarker}{}%
\end{pgfscope}%
\begin{pgfscope}%
\pgfsys@transformshift{2.493560in}{3.405751in}%
\pgfsys@useobject{currentmarker}{}%
\end{pgfscope}%
\begin{pgfscope}%
\pgfsys@transformshift{2.474312in}{3.251860in}%
\pgfsys@useobject{currentmarker}{}%
\end{pgfscope}%
\begin{pgfscope}%
\pgfsys@transformshift{2.455770in}{3.294024in}%
\pgfsys@useobject{currentmarker}{}%
\end{pgfscope}%
\begin{pgfscope}%
\pgfsys@transformshift{2.437460in}{3.193247in}%
\pgfsys@useobject{currentmarker}{}%
\end{pgfscope}%
\begin{pgfscope}%
\pgfsys@transformshift{2.415866in}{3.152417in}%
\pgfsys@useobject{currentmarker}{}%
\end{pgfscope}%
\begin{pgfscope}%
\pgfsys@transformshift{2.397087in}{3.141064in}%
\pgfsys@useobject{currentmarker}{}%
\end{pgfscope}%
\begin{pgfscope}%
\pgfsys@transformshift{2.381125in}{3.138921in}%
\pgfsys@useobject{currentmarker}{}%
\end{pgfscope}%
\begin{pgfscope}%
\pgfsys@transformshift{2.359764in}{3.140416in}%
\pgfsys@useobject{currentmarker}{}%
\end{pgfscope}%
\begin{pgfscope}%
\pgfsys@transformshift{2.341221in}{3.147817in}%
\pgfsys@useobject{currentmarker}{}%
\end{pgfscope}%
\begin{pgfscope}%
\pgfsys@transformshift{2.322442in}{3.174354in}%
\pgfsys@useobject{currentmarker}{}%
\end{pgfscope}%
\begin{pgfscope}%
\pgfsys@transformshift{2.303195in}{3.248367in}%
\pgfsys@useobject{currentmarker}{}%
\end{pgfscope}%
\begin{pgfscope}%
\pgfsys@transformshift{2.284416in}{3.407658in}%
\pgfsys@useobject{currentmarker}{}%
\end{pgfscope}%
\begin{pgfscope}%
\pgfsys@transformshift{2.266108in}{3.465482in}%
\pgfsys@useobject{currentmarker}{}%
\end{pgfscope}%
\begin{pgfscope}%
\pgfsys@transformshift{2.243809in}{3.386457in}%
\pgfsys@useobject{currentmarker}{}%
\end{pgfscope}%
\begin{pgfscope}%
\pgfsys@transformshift{2.228787in}{3.242721in}%
\pgfsys@useobject{currentmarker}{}%
\end{pgfscope}%
\begin{pgfscope}%
\pgfsys@transformshift{2.207191in}{3.176039in}%
\pgfsys@useobject{currentmarker}{}%
\end{pgfscope}%
\begin{pgfscope}%
\pgfsys@transformshift{2.182544in}{3.149056in}%
\pgfsys@useobject{currentmarker}{}%
\end{pgfscope}%
\begin{pgfscope}%
\pgfsys@transformshift{2.169869in}{3.142419in}%
\pgfsys@useobject{currentmarker}{}%
\end{pgfscope}%
\begin{pgfscope}%
\pgfsys@transformshift{2.150856in}{3.138588in}%
\pgfsys@useobject{currentmarker}{}%
\end{pgfscope}%
\begin{pgfscope}%
\pgfsys@transformshift{2.130669in}{3.140440in}%
\pgfsys@useobject{currentmarker}{}%
\end{pgfscope}%
\begin{pgfscope}%
\pgfsys@transformshift{2.108135in}{3.151513in}%
\pgfsys@useobject{currentmarker}{}%
\end{pgfscope}%
\begin{pgfscope}%
\pgfsys@transformshift{2.094051in}{3.169378in}%
\pgfsys@useobject{currentmarker}{}%
\end{pgfscope}%
\begin{pgfscope}%
\pgfsys@transformshift{2.071283in}{3.248212in}%
\pgfsys@useobject{currentmarker}{}%
\end{pgfscope}%
\begin{pgfscope}%
\pgfsys@transformshift{2.051565in}{3.393115in}%
\pgfsys@useobject{currentmarker}{}%
\end{pgfscope}%
\begin{pgfscope}%
\pgfsys@transformshift{2.033022in}{3.462125in}%
\pgfsys@useobject{currentmarker}{}%
\end{pgfscope}%
\begin{pgfscope}%
\pgfsys@transformshift{2.014009in}{3.427360in}%
\pgfsys@useobject{currentmarker}{}%
\end{pgfscope}%
\begin{pgfscope}%
\pgfsys@transformshift{1.993822in}{3.370003in}%
\pgfsys@useobject{currentmarker}{}%
\end{pgfscope}%
\begin{pgfscope}%
\pgfsys@transformshift{1.974808in}{3.248082in}%
\pgfsys@useobject{currentmarker}{}%
\end{pgfscope}%
\begin{pgfscope}%
\pgfsys@transformshift{1.958846in}{3.187961in}%
\pgfsys@useobject{currentmarker}{}%
\end{pgfscope}%
\begin{pgfscope}%
\pgfsys@transformshift{1.936547in}{3.151589in}%
\pgfsys@useobject{currentmarker}{}%
\end{pgfscope}%
\begin{pgfscope}%
\pgfsys@transformshift{1.917768in}{3.142114in}%
\pgfsys@useobject{currentmarker}{}%
\end{pgfscope}%
\begin{pgfscope}%
\pgfsys@transformshift{1.901338in}{3.138923in}%
\pgfsys@useobject{currentmarker}{}%
\end{pgfscope}%
\begin{pgfscope}%
\pgfsys@transformshift{1.880447in}{3.141383in}%
\pgfsys@useobject{currentmarker}{}%
\end{pgfscope}%
\begin{pgfscope}%
\pgfsys@transformshift{1.860496in}{3.150148in}%
\pgfsys@useobject{currentmarker}{}%
\end{pgfscope}%
\begin{pgfscope}%
\pgfsys@transformshift{1.842186in}{3.163080in}%
\pgfsys@useobject{currentmarker}{}%
\end{pgfscope}%
\begin{pgfscope}%
\pgfsys@transformshift{1.820356in}{3.164070in}%
\pgfsys@useobject{currentmarker}{}%
\end{pgfscope}%
\begin{pgfscope}%
\pgfsys@transformshift{1.801343in}{3.215277in}%
\pgfsys@useobject{currentmarker}{}%
\end{pgfscope}%
\begin{pgfscope}%
\pgfsys@transformshift{1.782330in}{3.347850in}%
\pgfsys@useobject{currentmarker}{}%
\end{pgfscope}%
\begin{pgfscope}%
\pgfsys@transformshift{1.763787in}{3.454948in}%
\pgfsys@useobject{currentmarker}{}%
\end{pgfscope}%
\begin{pgfscope}%
\pgfsys@transformshift{1.744539in}{3.472473in}%
\pgfsys@useobject{currentmarker}{}%
\end{pgfscope}%
\begin{pgfscope}%
\pgfsys@transformshift{1.725760in}{3.356029in}%
\pgfsys@useobject{currentmarker}{}%
\end{pgfscope}%
\begin{pgfscope}%
\pgfsys@transformshift{1.704636in}{3.234926in}%
\pgfsys@useobject{currentmarker}{}%
\end{pgfscope}%
\begin{pgfscope}%
\pgfsys@transformshift{1.686560in}{3.178617in}%
\pgfsys@useobject{currentmarker}{}%
\end{pgfscope}%
\begin{pgfscope}%
\pgfsys@transformshift{1.671069in}{3.155609in}%
\pgfsys@useobject{currentmarker}{}%
\end{pgfscope}%
\begin{pgfscope}%
\pgfsys@transformshift{1.650178in}{3.143242in}%
\pgfsys@useobject{currentmarker}{}%
\end{pgfscope}%
\begin{pgfscope}%
\pgfsys@transformshift{1.629522in}{3.140196in}%
\pgfsys@useobject{currentmarker}{}%
\end{pgfscope}%
\begin{pgfscope}%
\pgfsys@transformshift{1.611917in}{3.139329in}%
\pgfsys@useobject{currentmarker}{}%
\end{pgfscope}%
\begin{pgfscope}%
\pgfsys@transformshift{1.593373in}{3.143685in}%
\pgfsys@useobject{currentmarker}{}%
\end{pgfscope}%
\begin{pgfscope}%
\pgfsys@transformshift{1.574594in}{3.153288in}%
\pgfsys@useobject{currentmarker}{}%
\end{pgfscope}%
\end{pgfscope}%
\begin{pgfscope}%
\pgfsetrectcap%
\pgfsetmiterjoin%
\pgfsetlinewidth{0.501875pt}%
\definecolor{currentstroke}{rgb}{0.000000,0.000000,0.000000}%
\pgfsetstrokecolor{currentstroke}%
\pgfsetdash{}{0pt}%
\pgfpathmoveto{\pgfqpoint{0.444748in}{3.117349in}}%
\pgfpathlineto{\pgfqpoint{0.444748in}{3.584600in}}%
\pgfusepath{stroke}%
\end{pgfscope}%
\begin{pgfscope}%
\pgfsetrectcap%
\pgfsetmiterjoin%
\pgfsetlinewidth{0.501875pt}%
\definecolor{currentstroke}{rgb}{0.000000,0.000000,0.000000}%
\pgfsetstrokecolor{currentstroke}%
\pgfsetdash{}{0pt}%
\pgfpathmoveto{\pgfqpoint{4.676167in}{3.117349in}}%
\pgfpathlineto{\pgfqpoint{4.676167in}{3.584600in}}%
\pgfusepath{stroke}%
\end{pgfscope}%
\begin{pgfscope}%
\pgfsetrectcap%
\pgfsetmiterjoin%
\pgfsetlinewidth{0.501875pt}%
\definecolor{currentstroke}{rgb}{0.000000,0.000000,0.000000}%
\pgfsetstrokecolor{currentstroke}%
\pgfsetdash{}{0pt}%
\pgfpathmoveto{\pgfqpoint{0.444748in}{3.117349in}}%
\pgfpathlineto{\pgfqpoint{4.676167in}{3.117349in}}%
\pgfusepath{stroke}%
\end{pgfscope}%
\begin{pgfscope}%
\pgfsetrectcap%
\pgfsetmiterjoin%
\pgfsetlinewidth{0.501875pt}%
\definecolor{currentstroke}{rgb}{0.000000,0.000000,0.000000}%
\pgfsetstrokecolor{currentstroke}%
\pgfsetdash{}{0pt}%
\pgfpathmoveto{\pgfqpoint{0.444748in}{3.584600in}}%
\pgfpathlineto{\pgfqpoint{4.676167in}{3.584600in}}%
\pgfusepath{stroke}%
\end{pgfscope}%
\begin{pgfscope}%
\definecolor{textcolor}{rgb}{0.000000,0.000000,0.000000}%
\pgfsetstrokecolor{textcolor}%
\pgfsetfillcolor{textcolor}%
\pgftext[x=2.560458in,y=3.667933in,,base]{\color{textcolor}\rmfamily\fontsize{12.000000}{14.400000}\selectfont T = \qty{3.0}{\kelvin}}%
\end{pgfscope}%
\begin{pgfscope}%
\pgfsetbuttcap%
\pgfsetmiterjoin%
\definecolor{currentfill}{rgb}{1.000000,1.000000,1.000000}%
\pgfsetfillcolor{currentfill}%
\pgfsetlinewidth{0.000000pt}%
\definecolor{currentstroke}{rgb}{0.000000,0.000000,0.000000}%
\pgfsetstrokecolor{currentstroke}%
\pgfsetstrokeopacity{0.000000}%
\pgfsetdash{}{0pt}%
\pgfpathmoveto{\pgfqpoint{0.444748in}{2.222124in}}%
\pgfpathlineto{\pgfqpoint{4.676167in}{2.222124in}}%
\pgfpathlineto{\pgfqpoint{4.676167in}{2.689374in}}%
\pgfpathlineto{\pgfqpoint{0.444748in}{2.689374in}}%
\pgfpathlineto{\pgfqpoint{0.444748in}{2.222124in}}%
\pgfpathclose%
\pgfusepath{fill}%
\end{pgfscope}%
\begin{pgfscope}%
\pgfsetbuttcap%
\pgfsetroundjoin%
\definecolor{currentfill}{rgb}{0.000000,0.000000,0.000000}%
\pgfsetfillcolor{currentfill}%
\pgfsetlinewidth{0.501875pt}%
\definecolor{currentstroke}{rgb}{0.000000,0.000000,0.000000}%
\pgfsetstrokecolor{currentstroke}%
\pgfsetdash{}{0pt}%
\pgfsys@defobject{currentmarker}{\pgfqpoint{0.000000in}{0.000000in}}{\pgfqpoint{0.000000in}{0.041667in}}{%
\pgfpathmoveto{\pgfqpoint{0.000000in}{0.000000in}}%
\pgfpathlineto{\pgfqpoint{0.000000in}{0.041667in}}%
\pgfusepath{stroke,fill}%
}%
\begin{pgfscope}%
\pgfsys@transformshift{0.643886in}{2.222124in}%
\pgfsys@useobject{currentmarker}{}%
\end{pgfscope}%
\end{pgfscope}%
\begin{pgfscope}%
\pgfsetbuttcap%
\pgfsetroundjoin%
\definecolor{currentfill}{rgb}{0.000000,0.000000,0.000000}%
\pgfsetfillcolor{currentfill}%
\pgfsetlinewidth{0.501875pt}%
\definecolor{currentstroke}{rgb}{0.000000,0.000000,0.000000}%
\pgfsetstrokecolor{currentstroke}%
\pgfsetdash{}{0pt}%
\pgfsys@defobject{currentmarker}{\pgfqpoint{0.000000in}{-0.041667in}}{\pgfqpoint{0.000000in}{0.000000in}}{%
\pgfpathmoveto{\pgfqpoint{0.000000in}{0.000000in}}%
\pgfpathlineto{\pgfqpoint{0.000000in}{-0.041667in}}%
\pgfusepath{stroke,fill}%
}%
\begin{pgfscope}%
\pgfsys@transformshift{0.643886in}{2.689374in}%
\pgfsys@useobject{currentmarker}{}%
\end{pgfscope}%
\end{pgfscope}%
\begin{pgfscope}%
\pgfsetbuttcap%
\pgfsetroundjoin%
\definecolor{currentfill}{rgb}{0.000000,0.000000,0.000000}%
\pgfsetfillcolor{currentfill}%
\pgfsetlinewidth{0.501875pt}%
\definecolor{currentstroke}{rgb}{0.000000,0.000000,0.000000}%
\pgfsetstrokecolor{currentstroke}%
\pgfsetdash{}{0pt}%
\pgfsys@defobject{currentmarker}{\pgfqpoint{0.000000in}{0.000000in}}{\pgfqpoint{0.000000in}{0.041667in}}{%
\pgfpathmoveto{\pgfqpoint{0.000000in}{0.000000in}}%
\pgfpathlineto{\pgfqpoint{0.000000in}{0.041667in}}%
\pgfusepath{stroke,fill}%
}%
\begin{pgfscope}%
\pgfsys@transformshift{1.124261in}{2.222124in}%
\pgfsys@useobject{currentmarker}{}%
\end{pgfscope}%
\end{pgfscope}%
\begin{pgfscope}%
\pgfsetbuttcap%
\pgfsetroundjoin%
\definecolor{currentfill}{rgb}{0.000000,0.000000,0.000000}%
\pgfsetfillcolor{currentfill}%
\pgfsetlinewidth{0.501875pt}%
\definecolor{currentstroke}{rgb}{0.000000,0.000000,0.000000}%
\pgfsetstrokecolor{currentstroke}%
\pgfsetdash{}{0pt}%
\pgfsys@defobject{currentmarker}{\pgfqpoint{0.000000in}{-0.041667in}}{\pgfqpoint{0.000000in}{0.000000in}}{%
\pgfpathmoveto{\pgfqpoint{0.000000in}{0.000000in}}%
\pgfpathlineto{\pgfqpoint{0.000000in}{-0.041667in}}%
\pgfusepath{stroke,fill}%
}%
\begin{pgfscope}%
\pgfsys@transformshift{1.124261in}{2.689374in}%
\pgfsys@useobject{currentmarker}{}%
\end{pgfscope}%
\end{pgfscope}%
\begin{pgfscope}%
\pgfsetbuttcap%
\pgfsetroundjoin%
\definecolor{currentfill}{rgb}{0.000000,0.000000,0.000000}%
\pgfsetfillcolor{currentfill}%
\pgfsetlinewidth{0.501875pt}%
\definecolor{currentstroke}{rgb}{0.000000,0.000000,0.000000}%
\pgfsetstrokecolor{currentstroke}%
\pgfsetdash{}{0pt}%
\pgfsys@defobject{currentmarker}{\pgfqpoint{0.000000in}{0.000000in}}{\pgfqpoint{0.000000in}{0.041667in}}{%
\pgfpathmoveto{\pgfqpoint{0.000000in}{0.000000in}}%
\pgfpathlineto{\pgfqpoint{0.000000in}{0.041667in}}%
\pgfusepath{stroke,fill}%
}%
\begin{pgfscope}%
\pgfsys@transformshift{1.604637in}{2.222124in}%
\pgfsys@useobject{currentmarker}{}%
\end{pgfscope}%
\end{pgfscope}%
\begin{pgfscope}%
\pgfsetbuttcap%
\pgfsetroundjoin%
\definecolor{currentfill}{rgb}{0.000000,0.000000,0.000000}%
\pgfsetfillcolor{currentfill}%
\pgfsetlinewidth{0.501875pt}%
\definecolor{currentstroke}{rgb}{0.000000,0.000000,0.000000}%
\pgfsetstrokecolor{currentstroke}%
\pgfsetdash{}{0pt}%
\pgfsys@defobject{currentmarker}{\pgfqpoint{0.000000in}{-0.041667in}}{\pgfqpoint{0.000000in}{0.000000in}}{%
\pgfpathmoveto{\pgfqpoint{0.000000in}{0.000000in}}%
\pgfpathlineto{\pgfqpoint{0.000000in}{-0.041667in}}%
\pgfusepath{stroke,fill}%
}%
\begin{pgfscope}%
\pgfsys@transformshift{1.604637in}{2.689374in}%
\pgfsys@useobject{currentmarker}{}%
\end{pgfscope}%
\end{pgfscope}%
\begin{pgfscope}%
\pgfsetbuttcap%
\pgfsetroundjoin%
\definecolor{currentfill}{rgb}{0.000000,0.000000,0.000000}%
\pgfsetfillcolor{currentfill}%
\pgfsetlinewidth{0.501875pt}%
\definecolor{currentstroke}{rgb}{0.000000,0.000000,0.000000}%
\pgfsetstrokecolor{currentstroke}%
\pgfsetdash{}{0pt}%
\pgfsys@defobject{currentmarker}{\pgfqpoint{0.000000in}{0.000000in}}{\pgfqpoint{0.000000in}{0.041667in}}{%
\pgfpathmoveto{\pgfqpoint{0.000000in}{0.000000in}}%
\pgfpathlineto{\pgfqpoint{0.000000in}{0.041667in}}%
\pgfusepath{stroke,fill}%
}%
\begin{pgfscope}%
\pgfsys@transformshift{2.085012in}{2.222124in}%
\pgfsys@useobject{currentmarker}{}%
\end{pgfscope}%
\end{pgfscope}%
\begin{pgfscope}%
\pgfsetbuttcap%
\pgfsetroundjoin%
\definecolor{currentfill}{rgb}{0.000000,0.000000,0.000000}%
\pgfsetfillcolor{currentfill}%
\pgfsetlinewidth{0.501875pt}%
\definecolor{currentstroke}{rgb}{0.000000,0.000000,0.000000}%
\pgfsetstrokecolor{currentstroke}%
\pgfsetdash{}{0pt}%
\pgfsys@defobject{currentmarker}{\pgfqpoint{0.000000in}{-0.041667in}}{\pgfqpoint{0.000000in}{0.000000in}}{%
\pgfpathmoveto{\pgfqpoint{0.000000in}{0.000000in}}%
\pgfpathlineto{\pgfqpoint{0.000000in}{-0.041667in}}%
\pgfusepath{stroke,fill}%
}%
\begin{pgfscope}%
\pgfsys@transformshift{2.085012in}{2.689374in}%
\pgfsys@useobject{currentmarker}{}%
\end{pgfscope}%
\end{pgfscope}%
\begin{pgfscope}%
\pgfsetbuttcap%
\pgfsetroundjoin%
\definecolor{currentfill}{rgb}{0.000000,0.000000,0.000000}%
\pgfsetfillcolor{currentfill}%
\pgfsetlinewidth{0.501875pt}%
\definecolor{currentstroke}{rgb}{0.000000,0.000000,0.000000}%
\pgfsetstrokecolor{currentstroke}%
\pgfsetdash{}{0pt}%
\pgfsys@defobject{currentmarker}{\pgfqpoint{0.000000in}{0.000000in}}{\pgfqpoint{0.000000in}{0.041667in}}{%
\pgfpathmoveto{\pgfqpoint{0.000000in}{0.000000in}}%
\pgfpathlineto{\pgfqpoint{0.000000in}{0.041667in}}%
\pgfusepath{stroke,fill}%
}%
\begin{pgfscope}%
\pgfsys@transformshift{2.565388in}{2.222124in}%
\pgfsys@useobject{currentmarker}{}%
\end{pgfscope}%
\end{pgfscope}%
\begin{pgfscope}%
\pgfsetbuttcap%
\pgfsetroundjoin%
\definecolor{currentfill}{rgb}{0.000000,0.000000,0.000000}%
\pgfsetfillcolor{currentfill}%
\pgfsetlinewidth{0.501875pt}%
\definecolor{currentstroke}{rgb}{0.000000,0.000000,0.000000}%
\pgfsetstrokecolor{currentstroke}%
\pgfsetdash{}{0pt}%
\pgfsys@defobject{currentmarker}{\pgfqpoint{0.000000in}{-0.041667in}}{\pgfqpoint{0.000000in}{0.000000in}}{%
\pgfpathmoveto{\pgfqpoint{0.000000in}{0.000000in}}%
\pgfpathlineto{\pgfqpoint{0.000000in}{-0.041667in}}%
\pgfusepath{stroke,fill}%
}%
\begin{pgfscope}%
\pgfsys@transformshift{2.565388in}{2.689374in}%
\pgfsys@useobject{currentmarker}{}%
\end{pgfscope}%
\end{pgfscope}%
\begin{pgfscope}%
\pgfsetbuttcap%
\pgfsetroundjoin%
\definecolor{currentfill}{rgb}{0.000000,0.000000,0.000000}%
\pgfsetfillcolor{currentfill}%
\pgfsetlinewidth{0.501875pt}%
\definecolor{currentstroke}{rgb}{0.000000,0.000000,0.000000}%
\pgfsetstrokecolor{currentstroke}%
\pgfsetdash{}{0pt}%
\pgfsys@defobject{currentmarker}{\pgfqpoint{0.000000in}{0.000000in}}{\pgfqpoint{0.000000in}{0.041667in}}{%
\pgfpathmoveto{\pgfqpoint{0.000000in}{0.000000in}}%
\pgfpathlineto{\pgfqpoint{0.000000in}{0.041667in}}%
\pgfusepath{stroke,fill}%
}%
\begin{pgfscope}%
\pgfsys@transformshift{3.045763in}{2.222124in}%
\pgfsys@useobject{currentmarker}{}%
\end{pgfscope}%
\end{pgfscope}%
\begin{pgfscope}%
\pgfsetbuttcap%
\pgfsetroundjoin%
\definecolor{currentfill}{rgb}{0.000000,0.000000,0.000000}%
\pgfsetfillcolor{currentfill}%
\pgfsetlinewidth{0.501875pt}%
\definecolor{currentstroke}{rgb}{0.000000,0.000000,0.000000}%
\pgfsetstrokecolor{currentstroke}%
\pgfsetdash{}{0pt}%
\pgfsys@defobject{currentmarker}{\pgfqpoint{0.000000in}{-0.041667in}}{\pgfqpoint{0.000000in}{0.000000in}}{%
\pgfpathmoveto{\pgfqpoint{0.000000in}{0.000000in}}%
\pgfpathlineto{\pgfqpoint{0.000000in}{-0.041667in}}%
\pgfusepath{stroke,fill}%
}%
\begin{pgfscope}%
\pgfsys@transformshift{3.045763in}{2.689374in}%
\pgfsys@useobject{currentmarker}{}%
\end{pgfscope}%
\end{pgfscope}%
\begin{pgfscope}%
\pgfsetbuttcap%
\pgfsetroundjoin%
\definecolor{currentfill}{rgb}{0.000000,0.000000,0.000000}%
\pgfsetfillcolor{currentfill}%
\pgfsetlinewidth{0.501875pt}%
\definecolor{currentstroke}{rgb}{0.000000,0.000000,0.000000}%
\pgfsetstrokecolor{currentstroke}%
\pgfsetdash{}{0pt}%
\pgfsys@defobject{currentmarker}{\pgfqpoint{0.000000in}{0.000000in}}{\pgfqpoint{0.000000in}{0.041667in}}{%
\pgfpathmoveto{\pgfqpoint{0.000000in}{0.000000in}}%
\pgfpathlineto{\pgfqpoint{0.000000in}{0.041667in}}%
\pgfusepath{stroke,fill}%
}%
\begin{pgfscope}%
\pgfsys@transformshift{3.526138in}{2.222124in}%
\pgfsys@useobject{currentmarker}{}%
\end{pgfscope}%
\end{pgfscope}%
\begin{pgfscope}%
\pgfsetbuttcap%
\pgfsetroundjoin%
\definecolor{currentfill}{rgb}{0.000000,0.000000,0.000000}%
\pgfsetfillcolor{currentfill}%
\pgfsetlinewidth{0.501875pt}%
\definecolor{currentstroke}{rgb}{0.000000,0.000000,0.000000}%
\pgfsetstrokecolor{currentstroke}%
\pgfsetdash{}{0pt}%
\pgfsys@defobject{currentmarker}{\pgfqpoint{0.000000in}{-0.041667in}}{\pgfqpoint{0.000000in}{0.000000in}}{%
\pgfpathmoveto{\pgfqpoint{0.000000in}{0.000000in}}%
\pgfpathlineto{\pgfqpoint{0.000000in}{-0.041667in}}%
\pgfusepath{stroke,fill}%
}%
\begin{pgfscope}%
\pgfsys@transformshift{3.526138in}{2.689374in}%
\pgfsys@useobject{currentmarker}{}%
\end{pgfscope}%
\end{pgfscope}%
\begin{pgfscope}%
\pgfsetbuttcap%
\pgfsetroundjoin%
\definecolor{currentfill}{rgb}{0.000000,0.000000,0.000000}%
\pgfsetfillcolor{currentfill}%
\pgfsetlinewidth{0.501875pt}%
\definecolor{currentstroke}{rgb}{0.000000,0.000000,0.000000}%
\pgfsetstrokecolor{currentstroke}%
\pgfsetdash{}{0pt}%
\pgfsys@defobject{currentmarker}{\pgfqpoint{0.000000in}{0.000000in}}{\pgfqpoint{0.000000in}{0.041667in}}{%
\pgfpathmoveto{\pgfqpoint{0.000000in}{0.000000in}}%
\pgfpathlineto{\pgfqpoint{0.000000in}{0.041667in}}%
\pgfusepath{stroke,fill}%
}%
\begin{pgfscope}%
\pgfsys@transformshift{4.006514in}{2.222124in}%
\pgfsys@useobject{currentmarker}{}%
\end{pgfscope}%
\end{pgfscope}%
\begin{pgfscope}%
\pgfsetbuttcap%
\pgfsetroundjoin%
\definecolor{currentfill}{rgb}{0.000000,0.000000,0.000000}%
\pgfsetfillcolor{currentfill}%
\pgfsetlinewidth{0.501875pt}%
\definecolor{currentstroke}{rgb}{0.000000,0.000000,0.000000}%
\pgfsetstrokecolor{currentstroke}%
\pgfsetdash{}{0pt}%
\pgfsys@defobject{currentmarker}{\pgfqpoint{0.000000in}{-0.041667in}}{\pgfqpoint{0.000000in}{0.000000in}}{%
\pgfpathmoveto{\pgfqpoint{0.000000in}{0.000000in}}%
\pgfpathlineto{\pgfqpoint{0.000000in}{-0.041667in}}%
\pgfusepath{stroke,fill}%
}%
\begin{pgfscope}%
\pgfsys@transformshift{4.006514in}{2.689374in}%
\pgfsys@useobject{currentmarker}{}%
\end{pgfscope}%
\end{pgfscope}%
\begin{pgfscope}%
\pgfsetbuttcap%
\pgfsetroundjoin%
\definecolor{currentfill}{rgb}{0.000000,0.000000,0.000000}%
\pgfsetfillcolor{currentfill}%
\pgfsetlinewidth{0.501875pt}%
\definecolor{currentstroke}{rgb}{0.000000,0.000000,0.000000}%
\pgfsetstrokecolor{currentstroke}%
\pgfsetdash{}{0pt}%
\pgfsys@defobject{currentmarker}{\pgfqpoint{0.000000in}{0.000000in}}{\pgfqpoint{0.000000in}{0.041667in}}{%
\pgfpathmoveto{\pgfqpoint{0.000000in}{0.000000in}}%
\pgfpathlineto{\pgfqpoint{0.000000in}{0.041667in}}%
\pgfusepath{stroke,fill}%
}%
\begin{pgfscope}%
\pgfsys@transformshift{4.486889in}{2.222124in}%
\pgfsys@useobject{currentmarker}{}%
\end{pgfscope}%
\end{pgfscope}%
\begin{pgfscope}%
\pgfsetbuttcap%
\pgfsetroundjoin%
\definecolor{currentfill}{rgb}{0.000000,0.000000,0.000000}%
\pgfsetfillcolor{currentfill}%
\pgfsetlinewidth{0.501875pt}%
\definecolor{currentstroke}{rgb}{0.000000,0.000000,0.000000}%
\pgfsetstrokecolor{currentstroke}%
\pgfsetdash{}{0pt}%
\pgfsys@defobject{currentmarker}{\pgfqpoint{0.000000in}{-0.041667in}}{\pgfqpoint{0.000000in}{0.000000in}}{%
\pgfpathmoveto{\pgfqpoint{0.000000in}{0.000000in}}%
\pgfpathlineto{\pgfqpoint{0.000000in}{-0.041667in}}%
\pgfusepath{stroke,fill}%
}%
\begin{pgfscope}%
\pgfsys@transformshift{4.486889in}{2.689374in}%
\pgfsys@useobject{currentmarker}{}%
\end{pgfscope}%
\end{pgfscope}%
\begin{pgfscope}%
\pgfsetbuttcap%
\pgfsetroundjoin%
\definecolor{currentfill}{rgb}{0.000000,0.000000,0.000000}%
\pgfsetfillcolor{currentfill}%
\pgfsetlinewidth{0.501875pt}%
\definecolor{currentstroke}{rgb}{0.000000,0.000000,0.000000}%
\pgfsetstrokecolor{currentstroke}%
\pgfsetdash{}{0pt}%
\pgfsys@defobject{currentmarker}{\pgfqpoint{0.000000in}{0.000000in}}{\pgfqpoint{0.000000in}{0.020833in}}{%
\pgfpathmoveto{\pgfqpoint{0.000000in}{0.000000in}}%
\pgfpathlineto{\pgfqpoint{0.000000in}{0.020833in}}%
\pgfusepath{stroke,fill}%
}%
\begin{pgfscope}%
\pgfsys@transformshift{0.451736in}{2.222124in}%
\pgfsys@useobject{currentmarker}{}%
\end{pgfscope}%
\end{pgfscope}%
\begin{pgfscope}%
\pgfsetbuttcap%
\pgfsetroundjoin%
\definecolor{currentfill}{rgb}{0.000000,0.000000,0.000000}%
\pgfsetfillcolor{currentfill}%
\pgfsetlinewidth{0.501875pt}%
\definecolor{currentstroke}{rgb}{0.000000,0.000000,0.000000}%
\pgfsetstrokecolor{currentstroke}%
\pgfsetdash{}{0pt}%
\pgfsys@defobject{currentmarker}{\pgfqpoint{0.000000in}{-0.020833in}}{\pgfqpoint{0.000000in}{0.000000in}}{%
\pgfpathmoveto{\pgfqpoint{0.000000in}{0.000000in}}%
\pgfpathlineto{\pgfqpoint{0.000000in}{-0.020833in}}%
\pgfusepath{stroke,fill}%
}%
\begin{pgfscope}%
\pgfsys@transformshift{0.451736in}{2.689374in}%
\pgfsys@useobject{currentmarker}{}%
\end{pgfscope}%
\end{pgfscope}%
\begin{pgfscope}%
\pgfsetbuttcap%
\pgfsetroundjoin%
\definecolor{currentfill}{rgb}{0.000000,0.000000,0.000000}%
\pgfsetfillcolor{currentfill}%
\pgfsetlinewidth{0.501875pt}%
\definecolor{currentstroke}{rgb}{0.000000,0.000000,0.000000}%
\pgfsetstrokecolor{currentstroke}%
\pgfsetdash{}{0pt}%
\pgfsys@defobject{currentmarker}{\pgfqpoint{0.000000in}{0.000000in}}{\pgfqpoint{0.000000in}{0.020833in}}{%
\pgfpathmoveto{\pgfqpoint{0.000000in}{0.000000in}}%
\pgfpathlineto{\pgfqpoint{0.000000in}{0.020833in}}%
\pgfusepath{stroke,fill}%
}%
\begin{pgfscope}%
\pgfsys@transformshift{0.547811in}{2.222124in}%
\pgfsys@useobject{currentmarker}{}%
\end{pgfscope}%
\end{pgfscope}%
\begin{pgfscope}%
\pgfsetbuttcap%
\pgfsetroundjoin%
\definecolor{currentfill}{rgb}{0.000000,0.000000,0.000000}%
\pgfsetfillcolor{currentfill}%
\pgfsetlinewidth{0.501875pt}%
\definecolor{currentstroke}{rgb}{0.000000,0.000000,0.000000}%
\pgfsetstrokecolor{currentstroke}%
\pgfsetdash{}{0pt}%
\pgfsys@defobject{currentmarker}{\pgfqpoint{0.000000in}{-0.020833in}}{\pgfqpoint{0.000000in}{0.000000in}}{%
\pgfpathmoveto{\pgfqpoint{0.000000in}{0.000000in}}%
\pgfpathlineto{\pgfqpoint{0.000000in}{-0.020833in}}%
\pgfusepath{stroke,fill}%
}%
\begin{pgfscope}%
\pgfsys@transformshift{0.547811in}{2.689374in}%
\pgfsys@useobject{currentmarker}{}%
\end{pgfscope}%
\end{pgfscope}%
\begin{pgfscope}%
\pgfsetbuttcap%
\pgfsetroundjoin%
\definecolor{currentfill}{rgb}{0.000000,0.000000,0.000000}%
\pgfsetfillcolor{currentfill}%
\pgfsetlinewidth{0.501875pt}%
\definecolor{currentstroke}{rgb}{0.000000,0.000000,0.000000}%
\pgfsetstrokecolor{currentstroke}%
\pgfsetdash{}{0pt}%
\pgfsys@defobject{currentmarker}{\pgfqpoint{0.000000in}{0.000000in}}{\pgfqpoint{0.000000in}{0.020833in}}{%
\pgfpathmoveto{\pgfqpoint{0.000000in}{0.000000in}}%
\pgfpathlineto{\pgfqpoint{0.000000in}{0.020833in}}%
\pgfusepath{stroke,fill}%
}%
\begin{pgfscope}%
\pgfsys@transformshift{0.739961in}{2.222124in}%
\pgfsys@useobject{currentmarker}{}%
\end{pgfscope}%
\end{pgfscope}%
\begin{pgfscope}%
\pgfsetbuttcap%
\pgfsetroundjoin%
\definecolor{currentfill}{rgb}{0.000000,0.000000,0.000000}%
\pgfsetfillcolor{currentfill}%
\pgfsetlinewidth{0.501875pt}%
\definecolor{currentstroke}{rgb}{0.000000,0.000000,0.000000}%
\pgfsetstrokecolor{currentstroke}%
\pgfsetdash{}{0pt}%
\pgfsys@defobject{currentmarker}{\pgfqpoint{0.000000in}{-0.020833in}}{\pgfqpoint{0.000000in}{0.000000in}}{%
\pgfpathmoveto{\pgfqpoint{0.000000in}{0.000000in}}%
\pgfpathlineto{\pgfqpoint{0.000000in}{-0.020833in}}%
\pgfusepath{stroke,fill}%
}%
\begin{pgfscope}%
\pgfsys@transformshift{0.739961in}{2.689374in}%
\pgfsys@useobject{currentmarker}{}%
\end{pgfscope}%
\end{pgfscope}%
\begin{pgfscope}%
\pgfsetbuttcap%
\pgfsetroundjoin%
\definecolor{currentfill}{rgb}{0.000000,0.000000,0.000000}%
\pgfsetfillcolor{currentfill}%
\pgfsetlinewidth{0.501875pt}%
\definecolor{currentstroke}{rgb}{0.000000,0.000000,0.000000}%
\pgfsetstrokecolor{currentstroke}%
\pgfsetdash{}{0pt}%
\pgfsys@defobject{currentmarker}{\pgfqpoint{0.000000in}{0.000000in}}{\pgfqpoint{0.000000in}{0.020833in}}{%
\pgfpathmoveto{\pgfqpoint{0.000000in}{0.000000in}}%
\pgfpathlineto{\pgfqpoint{0.000000in}{0.020833in}}%
\pgfusepath{stroke,fill}%
}%
\begin{pgfscope}%
\pgfsys@transformshift{0.836036in}{2.222124in}%
\pgfsys@useobject{currentmarker}{}%
\end{pgfscope}%
\end{pgfscope}%
\begin{pgfscope}%
\pgfsetbuttcap%
\pgfsetroundjoin%
\definecolor{currentfill}{rgb}{0.000000,0.000000,0.000000}%
\pgfsetfillcolor{currentfill}%
\pgfsetlinewidth{0.501875pt}%
\definecolor{currentstroke}{rgb}{0.000000,0.000000,0.000000}%
\pgfsetstrokecolor{currentstroke}%
\pgfsetdash{}{0pt}%
\pgfsys@defobject{currentmarker}{\pgfqpoint{0.000000in}{-0.020833in}}{\pgfqpoint{0.000000in}{0.000000in}}{%
\pgfpathmoveto{\pgfqpoint{0.000000in}{0.000000in}}%
\pgfpathlineto{\pgfqpoint{0.000000in}{-0.020833in}}%
\pgfusepath{stroke,fill}%
}%
\begin{pgfscope}%
\pgfsys@transformshift{0.836036in}{2.689374in}%
\pgfsys@useobject{currentmarker}{}%
\end{pgfscope}%
\end{pgfscope}%
\begin{pgfscope}%
\pgfsetbuttcap%
\pgfsetroundjoin%
\definecolor{currentfill}{rgb}{0.000000,0.000000,0.000000}%
\pgfsetfillcolor{currentfill}%
\pgfsetlinewidth{0.501875pt}%
\definecolor{currentstroke}{rgb}{0.000000,0.000000,0.000000}%
\pgfsetstrokecolor{currentstroke}%
\pgfsetdash{}{0pt}%
\pgfsys@defobject{currentmarker}{\pgfqpoint{0.000000in}{0.000000in}}{\pgfqpoint{0.000000in}{0.020833in}}{%
\pgfpathmoveto{\pgfqpoint{0.000000in}{0.000000in}}%
\pgfpathlineto{\pgfqpoint{0.000000in}{0.020833in}}%
\pgfusepath{stroke,fill}%
}%
\begin{pgfscope}%
\pgfsys@transformshift{0.932111in}{2.222124in}%
\pgfsys@useobject{currentmarker}{}%
\end{pgfscope}%
\end{pgfscope}%
\begin{pgfscope}%
\pgfsetbuttcap%
\pgfsetroundjoin%
\definecolor{currentfill}{rgb}{0.000000,0.000000,0.000000}%
\pgfsetfillcolor{currentfill}%
\pgfsetlinewidth{0.501875pt}%
\definecolor{currentstroke}{rgb}{0.000000,0.000000,0.000000}%
\pgfsetstrokecolor{currentstroke}%
\pgfsetdash{}{0pt}%
\pgfsys@defobject{currentmarker}{\pgfqpoint{0.000000in}{-0.020833in}}{\pgfqpoint{0.000000in}{0.000000in}}{%
\pgfpathmoveto{\pgfqpoint{0.000000in}{0.000000in}}%
\pgfpathlineto{\pgfqpoint{0.000000in}{-0.020833in}}%
\pgfusepath{stroke,fill}%
}%
\begin{pgfscope}%
\pgfsys@transformshift{0.932111in}{2.689374in}%
\pgfsys@useobject{currentmarker}{}%
\end{pgfscope}%
\end{pgfscope}%
\begin{pgfscope}%
\pgfsetbuttcap%
\pgfsetroundjoin%
\definecolor{currentfill}{rgb}{0.000000,0.000000,0.000000}%
\pgfsetfillcolor{currentfill}%
\pgfsetlinewidth{0.501875pt}%
\definecolor{currentstroke}{rgb}{0.000000,0.000000,0.000000}%
\pgfsetstrokecolor{currentstroke}%
\pgfsetdash{}{0pt}%
\pgfsys@defobject{currentmarker}{\pgfqpoint{0.000000in}{0.000000in}}{\pgfqpoint{0.000000in}{0.020833in}}{%
\pgfpathmoveto{\pgfqpoint{0.000000in}{0.000000in}}%
\pgfpathlineto{\pgfqpoint{0.000000in}{0.020833in}}%
\pgfusepath{stroke,fill}%
}%
\begin{pgfscope}%
\pgfsys@transformshift{1.028186in}{2.222124in}%
\pgfsys@useobject{currentmarker}{}%
\end{pgfscope}%
\end{pgfscope}%
\begin{pgfscope}%
\pgfsetbuttcap%
\pgfsetroundjoin%
\definecolor{currentfill}{rgb}{0.000000,0.000000,0.000000}%
\pgfsetfillcolor{currentfill}%
\pgfsetlinewidth{0.501875pt}%
\definecolor{currentstroke}{rgb}{0.000000,0.000000,0.000000}%
\pgfsetstrokecolor{currentstroke}%
\pgfsetdash{}{0pt}%
\pgfsys@defobject{currentmarker}{\pgfqpoint{0.000000in}{-0.020833in}}{\pgfqpoint{0.000000in}{0.000000in}}{%
\pgfpathmoveto{\pgfqpoint{0.000000in}{0.000000in}}%
\pgfpathlineto{\pgfqpoint{0.000000in}{-0.020833in}}%
\pgfusepath{stroke,fill}%
}%
\begin{pgfscope}%
\pgfsys@transformshift{1.028186in}{2.689374in}%
\pgfsys@useobject{currentmarker}{}%
\end{pgfscope}%
\end{pgfscope}%
\begin{pgfscope}%
\pgfsetbuttcap%
\pgfsetroundjoin%
\definecolor{currentfill}{rgb}{0.000000,0.000000,0.000000}%
\pgfsetfillcolor{currentfill}%
\pgfsetlinewidth{0.501875pt}%
\definecolor{currentstroke}{rgb}{0.000000,0.000000,0.000000}%
\pgfsetstrokecolor{currentstroke}%
\pgfsetdash{}{0pt}%
\pgfsys@defobject{currentmarker}{\pgfqpoint{0.000000in}{0.000000in}}{\pgfqpoint{0.000000in}{0.020833in}}{%
\pgfpathmoveto{\pgfqpoint{0.000000in}{0.000000in}}%
\pgfpathlineto{\pgfqpoint{0.000000in}{0.020833in}}%
\pgfusepath{stroke,fill}%
}%
\begin{pgfscope}%
\pgfsys@transformshift{1.220336in}{2.222124in}%
\pgfsys@useobject{currentmarker}{}%
\end{pgfscope}%
\end{pgfscope}%
\begin{pgfscope}%
\pgfsetbuttcap%
\pgfsetroundjoin%
\definecolor{currentfill}{rgb}{0.000000,0.000000,0.000000}%
\pgfsetfillcolor{currentfill}%
\pgfsetlinewidth{0.501875pt}%
\definecolor{currentstroke}{rgb}{0.000000,0.000000,0.000000}%
\pgfsetstrokecolor{currentstroke}%
\pgfsetdash{}{0pt}%
\pgfsys@defobject{currentmarker}{\pgfqpoint{0.000000in}{-0.020833in}}{\pgfqpoint{0.000000in}{0.000000in}}{%
\pgfpathmoveto{\pgfqpoint{0.000000in}{0.000000in}}%
\pgfpathlineto{\pgfqpoint{0.000000in}{-0.020833in}}%
\pgfusepath{stroke,fill}%
}%
\begin{pgfscope}%
\pgfsys@transformshift{1.220336in}{2.689374in}%
\pgfsys@useobject{currentmarker}{}%
\end{pgfscope}%
\end{pgfscope}%
\begin{pgfscope}%
\pgfsetbuttcap%
\pgfsetroundjoin%
\definecolor{currentfill}{rgb}{0.000000,0.000000,0.000000}%
\pgfsetfillcolor{currentfill}%
\pgfsetlinewidth{0.501875pt}%
\definecolor{currentstroke}{rgb}{0.000000,0.000000,0.000000}%
\pgfsetstrokecolor{currentstroke}%
\pgfsetdash{}{0pt}%
\pgfsys@defobject{currentmarker}{\pgfqpoint{0.000000in}{0.000000in}}{\pgfqpoint{0.000000in}{0.020833in}}{%
\pgfpathmoveto{\pgfqpoint{0.000000in}{0.000000in}}%
\pgfpathlineto{\pgfqpoint{0.000000in}{0.020833in}}%
\pgfusepath{stroke,fill}%
}%
\begin{pgfscope}%
\pgfsys@transformshift{1.316411in}{2.222124in}%
\pgfsys@useobject{currentmarker}{}%
\end{pgfscope}%
\end{pgfscope}%
\begin{pgfscope}%
\pgfsetbuttcap%
\pgfsetroundjoin%
\definecolor{currentfill}{rgb}{0.000000,0.000000,0.000000}%
\pgfsetfillcolor{currentfill}%
\pgfsetlinewidth{0.501875pt}%
\definecolor{currentstroke}{rgb}{0.000000,0.000000,0.000000}%
\pgfsetstrokecolor{currentstroke}%
\pgfsetdash{}{0pt}%
\pgfsys@defobject{currentmarker}{\pgfqpoint{0.000000in}{-0.020833in}}{\pgfqpoint{0.000000in}{0.000000in}}{%
\pgfpathmoveto{\pgfqpoint{0.000000in}{0.000000in}}%
\pgfpathlineto{\pgfqpoint{0.000000in}{-0.020833in}}%
\pgfusepath{stroke,fill}%
}%
\begin{pgfscope}%
\pgfsys@transformshift{1.316411in}{2.689374in}%
\pgfsys@useobject{currentmarker}{}%
\end{pgfscope}%
\end{pgfscope}%
\begin{pgfscope}%
\pgfsetbuttcap%
\pgfsetroundjoin%
\definecolor{currentfill}{rgb}{0.000000,0.000000,0.000000}%
\pgfsetfillcolor{currentfill}%
\pgfsetlinewidth{0.501875pt}%
\definecolor{currentstroke}{rgb}{0.000000,0.000000,0.000000}%
\pgfsetstrokecolor{currentstroke}%
\pgfsetdash{}{0pt}%
\pgfsys@defobject{currentmarker}{\pgfqpoint{0.000000in}{0.000000in}}{\pgfqpoint{0.000000in}{0.020833in}}{%
\pgfpathmoveto{\pgfqpoint{0.000000in}{0.000000in}}%
\pgfpathlineto{\pgfqpoint{0.000000in}{0.020833in}}%
\pgfusepath{stroke,fill}%
}%
\begin{pgfscope}%
\pgfsys@transformshift{1.412487in}{2.222124in}%
\pgfsys@useobject{currentmarker}{}%
\end{pgfscope}%
\end{pgfscope}%
\begin{pgfscope}%
\pgfsetbuttcap%
\pgfsetroundjoin%
\definecolor{currentfill}{rgb}{0.000000,0.000000,0.000000}%
\pgfsetfillcolor{currentfill}%
\pgfsetlinewidth{0.501875pt}%
\definecolor{currentstroke}{rgb}{0.000000,0.000000,0.000000}%
\pgfsetstrokecolor{currentstroke}%
\pgfsetdash{}{0pt}%
\pgfsys@defobject{currentmarker}{\pgfqpoint{0.000000in}{-0.020833in}}{\pgfqpoint{0.000000in}{0.000000in}}{%
\pgfpathmoveto{\pgfqpoint{0.000000in}{0.000000in}}%
\pgfpathlineto{\pgfqpoint{0.000000in}{-0.020833in}}%
\pgfusepath{stroke,fill}%
}%
\begin{pgfscope}%
\pgfsys@transformshift{1.412487in}{2.689374in}%
\pgfsys@useobject{currentmarker}{}%
\end{pgfscope}%
\end{pgfscope}%
\begin{pgfscope}%
\pgfsetbuttcap%
\pgfsetroundjoin%
\definecolor{currentfill}{rgb}{0.000000,0.000000,0.000000}%
\pgfsetfillcolor{currentfill}%
\pgfsetlinewidth{0.501875pt}%
\definecolor{currentstroke}{rgb}{0.000000,0.000000,0.000000}%
\pgfsetstrokecolor{currentstroke}%
\pgfsetdash{}{0pt}%
\pgfsys@defobject{currentmarker}{\pgfqpoint{0.000000in}{0.000000in}}{\pgfqpoint{0.000000in}{0.020833in}}{%
\pgfpathmoveto{\pgfqpoint{0.000000in}{0.000000in}}%
\pgfpathlineto{\pgfqpoint{0.000000in}{0.020833in}}%
\pgfusepath{stroke,fill}%
}%
\begin{pgfscope}%
\pgfsys@transformshift{1.508562in}{2.222124in}%
\pgfsys@useobject{currentmarker}{}%
\end{pgfscope}%
\end{pgfscope}%
\begin{pgfscope}%
\pgfsetbuttcap%
\pgfsetroundjoin%
\definecolor{currentfill}{rgb}{0.000000,0.000000,0.000000}%
\pgfsetfillcolor{currentfill}%
\pgfsetlinewidth{0.501875pt}%
\definecolor{currentstroke}{rgb}{0.000000,0.000000,0.000000}%
\pgfsetstrokecolor{currentstroke}%
\pgfsetdash{}{0pt}%
\pgfsys@defobject{currentmarker}{\pgfqpoint{0.000000in}{-0.020833in}}{\pgfqpoint{0.000000in}{0.000000in}}{%
\pgfpathmoveto{\pgfqpoint{0.000000in}{0.000000in}}%
\pgfpathlineto{\pgfqpoint{0.000000in}{-0.020833in}}%
\pgfusepath{stroke,fill}%
}%
\begin{pgfscope}%
\pgfsys@transformshift{1.508562in}{2.689374in}%
\pgfsys@useobject{currentmarker}{}%
\end{pgfscope}%
\end{pgfscope}%
\begin{pgfscope}%
\pgfsetbuttcap%
\pgfsetroundjoin%
\definecolor{currentfill}{rgb}{0.000000,0.000000,0.000000}%
\pgfsetfillcolor{currentfill}%
\pgfsetlinewidth{0.501875pt}%
\definecolor{currentstroke}{rgb}{0.000000,0.000000,0.000000}%
\pgfsetstrokecolor{currentstroke}%
\pgfsetdash{}{0pt}%
\pgfsys@defobject{currentmarker}{\pgfqpoint{0.000000in}{0.000000in}}{\pgfqpoint{0.000000in}{0.020833in}}{%
\pgfpathmoveto{\pgfqpoint{0.000000in}{0.000000in}}%
\pgfpathlineto{\pgfqpoint{0.000000in}{0.020833in}}%
\pgfusepath{stroke,fill}%
}%
\begin{pgfscope}%
\pgfsys@transformshift{1.700712in}{2.222124in}%
\pgfsys@useobject{currentmarker}{}%
\end{pgfscope}%
\end{pgfscope}%
\begin{pgfscope}%
\pgfsetbuttcap%
\pgfsetroundjoin%
\definecolor{currentfill}{rgb}{0.000000,0.000000,0.000000}%
\pgfsetfillcolor{currentfill}%
\pgfsetlinewidth{0.501875pt}%
\definecolor{currentstroke}{rgb}{0.000000,0.000000,0.000000}%
\pgfsetstrokecolor{currentstroke}%
\pgfsetdash{}{0pt}%
\pgfsys@defobject{currentmarker}{\pgfqpoint{0.000000in}{-0.020833in}}{\pgfqpoint{0.000000in}{0.000000in}}{%
\pgfpathmoveto{\pgfqpoint{0.000000in}{0.000000in}}%
\pgfpathlineto{\pgfqpoint{0.000000in}{-0.020833in}}%
\pgfusepath{stroke,fill}%
}%
\begin{pgfscope}%
\pgfsys@transformshift{1.700712in}{2.689374in}%
\pgfsys@useobject{currentmarker}{}%
\end{pgfscope}%
\end{pgfscope}%
\begin{pgfscope}%
\pgfsetbuttcap%
\pgfsetroundjoin%
\definecolor{currentfill}{rgb}{0.000000,0.000000,0.000000}%
\pgfsetfillcolor{currentfill}%
\pgfsetlinewidth{0.501875pt}%
\definecolor{currentstroke}{rgb}{0.000000,0.000000,0.000000}%
\pgfsetstrokecolor{currentstroke}%
\pgfsetdash{}{0pt}%
\pgfsys@defobject{currentmarker}{\pgfqpoint{0.000000in}{0.000000in}}{\pgfqpoint{0.000000in}{0.020833in}}{%
\pgfpathmoveto{\pgfqpoint{0.000000in}{0.000000in}}%
\pgfpathlineto{\pgfqpoint{0.000000in}{0.020833in}}%
\pgfusepath{stroke,fill}%
}%
\begin{pgfscope}%
\pgfsys@transformshift{1.796787in}{2.222124in}%
\pgfsys@useobject{currentmarker}{}%
\end{pgfscope}%
\end{pgfscope}%
\begin{pgfscope}%
\pgfsetbuttcap%
\pgfsetroundjoin%
\definecolor{currentfill}{rgb}{0.000000,0.000000,0.000000}%
\pgfsetfillcolor{currentfill}%
\pgfsetlinewidth{0.501875pt}%
\definecolor{currentstroke}{rgb}{0.000000,0.000000,0.000000}%
\pgfsetstrokecolor{currentstroke}%
\pgfsetdash{}{0pt}%
\pgfsys@defobject{currentmarker}{\pgfqpoint{0.000000in}{-0.020833in}}{\pgfqpoint{0.000000in}{0.000000in}}{%
\pgfpathmoveto{\pgfqpoint{0.000000in}{0.000000in}}%
\pgfpathlineto{\pgfqpoint{0.000000in}{-0.020833in}}%
\pgfusepath{stroke,fill}%
}%
\begin{pgfscope}%
\pgfsys@transformshift{1.796787in}{2.689374in}%
\pgfsys@useobject{currentmarker}{}%
\end{pgfscope}%
\end{pgfscope}%
\begin{pgfscope}%
\pgfsetbuttcap%
\pgfsetroundjoin%
\definecolor{currentfill}{rgb}{0.000000,0.000000,0.000000}%
\pgfsetfillcolor{currentfill}%
\pgfsetlinewidth{0.501875pt}%
\definecolor{currentstroke}{rgb}{0.000000,0.000000,0.000000}%
\pgfsetstrokecolor{currentstroke}%
\pgfsetdash{}{0pt}%
\pgfsys@defobject{currentmarker}{\pgfqpoint{0.000000in}{0.000000in}}{\pgfqpoint{0.000000in}{0.020833in}}{%
\pgfpathmoveto{\pgfqpoint{0.000000in}{0.000000in}}%
\pgfpathlineto{\pgfqpoint{0.000000in}{0.020833in}}%
\pgfusepath{stroke,fill}%
}%
\begin{pgfscope}%
\pgfsys@transformshift{1.892862in}{2.222124in}%
\pgfsys@useobject{currentmarker}{}%
\end{pgfscope}%
\end{pgfscope}%
\begin{pgfscope}%
\pgfsetbuttcap%
\pgfsetroundjoin%
\definecolor{currentfill}{rgb}{0.000000,0.000000,0.000000}%
\pgfsetfillcolor{currentfill}%
\pgfsetlinewidth{0.501875pt}%
\definecolor{currentstroke}{rgb}{0.000000,0.000000,0.000000}%
\pgfsetstrokecolor{currentstroke}%
\pgfsetdash{}{0pt}%
\pgfsys@defobject{currentmarker}{\pgfqpoint{0.000000in}{-0.020833in}}{\pgfqpoint{0.000000in}{0.000000in}}{%
\pgfpathmoveto{\pgfqpoint{0.000000in}{0.000000in}}%
\pgfpathlineto{\pgfqpoint{0.000000in}{-0.020833in}}%
\pgfusepath{stroke,fill}%
}%
\begin{pgfscope}%
\pgfsys@transformshift{1.892862in}{2.689374in}%
\pgfsys@useobject{currentmarker}{}%
\end{pgfscope}%
\end{pgfscope}%
\begin{pgfscope}%
\pgfsetbuttcap%
\pgfsetroundjoin%
\definecolor{currentfill}{rgb}{0.000000,0.000000,0.000000}%
\pgfsetfillcolor{currentfill}%
\pgfsetlinewidth{0.501875pt}%
\definecolor{currentstroke}{rgb}{0.000000,0.000000,0.000000}%
\pgfsetstrokecolor{currentstroke}%
\pgfsetdash{}{0pt}%
\pgfsys@defobject{currentmarker}{\pgfqpoint{0.000000in}{0.000000in}}{\pgfqpoint{0.000000in}{0.020833in}}{%
\pgfpathmoveto{\pgfqpoint{0.000000in}{0.000000in}}%
\pgfpathlineto{\pgfqpoint{0.000000in}{0.020833in}}%
\pgfusepath{stroke,fill}%
}%
\begin{pgfscope}%
\pgfsys@transformshift{1.988937in}{2.222124in}%
\pgfsys@useobject{currentmarker}{}%
\end{pgfscope}%
\end{pgfscope}%
\begin{pgfscope}%
\pgfsetbuttcap%
\pgfsetroundjoin%
\definecolor{currentfill}{rgb}{0.000000,0.000000,0.000000}%
\pgfsetfillcolor{currentfill}%
\pgfsetlinewidth{0.501875pt}%
\definecolor{currentstroke}{rgb}{0.000000,0.000000,0.000000}%
\pgfsetstrokecolor{currentstroke}%
\pgfsetdash{}{0pt}%
\pgfsys@defobject{currentmarker}{\pgfqpoint{0.000000in}{-0.020833in}}{\pgfqpoint{0.000000in}{0.000000in}}{%
\pgfpathmoveto{\pgfqpoint{0.000000in}{0.000000in}}%
\pgfpathlineto{\pgfqpoint{0.000000in}{-0.020833in}}%
\pgfusepath{stroke,fill}%
}%
\begin{pgfscope}%
\pgfsys@transformshift{1.988937in}{2.689374in}%
\pgfsys@useobject{currentmarker}{}%
\end{pgfscope}%
\end{pgfscope}%
\begin{pgfscope}%
\pgfsetbuttcap%
\pgfsetroundjoin%
\definecolor{currentfill}{rgb}{0.000000,0.000000,0.000000}%
\pgfsetfillcolor{currentfill}%
\pgfsetlinewidth{0.501875pt}%
\definecolor{currentstroke}{rgb}{0.000000,0.000000,0.000000}%
\pgfsetstrokecolor{currentstroke}%
\pgfsetdash{}{0pt}%
\pgfsys@defobject{currentmarker}{\pgfqpoint{0.000000in}{0.000000in}}{\pgfqpoint{0.000000in}{0.020833in}}{%
\pgfpathmoveto{\pgfqpoint{0.000000in}{0.000000in}}%
\pgfpathlineto{\pgfqpoint{0.000000in}{0.020833in}}%
\pgfusepath{stroke,fill}%
}%
\begin{pgfscope}%
\pgfsys@transformshift{2.181087in}{2.222124in}%
\pgfsys@useobject{currentmarker}{}%
\end{pgfscope}%
\end{pgfscope}%
\begin{pgfscope}%
\pgfsetbuttcap%
\pgfsetroundjoin%
\definecolor{currentfill}{rgb}{0.000000,0.000000,0.000000}%
\pgfsetfillcolor{currentfill}%
\pgfsetlinewidth{0.501875pt}%
\definecolor{currentstroke}{rgb}{0.000000,0.000000,0.000000}%
\pgfsetstrokecolor{currentstroke}%
\pgfsetdash{}{0pt}%
\pgfsys@defobject{currentmarker}{\pgfqpoint{0.000000in}{-0.020833in}}{\pgfqpoint{0.000000in}{0.000000in}}{%
\pgfpathmoveto{\pgfqpoint{0.000000in}{0.000000in}}%
\pgfpathlineto{\pgfqpoint{0.000000in}{-0.020833in}}%
\pgfusepath{stroke,fill}%
}%
\begin{pgfscope}%
\pgfsys@transformshift{2.181087in}{2.689374in}%
\pgfsys@useobject{currentmarker}{}%
\end{pgfscope}%
\end{pgfscope}%
\begin{pgfscope}%
\pgfsetbuttcap%
\pgfsetroundjoin%
\definecolor{currentfill}{rgb}{0.000000,0.000000,0.000000}%
\pgfsetfillcolor{currentfill}%
\pgfsetlinewidth{0.501875pt}%
\definecolor{currentstroke}{rgb}{0.000000,0.000000,0.000000}%
\pgfsetstrokecolor{currentstroke}%
\pgfsetdash{}{0pt}%
\pgfsys@defobject{currentmarker}{\pgfqpoint{0.000000in}{0.000000in}}{\pgfqpoint{0.000000in}{0.020833in}}{%
\pgfpathmoveto{\pgfqpoint{0.000000in}{0.000000in}}%
\pgfpathlineto{\pgfqpoint{0.000000in}{0.020833in}}%
\pgfusepath{stroke,fill}%
}%
\begin{pgfscope}%
\pgfsys@transformshift{2.277162in}{2.222124in}%
\pgfsys@useobject{currentmarker}{}%
\end{pgfscope}%
\end{pgfscope}%
\begin{pgfscope}%
\pgfsetbuttcap%
\pgfsetroundjoin%
\definecolor{currentfill}{rgb}{0.000000,0.000000,0.000000}%
\pgfsetfillcolor{currentfill}%
\pgfsetlinewidth{0.501875pt}%
\definecolor{currentstroke}{rgb}{0.000000,0.000000,0.000000}%
\pgfsetstrokecolor{currentstroke}%
\pgfsetdash{}{0pt}%
\pgfsys@defobject{currentmarker}{\pgfqpoint{0.000000in}{-0.020833in}}{\pgfqpoint{0.000000in}{0.000000in}}{%
\pgfpathmoveto{\pgfqpoint{0.000000in}{0.000000in}}%
\pgfpathlineto{\pgfqpoint{0.000000in}{-0.020833in}}%
\pgfusepath{stroke,fill}%
}%
\begin{pgfscope}%
\pgfsys@transformshift{2.277162in}{2.689374in}%
\pgfsys@useobject{currentmarker}{}%
\end{pgfscope}%
\end{pgfscope}%
\begin{pgfscope}%
\pgfsetbuttcap%
\pgfsetroundjoin%
\definecolor{currentfill}{rgb}{0.000000,0.000000,0.000000}%
\pgfsetfillcolor{currentfill}%
\pgfsetlinewidth{0.501875pt}%
\definecolor{currentstroke}{rgb}{0.000000,0.000000,0.000000}%
\pgfsetstrokecolor{currentstroke}%
\pgfsetdash{}{0pt}%
\pgfsys@defobject{currentmarker}{\pgfqpoint{0.000000in}{0.000000in}}{\pgfqpoint{0.000000in}{0.020833in}}{%
\pgfpathmoveto{\pgfqpoint{0.000000in}{0.000000in}}%
\pgfpathlineto{\pgfqpoint{0.000000in}{0.020833in}}%
\pgfusepath{stroke,fill}%
}%
\begin{pgfscope}%
\pgfsys@transformshift{2.373237in}{2.222124in}%
\pgfsys@useobject{currentmarker}{}%
\end{pgfscope}%
\end{pgfscope}%
\begin{pgfscope}%
\pgfsetbuttcap%
\pgfsetroundjoin%
\definecolor{currentfill}{rgb}{0.000000,0.000000,0.000000}%
\pgfsetfillcolor{currentfill}%
\pgfsetlinewidth{0.501875pt}%
\definecolor{currentstroke}{rgb}{0.000000,0.000000,0.000000}%
\pgfsetstrokecolor{currentstroke}%
\pgfsetdash{}{0pt}%
\pgfsys@defobject{currentmarker}{\pgfqpoint{0.000000in}{-0.020833in}}{\pgfqpoint{0.000000in}{0.000000in}}{%
\pgfpathmoveto{\pgfqpoint{0.000000in}{0.000000in}}%
\pgfpathlineto{\pgfqpoint{0.000000in}{-0.020833in}}%
\pgfusepath{stroke,fill}%
}%
\begin{pgfscope}%
\pgfsys@transformshift{2.373237in}{2.689374in}%
\pgfsys@useobject{currentmarker}{}%
\end{pgfscope}%
\end{pgfscope}%
\begin{pgfscope}%
\pgfsetbuttcap%
\pgfsetroundjoin%
\definecolor{currentfill}{rgb}{0.000000,0.000000,0.000000}%
\pgfsetfillcolor{currentfill}%
\pgfsetlinewidth{0.501875pt}%
\definecolor{currentstroke}{rgb}{0.000000,0.000000,0.000000}%
\pgfsetstrokecolor{currentstroke}%
\pgfsetdash{}{0pt}%
\pgfsys@defobject{currentmarker}{\pgfqpoint{0.000000in}{0.000000in}}{\pgfqpoint{0.000000in}{0.020833in}}{%
\pgfpathmoveto{\pgfqpoint{0.000000in}{0.000000in}}%
\pgfpathlineto{\pgfqpoint{0.000000in}{0.020833in}}%
\pgfusepath{stroke,fill}%
}%
\begin{pgfscope}%
\pgfsys@transformshift{2.469312in}{2.222124in}%
\pgfsys@useobject{currentmarker}{}%
\end{pgfscope}%
\end{pgfscope}%
\begin{pgfscope}%
\pgfsetbuttcap%
\pgfsetroundjoin%
\definecolor{currentfill}{rgb}{0.000000,0.000000,0.000000}%
\pgfsetfillcolor{currentfill}%
\pgfsetlinewidth{0.501875pt}%
\definecolor{currentstroke}{rgb}{0.000000,0.000000,0.000000}%
\pgfsetstrokecolor{currentstroke}%
\pgfsetdash{}{0pt}%
\pgfsys@defobject{currentmarker}{\pgfqpoint{0.000000in}{-0.020833in}}{\pgfqpoint{0.000000in}{0.000000in}}{%
\pgfpathmoveto{\pgfqpoint{0.000000in}{0.000000in}}%
\pgfpathlineto{\pgfqpoint{0.000000in}{-0.020833in}}%
\pgfusepath{stroke,fill}%
}%
\begin{pgfscope}%
\pgfsys@transformshift{2.469312in}{2.689374in}%
\pgfsys@useobject{currentmarker}{}%
\end{pgfscope}%
\end{pgfscope}%
\begin{pgfscope}%
\pgfsetbuttcap%
\pgfsetroundjoin%
\definecolor{currentfill}{rgb}{0.000000,0.000000,0.000000}%
\pgfsetfillcolor{currentfill}%
\pgfsetlinewidth{0.501875pt}%
\definecolor{currentstroke}{rgb}{0.000000,0.000000,0.000000}%
\pgfsetstrokecolor{currentstroke}%
\pgfsetdash{}{0pt}%
\pgfsys@defobject{currentmarker}{\pgfqpoint{0.000000in}{0.000000in}}{\pgfqpoint{0.000000in}{0.020833in}}{%
\pgfpathmoveto{\pgfqpoint{0.000000in}{0.000000in}}%
\pgfpathlineto{\pgfqpoint{0.000000in}{0.020833in}}%
\pgfusepath{stroke,fill}%
}%
\begin{pgfscope}%
\pgfsys@transformshift{2.661463in}{2.222124in}%
\pgfsys@useobject{currentmarker}{}%
\end{pgfscope}%
\end{pgfscope}%
\begin{pgfscope}%
\pgfsetbuttcap%
\pgfsetroundjoin%
\definecolor{currentfill}{rgb}{0.000000,0.000000,0.000000}%
\pgfsetfillcolor{currentfill}%
\pgfsetlinewidth{0.501875pt}%
\definecolor{currentstroke}{rgb}{0.000000,0.000000,0.000000}%
\pgfsetstrokecolor{currentstroke}%
\pgfsetdash{}{0pt}%
\pgfsys@defobject{currentmarker}{\pgfqpoint{0.000000in}{-0.020833in}}{\pgfqpoint{0.000000in}{0.000000in}}{%
\pgfpathmoveto{\pgfqpoint{0.000000in}{0.000000in}}%
\pgfpathlineto{\pgfqpoint{0.000000in}{-0.020833in}}%
\pgfusepath{stroke,fill}%
}%
\begin{pgfscope}%
\pgfsys@transformshift{2.661463in}{2.689374in}%
\pgfsys@useobject{currentmarker}{}%
\end{pgfscope}%
\end{pgfscope}%
\begin{pgfscope}%
\pgfsetbuttcap%
\pgfsetroundjoin%
\definecolor{currentfill}{rgb}{0.000000,0.000000,0.000000}%
\pgfsetfillcolor{currentfill}%
\pgfsetlinewidth{0.501875pt}%
\definecolor{currentstroke}{rgb}{0.000000,0.000000,0.000000}%
\pgfsetstrokecolor{currentstroke}%
\pgfsetdash{}{0pt}%
\pgfsys@defobject{currentmarker}{\pgfqpoint{0.000000in}{0.000000in}}{\pgfqpoint{0.000000in}{0.020833in}}{%
\pgfpathmoveto{\pgfqpoint{0.000000in}{0.000000in}}%
\pgfpathlineto{\pgfqpoint{0.000000in}{0.020833in}}%
\pgfusepath{stroke,fill}%
}%
\begin{pgfscope}%
\pgfsys@transformshift{2.757538in}{2.222124in}%
\pgfsys@useobject{currentmarker}{}%
\end{pgfscope}%
\end{pgfscope}%
\begin{pgfscope}%
\pgfsetbuttcap%
\pgfsetroundjoin%
\definecolor{currentfill}{rgb}{0.000000,0.000000,0.000000}%
\pgfsetfillcolor{currentfill}%
\pgfsetlinewidth{0.501875pt}%
\definecolor{currentstroke}{rgb}{0.000000,0.000000,0.000000}%
\pgfsetstrokecolor{currentstroke}%
\pgfsetdash{}{0pt}%
\pgfsys@defobject{currentmarker}{\pgfqpoint{0.000000in}{-0.020833in}}{\pgfqpoint{0.000000in}{0.000000in}}{%
\pgfpathmoveto{\pgfqpoint{0.000000in}{0.000000in}}%
\pgfpathlineto{\pgfqpoint{0.000000in}{-0.020833in}}%
\pgfusepath{stroke,fill}%
}%
\begin{pgfscope}%
\pgfsys@transformshift{2.757538in}{2.689374in}%
\pgfsys@useobject{currentmarker}{}%
\end{pgfscope}%
\end{pgfscope}%
\begin{pgfscope}%
\pgfsetbuttcap%
\pgfsetroundjoin%
\definecolor{currentfill}{rgb}{0.000000,0.000000,0.000000}%
\pgfsetfillcolor{currentfill}%
\pgfsetlinewidth{0.501875pt}%
\definecolor{currentstroke}{rgb}{0.000000,0.000000,0.000000}%
\pgfsetstrokecolor{currentstroke}%
\pgfsetdash{}{0pt}%
\pgfsys@defobject{currentmarker}{\pgfqpoint{0.000000in}{0.000000in}}{\pgfqpoint{0.000000in}{0.020833in}}{%
\pgfpathmoveto{\pgfqpoint{0.000000in}{0.000000in}}%
\pgfpathlineto{\pgfqpoint{0.000000in}{0.020833in}}%
\pgfusepath{stroke,fill}%
}%
\begin{pgfscope}%
\pgfsys@transformshift{2.853613in}{2.222124in}%
\pgfsys@useobject{currentmarker}{}%
\end{pgfscope}%
\end{pgfscope}%
\begin{pgfscope}%
\pgfsetbuttcap%
\pgfsetroundjoin%
\definecolor{currentfill}{rgb}{0.000000,0.000000,0.000000}%
\pgfsetfillcolor{currentfill}%
\pgfsetlinewidth{0.501875pt}%
\definecolor{currentstroke}{rgb}{0.000000,0.000000,0.000000}%
\pgfsetstrokecolor{currentstroke}%
\pgfsetdash{}{0pt}%
\pgfsys@defobject{currentmarker}{\pgfqpoint{0.000000in}{-0.020833in}}{\pgfqpoint{0.000000in}{0.000000in}}{%
\pgfpathmoveto{\pgfqpoint{0.000000in}{0.000000in}}%
\pgfpathlineto{\pgfqpoint{0.000000in}{-0.020833in}}%
\pgfusepath{stroke,fill}%
}%
\begin{pgfscope}%
\pgfsys@transformshift{2.853613in}{2.689374in}%
\pgfsys@useobject{currentmarker}{}%
\end{pgfscope}%
\end{pgfscope}%
\begin{pgfscope}%
\pgfsetbuttcap%
\pgfsetroundjoin%
\definecolor{currentfill}{rgb}{0.000000,0.000000,0.000000}%
\pgfsetfillcolor{currentfill}%
\pgfsetlinewidth{0.501875pt}%
\definecolor{currentstroke}{rgb}{0.000000,0.000000,0.000000}%
\pgfsetstrokecolor{currentstroke}%
\pgfsetdash{}{0pt}%
\pgfsys@defobject{currentmarker}{\pgfqpoint{0.000000in}{0.000000in}}{\pgfqpoint{0.000000in}{0.020833in}}{%
\pgfpathmoveto{\pgfqpoint{0.000000in}{0.000000in}}%
\pgfpathlineto{\pgfqpoint{0.000000in}{0.020833in}}%
\pgfusepath{stroke,fill}%
}%
\begin{pgfscope}%
\pgfsys@transformshift{2.949688in}{2.222124in}%
\pgfsys@useobject{currentmarker}{}%
\end{pgfscope}%
\end{pgfscope}%
\begin{pgfscope}%
\pgfsetbuttcap%
\pgfsetroundjoin%
\definecolor{currentfill}{rgb}{0.000000,0.000000,0.000000}%
\pgfsetfillcolor{currentfill}%
\pgfsetlinewidth{0.501875pt}%
\definecolor{currentstroke}{rgb}{0.000000,0.000000,0.000000}%
\pgfsetstrokecolor{currentstroke}%
\pgfsetdash{}{0pt}%
\pgfsys@defobject{currentmarker}{\pgfqpoint{0.000000in}{-0.020833in}}{\pgfqpoint{0.000000in}{0.000000in}}{%
\pgfpathmoveto{\pgfqpoint{0.000000in}{0.000000in}}%
\pgfpathlineto{\pgfqpoint{0.000000in}{-0.020833in}}%
\pgfusepath{stroke,fill}%
}%
\begin{pgfscope}%
\pgfsys@transformshift{2.949688in}{2.689374in}%
\pgfsys@useobject{currentmarker}{}%
\end{pgfscope}%
\end{pgfscope}%
\begin{pgfscope}%
\pgfsetbuttcap%
\pgfsetroundjoin%
\definecolor{currentfill}{rgb}{0.000000,0.000000,0.000000}%
\pgfsetfillcolor{currentfill}%
\pgfsetlinewidth{0.501875pt}%
\definecolor{currentstroke}{rgb}{0.000000,0.000000,0.000000}%
\pgfsetstrokecolor{currentstroke}%
\pgfsetdash{}{0pt}%
\pgfsys@defobject{currentmarker}{\pgfqpoint{0.000000in}{0.000000in}}{\pgfqpoint{0.000000in}{0.020833in}}{%
\pgfpathmoveto{\pgfqpoint{0.000000in}{0.000000in}}%
\pgfpathlineto{\pgfqpoint{0.000000in}{0.020833in}}%
\pgfusepath{stroke,fill}%
}%
\begin{pgfscope}%
\pgfsys@transformshift{3.141838in}{2.222124in}%
\pgfsys@useobject{currentmarker}{}%
\end{pgfscope}%
\end{pgfscope}%
\begin{pgfscope}%
\pgfsetbuttcap%
\pgfsetroundjoin%
\definecolor{currentfill}{rgb}{0.000000,0.000000,0.000000}%
\pgfsetfillcolor{currentfill}%
\pgfsetlinewidth{0.501875pt}%
\definecolor{currentstroke}{rgb}{0.000000,0.000000,0.000000}%
\pgfsetstrokecolor{currentstroke}%
\pgfsetdash{}{0pt}%
\pgfsys@defobject{currentmarker}{\pgfqpoint{0.000000in}{-0.020833in}}{\pgfqpoint{0.000000in}{0.000000in}}{%
\pgfpathmoveto{\pgfqpoint{0.000000in}{0.000000in}}%
\pgfpathlineto{\pgfqpoint{0.000000in}{-0.020833in}}%
\pgfusepath{stroke,fill}%
}%
\begin{pgfscope}%
\pgfsys@transformshift{3.141838in}{2.689374in}%
\pgfsys@useobject{currentmarker}{}%
\end{pgfscope}%
\end{pgfscope}%
\begin{pgfscope}%
\pgfsetbuttcap%
\pgfsetroundjoin%
\definecolor{currentfill}{rgb}{0.000000,0.000000,0.000000}%
\pgfsetfillcolor{currentfill}%
\pgfsetlinewidth{0.501875pt}%
\definecolor{currentstroke}{rgb}{0.000000,0.000000,0.000000}%
\pgfsetstrokecolor{currentstroke}%
\pgfsetdash{}{0pt}%
\pgfsys@defobject{currentmarker}{\pgfqpoint{0.000000in}{0.000000in}}{\pgfqpoint{0.000000in}{0.020833in}}{%
\pgfpathmoveto{\pgfqpoint{0.000000in}{0.000000in}}%
\pgfpathlineto{\pgfqpoint{0.000000in}{0.020833in}}%
\pgfusepath{stroke,fill}%
}%
\begin{pgfscope}%
\pgfsys@transformshift{3.237913in}{2.222124in}%
\pgfsys@useobject{currentmarker}{}%
\end{pgfscope}%
\end{pgfscope}%
\begin{pgfscope}%
\pgfsetbuttcap%
\pgfsetroundjoin%
\definecolor{currentfill}{rgb}{0.000000,0.000000,0.000000}%
\pgfsetfillcolor{currentfill}%
\pgfsetlinewidth{0.501875pt}%
\definecolor{currentstroke}{rgb}{0.000000,0.000000,0.000000}%
\pgfsetstrokecolor{currentstroke}%
\pgfsetdash{}{0pt}%
\pgfsys@defobject{currentmarker}{\pgfqpoint{0.000000in}{-0.020833in}}{\pgfqpoint{0.000000in}{0.000000in}}{%
\pgfpathmoveto{\pgfqpoint{0.000000in}{0.000000in}}%
\pgfpathlineto{\pgfqpoint{0.000000in}{-0.020833in}}%
\pgfusepath{stroke,fill}%
}%
\begin{pgfscope}%
\pgfsys@transformshift{3.237913in}{2.689374in}%
\pgfsys@useobject{currentmarker}{}%
\end{pgfscope}%
\end{pgfscope}%
\begin{pgfscope}%
\pgfsetbuttcap%
\pgfsetroundjoin%
\definecolor{currentfill}{rgb}{0.000000,0.000000,0.000000}%
\pgfsetfillcolor{currentfill}%
\pgfsetlinewidth{0.501875pt}%
\definecolor{currentstroke}{rgb}{0.000000,0.000000,0.000000}%
\pgfsetstrokecolor{currentstroke}%
\pgfsetdash{}{0pt}%
\pgfsys@defobject{currentmarker}{\pgfqpoint{0.000000in}{0.000000in}}{\pgfqpoint{0.000000in}{0.020833in}}{%
\pgfpathmoveto{\pgfqpoint{0.000000in}{0.000000in}}%
\pgfpathlineto{\pgfqpoint{0.000000in}{0.020833in}}%
\pgfusepath{stroke,fill}%
}%
\begin{pgfscope}%
\pgfsys@transformshift{3.333988in}{2.222124in}%
\pgfsys@useobject{currentmarker}{}%
\end{pgfscope}%
\end{pgfscope}%
\begin{pgfscope}%
\pgfsetbuttcap%
\pgfsetroundjoin%
\definecolor{currentfill}{rgb}{0.000000,0.000000,0.000000}%
\pgfsetfillcolor{currentfill}%
\pgfsetlinewidth{0.501875pt}%
\definecolor{currentstroke}{rgb}{0.000000,0.000000,0.000000}%
\pgfsetstrokecolor{currentstroke}%
\pgfsetdash{}{0pt}%
\pgfsys@defobject{currentmarker}{\pgfqpoint{0.000000in}{-0.020833in}}{\pgfqpoint{0.000000in}{0.000000in}}{%
\pgfpathmoveto{\pgfqpoint{0.000000in}{0.000000in}}%
\pgfpathlineto{\pgfqpoint{0.000000in}{-0.020833in}}%
\pgfusepath{stroke,fill}%
}%
\begin{pgfscope}%
\pgfsys@transformshift{3.333988in}{2.689374in}%
\pgfsys@useobject{currentmarker}{}%
\end{pgfscope}%
\end{pgfscope}%
\begin{pgfscope}%
\pgfsetbuttcap%
\pgfsetroundjoin%
\definecolor{currentfill}{rgb}{0.000000,0.000000,0.000000}%
\pgfsetfillcolor{currentfill}%
\pgfsetlinewidth{0.501875pt}%
\definecolor{currentstroke}{rgb}{0.000000,0.000000,0.000000}%
\pgfsetstrokecolor{currentstroke}%
\pgfsetdash{}{0pt}%
\pgfsys@defobject{currentmarker}{\pgfqpoint{0.000000in}{0.000000in}}{\pgfqpoint{0.000000in}{0.020833in}}{%
\pgfpathmoveto{\pgfqpoint{0.000000in}{0.000000in}}%
\pgfpathlineto{\pgfqpoint{0.000000in}{0.020833in}}%
\pgfusepath{stroke,fill}%
}%
\begin{pgfscope}%
\pgfsys@transformshift{3.430063in}{2.222124in}%
\pgfsys@useobject{currentmarker}{}%
\end{pgfscope}%
\end{pgfscope}%
\begin{pgfscope}%
\pgfsetbuttcap%
\pgfsetroundjoin%
\definecolor{currentfill}{rgb}{0.000000,0.000000,0.000000}%
\pgfsetfillcolor{currentfill}%
\pgfsetlinewidth{0.501875pt}%
\definecolor{currentstroke}{rgb}{0.000000,0.000000,0.000000}%
\pgfsetstrokecolor{currentstroke}%
\pgfsetdash{}{0pt}%
\pgfsys@defobject{currentmarker}{\pgfqpoint{0.000000in}{-0.020833in}}{\pgfqpoint{0.000000in}{0.000000in}}{%
\pgfpathmoveto{\pgfqpoint{0.000000in}{0.000000in}}%
\pgfpathlineto{\pgfqpoint{0.000000in}{-0.020833in}}%
\pgfusepath{stroke,fill}%
}%
\begin{pgfscope}%
\pgfsys@transformshift{3.430063in}{2.689374in}%
\pgfsys@useobject{currentmarker}{}%
\end{pgfscope}%
\end{pgfscope}%
\begin{pgfscope}%
\pgfsetbuttcap%
\pgfsetroundjoin%
\definecolor{currentfill}{rgb}{0.000000,0.000000,0.000000}%
\pgfsetfillcolor{currentfill}%
\pgfsetlinewidth{0.501875pt}%
\definecolor{currentstroke}{rgb}{0.000000,0.000000,0.000000}%
\pgfsetstrokecolor{currentstroke}%
\pgfsetdash{}{0pt}%
\pgfsys@defobject{currentmarker}{\pgfqpoint{0.000000in}{0.000000in}}{\pgfqpoint{0.000000in}{0.020833in}}{%
\pgfpathmoveto{\pgfqpoint{0.000000in}{0.000000in}}%
\pgfpathlineto{\pgfqpoint{0.000000in}{0.020833in}}%
\pgfusepath{stroke,fill}%
}%
\begin{pgfscope}%
\pgfsys@transformshift{3.622213in}{2.222124in}%
\pgfsys@useobject{currentmarker}{}%
\end{pgfscope}%
\end{pgfscope}%
\begin{pgfscope}%
\pgfsetbuttcap%
\pgfsetroundjoin%
\definecolor{currentfill}{rgb}{0.000000,0.000000,0.000000}%
\pgfsetfillcolor{currentfill}%
\pgfsetlinewidth{0.501875pt}%
\definecolor{currentstroke}{rgb}{0.000000,0.000000,0.000000}%
\pgfsetstrokecolor{currentstroke}%
\pgfsetdash{}{0pt}%
\pgfsys@defobject{currentmarker}{\pgfqpoint{0.000000in}{-0.020833in}}{\pgfqpoint{0.000000in}{0.000000in}}{%
\pgfpathmoveto{\pgfqpoint{0.000000in}{0.000000in}}%
\pgfpathlineto{\pgfqpoint{0.000000in}{-0.020833in}}%
\pgfusepath{stroke,fill}%
}%
\begin{pgfscope}%
\pgfsys@transformshift{3.622213in}{2.689374in}%
\pgfsys@useobject{currentmarker}{}%
\end{pgfscope}%
\end{pgfscope}%
\begin{pgfscope}%
\pgfsetbuttcap%
\pgfsetroundjoin%
\definecolor{currentfill}{rgb}{0.000000,0.000000,0.000000}%
\pgfsetfillcolor{currentfill}%
\pgfsetlinewidth{0.501875pt}%
\definecolor{currentstroke}{rgb}{0.000000,0.000000,0.000000}%
\pgfsetstrokecolor{currentstroke}%
\pgfsetdash{}{0pt}%
\pgfsys@defobject{currentmarker}{\pgfqpoint{0.000000in}{0.000000in}}{\pgfqpoint{0.000000in}{0.020833in}}{%
\pgfpathmoveto{\pgfqpoint{0.000000in}{0.000000in}}%
\pgfpathlineto{\pgfqpoint{0.000000in}{0.020833in}}%
\pgfusepath{stroke,fill}%
}%
\begin{pgfscope}%
\pgfsys@transformshift{3.718289in}{2.222124in}%
\pgfsys@useobject{currentmarker}{}%
\end{pgfscope}%
\end{pgfscope}%
\begin{pgfscope}%
\pgfsetbuttcap%
\pgfsetroundjoin%
\definecolor{currentfill}{rgb}{0.000000,0.000000,0.000000}%
\pgfsetfillcolor{currentfill}%
\pgfsetlinewidth{0.501875pt}%
\definecolor{currentstroke}{rgb}{0.000000,0.000000,0.000000}%
\pgfsetstrokecolor{currentstroke}%
\pgfsetdash{}{0pt}%
\pgfsys@defobject{currentmarker}{\pgfqpoint{0.000000in}{-0.020833in}}{\pgfqpoint{0.000000in}{0.000000in}}{%
\pgfpathmoveto{\pgfqpoint{0.000000in}{0.000000in}}%
\pgfpathlineto{\pgfqpoint{0.000000in}{-0.020833in}}%
\pgfusepath{stroke,fill}%
}%
\begin{pgfscope}%
\pgfsys@transformshift{3.718289in}{2.689374in}%
\pgfsys@useobject{currentmarker}{}%
\end{pgfscope}%
\end{pgfscope}%
\begin{pgfscope}%
\pgfsetbuttcap%
\pgfsetroundjoin%
\definecolor{currentfill}{rgb}{0.000000,0.000000,0.000000}%
\pgfsetfillcolor{currentfill}%
\pgfsetlinewidth{0.501875pt}%
\definecolor{currentstroke}{rgb}{0.000000,0.000000,0.000000}%
\pgfsetstrokecolor{currentstroke}%
\pgfsetdash{}{0pt}%
\pgfsys@defobject{currentmarker}{\pgfqpoint{0.000000in}{0.000000in}}{\pgfqpoint{0.000000in}{0.020833in}}{%
\pgfpathmoveto{\pgfqpoint{0.000000in}{0.000000in}}%
\pgfpathlineto{\pgfqpoint{0.000000in}{0.020833in}}%
\pgfusepath{stroke,fill}%
}%
\begin{pgfscope}%
\pgfsys@transformshift{3.814364in}{2.222124in}%
\pgfsys@useobject{currentmarker}{}%
\end{pgfscope}%
\end{pgfscope}%
\begin{pgfscope}%
\pgfsetbuttcap%
\pgfsetroundjoin%
\definecolor{currentfill}{rgb}{0.000000,0.000000,0.000000}%
\pgfsetfillcolor{currentfill}%
\pgfsetlinewidth{0.501875pt}%
\definecolor{currentstroke}{rgb}{0.000000,0.000000,0.000000}%
\pgfsetstrokecolor{currentstroke}%
\pgfsetdash{}{0pt}%
\pgfsys@defobject{currentmarker}{\pgfqpoint{0.000000in}{-0.020833in}}{\pgfqpoint{0.000000in}{0.000000in}}{%
\pgfpathmoveto{\pgfqpoint{0.000000in}{0.000000in}}%
\pgfpathlineto{\pgfqpoint{0.000000in}{-0.020833in}}%
\pgfusepath{stroke,fill}%
}%
\begin{pgfscope}%
\pgfsys@transformshift{3.814364in}{2.689374in}%
\pgfsys@useobject{currentmarker}{}%
\end{pgfscope}%
\end{pgfscope}%
\begin{pgfscope}%
\pgfsetbuttcap%
\pgfsetroundjoin%
\definecolor{currentfill}{rgb}{0.000000,0.000000,0.000000}%
\pgfsetfillcolor{currentfill}%
\pgfsetlinewidth{0.501875pt}%
\definecolor{currentstroke}{rgb}{0.000000,0.000000,0.000000}%
\pgfsetstrokecolor{currentstroke}%
\pgfsetdash{}{0pt}%
\pgfsys@defobject{currentmarker}{\pgfqpoint{0.000000in}{0.000000in}}{\pgfqpoint{0.000000in}{0.020833in}}{%
\pgfpathmoveto{\pgfqpoint{0.000000in}{0.000000in}}%
\pgfpathlineto{\pgfqpoint{0.000000in}{0.020833in}}%
\pgfusepath{stroke,fill}%
}%
\begin{pgfscope}%
\pgfsys@transformshift{3.910439in}{2.222124in}%
\pgfsys@useobject{currentmarker}{}%
\end{pgfscope}%
\end{pgfscope}%
\begin{pgfscope}%
\pgfsetbuttcap%
\pgfsetroundjoin%
\definecolor{currentfill}{rgb}{0.000000,0.000000,0.000000}%
\pgfsetfillcolor{currentfill}%
\pgfsetlinewidth{0.501875pt}%
\definecolor{currentstroke}{rgb}{0.000000,0.000000,0.000000}%
\pgfsetstrokecolor{currentstroke}%
\pgfsetdash{}{0pt}%
\pgfsys@defobject{currentmarker}{\pgfqpoint{0.000000in}{-0.020833in}}{\pgfqpoint{0.000000in}{0.000000in}}{%
\pgfpathmoveto{\pgfqpoint{0.000000in}{0.000000in}}%
\pgfpathlineto{\pgfqpoint{0.000000in}{-0.020833in}}%
\pgfusepath{stroke,fill}%
}%
\begin{pgfscope}%
\pgfsys@transformshift{3.910439in}{2.689374in}%
\pgfsys@useobject{currentmarker}{}%
\end{pgfscope}%
\end{pgfscope}%
\begin{pgfscope}%
\pgfsetbuttcap%
\pgfsetroundjoin%
\definecolor{currentfill}{rgb}{0.000000,0.000000,0.000000}%
\pgfsetfillcolor{currentfill}%
\pgfsetlinewidth{0.501875pt}%
\definecolor{currentstroke}{rgb}{0.000000,0.000000,0.000000}%
\pgfsetstrokecolor{currentstroke}%
\pgfsetdash{}{0pt}%
\pgfsys@defobject{currentmarker}{\pgfqpoint{0.000000in}{0.000000in}}{\pgfqpoint{0.000000in}{0.020833in}}{%
\pgfpathmoveto{\pgfqpoint{0.000000in}{0.000000in}}%
\pgfpathlineto{\pgfqpoint{0.000000in}{0.020833in}}%
\pgfusepath{stroke,fill}%
}%
\begin{pgfscope}%
\pgfsys@transformshift{4.102589in}{2.222124in}%
\pgfsys@useobject{currentmarker}{}%
\end{pgfscope}%
\end{pgfscope}%
\begin{pgfscope}%
\pgfsetbuttcap%
\pgfsetroundjoin%
\definecolor{currentfill}{rgb}{0.000000,0.000000,0.000000}%
\pgfsetfillcolor{currentfill}%
\pgfsetlinewidth{0.501875pt}%
\definecolor{currentstroke}{rgb}{0.000000,0.000000,0.000000}%
\pgfsetstrokecolor{currentstroke}%
\pgfsetdash{}{0pt}%
\pgfsys@defobject{currentmarker}{\pgfqpoint{0.000000in}{-0.020833in}}{\pgfqpoint{0.000000in}{0.000000in}}{%
\pgfpathmoveto{\pgfqpoint{0.000000in}{0.000000in}}%
\pgfpathlineto{\pgfqpoint{0.000000in}{-0.020833in}}%
\pgfusepath{stroke,fill}%
}%
\begin{pgfscope}%
\pgfsys@transformshift{4.102589in}{2.689374in}%
\pgfsys@useobject{currentmarker}{}%
\end{pgfscope}%
\end{pgfscope}%
\begin{pgfscope}%
\pgfsetbuttcap%
\pgfsetroundjoin%
\definecolor{currentfill}{rgb}{0.000000,0.000000,0.000000}%
\pgfsetfillcolor{currentfill}%
\pgfsetlinewidth{0.501875pt}%
\definecolor{currentstroke}{rgb}{0.000000,0.000000,0.000000}%
\pgfsetstrokecolor{currentstroke}%
\pgfsetdash{}{0pt}%
\pgfsys@defobject{currentmarker}{\pgfqpoint{0.000000in}{0.000000in}}{\pgfqpoint{0.000000in}{0.020833in}}{%
\pgfpathmoveto{\pgfqpoint{0.000000in}{0.000000in}}%
\pgfpathlineto{\pgfqpoint{0.000000in}{0.020833in}}%
\pgfusepath{stroke,fill}%
}%
\begin{pgfscope}%
\pgfsys@transformshift{4.198664in}{2.222124in}%
\pgfsys@useobject{currentmarker}{}%
\end{pgfscope}%
\end{pgfscope}%
\begin{pgfscope}%
\pgfsetbuttcap%
\pgfsetroundjoin%
\definecolor{currentfill}{rgb}{0.000000,0.000000,0.000000}%
\pgfsetfillcolor{currentfill}%
\pgfsetlinewidth{0.501875pt}%
\definecolor{currentstroke}{rgb}{0.000000,0.000000,0.000000}%
\pgfsetstrokecolor{currentstroke}%
\pgfsetdash{}{0pt}%
\pgfsys@defobject{currentmarker}{\pgfqpoint{0.000000in}{-0.020833in}}{\pgfqpoint{0.000000in}{0.000000in}}{%
\pgfpathmoveto{\pgfqpoint{0.000000in}{0.000000in}}%
\pgfpathlineto{\pgfqpoint{0.000000in}{-0.020833in}}%
\pgfusepath{stroke,fill}%
}%
\begin{pgfscope}%
\pgfsys@transformshift{4.198664in}{2.689374in}%
\pgfsys@useobject{currentmarker}{}%
\end{pgfscope}%
\end{pgfscope}%
\begin{pgfscope}%
\pgfsetbuttcap%
\pgfsetroundjoin%
\definecolor{currentfill}{rgb}{0.000000,0.000000,0.000000}%
\pgfsetfillcolor{currentfill}%
\pgfsetlinewidth{0.501875pt}%
\definecolor{currentstroke}{rgb}{0.000000,0.000000,0.000000}%
\pgfsetstrokecolor{currentstroke}%
\pgfsetdash{}{0pt}%
\pgfsys@defobject{currentmarker}{\pgfqpoint{0.000000in}{0.000000in}}{\pgfqpoint{0.000000in}{0.020833in}}{%
\pgfpathmoveto{\pgfqpoint{0.000000in}{0.000000in}}%
\pgfpathlineto{\pgfqpoint{0.000000in}{0.020833in}}%
\pgfusepath{stroke,fill}%
}%
\begin{pgfscope}%
\pgfsys@transformshift{4.294739in}{2.222124in}%
\pgfsys@useobject{currentmarker}{}%
\end{pgfscope}%
\end{pgfscope}%
\begin{pgfscope}%
\pgfsetbuttcap%
\pgfsetroundjoin%
\definecolor{currentfill}{rgb}{0.000000,0.000000,0.000000}%
\pgfsetfillcolor{currentfill}%
\pgfsetlinewidth{0.501875pt}%
\definecolor{currentstroke}{rgb}{0.000000,0.000000,0.000000}%
\pgfsetstrokecolor{currentstroke}%
\pgfsetdash{}{0pt}%
\pgfsys@defobject{currentmarker}{\pgfqpoint{0.000000in}{-0.020833in}}{\pgfqpoint{0.000000in}{0.000000in}}{%
\pgfpathmoveto{\pgfqpoint{0.000000in}{0.000000in}}%
\pgfpathlineto{\pgfqpoint{0.000000in}{-0.020833in}}%
\pgfusepath{stroke,fill}%
}%
\begin{pgfscope}%
\pgfsys@transformshift{4.294739in}{2.689374in}%
\pgfsys@useobject{currentmarker}{}%
\end{pgfscope}%
\end{pgfscope}%
\begin{pgfscope}%
\pgfsetbuttcap%
\pgfsetroundjoin%
\definecolor{currentfill}{rgb}{0.000000,0.000000,0.000000}%
\pgfsetfillcolor{currentfill}%
\pgfsetlinewidth{0.501875pt}%
\definecolor{currentstroke}{rgb}{0.000000,0.000000,0.000000}%
\pgfsetstrokecolor{currentstroke}%
\pgfsetdash{}{0pt}%
\pgfsys@defobject{currentmarker}{\pgfqpoint{0.000000in}{0.000000in}}{\pgfqpoint{0.000000in}{0.020833in}}{%
\pgfpathmoveto{\pgfqpoint{0.000000in}{0.000000in}}%
\pgfpathlineto{\pgfqpoint{0.000000in}{0.020833in}}%
\pgfusepath{stroke,fill}%
}%
\begin{pgfscope}%
\pgfsys@transformshift{4.390814in}{2.222124in}%
\pgfsys@useobject{currentmarker}{}%
\end{pgfscope}%
\end{pgfscope}%
\begin{pgfscope}%
\pgfsetbuttcap%
\pgfsetroundjoin%
\definecolor{currentfill}{rgb}{0.000000,0.000000,0.000000}%
\pgfsetfillcolor{currentfill}%
\pgfsetlinewidth{0.501875pt}%
\definecolor{currentstroke}{rgb}{0.000000,0.000000,0.000000}%
\pgfsetstrokecolor{currentstroke}%
\pgfsetdash{}{0pt}%
\pgfsys@defobject{currentmarker}{\pgfqpoint{0.000000in}{-0.020833in}}{\pgfqpoint{0.000000in}{0.000000in}}{%
\pgfpathmoveto{\pgfqpoint{0.000000in}{0.000000in}}%
\pgfpathlineto{\pgfqpoint{0.000000in}{-0.020833in}}%
\pgfusepath{stroke,fill}%
}%
\begin{pgfscope}%
\pgfsys@transformshift{4.390814in}{2.689374in}%
\pgfsys@useobject{currentmarker}{}%
\end{pgfscope}%
\end{pgfscope}%
\begin{pgfscope}%
\pgfsetbuttcap%
\pgfsetroundjoin%
\definecolor{currentfill}{rgb}{0.000000,0.000000,0.000000}%
\pgfsetfillcolor{currentfill}%
\pgfsetlinewidth{0.501875pt}%
\definecolor{currentstroke}{rgb}{0.000000,0.000000,0.000000}%
\pgfsetstrokecolor{currentstroke}%
\pgfsetdash{}{0pt}%
\pgfsys@defobject{currentmarker}{\pgfqpoint{0.000000in}{0.000000in}}{\pgfqpoint{0.000000in}{0.020833in}}{%
\pgfpathmoveto{\pgfqpoint{0.000000in}{0.000000in}}%
\pgfpathlineto{\pgfqpoint{0.000000in}{0.020833in}}%
\pgfusepath{stroke,fill}%
}%
\begin{pgfscope}%
\pgfsys@transformshift{4.582964in}{2.222124in}%
\pgfsys@useobject{currentmarker}{}%
\end{pgfscope}%
\end{pgfscope}%
\begin{pgfscope}%
\pgfsetbuttcap%
\pgfsetroundjoin%
\definecolor{currentfill}{rgb}{0.000000,0.000000,0.000000}%
\pgfsetfillcolor{currentfill}%
\pgfsetlinewidth{0.501875pt}%
\definecolor{currentstroke}{rgb}{0.000000,0.000000,0.000000}%
\pgfsetstrokecolor{currentstroke}%
\pgfsetdash{}{0pt}%
\pgfsys@defobject{currentmarker}{\pgfqpoint{0.000000in}{-0.020833in}}{\pgfqpoint{0.000000in}{0.000000in}}{%
\pgfpathmoveto{\pgfqpoint{0.000000in}{0.000000in}}%
\pgfpathlineto{\pgfqpoint{0.000000in}{-0.020833in}}%
\pgfusepath{stroke,fill}%
}%
\begin{pgfscope}%
\pgfsys@transformshift{4.582964in}{2.689374in}%
\pgfsys@useobject{currentmarker}{}%
\end{pgfscope}%
\end{pgfscope}%
\begin{pgfscope}%
\pgfsetbuttcap%
\pgfsetroundjoin%
\definecolor{currentfill}{rgb}{0.000000,0.000000,0.000000}%
\pgfsetfillcolor{currentfill}%
\pgfsetlinewidth{0.501875pt}%
\definecolor{currentstroke}{rgb}{0.000000,0.000000,0.000000}%
\pgfsetstrokecolor{currentstroke}%
\pgfsetdash{}{0pt}%
\pgfsys@defobject{currentmarker}{\pgfqpoint{0.000000in}{0.000000in}}{\pgfqpoint{0.041667in}{0.000000in}}{%
\pgfpathmoveto{\pgfqpoint{0.000000in}{0.000000in}}%
\pgfpathlineto{\pgfqpoint{0.041667in}{0.000000in}}%
\pgfusepath{stroke,fill}%
}%
\begin{pgfscope}%
\pgfsys@transformshift{0.444748in}{2.240156in}%
\pgfsys@useobject{currentmarker}{}%
\end{pgfscope}%
\end{pgfscope}%
\begin{pgfscope}%
\pgfsetbuttcap%
\pgfsetroundjoin%
\definecolor{currentfill}{rgb}{0.000000,0.000000,0.000000}%
\pgfsetfillcolor{currentfill}%
\pgfsetlinewidth{0.501875pt}%
\definecolor{currentstroke}{rgb}{0.000000,0.000000,0.000000}%
\pgfsetstrokecolor{currentstroke}%
\pgfsetdash{}{0pt}%
\pgfsys@defobject{currentmarker}{\pgfqpoint{-0.041667in}{0.000000in}}{\pgfqpoint{-0.000000in}{0.000000in}}{%
\pgfpathmoveto{\pgfqpoint{-0.000000in}{0.000000in}}%
\pgfpathlineto{\pgfqpoint{-0.041667in}{0.000000in}}%
\pgfusepath{stroke,fill}%
}%
\begin{pgfscope}%
\pgfsys@transformshift{4.676167in}{2.240156in}%
\pgfsys@useobject{currentmarker}{}%
\end{pgfscope}%
\end{pgfscope}%
\begin{pgfscope}%
\definecolor{textcolor}{rgb}{0.000000,0.000000,0.000000}%
\pgfsetstrokecolor{textcolor}%
\pgfsetfillcolor{textcolor}%
\pgftext[x=0.326693in, y=2.191938in, left, base]{\color{textcolor}\rmfamily\fontsize{10.000000}{12.000000}\selectfont \(\displaystyle {0}\)}%
\end{pgfscope}%
\begin{pgfscope}%
\pgfsetbuttcap%
\pgfsetroundjoin%
\definecolor{currentfill}{rgb}{0.000000,0.000000,0.000000}%
\pgfsetfillcolor{currentfill}%
\pgfsetlinewidth{0.501875pt}%
\definecolor{currentstroke}{rgb}{0.000000,0.000000,0.000000}%
\pgfsetstrokecolor{currentstroke}%
\pgfsetdash{}{0pt}%
\pgfsys@defobject{currentmarker}{\pgfqpoint{0.000000in}{0.000000in}}{\pgfqpoint{0.041667in}{0.000000in}}{%
\pgfpathmoveto{\pgfqpoint{0.000000in}{0.000000in}}%
\pgfpathlineto{\pgfqpoint{0.041667in}{0.000000in}}%
\pgfusepath{stroke,fill}%
}%
\begin{pgfscope}%
\pgfsys@transformshift{0.444748in}{2.598322in}%
\pgfsys@useobject{currentmarker}{}%
\end{pgfscope}%
\end{pgfscope}%
\begin{pgfscope}%
\pgfsetbuttcap%
\pgfsetroundjoin%
\definecolor{currentfill}{rgb}{0.000000,0.000000,0.000000}%
\pgfsetfillcolor{currentfill}%
\pgfsetlinewidth{0.501875pt}%
\definecolor{currentstroke}{rgb}{0.000000,0.000000,0.000000}%
\pgfsetstrokecolor{currentstroke}%
\pgfsetdash{}{0pt}%
\pgfsys@defobject{currentmarker}{\pgfqpoint{-0.041667in}{0.000000in}}{\pgfqpoint{-0.000000in}{0.000000in}}{%
\pgfpathmoveto{\pgfqpoint{-0.000000in}{0.000000in}}%
\pgfpathlineto{\pgfqpoint{-0.041667in}{0.000000in}}%
\pgfusepath{stroke,fill}%
}%
\begin{pgfscope}%
\pgfsys@transformshift{4.676167in}{2.598322in}%
\pgfsys@useobject{currentmarker}{}%
\end{pgfscope}%
\end{pgfscope}%
\begin{pgfscope}%
\definecolor{textcolor}{rgb}{0.000000,0.000000,0.000000}%
\pgfsetstrokecolor{textcolor}%
\pgfsetfillcolor{textcolor}%
\pgftext[x=0.257248in, y=2.550104in, left, base]{\color{textcolor}\rmfamily\fontsize{10.000000}{12.000000}\selectfont \(\displaystyle {50}\)}%
\end{pgfscope}%
\begin{pgfscope}%
\pgfsetbuttcap%
\pgfsetroundjoin%
\definecolor{currentfill}{rgb}{0.000000,0.000000,0.000000}%
\pgfsetfillcolor{currentfill}%
\pgfsetlinewidth{0.501875pt}%
\definecolor{currentstroke}{rgb}{0.000000,0.000000,0.000000}%
\pgfsetstrokecolor{currentstroke}%
\pgfsetdash{}{0pt}%
\pgfsys@defobject{currentmarker}{\pgfqpoint{0.000000in}{0.000000in}}{\pgfqpoint{0.020833in}{0.000000in}}{%
\pgfpathmoveto{\pgfqpoint{0.000000in}{0.000000in}}%
\pgfpathlineto{\pgfqpoint{0.020833in}{0.000000in}}%
\pgfusepath{stroke,fill}%
}%
\begin{pgfscope}%
\pgfsys@transformshift{0.444748in}{2.311789in}%
\pgfsys@useobject{currentmarker}{}%
\end{pgfscope}%
\end{pgfscope}%
\begin{pgfscope}%
\pgfsetbuttcap%
\pgfsetroundjoin%
\definecolor{currentfill}{rgb}{0.000000,0.000000,0.000000}%
\pgfsetfillcolor{currentfill}%
\pgfsetlinewidth{0.501875pt}%
\definecolor{currentstroke}{rgb}{0.000000,0.000000,0.000000}%
\pgfsetstrokecolor{currentstroke}%
\pgfsetdash{}{0pt}%
\pgfsys@defobject{currentmarker}{\pgfqpoint{-0.020833in}{0.000000in}}{\pgfqpoint{-0.000000in}{0.000000in}}{%
\pgfpathmoveto{\pgfqpoint{-0.000000in}{0.000000in}}%
\pgfpathlineto{\pgfqpoint{-0.020833in}{0.000000in}}%
\pgfusepath{stroke,fill}%
}%
\begin{pgfscope}%
\pgfsys@transformshift{4.676167in}{2.311789in}%
\pgfsys@useobject{currentmarker}{}%
\end{pgfscope}%
\end{pgfscope}%
\begin{pgfscope}%
\pgfsetbuttcap%
\pgfsetroundjoin%
\definecolor{currentfill}{rgb}{0.000000,0.000000,0.000000}%
\pgfsetfillcolor{currentfill}%
\pgfsetlinewidth{0.501875pt}%
\definecolor{currentstroke}{rgb}{0.000000,0.000000,0.000000}%
\pgfsetstrokecolor{currentstroke}%
\pgfsetdash{}{0pt}%
\pgfsys@defobject{currentmarker}{\pgfqpoint{0.000000in}{0.000000in}}{\pgfqpoint{0.020833in}{0.000000in}}{%
\pgfpathmoveto{\pgfqpoint{0.000000in}{0.000000in}}%
\pgfpathlineto{\pgfqpoint{0.020833in}{0.000000in}}%
\pgfusepath{stroke,fill}%
}%
\begin{pgfscope}%
\pgfsys@transformshift{0.444748in}{2.383422in}%
\pgfsys@useobject{currentmarker}{}%
\end{pgfscope}%
\end{pgfscope}%
\begin{pgfscope}%
\pgfsetbuttcap%
\pgfsetroundjoin%
\definecolor{currentfill}{rgb}{0.000000,0.000000,0.000000}%
\pgfsetfillcolor{currentfill}%
\pgfsetlinewidth{0.501875pt}%
\definecolor{currentstroke}{rgb}{0.000000,0.000000,0.000000}%
\pgfsetstrokecolor{currentstroke}%
\pgfsetdash{}{0pt}%
\pgfsys@defobject{currentmarker}{\pgfqpoint{-0.020833in}{0.000000in}}{\pgfqpoint{-0.000000in}{0.000000in}}{%
\pgfpathmoveto{\pgfqpoint{-0.000000in}{0.000000in}}%
\pgfpathlineto{\pgfqpoint{-0.020833in}{0.000000in}}%
\pgfusepath{stroke,fill}%
}%
\begin{pgfscope}%
\pgfsys@transformshift{4.676167in}{2.383422in}%
\pgfsys@useobject{currentmarker}{}%
\end{pgfscope}%
\end{pgfscope}%
\begin{pgfscope}%
\pgfsetbuttcap%
\pgfsetroundjoin%
\definecolor{currentfill}{rgb}{0.000000,0.000000,0.000000}%
\pgfsetfillcolor{currentfill}%
\pgfsetlinewidth{0.501875pt}%
\definecolor{currentstroke}{rgb}{0.000000,0.000000,0.000000}%
\pgfsetstrokecolor{currentstroke}%
\pgfsetdash{}{0pt}%
\pgfsys@defobject{currentmarker}{\pgfqpoint{0.000000in}{0.000000in}}{\pgfqpoint{0.020833in}{0.000000in}}{%
\pgfpathmoveto{\pgfqpoint{0.000000in}{0.000000in}}%
\pgfpathlineto{\pgfqpoint{0.020833in}{0.000000in}}%
\pgfusepath{stroke,fill}%
}%
\begin{pgfscope}%
\pgfsys@transformshift{0.444748in}{2.455055in}%
\pgfsys@useobject{currentmarker}{}%
\end{pgfscope}%
\end{pgfscope}%
\begin{pgfscope}%
\pgfsetbuttcap%
\pgfsetroundjoin%
\definecolor{currentfill}{rgb}{0.000000,0.000000,0.000000}%
\pgfsetfillcolor{currentfill}%
\pgfsetlinewidth{0.501875pt}%
\definecolor{currentstroke}{rgb}{0.000000,0.000000,0.000000}%
\pgfsetstrokecolor{currentstroke}%
\pgfsetdash{}{0pt}%
\pgfsys@defobject{currentmarker}{\pgfqpoint{-0.020833in}{0.000000in}}{\pgfqpoint{-0.000000in}{0.000000in}}{%
\pgfpathmoveto{\pgfqpoint{-0.000000in}{0.000000in}}%
\pgfpathlineto{\pgfqpoint{-0.020833in}{0.000000in}}%
\pgfusepath{stroke,fill}%
}%
\begin{pgfscope}%
\pgfsys@transformshift{4.676167in}{2.455055in}%
\pgfsys@useobject{currentmarker}{}%
\end{pgfscope}%
\end{pgfscope}%
\begin{pgfscope}%
\pgfsetbuttcap%
\pgfsetroundjoin%
\definecolor{currentfill}{rgb}{0.000000,0.000000,0.000000}%
\pgfsetfillcolor{currentfill}%
\pgfsetlinewidth{0.501875pt}%
\definecolor{currentstroke}{rgb}{0.000000,0.000000,0.000000}%
\pgfsetstrokecolor{currentstroke}%
\pgfsetdash{}{0pt}%
\pgfsys@defobject{currentmarker}{\pgfqpoint{0.000000in}{0.000000in}}{\pgfqpoint{0.020833in}{0.000000in}}{%
\pgfpathmoveto{\pgfqpoint{0.000000in}{0.000000in}}%
\pgfpathlineto{\pgfqpoint{0.020833in}{0.000000in}}%
\pgfusepath{stroke,fill}%
}%
\begin{pgfscope}%
\pgfsys@transformshift{0.444748in}{2.526688in}%
\pgfsys@useobject{currentmarker}{}%
\end{pgfscope}%
\end{pgfscope}%
\begin{pgfscope}%
\pgfsetbuttcap%
\pgfsetroundjoin%
\definecolor{currentfill}{rgb}{0.000000,0.000000,0.000000}%
\pgfsetfillcolor{currentfill}%
\pgfsetlinewidth{0.501875pt}%
\definecolor{currentstroke}{rgb}{0.000000,0.000000,0.000000}%
\pgfsetstrokecolor{currentstroke}%
\pgfsetdash{}{0pt}%
\pgfsys@defobject{currentmarker}{\pgfqpoint{-0.020833in}{0.000000in}}{\pgfqpoint{-0.000000in}{0.000000in}}{%
\pgfpathmoveto{\pgfqpoint{-0.000000in}{0.000000in}}%
\pgfpathlineto{\pgfqpoint{-0.020833in}{0.000000in}}%
\pgfusepath{stroke,fill}%
}%
\begin{pgfscope}%
\pgfsys@transformshift{4.676167in}{2.526688in}%
\pgfsys@useobject{currentmarker}{}%
\end{pgfscope}%
\end{pgfscope}%
\begin{pgfscope}%
\pgfsetbuttcap%
\pgfsetroundjoin%
\definecolor{currentfill}{rgb}{0.000000,0.000000,0.000000}%
\pgfsetfillcolor{currentfill}%
\pgfsetlinewidth{0.501875pt}%
\definecolor{currentstroke}{rgb}{0.000000,0.000000,0.000000}%
\pgfsetstrokecolor{currentstroke}%
\pgfsetdash{}{0pt}%
\pgfsys@defobject{currentmarker}{\pgfqpoint{0.000000in}{0.000000in}}{\pgfqpoint{0.020833in}{0.000000in}}{%
\pgfpathmoveto{\pgfqpoint{0.000000in}{0.000000in}}%
\pgfpathlineto{\pgfqpoint{0.020833in}{0.000000in}}%
\pgfusepath{stroke,fill}%
}%
\begin{pgfscope}%
\pgfsys@transformshift{0.444748in}{2.669955in}%
\pgfsys@useobject{currentmarker}{}%
\end{pgfscope}%
\end{pgfscope}%
\begin{pgfscope}%
\pgfsetbuttcap%
\pgfsetroundjoin%
\definecolor{currentfill}{rgb}{0.000000,0.000000,0.000000}%
\pgfsetfillcolor{currentfill}%
\pgfsetlinewidth{0.501875pt}%
\definecolor{currentstroke}{rgb}{0.000000,0.000000,0.000000}%
\pgfsetstrokecolor{currentstroke}%
\pgfsetdash{}{0pt}%
\pgfsys@defobject{currentmarker}{\pgfqpoint{-0.020833in}{0.000000in}}{\pgfqpoint{-0.000000in}{0.000000in}}{%
\pgfpathmoveto{\pgfqpoint{-0.000000in}{0.000000in}}%
\pgfpathlineto{\pgfqpoint{-0.020833in}{0.000000in}}%
\pgfusepath{stroke,fill}%
}%
\begin{pgfscope}%
\pgfsys@transformshift{4.676167in}{2.669955in}%
\pgfsys@useobject{currentmarker}{}%
\end{pgfscope}%
\end{pgfscope}%
\begin{pgfscope}%
\definecolor{textcolor}{rgb}{0.000000,0.000000,0.000000}%
\pgfsetstrokecolor{textcolor}%
\pgfsetfillcolor{textcolor}%
\pgftext[x=0.201692in,y=2.455749in,,bottom,rotate=90.000000]{\color{textcolor}\rmfamily\fontsize{12.000000}{14.400000}\selectfont \(\displaystyle V_s\) (\unit{\micro\volt})}%
\end{pgfscope}%
\begin{pgfscope}%
\pgfpathrectangle{\pgfqpoint{0.444748in}{2.222124in}}{\pgfqpoint{4.231419in}{0.467251in}}%
\pgfusepath{clip}%
\pgfsetbuttcap%
\pgfsetroundjoin%
\pgfsetlinewidth{1.003750pt}%
\definecolor{currentstroke}{rgb}{0.047059,0.364706,0.647059}%
\pgfsetstrokecolor{currentstroke}%
\pgfsetdash{{3.700000pt}{1.600000pt}}{0.000000pt}%
\pgfpathmoveto{\pgfqpoint{0.646240in}{2.434724in}}%
\pgfpathlineto{\pgfqpoint{0.656568in}{2.358099in}}%
\pgfpathlineto{\pgfqpoint{0.674878in}{2.287536in}}%
\pgfpathlineto{\pgfqpoint{0.695063in}{2.253612in}}%
\pgfpathlineto{\pgfqpoint{0.714782in}{2.262780in}}%
\pgfpathlineto{\pgfqpoint{0.734733in}{2.302961in}}%
\pgfpathlineto{\pgfqpoint{0.755389in}{2.398987in}}%
\pgfpathlineto{\pgfqpoint{0.771821in}{2.556333in}}%
\pgfpathlineto{\pgfqpoint{0.789661in}{2.267074in}}%
\pgfpathlineto{\pgfqpoint{0.813603in}{2.251985in}}%
\pgfpathlineto{\pgfqpoint{0.829330in}{2.274037in}}%
\pgfpathlineto{\pgfqpoint{0.849281in}{2.333928in}}%
\pgfpathlineto{\pgfqpoint{0.886839in}{2.618798in}}%
\pgfpathlineto{\pgfqpoint{0.906321in}{2.629245in}}%
\pgfpathlineto{\pgfqpoint{0.945990in}{2.337482in}}%
\pgfpathlineto{\pgfqpoint{0.962655in}{2.275173in}}%
\pgfpathlineto{\pgfqpoint{0.981200in}{2.250313in}}%
\pgfpathlineto{\pgfqpoint{1.003499in}{2.259721in}}%
\pgfpathlineto{\pgfqpoint{1.025563in}{2.308693in}}%
\pgfpathlineto{\pgfqpoint{1.041994in}{2.385622in}}%
\pgfpathlineto{\pgfqpoint{1.058425in}{2.539909in}}%
\pgfpathlineto{\pgfqpoint{1.079550in}{2.636511in}}%
\pgfpathlineto{\pgfqpoint{1.096686in}{2.574493in}}%
\pgfpathlineto{\pgfqpoint{1.118750in}{2.391243in}}%
\pgfpathlineto{\pgfqpoint{1.138938in}{2.290014in}}%
\pgfpathlineto{\pgfqpoint{1.157246in}{2.257846in}}%
\pgfpathlineto{\pgfqpoint{1.176025in}{2.247415in}}%
\pgfpathlineto{\pgfqpoint{1.193629in}{2.264725in}}%
\pgfpathlineto{\pgfqpoint{1.215460in}{2.311783in}}%
\pgfpathlineto{\pgfqpoint{1.254189in}{2.524584in}}%
\pgfpathlineto{\pgfqpoint{1.269681in}{2.616174in}}%
\pgfpathlineto{\pgfqpoint{1.290102in}{2.600181in}}%
\pgfpathlineto{\pgfqpoint{1.311229in}{2.420617in}}%
\pgfpathlineto{\pgfqpoint{1.329537in}{2.327614in}}%
\pgfpathlineto{\pgfqpoint{1.346202in}{2.273337in}}%
\pgfpathlineto{\pgfqpoint{1.367329in}{2.250142in}}%
\pgfpathlineto{\pgfqpoint{1.386577in}{2.248862in}}%
\pgfpathlineto{\pgfqpoint{1.407468in}{2.274377in}}%
\pgfpathlineto{\pgfqpoint{1.424133in}{2.314498in}}%
\pgfpathlineto{\pgfqpoint{1.445260in}{2.419881in}}%
\pgfpathlineto{\pgfqpoint{1.464742in}{2.543251in}}%
\pgfpathlineto{\pgfqpoint{1.484929in}{2.623557in}}%
\pgfpathlineto{\pgfqpoint{1.506757in}{2.549471in}}%
\pgfpathlineto{\pgfqpoint{1.520373in}{2.455193in}}%
\pgfpathlineto{\pgfqpoint{1.541029in}{2.319949in}}%
\pgfpathlineto{\pgfqpoint{1.559806in}{2.272995in}}%
\pgfpathlineto{\pgfqpoint{1.580462in}{2.248982in}}%
\pgfpathlineto{\pgfqpoint{1.599241in}{2.245254in}}%
\pgfpathlineto{\pgfqpoint{1.616143in}{2.252101in}}%
\pgfpathlineto{\pgfqpoint{1.637737in}{2.279336in}}%
\pgfpathlineto{\pgfqpoint{1.655576in}{2.323007in}}%
\pgfpathlineto{\pgfqpoint{1.675997in}{2.416959in}}%
\pgfpathlineto{\pgfqpoint{1.692428in}{2.544708in}}%
\pgfpathlineto{\pgfqpoint{1.716372in}{2.618179in}}%
\pgfpathlineto{\pgfqpoint{1.732334in}{2.579966in}}%
\pgfpathlineto{\pgfqpoint{1.754867in}{2.427387in}}%
\pgfpathlineto{\pgfqpoint{1.771767in}{2.332722in}}%
\pgfpathlineto{\pgfqpoint{1.789843in}{2.284915in}}%
\pgfpathlineto{\pgfqpoint{1.808151in}{2.263604in}}%
\pgfpathlineto{\pgfqpoint{1.828807in}{2.247416in}}%
\pgfpathlineto{\pgfqpoint{1.846880in}{2.247974in}}%
\pgfpathlineto{\pgfqpoint{1.867771in}{2.270029in}}%
\pgfpathlineto{\pgfqpoint{1.885141in}{2.308908in}}%
\pgfpathlineto{\pgfqpoint{1.924811in}{2.428900in}}%
\pgfpathlineto{\pgfqpoint{1.946172in}{2.569457in}}%
\pgfpathlineto{\pgfqpoint{1.963777in}{2.614238in}}%
\pgfpathlineto{\pgfqpoint{1.982319in}{2.561511in}}%
\pgfpathlineto{\pgfqpoint{2.002507in}{2.441113in}}%
\pgfpathlineto{\pgfqpoint{2.020346in}{2.581264in}}%
\pgfpathlineto{\pgfqpoint{2.038654in}{2.616674in}}%
\pgfpathlineto{\pgfqpoint{2.062127in}{2.548514in}}%
\pgfpathlineto{\pgfqpoint{2.077620in}{2.423751in}}%
\pgfpathlineto{\pgfqpoint{2.099450in}{2.307998in}}%
\pgfpathlineto{\pgfqpoint{2.116584in}{2.265476in}}%
\pgfpathlineto{\pgfqpoint{2.134424in}{2.246998in}}%
\pgfpathlineto{\pgfqpoint{2.152734in}{2.247937in}}%
\pgfpathlineto{\pgfqpoint{2.173624in}{2.266934in}}%
\pgfpathlineto{\pgfqpoint{2.194749in}{2.317378in}}%
\pgfpathlineto{\pgfqpoint{2.213528in}{2.414140in}}%
\pgfpathlineto{\pgfqpoint{2.232776in}{2.548631in}}%
\pgfpathlineto{\pgfqpoint{2.250615in}{2.614017in}}%
\pgfpathlineto{\pgfqpoint{2.269628in}{2.567441in}}%
\pgfpathlineto{\pgfqpoint{2.290989in}{2.408738in}}%
\pgfpathlineto{\pgfqpoint{2.306951in}{2.311927in}}%
\pgfpathlineto{\pgfqpoint{2.327607in}{2.268124in}}%
\pgfpathlineto{\pgfqpoint{2.348498in}{2.246599in}}%
\pgfpathlineto{\pgfqpoint{2.365868in}{2.245248in}}%
\pgfpathlineto{\pgfqpoint{2.384176in}{2.259599in}}%
\pgfpathlineto{\pgfqpoint{2.405772in}{2.295584in}}%
\pgfpathlineto{\pgfqpoint{2.423141in}{2.363952in}}%
\pgfpathlineto{\pgfqpoint{2.462341in}{2.598132in}}%
\pgfpathlineto{\pgfqpoint{2.480415in}{2.613591in}}%
\pgfpathlineto{\pgfqpoint{2.501307in}{2.543821in}}%
\pgfpathlineto{\pgfqpoint{2.520555in}{2.408015in}}%
\pgfpathlineto{\pgfqpoint{2.540740in}{2.304749in}}%
\pgfpathlineto{\pgfqpoint{2.559050in}{2.268266in}}%
\pgfpathlineto{\pgfqpoint{2.576655in}{2.249855in}}%
\pgfpathlineto{\pgfqpoint{2.598249in}{2.245018in}}%
\pgfpathlineto{\pgfqpoint{2.615619in}{2.256429in}}%
\pgfpathlineto{\pgfqpoint{2.640267in}{2.278708in}}%
\pgfpathlineto{\pgfqpoint{2.655289in}{2.317381in}}%
\pgfpathlineto{\pgfqpoint{2.672425in}{2.385036in}}%
\pgfpathlineto{\pgfqpoint{2.693784in}{2.453004in}}%
\pgfpathlineto{\pgfqpoint{2.712797in}{2.575893in}}%
\pgfpathlineto{\pgfqpoint{2.733688in}{2.609279in}}%
\pgfpathlineto{\pgfqpoint{2.769837in}{2.454085in}}%
\pgfpathlineto{\pgfqpoint{2.792840in}{2.304961in}}%
\pgfpathlineto{\pgfqpoint{2.806455in}{2.278973in}}%
\pgfpathlineto{\pgfqpoint{2.827815in}{2.251989in}}%
\pgfpathlineto{\pgfqpoint{2.847533in}{2.245893in}}%
\pgfpathlineto{\pgfqpoint{2.866076in}{2.248614in}}%
\pgfpathlineto{\pgfqpoint{2.886497in}{2.273530in}}%
\pgfpathlineto{\pgfqpoint{2.906450in}{2.317350in}}%
\pgfpathlineto{\pgfqpoint{2.923819in}{2.397753in}}%
\pgfpathlineto{\pgfqpoint{2.943066in}{2.503593in}}%
\pgfpathlineto{\pgfqpoint{2.962080in}{2.569444in}}%
\pgfpathlineto{\pgfqpoint{2.982970in}{2.617266in}}%
\pgfpathlineto{\pgfqpoint{3.000341in}{2.545396in}}%
\pgfpathlineto{\pgfqpoint{3.018885in}{2.415288in}}%
\pgfpathlineto{\pgfqpoint{3.039072in}{2.317353in}}%
\pgfpathlineto{\pgfqpoint{3.057146in}{2.273215in}}%
\pgfpathlineto{\pgfqpoint{3.075454in}{2.254654in}}%
\pgfpathlineto{\pgfqpoint{3.096110in}{2.246118in}}%
\pgfpathlineto{\pgfqpoint{3.117471in}{2.259125in}}%
\pgfpathlineto{\pgfqpoint{3.135311in}{2.281962in}}%
\pgfpathlineto{\pgfqpoint{3.135545in}{2.310327in}}%
\pgfpathlineto{\pgfqpoint{3.154090in}{2.323208in}}%
\pgfpathlineto{\pgfqpoint{3.175215in}{2.410208in}}%
\pgfpathlineto{\pgfqpoint{3.192114in}{2.246133in}}%
\pgfpathlineto{\pgfqpoint{3.214415in}{2.261520in}}%
\pgfpathlineto{\pgfqpoint{3.231549in}{2.295592in}}%
\pgfpathlineto{\pgfqpoint{3.251031in}{2.366591in}}%
\pgfpathlineto{\pgfqpoint{3.270984in}{2.495934in}}%
\pgfpathlineto{\pgfqpoint{3.290935in}{2.603889in}}%
\pgfpathlineto{\pgfqpoint{3.309948in}{2.619615in}}%
\pgfpathlineto{\pgfqpoint{3.328493in}{2.523061in}}%
\pgfpathlineto{\pgfqpoint{3.349149in}{2.394136in}}%
\pgfpathlineto{\pgfqpoint{3.366754in}{2.310906in}}%
\pgfpathlineto{\pgfqpoint{3.384593in}{2.274283in}}%
\pgfpathlineto{\pgfqpoint{3.406892in}{2.260462in}}%
\pgfpathlineto{\pgfqpoint{3.424731in}{2.247263in}}%
\pgfpathlineto{\pgfqpoint{3.442807in}{2.251660in}}%
\pgfpathlineto{\pgfqpoint{3.462289in}{2.274914in}}%
\pgfpathlineto{\pgfqpoint{3.482005in}{2.314085in}}%
\pgfpathlineto{\pgfqpoint{3.517684in}{2.484067in}}%
\pgfpathlineto{\pgfqpoint{3.539985in}{2.519658in}}%
\pgfpathlineto{\pgfqpoint{3.557353in}{2.528071in}}%
\pgfpathlineto{\pgfqpoint{3.581061in}{2.631257in}}%
\pgfpathlineto{\pgfqpoint{3.595614in}{2.614263in}}%
\pgfpathlineto{\pgfqpoint{3.635049in}{2.378346in}}%
\pgfpathlineto{\pgfqpoint{3.652419in}{2.304367in}}%
\pgfpathlineto{\pgfqpoint{3.672370in}{2.268602in}}%
\pgfpathlineto{\pgfqpoint{3.691149in}{2.254634in}}%
\pgfpathlineto{\pgfqpoint{3.711102in}{2.249151in}}%
\pgfpathlineto{\pgfqpoint{3.729176in}{2.265587in}}%
\pgfpathlineto{\pgfqpoint{3.751475in}{2.301723in}}%
\pgfpathlineto{\pgfqpoint{3.769548in}{2.353442in}}%
\pgfpathlineto{\pgfqpoint{3.789032in}{2.457656in}}%
\pgfpathlineto{\pgfqpoint{3.808046in}{2.572502in}}%
\pgfpathlineto{\pgfqpoint{3.826823in}{2.639300in}}%
\pgfpathlineto{\pgfqpoint{3.846070in}{2.627062in}}%
\pgfpathlineto{\pgfqpoint{3.864615in}{2.549533in}}%
\pgfpathlineto{\pgfqpoint{3.884097in}{2.427136in}}%
\pgfpathlineto{\pgfqpoint{3.903110in}{2.351847in}}%
\pgfpathlineto{\pgfqpoint{3.925880in}{2.288802in}}%
\pgfpathlineto{\pgfqpoint{3.944422in}{2.259238in}}%
\pgfpathlineto{\pgfqpoint{3.964610in}{2.257435in}}%
\pgfpathlineto{\pgfqpoint{3.982214in}{2.249899in}}%
\pgfpathlineto{\pgfqpoint{4.001462in}{2.264156in}}%
\pgfpathlineto{\pgfqpoint{4.019770in}{2.293312in}}%
\pgfpathlineto{\pgfqpoint{4.040426in}{2.351107in}}%
\pgfpathlineto{\pgfqpoint{4.055685in}{2.432592in}}%
\pgfpathlineto{\pgfqpoint{4.077279in}{2.531283in}}%
\pgfpathlineto{\pgfqpoint{4.098169in}{2.615257in}}%
\pgfpathlineto{\pgfqpoint{4.117183in}{2.654400in}}%
\pgfpathlineto{\pgfqpoint{4.135024in}{2.611802in}}%
\pgfpathlineto{\pgfqpoint{4.173519in}{2.376935in}}%
\pgfpathlineto{\pgfqpoint{4.193001in}{2.305961in}}%
\pgfpathlineto{\pgfqpoint{4.211075in}{2.604246in}}%
\pgfpathlineto{\pgfqpoint{4.229619in}{2.497077in}}%
\pgfpathlineto{\pgfqpoint{4.248162in}{2.371483in}}%
\pgfpathlineto{\pgfqpoint{4.272340in}{2.292601in}}%
\pgfpathlineto{\pgfqpoint{4.288771in}{2.262380in}}%
\pgfpathlineto{\pgfqpoint{4.308722in}{2.252130in}}%
\pgfpathlineto{\pgfqpoint{4.329378in}{2.271933in}}%
\pgfpathlineto{\pgfqpoint{4.344637in}{2.297965in}}%
\pgfpathlineto{\pgfqpoint{4.363179in}{2.355908in}}%
\pgfpathlineto{\pgfqpoint{4.385009in}{2.476867in}}%
\pgfpathlineto{\pgfqpoint{4.404023in}{2.467288in}}%
\pgfpathlineto{\pgfqpoint{4.422331in}{2.595893in}}%
\pgfpathlineto{\pgfqpoint{4.442518in}{2.665901in}}%
\pgfpathlineto{\pgfqpoint{4.460828in}{2.652432in}}%
\pgfpathlineto{\pgfqpoint{4.480076in}{2.554861in}}%
\pgfpathlineto{\pgfqpoint{4.478902in}{2.578690in}}%
\pgfpathlineto{\pgfqpoint{4.474676in}{2.620859in}}%
\pgfpathlineto{\pgfqpoint{4.454254in}{2.668136in}}%
\pgfpathlineto{\pgfqpoint{4.436650in}{2.576646in}}%
\pgfpathlineto{\pgfqpoint{4.418107in}{2.403272in}}%
\pgfpathlineto{\pgfqpoint{4.396277in}{2.299478in}}%
\pgfpathlineto{\pgfqpoint{4.377029in}{2.258824in}}%
\pgfpathlineto{\pgfqpoint{4.358954in}{2.255939in}}%
\pgfpathlineto{\pgfqpoint{4.340646in}{2.287284in}}%
\pgfpathlineto{\pgfqpoint{4.319050in}{2.384785in}}%
\pgfpathlineto{\pgfqpoint{4.299802in}{2.552702in}}%
\pgfpathlineto{\pgfqpoint{4.281260in}{2.653381in}}%
\pgfpathlineto{\pgfqpoint{4.262715in}{2.615521in}}%
\pgfpathlineto{\pgfqpoint{4.243936in}{2.440449in}}%
\pgfpathlineto{\pgfqpoint{4.225394in}{2.326692in}}%
\pgfpathlineto{\pgfqpoint{4.207555in}{2.271057in}}%
\pgfpathlineto{\pgfqpoint{4.184551in}{2.249630in}}%
\pgfpathlineto{\pgfqpoint{4.166946in}{2.269006in}}%
\pgfpathlineto{\pgfqpoint{4.147932in}{2.321176in}}%
\pgfpathlineto{\pgfqpoint{4.128216in}{2.440584in}}%
\pgfpathlineto{\pgfqpoint{4.110377in}{2.590322in}}%
\pgfpathlineto{\pgfqpoint{4.089250in}{2.644072in}}%
\pgfpathlineto{\pgfqpoint{4.072585in}{2.564095in}}%
\pgfpathlineto{\pgfqpoint{4.051225in}{2.377113in}}%
\pgfpathlineto{\pgfqpoint{4.033620in}{2.296155in}}%
\pgfpathlineto{\pgfqpoint{4.010851in}{2.256092in}}%
\pgfpathlineto{\pgfqpoint{3.993717in}{2.250707in}}%
\pgfpathlineto{\pgfqpoint{3.973764in}{2.277243in}}%
\pgfpathlineto{\pgfqpoint{3.955219in}{2.337492in}}%
\pgfpathlineto{\pgfqpoint{3.936442in}{2.475014in}}%
\pgfpathlineto{\pgfqpoint{3.918367in}{2.616577in}}%
\pgfpathlineto{\pgfqpoint{3.897242in}{2.617320in}}%
\pgfpathlineto{\pgfqpoint{3.877289in}{2.469214in}}%
\pgfpathlineto{\pgfqpoint{3.858278in}{2.353283in}}%
\pgfpathlineto{\pgfqpoint{3.839733in}{2.285265in}}%
\pgfpathlineto{\pgfqpoint{3.821425in}{2.254286in}}%
\pgfpathlineto{\pgfqpoint{3.799595in}{2.250685in}}%
\pgfpathlineto{\pgfqpoint{3.781051in}{2.270052in}}%
\pgfpathlineto{\pgfqpoint{3.765794in}{2.311109in}}%
\pgfpathlineto{\pgfqpoint{3.741616in}{2.465294in}}%
\pgfpathlineto{\pgfqpoint{3.725185in}{2.552010in}}%
\pgfpathlineto{\pgfqpoint{3.704529in}{2.628365in}}%
\pgfpathlineto{\pgfqpoint{3.685281in}{2.579172in}}%
\pgfpathlineto{\pgfqpoint{3.666502in}{2.427197in}}%
\pgfpathlineto{\pgfqpoint{3.647960in}{2.321114in}}%
\pgfpathlineto{\pgfqpoint{3.628712in}{2.270771in}}%
\pgfpathlineto{\pgfqpoint{3.609933in}{2.248354in}}%
\pgfpathlineto{\pgfqpoint{3.590451in}{2.247870in}}%
\pgfpathlineto{\pgfqpoint{3.567918in}{2.268322in}}%
\pgfpathlineto{\pgfqpoint{3.551016in}{2.305603in}}%
\pgfpathlineto{\pgfqpoint{3.534585in}{2.389648in}}%
\pgfpathlineto{\pgfqpoint{3.513460in}{2.542837in}}%
\pgfpathlineto{\pgfqpoint{3.493976in}{2.619144in}}%
\pgfpathlineto{\pgfqpoint{3.475199in}{2.606867in}}%
\pgfpathlineto{\pgfqpoint{3.451021in}{2.535244in}}%
\pgfpathlineto{\pgfqpoint{3.436468in}{2.402894in}}%
\pgfpathlineto{\pgfqpoint{3.417220in}{2.313496in}}%
\pgfpathlineto{\pgfqpoint{3.401258in}{2.321950in}}%
\pgfpathlineto{\pgfqpoint{3.378959in}{2.619000in}}%
\pgfpathlineto{\pgfqpoint{3.357365in}{2.559496in}}%
\pgfpathlineto{\pgfqpoint{3.343046in}{2.433001in}}%
\pgfpathlineto{\pgfqpoint{3.320747in}{2.316020in}}%
\pgfpathlineto{\pgfqpoint{3.302672in}{2.266631in}}%
\pgfpathlineto{\pgfqpoint{3.282721in}{2.249469in}}%
\pgfpathlineto{\pgfqpoint{3.266055in}{2.247422in}}%
\pgfpathlineto{\pgfqpoint{3.243520in}{2.272404in}}%
\pgfpathlineto{\pgfqpoint{3.225446in}{2.324303in}}%
\pgfpathlineto{\pgfqpoint{3.209719in}{2.420782in}}%
\pgfpathlineto{\pgfqpoint{3.186482in}{2.586337in}}%
\pgfpathlineto{\pgfqpoint{3.166764in}{2.606532in}}%
\pgfpathlineto{\pgfqpoint{3.149393in}{2.522807in}}%
\pgfpathlineto{\pgfqpoint{3.128034in}{2.373377in}}%
\pgfpathlineto{\pgfqpoint{3.109490in}{2.300429in}}%
\pgfpathlineto{\pgfqpoint{3.090242in}{2.260915in}}%
\pgfpathlineto{\pgfqpoint{3.068882in}{2.245116in}}%
\pgfpathlineto{\pgfqpoint{3.053389in}{2.252591in}}%
\pgfpathlineto{\pgfqpoint{3.031796in}{2.279531in}}%
\pgfpathlineto{\pgfqpoint{3.013954in}{2.340943in}}%
\pgfpathlineto{\pgfqpoint{2.994943in}{2.462763in}}%
\pgfpathlineto{\pgfqpoint{2.973113in}{2.593357in}}%
\pgfpathlineto{\pgfqpoint{2.955743in}{2.612022in}}%
\pgfpathlineto{\pgfqpoint{2.936495in}{2.510480in}}%
\pgfpathlineto{\pgfqpoint{2.918656in}{2.428365in}}%
\pgfpathlineto{\pgfqpoint{2.899877in}{2.319458in}}%
\pgfpathlineto{\pgfqpoint{2.881569in}{2.268688in}}%
\pgfpathlineto{\pgfqpoint{2.852696in}{2.246165in}}%
\pgfpathlineto{\pgfqpoint{2.840960in}{2.246189in}}%
\pgfpathlineto{\pgfqpoint{2.821946in}{2.263897in}}%
\pgfpathlineto{\pgfqpoint{2.803638in}{2.295590in}}%
\pgfpathlineto{\pgfqpoint{2.783217in}{2.377317in}}%
\pgfpathlineto{\pgfqpoint{2.764203in}{2.478273in}}%
\pgfpathlineto{\pgfqpoint{2.744251in}{2.596819in}}%
\pgfpathlineto{\pgfqpoint{2.726646in}{2.603192in}}%
\pgfpathlineto{\pgfqpoint{2.707634in}{2.492594in}}%
\pgfpathlineto{\pgfqpoint{2.686039in}{2.360021in}}%
\pgfpathlineto{\pgfqpoint{2.667729in}{2.294254in}}%
\pgfpathlineto{\pgfqpoint{2.649186in}{2.263066in}}%
\pgfpathlineto{\pgfqpoint{2.627122in}{2.247154in}}%
\pgfpathlineto{\pgfqpoint{2.606700in}{2.248329in}}%
\pgfpathlineto{\pgfqpoint{2.588392in}{2.265819in}}%
\pgfpathlineto{\pgfqpoint{2.569613in}{2.305512in}}%
\pgfpathlineto{\pgfqpoint{2.553417in}{2.384968in}}%
\pgfpathlineto{\pgfqpoint{2.535107in}{2.528051in}}%
\pgfpathlineto{\pgfqpoint{2.514216in}{2.608388in}}%
\pgfpathlineto{\pgfqpoint{2.493091in}{2.561811in}}%
\pgfpathlineto{\pgfqpoint{2.455533in}{2.347080in}}%
\pgfpathlineto{\pgfqpoint{2.438399in}{2.293309in}}%
\pgfpathlineto{\pgfqpoint{2.415866in}{2.258223in}}%
\pgfpathlineto{\pgfqpoint{2.402250in}{2.246086in}}%
\pgfpathlineto{\pgfqpoint{2.380656in}{2.247370in}}%
\pgfpathlineto{\pgfqpoint{2.361172in}{2.259593in}}%
\pgfpathlineto{\pgfqpoint{2.342395in}{2.292608in}}%
\pgfpathlineto{\pgfqpoint{2.317748in}{2.400265in}}%
\pgfpathlineto{\pgfqpoint{2.283713in}{2.591012in}}%
\pgfpathlineto{\pgfqpoint{2.266342in}{2.613826in}}%
\pgfpathlineto{\pgfqpoint{2.246389in}{2.540195in}}%
\pgfpathlineto{\pgfqpoint{2.224561in}{2.402451in}}%
\pgfpathlineto{\pgfqpoint{2.206486in}{2.349860in}}%
\pgfpathlineto{\pgfqpoint{2.188178in}{2.606926in}}%
\pgfpathlineto{\pgfqpoint{2.169633in}{2.500188in}}%
\pgfpathlineto{\pgfqpoint{2.146396in}{2.344306in}}%
\pgfpathlineto{\pgfqpoint{2.127852in}{2.285686in}}%
\pgfpathlineto{\pgfqpoint{2.112125in}{2.263544in}}%
\pgfpathlineto{\pgfqpoint{2.091000in}{2.247214in}}%
\pgfpathlineto{\pgfqpoint{2.071986in}{2.246113in}}%
\pgfpathlineto{\pgfqpoint{2.054616in}{2.257889in}}%
\pgfpathlineto{\pgfqpoint{2.035839in}{2.286722in}}%
\pgfpathlineto{\pgfqpoint{2.015652in}{2.350725in}}%
\pgfpathlineto{\pgfqpoint{1.995464in}{2.509824in}}%
\pgfpathlineto{\pgfqpoint{1.974574in}{2.609077in}}%
\pgfpathlineto{\pgfqpoint{1.956969in}{2.561024in}}%
\pgfpathlineto{\pgfqpoint{1.938895in}{2.617157in}}%
\pgfpathlineto{\pgfqpoint{1.917065in}{2.579494in}}%
\pgfpathlineto{\pgfqpoint{1.902512in}{2.477373in}}%
\pgfpathlineto{\pgfqpoint{1.879273in}{2.333494in}}%
\pgfpathlineto{\pgfqpoint{1.862373in}{2.279779in}}%
\pgfpathlineto{\pgfqpoint{1.843360in}{2.254387in}}%
\pgfpathlineto{\pgfqpoint{1.821061in}{2.246610in}}%
\pgfpathlineto{\pgfqpoint{1.802517in}{2.256597in}}%
\pgfpathlineto{\pgfqpoint{1.784443in}{2.283013in}}%
\pgfpathlineto{\pgfqpoint{1.764256in}{2.315532in}}%
\pgfpathlineto{\pgfqpoint{1.745713in}{2.401205in}}%
\pgfpathlineto{\pgfqpoint{1.724821in}{2.487132in}}%
\pgfpathlineto{\pgfqpoint{1.708625in}{2.608822in}}%
\pgfpathlineto{\pgfqpoint{1.688674in}{2.612158in}}%
\pgfpathlineto{\pgfqpoint{1.669660in}{2.531252in}}%
\pgfpathlineto{\pgfqpoint{1.650413in}{2.402284in}}%
\pgfpathlineto{\pgfqpoint{1.628817in}{2.322263in}}%
\pgfpathlineto{\pgfqpoint{1.612152in}{2.279818in}}%
\pgfpathlineto{\pgfqpoint{1.591964in}{2.253209in}}%
\pgfpathlineto{\pgfqpoint{1.575065in}{2.246813in}}%
\pgfpathlineto{\pgfqpoint{1.553469in}{2.261034in}}%
\pgfpathlineto{\pgfqpoint{1.533753in}{2.282644in}}%
\pgfpathlineto{\pgfqpoint{1.515677in}{2.329676in}}%
\pgfpathlineto{\pgfqpoint{1.497369in}{2.403599in}}%
\pgfpathlineto{\pgfqpoint{1.477652in}{2.527745in}}%
\pgfpathlineto{\pgfqpoint{1.458405in}{2.618044in}}%
\pgfpathlineto{\pgfqpoint{1.437043in}{2.628091in}}%
\pgfpathlineto{\pgfqpoint{1.419673in}{2.572635in}}%
\pgfpathlineto{\pgfqpoint{1.399488in}{2.436858in}}%
\pgfpathlineto{\pgfqpoint{1.380474in}{2.349280in}}%
\pgfpathlineto{\pgfqpoint{1.362635in}{2.298192in}}%
\pgfpathlineto{\pgfqpoint{1.341743in}{2.271049in}}%
\pgfpathlineto{\pgfqpoint{1.323669in}{2.257101in}}%
\pgfpathlineto{\pgfqpoint{1.302544in}{2.248465in}}%
\pgfpathlineto{\pgfqpoint{1.285642in}{2.257236in}}%
\pgfpathlineto{\pgfqpoint{1.264752in}{2.289177in}}%
\pgfpathlineto{\pgfqpoint{1.244330in}{2.348826in}}%
\pgfpathlineto{\pgfqpoint{1.190344in}{2.611793in}}%
\pgfpathlineto{\pgfqpoint{1.168043in}{2.639998in}}%
\pgfpathlineto{\pgfqpoint{1.148326in}{2.612238in}}%
\pgfpathlineto{\pgfqpoint{1.129079in}{2.511838in}}%
\pgfpathlineto{\pgfqpoint{1.109362in}{2.429264in}}%
\pgfpathlineto{\pgfqpoint{1.093166in}{2.345847in}}%
\pgfpathlineto{\pgfqpoint{1.073447in}{2.295079in}}%
\pgfpathlineto{\pgfqpoint{1.053025in}{2.265585in}}%
\pgfpathlineto{\pgfqpoint{1.035186in}{2.251037in}}%
\pgfpathlineto{\pgfqpoint{1.017113in}{2.312565in}}%
\pgfpathlineto{\pgfqpoint{0.993640in}{2.270980in}}%
\pgfpathlineto{\pgfqpoint{0.978148in}{2.254700in}}%
\pgfpathlineto{\pgfqpoint{0.957727in}{2.254216in}}%
\pgfpathlineto{\pgfqpoint{0.937071in}{2.274084in}}%
\pgfpathlineto{\pgfqpoint{0.920403in}{2.307293in}}%
\pgfpathlineto{\pgfqpoint{0.899982in}{2.392954in}}%
\pgfpathlineto{\pgfqpoint{0.880970in}{2.512718in}}%
\pgfpathlineto{\pgfqpoint{0.860314in}{2.618028in}}%
\pgfpathlineto{\pgfqpoint{0.843413in}{2.655572in}}%
\pgfpathlineto{\pgfqpoint{0.825573in}{2.623857in}}%
\pgfpathlineto{\pgfqpoint{0.804917in}{2.537771in}}%
\pgfpathlineto{\pgfqpoint{0.786844in}{2.427557in}}%
\pgfpathlineto{\pgfqpoint{0.764779in}{2.347835in}}%
\pgfpathlineto{\pgfqpoint{0.746469in}{2.300497in}}%
\pgfpathlineto{\pgfqpoint{0.726518in}{2.271279in}}%
\pgfpathlineto{\pgfqpoint{0.708679in}{2.254234in}}%
\pgfpathlineto{\pgfqpoint{0.690603in}{2.258893in}}%
\pgfpathlineto{\pgfqpoint{0.669244in}{2.288571in}}%
\pgfpathlineto{\pgfqpoint{0.651405in}{2.326860in}}%
\pgfpathlineto{\pgfqpoint{0.651639in}{2.324712in}}%
\pgfpathlineto{\pgfqpoint{0.658447in}{2.298425in}}%
\pgfpathlineto{\pgfqpoint{0.673704in}{2.268129in}}%
\pgfpathlineto{\pgfqpoint{0.694829in}{2.255064in}}%
\pgfpathlineto{\pgfqpoint{0.713373in}{2.281890in}}%
\pgfpathlineto{\pgfqpoint{0.734029in}{2.352140in}}%
\pgfpathlineto{\pgfqpoint{0.776516in}{2.649386in}}%
\pgfpathlineto{\pgfqpoint{0.791538in}{2.641175in}}%
\pgfpathlineto{\pgfqpoint{0.809143in}{2.518628in}}%
\pgfpathlineto{\pgfqpoint{0.829799in}{2.355201in}}%
\pgfpathlineto{\pgfqpoint{0.847638in}{2.286682in}}%
\pgfpathlineto{\pgfqpoint{0.865712in}{2.254873in}}%
\pgfpathlineto{\pgfqpoint{0.888011in}{2.261063in}}%
\pgfpathlineto{\pgfqpoint{0.906790in}{2.293405in}}%
\pgfpathlineto{\pgfqpoint{0.923691in}{2.368368in}}%
\pgfpathlineto{\pgfqpoint{0.944111in}{2.521064in}}%
\pgfpathlineto{\pgfqpoint{0.965941in}{2.643769in}}%
\pgfpathlineto{\pgfqpoint{0.981200in}{2.611431in}}%
\pgfpathlineto{\pgfqpoint{1.001856in}{2.451561in}}%
\pgfpathlineto{\pgfqpoint{1.021572in}{2.315952in}}%
\pgfpathlineto{\pgfqpoint{1.039882in}{2.268422in}}%
\pgfpathlineto{\pgfqpoint{1.061242in}{2.249216in}}%
\pgfpathlineto{\pgfqpoint{1.081195in}{2.269915in}}%
\pgfpathlineto{\pgfqpoint{1.099503in}{2.316635in}}%
\pgfpathlineto{\pgfqpoint{1.116402in}{2.412318in}}%
\pgfpathlineto{\pgfqpoint{1.137295in}{2.584520in}}%
\pgfpathlineto{\pgfqpoint{1.157717in}{2.633081in}}%
\pgfpathlineto{\pgfqpoint{1.175319in}{2.547703in}}%
\pgfpathlineto{\pgfqpoint{1.195038in}{2.384104in}}%
\pgfpathlineto{\pgfqpoint{1.214754in}{2.298924in}}%
\pgfpathlineto{\pgfqpoint{1.231422in}{2.259216in}}%
\pgfpathlineto{\pgfqpoint{1.252078in}{2.247819in}}%
\pgfpathlineto{\pgfqpoint{1.270620in}{2.266931in}}%
\pgfpathlineto{\pgfqpoint{1.289868in}{2.314754in}}%
\pgfpathlineto{\pgfqpoint{1.311464in}{2.415135in}}%
\pgfpathlineto{\pgfqpoint{1.329303in}{2.575984in}}%
\pgfpathlineto{\pgfqpoint{1.349959in}{2.628202in}}%
\pgfpathlineto{\pgfqpoint{1.366624in}{2.540276in}}%
\pgfpathlineto{\pgfqpoint{1.384934in}{2.413793in}}%
\pgfpathlineto{\pgfqpoint{1.406764in}{2.310397in}}%
\pgfpathlineto{\pgfqpoint{1.426012in}{2.264702in}}%
\pgfpathlineto{\pgfqpoint{1.442912in}{2.247495in}}%
\pgfpathlineto{\pgfqpoint{1.462394in}{2.251465in}}%
\pgfpathlineto{\pgfqpoint{1.483755in}{2.280876in}}%
\pgfpathlineto{\pgfqpoint{1.502298in}{2.296379in}}%
\pgfpathlineto{\pgfqpoint{1.519433in}{2.349065in}}%
\pgfpathlineto{\pgfqpoint{1.558868in}{2.613370in}}%
\pgfpathlineto{\pgfqpoint{1.580933in}{2.592403in}}%
\pgfpathlineto{\pgfqpoint{1.598067in}{2.453413in}}%
\pgfpathlineto{\pgfqpoint{1.615437in}{2.336182in}}%
\pgfpathlineto{\pgfqpoint{1.636094in}{2.273644in}}%
\pgfpathlineto{\pgfqpoint{1.654404in}{2.249449in}}%
\pgfpathlineto{\pgfqpoint{1.675060in}{2.249762in}}%
\pgfpathlineto{\pgfqpoint{1.693133in}{2.271312in}}%
\pgfpathlineto{\pgfqpoint{1.714258in}{2.325074in}}%
\pgfpathlineto{\pgfqpoint{1.732098in}{2.412280in}}%
\pgfpathlineto{\pgfqpoint{1.752754in}{2.580978in}}%
\pgfpathlineto{\pgfqpoint{1.773175in}{2.617616in}}%
\pgfpathlineto{\pgfqpoint{1.790780in}{2.540573in}}%
\pgfpathlineto{\pgfqpoint{1.808619in}{2.401372in}}%
\pgfpathlineto{\pgfqpoint{1.827633in}{2.337153in}}%
\pgfpathlineto{\pgfqpoint{1.849463in}{2.273703in}}%
\pgfpathlineto{\pgfqpoint{1.867537in}{2.253613in}}%
\pgfpathlineto{\pgfqpoint{1.886081in}{2.245292in}}%
\pgfpathlineto{\pgfqpoint{1.906737in}{2.263320in}}%
\pgfpathlineto{\pgfqpoint{1.923402in}{2.295045in}}%
\pgfpathlineto{\pgfqpoint{1.944998in}{2.384149in}}%
\pgfpathlineto{\pgfqpoint{1.962603in}{2.462133in}}%
\pgfpathlineto{\pgfqpoint{1.983493in}{2.578096in}}%
\pgfpathlineto{\pgfqpoint{2.001567in}{2.614933in}}%
\pgfpathlineto{\pgfqpoint{2.020111in}{2.578691in}}%
\pgfpathlineto{\pgfqpoint{2.038654in}{2.439904in}}%
\pgfpathlineto{\pgfqpoint{2.059546in}{2.346844in}}%
\pgfpathlineto{\pgfqpoint{2.080906in}{2.282075in}}%
\pgfpathlineto{\pgfqpoint{2.095225in}{2.257067in}}%
\pgfpathlineto{\pgfqpoint{2.117524in}{2.245429in}}%
\pgfpathlineto{\pgfqpoint{2.136068in}{2.255948in}}%
\pgfpathlineto{\pgfqpoint{2.156019in}{2.281504in}}%
\pgfpathlineto{\pgfqpoint{2.173624in}{2.335448in}}%
\pgfpathlineto{\pgfqpoint{2.193577in}{2.426065in}}%
\pgfpathlineto{\pgfqpoint{2.214233in}{2.573121in}}%
\pgfpathlineto{\pgfqpoint{2.232541in}{2.603956in}}%
\pgfpathlineto{\pgfqpoint{2.252023in}{2.598973in}}%
\pgfpathlineto{\pgfqpoint{2.270802in}{2.492131in}}%
\pgfpathlineto{\pgfqpoint{2.290989in}{2.341388in}}%
\pgfpathlineto{\pgfqpoint{2.306012in}{2.287067in}}%
\pgfpathlineto{\pgfqpoint{2.327371in}{2.261348in}}%
\pgfpathlineto{\pgfqpoint{2.348027in}{2.247596in}}%
\pgfpathlineto{\pgfqpoint{2.370328in}{2.247882in}}%
\pgfpathlineto{\pgfqpoint{2.384880in}{2.261097in}}%
\pgfpathlineto{\pgfqpoint{2.406241in}{2.300579in}}%
\pgfpathlineto{\pgfqpoint{2.423377in}{2.371787in}}%
\pgfpathlineto{\pgfqpoint{2.444502in}{2.526212in}}%
\pgfpathlineto{\pgfqpoint{2.461872in}{2.595982in}}%
\pgfpathlineto{\pgfqpoint{2.480649in}{2.608092in}}%
\pgfpathlineto{\pgfqpoint{2.500133in}{2.545281in}}%
\pgfpathlineto{\pgfqpoint{2.520555in}{2.401075in}}%
\pgfpathlineto{\pgfqpoint{2.540740in}{2.301259in}}%
\pgfpathlineto{\pgfqpoint{2.558580in}{2.267124in}}%
\pgfpathlineto{\pgfqpoint{2.598249in}{2.246138in}}%
\pgfpathlineto{\pgfqpoint{2.615854in}{2.250734in}}%
\pgfpathlineto{\pgfqpoint{2.634867in}{2.268226in}}%
\pgfpathlineto{\pgfqpoint{2.655289in}{2.318164in}}%
\pgfpathlineto{\pgfqpoint{2.673128in}{2.404826in}}%
\pgfpathlineto{\pgfqpoint{2.691438in}{2.309411in}}%
\pgfpathlineto{\pgfqpoint{2.711389in}{2.406155in}}%
\pgfpathlineto{\pgfqpoint{2.730871in}{2.547406in}}%
\pgfpathlineto{\pgfqpoint{2.751293in}{2.612911in}}%
\pgfpathlineto{\pgfqpoint{2.767960in}{2.583768in}}%
\pgfpathlineto{\pgfqpoint{2.787911in}{2.500853in}}%
\pgfpathlineto{\pgfqpoint{2.808801in}{2.357054in}}%
\pgfpathlineto{\pgfqpoint{2.830397in}{2.297812in}}%
\pgfpathlineto{\pgfqpoint{2.848471in}{2.262928in}}%
\pgfpathlineto{\pgfqpoint{2.865841in}{2.247504in}}%
\pgfpathlineto{\pgfqpoint{2.884386in}{2.248929in}}%
\pgfpathlineto{\pgfqpoint{2.905276in}{2.272329in}}%
\pgfpathlineto{\pgfqpoint{2.923819in}{2.313690in}}%
\pgfpathlineto{\pgfqpoint{2.944240in}{2.407589in}}%
\pgfpathlineto{\pgfqpoint{2.963254in}{2.528591in}}%
\pgfpathlineto{\pgfqpoint{2.980624in}{2.608575in}}%
\pgfpathlineto{\pgfqpoint{3.001280in}{2.608770in}}%
\pgfpathlineto{\pgfqpoint{3.018180in}{2.564630in}}%
\pgfpathlineto{\pgfqpoint{3.039776in}{2.428803in}}%
\pgfpathlineto{\pgfqpoint{3.058320in}{2.322825in}}%
\pgfpathlineto{\pgfqpoint{3.077097in}{2.279851in}}%
\pgfpathlineto{\pgfqpoint{3.095172in}{2.255659in}}%
\pgfpathlineto{\pgfqpoint{3.116297in}{2.245994in}}%
\pgfpathlineto{\pgfqpoint{3.136719in}{2.256560in}}%
\pgfpathlineto{\pgfqpoint{3.155732in}{2.284333in}}%
\pgfpathlineto{\pgfqpoint{3.174746in}{2.332299in}}%
\pgfpathlineto{\pgfqpoint{3.194697in}{2.447865in}}%
\pgfpathlineto{\pgfqpoint{3.212770in}{2.522298in}}%
\pgfpathlineto{\pgfqpoint{3.230611in}{2.613956in}}%
\pgfpathlineto{\pgfqpoint{3.251971in}{2.608656in}}%
\pgfpathlineto{\pgfqpoint{3.269810in}{2.529612in}}%
\pgfpathlineto{\pgfqpoint{3.288354in}{2.398294in}}%
\pgfpathlineto{\pgfqpoint{3.309948in}{2.316824in}}%
\pgfpathlineto{\pgfqpoint{3.326850in}{2.273432in}}%
\pgfpathlineto{\pgfqpoint{3.346098in}{2.252186in}}%
\pgfpathlineto{\pgfqpoint{3.367928in}{2.247945in}}%
\pgfpathlineto{\pgfqpoint{3.404546in}{2.279559in}}%
\pgfpathlineto{\pgfqpoint{3.422385in}{2.314107in}}%
\pgfpathlineto{\pgfqpoint{3.442101in}{2.388666in}}%
\pgfpathlineto{\pgfqpoint{3.463697in}{2.478753in}}%
\pgfpathlineto{\pgfqpoint{3.479188in}{2.579980in}}%
\pgfpathlineto{\pgfqpoint{3.501019in}{2.631115in}}%
\pgfpathlineto{\pgfqpoint{3.522380in}{2.603179in}}%
\pgfpathlineto{\pgfqpoint{3.521909in}{2.547600in}}%
\pgfpathlineto{\pgfqpoint{3.537637in}{2.515570in}}%
\pgfpathlineto{\pgfqpoint{3.558762in}{2.381459in}}%
\pgfpathlineto{\pgfqpoint{3.578949in}{2.323318in}}%
\pgfpathlineto{\pgfqpoint{3.597728in}{2.278203in}}%
\pgfpathlineto{\pgfqpoint{3.612984in}{2.258779in}}%
\pgfpathlineto{\pgfqpoint{3.635518in}{2.247731in}}%
\pgfpathlineto{\pgfqpoint{3.651480in}{2.255088in}}%
\pgfpathlineto{\pgfqpoint{3.674015in}{2.270503in}}%
\pgfpathlineto{\pgfqpoint{3.693263in}{2.311595in}}%
\pgfpathlineto{\pgfqpoint{3.712511in}{2.355864in}}%
\pgfpathlineto{\pgfqpoint{3.732932in}{2.454131in}}%
\pgfpathlineto{\pgfqpoint{3.751475in}{2.529039in}}%
\pgfpathlineto{\pgfqpoint{3.769785in}{2.623853in}}%
\pgfpathlineto{\pgfqpoint{3.786684in}{2.639035in}}%
\pgfpathlineto{\pgfqpoint{3.809218in}{2.614695in}}%
\pgfpathlineto{\pgfqpoint{3.827057in}{2.499797in}}%
\pgfpathlineto{\pgfqpoint{3.847479in}{2.379900in}}%
\pgfpathlineto{\pgfqpoint{3.866258in}{2.322300in}}%
\pgfpathlineto{\pgfqpoint{3.883862in}{2.308874in}}%
\pgfpathlineto{\pgfqpoint{3.902641in}{2.332377in}}%
\pgfpathlineto{\pgfqpoint{3.922123in}{2.279146in}}%
\pgfpathlineto{\pgfqpoint{3.940902in}{2.257111in}}%
\pgfpathlineto{\pgfqpoint{3.962967in}{2.250360in}}%
\pgfpathlineto{\pgfqpoint{3.986440in}{2.273081in}}%
\pgfpathlineto{\pgfqpoint{3.997706in}{2.295099in}}%
\pgfpathlineto{\pgfqpoint{4.019536in}{2.347193in}}%
\pgfpathlineto{\pgfqpoint{4.039489in}{2.430652in}}%
\pgfpathlineto{\pgfqpoint{4.057562in}{2.560223in}}%
\pgfpathlineto{\pgfqpoint{4.076341in}{2.646089in}}%
\pgfpathlineto{\pgfqpoint{4.095589in}{2.650367in}}%
\pgfpathlineto{\pgfqpoint{4.117653in}{2.583815in}}%
\pgfpathlineto{\pgfqpoint{4.134787in}{2.458858in}}%
\pgfpathlineto{\pgfqpoint{4.154975in}{2.372901in}}%
\pgfpathlineto{\pgfqpoint{4.173048in}{2.332113in}}%
\pgfpathlineto{\pgfqpoint{4.192062in}{2.285802in}}%
\pgfpathlineto{\pgfqpoint{4.210840in}{2.260200in}}%
\pgfpathlineto{\pgfqpoint{4.234079in}{2.251754in}}%
\pgfpathlineto{\pgfqpoint{4.249336in}{2.260534in}}%
\pgfpathlineto{\pgfqpoint{4.267644in}{2.284979in}}%
\pgfpathlineto{\pgfqpoint{4.288536in}{2.323368in}}%
\pgfpathlineto{\pgfqpoint{4.308722in}{2.385292in}}%
\pgfpathlineto{\pgfqpoint{4.327501in}{2.494525in}}%
\pgfpathlineto{\pgfqpoint{4.345574in}{2.538544in}}%
\pgfpathlineto{\pgfqpoint{4.364822in}{2.643772in}}%
\pgfpathlineto{\pgfqpoint{4.383132in}{2.665785in}}%
\pgfpathlineto{\pgfqpoint{4.402614in}{2.627843in}}%
\pgfpathlineto{\pgfqpoint{4.420453in}{2.529293in}}%
\pgfpathlineto{\pgfqpoint{4.439467in}{2.405993in}}%
\pgfpathlineto{\pgfqpoint{4.463174in}{2.314436in}}%
\pgfpathlineto{\pgfqpoint{4.480779in}{2.276957in}}%
\pgfpathlineto{\pgfqpoint{4.474207in}{2.291836in}}%
\pgfpathlineto{\pgfqpoint{4.453551in}{2.383294in}}%
\pgfpathlineto{\pgfqpoint{4.435007in}{2.540977in}}%
\pgfpathlineto{\pgfqpoint{4.417168in}{2.655563in}}%
\pgfpathlineto{\pgfqpoint{4.398623in}{2.643812in}}%
\pgfpathlineto{\pgfqpoint{4.377498in}{2.463985in}}%
\pgfpathlineto{\pgfqpoint{4.358016in}{2.341115in}}%
\pgfpathlineto{\pgfqpoint{4.336420in}{2.274359in}}%
\pgfpathlineto{\pgfqpoint{4.336420in}{2.258480in}}%
\pgfpathlineto{\pgfqpoint{4.322572in}{2.253054in}}%
\pgfpathlineto{\pgfqpoint{4.300742in}{2.268810in}}%
\pgfpathlineto{\pgfqpoint{4.280789in}{2.316540in}}%
\pgfpathlineto{\pgfqpoint{4.262012in}{2.451928in}}%
\pgfpathlineto{\pgfqpoint{4.242294in}{2.608566in}}%
\pgfpathlineto{\pgfqpoint{4.223985in}{2.653982in}}%
\pgfpathlineto{\pgfqpoint{4.204738in}{2.556969in}}%
\pgfpathlineto{\pgfqpoint{4.186193in}{2.394791in}}%
\pgfpathlineto{\pgfqpoint{4.167415in}{2.302610in}}%
\pgfpathlineto{\pgfqpoint{4.148638in}{2.261909in}}%
\pgfpathlineto{\pgfqpoint{4.130093in}{2.251161in}}%
\pgfpathlineto{\pgfqpoint{4.106620in}{2.286219in}}%
\pgfpathlineto{\pgfqpoint{4.091598in}{2.346536in}}%
\pgfpathlineto{\pgfqpoint{4.070473in}{2.499261in}}%
\pgfpathlineto{\pgfqpoint{4.051694in}{2.616585in}}%
\pgfpathlineto{\pgfqpoint{4.034324in}{2.447840in}}%
\pgfpathlineto{\pgfqpoint{4.012025in}{2.314028in}}%
\pgfpathlineto{\pgfqpoint{3.994185in}{2.265999in}}%
\pgfpathlineto{\pgfqpoint{3.975875in}{2.248514in}}%
\pgfpathlineto{\pgfqpoint{3.955219in}{2.263339in}}%
\pgfpathlineto{\pgfqpoint{3.936911in}{2.309763in}}%
\pgfpathlineto{\pgfqpoint{3.917898in}{2.413367in}}%
\pgfpathlineto{\pgfqpoint{3.895833in}{2.597136in}}%
\pgfpathlineto{\pgfqpoint{3.877760in}{2.566888in}}%
\pgfpathlineto{\pgfqpoint{3.859215in}{2.635400in}}%
\pgfpathlineto{\pgfqpoint{3.839733in}{2.556019in}}%
\pgfpathlineto{\pgfqpoint{3.822597in}{2.399436in}}%
\pgfpathlineto{\pgfqpoint{3.799361in}{2.299239in}}%
\pgfpathlineto{\pgfqpoint{3.783633in}{2.261683in}}%
\pgfpathlineto{\pgfqpoint{3.762742in}{2.247272in}}%
\pgfpathlineto{\pgfqpoint{3.743024in}{2.265171in}}%
\pgfpathlineto{\pgfqpoint{3.724716in}{2.316243in}}%
\pgfpathlineto{\pgfqpoint{3.705937in}{2.418420in}}%
\pgfpathlineto{\pgfqpoint{3.685750in}{2.566483in}}%
\pgfpathlineto{\pgfqpoint{3.667442in}{2.628920in}}%
\pgfpathlineto{\pgfqpoint{3.648663in}{2.550101in}}%
\pgfpathlineto{\pgfqpoint{3.630589in}{2.409052in}}%
\pgfpathlineto{\pgfqpoint{3.606882in}{2.294972in}}%
\pgfpathlineto{\pgfqpoint{3.590920in}{2.262596in}}%
\pgfpathlineto{\pgfqpoint{3.571907in}{2.247685in}}%
\pgfpathlineto{\pgfqpoint{3.549842in}{2.253741in}}%
\pgfpathlineto{\pgfqpoint{3.528012in}{2.290515in}}%
\pgfpathlineto{\pgfqpoint{3.512990in}{2.344708in}}%
\pgfpathlineto{\pgfqpoint{3.496793in}{2.429087in}}%
\pgfpathlineto{\pgfqpoint{3.475434in}{2.575357in}}%
\pgfpathlineto{\pgfqpoint{3.456655in}{2.620990in}}%
\pgfpathlineto{\pgfqpoint{3.438816in}{2.571390in}}%
\pgfpathlineto{\pgfqpoint{3.415812in}{2.396268in}}%
\pgfpathlineto{\pgfqpoint{3.397503in}{2.316399in}}%
\pgfpathlineto{\pgfqpoint{3.377082in}{2.264950in}}%
\pgfpathlineto{\pgfqpoint{3.359008in}{2.247345in}}%
\pgfpathlineto{\pgfqpoint{3.341638in}{2.249581in}}%
\pgfpathlineto{\pgfqpoint{3.322390in}{2.271872in}}%
\pgfpathlineto{\pgfqpoint{3.301734in}{2.327880in}}%
\pgfpathlineto{\pgfqpoint{3.282015in}{2.387159in}}%
\pgfpathlineto{\pgfqpoint{3.263239in}{2.534684in}}%
\pgfpathlineto{\pgfqpoint{3.241174in}{2.611419in}}%
\pgfpathlineto{\pgfqpoint{3.226386in}{2.601616in}}%
\pgfpathlineto{\pgfqpoint{3.186246in}{2.344460in}}%
\pgfpathlineto{\pgfqpoint{3.170520in}{2.286806in}}%
\pgfpathlineto{\pgfqpoint{3.146342in}{2.255042in}}%
\pgfpathlineto{\pgfqpoint{3.128503in}{2.245711in}}%
\pgfpathlineto{\pgfqpoint{3.112306in}{2.250331in}}%
\pgfpathlineto{\pgfqpoint{3.090713in}{2.276237in}}%
\pgfpathlineto{\pgfqpoint{3.068882in}{2.342178in}}%
\pgfpathlineto{\pgfqpoint{3.053624in}{2.418029in}}%
\pgfpathlineto{\pgfqpoint{3.032264in}{2.579090in}}%
\pgfpathlineto{\pgfqpoint{3.031090in}{2.596405in}}%
\pgfpathlineto{\pgfqpoint{3.013720in}{2.612394in}}%
\pgfpathlineto{\pgfqpoint{2.994003in}{2.578736in}}%
\pgfpathlineto{\pgfqpoint{2.975930in}{2.443055in}}%
\pgfpathlineto{\pgfqpoint{2.956446in}{2.358007in}}%
\pgfpathlineto{\pgfqpoint{2.938372in}{2.291526in}}%
\pgfpathlineto{\pgfqpoint{2.918890in}{2.258507in}}%
\pgfpathlineto{\pgfqpoint{2.892366in}{2.245259in}}%
\pgfpathlineto{\pgfqpoint{2.880629in}{2.246087in}}%
\pgfpathlineto{\pgfqpoint{2.859739in}{2.264266in}}%
\pgfpathlineto{\pgfqpoint{2.841665in}{2.303127in}}%
\pgfpathlineto{\pgfqpoint{2.822652in}{2.335539in}}%
\pgfpathlineto{\pgfqpoint{2.801525in}{2.474989in}}%
\pgfpathlineto{\pgfqpoint{2.781574in}{2.596282in}}%
\pgfpathlineto{\pgfqpoint{2.764672in}{2.603613in}}%
\pgfpathlineto{\pgfqpoint{2.762795in}{2.567191in}}%
\pgfpathlineto{\pgfqpoint{2.745425in}{2.520452in}}%
\pgfpathlineto{\pgfqpoint{2.725474in}{2.402794in}}%
\pgfpathlineto{\pgfqpoint{2.705521in}{2.304730in}}%
\pgfpathlineto{\pgfqpoint{2.685570in}{2.367635in}}%
\pgfpathlineto{\pgfqpoint{2.667965in}{2.509385in}}%
\pgfpathlineto{\pgfqpoint{2.648717in}{2.613383in}}%
\pgfpathlineto{\pgfqpoint{2.629938in}{2.574621in}}%
\pgfpathlineto{\pgfqpoint{2.608812in}{2.420560in}}%
\pgfpathlineto{\pgfqpoint{2.590738in}{2.323814in}}%
\pgfpathlineto{\pgfqpoint{2.574307in}{2.276298in}}%
\pgfpathlineto{\pgfqpoint{2.552948in}{2.247774in}}%
\pgfpathlineto{\pgfqpoint{2.533934in}{2.245474in}}%
\pgfpathlineto{\pgfqpoint{2.512808in}{2.265993in}}%
\pgfpathlineto{\pgfqpoint{2.493560in}{2.287432in}}%
\pgfpathlineto{\pgfqpoint{2.493794in}{2.330470in}}%
\pgfpathlineto{\pgfqpoint{2.475721in}{2.363845in}}%
\pgfpathlineto{\pgfqpoint{2.456707in}{2.502656in}}%
\pgfpathlineto{\pgfqpoint{2.438868in}{2.572978in}}%
\pgfpathlineto{\pgfqpoint{2.419621in}{2.612835in}}%
\pgfpathlineto{\pgfqpoint{2.398496in}{2.505697in}}%
\pgfpathlineto{\pgfqpoint{2.379248in}{2.375260in}}%
\pgfpathlineto{\pgfqpoint{2.361409in}{2.316968in}}%
\pgfpathlineto{\pgfqpoint{2.343567in}{2.270113in}}%
\pgfpathlineto{\pgfqpoint{2.320800in}{2.246855in}}%
\pgfpathlineto{\pgfqpoint{2.302492in}{2.248742in}}%
\pgfpathlineto{\pgfqpoint{2.285356in}{2.265273in}}%
\pgfpathlineto{\pgfqpoint{2.264699in}{2.317290in}}%
\pgfpathlineto{\pgfqpoint{2.245921in}{2.424873in}}%
\pgfpathlineto{\pgfqpoint{2.224561in}{2.563803in}}%
\pgfpathlineto{\pgfqpoint{2.225264in}{2.607197in}}%
\pgfpathlineto{\pgfqpoint{2.205782in}{2.615708in}}%
\pgfpathlineto{\pgfqpoint{2.188412in}{2.548566in}}%
\pgfpathlineto{\pgfqpoint{2.169633in}{2.421953in}}%
\pgfpathlineto{\pgfqpoint{2.147805in}{2.328282in}}%
\pgfpathlineto{\pgfqpoint{2.129026in}{2.277659in}}%
\pgfpathlineto{\pgfqpoint{2.110952in}{2.252854in}}%
\pgfpathlineto{\pgfqpoint{2.094051in}{2.245571in}}%
\pgfpathlineto{\pgfqpoint{2.071047in}{2.260031in}}%
\pgfpathlineto{\pgfqpoint{2.052739in}{2.290679in}}%
\pgfpathlineto{\pgfqpoint{2.033491in}{2.361974in}}%
\pgfpathlineto{\pgfqpoint{2.013069in}{2.509857in}}%
\pgfpathlineto{\pgfqpoint{1.997107in}{2.601365in}}%
\pgfpathlineto{\pgfqpoint{1.971991in}{2.598474in}}%
\pgfpathlineto{\pgfqpoint{1.959552in}{2.541390in}}%
\pgfpathlineto{\pgfqpoint{1.938190in}{2.422781in}}%
\pgfpathlineto{\pgfqpoint{1.919177in}{2.330425in}}%
\pgfpathlineto{\pgfqpoint{1.898521in}{2.275199in}}%
\pgfpathlineto{\pgfqpoint{1.880213in}{2.252424in}}%
\pgfpathlineto{\pgfqpoint{1.860260in}{2.246310in}}%
\pgfpathlineto{\pgfqpoint{1.842655in}{2.253029in}}%
\pgfpathlineto{\pgfqpoint{1.821999in}{2.281458in}}%
\pgfpathlineto{\pgfqpoint{1.802517in}{2.320621in}}%
\pgfpathlineto{\pgfqpoint{1.782095in}{2.426303in}}%
\pgfpathlineto{\pgfqpoint{1.765664in}{2.553345in}}%
\pgfpathlineto{\pgfqpoint{1.744774in}{2.607828in}}%
\pgfpathlineto{\pgfqpoint{1.727874in}{2.622724in}}%
\pgfpathlineto{\pgfqpoint{1.707687in}{2.549292in}}%
\pgfpathlineto{\pgfqpoint{1.668721in}{2.334753in}}%
\pgfpathlineto{\pgfqpoint{1.645953in}{2.284552in}}%
\pgfpathlineto{\pgfqpoint{1.631634in}{2.264697in}}%
\pgfpathlineto{\pgfqpoint{1.610274in}{2.248181in}}%
\pgfpathlineto{\pgfqpoint{1.589147in}{2.253073in}}%
\pgfpathlineto{\pgfqpoint{1.573186in}{2.268048in}}%
\pgfpathlineto{\pgfqpoint{1.552529in}{2.297116in}}%
\pgfpathlineto{\pgfqpoint{1.534456in}{2.330234in}}%
\pgfpathlineto{\pgfqpoint{1.515208in}{2.268739in}}%
\pgfpathlineto{\pgfqpoint{1.498543in}{2.250209in}}%
\pgfpathlineto{\pgfqpoint{1.477182in}{2.252850in}}%
\pgfpathlineto{\pgfqpoint{1.456291in}{2.270996in}}%
\pgfpathlineto{\pgfqpoint{1.438686in}{2.309734in}}%
\pgfpathlineto{\pgfqpoint{1.418735in}{2.408625in}}%
\pgfpathlineto{\pgfqpoint{1.401834in}{2.528765in}}%
\pgfpathlineto{\pgfqpoint{1.379769in}{2.626246in}}%
\pgfpathlineto{\pgfqpoint{1.362399in}{2.624086in}}%
\pgfpathlineto{\pgfqpoint{1.322026in}{2.400596in}}%
\pgfpathlineto{\pgfqpoint{1.303952in}{2.322030in}}%
\pgfpathlineto{\pgfqpoint{1.285174in}{2.277759in}}%
\pgfpathlineto{\pgfqpoint{1.266395in}{2.266690in}}%
\pgfpathlineto{\pgfqpoint{1.246913in}{2.248787in}}%
\pgfpathlineto{\pgfqpoint{1.227899in}{2.254736in}}%
\pgfpathlineto{\pgfqpoint{1.207243in}{2.281411in}}%
\pgfpathlineto{\pgfqpoint{1.183770in}{2.351710in}}%
\pgfpathlineto{\pgfqpoint{1.168748in}{2.435217in}}%
\pgfpathlineto{\pgfqpoint{1.151612in}{2.537797in}}%
\pgfpathlineto{\pgfqpoint{1.134007in}{2.630692in}}%
\pgfpathlineto{\pgfqpoint{1.109831in}{2.626505in}}%
\pgfpathlineto{\pgfqpoint{1.093634in}{2.549273in}}%
\pgfpathlineto{\pgfqpoint{1.073682in}{2.398716in}}%
\pgfpathlineto{\pgfqpoint{1.053496in}{2.331148in}}%
\pgfpathlineto{\pgfqpoint{1.035891in}{2.293131in}}%
\pgfpathlineto{\pgfqpoint{1.015235in}{2.263588in}}%
\pgfpathlineto{\pgfqpoint{0.994345in}{2.250472in}}%
\pgfpathlineto{\pgfqpoint{0.976740in}{2.262310in}}%
\pgfpathlineto{\pgfqpoint{0.959838in}{2.278420in}}%
\pgfpathlineto{\pgfqpoint{0.941060in}{2.321170in}}%
\pgfpathlineto{\pgfqpoint{0.921109in}{2.378485in}}%
\pgfpathlineto{\pgfqpoint{0.895758in}{2.522488in}}%
\pgfpathlineto{\pgfqpoint{0.881439in}{2.611836in}}%
\pgfpathlineto{\pgfqpoint{0.860078in}{2.655071in}}%
\pgfpathlineto{\pgfqpoint{0.842004in}{2.634220in}}%
\pgfpathlineto{\pgfqpoint{0.823696in}{2.566316in}}%
\pgfpathlineto{\pgfqpoint{0.806326in}{2.448199in}}%
\pgfpathlineto{\pgfqpoint{0.784730in}{2.362350in}}%
\pgfpathlineto{\pgfqpoint{0.766891in}{2.310560in}}%
\pgfpathlineto{\pgfqpoint{0.744123in}{2.274963in}}%
\pgfpathlineto{\pgfqpoint{0.726753in}{2.255618in}}%
\pgfpathlineto{\pgfqpoint{0.709382in}{2.254089in}}%
\pgfpathlineto{\pgfqpoint{0.689195in}{2.274374in}}%
\pgfpathlineto{\pgfqpoint{0.670418in}{2.308889in}}%
\pgfpathlineto{\pgfqpoint{0.651170in}{2.350941in}}%
\pgfpathlineto{\pgfqpoint{0.651170in}{2.346431in}}%
\pgfpathlineto{\pgfqpoint{0.655865in}{2.336113in}}%
\pgfpathlineto{\pgfqpoint{0.675347in}{2.272616in}}%
\pgfpathlineto{\pgfqpoint{0.696942in}{2.252745in}}%
\pgfpathlineto{\pgfqpoint{0.714782in}{2.274218in}}%
\pgfpathlineto{\pgfqpoint{0.732621in}{2.322041in}}%
\pgfpathlineto{\pgfqpoint{0.753511in}{2.442606in}}%
\pgfpathlineto{\pgfqpoint{0.770647in}{2.602124in}}%
\pgfpathlineto{\pgfqpoint{0.789895in}{2.655273in}}%
\pgfpathlineto{\pgfqpoint{0.810317in}{2.553045in}}%
\pgfpathlineto{\pgfqpoint{0.828156in}{2.400417in}}%
\pgfpathlineto{\pgfqpoint{0.848578in}{2.294785in}}%
\pgfpathlineto{\pgfqpoint{0.866417in}{2.259478in}}%
\pgfpathlineto{\pgfqpoint{0.887776in}{2.255210in}}%
\pgfpathlineto{\pgfqpoint{0.904912in}{2.282232in}}%
\pgfpathlineto{\pgfqpoint{0.930028in}{2.450810in}}%
\pgfpathlineto{\pgfqpoint{0.947633in}{2.613646in}}%
\pgfpathlineto{\pgfqpoint{0.962655in}{2.644280in}}%
\pgfpathlineto{\pgfqpoint{0.983311in}{2.529135in}}%
\pgfpathlineto{\pgfqpoint{1.004202in}{2.360400in}}%
\pgfpathlineto{\pgfqpoint{1.022746in}{2.286643in}}%
\pgfpathlineto{\pgfqpoint{1.040820in}{2.255987in}}%
\pgfpathlineto{\pgfqpoint{1.061711in}{2.254005in}}%
\pgfpathlineto{\pgfqpoint{1.079786in}{2.283895in}}%
\pgfpathlineto{\pgfqpoint{1.096451in}{2.341731in}}%
\pgfpathlineto{\pgfqpoint{1.115934in}{2.480496in}}%
\pgfpathlineto{\pgfqpoint{1.140581in}{2.631385in}}%
\pgfpathlineto{\pgfqpoint{1.156777in}{2.605507in}}%
\pgfpathlineto{\pgfqpoint{1.174382in}{2.456729in}}%
\pgfpathlineto{\pgfqpoint{1.195741in}{2.324308in}}%
\pgfpathlineto{\pgfqpoint{1.213580in}{2.271258in}}%
\pgfpathlineto{\pgfqpoint{1.235176in}{2.248698in}}%
\pgfpathlineto{\pgfqpoint{1.253015in}{2.255440in}}%
\pgfpathlineto{\pgfqpoint{1.271794in}{2.278507in}}%
\pgfpathlineto{\pgfqpoint{1.291981in}{2.348352in}}%
\pgfpathlineto{\pgfqpoint{1.328834in}{2.602629in}}%
\pgfpathlineto{\pgfqpoint{1.349490in}{2.610095in}}%
\pgfpathlineto{\pgfqpoint{1.385872in}{2.354423in}}%
\pgfpathlineto{\pgfqpoint{1.405590in}{2.280799in}}%
\pgfpathlineto{\pgfqpoint{1.430001in}{2.248499in}}%
\pgfpathlineto{\pgfqpoint{1.445025in}{2.248284in}}%
\pgfpathlineto{\pgfqpoint{1.464976in}{2.277017in}}%
\pgfpathlineto{\pgfqpoint{1.483286in}{2.312784in}}%
\pgfpathlineto{\pgfqpoint{1.501594in}{2.391185in}}%
\pgfpathlineto{\pgfqpoint{1.519433in}{2.540109in}}%
\pgfpathlineto{\pgfqpoint{1.540793in}{2.623705in}}%
\pgfpathlineto{\pgfqpoint{1.559806in}{2.567753in}}%
\pgfpathlineto{\pgfqpoint{1.579993in}{2.469679in}}%
\pgfpathlineto{\pgfqpoint{1.597598in}{2.347268in}}%
\pgfpathlineto{\pgfqpoint{1.617315in}{2.275382in}}%
\pgfpathlineto{\pgfqpoint{1.635625in}{2.252872in}}%
\pgfpathlineto{\pgfqpoint{1.653698in}{2.246077in}}%
\pgfpathlineto{\pgfqpoint{1.675060in}{2.261241in}}%
\pgfpathlineto{\pgfqpoint{1.695481in}{2.291222in}}%
\pgfpathlineto{\pgfqpoint{1.712147in}{2.329242in}}%
\pgfpathlineto{\pgfqpoint{1.731629in}{2.466496in}}%
\pgfpathlineto{\pgfqpoint{1.754397in}{2.606459in}}%
\pgfpathlineto{\pgfqpoint{1.771767in}{2.618415in}}%
\pgfpathlineto{\pgfqpoint{1.790780in}{2.550564in}}%
\pgfpathlineto{\pgfqpoint{1.810733in}{2.396405in}}%
\pgfpathlineto{\pgfqpoint{1.828338in}{2.314843in}}%
\pgfpathlineto{\pgfqpoint{1.847351in}{2.271380in}}%
\pgfpathlineto{\pgfqpoint{1.866833in}{2.248906in}}%
\pgfpathlineto{\pgfqpoint{1.885847in}{2.247242in}}%
\pgfpathlineto{\pgfqpoint{1.905798in}{2.268567in}}%
\pgfpathlineto{\pgfqpoint{1.923873in}{2.304036in}}%
\pgfpathlineto{\pgfqpoint{1.946172in}{2.377993in}}%
\pgfpathlineto{\pgfqpoint{1.963072in}{2.428811in}}%
\pgfpathlineto{\pgfqpoint{1.983962in}{2.580518in}}%
\pgfpathlineto{\pgfqpoint{2.002038in}{2.614205in}}%
\pgfpathlineto{\pgfqpoint{2.019877in}{2.547610in}}%
\pgfpathlineto{\pgfqpoint{2.041471in}{2.385655in}}%
\pgfpathlineto{\pgfqpoint{2.059310in}{2.302876in}}%
\pgfpathlineto{\pgfqpoint{2.077855in}{2.265282in}}%
\pgfpathlineto{\pgfqpoint{2.096868in}{2.585424in}}%
\pgfpathlineto{\pgfqpoint{2.117524in}{2.417392in}}%
\pgfpathlineto{\pgfqpoint{2.137240in}{2.306477in}}%
\pgfpathlineto{\pgfqpoint{2.154845in}{2.264813in}}%
\pgfpathlineto{\pgfqpoint{2.176675in}{2.245863in}}%
\pgfpathlineto{\pgfqpoint{2.194280in}{2.251895in}}%
\pgfpathlineto{\pgfqpoint{2.212354in}{2.278346in}}%
\pgfpathlineto{\pgfqpoint{2.234184in}{2.330831in}}%
\pgfpathlineto{\pgfqpoint{2.252258in}{2.435826in}}%
\pgfpathlineto{\pgfqpoint{2.289815in}{2.614828in}}%
\pgfpathlineto{\pgfqpoint{2.307420in}{2.575774in}}%
\pgfpathlineto{\pgfqpoint{2.327842in}{2.427065in}}%
\pgfpathlineto{\pgfqpoint{2.348967in}{2.307474in}}%
\pgfpathlineto{\pgfqpoint{2.367277in}{2.269329in}}%
\pgfpathlineto{\pgfqpoint{2.386525in}{2.248555in}}%
\pgfpathlineto{\pgfqpoint{2.404364in}{2.246238in}}%
\pgfpathlineto{\pgfqpoint{2.423846in}{2.255202in}}%
\pgfpathlineto{\pgfqpoint{2.442154in}{2.282149in}}%
\pgfpathlineto{\pgfqpoint{2.460933in}{2.346800in}}%
\pgfpathlineto{\pgfqpoint{2.486283in}{2.513703in}}%
\pgfpathlineto{\pgfqpoint{2.501542in}{2.597944in}}%
\pgfpathlineto{\pgfqpoint{2.518676in}{2.606907in}}%
\pgfpathlineto{\pgfqpoint{2.540037in}{2.513206in}}%
\pgfpathlineto{\pgfqpoint{2.558345in}{2.377365in}}%
\pgfpathlineto{\pgfqpoint{2.579001in}{2.291234in}}%
\pgfpathlineto{\pgfqpoint{2.598249in}{2.260382in}}%
\pgfpathlineto{\pgfqpoint{2.618202in}{2.245174in}}%
\pgfpathlineto{\pgfqpoint{2.637684in}{2.248488in}}%
\pgfpathlineto{\pgfqpoint{2.655758in}{2.268586in}}%
\pgfpathlineto{\pgfqpoint{2.674771in}{2.295576in}}%
\pgfpathlineto{\pgfqpoint{2.693081in}{2.350258in}}%
\pgfpathlineto{\pgfqpoint{2.714675in}{2.471911in}}%
\pgfpathlineto{\pgfqpoint{2.733219in}{2.587830in}}%
\pgfpathlineto{\pgfqpoint{2.752232in}{2.615830in}}%
\pgfpathlineto{\pgfqpoint{2.768663in}{2.561518in}}%
\pgfpathlineto{\pgfqpoint{2.789554in}{2.404311in}}%
\pgfpathlineto{\pgfqpoint{2.807159in}{2.312235in}}%
\pgfpathlineto{\pgfqpoint{2.826406in}{2.268715in}}%
\pgfpathlineto{\pgfqpoint{2.847768in}{2.249685in}}%
\pgfpathlineto{\pgfqpoint{2.867484in}{2.248345in}}%
\pgfpathlineto{\pgfqpoint{2.885792in}{2.263818in}}%
\pgfpathlineto{\pgfqpoint{2.904571in}{2.293940in}}%
\pgfpathlineto{\pgfqpoint{2.920767in}{2.339804in}}%
\pgfpathlineto{\pgfqpoint{2.943537in}{2.463072in}}%
\pgfpathlineto{\pgfqpoint{2.960202in}{2.575753in}}%
\pgfpathlineto{\pgfqpoint{2.982501in}{2.619628in}}%
\pgfpathlineto{\pgfqpoint{3.001046in}{2.579252in}}%
\pgfpathlineto{\pgfqpoint{3.018180in}{2.487359in}}%
\pgfpathlineto{\pgfqpoint{3.040010in}{2.350071in}}%
\pgfpathlineto{\pgfqpoint{3.059023in}{2.294956in}}%
\pgfpathlineto{\pgfqpoint{3.099867in}{2.248091in}}%
\pgfpathlineto{\pgfqpoint{3.115358in}{2.246679in}}%
\pgfpathlineto{\pgfqpoint{3.136719in}{2.261017in}}%
\pgfpathlineto{\pgfqpoint{3.154793in}{2.286735in}}%
\pgfpathlineto{\pgfqpoint{3.172398in}{2.333995in}}%
\pgfpathlineto{\pgfqpoint{3.194931in}{2.439939in}}%
\pgfpathlineto{\pgfqpoint{3.211598in}{2.566014in}}%
\pgfpathlineto{\pgfqpoint{3.231080in}{2.623284in}}%
\pgfpathlineto{\pgfqpoint{3.250797in}{2.338045in}}%
\pgfpathlineto{\pgfqpoint{3.268872in}{2.408514in}}%
\pgfpathlineto{\pgfqpoint{3.288823in}{2.553566in}}%
\pgfpathlineto{\pgfqpoint{3.306897in}{2.622052in}}%
\pgfpathlineto{\pgfqpoint{3.326145in}{2.625611in}}%
\pgfpathlineto{\pgfqpoint{3.348446in}{2.519251in}}%
\pgfpathlineto{\pgfqpoint{3.367457in}{2.383682in}}%
\pgfpathlineto{\pgfqpoint{3.384124in}{2.328427in}}%
\pgfpathlineto{\pgfqpoint{3.405718in}{2.271548in}}%
\pgfpathlineto{\pgfqpoint{3.425905in}{2.249381in}}%
\pgfpathlineto{\pgfqpoint{3.441633in}{2.247823in}}%
\pgfpathlineto{\pgfqpoint{3.463932in}{2.266668in}}%
\pgfpathlineto{\pgfqpoint{3.482005in}{2.303751in}}%
\pgfpathlineto{\pgfqpoint{3.501724in}{2.372276in}}%
\pgfpathlineto{\pgfqpoint{3.539748in}{2.610902in}}%
\pgfpathlineto{\pgfqpoint{3.558996in}{2.627529in}}%
\pgfpathlineto{\pgfqpoint{3.578009in}{2.557192in}}%
\pgfpathlineto{\pgfqpoint{3.598197in}{2.434751in}}%
\pgfpathlineto{\pgfqpoint{3.618149in}{2.347984in}}%
\pgfpathlineto{\pgfqpoint{3.635283in}{2.287993in}}%
\pgfpathlineto{\pgfqpoint{3.655236in}{2.261034in}}%
\pgfpathlineto{\pgfqpoint{3.673779in}{2.248800in}}%
\pgfpathlineto{\pgfqpoint{3.689272in}{2.251089in}}%
\pgfpathlineto{\pgfqpoint{3.712745in}{2.275829in}}%
\pgfpathlineto{\pgfqpoint{3.732227in}{2.299326in}}%
\pgfpathlineto{\pgfqpoint{3.750301in}{2.359397in}}%
\pgfpathlineto{\pgfqpoint{3.770019in}{2.447902in}}%
\pgfpathlineto{\pgfqpoint{3.787858in}{2.569881in}}%
\pgfpathlineto{\pgfqpoint{3.807106in}{2.629108in}}%
\pgfpathlineto{\pgfqpoint{3.827057in}{2.635611in}}%
\pgfpathlineto{\pgfqpoint{3.849358in}{2.527984in}}%
\pgfpathlineto{\pgfqpoint{3.864146in}{2.425190in}}%
\pgfpathlineto{\pgfqpoint{3.882688in}{2.335086in}}%
\pgfpathlineto{\pgfqpoint{3.905458in}{2.292335in}}%
\pgfpathlineto{\pgfqpoint{3.920949in}{2.266495in}}%
\pgfpathlineto{\pgfqpoint{3.942311in}{2.249913in}}%
\pgfpathlineto{\pgfqpoint{3.961088in}{2.254604in}}%
\pgfpathlineto{\pgfqpoint{3.980335in}{2.274928in}}%
\pgfpathlineto{\pgfqpoint{3.999348in}{2.303230in}}%
\pgfpathlineto{\pgfqpoint{4.021413in}{2.367479in}}%
\pgfpathlineto{\pgfqpoint{4.038315in}{2.464471in}}%
\pgfpathlineto{\pgfqpoint{4.077044in}{2.648310in}}%
\pgfpathlineto{\pgfqpoint{4.096292in}{2.638459in}}%
\pgfpathlineto{\pgfqpoint{4.117653in}{2.596629in}}%
\pgfpathlineto{\pgfqpoint{4.133850in}{2.489366in}}%
\pgfpathlineto{\pgfqpoint{4.152863in}{2.378113in}}%
\pgfpathlineto{\pgfqpoint{4.172111in}{2.316180in}}%
\pgfpathlineto{\pgfqpoint{4.193705in}{2.272601in}}%
\pgfpathlineto{\pgfqpoint{4.212015in}{2.257810in}}%
\pgfpathlineto{\pgfqpoint{4.231028in}{2.253643in}}%
\pgfpathlineto{\pgfqpoint{4.250744in}{2.269501in}}%
\pgfpathlineto{\pgfqpoint{4.270226in}{2.299565in}}%
\pgfpathlineto{\pgfqpoint{4.308958in}{2.408931in}}%
\pgfpathlineto{\pgfqpoint{4.328440in}{2.527237in}}%
\pgfpathlineto{\pgfqpoint{4.346983in}{2.624615in}}%
\pgfpathlineto{\pgfqpoint{4.365996in}{2.666622in}}%
\pgfpathlineto{\pgfqpoint{4.383601in}{2.635727in}}%
\pgfpathlineto{\pgfqpoint{4.402145in}{2.568708in}}%
\pgfpathlineto{\pgfqpoint{4.422096in}{2.440942in}}%
\pgfpathlineto{\pgfqpoint{4.443223in}{2.345964in}}%
\pgfpathlineto{\pgfqpoint{4.462000in}{2.296916in}}%
\pgfpathlineto{\pgfqpoint{4.477493in}{2.271224in}}%
\pgfpathlineto{\pgfqpoint{4.480310in}{2.276371in}}%
\pgfpathlineto{\pgfqpoint{4.474442in}{2.288807in}}%
\pgfpathlineto{\pgfqpoint{4.452846in}{2.384067in}}%
\pgfpathlineto{\pgfqpoint{4.435007in}{2.533063in}}%
\pgfpathlineto{\pgfqpoint{4.415993in}{2.655142in}}%
\pgfpathlineto{\pgfqpoint{4.396746in}{2.645333in}}%
\pgfpathlineto{\pgfqpoint{4.360128in}{2.355233in}}%
\pgfpathlineto{\pgfqpoint{4.340177in}{2.267079in}}%
\pgfpathlineto{\pgfqpoint{4.318815in}{2.253806in}}%
\pgfpathlineto{\pgfqpoint{4.302385in}{2.280551in}}%
\pgfpathlineto{\pgfqpoint{4.282197in}{2.348262in}}%
\pgfpathlineto{\pgfqpoint{4.263889in}{2.487825in}}%
\pgfpathlineto{\pgfqpoint{4.241356in}{2.643344in}}%
\pgfpathlineto{\pgfqpoint{4.223280in}{2.635891in}}%
\pgfpathlineto{\pgfqpoint{4.205441in}{2.491143in}}%
\pgfpathlineto{\pgfqpoint{4.185725in}{2.347116in}}%
\pgfpathlineto{\pgfqpoint{4.166946in}{2.282913in}}%
\pgfpathlineto{\pgfqpoint{4.148403in}{2.253061in}}%
\pgfpathlineto{\pgfqpoint{4.127042in}{2.260976in}}%
\pgfpathlineto{\pgfqpoint{4.108263in}{2.301734in}}%
\pgfpathlineto{\pgfqpoint{4.089721in}{2.399357in}}%
\pgfpathlineto{\pgfqpoint{4.071176in}{2.555988in}}%
\pgfpathlineto{\pgfqpoint{4.049346in}{2.644811in}}%
\pgfpathlineto{\pgfqpoint{4.029864in}{2.574245in}}%
\pgfpathlineto{\pgfqpoint{4.011321in}{2.404771in}}%
\pgfpathlineto{\pgfqpoint{3.992777in}{2.305487in}}%
\pgfpathlineto{\pgfqpoint{3.974703in}{2.263555in}}%
\pgfpathlineto{\pgfqpoint{3.956159in}{2.247959in}}%
\pgfpathlineto{\pgfqpoint{3.937380in}{2.258550in}}%
\pgfpathlineto{\pgfqpoint{3.915081in}{2.305256in}}%
\pgfpathlineto{\pgfqpoint{3.896773in}{2.409374in}}%
\pgfpathlineto{\pgfqpoint{3.878229in}{2.560453in}}%
\pgfpathlineto{\pgfqpoint{3.859450in}{2.635479in}}%
\pgfpathlineto{\pgfqpoint{3.839968in}{2.586129in}}%
\pgfpathlineto{\pgfqpoint{3.822128in}{2.427062in}}%
\pgfpathlineto{\pgfqpoint{3.800533in}{2.306021in}}%
\pgfpathlineto{\pgfqpoint{3.781990in}{2.265832in}}%
\pgfpathlineto{\pgfqpoint{3.764151in}{2.247895in}}%
\pgfpathlineto{\pgfqpoint{3.742321in}{2.266306in}}%
\pgfpathlineto{\pgfqpoint{3.725656in}{2.287593in}}%
\pgfpathlineto{\pgfqpoint{3.701712in}{2.396312in}}%
\pgfpathlineto{\pgfqpoint{3.685281in}{2.528898in}}%
\pgfpathlineto{\pgfqpoint{3.685986in}{2.597971in}}%
\pgfpathlineto{\pgfqpoint{3.667676in}{2.622803in}}%
\pgfpathlineto{\pgfqpoint{3.650777in}{2.612575in}}%
\pgfpathlineto{\pgfqpoint{3.608056in}{2.329847in}}%
\pgfpathlineto{\pgfqpoint{3.591625in}{2.298423in}}%
\pgfpathlineto{\pgfqpoint{3.573081in}{2.268011in}}%
\pgfpathlineto{\pgfqpoint{3.551721in}{2.246451in}}%
\pgfpathlineto{\pgfqpoint{3.531534in}{2.254916in}}%
\pgfpathlineto{\pgfqpoint{3.510878in}{2.289855in}}%
\pgfpathlineto{\pgfqpoint{3.494681in}{2.350291in}}%
\pgfpathlineto{\pgfqpoint{3.476137in}{2.462501in}}%
\pgfpathlineto{\pgfqpoint{3.456421in}{2.345312in}}%
\pgfpathlineto{\pgfqpoint{3.434825in}{2.510511in}}%
\pgfpathlineto{\pgfqpoint{3.418628in}{2.592702in}}%
\pgfpathlineto{\pgfqpoint{3.398677in}{2.608813in}}%
\pgfpathlineto{\pgfqpoint{3.376142in}{2.457567in}}%
\pgfpathlineto{\pgfqpoint{3.362528in}{2.344469in}}%
\pgfpathlineto{\pgfqpoint{3.340698in}{2.290106in}}%
\pgfpathlineto{\pgfqpoint{3.319339in}{2.253567in}}%
\pgfpathlineto{\pgfqpoint{3.300560in}{2.246116in}}%
\pgfpathlineto{\pgfqpoint{3.281547in}{2.262788in}}%
\pgfpathlineto{\pgfqpoint{3.263004in}{2.297507in}}%
\pgfpathlineto{\pgfqpoint{3.245399in}{2.341323in}}%
\pgfpathlineto{\pgfqpoint{3.225915in}{2.462381in}}%
\pgfpathlineto{\pgfqpoint{3.204790in}{2.606717in}}%
\pgfpathlineto{\pgfqpoint{3.185777in}{2.604594in}}%
\pgfpathlineto{\pgfqpoint{3.167235in}{2.470453in}}%
\pgfpathlineto{\pgfqpoint{3.147750in}{2.370776in}}%
\pgfpathlineto{\pgfqpoint{3.127800in}{2.297742in}}%
\pgfpathlineto{\pgfqpoint{3.109021in}{2.260634in}}%
\pgfpathlineto{\pgfqpoint{3.090713in}{2.245558in}}%
\pgfpathlineto{\pgfqpoint{3.072168in}{2.252698in}}%
\pgfpathlineto{\pgfqpoint{3.053389in}{2.278876in}}%
\pgfpathlineto{\pgfqpoint{3.031796in}{2.351037in}}%
\pgfpathlineto{\pgfqpoint{3.014660in}{2.455760in}}%
\pgfpathlineto{\pgfqpoint{2.995412in}{2.593866in}}%
\pgfpathlineto{\pgfqpoint{2.974990in}{2.613144in}}%
\pgfpathlineto{\pgfqpoint{2.956682in}{2.545526in}}%
\pgfpathlineto{\pgfqpoint{2.917716in}{2.323341in}}%
\pgfpathlineto{\pgfqpoint{2.898937in}{2.274254in}}%
\pgfpathlineto{\pgfqpoint{2.880395in}{2.250709in}}%
\pgfpathlineto{\pgfqpoint{2.860676in}{2.246723in}}%
\pgfpathlineto{\pgfqpoint{2.836500in}{2.261102in}}%
\pgfpathlineto{\pgfqpoint{2.821243in}{2.286376in}}%
\pgfpathlineto{\pgfqpoint{2.802464in}{2.344305in}}%
\pgfpathlineto{\pgfqpoint{2.783685in}{2.458752in}}%
\pgfpathlineto{\pgfqpoint{2.762092in}{2.598498in}}%
\pgfpathlineto{\pgfqpoint{2.743313in}{2.612873in}}%
\pgfpathlineto{\pgfqpoint{2.728291in}{2.551545in}}%
\pgfpathlineto{\pgfqpoint{2.688150in}{2.366939in}}%
\pgfpathlineto{\pgfqpoint{2.669137in}{2.292856in}}%
\pgfpathlineto{\pgfqpoint{2.648481in}{2.257328in}}%
\pgfpathlineto{\pgfqpoint{2.628530in}{2.246406in}}%
\pgfpathlineto{\pgfqpoint{2.610220in}{2.245673in}}%
\pgfpathlineto{\pgfqpoint{2.593086in}{2.258451in}}%
\pgfpathlineto{\pgfqpoint{2.573604in}{2.294358in}}%
\pgfpathlineto{\pgfqpoint{2.552477in}{2.357931in}}%
\pgfpathlineto{\pgfqpoint{2.533229in}{2.494268in}}%
\pgfpathlineto{\pgfqpoint{2.516095in}{2.598781in}}%
\pgfpathlineto{\pgfqpoint{2.497082in}{2.601165in}}%
\pgfpathlineto{\pgfqpoint{2.456004in}{2.363353in}}%
\pgfpathlineto{\pgfqpoint{2.437694in}{2.297204in}}%
\pgfpathlineto{\pgfqpoint{2.416804in}{2.260916in}}%
\pgfpathlineto{\pgfqpoint{2.398261in}{2.246406in}}%
\pgfpathlineto{\pgfqpoint{2.381360in}{2.247676in}}%
\pgfpathlineto{\pgfqpoint{2.360469in}{2.268608in}}%
\pgfpathlineto{\pgfqpoint{2.337465in}{2.313904in}}%
\pgfpathlineto{\pgfqpoint{2.323382in}{2.266461in}}%
\pgfpathlineto{\pgfqpoint{2.302255in}{2.303353in}}%
\pgfpathlineto{\pgfqpoint{2.280896in}{2.421080in}}%
\pgfpathlineto{\pgfqpoint{2.264934in}{2.544336in}}%
\pgfpathlineto{\pgfqpoint{2.245217in}{2.614753in}}%
\pgfpathlineto{\pgfqpoint{2.229255in}{2.578794in}}%
\pgfpathlineto{\pgfqpoint{2.187238in}{2.334084in}}%
\pgfpathlineto{\pgfqpoint{2.168930in}{2.306392in}}%
\pgfpathlineto{\pgfqpoint{2.148039in}{2.263638in}}%
\pgfpathlineto{\pgfqpoint{2.130200in}{2.247171in}}%
\pgfpathlineto{\pgfqpoint{2.112359in}{2.245636in}}%
\pgfpathlineto{\pgfqpoint{2.089122in}{2.261166in}}%
\pgfpathlineto{\pgfqpoint{2.074803in}{2.285496in}}%
\pgfpathlineto{\pgfqpoint{2.053207in}{2.355795in}}%
\pgfpathlineto{\pgfqpoint{2.034194in}{2.492161in}}%
\pgfpathlineto{\pgfqpoint{2.012600in}{2.598384in}}%
\pgfpathlineto{\pgfqpoint{1.996638in}{2.617094in}}%
\pgfpathlineto{\pgfqpoint{1.975748in}{2.549296in}}%
\pgfpathlineto{\pgfqpoint{1.956266in}{2.410808in}}%
\pgfpathlineto{\pgfqpoint{1.939364in}{2.327292in}}%
\pgfpathlineto{\pgfqpoint{1.919648in}{2.278665in}}%
\pgfpathlineto{\pgfqpoint{1.898052in}{2.251748in}}%
\pgfpathlineto{\pgfqpoint{1.876222in}{2.246578in}}%
\pgfpathlineto{\pgfqpoint{1.860496in}{2.264363in}}%
\pgfpathlineto{\pgfqpoint{1.844064in}{2.258856in}}%
\pgfpathlineto{\pgfqpoint{1.822235in}{2.246719in}}%
\pgfpathlineto{\pgfqpoint{1.803691in}{2.255220in}}%
\pgfpathlineto{\pgfqpoint{1.783035in}{2.287068in}}%
\pgfpathlineto{\pgfqpoint{1.764961in}{2.334796in}}%
\pgfpathlineto{\pgfqpoint{1.726935in}{2.555955in}}%
\pgfpathlineto{\pgfqpoint{1.703696in}{2.625228in}}%
\pgfpathlineto{\pgfqpoint{1.687031in}{2.588401in}}%
\pgfpathlineto{\pgfqpoint{1.670600in}{2.482813in}}%
\pgfpathlineto{\pgfqpoint{1.650647in}{2.374683in}}%
\pgfpathlineto{\pgfqpoint{1.630460in}{2.297325in}}%
\pgfpathlineto{\pgfqpoint{1.610509in}{2.268734in}}%
\pgfpathlineto{\pgfqpoint{1.593138in}{2.252463in}}%
\pgfpathlineto{\pgfqpoint{1.574125in}{2.247499in}}%
\pgfpathlineto{\pgfqpoint{1.535161in}{2.280059in}}%
\pgfpathlineto{\pgfqpoint{1.512862in}{2.349848in}}%
\pgfpathlineto{\pgfqpoint{1.495726in}{2.408152in}}%
\pgfpathlineto{\pgfqpoint{1.475070in}{2.548490in}}%
\pgfpathlineto{\pgfqpoint{1.457465in}{2.621313in}}%
\pgfpathlineto{\pgfqpoint{1.439157in}{2.627768in}}%
\pgfpathlineto{\pgfqpoint{1.421552in}{2.591458in}}%
\pgfpathlineto{\pgfqpoint{1.399251in}{2.521239in}}%
\pgfpathlineto{\pgfqpoint{1.380004in}{2.392982in}}%
\pgfpathlineto{\pgfqpoint{1.364044in}{2.317310in}}%
\pgfpathlineto{\pgfqpoint{1.343387in}{2.277502in}}%
\pgfpathlineto{\pgfqpoint{1.321792in}{2.257236in}}%
\pgfpathlineto{\pgfqpoint{1.304187in}{2.248331in}}%
\pgfpathlineto{\pgfqpoint{1.284234in}{2.259669in}}%
\pgfpathlineto{\pgfqpoint{1.266866in}{2.280748in}}%
\pgfpathlineto{\pgfqpoint{1.245739in}{2.326937in}}%
\pgfpathlineto{\pgfqpoint{1.227431in}{2.388373in}}%
\pgfpathlineto{\pgfqpoint{1.207478in}{2.497201in}}%
\pgfpathlineto{\pgfqpoint{1.188701in}{2.298228in}}%
\pgfpathlineto{\pgfqpoint{1.170860in}{2.263765in}}%
\pgfpathlineto{\pgfqpoint{1.149266in}{2.248122in}}%
\pgfpathlineto{\pgfqpoint{1.130487in}{2.258913in}}%
\pgfpathlineto{\pgfqpoint{1.110770in}{2.301157in}}%
\pgfpathlineto{\pgfqpoint{1.092460in}{2.355799in}}%
\pgfpathlineto{\pgfqpoint{1.072273in}{2.498176in}}%
\pgfpathlineto{\pgfqpoint{1.054200in}{2.614004in}}%
\pgfpathlineto{\pgfqpoint{1.031900in}{2.642260in}}%
\pgfpathlineto{\pgfqpoint{1.016878in}{2.579545in}}%
\pgfpathlineto{\pgfqpoint{0.994814in}{2.426206in}}%
\pgfpathlineto{\pgfqpoint{0.977912in}{2.355500in}}%
\pgfpathlineto{\pgfqpoint{0.956084in}{2.278521in}}%
\pgfpathlineto{\pgfqpoint{0.937539in}{2.258301in}}%
\pgfpathlineto{\pgfqpoint{0.919935in}{2.251482in}}%
\pgfpathlineto{\pgfqpoint{0.899747in}{2.268487in}}%
\pgfpathlineto{\pgfqpoint{0.881205in}{2.304032in}}%
\pgfpathlineto{\pgfqpoint{0.861957in}{2.391094in}}%
\pgfpathlineto{\pgfqpoint{0.842944in}{2.483951in}}%
\pgfpathlineto{\pgfqpoint{0.824870in}{2.606897in}}%
\pgfpathlineto{\pgfqpoint{0.804448in}{2.655746in}}%
\pgfpathlineto{\pgfqpoint{0.780505in}{2.578228in}}%
\pgfpathlineto{\pgfqpoint{0.764074in}{2.468963in}}%
\pgfpathlineto{\pgfqpoint{0.746000in}{2.364978in}}%
\pgfpathlineto{\pgfqpoint{0.728864in}{2.305791in}}%
\pgfpathlineto{\pgfqpoint{0.708208in}{2.268330in}}%
\pgfpathlineto{\pgfqpoint{0.686614in}{2.253424in}}%
\pgfpathlineto{\pgfqpoint{0.668539in}{2.258107in}}%
\pgfpathlineto{\pgfqpoint{0.650934in}{2.269977in}}%
\pgfpathlineto{\pgfqpoint{0.657507in}{2.260118in}}%
\pgfpathlineto{\pgfqpoint{0.675815in}{2.256549in}}%
\pgfpathlineto{\pgfqpoint{0.693891in}{2.285669in}}%
\pgfpathlineto{\pgfqpoint{0.713608in}{2.351242in}}%
\pgfpathlineto{\pgfqpoint{0.751634in}{2.641508in}}%
\pgfpathlineto{\pgfqpoint{0.772525in}{2.630876in}}%
\pgfpathlineto{\pgfqpoint{0.791303in}{2.493838in}}%
\pgfpathlineto{\pgfqpoint{0.809143in}{2.349847in}}%
\pgfpathlineto{\pgfqpoint{0.831676in}{2.275978in}}%
\pgfpathlineto{\pgfqpoint{0.848107in}{2.252412in}}%
\pgfpathlineto{\pgfqpoint{0.866182in}{2.261939in}}%
\pgfpathlineto{\pgfqpoint{0.886602in}{2.321365in}}%
\pgfpathlineto{\pgfqpoint{0.906086in}{2.399947in}}%
\pgfpathlineto{\pgfqpoint{0.923220in}{2.554165in}}%
\pgfpathlineto{\pgfqpoint{0.945519in}{2.645849in}}%
\pgfpathlineto{\pgfqpoint{0.961952in}{2.579399in}}%
\pgfpathlineto{\pgfqpoint{0.984017in}{2.395195in}}%
\pgfpathlineto{\pgfqpoint{1.002559in}{2.305193in}}%
\pgfpathlineto{\pgfqpoint{1.022746in}{2.258135in}}%
\pgfpathlineto{\pgfqpoint{1.040586in}{2.250179in}}%
\pgfpathlineto{\pgfqpoint{1.057956in}{2.272785in}}%
\pgfpathlineto{\pgfqpoint{1.079786in}{2.334783in}}%
\pgfpathlineto{\pgfqpoint{1.115230in}{2.607150in}}%
\pgfpathlineto{\pgfqpoint{1.135886in}{2.622328in}}%
\pgfpathlineto{\pgfqpoint{1.156308in}{2.474401in}}%
\pgfpathlineto{\pgfqpoint{1.175085in}{2.348147in}}%
\pgfpathlineto{\pgfqpoint{1.195741in}{2.273888in}}%
\pgfpathlineto{\pgfqpoint{1.214520in}{2.249845in}}%
\pgfpathlineto{\pgfqpoint{1.232125in}{2.254268in}}%
\pgfpathlineto{\pgfqpoint{1.252547in}{2.285245in}}%
\pgfpathlineto{\pgfqpoint{1.271794in}{2.350901in}}%
\pgfpathlineto{\pgfqpoint{1.288459in}{2.488427in}}%
\pgfpathlineto{\pgfqpoint{1.309821in}{2.616251in}}%
\pgfpathlineto{\pgfqpoint{1.332120in}{2.596792in}}%
\pgfpathlineto{\pgfqpoint{1.348785in}{2.468615in}}%
\pgfpathlineto{\pgfqpoint{1.366390in}{2.351493in}}%
\pgfpathlineto{\pgfqpoint{1.387517in}{2.277756in}}%
\pgfpathlineto{\pgfqpoint{1.406294in}{2.252350in}}%
\pgfpathlineto{\pgfqpoint{1.426481in}{2.249157in}}%
\pgfpathlineto{\pgfqpoint{1.444086in}{2.270200in}}%
\pgfpathlineto{\pgfqpoint{1.462864in}{2.314635in}}%
\pgfpathlineto{\pgfqpoint{1.482581in}{2.435020in}}%
\pgfpathlineto{\pgfqpoint{1.500889in}{2.578002in}}%
\pgfpathlineto{\pgfqpoint{1.522250in}{2.624230in}}%
\pgfpathlineto{\pgfqpoint{1.540324in}{2.584490in}}%
\pgfpathlineto{\pgfqpoint{1.557226in}{2.452213in}}%
\pgfpathlineto{\pgfqpoint{1.579993in}{2.325665in}}%
\pgfpathlineto{\pgfqpoint{1.597129in}{2.273593in}}%
\pgfpathlineto{\pgfqpoint{1.617786in}{2.253885in}}%
\pgfpathlineto{\pgfqpoint{1.635859in}{2.246212in}}%
\pgfpathlineto{\pgfqpoint{1.659332in}{2.269037in}}%
\pgfpathlineto{\pgfqpoint{1.674120in}{2.299390in}}%
\pgfpathlineto{\pgfqpoint{1.694307in}{2.381365in}}%
\pgfpathlineto{\pgfqpoint{1.712850in}{2.517395in}}%
\pgfpathlineto{\pgfqpoint{1.732568in}{2.611817in}}%
\pgfpathlineto{\pgfqpoint{1.751582in}{2.593654in}}%
\pgfpathlineto{\pgfqpoint{1.770124in}{2.487374in}}%
\pgfpathlineto{\pgfqpoint{1.792189in}{2.381286in}}%
\pgfpathlineto{\pgfqpoint{1.809794in}{2.309897in}}%
\pgfpathlineto{\pgfqpoint{1.827633in}{2.275017in}}%
\pgfpathlineto{\pgfqpoint{1.849697in}{2.249539in}}%
\pgfpathlineto{\pgfqpoint{1.868007in}{2.246785in}}%
\pgfpathlineto{\pgfqpoint{1.888193in}{2.260070in}}%
\pgfpathlineto{\pgfqpoint{1.903686in}{2.287785in}}%
\pgfpathlineto{\pgfqpoint{1.923873in}{2.349630in}}%
\pgfpathlineto{\pgfqpoint{1.944764in}{2.469094in}}%
\pgfpathlineto{\pgfqpoint{1.965889in}{2.280653in}}%
\pgfpathlineto{\pgfqpoint{1.982085in}{2.337041in}}%
\pgfpathlineto{\pgfqpoint{2.002272in}{2.478127in}}%
\pgfpathlineto{\pgfqpoint{2.020346in}{2.591588in}}%
\pgfpathlineto{\pgfqpoint{2.037246in}{2.613164in}}%
\pgfpathlineto{\pgfqpoint{2.059076in}{2.499841in}}%
\pgfpathlineto{\pgfqpoint{2.076915in}{2.372498in}}%
\pgfpathlineto{\pgfqpoint{2.097337in}{2.285276in}}%
\pgfpathlineto{\pgfqpoint{2.115647in}{2.257124in}}%
\pgfpathlineto{\pgfqpoint{2.137477in}{2.244998in}}%
\pgfpathlineto{\pgfqpoint{2.154376in}{2.254684in}}%
\pgfpathlineto{\pgfqpoint{2.171981in}{2.282087in}}%
\pgfpathlineto{\pgfqpoint{2.193577in}{2.345022in}}%
\pgfpathlineto{\pgfqpoint{2.229490in}{2.576293in}}%
\pgfpathlineto{\pgfqpoint{2.251320in}{2.611445in}}%
\pgfpathlineto{\pgfqpoint{2.272211in}{2.513505in}}%
\pgfpathlineto{\pgfqpoint{2.290989in}{2.373494in}}%
\pgfpathlineto{\pgfqpoint{2.307420in}{2.299576in}}%
\pgfpathlineto{\pgfqpoint{2.328780in}{2.261111in}}%
\pgfpathlineto{\pgfqpoint{2.347090in}{2.246886in}}%
\pgfpathlineto{\pgfqpoint{2.364694in}{2.247655in}}%
\pgfpathlineto{\pgfqpoint{2.386525in}{2.269194in}}%
\pgfpathlineto{\pgfqpoint{2.405772in}{2.292418in}}%
\pgfpathlineto{\pgfqpoint{2.423611in}{2.281627in}}%
\pgfpathlineto{\pgfqpoint{2.444033in}{2.326353in}}%
\pgfpathlineto{\pgfqpoint{2.462810in}{2.408697in}}%
\pgfpathlineto{\pgfqpoint{2.482058in}{2.558977in}}%
\pgfpathlineto{\pgfqpoint{2.503185in}{2.613210in}}%
\pgfpathlineto{\pgfqpoint{2.521493in}{2.546593in}}%
\pgfpathlineto{\pgfqpoint{2.539098in}{2.407387in}}%
\pgfpathlineto{\pgfqpoint{2.557408in}{2.316171in}}%
\pgfpathlineto{\pgfqpoint{2.578064in}{2.267668in}}%
\pgfpathlineto{\pgfqpoint{2.596606in}{2.248151in}}%
\pgfpathlineto{\pgfqpoint{2.616325in}{2.247451in}}%
\pgfpathlineto{\pgfqpoint{2.635336in}{2.266973in}}%
\pgfpathlineto{\pgfqpoint{2.653880in}{2.300249in}}%
\pgfpathlineto{\pgfqpoint{2.674537in}{2.375959in}}%
\pgfpathlineto{\pgfqpoint{2.695427in}{2.506390in}}%
\pgfpathlineto{\pgfqpoint{2.713503in}{2.565527in}}%
\pgfpathlineto{\pgfqpoint{2.730871in}{2.607203in}}%
\pgfpathlineto{\pgfqpoint{2.751293in}{2.587733in}}%
\pgfpathlineto{\pgfqpoint{2.769837in}{2.517083in}}%
\pgfpathlineto{\pgfqpoint{2.788145in}{2.377674in}}%
\pgfpathlineto{\pgfqpoint{2.810444in}{2.289973in}}%
\pgfpathlineto{\pgfqpoint{2.828049in}{2.268575in}}%
\pgfpathlineto{\pgfqpoint{2.851288in}{2.246047in}}%
\pgfpathlineto{\pgfqpoint{2.866544in}{2.249223in}}%
\pgfpathlineto{\pgfqpoint{2.884854in}{2.258816in}}%
\pgfpathlineto{\pgfqpoint{2.905745in}{2.295664in}}%
\pgfpathlineto{\pgfqpoint{2.925227in}{2.355163in}}%
\pgfpathlineto{\pgfqpoint{2.942129in}{2.472270in}}%
\pgfpathlineto{\pgfqpoint{2.963019in}{2.608204in}}%
\pgfpathlineto{\pgfqpoint{2.982267in}{2.613562in}}%
\pgfpathlineto{\pgfqpoint{2.999167in}{2.534657in}}%
\pgfpathlineto{\pgfqpoint{3.021702in}{2.383526in}}%
\pgfpathlineto{\pgfqpoint{3.039072in}{2.310624in}}%
\pgfpathlineto{\pgfqpoint{3.059728in}{2.270149in}}%
\pgfpathlineto{\pgfqpoint{3.077568in}{2.251565in}}%
\pgfpathlineto{\pgfqpoint{3.096815in}{2.246988in}}%
\pgfpathlineto{\pgfqpoint{3.118646in}{2.263896in}}%
\pgfpathlineto{\pgfqpoint{3.135545in}{2.291525in}}%
\pgfpathlineto{\pgfqpoint{3.154793in}{2.342012in}}%
\pgfpathlineto{\pgfqpoint{3.174509in}{2.436220in}}%
\pgfpathlineto{\pgfqpoint{3.191645in}{2.525721in}}%
\pgfpathlineto{\pgfqpoint{3.213475in}{2.607238in}}%
\pgfpathlineto{\pgfqpoint{3.230375in}{2.587341in}}%
\pgfpathlineto{\pgfqpoint{3.249388in}{2.623212in}}%
\pgfpathlineto{\pgfqpoint{3.266759in}{2.578951in}}%
\pgfpathlineto{\pgfqpoint{3.289292in}{2.428462in}}%
\pgfpathlineto{\pgfqpoint{3.309245in}{2.337732in}}%
\pgfpathlineto{\pgfqpoint{3.327319in}{2.293116in}}%
\pgfpathlineto{\pgfqpoint{3.345863in}{2.265478in}}%
\pgfpathlineto{\pgfqpoint{3.364642in}{2.247386in}}%
\pgfpathlineto{\pgfqpoint{3.382716in}{2.251176in}}%
\pgfpathlineto{\pgfqpoint{3.404780in}{2.263517in}}%
\pgfpathlineto{\pgfqpoint{3.423323in}{2.296206in}}%
\pgfpathlineto{\pgfqpoint{3.445387in}{2.366861in}}%
\pgfpathlineto{\pgfqpoint{3.460880in}{2.449458in}}%
\pgfpathlineto{\pgfqpoint{3.482711in}{2.573997in}}%
\pgfpathlineto{\pgfqpoint{3.498436in}{2.631168in}}%
\pgfpathlineto{\pgfqpoint{3.519329in}{2.619184in}}%
\pgfpathlineto{\pgfqpoint{3.538576in}{2.546067in}}%
\pgfpathlineto{\pgfqpoint{3.557353in}{2.434515in}}%
\pgfpathlineto{\pgfqpoint{3.577072in}{2.330532in}}%
\pgfpathlineto{\pgfqpoint{3.594911in}{2.551143in}}%
\pgfpathlineto{\pgfqpoint{3.615567in}{2.633374in}}%
\pgfpathlineto{\pgfqpoint{3.633875in}{2.613539in}}%
\pgfpathlineto{\pgfqpoint{3.673310in}{2.360138in}}%
\pgfpathlineto{\pgfqpoint{3.691854in}{2.293226in}}%
\pgfpathlineto{\pgfqpoint{3.711102in}{2.260785in}}%
\pgfpathlineto{\pgfqpoint{3.733636in}{2.248472in}}%
\pgfpathlineto{\pgfqpoint{3.750535in}{2.259497in}}%
\pgfpathlineto{\pgfqpoint{3.768611in}{2.285904in}}%
\pgfpathlineto{\pgfqpoint{3.787858in}{2.336136in}}%
\pgfpathlineto{\pgfqpoint{3.809452in}{2.436735in}}%
\pgfpathlineto{\pgfqpoint{3.826119in}{2.550575in}}%
\pgfpathlineto{\pgfqpoint{3.851470in}{2.642896in}}%
\pgfpathlineto{\pgfqpoint{3.867432in}{2.616631in}}%
\pgfpathlineto{\pgfqpoint{3.886445in}{2.511592in}}%
\pgfpathlineto{\pgfqpoint{3.901936in}{2.410598in}}%
\pgfpathlineto{\pgfqpoint{3.922592in}{2.323717in}}%
\pgfpathlineto{\pgfqpoint{3.941371in}{2.280379in}}%
\pgfpathlineto{\pgfqpoint{3.962967in}{2.256429in}}%
\pgfpathlineto{\pgfqpoint{3.982449in}{2.249789in}}%
\pgfpathlineto{\pgfqpoint{4.000991in}{2.261650in}}%
\pgfpathlineto{\pgfqpoint{4.019536in}{2.288952in}}%
\pgfpathlineto{\pgfqpoint{4.038783in}{2.336868in}}%
\pgfpathlineto{\pgfqpoint{4.057328in}{2.410072in}}%
\pgfpathlineto{\pgfqpoint{4.076105in}{2.502927in}}%
\pgfpathlineto{\pgfqpoint{4.096527in}{2.621959in}}%
\pgfpathlineto{\pgfqpoint{4.115071in}{2.652428in}}%
\pgfpathlineto{\pgfqpoint{4.136430in}{2.578033in}}%
\pgfpathlineto{\pgfqpoint{4.155678in}{2.530081in}}%
\pgfpathlineto{\pgfqpoint{4.177274in}{2.400623in}}%
\pgfpathlineto{\pgfqpoint{4.193236in}{2.331682in}}%
\pgfpathlineto{\pgfqpoint{4.209198in}{2.288608in}}%
\pgfpathlineto{\pgfqpoint{4.230793in}{2.263448in}}%
\pgfpathlineto{\pgfqpoint{4.249805in}{2.252414in}}%
\pgfpathlineto{\pgfqpoint{4.269052in}{2.265016in}}%
\pgfpathlineto{\pgfqpoint{4.290179in}{2.290905in}}%
\pgfpathlineto{\pgfqpoint{4.306376in}{2.328745in}}%
\pgfpathlineto{\pgfqpoint{4.330318in}{2.439269in}}%
\pgfpathlineto{\pgfqpoint{4.346279in}{2.545467in}}%
\pgfpathlineto{\pgfqpoint{4.366701in}{2.645505in}}%
\pgfpathlineto{\pgfqpoint{4.385244in}{2.667183in}}%
\pgfpathlineto{\pgfqpoint{4.404023in}{2.649730in}}%
\pgfpathlineto{\pgfqpoint{4.422331in}{2.563822in}}%
\pgfpathlineto{\pgfqpoint{4.444632in}{2.437913in}}%
\pgfpathlineto{\pgfqpoint{4.461297in}{2.368856in}}%
\pgfpathlineto{\pgfqpoint{4.479841in}{2.307590in}}%
\pgfpathlineto{\pgfqpoint{4.473973in}{2.329457in}}%
\pgfpathlineto{\pgfqpoint{4.454489in}{2.452115in}}%
\pgfpathlineto{\pgfqpoint{4.434538in}{2.615966in}}%
\pgfpathlineto{\pgfqpoint{4.417402in}{2.667229in}}%
\pgfpathlineto{\pgfqpoint{4.398859in}{2.562600in}}%
\pgfpathlineto{\pgfqpoint{4.377733in}{2.379923in}}%
\pgfpathlineto{\pgfqpoint{4.360128in}{2.301369in}}%
\pgfpathlineto{\pgfqpoint{4.339003in}{2.257549in}}%
\pgfpathlineto{\pgfqpoint{4.320929in}{2.257437in}}%
\pgfpathlineto{\pgfqpoint{4.302150in}{2.293126in}}%
\pgfpathlineto{\pgfqpoint{4.281025in}{2.395804in}}%
\pgfpathlineto{\pgfqpoint{4.261541in}{2.550886in}}%
\pgfpathlineto{\pgfqpoint{4.243233in}{2.652050in}}%
\pgfpathlineto{\pgfqpoint{4.225394in}{2.611448in}}%
\pgfpathlineto{\pgfqpoint{4.203095in}{2.418901in}}%
\pgfpathlineto{\pgfqpoint{4.186899in}{2.332038in}}%
\pgfpathlineto{\pgfqpoint{4.168354in}{2.276072in}}%
\pgfpathlineto{\pgfqpoint{4.147229in}{2.249562in}}%
\pgfpathlineto{\pgfqpoint{4.128919in}{2.266564in}}%
\pgfpathlineto{\pgfqpoint{4.109203in}{2.317811in}}%
\pgfpathlineto{\pgfqpoint{4.090189in}{2.427228in}}%
\pgfpathlineto{\pgfqpoint{4.068830in}{2.608679in}}%
\pgfpathlineto{\pgfqpoint{4.050520in}{2.641184in}}%
\pgfpathlineto{\pgfqpoint{4.031741in}{2.533889in}}%
\pgfpathlineto{\pgfqpoint{4.014842in}{2.381289in}}%
\pgfpathlineto{\pgfqpoint{3.992777in}{2.291673in}}%
\pgfpathlineto{\pgfqpoint{3.974703in}{2.254536in}}%
\pgfpathlineto{\pgfqpoint{3.954282in}{2.250969in}}%
\pgfpathlineto{\pgfqpoint{3.938320in}{2.271499in}}%
\pgfpathlineto{\pgfqpoint{3.916490in}{2.345518in}}%
\pgfpathlineto{\pgfqpoint{3.898181in}{2.469070in}}%
\pgfpathlineto{\pgfqpoint{3.878934in}{2.610178in}}%
\pgfpathlineto{\pgfqpoint{3.860155in}{2.624678in}}%
\pgfpathlineto{\pgfqpoint{3.819782in}{2.342315in}}%
\pgfpathlineto{\pgfqpoint{3.802177in}{2.280961in}}%
\pgfpathlineto{\pgfqpoint{3.782693in}{2.252665in}}%
\pgfpathlineto{\pgfqpoint{3.762037in}{2.250661in}}%
\pgfpathlineto{\pgfqpoint{3.741852in}{2.281715in}}%
\pgfpathlineto{\pgfqpoint{3.724716in}{2.344575in}}%
\pgfpathlineto{\pgfqpoint{3.706172in}{2.479387in}}%
\pgfpathlineto{\pgfqpoint{3.686221in}{2.600706in}}%
\pgfpathlineto{\pgfqpoint{3.667207in}{2.618053in}}%
\pgfpathlineto{\pgfqpoint{3.647254in}{2.516192in}}%
\pgfpathlineto{\pgfqpoint{3.628478in}{2.384599in}}%
\pgfpathlineto{\pgfqpoint{3.608759in}{2.297846in}}%
\pgfpathlineto{\pgfqpoint{3.593502in}{2.268793in}}%
\pgfpathlineto{\pgfqpoint{3.571438in}{2.251210in}}%
\pgfpathlineto{\pgfqpoint{3.552190in}{2.249661in}}%
\pgfpathlineto{\pgfqpoint{3.533177in}{2.268812in}}%
\pgfpathlineto{\pgfqpoint{3.512521in}{2.336754in}}%
\pgfpathlineto{\pgfqpoint{3.496793in}{2.411771in}}%
\pgfpathlineto{\pgfqpoint{3.475903in}{2.576952in}}%
\pgfpathlineto{\pgfqpoint{3.458532in}{2.623983in}}%
\pgfpathlineto{\pgfqpoint{3.438347in}{2.591137in}}%
\pgfpathlineto{\pgfqpoint{3.416046in}{2.429952in}}%
\pgfpathlineto{\pgfqpoint{3.398912in}{2.325202in}}%
\pgfpathlineto{\pgfqpoint{3.378725in}{2.273991in}}%
\pgfpathlineto{\pgfqpoint{3.359711in}{2.250670in}}%
\pgfpathlineto{\pgfqpoint{3.338118in}{2.248495in}}%
\pgfpathlineto{\pgfqpoint{3.322859in}{2.264239in}}%
\pgfpathlineto{\pgfqpoint{3.300794in}{2.290777in}}%
\pgfpathlineto{\pgfqpoint{3.283424in}{2.364334in}}%
\pgfpathlineto{\pgfqpoint{3.263707in}{2.477002in}}%
\pgfpathlineto{\pgfqpoint{3.244460in}{2.602764in}}%
\pgfpathlineto{\pgfqpoint{3.227089in}{2.612161in}}%
\pgfpathlineto{\pgfqpoint{3.205259in}{2.484191in}}%
\pgfpathlineto{\pgfqpoint{3.186482in}{2.358323in}}%
\pgfpathlineto{\pgfqpoint{3.167469in}{2.307071in}}%
\pgfpathlineto{\pgfqpoint{3.145404in}{2.261087in}}%
\pgfpathlineto{\pgfqpoint{3.131320in}{2.247573in}}%
\pgfpathlineto{\pgfqpoint{3.109255in}{2.250963in}}%
\pgfpathlineto{\pgfqpoint{3.090713in}{2.274766in}}%
\pgfpathlineto{\pgfqpoint{3.071934in}{2.326251in}}%
\pgfpathlineto{\pgfqpoint{3.053155in}{2.432230in}}%
\pgfpathlineto{\pgfqpoint{3.034612in}{2.570479in}}%
\pgfpathlineto{\pgfqpoint{3.015128in}{2.320494in}}%
\pgfpathlineto{\pgfqpoint{2.994472in}{2.455701in}}%
\pgfpathlineto{\pgfqpoint{2.975459in}{2.566600in}}%
\pgfpathlineto{\pgfqpoint{2.957151in}{2.617734in}}%
\pgfpathlineto{\pgfqpoint{2.936729in}{2.538387in}}%
\pgfpathlineto{\pgfqpoint{2.917247in}{2.396845in}}%
\pgfpathlineto{\pgfqpoint{2.898937in}{2.311046in}}%
\pgfpathlineto{\pgfqpoint{2.876638in}{2.262167in}}%
\pgfpathlineto{\pgfqpoint{2.860913in}{2.247357in}}%
\pgfpathlineto{\pgfqpoint{2.840020in}{2.250991in}}%
\pgfpathlineto{\pgfqpoint{2.817487in}{2.277143in}}%
\pgfpathlineto{\pgfqpoint{2.802699in}{2.314101in}}%
\pgfpathlineto{\pgfqpoint{2.783685in}{2.396419in}}%
\pgfpathlineto{\pgfqpoint{2.762092in}{2.555495in}}%
\pgfpathlineto{\pgfqpoint{2.746833in}{2.612834in}}%
\pgfpathlineto{\pgfqpoint{2.726177in}{2.580555in}}%
\pgfpathlineto{\pgfqpoint{2.706226in}{2.443473in}}%
\pgfpathlineto{\pgfqpoint{2.688385in}{2.336183in}}%
\pgfpathlineto{\pgfqpoint{2.669373in}{2.283643in}}%
\pgfpathlineto{\pgfqpoint{2.648012in}{2.252187in}}%
\pgfpathlineto{\pgfqpoint{2.632050in}{2.244957in}}%
\pgfpathlineto{\pgfqpoint{2.610691in}{2.248502in}}%
\pgfpathlineto{\pgfqpoint{2.591912in}{2.267912in}}%
\pgfpathlineto{\pgfqpoint{2.570082in}{2.315514in}}%
\pgfpathlineto{\pgfqpoint{2.552243in}{2.370828in}}%
\pgfpathlineto{\pgfqpoint{2.533229in}{2.512575in}}%
\pgfpathlineto{\pgfqpoint{2.514216in}{2.605319in}}%
\pgfpathlineto{\pgfqpoint{2.495908in}{2.590926in}}%
\pgfpathlineto{\pgfqpoint{2.474547in}{2.477024in}}%
\pgfpathlineto{\pgfqpoint{2.456473in}{2.366664in}}%
\pgfpathlineto{\pgfqpoint{2.437694in}{2.297394in}}%
\pgfpathlineto{\pgfqpoint{2.419386in}{2.261814in}}%
\pgfpathlineto{\pgfqpoint{2.397556in}{2.245452in}}%
\pgfpathlineto{\pgfqpoint{2.379482in}{2.249219in}}%
\pgfpathlineto{\pgfqpoint{2.361172in}{2.266569in}}%
\pgfpathlineto{\pgfqpoint{2.339578in}{2.317807in}}%
\pgfpathlineto{\pgfqpoint{2.317983in}{2.417517in}}%
\pgfpathlineto{\pgfqpoint{2.302021in}{2.514667in}}%
\pgfpathlineto{\pgfqpoint{2.284182in}{2.594411in}}%
\pgfpathlineto{\pgfqpoint{2.264934in}{2.611694in}}%
\pgfpathlineto{\pgfqpoint{2.225499in}{2.456252in}}%
\pgfpathlineto{\pgfqpoint{2.205548in}{2.365638in}}%
\pgfpathlineto{\pgfqpoint{2.187472in}{2.302624in}}%
\pgfpathlineto{\pgfqpoint{2.169633in}{2.266932in}}%
\pgfpathlineto{\pgfqpoint{2.147569in}{2.246585in}}%
\pgfpathlineto{\pgfqpoint{2.129964in}{2.251689in}}%
\pgfpathlineto{\pgfqpoint{2.110482in}{2.268026in}}%
\pgfpathlineto{\pgfqpoint{2.092642in}{2.298949in}}%
\pgfpathlineto{\pgfqpoint{2.070578in}{2.389116in}}%
\pgfpathlineto{\pgfqpoint{2.052739in}{2.516293in}}%
\pgfpathlineto{\pgfqpoint{2.033725in}{2.550135in}}%
\pgfpathlineto{\pgfqpoint{2.015886in}{2.260900in}}%
\pgfpathlineto{\pgfqpoint{1.996873in}{2.298822in}}%
\pgfpathlineto{\pgfqpoint{1.976217in}{2.382183in}}%
\pgfpathlineto{\pgfqpoint{1.955326in}{2.548372in}}%
\pgfpathlineto{\pgfqpoint{1.937487in}{2.615577in}}%
\pgfpathlineto{\pgfqpoint{1.916596in}{2.581966in}}%
\pgfpathlineto{\pgfqpoint{1.898992in}{2.492858in}}%
\pgfpathlineto{\pgfqpoint{1.875987in}{2.347203in}}%
\pgfpathlineto{\pgfqpoint{1.861668in}{2.293238in}}%
\pgfpathlineto{\pgfqpoint{1.841952in}{2.311249in}}%
\pgfpathlineto{\pgfqpoint{1.824113in}{2.268753in}}%
\pgfpathlineto{\pgfqpoint{1.802986in}{2.250612in}}%
\pgfpathlineto{\pgfqpoint{1.784678in}{2.248090in}}%
\pgfpathlineto{\pgfqpoint{1.764727in}{2.267144in}}%
\pgfpathlineto{\pgfqpoint{1.744305in}{2.311452in}}%
\pgfpathlineto{\pgfqpoint{1.726229in}{2.376518in}}%
\pgfpathlineto{\pgfqpoint{1.707216in}{2.512524in}}%
\pgfpathlineto{\pgfqpoint{1.689142in}{2.596838in}}%
\pgfpathlineto{\pgfqpoint{1.670364in}{2.625971in}}%
\pgfpathlineto{\pgfqpoint{1.648535in}{2.554832in}}%
\pgfpathlineto{\pgfqpoint{1.630931in}{2.421994in}}%
\pgfpathlineto{\pgfqpoint{1.609569in}{2.322271in}}%
\pgfpathlineto{\pgfqpoint{1.592435in}{2.281336in}}%
\pgfpathlineto{\pgfqpoint{1.571779in}{2.254291in}}%
\pgfpathlineto{\pgfqpoint{1.554643in}{2.247165in}}%
\pgfpathlineto{\pgfqpoint{1.531639in}{2.259875in}}%
\pgfpathlineto{\pgfqpoint{1.515677in}{2.280881in}}%
\pgfpathlineto{\pgfqpoint{1.495257in}{2.333803in}}%
\pgfpathlineto{\pgfqpoint{1.474835in}{2.450695in}}%
\pgfpathlineto{\pgfqpoint{1.457465in}{2.508068in}}%
\pgfpathlineto{\pgfqpoint{1.437512in}{2.611740in}}%
\pgfpathlineto{\pgfqpoint{1.419204in}{2.631690in}}%
\pgfpathlineto{\pgfqpoint{1.398548in}{2.577011in}}%
\pgfpathlineto{\pgfqpoint{1.381412in}{2.500148in}}%
\pgfpathlineto{\pgfqpoint{1.361227in}{2.377654in}}%
\pgfpathlineto{\pgfqpoint{1.337519in}{2.298019in}}%
\pgfpathlineto{\pgfqpoint{1.323200in}{2.271644in}}%
\pgfpathlineto{\pgfqpoint{1.302544in}{2.251248in}}%
\pgfpathlineto{\pgfqpoint{1.285408in}{2.249565in}}%
\pgfpathlineto{\pgfqpoint{1.265221in}{2.262088in}}%
\pgfpathlineto{\pgfqpoint{1.247147in}{2.279926in}}%
\pgfpathlineto{\pgfqpoint{1.226257in}{2.332279in}}%
\pgfpathlineto{\pgfqpoint{1.207948in}{2.448907in}}%
\pgfpathlineto{\pgfqpoint{1.187761in}{2.560462in}}%
\pgfpathlineto{\pgfqpoint{1.168043in}{2.634756in}}%
\pgfpathlineto{\pgfqpoint{1.150440in}{2.641011in}}%
\pgfpathlineto{\pgfqpoint{1.132130in}{2.580943in}}%
\pgfpathlineto{\pgfqpoint{1.111943in}{2.459576in}}%
\pgfpathlineto{\pgfqpoint{1.091992in}{2.350214in}}%
\pgfpathlineto{\pgfqpoint{1.074152in}{2.311127in}}%
\pgfpathlineto{\pgfqpoint{1.053965in}{2.278766in}}%
\pgfpathlineto{\pgfqpoint{1.033075in}{2.254478in}}%
\pgfpathlineto{\pgfqpoint{1.015704in}{2.250886in}}%
\pgfpathlineto{\pgfqpoint{0.997631in}{2.265543in}}%
\pgfpathlineto{\pgfqpoint{0.977209in}{2.274346in}}%
\pgfpathlineto{\pgfqpoint{0.956318in}{2.318191in}}%
\pgfpathlineto{\pgfqpoint{0.938948in}{2.391296in}}%
\pgfpathlineto{\pgfqpoint{0.902799in}{2.620107in}}%
\pgfpathlineto{\pgfqpoint{0.882613in}{2.652932in}}%
\pgfpathlineto{\pgfqpoint{0.861723in}{2.624753in}}%
\pgfpathlineto{\pgfqpoint{0.839893in}{2.559079in}}%
\pgfpathlineto{\pgfqpoint{0.823462in}{2.437763in}}%
\pgfpathlineto{\pgfqpoint{0.802569in}{2.339568in}}%
\pgfpathlineto{\pgfqpoint{0.784964in}{2.295782in}}%
\pgfpathlineto{\pgfqpoint{0.763371in}{2.266028in}}%
\pgfpathlineto{\pgfqpoint{0.746235in}{2.252946in}}%
\pgfpathlineto{\pgfqpoint{0.728395in}{2.290069in}}%
\pgfpathlineto{\pgfqpoint{0.707505in}{2.264562in}}%
\pgfpathlineto{\pgfqpoint{0.688257in}{2.252661in}}%
\pgfpathlineto{\pgfqpoint{0.670418in}{2.266746in}}%
\pgfpathlineto{\pgfqpoint{0.649291in}{2.294417in}}%
\pgfpathlineto{\pgfqpoint{0.650231in}{2.296364in}}%
\pgfpathlineto{\pgfqpoint{0.657742in}{2.273296in}}%
\pgfpathlineto{\pgfqpoint{0.675112in}{2.253202in}}%
\pgfpathlineto{\pgfqpoint{0.696942in}{2.276841in}}%
\pgfpathlineto{\pgfqpoint{0.712904in}{2.320418in}}%
\pgfpathlineto{\pgfqpoint{0.734498in}{2.441060in}}%
\pgfpathlineto{\pgfqpoint{0.752808in}{2.604486in}}%
\pgfpathlineto{\pgfqpoint{0.772056in}{2.654907in}}%
\pgfpathlineto{\pgfqpoint{0.792241in}{2.549571in}}%
\pgfpathlineto{\pgfqpoint{0.809612in}{2.399161in}}%
\pgfpathlineto{\pgfqpoint{0.828156in}{2.304388in}}%
\pgfpathlineto{\pgfqpoint{0.850221in}{2.258082in}}%
\pgfpathlineto{\pgfqpoint{0.868060in}{2.253262in}}%
\pgfpathlineto{\pgfqpoint{0.886368in}{2.280635in}}%
\pgfpathlineto{\pgfqpoint{0.905147in}{2.342563in}}%
\pgfpathlineto{\pgfqpoint{0.926037in}{2.489099in}}%
\pgfpathlineto{\pgfqpoint{0.944347in}{2.628242in}}%
\pgfpathlineto{\pgfqpoint{0.961952in}{2.631604in}}%
\pgfpathlineto{\pgfqpoint{0.980026in}{2.496852in}}%
\pgfpathlineto{\pgfqpoint{1.001621in}{2.353023in}}%
\pgfpathlineto{\pgfqpoint{1.022041in}{2.276415in}}%
\pgfpathlineto{\pgfqpoint{1.040586in}{2.250903in}}%
\pgfpathlineto{\pgfqpoint{1.058425in}{2.256873in}}%
\pgfpathlineto{\pgfqpoint{1.080255in}{2.296224in}}%
\pgfpathlineto{\pgfqpoint{1.097155in}{2.368382in}}%
\pgfpathlineto{\pgfqpoint{1.117811in}{2.527521in}}%
\pgfpathlineto{\pgfqpoint{1.135886in}{2.631818in}}%
\pgfpathlineto{\pgfqpoint{1.156308in}{2.584096in}}%
\pgfpathlineto{\pgfqpoint{1.175556in}{2.425409in}}%
\pgfpathlineto{\pgfqpoint{1.195272in}{2.311860in}}%
\pgfpathlineto{\pgfqpoint{1.213580in}{2.266262in}}%
\pgfpathlineto{\pgfqpoint{1.232125in}{2.247695in}}%
\pgfpathlineto{\pgfqpoint{1.251607in}{2.261778in}}%
\pgfpathlineto{\pgfqpoint{1.271089in}{2.296413in}}%
\pgfpathlineto{\pgfqpoint{1.290339in}{2.370991in}}%
\pgfpathlineto{\pgfqpoint{1.308881in}{2.512911in}}%
\pgfpathlineto{\pgfqpoint{1.329303in}{2.623945in}}%
\pgfpathlineto{\pgfqpoint{1.347376in}{2.607674in}}%
\pgfpathlineto{\pgfqpoint{1.368503in}{2.447711in}}%
\pgfpathlineto{\pgfqpoint{1.386106in}{2.333192in}}%
\pgfpathlineto{\pgfqpoint{1.405120in}{2.277240in}}%
\pgfpathlineto{\pgfqpoint{1.425541in}{2.250102in}}%
\pgfpathlineto{\pgfqpoint{1.446432in}{2.251654in}}%
\pgfpathlineto{\pgfqpoint{1.481407in}{2.295401in}}%
\pgfpathlineto{\pgfqpoint{1.502298in}{2.391280in}}%
\pgfpathlineto{\pgfqpoint{1.520608in}{2.540238in}}%
\pgfpathlineto{\pgfqpoint{1.542672in}{2.623610in}}%
\pgfpathlineto{\pgfqpoint{1.560277in}{2.578079in}}%
\pgfpathlineto{\pgfqpoint{1.597364in}{2.352357in}}%
\pgfpathlineto{\pgfqpoint{1.615672in}{2.285922in}}%
\pgfpathlineto{\pgfqpoint{1.637033in}{2.601663in}}%
\pgfpathlineto{\pgfqpoint{1.673417in}{2.334756in}}%
\pgfpathlineto{\pgfqpoint{1.691491in}{2.270727in}}%
\pgfpathlineto{\pgfqpoint{1.713084in}{2.249295in}}%
\pgfpathlineto{\pgfqpoint{1.732803in}{2.250119in}}%
\pgfpathlineto{\pgfqpoint{1.753693in}{2.278619in}}%
\pgfpathlineto{\pgfqpoint{1.772238in}{2.334627in}}%
\pgfpathlineto{\pgfqpoint{1.811436in}{2.580014in}}%
\pgfpathlineto{\pgfqpoint{1.829276in}{2.615707in}}%
\pgfpathlineto{\pgfqpoint{1.847115in}{2.548967in}}%
\pgfpathlineto{\pgfqpoint{1.868007in}{2.398865in}}%
\pgfpathlineto{\pgfqpoint{1.885847in}{2.308969in}}%
\pgfpathlineto{\pgfqpoint{1.903920in}{2.266957in}}%
\pgfpathlineto{\pgfqpoint{1.925516in}{2.245600in}}%
\pgfpathlineto{\pgfqpoint{1.943121in}{2.249549in}}%
\pgfpathlineto{\pgfqpoint{1.964246in}{2.277854in}}%
\pgfpathlineto{\pgfqpoint{1.982319in}{2.325584in}}%
\pgfpathlineto{\pgfqpoint{2.003210in}{2.452760in}}%
\pgfpathlineto{\pgfqpoint{2.020580in}{2.581430in}}%
\pgfpathlineto{\pgfqpoint{2.041707in}{2.609944in}}%
\pgfpathlineto{\pgfqpoint{2.060719in}{2.517448in}}%
\pgfpathlineto{\pgfqpoint{2.078560in}{2.381424in}}%
\pgfpathlineto{\pgfqpoint{2.095459in}{2.301635in}}%
\pgfpathlineto{\pgfqpoint{2.117758in}{2.269202in}}%
\pgfpathlineto{\pgfqpoint{2.137711in}{2.248668in}}%
\pgfpathlineto{\pgfqpoint{2.156488in}{2.246127in}}%
\pgfpathlineto{\pgfqpoint{2.174093in}{2.262360in}}%
\pgfpathlineto{\pgfqpoint{2.195220in}{2.305606in}}%
\pgfpathlineto{\pgfqpoint{2.213293in}{2.382853in}}%
\pgfpathlineto{\pgfqpoint{2.231133in}{2.501796in}}%
\pgfpathlineto{\pgfqpoint{2.252023in}{2.609422in}}%
\pgfpathlineto{\pgfqpoint{2.269159in}{2.589778in}}%
\pgfpathlineto{\pgfqpoint{2.309766in}{2.330721in}}%
\pgfpathlineto{\pgfqpoint{2.327137in}{2.278618in}}%
\pgfpathlineto{\pgfqpoint{2.347793in}{2.252210in}}%
\pgfpathlineto{\pgfqpoint{2.365868in}{2.243362in}}%
\pgfpathlineto{\pgfqpoint{2.387697in}{2.253736in}}%
\pgfpathlineto{\pgfqpoint{2.405067in}{2.278565in}}%
\pgfpathlineto{\pgfqpoint{2.423141in}{2.281899in}}%
\pgfpathlineto{\pgfqpoint{2.440511in}{2.314944in}}%
\pgfpathlineto{\pgfqpoint{2.462341in}{2.434466in}}%
\pgfpathlineto{\pgfqpoint{2.479477in}{2.565162in}}%
\pgfpathlineto{\pgfqpoint{2.503888in}{2.603982in}}%
\pgfpathlineto{\pgfqpoint{2.519615in}{2.585632in}}%
\pgfpathlineto{\pgfqpoint{2.543557in}{2.424206in}}%
\pgfpathlineto{\pgfqpoint{2.559754in}{2.330556in}}%
\pgfpathlineto{\pgfqpoint{2.576890in}{2.275842in}}%
\pgfpathlineto{\pgfqpoint{2.598015in}{2.249834in}}%
\pgfpathlineto{\pgfqpoint{2.615619in}{2.244366in}}%
\pgfpathlineto{\pgfqpoint{2.638153in}{2.261138in}}%
\pgfpathlineto{\pgfqpoint{2.652941in}{2.292098in}}%
\pgfpathlineto{\pgfqpoint{2.672894in}{2.350377in}}%
\pgfpathlineto{\pgfqpoint{2.711858in}{2.589551in}}%
\pgfpathlineto{\pgfqpoint{2.732750in}{2.605800in}}%
\pgfpathlineto{\pgfqpoint{2.750824in}{2.511191in}}%
\pgfpathlineto{\pgfqpoint{2.771480in}{2.611965in}}%
\pgfpathlineto{\pgfqpoint{2.790023in}{2.586126in}}%
\pgfpathlineto{\pgfqpoint{2.807627in}{2.459356in}}%
\pgfpathlineto{\pgfqpoint{2.829223in}{2.465665in}}%
\pgfpathlineto{\pgfqpoint{2.847297in}{2.355627in}}%
\pgfpathlineto{\pgfqpoint{2.865607in}{2.286302in}}%
\pgfpathlineto{\pgfqpoint{2.882977in}{2.255076in}}%
\pgfpathlineto{\pgfqpoint{2.905511in}{2.246203in}}%
\pgfpathlineto{\pgfqpoint{2.923584in}{2.247965in}}%
\pgfpathlineto{\pgfqpoint{2.943772in}{2.271856in}}%
\pgfpathlineto{\pgfqpoint{2.962080in}{2.314843in}}%
\pgfpathlineto{\pgfqpoint{2.982970in}{2.351540in}}%
\pgfpathlineto{\pgfqpoint{3.019119in}{2.583959in}}%
\pgfpathlineto{\pgfqpoint{3.040479in}{2.609969in}}%
\pgfpathlineto{\pgfqpoint{3.057615in}{2.512558in}}%
\pgfpathlineto{\pgfqpoint{3.076159in}{2.381258in}}%
\pgfpathlineto{\pgfqpoint{3.097284in}{2.292062in}}%
\pgfpathlineto{\pgfqpoint{3.115829in}{2.260231in}}%
\pgfpathlineto{\pgfqpoint{3.134137in}{2.249866in}}%
\pgfpathlineto{\pgfqpoint{3.154558in}{2.246047in}}%
\pgfpathlineto{\pgfqpoint{3.173337in}{2.262623in}}%
\pgfpathlineto{\pgfqpoint{3.194697in}{2.301782in}}%
\pgfpathlineto{\pgfqpoint{3.212536in}{2.365570in}}%
\pgfpathlineto{\pgfqpoint{3.232958in}{2.501554in}}%
\pgfpathlineto{\pgfqpoint{3.251502in}{2.601981in}}%
\pgfpathlineto{\pgfqpoint{3.269576in}{2.623255in}}%
\pgfpathlineto{\pgfqpoint{3.291406in}{2.548969in}}%
\pgfpathlineto{\pgfqpoint{3.308305in}{2.448405in}}%
\pgfpathlineto{\pgfqpoint{3.325910in}{2.350917in}}%
\pgfpathlineto{\pgfqpoint{3.347506in}{2.283439in}}%
\pgfpathlineto{\pgfqpoint{3.366048in}{2.257165in}}%
\pgfpathlineto{\pgfqpoint{3.386236in}{2.246357in}}%
\pgfpathlineto{\pgfqpoint{3.405015in}{2.252409in}}%
\pgfpathlineto{\pgfqpoint{3.423088in}{2.264274in}}%
\pgfpathlineto{\pgfqpoint{3.441633in}{2.297440in}}%
\pgfpathlineto{\pgfqpoint{3.462054in}{2.346088in}}%
\pgfpathlineto{\pgfqpoint{3.480597in}{2.443567in}}%
\pgfpathlineto{\pgfqpoint{3.498670in}{2.581911in}}%
\pgfpathlineto{\pgfqpoint{3.520032in}{2.630421in}}%
\pgfpathlineto{\pgfqpoint{3.538340in}{2.589946in}}%
\pgfpathlineto{\pgfqpoint{3.557353in}{2.479659in}}%
\pgfpathlineto{\pgfqpoint{3.580123in}{2.360170in}}%
\pgfpathlineto{\pgfqpoint{3.594676in}{2.305143in}}%
\pgfpathlineto{\pgfqpoint{3.615333in}{2.271337in}}%
\pgfpathlineto{\pgfqpoint{3.633641in}{2.254263in}}%
\pgfpathlineto{\pgfqpoint{3.653593in}{2.246910in}}%
\pgfpathlineto{\pgfqpoint{3.674484in}{2.258230in}}%
\pgfpathlineto{\pgfqpoint{3.693732in}{2.285392in}}%
\pgfpathlineto{\pgfqpoint{3.729410in}{2.384400in}}%
\pgfpathlineto{\pgfqpoint{3.751475in}{2.503362in}}%
\pgfpathlineto{\pgfqpoint{3.769785in}{2.609599in}}%
\pgfpathlineto{\pgfqpoint{3.789267in}{2.638854in}}%
\pgfpathlineto{\pgfqpoint{3.808046in}{2.598925in}}%
\pgfpathlineto{\pgfqpoint{3.827528in}{2.487515in}}%
\pgfpathlineto{\pgfqpoint{3.847010in}{2.521030in}}%
\pgfpathlineto{\pgfqpoint{3.866258in}{2.642088in}}%
\pgfpathlineto{\pgfqpoint{3.884802in}{2.589284in}}%
\pgfpathlineto{\pgfqpoint{3.904519in}{2.458062in}}%
\pgfpathlineto{\pgfqpoint{3.923532in}{2.349030in}}%
\pgfpathlineto{\pgfqpoint{3.943014in}{2.300148in}}%
\pgfpathlineto{\pgfqpoint{3.961324in}{2.265506in}}%
\pgfpathlineto{\pgfqpoint{3.980806in}{2.250456in}}%
\pgfpathlineto{\pgfqpoint{3.999585in}{2.252995in}}%
\pgfpathlineto{\pgfqpoint{4.020005in}{2.273496in}}%
\pgfpathlineto{\pgfqpoint{4.037609in}{2.307533in}}%
\pgfpathlineto{\pgfqpoint{4.056857in}{2.373218in}}%
\pgfpathlineto{\pgfqpoint{4.076105in}{2.465760in}}%
\pgfpathlineto{\pgfqpoint{4.094415in}{2.600535in}}%
\pgfpathlineto{\pgfqpoint{4.114602in}{2.653201in}}%
\pgfpathlineto{\pgfqpoint{4.132676in}{2.627392in}}%
\pgfpathlineto{\pgfqpoint{4.154975in}{2.516509in}}%
\pgfpathlineto{\pgfqpoint{4.170702in}{2.418605in}}%
\pgfpathlineto{\pgfqpoint{4.192062in}{2.320704in}}%
\pgfpathlineto{\pgfqpoint{4.211780in}{2.283677in}}%
\pgfpathlineto{\pgfqpoint{4.230323in}{2.264049in}}%
\pgfpathlineto{\pgfqpoint{4.248162in}{2.251342in}}%
\pgfpathlineto{\pgfqpoint{4.266001in}{2.258774in}}%
\pgfpathlineto{\pgfqpoint{4.290414in}{2.278835in}}%
\pgfpathlineto{\pgfqpoint{4.308958in}{2.322079in}}%
\pgfpathlineto{\pgfqpoint{4.327266in}{2.380066in}}%
\pgfpathlineto{\pgfqpoint{4.346514in}{2.463214in}}%
\pgfpathlineto{\pgfqpoint{4.365293in}{2.601546in}}%
\pgfpathlineto{\pgfqpoint{4.384306in}{2.661443in}}%
\pgfpathlineto{\pgfqpoint{4.402848in}{2.653977in}}%
\pgfpathlineto{\pgfqpoint{4.421862in}{2.597055in}}%
\pgfpathlineto{\pgfqpoint{4.462471in}{2.350737in}}%
\pgfpathlineto{\pgfqpoint{4.482422in}{2.303919in}}%
\pgfpathlineto{\pgfqpoint{4.474676in}{2.330092in}}%
\pgfpathlineto{\pgfqpoint{4.455663in}{2.461569in}}%
\pgfpathlineto{\pgfqpoint{4.432190in}{2.644566in}}%
\pgfpathlineto{\pgfqpoint{4.416933in}{2.664407in}}%
\pgfpathlineto{\pgfqpoint{4.393694in}{2.508161in}}%
\pgfpathlineto{\pgfqpoint{4.377264in}{2.380699in}}%
\pgfpathlineto{\pgfqpoint{4.356373in}{2.291905in}}%
\pgfpathlineto{\pgfqpoint{4.338534in}{2.257777in}}%
\pgfpathlineto{\pgfqpoint{4.319989in}{2.254855in}}%
\pgfpathlineto{\pgfqpoint{4.302385in}{2.286249in}}%
\pgfpathlineto{\pgfqpoint{4.276329in}{2.414339in}}%
\pgfpathlineto{\pgfqpoint{4.262012in}{2.547728in}}%
\pgfpathlineto{\pgfqpoint{4.242528in}{2.650851in}}%
\pgfpathlineto{\pgfqpoint{4.224220in}{2.615352in}}%
\pgfpathlineto{\pgfqpoint{4.205207in}{2.451254in}}%
\pgfpathlineto{\pgfqpoint{4.184082in}{2.317713in}}%
\pgfpathlineto{\pgfqpoint{4.165303in}{2.266855in}}%
\pgfpathlineto{\pgfqpoint{4.146758in}{2.249329in}}%
\pgfpathlineto{\pgfqpoint{4.127982in}{2.265849in}}%
\pgfpathlineto{\pgfqpoint{4.108968in}{2.314323in}}%
\pgfpathlineto{\pgfqpoint{4.090893in}{2.423183in}}%
\pgfpathlineto{\pgfqpoint{4.069299in}{2.605826in}}%
\pgfpathlineto{\pgfqpoint{4.050051in}{2.641596in}}%
\pgfpathlineto{\pgfqpoint{4.034558in}{2.558674in}}%
\pgfpathlineto{\pgfqpoint{4.012964in}{2.377707in}}%
\pgfpathlineto{\pgfqpoint{3.991603in}{2.287723in}}%
\pgfpathlineto{\pgfqpoint{3.975641in}{2.258679in}}%
\pgfpathlineto{\pgfqpoint{3.957333in}{2.248198in}}%
\pgfpathlineto{\pgfqpoint{3.936442in}{2.274309in}}%
\pgfpathlineto{\pgfqpoint{3.916724in}{2.336187in}}%
\pgfpathlineto{\pgfqpoint{3.875646in}{2.623692in}}%
\pgfpathlineto{\pgfqpoint{3.858041in}{2.613842in}}%
\pgfpathlineto{\pgfqpoint{3.842316in}{2.492787in}}%
\pgfpathlineto{\pgfqpoint{3.817669in}{2.321835in}}%
\pgfpathlineto{\pgfqpoint{3.801472in}{2.282690in}}%
\pgfpathlineto{\pgfqpoint{3.783633in}{2.252453in}}%
\pgfpathlineto{\pgfqpoint{3.766028in}{2.248335in}}%
\pgfpathlineto{\pgfqpoint{3.743258in}{2.276791in}}%
\pgfpathlineto{\pgfqpoint{3.724247in}{2.337001in}}%
\pgfpathlineto{\pgfqpoint{3.687158in}{2.605288in}}%
\pgfpathlineto{\pgfqpoint{3.668850in}{2.627046in}}%
\pgfpathlineto{\pgfqpoint{3.646786in}{2.513221in}}%
\pgfpathlineto{\pgfqpoint{3.627772in}{2.374917in}}%
\pgfpathlineto{\pgfqpoint{3.610402in}{2.301779in}}%
\pgfpathlineto{\pgfqpoint{3.588574in}{2.263795in}}%
\pgfpathlineto{\pgfqpoint{3.572377in}{2.246586in}}%
\pgfpathlineto{\pgfqpoint{3.551016in}{2.249752in}}%
\pgfpathlineto{\pgfqpoint{3.531768in}{2.272162in}}%
\pgfpathlineto{\pgfqpoint{3.513224in}{2.321977in}}%
\pgfpathlineto{\pgfqpoint{3.494916in}{2.424905in}}%
\pgfpathlineto{\pgfqpoint{3.474494in}{2.592570in}}%
\pgfpathlineto{\pgfqpoint{3.455715in}{2.622534in}}%
\pgfpathlineto{\pgfqpoint{3.436702in}{2.545541in}}%
\pgfpathlineto{\pgfqpoint{3.417925in}{2.401111in}}%
\pgfpathlineto{\pgfqpoint{3.398912in}{2.316111in}}%
\pgfpathlineto{\pgfqpoint{3.380133in}{2.270063in}}%
\pgfpathlineto{\pgfqpoint{3.359477in}{2.245931in}}%
\pgfpathlineto{\pgfqpoint{3.339760in}{2.248412in}}%
\pgfpathlineto{\pgfqpoint{3.320513in}{2.270766in}}%
\pgfpathlineto{\pgfqpoint{3.302437in}{2.321929in}}%
\pgfpathlineto{\pgfqpoint{3.280138in}{2.450138in}}%
\pgfpathlineto{\pgfqpoint{3.261596in}{2.568595in}}%
\pgfpathlineto{\pgfqpoint{3.242817in}{2.617173in}}%
\pgfpathlineto{\pgfqpoint{3.224507in}{2.554237in}}%
\pgfpathlineto{\pgfqpoint{3.205730in}{2.428764in}}%
\pgfpathlineto{\pgfqpoint{3.186951in}{2.332725in}}%
\pgfpathlineto{\pgfqpoint{3.168877in}{2.288562in}}%
\pgfpathlineto{\pgfqpoint{3.147047in}{2.253453in}}%
\pgfpathlineto{\pgfqpoint{3.128737in}{2.244178in}}%
\pgfpathlineto{\pgfqpoint{3.109021in}{2.257086in}}%
\pgfpathlineto{\pgfqpoint{3.091416in}{2.286794in}}%
\pgfpathlineto{\pgfqpoint{3.072871in}{2.360863in}}%
\pgfpathlineto{\pgfqpoint{3.036019in}{2.607876in}}%
\pgfpathlineto{\pgfqpoint{3.013720in}{2.608122in}}%
\pgfpathlineto{\pgfqpoint{2.995646in}{2.591524in}}%
\pgfpathlineto{\pgfqpoint{2.976633in}{2.465928in}}%
\pgfpathlineto{\pgfqpoint{2.957151in}{2.347500in}}%
\pgfpathlineto{\pgfqpoint{2.939077in}{2.288177in}}%
\pgfpathlineto{\pgfqpoint{2.916778in}{2.254613in}}%
\pgfpathlineto{\pgfqpoint{2.899642in}{2.245237in}}%
\pgfpathlineto{\pgfqpoint{2.877812in}{2.248156in}}%
\pgfpathlineto{\pgfqpoint{2.859504in}{2.262791in}}%
\pgfpathlineto{\pgfqpoint{2.840020in}{2.285212in}}%
\pgfpathlineto{\pgfqpoint{2.823120in}{2.342448in}}%
\pgfpathlineto{\pgfqpoint{2.802933in}{2.482439in}}%
\pgfpathlineto{\pgfqpoint{2.782511in}{2.596174in}}%
\pgfpathlineto{\pgfqpoint{2.763735in}{2.603506in}}%
\pgfpathlineto{\pgfqpoint{2.747067in}{2.530819in}}%
\pgfpathlineto{\pgfqpoint{2.725474in}{2.388370in}}%
\pgfpathlineto{\pgfqpoint{2.706460in}{2.307690in}}%
\pgfpathlineto{\pgfqpoint{2.687916in}{2.297256in}}%
\pgfpathlineto{\pgfqpoint{2.688150in}{2.272861in}}%
\pgfpathlineto{\pgfqpoint{2.670311in}{2.264255in}}%
\pgfpathlineto{\pgfqpoint{2.647543in}{2.245363in}}%
\pgfpathlineto{\pgfqpoint{2.628764in}{2.249147in}}%
\pgfpathlineto{\pgfqpoint{2.610925in}{2.269097in}}%
\pgfpathlineto{\pgfqpoint{2.592146in}{2.302756in}}%
\pgfpathlineto{\pgfqpoint{2.575247in}{2.386467in}}%
\pgfpathlineto{\pgfqpoint{2.551539in}{2.483314in}}%
\pgfpathlineto{\pgfqpoint{2.532995in}{2.593089in}}%
\pgfpathlineto{\pgfqpoint{2.514216in}{2.599267in}}%
\pgfpathlineto{\pgfqpoint{2.497551in}{2.509048in}}%
\pgfpathlineto{\pgfqpoint{2.474078in}{2.356499in}}%
\pgfpathlineto{\pgfqpoint{2.457647in}{2.293149in}}%
\pgfpathlineto{\pgfqpoint{2.437460in}{2.258586in}}%
\pgfpathlineto{\pgfqpoint{2.419152in}{2.258469in}}%
\pgfpathlineto{\pgfqpoint{2.397321in}{2.246771in}}%
\pgfpathlineto{\pgfqpoint{2.378777in}{2.245416in}}%
\pgfpathlineto{\pgfqpoint{2.361172in}{2.261542in}}%
\pgfpathlineto{\pgfqpoint{2.338873in}{2.290041in}}%
\pgfpathlineto{\pgfqpoint{2.322911in}{2.347735in}}%
\pgfpathlineto{\pgfqpoint{2.301083in}{2.508962in}}%
\pgfpathlineto{\pgfqpoint{2.279018in}{2.600577in}}%
\pgfpathlineto{\pgfqpoint{2.263760in}{2.607105in}}%
\pgfpathlineto{\pgfqpoint{2.245921in}{2.547142in}}%
\pgfpathlineto{\pgfqpoint{2.227142in}{2.445463in}}%
\pgfpathlineto{\pgfqpoint{2.208365in}{2.336364in}}%
\pgfpathlineto{\pgfqpoint{2.187004in}{2.280396in}}%
\pgfpathlineto{\pgfqpoint{2.168695in}{2.257282in}}%
\pgfpathlineto{\pgfqpoint{2.149917in}{2.247869in}}%
\pgfpathlineto{\pgfqpoint{2.131843in}{2.246186in}}%
\pgfpathlineto{\pgfqpoint{2.110482in}{2.265416in}}%
\pgfpathlineto{\pgfqpoint{2.092174in}{2.299745in}}%
\pgfpathlineto{\pgfqpoint{2.073160in}{2.378482in}}%
\pgfpathlineto{\pgfqpoint{2.050861in}{2.500425in}}%
\pgfpathlineto{\pgfqpoint{2.033022in}{2.587472in}}%
\pgfpathlineto{\pgfqpoint{2.014946in}{2.615812in}}%
\pgfpathlineto{\pgfqpoint{1.995699in}{2.596504in}}%
\pgfpathlineto{\pgfqpoint{1.975748in}{2.487969in}}%
\pgfpathlineto{\pgfqpoint{1.956969in}{2.358753in}}%
\pgfpathlineto{\pgfqpoint{1.938190in}{2.292577in}}%
\pgfpathlineto{\pgfqpoint{1.921056in}{2.278304in}}%
\pgfpathlineto{\pgfqpoint{1.897112in}{2.470788in}}%
\pgfpathlineto{\pgfqpoint{1.878804in}{2.347865in}}%
\pgfpathlineto{\pgfqpoint{1.860731in}{2.288130in}}%
\pgfpathlineto{\pgfqpoint{1.841717in}{2.255167in}}%
\pgfpathlineto{\pgfqpoint{1.820827in}{2.244944in}}%
\pgfpathlineto{\pgfqpoint{1.801814in}{2.252858in}}%
\pgfpathlineto{\pgfqpoint{1.783974in}{2.277515in}}%
\pgfpathlineto{\pgfqpoint{1.765664in}{2.333730in}}%
\pgfpathlineto{\pgfqpoint{1.744539in}{2.458432in}}%
\pgfpathlineto{\pgfqpoint{1.725760in}{2.586409in}}%
\pgfpathlineto{\pgfqpoint{1.706982in}{2.616275in}}%
\pgfpathlineto{\pgfqpoint{1.688439in}{2.620092in}}%
\pgfpathlineto{\pgfqpoint{1.670364in}{2.552947in}}%
\pgfpathlineto{\pgfqpoint{1.650881in}{2.420981in}}%
\pgfpathlineto{\pgfqpoint{1.627408in}{2.323594in}}%
\pgfpathlineto{\pgfqpoint{1.612855in}{2.286456in}}%
\pgfpathlineto{\pgfqpoint{1.591730in}{2.263959in}}%
\pgfpathlineto{\pgfqpoint{1.574360in}{2.248955in}}%
\pgfpathlineto{\pgfqpoint{1.554174in}{2.247880in}}%
\pgfpathlineto{\pgfqpoint{1.533753in}{2.266308in}}%
\pgfpathlineto{\pgfqpoint{1.515913in}{2.300475in}}%
\pgfpathlineto{\pgfqpoint{1.496195in}{2.378649in}}%
\pgfpathlineto{\pgfqpoint{1.475070in}{2.518103in}}%
\pgfpathlineto{\pgfqpoint{1.457699in}{2.605413in}}%
\pgfpathlineto{\pgfqpoint{1.438217in}{2.629660in}}%
\pgfpathlineto{\pgfqpoint{1.420144in}{2.595455in}}%
\pgfpathlineto{\pgfqpoint{1.420378in}{2.502533in}}%
\pgfpathlineto{\pgfqpoint{1.396434in}{2.434211in}}%
\pgfpathlineto{\pgfqpoint{1.380943in}{2.357019in}}%
\pgfpathlineto{\pgfqpoint{1.360756in}{2.294587in}}%
\pgfpathlineto{\pgfqpoint{1.343151in}{2.269466in}}%
\pgfpathlineto{\pgfqpoint{1.324374in}{2.251213in}}%
\pgfpathlineto{\pgfqpoint{1.305595in}{2.247653in}}%
\pgfpathlineto{\pgfqpoint{1.283765in}{2.262238in}}%
\pgfpathlineto{\pgfqpoint{1.265926in}{2.288730in}}%
\pgfpathlineto{\pgfqpoint{1.245270in}{2.350368in}}%
\pgfpathlineto{\pgfqpoint{1.208417in}{2.554389in}}%
\pgfpathlineto{\pgfqpoint{1.188464in}{2.630742in}}%
\pgfpathlineto{\pgfqpoint{1.168748in}{2.638589in}}%
\pgfpathlineto{\pgfqpoint{1.147152in}{2.617889in}}%
\pgfpathlineto{\pgfqpoint{1.133069in}{2.534503in}}%
\pgfpathlineto{\pgfqpoint{1.094809in}{2.343033in}}%
\pgfpathlineto{\pgfqpoint{1.073682in}{2.290209in}}%
\pgfpathlineto{\pgfqpoint{1.053731in}{2.262504in}}%
\pgfpathlineto{\pgfqpoint{1.032606in}{2.248785in}}%
\pgfpathlineto{\pgfqpoint{1.015001in}{2.254056in}}%
\pgfpathlineto{\pgfqpoint{0.997631in}{2.256943in}}%
\pgfpathlineto{\pgfqpoint{0.976740in}{2.281434in}}%
\pgfpathlineto{\pgfqpoint{0.959135in}{2.319749in}}%
\pgfpathlineto{\pgfqpoint{0.938479in}{2.415536in}}%
\pgfpathlineto{\pgfqpoint{0.917823in}{2.552392in}}%
\pgfpathlineto{\pgfqpoint{0.899513in}{2.635262in}}%
\pgfpathlineto{\pgfqpoint{0.882142in}{2.651472in}}%
\pgfpathlineto{\pgfqpoint{0.861723in}{2.592880in}}%
\pgfpathlineto{\pgfqpoint{0.842239in}{2.507925in}}%
\pgfpathlineto{\pgfqpoint{0.824870in}{2.389325in}}%
\pgfpathlineto{\pgfqpoint{0.805386in}{2.347927in}}%
\pgfpathlineto{\pgfqpoint{0.784027in}{2.291472in}}%
\pgfpathlineto{\pgfqpoint{0.765014in}{2.266414in}}%
\pgfpathlineto{\pgfqpoint{0.746235in}{2.340896in}}%
\pgfpathlineto{\pgfqpoint{0.726284in}{2.298490in}}%
\pgfpathlineto{\pgfqpoint{0.708208in}{2.264618in}}%
\pgfpathlineto{\pgfqpoint{0.690603in}{2.251402in}}%
\pgfpathlineto{\pgfqpoint{0.669010in}{2.263838in}}%
\pgfpathlineto{\pgfqpoint{0.648588in}{2.295523in}}%
\pgfpathlineto{\pgfqpoint{0.651405in}{2.291066in}}%
\pgfpathlineto{\pgfqpoint{0.656802in}{2.275455in}}%
\pgfpathlineto{\pgfqpoint{0.675815in}{2.251158in}}%
\pgfpathlineto{\pgfqpoint{0.695534in}{2.268705in}}%
\pgfpathlineto{\pgfqpoint{0.715016in}{2.317293in}}%
\pgfpathlineto{\pgfqpoint{0.732855in}{2.415837in}}%
\pgfpathlineto{\pgfqpoint{0.751869in}{2.584478in}}%
\pgfpathlineto{\pgfqpoint{0.771351in}{2.654755in}}%
\pgfpathlineto{\pgfqpoint{0.789424in}{2.587992in}}%
\pgfpathlineto{\pgfqpoint{0.809612in}{2.402748in}}%
\pgfpathlineto{\pgfqpoint{0.829094in}{2.306950in}}%
\pgfpathlineto{\pgfqpoint{0.850455in}{2.257545in}}%
\pgfpathlineto{\pgfqpoint{0.867120in}{2.250897in}}%
\pgfpathlineto{\pgfqpoint{0.888247in}{2.280477in}}%
\pgfpathlineto{\pgfqpoint{0.906555in}{2.338003in}}%
\pgfpathlineto{\pgfqpoint{0.925100in}{2.460351in}}%
\pgfpathlineto{\pgfqpoint{0.945990in}{2.622545in}}%
\pgfpathlineto{\pgfqpoint{0.964064in}{2.632711in}}%
\pgfpathlineto{\pgfqpoint{0.981669in}{2.504523in}}%
\pgfpathlineto{\pgfqpoint{1.002559in}{2.343505in}}%
\pgfpathlineto{\pgfqpoint{1.022278in}{2.273453in}}%
\pgfpathlineto{\pgfqpoint{1.041289in}{2.249490in}}%
\pgfpathlineto{\pgfqpoint{1.059365in}{2.255657in}}%
\pgfpathlineto{\pgfqpoint{1.079550in}{2.295444in}}%
\pgfpathlineto{\pgfqpoint{1.097625in}{2.367726in}}%
\pgfpathlineto{\pgfqpoint{1.118282in}{2.534050in}}%
\pgfpathlineto{\pgfqpoint{1.136590in}{2.632032in}}%
\pgfpathlineto{\pgfqpoint{1.154900in}{2.597621in}}%
\pgfpathlineto{\pgfqpoint{1.176025in}{2.426039in}}%
\pgfpathlineto{\pgfqpoint{1.196212in}{2.313364in}}%
\pgfpathlineto{\pgfqpoint{1.214754in}{2.262732in}}%
\pgfpathlineto{\pgfqpoint{1.232594in}{2.245981in}}%
\pgfpathlineto{\pgfqpoint{1.251138in}{2.256685in}}%
\pgfpathlineto{\pgfqpoint{1.271794in}{2.296234in}}%
\pgfpathlineto{\pgfqpoint{1.289868in}{2.370904in}}%
\pgfpathlineto{\pgfqpoint{1.307707in}{2.517668in}}%
\pgfpathlineto{\pgfqpoint{1.329772in}{2.623249in}}%
\pgfpathlineto{\pgfqpoint{1.347847in}{2.588919in}}%
\pgfpathlineto{\pgfqpoint{1.367798in}{2.415612in}}%
\pgfpathlineto{\pgfqpoint{1.384934in}{2.324166in}}%
\pgfpathlineto{\pgfqpoint{1.402773in}{2.271637in}}%
\pgfpathlineto{\pgfqpoint{1.424133in}{2.247798in}}%
\pgfpathlineto{\pgfqpoint{1.445494in}{2.253730in}}%
\pgfpathlineto{\pgfqpoint{1.462628in}{2.280995in}}%
\pgfpathlineto{\pgfqpoint{1.483521in}{2.331650in}}%
\pgfpathlineto{\pgfqpoint{1.502063in}{2.440189in}}%
\pgfpathlineto{\pgfqpoint{1.522719in}{2.549216in}}%
\pgfpathlineto{\pgfqpoint{1.541264in}{2.621866in}}%
\pgfpathlineto{\pgfqpoint{1.559337in}{2.565072in}}%
\pgfpathlineto{\pgfqpoint{1.577177in}{2.420229in}}%
\pgfpathlineto{\pgfqpoint{1.598067in}{2.322092in}}%
\pgfpathlineto{\pgfqpoint{1.615908in}{2.277471in}}%
\pgfpathlineto{\pgfqpoint{1.636328in}{2.248833in}}%
\pgfpathlineto{\pgfqpoint{1.653698in}{2.245131in}}%
\pgfpathlineto{\pgfqpoint{1.675529in}{2.262937in}}%
\pgfpathlineto{\pgfqpoint{1.696654in}{2.300142in}}%
\pgfpathlineto{\pgfqpoint{1.711441in}{2.340280in}}%
\pgfpathlineto{\pgfqpoint{1.732098in}{2.475993in}}%
\pgfpathlineto{\pgfqpoint{1.752990in}{2.603754in}}%
\pgfpathlineto{\pgfqpoint{1.770829in}{2.611211in}}%
\pgfpathlineto{\pgfqpoint{1.791954in}{2.508597in}}%
\pgfpathlineto{\pgfqpoint{1.810262in}{2.451390in}}%
\pgfpathlineto{\pgfqpoint{1.828104in}{2.334021in}}%
\pgfpathlineto{\pgfqpoint{1.845472in}{2.274256in}}%
\pgfpathlineto{\pgfqpoint{1.867771in}{2.248262in}}%
\pgfpathlineto{\pgfqpoint{1.885376in}{2.245576in}}%
\pgfpathlineto{\pgfqpoint{1.906268in}{2.264301in}}%
\pgfpathlineto{\pgfqpoint{1.923873in}{2.297809in}}%
\pgfpathlineto{\pgfqpoint{1.944529in}{2.356109in}}%
\pgfpathlineto{\pgfqpoint{1.962368in}{2.487837in}}%
\pgfpathlineto{\pgfqpoint{1.984197in}{2.610197in}}%
\pgfpathlineto{\pgfqpoint{2.001333in}{2.596306in}}%
\pgfpathlineto{\pgfqpoint{2.019406in}{2.507846in}}%
\pgfpathlineto{\pgfqpoint{2.056024in}{2.286726in}}%
\pgfpathlineto{\pgfqpoint{2.077620in}{2.267959in}}%
\pgfpathlineto{\pgfqpoint{2.097807in}{2.244965in}}%
\pgfpathlineto{\pgfqpoint{2.118698in}{2.247921in}}%
\pgfpathlineto{\pgfqpoint{2.138415in}{2.264775in}}%
\pgfpathlineto{\pgfqpoint{2.154845in}{2.296496in}}%
\pgfpathlineto{\pgfqpoint{2.175501in}{2.376593in}}%
\pgfpathlineto{\pgfqpoint{2.193811in}{2.491680in}}%
\pgfpathlineto{\pgfqpoint{2.211651in}{2.566590in}}%
\pgfpathlineto{\pgfqpoint{2.230193in}{2.610807in}}%
\pgfpathlineto{\pgfqpoint{2.251086in}{2.536241in}}%
\pgfpathlineto{\pgfqpoint{2.271976in}{2.395365in}}%
\pgfpathlineto{\pgfqpoint{2.290755in}{2.304770in}}%
\pgfpathlineto{\pgfqpoint{2.309532in}{2.265731in}}%
\pgfpathlineto{\pgfqpoint{2.326902in}{2.246586in}}%
\pgfpathlineto{\pgfqpoint{2.348027in}{2.246082in}}%
\pgfpathlineto{\pgfqpoint{2.369389in}{2.260554in}}%
\pgfpathlineto{\pgfqpoint{2.386993in}{2.276259in}}%
\pgfpathlineto{\pgfqpoint{2.404598in}{2.311474in}}%
\pgfpathlineto{\pgfqpoint{2.422906in}{2.390475in}}%
\pgfpathlineto{\pgfqpoint{2.442625in}{2.508175in}}%
\pgfpathlineto{\pgfqpoint{2.461402in}{2.604890in}}%
\pgfpathlineto{\pgfqpoint{2.482529in}{2.574852in}}%
\pgfpathlineto{\pgfqpoint{2.500837in}{2.447832in}}%
\pgfpathlineto{\pgfqpoint{2.518207in}{2.341188in}}%
\pgfpathlineto{\pgfqpoint{2.540506in}{2.278785in}}%
\pgfpathlineto{\pgfqpoint{2.559754in}{2.252645in}}%
\pgfpathlineto{\pgfqpoint{2.577124in}{2.247037in}}%
\pgfpathlineto{\pgfqpoint{2.598483in}{2.247220in}}%
\pgfpathlineto{\pgfqpoint{2.616325in}{2.260577in}}%
\pgfpathlineto{\pgfqpoint{2.634867in}{2.289167in}}%
\pgfpathlineto{\pgfqpoint{2.654349in}{2.350766in}}%
\pgfpathlineto{\pgfqpoint{2.675005in}{2.352859in}}%
\pgfpathlineto{\pgfqpoint{2.694724in}{2.480488in}}%
\pgfpathlineto{\pgfqpoint{2.714206in}{2.594519in}}%
\pgfpathlineto{\pgfqpoint{2.732985in}{2.610498in}}%
\pgfpathlineto{\pgfqpoint{2.750824in}{2.523946in}}%
\pgfpathlineto{\pgfqpoint{2.768898in}{2.396502in}}%
\pgfpathlineto{\pgfqpoint{2.789788in}{2.302669in}}%
\pgfpathlineto{\pgfqpoint{2.807627in}{2.265508in}}%
\pgfpathlineto{\pgfqpoint{2.826406in}{2.247498in}}%
\pgfpathlineto{\pgfqpoint{2.847297in}{2.247489in}}%
\pgfpathlineto{\pgfqpoint{2.866781in}{2.265295in}}%
\pgfpathlineto{\pgfqpoint{2.884620in}{2.297404in}}%
\pgfpathlineto{\pgfqpoint{2.908562in}{2.357063in}}%
\pgfpathlineto{\pgfqpoint{2.924289in}{2.453886in}}%
\pgfpathlineto{\pgfqpoint{2.944475in}{2.565666in}}%
\pgfpathlineto{\pgfqpoint{2.962550in}{2.616986in}}%
\pgfpathlineto{\pgfqpoint{2.981562in}{2.584874in}}%
\pgfpathlineto{\pgfqpoint{3.002220in}{2.449720in}}%
\pgfpathlineto{\pgfqpoint{3.019588in}{2.366694in}}%
\pgfpathlineto{\pgfqpoint{3.037664in}{2.301289in}}%
\pgfpathlineto{\pgfqpoint{3.037898in}{2.276124in}}%
\pgfpathlineto{\pgfqpoint{3.062780in}{2.268645in}}%
\pgfpathlineto{\pgfqpoint{3.077333in}{2.253860in}}%
\pgfpathlineto{\pgfqpoint{3.095407in}{2.528753in}}%
\pgfpathlineto{\pgfqpoint{3.116532in}{2.363589in}}%
\pgfpathlineto{\pgfqpoint{3.134606in}{2.294845in}}%
\pgfpathlineto{\pgfqpoint{3.152915in}{2.259086in}}%
\pgfpathlineto{\pgfqpoint{3.171458in}{2.246071in}}%
\pgfpathlineto{\pgfqpoint{3.192114in}{2.253514in}}%
\pgfpathlineto{\pgfqpoint{3.210424in}{2.275417in}}%
\pgfpathlineto{\pgfqpoint{3.232489in}{2.330761in}}%
\pgfpathlineto{\pgfqpoint{3.252205in}{2.438759in}}%
\pgfpathlineto{\pgfqpoint{3.271453in}{2.554741in}}%
\pgfpathlineto{\pgfqpoint{3.289058in}{2.622229in}}%
\pgfpathlineto{\pgfqpoint{3.306897in}{2.596673in}}%
\pgfpathlineto{\pgfqpoint{3.328024in}{2.499581in}}%
\pgfpathlineto{\pgfqpoint{3.348680in}{2.357100in}}%
\pgfpathlineto{\pgfqpoint{3.364171in}{2.302158in}}%
\pgfpathlineto{\pgfqpoint{3.385062in}{2.263821in}}%
\pgfpathlineto{\pgfqpoint{3.405483in}{2.248122in}}%
\pgfpathlineto{\pgfqpoint{3.424731in}{2.249051in}}%
\pgfpathlineto{\pgfqpoint{3.444450in}{2.269733in}}%
\pgfpathlineto{\pgfqpoint{3.460175in}{2.294293in}}%
\pgfpathlineto{\pgfqpoint{3.480831in}{2.369214in}}%
\pgfpathlineto{\pgfqpoint{3.500784in}{2.487922in}}%
\pgfpathlineto{\pgfqpoint{3.520971in}{2.615838in}}%
\pgfpathlineto{\pgfqpoint{3.540688in}{2.629970in}}%
\pgfpathlineto{\pgfqpoint{3.558527in}{2.573862in}}%
\pgfpathlineto{\pgfqpoint{3.577540in}{2.454643in}}%
\pgfpathlineto{\pgfqpoint{3.598431in}{2.336817in}}%
\pgfpathlineto{\pgfqpoint{3.616270in}{2.299758in}}%
\pgfpathlineto{\pgfqpoint{3.634109in}{2.275243in}}%
\pgfpathlineto{\pgfqpoint{3.653123in}{2.253057in}}%
\pgfpathlineto{\pgfqpoint{3.673779in}{2.249722in}}%
\pgfpathlineto{\pgfqpoint{3.693966in}{2.265151in}}%
\pgfpathlineto{\pgfqpoint{3.713214in}{2.295635in}}%
\pgfpathlineto{\pgfqpoint{3.733636in}{2.339422in}}%
\pgfpathlineto{\pgfqpoint{3.750772in}{2.423726in}}%
\pgfpathlineto{\pgfqpoint{3.769785in}{2.545111in}}%
\pgfpathlineto{\pgfqpoint{3.789267in}{2.628799in}}%
\pgfpathlineto{\pgfqpoint{3.808280in}{2.635289in}}%
\pgfpathlineto{\pgfqpoint{3.826823in}{2.567357in}}%
\pgfpathlineto{\pgfqpoint{3.846305in}{2.471787in}}%
\pgfpathlineto{\pgfqpoint{3.866963in}{2.360495in}}%
\pgfpathlineto{\pgfqpoint{3.885505in}{2.300855in}}%
\pgfpathlineto{\pgfqpoint{3.904753in}{2.268469in}}%
\pgfpathlineto{\pgfqpoint{3.923063in}{2.257257in}}%
\pgfpathlineto{\pgfqpoint{3.942779in}{2.248873in}}%
\pgfpathlineto{\pgfqpoint{3.960384in}{2.259107in}}%
\pgfpathlineto{\pgfqpoint{3.980572in}{2.272282in}}%
\pgfpathlineto{\pgfqpoint{3.999348in}{2.310615in}}%
\pgfpathlineto{\pgfqpoint{4.019301in}{2.375462in}}%
\pgfpathlineto{\pgfqpoint{4.037375in}{2.491037in}}%
\pgfpathlineto{\pgfqpoint{4.056623in}{2.606623in}}%
\pgfpathlineto{\pgfqpoint{4.075402in}{2.650638in}}%
\pgfpathlineto{\pgfqpoint{4.097701in}{2.640046in}}%
\pgfpathlineto{\pgfqpoint{4.116245in}{2.557977in}}%
\pgfpathlineto{\pgfqpoint{4.136430in}{2.422534in}}%
\pgfpathlineto{\pgfqpoint{4.154975in}{2.349175in}}%
\pgfpathlineto{\pgfqpoint{4.174457in}{2.294729in}}%
\pgfpathlineto{\pgfqpoint{4.192531in}{2.271263in}}%
\pgfpathlineto{\pgfqpoint{4.212249in}{2.251648in}}%
\pgfpathlineto{\pgfqpoint{4.229149in}{2.515648in}}%
\pgfpathlineto{\pgfqpoint{4.250041in}{2.641239in}}%
\pgfpathlineto{\pgfqpoint{4.268584in}{2.649742in}}%
\pgfpathlineto{\pgfqpoint{4.287597in}{2.560446in}}%
\pgfpathlineto{\pgfqpoint{4.308487in}{2.393777in}}%
\pgfpathlineto{\pgfqpoint{4.327266in}{2.317759in}}%
\pgfpathlineto{\pgfqpoint{4.346983in}{2.268372in}}%
\pgfpathlineto{\pgfqpoint{4.364353in}{2.251733in}}%
\pgfpathlineto{\pgfqpoint{4.389000in}{2.264027in}}%
\pgfpathlineto{\pgfqpoint{4.402380in}{2.292349in}}%
\pgfpathlineto{\pgfqpoint{4.421158in}{2.341920in}}%
\pgfpathlineto{\pgfqpoint{4.443926in}{2.477747in}}%
\pgfpathlineto{\pgfqpoint{4.462236in}{2.619894in}}%
\pgfpathlineto{\pgfqpoint{4.481013in}{2.667708in}}%
\pgfpathlineto{\pgfqpoint{4.473268in}{2.653044in}}%
\pgfpathlineto{\pgfqpoint{4.433833in}{2.348663in}}%
\pgfpathlineto{\pgfqpoint{4.415993in}{2.284274in}}%
\pgfpathlineto{\pgfqpoint{4.397920in}{2.254720in}}%
\pgfpathlineto{\pgfqpoint{4.377264in}{2.262404in}}%
\pgfpathlineto{\pgfqpoint{4.359424in}{2.303268in}}%
\pgfpathlineto{\pgfqpoint{4.342054in}{2.401446in}}%
\pgfpathlineto{\pgfqpoint{4.322101in}{2.573705in}}%
\pgfpathlineto{\pgfqpoint{4.300976in}{2.654700in}}%
\pgfpathlineto{\pgfqpoint{4.282903in}{2.557836in}}%
\pgfpathlineto{\pgfqpoint{4.260604in}{2.361069in}}%
\pgfpathlineto{\pgfqpoint{4.243936in}{2.291975in}}%
\pgfpathlineto{\pgfqpoint{4.224923in}{2.255841in}}%
\pgfpathlineto{\pgfqpoint{4.203095in}{2.253474in}}%
\pgfpathlineto{\pgfqpoint{4.185256in}{2.288743in}}%
\pgfpathlineto{\pgfqpoint{4.164600in}{2.377785in}}%
\pgfpathlineto{\pgfqpoint{4.146290in}{2.542716in}}%
\pgfpathlineto{\pgfqpoint{4.127511in}{2.642952in}}%
\pgfpathlineto{\pgfqpoint{4.105212in}{2.560215in}}%
\pgfpathlineto{\pgfqpoint{4.089955in}{2.413799in}}%
\pgfpathlineto{\pgfqpoint{4.071645in}{2.307347in}}%
\pgfpathlineto{\pgfqpoint{4.053103in}{2.264166in}}%
\pgfpathlineto{\pgfqpoint{4.031272in}{2.247619in}}%
\pgfpathlineto{\pgfqpoint{4.012259in}{2.262362in}}%
\pgfpathlineto{\pgfqpoint{3.993951in}{2.308407in}}%
\pgfpathlineto{\pgfqpoint{3.974938in}{2.419741in}}%
\pgfpathlineto{\pgfqpoint{3.953342in}{2.596136in}}%
\pgfpathlineto{\pgfqpoint{3.934094in}{2.632987in}}%
\pgfpathlineto{\pgfqpoint{3.916021in}{2.521504in}}%
\pgfpathlineto{\pgfqpoint{3.897242in}{2.366420in}}%
\pgfpathlineto{\pgfqpoint{3.878934in}{2.291297in}}%
\pgfpathlineto{\pgfqpoint{3.860155in}{2.258296in}}%
\pgfpathlineto{\pgfqpoint{3.840673in}{2.245899in}}%
\pgfpathlineto{\pgfqpoint{3.822597in}{2.262129in}}%
\pgfpathlineto{\pgfqpoint{3.800769in}{2.313683in}}%
\pgfpathlineto{\pgfqpoint{3.785276in}{2.380676in}}%
\pgfpathlineto{\pgfqpoint{3.765325in}{2.530284in}}%
\pgfpathlineto{\pgfqpoint{3.740443in}{2.628593in}}%
\pgfpathlineto{\pgfqpoint{3.723776in}{2.605867in}}%
\pgfpathlineto{\pgfqpoint{3.701243in}{2.431503in}}%
\pgfpathlineto{\pgfqpoint{3.686455in}{2.346592in}}%
\pgfpathlineto{\pgfqpoint{3.667676in}{2.286019in}}%
\pgfpathlineto{\pgfqpoint{3.649368in}{2.255342in}}%
\pgfpathlineto{\pgfqpoint{3.627069in}{2.618537in}}%
\pgfpathlineto{\pgfqpoint{3.608994in}{2.506203in}}%
\pgfpathlineto{\pgfqpoint{3.589982in}{2.363030in}}%
\pgfpathlineto{\pgfqpoint{3.568621in}{2.285979in}}%
\pgfpathlineto{\pgfqpoint{3.553128in}{2.259367in}}%
\pgfpathlineto{\pgfqpoint{3.531300in}{2.244862in}}%
\pgfpathlineto{\pgfqpoint{3.515338in}{2.251876in}}%
\pgfpathlineto{\pgfqpoint{3.493742in}{2.279258in}}%
\pgfpathlineto{\pgfqpoint{3.474025in}{2.340415in}}%
\pgfpathlineto{\pgfqpoint{3.452430in}{2.487118in}}%
\pgfpathlineto{\pgfqpoint{3.436938in}{2.600870in}}%
\pgfpathlineto{\pgfqpoint{3.417925in}{2.618877in}}%
\pgfpathlineto{\pgfqpoint{3.397738in}{2.527262in}}%
\pgfpathlineto{\pgfqpoint{3.379664in}{2.370485in}}%
\pgfpathlineto{\pgfqpoint{3.357129in}{2.289778in}}%
\pgfpathlineto{\pgfqpoint{3.340933in}{2.261999in}}%
\pgfpathlineto{\pgfqpoint{3.321921in}{2.244592in}}%
\pgfpathlineto{\pgfqpoint{3.301263in}{2.254173in}}%
\pgfpathlineto{\pgfqpoint{3.282252in}{2.282186in}}%
\pgfpathlineto{\pgfqpoint{3.264411in}{2.346150in}}%
\pgfpathlineto{\pgfqpoint{3.245634in}{2.474230in}}%
\pgfpathlineto{\pgfqpoint{3.223569in}{2.604277in}}%
\pgfpathlineto{\pgfqpoint{3.205964in}{2.608423in}}%
\pgfpathlineto{\pgfqpoint{3.186951in}{2.501983in}}%
\pgfpathlineto{\pgfqpoint{3.167703in}{2.370178in}}%
\pgfpathlineto{\pgfqpoint{3.149159in}{2.297794in}}%
\pgfpathlineto{\pgfqpoint{3.129208in}{2.259268in}}%
\pgfpathlineto{\pgfqpoint{3.110429in}{2.244980in}}%
\pgfpathlineto{\pgfqpoint{3.090713in}{2.250830in}}%
\pgfpathlineto{\pgfqpoint{3.072168in}{2.270801in}}%
\pgfpathlineto{\pgfqpoint{3.050338in}{2.306986in}}%
\pgfpathlineto{\pgfqpoint{3.031796in}{2.413905in}}%
\pgfpathlineto{\pgfqpoint{3.013954in}{2.554165in}}%
\pgfpathlineto{\pgfqpoint{2.994003in}{2.615468in}}%
\pgfpathlineto{\pgfqpoint{2.975695in}{2.565702in}}%
\pgfpathlineto{\pgfqpoint{2.958091in}{2.440535in}}%
\pgfpathlineto{\pgfqpoint{2.935555in}{2.328070in}}%
\pgfpathlineto{\pgfqpoint{2.917247in}{2.275768in}}%
\pgfpathlineto{\pgfqpoint{2.898937in}{2.251509in}}%
\pgfpathlineto{\pgfqpoint{2.879926in}{2.244815in}}%
\pgfpathlineto{\pgfqpoint{2.859739in}{2.253622in}}%
\pgfpathlineto{\pgfqpoint{2.840020in}{2.270537in}}%
\pgfpathlineto{\pgfqpoint{2.821712in}{2.318356in}}%
\pgfpathlineto{\pgfqpoint{2.802464in}{2.414783in}}%
\pgfpathlineto{\pgfqpoint{2.784156in}{2.542646in}}%
\pgfpathlineto{\pgfqpoint{2.766081in}{2.612054in}}%
\pgfpathlineto{\pgfqpoint{2.743782in}{2.550352in}}%
\pgfpathlineto{\pgfqpoint{2.725003in}{2.524077in}}%
\pgfpathlineto{\pgfqpoint{2.708338in}{2.393563in}}%
\pgfpathlineto{\pgfqpoint{2.686507in}{2.299381in}}%
\pgfpathlineto{\pgfqpoint{2.667965in}{2.267771in}}%
\pgfpathlineto{\pgfqpoint{2.649186in}{2.246882in}}%
\pgfpathlineto{\pgfqpoint{2.629233in}{2.245685in}}%
\pgfpathlineto{\pgfqpoint{2.610220in}{2.260786in}}%
\pgfpathlineto{\pgfqpoint{2.589564in}{2.301090in}}%
\pgfpathlineto{\pgfqpoint{2.573604in}{2.366561in}}%
\pgfpathlineto{\pgfqpoint{2.551774in}{2.522465in}}%
\pgfpathlineto{\pgfqpoint{2.532995in}{2.596625in}}%
\pgfpathlineto{\pgfqpoint{2.514216in}{2.599043in}}%
\pgfpathlineto{\pgfqpoint{2.496377in}{2.550419in}}%
\pgfpathlineto{\pgfqpoint{2.496142in}{2.599148in}}%
\pgfpathlineto{\pgfqpoint{2.474547in}{2.612100in}}%
\pgfpathlineto{\pgfqpoint{2.459055in}{2.571591in}}%
\pgfpathlineto{\pgfqpoint{2.433234in}{2.383513in}}%
\pgfpathlineto{\pgfqpoint{2.417272in}{2.306193in}}%
\pgfpathlineto{\pgfqpoint{2.400138in}{2.270591in}}%
\pgfpathlineto{\pgfqpoint{2.379011in}{2.247519in}}%
\pgfpathlineto{\pgfqpoint{2.360469in}{2.245061in}}%
\pgfpathlineto{\pgfqpoint{2.341456in}{2.259489in}}%
\pgfpathlineto{\pgfqpoint{2.322911in}{2.282794in}}%
\pgfpathlineto{\pgfqpoint{2.302960in}{2.342208in}}%
\pgfpathlineto{\pgfqpoint{2.283007in}{2.484228in}}%
\pgfpathlineto{\pgfqpoint{2.262351in}{2.608041in}}%
\pgfpathlineto{\pgfqpoint{2.245921in}{2.613691in}}%
\pgfpathlineto{\pgfqpoint{2.224090in}{2.529688in}}%
\pgfpathlineto{\pgfqpoint{2.207660in}{2.406196in}}%
\pgfpathlineto{\pgfqpoint{2.187238in}{2.316644in}}%
\pgfpathlineto{\pgfqpoint{2.169164in}{2.271991in}}%
\pgfpathlineto{\pgfqpoint{2.150151in}{2.250814in}}%
\pgfpathlineto{\pgfqpoint{2.126678in}{2.247614in}}%
\pgfpathlineto{\pgfqpoint{2.110716in}{2.253506in}}%
\pgfpathlineto{\pgfqpoint{2.088888in}{2.276295in}}%
\pgfpathlineto{\pgfqpoint{2.073160in}{2.319097in}}%
\pgfpathlineto{\pgfqpoint{2.051330in}{2.428420in}}%
\pgfpathlineto{\pgfqpoint{2.033960in}{2.524223in}}%
\pgfpathlineto{\pgfqpoint{2.014712in}{2.612742in}}%
\pgfpathlineto{\pgfqpoint{1.996404in}{2.591065in}}%
\pgfpathlineto{\pgfqpoint{1.975043in}{2.466907in}}%
\pgfpathlineto{\pgfqpoint{1.956266in}{2.376169in}}%
\pgfpathlineto{\pgfqpoint{1.937721in}{2.305704in}}%
\pgfpathlineto{\pgfqpoint{1.915891in}{2.262898in}}%
\pgfpathlineto{\pgfqpoint{1.901808in}{2.250585in}}%
\pgfpathlineto{\pgfqpoint{1.879508in}{2.246279in}}%
\pgfpathlineto{\pgfqpoint{1.860496in}{2.258331in}}%
\pgfpathlineto{\pgfqpoint{1.840778in}{2.284436in}}%
\pgfpathlineto{\pgfqpoint{1.821999in}{2.325397in}}%
\pgfpathlineto{\pgfqpoint{1.803691in}{2.396446in}}%
\pgfpathlineto{\pgfqpoint{1.783035in}{2.524939in}}%
\pgfpathlineto{\pgfqpoint{1.765195in}{2.606809in}}%
\pgfpathlineto{\pgfqpoint{1.743365in}{2.607787in}}%
\pgfpathlineto{\pgfqpoint{1.727874in}{2.553465in}}%
\pgfpathlineto{\pgfqpoint{1.706278in}{2.415033in}}%
\pgfpathlineto{\pgfqpoint{1.687500in}{2.325948in}}%
\pgfpathlineto{\pgfqpoint{1.668957in}{2.286800in}}%
\pgfpathlineto{\pgfqpoint{1.650647in}{2.259951in}}%
\pgfpathlineto{\pgfqpoint{1.630931in}{2.248190in}}%
\pgfpathlineto{\pgfqpoint{1.613326in}{2.248784in}}%
\pgfpathlineto{\pgfqpoint{1.592670in}{2.262037in}}%
\pgfpathlineto{\pgfqpoint{1.574594in}{2.293233in}}%
\pgfpathlineto{\pgfqpoint{1.550887in}{2.358095in}}%
\pgfpathlineto{\pgfqpoint{1.533753in}{2.437965in}}%
\pgfpathlineto{\pgfqpoint{1.513096in}{2.569865in}}%
\pgfpathlineto{\pgfqpoint{1.495257in}{2.623633in}}%
\pgfpathlineto{\pgfqpoint{1.476947in}{2.620849in}}%
\pgfpathlineto{\pgfqpoint{1.459108in}{2.550618in}}%
\pgfpathlineto{\pgfqpoint{1.438921in}{2.418088in}}%
\pgfpathlineto{\pgfqpoint{1.419673in}{2.345663in}}%
\pgfpathlineto{\pgfqpoint{1.396671in}{2.409206in}}%
\pgfpathlineto{\pgfqpoint{1.379769in}{2.332683in}}%
\pgfpathlineto{\pgfqpoint{1.359113in}{2.281532in}}%
\pgfpathlineto{\pgfqpoint{1.342448in}{2.258275in}}%
\pgfpathlineto{\pgfqpoint{1.324138in}{2.247514in}}%
\pgfpathlineto{\pgfqpoint{1.304656in}{2.252239in}}%
\pgfpathlineto{\pgfqpoint{1.284234in}{2.271866in}}%
\pgfpathlineto{\pgfqpoint{1.264049in}{2.309531in}}%
\pgfpathlineto{\pgfqpoint{1.246678in}{2.344831in}}%
\pgfpathlineto{\pgfqpoint{1.225788in}{2.467525in}}%
\pgfpathlineto{\pgfqpoint{1.207243in}{2.522740in}}%
\pgfpathlineto{\pgfqpoint{1.187527in}{2.621125in}}%
\pgfpathlineto{\pgfqpoint{1.170391in}{2.638660in}}%
\pgfpathlineto{\pgfqpoint{1.149735in}{2.636918in}}%
\pgfpathlineto{\pgfqpoint{1.132130in}{2.577886in}}%
\pgfpathlineto{\pgfqpoint{1.112179in}{2.435622in}}%
\pgfpathlineto{\pgfqpoint{1.091757in}{2.335237in}}%
\pgfpathlineto{\pgfqpoint{1.071101in}{2.288567in}}%
\pgfpathlineto{\pgfqpoint{1.053025in}{2.266706in}}%
\pgfpathlineto{\pgfqpoint{1.036595in}{2.251677in}}%
\pgfpathlineto{\pgfqpoint{1.015704in}{2.250369in}}%
\pgfpathlineto{\pgfqpoint{0.995517in}{2.269524in}}%
\pgfpathlineto{\pgfqpoint{0.979321in}{2.298980in}}%
\pgfpathlineto{\pgfqpoint{0.958196in}{2.361887in}}%
\pgfpathlineto{\pgfqpoint{0.940356in}{2.460791in}}%
\pgfpathlineto{\pgfqpoint{0.918761in}{2.591982in}}%
\pgfpathlineto{\pgfqpoint{0.898339in}{2.649438in}}%
\pgfpathlineto{\pgfqpoint{0.880265in}{2.642469in}}%
\pgfpathlineto{\pgfqpoint{0.862426in}{2.564985in}}%
\pgfpathlineto{\pgfqpoint{0.824870in}{2.374113in}}%
\pgfpathlineto{\pgfqpoint{0.786138in}{2.287199in}}%
\pgfpathlineto{\pgfqpoint{0.763136in}{2.259667in}}%
\pgfpathlineto{\pgfqpoint{0.744826in}{2.251487in}}%
\pgfpathlineto{\pgfqpoint{0.727692in}{2.257221in}}%
\pgfpathlineto{\pgfqpoint{0.707505in}{2.280544in}}%
\pgfpathlineto{\pgfqpoint{0.689195in}{2.310909in}}%
\pgfpathlineto{\pgfqpoint{0.667835in}{2.395309in}}%
\pgfpathlineto{\pgfqpoint{0.649996in}{2.488064in}}%
\pgfpathlineto{\pgfqpoint{0.651170in}{2.483180in}}%
\pgfpathlineto{\pgfqpoint{0.657039in}{2.428515in}}%
\pgfpathlineto{\pgfqpoint{0.676755in}{2.315005in}}%
\pgfpathlineto{\pgfqpoint{0.696472in}{2.264413in}}%
\pgfpathlineto{\pgfqpoint{0.714547in}{2.251047in}}%
\pgfpathlineto{\pgfqpoint{0.732855in}{2.272342in}}%
\pgfpathlineto{\pgfqpoint{0.754685in}{2.335151in}}%
\pgfpathlineto{\pgfqpoint{0.776516in}{2.484104in}}%
\pgfpathlineto{\pgfqpoint{0.791772in}{2.617536in}}%
\pgfpathlineto{\pgfqpoint{0.810317in}{2.648360in}}%
\pgfpathlineto{\pgfqpoint{0.829564in}{2.539450in}}%
\pgfpathlineto{\pgfqpoint{0.848812in}{2.379751in}}%
\pgfpathlineto{\pgfqpoint{0.866651in}{2.292267in}}%
\pgfpathlineto{\pgfqpoint{0.888247in}{2.253534in}}%
\pgfpathlineto{\pgfqpoint{0.906555in}{2.254399in}}%
\pgfpathlineto{\pgfqpoint{0.925334in}{2.282551in}}%
\pgfpathlineto{\pgfqpoint{0.944347in}{2.341940in}}%
\pgfpathlineto{\pgfqpoint{0.961716in}{2.471104in}}%
\pgfpathlineto{\pgfqpoint{0.983546in}{2.347607in}}%
\pgfpathlineto{\pgfqpoint{1.001856in}{2.279959in}}%
\pgfpathlineto{\pgfqpoint{1.023215in}{2.248802in}}%
\pgfpathlineto{\pgfqpoint{1.041289in}{2.254851in}}%
\pgfpathlineto{\pgfqpoint{1.059130in}{2.284949in}}%
\pgfpathlineto{\pgfqpoint{1.080255in}{2.372352in}}%
\pgfpathlineto{\pgfqpoint{1.101146in}{2.544450in}}%
\pgfpathlineto{\pgfqpoint{1.116168in}{2.628914in}}%
\pgfpathlineto{\pgfqpoint{1.137059in}{2.587070in}}%
\pgfpathlineto{\pgfqpoint{1.161002in}{2.392114in}}%
\pgfpathlineto{\pgfqpoint{1.172268in}{2.313491in}}%
\pgfpathlineto{\pgfqpoint{1.193395in}{2.262389in}}%
\pgfpathlineto{\pgfqpoint{1.217337in}{2.247053in}}%
\pgfpathlineto{\pgfqpoint{1.232125in}{2.263453in}}%
\pgfpathlineto{\pgfqpoint{1.249964in}{2.298649in}}%
\pgfpathlineto{\pgfqpoint{1.270855in}{2.397440in}}%
\pgfpathlineto{\pgfqpoint{1.289399in}{2.538223in}}%
\pgfpathlineto{\pgfqpoint{1.310758in}{2.627060in}}%
\pgfpathlineto{\pgfqpoint{1.331180in}{2.554329in}}%
\pgfpathlineto{\pgfqpoint{1.348785in}{2.401265in}}%
\pgfpathlineto{\pgfqpoint{1.366859in}{2.302112in}}%
\pgfpathlineto{\pgfqpoint{1.387751in}{2.257923in}}%
\pgfpathlineto{\pgfqpoint{1.406059in}{2.245748in}}%
\pgfpathlineto{\pgfqpoint{1.424367in}{2.251586in}}%
\pgfpathlineto{\pgfqpoint{1.444554in}{2.282433in}}%
\pgfpathlineto{\pgfqpoint{1.465916in}{2.354824in}}%
\pgfpathlineto{\pgfqpoint{1.483989in}{2.488908in}}%
\pgfpathlineto{\pgfqpoint{1.501829in}{2.589713in}}%
\pgfpathlineto{\pgfqpoint{1.522485in}{2.619110in}}%
\pgfpathlineto{\pgfqpoint{1.540558in}{2.529539in}}%
\pgfpathlineto{\pgfqpoint{1.558634in}{2.383515in}}%
\pgfpathlineto{\pgfqpoint{1.579759in}{2.286155in}}%
\pgfpathlineto{\pgfqpoint{1.597833in}{2.262004in}}%
\pgfpathlineto{\pgfqpoint{1.615908in}{2.245866in}}%
\pgfpathlineto{\pgfqpoint{1.636564in}{2.251035in}}%
\pgfpathlineto{\pgfqpoint{1.657221in}{2.274528in}}%
\pgfpathlineto{\pgfqpoint{1.674823in}{2.321952in}}%
\pgfpathlineto{\pgfqpoint{1.692899in}{2.420461in}}%
\pgfpathlineto{\pgfqpoint{1.714024in}{2.548696in}}%
\pgfpathlineto{\pgfqpoint{1.731394in}{2.617416in}}%
\pgfpathlineto{\pgfqpoint{1.751111in}{2.580760in}}%
\pgfpathlineto{\pgfqpoint{1.771064in}{2.479439in}}%
\pgfpathlineto{\pgfqpoint{1.791485in}{2.330619in}}%
\pgfpathlineto{\pgfqpoint{1.810262in}{2.273386in}}%
\pgfpathlineto{\pgfqpoint{1.828104in}{2.253374in}}%
\pgfpathlineto{\pgfqpoint{1.848523in}{2.244971in}}%
\pgfpathlineto{\pgfqpoint{1.866833in}{2.253319in}}%
\pgfpathlineto{\pgfqpoint{1.887958in}{2.275495in}}%
\pgfpathlineto{\pgfqpoint{1.904155in}{2.315595in}}%
\pgfpathlineto{\pgfqpoint{1.924811in}{2.415590in}}%
\pgfpathlineto{\pgfqpoint{1.942884in}{2.554567in}}%
\pgfpathlineto{\pgfqpoint{1.963777in}{2.613385in}}%
\pgfpathlineto{\pgfqpoint{1.982554in}{2.560514in}}%
\pgfpathlineto{\pgfqpoint{2.002272in}{2.462670in}}%
\pgfpathlineto{\pgfqpoint{2.020111in}{2.340399in}}%
\pgfpathlineto{\pgfqpoint{2.039125in}{2.285458in}}%
\pgfpathlineto{\pgfqpoint{2.059076in}{2.262878in}}%
\pgfpathlineto{\pgfqpoint{2.080437in}{2.248250in}}%
\pgfpathlineto{\pgfqpoint{2.096633in}{2.244718in}}%
\pgfpathlineto{\pgfqpoint{2.117055in}{2.260706in}}%
\pgfpathlineto{\pgfqpoint{2.134189in}{2.289037in}}%
\pgfpathlineto{\pgfqpoint{2.156959in}{2.368941in}}%
\pgfpathlineto{\pgfqpoint{2.173390in}{2.476611in}}%
\pgfpathlineto{\pgfqpoint{2.194280in}{2.597086in}}%
\pgfpathlineto{\pgfqpoint{2.212119in}{2.602872in}}%
\pgfpathlineto{\pgfqpoint{2.233481in}{2.487617in}}%
\pgfpathlineto{\pgfqpoint{2.252258in}{2.365250in}}%
\pgfpathlineto{\pgfqpoint{2.270802in}{2.294843in}}%
\pgfpathlineto{\pgfqpoint{2.292163in}{2.298045in}}%
\pgfpathlineto{\pgfqpoint{2.310237in}{2.262585in}}%
\pgfpathlineto{\pgfqpoint{2.331128in}{2.244804in}}%
\pgfpathlineto{\pgfqpoint{2.351549in}{2.249654in}}%
\pgfpathlineto{\pgfqpoint{2.367041in}{2.265662in}}%
\pgfpathlineto{\pgfqpoint{2.383708in}{2.303065in}}%
\pgfpathlineto{\pgfqpoint{2.405067in}{2.366506in}}%
\pgfpathlineto{\pgfqpoint{2.425958in}{2.472410in}}%
\pgfpathlineto{\pgfqpoint{2.443797in}{2.584190in}}%
\pgfpathlineto{\pgfqpoint{2.465158in}{2.610373in}}%
\pgfpathlineto{\pgfqpoint{2.482529in}{2.574511in}}%
\pgfpathlineto{\pgfqpoint{2.500602in}{2.489757in}}%
\pgfpathlineto{\pgfqpoint{2.519147in}{2.376633in}}%
\pgfpathlineto{\pgfqpoint{2.539803in}{2.311044in}}%
\pgfpathlineto{\pgfqpoint{2.557642in}{2.265620in}}%
\pgfpathlineto{\pgfqpoint{2.582287in}{2.245580in}}%
\pgfpathlineto{\pgfqpoint{2.597780in}{2.245672in}}%
\pgfpathlineto{\pgfqpoint{2.614916in}{2.261878in}}%
\pgfpathlineto{\pgfqpoint{2.637215in}{2.291594in}}%
\pgfpathlineto{\pgfqpoint{2.654586in}{2.360959in}}%
\pgfpathlineto{\pgfqpoint{2.675711in}{2.494234in}}%
\pgfpathlineto{\pgfqpoint{2.694019in}{2.596241in}}%
\pgfpathlineto{\pgfqpoint{2.711858in}{2.604334in}}%
\pgfpathlineto{\pgfqpoint{2.750119in}{2.393099in}}%
\pgfpathlineto{\pgfqpoint{2.768898in}{2.320335in}}%
\pgfpathlineto{\pgfqpoint{2.789788in}{2.272797in}}%
\pgfpathlineto{\pgfqpoint{2.808098in}{2.253232in}}%
\pgfpathlineto{\pgfqpoint{2.829223in}{2.244095in}}%
\pgfpathlineto{\pgfqpoint{2.846594in}{2.252282in}}%
\pgfpathlineto{\pgfqpoint{2.864902in}{2.266763in}}%
\pgfpathlineto{\pgfqpoint{2.886497in}{2.306044in}}%
\pgfpathlineto{\pgfqpoint{2.904102in}{2.381950in}}%
\pgfpathlineto{\pgfqpoint{2.921941in}{2.452680in}}%
\pgfpathlineto{\pgfqpoint{2.944006in}{2.590128in}}%
\pgfpathlineto{\pgfqpoint{2.964193in}{2.608736in}}%
\pgfpathlineto{\pgfqpoint{2.981798in}{2.571293in}}%
\pgfpathlineto{\pgfqpoint{3.000575in}{2.497806in}}%
\pgfpathlineto{\pgfqpoint{3.017945in}{2.377577in}}%
\pgfpathlineto{\pgfqpoint{3.040010in}{2.293633in}}%
\pgfpathlineto{\pgfqpoint{3.056675in}{2.268000in}}%
\pgfpathlineto{\pgfqpoint{3.078976in}{2.248436in}}%
\pgfpathlineto{\pgfqpoint{3.096815in}{2.246261in}}%
\pgfpathlineto{\pgfqpoint{3.117706in}{2.258810in}}%
\pgfpathlineto{\pgfqpoint{3.136485in}{2.282585in}}%
\pgfpathlineto{\pgfqpoint{3.154793in}{2.310990in}}%
\pgfpathlineto{\pgfqpoint{3.173337in}{2.397010in}}%
\pgfpathlineto{\pgfqpoint{3.191411in}{2.516568in}}%
\pgfpathlineto{\pgfqpoint{3.212536in}{2.613759in}}%
\pgfpathlineto{\pgfqpoint{3.231549in}{2.613886in}}%
\pgfpathlineto{\pgfqpoint{3.247277in}{2.522706in}}%
\pgfpathlineto{\pgfqpoint{3.272393in}{2.390964in}}%
\pgfpathlineto{\pgfqpoint{3.289058in}{2.330234in}}%
\pgfpathlineto{\pgfqpoint{3.307837in}{2.283766in}}%
\pgfpathlineto{\pgfqpoint{3.328258in}{2.264537in}}%
\pgfpathlineto{\pgfqpoint{3.346566in}{2.261821in}}%
\pgfpathlineto{\pgfqpoint{3.364171in}{2.246992in}}%
\pgfpathlineto{\pgfqpoint{3.385767in}{2.252493in}}%
\pgfpathlineto{\pgfqpoint{3.403841in}{2.271283in}}%
\pgfpathlineto{\pgfqpoint{3.427314in}{2.306373in}}%
\pgfpathlineto{\pgfqpoint{3.443510in}{2.376481in}}%
\pgfpathlineto{\pgfqpoint{3.464166in}{2.483506in}}%
\pgfpathlineto{\pgfqpoint{3.482474in}{2.595923in}}%
\pgfpathlineto{\pgfqpoint{3.500079in}{2.630456in}}%
\pgfpathlineto{\pgfqpoint{3.518623in}{2.598019in}}%
\pgfpathlineto{\pgfqpoint{3.538811in}{2.465961in}}%
\pgfpathlineto{\pgfqpoint{3.558058in}{2.356212in}}%
\pgfpathlineto{\pgfqpoint{3.578009in}{2.298282in}}%
\pgfpathlineto{\pgfqpoint{3.615098in}{2.253252in}}%
\pgfpathlineto{\pgfqpoint{3.635518in}{2.253066in}}%
\pgfpathlineto{\pgfqpoint{3.653123in}{2.249558in}}%
\pgfpathlineto{\pgfqpoint{3.674484in}{2.267646in}}%
\pgfpathlineto{\pgfqpoint{3.693263in}{2.299358in}}%
\pgfpathlineto{\pgfqpoint{3.712040in}{2.362246in}}%
\pgfpathlineto{\pgfqpoint{3.729410in}{2.456950in}}%
\pgfpathlineto{\pgfqpoint{3.749127in}{2.519061in}}%
\pgfpathlineto{\pgfqpoint{3.770254in}{2.628925in}}%
\pgfpathlineto{\pgfqpoint{3.789736in}{2.632764in}}%
\pgfpathlineto{\pgfqpoint{3.809689in}{2.562416in}}%
\pgfpathlineto{\pgfqpoint{3.828702in}{2.446914in}}%
\pgfpathlineto{\pgfqpoint{3.845133in}{2.358216in}}%
\pgfpathlineto{\pgfqpoint{3.864380in}{2.301962in}}%
\pgfpathlineto{\pgfqpoint{3.889496in}{2.260399in}}%
\pgfpathlineto{\pgfqpoint{3.905222in}{2.253043in}}%
\pgfpathlineto{\pgfqpoint{3.924940in}{2.248194in}}%
\pgfpathlineto{\pgfqpoint{3.939494in}{2.257375in}}%
\pgfpathlineto{\pgfqpoint{3.962027in}{2.282528in}}%
\pgfpathlineto{\pgfqpoint{3.981040in}{2.326918in}}%
\pgfpathlineto{\pgfqpoint{4.000054in}{2.380769in}}%
\pgfpathlineto{\pgfqpoint{4.019301in}{2.498381in}}%
\pgfpathlineto{\pgfqpoint{4.038315in}{2.604185in}}%
\pgfpathlineto{\pgfqpoint{4.057562in}{2.646460in}}%
\pgfpathlineto{\pgfqpoint{4.076105in}{2.646505in}}%
\pgfpathlineto{\pgfqpoint{4.098640in}{2.591008in}}%
\pgfpathlineto{\pgfqpoint{4.116479in}{2.504904in}}%
\pgfpathlineto{\pgfqpoint{4.135024in}{2.398195in}}%
\pgfpathlineto{\pgfqpoint{4.154506in}{2.327779in}}%
\pgfpathlineto{\pgfqpoint{4.173519in}{2.300139in}}%
\pgfpathlineto{\pgfqpoint{4.189950in}{2.269859in}}%
\pgfpathlineto{\pgfqpoint{4.211544in}{2.252261in}}%
\pgfpathlineto{\pgfqpoint{4.231965in}{2.255435in}}%
\pgfpathlineto{\pgfqpoint{4.250510in}{2.279492in}}%
\pgfpathlineto{\pgfqpoint{4.268115in}{2.649082in}}%
\pgfpathlineto{\pgfqpoint{4.290414in}{2.526389in}}%
\pgfpathlineto{\pgfqpoint{4.309661in}{2.390617in}}%
\pgfpathlineto{\pgfqpoint{4.324918in}{2.314931in}}%
\pgfpathlineto{\pgfqpoint{4.350034in}{2.267908in}}%
\pgfpathlineto{\pgfqpoint{4.365293in}{2.251955in}}%
\pgfpathlineto{\pgfqpoint{4.385009in}{2.259603in}}%
\pgfpathlineto{\pgfqpoint{4.404962in}{2.291419in}}%
\pgfpathlineto{\pgfqpoint{4.422801in}{2.350081in}}%
\pgfpathlineto{\pgfqpoint{4.441344in}{2.445750in}}%
\pgfpathlineto{\pgfqpoint{4.460357in}{2.589737in}}%
\pgfpathlineto{\pgfqpoint{4.479605in}{2.665025in}}%
\pgfpathlineto{\pgfqpoint{4.474207in}{2.650943in}}%
\pgfpathlineto{\pgfqpoint{4.436884in}{2.353717in}}%
\pgfpathlineto{\pgfqpoint{4.416699in}{2.283793in}}%
\pgfpathlineto{\pgfqpoint{4.396511in}{2.252431in}}%
\pgfpathlineto{\pgfqpoint{4.378203in}{2.262705in}}%
\pgfpathlineto{\pgfqpoint{4.358016in}{2.316728in}}%
\pgfpathlineto{\pgfqpoint{4.339237in}{2.426894in}}%
\pgfpathlineto{\pgfqpoint{4.320458in}{2.600898in}}%
\pgfpathlineto{\pgfqpoint{4.302619in}{2.655549in}}%
\pgfpathlineto{\pgfqpoint{4.281963in}{2.535016in}}%
\pgfpathlineto{\pgfqpoint{4.260838in}{2.356087in}}%
\pgfpathlineto{\pgfqpoint{4.242528in}{2.285039in}}%
\pgfpathlineto{\pgfqpoint{4.224689in}{2.253757in}}%
\pgfpathlineto{\pgfqpoint{4.206381in}{2.256166in}}%
\pgfpathlineto{\pgfqpoint{4.186899in}{2.291455in}}%
\pgfpathlineto{\pgfqpoint{4.166477in}{2.394806in}}%
\pgfpathlineto{\pgfqpoint{4.146758in}{2.559274in}}%
\pgfpathlineto{\pgfqpoint{4.128216in}{2.644939in}}%
\pgfpathlineto{\pgfqpoint{4.109906in}{2.572461in}}%
\pgfpathlineto{\pgfqpoint{4.093006in}{2.404984in}}%
\pgfpathlineto{\pgfqpoint{4.067656in}{2.289178in}}%
\pgfpathlineto{\pgfqpoint{4.051460in}{2.260366in}}%
\pgfpathlineto{\pgfqpoint{4.033384in}{2.248763in}}%
\pgfpathlineto{\pgfqpoint{4.013199in}{2.268770in}}%
\pgfpathlineto{\pgfqpoint{3.994185in}{2.316258in}}%
\pgfpathlineto{\pgfqpoint{3.972824in}{2.458529in}}%
\pgfpathlineto{\pgfqpoint{3.954282in}{2.589913in}}%
\pgfpathlineto{\pgfqpoint{3.938554in}{2.635389in}}%
\pgfpathlineto{\pgfqpoint{3.917195in}{2.522374in}}%
\pgfpathlineto{\pgfqpoint{3.899590in}{2.364861in}}%
\pgfpathlineto{\pgfqpoint{3.875646in}{2.286207in}}%
\pgfpathlineto{\pgfqpoint{3.860389in}{2.258515in}}%
\pgfpathlineto{\pgfqpoint{3.841142in}{2.246469in}}%
\pgfpathlineto{\pgfqpoint{3.822834in}{2.259604in}}%
\pgfpathlineto{\pgfqpoint{3.801941in}{2.285818in}}%
\pgfpathlineto{\pgfqpoint{3.779173in}{2.373505in}}%
\pgfpathlineto{\pgfqpoint{3.764854in}{2.487985in}}%
\pgfpathlineto{\pgfqpoint{3.741147in}{2.613604in}}%
\pgfpathlineto{\pgfqpoint{3.721665in}{2.608007in}}%
\pgfpathlineto{\pgfqpoint{3.684578in}{2.355530in}}%
\pgfpathlineto{\pgfqpoint{3.666033in}{2.298407in}}%
\pgfpathlineto{\pgfqpoint{3.647489in}{2.600102in}}%
\pgfpathlineto{\pgfqpoint{3.628478in}{2.613217in}}%
\pgfpathlineto{\pgfqpoint{3.611107in}{2.486809in}}%
\pgfpathlineto{\pgfqpoint{3.590685in}{2.347440in}}%
\pgfpathlineto{\pgfqpoint{3.571907in}{2.282626in}}%
\pgfpathlineto{\pgfqpoint{3.549608in}{2.253053in}}%
\pgfpathlineto{\pgfqpoint{3.532237in}{2.246497in}}%
\pgfpathlineto{\pgfqpoint{3.513929in}{2.260733in}}%
\pgfpathlineto{\pgfqpoint{3.495150in}{2.298542in}}%
\pgfpathlineto{\pgfqpoint{3.473555in}{2.412116in}}%
\pgfpathlineto{\pgfqpoint{3.454778in}{2.561612in}}%
\pgfpathlineto{\pgfqpoint{3.437642in}{2.620930in}}%
\pgfpathlineto{\pgfqpoint{3.417454in}{2.556371in}}%
\pgfpathlineto{\pgfqpoint{3.400086in}{2.420925in}}%
\pgfpathlineto{\pgfqpoint{3.374265in}{2.319511in}}%
\pgfpathlineto{\pgfqpoint{3.359946in}{2.280503in}}%
\pgfpathlineto{\pgfqpoint{3.338352in}{2.251553in}}%
\pgfpathlineto{\pgfqpoint{3.320042in}{2.245758in}}%
\pgfpathlineto{\pgfqpoint{3.300794in}{2.260791in}}%
\pgfpathlineto{\pgfqpoint{3.283424in}{2.285569in}}%
\pgfpathlineto{\pgfqpoint{3.263239in}{2.348453in}}%
\pgfpathlineto{\pgfqpoint{3.246337in}{2.439810in}}%
\pgfpathlineto{\pgfqpoint{3.225212in}{2.596187in}}%
\pgfpathlineto{\pgfqpoint{3.208076in}{2.612847in}}%
\pgfpathlineto{\pgfqpoint{3.185308in}{2.488108in}}%
\pgfpathlineto{\pgfqpoint{3.166295in}{2.363737in}}%
\pgfpathlineto{\pgfqpoint{3.148221in}{2.291663in}}%
\pgfpathlineto{\pgfqpoint{3.129911in}{2.260517in}}%
\pgfpathlineto{\pgfqpoint{3.107612in}{2.244962in}}%
\pgfpathlineto{\pgfqpoint{3.092824in}{2.250445in}}%
\pgfpathlineto{\pgfqpoint{3.070056in}{2.275127in}}%
\pgfpathlineto{\pgfqpoint{3.051278in}{2.320703in}}%
\pgfpathlineto{\pgfqpoint{3.033204in}{2.429667in}}%
\pgfpathlineto{\pgfqpoint{3.014425in}{2.514290in}}%
\pgfpathlineto{\pgfqpoint{2.996115in}{2.588108in}}%
\pgfpathlineto{\pgfqpoint{2.978747in}{2.605516in}}%
\pgfpathlineto{\pgfqpoint{2.955508in}{2.479579in}}%
\pgfpathlineto{\pgfqpoint{2.938372in}{2.361871in}}%
\pgfpathlineto{\pgfqpoint{2.915604in}{2.288373in}}%
\pgfpathlineto{\pgfqpoint{2.899877in}{2.259639in}}%
\pgfpathlineto{\pgfqpoint{2.881332in}{2.244504in}}%
\pgfpathlineto{\pgfqpoint{2.859268in}{2.246581in}}%
\pgfpathlineto{\pgfqpoint{2.841194in}{2.263128in}}%
\pgfpathlineto{\pgfqpoint{2.822415in}{2.302109in}}%
\pgfpathlineto{\pgfqpoint{2.800351in}{2.406150in}}%
\pgfpathlineto{\pgfqpoint{2.785328in}{2.517035in}}%
\pgfpathlineto{\pgfqpoint{2.759744in}{2.611604in}}%
\pgfpathlineto{\pgfqpoint{2.747773in}{2.592183in}}%
\pgfpathlineto{\pgfqpoint{2.726177in}{2.474855in}}%
\pgfpathlineto{\pgfqpoint{2.706226in}{2.353794in}}%
\pgfpathlineto{\pgfqpoint{2.686273in}{2.288594in}}%
\pgfpathlineto{\pgfqpoint{2.668434in}{2.257185in}}%
\pgfpathlineto{\pgfqpoint{2.645900in}{2.244816in}}%
\pgfpathlineto{\pgfqpoint{2.626887in}{2.244798in}}%
\pgfpathlineto{\pgfqpoint{2.609282in}{2.258264in}}%
\pgfpathlineto{\pgfqpoint{2.590738in}{2.282165in}}%
\pgfpathlineto{\pgfqpoint{2.571725in}{2.344655in}}%
\pgfpathlineto{\pgfqpoint{2.553651in}{2.467934in}}%
\pgfpathlineto{\pgfqpoint{2.533934in}{2.472590in}}%
\pgfpathlineto{\pgfqpoint{2.516095in}{2.592637in}}%
\pgfpathlineto{\pgfqpoint{2.494734in}{2.599161in}}%
\pgfpathlineto{\pgfqpoint{2.475486in}{2.505657in}}%
\pgfpathlineto{\pgfqpoint{2.459055in}{2.386450in}}%
\pgfpathlineto{\pgfqpoint{2.437460in}{2.299360in}}%
\pgfpathlineto{\pgfqpoint{2.418212in}{2.272473in}}%
\pgfpathlineto{\pgfqpoint{2.399433in}{2.250295in}}%
\pgfpathlineto{\pgfqpoint{2.379717in}{2.244154in}}%
\pgfpathlineto{\pgfqpoint{2.360235in}{2.259623in}}%
\pgfpathlineto{\pgfqpoint{2.339344in}{2.294629in}}%
\pgfpathlineto{\pgfqpoint{2.321034in}{2.371776in}}%
\pgfpathlineto{\pgfqpoint{2.303664in}{2.491930in}}%
\pgfpathlineto{\pgfqpoint{2.285824in}{2.601702in}}%
\pgfpathlineto{\pgfqpoint{2.265873in}{2.609908in}}%
\pgfpathlineto{\pgfqpoint{2.244983in}{2.515419in}}%
\pgfpathlineto{\pgfqpoint{2.222448in}{2.366137in}}%
\pgfpathlineto{\pgfqpoint{2.200617in}{2.315039in}}%
\pgfpathlineto{\pgfqpoint{2.189820in}{2.284138in}}%
\pgfpathlineto{\pgfqpoint{2.166347in}{2.257778in}}%
\pgfpathlineto{\pgfqpoint{2.148039in}{2.246182in}}%
\pgfpathlineto{\pgfqpoint{2.128086in}{2.254727in}}%
\pgfpathlineto{\pgfqpoint{2.112125in}{2.269533in}}%
\pgfpathlineto{\pgfqpoint{2.090765in}{2.315278in}}%
\pgfpathlineto{\pgfqpoint{2.070578in}{2.382681in}}%
\pgfpathlineto{\pgfqpoint{2.052270in}{2.512280in}}%
\pgfpathlineto{\pgfqpoint{2.033725in}{2.594420in}}%
\pgfpathlineto{\pgfqpoint{2.015183in}{2.613439in}}%
\pgfpathlineto{\pgfqpoint{1.997342in}{2.586998in}}%
\pgfpathlineto{\pgfqpoint{1.972931in}{2.414690in}}%
\pgfpathlineto{\pgfqpoint{1.956500in}{2.383880in}}%
\pgfpathlineto{\pgfqpoint{1.937487in}{2.314339in}}%
\pgfpathlineto{\pgfqpoint{1.919177in}{2.274372in}}%
\pgfpathlineto{\pgfqpoint{1.901338in}{2.251771in}}%
\pgfpathlineto{\pgfqpoint{1.878804in}{2.246558in}}%
\pgfpathlineto{\pgfqpoint{1.860731in}{2.258009in}}%
\pgfpathlineto{\pgfqpoint{1.845238in}{2.281127in}}%
\pgfpathlineto{\pgfqpoint{1.824113in}{2.334434in}}%
\pgfpathlineto{\pgfqpoint{1.802282in}{2.463088in}}%
\pgfpathlineto{\pgfqpoint{1.783738in}{2.553618in}}%
\pgfpathlineto{\pgfqpoint{1.765664in}{2.615317in}}%
\pgfpathlineto{\pgfqpoint{1.747122in}{2.603323in}}%
\pgfpathlineto{\pgfqpoint{1.727169in}{2.512222in}}%
\pgfpathlineto{\pgfqpoint{1.707452in}{2.383667in}}%
\pgfpathlineto{\pgfqpoint{1.688908in}{2.317130in}}%
\pgfpathlineto{\pgfqpoint{1.666843in}{2.287344in}}%
\pgfpathlineto{\pgfqpoint{1.652056in}{2.264380in}}%
\pgfpathlineto{\pgfqpoint{1.629757in}{2.247773in}}%
\pgfpathlineto{\pgfqpoint{1.608395in}{2.251826in}}%
\pgfpathlineto{\pgfqpoint{1.590322in}{2.262042in}}%
\pgfpathlineto{\pgfqpoint{1.571779in}{2.290101in}}%
\pgfpathlineto{\pgfqpoint{1.553703in}{2.346216in}}%
\pgfpathlineto{\pgfqpoint{1.534690in}{2.438882in}}%
\pgfpathlineto{\pgfqpoint{1.515208in}{2.570917in}}%
\pgfpathlineto{\pgfqpoint{1.497603in}{2.612238in}}%
\pgfpathlineto{\pgfqpoint{1.477652in}{2.627376in}}%
\pgfpathlineto{\pgfqpoint{1.458168in}{2.588473in}}%
\pgfpathlineto{\pgfqpoint{1.438686in}{2.460818in}}%
\pgfpathlineto{\pgfqpoint{1.418030in}{2.369657in}}%
\pgfpathlineto{\pgfqpoint{1.399956in}{2.308456in}}%
\pgfpathlineto{\pgfqpoint{1.359582in}{2.262542in}}%
\pgfpathlineto{\pgfqpoint{1.341979in}{2.249005in}}%
\pgfpathlineto{\pgfqpoint{1.324843in}{2.249875in}}%
\pgfpathlineto{\pgfqpoint{1.304421in}{2.252668in}}%
\pgfpathlineto{\pgfqpoint{1.281653in}{2.269660in}}%
\pgfpathlineto{\pgfqpoint{1.266395in}{2.297719in}}%
\pgfpathlineto{\pgfqpoint{1.246678in}{2.371108in}}%
\pgfpathlineto{\pgfqpoint{1.225082in}{2.507303in}}%
\pgfpathlineto{\pgfqpoint{1.208417in}{2.568564in}}%
\pgfpathlineto{\pgfqpoint{1.187996in}{2.636624in}}%
\pgfpathlineto{\pgfqpoint{1.164054in}{2.614005in}}%
\pgfpathlineto{\pgfqpoint{1.149266in}{2.547645in}}%
\pgfpathlineto{\pgfqpoint{1.131661in}{2.436258in}}%
\pgfpathlineto{\pgfqpoint{1.111474in}{2.336862in}}%
\pgfpathlineto{\pgfqpoint{1.090818in}{2.291537in}}%
\pgfpathlineto{\pgfqpoint{1.051148in}{2.255751in}}%
\pgfpathlineto{\pgfqpoint{1.030492in}{2.250078in}}%
\pgfpathlineto{\pgfqpoint{1.010070in}{2.261804in}}%
\pgfpathlineto{\pgfqpoint{0.995048in}{2.278689in}}%
\pgfpathlineto{\pgfqpoint{0.976504in}{2.307390in}}%
\pgfpathlineto{\pgfqpoint{0.955379in}{2.361239in}}%
\pgfpathlineto{\pgfqpoint{0.938713in}{2.446478in}}%
\pgfpathlineto{\pgfqpoint{0.921109in}{2.565962in}}%
\pgfpathlineto{\pgfqpoint{0.903035in}{2.634657in}}%
\pgfpathlineto{\pgfqpoint{0.882379in}{2.650118in}}%
\pgfpathlineto{\pgfqpoint{0.861252in}{2.574592in}}%
\pgfpathlineto{\pgfqpoint{0.843178in}{2.461698in}}%
\pgfpathlineto{\pgfqpoint{0.823225in}{2.358147in}}%
\pgfpathlineto{\pgfqpoint{0.803509in}{2.300419in}}%
\pgfpathlineto{\pgfqpoint{0.784261in}{2.277393in}}%
\pgfpathlineto{\pgfqpoint{0.766188in}{2.261530in}}%
\pgfpathlineto{\pgfqpoint{0.748112in}{2.252118in}}%
\pgfpathlineto{\pgfqpoint{0.725813in}{2.262203in}}%
\pgfpathlineto{\pgfqpoint{0.707739in}{2.282794in}}%
\pgfpathlineto{\pgfqpoint{0.690840in}{2.324179in}}%
\pgfpathlineto{\pgfqpoint{0.669713in}{2.389037in}}%
\pgfpathlineto{\pgfqpoint{0.652108in}{2.472879in}}%
\pgfpathlineto{\pgfqpoint{0.651405in}{2.468013in}}%
\pgfpathlineto{\pgfqpoint{0.654456in}{2.428709in}}%
\pgfpathlineto{\pgfqpoint{0.676755in}{2.305156in}}%
\pgfpathlineto{\pgfqpoint{0.695768in}{2.262265in}}%
\pgfpathlineto{\pgfqpoint{0.714311in}{2.252900in}}%
\pgfpathlineto{\pgfqpoint{0.733795in}{2.277825in}}%
\pgfpathlineto{\pgfqpoint{0.750929in}{2.334352in}}%
\pgfpathlineto{\pgfqpoint{0.770647in}{2.455148in}}%
\pgfpathlineto{\pgfqpoint{0.795293in}{2.643358in}}%
\pgfpathlineto{\pgfqpoint{0.811020in}{2.641951in}}%
\pgfpathlineto{\pgfqpoint{0.811489in}{2.585011in}}%
\pgfpathlineto{\pgfqpoint{0.830973in}{2.521127in}}%
\pgfpathlineto{\pgfqpoint{0.848107in}{2.368202in}}%
\pgfpathlineto{\pgfqpoint{0.866886in}{2.288057in}}%
\pgfpathlineto{\pgfqpoint{0.888247in}{2.252342in}}%
\pgfpathlineto{\pgfqpoint{0.906790in}{2.255711in}}%
\pgfpathlineto{\pgfqpoint{0.925100in}{2.284735in}}%
\pgfpathlineto{\pgfqpoint{0.944347in}{2.351826in}}%
\pgfpathlineto{\pgfqpoint{0.965238in}{2.509526in}}%
\pgfpathlineto{\pgfqpoint{0.983311in}{2.632092in}}%
\pgfpathlineto{\pgfqpoint{1.001856in}{2.617859in}}%
\pgfpathlineto{\pgfqpoint{1.022512in}{2.459574in}}%
\pgfpathlineto{\pgfqpoint{1.040820in}{2.335600in}}%
\pgfpathlineto{\pgfqpoint{1.058425in}{2.275933in}}%
\pgfpathlineto{\pgfqpoint{1.079315in}{2.249007in}}%
\pgfpathlineto{\pgfqpoint{1.098094in}{2.255922in}}%
\pgfpathlineto{\pgfqpoint{1.115699in}{2.284170in}}%
\pgfpathlineto{\pgfqpoint{1.134947in}{2.361537in}}%
\pgfpathlineto{\pgfqpoint{1.157246in}{2.535916in}}%
\pgfpathlineto{\pgfqpoint{1.175085in}{2.630122in}}%
\pgfpathlineto{\pgfqpoint{1.192924in}{2.596905in}}%
\pgfpathlineto{\pgfqpoint{1.213346in}{2.427961in}}%
\pgfpathlineto{\pgfqpoint{1.233768in}{2.314425in}}%
\pgfpathlineto{\pgfqpoint{1.250198in}{2.266269in}}%
\pgfpathlineto{\pgfqpoint{1.271560in}{2.246632in}}%
\pgfpathlineto{\pgfqpoint{1.292450in}{2.258780in}}%
\pgfpathlineto{\pgfqpoint{1.311932in}{2.297130in}}%
\pgfpathlineto{\pgfqpoint{1.327894in}{2.357934in}}%
\pgfpathlineto{\pgfqpoint{1.353714in}{2.554429in}}%
\pgfpathlineto{\pgfqpoint{1.365921in}{2.611393in}}%
\pgfpathlineto{\pgfqpoint{1.388454in}{2.595833in}}%
\pgfpathlineto{\pgfqpoint{1.404651in}{2.459751in}}%
\pgfpathlineto{\pgfqpoint{1.426012in}{2.321405in}}%
\pgfpathlineto{\pgfqpoint{1.445260in}{2.267597in}}%
\pgfpathlineto{\pgfqpoint{1.462159in}{2.249659in}}%
\pgfpathlineto{\pgfqpoint{1.482347in}{2.249379in}}%
\pgfpathlineto{\pgfqpoint{1.501125in}{2.270585in}}%
\pgfpathlineto{\pgfqpoint{1.519199in}{2.315004in}}%
\pgfpathlineto{\pgfqpoint{1.540090in}{2.426696in}}%
\pgfpathlineto{\pgfqpoint{1.557929in}{2.570985in}}%
\pgfpathlineto{\pgfqpoint{1.580462in}{2.615202in}}%
\pgfpathlineto{\pgfqpoint{1.598538in}{2.522032in}}%
\pgfpathlineto{\pgfqpoint{1.619428in}{2.394885in}}%
\pgfpathlineto{\pgfqpoint{1.636564in}{2.304569in}}%
\pgfpathlineto{\pgfqpoint{1.652995in}{2.265248in}}%
\pgfpathlineto{\pgfqpoint{1.673651in}{2.246010in}}%
\pgfpathlineto{\pgfqpoint{1.694776in}{2.252797in}}%
\pgfpathlineto{\pgfqpoint{1.712850in}{2.273941in}}%
\pgfpathlineto{\pgfqpoint{1.733977in}{2.314378in}}%
\pgfpathlineto{\pgfqpoint{1.752754in}{2.403119in}}%
\pgfpathlineto{\pgfqpoint{1.769890in}{2.508644in}}%
\pgfpathlineto{\pgfqpoint{1.790780in}{2.614055in}}%
\pgfpathlineto{\pgfqpoint{1.809090in}{2.601527in}}%
\pgfpathlineto{\pgfqpoint{1.826695in}{2.493158in}}%
\pgfpathlineto{\pgfqpoint{1.847820in}{2.342848in}}%
\pgfpathlineto{\pgfqpoint{1.865894in}{2.277276in}}%
\pgfpathlineto{\pgfqpoint{1.886550in}{2.254739in}}%
\pgfpathlineto{\pgfqpoint{1.905563in}{2.245356in}}%
\pgfpathlineto{\pgfqpoint{1.925516in}{2.253193in}}%
\pgfpathlineto{\pgfqpoint{1.944529in}{2.278842in}}%
\pgfpathlineto{\pgfqpoint{1.965654in}{2.344500in}}%
\pgfpathlineto{\pgfqpoint{1.979502in}{2.447085in}}%
\pgfpathlineto{\pgfqpoint{2.001098in}{2.546144in}}%
\pgfpathlineto{\pgfqpoint{2.022223in}{2.614812in}}%
\pgfpathlineto{\pgfqpoint{2.039359in}{2.595879in}}%
\pgfpathlineto{\pgfqpoint{2.057198in}{2.469415in}}%
\pgfpathlineto{\pgfqpoint{2.079263in}{2.337738in}}%
\pgfpathlineto{\pgfqpoint{2.100154in}{2.271723in}}%
\pgfpathlineto{\pgfqpoint{2.118464in}{2.250441in}}%
\pgfpathlineto{\pgfqpoint{2.136068in}{2.244172in}}%
\pgfpathlineto{\pgfqpoint{2.154142in}{2.257835in}}%
\pgfpathlineto{\pgfqpoint{2.175501in}{2.291124in}}%
\pgfpathlineto{\pgfqpoint{2.192872in}{2.356012in}}%
\pgfpathlineto{\pgfqpoint{2.214233in}{2.498231in}}%
\pgfpathlineto{\pgfqpoint{2.231602in}{2.530280in}}%
\pgfpathlineto{\pgfqpoint{2.256249in}{2.303723in}}%
\pgfpathlineto{\pgfqpoint{2.270099in}{2.373250in}}%
\pgfpathlineto{\pgfqpoint{2.289347in}{2.457042in}}%
\pgfpathlineto{\pgfqpoint{2.309532in}{2.600520in}}%
\pgfpathlineto{\pgfqpoint{2.326433in}{2.603592in}}%
\pgfpathlineto{\pgfqpoint{2.367980in}{2.329049in}}%
\pgfpathlineto{\pgfqpoint{2.385116in}{2.271839in}}%
\pgfpathlineto{\pgfqpoint{2.406007in}{2.248406in}}%
\pgfpathlineto{\pgfqpoint{2.424315in}{2.245100in}}%
\pgfpathlineto{\pgfqpoint{2.444971in}{2.262838in}}%
\pgfpathlineto{\pgfqpoint{2.462576in}{2.297783in}}%
\pgfpathlineto{\pgfqpoint{2.481120in}{2.376648in}}%
\pgfpathlineto{\pgfqpoint{2.518441in}{2.593766in}}%
\pgfpathlineto{\pgfqpoint{2.538629in}{2.601012in}}%
\pgfpathlineto{\pgfqpoint{2.559519in}{2.500943in}}%
\pgfpathlineto{\pgfqpoint{2.577593in}{2.369582in}}%
\pgfpathlineto{\pgfqpoint{2.595668in}{2.294728in}}%
\pgfpathlineto{\pgfqpoint{2.616325in}{2.257684in}}%
\pgfpathlineto{\pgfqpoint{2.634164in}{2.244785in}}%
\pgfpathlineto{\pgfqpoint{2.655994in}{2.251020in}}%
\pgfpathlineto{\pgfqpoint{2.673362in}{2.273267in}}%
\pgfpathlineto{\pgfqpoint{2.691438in}{2.306120in}}%
\pgfpathlineto{\pgfqpoint{2.713032in}{2.407844in}}%
\pgfpathlineto{\pgfqpoint{2.734159in}{2.546178in}}%
\pgfpathlineto{\pgfqpoint{2.751998in}{2.607808in}}%
\pgfpathlineto{\pgfqpoint{2.770306in}{2.589841in}}%
\pgfpathlineto{\pgfqpoint{2.812792in}{2.323852in}}%
\pgfpathlineto{\pgfqpoint{2.827580in}{2.282167in}}%
\pgfpathlineto{\pgfqpoint{2.848940in}{2.251829in}}%
\pgfpathlineto{\pgfqpoint{2.866781in}{2.246276in}}%
\pgfpathlineto{\pgfqpoint{2.884386in}{2.247091in}}%
\pgfpathlineto{\pgfqpoint{2.906450in}{2.270691in}}%
\pgfpathlineto{\pgfqpoint{2.923115in}{2.311965in}}%
\pgfpathlineto{\pgfqpoint{2.946354in}{2.377037in}}%
\pgfpathlineto{\pgfqpoint{2.959733in}{2.467249in}}%
\pgfpathlineto{\pgfqpoint{2.981093in}{2.580511in}}%
\pgfpathlineto{\pgfqpoint{3.002454in}{2.610765in}}%
\pgfpathlineto{\pgfqpoint{3.019354in}{2.548265in}}%
\pgfpathlineto{\pgfqpoint{3.037664in}{2.409275in}}%
\pgfpathlineto{\pgfqpoint{3.056441in}{2.315451in}}%
\pgfpathlineto{\pgfqpoint{3.076159in}{2.270837in}}%
\pgfpathlineto{\pgfqpoint{3.098458in}{2.251090in}}%
\pgfpathlineto{\pgfqpoint{3.115358in}{2.245115in}}%
\pgfpathlineto{\pgfqpoint{3.133197in}{2.253599in}}%
\pgfpathlineto{\pgfqpoint{3.154793in}{2.277130in}}%
\pgfpathlineto{\pgfqpoint{3.174980in}{2.323214in}}%
\pgfpathlineto{\pgfqpoint{3.193993in}{2.408301in}}%
\pgfpathlineto{\pgfqpoint{3.212301in}{2.540576in}}%
\pgfpathlineto{\pgfqpoint{3.232489in}{2.616570in}}%
\pgfpathlineto{\pgfqpoint{3.250797in}{2.612398in}}%
\pgfpathlineto{\pgfqpoint{3.272393in}{2.525662in}}%
\pgfpathlineto{\pgfqpoint{3.287180in}{2.424336in}}%
\pgfpathlineto{\pgfqpoint{3.307837in}{2.332272in}}%
\pgfpathlineto{\pgfqpoint{3.327084in}{2.281779in}}%
\pgfpathlineto{\pgfqpoint{3.345392in}{2.257822in}}%
\pgfpathlineto{\pgfqpoint{3.367222in}{2.247530in}}%
\pgfpathlineto{\pgfqpoint{3.384358in}{2.249022in}}%
\pgfpathlineto{\pgfqpoint{3.401729in}{2.265054in}}%
\pgfpathlineto{\pgfqpoint{3.423323in}{2.257534in}}%
\pgfpathlineto{\pgfqpoint{3.443510in}{2.288930in}}%
\pgfpathlineto{\pgfqpoint{3.462054in}{2.337004in}}%
\pgfpathlineto{\pgfqpoint{3.481771in}{2.411526in}}%
\pgfpathlineto{\pgfqpoint{3.499376in}{2.528320in}}%
\pgfpathlineto{\pgfqpoint{3.520266in}{2.624259in}}%
\pgfpathlineto{\pgfqpoint{3.537637in}{2.612909in}}%
\pgfpathlineto{\pgfqpoint{3.556415in}{2.519846in}}%
\pgfpathlineto{\pgfqpoint{3.577540in}{2.386120in}}%
\pgfpathlineto{\pgfqpoint{3.599605in}{2.306908in}}%
\pgfpathlineto{\pgfqpoint{3.616975in}{2.274464in}}%
\pgfpathlineto{\pgfqpoint{3.635049in}{2.253219in}}%
\pgfpathlineto{\pgfqpoint{3.656410in}{2.257658in}}%
\pgfpathlineto{\pgfqpoint{3.672370in}{2.247251in}}%
\pgfpathlineto{\pgfqpoint{3.691149in}{2.253066in}}%
\pgfpathlineto{\pgfqpoint{3.713214in}{2.265619in}}%
\pgfpathlineto{\pgfqpoint{3.732696in}{2.305891in}}%
\pgfpathlineto{\pgfqpoint{3.748658in}{2.352686in}}%
\pgfpathlineto{\pgfqpoint{3.766732in}{2.454609in}}%
\pgfpathlineto{\pgfqpoint{3.789032in}{2.605786in}}%
\pgfpathlineto{\pgfqpoint{3.808280in}{2.641280in}}%
\pgfpathlineto{\pgfqpoint{3.827293in}{2.615777in}}%
\pgfpathlineto{\pgfqpoint{3.846541in}{2.505388in}}%
\pgfpathlineto{\pgfqpoint{3.865554in}{2.382972in}}%
\pgfpathlineto{\pgfqpoint{3.884331in}{2.314158in}}%
\pgfpathlineto{\pgfqpoint{3.903815in}{2.273288in}}%
\pgfpathlineto{\pgfqpoint{3.922358in}{2.254527in}}%
\pgfpathlineto{\pgfqpoint{3.943483in}{2.248626in}}%
\pgfpathlineto{\pgfqpoint{3.962967in}{2.258930in}}%
\pgfpathlineto{\pgfqpoint{3.981040in}{2.282323in}}%
\pgfpathlineto{\pgfqpoint{3.999348in}{2.321759in}}%
\pgfpathlineto{\pgfqpoint{4.018596in}{2.399361in}}%
\pgfpathlineto{\pgfqpoint{4.038080in}{2.527367in}}%
\pgfpathlineto{\pgfqpoint{4.056623in}{2.616000in}}%
\pgfpathlineto{\pgfqpoint{4.077513in}{2.651204in}}%
\pgfpathlineto{\pgfqpoint{4.096527in}{2.604508in}}%
\pgfpathlineto{\pgfqpoint{4.117888in}{2.653950in}}%
\pgfpathlineto{\pgfqpoint{4.133850in}{2.631328in}}%
\pgfpathlineto{\pgfqpoint{4.151689in}{2.531450in}}%
\pgfpathlineto{\pgfqpoint{4.174691in}{2.410867in}}%
\pgfpathlineto{\pgfqpoint{4.189479in}{2.345464in}}%
\pgfpathlineto{\pgfqpoint{4.209666in}{2.292827in}}%
\pgfpathlineto{\pgfqpoint{4.231262in}{2.258512in}}%
\pgfpathlineto{\pgfqpoint{4.250041in}{2.250795in}}%
\pgfpathlineto{\pgfqpoint{4.269523in}{2.260897in}}%
\pgfpathlineto{\pgfqpoint{4.288771in}{2.286792in}}%
\pgfpathlineto{\pgfqpoint{4.307313in}{2.335495in}}%
\pgfpathlineto{\pgfqpoint{4.346045in}{2.519232in}}%
\pgfpathlineto{\pgfqpoint{4.346514in}{2.567184in}}%
\pgfpathlineto{\pgfqpoint{4.363884in}{2.602413in}}%
\pgfpathlineto{\pgfqpoint{4.385244in}{2.666248in}}%
\pgfpathlineto{\pgfqpoint{4.404257in}{2.630655in}}%
\pgfpathlineto{\pgfqpoint{4.423270in}{2.539833in}}%
\pgfpathlineto{\pgfqpoint{4.441815in}{2.426766in}}%
\pgfpathlineto{\pgfqpoint{4.460828in}{2.340771in}}%
\pgfpathlineto{\pgfqpoint{4.480310in}{2.293381in}}%
\pgfpathlineto{\pgfqpoint{4.473737in}{2.311205in}}%
\pgfpathlineto{\pgfqpoint{4.456837in}{2.405185in}}%
\pgfpathlineto{\pgfqpoint{4.435712in}{2.590563in}}%
\pgfpathlineto{\pgfqpoint{4.417871in}{2.269700in}}%
\pgfpathlineto{\pgfqpoint{4.396980in}{2.250008in}}%
\pgfpathlineto{\pgfqpoint{4.373273in}{2.284174in}}%
\pgfpathlineto{\pgfqpoint{4.360833in}{2.317980in}}%
\pgfpathlineto{\pgfqpoint{4.340411in}{2.421839in}}%
\pgfpathlineto{\pgfqpoint{4.322101in}{2.585451in}}%
\pgfpathlineto{\pgfqpoint{4.301916in}{2.655150in}}%
\pgfpathlineto{\pgfqpoint{4.284545in}{2.543177in}}%
\pgfpathlineto{\pgfqpoint{4.260369in}{2.357202in}}%
\pgfpathlineto{\pgfqpoint{4.245816in}{2.294394in}}%
\pgfpathlineto{\pgfqpoint{4.224923in}{2.255039in}}%
\pgfpathlineto{\pgfqpoint{4.203329in}{2.255118in}}%
\pgfpathlineto{\pgfqpoint{4.185019in}{2.286398in}}%
\pgfpathlineto{\pgfqpoint{4.167180in}{2.361444in}}%
\pgfpathlineto{\pgfqpoint{4.149575in}{2.510958in}}%
\pgfpathlineto{\pgfqpoint{4.128216in}{2.640982in}}%
\pgfpathlineto{\pgfqpoint{4.111080in}{2.612349in}}%
\pgfpathlineto{\pgfqpoint{4.088781in}{2.410668in}}%
\pgfpathlineto{\pgfqpoint{4.070942in}{2.311270in}}%
\pgfpathlineto{\pgfqpoint{4.052397in}{2.265425in}}%
\pgfpathlineto{\pgfqpoint{4.031976in}{2.247424in}}%
\pgfpathlineto{\pgfqpoint{4.013199in}{2.264349in}}%
\pgfpathlineto{\pgfqpoint{3.994420in}{2.312598in}}%
\pgfpathlineto{\pgfqpoint{3.973059in}{2.447919in}}%
\pgfpathlineto{\pgfqpoint{3.958036in}{2.578636in}}%
\pgfpathlineto{\pgfqpoint{3.935268in}{2.630570in}}%
\pgfpathlineto{\pgfqpoint{3.917195in}{2.536362in}}%
\pgfpathlineto{\pgfqpoint{3.897947in}{2.373300in}}%
\pgfpathlineto{\pgfqpoint{3.879637in}{2.294682in}}%
\pgfpathlineto{\pgfqpoint{3.857572in}{2.274351in}}%
\pgfpathlineto{\pgfqpoint{3.839733in}{2.248423in}}%
\pgfpathlineto{\pgfqpoint{3.820486in}{2.251124in}}%
\pgfpathlineto{\pgfqpoint{3.801238in}{2.279074in}}%
\pgfpathlineto{\pgfqpoint{3.783399in}{2.343543in}}%
\pgfpathlineto{\pgfqpoint{3.761568in}{2.466586in}}%
\pgfpathlineto{\pgfqpoint{3.743729in}{2.602671in}}%
\pgfpathlineto{\pgfqpoint{3.724716in}{2.622370in}}%
\pgfpathlineto{\pgfqpoint{3.708285in}{2.503917in}}%
\pgfpathlineto{\pgfqpoint{3.684812in}{2.354530in}}%
\pgfpathlineto{\pgfqpoint{3.668147in}{2.294044in}}%
\pgfpathlineto{\pgfqpoint{3.648897in}{2.265847in}}%
\pgfpathlineto{\pgfqpoint{3.629181in}{2.246825in}}%
\pgfpathlineto{\pgfqpoint{3.607116in}{2.253090in}}%
\pgfpathlineto{\pgfqpoint{3.585520in}{2.287681in}}%
\pgfpathlineto{\pgfqpoint{3.570264in}{2.324567in}}%
\pgfpathlineto{\pgfqpoint{3.549608in}{2.451929in}}%
\pgfpathlineto{\pgfqpoint{3.533177in}{2.562214in}}%
\pgfpathlineto{\pgfqpoint{3.514867in}{2.621762in}}%
\pgfpathlineto{\pgfqpoint{3.494447in}{2.612177in}}%
\pgfpathlineto{\pgfqpoint{3.474260in}{2.508786in}}%
\pgfpathlineto{\pgfqpoint{3.457358in}{2.387437in}}%
\pgfpathlineto{\pgfqpoint{3.438347in}{2.304258in}}%
\pgfpathlineto{\pgfqpoint{3.412995in}{2.258382in}}%
\pgfpathlineto{\pgfqpoint{3.396564in}{2.251473in}}%
\pgfpathlineto{\pgfqpoint{3.378021in}{2.245362in}}%
\pgfpathlineto{\pgfqpoint{3.359008in}{2.258747in}}%
\pgfpathlineto{\pgfqpoint{3.339760in}{2.291769in}}%
\pgfpathlineto{\pgfqpoint{3.320982in}{2.369240in}}%
\pgfpathlineto{\pgfqpoint{3.302203in}{2.493115in}}%
\pgfpathlineto{\pgfqpoint{3.280843in}{2.596654in}}%
\pgfpathlineto{\pgfqpoint{3.264176in}{2.612219in}}%
\pgfpathlineto{\pgfqpoint{3.243520in}{2.551449in}}%
\pgfpathlineto{\pgfqpoint{3.223569in}{2.395624in}}%
\pgfpathlineto{\pgfqpoint{3.205259in}{2.312258in}}%
\pgfpathlineto{\pgfqpoint{3.184603in}{2.266760in}}%
\pgfpathlineto{\pgfqpoint{3.165826in}{2.247332in}}%
\pgfpathlineto{\pgfqpoint{3.147516in}{2.450135in}}%
\pgfpathlineto{\pgfqpoint{3.129677in}{2.593883in}}%
\pgfpathlineto{\pgfqpoint{3.109960in}{2.605001in}}%
\pgfpathlineto{\pgfqpoint{3.066769in}{2.330932in}}%
\pgfpathlineto{\pgfqpoint{3.054563in}{2.288295in}}%
\pgfpathlineto{\pgfqpoint{3.035550in}{2.254902in}}%
\pgfpathlineto{\pgfqpoint{3.013720in}{2.244523in}}%
\pgfpathlineto{\pgfqpoint{2.995412in}{2.255682in}}%
\pgfpathlineto{\pgfqpoint{2.975695in}{2.295394in}}%
\pgfpathlineto{\pgfqpoint{2.955977in}{2.378670in}}%
\pgfpathlineto{\pgfqpoint{2.936495in}{2.528959in}}%
\pgfpathlineto{\pgfqpoint{2.918656in}{2.613712in}}%
\pgfpathlineto{\pgfqpoint{2.900346in}{2.576247in}}%
\pgfpathlineto{\pgfqpoint{2.879926in}{2.495010in}}%
\pgfpathlineto{\pgfqpoint{2.858799in}{2.351635in}}%
\pgfpathlineto{\pgfqpoint{2.840491in}{2.285180in}}%
\pgfpathlineto{\pgfqpoint{2.821712in}{2.255213in}}%
\pgfpathlineto{\pgfqpoint{2.803404in}{2.244525in}}%
\pgfpathlineto{\pgfqpoint{2.783685in}{2.253812in}}%
\pgfpathlineto{\pgfqpoint{2.763264in}{2.287012in}}%
\pgfpathlineto{\pgfqpoint{2.744721in}{2.360030in}}%
\pgfpathlineto{\pgfqpoint{2.744016in}{2.441145in}}%
\pgfpathlineto{\pgfqpoint{2.725708in}{2.498901in}}%
\pgfpathlineto{\pgfqpoint{2.707398in}{2.605429in}}%
\pgfpathlineto{\pgfqpoint{2.689090in}{2.608593in}}%
\pgfpathlineto{\pgfqpoint{2.666557in}{2.491001in}}%
\pgfpathlineto{\pgfqpoint{2.644961in}{2.343987in}}%
\pgfpathlineto{\pgfqpoint{2.629704in}{2.293645in}}%
\pgfpathlineto{\pgfqpoint{2.610456in}{2.259723in}}%
\pgfpathlineto{\pgfqpoint{2.589564in}{2.244610in}}%
\pgfpathlineto{\pgfqpoint{2.571490in}{2.251306in}}%
\pgfpathlineto{\pgfqpoint{2.552008in}{2.273003in}}%
\pgfpathlineto{\pgfqpoint{2.532760in}{2.323319in}}%
\pgfpathlineto{\pgfqpoint{2.493560in}{2.575993in}}%
\pgfpathlineto{\pgfqpoint{2.475721in}{2.610396in}}%
\pgfpathlineto{\pgfqpoint{2.456473in}{2.581069in}}%
\pgfpathlineto{\pgfqpoint{2.437460in}{2.479690in}}%
\pgfpathlineto{\pgfqpoint{2.418681in}{2.357280in}}%
\pgfpathlineto{\pgfqpoint{2.397556in}{2.282010in}}%
\pgfpathlineto{\pgfqpoint{2.378777in}{2.266924in}}%
\pgfpathlineto{\pgfqpoint{2.360938in}{2.247384in}}%
\pgfpathlineto{\pgfqpoint{2.341456in}{2.246169in}}%
\pgfpathlineto{\pgfqpoint{2.323617in}{2.259104in}}%
\pgfpathlineto{\pgfqpoint{2.302021in}{2.301023in}}%
\pgfpathlineto{\pgfqpoint{2.286530in}{2.367818in}}%
\pgfpathlineto{\pgfqpoint{2.263994in}{2.525214in}}%
\pgfpathlineto{\pgfqpoint{2.246626in}{2.606325in}}%
\pgfpathlineto{\pgfqpoint{2.224325in}{2.611160in}}%
\pgfpathlineto{\pgfqpoint{2.205548in}{2.533012in}}%
\pgfpathlineto{\pgfqpoint{2.187004in}{2.401886in}}%
\pgfpathlineto{\pgfqpoint{2.168930in}{2.320823in}}%
\pgfpathlineto{\pgfqpoint{2.147569in}{2.268787in}}%
\pgfpathlineto{\pgfqpoint{2.128321in}{2.252208in}}%
\pgfpathlineto{\pgfqpoint{2.112125in}{2.245331in}}%
\pgfpathlineto{\pgfqpoint{2.091939in}{2.256629in}}%
\pgfpathlineto{\pgfqpoint{2.070343in}{2.288580in}}%
\pgfpathlineto{\pgfqpoint{2.052973in}{2.355649in}}%
\pgfpathlineto{\pgfqpoint{2.033491in}{2.490363in}}%
\pgfpathlineto{\pgfqpoint{2.014009in}{2.599479in}}%
\pgfpathlineto{\pgfqpoint{1.995933in}{2.614660in}}%
\pgfpathlineto{\pgfqpoint{1.974105in}{2.593779in}}%
\pgfpathlineto{\pgfqpoint{1.955795in}{2.481402in}}%
\pgfpathlineto{\pgfqpoint{1.938425in}{2.607486in}}%
\pgfpathlineto{\pgfqpoint{1.915422in}{2.584543in}}%
\pgfpathlineto{\pgfqpoint{1.900634in}{2.498350in}}%
\pgfpathlineto{\pgfqpoint{1.881856in}{2.359548in}}%
\pgfpathlineto{\pgfqpoint{1.860496in}{2.289182in}}%
\pgfpathlineto{\pgfqpoint{1.841012in}{2.262753in}}%
\pgfpathlineto{\pgfqpoint{1.822939in}{2.247606in}}%
\pgfpathlineto{\pgfqpoint{1.804394in}{2.251010in}}%
\pgfpathlineto{\pgfqpoint{1.782330in}{2.277906in}}%
\pgfpathlineto{\pgfqpoint{1.767307in}{2.313676in}}%
\pgfpathlineto{\pgfqpoint{1.745713in}{2.421892in}}%
\pgfpathlineto{\pgfqpoint{1.726935in}{2.550065in}}%
\pgfpathlineto{\pgfqpoint{1.708861in}{2.619954in}}%
\pgfpathlineto{\pgfqpoint{1.687265in}{2.613606in}}%
\pgfpathlineto{\pgfqpoint{1.668957in}{2.522129in}}%
\pgfpathlineto{\pgfqpoint{1.650647in}{2.403202in}}%
\pgfpathlineto{\pgfqpoint{1.629522in}{2.323462in}}%
\pgfpathlineto{\pgfqpoint{1.610509in}{2.279750in}}%
\pgfpathlineto{\pgfqpoint{1.591496in}{2.254985in}}%
\pgfpathlineto{\pgfqpoint{1.573186in}{2.246266in}}%
\pgfpathlineto{\pgfqpoint{1.554878in}{2.252352in}}%
\pgfpathlineto{\pgfqpoint{1.534221in}{2.280505in}}%
\pgfpathlineto{\pgfqpoint{1.515443in}{2.330511in}}%
\pgfpathlineto{\pgfqpoint{1.496195in}{2.428825in}}%
\pgfpathlineto{\pgfqpoint{1.476478in}{2.568625in}}%
\pgfpathlineto{\pgfqpoint{1.456760in}{2.628726in}}%
\pgfpathlineto{\pgfqpoint{1.438686in}{2.619340in}}%
\pgfpathlineto{\pgfqpoint{1.420847in}{2.541199in}}%
\pgfpathlineto{\pgfqpoint{1.398314in}{2.420555in}}%
\pgfpathlineto{\pgfqpoint{1.379300in}{2.349042in}}%
\pgfpathlineto{\pgfqpoint{1.361227in}{2.299619in}}%
\pgfpathlineto{\pgfqpoint{1.340334in}{2.264592in}}%
\pgfpathlineto{\pgfqpoint{1.323200in}{2.342791in}}%
\pgfpathlineto{\pgfqpoint{1.304187in}{2.290558in}}%
\pgfpathlineto{\pgfqpoint{1.284234in}{2.261977in}}%
\pgfpathlineto{\pgfqpoint{1.267100in}{2.249074in}}%
\pgfpathlineto{\pgfqpoint{1.246209in}{2.250071in}}%
\pgfpathlineto{\pgfqpoint{1.226257in}{2.273059in}}%
\pgfpathlineto{\pgfqpoint{1.209357in}{2.310001in}}%
\pgfpathlineto{\pgfqpoint{1.188701in}{2.387540in}}%
\pgfpathlineto{\pgfqpoint{1.168279in}{2.491701in}}%
\pgfpathlineto{\pgfqpoint{1.151143in}{2.593578in}}%
\pgfpathlineto{\pgfqpoint{1.129547in}{2.641991in}}%
\pgfpathlineto{\pgfqpoint{1.109596in}{2.608571in}}%
\pgfpathlineto{\pgfqpoint{1.072039in}{2.375529in}}%
\pgfpathlineto{\pgfqpoint{1.054434in}{2.315523in}}%
\pgfpathlineto{\pgfqpoint{1.034952in}{2.282228in}}%
\pgfpathlineto{\pgfqpoint{1.016878in}{2.258941in}}%
\pgfpathlineto{\pgfqpoint{0.995988in}{2.249792in}}%
\pgfpathlineto{\pgfqpoint{0.977912in}{2.259283in}}%
\pgfpathlineto{\pgfqpoint{0.956787in}{2.281018in}}%
\pgfpathlineto{\pgfqpoint{0.940122in}{2.318841in}}%
\pgfpathlineto{\pgfqpoint{0.919700in}{2.389521in}}%
\pgfpathlineto{\pgfqpoint{0.899747in}{2.526391in}}%
\pgfpathlineto{\pgfqpoint{0.881439in}{2.623729in}}%
\pgfpathlineto{\pgfqpoint{0.861017in}{2.653376in}}%
\pgfpathlineto{\pgfqpoint{0.841535in}{2.617717in}}%
\pgfpathlineto{\pgfqpoint{0.823931in}{2.502994in}}%
\pgfpathlineto{\pgfqpoint{0.802569in}{2.380791in}}%
\pgfpathlineto{\pgfqpoint{0.784027in}{2.323016in}}%
\pgfpathlineto{\pgfqpoint{0.767125in}{2.288231in}}%
\pgfpathlineto{\pgfqpoint{0.747409in}{2.264269in}}%
\pgfpathlineto{\pgfqpoint{0.727456in}{2.251369in}}%
\pgfpathlineto{\pgfqpoint{0.707739in}{2.259084in}}%
\pgfpathlineto{\pgfqpoint{0.687318in}{2.282215in}}%
\pgfpathlineto{\pgfqpoint{0.669478in}{2.323191in}}%
\pgfpathlineto{\pgfqpoint{0.652342in}{2.386083in}}%
\pgfpathlineto{\pgfqpoint{0.651874in}{2.383743in}}%
\pgfpathlineto{\pgfqpoint{0.655865in}{2.361854in}}%
\pgfpathlineto{\pgfqpoint{0.674643in}{2.285749in}}%
\pgfpathlineto{\pgfqpoint{0.699054in}{2.252183in}}%
\pgfpathlineto{\pgfqpoint{0.712434in}{2.264644in}}%
\pgfpathlineto{\pgfqpoint{0.735203in}{2.319910in}}%
\pgfpathlineto{\pgfqpoint{0.754685in}{2.419123in}}%
\pgfpathlineto{\pgfqpoint{0.772290in}{2.588410in}}%
\pgfpathlineto{\pgfqpoint{0.791303in}{2.654966in}}%
\pgfpathlineto{\pgfqpoint{0.809846in}{2.577813in}}%
\pgfpathlineto{\pgfqpoint{0.828156in}{2.414653in}}%
\pgfpathlineto{\pgfqpoint{0.850221in}{2.298296in}}%
\pgfpathlineto{\pgfqpoint{0.868294in}{2.259745in}}%
\pgfpathlineto{\pgfqpoint{0.887776in}{2.252196in}}%
\pgfpathlineto{\pgfqpoint{0.906555in}{2.277901in}}%
\pgfpathlineto{\pgfqpoint{0.925100in}{2.336738in}}%
\pgfpathlineto{\pgfqpoint{0.943876in}{2.464390in}}%
\pgfpathlineto{\pgfqpoint{0.962655in}{2.613858in}}%
\pgfpathlineto{\pgfqpoint{0.984486in}{2.626889in}}%
\pgfpathlineto{\pgfqpoint{1.004673in}{2.494811in}}%
\pgfpathlineto{\pgfqpoint{1.022278in}{2.347596in}}%
\pgfpathlineto{\pgfqpoint{1.040586in}{2.280106in}}%
\pgfpathlineto{\pgfqpoint{1.057956in}{2.251448in}}%
\pgfpathlineto{\pgfqpoint{1.077673in}{2.254341in}}%
\pgfpathlineto{\pgfqpoint{1.095746in}{2.286864in}}%
\pgfpathlineto{\pgfqpoint{1.119454in}{2.365582in}}%
\pgfpathlineto{\pgfqpoint{1.136355in}{2.513281in}}%
\pgfpathlineto{\pgfqpoint{1.155134in}{2.626950in}}%
\pgfpathlineto{\pgfqpoint{1.173911in}{2.604702in}}%
\pgfpathlineto{\pgfqpoint{1.191987in}{2.457389in}}%
\pgfpathlineto{\pgfqpoint{1.213817in}{2.321400in}}%
\pgfpathlineto{\pgfqpoint{1.233768in}{2.266467in}}%
\pgfpathlineto{\pgfqpoint{1.252547in}{2.248382in}}%
\pgfpathlineto{\pgfqpoint{1.270151in}{2.252787in}}%
\pgfpathlineto{\pgfqpoint{1.290102in}{2.284892in}}%
\pgfpathlineto{\pgfqpoint{1.308647in}{2.349392in}}%
\pgfpathlineto{\pgfqpoint{1.326486in}{2.477066in}}%
\pgfpathlineto{\pgfqpoint{1.347376in}{2.618285in}}%
\pgfpathlineto{\pgfqpoint{1.367564in}{2.594984in}}%
\pgfpathlineto{\pgfqpoint{1.386106in}{2.448751in}}%
\pgfpathlineto{\pgfqpoint{1.404416in}{2.334618in}}%
\pgfpathlineto{\pgfqpoint{1.426246in}{2.269296in}}%
\pgfpathlineto{\pgfqpoint{1.442677in}{2.248552in}}%
\pgfpathlineto{\pgfqpoint{1.464742in}{2.254202in}}%
\pgfpathlineto{\pgfqpoint{1.481173in}{2.280545in}}%
\pgfpathlineto{\pgfqpoint{1.502298in}{2.345669in}}%
\pgfpathlineto{\pgfqpoint{1.542201in}{2.600810in}}%
\pgfpathlineto{\pgfqpoint{1.558163in}{2.616670in}}%
\pgfpathlineto{\pgfqpoint{1.578819in}{2.481893in}}%
\pgfpathlineto{\pgfqpoint{1.596659in}{2.605922in}}%
\pgfpathlineto{\pgfqpoint{1.618489in}{2.609996in}}%
\pgfpathlineto{\pgfqpoint{1.636799in}{2.535406in}}%
\pgfpathlineto{\pgfqpoint{1.656281in}{2.378482in}}%
\pgfpathlineto{\pgfqpoint{1.675294in}{2.285761in}}%
\pgfpathlineto{\pgfqpoint{1.693837in}{2.255263in}}%
\pgfpathlineto{\pgfqpoint{1.714493in}{2.246507in}}%
\pgfpathlineto{\pgfqpoint{1.732098in}{2.260916in}}%
\pgfpathlineto{\pgfqpoint{1.752754in}{2.296603in}}%
\pgfpathlineto{\pgfqpoint{1.771064in}{2.375577in}}%
\pgfpathlineto{\pgfqpoint{1.789606in}{2.507877in}}%
\pgfpathlineto{\pgfqpoint{1.809794in}{2.613019in}}%
\pgfpathlineto{\pgfqpoint{1.828807in}{2.589909in}}%
\pgfpathlineto{\pgfqpoint{1.849932in}{2.456467in}}%
\pgfpathlineto{\pgfqpoint{1.868242in}{2.331562in}}%
\pgfpathlineto{\pgfqpoint{1.868476in}{2.292033in}}%
\pgfpathlineto{\pgfqpoint{1.888427in}{2.273050in}}%
\pgfpathlineto{\pgfqpoint{1.907677in}{2.249746in}}%
\pgfpathlineto{\pgfqpoint{1.924576in}{2.244942in}}%
\pgfpathlineto{\pgfqpoint{1.945232in}{2.260295in}}%
\pgfpathlineto{\pgfqpoint{1.963072in}{2.289979in}}%
\pgfpathlineto{\pgfqpoint{1.981616in}{2.354705in}}%
\pgfpathlineto{\pgfqpoint{2.002507in}{2.509041in}}%
\pgfpathlineto{\pgfqpoint{2.023866in}{2.613492in}}%
\pgfpathlineto{\pgfqpoint{2.037951in}{2.595528in}}%
\pgfpathlineto{\pgfqpoint{2.059546in}{2.500690in}}%
\pgfpathlineto{\pgfqpoint{2.077386in}{2.370074in}}%
\pgfpathlineto{\pgfqpoint{2.098276in}{2.285517in}}%
\pgfpathlineto{\pgfqpoint{2.117524in}{2.255888in}}%
\pgfpathlineto{\pgfqpoint{2.140763in}{2.245304in}}%
\pgfpathlineto{\pgfqpoint{2.155550in}{2.249915in}}%
\pgfpathlineto{\pgfqpoint{2.175501in}{2.270472in}}%
\pgfpathlineto{\pgfqpoint{2.193577in}{2.314660in}}%
\pgfpathlineto{\pgfqpoint{2.214936in}{2.407980in}}%
\pgfpathlineto{\pgfqpoint{2.232776in}{2.549192in}}%
\pgfpathlineto{\pgfqpoint{2.251086in}{2.568601in}}%
\pgfpathlineto{\pgfqpoint{2.275028in}{2.601921in}}%
\pgfpathlineto{\pgfqpoint{2.290050in}{2.513887in}}%
\pgfpathlineto{\pgfqpoint{2.307889in}{2.378302in}}%
\pgfpathlineto{\pgfqpoint{2.328311in}{2.291553in}}%
\pgfpathlineto{\pgfqpoint{2.345916in}{2.267073in}}%
\pgfpathlineto{\pgfqpoint{2.367511in}{2.247260in}}%
\pgfpathlineto{\pgfqpoint{2.384880in}{2.245855in}}%
\pgfpathlineto{\pgfqpoint{2.406944in}{2.263007in}}%
\pgfpathlineto{\pgfqpoint{2.424315in}{2.297876in}}%
\pgfpathlineto{\pgfqpoint{2.444033in}{2.375889in}}%
\pgfpathlineto{\pgfqpoint{2.462341in}{2.492766in}}%
\pgfpathlineto{\pgfqpoint{2.481355in}{2.602418in}}%
\pgfpathlineto{\pgfqpoint{2.502950in}{2.600944in}}%
\pgfpathlineto{\pgfqpoint{2.521493in}{2.493902in}}%
\pgfpathlineto{\pgfqpoint{2.542149in}{2.346614in}}%
\pgfpathlineto{\pgfqpoint{2.558580in}{2.290008in}}%
\pgfpathlineto{\pgfqpoint{2.578767in}{2.255698in}}%
\pgfpathlineto{\pgfqpoint{2.596372in}{2.263896in}}%
\pgfpathlineto{\pgfqpoint{2.635336in}{2.244323in}}%
\pgfpathlineto{\pgfqpoint{2.653412in}{2.253256in}}%
\pgfpathlineto{\pgfqpoint{2.674068in}{2.283589in}}%
\pgfpathlineto{\pgfqpoint{2.692376in}{2.340717in}}%
\pgfpathlineto{\pgfqpoint{2.712797in}{2.467329in}}%
\pgfpathlineto{\pgfqpoint{2.731576in}{2.574548in}}%
\pgfpathlineto{\pgfqpoint{2.753641in}{2.610131in}}%
\pgfpathlineto{\pgfqpoint{2.773592in}{2.498302in}}%
\pgfpathlineto{\pgfqpoint{2.788850in}{2.491324in}}%
\pgfpathlineto{\pgfqpoint{2.810444in}{2.339459in}}%
\pgfpathlineto{\pgfqpoint{2.829458in}{2.278299in}}%
\pgfpathlineto{\pgfqpoint{2.844951in}{2.256108in}}%
\pgfpathlineto{\pgfqpoint{2.866544in}{2.245382in}}%
\pgfpathlineto{\pgfqpoint{2.888375in}{2.256694in}}%
\pgfpathlineto{\pgfqpoint{2.903163in}{2.277255in}}%
\pgfpathlineto{\pgfqpoint{2.924524in}{2.321045in}}%
\pgfpathlineto{\pgfqpoint{2.942363in}{2.405136in}}%
\pgfpathlineto{\pgfqpoint{2.963488in}{2.549339in}}%
\pgfpathlineto{\pgfqpoint{2.981562in}{2.598635in}}%
\pgfpathlineto{\pgfqpoint{2.998698in}{2.609947in}}%
\pgfpathlineto{\pgfqpoint{3.019588in}{2.497787in}}%
\pgfpathlineto{\pgfqpoint{3.037427in}{2.375609in}}%
\pgfpathlineto{\pgfqpoint{3.056206in}{2.304526in}}%
\pgfpathlineto{\pgfqpoint{3.076862in}{2.269131in}}%
\pgfpathlineto{\pgfqpoint{3.094936in}{2.249715in}}%
\pgfpathlineto{\pgfqpoint{3.116297in}{2.469258in}}%
\pgfpathlineto{\pgfqpoint{3.135311in}{2.528157in}}%
\pgfpathlineto{\pgfqpoint{3.156201in}{2.385961in}}%
\pgfpathlineto{\pgfqpoint{3.176389in}{2.292972in}}%
\pgfpathlineto{\pgfqpoint{3.191411in}{2.269892in}}%
\pgfpathlineto{\pgfqpoint{3.214884in}{2.245715in}}%
\pgfpathlineto{\pgfqpoint{3.229906in}{2.248806in}}%
\pgfpathlineto{\pgfqpoint{3.251031in}{2.270082in}}%
\pgfpathlineto{\pgfqpoint{3.269341in}{2.311229in}}%
\pgfpathlineto{\pgfqpoint{3.293049in}{2.456423in}}%
\pgfpathlineto{\pgfqpoint{3.308071in}{2.563022in}}%
\pgfpathlineto{\pgfqpoint{3.327319in}{2.624542in}}%
\pgfpathlineto{\pgfqpoint{3.345863in}{2.596319in}}%
\pgfpathlineto{\pgfqpoint{3.366285in}{2.488652in}}%
\pgfpathlineto{\pgfqpoint{3.384124in}{2.357246in}}%
\pgfpathlineto{\pgfqpoint{3.405015in}{2.285464in}}%
\pgfpathlineto{\pgfqpoint{3.423559in}{2.259583in}}%
\pgfpathlineto{\pgfqpoint{3.441162in}{2.246931in}}%
\pgfpathlineto{\pgfqpoint{3.462758in}{2.261111in}}%
\pgfpathlineto{\pgfqpoint{3.480362in}{2.290483in}}%
\pgfpathlineto{\pgfqpoint{3.501019in}{2.356920in}}%
\pgfpathlineto{\pgfqpoint{3.517449in}{2.401118in}}%
\pgfpathlineto{\pgfqpoint{3.537168in}{2.529095in}}%
\pgfpathlineto{\pgfqpoint{3.559232in}{2.628723in}}%
\pgfpathlineto{\pgfqpoint{3.579652in}{2.597605in}}%
\pgfpathlineto{\pgfqpoint{3.598197in}{2.496175in}}%
\pgfpathlineto{\pgfqpoint{3.616036in}{2.375025in}}%
\pgfpathlineto{\pgfqpoint{3.633641in}{2.312425in}}%
\pgfpathlineto{\pgfqpoint{3.654766in}{2.271025in}}%
\pgfpathlineto{\pgfqpoint{3.675893in}{2.250531in}}%
\pgfpathlineto{\pgfqpoint{3.690446in}{2.248507in}}%
\pgfpathlineto{\pgfqpoint{3.712040in}{2.270188in}}%
\pgfpathlineto{\pgfqpoint{3.731287in}{2.297643in}}%
\pgfpathlineto{\pgfqpoint{3.751006in}{2.360360in}}%
\pgfpathlineto{\pgfqpoint{3.769785in}{2.459818in}}%
\pgfpathlineto{\pgfqpoint{3.789970in}{2.590007in}}%
\pgfpathlineto{\pgfqpoint{3.808280in}{2.641389in}}%
\pgfpathlineto{\pgfqpoint{3.827528in}{2.619436in}}%
\pgfpathlineto{\pgfqpoint{3.845601in}{2.509919in}}%
\pgfpathlineto{\pgfqpoint{3.865789in}{2.370595in}}%
\pgfpathlineto{\pgfqpoint{3.884802in}{2.316909in}}%
\pgfpathlineto{\pgfqpoint{3.903110in}{2.278095in}}%
\pgfpathlineto{\pgfqpoint{3.922358in}{2.260559in}}%
\pgfpathlineto{\pgfqpoint{3.940431in}{2.249722in}}%
\pgfpathlineto{\pgfqpoint{3.961324in}{2.254553in}}%
\pgfpathlineto{\pgfqpoint{3.980572in}{2.276045in}}%
\pgfpathlineto{\pgfqpoint{4.002400in}{2.327290in}}%
\pgfpathlineto{\pgfqpoint{4.020241in}{2.398539in}}%
\pgfpathlineto{\pgfqpoint{4.039018in}{2.507527in}}%
\pgfpathlineto{\pgfqpoint{4.057797in}{2.609434in}}%
\pgfpathlineto{\pgfqpoint{4.076576in}{2.648799in}}%
\pgfpathlineto{\pgfqpoint{4.095589in}{2.649379in}}%
\pgfpathlineto{\pgfqpoint{4.117888in}{2.582737in}}%
\pgfpathlineto{\pgfqpoint{4.135024in}{2.475367in}}%
\pgfpathlineto{\pgfqpoint{4.153332in}{2.381330in}}%
\pgfpathlineto{\pgfqpoint{4.172345in}{2.311274in}}%
\pgfpathlineto{\pgfqpoint{4.213189in}{2.257080in}}%
\pgfpathlineto{\pgfqpoint{4.232671in}{2.252455in}}%
\pgfpathlineto{\pgfqpoint{4.251213in}{2.260060in}}%
\pgfpathlineto{\pgfqpoint{4.269523in}{2.287029in}}%
\pgfpathlineto{\pgfqpoint{4.293231in}{2.344671in}}%
\pgfpathlineto{\pgfqpoint{4.308487in}{2.393888in}}%
\pgfpathlineto{\pgfqpoint{4.346748in}{2.608357in}}%
\pgfpathlineto{\pgfqpoint{4.365293in}{2.660720in}}%
\pgfpathlineto{\pgfqpoint{4.383835in}{2.650593in}}%
\pgfpathlineto{\pgfqpoint{4.403554in}{2.263225in}}%
\pgfpathlineto{\pgfqpoint{4.422096in}{2.295733in}}%
\pgfpathlineto{\pgfqpoint{4.440875in}{2.366121in}}%
\pgfpathlineto{\pgfqpoint{4.460357in}{2.492200in}}%
\pgfpathlineto{\pgfqpoint{4.479136in}{2.633483in}}%
\pgfpathlineto{\pgfqpoint{4.481013in}{2.638480in}}%
\pgfpathlineto{\pgfqpoint{4.475145in}{2.603682in}}%
\pgfpathlineto{\pgfqpoint{4.452143in}{2.391916in}}%
\pgfpathlineto{\pgfqpoint{4.435007in}{2.304330in}}%
\pgfpathlineto{\pgfqpoint{4.416699in}{2.261908in}}%
\pgfpathlineto{\pgfqpoint{4.398154in}{2.253373in}}%
\pgfpathlineto{\pgfqpoint{4.376795in}{2.286320in}}%
\pgfpathlineto{\pgfqpoint{4.359424in}{2.364307in}}%
\pgfpathlineto{\pgfqpoint{4.339942in}{2.532908in}}%
\pgfpathlineto{\pgfqpoint{4.317641in}{2.653920in}}%
\pgfpathlineto{\pgfqpoint{4.303090in}{2.626235in}}%
\pgfpathlineto{\pgfqpoint{4.278208in}{2.409531in}}%
\pgfpathlineto{\pgfqpoint{4.263889in}{2.322737in}}%
\pgfpathlineto{\pgfqpoint{4.244407in}{2.269488in}}%
\pgfpathlineto{\pgfqpoint{4.226097in}{2.249624in}}%
\pgfpathlineto{\pgfqpoint{4.205675in}{2.268666in}}%
\pgfpathlineto{\pgfqpoint{4.188071in}{2.319808in}}%
\pgfpathlineto{\pgfqpoint{4.166477in}{2.461349in}}%
\pgfpathlineto{\pgfqpoint{4.149812in}{2.607851in}}%
\pgfpathlineto{\pgfqpoint{4.127982in}{2.637343in}}%
\pgfpathlineto{\pgfqpoint{4.109437in}{2.505184in}}%
\pgfpathlineto{\pgfqpoint{4.088781in}{2.345664in}}%
\pgfpathlineto{\pgfqpoint{4.067890in}{2.278693in}}%
\pgfpathlineto{\pgfqpoint{4.050286in}{2.256154in}}%
\pgfpathlineto{\pgfqpoint{4.033150in}{2.249910in}}%
\pgfpathlineto{\pgfqpoint{4.014373in}{2.274826in}}%
\pgfpathlineto{\pgfqpoint{3.993951in}{2.333033in}}%
\pgfpathlineto{\pgfqpoint{3.975875in}{2.456652in}}%
\pgfpathlineto{\pgfqpoint{3.954750in}{2.615801in}}%
\pgfpathlineto{\pgfqpoint{3.933625in}{2.617307in}}%
\pgfpathlineto{\pgfqpoint{3.916255in}{2.479488in}}%
\pgfpathlineto{\pgfqpoint{3.897947in}{2.344098in}}%
\pgfpathlineto{\pgfqpoint{3.878934in}{2.281823in}}%
\pgfpathlineto{\pgfqpoint{3.854990in}{2.249229in}}%
\pgfpathlineto{\pgfqpoint{3.839030in}{2.249568in}}%
\pgfpathlineto{\pgfqpoint{3.819782in}{2.450461in}}%
\pgfpathlineto{\pgfqpoint{3.801472in}{2.328223in}}%
\pgfpathlineto{\pgfqpoint{3.782693in}{2.273181in}}%
\pgfpathlineto{\pgfqpoint{3.765559in}{2.250803in}}%
\pgfpathlineto{\pgfqpoint{3.743495in}{2.255078in}}%
\pgfpathlineto{\pgfqpoint{3.725656in}{2.285701in}}%
\pgfpathlineto{\pgfqpoint{3.705234in}{2.373668in}}%
\pgfpathlineto{\pgfqpoint{3.685986in}{2.530484in}}%
\pgfpathlineto{\pgfqpoint{3.666738in}{2.618384in}}%
\pgfpathlineto{\pgfqpoint{3.648428in}{2.600856in}}%
\pgfpathlineto{\pgfqpoint{3.629415in}{2.466920in}}%
\pgfpathlineto{\pgfqpoint{3.607351in}{2.328095in}}%
\pgfpathlineto{\pgfqpoint{3.591859in}{2.280276in}}%
\pgfpathlineto{\pgfqpoint{3.569560in}{2.250943in}}%
\pgfpathlineto{\pgfqpoint{3.551250in}{2.250335in}}%
\pgfpathlineto{\pgfqpoint{3.529420in}{2.271621in}}%
\pgfpathlineto{\pgfqpoint{3.514398in}{2.312332in}}%
\pgfpathlineto{\pgfqpoint{3.496324in}{2.396772in}}%
\pgfpathlineto{\pgfqpoint{3.474494in}{2.525920in}}%
\pgfpathlineto{\pgfqpoint{3.455481in}{2.618956in}}%
\pgfpathlineto{\pgfqpoint{3.436702in}{2.583964in}}%
\pgfpathlineto{\pgfqpoint{3.418160in}{2.499375in}}%
\pgfpathlineto{\pgfqpoint{3.398912in}{2.364299in}}%
\pgfpathlineto{\pgfqpoint{3.376847in}{2.285269in}}%
\pgfpathlineto{\pgfqpoint{3.359008in}{2.254625in}}%
\pgfpathlineto{\pgfqpoint{3.340464in}{2.245038in}}%
\pgfpathlineto{\pgfqpoint{3.321685in}{2.258016in}}%
\pgfpathlineto{\pgfqpoint{3.302672in}{2.290445in}}%
\pgfpathlineto{\pgfqpoint{3.278261in}{2.381693in}}%
\pgfpathlineto{\pgfqpoint{3.265350in}{2.477521in}}%
\pgfpathlineto{\pgfqpoint{3.243991in}{2.375478in}}%
\pgfpathlineto{\pgfqpoint{3.226620in}{2.495291in}}%
\pgfpathlineto{\pgfqpoint{3.203851in}{2.614547in}}%
\pgfpathlineto{\pgfqpoint{3.185074in}{2.581068in}}%
\pgfpathlineto{\pgfqpoint{3.169346in}{2.471990in}}%
\pgfpathlineto{\pgfqpoint{3.147985in}{2.338883in}}%
\pgfpathlineto{\pgfqpoint{3.127565in}{2.277680in}}%
\pgfpathlineto{\pgfqpoint{3.109255in}{2.252307in}}%
\pgfpathlineto{\pgfqpoint{3.091416in}{2.245155in}}%
\pgfpathlineto{\pgfqpoint{3.071699in}{2.260206in}}%
\pgfpathlineto{\pgfqpoint{3.051512in}{2.289823in}}%
\pgfpathlineto{\pgfqpoint{3.032968in}{2.363584in}}%
\pgfpathlineto{\pgfqpoint{3.014191in}{2.499479in}}%
\pgfpathlineto{\pgfqpoint{2.997055in}{2.606039in}}%
\pgfpathlineto{\pgfqpoint{2.978276in}{2.603310in}}%
\pgfpathlineto{\pgfqpoint{2.957151in}{2.509399in}}%
\pgfpathlineto{\pgfqpoint{2.936729in}{2.363673in}}%
\pgfpathlineto{\pgfqpoint{2.917482in}{2.301352in}}%
\pgfpathlineto{\pgfqpoint{2.901051in}{2.268287in}}%
\pgfpathlineto{\pgfqpoint{2.878752in}{2.246936in}}%
\pgfpathlineto{\pgfqpoint{2.857156in}{2.247673in}}%
\pgfpathlineto{\pgfqpoint{2.839082in}{2.264581in}}%
\pgfpathlineto{\pgfqpoint{2.820304in}{2.305611in}}%
\pgfpathlineto{\pgfqpoint{2.800821in}{2.390378in}}%
\pgfpathlineto{\pgfqpoint{2.786034in}{2.510489in}}%
\pgfpathlineto{\pgfqpoint{2.765377in}{2.605637in}}%
\pgfpathlineto{\pgfqpoint{2.745190in}{2.606110in}}%
\pgfpathlineto{\pgfqpoint{2.724065in}{2.499630in}}%
\pgfpathlineto{\pgfqpoint{2.702001in}{2.367242in}}%
\pgfpathlineto{\pgfqpoint{2.687213in}{2.318261in}}%
\pgfpathlineto{\pgfqpoint{2.668434in}{2.274957in}}%
\pgfpathlineto{\pgfqpoint{2.649421in}{2.258400in}}%
\pgfpathlineto{\pgfqpoint{2.628061in}{2.244598in}}%
\pgfpathlineto{\pgfqpoint{2.610925in}{2.250370in}}%
\pgfpathlineto{\pgfqpoint{2.590738in}{2.272478in}}%
\pgfpathlineto{\pgfqpoint{2.569378in}{2.303654in}}%
\pgfpathlineto{\pgfqpoint{2.550834in}{2.397499in}}%
\pgfpathlineto{\pgfqpoint{2.532526in}{2.523575in}}%
\pgfpathlineto{\pgfqpoint{2.516095in}{2.608995in}}%
\pgfpathlineto{\pgfqpoint{2.494968in}{2.576213in}}%
\pgfpathlineto{\pgfqpoint{2.455533in}{2.328894in}}%
\pgfpathlineto{\pgfqpoint{2.437694in}{2.284248in}}%
\pgfpathlineto{\pgfqpoint{2.421263in}{2.259406in}}%
\pgfpathlineto{\pgfqpoint{2.398964in}{2.244539in}}%
\pgfpathlineto{\pgfqpoint{2.377369in}{2.252685in}}%
\pgfpathlineto{\pgfqpoint{2.358826in}{2.276407in}}%
\pgfpathlineto{\pgfqpoint{2.343333in}{2.308191in}}%
\pgfpathlineto{\pgfqpoint{2.321739in}{2.416893in}}%
\pgfpathlineto{\pgfqpoint{2.303664in}{2.539147in}}%
\pgfpathlineto{\pgfqpoint{2.281365in}{2.612201in}}%
\pgfpathlineto{\pgfqpoint{2.263525in}{2.579574in}}%
\pgfpathlineto{\pgfqpoint{2.225970in}{2.372097in}}%
\pgfpathlineto{\pgfqpoint{2.207894in}{2.305290in}}%
\pgfpathlineto{\pgfqpoint{2.188412in}{2.269542in}}%
\pgfpathlineto{\pgfqpoint{2.167990in}{2.252141in}}%
\pgfpathlineto{\pgfqpoint{2.148743in}{2.245217in}}%
\pgfpathlineto{\pgfqpoint{2.126912in}{2.259828in}}%
\pgfpathlineto{\pgfqpoint{2.111421in}{2.279712in}}%
\pgfpathlineto{\pgfqpoint{2.089825in}{2.302262in}}%
\pgfpathlineto{\pgfqpoint{2.071283in}{2.395285in}}%
\pgfpathlineto{\pgfqpoint{2.053207in}{2.507343in}}%
\pgfpathlineto{\pgfqpoint{2.034665in}{2.559430in}}%
\pgfpathlineto{\pgfqpoint{2.015652in}{2.616667in}}%
\pgfpathlineto{\pgfqpoint{1.994290in}{2.563590in}}%
\pgfpathlineto{\pgfqpoint{1.958143in}{2.345102in}}%
\pgfpathlineto{\pgfqpoint{1.937956in}{2.290315in}}%
\pgfpathlineto{\pgfqpoint{1.920820in}{2.276806in}}%
\pgfpathlineto{\pgfqpoint{1.898992in}{2.251501in}}%
\pgfpathlineto{\pgfqpoint{1.880213in}{2.245850in}}%
\pgfpathlineto{\pgfqpoint{1.858617in}{2.259749in}}%
\pgfpathlineto{\pgfqpoint{1.840074in}{2.289600in}}%
\pgfpathlineto{\pgfqpoint{1.821530in}{2.349229in}}%
\pgfpathlineto{\pgfqpoint{1.783504in}{2.556293in}}%
\pgfpathlineto{\pgfqpoint{1.764961in}{2.617586in}}%
\pgfpathlineto{\pgfqpoint{1.744070in}{2.600204in}}%
\pgfpathlineto{\pgfqpoint{1.728109in}{2.512466in}}%
\pgfpathlineto{\pgfqpoint{1.710738in}{2.397599in}}%
\pgfpathlineto{\pgfqpoint{1.687734in}{2.336709in}}%
\pgfpathlineto{\pgfqpoint{1.670364in}{2.296862in}}%
\pgfpathlineto{\pgfqpoint{1.649004in}{2.265048in}}%
\pgfpathlineto{\pgfqpoint{1.629757in}{2.251982in}}%
\pgfpathlineto{\pgfqpoint{1.611447in}{2.247850in}}%
\pgfpathlineto{\pgfqpoint{1.592904in}{2.257999in}}%
\pgfpathlineto{\pgfqpoint{1.571543in}{2.289409in}}%
\pgfpathlineto{\pgfqpoint{1.554643in}{2.340595in}}%
\pgfpathlineto{\pgfqpoint{1.531170in}{2.328674in}}%
\pgfpathlineto{\pgfqpoint{1.512157in}{2.420470in}}%
\pgfpathlineto{\pgfqpoint{1.494552in}{2.540015in}}%
\pgfpathlineto{\pgfqpoint{1.479061in}{2.607942in}}%
\pgfpathlineto{\pgfqpoint{1.459577in}{2.627704in}}%
\pgfpathlineto{\pgfqpoint{1.438686in}{2.551058in}}%
\pgfpathlineto{\pgfqpoint{1.418265in}{2.419898in}}%
\pgfpathlineto{\pgfqpoint{1.395731in}{2.328226in}}%
\pgfpathlineto{\pgfqpoint{1.379066in}{2.302673in}}%
\pgfpathlineto{\pgfqpoint{1.360521in}{2.277704in}}%
\pgfpathlineto{\pgfqpoint{1.340334in}{2.252992in}}%
\pgfpathlineto{\pgfqpoint{1.322261in}{2.248218in}}%
\pgfpathlineto{\pgfqpoint{1.303482in}{2.257287in}}%
\pgfpathlineto{\pgfqpoint{1.284470in}{2.281824in}}%
\pgfpathlineto{\pgfqpoint{1.263343in}{2.325305in}}%
\pgfpathlineto{\pgfqpoint{1.245035in}{2.397061in}}%
\pgfpathlineto{\pgfqpoint{1.226960in}{2.495249in}}%
\pgfpathlineto{\pgfqpoint{1.208183in}{2.559380in}}%
\pgfpathlineto{\pgfqpoint{1.188464in}{2.631790in}}%
\pgfpathlineto{\pgfqpoint{1.168043in}{2.634113in}}%
\pgfpathlineto{\pgfqpoint{1.149500in}{2.562794in}}%
\pgfpathlineto{\pgfqpoint{1.130018in}{2.437072in}}%
\pgfpathlineto{\pgfqpoint{1.111474in}{2.349868in}}%
\pgfpathlineto{\pgfqpoint{1.091523in}{2.301002in}}%
\pgfpathlineto{\pgfqpoint{1.053496in}{2.255588in}}%
\pgfpathlineto{\pgfqpoint{1.035891in}{2.249998in}}%
\pgfpathlineto{\pgfqpoint{1.016173in}{2.265812in}}%
\pgfpathlineto{\pgfqpoint{0.995517in}{2.292040in}}%
\pgfpathlineto{\pgfqpoint{0.976740in}{2.372935in}}%
\pgfpathlineto{\pgfqpoint{0.957492in}{2.307059in}}%
\pgfpathlineto{\pgfqpoint{0.940356in}{2.274625in}}%
\pgfpathlineto{\pgfqpoint{0.917823in}{2.250966in}}%
\pgfpathlineto{\pgfqpoint{0.901156in}{2.254389in}}%
\pgfpathlineto{\pgfqpoint{0.880265in}{2.283626in}}%
\pgfpathlineto{\pgfqpoint{0.860783in}{2.340220in}}%
\pgfpathlineto{\pgfqpoint{0.844118in}{2.392509in}}%
\pgfpathlineto{\pgfqpoint{0.823696in}{2.516603in}}%
\pgfpathlineto{\pgfqpoint{0.803040in}{2.611793in}}%
\pgfpathlineto{\pgfqpoint{0.784027in}{2.655772in}}%
\pgfpathlineto{\pgfqpoint{0.763371in}{2.588322in}}%
\pgfpathlineto{\pgfqpoint{0.747174in}{2.472848in}}%
\pgfpathlineto{\pgfqpoint{0.727456in}{2.354903in}}%
\pgfpathlineto{\pgfqpoint{0.707036in}{2.294094in}}%
\pgfpathlineto{\pgfqpoint{0.688960in}{2.265747in}}%
\pgfpathlineto{\pgfqpoint{0.665958in}{2.252396in}}%
\pgfpathlineto{\pgfqpoint{0.650231in}{2.273404in}}%
\pgfpathlineto{\pgfqpoint{0.650934in}{2.274600in}}%
\pgfpathlineto{\pgfqpoint{0.656099in}{2.285592in}}%
\pgfpathlineto{\pgfqpoint{0.676521in}{2.365114in}}%
\pgfpathlineto{\pgfqpoint{0.712199in}{2.636779in}}%
\pgfpathlineto{\pgfqpoint{0.734498in}{2.619200in}}%
\pgfpathlineto{\pgfqpoint{0.752808in}{2.468424in}}%
\pgfpathlineto{\pgfqpoint{0.771351in}{2.362620in}}%
\pgfpathlineto{\pgfqpoint{0.790129in}{2.284151in}}%
\pgfpathlineto{\pgfqpoint{0.809612in}{2.253102in}}%
\pgfpathlineto{\pgfqpoint{0.828390in}{2.257652in}}%
\pgfpathlineto{\pgfqpoint{0.850455in}{2.305682in}}%
\pgfpathlineto{\pgfqpoint{0.868997in}{2.397505in}}%
\pgfpathlineto{\pgfqpoint{0.887776in}{2.563830in}}%
\pgfpathlineto{\pgfqpoint{0.906790in}{2.645948in}}%
\pgfpathlineto{\pgfqpoint{0.925803in}{2.574452in}}%
\pgfpathlineto{\pgfqpoint{0.946928in}{2.384104in}}%
\pgfpathlineto{\pgfqpoint{0.963595in}{2.304562in}}%
\pgfpathlineto{\pgfqpoint{0.983546in}{2.261812in}}%
\pgfpathlineto{\pgfqpoint{1.000916in}{2.249121in}}%
\pgfpathlineto{\pgfqpoint{1.026501in}{2.284377in}}%
\pgfpathlineto{\pgfqpoint{1.042934in}{2.329115in}}%
\pgfpathlineto{\pgfqpoint{1.061007in}{2.447994in}}%
\pgfpathlineto{\pgfqpoint{1.080255in}{2.605280in}}%
\pgfpathlineto{\pgfqpoint{1.098563in}{2.627734in}}%
\pgfpathlineto{\pgfqpoint{1.141049in}{2.333348in}}%
\pgfpathlineto{\pgfqpoint{1.154663in}{2.281824in}}%
\pgfpathlineto{\pgfqpoint{1.173442in}{2.251169in}}%
\pgfpathlineto{\pgfqpoint{1.195507in}{2.252066in}}%
\pgfpathlineto{\pgfqpoint{1.213112in}{2.279226in}}%
\pgfpathlineto{\pgfqpoint{1.233533in}{2.334942in}}%
\pgfpathlineto{\pgfqpoint{1.253955in}{2.445659in}}%
\pgfpathlineto{\pgfqpoint{1.272497in}{2.577878in}}%
\pgfpathlineto{\pgfqpoint{1.288694in}{2.628508in}}%
\pgfpathlineto{\pgfqpoint{1.310055in}{2.524726in}}%
\pgfpathlineto{\pgfqpoint{1.328129in}{2.378497in}}%
\pgfpathlineto{\pgfqpoint{1.349725in}{2.288141in}}%
\pgfpathlineto{\pgfqpoint{1.367095in}{2.256179in}}%
\pgfpathlineto{\pgfqpoint{1.387751in}{2.246686in}}%
\pgfpathlineto{\pgfqpoint{1.406528in}{2.263174in}}%
\pgfpathlineto{\pgfqpoint{1.423664in}{2.298181in}}%
\pgfpathlineto{\pgfqpoint{1.444789in}{2.393605in}}%
\pgfpathlineto{\pgfqpoint{1.462628in}{2.508648in}}%
\pgfpathlineto{\pgfqpoint{1.487041in}{2.622717in}}%
\pgfpathlineto{\pgfqpoint{1.501829in}{2.610138in}}%
\pgfpathlineto{\pgfqpoint{1.522016in}{2.598374in}}%
\pgfpathlineto{\pgfqpoint{1.539855in}{2.466173in}}%
\pgfpathlineto{\pgfqpoint{1.559806in}{2.330138in}}%
\pgfpathlineto{\pgfqpoint{1.579525in}{2.277671in}}%
\pgfpathlineto{\pgfqpoint{1.599946in}{2.250118in}}%
\pgfpathlineto{\pgfqpoint{1.617786in}{2.246763in}}%
\pgfpathlineto{\pgfqpoint{1.634451in}{2.260185in}}%
\pgfpathlineto{\pgfqpoint{1.656515in}{2.282453in}}%
\pgfpathlineto{\pgfqpoint{1.674589in}{2.323653in}}%
\pgfpathlineto{\pgfqpoint{1.694307in}{2.431019in}}%
\pgfpathlineto{\pgfqpoint{1.714258in}{2.426199in}}%
\pgfpathlineto{\pgfqpoint{1.729986in}{2.560231in}}%
\pgfpathlineto{\pgfqpoint{1.750642in}{2.617418in}}%
\pgfpathlineto{\pgfqpoint{1.771064in}{2.534482in}}%
\pgfpathlineto{\pgfqpoint{1.789606in}{2.388435in}}%
\pgfpathlineto{\pgfqpoint{1.809559in}{2.300899in}}%
\pgfpathlineto{\pgfqpoint{1.829276in}{2.295402in}}%
\pgfpathlineto{\pgfqpoint{1.846412in}{2.565136in}}%
\pgfpathlineto{\pgfqpoint{1.867537in}{2.612057in}}%
\pgfpathlineto{\pgfqpoint{1.888898in}{2.526224in}}%
\pgfpathlineto{\pgfqpoint{1.888193in}{2.431288in}}%
\pgfpathlineto{\pgfqpoint{1.907206in}{2.376537in}}%
\pgfpathlineto{\pgfqpoint{1.924576in}{2.292945in}}%
\pgfpathlineto{\pgfqpoint{1.943121in}{2.257816in}}%
\pgfpathlineto{\pgfqpoint{1.963541in}{2.244798in}}%
\pgfpathlineto{\pgfqpoint{1.983025in}{2.256169in}}%
\pgfpathlineto{\pgfqpoint{2.002741in}{2.289819in}}%
\pgfpathlineto{\pgfqpoint{2.020111in}{2.356771in}}%
\pgfpathlineto{\pgfqpoint{2.041471in}{2.516717in}}%
\pgfpathlineto{\pgfqpoint{2.059310in}{2.606300in}}%
\pgfpathlineto{\pgfqpoint{2.077386in}{2.594247in}}%
\pgfpathlineto{\pgfqpoint{2.098276in}{2.464537in}}%
\pgfpathlineto{\pgfqpoint{2.116115in}{2.338774in}}%
\pgfpathlineto{\pgfqpoint{2.139589in}{2.269172in}}%
\pgfpathlineto{\pgfqpoint{2.158602in}{2.248035in}}%
\pgfpathlineto{\pgfqpoint{2.175972in}{2.245893in}}%
\pgfpathlineto{\pgfqpoint{2.193577in}{2.255336in}}%
\pgfpathlineto{\pgfqpoint{2.211651in}{2.282222in}}%
\pgfpathlineto{\pgfqpoint{2.233010in}{2.354832in}}%
\pgfpathlineto{\pgfqpoint{2.250849in}{2.450264in}}%
\pgfpathlineto{\pgfqpoint{2.272679in}{2.592940in}}%
\pgfpathlineto{\pgfqpoint{2.289815in}{2.612871in}}%
\pgfpathlineto{\pgfqpoint{2.308360in}{2.549917in}}%
\pgfpathlineto{\pgfqpoint{2.329485in}{2.380169in}}%
\pgfpathlineto{\pgfqpoint{2.348027in}{2.300790in}}%
\pgfpathlineto{\pgfqpoint{2.365398in}{2.264565in}}%
\pgfpathlineto{\pgfqpoint{2.387228in}{2.245301in}}%
\pgfpathlineto{\pgfqpoint{2.404129in}{2.248384in}}%
\pgfpathlineto{\pgfqpoint{2.424785in}{2.272221in}}%
\pgfpathlineto{\pgfqpoint{2.443328in}{2.317840in}}%
\pgfpathlineto{\pgfqpoint{2.481823in}{2.502674in}}%
\pgfpathlineto{\pgfqpoint{2.499663in}{2.571740in}}%
\pgfpathlineto{\pgfqpoint{2.520789in}{2.610351in}}%
\pgfpathlineto{\pgfqpoint{2.539566in}{2.527140in}}%
\pgfpathlineto{\pgfqpoint{2.557642in}{2.398480in}}%
\pgfpathlineto{\pgfqpoint{2.578064in}{2.304721in}}%
\pgfpathlineto{\pgfqpoint{2.596137in}{2.264653in}}%
\pgfpathlineto{\pgfqpoint{2.617733in}{2.246854in}}%
\pgfpathlineto{\pgfqpoint{2.635572in}{2.247324in}}%
\pgfpathlineto{\pgfqpoint{2.653412in}{2.264430in}}%
\pgfpathlineto{\pgfqpoint{2.674068in}{2.302157in}}%
\pgfpathlineto{\pgfqpoint{2.691907in}{2.382110in}}%
\pgfpathlineto{\pgfqpoint{2.713971in}{2.521632in}}%
\pgfpathlineto{\pgfqpoint{2.731342in}{2.605908in}}%
\pgfpathlineto{\pgfqpoint{2.752467in}{2.593907in}}%
\pgfpathlineto{\pgfqpoint{2.788616in}{2.358709in}}%
\pgfpathlineto{\pgfqpoint{2.807864in}{2.291886in}}%
\pgfpathlineto{\pgfqpoint{2.828989in}{2.258459in}}%
\pgfpathlineto{\pgfqpoint{2.846359in}{2.245729in}}%
\pgfpathlineto{\pgfqpoint{2.866781in}{2.248894in}}%
\pgfpathlineto{\pgfqpoint{2.886966in}{2.262403in}}%
\pgfpathlineto{\pgfqpoint{2.904337in}{2.290418in}}%
\pgfpathlineto{\pgfqpoint{2.924993in}{2.367759in}}%
\pgfpathlineto{\pgfqpoint{2.943066in}{2.474419in}}%
\pgfpathlineto{\pgfqpoint{2.963488in}{2.602298in}}%
\pgfpathlineto{\pgfqpoint{2.979919in}{2.616377in}}%
\pgfpathlineto{\pgfqpoint{3.001046in}{2.587354in}}%
\pgfpathlineto{\pgfqpoint{3.022640in}{2.450186in}}%
\pgfpathlineto{\pgfqpoint{3.039072in}{2.500508in}}%
\pgfpathlineto{\pgfqpoint{3.058084in}{2.374557in}}%
\pgfpathlineto{\pgfqpoint{3.078271in}{2.294679in}}%
\pgfpathlineto{\pgfqpoint{3.097284in}{2.263612in}}%
\pgfpathlineto{\pgfqpoint{3.113715in}{2.248170in}}%
\pgfpathlineto{\pgfqpoint{3.134842in}{2.251395in}}%
\pgfpathlineto{\pgfqpoint{3.156201in}{2.265098in}}%
\pgfpathlineto{\pgfqpoint{3.174509in}{2.300337in}}%
\pgfpathlineto{\pgfqpoint{3.191880in}{2.352844in}}%
\pgfpathlineto{\pgfqpoint{3.212301in}{2.453185in}}%
\pgfpathlineto{\pgfqpoint{3.232254in}{2.561086in}}%
\pgfpathlineto{\pgfqpoint{3.250562in}{2.622879in}}%
\pgfpathlineto{\pgfqpoint{3.267698in}{2.587199in}}%
\pgfpathlineto{\pgfqpoint{3.287884in}{2.476481in}}%
\pgfpathlineto{\pgfqpoint{3.309714in}{2.402507in}}%
\pgfpathlineto{\pgfqpoint{3.324736in}{2.322070in}}%
\pgfpathlineto{\pgfqpoint{3.348209in}{2.268380in}}%
\pgfpathlineto{\pgfqpoint{3.364876in}{2.253450in}}%
\pgfpathlineto{\pgfqpoint{3.384827in}{2.248081in}}%
\pgfpathlineto{\pgfqpoint{3.403606in}{2.262090in}}%
\pgfpathlineto{\pgfqpoint{3.423559in}{2.296716in}}%
\pgfpathlineto{\pgfqpoint{3.443041in}{2.342052in}}%
\pgfpathlineto{\pgfqpoint{3.463463in}{2.404025in}}%
\pgfpathlineto{\pgfqpoint{3.481302in}{2.526786in}}%
\pgfpathlineto{\pgfqpoint{3.499141in}{2.619367in}}%
\pgfpathlineto{\pgfqpoint{3.520501in}{2.621277in}}%
\pgfpathlineto{\pgfqpoint{3.537871in}{2.542989in}}%
\pgfpathlineto{\pgfqpoint{3.577306in}{2.344371in}}%
\pgfpathlineto{\pgfqpoint{3.595145in}{2.296552in}}%
\pgfpathlineto{\pgfqpoint{3.616505in}{2.264135in}}%
\pgfpathlineto{\pgfqpoint{3.634346in}{2.250632in}}%
\pgfpathlineto{\pgfqpoint{3.654062in}{2.249544in}}%
\pgfpathlineto{\pgfqpoint{3.673779in}{2.268170in}}%
\pgfpathlineto{\pgfqpoint{3.692558in}{2.289775in}}%
\pgfpathlineto{\pgfqpoint{3.713919in}{2.332635in}}%
\pgfpathlineto{\pgfqpoint{3.728471in}{2.396915in}}%
\pgfpathlineto{\pgfqpoint{3.749127in}{2.517791in}}%
\pgfpathlineto{\pgfqpoint{3.770019in}{2.263645in}}%
\pgfpathlineto{\pgfqpoint{3.791613in}{2.299243in}}%
\pgfpathlineto{\pgfqpoint{3.804524in}{2.363273in}}%
\pgfpathlineto{\pgfqpoint{3.827057in}{2.464755in}}%
\pgfpathlineto{\pgfqpoint{3.846776in}{2.593960in}}%
\pgfpathlineto{\pgfqpoint{3.866492in}{2.642535in}}%
\pgfpathlineto{\pgfqpoint{3.885505in}{2.608899in}}%
\pgfpathlineto{\pgfqpoint{3.902407in}{2.518501in}}%
\pgfpathlineto{\pgfqpoint{3.924235in}{2.379734in}}%
\pgfpathlineto{\pgfqpoint{3.942074in}{2.313832in}}%
\pgfpathlineto{\pgfqpoint{3.960150in}{2.271193in}}%
\pgfpathlineto{\pgfqpoint{3.979632in}{2.252857in}}%
\pgfpathlineto{\pgfqpoint{3.997471in}{2.250438in}}%
\pgfpathlineto{\pgfqpoint{4.018362in}{2.269684in}}%
\pgfpathlineto{\pgfqpoint{4.037375in}{2.307784in}}%
\pgfpathlineto{\pgfqpoint{4.057797in}{2.367131in}}%
\pgfpathlineto{\pgfqpoint{4.079392in}{2.486704in}}%
\pgfpathlineto{\pgfqpoint{4.095118in}{2.594718in}}%
\pgfpathlineto{\pgfqpoint{4.113194in}{2.653270in}}%
\pgfpathlineto{\pgfqpoint{4.133379in}{2.618695in}}%
\pgfpathlineto{\pgfqpoint{4.152158in}{2.524799in}}%
\pgfpathlineto{\pgfqpoint{4.174691in}{2.384458in}}%
\pgfpathlineto{\pgfqpoint{4.192062in}{2.328264in}}%
\pgfpathlineto{\pgfqpoint{4.210606in}{2.282192in}}%
\pgfpathlineto{\pgfqpoint{4.229385in}{2.259638in}}%
\pgfpathlineto{\pgfqpoint{4.248867in}{2.252411in}}%
\pgfpathlineto{\pgfqpoint{4.267644in}{2.260796in}}%
\pgfpathlineto{\pgfqpoint{4.287831in}{2.285003in}}%
\pgfpathlineto{\pgfqpoint{4.308487in}{2.326437in}}%
\pgfpathlineto{\pgfqpoint{4.331021in}{2.434686in}}%
\pgfpathlineto{\pgfqpoint{4.344871in}{2.519720in}}%
\pgfpathlineto{\pgfqpoint{4.362710in}{2.616844in}}%
\pgfpathlineto{\pgfqpoint{4.385244in}{2.665577in}}%
\pgfpathlineto{\pgfqpoint{4.405900in}{2.631565in}}%
\pgfpathlineto{\pgfqpoint{4.423505in}{2.464655in}}%
\pgfpathlineto{\pgfqpoint{4.442752in}{2.593862in}}%
\pgfpathlineto{\pgfqpoint{4.461766in}{2.665758in}}%
\pgfpathlineto{\pgfqpoint{4.481484in}{2.646140in}}%
\pgfpathlineto{\pgfqpoint{4.481953in}{2.643105in}}%
\pgfpathlineto{\pgfqpoint{4.473737in}{2.666685in}}%
\pgfpathlineto{\pgfqpoint{4.455897in}{2.629842in}}%
\pgfpathlineto{\pgfqpoint{4.435241in}{2.442481in}}%
\pgfpathlineto{\pgfqpoint{4.417871in}{2.331032in}}%
\pgfpathlineto{\pgfqpoint{4.398389in}{2.273442in}}%
\pgfpathlineto{\pgfqpoint{4.377264in}{2.251359in}}%
\pgfpathlineto{\pgfqpoint{4.358954in}{2.272395in}}%
\pgfpathlineto{\pgfqpoint{4.338768in}{2.336059in}}%
\pgfpathlineto{\pgfqpoint{4.300976in}{2.631840in}}%
\pgfpathlineto{\pgfqpoint{4.281963in}{2.646533in}}%
\pgfpathlineto{\pgfqpoint{4.264358in}{2.524107in}}%
\pgfpathlineto{\pgfqpoint{4.243936in}{2.355132in}}%
\pgfpathlineto{\pgfqpoint{4.223046in}{2.280348in}}%
\pgfpathlineto{\pgfqpoint{4.205207in}{2.252274in}}%
\pgfpathlineto{\pgfqpoint{4.186664in}{2.257490in}}%
\pgfpathlineto{\pgfqpoint{4.166242in}{2.297924in}}%
\pgfpathlineto{\pgfqpoint{4.148167in}{2.384136in}}%
\pgfpathlineto{\pgfqpoint{4.127982in}{2.569520in}}%
\pgfpathlineto{\pgfqpoint{4.109437in}{2.645936in}}%
\pgfpathlineto{\pgfqpoint{4.089250in}{2.568366in}}%
\pgfpathlineto{\pgfqpoint{4.070942in}{2.412771in}}%
\pgfpathlineto{\pgfqpoint{4.052868in}{2.310370in}}%
\pgfpathlineto{\pgfqpoint{4.031741in}{2.259091in}}%
\pgfpathlineto{\pgfqpoint{4.012493in}{2.248558in}}%
\pgfpathlineto{\pgfqpoint{3.993480in}{2.270190in}}%
\pgfpathlineto{\pgfqpoint{3.971181in}{2.336981in}}%
\pgfpathlineto{\pgfqpoint{3.956159in}{2.437249in}}%
\pgfpathlineto{\pgfqpoint{3.935737in}{2.609615in}}%
\pgfpathlineto{\pgfqpoint{3.919775in}{2.631719in}}%
\pgfpathlineto{\pgfqpoint{3.897476in}{2.512537in}}%
\pgfpathlineto{\pgfqpoint{3.878697in}{2.367537in}}%
\pgfpathlineto{\pgfqpoint{3.857104in}{2.282563in}}%
\pgfpathlineto{\pgfqpoint{3.841376in}{2.257895in}}%
\pgfpathlineto{\pgfqpoint{3.819311in}{2.246789in}}%
\pgfpathlineto{\pgfqpoint{3.801238in}{2.258265in}}%
\pgfpathlineto{\pgfqpoint{3.783399in}{2.288481in}}%
\pgfpathlineto{\pgfqpoint{3.764151in}{2.375954in}}%
\pgfpathlineto{\pgfqpoint{3.744432in}{2.528627in}}%
\pgfpathlineto{\pgfqpoint{3.726828in}{2.625236in}}%
\pgfpathlineto{\pgfqpoint{3.706877in}{2.593132in}}%
\pgfpathlineto{\pgfqpoint{3.688332in}{2.476116in}}%
\pgfpathlineto{\pgfqpoint{3.666738in}{2.374210in}}%
\pgfpathlineto{\pgfqpoint{3.647960in}{2.361260in}}%
\pgfpathlineto{\pgfqpoint{3.628712in}{2.287271in}}%
\pgfpathlineto{\pgfqpoint{3.610873in}{2.256353in}}%
\pgfpathlineto{\pgfqpoint{3.588808in}{2.246095in}}%
\pgfpathlineto{\pgfqpoint{3.570498in}{2.263730in}}%
\pgfpathlineto{\pgfqpoint{3.554536in}{2.304574in}}%
\pgfpathlineto{\pgfqpoint{3.533646in}{2.409601in}}%
\pgfpathlineto{\pgfqpoint{3.514632in}{2.450416in}}%
\pgfpathlineto{\pgfqpoint{3.495855in}{2.584883in}}%
\pgfpathlineto{\pgfqpoint{3.477311in}{2.277476in}}%
\pgfpathlineto{\pgfqpoint{3.454778in}{2.349598in}}%
\pgfpathlineto{\pgfqpoint{3.436938in}{2.486301in}}%
\pgfpathlineto{\pgfqpoint{3.418160in}{2.610041in}}%
\pgfpathlineto{\pgfqpoint{3.397972in}{2.589207in}}%
\pgfpathlineto{\pgfqpoint{3.378256in}{2.482500in}}%
\pgfpathlineto{\pgfqpoint{3.359477in}{2.351632in}}%
\pgfpathlineto{\pgfqpoint{3.341169in}{2.287793in}}%
\pgfpathlineto{\pgfqpoint{3.321921in}{2.255335in}}%
\pgfpathlineto{\pgfqpoint{3.300091in}{2.245291in}}%
\pgfpathlineto{\pgfqpoint{3.281078in}{2.255737in}}%
\pgfpathlineto{\pgfqpoint{3.262768in}{2.284851in}}%
\pgfpathlineto{\pgfqpoint{3.244225in}{2.353997in}}%
\pgfpathlineto{\pgfqpoint{3.227794in}{2.479425in}}%
\pgfpathlineto{\pgfqpoint{3.205259in}{2.606780in}}%
\pgfpathlineto{\pgfqpoint{3.189534in}{2.609255in}}%
\pgfpathlineto{\pgfqpoint{3.168172in}{2.491448in}}%
\pgfpathlineto{\pgfqpoint{3.148690in}{2.357472in}}%
\pgfpathlineto{\pgfqpoint{3.126626in}{2.281895in}}%
\pgfpathlineto{\pgfqpoint{3.112072in}{2.259310in}}%
\pgfpathlineto{\pgfqpoint{3.089068in}{2.244348in}}%
\pgfpathlineto{\pgfqpoint{3.073811in}{2.251923in}}%
\pgfpathlineto{\pgfqpoint{3.053155in}{2.280620in}}%
\pgfpathlineto{\pgfqpoint{3.034612in}{2.322525in}}%
\pgfpathlineto{\pgfqpoint{3.013017in}{2.425748in}}%
\pgfpathlineto{\pgfqpoint{2.993298in}{2.577844in}}%
\pgfpathlineto{\pgfqpoint{2.974756in}{2.615894in}}%
\pgfpathlineto{\pgfqpoint{2.954568in}{2.569409in}}%
\pgfpathlineto{\pgfqpoint{2.938372in}{2.469043in}}%
\pgfpathlineto{\pgfqpoint{2.918890in}{2.338123in}}%
\pgfpathlineto{\pgfqpoint{2.900111in}{2.345603in}}%
\pgfpathlineto{\pgfqpoint{2.879455in}{2.274790in}}%
\pgfpathlineto{\pgfqpoint{2.860913in}{2.251579in}}%
\pgfpathlineto{\pgfqpoint{2.839082in}{2.245886in}}%
\pgfpathlineto{\pgfqpoint{2.819364in}{2.263659in}}%
\pgfpathlineto{\pgfqpoint{2.805047in}{2.295292in}}%
\pgfpathlineto{\pgfqpoint{2.804576in}{2.339983in}}%
\pgfpathlineto{\pgfqpoint{2.783451in}{2.363404in}}%
\pgfpathlineto{\pgfqpoint{2.764438in}{2.492345in}}%
\pgfpathlineto{\pgfqpoint{2.746599in}{2.595923in}}%
\pgfpathlineto{\pgfqpoint{2.724534in}{2.597632in}}%
\pgfpathlineto{\pgfqpoint{2.705990in}{2.495170in}}%
\pgfpathlineto{\pgfqpoint{2.684161in}{2.345907in}}%
\pgfpathlineto{\pgfqpoint{2.668668in}{2.295759in}}%
\pgfpathlineto{\pgfqpoint{2.648717in}{2.256888in}}%
\pgfpathlineto{\pgfqpoint{2.627590in}{2.244538in}}%
\pgfpathlineto{\pgfqpoint{2.612099in}{2.249515in}}%
\pgfpathlineto{\pgfqpoint{2.593555in}{2.274222in}}%
\pgfpathlineto{\pgfqpoint{2.566796in}{2.347759in}}%
\pgfpathlineto{\pgfqpoint{2.549894in}{2.448879in}}%
\pgfpathlineto{\pgfqpoint{2.532055in}{2.570617in}}%
\pgfpathlineto{\pgfqpoint{2.513513in}{2.611519in}}%
\pgfpathlineto{\pgfqpoint{2.494734in}{2.550383in}}%
\pgfpathlineto{\pgfqpoint{2.476426in}{2.457061in}}%
\pgfpathlineto{\pgfqpoint{2.455533in}{2.336961in}}%
\pgfpathlineto{\pgfqpoint{2.438634in}{2.287041in}}%
\pgfpathlineto{\pgfqpoint{2.417272in}{2.259927in}}%
\pgfpathlineto{\pgfqpoint{2.399199in}{2.250351in}}%
\pgfpathlineto{\pgfqpoint{2.380656in}{2.244817in}}%
\pgfpathlineto{\pgfqpoint{2.359529in}{2.258847in}}%
\pgfpathlineto{\pgfqpoint{2.340282in}{2.289610in}}%
\pgfpathlineto{\pgfqpoint{2.321974in}{2.348395in}}%
\pgfpathlineto{\pgfqpoint{2.303429in}{2.468662in}}%
\pgfpathlineto{\pgfqpoint{2.282304in}{2.581826in}}%
\pgfpathlineto{\pgfqpoint{2.265873in}{2.614100in}}%
\pgfpathlineto{\pgfqpoint{2.246155in}{2.539176in}}%
\pgfpathlineto{\pgfqpoint{2.225970in}{2.425604in}}%
\pgfpathlineto{\pgfqpoint{2.205077in}{2.349464in}}%
\pgfpathlineto{\pgfqpoint{2.186300in}{2.290474in}}%
\pgfpathlineto{\pgfqpoint{2.167990in}{2.274060in}}%
\pgfpathlineto{\pgfqpoint{2.148274in}{2.258632in}}%
\pgfpathlineto{\pgfqpoint{2.130200in}{2.251261in}}%
\pgfpathlineto{\pgfqpoint{2.112359in}{2.246352in}}%
\pgfpathlineto{\pgfqpoint{2.090060in}{2.260576in}}%
\pgfpathlineto{\pgfqpoint{2.071283in}{2.295772in}}%
\pgfpathlineto{\pgfqpoint{2.053913in}{2.375070in}}%
\pgfpathlineto{\pgfqpoint{2.036542in}{2.528986in}}%
\pgfpathlineto{\pgfqpoint{2.016589in}{2.609055in}}%
\pgfpathlineto{\pgfqpoint{1.994056in}{2.607245in}}%
\pgfpathlineto{\pgfqpoint{1.975982in}{2.517219in}}%
\pgfpathlineto{\pgfqpoint{1.957438in}{2.409554in}}%
\pgfpathlineto{\pgfqpoint{1.935373in}{2.312886in}}%
\pgfpathlineto{\pgfqpoint{1.917768in}{2.275345in}}%
\pgfpathlineto{\pgfqpoint{1.900164in}{2.257926in}}%
\pgfpathlineto{\pgfqpoint{1.876456in}{2.247687in}}%
\pgfpathlineto{\pgfqpoint{1.859088in}{2.255441in}}%
\pgfpathlineto{\pgfqpoint{1.840778in}{2.280128in}}%
\pgfpathlineto{\pgfqpoint{1.822235in}{2.321988in}}%
\pgfpathlineto{\pgfqpoint{1.803456in}{2.378434in}}%
\pgfpathlineto{\pgfqpoint{1.782330in}{2.512453in}}%
\pgfpathlineto{\pgfqpoint{1.763787in}{2.469421in}}%
\pgfpathlineto{\pgfqpoint{1.747122in}{2.587435in}}%
\pgfpathlineto{\pgfqpoint{1.726700in}{2.626316in}}%
\pgfpathlineto{\pgfqpoint{1.707687in}{2.579844in}}%
\pgfpathlineto{\pgfqpoint{1.686560in}{2.492008in}}%
\pgfpathlineto{\pgfqpoint{1.670834in}{2.410187in}}%
\pgfpathlineto{\pgfqpoint{1.648535in}{2.612440in}}%
\pgfpathlineto{\pgfqpoint{1.630460in}{2.615505in}}%
\pgfpathlineto{\pgfqpoint{1.612152in}{2.527776in}}%
\pgfpathlineto{\pgfqpoint{1.612152in}{2.527776in}}%
\pgfusepath{stroke}%
\end{pgfscope}%
\begin{pgfscope}%
\pgfpathrectangle{\pgfqpoint{0.444748in}{2.222124in}}{\pgfqpoint{4.231419in}{0.467251in}}%
\pgfusepath{clip}%
\pgfsetbuttcap%
\pgfsetroundjoin%
\definecolor{currentfill}{rgb}{0.047059,0.364706,0.647059}%
\pgfsetfillcolor{currentfill}%
\pgfsetlinewidth{1.003750pt}%
\definecolor{currentstroke}{rgb}{0.047059,0.364706,0.647059}%
\pgfsetstrokecolor{currentstroke}%
\pgfsetdash{}{0pt}%
\pgfsys@defobject{currentmarker}{\pgfqpoint{-0.010417in}{-0.010417in}}{\pgfqpoint{0.010417in}{0.010417in}}{%
\pgfpathmoveto{\pgfqpoint{0.000000in}{-0.010417in}}%
\pgfpathcurveto{\pgfqpoint{0.002763in}{-0.010417in}}{\pgfqpoint{0.005412in}{-0.009319in}}{\pgfqpoint{0.007366in}{-0.007366in}}%
\pgfpathcurveto{\pgfqpoint{0.009319in}{-0.005412in}}{\pgfqpoint{0.010417in}{-0.002763in}}{\pgfqpoint{0.010417in}{0.000000in}}%
\pgfpathcurveto{\pgfqpoint{0.010417in}{0.002763in}}{\pgfqpoint{0.009319in}{0.005412in}}{\pgfqpoint{0.007366in}{0.007366in}}%
\pgfpathcurveto{\pgfqpoint{0.005412in}{0.009319in}}{\pgfqpoint{0.002763in}{0.010417in}}{\pgfqpoint{0.000000in}{0.010417in}}%
\pgfpathcurveto{\pgfqpoint{-0.002763in}{0.010417in}}{\pgfqpoint{-0.005412in}{0.009319in}}{\pgfqpoint{-0.007366in}{0.007366in}}%
\pgfpathcurveto{\pgfqpoint{-0.009319in}{0.005412in}}{\pgfqpoint{-0.010417in}{0.002763in}}{\pgfqpoint{-0.010417in}{0.000000in}}%
\pgfpathcurveto{\pgfqpoint{-0.010417in}{-0.002763in}}{\pgfqpoint{-0.009319in}{-0.005412in}}{\pgfqpoint{-0.007366in}{-0.007366in}}%
\pgfpathcurveto{\pgfqpoint{-0.005412in}{-0.009319in}}{\pgfqpoint{-0.002763in}{-0.010417in}}{\pgfqpoint{0.000000in}{-0.010417in}}%
\pgfpathlineto{\pgfqpoint{0.000000in}{-0.010417in}}%
\pgfpathclose%
\pgfusepath{stroke,fill}%
}%
\begin{pgfscope}%
\pgfsys@transformshift{0.646240in}{2.434724in}%
\pgfsys@useobject{currentmarker}{}%
\end{pgfscope}%
\begin{pgfscope}%
\pgfsys@transformshift{0.656568in}{2.358099in}%
\pgfsys@useobject{currentmarker}{}%
\end{pgfscope}%
\begin{pgfscope}%
\pgfsys@transformshift{0.674878in}{2.287536in}%
\pgfsys@useobject{currentmarker}{}%
\end{pgfscope}%
\begin{pgfscope}%
\pgfsys@transformshift{0.695063in}{2.253612in}%
\pgfsys@useobject{currentmarker}{}%
\end{pgfscope}%
\begin{pgfscope}%
\pgfsys@transformshift{0.714782in}{2.262780in}%
\pgfsys@useobject{currentmarker}{}%
\end{pgfscope}%
\begin{pgfscope}%
\pgfsys@transformshift{0.734733in}{2.302961in}%
\pgfsys@useobject{currentmarker}{}%
\end{pgfscope}%
\begin{pgfscope}%
\pgfsys@transformshift{0.755389in}{2.398987in}%
\pgfsys@useobject{currentmarker}{}%
\end{pgfscope}%
\begin{pgfscope}%
\pgfsys@transformshift{0.771821in}{2.556333in}%
\pgfsys@useobject{currentmarker}{}%
\end{pgfscope}%
\begin{pgfscope}%
\pgfsys@transformshift{0.789661in}{2.267074in}%
\pgfsys@useobject{currentmarker}{}%
\end{pgfscope}%
\begin{pgfscope}%
\pgfsys@transformshift{0.813603in}{2.251985in}%
\pgfsys@useobject{currentmarker}{}%
\end{pgfscope}%
\begin{pgfscope}%
\pgfsys@transformshift{0.829330in}{2.274037in}%
\pgfsys@useobject{currentmarker}{}%
\end{pgfscope}%
\begin{pgfscope}%
\pgfsys@transformshift{0.849281in}{2.333928in}%
\pgfsys@useobject{currentmarker}{}%
\end{pgfscope}%
\begin{pgfscope}%
\pgfsys@transformshift{0.867591in}{2.469574in}%
\pgfsys@useobject{currentmarker}{}%
\end{pgfscope}%
\begin{pgfscope}%
\pgfsys@transformshift{0.886839in}{2.618798in}%
\pgfsys@useobject{currentmarker}{}%
\end{pgfscope}%
\begin{pgfscope}%
\pgfsys@transformshift{0.906321in}{2.629245in}%
\pgfsys@useobject{currentmarker}{}%
\end{pgfscope}%
\begin{pgfscope}%
\pgfsys@transformshift{0.924160in}{2.498589in}%
\pgfsys@useobject{currentmarker}{}%
\end{pgfscope}%
\begin{pgfscope}%
\pgfsys@transformshift{0.945990in}{2.337482in}%
\pgfsys@useobject{currentmarker}{}%
\end{pgfscope}%
\begin{pgfscope}%
\pgfsys@transformshift{0.962655in}{2.275173in}%
\pgfsys@useobject{currentmarker}{}%
\end{pgfscope}%
\begin{pgfscope}%
\pgfsys@transformshift{0.981200in}{2.250313in}%
\pgfsys@useobject{currentmarker}{}%
\end{pgfscope}%
\begin{pgfscope}%
\pgfsys@transformshift{1.003499in}{2.259721in}%
\pgfsys@useobject{currentmarker}{}%
\end{pgfscope}%
\begin{pgfscope}%
\pgfsys@transformshift{1.025563in}{2.308693in}%
\pgfsys@useobject{currentmarker}{}%
\end{pgfscope}%
\begin{pgfscope}%
\pgfsys@transformshift{1.041994in}{2.385622in}%
\pgfsys@useobject{currentmarker}{}%
\end{pgfscope}%
\begin{pgfscope}%
\pgfsys@transformshift{1.058425in}{2.539909in}%
\pgfsys@useobject{currentmarker}{}%
\end{pgfscope}%
\begin{pgfscope}%
\pgfsys@transformshift{1.079550in}{2.636511in}%
\pgfsys@useobject{currentmarker}{}%
\end{pgfscope}%
\begin{pgfscope}%
\pgfsys@transformshift{1.096686in}{2.574493in}%
\pgfsys@useobject{currentmarker}{}%
\end{pgfscope}%
\begin{pgfscope}%
\pgfsys@transformshift{1.118750in}{2.391243in}%
\pgfsys@useobject{currentmarker}{}%
\end{pgfscope}%
\begin{pgfscope}%
\pgfsys@transformshift{1.138938in}{2.290014in}%
\pgfsys@useobject{currentmarker}{}%
\end{pgfscope}%
\begin{pgfscope}%
\pgfsys@transformshift{1.157246in}{2.257846in}%
\pgfsys@useobject{currentmarker}{}%
\end{pgfscope}%
\begin{pgfscope}%
\pgfsys@transformshift{1.176025in}{2.247415in}%
\pgfsys@useobject{currentmarker}{}%
\end{pgfscope}%
\begin{pgfscope}%
\pgfsys@transformshift{1.193629in}{2.264725in}%
\pgfsys@useobject{currentmarker}{}%
\end{pgfscope}%
\begin{pgfscope}%
\pgfsys@transformshift{1.215460in}{2.311783in}%
\pgfsys@useobject{currentmarker}{}%
\end{pgfscope}%
\begin{pgfscope}%
\pgfsys@transformshift{1.233064in}{2.406802in}%
\pgfsys@useobject{currentmarker}{}%
\end{pgfscope}%
\begin{pgfscope}%
\pgfsys@transformshift{1.254189in}{2.524584in}%
\pgfsys@useobject{currentmarker}{}%
\end{pgfscope}%
\begin{pgfscope}%
\pgfsys@transformshift{1.269681in}{2.616174in}%
\pgfsys@useobject{currentmarker}{}%
\end{pgfscope}%
\begin{pgfscope}%
\pgfsys@transformshift{1.290102in}{2.600181in}%
\pgfsys@useobject{currentmarker}{}%
\end{pgfscope}%
\begin{pgfscope}%
\pgfsys@transformshift{1.311229in}{2.420617in}%
\pgfsys@useobject{currentmarker}{}%
\end{pgfscope}%
\begin{pgfscope}%
\pgfsys@transformshift{1.329537in}{2.327614in}%
\pgfsys@useobject{currentmarker}{}%
\end{pgfscope}%
\begin{pgfscope}%
\pgfsys@transformshift{1.346202in}{2.273337in}%
\pgfsys@useobject{currentmarker}{}%
\end{pgfscope}%
\begin{pgfscope}%
\pgfsys@transformshift{1.367329in}{2.250142in}%
\pgfsys@useobject{currentmarker}{}%
\end{pgfscope}%
\begin{pgfscope}%
\pgfsys@transformshift{1.386577in}{2.248862in}%
\pgfsys@useobject{currentmarker}{}%
\end{pgfscope}%
\begin{pgfscope}%
\pgfsys@transformshift{1.407468in}{2.274377in}%
\pgfsys@useobject{currentmarker}{}%
\end{pgfscope}%
\begin{pgfscope}%
\pgfsys@transformshift{1.424133in}{2.314498in}%
\pgfsys@useobject{currentmarker}{}%
\end{pgfscope}%
\begin{pgfscope}%
\pgfsys@transformshift{1.445260in}{2.419881in}%
\pgfsys@useobject{currentmarker}{}%
\end{pgfscope}%
\begin{pgfscope}%
\pgfsys@transformshift{1.464742in}{2.543251in}%
\pgfsys@useobject{currentmarker}{}%
\end{pgfscope}%
\begin{pgfscope}%
\pgfsys@transformshift{1.484929in}{2.623557in}%
\pgfsys@useobject{currentmarker}{}%
\end{pgfscope}%
\begin{pgfscope}%
\pgfsys@transformshift{1.506757in}{2.549471in}%
\pgfsys@useobject{currentmarker}{}%
\end{pgfscope}%
\begin{pgfscope}%
\pgfsys@transformshift{1.520373in}{2.455193in}%
\pgfsys@useobject{currentmarker}{}%
\end{pgfscope}%
\begin{pgfscope}%
\pgfsys@transformshift{1.541029in}{2.319949in}%
\pgfsys@useobject{currentmarker}{}%
\end{pgfscope}%
\begin{pgfscope}%
\pgfsys@transformshift{1.559806in}{2.272995in}%
\pgfsys@useobject{currentmarker}{}%
\end{pgfscope}%
\begin{pgfscope}%
\pgfsys@transformshift{1.580462in}{2.248982in}%
\pgfsys@useobject{currentmarker}{}%
\end{pgfscope}%
\begin{pgfscope}%
\pgfsys@transformshift{1.599241in}{2.245254in}%
\pgfsys@useobject{currentmarker}{}%
\end{pgfscope}%
\begin{pgfscope}%
\pgfsys@transformshift{1.616143in}{2.252101in}%
\pgfsys@useobject{currentmarker}{}%
\end{pgfscope}%
\begin{pgfscope}%
\pgfsys@transformshift{1.637737in}{2.279336in}%
\pgfsys@useobject{currentmarker}{}%
\end{pgfscope}%
\begin{pgfscope}%
\pgfsys@transformshift{1.655576in}{2.323007in}%
\pgfsys@useobject{currentmarker}{}%
\end{pgfscope}%
\begin{pgfscope}%
\pgfsys@transformshift{1.675997in}{2.416959in}%
\pgfsys@useobject{currentmarker}{}%
\end{pgfscope}%
\begin{pgfscope}%
\pgfsys@transformshift{1.692428in}{2.544708in}%
\pgfsys@useobject{currentmarker}{}%
\end{pgfscope}%
\begin{pgfscope}%
\pgfsys@transformshift{1.716372in}{2.618179in}%
\pgfsys@useobject{currentmarker}{}%
\end{pgfscope}%
\begin{pgfscope}%
\pgfsys@transformshift{1.732334in}{2.579966in}%
\pgfsys@useobject{currentmarker}{}%
\end{pgfscope}%
\begin{pgfscope}%
\pgfsys@transformshift{1.754867in}{2.427387in}%
\pgfsys@useobject{currentmarker}{}%
\end{pgfscope}%
\begin{pgfscope}%
\pgfsys@transformshift{1.771767in}{2.332722in}%
\pgfsys@useobject{currentmarker}{}%
\end{pgfscope}%
\begin{pgfscope}%
\pgfsys@transformshift{1.789843in}{2.284915in}%
\pgfsys@useobject{currentmarker}{}%
\end{pgfscope}%
\begin{pgfscope}%
\pgfsys@transformshift{1.808151in}{2.263604in}%
\pgfsys@useobject{currentmarker}{}%
\end{pgfscope}%
\begin{pgfscope}%
\pgfsys@transformshift{1.828807in}{2.247416in}%
\pgfsys@useobject{currentmarker}{}%
\end{pgfscope}%
\begin{pgfscope}%
\pgfsys@transformshift{1.846880in}{2.247974in}%
\pgfsys@useobject{currentmarker}{}%
\end{pgfscope}%
\begin{pgfscope}%
\pgfsys@transformshift{1.867771in}{2.270029in}%
\pgfsys@useobject{currentmarker}{}%
\end{pgfscope}%
\begin{pgfscope}%
\pgfsys@transformshift{1.885141in}{2.308908in}%
\pgfsys@useobject{currentmarker}{}%
\end{pgfscope}%
\begin{pgfscope}%
\pgfsys@transformshift{1.906737in}{2.375266in}%
\pgfsys@useobject{currentmarker}{}%
\end{pgfscope}%
\begin{pgfscope}%
\pgfsys@transformshift{1.924811in}{2.428900in}%
\pgfsys@useobject{currentmarker}{}%
\end{pgfscope}%
\begin{pgfscope}%
\pgfsys@transformshift{1.946172in}{2.569457in}%
\pgfsys@useobject{currentmarker}{}%
\end{pgfscope}%
\begin{pgfscope}%
\pgfsys@transformshift{1.963777in}{2.614238in}%
\pgfsys@useobject{currentmarker}{}%
\end{pgfscope}%
\begin{pgfscope}%
\pgfsys@transformshift{1.982319in}{2.561511in}%
\pgfsys@useobject{currentmarker}{}%
\end{pgfscope}%
\begin{pgfscope}%
\pgfsys@transformshift{2.002507in}{2.441113in}%
\pgfsys@useobject{currentmarker}{}%
\end{pgfscope}%
\begin{pgfscope}%
\pgfsys@transformshift{2.020346in}{2.581264in}%
\pgfsys@useobject{currentmarker}{}%
\end{pgfscope}%
\begin{pgfscope}%
\pgfsys@transformshift{2.038654in}{2.616674in}%
\pgfsys@useobject{currentmarker}{}%
\end{pgfscope}%
\begin{pgfscope}%
\pgfsys@transformshift{2.062127in}{2.548514in}%
\pgfsys@useobject{currentmarker}{}%
\end{pgfscope}%
\begin{pgfscope}%
\pgfsys@transformshift{2.077620in}{2.423751in}%
\pgfsys@useobject{currentmarker}{}%
\end{pgfscope}%
\begin{pgfscope}%
\pgfsys@transformshift{2.099450in}{2.307998in}%
\pgfsys@useobject{currentmarker}{}%
\end{pgfscope}%
\begin{pgfscope}%
\pgfsys@transformshift{2.116584in}{2.265476in}%
\pgfsys@useobject{currentmarker}{}%
\end{pgfscope}%
\begin{pgfscope}%
\pgfsys@transformshift{2.134424in}{2.246998in}%
\pgfsys@useobject{currentmarker}{}%
\end{pgfscope}%
\begin{pgfscope}%
\pgfsys@transformshift{2.152734in}{2.247937in}%
\pgfsys@useobject{currentmarker}{}%
\end{pgfscope}%
\begin{pgfscope}%
\pgfsys@transformshift{2.173624in}{2.266934in}%
\pgfsys@useobject{currentmarker}{}%
\end{pgfscope}%
\begin{pgfscope}%
\pgfsys@transformshift{2.194749in}{2.317378in}%
\pgfsys@useobject{currentmarker}{}%
\end{pgfscope}%
\begin{pgfscope}%
\pgfsys@transformshift{2.213528in}{2.414140in}%
\pgfsys@useobject{currentmarker}{}%
\end{pgfscope}%
\begin{pgfscope}%
\pgfsys@transformshift{2.232776in}{2.548631in}%
\pgfsys@useobject{currentmarker}{}%
\end{pgfscope}%
\begin{pgfscope}%
\pgfsys@transformshift{2.250615in}{2.614017in}%
\pgfsys@useobject{currentmarker}{}%
\end{pgfscope}%
\begin{pgfscope}%
\pgfsys@transformshift{2.269628in}{2.567441in}%
\pgfsys@useobject{currentmarker}{}%
\end{pgfscope}%
\begin{pgfscope}%
\pgfsys@transformshift{2.290989in}{2.408738in}%
\pgfsys@useobject{currentmarker}{}%
\end{pgfscope}%
\begin{pgfscope}%
\pgfsys@transformshift{2.306951in}{2.311927in}%
\pgfsys@useobject{currentmarker}{}%
\end{pgfscope}%
\begin{pgfscope}%
\pgfsys@transformshift{2.327607in}{2.268124in}%
\pgfsys@useobject{currentmarker}{}%
\end{pgfscope}%
\begin{pgfscope}%
\pgfsys@transformshift{2.348498in}{2.246599in}%
\pgfsys@useobject{currentmarker}{}%
\end{pgfscope}%
\begin{pgfscope}%
\pgfsys@transformshift{2.365868in}{2.245248in}%
\pgfsys@useobject{currentmarker}{}%
\end{pgfscope}%
\begin{pgfscope}%
\pgfsys@transformshift{2.384176in}{2.259599in}%
\pgfsys@useobject{currentmarker}{}%
\end{pgfscope}%
\begin{pgfscope}%
\pgfsys@transformshift{2.405772in}{2.295584in}%
\pgfsys@useobject{currentmarker}{}%
\end{pgfscope}%
\begin{pgfscope}%
\pgfsys@transformshift{2.423141in}{2.363952in}%
\pgfsys@useobject{currentmarker}{}%
\end{pgfscope}%
\begin{pgfscope}%
\pgfsys@transformshift{2.441216in}{2.474156in}%
\pgfsys@useobject{currentmarker}{}%
\end{pgfscope}%
\begin{pgfscope}%
\pgfsys@transformshift{2.462341in}{2.598132in}%
\pgfsys@useobject{currentmarker}{}%
\end{pgfscope}%
\begin{pgfscope}%
\pgfsys@transformshift{2.480415in}{2.613591in}%
\pgfsys@useobject{currentmarker}{}%
\end{pgfscope}%
\begin{pgfscope}%
\pgfsys@transformshift{2.501307in}{2.543821in}%
\pgfsys@useobject{currentmarker}{}%
\end{pgfscope}%
\begin{pgfscope}%
\pgfsys@transformshift{2.520555in}{2.408015in}%
\pgfsys@useobject{currentmarker}{}%
\end{pgfscope}%
\begin{pgfscope}%
\pgfsys@transformshift{2.540740in}{2.304749in}%
\pgfsys@useobject{currentmarker}{}%
\end{pgfscope}%
\begin{pgfscope}%
\pgfsys@transformshift{2.559050in}{2.268266in}%
\pgfsys@useobject{currentmarker}{}%
\end{pgfscope}%
\begin{pgfscope}%
\pgfsys@transformshift{2.576655in}{2.249855in}%
\pgfsys@useobject{currentmarker}{}%
\end{pgfscope}%
\begin{pgfscope}%
\pgfsys@transformshift{2.598249in}{2.245018in}%
\pgfsys@useobject{currentmarker}{}%
\end{pgfscope}%
\begin{pgfscope}%
\pgfsys@transformshift{2.615619in}{2.256429in}%
\pgfsys@useobject{currentmarker}{}%
\end{pgfscope}%
\begin{pgfscope}%
\pgfsys@transformshift{2.640267in}{2.278708in}%
\pgfsys@useobject{currentmarker}{}%
\end{pgfscope}%
\begin{pgfscope}%
\pgfsys@transformshift{2.655289in}{2.317381in}%
\pgfsys@useobject{currentmarker}{}%
\end{pgfscope}%
\begin{pgfscope}%
\pgfsys@transformshift{2.672425in}{2.385036in}%
\pgfsys@useobject{currentmarker}{}%
\end{pgfscope}%
\begin{pgfscope}%
\pgfsys@transformshift{2.693784in}{2.453004in}%
\pgfsys@useobject{currentmarker}{}%
\end{pgfscope}%
\begin{pgfscope}%
\pgfsys@transformshift{2.712797in}{2.575893in}%
\pgfsys@useobject{currentmarker}{}%
\end{pgfscope}%
\begin{pgfscope}%
\pgfsys@transformshift{2.733688in}{2.609279in}%
\pgfsys@useobject{currentmarker}{}%
\end{pgfscope}%
\begin{pgfscope}%
\pgfsys@transformshift{2.751764in}{2.533745in}%
\pgfsys@useobject{currentmarker}{}%
\end{pgfscope}%
\begin{pgfscope}%
\pgfsys@transformshift{2.769837in}{2.454085in}%
\pgfsys@useobject{currentmarker}{}%
\end{pgfscope}%
\begin{pgfscope}%
\pgfsys@transformshift{2.792840in}{2.304961in}%
\pgfsys@useobject{currentmarker}{}%
\end{pgfscope}%
\begin{pgfscope}%
\pgfsys@transformshift{2.806455in}{2.278973in}%
\pgfsys@useobject{currentmarker}{}%
\end{pgfscope}%
\begin{pgfscope}%
\pgfsys@transformshift{2.827815in}{2.251989in}%
\pgfsys@useobject{currentmarker}{}%
\end{pgfscope}%
\begin{pgfscope}%
\pgfsys@transformshift{2.847533in}{2.245893in}%
\pgfsys@useobject{currentmarker}{}%
\end{pgfscope}%
\begin{pgfscope}%
\pgfsys@transformshift{2.866076in}{2.248614in}%
\pgfsys@useobject{currentmarker}{}%
\end{pgfscope}%
\begin{pgfscope}%
\pgfsys@transformshift{2.886497in}{2.273530in}%
\pgfsys@useobject{currentmarker}{}%
\end{pgfscope}%
\begin{pgfscope}%
\pgfsys@transformshift{2.906450in}{2.317350in}%
\pgfsys@useobject{currentmarker}{}%
\end{pgfscope}%
\begin{pgfscope}%
\pgfsys@transformshift{2.923819in}{2.397753in}%
\pgfsys@useobject{currentmarker}{}%
\end{pgfscope}%
\begin{pgfscope}%
\pgfsys@transformshift{2.943066in}{2.503593in}%
\pgfsys@useobject{currentmarker}{}%
\end{pgfscope}%
\begin{pgfscope}%
\pgfsys@transformshift{2.962080in}{2.569444in}%
\pgfsys@useobject{currentmarker}{}%
\end{pgfscope}%
\begin{pgfscope}%
\pgfsys@transformshift{2.982970in}{2.617266in}%
\pgfsys@useobject{currentmarker}{}%
\end{pgfscope}%
\begin{pgfscope}%
\pgfsys@transformshift{3.000341in}{2.545396in}%
\pgfsys@useobject{currentmarker}{}%
\end{pgfscope}%
\begin{pgfscope}%
\pgfsys@transformshift{3.018885in}{2.415288in}%
\pgfsys@useobject{currentmarker}{}%
\end{pgfscope}%
\begin{pgfscope}%
\pgfsys@transformshift{3.039072in}{2.317353in}%
\pgfsys@useobject{currentmarker}{}%
\end{pgfscope}%
\begin{pgfscope}%
\pgfsys@transformshift{3.057146in}{2.273215in}%
\pgfsys@useobject{currentmarker}{}%
\end{pgfscope}%
\begin{pgfscope}%
\pgfsys@transformshift{3.075454in}{2.254654in}%
\pgfsys@useobject{currentmarker}{}%
\end{pgfscope}%
\begin{pgfscope}%
\pgfsys@transformshift{3.096110in}{2.246118in}%
\pgfsys@useobject{currentmarker}{}%
\end{pgfscope}%
\begin{pgfscope}%
\pgfsys@transformshift{3.117471in}{2.259125in}%
\pgfsys@useobject{currentmarker}{}%
\end{pgfscope}%
\begin{pgfscope}%
\pgfsys@transformshift{3.135311in}{2.281962in}%
\pgfsys@useobject{currentmarker}{}%
\end{pgfscope}%
\begin{pgfscope}%
\pgfsys@transformshift{3.135545in}{2.310327in}%
\pgfsys@useobject{currentmarker}{}%
\end{pgfscope}%
\begin{pgfscope}%
\pgfsys@transformshift{3.154090in}{2.323208in}%
\pgfsys@useobject{currentmarker}{}%
\end{pgfscope}%
\begin{pgfscope}%
\pgfsys@transformshift{3.175215in}{2.410208in}%
\pgfsys@useobject{currentmarker}{}%
\end{pgfscope}%
\begin{pgfscope}%
\pgfsys@transformshift{3.192114in}{2.246133in}%
\pgfsys@useobject{currentmarker}{}%
\end{pgfscope}%
\begin{pgfscope}%
\pgfsys@transformshift{3.214415in}{2.261520in}%
\pgfsys@useobject{currentmarker}{}%
\end{pgfscope}%
\begin{pgfscope}%
\pgfsys@transformshift{3.231549in}{2.295592in}%
\pgfsys@useobject{currentmarker}{}%
\end{pgfscope}%
\begin{pgfscope}%
\pgfsys@transformshift{3.251031in}{2.366591in}%
\pgfsys@useobject{currentmarker}{}%
\end{pgfscope}%
\begin{pgfscope}%
\pgfsys@transformshift{3.270984in}{2.495934in}%
\pgfsys@useobject{currentmarker}{}%
\end{pgfscope}%
\begin{pgfscope}%
\pgfsys@transformshift{3.290935in}{2.603889in}%
\pgfsys@useobject{currentmarker}{}%
\end{pgfscope}%
\begin{pgfscope}%
\pgfsys@transformshift{3.309948in}{2.619615in}%
\pgfsys@useobject{currentmarker}{}%
\end{pgfscope}%
\begin{pgfscope}%
\pgfsys@transformshift{3.328493in}{2.523061in}%
\pgfsys@useobject{currentmarker}{}%
\end{pgfscope}%
\begin{pgfscope}%
\pgfsys@transformshift{3.349149in}{2.394136in}%
\pgfsys@useobject{currentmarker}{}%
\end{pgfscope}%
\begin{pgfscope}%
\pgfsys@transformshift{3.366754in}{2.310906in}%
\pgfsys@useobject{currentmarker}{}%
\end{pgfscope}%
\begin{pgfscope}%
\pgfsys@transformshift{3.384593in}{2.274283in}%
\pgfsys@useobject{currentmarker}{}%
\end{pgfscope}%
\begin{pgfscope}%
\pgfsys@transformshift{3.406892in}{2.260462in}%
\pgfsys@useobject{currentmarker}{}%
\end{pgfscope}%
\begin{pgfscope}%
\pgfsys@transformshift{3.424731in}{2.247263in}%
\pgfsys@useobject{currentmarker}{}%
\end{pgfscope}%
\begin{pgfscope}%
\pgfsys@transformshift{3.442807in}{2.251660in}%
\pgfsys@useobject{currentmarker}{}%
\end{pgfscope}%
\begin{pgfscope}%
\pgfsys@transformshift{3.462289in}{2.274914in}%
\pgfsys@useobject{currentmarker}{}%
\end{pgfscope}%
\begin{pgfscope}%
\pgfsys@transformshift{3.482005in}{2.314085in}%
\pgfsys@useobject{currentmarker}{}%
\end{pgfscope}%
\begin{pgfscope}%
\pgfsys@transformshift{3.500315in}{2.399247in}%
\pgfsys@useobject{currentmarker}{}%
\end{pgfscope}%
\begin{pgfscope}%
\pgfsys@transformshift{3.517684in}{2.484067in}%
\pgfsys@useobject{currentmarker}{}%
\end{pgfscope}%
\begin{pgfscope}%
\pgfsys@transformshift{3.539985in}{2.519658in}%
\pgfsys@useobject{currentmarker}{}%
\end{pgfscope}%
\begin{pgfscope}%
\pgfsys@transformshift{3.557353in}{2.528071in}%
\pgfsys@useobject{currentmarker}{}%
\end{pgfscope}%
\begin{pgfscope}%
\pgfsys@transformshift{3.581061in}{2.631257in}%
\pgfsys@useobject{currentmarker}{}%
\end{pgfscope}%
\begin{pgfscope}%
\pgfsys@transformshift{3.595614in}{2.614263in}%
\pgfsys@useobject{currentmarker}{}%
\end{pgfscope}%
\begin{pgfscope}%
\pgfsys@transformshift{3.617913in}{2.478508in}%
\pgfsys@useobject{currentmarker}{}%
\end{pgfscope}%
\begin{pgfscope}%
\pgfsys@transformshift{3.635049in}{2.378346in}%
\pgfsys@useobject{currentmarker}{}%
\end{pgfscope}%
\begin{pgfscope}%
\pgfsys@transformshift{3.652419in}{2.304367in}%
\pgfsys@useobject{currentmarker}{}%
\end{pgfscope}%
\begin{pgfscope}%
\pgfsys@transformshift{3.672370in}{2.268602in}%
\pgfsys@useobject{currentmarker}{}%
\end{pgfscope}%
\begin{pgfscope}%
\pgfsys@transformshift{3.691149in}{2.254634in}%
\pgfsys@useobject{currentmarker}{}%
\end{pgfscope}%
\begin{pgfscope}%
\pgfsys@transformshift{3.711102in}{2.249151in}%
\pgfsys@useobject{currentmarker}{}%
\end{pgfscope}%
\begin{pgfscope}%
\pgfsys@transformshift{3.729176in}{2.265587in}%
\pgfsys@useobject{currentmarker}{}%
\end{pgfscope}%
\begin{pgfscope}%
\pgfsys@transformshift{3.751475in}{2.301723in}%
\pgfsys@useobject{currentmarker}{}%
\end{pgfscope}%
\begin{pgfscope}%
\pgfsys@transformshift{3.769548in}{2.353442in}%
\pgfsys@useobject{currentmarker}{}%
\end{pgfscope}%
\begin{pgfscope}%
\pgfsys@transformshift{3.789032in}{2.457656in}%
\pgfsys@useobject{currentmarker}{}%
\end{pgfscope}%
\begin{pgfscope}%
\pgfsys@transformshift{3.808046in}{2.572502in}%
\pgfsys@useobject{currentmarker}{}%
\end{pgfscope}%
\begin{pgfscope}%
\pgfsys@transformshift{3.826823in}{2.639300in}%
\pgfsys@useobject{currentmarker}{}%
\end{pgfscope}%
\begin{pgfscope}%
\pgfsys@transformshift{3.846070in}{2.627062in}%
\pgfsys@useobject{currentmarker}{}%
\end{pgfscope}%
\begin{pgfscope}%
\pgfsys@transformshift{3.864615in}{2.549533in}%
\pgfsys@useobject{currentmarker}{}%
\end{pgfscope}%
\begin{pgfscope}%
\pgfsys@transformshift{3.884097in}{2.427136in}%
\pgfsys@useobject{currentmarker}{}%
\end{pgfscope}%
\begin{pgfscope}%
\pgfsys@transformshift{3.903110in}{2.351847in}%
\pgfsys@useobject{currentmarker}{}%
\end{pgfscope}%
\begin{pgfscope}%
\pgfsys@transformshift{3.925880in}{2.288802in}%
\pgfsys@useobject{currentmarker}{}%
\end{pgfscope}%
\begin{pgfscope}%
\pgfsys@transformshift{3.944422in}{2.259238in}%
\pgfsys@useobject{currentmarker}{}%
\end{pgfscope}%
\begin{pgfscope}%
\pgfsys@transformshift{3.964610in}{2.257435in}%
\pgfsys@useobject{currentmarker}{}%
\end{pgfscope}%
\begin{pgfscope}%
\pgfsys@transformshift{3.982214in}{2.249899in}%
\pgfsys@useobject{currentmarker}{}%
\end{pgfscope}%
\begin{pgfscope}%
\pgfsys@transformshift{4.001462in}{2.264156in}%
\pgfsys@useobject{currentmarker}{}%
\end{pgfscope}%
\begin{pgfscope}%
\pgfsys@transformshift{4.019770in}{2.293312in}%
\pgfsys@useobject{currentmarker}{}%
\end{pgfscope}%
\begin{pgfscope}%
\pgfsys@transformshift{4.040426in}{2.351107in}%
\pgfsys@useobject{currentmarker}{}%
\end{pgfscope}%
\begin{pgfscope}%
\pgfsys@transformshift{4.055685in}{2.432592in}%
\pgfsys@useobject{currentmarker}{}%
\end{pgfscope}%
\begin{pgfscope}%
\pgfsys@transformshift{4.077279in}{2.531283in}%
\pgfsys@useobject{currentmarker}{}%
\end{pgfscope}%
\begin{pgfscope}%
\pgfsys@transformshift{4.098169in}{2.615257in}%
\pgfsys@useobject{currentmarker}{}%
\end{pgfscope}%
\begin{pgfscope}%
\pgfsys@transformshift{4.117183in}{2.654400in}%
\pgfsys@useobject{currentmarker}{}%
\end{pgfscope}%
\begin{pgfscope}%
\pgfsys@transformshift{4.135024in}{2.611802in}%
\pgfsys@useobject{currentmarker}{}%
\end{pgfscope}%
\begin{pgfscope}%
\pgfsys@transformshift{4.150984in}{2.515346in}%
\pgfsys@useobject{currentmarker}{}%
\end{pgfscope}%
\begin{pgfscope}%
\pgfsys@transformshift{4.173519in}{2.376935in}%
\pgfsys@useobject{currentmarker}{}%
\end{pgfscope}%
\begin{pgfscope}%
\pgfsys@transformshift{4.193001in}{2.305961in}%
\pgfsys@useobject{currentmarker}{}%
\end{pgfscope}%
\begin{pgfscope}%
\pgfsys@transformshift{4.211075in}{2.604246in}%
\pgfsys@useobject{currentmarker}{}%
\end{pgfscope}%
\begin{pgfscope}%
\pgfsys@transformshift{4.229619in}{2.497077in}%
\pgfsys@useobject{currentmarker}{}%
\end{pgfscope}%
\begin{pgfscope}%
\pgfsys@transformshift{4.248162in}{2.371483in}%
\pgfsys@useobject{currentmarker}{}%
\end{pgfscope}%
\begin{pgfscope}%
\pgfsys@transformshift{4.272340in}{2.292601in}%
\pgfsys@useobject{currentmarker}{}%
\end{pgfscope}%
\begin{pgfscope}%
\pgfsys@transformshift{4.288771in}{2.262380in}%
\pgfsys@useobject{currentmarker}{}%
\end{pgfscope}%
\begin{pgfscope}%
\pgfsys@transformshift{4.308722in}{2.252130in}%
\pgfsys@useobject{currentmarker}{}%
\end{pgfscope}%
\begin{pgfscope}%
\pgfsys@transformshift{4.329378in}{2.271933in}%
\pgfsys@useobject{currentmarker}{}%
\end{pgfscope}%
\begin{pgfscope}%
\pgfsys@transformshift{4.344637in}{2.297965in}%
\pgfsys@useobject{currentmarker}{}%
\end{pgfscope}%
\begin{pgfscope}%
\pgfsys@transformshift{4.363179in}{2.355908in}%
\pgfsys@useobject{currentmarker}{}%
\end{pgfscope}%
\begin{pgfscope}%
\pgfsys@transformshift{4.385009in}{2.476867in}%
\pgfsys@useobject{currentmarker}{}%
\end{pgfscope}%
\begin{pgfscope}%
\pgfsys@transformshift{4.404023in}{2.467288in}%
\pgfsys@useobject{currentmarker}{}%
\end{pgfscope}%
\begin{pgfscope}%
\pgfsys@transformshift{4.422331in}{2.595893in}%
\pgfsys@useobject{currentmarker}{}%
\end{pgfscope}%
\begin{pgfscope}%
\pgfsys@transformshift{4.442518in}{2.665901in}%
\pgfsys@useobject{currentmarker}{}%
\end{pgfscope}%
\begin{pgfscope}%
\pgfsys@transformshift{4.460828in}{2.652432in}%
\pgfsys@useobject{currentmarker}{}%
\end{pgfscope}%
\begin{pgfscope}%
\pgfsys@transformshift{4.480076in}{2.554861in}%
\pgfsys@useobject{currentmarker}{}%
\end{pgfscope}%
\begin{pgfscope}%
\pgfsys@transformshift{4.478902in}{2.578690in}%
\pgfsys@useobject{currentmarker}{}%
\end{pgfscope}%
\begin{pgfscope}%
\pgfsys@transformshift{4.474676in}{2.620859in}%
\pgfsys@useobject{currentmarker}{}%
\end{pgfscope}%
\begin{pgfscope}%
\pgfsys@transformshift{4.454254in}{2.668136in}%
\pgfsys@useobject{currentmarker}{}%
\end{pgfscope}%
\begin{pgfscope}%
\pgfsys@transformshift{4.436650in}{2.576646in}%
\pgfsys@useobject{currentmarker}{}%
\end{pgfscope}%
\begin{pgfscope}%
\pgfsys@transformshift{4.418107in}{2.403272in}%
\pgfsys@useobject{currentmarker}{}%
\end{pgfscope}%
\begin{pgfscope}%
\pgfsys@transformshift{4.396277in}{2.299478in}%
\pgfsys@useobject{currentmarker}{}%
\end{pgfscope}%
\begin{pgfscope}%
\pgfsys@transformshift{4.377029in}{2.258824in}%
\pgfsys@useobject{currentmarker}{}%
\end{pgfscope}%
\begin{pgfscope}%
\pgfsys@transformshift{4.358954in}{2.255939in}%
\pgfsys@useobject{currentmarker}{}%
\end{pgfscope}%
\begin{pgfscope}%
\pgfsys@transformshift{4.340646in}{2.287284in}%
\pgfsys@useobject{currentmarker}{}%
\end{pgfscope}%
\begin{pgfscope}%
\pgfsys@transformshift{4.319050in}{2.384785in}%
\pgfsys@useobject{currentmarker}{}%
\end{pgfscope}%
\begin{pgfscope}%
\pgfsys@transformshift{4.299802in}{2.552702in}%
\pgfsys@useobject{currentmarker}{}%
\end{pgfscope}%
\begin{pgfscope}%
\pgfsys@transformshift{4.281260in}{2.653381in}%
\pgfsys@useobject{currentmarker}{}%
\end{pgfscope}%
\begin{pgfscope}%
\pgfsys@transformshift{4.262715in}{2.615521in}%
\pgfsys@useobject{currentmarker}{}%
\end{pgfscope}%
\begin{pgfscope}%
\pgfsys@transformshift{4.243936in}{2.440449in}%
\pgfsys@useobject{currentmarker}{}%
\end{pgfscope}%
\begin{pgfscope}%
\pgfsys@transformshift{4.225394in}{2.326692in}%
\pgfsys@useobject{currentmarker}{}%
\end{pgfscope}%
\begin{pgfscope}%
\pgfsys@transformshift{4.207555in}{2.271057in}%
\pgfsys@useobject{currentmarker}{}%
\end{pgfscope}%
\begin{pgfscope}%
\pgfsys@transformshift{4.184551in}{2.249630in}%
\pgfsys@useobject{currentmarker}{}%
\end{pgfscope}%
\begin{pgfscope}%
\pgfsys@transformshift{4.166946in}{2.269006in}%
\pgfsys@useobject{currentmarker}{}%
\end{pgfscope}%
\begin{pgfscope}%
\pgfsys@transformshift{4.147932in}{2.321176in}%
\pgfsys@useobject{currentmarker}{}%
\end{pgfscope}%
\begin{pgfscope}%
\pgfsys@transformshift{4.128216in}{2.440584in}%
\pgfsys@useobject{currentmarker}{}%
\end{pgfscope}%
\begin{pgfscope}%
\pgfsys@transformshift{4.110377in}{2.590322in}%
\pgfsys@useobject{currentmarker}{}%
\end{pgfscope}%
\begin{pgfscope}%
\pgfsys@transformshift{4.089250in}{2.644072in}%
\pgfsys@useobject{currentmarker}{}%
\end{pgfscope}%
\begin{pgfscope}%
\pgfsys@transformshift{4.072585in}{2.564095in}%
\pgfsys@useobject{currentmarker}{}%
\end{pgfscope}%
\begin{pgfscope}%
\pgfsys@transformshift{4.051225in}{2.377113in}%
\pgfsys@useobject{currentmarker}{}%
\end{pgfscope}%
\begin{pgfscope}%
\pgfsys@transformshift{4.033620in}{2.296155in}%
\pgfsys@useobject{currentmarker}{}%
\end{pgfscope}%
\begin{pgfscope}%
\pgfsys@transformshift{4.010851in}{2.256092in}%
\pgfsys@useobject{currentmarker}{}%
\end{pgfscope}%
\begin{pgfscope}%
\pgfsys@transformshift{3.993717in}{2.250707in}%
\pgfsys@useobject{currentmarker}{}%
\end{pgfscope}%
\begin{pgfscope}%
\pgfsys@transformshift{3.973764in}{2.277243in}%
\pgfsys@useobject{currentmarker}{}%
\end{pgfscope}%
\begin{pgfscope}%
\pgfsys@transformshift{3.955219in}{2.337492in}%
\pgfsys@useobject{currentmarker}{}%
\end{pgfscope}%
\begin{pgfscope}%
\pgfsys@transformshift{3.936442in}{2.475014in}%
\pgfsys@useobject{currentmarker}{}%
\end{pgfscope}%
\begin{pgfscope}%
\pgfsys@transformshift{3.918367in}{2.616577in}%
\pgfsys@useobject{currentmarker}{}%
\end{pgfscope}%
\begin{pgfscope}%
\pgfsys@transformshift{3.897242in}{2.617320in}%
\pgfsys@useobject{currentmarker}{}%
\end{pgfscope}%
\begin{pgfscope}%
\pgfsys@transformshift{3.877289in}{2.469214in}%
\pgfsys@useobject{currentmarker}{}%
\end{pgfscope}%
\begin{pgfscope}%
\pgfsys@transformshift{3.858278in}{2.353283in}%
\pgfsys@useobject{currentmarker}{}%
\end{pgfscope}%
\begin{pgfscope}%
\pgfsys@transformshift{3.839733in}{2.285265in}%
\pgfsys@useobject{currentmarker}{}%
\end{pgfscope}%
\begin{pgfscope}%
\pgfsys@transformshift{3.821425in}{2.254286in}%
\pgfsys@useobject{currentmarker}{}%
\end{pgfscope}%
\begin{pgfscope}%
\pgfsys@transformshift{3.799595in}{2.250685in}%
\pgfsys@useobject{currentmarker}{}%
\end{pgfscope}%
\begin{pgfscope}%
\pgfsys@transformshift{3.781051in}{2.270052in}%
\pgfsys@useobject{currentmarker}{}%
\end{pgfscope}%
\begin{pgfscope}%
\pgfsys@transformshift{3.765794in}{2.311109in}%
\pgfsys@useobject{currentmarker}{}%
\end{pgfscope}%
\begin{pgfscope}%
\pgfsys@transformshift{3.741616in}{2.465294in}%
\pgfsys@useobject{currentmarker}{}%
\end{pgfscope}%
\begin{pgfscope}%
\pgfsys@transformshift{3.725185in}{2.552010in}%
\pgfsys@useobject{currentmarker}{}%
\end{pgfscope}%
\begin{pgfscope}%
\pgfsys@transformshift{3.704529in}{2.628365in}%
\pgfsys@useobject{currentmarker}{}%
\end{pgfscope}%
\begin{pgfscope}%
\pgfsys@transformshift{3.685281in}{2.579172in}%
\pgfsys@useobject{currentmarker}{}%
\end{pgfscope}%
\begin{pgfscope}%
\pgfsys@transformshift{3.666502in}{2.427197in}%
\pgfsys@useobject{currentmarker}{}%
\end{pgfscope}%
\begin{pgfscope}%
\pgfsys@transformshift{3.647960in}{2.321114in}%
\pgfsys@useobject{currentmarker}{}%
\end{pgfscope}%
\begin{pgfscope}%
\pgfsys@transformshift{3.628712in}{2.270771in}%
\pgfsys@useobject{currentmarker}{}%
\end{pgfscope}%
\begin{pgfscope}%
\pgfsys@transformshift{3.609933in}{2.248354in}%
\pgfsys@useobject{currentmarker}{}%
\end{pgfscope}%
\begin{pgfscope}%
\pgfsys@transformshift{3.590451in}{2.247870in}%
\pgfsys@useobject{currentmarker}{}%
\end{pgfscope}%
\begin{pgfscope}%
\pgfsys@transformshift{3.567918in}{2.268322in}%
\pgfsys@useobject{currentmarker}{}%
\end{pgfscope}%
\begin{pgfscope}%
\pgfsys@transformshift{3.551016in}{2.305603in}%
\pgfsys@useobject{currentmarker}{}%
\end{pgfscope}%
\begin{pgfscope}%
\pgfsys@transformshift{3.534585in}{2.389648in}%
\pgfsys@useobject{currentmarker}{}%
\end{pgfscope}%
\begin{pgfscope}%
\pgfsys@transformshift{3.513460in}{2.542837in}%
\pgfsys@useobject{currentmarker}{}%
\end{pgfscope}%
\begin{pgfscope}%
\pgfsys@transformshift{3.493976in}{2.619144in}%
\pgfsys@useobject{currentmarker}{}%
\end{pgfscope}%
\begin{pgfscope}%
\pgfsys@transformshift{3.475199in}{2.606867in}%
\pgfsys@useobject{currentmarker}{}%
\end{pgfscope}%
\begin{pgfscope}%
\pgfsys@transformshift{3.451021in}{2.535244in}%
\pgfsys@useobject{currentmarker}{}%
\end{pgfscope}%
\begin{pgfscope}%
\pgfsys@transformshift{3.436468in}{2.402894in}%
\pgfsys@useobject{currentmarker}{}%
\end{pgfscope}%
\begin{pgfscope}%
\pgfsys@transformshift{3.417220in}{2.313496in}%
\pgfsys@useobject{currentmarker}{}%
\end{pgfscope}%
\begin{pgfscope}%
\pgfsys@transformshift{3.401258in}{2.321950in}%
\pgfsys@useobject{currentmarker}{}%
\end{pgfscope}%
\begin{pgfscope}%
\pgfsys@transformshift{3.378959in}{2.619000in}%
\pgfsys@useobject{currentmarker}{}%
\end{pgfscope}%
\begin{pgfscope}%
\pgfsys@transformshift{3.357365in}{2.559496in}%
\pgfsys@useobject{currentmarker}{}%
\end{pgfscope}%
\begin{pgfscope}%
\pgfsys@transformshift{3.343046in}{2.433001in}%
\pgfsys@useobject{currentmarker}{}%
\end{pgfscope}%
\begin{pgfscope}%
\pgfsys@transformshift{3.320747in}{2.316020in}%
\pgfsys@useobject{currentmarker}{}%
\end{pgfscope}%
\begin{pgfscope}%
\pgfsys@transformshift{3.302672in}{2.266631in}%
\pgfsys@useobject{currentmarker}{}%
\end{pgfscope}%
\begin{pgfscope}%
\pgfsys@transformshift{3.282721in}{2.249469in}%
\pgfsys@useobject{currentmarker}{}%
\end{pgfscope}%
\begin{pgfscope}%
\pgfsys@transformshift{3.266055in}{2.247422in}%
\pgfsys@useobject{currentmarker}{}%
\end{pgfscope}%
\begin{pgfscope}%
\pgfsys@transformshift{3.243520in}{2.272404in}%
\pgfsys@useobject{currentmarker}{}%
\end{pgfscope}%
\begin{pgfscope}%
\pgfsys@transformshift{3.225446in}{2.324303in}%
\pgfsys@useobject{currentmarker}{}%
\end{pgfscope}%
\begin{pgfscope}%
\pgfsys@transformshift{3.209719in}{2.420782in}%
\pgfsys@useobject{currentmarker}{}%
\end{pgfscope}%
\begin{pgfscope}%
\pgfsys@transformshift{3.186482in}{2.586337in}%
\pgfsys@useobject{currentmarker}{}%
\end{pgfscope}%
\begin{pgfscope}%
\pgfsys@transformshift{3.166764in}{2.606532in}%
\pgfsys@useobject{currentmarker}{}%
\end{pgfscope}%
\begin{pgfscope}%
\pgfsys@transformshift{3.149393in}{2.522807in}%
\pgfsys@useobject{currentmarker}{}%
\end{pgfscope}%
\begin{pgfscope}%
\pgfsys@transformshift{3.128034in}{2.373377in}%
\pgfsys@useobject{currentmarker}{}%
\end{pgfscope}%
\begin{pgfscope}%
\pgfsys@transformshift{3.109490in}{2.300429in}%
\pgfsys@useobject{currentmarker}{}%
\end{pgfscope}%
\begin{pgfscope}%
\pgfsys@transformshift{3.090242in}{2.260915in}%
\pgfsys@useobject{currentmarker}{}%
\end{pgfscope}%
\begin{pgfscope}%
\pgfsys@transformshift{3.068882in}{2.245116in}%
\pgfsys@useobject{currentmarker}{}%
\end{pgfscope}%
\begin{pgfscope}%
\pgfsys@transformshift{3.053389in}{2.252591in}%
\pgfsys@useobject{currentmarker}{}%
\end{pgfscope}%
\begin{pgfscope}%
\pgfsys@transformshift{3.031796in}{2.279531in}%
\pgfsys@useobject{currentmarker}{}%
\end{pgfscope}%
\begin{pgfscope}%
\pgfsys@transformshift{3.013954in}{2.340943in}%
\pgfsys@useobject{currentmarker}{}%
\end{pgfscope}%
\begin{pgfscope}%
\pgfsys@transformshift{2.994943in}{2.462763in}%
\pgfsys@useobject{currentmarker}{}%
\end{pgfscope}%
\begin{pgfscope}%
\pgfsys@transformshift{2.973113in}{2.593357in}%
\pgfsys@useobject{currentmarker}{}%
\end{pgfscope}%
\begin{pgfscope}%
\pgfsys@transformshift{2.955743in}{2.612022in}%
\pgfsys@useobject{currentmarker}{}%
\end{pgfscope}%
\begin{pgfscope}%
\pgfsys@transformshift{2.936495in}{2.510480in}%
\pgfsys@useobject{currentmarker}{}%
\end{pgfscope}%
\begin{pgfscope}%
\pgfsys@transformshift{2.918656in}{2.428365in}%
\pgfsys@useobject{currentmarker}{}%
\end{pgfscope}%
\begin{pgfscope}%
\pgfsys@transformshift{2.899877in}{2.319458in}%
\pgfsys@useobject{currentmarker}{}%
\end{pgfscope}%
\begin{pgfscope}%
\pgfsys@transformshift{2.881569in}{2.268688in}%
\pgfsys@useobject{currentmarker}{}%
\end{pgfscope}%
\begin{pgfscope}%
\pgfsys@transformshift{2.852696in}{2.246165in}%
\pgfsys@useobject{currentmarker}{}%
\end{pgfscope}%
\begin{pgfscope}%
\pgfsys@transformshift{2.840960in}{2.246189in}%
\pgfsys@useobject{currentmarker}{}%
\end{pgfscope}%
\begin{pgfscope}%
\pgfsys@transformshift{2.821946in}{2.263897in}%
\pgfsys@useobject{currentmarker}{}%
\end{pgfscope}%
\begin{pgfscope}%
\pgfsys@transformshift{2.803638in}{2.295590in}%
\pgfsys@useobject{currentmarker}{}%
\end{pgfscope}%
\begin{pgfscope}%
\pgfsys@transformshift{2.783217in}{2.377317in}%
\pgfsys@useobject{currentmarker}{}%
\end{pgfscope}%
\begin{pgfscope}%
\pgfsys@transformshift{2.764203in}{2.478273in}%
\pgfsys@useobject{currentmarker}{}%
\end{pgfscope}%
\begin{pgfscope}%
\pgfsys@transformshift{2.744251in}{2.596819in}%
\pgfsys@useobject{currentmarker}{}%
\end{pgfscope}%
\begin{pgfscope}%
\pgfsys@transformshift{2.726646in}{2.603192in}%
\pgfsys@useobject{currentmarker}{}%
\end{pgfscope}%
\begin{pgfscope}%
\pgfsys@transformshift{2.707634in}{2.492594in}%
\pgfsys@useobject{currentmarker}{}%
\end{pgfscope}%
\begin{pgfscope}%
\pgfsys@transformshift{2.686039in}{2.360021in}%
\pgfsys@useobject{currentmarker}{}%
\end{pgfscope}%
\begin{pgfscope}%
\pgfsys@transformshift{2.667729in}{2.294254in}%
\pgfsys@useobject{currentmarker}{}%
\end{pgfscope}%
\begin{pgfscope}%
\pgfsys@transformshift{2.649186in}{2.263066in}%
\pgfsys@useobject{currentmarker}{}%
\end{pgfscope}%
\begin{pgfscope}%
\pgfsys@transformshift{2.627122in}{2.247154in}%
\pgfsys@useobject{currentmarker}{}%
\end{pgfscope}%
\begin{pgfscope}%
\pgfsys@transformshift{2.606700in}{2.248329in}%
\pgfsys@useobject{currentmarker}{}%
\end{pgfscope}%
\begin{pgfscope}%
\pgfsys@transformshift{2.588392in}{2.265819in}%
\pgfsys@useobject{currentmarker}{}%
\end{pgfscope}%
\begin{pgfscope}%
\pgfsys@transformshift{2.569613in}{2.305512in}%
\pgfsys@useobject{currentmarker}{}%
\end{pgfscope}%
\begin{pgfscope}%
\pgfsys@transformshift{2.553417in}{2.384968in}%
\pgfsys@useobject{currentmarker}{}%
\end{pgfscope}%
\begin{pgfscope}%
\pgfsys@transformshift{2.535107in}{2.528051in}%
\pgfsys@useobject{currentmarker}{}%
\end{pgfscope}%
\begin{pgfscope}%
\pgfsys@transformshift{2.514216in}{2.608388in}%
\pgfsys@useobject{currentmarker}{}%
\end{pgfscope}%
\begin{pgfscope}%
\pgfsys@transformshift{2.493091in}{2.561811in}%
\pgfsys@useobject{currentmarker}{}%
\end{pgfscope}%
\begin{pgfscope}%
\pgfsys@transformshift{2.474781in}{2.460172in}%
\pgfsys@useobject{currentmarker}{}%
\end{pgfscope}%
\begin{pgfscope}%
\pgfsys@transformshift{2.455533in}{2.347080in}%
\pgfsys@useobject{currentmarker}{}%
\end{pgfscope}%
\begin{pgfscope}%
\pgfsys@transformshift{2.438399in}{2.293309in}%
\pgfsys@useobject{currentmarker}{}%
\end{pgfscope}%
\begin{pgfscope}%
\pgfsys@transformshift{2.415866in}{2.258223in}%
\pgfsys@useobject{currentmarker}{}%
\end{pgfscope}%
\begin{pgfscope}%
\pgfsys@transformshift{2.402250in}{2.246086in}%
\pgfsys@useobject{currentmarker}{}%
\end{pgfscope}%
\begin{pgfscope}%
\pgfsys@transformshift{2.380656in}{2.247370in}%
\pgfsys@useobject{currentmarker}{}%
\end{pgfscope}%
\begin{pgfscope}%
\pgfsys@transformshift{2.361172in}{2.259593in}%
\pgfsys@useobject{currentmarker}{}%
\end{pgfscope}%
\begin{pgfscope}%
\pgfsys@transformshift{2.342395in}{2.292608in}%
\pgfsys@useobject{currentmarker}{}%
\end{pgfscope}%
\begin{pgfscope}%
\pgfsys@transformshift{2.317748in}{2.400265in}%
\pgfsys@useobject{currentmarker}{}%
\end{pgfscope}%
\begin{pgfscope}%
\pgfsys@transformshift{2.303664in}{2.478349in}%
\pgfsys@useobject{currentmarker}{}%
\end{pgfscope}%
\begin{pgfscope}%
\pgfsys@transformshift{2.283713in}{2.591012in}%
\pgfsys@useobject{currentmarker}{}%
\end{pgfscope}%
\begin{pgfscope}%
\pgfsys@transformshift{2.266342in}{2.613826in}%
\pgfsys@useobject{currentmarker}{}%
\end{pgfscope}%
\begin{pgfscope}%
\pgfsys@transformshift{2.246389in}{2.540195in}%
\pgfsys@useobject{currentmarker}{}%
\end{pgfscope}%
\begin{pgfscope}%
\pgfsys@transformshift{2.224561in}{2.402451in}%
\pgfsys@useobject{currentmarker}{}%
\end{pgfscope}%
\begin{pgfscope}%
\pgfsys@transformshift{2.206486in}{2.349860in}%
\pgfsys@useobject{currentmarker}{}%
\end{pgfscope}%
\begin{pgfscope}%
\pgfsys@transformshift{2.188178in}{2.606926in}%
\pgfsys@useobject{currentmarker}{}%
\end{pgfscope}%
\begin{pgfscope}%
\pgfsys@transformshift{2.169633in}{2.500188in}%
\pgfsys@useobject{currentmarker}{}%
\end{pgfscope}%
\begin{pgfscope}%
\pgfsys@transformshift{2.146396in}{2.344306in}%
\pgfsys@useobject{currentmarker}{}%
\end{pgfscope}%
\begin{pgfscope}%
\pgfsys@transformshift{2.127852in}{2.285686in}%
\pgfsys@useobject{currentmarker}{}%
\end{pgfscope}%
\begin{pgfscope}%
\pgfsys@transformshift{2.112125in}{2.263544in}%
\pgfsys@useobject{currentmarker}{}%
\end{pgfscope}%
\begin{pgfscope}%
\pgfsys@transformshift{2.091000in}{2.247214in}%
\pgfsys@useobject{currentmarker}{}%
\end{pgfscope}%
\begin{pgfscope}%
\pgfsys@transformshift{2.071986in}{2.246113in}%
\pgfsys@useobject{currentmarker}{}%
\end{pgfscope}%
\begin{pgfscope}%
\pgfsys@transformshift{2.054616in}{2.257889in}%
\pgfsys@useobject{currentmarker}{}%
\end{pgfscope}%
\begin{pgfscope}%
\pgfsys@transformshift{2.035839in}{2.286722in}%
\pgfsys@useobject{currentmarker}{}%
\end{pgfscope}%
\begin{pgfscope}%
\pgfsys@transformshift{2.015652in}{2.350725in}%
\pgfsys@useobject{currentmarker}{}%
\end{pgfscope}%
\begin{pgfscope}%
\pgfsys@transformshift{1.995464in}{2.509824in}%
\pgfsys@useobject{currentmarker}{}%
\end{pgfscope}%
\begin{pgfscope}%
\pgfsys@transformshift{1.974574in}{2.609077in}%
\pgfsys@useobject{currentmarker}{}%
\end{pgfscope}%
\begin{pgfscope}%
\pgfsys@transformshift{1.956969in}{2.561024in}%
\pgfsys@useobject{currentmarker}{}%
\end{pgfscope}%
\begin{pgfscope}%
\pgfsys@transformshift{1.938895in}{2.617157in}%
\pgfsys@useobject{currentmarker}{}%
\end{pgfscope}%
\begin{pgfscope}%
\pgfsys@transformshift{1.917065in}{2.579494in}%
\pgfsys@useobject{currentmarker}{}%
\end{pgfscope}%
\begin{pgfscope}%
\pgfsys@transformshift{1.902512in}{2.477373in}%
\pgfsys@useobject{currentmarker}{}%
\end{pgfscope}%
\begin{pgfscope}%
\pgfsys@transformshift{1.879273in}{2.333494in}%
\pgfsys@useobject{currentmarker}{}%
\end{pgfscope}%
\begin{pgfscope}%
\pgfsys@transformshift{1.862373in}{2.279779in}%
\pgfsys@useobject{currentmarker}{}%
\end{pgfscope}%
\begin{pgfscope}%
\pgfsys@transformshift{1.843360in}{2.254387in}%
\pgfsys@useobject{currentmarker}{}%
\end{pgfscope}%
\begin{pgfscope}%
\pgfsys@transformshift{1.821061in}{2.246610in}%
\pgfsys@useobject{currentmarker}{}%
\end{pgfscope}%
\begin{pgfscope}%
\pgfsys@transformshift{1.802517in}{2.256597in}%
\pgfsys@useobject{currentmarker}{}%
\end{pgfscope}%
\begin{pgfscope}%
\pgfsys@transformshift{1.784443in}{2.283013in}%
\pgfsys@useobject{currentmarker}{}%
\end{pgfscope}%
\begin{pgfscope}%
\pgfsys@transformshift{1.764256in}{2.315532in}%
\pgfsys@useobject{currentmarker}{}%
\end{pgfscope}%
\begin{pgfscope}%
\pgfsys@transformshift{1.745713in}{2.401205in}%
\pgfsys@useobject{currentmarker}{}%
\end{pgfscope}%
\begin{pgfscope}%
\pgfsys@transformshift{1.724821in}{2.487132in}%
\pgfsys@useobject{currentmarker}{}%
\end{pgfscope}%
\begin{pgfscope}%
\pgfsys@transformshift{1.708625in}{2.608822in}%
\pgfsys@useobject{currentmarker}{}%
\end{pgfscope}%
\begin{pgfscope}%
\pgfsys@transformshift{1.688674in}{2.612158in}%
\pgfsys@useobject{currentmarker}{}%
\end{pgfscope}%
\begin{pgfscope}%
\pgfsys@transformshift{1.669660in}{2.531252in}%
\pgfsys@useobject{currentmarker}{}%
\end{pgfscope}%
\begin{pgfscope}%
\pgfsys@transformshift{1.650413in}{2.402284in}%
\pgfsys@useobject{currentmarker}{}%
\end{pgfscope}%
\begin{pgfscope}%
\pgfsys@transformshift{1.628817in}{2.322263in}%
\pgfsys@useobject{currentmarker}{}%
\end{pgfscope}%
\begin{pgfscope}%
\pgfsys@transformshift{1.612152in}{2.279818in}%
\pgfsys@useobject{currentmarker}{}%
\end{pgfscope}%
\begin{pgfscope}%
\pgfsys@transformshift{1.591964in}{2.253209in}%
\pgfsys@useobject{currentmarker}{}%
\end{pgfscope}%
\begin{pgfscope}%
\pgfsys@transformshift{1.575065in}{2.246813in}%
\pgfsys@useobject{currentmarker}{}%
\end{pgfscope}%
\begin{pgfscope}%
\pgfsys@transformshift{1.553469in}{2.261034in}%
\pgfsys@useobject{currentmarker}{}%
\end{pgfscope}%
\begin{pgfscope}%
\pgfsys@transformshift{1.533753in}{2.282644in}%
\pgfsys@useobject{currentmarker}{}%
\end{pgfscope}%
\begin{pgfscope}%
\pgfsys@transformshift{1.515677in}{2.329676in}%
\pgfsys@useobject{currentmarker}{}%
\end{pgfscope}%
\begin{pgfscope}%
\pgfsys@transformshift{1.497369in}{2.403599in}%
\pgfsys@useobject{currentmarker}{}%
\end{pgfscope}%
\begin{pgfscope}%
\pgfsys@transformshift{1.477652in}{2.527745in}%
\pgfsys@useobject{currentmarker}{}%
\end{pgfscope}%
\begin{pgfscope}%
\pgfsys@transformshift{1.458405in}{2.618044in}%
\pgfsys@useobject{currentmarker}{}%
\end{pgfscope}%
\begin{pgfscope}%
\pgfsys@transformshift{1.437043in}{2.628091in}%
\pgfsys@useobject{currentmarker}{}%
\end{pgfscope}%
\begin{pgfscope}%
\pgfsys@transformshift{1.419673in}{2.572635in}%
\pgfsys@useobject{currentmarker}{}%
\end{pgfscope}%
\begin{pgfscope}%
\pgfsys@transformshift{1.399488in}{2.436858in}%
\pgfsys@useobject{currentmarker}{}%
\end{pgfscope}%
\begin{pgfscope}%
\pgfsys@transformshift{1.380474in}{2.349280in}%
\pgfsys@useobject{currentmarker}{}%
\end{pgfscope}%
\begin{pgfscope}%
\pgfsys@transformshift{1.362635in}{2.298192in}%
\pgfsys@useobject{currentmarker}{}%
\end{pgfscope}%
\begin{pgfscope}%
\pgfsys@transformshift{1.341743in}{2.271049in}%
\pgfsys@useobject{currentmarker}{}%
\end{pgfscope}%
\begin{pgfscope}%
\pgfsys@transformshift{1.323669in}{2.257101in}%
\pgfsys@useobject{currentmarker}{}%
\end{pgfscope}%
\begin{pgfscope}%
\pgfsys@transformshift{1.302544in}{2.248465in}%
\pgfsys@useobject{currentmarker}{}%
\end{pgfscope}%
\begin{pgfscope}%
\pgfsys@transformshift{1.285642in}{2.257236in}%
\pgfsys@useobject{currentmarker}{}%
\end{pgfscope}%
\begin{pgfscope}%
\pgfsys@transformshift{1.264752in}{2.289177in}%
\pgfsys@useobject{currentmarker}{}%
\end{pgfscope}%
\begin{pgfscope}%
\pgfsys@transformshift{1.244330in}{2.348826in}%
\pgfsys@useobject{currentmarker}{}%
\end{pgfscope}%
\begin{pgfscope}%
\pgfsys@transformshift{1.226257in}{2.437079in}%
\pgfsys@useobject{currentmarker}{}%
\end{pgfscope}%
\begin{pgfscope}%
\pgfsys@transformshift{1.205600in}{2.537680in}%
\pgfsys@useobject{currentmarker}{}%
\end{pgfscope}%
\begin{pgfscope}%
\pgfsys@transformshift{1.190344in}{2.611793in}%
\pgfsys@useobject{currentmarker}{}%
\end{pgfscope}%
\begin{pgfscope}%
\pgfsys@transformshift{1.168043in}{2.639998in}%
\pgfsys@useobject{currentmarker}{}%
\end{pgfscope}%
\begin{pgfscope}%
\pgfsys@transformshift{1.148326in}{2.612238in}%
\pgfsys@useobject{currentmarker}{}%
\end{pgfscope}%
\begin{pgfscope}%
\pgfsys@transformshift{1.129079in}{2.511838in}%
\pgfsys@useobject{currentmarker}{}%
\end{pgfscope}%
\begin{pgfscope}%
\pgfsys@transformshift{1.109362in}{2.429264in}%
\pgfsys@useobject{currentmarker}{}%
\end{pgfscope}%
\begin{pgfscope}%
\pgfsys@transformshift{1.093166in}{2.345847in}%
\pgfsys@useobject{currentmarker}{}%
\end{pgfscope}%
\begin{pgfscope}%
\pgfsys@transformshift{1.073447in}{2.295079in}%
\pgfsys@useobject{currentmarker}{}%
\end{pgfscope}%
\begin{pgfscope}%
\pgfsys@transformshift{1.053025in}{2.265585in}%
\pgfsys@useobject{currentmarker}{}%
\end{pgfscope}%
\begin{pgfscope}%
\pgfsys@transformshift{1.035186in}{2.251037in}%
\pgfsys@useobject{currentmarker}{}%
\end{pgfscope}%
\begin{pgfscope}%
\pgfsys@transformshift{1.017113in}{2.312565in}%
\pgfsys@useobject{currentmarker}{}%
\end{pgfscope}%
\begin{pgfscope}%
\pgfsys@transformshift{0.993640in}{2.270980in}%
\pgfsys@useobject{currentmarker}{}%
\end{pgfscope}%
\begin{pgfscope}%
\pgfsys@transformshift{0.978148in}{2.254700in}%
\pgfsys@useobject{currentmarker}{}%
\end{pgfscope}%
\begin{pgfscope}%
\pgfsys@transformshift{0.957727in}{2.254216in}%
\pgfsys@useobject{currentmarker}{}%
\end{pgfscope}%
\begin{pgfscope}%
\pgfsys@transformshift{0.937071in}{2.274084in}%
\pgfsys@useobject{currentmarker}{}%
\end{pgfscope}%
\begin{pgfscope}%
\pgfsys@transformshift{0.920403in}{2.307293in}%
\pgfsys@useobject{currentmarker}{}%
\end{pgfscope}%
\begin{pgfscope}%
\pgfsys@transformshift{0.899982in}{2.392954in}%
\pgfsys@useobject{currentmarker}{}%
\end{pgfscope}%
\begin{pgfscope}%
\pgfsys@transformshift{0.880970in}{2.512718in}%
\pgfsys@useobject{currentmarker}{}%
\end{pgfscope}%
\begin{pgfscope}%
\pgfsys@transformshift{0.860314in}{2.618028in}%
\pgfsys@useobject{currentmarker}{}%
\end{pgfscope}%
\begin{pgfscope}%
\pgfsys@transformshift{0.843413in}{2.655572in}%
\pgfsys@useobject{currentmarker}{}%
\end{pgfscope}%
\begin{pgfscope}%
\pgfsys@transformshift{0.825573in}{2.623857in}%
\pgfsys@useobject{currentmarker}{}%
\end{pgfscope}%
\begin{pgfscope}%
\pgfsys@transformshift{0.804917in}{2.537771in}%
\pgfsys@useobject{currentmarker}{}%
\end{pgfscope}%
\begin{pgfscope}%
\pgfsys@transformshift{0.786844in}{2.427557in}%
\pgfsys@useobject{currentmarker}{}%
\end{pgfscope}%
\begin{pgfscope}%
\pgfsys@transformshift{0.764779in}{2.347835in}%
\pgfsys@useobject{currentmarker}{}%
\end{pgfscope}%
\begin{pgfscope}%
\pgfsys@transformshift{0.746469in}{2.300497in}%
\pgfsys@useobject{currentmarker}{}%
\end{pgfscope}%
\begin{pgfscope}%
\pgfsys@transformshift{0.726518in}{2.271279in}%
\pgfsys@useobject{currentmarker}{}%
\end{pgfscope}%
\begin{pgfscope}%
\pgfsys@transformshift{0.708679in}{2.254234in}%
\pgfsys@useobject{currentmarker}{}%
\end{pgfscope}%
\begin{pgfscope}%
\pgfsys@transformshift{0.690603in}{2.258893in}%
\pgfsys@useobject{currentmarker}{}%
\end{pgfscope}%
\begin{pgfscope}%
\pgfsys@transformshift{0.669244in}{2.288571in}%
\pgfsys@useobject{currentmarker}{}%
\end{pgfscope}%
\begin{pgfscope}%
\pgfsys@transformshift{0.651405in}{2.326860in}%
\pgfsys@useobject{currentmarker}{}%
\end{pgfscope}%
\begin{pgfscope}%
\pgfsys@transformshift{0.651639in}{2.324712in}%
\pgfsys@useobject{currentmarker}{}%
\end{pgfscope}%
\begin{pgfscope}%
\pgfsys@transformshift{0.658447in}{2.298425in}%
\pgfsys@useobject{currentmarker}{}%
\end{pgfscope}%
\begin{pgfscope}%
\pgfsys@transformshift{0.673704in}{2.268129in}%
\pgfsys@useobject{currentmarker}{}%
\end{pgfscope}%
\begin{pgfscope}%
\pgfsys@transformshift{0.694829in}{2.255064in}%
\pgfsys@useobject{currentmarker}{}%
\end{pgfscope}%
\begin{pgfscope}%
\pgfsys@transformshift{0.713373in}{2.281890in}%
\pgfsys@useobject{currentmarker}{}%
\end{pgfscope}%
\begin{pgfscope}%
\pgfsys@transformshift{0.734029in}{2.352140in}%
\pgfsys@useobject{currentmarker}{}%
\end{pgfscope}%
\begin{pgfscope}%
\pgfsys@transformshift{0.751869in}{2.478482in}%
\pgfsys@useobject{currentmarker}{}%
\end{pgfscope}%
\begin{pgfscope}%
\pgfsys@transformshift{0.776516in}{2.649386in}%
\pgfsys@useobject{currentmarker}{}%
\end{pgfscope}%
\begin{pgfscope}%
\pgfsys@transformshift{0.791538in}{2.641175in}%
\pgfsys@useobject{currentmarker}{}%
\end{pgfscope}%
\begin{pgfscope}%
\pgfsys@transformshift{0.809143in}{2.518628in}%
\pgfsys@useobject{currentmarker}{}%
\end{pgfscope}%
\begin{pgfscope}%
\pgfsys@transformshift{0.829799in}{2.355201in}%
\pgfsys@useobject{currentmarker}{}%
\end{pgfscope}%
\begin{pgfscope}%
\pgfsys@transformshift{0.847638in}{2.286682in}%
\pgfsys@useobject{currentmarker}{}%
\end{pgfscope}%
\begin{pgfscope}%
\pgfsys@transformshift{0.865712in}{2.254873in}%
\pgfsys@useobject{currentmarker}{}%
\end{pgfscope}%
\begin{pgfscope}%
\pgfsys@transformshift{0.888011in}{2.261063in}%
\pgfsys@useobject{currentmarker}{}%
\end{pgfscope}%
\begin{pgfscope}%
\pgfsys@transformshift{0.906790in}{2.293405in}%
\pgfsys@useobject{currentmarker}{}%
\end{pgfscope}%
\begin{pgfscope}%
\pgfsys@transformshift{0.923691in}{2.368368in}%
\pgfsys@useobject{currentmarker}{}%
\end{pgfscope}%
\begin{pgfscope}%
\pgfsys@transformshift{0.944111in}{2.521064in}%
\pgfsys@useobject{currentmarker}{}%
\end{pgfscope}%
\begin{pgfscope}%
\pgfsys@transformshift{0.965941in}{2.643769in}%
\pgfsys@useobject{currentmarker}{}%
\end{pgfscope}%
\begin{pgfscope}%
\pgfsys@transformshift{0.981200in}{2.611431in}%
\pgfsys@useobject{currentmarker}{}%
\end{pgfscope}%
\begin{pgfscope}%
\pgfsys@transformshift{1.001856in}{2.451561in}%
\pgfsys@useobject{currentmarker}{}%
\end{pgfscope}%
\begin{pgfscope}%
\pgfsys@transformshift{1.021572in}{2.315952in}%
\pgfsys@useobject{currentmarker}{}%
\end{pgfscope}%
\begin{pgfscope}%
\pgfsys@transformshift{1.039882in}{2.268422in}%
\pgfsys@useobject{currentmarker}{}%
\end{pgfscope}%
\begin{pgfscope}%
\pgfsys@transformshift{1.061242in}{2.249216in}%
\pgfsys@useobject{currentmarker}{}%
\end{pgfscope}%
\begin{pgfscope}%
\pgfsys@transformshift{1.081195in}{2.269915in}%
\pgfsys@useobject{currentmarker}{}%
\end{pgfscope}%
\begin{pgfscope}%
\pgfsys@transformshift{1.099503in}{2.316635in}%
\pgfsys@useobject{currentmarker}{}%
\end{pgfscope}%
\begin{pgfscope}%
\pgfsys@transformshift{1.116402in}{2.412318in}%
\pgfsys@useobject{currentmarker}{}%
\end{pgfscope}%
\begin{pgfscope}%
\pgfsys@transformshift{1.137295in}{2.584520in}%
\pgfsys@useobject{currentmarker}{}%
\end{pgfscope}%
\begin{pgfscope}%
\pgfsys@transformshift{1.157717in}{2.633081in}%
\pgfsys@useobject{currentmarker}{}%
\end{pgfscope}%
\begin{pgfscope}%
\pgfsys@transformshift{1.175319in}{2.547703in}%
\pgfsys@useobject{currentmarker}{}%
\end{pgfscope}%
\begin{pgfscope}%
\pgfsys@transformshift{1.195038in}{2.384104in}%
\pgfsys@useobject{currentmarker}{}%
\end{pgfscope}%
\begin{pgfscope}%
\pgfsys@transformshift{1.214754in}{2.298924in}%
\pgfsys@useobject{currentmarker}{}%
\end{pgfscope}%
\begin{pgfscope}%
\pgfsys@transformshift{1.231422in}{2.259216in}%
\pgfsys@useobject{currentmarker}{}%
\end{pgfscope}%
\begin{pgfscope}%
\pgfsys@transformshift{1.252078in}{2.247819in}%
\pgfsys@useobject{currentmarker}{}%
\end{pgfscope}%
\begin{pgfscope}%
\pgfsys@transformshift{1.270620in}{2.266931in}%
\pgfsys@useobject{currentmarker}{}%
\end{pgfscope}%
\begin{pgfscope}%
\pgfsys@transformshift{1.289868in}{2.314754in}%
\pgfsys@useobject{currentmarker}{}%
\end{pgfscope}%
\begin{pgfscope}%
\pgfsys@transformshift{1.311464in}{2.415135in}%
\pgfsys@useobject{currentmarker}{}%
\end{pgfscope}%
\begin{pgfscope}%
\pgfsys@transformshift{1.329303in}{2.575984in}%
\pgfsys@useobject{currentmarker}{}%
\end{pgfscope}%
\begin{pgfscope}%
\pgfsys@transformshift{1.349959in}{2.628202in}%
\pgfsys@useobject{currentmarker}{}%
\end{pgfscope}%
\begin{pgfscope}%
\pgfsys@transformshift{1.366624in}{2.540276in}%
\pgfsys@useobject{currentmarker}{}%
\end{pgfscope}%
\begin{pgfscope}%
\pgfsys@transformshift{1.384934in}{2.413793in}%
\pgfsys@useobject{currentmarker}{}%
\end{pgfscope}%
\begin{pgfscope}%
\pgfsys@transformshift{1.406764in}{2.310397in}%
\pgfsys@useobject{currentmarker}{}%
\end{pgfscope}%
\begin{pgfscope}%
\pgfsys@transformshift{1.426012in}{2.264702in}%
\pgfsys@useobject{currentmarker}{}%
\end{pgfscope}%
\begin{pgfscope}%
\pgfsys@transformshift{1.442912in}{2.247495in}%
\pgfsys@useobject{currentmarker}{}%
\end{pgfscope}%
\begin{pgfscope}%
\pgfsys@transformshift{1.462394in}{2.251465in}%
\pgfsys@useobject{currentmarker}{}%
\end{pgfscope}%
\begin{pgfscope}%
\pgfsys@transformshift{1.483755in}{2.280876in}%
\pgfsys@useobject{currentmarker}{}%
\end{pgfscope}%
\begin{pgfscope}%
\pgfsys@transformshift{1.502298in}{2.296379in}%
\pgfsys@useobject{currentmarker}{}%
\end{pgfscope}%
\begin{pgfscope}%
\pgfsys@transformshift{1.519433in}{2.349065in}%
\pgfsys@useobject{currentmarker}{}%
\end{pgfscope}%
\begin{pgfscope}%
\pgfsys@transformshift{1.541733in}{2.498075in}%
\pgfsys@useobject{currentmarker}{}%
\end{pgfscope}%
\begin{pgfscope}%
\pgfsys@transformshift{1.558868in}{2.613370in}%
\pgfsys@useobject{currentmarker}{}%
\end{pgfscope}%
\begin{pgfscope}%
\pgfsys@transformshift{1.580933in}{2.592403in}%
\pgfsys@useobject{currentmarker}{}%
\end{pgfscope}%
\begin{pgfscope}%
\pgfsys@transformshift{1.598067in}{2.453413in}%
\pgfsys@useobject{currentmarker}{}%
\end{pgfscope}%
\begin{pgfscope}%
\pgfsys@transformshift{1.615437in}{2.336182in}%
\pgfsys@useobject{currentmarker}{}%
\end{pgfscope}%
\begin{pgfscope}%
\pgfsys@transformshift{1.636094in}{2.273644in}%
\pgfsys@useobject{currentmarker}{}%
\end{pgfscope}%
\begin{pgfscope}%
\pgfsys@transformshift{1.654404in}{2.249449in}%
\pgfsys@useobject{currentmarker}{}%
\end{pgfscope}%
\begin{pgfscope}%
\pgfsys@transformshift{1.675060in}{2.249762in}%
\pgfsys@useobject{currentmarker}{}%
\end{pgfscope}%
\begin{pgfscope}%
\pgfsys@transformshift{1.693133in}{2.271312in}%
\pgfsys@useobject{currentmarker}{}%
\end{pgfscope}%
\begin{pgfscope}%
\pgfsys@transformshift{1.714258in}{2.325074in}%
\pgfsys@useobject{currentmarker}{}%
\end{pgfscope}%
\begin{pgfscope}%
\pgfsys@transformshift{1.732098in}{2.412280in}%
\pgfsys@useobject{currentmarker}{}%
\end{pgfscope}%
\begin{pgfscope}%
\pgfsys@transformshift{1.752754in}{2.580978in}%
\pgfsys@useobject{currentmarker}{}%
\end{pgfscope}%
\begin{pgfscope}%
\pgfsys@transformshift{1.773175in}{2.617616in}%
\pgfsys@useobject{currentmarker}{}%
\end{pgfscope}%
\begin{pgfscope}%
\pgfsys@transformshift{1.790780in}{2.540573in}%
\pgfsys@useobject{currentmarker}{}%
\end{pgfscope}%
\begin{pgfscope}%
\pgfsys@transformshift{1.808619in}{2.401372in}%
\pgfsys@useobject{currentmarker}{}%
\end{pgfscope}%
\begin{pgfscope}%
\pgfsys@transformshift{1.827633in}{2.337153in}%
\pgfsys@useobject{currentmarker}{}%
\end{pgfscope}%
\begin{pgfscope}%
\pgfsys@transformshift{1.849463in}{2.273703in}%
\pgfsys@useobject{currentmarker}{}%
\end{pgfscope}%
\begin{pgfscope}%
\pgfsys@transformshift{1.867537in}{2.253613in}%
\pgfsys@useobject{currentmarker}{}%
\end{pgfscope}%
\begin{pgfscope}%
\pgfsys@transformshift{1.886081in}{2.245292in}%
\pgfsys@useobject{currentmarker}{}%
\end{pgfscope}%
\begin{pgfscope}%
\pgfsys@transformshift{1.906737in}{2.263320in}%
\pgfsys@useobject{currentmarker}{}%
\end{pgfscope}%
\begin{pgfscope}%
\pgfsys@transformshift{1.923402in}{2.295045in}%
\pgfsys@useobject{currentmarker}{}%
\end{pgfscope}%
\begin{pgfscope}%
\pgfsys@transformshift{1.944998in}{2.384149in}%
\pgfsys@useobject{currentmarker}{}%
\end{pgfscope}%
\begin{pgfscope}%
\pgfsys@transformshift{1.962603in}{2.462133in}%
\pgfsys@useobject{currentmarker}{}%
\end{pgfscope}%
\begin{pgfscope}%
\pgfsys@transformshift{1.983493in}{2.578096in}%
\pgfsys@useobject{currentmarker}{}%
\end{pgfscope}%
\begin{pgfscope}%
\pgfsys@transformshift{2.001567in}{2.614933in}%
\pgfsys@useobject{currentmarker}{}%
\end{pgfscope}%
\begin{pgfscope}%
\pgfsys@transformshift{2.020111in}{2.578691in}%
\pgfsys@useobject{currentmarker}{}%
\end{pgfscope}%
\begin{pgfscope}%
\pgfsys@transformshift{2.038654in}{2.439904in}%
\pgfsys@useobject{currentmarker}{}%
\end{pgfscope}%
\begin{pgfscope}%
\pgfsys@transformshift{2.059546in}{2.346844in}%
\pgfsys@useobject{currentmarker}{}%
\end{pgfscope}%
\begin{pgfscope}%
\pgfsys@transformshift{2.080906in}{2.282075in}%
\pgfsys@useobject{currentmarker}{}%
\end{pgfscope}%
\begin{pgfscope}%
\pgfsys@transformshift{2.095225in}{2.257067in}%
\pgfsys@useobject{currentmarker}{}%
\end{pgfscope}%
\begin{pgfscope}%
\pgfsys@transformshift{2.117524in}{2.245429in}%
\pgfsys@useobject{currentmarker}{}%
\end{pgfscope}%
\begin{pgfscope}%
\pgfsys@transformshift{2.136068in}{2.255948in}%
\pgfsys@useobject{currentmarker}{}%
\end{pgfscope}%
\begin{pgfscope}%
\pgfsys@transformshift{2.156019in}{2.281504in}%
\pgfsys@useobject{currentmarker}{}%
\end{pgfscope}%
\begin{pgfscope}%
\pgfsys@transformshift{2.173624in}{2.335448in}%
\pgfsys@useobject{currentmarker}{}%
\end{pgfscope}%
\begin{pgfscope}%
\pgfsys@transformshift{2.193577in}{2.426065in}%
\pgfsys@useobject{currentmarker}{}%
\end{pgfscope}%
\begin{pgfscope}%
\pgfsys@transformshift{2.214233in}{2.573121in}%
\pgfsys@useobject{currentmarker}{}%
\end{pgfscope}%
\begin{pgfscope}%
\pgfsys@transformshift{2.232541in}{2.603956in}%
\pgfsys@useobject{currentmarker}{}%
\end{pgfscope}%
\begin{pgfscope}%
\pgfsys@transformshift{2.252023in}{2.598973in}%
\pgfsys@useobject{currentmarker}{}%
\end{pgfscope}%
\begin{pgfscope}%
\pgfsys@transformshift{2.270802in}{2.492131in}%
\pgfsys@useobject{currentmarker}{}%
\end{pgfscope}%
\begin{pgfscope}%
\pgfsys@transformshift{2.290989in}{2.341388in}%
\pgfsys@useobject{currentmarker}{}%
\end{pgfscope}%
\begin{pgfscope}%
\pgfsys@transformshift{2.306012in}{2.287067in}%
\pgfsys@useobject{currentmarker}{}%
\end{pgfscope}%
\begin{pgfscope}%
\pgfsys@transformshift{2.327371in}{2.261348in}%
\pgfsys@useobject{currentmarker}{}%
\end{pgfscope}%
\begin{pgfscope}%
\pgfsys@transformshift{2.348027in}{2.247596in}%
\pgfsys@useobject{currentmarker}{}%
\end{pgfscope}%
\begin{pgfscope}%
\pgfsys@transformshift{2.370328in}{2.247882in}%
\pgfsys@useobject{currentmarker}{}%
\end{pgfscope}%
\begin{pgfscope}%
\pgfsys@transformshift{2.384880in}{2.261097in}%
\pgfsys@useobject{currentmarker}{}%
\end{pgfscope}%
\begin{pgfscope}%
\pgfsys@transformshift{2.406241in}{2.300579in}%
\pgfsys@useobject{currentmarker}{}%
\end{pgfscope}%
\begin{pgfscope}%
\pgfsys@transformshift{2.423377in}{2.371787in}%
\pgfsys@useobject{currentmarker}{}%
\end{pgfscope}%
\begin{pgfscope}%
\pgfsys@transformshift{2.444502in}{2.526212in}%
\pgfsys@useobject{currentmarker}{}%
\end{pgfscope}%
\begin{pgfscope}%
\pgfsys@transformshift{2.461872in}{2.595982in}%
\pgfsys@useobject{currentmarker}{}%
\end{pgfscope}%
\begin{pgfscope}%
\pgfsys@transformshift{2.480649in}{2.608092in}%
\pgfsys@useobject{currentmarker}{}%
\end{pgfscope}%
\begin{pgfscope}%
\pgfsys@transformshift{2.500133in}{2.545281in}%
\pgfsys@useobject{currentmarker}{}%
\end{pgfscope}%
\begin{pgfscope}%
\pgfsys@transformshift{2.520555in}{2.401075in}%
\pgfsys@useobject{currentmarker}{}%
\end{pgfscope}%
\begin{pgfscope}%
\pgfsys@transformshift{2.540740in}{2.301259in}%
\pgfsys@useobject{currentmarker}{}%
\end{pgfscope}%
\begin{pgfscope}%
\pgfsys@transformshift{2.558580in}{2.267124in}%
\pgfsys@useobject{currentmarker}{}%
\end{pgfscope}%
\begin{pgfscope}%
\pgfsys@transformshift{2.577827in}{2.256893in}%
\pgfsys@useobject{currentmarker}{}%
\end{pgfscope}%
\begin{pgfscope}%
\pgfsys@transformshift{2.598249in}{2.246138in}%
\pgfsys@useobject{currentmarker}{}%
\end{pgfscope}%
\begin{pgfscope}%
\pgfsys@transformshift{2.615854in}{2.250734in}%
\pgfsys@useobject{currentmarker}{}%
\end{pgfscope}%
\begin{pgfscope}%
\pgfsys@transformshift{2.634867in}{2.268226in}%
\pgfsys@useobject{currentmarker}{}%
\end{pgfscope}%
\begin{pgfscope}%
\pgfsys@transformshift{2.655289in}{2.318164in}%
\pgfsys@useobject{currentmarker}{}%
\end{pgfscope}%
\begin{pgfscope}%
\pgfsys@transformshift{2.673128in}{2.404826in}%
\pgfsys@useobject{currentmarker}{}%
\end{pgfscope}%
\begin{pgfscope}%
\pgfsys@transformshift{2.691438in}{2.309411in}%
\pgfsys@useobject{currentmarker}{}%
\end{pgfscope}%
\begin{pgfscope}%
\pgfsys@transformshift{2.711389in}{2.406155in}%
\pgfsys@useobject{currentmarker}{}%
\end{pgfscope}%
\begin{pgfscope}%
\pgfsys@transformshift{2.730871in}{2.547406in}%
\pgfsys@useobject{currentmarker}{}%
\end{pgfscope}%
\begin{pgfscope}%
\pgfsys@transformshift{2.751293in}{2.612911in}%
\pgfsys@useobject{currentmarker}{}%
\end{pgfscope}%
\begin{pgfscope}%
\pgfsys@transformshift{2.767960in}{2.583768in}%
\pgfsys@useobject{currentmarker}{}%
\end{pgfscope}%
\begin{pgfscope}%
\pgfsys@transformshift{2.787911in}{2.500853in}%
\pgfsys@useobject{currentmarker}{}%
\end{pgfscope}%
\begin{pgfscope}%
\pgfsys@transformshift{2.808801in}{2.357054in}%
\pgfsys@useobject{currentmarker}{}%
\end{pgfscope}%
\begin{pgfscope}%
\pgfsys@transformshift{2.830397in}{2.297812in}%
\pgfsys@useobject{currentmarker}{}%
\end{pgfscope}%
\begin{pgfscope}%
\pgfsys@transformshift{2.848471in}{2.262928in}%
\pgfsys@useobject{currentmarker}{}%
\end{pgfscope}%
\begin{pgfscope}%
\pgfsys@transformshift{2.865841in}{2.247504in}%
\pgfsys@useobject{currentmarker}{}%
\end{pgfscope}%
\begin{pgfscope}%
\pgfsys@transformshift{2.884386in}{2.248929in}%
\pgfsys@useobject{currentmarker}{}%
\end{pgfscope}%
\begin{pgfscope}%
\pgfsys@transformshift{2.905276in}{2.272329in}%
\pgfsys@useobject{currentmarker}{}%
\end{pgfscope}%
\begin{pgfscope}%
\pgfsys@transformshift{2.923819in}{2.313690in}%
\pgfsys@useobject{currentmarker}{}%
\end{pgfscope}%
\begin{pgfscope}%
\pgfsys@transformshift{2.944240in}{2.407589in}%
\pgfsys@useobject{currentmarker}{}%
\end{pgfscope}%
\begin{pgfscope}%
\pgfsys@transformshift{2.963254in}{2.528591in}%
\pgfsys@useobject{currentmarker}{}%
\end{pgfscope}%
\begin{pgfscope}%
\pgfsys@transformshift{2.980624in}{2.608575in}%
\pgfsys@useobject{currentmarker}{}%
\end{pgfscope}%
\begin{pgfscope}%
\pgfsys@transformshift{3.001280in}{2.608770in}%
\pgfsys@useobject{currentmarker}{}%
\end{pgfscope}%
\begin{pgfscope}%
\pgfsys@transformshift{3.018180in}{2.564630in}%
\pgfsys@useobject{currentmarker}{}%
\end{pgfscope}%
\begin{pgfscope}%
\pgfsys@transformshift{3.039776in}{2.428803in}%
\pgfsys@useobject{currentmarker}{}%
\end{pgfscope}%
\begin{pgfscope}%
\pgfsys@transformshift{3.058320in}{2.322825in}%
\pgfsys@useobject{currentmarker}{}%
\end{pgfscope}%
\begin{pgfscope}%
\pgfsys@transformshift{3.077097in}{2.279851in}%
\pgfsys@useobject{currentmarker}{}%
\end{pgfscope}%
\begin{pgfscope}%
\pgfsys@transformshift{3.095172in}{2.255659in}%
\pgfsys@useobject{currentmarker}{}%
\end{pgfscope}%
\begin{pgfscope}%
\pgfsys@transformshift{3.116297in}{2.245994in}%
\pgfsys@useobject{currentmarker}{}%
\end{pgfscope}%
\begin{pgfscope}%
\pgfsys@transformshift{3.136719in}{2.256560in}%
\pgfsys@useobject{currentmarker}{}%
\end{pgfscope}%
\begin{pgfscope}%
\pgfsys@transformshift{3.155732in}{2.284333in}%
\pgfsys@useobject{currentmarker}{}%
\end{pgfscope}%
\begin{pgfscope}%
\pgfsys@transformshift{3.174746in}{2.332299in}%
\pgfsys@useobject{currentmarker}{}%
\end{pgfscope}%
\begin{pgfscope}%
\pgfsys@transformshift{3.194697in}{2.447865in}%
\pgfsys@useobject{currentmarker}{}%
\end{pgfscope}%
\begin{pgfscope}%
\pgfsys@transformshift{3.212770in}{2.522298in}%
\pgfsys@useobject{currentmarker}{}%
\end{pgfscope}%
\begin{pgfscope}%
\pgfsys@transformshift{3.230611in}{2.613956in}%
\pgfsys@useobject{currentmarker}{}%
\end{pgfscope}%
\begin{pgfscope}%
\pgfsys@transformshift{3.251971in}{2.608656in}%
\pgfsys@useobject{currentmarker}{}%
\end{pgfscope}%
\begin{pgfscope}%
\pgfsys@transformshift{3.269810in}{2.529612in}%
\pgfsys@useobject{currentmarker}{}%
\end{pgfscope}%
\begin{pgfscope}%
\pgfsys@transformshift{3.288354in}{2.398294in}%
\pgfsys@useobject{currentmarker}{}%
\end{pgfscope}%
\begin{pgfscope}%
\pgfsys@transformshift{3.309948in}{2.316824in}%
\pgfsys@useobject{currentmarker}{}%
\end{pgfscope}%
\begin{pgfscope}%
\pgfsys@transformshift{3.326850in}{2.273432in}%
\pgfsys@useobject{currentmarker}{}%
\end{pgfscope}%
\begin{pgfscope}%
\pgfsys@transformshift{3.346098in}{2.252186in}%
\pgfsys@useobject{currentmarker}{}%
\end{pgfscope}%
\begin{pgfscope}%
\pgfsys@transformshift{3.367928in}{2.247945in}%
\pgfsys@useobject{currentmarker}{}%
\end{pgfscope}%
\begin{pgfscope}%
\pgfsys@transformshift{3.385298in}{2.262262in}%
\pgfsys@useobject{currentmarker}{}%
\end{pgfscope}%
\begin{pgfscope}%
\pgfsys@transformshift{3.404546in}{2.279559in}%
\pgfsys@useobject{currentmarker}{}%
\end{pgfscope}%
\begin{pgfscope}%
\pgfsys@transformshift{3.422385in}{2.314107in}%
\pgfsys@useobject{currentmarker}{}%
\end{pgfscope}%
\begin{pgfscope}%
\pgfsys@transformshift{3.442101in}{2.388666in}%
\pgfsys@useobject{currentmarker}{}%
\end{pgfscope}%
\begin{pgfscope}%
\pgfsys@transformshift{3.463697in}{2.478753in}%
\pgfsys@useobject{currentmarker}{}%
\end{pgfscope}%
\begin{pgfscope}%
\pgfsys@transformshift{3.479188in}{2.579980in}%
\pgfsys@useobject{currentmarker}{}%
\end{pgfscope}%
\begin{pgfscope}%
\pgfsys@transformshift{3.501019in}{2.631115in}%
\pgfsys@useobject{currentmarker}{}%
\end{pgfscope}%
\begin{pgfscope}%
\pgfsys@transformshift{3.522380in}{2.603179in}%
\pgfsys@useobject{currentmarker}{}%
\end{pgfscope}%
\begin{pgfscope}%
\pgfsys@transformshift{3.521909in}{2.547600in}%
\pgfsys@useobject{currentmarker}{}%
\end{pgfscope}%
\begin{pgfscope}%
\pgfsys@transformshift{3.537637in}{2.515570in}%
\pgfsys@useobject{currentmarker}{}%
\end{pgfscope}%
\begin{pgfscope}%
\pgfsys@transformshift{3.558762in}{2.381459in}%
\pgfsys@useobject{currentmarker}{}%
\end{pgfscope}%
\begin{pgfscope}%
\pgfsys@transformshift{3.578949in}{2.323318in}%
\pgfsys@useobject{currentmarker}{}%
\end{pgfscope}%
\begin{pgfscope}%
\pgfsys@transformshift{3.597728in}{2.278203in}%
\pgfsys@useobject{currentmarker}{}%
\end{pgfscope}%
\begin{pgfscope}%
\pgfsys@transformshift{3.612984in}{2.258779in}%
\pgfsys@useobject{currentmarker}{}%
\end{pgfscope}%
\begin{pgfscope}%
\pgfsys@transformshift{3.635518in}{2.247731in}%
\pgfsys@useobject{currentmarker}{}%
\end{pgfscope}%
\begin{pgfscope}%
\pgfsys@transformshift{3.651480in}{2.255088in}%
\pgfsys@useobject{currentmarker}{}%
\end{pgfscope}%
\begin{pgfscope}%
\pgfsys@transformshift{3.674015in}{2.270503in}%
\pgfsys@useobject{currentmarker}{}%
\end{pgfscope}%
\begin{pgfscope}%
\pgfsys@transformshift{3.693263in}{2.311595in}%
\pgfsys@useobject{currentmarker}{}%
\end{pgfscope}%
\begin{pgfscope}%
\pgfsys@transformshift{3.712511in}{2.355864in}%
\pgfsys@useobject{currentmarker}{}%
\end{pgfscope}%
\begin{pgfscope}%
\pgfsys@transformshift{3.732932in}{2.454131in}%
\pgfsys@useobject{currentmarker}{}%
\end{pgfscope}%
\begin{pgfscope}%
\pgfsys@transformshift{3.751475in}{2.529039in}%
\pgfsys@useobject{currentmarker}{}%
\end{pgfscope}%
\begin{pgfscope}%
\pgfsys@transformshift{3.769785in}{2.623853in}%
\pgfsys@useobject{currentmarker}{}%
\end{pgfscope}%
\begin{pgfscope}%
\pgfsys@transformshift{3.786684in}{2.639035in}%
\pgfsys@useobject{currentmarker}{}%
\end{pgfscope}%
\begin{pgfscope}%
\pgfsys@transformshift{3.809218in}{2.614695in}%
\pgfsys@useobject{currentmarker}{}%
\end{pgfscope}%
\begin{pgfscope}%
\pgfsys@transformshift{3.827057in}{2.499797in}%
\pgfsys@useobject{currentmarker}{}%
\end{pgfscope}%
\begin{pgfscope}%
\pgfsys@transformshift{3.847479in}{2.379900in}%
\pgfsys@useobject{currentmarker}{}%
\end{pgfscope}%
\begin{pgfscope}%
\pgfsys@transformshift{3.866258in}{2.322300in}%
\pgfsys@useobject{currentmarker}{}%
\end{pgfscope}%
\begin{pgfscope}%
\pgfsys@transformshift{3.883862in}{2.308874in}%
\pgfsys@useobject{currentmarker}{}%
\end{pgfscope}%
\begin{pgfscope}%
\pgfsys@transformshift{3.902641in}{2.332377in}%
\pgfsys@useobject{currentmarker}{}%
\end{pgfscope}%
\begin{pgfscope}%
\pgfsys@transformshift{3.922123in}{2.279146in}%
\pgfsys@useobject{currentmarker}{}%
\end{pgfscope}%
\begin{pgfscope}%
\pgfsys@transformshift{3.940902in}{2.257111in}%
\pgfsys@useobject{currentmarker}{}%
\end{pgfscope}%
\begin{pgfscope}%
\pgfsys@transformshift{3.962967in}{2.250360in}%
\pgfsys@useobject{currentmarker}{}%
\end{pgfscope}%
\begin{pgfscope}%
\pgfsys@transformshift{3.986440in}{2.273081in}%
\pgfsys@useobject{currentmarker}{}%
\end{pgfscope}%
\begin{pgfscope}%
\pgfsys@transformshift{3.997706in}{2.295099in}%
\pgfsys@useobject{currentmarker}{}%
\end{pgfscope}%
\begin{pgfscope}%
\pgfsys@transformshift{4.019536in}{2.347193in}%
\pgfsys@useobject{currentmarker}{}%
\end{pgfscope}%
\begin{pgfscope}%
\pgfsys@transformshift{4.039489in}{2.430652in}%
\pgfsys@useobject{currentmarker}{}%
\end{pgfscope}%
\begin{pgfscope}%
\pgfsys@transformshift{4.057562in}{2.560223in}%
\pgfsys@useobject{currentmarker}{}%
\end{pgfscope}%
\begin{pgfscope}%
\pgfsys@transformshift{4.076341in}{2.646089in}%
\pgfsys@useobject{currentmarker}{}%
\end{pgfscope}%
\begin{pgfscope}%
\pgfsys@transformshift{4.095589in}{2.650367in}%
\pgfsys@useobject{currentmarker}{}%
\end{pgfscope}%
\begin{pgfscope}%
\pgfsys@transformshift{4.117653in}{2.583815in}%
\pgfsys@useobject{currentmarker}{}%
\end{pgfscope}%
\begin{pgfscope}%
\pgfsys@transformshift{4.134787in}{2.458858in}%
\pgfsys@useobject{currentmarker}{}%
\end{pgfscope}%
\begin{pgfscope}%
\pgfsys@transformshift{4.154975in}{2.372901in}%
\pgfsys@useobject{currentmarker}{}%
\end{pgfscope}%
\begin{pgfscope}%
\pgfsys@transformshift{4.173048in}{2.332113in}%
\pgfsys@useobject{currentmarker}{}%
\end{pgfscope}%
\begin{pgfscope}%
\pgfsys@transformshift{4.192062in}{2.285802in}%
\pgfsys@useobject{currentmarker}{}%
\end{pgfscope}%
\begin{pgfscope}%
\pgfsys@transformshift{4.210840in}{2.260200in}%
\pgfsys@useobject{currentmarker}{}%
\end{pgfscope}%
\begin{pgfscope}%
\pgfsys@transformshift{4.234079in}{2.251754in}%
\pgfsys@useobject{currentmarker}{}%
\end{pgfscope}%
\begin{pgfscope}%
\pgfsys@transformshift{4.249336in}{2.260534in}%
\pgfsys@useobject{currentmarker}{}%
\end{pgfscope}%
\begin{pgfscope}%
\pgfsys@transformshift{4.267644in}{2.284979in}%
\pgfsys@useobject{currentmarker}{}%
\end{pgfscope}%
\begin{pgfscope}%
\pgfsys@transformshift{4.288536in}{2.323368in}%
\pgfsys@useobject{currentmarker}{}%
\end{pgfscope}%
\begin{pgfscope}%
\pgfsys@transformshift{4.308722in}{2.385292in}%
\pgfsys@useobject{currentmarker}{}%
\end{pgfscope}%
\begin{pgfscope}%
\pgfsys@transformshift{4.327501in}{2.494525in}%
\pgfsys@useobject{currentmarker}{}%
\end{pgfscope}%
\begin{pgfscope}%
\pgfsys@transformshift{4.345574in}{2.538544in}%
\pgfsys@useobject{currentmarker}{}%
\end{pgfscope}%
\begin{pgfscope}%
\pgfsys@transformshift{4.364822in}{2.643772in}%
\pgfsys@useobject{currentmarker}{}%
\end{pgfscope}%
\begin{pgfscope}%
\pgfsys@transformshift{4.383132in}{2.665785in}%
\pgfsys@useobject{currentmarker}{}%
\end{pgfscope}%
\begin{pgfscope}%
\pgfsys@transformshift{4.402614in}{2.627843in}%
\pgfsys@useobject{currentmarker}{}%
\end{pgfscope}%
\begin{pgfscope}%
\pgfsys@transformshift{4.420453in}{2.529293in}%
\pgfsys@useobject{currentmarker}{}%
\end{pgfscope}%
\begin{pgfscope}%
\pgfsys@transformshift{4.439467in}{2.405993in}%
\pgfsys@useobject{currentmarker}{}%
\end{pgfscope}%
\begin{pgfscope}%
\pgfsys@transformshift{4.463174in}{2.314436in}%
\pgfsys@useobject{currentmarker}{}%
\end{pgfscope}%
\begin{pgfscope}%
\pgfsys@transformshift{4.480779in}{2.276957in}%
\pgfsys@useobject{currentmarker}{}%
\end{pgfscope}%
\begin{pgfscope}%
\pgfsys@transformshift{4.479841in}{2.277814in}%
\pgfsys@useobject{currentmarker}{}%
\end{pgfscope}%
\begin{pgfscope}%
\pgfsys@transformshift{4.474207in}{2.291836in}%
\pgfsys@useobject{currentmarker}{}%
\end{pgfscope}%
\begin{pgfscope}%
\pgfsys@transformshift{4.453551in}{2.383294in}%
\pgfsys@useobject{currentmarker}{}%
\end{pgfscope}%
\begin{pgfscope}%
\pgfsys@transformshift{4.435007in}{2.540977in}%
\pgfsys@useobject{currentmarker}{}%
\end{pgfscope}%
\begin{pgfscope}%
\pgfsys@transformshift{4.417168in}{2.655563in}%
\pgfsys@useobject{currentmarker}{}%
\end{pgfscope}%
\begin{pgfscope}%
\pgfsys@transformshift{4.398623in}{2.643812in}%
\pgfsys@useobject{currentmarker}{}%
\end{pgfscope}%
\begin{pgfscope}%
\pgfsys@transformshift{4.377498in}{2.463985in}%
\pgfsys@useobject{currentmarker}{}%
\end{pgfscope}%
\begin{pgfscope}%
\pgfsys@transformshift{4.358016in}{2.341115in}%
\pgfsys@useobject{currentmarker}{}%
\end{pgfscope}%
\begin{pgfscope}%
\pgfsys@transformshift{4.336420in}{2.274359in}%
\pgfsys@useobject{currentmarker}{}%
\end{pgfscope}%
\begin{pgfscope}%
\pgfsys@transformshift{4.336420in}{2.258480in}%
\pgfsys@useobject{currentmarker}{}%
\end{pgfscope}%
\begin{pgfscope}%
\pgfsys@transformshift{4.322572in}{2.253054in}%
\pgfsys@useobject{currentmarker}{}%
\end{pgfscope}%
\begin{pgfscope}%
\pgfsys@transformshift{4.300742in}{2.268810in}%
\pgfsys@useobject{currentmarker}{}%
\end{pgfscope}%
\begin{pgfscope}%
\pgfsys@transformshift{4.280789in}{2.316540in}%
\pgfsys@useobject{currentmarker}{}%
\end{pgfscope}%
\begin{pgfscope}%
\pgfsys@transformshift{4.262012in}{2.451928in}%
\pgfsys@useobject{currentmarker}{}%
\end{pgfscope}%
\begin{pgfscope}%
\pgfsys@transformshift{4.242294in}{2.608566in}%
\pgfsys@useobject{currentmarker}{}%
\end{pgfscope}%
\begin{pgfscope}%
\pgfsys@transformshift{4.223985in}{2.653982in}%
\pgfsys@useobject{currentmarker}{}%
\end{pgfscope}%
\begin{pgfscope}%
\pgfsys@transformshift{4.204738in}{2.556969in}%
\pgfsys@useobject{currentmarker}{}%
\end{pgfscope}%
\begin{pgfscope}%
\pgfsys@transformshift{4.186193in}{2.394791in}%
\pgfsys@useobject{currentmarker}{}%
\end{pgfscope}%
\begin{pgfscope}%
\pgfsys@transformshift{4.167415in}{2.302610in}%
\pgfsys@useobject{currentmarker}{}%
\end{pgfscope}%
\begin{pgfscope}%
\pgfsys@transformshift{4.148638in}{2.261909in}%
\pgfsys@useobject{currentmarker}{}%
\end{pgfscope}%
\begin{pgfscope}%
\pgfsys@transformshift{4.130093in}{2.251161in}%
\pgfsys@useobject{currentmarker}{}%
\end{pgfscope}%
\begin{pgfscope}%
\pgfsys@transformshift{4.106620in}{2.286219in}%
\pgfsys@useobject{currentmarker}{}%
\end{pgfscope}%
\begin{pgfscope}%
\pgfsys@transformshift{4.091598in}{2.346536in}%
\pgfsys@useobject{currentmarker}{}%
\end{pgfscope}%
\begin{pgfscope}%
\pgfsys@transformshift{4.070473in}{2.499261in}%
\pgfsys@useobject{currentmarker}{}%
\end{pgfscope}%
\begin{pgfscope}%
\pgfsys@transformshift{4.051694in}{2.616585in}%
\pgfsys@useobject{currentmarker}{}%
\end{pgfscope}%
\begin{pgfscope}%
\pgfsys@transformshift{4.034324in}{2.447840in}%
\pgfsys@useobject{currentmarker}{}%
\end{pgfscope}%
\begin{pgfscope}%
\pgfsys@transformshift{4.012025in}{2.314028in}%
\pgfsys@useobject{currentmarker}{}%
\end{pgfscope}%
\begin{pgfscope}%
\pgfsys@transformshift{3.994185in}{2.265999in}%
\pgfsys@useobject{currentmarker}{}%
\end{pgfscope}%
\begin{pgfscope}%
\pgfsys@transformshift{3.975875in}{2.248514in}%
\pgfsys@useobject{currentmarker}{}%
\end{pgfscope}%
\begin{pgfscope}%
\pgfsys@transformshift{3.955219in}{2.263339in}%
\pgfsys@useobject{currentmarker}{}%
\end{pgfscope}%
\begin{pgfscope}%
\pgfsys@transformshift{3.936911in}{2.309763in}%
\pgfsys@useobject{currentmarker}{}%
\end{pgfscope}%
\begin{pgfscope}%
\pgfsys@transformshift{3.917898in}{2.413367in}%
\pgfsys@useobject{currentmarker}{}%
\end{pgfscope}%
\begin{pgfscope}%
\pgfsys@transformshift{3.895833in}{2.597136in}%
\pgfsys@useobject{currentmarker}{}%
\end{pgfscope}%
\begin{pgfscope}%
\pgfsys@transformshift{3.877760in}{2.566888in}%
\pgfsys@useobject{currentmarker}{}%
\end{pgfscope}%
\begin{pgfscope}%
\pgfsys@transformshift{3.859215in}{2.635400in}%
\pgfsys@useobject{currentmarker}{}%
\end{pgfscope}%
\begin{pgfscope}%
\pgfsys@transformshift{3.839733in}{2.556019in}%
\pgfsys@useobject{currentmarker}{}%
\end{pgfscope}%
\begin{pgfscope}%
\pgfsys@transformshift{3.822597in}{2.399436in}%
\pgfsys@useobject{currentmarker}{}%
\end{pgfscope}%
\begin{pgfscope}%
\pgfsys@transformshift{3.799361in}{2.299239in}%
\pgfsys@useobject{currentmarker}{}%
\end{pgfscope}%
\begin{pgfscope}%
\pgfsys@transformshift{3.783633in}{2.261683in}%
\pgfsys@useobject{currentmarker}{}%
\end{pgfscope}%
\begin{pgfscope}%
\pgfsys@transformshift{3.762742in}{2.247272in}%
\pgfsys@useobject{currentmarker}{}%
\end{pgfscope}%
\begin{pgfscope}%
\pgfsys@transformshift{3.743024in}{2.265171in}%
\pgfsys@useobject{currentmarker}{}%
\end{pgfscope}%
\begin{pgfscope}%
\pgfsys@transformshift{3.724716in}{2.316243in}%
\pgfsys@useobject{currentmarker}{}%
\end{pgfscope}%
\begin{pgfscope}%
\pgfsys@transformshift{3.705937in}{2.418420in}%
\pgfsys@useobject{currentmarker}{}%
\end{pgfscope}%
\begin{pgfscope}%
\pgfsys@transformshift{3.685750in}{2.566483in}%
\pgfsys@useobject{currentmarker}{}%
\end{pgfscope}%
\begin{pgfscope}%
\pgfsys@transformshift{3.667442in}{2.628920in}%
\pgfsys@useobject{currentmarker}{}%
\end{pgfscope}%
\begin{pgfscope}%
\pgfsys@transformshift{3.648663in}{2.550101in}%
\pgfsys@useobject{currentmarker}{}%
\end{pgfscope}%
\begin{pgfscope}%
\pgfsys@transformshift{3.630589in}{2.409052in}%
\pgfsys@useobject{currentmarker}{}%
\end{pgfscope}%
\begin{pgfscope}%
\pgfsys@transformshift{3.606882in}{2.294972in}%
\pgfsys@useobject{currentmarker}{}%
\end{pgfscope}%
\begin{pgfscope}%
\pgfsys@transformshift{3.590920in}{2.262596in}%
\pgfsys@useobject{currentmarker}{}%
\end{pgfscope}%
\begin{pgfscope}%
\pgfsys@transformshift{3.571907in}{2.247685in}%
\pgfsys@useobject{currentmarker}{}%
\end{pgfscope}%
\begin{pgfscope}%
\pgfsys@transformshift{3.549842in}{2.253741in}%
\pgfsys@useobject{currentmarker}{}%
\end{pgfscope}%
\begin{pgfscope}%
\pgfsys@transformshift{3.528012in}{2.290515in}%
\pgfsys@useobject{currentmarker}{}%
\end{pgfscope}%
\begin{pgfscope}%
\pgfsys@transformshift{3.512990in}{2.344708in}%
\pgfsys@useobject{currentmarker}{}%
\end{pgfscope}%
\begin{pgfscope}%
\pgfsys@transformshift{3.496793in}{2.429087in}%
\pgfsys@useobject{currentmarker}{}%
\end{pgfscope}%
\begin{pgfscope}%
\pgfsys@transformshift{3.475434in}{2.575357in}%
\pgfsys@useobject{currentmarker}{}%
\end{pgfscope}%
\begin{pgfscope}%
\pgfsys@transformshift{3.456655in}{2.620990in}%
\pgfsys@useobject{currentmarker}{}%
\end{pgfscope}%
\begin{pgfscope}%
\pgfsys@transformshift{3.438816in}{2.571390in}%
\pgfsys@useobject{currentmarker}{}%
\end{pgfscope}%
\begin{pgfscope}%
\pgfsys@transformshift{3.415812in}{2.396268in}%
\pgfsys@useobject{currentmarker}{}%
\end{pgfscope}%
\begin{pgfscope}%
\pgfsys@transformshift{3.397503in}{2.316399in}%
\pgfsys@useobject{currentmarker}{}%
\end{pgfscope}%
\begin{pgfscope}%
\pgfsys@transformshift{3.377082in}{2.264950in}%
\pgfsys@useobject{currentmarker}{}%
\end{pgfscope}%
\begin{pgfscope}%
\pgfsys@transformshift{3.359008in}{2.247345in}%
\pgfsys@useobject{currentmarker}{}%
\end{pgfscope}%
\begin{pgfscope}%
\pgfsys@transformshift{3.341638in}{2.249581in}%
\pgfsys@useobject{currentmarker}{}%
\end{pgfscope}%
\begin{pgfscope}%
\pgfsys@transformshift{3.322390in}{2.271872in}%
\pgfsys@useobject{currentmarker}{}%
\end{pgfscope}%
\begin{pgfscope}%
\pgfsys@transformshift{3.301734in}{2.327880in}%
\pgfsys@useobject{currentmarker}{}%
\end{pgfscope}%
\begin{pgfscope}%
\pgfsys@transformshift{3.282015in}{2.387159in}%
\pgfsys@useobject{currentmarker}{}%
\end{pgfscope}%
\begin{pgfscope}%
\pgfsys@transformshift{3.263239in}{2.534684in}%
\pgfsys@useobject{currentmarker}{}%
\end{pgfscope}%
\begin{pgfscope}%
\pgfsys@transformshift{3.241174in}{2.611419in}%
\pgfsys@useobject{currentmarker}{}%
\end{pgfscope}%
\begin{pgfscope}%
\pgfsys@transformshift{3.226386in}{2.601616in}%
\pgfsys@useobject{currentmarker}{}%
\end{pgfscope}%
\begin{pgfscope}%
\pgfsys@transformshift{3.204790in}{2.465905in}%
\pgfsys@useobject{currentmarker}{}%
\end{pgfscope}%
\begin{pgfscope}%
\pgfsys@transformshift{3.186246in}{2.344460in}%
\pgfsys@useobject{currentmarker}{}%
\end{pgfscope}%
\begin{pgfscope}%
\pgfsys@transformshift{3.170520in}{2.286806in}%
\pgfsys@useobject{currentmarker}{}%
\end{pgfscope}%
\begin{pgfscope}%
\pgfsys@transformshift{3.146342in}{2.255042in}%
\pgfsys@useobject{currentmarker}{}%
\end{pgfscope}%
\begin{pgfscope}%
\pgfsys@transformshift{3.128503in}{2.245711in}%
\pgfsys@useobject{currentmarker}{}%
\end{pgfscope}%
\begin{pgfscope}%
\pgfsys@transformshift{3.112306in}{2.250331in}%
\pgfsys@useobject{currentmarker}{}%
\end{pgfscope}%
\begin{pgfscope}%
\pgfsys@transformshift{3.090713in}{2.276237in}%
\pgfsys@useobject{currentmarker}{}%
\end{pgfscope}%
\begin{pgfscope}%
\pgfsys@transformshift{3.068882in}{2.342178in}%
\pgfsys@useobject{currentmarker}{}%
\end{pgfscope}%
\begin{pgfscope}%
\pgfsys@transformshift{3.053624in}{2.418029in}%
\pgfsys@useobject{currentmarker}{}%
\end{pgfscope}%
\begin{pgfscope}%
\pgfsys@transformshift{3.032264in}{2.579090in}%
\pgfsys@useobject{currentmarker}{}%
\end{pgfscope}%
\begin{pgfscope}%
\pgfsys@transformshift{3.031090in}{2.596405in}%
\pgfsys@useobject{currentmarker}{}%
\end{pgfscope}%
\begin{pgfscope}%
\pgfsys@transformshift{3.013720in}{2.612394in}%
\pgfsys@useobject{currentmarker}{}%
\end{pgfscope}%
\begin{pgfscope}%
\pgfsys@transformshift{2.994003in}{2.578736in}%
\pgfsys@useobject{currentmarker}{}%
\end{pgfscope}%
\begin{pgfscope}%
\pgfsys@transformshift{2.975930in}{2.443055in}%
\pgfsys@useobject{currentmarker}{}%
\end{pgfscope}%
\begin{pgfscope}%
\pgfsys@transformshift{2.956446in}{2.358007in}%
\pgfsys@useobject{currentmarker}{}%
\end{pgfscope}%
\begin{pgfscope}%
\pgfsys@transformshift{2.938372in}{2.291526in}%
\pgfsys@useobject{currentmarker}{}%
\end{pgfscope}%
\begin{pgfscope}%
\pgfsys@transformshift{2.918890in}{2.258507in}%
\pgfsys@useobject{currentmarker}{}%
\end{pgfscope}%
\begin{pgfscope}%
\pgfsys@transformshift{2.892366in}{2.245259in}%
\pgfsys@useobject{currentmarker}{}%
\end{pgfscope}%
\begin{pgfscope}%
\pgfsys@transformshift{2.880629in}{2.246087in}%
\pgfsys@useobject{currentmarker}{}%
\end{pgfscope}%
\begin{pgfscope}%
\pgfsys@transformshift{2.859739in}{2.264266in}%
\pgfsys@useobject{currentmarker}{}%
\end{pgfscope}%
\begin{pgfscope}%
\pgfsys@transformshift{2.841665in}{2.303127in}%
\pgfsys@useobject{currentmarker}{}%
\end{pgfscope}%
\begin{pgfscope}%
\pgfsys@transformshift{2.822652in}{2.335539in}%
\pgfsys@useobject{currentmarker}{}%
\end{pgfscope}%
\begin{pgfscope}%
\pgfsys@transformshift{2.801525in}{2.474989in}%
\pgfsys@useobject{currentmarker}{}%
\end{pgfscope}%
\begin{pgfscope}%
\pgfsys@transformshift{2.781574in}{2.596282in}%
\pgfsys@useobject{currentmarker}{}%
\end{pgfscope}%
\begin{pgfscope}%
\pgfsys@transformshift{2.764672in}{2.603613in}%
\pgfsys@useobject{currentmarker}{}%
\end{pgfscope}%
\begin{pgfscope}%
\pgfsys@transformshift{2.762795in}{2.567191in}%
\pgfsys@useobject{currentmarker}{}%
\end{pgfscope}%
\begin{pgfscope}%
\pgfsys@transformshift{2.745425in}{2.520452in}%
\pgfsys@useobject{currentmarker}{}%
\end{pgfscope}%
\begin{pgfscope}%
\pgfsys@transformshift{2.725474in}{2.402794in}%
\pgfsys@useobject{currentmarker}{}%
\end{pgfscope}%
\begin{pgfscope}%
\pgfsys@transformshift{2.705521in}{2.304730in}%
\pgfsys@useobject{currentmarker}{}%
\end{pgfscope}%
\begin{pgfscope}%
\pgfsys@transformshift{2.685570in}{2.367635in}%
\pgfsys@useobject{currentmarker}{}%
\end{pgfscope}%
\begin{pgfscope}%
\pgfsys@transformshift{2.667965in}{2.509385in}%
\pgfsys@useobject{currentmarker}{}%
\end{pgfscope}%
\begin{pgfscope}%
\pgfsys@transformshift{2.648717in}{2.613383in}%
\pgfsys@useobject{currentmarker}{}%
\end{pgfscope}%
\begin{pgfscope}%
\pgfsys@transformshift{2.629938in}{2.574621in}%
\pgfsys@useobject{currentmarker}{}%
\end{pgfscope}%
\begin{pgfscope}%
\pgfsys@transformshift{2.608812in}{2.420560in}%
\pgfsys@useobject{currentmarker}{}%
\end{pgfscope}%
\begin{pgfscope}%
\pgfsys@transformshift{2.590738in}{2.323814in}%
\pgfsys@useobject{currentmarker}{}%
\end{pgfscope}%
\begin{pgfscope}%
\pgfsys@transformshift{2.574307in}{2.276298in}%
\pgfsys@useobject{currentmarker}{}%
\end{pgfscope}%
\begin{pgfscope}%
\pgfsys@transformshift{2.552948in}{2.247774in}%
\pgfsys@useobject{currentmarker}{}%
\end{pgfscope}%
\begin{pgfscope}%
\pgfsys@transformshift{2.533934in}{2.245474in}%
\pgfsys@useobject{currentmarker}{}%
\end{pgfscope}%
\begin{pgfscope}%
\pgfsys@transformshift{2.512808in}{2.265993in}%
\pgfsys@useobject{currentmarker}{}%
\end{pgfscope}%
\begin{pgfscope}%
\pgfsys@transformshift{2.493560in}{2.287432in}%
\pgfsys@useobject{currentmarker}{}%
\end{pgfscope}%
\begin{pgfscope}%
\pgfsys@transformshift{2.493794in}{2.330470in}%
\pgfsys@useobject{currentmarker}{}%
\end{pgfscope}%
\begin{pgfscope}%
\pgfsys@transformshift{2.475721in}{2.363845in}%
\pgfsys@useobject{currentmarker}{}%
\end{pgfscope}%
\begin{pgfscope}%
\pgfsys@transformshift{2.456707in}{2.502656in}%
\pgfsys@useobject{currentmarker}{}%
\end{pgfscope}%
\begin{pgfscope}%
\pgfsys@transformshift{2.438868in}{2.572978in}%
\pgfsys@useobject{currentmarker}{}%
\end{pgfscope}%
\begin{pgfscope}%
\pgfsys@transformshift{2.419621in}{2.612835in}%
\pgfsys@useobject{currentmarker}{}%
\end{pgfscope}%
\begin{pgfscope}%
\pgfsys@transformshift{2.398496in}{2.505697in}%
\pgfsys@useobject{currentmarker}{}%
\end{pgfscope}%
\begin{pgfscope}%
\pgfsys@transformshift{2.379248in}{2.375260in}%
\pgfsys@useobject{currentmarker}{}%
\end{pgfscope}%
\begin{pgfscope}%
\pgfsys@transformshift{2.361409in}{2.316968in}%
\pgfsys@useobject{currentmarker}{}%
\end{pgfscope}%
\begin{pgfscope}%
\pgfsys@transformshift{2.343567in}{2.270113in}%
\pgfsys@useobject{currentmarker}{}%
\end{pgfscope}%
\begin{pgfscope}%
\pgfsys@transformshift{2.320800in}{2.246855in}%
\pgfsys@useobject{currentmarker}{}%
\end{pgfscope}%
\begin{pgfscope}%
\pgfsys@transformshift{2.302492in}{2.248742in}%
\pgfsys@useobject{currentmarker}{}%
\end{pgfscope}%
\begin{pgfscope}%
\pgfsys@transformshift{2.285356in}{2.265273in}%
\pgfsys@useobject{currentmarker}{}%
\end{pgfscope}%
\begin{pgfscope}%
\pgfsys@transformshift{2.264699in}{2.317290in}%
\pgfsys@useobject{currentmarker}{}%
\end{pgfscope}%
\begin{pgfscope}%
\pgfsys@transformshift{2.245921in}{2.424873in}%
\pgfsys@useobject{currentmarker}{}%
\end{pgfscope}%
\begin{pgfscope}%
\pgfsys@transformshift{2.224561in}{2.563803in}%
\pgfsys@useobject{currentmarker}{}%
\end{pgfscope}%
\begin{pgfscope}%
\pgfsys@transformshift{2.225264in}{2.607197in}%
\pgfsys@useobject{currentmarker}{}%
\end{pgfscope}%
\begin{pgfscope}%
\pgfsys@transformshift{2.205782in}{2.615708in}%
\pgfsys@useobject{currentmarker}{}%
\end{pgfscope}%
\begin{pgfscope}%
\pgfsys@transformshift{2.188412in}{2.548566in}%
\pgfsys@useobject{currentmarker}{}%
\end{pgfscope}%
\begin{pgfscope}%
\pgfsys@transformshift{2.169633in}{2.421953in}%
\pgfsys@useobject{currentmarker}{}%
\end{pgfscope}%
\begin{pgfscope}%
\pgfsys@transformshift{2.147805in}{2.328282in}%
\pgfsys@useobject{currentmarker}{}%
\end{pgfscope}%
\begin{pgfscope}%
\pgfsys@transformshift{2.129026in}{2.277659in}%
\pgfsys@useobject{currentmarker}{}%
\end{pgfscope}%
\begin{pgfscope}%
\pgfsys@transformshift{2.110952in}{2.252854in}%
\pgfsys@useobject{currentmarker}{}%
\end{pgfscope}%
\begin{pgfscope}%
\pgfsys@transformshift{2.094051in}{2.245571in}%
\pgfsys@useobject{currentmarker}{}%
\end{pgfscope}%
\begin{pgfscope}%
\pgfsys@transformshift{2.071047in}{2.260031in}%
\pgfsys@useobject{currentmarker}{}%
\end{pgfscope}%
\begin{pgfscope}%
\pgfsys@transformshift{2.052739in}{2.290679in}%
\pgfsys@useobject{currentmarker}{}%
\end{pgfscope}%
\begin{pgfscope}%
\pgfsys@transformshift{2.033491in}{2.361974in}%
\pgfsys@useobject{currentmarker}{}%
\end{pgfscope}%
\begin{pgfscope}%
\pgfsys@transformshift{2.013069in}{2.509857in}%
\pgfsys@useobject{currentmarker}{}%
\end{pgfscope}%
\begin{pgfscope}%
\pgfsys@transformshift{1.997107in}{2.601365in}%
\pgfsys@useobject{currentmarker}{}%
\end{pgfscope}%
\begin{pgfscope}%
\pgfsys@transformshift{1.971991in}{2.598474in}%
\pgfsys@useobject{currentmarker}{}%
\end{pgfscope}%
\begin{pgfscope}%
\pgfsys@transformshift{1.959552in}{2.541390in}%
\pgfsys@useobject{currentmarker}{}%
\end{pgfscope}%
\begin{pgfscope}%
\pgfsys@transformshift{1.938190in}{2.422781in}%
\pgfsys@useobject{currentmarker}{}%
\end{pgfscope}%
\begin{pgfscope}%
\pgfsys@transformshift{1.919177in}{2.330425in}%
\pgfsys@useobject{currentmarker}{}%
\end{pgfscope}%
\begin{pgfscope}%
\pgfsys@transformshift{1.898521in}{2.275199in}%
\pgfsys@useobject{currentmarker}{}%
\end{pgfscope}%
\begin{pgfscope}%
\pgfsys@transformshift{1.880213in}{2.252424in}%
\pgfsys@useobject{currentmarker}{}%
\end{pgfscope}%
\begin{pgfscope}%
\pgfsys@transformshift{1.860260in}{2.246310in}%
\pgfsys@useobject{currentmarker}{}%
\end{pgfscope}%
\begin{pgfscope}%
\pgfsys@transformshift{1.842655in}{2.253029in}%
\pgfsys@useobject{currentmarker}{}%
\end{pgfscope}%
\begin{pgfscope}%
\pgfsys@transformshift{1.821999in}{2.281458in}%
\pgfsys@useobject{currentmarker}{}%
\end{pgfscope}%
\begin{pgfscope}%
\pgfsys@transformshift{1.802517in}{2.320621in}%
\pgfsys@useobject{currentmarker}{}%
\end{pgfscope}%
\begin{pgfscope}%
\pgfsys@transformshift{1.782095in}{2.426303in}%
\pgfsys@useobject{currentmarker}{}%
\end{pgfscope}%
\begin{pgfscope}%
\pgfsys@transformshift{1.765664in}{2.553345in}%
\pgfsys@useobject{currentmarker}{}%
\end{pgfscope}%
\begin{pgfscope}%
\pgfsys@transformshift{1.744774in}{2.607828in}%
\pgfsys@useobject{currentmarker}{}%
\end{pgfscope}%
\begin{pgfscope}%
\pgfsys@transformshift{1.727874in}{2.622724in}%
\pgfsys@useobject{currentmarker}{}%
\end{pgfscope}%
\begin{pgfscope}%
\pgfsys@transformshift{1.707687in}{2.549292in}%
\pgfsys@useobject{currentmarker}{}%
\end{pgfscope}%
\begin{pgfscope}%
\pgfsys@transformshift{1.688674in}{2.444890in}%
\pgfsys@useobject{currentmarker}{}%
\end{pgfscope}%
\begin{pgfscope}%
\pgfsys@transformshift{1.668721in}{2.334753in}%
\pgfsys@useobject{currentmarker}{}%
\end{pgfscope}%
\begin{pgfscope}%
\pgfsys@transformshift{1.645953in}{2.284552in}%
\pgfsys@useobject{currentmarker}{}%
\end{pgfscope}%
\begin{pgfscope}%
\pgfsys@transformshift{1.631634in}{2.264697in}%
\pgfsys@useobject{currentmarker}{}%
\end{pgfscope}%
\begin{pgfscope}%
\pgfsys@transformshift{1.610274in}{2.248181in}%
\pgfsys@useobject{currentmarker}{}%
\end{pgfscope}%
\begin{pgfscope}%
\pgfsys@transformshift{1.589147in}{2.253073in}%
\pgfsys@useobject{currentmarker}{}%
\end{pgfscope}%
\begin{pgfscope}%
\pgfsys@transformshift{1.573186in}{2.268048in}%
\pgfsys@useobject{currentmarker}{}%
\end{pgfscope}%
\begin{pgfscope}%
\pgfsys@transformshift{1.552529in}{2.297116in}%
\pgfsys@useobject{currentmarker}{}%
\end{pgfscope}%
\begin{pgfscope}%
\pgfsys@transformshift{1.534456in}{2.330234in}%
\pgfsys@useobject{currentmarker}{}%
\end{pgfscope}%
\begin{pgfscope}%
\pgfsys@transformshift{1.515208in}{2.268739in}%
\pgfsys@useobject{currentmarker}{}%
\end{pgfscope}%
\begin{pgfscope}%
\pgfsys@transformshift{1.498543in}{2.250209in}%
\pgfsys@useobject{currentmarker}{}%
\end{pgfscope}%
\begin{pgfscope}%
\pgfsys@transformshift{1.477182in}{2.252850in}%
\pgfsys@useobject{currentmarker}{}%
\end{pgfscope}%
\begin{pgfscope}%
\pgfsys@transformshift{1.456291in}{2.270996in}%
\pgfsys@useobject{currentmarker}{}%
\end{pgfscope}%
\begin{pgfscope}%
\pgfsys@transformshift{1.438686in}{2.309734in}%
\pgfsys@useobject{currentmarker}{}%
\end{pgfscope}%
\begin{pgfscope}%
\pgfsys@transformshift{1.418735in}{2.408625in}%
\pgfsys@useobject{currentmarker}{}%
\end{pgfscope}%
\begin{pgfscope}%
\pgfsys@transformshift{1.401834in}{2.528765in}%
\pgfsys@useobject{currentmarker}{}%
\end{pgfscope}%
\begin{pgfscope}%
\pgfsys@transformshift{1.379769in}{2.626246in}%
\pgfsys@useobject{currentmarker}{}%
\end{pgfscope}%
\begin{pgfscope}%
\pgfsys@transformshift{1.362399in}{2.624086in}%
\pgfsys@useobject{currentmarker}{}%
\end{pgfscope}%
\begin{pgfscope}%
\pgfsys@transformshift{1.341979in}{2.512344in}%
\pgfsys@useobject{currentmarker}{}%
\end{pgfscope}%
\begin{pgfscope}%
\pgfsys@transformshift{1.322026in}{2.400596in}%
\pgfsys@useobject{currentmarker}{}%
\end{pgfscope}%
\begin{pgfscope}%
\pgfsys@transformshift{1.303952in}{2.322030in}%
\pgfsys@useobject{currentmarker}{}%
\end{pgfscope}%
\begin{pgfscope}%
\pgfsys@transformshift{1.285174in}{2.277759in}%
\pgfsys@useobject{currentmarker}{}%
\end{pgfscope}%
\begin{pgfscope}%
\pgfsys@transformshift{1.266395in}{2.266690in}%
\pgfsys@useobject{currentmarker}{}%
\end{pgfscope}%
\begin{pgfscope}%
\pgfsys@transformshift{1.246913in}{2.248787in}%
\pgfsys@useobject{currentmarker}{}%
\end{pgfscope}%
\begin{pgfscope}%
\pgfsys@transformshift{1.227899in}{2.254736in}%
\pgfsys@useobject{currentmarker}{}%
\end{pgfscope}%
\begin{pgfscope}%
\pgfsys@transformshift{1.207243in}{2.281411in}%
\pgfsys@useobject{currentmarker}{}%
\end{pgfscope}%
\begin{pgfscope}%
\pgfsys@transformshift{1.183770in}{2.351710in}%
\pgfsys@useobject{currentmarker}{}%
\end{pgfscope}%
\begin{pgfscope}%
\pgfsys@transformshift{1.168748in}{2.435217in}%
\pgfsys@useobject{currentmarker}{}%
\end{pgfscope}%
\begin{pgfscope}%
\pgfsys@transformshift{1.151612in}{2.537797in}%
\pgfsys@useobject{currentmarker}{}%
\end{pgfscope}%
\begin{pgfscope}%
\pgfsys@transformshift{1.134007in}{2.630692in}%
\pgfsys@useobject{currentmarker}{}%
\end{pgfscope}%
\begin{pgfscope}%
\pgfsys@transformshift{1.109831in}{2.626505in}%
\pgfsys@useobject{currentmarker}{}%
\end{pgfscope}%
\begin{pgfscope}%
\pgfsys@transformshift{1.093634in}{2.549273in}%
\pgfsys@useobject{currentmarker}{}%
\end{pgfscope}%
\begin{pgfscope}%
\pgfsys@transformshift{1.073682in}{2.398716in}%
\pgfsys@useobject{currentmarker}{}%
\end{pgfscope}%
\begin{pgfscope}%
\pgfsys@transformshift{1.053496in}{2.331148in}%
\pgfsys@useobject{currentmarker}{}%
\end{pgfscope}%
\begin{pgfscope}%
\pgfsys@transformshift{1.035891in}{2.293131in}%
\pgfsys@useobject{currentmarker}{}%
\end{pgfscope}%
\begin{pgfscope}%
\pgfsys@transformshift{1.015235in}{2.263588in}%
\pgfsys@useobject{currentmarker}{}%
\end{pgfscope}%
\begin{pgfscope}%
\pgfsys@transformshift{0.994345in}{2.250472in}%
\pgfsys@useobject{currentmarker}{}%
\end{pgfscope}%
\begin{pgfscope}%
\pgfsys@transformshift{0.976740in}{2.262310in}%
\pgfsys@useobject{currentmarker}{}%
\end{pgfscope}%
\begin{pgfscope}%
\pgfsys@transformshift{0.959838in}{2.278420in}%
\pgfsys@useobject{currentmarker}{}%
\end{pgfscope}%
\begin{pgfscope}%
\pgfsys@transformshift{0.941060in}{2.321170in}%
\pgfsys@useobject{currentmarker}{}%
\end{pgfscope}%
\begin{pgfscope}%
\pgfsys@transformshift{0.921109in}{2.378485in}%
\pgfsys@useobject{currentmarker}{}%
\end{pgfscope}%
\begin{pgfscope}%
\pgfsys@transformshift{0.895758in}{2.522488in}%
\pgfsys@useobject{currentmarker}{}%
\end{pgfscope}%
\begin{pgfscope}%
\pgfsys@transformshift{0.881439in}{2.611836in}%
\pgfsys@useobject{currentmarker}{}%
\end{pgfscope}%
\begin{pgfscope}%
\pgfsys@transformshift{0.860078in}{2.655071in}%
\pgfsys@useobject{currentmarker}{}%
\end{pgfscope}%
\begin{pgfscope}%
\pgfsys@transformshift{0.842004in}{2.634220in}%
\pgfsys@useobject{currentmarker}{}%
\end{pgfscope}%
\begin{pgfscope}%
\pgfsys@transformshift{0.823696in}{2.566316in}%
\pgfsys@useobject{currentmarker}{}%
\end{pgfscope}%
\begin{pgfscope}%
\pgfsys@transformshift{0.806326in}{2.448199in}%
\pgfsys@useobject{currentmarker}{}%
\end{pgfscope}%
\begin{pgfscope}%
\pgfsys@transformshift{0.784730in}{2.362350in}%
\pgfsys@useobject{currentmarker}{}%
\end{pgfscope}%
\begin{pgfscope}%
\pgfsys@transformshift{0.766891in}{2.310560in}%
\pgfsys@useobject{currentmarker}{}%
\end{pgfscope}%
\begin{pgfscope}%
\pgfsys@transformshift{0.744123in}{2.274963in}%
\pgfsys@useobject{currentmarker}{}%
\end{pgfscope}%
\begin{pgfscope}%
\pgfsys@transformshift{0.726753in}{2.255618in}%
\pgfsys@useobject{currentmarker}{}%
\end{pgfscope}%
\begin{pgfscope}%
\pgfsys@transformshift{0.709382in}{2.254089in}%
\pgfsys@useobject{currentmarker}{}%
\end{pgfscope}%
\begin{pgfscope}%
\pgfsys@transformshift{0.689195in}{2.274374in}%
\pgfsys@useobject{currentmarker}{}%
\end{pgfscope}%
\begin{pgfscope}%
\pgfsys@transformshift{0.670418in}{2.308889in}%
\pgfsys@useobject{currentmarker}{}%
\end{pgfscope}%
\begin{pgfscope}%
\pgfsys@transformshift{0.651170in}{2.350941in}%
\pgfsys@useobject{currentmarker}{}%
\end{pgfscope}%
\begin{pgfscope}%
\pgfsys@transformshift{0.651170in}{2.346431in}%
\pgfsys@useobject{currentmarker}{}%
\end{pgfscope}%
\begin{pgfscope}%
\pgfsys@transformshift{0.655865in}{2.336113in}%
\pgfsys@useobject{currentmarker}{}%
\end{pgfscope}%
\begin{pgfscope}%
\pgfsys@transformshift{0.675347in}{2.272616in}%
\pgfsys@useobject{currentmarker}{}%
\end{pgfscope}%
\begin{pgfscope}%
\pgfsys@transformshift{0.696942in}{2.252745in}%
\pgfsys@useobject{currentmarker}{}%
\end{pgfscope}%
\begin{pgfscope}%
\pgfsys@transformshift{0.714782in}{2.274218in}%
\pgfsys@useobject{currentmarker}{}%
\end{pgfscope}%
\begin{pgfscope}%
\pgfsys@transformshift{0.732621in}{2.322041in}%
\pgfsys@useobject{currentmarker}{}%
\end{pgfscope}%
\begin{pgfscope}%
\pgfsys@transformshift{0.753511in}{2.442606in}%
\pgfsys@useobject{currentmarker}{}%
\end{pgfscope}%
\begin{pgfscope}%
\pgfsys@transformshift{0.770647in}{2.602124in}%
\pgfsys@useobject{currentmarker}{}%
\end{pgfscope}%
\begin{pgfscope}%
\pgfsys@transformshift{0.789895in}{2.655273in}%
\pgfsys@useobject{currentmarker}{}%
\end{pgfscope}%
\begin{pgfscope}%
\pgfsys@transformshift{0.810317in}{2.553045in}%
\pgfsys@useobject{currentmarker}{}%
\end{pgfscope}%
\begin{pgfscope}%
\pgfsys@transformshift{0.828156in}{2.400417in}%
\pgfsys@useobject{currentmarker}{}%
\end{pgfscope}%
\begin{pgfscope}%
\pgfsys@transformshift{0.848578in}{2.294785in}%
\pgfsys@useobject{currentmarker}{}%
\end{pgfscope}%
\begin{pgfscope}%
\pgfsys@transformshift{0.866417in}{2.259478in}%
\pgfsys@useobject{currentmarker}{}%
\end{pgfscope}%
\begin{pgfscope}%
\pgfsys@transformshift{0.887776in}{2.255210in}%
\pgfsys@useobject{currentmarker}{}%
\end{pgfscope}%
\begin{pgfscope}%
\pgfsys@transformshift{0.904912in}{2.282232in}%
\pgfsys@useobject{currentmarker}{}%
\end{pgfscope}%
\begin{pgfscope}%
\pgfsys@transformshift{0.930028in}{2.450810in}%
\pgfsys@useobject{currentmarker}{}%
\end{pgfscope}%
\begin{pgfscope}%
\pgfsys@transformshift{0.947633in}{2.613646in}%
\pgfsys@useobject{currentmarker}{}%
\end{pgfscope}%
\begin{pgfscope}%
\pgfsys@transformshift{0.962655in}{2.644280in}%
\pgfsys@useobject{currentmarker}{}%
\end{pgfscope}%
\begin{pgfscope}%
\pgfsys@transformshift{0.983311in}{2.529135in}%
\pgfsys@useobject{currentmarker}{}%
\end{pgfscope}%
\begin{pgfscope}%
\pgfsys@transformshift{1.004202in}{2.360400in}%
\pgfsys@useobject{currentmarker}{}%
\end{pgfscope}%
\begin{pgfscope}%
\pgfsys@transformshift{1.022746in}{2.286643in}%
\pgfsys@useobject{currentmarker}{}%
\end{pgfscope}%
\begin{pgfscope}%
\pgfsys@transformshift{1.040820in}{2.255987in}%
\pgfsys@useobject{currentmarker}{}%
\end{pgfscope}%
\begin{pgfscope}%
\pgfsys@transformshift{1.061711in}{2.254005in}%
\pgfsys@useobject{currentmarker}{}%
\end{pgfscope}%
\begin{pgfscope}%
\pgfsys@transformshift{1.079786in}{2.283895in}%
\pgfsys@useobject{currentmarker}{}%
\end{pgfscope}%
\begin{pgfscope}%
\pgfsys@transformshift{1.096451in}{2.341731in}%
\pgfsys@useobject{currentmarker}{}%
\end{pgfscope}%
\begin{pgfscope}%
\pgfsys@transformshift{1.115934in}{2.480496in}%
\pgfsys@useobject{currentmarker}{}%
\end{pgfscope}%
\begin{pgfscope}%
\pgfsys@transformshift{1.140581in}{2.631385in}%
\pgfsys@useobject{currentmarker}{}%
\end{pgfscope}%
\begin{pgfscope}%
\pgfsys@transformshift{1.156777in}{2.605507in}%
\pgfsys@useobject{currentmarker}{}%
\end{pgfscope}%
\begin{pgfscope}%
\pgfsys@transformshift{1.174382in}{2.456729in}%
\pgfsys@useobject{currentmarker}{}%
\end{pgfscope}%
\begin{pgfscope}%
\pgfsys@transformshift{1.195741in}{2.324308in}%
\pgfsys@useobject{currentmarker}{}%
\end{pgfscope}%
\begin{pgfscope}%
\pgfsys@transformshift{1.213580in}{2.271258in}%
\pgfsys@useobject{currentmarker}{}%
\end{pgfscope}%
\begin{pgfscope}%
\pgfsys@transformshift{1.235176in}{2.248698in}%
\pgfsys@useobject{currentmarker}{}%
\end{pgfscope}%
\begin{pgfscope}%
\pgfsys@transformshift{1.253015in}{2.255440in}%
\pgfsys@useobject{currentmarker}{}%
\end{pgfscope}%
\begin{pgfscope}%
\pgfsys@transformshift{1.271794in}{2.278507in}%
\pgfsys@useobject{currentmarker}{}%
\end{pgfscope}%
\begin{pgfscope}%
\pgfsys@transformshift{1.291981in}{2.348352in}%
\pgfsys@useobject{currentmarker}{}%
\end{pgfscope}%
\begin{pgfscope}%
\pgfsys@transformshift{1.310290in}{2.471841in}%
\pgfsys@useobject{currentmarker}{}%
\end{pgfscope}%
\begin{pgfscope}%
\pgfsys@transformshift{1.328834in}{2.602629in}%
\pgfsys@useobject{currentmarker}{}%
\end{pgfscope}%
\begin{pgfscope}%
\pgfsys@transformshift{1.349490in}{2.610095in}%
\pgfsys@useobject{currentmarker}{}%
\end{pgfscope}%
\begin{pgfscope}%
\pgfsys@transformshift{1.367564in}{2.482476in}%
\pgfsys@useobject{currentmarker}{}%
\end{pgfscope}%
\begin{pgfscope}%
\pgfsys@transformshift{1.385872in}{2.354423in}%
\pgfsys@useobject{currentmarker}{}%
\end{pgfscope}%
\begin{pgfscope}%
\pgfsys@transformshift{1.405590in}{2.280799in}%
\pgfsys@useobject{currentmarker}{}%
\end{pgfscope}%
\begin{pgfscope}%
\pgfsys@transformshift{1.430001in}{2.248499in}%
\pgfsys@useobject{currentmarker}{}%
\end{pgfscope}%
\begin{pgfscope}%
\pgfsys@transformshift{1.445025in}{2.248284in}%
\pgfsys@useobject{currentmarker}{}%
\end{pgfscope}%
\begin{pgfscope}%
\pgfsys@transformshift{1.464976in}{2.277017in}%
\pgfsys@useobject{currentmarker}{}%
\end{pgfscope}%
\begin{pgfscope}%
\pgfsys@transformshift{1.483286in}{2.312784in}%
\pgfsys@useobject{currentmarker}{}%
\end{pgfscope}%
\begin{pgfscope}%
\pgfsys@transformshift{1.501594in}{2.391185in}%
\pgfsys@useobject{currentmarker}{}%
\end{pgfscope}%
\begin{pgfscope}%
\pgfsys@transformshift{1.519433in}{2.540109in}%
\pgfsys@useobject{currentmarker}{}%
\end{pgfscope}%
\begin{pgfscope}%
\pgfsys@transformshift{1.540793in}{2.623705in}%
\pgfsys@useobject{currentmarker}{}%
\end{pgfscope}%
\begin{pgfscope}%
\pgfsys@transformshift{1.559806in}{2.567753in}%
\pgfsys@useobject{currentmarker}{}%
\end{pgfscope}%
\begin{pgfscope}%
\pgfsys@transformshift{1.579993in}{2.469679in}%
\pgfsys@useobject{currentmarker}{}%
\end{pgfscope}%
\begin{pgfscope}%
\pgfsys@transformshift{1.597598in}{2.347268in}%
\pgfsys@useobject{currentmarker}{}%
\end{pgfscope}%
\begin{pgfscope}%
\pgfsys@transformshift{1.617315in}{2.275382in}%
\pgfsys@useobject{currentmarker}{}%
\end{pgfscope}%
\begin{pgfscope}%
\pgfsys@transformshift{1.635625in}{2.252872in}%
\pgfsys@useobject{currentmarker}{}%
\end{pgfscope}%
\begin{pgfscope}%
\pgfsys@transformshift{1.653698in}{2.246077in}%
\pgfsys@useobject{currentmarker}{}%
\end{pgfscope}%
\begin{pgfscope}%
\pgfsys@transformshift{1.675060in}{2.261241in}%
\pgfsys@useobject{currentmarker}{}%
\end{pgfscope}%
\begin{pgfscope}%
\pgfsys@transformshift{1.695481in}{2.291222in}%
\pgfsys@useobject{currentmarker}{}%
\end{pgfscope}%
\begin{pgfscope}%
\pgfsys@transformshift{1.712147in}{2.329242in}%
\pgfsys@useobject{currentmarker}{}%
\end{pgfscope}%
\begin{pgfscope}%
\pgfsys@transformshift{1.731629in}{2.466496in}%
\pgfsys@useobject{currentmarker}{}%
\end{pgfscope}%
\begin{pgfscope}%
\pgfsys@transformshift{1.754397in}{2.606459in}%
\pgfsys@useobject{currentmarker}{}%
\end{pgfscope}%
\begin{pgfscope}%
\pgfsys@transformshift{1.771767in}{2.618415in}%
\pgfsys@useobject{currentmarker}{}%
\end{pgfscope}%
\begin{pgfscope}%
\pgfsys@transformshift{1.790780in}{2.550564in}%
\pgfsys@useobject{currentmarker}{}%
\end{pgfscope}%
\begin{pgfscope}%
\pgfsys@transformshift{1.810733in}{2.396405in}%
\pgfsys@useobject{currentmarker}{}%
\end{pgfscope}%
\begin{pgfscope}%
\pgfsys@transformshift{1.828338in}{2.314843in}%
\pgfsys@useobject{currentmarker}{}%
\end{pgfscope}%
\begin{pgfscope}%
\pgfsys@transformshift{1.847351in}{2.271380in}%
\pgfsys@useobject{currentmarker}{}%
\end{pgfscope}%
\begin{pgfscope}%
\pgfsys@transformshift{1.866833in}{2.248906in}%
\pgfsys@useobject{currentmarker}{}%
\end{pgfscope}%
\begin{pgfscope}%
\pgfsys@transformshift{1.885847in}{2.247242in}%
\pgfsys@useobject{currentmarker}{}%
\end{pgfscope}%
\begin{pgfscope}%
\pgfsys@transformshift{1.905798in}{2.268567in}%
\pgfsys@useobject{currentmarker}{}%
\end{pgfscope}%
\begin{pgfscope}%
\pgfsys@transformshift{1.923873in}{2.304036in}%
\pgfsys@useobject{currentmarker}{}%
\end{pgfscope}%
\begin{pgfscope}%
\pgfsys@transformshift{1.946172in}{2.377993in}%
\pgfsys@useobject{currentmarker}{}%
\end{pgfscope}%
\begin{pgfscope}%
\pgfsys@transformshift{1.963072in}{2.428811in}%
\pgfsys@useobject{currentmarker}{}%
\end{pgfscope}%
\begin{pgfscope}%
\pgfsys@transformshift{1.983962in}{2.580518in}%
\pgfsys@useobject{currentmarker}{}%
\end{pgfscope}%
\begin{pgfscope}%
\pgfsys@transformshift{2.002038in}{2.614205in}%
\pgfsys@useobject{currentmarker}{}%
\end{pgfscope}%
\begin{pgfscope}%
\pgfsys@transformshift{2.019877in}{2.547610in}%
\pgfsys@useobject{currentmarker}{}%
\end{pgfscope}%
\begin{pgfscope}%
\pgfsys@transformshift{2.041471in}{2.385655in}%
\pgfsys@useobject{currentmarker}{}%
\end{pgfscope}%
\begin{pgfscope}%
\pgfsys@transformshift{2.059310in}{2.302876in}%
\pgfsys@useobject{currentmarker}{}%
\end{pgfscope}%
\begin{pgfscope}%
\pgfsys@transformshift{2.077855in}{2.265282in}%
\pgfsys@useobject{currentmarker}{}%
\end{pgfscope}%
\begin{pgfscope}%
\pgfsys@transformshift{2.096868in}{2.585424in}%
\pgfsys@useobject{currentmarker}{}%
\end{pgfscope}%
\begin{pgfscope}%
\pgfsys@transformshift{2.117524in}{2.417392in}%
\pgfsys@useobject{currentmarker}{}%
\end{pgfscope}%
\begin{pgfscope}%
\pgfsys@transformshift{2.137240in}{2.306477in}%
\pgfsys@useobject{currentmarker}{}%
\end{pgfscope}%
\begin{pgfscope}%
\pgfsys@transformshift{2.154845in}{2.264813in}%
\pgfsys@useobject{currentmarker}{}%
\end{pgfscope}%
\begin{pgfscope}%
\pgfsys@transformshift{2.176675in}{2.245863in}%
\pgfsys@useobject{currentmarker}{}%
\end{pgfscope}%
\begin{pgfscope}%
\pgfsys@transformshift{2.194280in}{2.251895in}%
\pgfsys@useobject{currentmarker}{}%
\end{pgfscope}%
\begin{pgfscope}%
\pgfsys@transformshift{2.212354in}{2.278346in}%
\pgfsys@useobject{currentmarker}{}%
\end{pgfscope}%
\begin{pgfscope}%
\pgfsys@transformshift{2.234184in}{2.330831in}%
\pgfsys@useobject{currentmarker}{}%
\end{pgfscope}%
\begin{pgfscope}%
\pgfsys@transformshift{2.252258in}{2.435826in}%
\pgfsys@useobject{currentmarker}{}%
\end{pgfscope}%
\begin{pgfscope}%
\pgfsys@transformshift{2.268220in}{2.511654in}%
\pgfsys@useobject{currentmarker}{}%
\end{pgfscope}%
\begin{pgfscope}%
\pgfsys@transformshift{2.289815in}{2.614828in}%
\pgfsys@useobject{currentmarker}{}%
\end{pgfscope}%
\begin{pgfscope}%
\pgfsys@transformshift{2.307420in}{2.575774in}%
\pgfsys@useobject{currentmarker}{}%
\end{pgfscope}%
\begin{pgfscope}%
\pgfsys@transformshift{2.327842in}{2.427065in}%
\pgfsys@useobject{currentmarker}{}%
\end{pgfscope}%
\begin{pgfscope}%
\pgfsys@transformshift{2.348967in}{2.307474in}%
\pgfsys@useobject{currentmarker}{}%
\end{pgfscope}%
\begin{pgfscope}%
\pgfsys@transformshift{2.367277in}{2.269329in}%
\pgfsys@useobject{currentmarker}{}%
\end{pgfscope}%
\begin{pgfscope}%
\pgfsys@transformshift{2.386525in}{2.248555in}%
\pgfsys@useobject{currentmarker}{}%
\end{pgfscope}%
\begin{pgfscope}%
\pgfsys@transformshift{2.404364in}{2.246238in}%
\pgfsys@useobject{currentmarker}{}%
\end{pgfscope}%
\begin{pgfscope}%
\pgfsys@transformshift{2.423846in}{2.255202in}%
\pgfsys@useobject{currentmarker}{}%
\end{pgfscope}%
\begin{pgfscope}%
\pgfsys@transformshift{2.442154in}{2.282149in}%
\pgfsys@useobject{currentmarker}{}%
\end{pgfscope}%
\begin{pgfscope}%
\pgfsys@transformshift{2.460933in}{2.346800in}%
\pgfsys@useobject{currentmarker}{}%
\end{pgfscope}%
\begin{pgfscope}%
\pgfsys@transformshift{2.486283in}{2.513703in}%
\pgfsys@useobject{currentmarker}{}%
\end{pgfscope}%
\begin{pgfscope}%
\pgfsys@transformshift{2.501542in}{2.597944in}%
\pgfsys@useobject{currentmarker}{}%
\end{pgfscope}%
\begin{pgfscope}%
\pgfsys@transformshift{2.518676in}{2.606907in}%
\pgfsys@useobject{currentmarker}{}%
\end{pgfscope}%
\begin{pgfscope}%
\pgfsys@transformshift{2.540037in}{2.513206in}%
\pgfsys@useobject{currentmarker}{}%
\end{pgfscope}%
\begin{pgfscope}%
\pgfsys@transformshift{2.558345in}{2.377365in}%
\pgfsys@useobject{currentmarker}{}%
\end{pgfscope}%
\begin{pgfscope}%
\pgfsys@transformshift{2.579001in}{2.291234in}%
\pgfsys@useobject{currentmarker}{}%
\end{pgfscope}%
\begin{pgfscope}%
\pgfsys@transformshift{2.598249in}{2.260382in}%
\pgfsys@useobject{currentmarker}{}%
\end{pgfscope}%
\begin{pgfscope}%
\pgfsys@transformshift{2.618202in}{2.245174in}%
\pgfsys@useobject{currentmarker}{}%
\end{pgfscope}%
\begin{pgfscope}%
\pgfsys@transformshift{2.637684in}{2.248488in}%
\pgfsys@useobject{currentmarker}{}%
\end{pgfscope}%
\begin{pgfscope}%
\pgfsys@transformshift{2.655758in}{2.268586in}%
\pgfsys@useobject{currentmarker}{}%
\end{pgfscope}%
\begin{pgfscope}%
\pgfsys@transformshift{2.674771in}{2.295576in}%
\pgfsys@useobject{currentmarker}{}%
\end{pgfscope}%
\begin{pgfscope}%
\pgfsys@transformshift{2.693081in}{2.350258in}%
\pgfsys@useobject{currentmarker}{}%
\end{pgfscope}%
\begin{pgfscope}%
\pgfsys@transformshift{2.714675in}{2.471911in}%
\pgfsys@useobject{currentmarker}{}%
\end{pgfscope}%
\begin{pgfscope}%
\pgfsys@transformshift{2.733219in}{2.587830in}%
\pgfsys@useobject{currentmarker}{}%
\end{pgfscope}%
\begin{pgfscope}%
\pgfsys@transformshift{2.752232in}{2.615830in}%
\pgfsys@useobject{currentmarker}{}%
\end{pgfscope}%
\begin{pgfscope}%
\pgfsys@transformshift{2.768663in}{2.561518in}%
\pgfsys@useobject{currentmarker}{}%
\end{pgfscope}%
\begin{pgfscope}%
\pgfsys@transformshift{2.789554in}{2.404311in}%
\pgfsys@useobject{currentmarker}{}%
\end{pgfscope}%
\begin{pgfscope}%
\pgfsys@transformshift{2.807159in}{2.312235in}%
\pgfsys@useobject{currentmarker}{}%
\end{pgfscope}%
\begin{pgfscope}%
\pgfsys@transformshift{2.826406in}{2.268715in}%
\pgfsys@useobject{currentmarker}{}%
\end{pgfscope}%
\begin{pgfscope}%
\pgfsys@transformshift{2.847768in}{2.249685in}%
\pgfsys@useobject{currentmarker}{}%
\end{pgfscope}%
\begin{pgfscope}%
\pgfsys@transformshift{2.867484in}{2.248345in}%
\pgfsys@useobject{currentmarker}{}%
\end{pgfscope}%
\begin{pgfscope}%
\pgfsys@transformshift{2.885792in}{2.263818in}%
\pgfsys@useobject{currentmarker}{}%
\end{pgfscope}%
\begin{pgfscope}%
\pgfsys@transformshift{2.904571in}{2.293940in}%
\pgfsys@useobject{currentmarker}{}%
\end{pgfscope}%
\begin{pgfscope}%
\pgfsys@transformshift{2.920767in}{2.339804in}%
\pgfsys@useobject{currentmarker}{}%
\end{pgfscope}%
\begin{pgfscope}%
\pgfsys@transformshift{2.943537in}{2.463072in}%
\pgfsys@useobject{currentmarker}{}%
\end{pgfscope}%
\begin{pgfscope}%
\pgfsys@transformshift{2.960202in}{2.575753in}%
\pgfsys@useobject{currentmarker}{}%
\end{pgfscope}%
\begin{pgfscope}%
\pgfsys@transformshift{2.982501in}{2.619628in}%
\pgfsys@useobject{currentmarker}{}%
\end{pgfscope}%
\begin{pgfscope}%
\pgfsys@transformshift{3.001046in}{2.579252in}%
\pgfsys@useobject{currentmarker}{}%
\end{pgfscope}%
\begin{pgfscope}%
\pgfsys@transformshift{3.018180in}{2.487359in}%
\pgfsys@useobject{currentmarker}{}%
\end{pgfscope}%
\begin{pgfscope}%
\pgfsys@transformshift{3.040010in}{2.350071in}%
\pgfsys@useobject{currentmarker}{}%
\end{pgfscope}%
\begin{pgfscope}%
\pgfsys@transformshift{3.059023in}{2.294956in}%
\pgfsys@useobject{currentmarker}{}%
\end{pgfscope}%
\begin{pgfscope}%
\pgfsys@transformshift{3.079211in}{2.270974in}%
\pgfsys@useobject{currentmarker}{}%
\end{pgfscope}%
\begin{pgfscope}%
\pgfsys@transformshift{3.099867in}{2.248091in}%
\pgfsys@useobject{currentmarker}{}%
\end{pgfscope}%
\begin{pgfscope}%
\pgfsys@transformshift{3.115358in}{2.246679in}%
\pgfsys@useobject{currentmarker}{}%
\end{pgfscope}%
\begin{pgfscope}%
\pgfsys@transformshift{3.136719in}{2.261017in}%
\pgfsys@useobject{currentmarker}{}%
\end{pgfscope}%
\begin{pgfscope}%
\pgfsys@transformshift{3.154793in}{2.286735in}%
\pgfsys@useobject{currentmarker}{}%
\end{pgfscope}%
\begin{pgfscope}%
\pgfsys@transformshift{3.172398in}{2.333995in}%
\pgfsys@useobject{currentmarker}{}%
\end{pgfscope}%
\begin{pgfscope}%
\pgfsys@transformshift{3.194931in}{2.439939in}%
\pgfsys@useobject{currentmarker}{}%
\end{pgfscope}%
\begin{pgfscope}%
\pgfsys@transformshift{3.211598in}{2.566014in}%
\pgfsys@useobject{currentmarker}{}%
\end{pgfscope}%
\begin{pgfscope}%
\pgfsys@transformshift{3.231080in}{2.623284in}%
\pgfsys@useobject{currentmarker}{}%
\end{pgfscope}%
\begin{pgfscope}%
\pgfsys@transformshift{3.250797in}{2.338045in}%
\pgfsys@useobject{currentmarker}{}%
\end{pgfscope}%
\begin{pgfscope}%
\pgfsys@transformshift{3.268872in}{2.408514in}%
\pgfsys@useobject{currentmarker}{}%
\end{pgfscope}%
\begin{pgfscope}%
\pgfsys@transformshift{3.288823in}{2.553566in}%
\pgfsys@useobject{currentmarker}{}%
\end{pgfscope}%
\begin{pgfscope}%
\pgfsys@transformshift{3.306897in}{2.622052in}%
\pgfsys@useobject{currentmarker}{}%
\end{pgfscope}%
\begin{pgfscope}%
\pgfsys@transformshift{3.326145in}{2.625611in}%
\pgfsys@useobject{currentmarker}{}%
\end{pgfscope}%
\begin{pgfscope}%
\pgfsys@transformshift{3.348446in}{2.519251in}%
\pgfsys@useobject{currentmarker}{}%
\end{pgfscope}%
\begin{pgfscope}%
\pgfsys@transformshift{3.367457in}{2.383682in}%
\pgfsys@useobject{currentmarker}{}%
\end{pgfscope}%
\begin{pgfscope}%
\pgfsys@transformshift{3.384124in}{2.328427in}%
\pgfsys@useobject{currentmarker}{}%
\end{pgfscope}%
\begin{pgfscope}%
\pgfsys@transformshift{3.405718in}{2.271548in}%
\pgfsys@useobject{currentmarker}{}%
\end{pgfscope}%
\begin{pgfscope}%
\pgfsys@transformshift{3.425905in}{2.249381in}%
\pgfsys@useobject{currentmarker}{}%
\end{pgfscope}%
\begin{pgfscope}%
\pgfsys@transformshift{3.441633in}{2.247823in}%
\pgfsys@useobject{currentmarker}{}%
\end{pgfscope}%
\begin{pgfscope}%
\pgfsys@transformshift{3.463932in}{2.266668in}%
\pgfsys@useobject{currentmarker}{}%
\end{pgfscope}%
\begin{pgfscope}%
\pgfsys@transformshift{3.482005in}{2.303751in}%
\pgfsys@useobject{currentmarker}{}%
\end{pgfscope}%
\begin{pgfscope}%
\pgfsys@transformshift{3.501724in}{2.372276in}%
\pgfsys@useobject{currentmarker}{}%
\end{pgfscope}%
\begin{pgfscope}%
\pgfsys@transformshift{3.521440in}{2.493788in}%
\pgfsys@useobject{currentmarker}{}%
\end{pgfscope}%
\begin{pgfscope}%
\pgfsys@transformshift{3.539748in}{2.610902in}%
\pgfsys@useobject{currentmarker}{}%
\end{pgfscope}%
\begin{pgfscope}%
\pgfsys@transformshift{3.558996in}{2.627529in}%
\pgfsys@useobject{currentmarker}{}%
\end{pgfscope}%
\begin{pgfscope}%
\pgfsys@transformshift{3.578009in}{2.557192in}%
\pgfsys@useobject{currentmarker}{}%
\end{pgfscope}%
\begin{pgfscope}%
\pgfsys@transformshift{3.598197in}{2.434751in}%
\pgfsys@useobject{currentmarker}{}%
\end{pgfscope}%
\begin{pgfscope}%
\pgfsys@transformshift{3.618149in}{2.347984in}%
\pgfsys@useobject{currentmarker}{}%
\end{pgfscope}%
\begin{pgfscope}%
\pgfsys@transformshift{3.635283in}{2.287993in}%
\pgfsys@useobject{currentmarker}{}%
\end{pgfscope}%
\begin{pgfscope}%
\pgfsys@transformshift{3.655236in}{2.261034in}%
\pgfsys@useobject{currentmarker}{}%
\end{pgfscope}%
\begin{pgfscope}%
\pgfsys@transformshift{3.673779in}{2.248800in}%
\pgfsys@useobject{currentmarker}{}%
\end{pgfscope}%
\begin{pgfscope}%
\pgfsys@transformshift{3.689272in}{2.251089in}%
\pgfsys@useobject{currentmarker}{}%
\end{pgfscope}%
\begin{pgfscope}%
\pgfsys@transformshift{3.712745in}{2.275829in}%
\pgfsys@useobject{currentmarker}{}%
\end{pgfscope}%
\begin{pgfscope}%
\pgfsys@transformshift{3.732227in}{2.299326in}%
\pgfsys@useobject{currentmarker}{}%
\end{pgfscope}%
\begin{pgfscope}%
\pgfsys@transformshift{3.750301in}{2.359397in}%
\pgfsys@useobject{currentmarker}{}%
\end{pgfscope}%
\begin{pgfscope}%
\pgfsys@transformshift{3.770019in}{2.447902in}%
\pgfsys@useobject{currentmarker}{}%
\end{pgfscope}%
\begin{pgfscope}%
\pgfsys@transformshift{3.787858in}{2.569881in}%
\pgfsys@useobject{currentmarker}{}%
\end{pgfscope}%
\begin{pgfscope}%
\pgfsys@transformshift{3.807106in}{2.629108in}%
\pgfsys@useobject{currentmarker}{}%
\end{pgfscope}%
\begin{pgfscope}%
\pgfsys@transformshift{3.827057in}{2.635611in}%
\pgfsys@useobject{currentmarker}{}%
\end{pgfscope}%
\begin{pgfscope}%
\pgfsys@transformshift{3.849358in}{2.527984in}%
\pgfsys@useobject{currentmarker}{}%
\end{pgfscope}%
\begin{pgfscope}%
\pgfsys@transformshift{3.864146in}{2.425190in}%
\pgfsys@useobject{currentmarker}{}%
\end{pgfscope}%
\begin{pgfscope}%
\pgfsys@transformshift{3.882688in}{2.335086in}%
\pgfsys@useobject{currentmarker}{}%
\end{pgfscope}%
\begin{pgfscope}%
\pgfsys@transformshift{3.905458in}{2.292335in}%
\pgfsys@useobject{currentmarker}{}%
\end{pgfscope}%
\begin{pgfscope}%
\pgfsys@transformshift{3.920949in}{2.266495in}%
\pgfsys@useobject{currentmarker}{}%
\end{pgfscope}%
\begin{pgfscope}%
\pgfsys@transformshift{3.942311in}{2.249913in}%
\pgfsys@useobject{currentmarker}{}%
\end{pgfscope}%
\begin{pgfscope}%
\pgfsys@transformshift{3.961088in}{2.254604in}%
\pgfsys@useobject{currentmarker}{}%
\end{pgfscope}%
\begin{pgfscope}%
\pgfsys@transformshift{3.980335in}{2.274928in}%
\pgfsys@useobject{currentmarker}{}%
\end{pgfscope}%
\begin{pgfscope}%
\pgfsys@transformshift{3.999348in}{2.303230in}%
\pgfsys@useobject{currentmarker}{}%
\end{pgfscope}%
\begin{pgfscope}%
\pgfsys@transformshift{4.021413in}{2.367479in}%
\pgfsys@useobject{currentmarker}{}%
\end{pgfscope}%
\begin{pgfscope}%
\pgfsys@transformshift{4.038315in}{2.464471in}%
\pgfsys@useobject{currentmarker}{}%
\end{pgfscope}%
\begin{pgfscope}%
\pgfsys@transformshift{4.060379in}{2.567171in}%
\pgfsys@useobject{currentmarker}{}%
\end{pgfscope}%
\begin{pgfscope}%
\pgfsys@transformshift{4.077044in}{2.648310in}%
\pgfsys@useobject{currentmarker}{}%
\end{pgfscope}%
\begin{pgfscope}%
\pgfsys@transformshift{4.096292in}{2.638459in}%
\pgfsys@useobject{currentmarker}{}%
\end{pgfscope}%
\begin{pgfscope}%
\pgfsys@transformshift{4.117653in}{2.596629in}%
\pgfsys@useobject{currentmarker}{}%
\end{pgfscope}%
\begin{pgfscope}%
\pgfsys@transformshift{4.133850in}{2.489366in}%
\pgfsys@useobject{currentmarker}{}%
\end{pgfscope}%
\begin{pgfscope}%
\pgfsys@transformshift{4.152863in}{2.378113in}%
\pgfsys@useobject{currentmarker}{}%
\end{pgfscope}%
\begin{pgfscope}%
\pgfsys@transformshift{4.172111in}{2.316180in}%
\pgfsys@useobject{currentmarker}{}%
\end{pgfscope}%
\begin{pgfscope}%
\pgfsys@transformshift{4.193705in}{2.272601in}%
\pgfsys@useobject{currentmarker}{}%
\end{pgfscope}%
\begin{pgfscope}%
\pgfsys@transformshift{4.212015in}{2.257810in}%
\pgfsys@useobject{currentmarker}{}%
\end{pgfscope}%
\begin{pgfscope}%
\pgfsys@transformshift{4.231028in}{2.253643in}%
\pgfsys@useobject{currentmarker}{}%
\end{pgfscope}%
\begin{pgfscope}%
\pgfsys@transformshift{4.250744in}{2.269501in}%
\pgfsys@useobject{currentmarker}{}%
\end{pgfscope}%
\begin{pgfscope}%
\pgfsys@transformshift{4.270226in}{2.299565in}%
\pgfsys@useobject{currentmarker}{}%
\end{pgfscope}%
\begin{pgfscope}%
\pgfsys@transformshift{4.288300in}{2.350144in}%
\pgfsys@useobject{currentmarker}{}%
\end{pgfscope}%
\begin{pgfscope}%
\pgfsys@transformshift{4.308958in}{2.408931in}%
\pgfsys@useobject{currentmarker}{}%
\end{pgfscope}%
\begin{pgfscope}%
\pgfsys@transformshift{4.328440in}{2.527237in}%
\pgfsys@useobject{currentmarker}{}%
\end{pgfscope}%
\begin{pgfscope}%
\pgfsys@transformshift{4.346983in}{2.624615in}%
\pgfsys@useobject{currentmarker}{}%
\end{pgfscope}%
\begin{pgfscope}%
\pgfsys@transformshift{4.365996in}{2.666622in}%
\pgfsys@useobject{currentmarker}{}%
\end{pgfscope}%
\begin{pgfscope}%
\pgfsys@transformshift{4.383601in}{2.635727in}%
\pgfsys@useobject{currentmarker}{}%
\end{pgfscope}%
\begin{pgfscope}%
\pgfsys@transformshift{4.402145in}{2.568708in}%
\pgfsys@useobject{currentmarker}{}%
\end{pgfscope}%
\begin{pgfscope}%
\pgfsys@transformshift{4.422096in}{2.440942in}%
\pgfsys@useobject{currentmarker}{}%
\end{pgfscope}%
\begin{pgfscope}%
\pgfsys@transformshift{4.443223in}{2.345964in}%
\pgfsys@useobject{currentmarker}{}%
\end{pgfscope}%
\begin{pgfscope}%
\pgfsys@transformshift{4.462000in}{2.296916in}%
\pgfsys@useobject{currentmarker}{}%
\end{pgfscope}%
\begin{pgfscope}%
\pgfsys@transformshift{4.477493in}{2.271224in}%
\pgfsys@useobject{currentmarker}{}%
\end{pgfscope}%
\begin{pgfscope}%
\pgfsys@transformshift{4.480310in}{2.276371in}%
\pgfsys@useobject{currentmarker}{}%
\end{pgfscope}%
\begin{pgfscope}%
\pgfsys@transformshift{4.474442in}{2.288807in}%
\pgfsys@useobject{currentmarker}{}%
\end{pgfscope}%
\begin{pgfscope}%
\pgfsys@transformshift{4.452846in}{2.384067in}%
\pgfsys@useobject{currentmarker}{}%
\end{pgfscope}%
\begin{pgfscope}%
\pgfsys@transformshift{4.435007in}{2.533063in}%
\pgfsys@useobject{currentmarker}{}%
\end{pgfscope}%
\begin{pgfscope}%
\pgfsys@transformshift{4.415993in}{2.655142in}%
\pgfsys@useobject{currentmarker}{}%
\end{pgfscope}%
\begin{pgfscope}%
\pgfsys@transformshift{4.396746in}{2.645333in}%
\pgfsys@useobject{currentmarker}{}%
\end{pgfscope}%
\begin{pgfscope}%
\pgfsys@transformshift{4.377967in}{2.495928in}%
\pgfsys@useobject{currentmarker}{}%
\end{pgfscope}%
\begin{pgfscope}%
\pgfsys@transformshift{4.360128in}{2.355233in}%
\pgfsys@useobject{currentmarker}{}%
\end{pgfscope}%
\begin{pgfscope}%
\pgfsys@transformshift{4.340177in}{2.267079in}%
\pgfsys@useobject{currentmarker}{}%
\end{pgfscope}%
\begin{pgfscope}%
\pgfsys@transformshift{4.318815in}{2.253806in}%
\pgfsys@useobject{currentmarker}{}%
\end{pgfscope}%
\begin{pgfscope}%
\pgfsys@transformshift{4.302385in}{2.280551in}%
\pgfsys@useobject{currentmarker}{}%
\end{pgfscope}%
\begin{pgfscope}%
\pgfsys@transformshift{4.282197in}{2.348262in}%
\pgfsys@useobject{currentmarker}{}%
\end{pgfscope}%
\begin{pgfscope}%
\pgfsys@transformshift{4.263889in}{2.487825in}%
\pgfsys@useobject{currentmarker}{}%
\end{pgfscope}%
\begin{pgfscope}%
\pgfsys@transformshift{4.241356in}{2.643344in}%
\pgfsys@useobject{currentmarker}{}%
\end{pgfscope}%
\begin{pgfscope}%
\pgfsys@transformshift{4.223280in}{2.635891in}%
\pgfsys@useobject{currentmarker}{}%
\end{pgfscope}%
\begin{pgfscope}%
\pgfsys@transformshift{4.205441in}{2.491143in}%
\pgfsys@useobject{currentmarker}{}%
\end{pgfscope}%
\begin{pgfscope}%
\pgfsys@transformshift{4.185725in}{2.347116in}%
\pgfsys@useobject{currentmarker}{}%
\end{pgfscope}%
\begin{pgfscope}%
\pgfsys@transformshift{4.166946in}{2.282913in}%
\pgfsys@useobject{currentmarker}{}%
\end{pgfscope}%
\begin{pgfscope}%
\pgfsys@transformshift{4.148403in}{2.253061in}%
\pgfsys@useobject{currentmarker}{}%
\end{pgfscope}%
\begin{pgfscope}%
\pgfsys@transformshift{4.127042in}{2.260976in}%
\pgfsys@useobject{currentmarker}{}%
\end{pgfscope}%
\begin{pgfscope}%
\pgfsys@transformshift{4.108263in}{2.301734in}%
\pgfsys@useobject{currentmarker}{}%
\end{pgfscope}%
\begin{pgfscope}%
\pgfsys@transformshift{4.089721in}{2.399357in}%
\pgfsys@useobject{currentmarker}{}%
\end{pgfscope}%
\begin{pgfscope}%
\pgfsys@transformshift{4.071176in}{2.555988in}%
\pgfsys@useobject{currentmarker}{}%
\end{pgfscope}%
\begin{pgfscope}%
\pgfsys@transformshift{4.049346in}{2.644811in}%
\pgfsys@useobject{currentmarker}{}%
\end{pgfscope}%
\begin{pgfscope}%
\pgfsys@transformshift{4.029864in}{2.574245in}%
\pgfsys@useobject{currentmarker}{}%
\end{pgfscope}%
\begin{pgfscope}%
\pgfsys@transformshift{4.011321in}{2.404771in}%
\pgfsys@useobject{currentmarker}{}%
\end{pgfscope}%
\begin{pgfscope}%
\pgfsys@transformshift{3.992777in}{2.305487in}%
\pgfsys@useobject{currentmarker}{}%
\end{pgfscope}%
\begin{pgfscope}%
\pgfsys@transformshift{3.974703in}{2.263555in}%
\pgfsys@useobject{currentmarker}{}%
\end{pgfscope}%
\begin{pgfscope}%
\pgfsys@transformshift{3.956159in}{2.247959in}%
\pgfsys@useobject{currentmarker}{}%
\end{pgfscope}%
\begin{pgfscope}%
\pgfsys@transformshift{3.937380in}{2.258550in}%
\pgfsys@useobject{currentmarker}{}%
\end{pgfscope}%
\begin{pgfscope}%
\pgfsys@transformshift{3.915081in}{2.305256in}%
\pgfsys@useobject{currentmarker}{}%
\end{pgfscope}%
\begin{pgfscope}%
\pgfsys@transformshift{3.896773in}{2.409374in}%
\pgfsys@useobject{currentmarker}{}%
\end{pgfscope}%
\begin{pgfscope}%
\pgfsys@transformshift{3.878229in}{2.560453in}%
\pgfsys@useobject{currentmarker}{}%
\end{pgfscope}%
\begin{pgfscope}%
\pgfsys@transformshift{3.859450in}{2.635479in}%
\pgfsys@useobject{currentmarker}{}%
\end{pgfscope}%
\begin{pgfscope}%
\pgfsys@transformshift{3.839968in}{2.586129in}%
\pgfsys@useobject{currentmarker}{}%
\end{pgfscope}%
\begin{pgfscope}%
\pgfsys@transformshift{3.822128in}{2.427062in}%
\pgfsys@useobject{currentmarker}{}%
\end{pgfscope}%
\begin{pgfscope}%
\pgfsys@transformshift{3.800533in}{2.306021in}%
\pgfsys@useobject{currentmarker}{}%
\end{pgfscope}%
\begin{pgfscope}%
\pgfsys@transformshift{3.781990in}{2.265832in}%
\pgfsys@useobject{currentmarker}{}%
\end{pgfscope}%
\begin{pgfscope}%
\pgfsys@transformshift{3.764151in}{2.247895in}%
\pgfsys@useobject{currentmarker}{}%
\end{pgfscope}%
\begin{pgfscope}%
\pgfsys@transformshift{3.742321in}{2.266306in}%
\pgfsys@useobject{currentmarker}{}%
\end{pgfscope}%
\begin{pgfscope}%
\pgfsys@transformshift{3.725656in}{2.287593in}%
\pgfsys@useobject{currentmarker}{}%
\end{pgfscope}%
\begin{pgfscope}%
\pgfsys@transformshift{3.701712in}{2.396312in}%
\pgfsys@useobject{currentmarker}{}%
\end{pgfscope}%
\begin{pgfscope}%
\pgfsys@transformshift{3.685281in}{2.528898in}%
\pgfsys@useobject{currentmarker}{}%
\end{pgfscope}%
\begin{pgfscope}%
\pgfsys@transformshift{3.685986in}{2.597971in}%
\pgfsys@useobject{currentmarker}{}%
\end{pgfscope}%
\begin{pgfscope}%
\pgfsys@transformshift{3.667676in}{2.622803in}%
\pgfsys@useobject{currentmarker}{}%
\end{pgfscope}%
\begin{pgfscope}%
\pgfsys@transformshift{3.650777in}{2.612575in}%
\pgfsys@useobject{currentmarker}{}%
\end{pgfscope}%
\begin{pgfscope}%
\pgfsys@transformshift{3.627538in}{2.458338in}%
\pgfsys@useobject{currentmarker}{}%
\end{pgfscope}%
\begin{pgfscope}%
\pgfsys@transformshift{3.608056in}{2.329847in}%
\pgfsys@useobject{currentmarker}{}%
\end{pgfscope}%
\begin{pgfscope}%
\pgfsys@transformshift{3.591625in}{2.298423in}%
\pgfsys@useobject{currentmarker}{}%
\end{pgfscope}%
\begin{pgfscope}%
\pgfsys@transformshift{3.573081in}{2.268011in}%
\pgfsys@useobject{currentmarker}{}%
\end{pgfscope}%
\begin{pgfscope}%
\pgfsys@transformshift{3.551721in}{2.246451in}%
\pgfsys@useobject{currentmarker}{}%
\end{pgfscope}%
\begin{pgfscope}%
\pgfsys@transformshift{3.531534in}{2.254916in}%
\pgfsys@useobject{currentmarker}{}%
\end{pgfscope}%
\begin{pgfscope}%
\pgfsys@transformshift{3.510878in}{2.289855in}%
\pgfsys@useobject{currentmarker}{}%
\end{pgfscope}%
\begin{pgfscope}%
\pgfsys@transformshift{3.494681in}{2.350291in}%
\pgfsys@useobject{currentmarker}{}%
\end{pgfscope}%
\begin{pgfscope}%
\pgfsys@transformshift{3.476137in}{2.462501in}%
\pgfsys@useobject{currentmarker}{}%
\end{pgfscope}%
\begin{pgfscope}%
\pgfsys@transformshift{3.456421in}{2.345312in}%
\pgfsys@useobject{currentmarker}{}%
\end{pgfscope}%
\begin{pgfscope}%
\pgfsys@transformshift{3.434825in}{2.510511in}%
\pgfsys@useobject{currentmarker}{}%
\end{pgfscope}%
\begin{pgfscope}%
\pgfsys@transformshift{3.418628in}{2.592702in}%
\pgfsys@useobject{currentmarker}{}%
\end{pgfscope}%
\begin{pgfscope}%
\pgfsys@transformshift{3.398677in}{2.608813in}%
\pgfsys@useobject{currentmarker}{}%
\end{pgfscope}%
\begin{pgfscope}%
\pgfsys@transformshift{3.376142in}{2.457567in}%
\pgfsys@useobject{currentmarker}{}%
\end{pgfscope}%
\begin{pgfscope}%
\pgfsys@transformshift{3.362528in}{2.344469in}%
\pgfsys@useobject{currentmarker}{}%
\end{pgfscope}%
\begin{pgfscope}%
\pgfsys@transformshift{3.340698in}{2.290106in}%
\pgfsys@useobject{currentmarker}{}%
\end{pgfscope}%
\begin{pgfscope}%
\pgfsys@transformshift{3.319339in}{2.253567in}%
\pgfsys@useobject{currentmarker}{}%
\end{pgfscope}%
\begin{pgfscope}%
\pgfsys@transformshift{3.300560in}{2.246116in}%
\pgfsys@useobject{currentmarker}{}%
\end{pgfscope}%
\begin{pgfscope}%
\pgfsys@transformshift{3.281547in}{2.262788in}%
\pgfsys@useobject{currentmarker}{}%
\end{pgfscope}%
\begin{pgfscope}%
\pgfsys@transformshift{3.263004in}{2.297507in}%
\pgfsys@useobject{currentmarker}{}%
\end{pgfscope}%
\begin{pgfscope}%
\pgfsys@transformshift{3.245399in}{2.341323in}%
\pgfsys@useobject{currentmarker}{}%
\end{pgfscope}%
\begin{pgfscope}%
\pgfsys@transformshift{3.225915in}{2.462381in}%
\pgfsys@useobject{currentmarker}{}%
\end{pgfscope}%
\begin{pgfscope}%
\pgfsys@transformshift{3.204790in}{2.606717in}%
\pgfsys@useobject{currentmarker}{}%
\end{pgfscope}%
\begin{pgfscope}%
\pgfsys@transformshift{3.185777in}{2.604594in}%
\pgfsys@useobject{currentmarker}{}%
\end{pgfscope}%
\begin{pgfscope}%
\pgfsys@transformshift{3.167235in}{2.470453in}%
\pgfsys@useobject{currentmarker}{}%
\end{pgfscope}%
\begin{pgfscope}%
\pgfsys@transformshift{3.147750in}{2.370776in}%
\pgfsys@useobject{currentmarker}{}%
\end{pgfscope}%
\begin{pgfscope}%
\pgfsys@transformshift{3.127800in}{2.297742in}%
\pgfsys@useobject{currentmarker}{}%
\end{pgfscope}%
\begin{pgfscope}%
\pgfsys@transformshift{3.109021in}{2.260634in}%
\pgfsys@useobject{currentmarker}{}%
\end{pgfscope}%
\begin{pgfscope}%
\pgfsys@transformshift{3.090713in}{2.245558in}%
\pgfsys@useobject{currentmarker}{}%
\end{pgfscope}%
\begin{pgfscope}%
\pgfsys@transformshift{3.072168in}{2.252698in}%
\pgfsys@useobject{currentmarker}{}%
\end{pgfscope}%
\begin{pgfscope}%
\pgfsys@transformshift{3.053389in}{2.278876in}%
\pgfsys@useobject{currentmarker}{}%
\end{pgfscope}%
\begin{pgfscope}%
\pgfsys@transformshift{3.031796in}{2.351037in}%
\pgfsys@useobject{currentmarker}{}%
\end{pgfscope}%
\begin{pgfscope}%
\pgfsys@transformshift{3.014660in}{2.455760in}%
\pgfsys@useobject{currentmarker}{}%
\end{pgfscope}%
\begin{pgfscope}%
\pgfsys@transformshift{2.995412in}{2.593866in}%
\pgfsys@useobject{currentmarker}{}%
\end{pgfscope}%
\begin{pgfscope}%
\pgfsys@transformshift{2.974990in}{2.613144in}%
\pgfsys@useobject{currentmarker}{}%
\end{pgfscope}%
\begin{pgfscope}%
\pgfsys@transformshift{2.956682in}{2.545526in}%
\pgfsys@useobject{currentmarker}{}%
\end{pgfscope}%
\begin{pgfscope}%
\pgfsys@transformshift{2.938843in}{2.441206in}%
\pgfsys@useobject{currentmarker}{}%
\end{pgfscope}%
\begin{pgfscope}%
\pgfsys@transformshift{2.917716in}{2.323341in}%
\pgfsys@useobject{currentmarker}{}%
\end{pgfscope}%
\begin{pgfscope}%
\pgfsys@transformshift{2.898937in}{2.274254in}%
\pgfsys@useobject{currentmarker}{}%
\end{pgfscope}%
\begin{pgfscope}%
\pgfsys@transformshift{2.880395in}{2.250709in}%
\pgfsys@useobject{currentmarker}{}%
\end{pgfscope}%
\begin{pgfscope}%
\pgfsys@transformshift{2.860676in}{2.246723in}%
\pgfsys@useobject{currentmarker}{}%
\end{pgfscope}%
\begin{pgfscope}%
\pgfsys@transformshift{2.836500in}{2.261102in}%
\pgfsys@useobject{currentmarker}{}%
\end{pgfscope}%
\begin{pgfscope}%
\pgfsys@transformshift{2.821243in}{2.286376in}%
\pgfsys@useobject{currentmarker}{}%
\end{pgfscope}%
\begin{pgfscope}%
\pgfsys@transformshift{2.802464in}{2.344305in}%
\pgfsys@useobject{currentmarker}{}%
\end{pgfscope}%
\begin{pgfscope}%
\pgfsys@transformshift{2.783685in}{2.458752in}%
\pgfsys@useobject{currentmarker}{}%
\end{pgfscope}%
\begin{pgfscope}%
\pgfsys@transformshift{2.762092in}{2.598498in}%
\pgfsys@useobject{currentmarker}{}%
\end{pgfscope}%
\begin{pgfscope}%
\pgfsys@transformshift{2.743313in}{2.612873in}%
\pgfsys@useobject{currentmarker}{}%
\end{pgfscope}%
\begin{pgfscope}%
\pgfsys@transformshift{2.728291in}{2.551545in}%
\pgfsys@useobject{currentmarker}{}%
\end{pgfscope}%
\begin{pgfscope}%
\pgfsys@transformshift{2.705755in}{2.446482in}%
\pgfsys@useobject{currentmarker}{}%
\end{pgfscope}%
\begin{pgfscope}%
\pgfsys@transformshift{2.688150in}{2.366939in}%
\pgfsys@useobject{currentmarker}{}%
\end{pgfscope}%
\begin{pgfscope}%
\pgfsys@transformshift{2.669137in}{2.292856in}%
\pgfsys@useobject{currentmarker}{}%
\end{pgfscope}%
\begin{pgfscope}%
\pgfsys@transformshift{2.648481in}{2.257328in}%
\pgfsys@useobject{currentmarker}{}%
\end{pgfscope}%
\begin{pgfscope}%
\pgfsys@transformshift{2.628530in}{2.246406in}%
\pgfsys@useobject{currentmarker}{}%
\end{pgfscope}%
\begin{pgfscope}%
\pgfsys@transformshift{2.610220in}{2.245673in}%
\pgfsys@useobject{currentmarker}{}%
\end{pgfscope}%
\begin{pgfscope}%
\pgfsys@transformshift{2.593086in}{2.258451in}%
\pgfsys@useobject{currentmarker}{}%
\end{pgfscope}%
\begin{pgfscope}%
\pgfsys@transformshift{2.573604in}{2.294358in}%
\pgfsys@useobject{currentmarker}{}%
\end{pgfscope}%
\begin{pgfscope}%
\pgfsys@transformshift{2.552477in}{2.357931in}%
\pgfsys@useobject{currentmarker}{}%
\end{pgfscope}%
\begin{pgfscope}%
\pgfsys@transformshift{2.533229in}{2.494268in}%
\pgfsys@useobject{currentmarker}{}%
\end{pgfscope}%
\begin{pgfscope}%
\pgfsys@transformshift{2.516095in}{2.598781in}%
\pgfsys@useobject{currentmarker}{}%
\end{pgfscope}%
\begin{pgfscope}%
\pgfsys@transformshift{2.497082in}{2.601165in}%
\pgfsys@useobject{currentmarker}{}%
\end{pgfscope}%
\begin{pgfscope}%
\pgfsys@transformshift{2.477598in}{2.488285in}%
\pgfsys@useobject{currentmarker}{}%
\end{pgfscope}%
\begin{pgfscope}%
\pgfsys@transformshift{2.456004in}{2.363353in}%
\pgfsys@useobject{currentmarker}{}%
\end{pgfscope}%
\begin{pgfscope}%
\pgfsys@transformshift{2.437694in}{2.297204in}%
\pgfsys@useobject{currentmarker}{}%
\end{pgfscope}%
\begin{pgfscope}%
\pgfsys@transformshift{2.416804in}{2.260916in}%
\pgfsys@useobject{currentmarker}{}%
\end{pgfscope}%
\begin{pgfscope}%
\pgfsys@transformshift{2.398261in}{2.246406in}%
\pgfsys@useobject{currentmarker}{}%
\end{pgfscope}%
\begin{pgfscope}%
\pgfsys@transformshift{2.381360in}{2.247676in}%
\pgfsys@useobject{currentmarker}{}%
\end{pgfscope}%
\begin{pgfscope}%
\pgfsys@transformshift{2.360469in}{2.268608in}%
\pgfsys@useobject{currentmarker}{}%
\end{pgfscope}%
\begin{pgfscope}%
\pgfsys@transformshift{2.337465in}{2.313904in}%
\pgfsys@useobject{currentmarker}{}%
\end{pgfscope}%
\begin{pgfscope}%
\pgfsys@transformshift{2.323382in}{2.266461in}%
\pgfsys@useobject{currentmarker}{}%
\end{pgfscope}%
\begin{pgfscope}%
\pgfsys@transformshift{2.302255in}{2.303353in}%
\pgfsys@useobject{currentmarker}{}%
\end{pgfscope}%
\begin{pgfscope}%
\pgfsys@transformshift{2.280896in}{2.421080in}%
\pgfsys@useobject{currentmarker}{}%
\end{pgfscope}%
\begin{pgfscope}%
\pgfsys@transformshift{2.264934in}{2.544336in}%
\pgfsys@useobject{currentmarker}{}%
\end{pgfscope}%
\begin{pgfscope}%
\pgfsys@transformshift{2.245217in}{2.614753in}%
\pgfsys@useobject{currentmarker}{}%
\end{pgfscope}%
\begin{pgfscope}%
\pgfsys@transformshift{2.229255in}{2.578794in}%
\pgfsys@useobject{currentmarker}{}%
\end{pgfscope}%
\begin{pgfscope}%
\pgfsys@transformshift{2.205782in}{2.438381in}%
\pgfsys@useobject{currentmarker}{}%
\end{pgfscope}%
\begin{pgfscope}%
\pgfsys@transformshift{2.187238in}{2.334084in}%
\pgfsys@useobject{currentmarker}{}%
\end{pgfscope}%
\begin{pgfscope}%
\pgfsys@transformshift{2.168930in}{2.306392in}%
\pgfsys@useobject{currentmarker}{}%
\end{pgfscope}%
\begin{pgfscope}%
\pgfsys@transformshift{2.148039in}{2.263638in}%
\pgfsys@useobject{currentmarker}{}%
\end{pgfscope}%
\begin{pgfscope}%
\pgfsys@transformshift{2.130200in}{2.247171in}%
\pgfsys@useobject{currentmarker}{}%
\end{pgfscope}%
\begin{pgfscope}%
\pgfsys@transformshift{2.112359in}{2.245636in}%
\pgfsys@useobject{currentmarker}{}%
\end{pgfscope}%
\begin{pgfscope}%
\pgfsys@transformshift{2.089122in}{2.261166in}%
\pgfsys@useobject{currentmarker}{}%
\end{pgfscope}%
\begin{pgfscope}%
\pgfsys@transformshift{2.074803in}{2.285496in}%
\pgfsys@useobject{currentmarker}{}%
\end{pgfscope}%
\begin{pgfscope}%
\pgfsys@transformshift{2.053207in}{2.355795in}%
\pgfsys@useobject{currentmarker}{}%
\end{pgfscope}%
\begin{pgfscope}%
\pgfsys@transformshift{2.034194in}{2.492161in}%
\pgfsys@useobject{currentmarker}{}%
\end{pgfscope}%
\begin{pgfscope}%
\pgfsys@transformshift{2.012600in}{2.598384in}%
\pgfsys@useobject{currentmarker}{}%
\end{pgfscope}%
\begin{pgfscope}%
\pgfsys@transformshift{1.996638in}{2.617094in}%
\pgfsys@useobject{currentmarker}{}%
\end{pgfscope}%
\begin{pgfscope}%
\pgfsys@transformshift{1.975748in}{2.549296in}%
\pgfsys@useobject{currentmarker}{}%
\end{pgfscope}%
\begin{pgfscope}%
\pgfsys@transformshift{1.956266in}{2.410808in}%
\pgfsys@useobject{currentmarker}{}%
\end{pgfscope}%
\begin{pgfscope}%
\pgfsys@transformshift{1.939364in}{2.327292in}%
\pgfsys@useobject{currentmarker}{}%
\end{pgfscope}%
\begin{pgfscope}%
\pgfsys@transformshift{1.919648in}{2.278665in}%
\pgfsys@useobject{currentmarker}{}%
\end{pgfscope}%
\begin{pgfscope}%
\pgfsys@transformshift{1.898052in}{2.251748in}%
\pgfsys@useobject{currentmarker}{}%
\end{pgfscope}%
\begin{pgfscope}%
\pgfsys@transformshift{1.876222in}{2.246578in}%
\pgfsys@useobject{currentmarker}{}%
\end{pgfscope}%
\begin{pgfscope}%
\pgfsys@transformshift{1.860496in}{2.264363in}%
\pgfsys@useobject{currentmarker}{}%
\end{pgfscope}%
\begin{pgfscope}%
\pgfsys@transformshift{1.844064in}{2.258856in}%
\pgfsys@useobject{currentmarker}{}%
\end{pgfscope}%
\begin{pgfscope}%
\pgfsys@transformshift{1.822235in}{2.246719in}%
\pgfsys@useobject{currentmarker}{}%
\end{pgfscope}%
\begin{pgfscope}%
\pgfsys@transformshift{1.803691in}{2.255220in}%
\pgfsys@useobject{currentmarker}{}%
\end{pgfscope}%
\begin{pgfscope}%
\pgfsys@transformshift{1.783035in}{2.287068in}%
\pgfsys@useobject{currentmarker}{}%
\end{pgfscope}%
\begin{pgfscope}%
\pgfsys@transformshift{1.764961in}{2.334796in}%
\pgfsys@useobject{currentmarker}{}%
\end{pgfscope}%
\begin{pgfscope}%
\pgfsys@transformshift{1.747825in}{2.434099in}%
\pgfsys@useobject{currentmarker}{}%
\end{pgfscope}%
\begin{pgfscope}%
\pgfsys@transformshift{1.726935in}{2.555955in}%
\pgfsys@useobject{currentmarker}{}%
\end{pgfscope}%
\begin{pgfscope}%
\pgfsys@transformshift{1.703696in}{2.625228in}%
\pgfsys@useobject{currentmarker}{}%
\end{pgfscope}%
\begin{pgfscope}%
\pgfsys@transformshift{1.687031in}{2.588401in}%
\pgfsys@useobject{currentmarker}{}%
\end{pgfscope}%
\begin{pgfscope}%
\pgfsys@transformshift{1.670600in}{2.482813in}%
\pgfsys@useobject{currentmarker}{}%
\end{pgfscope}%
\begin{pgfscope}%
\pgfsys@transformshift{1.650647in}{2.374683in}%
\pgfsys@useobject{currentmarker}{}%
\end{pgfscope}%
\begin{pgfscope}%
\pgfsys@transformshift{1.630460in}{2.297325in}%
\pgfsys@useobject{currentmarker}{}%
\end{pgfscope}%
\begin{pgfscope}%
\pgfsys@transformshift{1.610509in}{2.268734in}%
\pgfsys@useobject{currentmarker}{}%
\end{pgfscope}%
\begin{pgfscope}%
\pgfsys@transformshift{1.593138in}{2.252463in}%
\pgfsys@useobject{currentmarker}{}%
\end{pgfscope}%
\begin{pgfscope}%
\pgfsys@transformshift{1.574125in}{2.247499in}%
\pgfsys@useobject{currentmarker}{}%
\end{pgfscope}%
\begin{pgfscope}%
\pgfsys@transformshift{1.553000in}{2.265445in}%
\pgfsys@useobject{currentmarker}{}%
\end{pgfscope}%
\begin{pgfscope}%
\pgfsys@transformshift{1.535161in}{2.280059in}%
\pgfsys@useobject{currentmarker}{}%
\end{pgfscope}%
\begin{pgfscope}%
\pgfsys@transformshift{1.512862in}{2.349848in}%
\pgfsys@useobject{currentmarker}{}%
\end{pgfscope}%
\begin{pgfscope}%
\pgfsys@transformshift{1.495726in}{2.408152in}%
\pgfsys@useobject{currentmarker}{}%
\end{pgfscope}%
\begin{pgfscope}%
\pgfsys@transformshift{1.475070in}{2.548490in}%
\pgfsys@useobject{currentmarker}{}%
\end{pgfscope}%
\begin{pgfscope}%
\pgfsys@transformshift{1.457465in}{2.621313in}%
\pgfsys@useobject{currentmarker}{}%
\end{pgfscope}%
\begin{pgfscope}%
\pgfsys@transformshift{1.439157in}{2.627768in}%
\pgfsys@useobject{currentmarker}{}%
\end{pgfscope}%
\begin{pgfscope}%
\pgfsys@transformshift{1.421552in}{2.591458in}%
\pgfsys@useobject{currentmarker}{}%
\end{pgfscope}%
\begin{pgfscope}%
\pgfsys@transformshift{1.399251in}{2.521239in}%
\pgfsys@useobject{currentmarker}{}%
\end{pgfscope}%
\begin{pgfscope}%
\pgfsys@transformshift{1.380004in}{2.392982in}%
\pgfsys@useobject{currentmarker}{}%
\end{pgfscope}%
\begin{pgfscope}%
\pgfsys@transformshift{1.364044in}{2.317310in}%
\pgfsys@useobject{currentmarker}{}%
\end{pgfscope}%
\begin{pgfscope}%
\pgfsys@transformshift{1.343387in}{2.277502in}%
\pgfsys@useobject{currentmarker}{}%
\end{pgfscope}%
\begin{pgfscope}%
\pgfsys@transformshift{1.321792in}{2.257236in}%
\pgfsys@useobject{currentmarker}{}%
\end{pgfscope}%
\begin{pgfscope}%
\pgfsys@transformshift{1.304187in}{2.248331in}%
\pgfsys@useobject{currentmarker}{}%
\end{pgfscope}%
\begin{pgfscope}%
\pgfsys@transformshift{1.284234in}{2.259669in}%
\pgfsys@useobject{currentmarker}{}%
\end{pgfscope}%
\begin{pgfscope}%
\pgfsys@transformshift{1.266866in}{2.280748in}%
\pgfsys@useobject{currentmarker}{}%
\end{pgfscope}%
\begin{pgfscope}%
\pgfsys@transformshift{1.245739in}{2.326937in}%
\pgfsys@useobject{currentmarker}{}%
\end{pgfscope}%
\begin{pgfscope}%
\pgfsys@transformshift{1.227431in}{2.388373in}%
\pgfsys@useobject{currentmarker}{}%
\end{pgfscope}%
\begin{pgfscope}%
\pgfsys@transformshift{1.207478in}{2.497201in}%
\pgfsys@useobject{currentmarker}{}%
\end{pgfscope}%
\begin{pgfscope}%
\pgfsys@transformshift{1.188701in}{2.298228in}%
\pgfsys@useobject{currentmarker}{}%
\end{pgfscope}%
\begin{pgfscope}%
\pgfsys@transformshift{1.170860in}{2.263765in}%
\pgfsys@useobject{currentmarker}{}%
\end{pgfscope}%
\begin{pgfscope}%
\pgfsys@transformshift{1.149266in}{2.248122in}%
\pgfsys@useobject{currentmarker}{}%
\end{pgfscope}%
\begin{pgfscope}%
\pgfsys@transformshift{1.130487in}{2.258913in}%
\pgfsys@useobject{currentmarker}{}%
\end{pgfscope}%
\begin{pgfscope}%
\pgfsys@transformshift{1.110770in}{2.301157in}%
\pgfsys@useobject{currentmarker}{}%
\end{pgfscope}%
\begin{pgfscope}%
\pgfsys@transformshift{1.092460in}{2.355799in}%
\pgfsys@useobject{currentmarker}{}%
\end{pgfscope}%
\begin{pgfscope}%
\pgfsys@transformshift{1.072273in}{2.498176in}%
\pgfsys@useobject{currentmarker}{}%
\end{pgfscope}%
\begin{pgfscope}%
\pgfsys@transformshift{1.054200in}{2.614004in}%
\pgfsys@useobject{currentmarker}{}%
\end{pgfscope}%
\begin{pgfscope}%
\pgfsys@transformshift{1.031900in}{2.642260in}%
\pgfsys@useobject{currentmarker}{}%
\end{pgfscope}%
\begin{pgfscope}%
\pgfsys@transformshift{1.016878in}{2.579545in}%
\pgfsys@useobject{currentmarker}{}%
\end{pgfscope}%
\begin{pgfscope}%
\pgfsys@transformshift{0.994814in}{2.426206in}%
\pgfsys@useobject{currentmarker}{}%
\end{pgfscope}%
\begin{pgfscope}%
\pgfsys@transformshift{0.977912in}{2.355500in}%
\pgfsys@useobject{currentmarker}{}%
\end{pgfscope}%
\begin{pgfscope}%
\pgfsys@transformshift{0.956084in}{2.278521in}%
\pgfsys@useobject{currentmarker}{}%
\end{pgfscope}%
\begin{pgfscope}%
\pgfsys@transformshift{0.937539in}{2.258301in}%
\pgfsys@useobject{currentmarker}{}%
\end{pgfscope}%
\begin{pgfscope}%
\pgfsys@transformshift{0.919935in}{2.251482in}%
\pgfsys@useobject{currentmarker}{}%
\end{pgfscope}%
\begin{pgfscope}%
\pgfsys@transformshift{0.899747in}{2.268487in}%
\pgfsys@useobject{currentmarker}{}%
\end{pgfscope}%
\begin{pgfscope}%
\pgfsys@transformshift{0.881205in}{2.304032in}%
\pgfsys@useobject{currentmarker}{}%
\end{pgfscope}%
\begin{pgfscope}%
\pgfsys@transformshift{0.861957in}{2.391094in}%
\pgfsys@useobject{currentmarker}{}%
\end{pgfscope}%
\begin{pgfscope}%
\pgfsys@transformshift{0.842944in}{2.483951in}%
\pgfsys@useobject{currentmarker}{}%
\end{pgfscope}%
\begin{pgfscope}%
\pgfsys@transformshift{0.824870in}{2.606897in}%
\pgfsys@useobject{currentmarker}{}%
\end{pgfscope}%
\begin{pgfscope}%
\pgfsys@transformshift{0.804448in}{2.655746in}%
\pgfsys@useobject{currentmarker}{}%
\end{pgfscope}%
\begin{pgfscope}%
\pgfsys@transformshift{0.780505in}{2.578228in}%
\pgfsys@useobject{currentmarker}{}%
\end{pgfscope}%
\begin{pgfscope}%
\pgfsys@transformshift{0.764074in}{2.468963in}%
\pgfsys@useobject{currentmarker}{}%
\end{pgfscope}%
\begin{pgfscope}%
\pgfsys@transformshift{0.746000in}{2.364978in}%
\pgfsys@useobject{currentmarker}{}%
\end{pgfscope}%
\begin{pgfscope}%
\pgfsys@transformshift{0.728864in}{2.305791in}%
\pgfsys@useobject{currentmarker}{}%
\end{pgfscope}%
\begin{pgfscope}%
\pgfsys@transformshift{0.708208in}{2.268330in}%
\pgfsys@useobject{currentmarker}{}%
\end{pgfscope}%
\begin{pgfscope}%
\pgfsys@transformshift{0.686614in}{2.253424in}%
\pgfsys@useobject{currentmarker}{}%
\end{pgfscope}%
\begin{pgfscope}%
\pgfsys@transformshift{0.668539in}{2.258107in}%
\pgfsys@useobject{currentmarker}{}%
\end{pgfscope}%
\begin{pgfscope}%
\pgfsys@transformshift{0.651639in}{2.269543in}%
\pgfsys@useobject{currentmarker}{}%
\end{pgfscope}%
\begin{pgfscope}%
\pgfsys@transformshift{0.650934in}{2.269977in}%
\pgfsys@useobject{currentmarker}{}%
\end{pgfscope}%
\begin{pgfscope}%
\pgfsys@transformshift{0.657507in}{2.260118in}%
\pgfsys@useobject{currentmarker}{}%
\end{pgfscope}%
\begin{pgfscope}%
\pgfsys@transformshift{0.675815in}{2.256549in}%
\pgfsys@useobject{currentmarker}{}%
\end{pgfscope}%
\begin{pgfscope}%
\pgfsys@transformshift{0.693891in}{2.285669in}%
\pgfsys@useobject{currentmarker}{}%
\end{pgfscope}%
\begin{pgfscope}%
\pgfsys@transformshift{0.713608in}{2.351242in}%
\pgfsys@useobject{currentmarker}{}%
\end{pgfscope}%
\begin{pgfscope}%
\pgfsys@transformshift{0.734029in}{2.504946in}%
\pgfsys@useobject{currentmarker}{}%
\end{pgfscope}%
\begin{pgfscope}%
\pgfsys@transformshift{0.751634in}{2.641508in}%
\pgfsys@useobject{currentmarker}{}%
\end{pgfscope}%
\begin{pgfscope}%
\pgfsys@transformshift{0.772525in}{2.630876in}%
\pgfsys@useobject{currentmarker}{}%
\end{pgfscope}%
\begin{pgfscope}%
\pgfsys@transformshift{0.791303in}{2.493838in}%
\pgfsys@useobject{currentmarker}{}%
\end{pgfscope}%
\begin{pgfscope}%
\pgfsys@transformshift{0.809143in}{2.349847in}%
\pgfsys@useobject{currentmarker}{}%
\end{pgfscope}%
\begin{pgfscope}%
\pgfsys@transformshift{0.831676in}{2.275978in}%
\pgfsys@useobject{currentmarker}{}%
\end{pgfscope}%
\begin{pgfscope}%
\pgfsys@transformshift{0.848107in}{2.252412in}%
\pgfsys@useobject{currentmarker}{}%
\end{pgfscope}%
\begin{pgfscope}%
\pgfsys@transformshift{0.866182in}{2.261939in}%
\pgfsys@useobject{currentmarker}{}%
\end{pgfscope}%
\begin{pgfscope}%
\pgfsys@transformshift{0.886602in}{2.321365in}%
\pgfsys@useobject{currentmarker}{}%
\end{pgfscope}%
\begin{pgfscope}%
\pgfsys@transformshift{0.906086in}{2.399947in}%
\pgfsys@useobject{currentmarker}{}%
\end{pgfscope}%
\begin{pgfscope}%
\pgfsys@transformshift{0.923220in}{2.554165in}%
\pgfsys@useobject{currentmarker}{}%
\end{pgfscope}%
\begin{pgfscope}%
\pgfsys@transformshift{0.945519in}{2.645849in}%
\pgfsys@useobject{currentmarker}{}%
\end{pgfscope}%
\begin{pgfscope}%
\pgfsys@transformshift{0.961952in}{2.579399in}%
\pgfsys@useobject{currentmarker}{}%
\end{pgfscope}%
\begin{pgfscope}%
\pgfsys@transformshift{0.984017in}{2.395195in}%
\pgfsys@useobject{currentmarker}{}%
\end{pgfscope}%
\begin{pgfscope}%
\pgfsys@transformshift{1.002559in}{2.305193in}%
\pgfsys@useobject{currentmarker}{}%
\end{pgfscope}%
\begin{pgfscope}%
\pgfsys@transformshift{1.022746in}{2.258135in}%
\pgfsys@useobject{currentmarker}{}%
\end{pgfscope}%
\begin{pgfscope}%
\pgfsys@transformshift{1.040586in}{2.250179in}%
\pgfsys@useobject{currentmarker}{}%
\end{pgfscope}%
\begin{pgfscope}%
\pgfsys@transformshift{1.057956in}{2.272785in}%
\pgfsys@useobject{currentmarker}{}%
\end{pgfscope}%
\begin{pgfscope}%
\pgfsys@transformshift{1.079786in}{2.334783in}%
\pgfsys@useobject{currentmarker}{}%
\end{pgfscope}%
\begin{pgfscope}%
\pgfsys@transformshift{1.097155in}{2.465291in}%
\pgfsys@useobject{currentmarker}{}%
\end{pgfscope}%
\begin{pgfscope}%
\pgfsys@transformshift{1.115230in}{2.607150in}%
\pgfsys@useobject{currentmarker}{}%
\end{pgfscope}%
\begin{pgfscope}%
\pgfsys@transformshift{1.135886in}{2.622328in}%
\pgfsys@useobject{currentmarker}{}%
\end{pgfscope}%
\begin{pgfscope}%
\pgfsys@transformshift{1.156308in}{2.474401in}%
\pgfsys@useobject{currentmarker}{}%
\end{pgfscope}%
\begin{pgfscope}%
\pgfsys@transformshift{1.175085in}{2.348147in}%
\pgfsys@useobject{currentmarker}{}%
\end{pgfscope}%
\begin{pgfscope}%
\pgfsys@transformshift{1.195741in}{2.273888in}%
\pgfsys@useobject{currentmarker}{}%
\end{pgfscope}%
\begin{pgfscope}%
\pgfsys@transformshift{1.214520in}{2.249845in}%
\pgfsys@useobject{currentmarker}{}%
\end{pgfscope}%
\begin{pgfscope}%
\pgfsys@transformshift{1.232125in}{2.254268in}%
\pgfsys@useobject{currentmarker}{}%
\end{pgfscope}%
\begin{pgfscope}%
\pgfsys@transformshift{1.252547in}{2.285245in}%
\pgfsys@useobject{currentmarker}{}%
\end{pgfscope}%
\begin{pgfscope}%
\pgfsys@transformshift{1.271794in}{2.350901in}%
\pgfsys@useobject{currentmarker}{}%
\end{pgfscope}%
\begin{pgfscope}%
\pgfsys@transformshift{1.288459in}{2.488427in}%
\pgfsys@useobject{currentmarker}{}%
\end{pgfscope}%
\begin{pgfscope}%
\pgfsys@transformshift{1.309821in}{2.616251in}%
\pgfsys@useobject{currentmarker}{}%
\end{pgfscope}%
\begin{pgfscope}%
\pgfsys@transformshift{1.332120in}{2.596792in}%
\pgfsys@useobject{currentmarker}{}%
\end{pgfscope}%
\begin{pgfscope}%
\pgfsys@transformshift{1.348785in}{2.468615in}%
\pgfsys@useobject{currentmarker}{}%
\end{pgfscope}%
\begin{pgfscope}%
\pgfsys@transformshift{1.366390in}{2.351493in}%
\pgfsys@useobject{currentmarker}{}%
\end{pgfscope}%
\begin{pgfscope}%
\pgfsys@transformshift{1.387517in}{2.277756in}%
\pgfsys@useobject{currentmarker}{}%
\end{pgfscope}%
\begin{pgfscope}%
\pgfsys@transformshift{1.406294in}{2.252350in}%
\pgfsys@useobject{currentmarker}{}%
\end{pgfscope}%
\begin{pgfscope}%
\pgfsys@transformshift{1.426481in}{2.249157in}%
\pgfsys@useobject{currentmarker}{}%
\end{pgfscope}%
\begin{pgfscope}%
\pgfsys@transformshift{1.444086in}{2.270200in}%
\pgfsys@useobject{currentmarker}{}%
\end{pgfscope}%
\begin{pgfscope}%
\pgfsys@transformshift{1.462864in}{2.314635in}%
\pgfsys@useobject{currentmarker}{}%
\end{pgfscope}%
\begin{pgfscope}%
\pgfsys@transformshift{1.482581in}{2.435020in}%
\pgfsys@useobject{currentmarker}{}%
\end{pgfscope}%
\begin{pgfscope}%
\pgfsys@transformshift{1.500889in}{2.578002in}%
\pgfsys@useobject{currentmarker}{}%
\end{pgfscope}%
\begin{pgfscope}%
\pgfsys@transformshift{1.522250in}{2.624230in}%
\pgfsys@useobject{currentmarker}{}%
\end{pgfscope}%
\begin{pgfscope}%
\pgfsys@transformshift{1.540324in}{2.584490in}%
\pgfsys@useobject{currentmarker}{}%
\end{pgfscope}%
\begin{pgfscope}%
\pgfsys@transformshift{1.557226in}{2.452213in}%
\pgfsys@useobject{currentmarker}{}%
\end{pgfscope}%
\begin{pgfscope}%
\pgfsys@transformshift{1.579993in}{2.325665in}%
\pgfsys@useobject{currentmarker}{}%
\end{pgfscope}%
\begin{pgfscope}%
\pgfsys@transformshift{1.597129in}{2.273593in}%
\pgfsys@useobject{currentmarker}{}%
\end{pgfscope}%
\begin{pgfscope}%
\pgfsys@transformshift{1.617786in}{2.253885in}%
\pgfsys@useobject{currentmarker}{}%
\end{pgfscope}%
\begin{pgfscope}%
\pgfsys@transformshift{1.635859in}{2.246212in}%
\pgfsys@useobject{currentmarker}{}%
\end{pgfscope}%
\begin{pgfscope}%
\pgfsys@transformshift{1.659332in}{2.269037in}%
\pgfsys@useobject{currentmarker}{}%
\end{pgfscope}%
\begin{pgfscope}%
\pgfsys@transformshift{1.674120in}{2.299390in}%
\pgfsys@useobject{currentmarker}{}%
\end{pgfscope}%
\begin{pgfscope}%
\pgfsys@transformshift{1.694307in}{2.381365in}%
\pgfsys@useobject{currentmarker}{}%
\end{pgfscope}%
\begin{pgfscope}%
\pgfsys@transformshift{1.712850in}{2.517395in}%
\pgfsys@useobject{currentmarker}{}%
\end{pgfscope}%
\begin{pgfscope}%
\pgfsys@transformshift{1.732568in}{2.611817in}%
\pgfsys@useobject{currentmarker}{}%
\end{pgfscope}%
\begin{pgfscope}%
\pgfsys@transformshift{1.751582in}{2.593654in}%
\pgfsys@useobject{currentmarker}{}%
\end{pgfscope}%
\begin{pgfscope}%
\pgfsys@transformshift{1.770124in}{2.487374in}%
\pgfsys@useobject{currentmarker}{}%
\end{pgfscope}%
\begin{pgfscope}%
\pgfsys@transformshift{1.792189in}{2.381286in}%
\pgfsys@useobject{currentmarker}{}%
\end{pgfscope}%
\begin{pgfscope}%
\pgfsys@transformshift{1.809794in}{2.309897in}%
\pgfsys@useobject{currentmarker}{}%
\end{pgfscope}%
\begin{pgfscope}%
\pgfsys@transformshift{1.827633in}{2.275017in}%
\pgfsys@useobject{currentmarker}{}%
\end{pgfscope}%
\begin{pgfscope}%
\pgfsys@transformshift{1.849697in}{2.249539in}%
\pgfsys@useobject{currentmarker}{}%
\end{pgfscope}%
\begin{pgfscope}%
\pgfsys@transformshift{1.868007in}{2.246785in}%
\pgfsys@useobject{currentmarker}{}%
\end{pgfscope}%
\begin{pgfscope}%
\pgfsys@transformshift{1.888193in}{2.260070in}%
\pgfsys@useobject{currentmarker}{}%
\end{pgfscope}%
\begin{pgfscope}%
\pgfsys@transformshift{1.903686in}{2.287785in}%
\pgfsys@useobject{currentmarker}{}%
\end{pgfscope}%
\begin{pgfscope}%
\pgfsys@transformshift{1.923873in}{2.349630in}%
\pgfsys@useobject{currentmarker}{}%
\end{pgfscope}%
\begin{pgfscope}%
\pgfsys@transformshift{1.944764in}{2.469094in}%
\pgfsys@useobject{currentmarker}{}%
\end{pgfscope}%
\begin{pgfscope}%
\pgfsys@transformshift{1.965889in}{2.280653in}%
\pgfsys@useobject{currentmarker}{}%
\end{pgfscope}%
\begin{pgfscope}%
\pgfsys@transformshift{1.982085in}{2.337041in}%
\pgfsys@useobject{currentmarker}{}%
\end{pgfscope}%
\begin{pgfscope}%
\pgfsys@transformshift{2.002272in}{2.478127in}%
\pgfsys@useobject{currentmarker}{}%
\end{pgfscope}%
\begin{pgfscope}%
\pgfsys@transformshift{2.020346in}{2.591588in}%
\pgfsys@useobject{currentmarker}{}%
\end{pgfscope}%
\begin{pgfscope}%
\pgfsys@transformshift{2.037246in}{2.613164in}%
\pgfsys@useobject{currentmarker}{}%
\end{pgfscope}%
\begin{pgfscope}%
\pgfsys@transformshift{2.059076in}{2.499841in}%
\pgfsys@useobject{currentmarker}{}%
\end{pgfscope}%
\begin{pgfscope}%
\pgfsys@transformshift{2.076915in}{2.372498in}%
\pgfsys@useobject{currentmarker}{}%
\end{pgfscope}%
\begin{pgfscope}%
\pgfsys@transformshift{2.097337in}{2.285276in}%
\pgfsys@useobject{currentmarker}{}%
\end{pgfscope}%
\begin{pgfscope}%
\pgfsys@transformshift{2.115647in}{2.257124in}%
\pgfsys@useobject{currentmarker}{}%
\end{pgfscope}%
\begin{pgfscope}%
\pgfsys@transformshift{2.137477in}{2.244998in}%
\pgfsys@useobject{currentmarker}{}%
\end{pgfscope}%
\begin{pgfscope}%
\pgfsys@transformshift{2.154376in}{2.254684in}%
\pgfsys@useobject{currentmarker}{}%
\end{pgfscope}%
\begin{pgfscope}%
\pgfsys@transformshift{2.171981in}{2.282087in}%
\pgfsys@useobject{currentmarker}{}%
\end{pgfscope}%
\begin{pgfscope}%
\pgfsys@transformshift{2.193577in}{2.345022in}%
\pgfsys@useobject{currentmarker}{}%
\end{pgfscope}%
\begin{pgfscope}%
\pgfsys@transformshift{2.210945in}{2.455669in}%
\pgfsys@useobject{currentmarker}{}%
\end{pgfscope}%
\begin{pgfscope}%
\pgfsys@transformshift{2.229490in}{2.576293in}%
\pgfsys@useobject{currentmarker}{}%
\end{pgfscope}%
\begin{pgfscope}%
\pgfsys@transformshift{2.251320in}{2.611445in}%
\pgfsys@useobject{currentmarker}{}%
\end{pgfscope}%
\begin{pgfscope}%
\pgfsys@transformshift{2.272211in}{2.513505in}%
\pgfsys@useobject{currentmarker}{}%
\end{pgfscope}%
\begin{pgfscope}%
\pgfsys@transformshift{2.290989in}{2.373494in}%
\pgfsys@useobject{currentmarker}{}%
\end{pgfscope}%
\begin{pgfscope}%
\pgfsys@transformshift{2.307420in}{2.299576in}%
\pgfsys@useobject{currentmarker}{}%
\end{pgfscope}%
\begin{pgfscope}%
\pgfsys@transformshift{2.328780in}{2.261111in}%
\pgfsys@useobject{currentmarker}{}%
\end{pgfscope}%
\begin{pgfscope}%
\pgfsys@transformshift{2.347090in}{2.246886in}%
\pgfsys@useobject{currentmarker}{}%
\end{pgfscope}%
\begin{pgfscope}%
\pgfsys@transformshift{2.364694in}{2.247655in}%
\pgfsys@useobject{currentmarker}{}%
\end{pgfscope}%
\begin{pgfscope}%
\pgfsys@transformshift{2.386525in}{2.269194in}%
\pgfsys@useobject{currentmarker}{}%
\end{pgfscope}%
\begin{pgfscope}%
\pgfsys@transformshift{2.405772in}{2.292418in}%
\pgfsys@useobject{currentmarker}{}%
\end{pgfscope}%
\begin{pgfscope}%
\pgfsys@transformshift{2.423611in}{2.281627in}%
\pgfsys@useobject{currentmarker}{}%
\end{pgfscope}%
\begin{pgfscope}%
\pgfsys@transformshift{2.444033in}{2.326353in}%
\pgfsys@useobject{currentmarker}{}%
\end{pgfscope}%
\begin{pgfscope}%
\pgfsys@transformshift{2.462810in}{2.408697in}%
\pgfsys@useobject{currentmarker}{}%
\end{pgfscope}%
\begin{pgfscope}%
\pgfsys@transformshift{2.482058in}{2.558977in}%
\pgfsys@useobject{currentmarker}{}%
\end{pgfscope}%
\begin{pgfscope}%
\pgfsys@transformshift{2.503185in}{2.613210in}%
\pgfsys@useobject{currentmarker}{}%
\end{pgfscope}%
\begin{pgfscope}%
\pgfsys@transformshift{2.521493in}{2.546593in}%
\pgfsys@useobject{currentmarker}{}%
\end{pgfscope}%
\begin{pgfscope}%
\pgfsys@transformshift{2.539098in}{2.407387in}%
\pgfsys@useobject{currentmarker}{}%
\end{pgfscope}%
\begin{pgfscope}%
\pgfsys@transformshift{2.557408in}{2.316171in}%
\pgfsys@useobject{currentmarker}{}%
\end{pgfscope}%
\begin{pgfscope}%
\pgfsys@transformshift{2.578064in}{2.267668in}%
\pgfsys@useobject{currentmarker}{}%
\end{pgfscope}%
\begin{pgfscope}%
\pgfsys@transformshift{2.596606in}{2.248151in}%
\pgfsys@useobject{currentmarker}{}%
\end{pgfscope}%
\begin{pgfscope}%
\pgfsys@transformshift{2.616325in}{2.247451in}%
\pgfsys@useobject{currentmarker}{}%
\end{pgfscope}%
\begin{pgfscope}%
\pgfsys@transformshift{2.635336in}{2.266973in}%
\pgfsys@useobject{currentmarker}{}%
\end{pgfscope}%
\begin{pgfscope}%
\pgfsys@transformshift{2.653880in}{2.300249in}%
\pgfsys@useobject{currentmarker}{}%
\end{pgfscope}%
\begin{pgfscope}%
\pgfsys@transformshift{2.674537in}{2.375959in}%
\pgfsys@useobject{currentmarker}{}%
\end{pgfscope}%
\begin{pgfscope}%
\pgfsys@transformshift{2.695427in}{2.506390in}%
\pgfsys@useobject{currentmarker}{}%
\end{pgfscope}%
\begin{pgfscope}%
\pgfsys@transformshift{2.713503in}{2.565527in}%
\pgfsys@useobject{currentmarker}{}%
\end{pgfscope}%
\begin{pgfscope}%
\pgfsys@transformshift{2.730871in}{2.607203in}%
\pgfsys@useobject{currentmarker}{}%
\end{pgfscope}%
\begin{pgfscope}%
\pgfsys@transformshift{2.751293in}{2.587733in}%
\pgfsys@useobject{currentmarker}{}%
\end{pgfscope}%
\begin{pgfscope}%
\pgfsys@transformshift{2.769837in}{2.517083in}%
\pgfsys@useobject{currentmarker}{}%
\end{pgfscope}%
\begin{pgfscope}%
\pgfsys@transformshift{2.788145in}{2.377674in}%
\pgfsys@useobject{currentmarker}{}%
\end{pgfscope}%
\begin{pgfscope}%
\pgfsys@transformshift{2.810444in}{2.289973in}%
\pgfsys@useobject{currentmarker}{}%
\end{pgfscope}%
\begin{pgfscope}%
\pgfsys@transformshift{2.828049in}{2.268575in}%
\pgfsys@useobject{currentmarker}{}%
\end{pgfscope}%
\begin{pgfscope}%
\pgfsys@transformshift{2.851288in}{2.246047in}%
\pgfsys@useobject{currentmarker}{}%
\end{pgfscope}%
\begin{pgfscope}%
\pgfsys@transformshift{2.866544in}{2.249223in}%
\pgfsys@useobject{currentmarker}{}%
\end{pgfscope}%
\begin{pgfscope}%
\pgfsys@transformshift{2.884854in}{2.258816in}%
\pgfsys@useobject{currentmarker}{}%
\end{pgfscope}%
\begin{pgfscope}%
\pgfsys@transformshift{2.905745in}{2.295664in}%
\pgfsys@useobject{currentmarker}{}%
\end{pgfscope}%
\begin{pgfscope}%
\pgfsys@transformshift{2.925227in}{2.355163in}%
\pgfsys@useobject{currentmarker}{}%
\end{pgfscope}%
\begin{pgfscope}%
\pgfsys@transformshift{2.942129in}{2.472270in}%
\pgfsys@useobject{currentmarker}{}%
\end{pgfscope}%
\begin{pgfscope}%
\pgfsys@transformshift{2.963019in}{2.608204in}%
\pgfsys@useobject{currentmarker}{}%
\end{pgfscope}%
\begin{pgfscope}%
\pgfsys@transformshift{2.982267in}{2.613562in}%
\pgfsys@useobject{currentmarker}{}%
\end{pgfscope}%
\begin{pgfscope}%
\pgfsys@transformshift{2.999167in}{2.534657in}%
\pgfsys@useobject{currentmarker}{}%
\end{pgfscope}%
\begin{pgfscope}%
\pgfsys@transformshift{3.021702in}{2.383526in}%
\pgfsys@useobject{currentmarker}{}%
\end{pgfscope}%
\begin{pgfscope}%
\pgfsys@transformshift{3.039072in}{2.310624in}%
\pgfsys@useobject{currentmarker}{}%
\end{pgfscope}%
\begin{pgfscope}%
\pgfsys@transformshift{3.059728in}{2.270149in}%
\pgfsys@useobject{currentmarker}{}%
\end{pgfscope}%
\begin{pgfscope}%
\pgfsys@transformshift{3.077568in}{2.251565in}%
\pgfsys@useobject{currentmarker}{}%
\end{pgfscope}%
\begin{pgfscope}%
\pgfsys@transformshift{3.096815in}{2.246988in}%
\pgfsys@useobject{currentmarker}{}%
\end{pgfscope}%
\begin{pgfscope}%
\pgfsys@transformshift{3.118646in}{2.263896in}%
\pgfsys@useobject{currentmarker}{}%
\end{pgfscope}%
\begin{pgfscope}%
\pgfsys@transformshift{3.135545in}{2.291525in}%
\pgfsys@useobject{currentmarker}{}%
\end{pgfscope}%
\begin{pgfscope}%
\pgfsys@transformshift{3.154793in}{2.342012in}%
\pgfsys@useobject{currentmarker}{}%
\end{pgfscope}%
\begin{pgfscope}%
\pgfsys@transformshift{3.174509in}{2.436220in}%
\pgfsys@useobject{currentmarker}{}%
\end{pgfscope}%
\begin{pgfscope}%
\pgfsys@transformshift{3.191645in}{2.525721in}%
\pgfsys@useobject{currentmarker}{}%
\end{pgfscope}%
\begin{pgfscope}%
\pgfsys@transformshift{3.213475in}{2.607238in}%
\pgfsys@useobject{currentmarker}{}%
\end{pgfscope}%
\begin{pgfscope}%
\pgfsys@transformshift{3.230375in}{2.587341in}%
\pgfsys@useobject{currentmarker}{}%
\end{pgfscope}%
\begin{pgfscope}%
\pgfsys@transformshift{3.249388in}{2.623212in}%
\pgfsys@useobject{currentmarker}{}%
\end{pgfscope}%
\begin{pgfscope}%
\pgfsys@transformshift{3.266759in}{2.578951in}%
\pgfsys@useobject{currentmarker}{}%
\end{pgfscope}%
\begin{pgfscope}%
\pgfsys@transformshift{3.289292in}{2.428462in}%
\pgfsys@useobject{currentmarker}{}%
\end{pgfscope}%
\begin{pgfscope}%
\pgfsys@transformshift{3.309245in}{2.337732in}%
\pgfsys@useobject{currentmarker}{}%
\end{pgfscope}%
\begin{pgfscope}%
\pgfsys@transformshift{3.327319in}{2.293116in}%
\pgfsys@useobject{currentmarker}{}%
\end{pgfscope}%
\begin{pgfscope}%
\pgfsys@transformshift{3.345863in}{2.265478in}%
\pgfsys@useobject{currentmarker}{}%
\end{pgfscope}%
\begin{pgfscope}%
\pgfsys@transformshift{3.364642in}{2.247386in}%
\pgfsys@useobject{currentmarker}{}%
\end{pgfscope}%
\begin{pgfscope}%
\pgfsys@transformshift{3.382716in}{2.251176in}%
\pgfsys@useobject{currentmarker}{}%
\end{pgfscope}%
\begin{pgfscope}%
\pgfsys@transformshift{3.404780in}{2.263517in}%
\pgfsys@useobject{currentmarker}{}%
\end{pgfscope}%
\begin{pgfscope}%
\pgfsys@transformshift{3.423323in}{2.296206in}%
\pgfsys@useobject{currentmarker}{}%
\end{pgfscope}%
\begin{pgfscope}%
\pgfsys@transformshift{3.445387in}{2.366861in}%
\pgfsys@useobject{currentmarker}{}%
\end{pgfscope}%
\begin{pgfscope}%
\pgfsys@transformshift{3.460880in}{2.449458in}%
\pgfsys@useobject{currentmarker}{}%
\end{pgfscope}%
\begin{pgfscope}%
\pgfsys@transformshift{3.482711in}{2.573997in}%
\pgfsys@useobject{currentmarker}{}%
\end{pgfscope}%
\begin{pgfscope}%
\pgfsys@transformshift{3.498436in}{2.631168in}%
\pgfsys@useobject{currentmarker}{}%
\end{pgfscope}%
\begin{pgfscope}%
\pgfsys@transformshift{3.519329in}{2.619184in}%
\pgfsys@useobject{currentmarker}{}%
\end{pgfscope}%
\begin{pgfscope}%
\pgfsys@transformshift{3.538576in}{2.546067in}%
\pgfsys@useobject{currentmarker}{}%
\end{pgfscope}%
\begin{pgfscope}%
\pgfsys@transformshift{3.557353in}{2.434515in}%
\pgfsys@useobject{currentmarker}{}%
\end{pgfscope}%
\begin{pgfscope}%
\pgfsys@transformshift{3.577072in}{2.330532in}%
\pgfsys@useobject{currentmarker}{}%
\end{pgfscope}%
\begin{pgfscope}%
\pgfsys@transformshift{3.594911in}{2.551143in}%
\pgfsys@useobject{currentmarker}{}%
\end{pgfscope}%
\begin{pgfscope}%
\pgfsys@transformshift{3.615567in}{2.633374in}%
\pgfsys@useobject{currentmarker}{}%
\end{pgfscope}%
\begin{pgfscope}%
\pgfsys@transformshift{3.633875in}{2.613539in}%
\pgfsys@useobject{currentmarker}{}%
\end{pgfscope}%
\begin{pgfscope}%
\pgfsys@transformshift{3.653828in}{2.485459in}%
\pgfsys@useobject{currentmarker}{}%
\end{pgfscope}%
\begin{pgfscope}%
\pgfsys@transformshift{3.673310in}{2.360138in}%
\pgfsys@useobject{currentmarker}{}%
\end{pgfscope}%
\begin{pgfscope}%
\pgfsys@transformshift{3.691854in}{2.293226in}%
\pgfsys@useobject{currentmarker}{}%
\end{pgfscope}%
\begin{pgfscope}%
\pgfsys@transformshift{3.711102in}{2.260785in}%
\pgfsys@useobject{currentmarker}{}%
\end{pgfscope}%
\begin{pgfscope}%
\pgfsys@transformshift{3.733636in}{2.248472in}%
\pgfsys@useobject{currentmarker}{}%
\end{pgfscope}%
\begin{pgfscope}%
\pgfsys@transformshift{3.750535in}{2.259497in}%
\pgfsys@useobject{currentmarker}{}%
\end{pgfscope}%
\begin{pgfscope}%
\pgfsys@transformshift{3.768611in}{2.285904in}%
\pgfsys@useobject{currentmarker}{}%
\end{pgfscope}%
\begin{pgfscope}%
\pgfsys@transformshift{3.787858in}{2.336136in}%
\pgfsys@useobject{currentmarker}{}%
\end{pgfscope}%
\begin{pgfscope}%
\pgfsys@transformshift{3.809452in}{2.436735in}%
\pgfsys@useobject{currentmarker}{}%
\end{pgfscope}%
\begin{pgfscope}%
\pgfsys@transformshift{3.826119in}{2.550575in}%
\pgfsys@useobject{currentmarker}{}%
\end{pgfscope}%
\begin{pgfscope}%
\pgfsys@transformshift{3.851470in}{2.642896in}%
\pgfsys@useobject{currentmarker}{}%
\end{pgfscope}%
\begin{pgfscope}%
\pgfsys@transformshift{3.867432in}{2.616631in}%
\pgfsys@useobject{currentmarker}{}%
\end{pgfscope}%
\begin{pgfscope}%
\pgfsys@transformshift{3.886445in}{2.511592in}%
\pgfsys@useobject{currentmarker}{}%
\end{pgfscope}%
\begin{pgfscope}%
\pgfsys@transformshift{3.901936in}{2.410598in}%
\pgfsys@useobject{currentmarker}{}%
\end{pgfscope}%
\begin{pgfscope}%
\pgfsys@transformshift{3.922592in}{2.323717in}%
\pgfsys@useobject{currentmarker}{}%
\end{pgfscope}%
\begin{pgfscope}%
\pgfsys@transformshift{3.941371in}{2.280379in}%
\pgfsys@useobject{currentmarker}{}%
\end{pgfscope}%
\begin{pgfscope}%
\pgfsys@transformshift{3.962967in}{2.256429in}%
\pgfsys@useobject{currentmarker}{}%
\end{pgfscope}%
\begin{pgfscope}%
\pgfsys@transformshift{3.982449in}{2.249789in}%
\pgfsys@useobject{currentmarker}{}%
\end{pgfscope}%
\begin{pgfscope}%
\pgfsys@transformshift{4.000991in}{2.261650in}%
\pgfsys@useobject{currentmarker}{}%
\end{pgfscope}%
\begin{pgfscope}%
\pgfsys@transformshift{4.019536in}{2.288952in}%
\pgfsys@useobject{currentmarker}{}%
\end{pgfscope}%
\begin{pgfscope}%
\pgfsys@transformshift{4.038783in}{2.336868in}%
\pgfsys@useobject{currentmarker}{}%
\end{pgfscope}%
\begin{pgfscope}%
\pgfsys@transformshift{4.057328in}{2.410072in}%
\pgfsys@useobject{currentmarker}{}%
\end{pgfscope}%
\begin{pgfscope}%
\pgfsys@transformshift{4.076105in}{2.502927in}%
\pgfsys@useobject{currentmarker}{}%
\end{pgfscope}%
\begin{pgfscope}%
\pgfsys@transformshift{4.096527in}{2.621959in}%
\pgfsys@useobject{currentmarker}{}%
\end{pgfscope}%
\begin{pgfscope}%
\pgfsys@transformshift{4.115071in}{2.652428in}%
\pgfsys@useobject{currentmarker}{}%
\end{pgfscope}%
\begin{pgfscope}%
\pgfsys@transformshift{4.136430in}{2.578033in}%
\pgfsys@useobject{currentmarker}{}%
\end{pgfscope}%
\begin{pgfscope}%
\pgfsys@transformshift{4.155678in}{2.530081in}%
\pgfsys@useobject{currentmarker}{}%
\end{pgfscope}%
\begin{pgfscope}%
\pgfsys@transformshift{4.177274in}{2.400623in}%
\pgfsys@useobject{currentmarker}{}%
\end{pgfscope}%
\begin{pgfscope}%
\pgfsys@transformshift{4.193236in}{2.331682in}%
\pgfsys@useobject{currentmarker}{}%
\end{pgfscope}%
\begin{pgfscope}%
\pgfsys@transformshift{4.209198in}{2.288608in}%
\pgfsys@useobject{currentmarker}{}%
\end{pgfscope}%
\begin{pgfscope}%
\pgfsys@transformshift{4.230793in}{2.263448in}%
\pgfsys@useobject{currentmarker}{}%
\end{pgfscope}%
\begin{pgfscope}%
\pgfsys@transformshift{4.249805in}{2.252414in}%
\pgfsys@useobject{currentmarker}{}%
\end{pgfscope}%
\begin{pgfscope}%
\pgfsys@transformshift{4.269052in}{2.265016in}%
\pgfsys@useobject{currentmarker}{}%
\end{pgfscope}%
\begin{pgfscope}%
\pgfsys@transformshift{4.290179in}{2.290905in}%
\pgfsys@useobject{currentmarker}{}%
\end{pgfscope}%
\begin{pgfscope}%
\pgfsys@transformshift{4.306376in}{2.328745in}%
\pgfsys@useobject{currentmarker}{}%
\end{pgfscope}%
\begin{pgfscope}%
\pgfsys@transformshift{4.330318in}{2.439269in}%
\pgfsys@useobject{currentmarker}{}%
\end{pgfscope}%
\begin{pgfscope}%
\pgfsys@transformshift{4.346279in}{2.545467in}%
\pgfsys@useobject{currentmarker}{}%
\end{pgfscope}%
\begin{pgfscope}%
\pgfsys@transformshift{4.366701in}{2.645505in}%
\pgfsys@useobject{currentmarker}{}%
\end{pgfscope}%
\begin{pgfscope}%
\pgfsys@transformshift{4.385244in}{2.667183in}%
\pgfsys@useobject{currentmarker}{}%
\end{pgfscope}%
\begin{pgfscope}%
\pgfsys@transformshift{4.404023in}{2.649730in}%
\pgfsys@useobject{currentmarker}{}%
\end{pgfscope}%
\begin{pgfscope}%
\pgfsys@transformshift{4.422331in}{2.563822in}%
\pgfsys@useobject{currentmarker}{}%
\end{pgfscope}%
\begin{pgfscope}%
\pgfsys@transformshift{4.444632in}{2.437913in}%
\pgfsys@useobject{currentmarker}{}%
\end{pgfscope}%
\begin{pgfscope}%
\pgfsys@transformshift{4.461297in}{2.368856in}%
\pgfsys@useobject{currentmarker}{}%
\end{pgfscope}%
\begin{pgfscope}%
\pgfsys@transformshift{4.479370in}{2.307875in}%
\pgfsys@useobject{currentmarker}{}%
\end{pgfscope}%
\begin{pgfscope}%
\pgfsys@transformshift{4.479841in}{2.307590in}%
\pgfsys@useobject{currentmarker}{}%
\end{pgfscope}%
\begin{pgfscope}%
\pgfsys@transformshift{4.473973in}{2.329457in}%
\pgfsys@useobject{currentmarker}{}%
\end{pgfscope}%
\begin{pgfscope}%
\pgfsys@transformshift{4.454489in}{2.452115in}%
\pgfsys@useobject{currentmarker}{}%
\end{pgfscope}%
\begin{pgfscope}%
\pgfsys@transformshift{4.434538in}{2.615966in}%
\pgfsys@useobject{currentmarker}{}%
\end{pgfscope}%
\begin{pgfscope}%
\pgfsys@transformshift{4.417402in}{2.667229in}%
\pgfsys@useobject{currentmarker}{}%
\end{pgfscope}%
\begin{pgfscope}%
\pgfsys@transformshift{4.398859in}{2.562600in}%
\pgfsys@useobject{currentmarker}{}%
\end{pgfscope}%
\begin{pgfscope}%
\pgfsys@transformshift{4.377733in}{2.379923in}%
\pgfsys@useobject{currentmarker}{}%
\end{pgfscope}%
\begin{pgfscope}%
\pgfsys@transformshift{4.360128in}{2.301369in}%
\pgfsys@useobject{currentmarker}{}%
\end{pgfscope}%
\begin{pgfscope}%
\pgfsys@transformshift{4.339003in}{2.257549in}%
\pgfsys@useobject{currentmarker}{}%
\end{pgfscope}%
\begin{pgfscope}%
\pgfsys@transformshift{4.320929in}{2.257437in}%
\pgfsys@useobject{currentmarker}{}%
\end{pgfscope}%
\begin{pgfscope}%
\pgfsys@transformshift{4.302150in}{2.293126in}%
\pgfsys@useobject{currentmarker}{}%
\end{pgfscope}%
\begin{pgfscope}%
\pgfsys@transformshift{4.281025in}{2.395804in}%
\pgfsys@useobject{currentmarker}{}%
\end{pgfscope}%
\begin{pgfscope}%
\pgfsys@transformshift{4.261541in}{2.550886in}%
\pgfsys@useobject{currentmarker}{}%
\end{pgfscope}%
\begin{pgfscope}%
\pgfsys@transformshift{4.243233in}{2.652050in}%
\pgfsys@useobject{currentmarker}{}%
\end{pgfscope}%
\begin{pgfscope}%
\pgfsys@transformshift{4.225394in}{2.611448in}%
\pgfsys@useobject{currentmarker}{}%
\end{pgfscope}%
\begin{pgfscope}%
\pgfsys@transformshift{4.203095in}{2.418901in}%
\pgfsys@useobject{currentmarker}{}%
\end{pgfscope}%
\begin{pgfscope}%
\pgfsys@transformshift{4.186899in}{2.332038in}%
\pgfsys@useobject{currentmarker}{}%
\end{pgfscope}%
\begin{pgfscope}%
\pgfsys@transformshift{4.168354in}{2.276072in}%
\pgfsys@useobject{currentmarker}{}%
\end{pgfscope}%
\begin{pgfscope}%
\pgfsys@transformshift{4.147229in}{2.249562in}%
\pgfsys@useobject{currentmarker}{}%
\end{pgfscope}%
\begin{pgfscope}%
\pgfsys@transformshift{4.128919in}{2.266564in}%
\pgfsys@useobject{currentmarker}{}%
\end{pgfscope}%
\begin{pgfscope}%
\pgfsys@transformshift{4.109203in}{2.317811in}%
\pgfsys@useobject{currentmarker}{}%
\end{pgfscope}%
\begin{pgfscope}%
\pgfsys@transformshift{4.090189in}{2.427228in}%
\pgfsys@useobject{currentmarker}{}%
\end{pgfscope}%
\begin{pgfscope}%
\pgfsys@transformshift{4.068830in}{2.608679in}%
\pgfsys@useobject{currentmarker}{}%
\end{pgfscope}%
\begin{pgfscope}%
\pgfsys@transformshift{4.050520in}{2.641184in}%
\pgfsys@useobject{currentmarker}{}%
\end{pgfscope}%
\begin{pgfscope}%
\pgfsys@transformshift{4.031741in}{2.533889in}%
\pgfsys@useobject{currentmarker}{}%
\end{pgfscope}%
\begin{pgfscope}%
\pgfsys@transformshift{4.014842in}{2.381289in}%
\pgfsys@useobject{currentmarker}{}%
\end{pgfscope}%
\begin{pgfscope}%
\pgfsys@transformshift{3.992777in}{2.291673in}%
\pgfsys@useobject{currentmarker}{}%
\end{pgfscope}%
\begin{pgfscope}%
\pgfsys@transformshift{3.974703in}{2.254536in}%
\pgfsys@useobject{currentmarker}{}%
\end{pgfscope}%
\begin{pgfscope}%
\pgfsys@transformshift{3.954282in}{2.250969in}%
\pgfsys@useobject{currentmarker}{}%
\end{pgfscope}%
\begin{pgfscope}%
\pgfsys@transformshift{3.938320in}{2.271499in}%
\pgfsys@useobject{currentmarker}{}%
\end{pgfscope}%
\begin{pgfscope}%
\pgfsys@transformshift{3.916490in}{2.345518in}%
\pgfsys@useobject{currentmarker}{}%
\end{pgfscope}%
\begin{pgfscope}%
\pgfsys@transformshift{3.898181in}{2.469070in}%
\pgfsys@useobject{currentmarker}{}%
\end{pgfscope}%
\begin{pgfscope}%
\pgfsys@transformshift{3.878934in}{2.610178in}%
\pgfsys@useobject{currentmarker}{}%
\end{pgfscope}%
\begin{pgfscope}%
\pgfsys@transformshift{3.860155in}{2.624678in}%
\pgfsys@useobject{currentmarker}{}%
\end{pgfscope}%
\begin{pgfscope}%
\pgfsys@transformshift{3.839030in}{2.478361in}%
\pgfsys@useobject{currentmarker}{}%
\end{pgfscope}%
\begin{pgfscope}%
\pgfsys@transformshift{3.819782in}{2.342315in}%
\pgfsys@useobject{currentmarker}{}%
\end{pgfscope}%
\begin{pgfscope}%
\pgfsys@transformshift{3.802177in}{2.280961in}%
\pgfsys@useobject{currentmarker}{}%
\end{pgfscope}%
\begin{pgfscope}%
\pgfsys@transformshift{3.782693in}{2.252665in}%
\pgfsys@useobject{currentmarker}{}%
\end{pgfscope}%
\begin{pgfscope}%
\pgfsys@transformshift{3.762037in}{2.250661in}%
\pgfsys@useobject{currentmarker}{}%
\end{pgfscope}%
\begin{pgfscope}%
\pgfsys@transformshift{3.741852in}{2.281715in}%
\pgfsys@useobject{currentmarker}{}%
\end{pgfscope}%
\begin{pgfscope}%
\pgfsys@transformshift{3.724716in}{2.344575in}%
\pgfsys@useobject{currentmarker}{}%
\end{pgfscope}%
\begin{pgfscope}%
\pgfsys@transformshift{3.706172in}{2.479387in}%
\pgfsys@useobject{currentmarker}{}%
\end{pgfscope}%
\begin{pgfscope}%
\pgfsys@transformshift{3.686221in}{2.600706in}%
\pgfsys@useobject{currentmarker}{}%
\end{pgfscope}%
\begin{pgfscope}%
\pgfsys@transformshift{3.667207in}{2.618053in}%
\pgfsys@useobject{currentmarker}{}%
\end{pgfscope}%
\begin{pgfscope}%
\pgfsys@transformshift{3.647254in}{2.516192in}%
\pgfsys@useobject{currentmarker}{}%
\end{pgfscope}%
\begin{pgfscope}%
\pgfsys@transformshift{3.628478in}{2.384599in}%
\pgfsys@useobject{currentmarker}{}%
\end{pgfscope}%
\begin{pgfscope}%
\pgfsys@transformshift{3.608759in}{2.297846in}%
\pgfsys@useobject{currentmarker}{}%
\end{pgfscope}%
\begin{pgfscope}%
\pgfsys@transformshift{3.593502in}{2.268793in}%
\pgfsys@useobject{currentmarker}{}%
\end{pgfscope}%
\begin{pgfscope}%
\pgfsys@transformshift{3.571438in}{2.251210in}%
\pgfsys@useobject{currentmarker}{}%
\end{pgfscope}%
\begin{pgfscope}%
\pgfsys@transformshift{3.552190in}{2.249661in}%
\pgfsys@useobject{currentmarker}{}%
\end{pgfscope}%
\begin{pgfscope}%
\pgfsys@transformshift{3.533177in}{2.268812in}%
\pgfsys@useobject{currentmarker}{}%
\end{pgfscope}%
\begin{pgfscope}%
\pgfsys@transformshift{3.512521in}{2.336754in}%
\pgfsys@useobject{currentmarker}{}%
\end{pgfscope}%
\begin{pgfscope}%
\pgfsys@transformshift{3.496793in}{2.411771in}%
\pgfsys@useobject{currentmarker}{}%
\end{pgfscope}%
\begin{pgfscope}%
\pgfsys@transformshift{3.475903in}{2.576952in}%
\pgfsys@useobject{currentmarker}{}%
\end{pgfscope}%
\begin{pgfscope}%
\pgfsys@transformshift{3.458532in}{2.623983in}%
\pgfsys@useobject{currentmarker}{}%
\end{pgfscope}%
\begin{pgfscope}%
\pgfsys@transformshift{3.438347in}{2.591137in}%
\pgfsys@useobject{currentmarker}{}%
\end{pgfscope}%
\begin{pgfscope}%
\pgfsys@transformshift{3.416046in}{2.429952in}%
\pgfsys@useobject{currentmarker}{}%
\end{pgfscope}%
\begin{pgfscope}%
\pgfsys@transformshift{3.398912in}{2.325202in}%
\pgfsys@useobject{currentmarker}{}%
\end{pgfscope}%
\begin{pgfscope}%
\pgfsys@transformshift{3.378725in}{2.273991in}%
\pgfsys@useobject{currentmarker}{}%
\end{pgfscope}%
\begin{pgfscope}%
\pgfsys@transformshift{3.359711in}{2.250670in}%
\pgfsys@useobject{currentmarker}{}%
\end{pgfscope}%
\begin{pgfscope}%
\pgfsys@transformshift{3.338118in}{2.248495in}%
\pgfsys@useobject{currentmarker}{}%
\end{pgfscope}%
\begin{pgfscope}%
\pgfsys@transformshift{3.322859in}{2.264239in}%
\pgfsys@useobject{currentmarker}{}%
\end{pgfscope}%
\begin{pgfscope}%
\pgfsys@transformshift{3.300794in}{2.290777in}%
\pgfsys@useobject{currentmarker}{}%
\end{pgfscope}%
\begin{pgfscope}%
\pgfsys@transformshift{3.283424in}{2.364334in}%
\pgfsys@useobject{currentmarker}{}%
\end{pgfscope}%
\begin{pgfscope}%
\pgfsys@transformshift{3.263707in}{2.477002in}%
\pgfsys@useobject{currentmarker}{}%
\end{pgfscope}%
\begin{pgfscope}%
\pgfsys@transformshift{3.244460in}{2.602764in}%
\pgfsys@useobject{currentmarker}{}%
\end{pgfscope}%
\begin{pgfscope}%
\pgfsys@transformshift{3.227089in}{2.612161in}%
\pgfsys@useobject{currentmarker}{}%
\end{pgfscope}%
\begin{pgfscope}%
\pgfsys@transformshift{3.205259in}{2.484191in}%
\pgfsys@useobject{currentmarker}{}%
\end{pgfscope}%
\begin{pgfscope}%
\pgfsys@transformshift{3.186482in}{2.358323in}%
\pgfsys@useobject{currentmarker}{}%
\end{pgfscope}%
\begin{pgfscope}%
\pgfsys@transformshift{3.167469in}{2.307071in}%
\pgfsys@useobject{currentmarker}{}%
\end{pgfscope}%
\begin{pgfscope}%
\pgfsys@transformshift{3.145404in}{2.261087in}%
\pgfsys@useobject{currentmarker}{}%
\end{pgfscope}%
\begin{pgfscope}%
\pgfsys@transformshift{3.131320in}{2.247573in}%
\pgfsys@useobject{currentmarker}{}%
\end{pgfscope}%
\begin{pgfscope}%
\pgfsys@transformshift{3.109255in}{2.250963in}%
\pgfsys@useobject{currentmarker}{}%
\end{pgfscope}%
\begin{pgfscope}%
\pgfsys@transformshift{3.090713in}{2.274766in}%
\pgfsys@useobject{currentmarker}{}%
\end{pgfscope}%
\begin{pgfscope}%
\pgfsys@transformshift{3.071934in}{2.326251in}%
\pgfsys@useobject{currentmarker}{}%
\end{pgfscope}%
\begin{pgfscope}%
\pgfsys@transformshift{3.053155in}{2.432230in}%
\pgfsys@useobject{currentmarker}{}%
\end{pgfscope}%
\begin{pgfscope}%
\pgfsys@transformshift{3.034612in}{2.570479in}%
\pgfsys@useobject{currentmarker}{}%
\end{pgfscope}%
\begin{pgfscope}%
\pgfsys@transformshift{3.015128in}{2.320494in}%
\pgfsys@useobject{currentmarker}{}%
\end{pgfscope}%
\begin{pgfscope}%
\pgfsys@transformshift{2.994472in}{2.455701in}%
\pgfsys@useobject{currentmarker}{}%
\end{pgfscope}%
\begin{pgfscope}%
\pgfsys@transformshift{2.975459in}{2.566600in}%
\pgfsys@useobject{currentmarker}{}%
\end{pgfscope}%
\begin{pgfscope}%
\pgfsys@transformshift{2.957151in}{2.617734in}%
\pgfsys@useobject{currentmarker}{}%
\end{pgfscope}%
\begin{pgfscope}%
\pgfsys@transformshift{2.936729in}{2.538387in}%
\pgfsys@useobject{currentmarker}{}%
\end{pgfscope}%
\begin{pgfscope}%
\pgfsys@transformshift{2.917247in}{2.396845in}%
\pgfsys@useobject{currentmarker}{}%
\end{pgfscope}%
\begin{pgfscope}%
\pgfsys@transformshift{2.898937in}{2.311046in}%
\pgfsys@useobject{currentmarker}{}%
\end{pgfscope}%
\begin{pgfscope}%
\pgfsys@transformshift{2.876638in}{2.262167in}%
\pgfsys@useobject{currentmarker}{}%
\end{pgfscope}%
\begin{pgfscope}%
\pgfsys@transformshift{2.860913in}{2.247357in}%
\pgfsys@useobject{currentmarker}{}%
\end{pgfscope}%
\begin{pgfscope}%
\pgfsys@transformshift{2.840020in}{2.250991in}%
\pgfsys@useobject{currentmarker}{}%
\end{pgfscope}%
\begin{pgfscope}%
\pgfsys@transformshift{2.817487in}{2.277143in}%
\pgfsys@useobject{currentmarker}{}%
\end{pgfscope}%
\begin{pgfscope}%
\pgfsys@transformshift{2.802699in}{2.314101in}%
\pgfsys@useobject{currentmarker}{}%
\end{pgfscope}%
\begin{pgfscope}%
\pgfsys@transformshift{2.783685in}{2.396419in}%
\pgfsys@useobject{currentmarker}{}%
\end{pgfscope}%
\begin{pgfscope}%
\pgfsys@transformshift{2.762092in}{2.555495in}%
\pgfsys@useobject{currentmarker}{}%
\end{pgfscope}%
\begin{pgfscope}%
\pgfsys@transformshift{2.746833in}{2.612834in}%
\pgfsys@useobject{currentmarker}{}%
\end{pgfscope}%
\begin{pgfscope}%
\pgfsys@transformshift{2.726177in}{2.580555in}%
\pgfsys@useobject{currentmarker}{}%
\end{pgfscope}%
\begin{pgfscope}%
\pgfsys@transformshift{2.706226in}{2.443473in}%
\pgfsys@useobject{currentmarker}{}%
\end{pgfscope}%
\begin{pgfscope}%
\pgfsys@transformshift{2.688385in}{2.336183in}%
\pgfsys@useobject{currentmarker}{}%
\end{pgfscope}%
\begin{pgfscope}%
\pgfsys@transformshift{2.669373in}{2.283643in}%
\pgfsys@useobject{currentmarker}{}%
\end{pgfscope}%
\begin{pgfscope}%
\pgfsys@transformshift{2.648012in}{2.252187in}%
\pgfsys@useobject{currentmarker}{}%
\end{pgfscope}%
\begin{pgfscope}%
\pgfsys@transformshift{2.632050in}{2.244957in}%
\pgfsys@useobject{currentmarker}{}%
\end{pgfscope}%
\begin{pgfscope}%
\pgfsys@transformshift{2.610691in}{2.248502in}%
\pgfsys@useobject{currentmarker}{}%
\end{pgfscope}%
\begin{pgfscope}%
\pgfsys@transformshift{2.591912in}{2.267912in}%
\pgfsys@useobject{currentmarker}{}%
\end{pgfscope}%
\begin{pgfscope}%
\pgfsys@transformshift{2.570082in}{2.315514in}%
\pgfsys@useobject{currentmarker}{}%
\end{pgfscope}%
\begin{pgfscope}%
\pgfsys@transformshift{2.552243in}{2.370828in}%
\pgfsys@useobject{currentmarker}{}%
\end{pgfscope}%
\begin{pgfscope}%
\pgfsys@transformshift{2.533229in}{2.512575in}%
\pgfsys@useobject{currentmarker}{}%
\end{pgfscope}%
\begin{pgfscope}%
\pgfsys@transformshift{2.514216in}{2.605319in}%
\pgfsys@useobject{currentmarker}{}%
\end{pgfscope}%
\begin{pgfscope}%
\pgfsys@transformshift{2.495908in}{2.590926in}%
\pgfsys@useobject{currentmarker}{}%
\end{pgfscope}%
\begin{pgfscope}%
\pgfsys@transformshift{2.474547in}{2.477024in}%
\pgfsys@useobject{currentmarker}{}%
\end{pgfscope}%
\begin{pgfscope}%
\pgfsys@transformshift{2.456473in}{2.366664in}%
\pgfsys@useobject{currentmarker}{}%
\end{pgfscope}%
\begin{pgfscope}%
\pgfsys@transformshift{2.437694in}{2.297394in}%
\pgfsys@useobject{currentmarker}{}%
\end{pgfscope}%
\begin{pgfscope}%
\pgfsys@transformshift{2.419386in}{2.261814in}%
\pgfsys@useobject{currentmarker}{}%
\end{pgfscope}%
\begin{pgfscope}%
\pgfsys@transformshift{2.397556in}{2.245452in}%
\pgfsys@useobject{currentmarker}{}%
\end{pgfscope}%
\begin{pgfscope}%
\pgfsys@transformshift{2.379482in}{2.249219in}%
\pgfsys@useobject{currentmarker}{}%
\end{pgfscope}%
\begin{pgfscope}%
\pgfsys@transformshift{2.361172in}{2.266569in}%
\pgfsys@useobject{currentmarker}{}%
\end{pgfscope}%
\begin{pgfscope}%
\pgfsys@transformshift{2.339578in}{2.317807in}%
\pgfsys@useobject{currentmarker}{}%
\end{pgfscope}%
\begin{pgfscope}%
\pgfsys@transformshift{2.317983in}{2.417517in}%
\pgfsys@useobject{currentmarker}{}%
\end{pgfscope}%
\begin{pgfscope}%
\pgfsys@transformshift{2.302021in}{2.514667in}%
\pgfsys@useobject{currentmarker}{}%
\end{pgfscope}%
\begin{pgfscope}%
\pgfsys@transformshift{2.284182in}{2.594411in}%
\pgfsys@useobject{currentmarker}{}%
\end{pgfscope}%
\begin{pgfscope}%
\pgfsys@transformshift{2.264934in}{2.611694in}%
\pgfsys@useobject{currentmarker}{}%
\end{pgfscope}%
\begin{pgfscope}%
\pgfsys@transformshift{2.246155in}{2.537928in}%
\pgfsys@useobject{currentmarker}{}%
\end{pgfscope}%
\begin{pgfscope}%
\pgfsys@transformshift{2.225499in}{2.456252in}%
\pgfsys@useobject{currentmarker}{}%
\end{pgfscope}%
\begin{pgfscope}%
\pgfsys@transformshift{2.205548in}{2.365638in}%
\pgfsys@useobject{currentmarker}{}%
\end{pgfscope}%
\begin{pgfscope}%
\pgfsys@transformshift{2.187472in}{2.302624in}%
\pgfsys@useobject{currentmarker}{}%
\end{pgfscope}%
\begin{pgfscope}%
\pgfsys@transformshift{2.169633in}{2.266932in}%
\pgfsys@useobject{currentmarker}{}%
\end{pgfscope}%
\begin{pgfscope}%
\pgfsys@transformshift{2.147569in}{2.246585in}%
\pgfsys@useobject{currentmarker}{}%
\end{pgfscope}%
\begin{pgfscope}%
\pgfsys@transformshift{2.129964in}{2.251689in}%
\pgfsys@useobject{currentmarker}{}%
\end{pgfscope}%
\begin{pgfscope}%
\pgfsys@transformshift{2.110482in}{2.268026in}%
\pgfsys@useobject{currentmarker}{}%
\end{pgfscope}%
\begin{pgfscope}%
\pgfsys@transformshift{2.092642in}{2.298949in}%
\pgfsys@useobject{currentmarker}{}%
\end{pgfscope}%
\begin{pgfscope}%
\pgfsys@transformshift{2.070578in}{2.389116in}%
\pgfsys@useobject{currentmarker}{}%
\end{pgfscope}%
\begin{pgfscope}%
\pgfsys@transformshift{2.052739in}{2.516293in}%
\pgfsys@useobject{currentmarker}{}%
\end{pgfscope}%
\begin{pgfscope}%
\pgfsys@transformshift{2.033725in}{2.550135in}%
\pgfsys@useobject{currentmarker}{}%
\end{pgfscope}%
\begin{pgfscope}%
\pgfsys@transformshift{2.015886in}{2.260900in}%
\pgfsys@useobject{currentmarker}{}%
\end{pgfscope}%
\begin{pgfscope}%
\pgfsys@transformshift{1.996873in}{2.298822in}%
\pgfsys@useobject{currentmarker}{}%
\end{pgfscope}%
\begin{pgfscope}%
\pgfsys@transformshift{1.976217in}{2.382183in}%
\pgfsys@useobject{currentmarker}{}%
\end{pgfscope}%
\begin{pgfscope}%
\pgfsys@transformshift{1.955326in}{2.548372in}%
\pgfsys@useobject{currentmarker}{}%
\end{pgfscope}%
\begin{pgfscope}%
\pgfsys@transformshift{1.937487in}{2.615577in}%
\pgfsys@useobject{currentmarker}{}%
\end{pgfscope}%
\begin{pgfscope}%
\pgfsys@transformshift{1.916596in}{2.581966in}%
\pgfsys@useobject{currentmarker}{}%
\end{pgfscope}%
\begin{pgfscope}%
\pgfsys@transformshift{1.898992in}{2.492858in}%
\pgfsys@useobject{currentmarker}{}%
\end{pgfscope}%
\begin{pgfscope}%
\pgfsys@transformshift{1.875987in}{2.347203in}%
\pgfsys@useobject{currentmarker}{}%
\end{pgfscope}%
\begin{pgfscope}%
\pgfsys@transformshift{1.861668in}{2.293238in}%
\pgfsys@useobject{currentmarker}{}%
\end{pgfscope}%
\begin{pgfscope}%
\pgfsys@transformshift{1.841952in}{2.311249in}%
\pgfsys@useobject{currentmarker}{}%
\end{pgfscope}%
\begin{pgfscope}%
\pgfsys@transformshift{1.824113in}{2.268753in}%
\pgfsys@useobject{currentmarker}{}%
\end{pgfscope}%
\begin{pgfscope}%
\pgfsys@transformshift{1.802986in}{2.250612in}%
\pgfsys@useobject{currentmarker}{}%
\end{pgfscope}%
\begin{pgfscope}%
\pgfsys@transformshift{1.784678in}{2.248090in}%
\pgfsys@useobject{currentmarker}{}%
\end{pgfscope}%
\begin{pgfscope}%
\pgfsys@transformshift{1.764727in}{2.267144in}%
\pgfsys@useobject{currentmarker}{}%
\end{pgfscope}%
\begin{pgfscope}%
\pgfsys@transformshift{1.744305in}{2.311452in}%
\pgfsys@useobject{currentmarker}{}%
\end{pgfscope}%
\begin{pgfscope}%
\pgfsys@transformshift{1.726229in}{2.376518in}%
\pgfsys@useobject{currentmarker}{}%
\end{pgfscope}%
\begin{pgfscope}%
\pgfsys@transformshift{1.707216in}{2.512524in}%
\pgfsys@useobject{currentmarker}{}%
\end{pgfscope}%
\begin{pgfscope}%
\pgfsys@transformshift{1.689142in}{2.596838in}%
\pgfsys@useobject{currentmarker}{}%
\end{pgfscope}%
\begin{pgfscope}%
\pgfsys@transformshift{1.670364in}{2.625971in}%
\pgfsys@useobject{currentmarker}{}%
\end{pgfscope}%
\begin{pgfscope}%
\pgfsys@transformshift{1.648535in}{2.554832in}%
\pgfsys@useobject{currentmarker}{}%
\end{pgfscope}%
\begin{pgfscope}%
\pgfsys@transformshift{1.630931in}{2.421994in}%
\pgfsys@useobject{currentmarker}{}%
\end{pgfscope}%
\begin{pgfscope}%
\pgfsys@transformshift{1.609569in}{2.322271in}%
\pgfsys@useobject{currentmarker}{}%
\end{pgfscope}%
\begin{pgfscope}%
\pgfsys@transformshift{1.592435in}{2.281336in}%
\pgfsys@useobject{currentmarker}{}%
\end{pgfscope}%
\begin{pgfscope}%
\pgfsys@transformshift{1.571779in}{2.254291in}%
\pgfsys@useobject{currentmarker}{}%
\end{pgfscope}%
\begin{pgfscope}%
\pgfsys@transformshift{1.554643in}{2.247165in}%
\pgfsys@useobject{currentmarker}{}%
\end{pgfscope}%
\begin{pgfscope}%
\pgfsys@transformshift{1.531639in}{2.259875in}%
\pgfsys@useobject{currentmarker}{}%
\end{pgfscope}%
\begin{pgfscope}%
\pgfsys@transformshift{1.515677in}{2.280881in}%
\pgfsys@useobject{currentmarker}{}%
\end{pgfscope}%
\begin{pgfscope}%
\pgfsys@transformshift{1.495257in}{2.333803in}%
\pgfsys@useobject{currentmarker}{}%
\end{pgfscope}%
\begin{pgfscope}%
\pgfsys@transformshift{1.474835in}{2.450695in}%
\pgfsys@useobject{currentmarker}{}%
\end{pgfscope}%
\begin{pgfscope}%
\pgfsys@transformshift{1.457465in}{2.508068in}%
\pgfsys@useobject{currentmarker}{}%
\end{pgfscope}%
\begin{pgfscope}%
\pgfsys@transformshift{1.437512in}{2.611740in}%
\pgfsys@useobject{currentmarker}{}%
\end{pgfscope}%
\begin{pgfscope}%
\pgfsys@transformshift{1.419204in}{2.631690in}%
\pgfsys@useobject{currentmarker}{}%
\end{pgfscope}%
\begin{pgfscope}%
\pgfsys@transformshift{1.398548in}{2.577011in}%
\pgfsys@useobject{currentmarker}{}%
\end{pgfscope}%
\begin{pgfscope}%
\pgfsys@transformshift{1.381412in}{2.500148in}%
\pgfsys@useobject{currentmarker}{}%
\end{pgfscope}%
\begin{pgfscope}%
\pgfsys@transformshift{1.361227in}{2.377654in}%
\pgfsys@useobject{currentmarker}{}%
\end{pgfscope}%
\begin{pgfscope}%
\pgfsys@transformshift{1.337519in}{2.298019in}%
\pgfsys@useobject{currentmarker}{}%
\end{pgfscope}%
\begin{pgfscope}%
\pgfsys@transformshift{1.323200in}{2.271644in}%
\pgfsys@useobject{currentmarker}{}%
\end{pgfscope}%
\begin{pgfscope}%
\pgfsys@transformshift{1.302544in}{2.251248in}%
\pgfsys@useobject{currentmarker}{}%
\end{pgfscope}%
\begin{pgfscope}%
\pgfsys@transformshift{1.285408in}{2.249565in}%
\pgfsys@useobject{currentmarker}{}%
\end{pgfscope}%
\begin{pgfscope}%
\pgfsys@transformshift{1.265221in}{2.262088in}%
\pgfsys@useobject{currentmarker}{}%
\end{pgfscope}%
\begin{pgfscope}%
\pgfsys@transformshift{1.247147in}{2.279926in}%
\pgfsys@useobject{currentmarker}{}%
\end{pgfscope}%
\begin{pgfscope}%
\pgfsys@transformshift{1.226257in}{2.332279in}%
\pgfsys@useobject{currentmarker}{}%
\end{pgfscope}%
\begin{pgfscope}%
\pgfsys@transformshift{1.207948in}{2.448907in}%
\pgfsys@useobject{currentmarker}{}%
\end{pgfscope}%
\begin{pgfscope}%
\pgfsys@transformshift{1.187761in}{2.560462in}%
\pgfsys@useobject{currentmarker}{}%
\end{pgfscope}%
\begin{pgfscope}%
\pgfsys@transformshift{1.168043in}{2.634756in}%
\pgfsys@useobject{currentmarker}{}%
\end{pgfscope}%
\begin{pgfscope}%
\pgfsys@transformshift{1.150440in}{2.641011in}%
\pgfsys@useobject{currentmarker}{}%
\end{pgfscope}%
\begin{pgfscope}%
\pgfsys@transformshift{1.132130in}{2.580943in}%
\pgfsys@useobject{currentmarker}{}%
\end{pgfscope}%
\begin{pgfscope}%
\pgfsys@transformshift{1.111943in}{2.459576in}%
\pgfsys@useobject{currentmarker}{}%
\end{pgfscope}%
\begin{pgfscope}%
\pgfsys@transformshift{1.091992in}{2.350214in}%
\pgfsys@useobject{currentmarker}{}%
\end{pgfscope}%
\begin{pgfscope}%
\pgfsys@transformshift{1.074152in}{2.311127in}%
\pgfsys@useobject{currentmarker}{}%
\end{pgfscope}%
\begin{pgfscope}%
\pgfsys@transformshift{1.053965in}{2.278766in}%
\pgfsys@useobject{currentmarker}{}%
\end{pgfscope}%
\begin{pgfscope}%
\pgfsys@transformshift{1.033075in}{2.254478in}%
\pgfsys@useobject{currentmarker}{}%
\end{pgfscope}%
\begin{pgfscope}%
\pgfsys@transformshift{1.015704in}{2.250886in}%
\pgfsys@useobject{currentmarker}{}%
\end{pgfscope}%
\begin{pgfscope}%
\pgfsys@transformshift{0.997631in}{2.265543in}%
\pgfsys@useobject{currentmarker}{}%
\end{pgfscope}%
\begin{pgfscope}%
\pgfsys@transformshift{0.977209in}{2.274346in}%
\pgfsys@useobject{currentmarker}{}%
\end{pgfscope}%
\begin{pgfscope}%
\pgfsys@transformshift{0.956318in}{2.318191in}%
\pgfsys@useobject{currentmarker}{}%
\end{pgfscope}%
\begin{pgfscope}%
\pgfsys@transformshift{0.938948in}{2.391296in}%
\pgfsys@useobject{currentmarker}{}%
\end{pgfscope}%
\begin{pgfscope}%
\pgfsys@transformshift{0.917352in}{2.523991in}%
\pgfsys@useobject{currentmarker}{}%
\end{pgfscope}%
\begin{pgfscope}%
\pgfsys@transformshift{0.902799in}{2.620107in}%
\pgfsys@useobject{currentmarker}{}%
\end{pgfscope}%
\begin{pgfscope}%
\pgfsys@transformshift{0.882613in}{2.652932in}%
\pgfsys@useobject{currentmarker}{}%
\end{pgfscope}%
\begin{pgfscope}%
\pgfsys@transformshift{0.861723in}{2.624753in}%
\pgfsys@useobject{currentmarker}{}%
\end{pgfscope}%
\begin{pgfscope}%
\pgfsys@transformshift{0.839893in}{2.559079in}%
\pgfsys@useobject{currentmarker}{}%
\end{pgfscope}%
\begin{pgfscope}%
\pgfsys@transformshift{0.823462in}{2.437763in}%
\pgfsys@useobject{currentmarker}{}%
\end{pgfscope}%
\begin{pgfscope}%
\pgfsys@transformshift{0.802569in}{2.339568in}%
\pgfsys@useobject{currentmarker}{}%
\end{pgfscope}%
\begin{pgfscope}%
\pgfsys@transformshift{0.784964in}{2.295782in}%
\pgfsys@useobject{currentmarker}{}%
\end{pgfscope}%
\begin{pgfscope}%
\pgfsys@transformshift{0.763371in}{2.266028in}%
\pgfsys@useobject{currentmarker}{}%
\end{pgfscope}%
\begin{pgfscope}%
\pgfsys@transformshift{0.746235in}{2.252946in}%
\pgfsys@useobject{currentmarker}{}%
\end{pgfscope}%
\begin{pgfscope}%
\pgfsys@transformshift{0.728395in}{2.290069in}%
\pgfsys@useobject{currentmarker}{}%
\end{pgfscope}%
\begin{pgfscope}%
\pgfsys@transformshift{0.707505in}{2.264562in}%
\pgfsys@useobject{currentmarker}{}%
\end{pgfscope}%
\begin{pgfscope}%
\pgfsys@transformshift{0.688257in}{2.252661in}%
\pgfsys@useobject{currentmarker}{}%
\end{pgfscope}%
\begin{pgfscope}%
\pgfsys@transformshift{0.670418in}{2.266746in}%
\pgfsys@useobject{currentmarker}{}%
\end{pgfscope}%
\begin{pgfscope}%
\pgfsys@transformshift{0.649291in}{2.294417in}%
\pgfsys@useobject{currentmarker}{}%
\end{pgfscope}%
\begin{pgfscope}%
\pgfsys@transformshift{0.650231in}{2.296364in}%
\pgfsys@useobject{currentmarker}{}%
\end{pgfscope}%
\begin{pgfscope}%
\pgfsys@transformshift{0.657742in}{2.273296in}%
\pgfsys@useobject{currentmarker}{}%
\end{pgfscope}%
\begin{pgfscope}%
\pgfsys@transformshift{0.675112in}{2.253202in}%
\pgfsys@useobject{currentmarker}{}%
\end{pgfscope}%
\begin{pgfscope}%
\pgfsys@transformshift{0.696942in}{2.276841in}%
\pgfsys@useobject{currentmarker}{}%
\end{pgfscope}%
\begin{pgfscope}%
\pgfsys@transformshift{0.712904in}{2.320418in}%
\pgfsys@useobject{currentmarker}{}%
\end{pgfscope}%
\begin{pgfscope}%
\pgfsys@transformshift{0.734498in}{2.441060in}%
\pgfsys@useobject{currentmarker}{}%
\end{pgfscope}%
\begin{pgfscope}%
\pgfsys@transformshift{0.752808in}{2.604486in}%
\pgfsys@useobject{currentmarker}{}%
\end{pgfscope}%
\begin{pgfscope}%
\pgfsys@transformshift{0.772056in}{2.654907in}%
\pgfsys@useobject{currentmarker}{}%
\end{pgfscope}%
\begin{pgfscope}%
\pgfsys@transformshift{0.792241in}{2.549571in}%
\pgfsys@useobject{currentmarker}{}%
\end{pgfscope}%
\begin{pgfscope}%
\pgfsys@transformshift{0.809612in}{2.399161in}%
\pgfsys@useobject{currentmarker}{}%
\end{pgfscope}%
\begin{pgfscope}%
\pgfsys@transformshift{0.828156in}{2.304388in}%
\pgfsys@useobject{currentmarker}{}%
\end{pgfscope}%
\begin{pgfscope}%
\pgfsys@transformshift{0.850221in}{2.258082in}%
\pgfsys@useobject{currentmarker}{}%
\end{pgfscope}%
\begin{pgfscope}%
\pgfsys@transformshift{0.868060in}{2.253262in}%
\pgfsys@useobject{currentmarker}{}%
\end{pgfscope}%
\begin{pgfscope}%
\pgfsys@transformshift{0.886368in}{2.280635in}%
\pgfsys@useobject{currentmarker}{}%
\end{pgfscope}%
\begin{pgfscope}%
\pgfsys@transformshift{0.905147in}{2.342563in}%
\pgfsys@useobject{currentmarker}{}%
\end{pgfscope}%
\begin{pgfscope}%
\pgfsys@transformshift{0.926037in}{2.489099in}%
\pgfsys@useobject{currentmarker}{}%
\end{pgfscope}%
\begin{pgfscope}%
\pgfsys@transformshift{0.944347in}{2.628242in}%
\pgfsys@useobject{currentmarker}{}%
\end{pgfscope}%
\begin{pgfscope}%
\pgfsys@transformshift{0.961952in}{2.631604in}%
\pgfsys@useobject{currentmarker}{}%
\end{pgfscope}%
\begin{pgfscope}%
\pgfsys@transformshift{0.980026in}{2.496852in}%
\pgfsys@useobject{currentmarker}{}%
\end{pgfscope}%
\begin{pgfscope}%
\pgfsys@transformshift{1.001621in}{2.353023in}%
\pgfsys@useobject{currentmarker}{}%
\end{pgfscope}%
\begin{pgfscope}%
\pgfsys@transformshift{1.022041in}{2.276415in}%
\pgfsys@useobject{currentmarker}{}%
\end{pgfscope}%
\begin{pgfscope}%
\pgfsys@transformshift{1.040586in}{2.250903in}%
\pgfsys@useobject{currentmarker}{}%
\end{pgfscope}%
\begin{pgfscope}%
\pgfsys@transformshift{1.058425in}{2.256873in}%
\pgfsys@useobject{currentmarker}{}%
\end{pgfscope}%
\begin{pgfscope}%
\pgfsys@transformshift{1.080255in}{2.296224in}%
\pgfsys@useobject{currentmarker}{}%
\end{pgfscope}%
\begin{pgfscope}%
\pgfsys@transformshift{1.097155in}{2.368382in}%
\pgfsys@useobject{currentmarker}{}%
\end{pgfscope}%
\begin{pgfscope}%
\pgfsys@transformshift{1.117811in}{2.527521in}%
\pgfsys@useobject{currentmarker}{}%
\end{pgfscope}%
\begin{pgfscope}%
\pgfsys@transformshift{1.135886in}{2.631818in}%
\pgfsys@useobject{currentmarker}{}%
\end{pgfscope}%
\begin{pgfscope}%
\pgfsys@transformshift{1.156308in}{2.584096in}%
\pgfsys@useobject{currentmarker}{}%
\end{pgfscope}%
\begin{pgfscope}%
\pgfsys@transformshift{1.175556in}{2.425409in}%
\pgfsys@useobject{currentmarker}{}%
\end{pgfscope}%
\begin{pgfscope}%
\pgfsys@transformshift{1.195272in}{2.311860in}%
\pgfsys@useobject{currentmarker}{}%
\end{pgfscope}%
\begin{pgfscope}%
\pgfsys@transformshift{1.213580in}{2.266262in}%
\pgfsys@useobject{currentmarker}{}%
\end{pgfscope}%
\begin{pgfscope}%
\pgfsys@transformshift{1.232125in}{2.247695in}%
\pgfsys@useobject{currentmarker}{}%
\end{pgfscope}%
\begin{pgfscope}%
\pgfsys@transformshift{1.251607in}{2.261778in}%
\pgfsys@useobject{currentmarker}{}%
\end{pgfscope}%
\begin{pgfscope}%
\pgfsys@transformshift{1.271089in}{2.296413in}%
\pgfsys@useobject{currentmarker}{}%
\end{pgfscope}%
\begin{pgfscope}%
\pgfsys@transformshift{1.290339in}{2.370991in}%
\pgfsys@useobject{currentmarker}{}%
\end{pgfscope}%
\begin{pgfscope}%
\pgfsys@transformshift{1.308881in}{2.512911in}%
\pgfsys@useobject{currentmarker}{}%
\end{pgfscope}%
\begin{pgfscope}%
\pgfsys@transformshift{1.329303in}{2.623945in}%
\pgfsys@useobject{currentmarker}{}%
\end{pgfscope}%
\begin{pgfscope}%
\pgfsys@transformshift{1.347376in}{2.607674in}%
\pgfsys@useobject{currentmarker}{}%
\end{pgfscope}%
\begin{pgfscope}%
\pgfsys@transformshift{1.368503in}{2.447711in}%
\pgfsys@useobject{currentmarker}{}%
\end{pgfscope}%
\begin{pgfscope}%
\pgfsys@transformshift{1.386106in}{2.333192in}%
\pgfsys@useobject{currentmarker}{}%
\end{pgfscope}%
\begin{pgfscope}%
\pgfsys@transformshift{1.405120in}{2.277240in}%
\pgfsys@useobject{currentmarker}{}%
\end{pgfscope}%
\begin{pgfscope}%
\pgfsys@transformshift{1.425541in}{2.250102in}%
\pgfsys@useobject{currentmarker}{}%
\end{pgfscope}%
\begin{pgfscope}%
\pgfsys@transformshift{1.446432in}{2.251654in}%
\pgfsys@useobject{currentmarker}{}%
\end{pgfscope}%
\begin{pgfscope}%
\pgfsys@transformshift{1.464742in}{2.274047in}%
\pgfsys@useobject{currentmarker}{}%
\end{pgfscope}%
\begin{pgfscope}%
\pgfsys@transformshift{1.481407in}{2.295401in}%
\pgfsys@useobject{currentmarker}{}%
\end{pgfscope}%
\begin{pgfscope}%
\pgfsys@transformshift{1.502298in}{2.391280in}%
\pgfsys@useobject{currentmarker}{}%
\end{pgfscope}%
\begin{pgfscope}%
\pgfsys@transformshift{1.520608in}{2.540238in}%
\pgfsys@useobject{currentmarker}{}%
\end{pgfscope}%
\begin{pgfscope}%
\pgfsys@transformshift{1.542672in}{2.623610in}%
\pgfsys@useobject{currentmarker}{}%
\end{pgfscope}%
\begin{pgfscope}%
\pgfsys@transformshift{1.560277in}{2.578079in}%
\pgfsys@useobject{currentmarker}{}%
\end{pgfscope}%
\begin{pgfscope}%
\pgfsys@transformshift{1.579054in}{2.466207in}%
\pgfsys@useobject{currentmarker}{}%
\end{pgfscope}%
\begin{pgfscope}%
\pgfsys@transformshift{1.597364in}{2.352357in}%
\pgfsys@useobject{currentmarker}{}%
\end{pgfscope}%
\begin{pgfscope}%
\pgfsys@transformshift{1.615672in}{2.285922in}%
\pgfsys@useobject{currentmarker}{}%
\end{pgfscope}%
\begin{pgfscope}%
\pgfsys@transformshift{1.637033in}{2.601663in}%
\pgfsys@useobject{currentmarker}{}%
\end{pgfscope}%
\begin{pgfscope}%
\pgfsys@transformshift{1.654872in}{2.467462in}%
\pgfsys@useobject{currentmarker}{}%
\end{pgfscope}%
\begin{pgfscope}%
\pgfsys@transformshift{1.673417in}{2.334756in}%
\pgfsys@useobject{currentmarker}{}%
\end{pgfscope}%
\begin{pgfscope}%
\pgfsys@transformshift{1.691491in}{2.270727in}%
\pgfsys@useobject{currentmarker}{}%
\end{pgfscope}%
\begin{pgfscope}%
\pgfsys@transformshift{1.713084in}{2.249295in}%
\pgfsys@useobject{currentmarker}{}%
\end{pgfscope}%
\begin{pgfscope}%
\pgfsys@transformshift{1.732803in}{2.250119in}%
\pgfsys@useobject{currentmarker}{}%
\end{pgfscope}%
\begin{pgfscope}%
\pgfsys@transformshift{1.753693in}{2.278619in}%
\pgfsys@useobject{currentmarker}{}%
\end{pgfscope}%
\begin{pgfscope}%
\pgfsys@transformshift{1.772238in}{2.334627in}%
\pgfsys@useobject{currentmarker}{}%
\end{pgfscope}%
\begin{pgfscope}%
\pgfsys@transformshift{1.790780in}{2.449785in}%
\pgfsys@useobject{currentmarker}{}%
\end{pgfscope}%
\begin{pgfscope}%
\pgfsys@transformshift{1.811436in}{2.580014in}%
\pgfsys@useobject{currentmarker}{}%
\end{pgfscope}%
\begin{pgfscope}%
\pgfsys@transformshift{1.829276in}{2.615707in}%
\pgfsys@useobject{currentmarker}{}%
\end{pgfscope}%
\begin{pgfscope}%
\pgfsys@transformshift{1.847115in}{2.548967in}%
\pgfsys@useobject{currentmarker}{}%
\end{pgfscope}%
\begin{pgfscope}%
\pgfsys@transformshift{1.868007in}{2.398865in}%
\pgfsys@useobject{currentmarker}{}%
\end{pgfscope}%
\begin{pgfscope}%
\pgfsys@transformshift{1.885847in}{2.308969in}%
\pgfsys@useobject{currentmarker}{}%
\end{pgfscope}%
\begin{pgfscope}%
\pgfsys@transformshift{1.903920in}{2.266957in}%
\pgfsys@useobject{currentmarker}{}%
\end{pgfscope}%
\begin{pgfscope}%
\pgfsys@transformshift{1.925516in}{2.245600in}%
\pgfsys@useobject{currentmarker}{}%
\end{pgfscope}%
\begin{pgfscope}%
\pgfsys@transformshift{1.943121in}{2.249549in}%
\pgfsys@useobject{currentmarker}{}%
\end{pgfscope}%
\begin{pgfscope}%
\pgfsys@transformshift{1.964246in}{2.277854in}%
\pgfsys@useobject{currentmarker}{}%
\end{pgfscope}%
\begin{pgfscope}%
\pgfsys@transformshift{1.982319in}{2.325584in}%
\pgfsys@useobject{currentmarker}{}%
\end{pgfscope}%
\begin{pgfscope}%
\pgfsys@transformshift{2.003210in}{2.452760in}%
\pgfsys@useobject{currentmarker}{}%
\end{pgfscope}%
\begin{pgfscope}%
\pgfsys@transformshift{2.020580in}{2.581430in}%
\pgfsys@useobject{currentmarker}{}%
\end{pgfscope}%
\begin{pgfscope}%
\pgfsys@transformshift{2.041707in}{2.609944in}%
\pgfsys@useobject{currentmarker}{}%
\end{pgfscope}%
\begin{pgfscope}%
\pgfsys@transformshift{2.060719in}{2.517448in}%
\pgfsys@useobject{currentmarker}{}%
\end{pgfscope}%
\begin{pgfscope}%
\pgfsys@transformshift{2.078560in}{2.381424in}%
\pgfsys@useobject{currentmarker}{}%
\end{pgfscope}%
\begin{pgfscope}%
\pgfsys@transformshift{2.095459in}{2.301635in}%
\pgfsys@useobject{currentmarker}{}%
\end{pgfscope}%
\begin{pgfscope}%
\pgfsys@transformshift{2.117758in}{2.269202in}%
\pgfsys@useobject{currentmarker}{}%
\end{pgfscope}%
\begin{pgfscope}%
\pgfsys@transformshift{2.137711in}{2.248668in}%
\pgfsys@useobject{currentmarker}{}%
\end{pgfscope}%
\begin{pgfscope}%
\pgfsys@transformshift{2.156488in}{2.246127in}%
\pgfsys@useobject{currentmarker}{}%
\end{pgfscope}%
\begin{pgfscope}%
\pgfsys@transformshift{2.174093in}{2.262360in}%
\pgfsys@useobject{currentmarker}{}%
\end{pgfscope}%
\begin{pgfscope}%
\pgfsys@transformshift{2.195220in}{2.305606in}%
\pgfsys@useobject{currentmarker}{}%
\end{pgfscope}%
\begin{pgfscope}%
\pgfsys@transformshift{2.213293in}{2.382853in}%
\pgfsys@useobject{currentmarker}{}%
\end{pgfscope}%
\begin{pgfscope}%
\pgfsys@transformshift{2.231133in}{2.501796in}%
\pgfsys@useobject{currentmarker}{}%
\end{pgfscope}%
\begin{pgfscope}%
\pgfsys@transformshift{2.252023in}{2.609422in}%
\pgfsys@useobject{currentmarker}{}%
\end{pgfscope}%
\begin{pgfscope}%
\pgfsys@transformshift{2.269159in}{2.589778in}%
\pgfsys@useobject{currentmarker}{}%
\end{pgfscope}%
\begin{pgfscope}%
\pgfsys@transformshift{2.291458in}{2.444539in}%
\pgfsys@useobject{currentmarker}{}%
\end{pgfscope}%
\begin{pgfscope}%
\pgfsys@transformshift{2.309766in}{2.330721in}%
\pgfsys@useobject{currentmarker}{}%
\end{pgfscope}%
\begin{pgfscope}%
\pgfsys@transformshift{2.327137in}{2.278618in}%
\pgfsys@useobject{currentmarker}{}%
\end{pgfscope}%
\begin{pgfscope}%
\pgfsys@transformshift{2.347793in}{2.252210in}%
\pgfsys@useobject{currentmarker}{}%
\end{pgfscope}%
\begin{pgfscope}%
\pgfsys@transformshift{2.365868in}{2.243362in}%
\pgfsys@useobject{currentmarker}{}%
\end{pgfscope}%
\begin{pgfscope}%
\pgfsys@transformshift{2.387697in}{2.253736in}%
\pgfsys@useobject{currentmarker}{}%
\end{pgfscope}%
\begin{pgfscope}%
\pgfsys@transformshift{2.405067in}{2.278565in}%
\pgfsys@useobject{currentmarker}{}%
\end{pgfscope}%
\begin{pgfscope}%
\pgfsys@transformshift{2.423141in}{2.281899in}%
\pgfsys@useobject{currentmarker}{}%
\end{pgfscope}%
\begin{pgfscope}%
\pgfsys@transformshift{2.440511in}{2.314944in}%
\pgfsys@useobject{currentmarker}{}%
\end{pgfscope}%
\begin{pgfscope}%
\pgfsys@transformshift{2.462341in}{2.434466in}%
\pgfsys@useobject{currentmarker}{}%
\end{pgfscope}%
\begin{pgfscope}%
\pgfsys@transformshift{2.479477in}{2.565162in}%
\pgfsys@useobject{currentmarker}{}%
\end{pgfscope}%
\begin{pgfscope}%
\pgfsys@transformshift{2.503888in}{2.603982in}%
\pgfsys@useobject{currentmarker}{}%
\end{pgfscope}%
\begin{pgfscope}%
\pgfsys@transformshift{2.519615in}{2.585632in}%
\pgfsys@useobject{currentmarker}{}%
\end{pgfscope}%
\begin{pgfscope}%
\pgfsys@transformshift{2.543557in}{2.424206in}%
\pgfsys@useobject{currentmarker}{}%
\end{pgfscope}%
\begin{pgfscope}%
\pgfsys@transformshift{2.559754in}{2.330556in}%
\pgfsys@useobject{currentmarker}{}%
\end{pgfscope}%
\begin{pgfscope}%
\pgfsys@transformshift{2.576890in}{2.275842in}%
\pgfsys@useobject{currentmarker}{}%
\end{pgfscope}%
\begin{pgfscope}%
\pgfsys@transformshift{2.598015in}{2.249834in}%
\pgfsys@useobject{currentmarker}{}%
\end{pgfscope}%
\begin{pgfscope}%
\pgfsys@transformshift{2.615619in}{2.244366in}%
\pgfsys@useobject{currentmarker}{}%
\end{pgfscope}%
\begin{pgfscope}%
\pgfsys@transformshift{2.638153in}{2.261138in}%
\pgfsys@useobject{currentmarker}{}%
\end{pgfscope}%
\begin{pgfscope}%
\pgfsys@transformshift{2.652941in}{2.292098in}%
\pgfsys@useobject{currentmarker}{}%
\end{pgfscope}%
\begin{pgfscope}%
\pgfsys@transformshift{2.672894in}{2.350377in}%
\pgfsys@useobject{currentmarker}{}%
\end{pgfscope}%
\begin{pgfscope}%
\pgfsys@transformshift{2.693784in}{2.478628in}%
\pgfsys@useobject{currentmarker}{}%
\end{pgfscope}%
\begin{pgfscope}%
\pgfsys@transformshift{2.711858in}{2.589551in}%
\pgfsys@useobject{currentmarker}{}%
\end{pgfscope}%
\begin{pgfscope}%
\pgfsys@transformshift{2.732750in}{2.605800in}%
\pgfsys@useobject{currentmarker}{}%
\end{pgfscope}%
\begin{pgfscope}%
\pgfsys@transformshift{2.750824in}{2.511191in}%
\pgfsys@useobject{currentmarker}{}%
\end{pgfscope}%
\begin{pgfscope}%
\pgfsys@transformshift{2.771480in}{2.611965in}%
\pgfsys@useobject{currentmarker}{}%
\end{pgfscope}%
\begin{pgfscope}%
\pgfsys@transformshift{2.790023in}{2.586126in}%
\pgfsys@useobject{currentmarker}{}%
\end{pgfscope}%
\begin{pgfscope}%
\pgfsys@transformshift{2.807627in}{2.459356in}%
\pgfsys@useobject{currentmarker}{}%
\end{pgfscope}%
\begin{pgfscope}%
\pgfsys@transformshift{2.829223in}{2.465665in}%
\pgfsys@useobject{currentmarker}{}%
\end{pgfscope}%
\begin{pgfscope}%
\pgfsys@transformshift{2.847297in}{2.355627in}%
\pgfsys@useobject{currentmarker}{}%
\end{pgfscope}%
\begin{pgfscope}%
\pgfsys@transformshift{2.865607in}{2.286302in}%
\pgfsys@useobject{currentmarker}{}%
\end{pgfscope}%
\begin{pgfscope}%
\pgfsys@transformshift{2.882977in}{2.255076in}%
\pgfsys@useobject{currentmarker}{}%
\end{pgfscope}%
\begin{pgfscope}%
\pgfsys@transformshift{2.905511in}{2.246203in}%
\pgfsys@useobject{currentmarker}{}%
\end{pgfscope}%
\begin{pgfscope}%
\pgfsys@transformshift{2.923584in}{2.247965in}%
\pgfsys@useobject{currentmarker}{}%
\end{pgfscope}%
\begin{pgfscope}%
\pgfsys@transformshift{2.943772in}{2.271856in}%
\pgfsys@useobject{currentmarker}{}%
\end{pgfscope}%
\begin{pgfscope}%
\pgfsys@transformshift{2.962080in}{2.314843in}%
\pgfsys@useobject{currentmarker}{}%
\end{pgfscope}%
\begin{pgfscope}%
\pgfsys@transformshift{2.982970in}{2.351540in}%
\pgfsys@useobject{currentmarker}{}%
\end{pgfscope}%
\begin{pgfscope}%
\pgfsys@transformshift{3.001749in}{2.474331in}%
\pgfsys@useobject{currentmarker}{}%
\end{pgfscope}%
\begin{pgfscope}%
\pgfsys@transformshift{3.019119in}{2.583959in}%
\pgfsys@useobject{currentmarker}{}%
\end{pgfscope}%
\begin{pgfscope}%
\pgfsys@transformshift{3.040479in}{2.609969in}%
\pgfsys@useobject{currentmarker}{}%
\end{pgfscope}%
\begin{pgfscope}%
\pgfsys@transformshift{3.057615in}{2.512558in}%
\pgfsys@useobject{currentmarker}{}%
\end{pgfscope}%
\begin{pgfscope}%
\pgfsys@transformshift{3.076159in}{2.381258in}%
\pgfsys@useobject{currentmarker}{}%
\end{pgfscope}%
\begin{pgfscope}%
\pgfsys@transformshift{3.097284in}{2.292062in}%
\pgfsys@useobject{currentmarker}{}%
\end{pgfscope}%
\begin{pgfscope}%
\pgfsys@transformshift{3.115829in}{2.260231in}%
\pgfsys@useobject{currentmarker}{}%
\end{pgfscope}%
\begin{pgfscope}%
\pgfsys@transformshift{3.134137in}{2.249866in}%
\pgfsys@useobject{currentmarker}{}%
\end{pgfscope}%
\begin{pgfscope}%
\pgfsys@transformshift{3.154558in}{2.246047in}%
\pgfsys@useobject{currentmarker}{}%
\end{pgfscope}%
\begin{pgfscope}%
\pgfsys@transformshift{3.173337in}{2.262623in}%
\pgfsys@useobject{currentmarker}{}%
\end{pgfscope}%
\begin{pgfscope}%
\pgfsys@transformshift{3.194697in}{2.301782in}%
\pgfsys@useobject{currentmarker}{}%
\end{pgfscope}%
\begin{pgfscope}%
\pgfsys@transformshift{3.212536in}{2.365570in}%
\pgfsys@useobject{currentmarker}{}%
\end{pgfscope}%
\begin{pgfscope}%
\pgfsys@transformshift{3.232958in}{2.501554in}%
\pgfsys@useobject{currentmarker}{}%
\end{pgfscope}%
\begin{pgfscope}%
\pgfsys@transformshift{3.251502in}{2.601981in}%
\pgfsys@useobject{currentmarker}{}%
\end{pgfscope}%
\begin{pgfscope}%
\pgfsys@transformshift{3.269576in}{2.623255in}%
\pgfsys@useobject{currentmarker}{}%
\end{pgfscope}%
\begin{pgfscope}%
\pgfsys@transformshift{3.291406in}{2.548969in}%
\pgfsys@useobject{currentmarker}{}%
\end{pgfscope}%
\begin{pgfscope}%
\pgfsys@transformshift{3.308305in}{2.448405in}%
\pgfsys@useobject{currentmarker}{}%
\end{pgfscope}%
\begin{pgfscope}%
\pgfsys@transformshift{3.325910in}{2.350917in}%
\pgfsys@useobject{currentmarker}{}%
\end{pgfscope}%
\begin{pgfscope}%
\pgfsys@transformshift{3.347506in}{2.283439in}%
\pgfsys@useobject{currentmarker}{}%
\end{pgfscope}%
\begin{pgfscope}%
\pgfsys@transformshift{3.366048in}{2.257165in}%
\pgfsys@useobject{currentmarker}{}%
\end{pgfscope}%
\begin{pgfscope}%
\pgfsys@transformshift{3.386236in}{2.246357in}%
\pgfsys@useobject{currentmarker}{}%
\end{pgfscope}%
\begin{pgfscope}%
\pgfsys@transformshift{3.405015in}{2.252409in}%
\pgfsys@useobject{currentmarker}{}%
\end{pgfscope}%
\begin{pgfscope}%
\pgfsys@transformshift{3.423088in}{2.264274in}%
\pgfsys@useobject{currentmarker}{}%
\end{pgfscope}%
\begin{pgfscope}%
\pgfsys@transformshift{3.441633in}{2.297440in}%
\pgfsys@useobject{currentmarker}{}%
\end{pgfscope}%
\begin{pgfscope}%
\pgfsys@transformshift{3.462054in}{2.346088in}%
\pgfsys@useobject{currentmarker}{}%
\end{pgfscope}%
\begin{pgfscope}%
\pgfsys@transformshift{3.480597in}{2.443567in}%
\pgfsys@useobject{currentmarker}{}%
\end{pgfscope}%
\begin{pgfscope}%
\pgfsys@transformshift{3.498670in}{2.581911in}%
\pgfsys@useobject{currentmarker}{}%
\end{pgfscope}%
\begin{pgfscope}%
\pgfsys@transformshift{3.520032in}{2.630421in}%
\pgfsys@useobject{currentmarker}{}%
\end{pgfscope}%
\begin{pgfscope}%
\pgfsys@transformshift{3.538340in}{2.589946in}%
\pgfsys@useobject{currentmarker}{}%
\end{pgfscope}%
\begin{pgfscope}%
\pgfsys@transformshift{3.557353in}{2.479659in}%
\pgfsys@useobject{currentmarker}{}%
\end{pgfscope}%
\begin{pgfscope}%
\pgfsys@transformshift{3.580123in}{2.360170in}%
\pgfsys@useobject{currentmarker}{}%
\end{pgfscope}%
\begin{pgfscope}%
\pgfsys@transformshift{3.594676in}{2.305143in}%
\pgfsys@useobject{currentmarker}{}%
\end{pgfscope}%
\begin{pgfscope}%
\pgfsys@transformshift{3.615333in}{2.271337in}%
\pgfsys@useobject{currentmarker}{}%
\end{pgfscope}%
\begin{pgfscope}%
\pgfsys@transformshift{3.633641in}{2.254263in}%
\pgfsys@useobject{currentmarker}{}%
\end{pgfscope}%
\begin{pgfscope}%
\pgfsys@transformshift{3.653593in}{2.246910in}%
\pgfsys@useobject{currentmarker}{}%
\end{pgfscope}%
\begin{pgfscope}%
\pgfsys@transformshift{3.674484in}{2.258230in}%
\pgfsys@useobject{currentmarker}{}%
\end{pgfscope}%
\begin{pgfscope}%
\pgfsys@transformshift{3.693732in}{2.285392in}%
\pgfsys@useobject{currentmarker}{}%
\end{pgfscope}%
\begin{pgfscope}%
\pgfsys@transformshift{3.712511in}{2.338864in}%
\pgfsys@useobject{currentmarker}{}%
\end{pgfscope}%
\begin{pgfscope}%
\pgfsys@transformshift{3.729410in}{2.384400in}%
\pgfsys@useobject{currentmarker}{}%
\end{pgfscope}%
\begin{pgfscope}%
\pgfsys@transformshift{3.751475in}{2.503362in}%
\pgfsys@useobject{currentmarker}{}%
\end{pgfscope}%
\begin{pgfscope}%
\pgfsys@transformshift{3.769785in}{2.609599in}%
\pgfsys@useobject{currentmarker}{}%
\end{pgfscope}%
\begin{pgfscope}%
\pgfsys@transformshift{3.789267in}{2.638854in}%
\pgfsys@useobject{currentmarker}{}%
\end{pgfscope}%
\begin{pgfscope}%
\pgfsys@transformshift{3.808046in}{2.598925in}%
\pgfsys@useobject{currentmarker}{}%
\end{pgfscope}%
\begin{pgfscope}%
\pgfsys@transformshift{3.827528in}{2.487515in}%
\pgfsys@useobject{currentmarker}{}%
\end{pgfscope}%
\begin{pgfscope}%
\pgfsys@transformshift{3.847010in}{2.521030in}%
\pgfsys@useobject{currentmarker}{}%
\end{pgfscope}%
\begin{pgfscope}%
\pgfsys@transformshift{3.866258in}{2.642088in}%
\pgfsys@useobject{currentmarker}{}%
\end{pgfscope}%
\begin{pgfscope}%
\pgfsys@transformshift{3.884802in}{2.589284in}%
\pgfsys@useobject{currentmarker}{}%
\end{pgfscope}%
\begin{pgfscope}%
\pgfsys@transformshift{3.904519in}{2.458062in}%
\pgfsys@useobject{currentmarker}{}%
\end{pgfscope}%
\begin{pgfscope}%
\pgfsys@transformshift{3.923532in}{2.349030in}%
\pgfsys@useobject{currentmarker}{}%
\end{pgfscope}%
\begin{pgfscope}%
\pgfsys@transformshift{3.943014in}{2.300148in}%
\pgfsys@useobject{currentmarker}{}%
\end{pgfscope}%
\begin{pgfscope}%
\pgfsys@transformshift{3.961324in}{2.265506in}%
\pgfsys@useobject{currentmarker}{}%
\end{pgfscope}%
\begin{pgfscope}%
\pgfsys@transformshift{3.980806in}{2.250456in}%
\pgfsys@useobject{currentmarker}{}%
\end{pgfscope}%
\begin{pgfscope}%
\pgfsys@transformshift{3.999585in}{2.252995in}%
\pgfsys@useobject{currentmarker}{}%
\end{pgfscope}%
\begin{pgfscope}%
\pgfsys@transformshift{4.020005in}{2.273496in}%
\pgfsys@useobject{currentmarker}{}%
\end{pgfscope}%
\begin{pgfscope}%
\pgfsys@transformshift{4.037609in}{2.307533in}%
\pgfsys@useobject{currentmarker}{}%
\end{pgfscope}%
\begin{pgfscope}%
\pgfsys@transformshift{4.056857in}{2.373218in}%
\pgfsys@useobject{currentmarker}{}%
\end{pgfscope}%
\begin{pgfscope}%
\pgfsys@transformshift{4.076105in}{2.465760in}%
\pgfsys@useobject{currentmarker}{}%
\end{pgfscope}%
\begin{pgfscope}%
\pgfsys@transformshift{4.094415in}{2.600535in}%
\pgfsys@useobject{currentmarker}{}%
\end{pgfscope}%
\begin{pgfscope}%
\pgfsys@transformshift{4.114602in}{2.653201in}%
\pgfsys@useobject{currentmarker}{}%
\end{pgfscope}%
\begin{pgfscope}%
\pgfsys@transformshift{4.132676in}{2.627392in}%
\pgfsys@useobject{currentmarker}{}%
\end{pgfscope}%
\begin{pgfscope}%
\pgfsys@transformshift{4.154975in}{2.516509in}%
\pgfsys@useobject{currentmarker}{}%
\end{pgfscope}%
\begin{pgfscope}%
\pgfsys@transformshift{4.170702in}{2.418605in}%
\pgfsys@useobject{currentmarker}{}%
\end{pgfscope}%
\begin{pgfscope}%
\pgfsys@transformshift{4.192062in}{2.320704in}%
\pgfsys@useobject{currentmarker}{}%
\end{pgfscope}%
\begin{pgfscope}%
\pgfsys@transformshift{4.211780in}{2.283677in}%
\pgfsys@useobject{currentmarker}{}%
\end{pgfscope}%
\begin{pgfscope}%
\pgfsys@transformshift{4.230323in}{2.264049in}%
\pgfsys@useobject{currentmarker}{}%
\end{pgfscope}%
\begin{pgfscope}%
\pgfsys@transformshift{4.248162in}{2.251342in}%
\pgfsys@useobject{currentmarker}{}%
\end{pgfscope}%
\begin{pgfscope}%
\pgfsys@transformshift{4.266001in}{2.258774in}%
\pgfsys@useobject{currentmarker}{}%
\end{pgfscope}%
\begin{pgfscope}%
\pgfsys@transformshift{4.290414in}{2.278835in}%
\pgfsys@useobject{currentmarker}{}%
\end{pgfscope}%
\begin{pgfscope}%
\pgfsys@transformshift{4.308958in}{2.322079in}%
\pgfsys@useobject{currentmarker}{}%
\end{pgfscope}%
\begin{pgfscope}%
\pgfsys@transformshift{4.327266in}{2.380066in}%
\pgfsys@useobject{currentmarker}{}%
\end{pgfscope}%
\begin{pgfscope}%
\pgfsys@transformshift{4.346514in}{2.463214in}%
\pgfsys@useobject{currentmarker}{}%
\end{pgfscope}%
\begin{pgfscope}%
\pgfsys@transformshift{4.365293in}{2.601546in}%
\pgfsys@useobject{currentmarker}{}%
\end{pgfscope}%
\begin{pgfscope}%
\pgfsys@transformshift{4.384306in}{2.661443in}%
\pgfsys@useobject{currentmarker}{}%
\end{pgfscope}%
\begin{pgfscope}%
\pgfsys@transformshift{4.402848in}{2.653977in}%
\pgfsys@useobject{currentmarker}{}%
\end{pgfscope}%
\begin{pgfscope}%
\pgfsys@transformshift{4.421862in}{2.597055in}%
\pgfsys@useobject{currentmarker}{}%
\end{pgfscope}%
\begin{pgfscope}%
\pgfsys@transformshift{4.441109in}{2.476988in}%
\pgfsys@useobject{currentmarker}{}%
\end{pgfscope}%
\begin{pgfscope}%
\pgfsys@transformshift{4.462471in}{2.350737in}%
\pgfsys@useobject{currentmarker}{}%
\end{pgfscope}%
\begin{pgfscope}%
\pgfsys@transformshift{4.481718in}{2.304596in}%
\pgfsys@useobject{currentmarker}{}%
\end{pgfscope}%
\begin{pgfscope}%
\pgfsys@transformshift{4.482422in}{2.303919in}%
\pgfsys@useobject{currentmarker}{}%
\end{pgfscope}%
\begin{pgfscope}%
\pgfsys@transformshift{4.474676in}{2.330092in}%
\pgfsys@useobject{currentmarker}{}%
\end{pgfscope}%
\begin{pgfscope}%
\pgfsys@transformshift{4.455663in}{2.461569in}%
\pgfsys@useobject{currentmarker}{}%
\end{pgfscope}%
\begin{pgfscope}%
\pgfsys@transformshift{4.432190in}{2.644566in}%
\pgfsys@useobject{currentmarker}{}%
\end{pgfscope}%
\begin{pgfscope}%
\pgfsys@transformshift{4.416933in}{2.664407in}%
\pgfsys@useobject{currentmarker}{}%
\end{pgfscope}%
\begin{pgfscope}%
\pgfsys@transformshift{4.393694in}{2.508161in}%
\pgfsys@useobject{currentmarker}{}%
\end{pgfscope}%
\begin{pgfscope}%
\pgfsys@transformshift{4.377264in}{2.380699in}%
\pgfsys@useobject{currentmarker}{}%
\end{pgfscope}%
\begin{pgfscope}%
\pgfsys@transformshift{4.356373in}{2.291905in}%
\pgfsys@useobject{currentmarker}{}%
\end{pgfscope}%
\begin{pgfscope}%
\pgfsys@transformshift{4.338534in}{2.257777in}%
\pgfsys@useobject{currentmarker}{}%
\end{pgfscope}%
\begin{pgfscope}%
\pgfsys@transformshift{4.319989in}{2.254855in}%
\pgfsys@useobject{currentmarker}{}%
\end{pgfscope}%
\begin{pgfscope}%
\pgfsys@transformshift{4.302385in}{2.286249in}%
\pgfsys@useobject{currentmarker}{}%
\end{pgfscope}%
\begin{pgfscope}%
\pgfsys@transformshift{4.276329in}{2.414339in}%
\pgfsys@useobject{currentmarker}{}%
\end{pgfscope}%
\begin{pgfscope}%
\pgfsys@transformshift{4.262012in}{2.547728in}%
\pgfsys@useobject{currentmarker}{}%
\end{pgfscope}%
\begin{pgfscope}%
\pgfsys@transformshift{4.242528in}{2.650851in}%
\pgfsys@useobject{currentmarker}{}%
\end{pgfscope}%
\begin{pgfscope}%
\pgfsys@transformshift{4.224220in}{2.615352in}%
\pgfsys@useobject{currentmarker}{}%
\end{pgfscope}%
\begin{pgfscope}%
\pgfsys@transformshift{4.205207in}{2.451254in}%
\pgfsys@useobject{currentmarker}{}%
\end{pgfscope}%
\begin{pgfscope}%
\pgfsys@transformshift{4.184082in}{2.317713in}%
\pgfsys@useobject{currentmarker}{}%
\end{pgfscope}%
\begin{pgfscope}%
\pgfsys@transformshift{4.165303in}{2.266855in}%
\pgfsys@useobject{currentmarker}{}%
\end{pgfscope}%
\begin{pgfscope}%
\pgfsys@transformshift{4.146758in}{2.249329in}%
\pgfsys@useobject{currentmarker}{}%
\end{pgfscope}%
\begin{pgfscope}%
\pgfsys@transformshift{4.127982in}{2.265849in}%
\pgfsys@useobject{currentmarker}{}%
\end{pgfscope}%
\begin{pgfscope}%
\pgfsys@transformshift{4.108968in}{2.314323in}%
\pgfsys@useobject{currentmarker}{}%
\end{pgfscope}%
\begin{pgfscope}%
\pgfsys@transformshift{4.090893in}{2.423183in}%
\pgfsys@useobject{currentmarker}{}%
\end{pgfscope}%
\begin{pgfscope}%
\pgfsys@transformshift{4.069299in}{2.605826in}%
\pgfsys@useobject{currentmarker}{}%
\end{pgfscope}%
\begin{pgfscope}%
\pgfsys@transformshift{4.050051in}{2.641596in}%
\pgfsys@useobject{currentmarker}{}%
\end{pgfscope}%
\begin{pgfscope}%
\pgfsys@transformshift{4.034558in}{2.558674in}%
\pgfsys@useobject{currentmarker}{}%
\end{pgfscope}%
\begin{pgfscope}%
\pgfsys@transformshift{4.012964in}{2.377707in}%
\pgfsys@useobject{currentmarker}{}%
\end{pgfscope}%
\begin{pgfscope}%
\pgfsys@transformshift{3.991603in}{2.287723in}%
\pgfsys@useobject{currentmarker}{}%
\end{pgfscope}%
\begin{pgfscope}%
\pgfsys@transformshift{3.975641in}{2.258679in}%
\pgfsys@useobject{currentmarker}{}%
\end{pgfscope}%
\begin{pgfscope}%
\pgfsys@transformshift{3.957333in}{2.248198in}%
\pgfsys@useobject{currentmarker}{}%
\end{pgfscope}%
\begin{pgfscope}%
\pgfsys@transformshift{3.936442in}{2.274309in}%
\pgfsys@useobject{currentmarker}{}%
\end{pgfscope}%
\begin{pgfscope}%
\pgfsys@transformshift{3.916724in}{2.336187in}%
\pgfsys@useobject{currentmarker}{}%
\end{pgfscope}%
\begin{pgfscope}%
\pgfsys@transformshift{3.898181in}{2.468251in}%
\pgfsys@useobject{currentmarker}{}%
\end{pgfscope}%
\begin{pgfscope}%
\pgfsys@transformshift{3.875646in}{2.623692in}%
\pgfsys@useobject{currentmarker}{}%
\end{pgfscope}%
\begin{pgfscope}%
\pgfsys@transformshift{3.858041in}{2.613842in}%
\pgfsys@useobject{currentmarker}{}%
\end{pgfscope}%
\begin{pgfscope}%
\pgfsys@transformshift{3.842316in}{2.492787in}%
\pgfsys@useobject{currentmarker}{}%
\end{pgfscope}%
\begin{pgfscope}%
\pgfsys@transformshift{3.817669in}{2.321835in}%
\pgfsys@useobject{currentmarker}{}%
\end{pgfscope}%
\begin{pgfscope}%
\pgfsys@transformshift{3.801472in}{2.282690in}%
\pgfsys@useobject{currentmarker}{}%
\end{pgfscope}%
\begin{pgfscope}%
\pgfsys@transformshift{3.783633in}{2.252453in}%
\pgfsys@useobject{currentmarker}{}%
\end{pgfscope}%
\begin{pgfscope}%
\pgfsys@transformshift{3.766028in}{2.248335in}%
\pgfsys@useobject{currentmarker}{}%
\end{pgfscope}%
\begin{pgfscope}%
\pgfsys@transformshift{3.743258in}{2.276791in}%
\pgfsys@useobject{currentmarker}{}%
\end{pgfscope}%
\begin{pgfscope}%
\pgfsys@transformshift{3.724247in}{2.337001in}%
\pgfsys@useobject{currentmarker}{}%
\end{pgfscope}%
\begin{pgfscope}%
\pgfsys@transformshift{3.705937in}{2.467366in}%
\pgfsys@useobject{currentmarker}{}%
\end{pgfscope}%
\begin{pgfscope}%
\pgfsys@transformshift{3.687158in}{2.605288in}%
\pgfsys@useobject{currentmarker}{}%
\end{pgfscope}%
\begin{pgfscope}%
\pgfsys@transformshift{3.668850in}{2.627046in}%
\pgfsys@useobject{currentmarker}{}%
\end{pgfscope}%
\begin{pgfscope}%
\pgfsys@transformshift{3.646786in}{2.513221in}%
\pgfsys@useobject{currentmarker}{}%
\end{pgfscope}%
\begin{pgfscope}%
\pgfsys@transformshift{3.627772in}{2.374917in}%
\pgfsys@useobject{currentmarker}{}%
\end{pgfscope}%
\begin{pgfscope}%
\pgfsys@transformshift{3.610402in}{2.301779in}%
\pgfsys@useobject{currentmarker}{}%
\end{pgfscope}%
\begin{pgfscope}%
\pgfsys@transformshift{3.588574in}{2.263795in}%
\pgfsys@useobject{currentmarker}{}%
\end{pgfscope}%
\begin{pgfscope}%
\pgfsys@transformshift{3.572377in}{2.246586in}%
\pgfsys@useobject{currentmarker}{}%
\end{pgfscope}%
\begin{pgfscope}%
\pgfsys@transformshift{3.551016in}{2.249752in}%
\pgfsys@useobject{currentmarker}{}%
\end{pgfscope}%
\begin{pgfscope}%
\pgfsys@transformshift{3.531768in}{2.272162in}%
\pgfsys@useobject{currentmarker}{}%
\end{pgfscope}%
\begin{pgfscope}%
\pgfsys@transformshift{3.513224in}{2.321977in}%
\pgfsys@useobject{currentmarker}{}%
\end{pgfscope}%
\begin{pgfscope}%
\pgfsys@transformshift{3.494916in}{2.424905in}%
\pgfsys@useobject{currentmarker}{}%
\end{pgfscope}%
\begin{pgfscope}%
\pgfsys@transformshift{3.474494in}{2.592570in}%
\pgfsys@useobject{currentmarker}{}%
\end{pgfscope}%
\begin{pgfscope}%
\pgfsys@transformshift{3.455715in}{2.622534in}%
\pgfsys@useobject{currentmarker}{}%
\end{pgfscope}%
\begin{pgfscope}%
\pgfsys@transformshift{3.436702in}{2.545541in}%
\pgfsys@useobject{currentmarker}{}%
\end{pgfscope}%
\begin{pgfscope}%
\pgfsys@transformshift{3.417925in}{2.401111in}%
\pgfsys@useobject{currentmarker}{}%
\end{pgfscope}%
\begin{pgfscope}%
\pgfsys@transformshift{3.398912in}{2.316111in}%
\pgfsys@useobject{currentmarker}{}%
\end{pgfscope}%
\begin{pgfscope}%
\pgfsys@transformshift{3.380133in}{2.270063in}%
\pgfsys@useobject{currentmarker}{}%
\end{pgfscope}%
\begin{pgfscope}%
\pgfsys@transformshift{3.359477in}{2.245931in}%
\pgfsys@useobject{currentmarker}{}%
\end{pgfscope}%
\begin{pgfscope}%
\pgfsys@transformshift{3.339760in}{2.248412in}%
\pgfsys@useobject{currentmarker}{}%
\end{pgfscope}%
\begin{pgfscope}%
\pgfsys@transformshift{3.320513in}{2.270766in}%
\pgfsys@useobject{currentmarker}{}%
\end{pgfscope}%
\begin{pgfscope}%
\pgfsys@transformshift{3.302437in}{2.321929in}%
\pgfsys@useobject{currentmarker}{}%
\end{pgfscope}%
\begin{pgfscope}%
\pgfsys@transformshift{3.280138in}{2.450138in}%
\pgfsys@useobject{currentmarker}{}%
\end{pgfscope}%
\begin{pgfscope}%
\pgfsys@transformshift{3.261596in}{2.568595in}%
\pgfsys@useobject{currentmarker}{}%
\end{pgfscope}%
\begin{pgfscope}%
\pgfsys@transformshift{3.242817in}{2.617173in}%
\pgfsys@useobject{currentmarker}{}%
\end{pgfscope}%
\begin{pgfscope}%
\pgfsys@transformshift{3.224507in}{2.554237in}%
\pgfsys@useobject{currentmarker}{}%
\end{pgfscope}%
\begin{pgfscope}%
\pgfsys@transformshift{3.205730in}{2.428764in}%
\pgfsys@useobject{currentmarker}{}%
\end{pgfscope}%
\begin{pgfscope}%
\pgfsys@transformshift{3.186951in}{2.332725in}%
\pgfsys@useobject{currentmarker}{}%
\end{pgfscope}%
\begin{pgfscope}%
\pgfsys@transformshift{3.168877in}{2.288562in}%
\pgfsys@useobject{currentmarker}{}%
\end{pgfscope}%
\begin{pgfscope}%
\pgfsys@transformshift{3.147047in}{2.253453in}%
\pgfsys@useobject{currentmarker}{}%
\end{pgfscope}%
\begin{pgfscope}%
\pgfsys@transformshift{3.128737in}{2.244178in}%
\pgfsys@useobject{currentmarker}{}%
\end{pgfscope}%
\begin{pgfscope}%
\pgfsys@transformshift{3.109021in}{2.257086in}%
\pgfsys@useobject{currentmarker}{}%
\end{pgfscope}%
\begin{pgfscope}%
\pgfsys@transformshift{3.091416in}{2.286794in}%
\pgfsys@useobject{currentmarker}{}%
\end{pgfscope}%
\begin{pgfscope}%
\pgfsys@transformshift{3.072871in}{2.360863in}%
\pgfsys@useobject{currentmarker}{}%
\end{pgfscope}%
\begin{pgfscope}%
\pgfsys@transformshift{3.051043in}{2.507044in}%
\pgfsys@useobject{currentmarker}{}%
\end{pgfscope}%
\begin{pgfscope}%
\pgfsys@transformshift{3.036019in}{2.607876in}%
\pgfsys@useobject{currentmarker}{}%
\end{pgfscope}%
\begin{pgfscope}%
\pgfsys@transformshift{3.013720in}{2.608122in}%
\pgfsys@useobject{currentmarker}{}%
\end{pgfscope}%
\begin{pgfscope}%
\pgfsys@transformshift{2.995646in}{2.591524in}%
\pgfsys@useobject{currentmarker}{}%
\end{pgfscope}%
\begin{pgfscope}%
\pgfsys@transformshift{2.976633in}{2.465928in}%
\pgfsys@useobject{currentmarker}{}%
\end{pgfscope}%
\begin{pgfscope}%
\pgfsys@transformshift{2.957151in}{2.347500in}%
\pgfsys@useobject{currentmarker}{}%
\end{pgfscope}%
\begin{pgfscope}%
\pgfsys@transformshift{2.939077in}{2.288177in}%
\pgfsys@useobject{currentmarker}{}%
\end{pgfscope}%
\begin{pgfscope}%
\pgfsys@transformshift{2.916778in}{2.254613in}%
\pgfsys@useobject{currentmarker}{}%
\end{pgfscope}%
\begin{pgfscope}%
\pgfsys@transformshift{2.899642in}{2.245237in}%
\pgfsys@useobject{currentmarker}{}%
\end{pgfscope}%
\begin{pgfscope}%
\pgfsys@transformshift{2.877812in}{2.248156in}%
\pgfsys@useobject{currentmarker}{}%
\end{pgfscope}%
\begin{pgfscope}%
\pgfsys@transformshift{2.859504in}{2.262791in}%
\pgfsys@useobject{currentmarker}{}%
\end{pgfscope}%
\begin{pgfscope}%
\pgfsys@transformshift{2.840020in}{2.285212in}%
\pgfsys@useobject{currentmarker}{}%
\end{pgfscope}%
\begin{pgfscope}%
\pgfsys@transformshift{2.823120in}{2.342448in}%
\pgfsys@useobject{currentmarker}{}%
\end{pgfscope}%
\begin{pgfscope}%
\pgfsys@transformshift{2.802933in}{2.482439in}%
\pgfsys@useobject{currentmarker}{}%
\end{pgfscope}%
\begin{pgfscope}%
\pgfsys@transformshift{2.782511in}{2.596174in}%
\pgfsys@useobject{currentmarker}{}%
\end{pgfscope}%
\begin{pgfscope}%
\pgfsys@transformshift{2.763735in}{2.603506in}%
\pgfsys@useobject{currentmarker}{}%
\end{pgfscope}%
\begin{pgfscope}%
\pgfsys@transformshift{2.747067in}{2.530819in}%
\pgfsys@useobject{currentmarker}{}%
\end{pgfscope}%
\begin{pgfscope}%
\pgfsys@transformshift{2.725474in}{2.388370in}%
\pgfsys@useobject{currentmarker}{}%
\end{pgfscope}%
\begin{pgfscope}%
\pgfsys@transformshift{2.706460in}{2.307690in}%
\pgfsys@useobject{currentmarker}{}%
\end{pgfscope}%
\begin{pgfscope}%
\pgfsys@transformshift{2.687916in}{2.297256in}%
\pgfsys@useobject{currentmarker}{}%
\end{pgfscope}%
\begin{pgfscope}%
\pgfsys@transformshift{2.688150in}{2.272861in}%
\pgfsys@useobject{currentmarker}{}%
\end{pgfscope}%
\begin{pgfscope}%
\pgfsys@transformshift{2.670311in}{2.264255in}%
\pgfsys@useobject{currentmarker}{}%
\end{pgfscope}%
\begin{pgfscope}%
\pgfsys@transformshift{2.647543in}{2.245363in}%
\pgfsys@useobject{currentmarker}{}%
\end{pgfscope}%
\begin{pgfscope}%
\pgfsys@transformshift{2.628764in}{2.249147in}%
\pgfsys@useobject{currentmarker}{}%
\end{pgfscope}%
\begin{pgfscope}%
\pgfsys@transformshift{2.610925in}{2.269097in}%
\pgfsys@useobject{currentmarker}{}%
\end{pgfscope}%
\begin{pgfscope}%
\pgfsys@transformshift{2.592146in}{2.302756in}%
\pgfsys@useobject{currentmarker}{}%
\end{pgfscope}%
\begin{pgfscope}%
\pgfsys@transformshift{2.575247in}{2.386467in}%
\pgfsys@useobject{currentmarker}{}%
\end{pgfscope}%
\begin{pgfscope}%
\pgfsys@transformshift{2.551539in}{2.483314in}%
\pgfsys@useobject{currentmarker}{}%
\end{pgfscope}%
\begin{pgfscope}%
\pgfsys@transformshift{2.532995in}{2.593089in}%
\pgfsys@useobject{currentmarker}{}%
\end{pgfscope}%
\begin{pgfscope}%
\pgfsys@transformshift{2.514216in}{2.599267in}%
\pgfsys@useobject{currentmarker}{}%
\end{pgfscope}%
\begin{pgfscope}%
\pgfsys@transformshift{2.497551in}{2.509048in}%
\pgfsys@useobject{currentmarker}{}%
\end{pgfscope}%
\begin{pgfscope}%
\pgfsys@transformshift{2.474078in}{2.356499in}%
\pgfsys@useobject{currentmarker}{}%
\end{pgfscope}%
\begin{pgfscope}%
\pgfsys@transformshift{2.457647in}{2.293149in}%
\pgfsys@useobject{currentmarker}{}%
\end{pgfscope}%
\begin{pgfscope}%
\pgfsys@transformshift{2.437460in}{2.258586in}%
\pgfsys@useobject{currentmarker}{}%
\end{pgfscope}%
\begin{pgfscope}%
\pgfsys@transformshift{2.419152in}{2.258469in}%
\pgfsys@useobject{currentmarker}{}%
\end{pgfscope}%
\begin{pgfscope}%
\pgfsys@transformshift{2.397321in}{2.246771in}%
\pgfsys@useobject{currentmarker}{}%
\end{pgfscope}%
\begin{pgfscope}%
\pgfsys@transformshift{2.378777in}{2.245416in}%
\pgfsys@useobject{currentmarker}{}%
\end{pgfscope}%
\begin{pgfscope}%
\pgfsys@transformshift{2.361172in}{2.261542in}%
\pgfsys@useobject{currentmarker}{}%
\end{pgfscope}%
\begin{pgfscope}%
\pgfsys@transformshift{2.338873in}{2.290041in}%
\pgfsys@useobject{currentmarker}{}%
\end{pgfscope}%
\begin{pgfscope}%
\pgfsys@transformshift{2.322911in}{2.347735in}%
\pgfsys@useobject{currentmarker}{}%
\end{pgfscope}%
\begin{pgfscope}%
\pgfsys@transformshift{2.301083in}{2.508962in}%
\pgfsys@useobject{currentmarker}{}%
\end{pgfscope}%
\begin{pgfscope}%
\pgfsys@transformshift{2.279018in}{2.600577in}%
\pgfsys@useobject{currentmarker}{}%
\end{pgfscope}%
\begin{pgfscope}%
\pgfsys@transformshift{2.263760in}{2.607105in}%
\pgfsys@useobject{currentmarker}{}%
\end{pgfscope}%
\begin{pgfscope}%
\pgfsys@transformshift{2.245921in}{2.547142in}%
\pgfsys@useobject{currentmarker}{}%
\end{pgfscope}%
\begin{pgfscope}%
\pgfsys@transformshift{2.227142in}{2.445463in}%
\pgfsys@useobject{currentmarker}{}%
\end{pgfscope}%
\begin{pgfscope}%
\pgfsys@transformshift{2.208365in}{2.336364in}%
\pgfsys@useobject{currentmarker}{}%
\end{pgfscope}%
\begin{pgfscope}%
\pgfsys@transformshift{2.187004in}{2.280396in}%
\pgfsys@useobject{currentmarker}{}%
\end{pgfscope}%
\begin{pgfscope}%
\pgfsys@transformshift{2.168695in}{2.257282in}%
\pgfsys@useobject{currentmarker}{}%
\end{pgfscope}%
\begin{pgfscope}%
\pgfsys@transformshift{2.149917in}{2.247869in}%
\pgfsys@useobject{currentmarker}{}%
\end{pgfscope}%
\begin{pgfscope}%
\pgfsys@transformshift{2.131843in}{2.246186in}%
\pgfsys@useobject{currentmarker}{}%
\end{pgfscope}%
\begin{pgfscope}%
\pgfsys@transformshift{2.110482in}{2.265416in}%
\pgfsys@useobject{currentmarker}{}%
\end{pgfscope}%
\begin{pgfscope}%
\pgfsys@transformshift{2.092174in}{2.299745in}%
\pgfsys@useobject{currentmarker}{}%
\end{pgfscope}%
\begin{pgfscope}%
\pgfsys@transformshift{2.073160in}{2.378482in}%
\pgfsys@useobject{currentmarker}{}%
\end{pgfscope}%
\begin{pgfscope}%
\pgfsys@transformshift{2.050861in}{2.500425in}%
\pgfsys@useobject{currentmarker}{}%
\end{pgfscope}%
\begin{pgfscope}%
\pgfsys@transformshift{2.033022in}{2.587472in}%
\pgfsys@useobject{currentmarker}{}%
\end{pgfscope}%
\begin{pgfscope}%
\pgfsys@transformshift{2.014946in}{2.615812in}%
\pgfsys@useobject{currentmarker}{}%
\end{pgfscope}%
\begin{pgfscope}%
\pgfsys@transformshift{1.995699in}{2.596504in}%
\pgfsys@useobject{currentmarker}{}%
\end{pgfscope}%
\begin{pgfscope}%
\pgfsys@transformshift{1.975748in}{2.487969in}%
\pgfsys@useobject{currentmarker}{}%
\end{pgfscope}%
\begin{pgfscope}%
\pgfsys@transformshift{1.956969in}{2.358753in}%
\pgfsys@useobject{currentmarker}{}%
\end{pgfscope}%
\begin{pgfscope}%
\pgfsys@transformshift{1.938190in}{2.292577in}%
\pgfsys@useobject{currentmarker}{}%
\end{pgfscope}%
\begin{pgfscope}%
\pgfsys@transformshift{1.921056in}{2.278304in}%
\pgfsys@useobject{currentmarker}{}%
\end{pgfscope}%
\begin{pgfscope}%
\pgfsys@transformshift{1.897112in}{2.470788in}%
\pgfsys@useobject{currentmarker}{}%
\end{pgfscope}%
\begin{pgfscope}%
\pgfsys@transformshift{1.878804in}{2.347865in}%
\pgfsys@useobject{currentmarker}{}%
\end{pgfscope}%
\begin{pgfscope}%
\pgfsys@transformshift{1.860731in}{2.288130in}%
\pgfsys@useobject{currentmarker}{}%
\end{pgfscope}%
\begin{pgfscope}%
\pgfsys@transformshift{1.841717in}{2.255167in}%
\pgfsys@useobject{currentmarker}{}%
\end{pgfscope}%
\begin{pgfscope}%
\pgfsys@transformshift{1.820827in}{2.244944in}%
\pgfsys@useobject{currentmarker}{}%
\end{pgfscope}%
\begin{pgfscope}%
\pgfsys@transformshift{1.801814in}{2.252858in}%
\pgfsys@useobject{currentmarker}{}%
\end{pgfscope}%
\begin{pgfscope}%
\pgfsys@transformshift{1.783974in}{2.277515in}%
\pgfsys@useobject{currentmarker}{}%
\end{pgfscope}%
\begin{pgfscope}%
\pgfsys@transformshift{1.765664in}{2.333730in}%
\pgfsys@useobject{currentmarker}{}%
\end{pgfscope}%
\begin{pgfscope}%
\pgfsys@transformshift{1.744539in}{2.458432in}%
\pgfsys@useobject{currentmarker}{}%
\end{pgfscope}%
\begin{pgfscope}%
\pgfsys@transformshift{1.725760in}{2.586409in}%
\pgfsys@useobject{currentmarker}{}%
\end{pgfscope}%
\begin{pgfscope}%
\pgfsys@transformshift{1.706982in}{2.616275in}%
\pgfsys@useobject{currentmarker}{}%
\end{pgfscope}%
\begin{pgfscope}%
\pgfsys@transformshift{1.688439in}{2.620092in}%
\pgfsys@useobject{currentmarker}{}%
\end{pgfscope}%
\begin{pgfscope}%
\pgfsys@transformshift{1.670364in}{2.552947in}%
\pgfsys@useobject{currentmarker}{}%
\end{pgfscope}%
\begin{pgfscope}%
\pgfsys@transformshift{1.650881in}{2.420981in}%
\pgfsys@useobject{currentmarker}{}%
\end{pgfscope}%
\begin{pgfscope}%
\pgfsys@transformshift{1.627408in}{2.323594in}%
\pgfsys@useobject{currentmarker}{}%
\end{pgfscope}%
\begin{pgfscope}%
\pgfsys@transformshift{1.612855in}{2.286456in}%
\pgfsys@useobject{currentmarker}{}%
\end{pgfscope}%
\begin{pgfscope}%
\pgfsys@transformshift{1.591730in}{2.263959in}%
\pgfsys@useobject{currentmarker}{}%
\end{pgfscope}%
\begin{pgfscope}%
\pgfsys@transformshift{1.574360in}{2.248955in}%
\pgfsys@useobject{currentmarker}{}%
\end{pgfscope}%
\begin{pgfscope}%
\pgfsys@transformshift{1.554174in}{2.247880in}%
\pgfsys@useobject{currentmarker}{}%
\end{pgfscope}%
\begin{pgfscope}%
\pgfsys@transformshift{1.533753in}{2.266308in}%
\pgfsys@useobject{currentmarker}{}%
\end{pgfscope}%
\begin{pgfscope}%
\pgfsys@transformshift{1.515913in}{2.300475in}%
\pgfsys@useobject{currentmarker}{}%
\end{pgfscope}%
\begin{pgfscope}%
\pgfsys@transformshift{1.496195in}{2.378649in}%
\pgfsys@useobject{currentmarker}{}%
\end{pgfscope}%
\begin{pgfscope}%
\pgfsys@transformshift{1.475070in}{2.518103in}%
\pgfsys@useobject{currentmarker}{}%
\end{pgfscope}%
\begin{pgfscope}%
\pgfsys@transformshift{1.457699in}{2.605413in}%
\pgfsys@useobject{currentmarker}{}%
\end{pgfscope}%
\begin{pgfscope}%
\pgfsys@transformshift{1.438217in}{2.629660in}%
\pgfsys@useobject{currentmarker}{}%
\end{pgfscope}%
\begin{pgfscope}%
\pgfsys@transformshift{1.420144in}{2.595455in}%
\pgfsys@useobject{currentmarker}{}%
\end{pgfscope}%
\begin{pgfscope}%
\pgfsys@transformshift{1.420378in}{2.502533in}%
\pgfsys@useobject{currentmarker}{}%
\end{pgfscope}%
\begin{pgfscope}%
\pgfsys@transformshift{1.396434in}{2.434211in}%
\pgfsys@useobject{currentmarker}{}%
\end{pgfscope}%
\begin{pgfscope}%
\pgfsys@transformshift{1.380943in}{2.357019in}%
\pgfsys@useobject{currentmarker}{}%
\end{pgfscope}%
\begin{pgfscope}%
\pgfsys@transformshift{1.360756in}{2.294587in}%
\pgfsys@useobject{currentmarker}{}%
\end{pgfscope}%
\begin{pgfscope}%
\pgfsys@transformshift{1.343151in}{2.269466in}%
\pgfsys@useobject{currentmarker}{}%
\end{pgfscope}%
\begin{pgfscope}%
\pgfsys@transformshift{1.324374in}{2.251213in}%
\pgfsys@useobject{currentmarker}{}%
\end{pgfscope}%
\begin{pgfscope}%
\pgfsys@transformshift{1.305595in}{2.247653in}%
\pgfsys@useobject{currentmarker}{}%
\end{pgfscope}%
\begin{pgfscope}%
\pgfsys@transformshift{1.283765in}{2.262238in}%
\pgfsys@useobject{currentmarker}{}%
\end{pgfscope}%
\begin{pgfscope}%
\pgfsys@transformshift{1.265926in}{2.288730in}%
\pgfsys@useobject{currentmarker}{}%
\end{pgfscope}%
\begin{pgfscope}%
\pgfsys@transformshift{1.245270in}{2.350368in}%
\pgfsys@useobject{currentmarker}{}%
\end{pgfscope}%
\begin{pgfscope}%
\pgfsys@transformshift{1.228134in}{2.447524in}%
\pgfsys@useobject{currentmarker}{}%
\end{pgfscope}%
\begin{pgfscope}%
\pgfsys@transformshift{1.208417in}{2.554389in}%
\pgfsys@useobject{currentmarker}{}%
\end{pgfscope}%
\begin{pgfscope}%
\pgfsys@transformshift{1.188464in}{2.630742in}%
\pgfsys@useobject{currentmarker}{}%
\end{pgfscope}%
\begin{pgfscope}%
\pgfsys@transformshift{1.168748in}{2.638589in}%
\pgfsys@useobject{currentmarker}{}%
\end{pgfscope}%
\begin{pgfscope}%
\pgfsys@transformshift{1.147152in}{2.617889in}%
\pgfsys@useobject{currentmarker}{}%
\end{pgfscope}%
\begin{pgfscope}%
\pgfsys@transformshift{1.133069in}{2.534503in}%
\pgfsys@useobject{currentmarker}{}%
\end{pgfscope}%
\begin{pgfscope}%
\pgfsys@transformshift{1.109831in}{2.416006in}%
\pgfsys@useobject{currentmarker}{}%
\end{pgfscope}%
\begin{pgfscope}%
\pgfsys@transformshift{1.094809in}{2.343033in}%
\pgfsys@useobject{currentmarker}{}%
\end{pgfscope}%
\begin{pgfscope}%
\pgfsys@transformshift{1.073682in}{2.290209in}%
\pgfsys@useobject{currentmarker}{}%
\end{pgfscope}%
\begin{pgfscope}%
\pgfsys@transformshift{1.053731in}{2.262504in}%
\pgfsys@useobject{currentmarker}{}%
\end{pgfscope}%
\begin{pgfscope}%
\pgfsys@transformshift{1.032606in}{2.248785in}%
\pgfsys@useobject{currentmarker}{}%
\end{pgfscope}%
\begin{pgfscope}%
\pgfsys@transformshift{1.015001in}{2.254056in}%
\pgfsys@useobject{currentmarker}{}%
\end{pgfscope}%
\begin{pgfscope}%
\pgfsys@transformshift{0.997631in}{2.256943in}%
\pgfsys@useobject{currentmarker}{}%
\end{pgfscope}%
\begin{pgfscope}%
\pgfsys@transformshift{0.976740in}{2.281434in}%
\pgfsys@useobject{currentmarker}{}%
\end{pgfscope}%
\begin{pgfscope}%
\pgfsys@transformshift{0.959135in}{2.319749in}%
\pgfsys@useobject{currentmarker}{}%
\end{pgfscope}%
\begin{pgfscope}%
\pgfsys@transformshift{0.938479in}{2.415536in}%
\pgfsys@useobject{currentmarker}{}%
\end{pgfscope}%
\begin{pgfscope}%
\pgfsys@transformshift{0.917823in}{2.552392in}%
\pgfsys@useobject{currentmarker}{}%
\end{pgfscope}%
\begin{pgfscope}%
\pgfsys@transformshift{0.899513in}{2.635262in}%
\pgfsys@useobject{currentmarker}{}%
\end{pgfscope}%
\begin{pgfscope}%
\pgfsys@transformshift{0.882142in}{2.651472in}%
\pgfsys@useobject{currentmarker}{}%
\end{pgfscope}%
\begin{pgfscope}%
\pgfsys@transformshift{0.861723in}{2.592880in}%
\pgfsys@useobject{currentmarker}{}%
\end{pgfscope}%
\begin{pgfscope}%
\pgfsys@transformshift{0.842239in}{2.507925in}%
\pgfsys@useobject{currentmarker}{}%
\end{pgfscope}%
\begin{pgfscope}%
\pgfsys@transformshift{0.824870in}{2.389325in}%
\pgfsys@useobject{currentmarker}{}%
\end{pgfscope}%
\begin{pgfscope}%
\pgfsys@transformshift{0.805386in}{2.347927in}%
\pgfsys@useobject{currentmarker}{}%
\end{pgfscope}%
\begin{pgfscope}%
\pgfsys@transformshift{0.784027in}{2.291472in}%
\pgfsys@useobject{currentmarker}{}%
\end{pgfscope}%
\begin{pgfscope}%
\pgfsys@transformshift{0.765014in}{2.266414in}%
\pgfsys@useobject{currentmarker}{}%
\end{pgfscope}%
\begin{pgfscope}%
\pgfsys@transformshift{0.746235in}{2.340896in}%
\pgfsys@useobject{currentmarker}{}%
\end{pgfscope}%
\begin{pgfscope}%
\pgfsys@transformshift{0.726284in}{2.298490in}%
\pgfsys@useobject{currentmarker}{}%
\end{pgfscope}%
\begin{pgfscope}%
\pgfsys@transformshift{0.708208in}{2.264618in}%
\pgfsys@useobject{currentmarker}{}%
\end{pgfscope}%
\begin{pgfscope}%
\pgfsys@transformshift{0.690603in}{2.251402in}%
\pgfsys@useobject{currentmarker}{}%
\end{pgfscope}%
\begin{pgfscope}%
\pgfsys@transformshift{0.669010in}{2.263838in}%
\pgfsys@useobject{currentmarker}{}%
\end{pgfscope}%
\begin{pgfscope}%
\pgfsys@transformshift{0.648588in}{2.295523in}%
\pgfsys@useobject{currentmarker}{}%
\end{pgfscope}%
\begin{pgfscope}%
\pgfsys@transformshift{0.651405in}{2.291066in}%
\pgfsys@useobject{currentmarker}{}%
\end{pgfscope}%
\begin{pgfscope}%
\pgfsys@transformshift{0.656802in}{2.275455in}%
\pgfsys@useobject{currentmarker}{}%
\end{pgfscope}%
\begin{pgfscope}%
\pgfsys@transformshift{0.675815in}{2.251158in}%
\pgfsys@useobject{currentmarker}{}%
\end{pgfscope}%
\begin{pgfscope}%
\pgfsys@transformshift{0.695534in}{2.268705in}%
\pgfsys@useobject{currentmarker}{}%
\end{pgfscope}%
\begin{pgfscope}%
\pgfsys@transformshift{0.715016in}{2.317293in}%
\pgfsys@useobject{currentmarker}{}%
\end{pgfscope}%
\begin{pgfscope}%
\pgfsys@transformshift{0.732855in}{2.415837in}%
\pgfsys@useobject{currentmarker}{}%
\end{pgfscope}%
\begin{pgfscope}%
\pgfsys@transformshift{0.751869in}{2.584478in}%
\pgfsys@useobject{currentmarker}{}%
\end{pgfscope}%
\begin{pgfscope}%
\pgfsys@transformshift{0.771351in}{2.654755in}%
\pgfsys@useobject{currentmarker}{}%
\end{pgfscope}%
\begin{pgfscope}%
\pgfsys@transformshift{0.789424in}{2.587992in}%
\pgfsys@useobject{currentmarker}{}%
\end{pgfscope}%
\begin{pgfscope}%
\pgfsys@transformshift{0.809612in}{2.402748in}%
\pgfsys@useobject{currentmarker}{}%
\end{pgfscope}%
\begin{pgfscope}%
\pgfsys@transformshift{0.829094in}{2.306950in}%
\pgfsys@useobject{currentmarker}{}%
\end{pgfscope}%
\begin{pgfscope}%
\pgfsys@transformshift{0.850455in}{2.257545in}%
\pgfsys@useobject{currentmarker}{}%
\end{pgfscope}%
\begin{pgfscope}%
\pgfsys@transformshift{0.867120in}{2.250897in}%
\pgfsys@useobject{currentmarker}{}%
\end{pgfscope}%
\begin{pgfscope}%
\pgfsys@transformshift{0.888247in}{2.280477in}%
\pgfsys@useobject{currentmarker}{}%
\end{pgfscope}%
\begin{pgfscope}%
\pgfsys@transformshift{0.906555in}{2.338003in}%
\pgfsys@useobject{currentmarker}{}%
\end{pgfscope}%
\begin{pgfscope}%
\pgfsys@transformshift{0.925100in}{2.460351in}%
\pgfsys@useobject{currentmarker}{}%
\end{pgfscope}%
\begin{pgfscope}%
\pgfsys@transformshift{0.945990in}{2.622545in}%
\pgfsys@useobject{currentmarker}{}%
\end{pgfscope}%
\begin{pgfscope}%
\pgfsys@transformshift{0.964064in}{2.632711in}%
\pgfsys@useobject{currentmarker}{}%
\end{pgfscope}%
\begin{pgfscope}%
\pgfsys@transformshift{0.981669in}{2.504523in}%
\pgfsys@useobject{currentmarker}{}%
\end{pgfscope}%
\begin{pgfscope}%
\pgfsys@transformshift{1.002559in}{2.343505in}%
\pgfsys@useobject{currentmarker}{}%
\end{pgfscope}%
\begin{pgfscope}%
\pgfsys@transformshift{1.022278in}{2.273453in}%
\pgfsys@useobject{currentmarker}{}%
\end{pgfscope}%
\begin{pgfscope}%
\pgfsys@transformshift{1.041289in}{2.249490in}%
\pgfsys@useobject{currentmarker}{}%
\end{pgfscope}%
\begin{pgfscope}%
\pgfsys@transformshift{1.059365in}{2.255657in}%
\pgfsys@useobject{currentmarker}{}%
\end{pgfscope}%
\begin{pgfscope}%
\pgfsys@transformshift{1.079550in}{2.295444in}%
\pgfsys@useobject{currentmarker}{}%
\end{pgfscope}%
\begin{pgfscope}%
\pgfsys@transformshift{1.097625in}{2.367726in}%
\pgfsys@useobject{currentmarker}{}%
\end{pgfscope}%
\begin{pgfscope}%
\pgfsys@transformshift{1.118282in}{2.534050in}%
\pgfsys@useobject{currentmarker}{}%
\end{pgfscope}%
\begin{pgfscope}%
\pgfsys@transformshift{1.136590in}{2.632032in}%
\pgfsys@useobject{currentmarker}{}%
\end{pgfscope}%
\begin{pgfscope}%
\pgfsys@transformshift{1.154900in}{2.597621in}%
\pgfsys@useobject{currentmarker}{}%
\end{pgfscope}%
\begin{pgfscope}%
\pgfsys@transformshift{1.176025in}{2.426039in}%
\pgfsys@useobject{currentmarker}{}%
\end{pgfscope}%
\begin{pgfscope}%
\pgfsys@transformshift{1.196212in}{2.313364in}%
\pgfsys@useobject{currentmarker}{}%
\end{pgfscope}%
\begin{pgfscope}%
\pgfsys@transformshift{1.214754in}{2.262732in}%
\pgfsys@useobject{currentmarker}{}%
\end{pgfscope}%
\begin{pgfscope}%
\pgfsys@transformshift{1.232594in}{2.245981in}%
\pgfsys@useobject{currentmarker}{}%
\end{pgfscope}%
\begin{pgfscope}%
\pgfsys@transformshift{1.251138in}{2.256685in}%
\pgfsys@useobject{currentmarker}{}%
\end{pgfscope}%
\begin{pgfscope}%
\pgfsys@transformshift{1.271794in}{2.296234in}%
\pgfsys@useobject{currentmarker}{}%
\end{pgfscope}%
\begin{pgfscope}%
\pgfsys@transformshift{1.289868in}{2.370904in}%
\pgfsys@useobject{currentmarker}{}%
\end{pgfscope}%
\begin{pgfscope}%
\pgfsys@transformshift{1.307707in}{2.517668in}%
\pgfsys@useobject{currentmarker}{}%
\end{pgfscope}%
\begin{pgfscope}%
\pgfsys@transformshift{1.329772in}{2.623249in}%
\pgfsys@useobject{currentmarker}{}%
\end{pgfscope}%
\begin{pgfscope}%
\pgfsys@transformshift{1.347847in}{2.588919in}%
\pgfsys@useobject{currentmarker}{}%
\end{pgfscope}%
\begin{pgfscope}%
\pgfsys@transformshift{1.367798in}{2.415612in}%
\pgfsys@useobject{currentmarker}{}%
\end{pgfscope}%
\begin{pgfscope}%
\pgfsys@transformshift{1.384934in}{2.324166in}%
\pgfsys@useobject{currentmarker}{}%
\end{pgfscope}%
\begin{pgfscope}%
\pgfsys@transformshift{1.402773in}{2.271637in}%
\pgfsys@useobject{currentmarker}{}%
\end{pgfscope}%
\begin{pgfscope}%
\pgfsys@transformshift{1.424133in}{2.247798in}%
\pgfsys@useobject{currentmarker}{}%
\end{pgfscope}%
\begin{pgfscope}%
\pgfsys@transformshift{1.445494in}{2.253730in}%
\pgfsys@useobject{currentmarker}{}%
\end{pgfscope}%
\begin{pgfscope}%
\pgfsys@transformshift{1.462628in}{2.280995in}%
\pgfsys@useobject{currentmarker}{}%
\end{pgfscope}%
\begin{pgfscope}%
\pgfsys@transformshift{1.483521in}{2.331650in}%
\pgfsys@useobject{currentmarker}{}%
\end{pgfscope}%
\begin{pgfscope}%
\pgfsys@transformshift{1.502063in}{2.440189in}%
\pgfsys@useobject{currentmarker}{}%
\end{pgfscope}%
\begin{pgfscope}%
\pgfsys@transformshift{1.522719in}{2.549216in}%
\pgfsys@useobject{currentmarker}{}%
\end{pgfscope}%
\begin{pgfscope}%
\pgfsys@transformshift{1.541264in}{2.621866in}%
\pgfsys@useobject{currentmarker}{}%
\end{pgfscope}%
\begin{pgfscope}%
\pgfsys@transformshift{1.559337in}{2.565072in}%
\pgfsys@useobject{currentmarker}{}%
\end{pgfscope}%
\begin{pgfscope}%
\pgfsys@transformshift{1.577177in}{2.420229in}%
\pgfsys@useobject{currentmarker}{}%
\end{pgfscope}%
\begin{pgfscope}%
\pgfsys@transformshift{1.598067in}{2.322092in}%
\pgfsys@useobject{currentmarker}{}%
\end{pgfscope}%
\begin{pgfscope}%
\pgfsys@transformshift{1.615908in}{2.277471in}%
\pgfsys@useobject{currentmarker}{}%
\end{pgfscope}%
\begin{pgfscope}%
\pgfsys@transformshift{1.636328in}{2.248833in}%
\pgfsys@useobject{currentmarker}{}%
\end{pgfscope}%
\begin{pgfscope}%
\pgfsys@transformshift{1.653698in}{2.245131in}%
\pgfsys@useobject{currentmarker}{}%
\end{pgfscope}%
\begin{pgfscope}%
\pgfsys@transformshift{1.675529in}{2.262937in}%
\pgfsys@useobject{currentmarker}{}%
\end{pgfscope}%
\begin{pgfscope}%
\pgfsys@transformshift{1.696654in}{2.300142in}%
\pgfsys@useobject{currentmarker}{}%
\end{pgfscope}%
\begin{pgfscope}%
\pgfsys@transformshift{1.711441in}{2.340280in}%
\pgfsys@useobject{currentmarker}{}%
\end{pgfscope}%
\begin{pgfscope}%
\pgfsys@transformshift{1.732098in}{2.475993in}%
\pgfsys@useobject{currentmarker}{}%
\end{pgfscope}%
\begin{pgfscope}%
\pgfsys@transformshift{1.752990in}{2.603754in}%
\pgfsys@useobject{currentmarker}{}%
\end{pgfscope}%
\begin{pgfscope}%
\pgfsys@transformshift{1.770829in}{2.611211in}%
\pgfsys@useobject{currentmarker}{}%
\end{pgfscope}%
\begin{pgfscope}%
\pgfsys@transformshift{1.791954in}{2.508597in}%
\pgfsys@useobject{currentmarker}{}%
\end{pgfscope}%
\begin{pgfscope}%
\pgfsys@transformshift{1.810262in}{2.451390in}%
\pgfsys@useobject{currentmarker}{}%
\end{pgfscope}%
\begin{pgfscope}%
\pgfsys@transformshift{1.828104in}{2.334021in}%
\pgfsys@useobject{currentmarker}{}%
\end{pgfscope}%
\begin{pgfscope}%
\pgfsys@transformshift{1.845472in}{2.274256in}%
\pgfsys@useobject{currentmarker}{}%
\end{pgfscope}%
\begin{pgfscope}%
\pgfsys@transformshift{1.867771in}{2.248262in}%
\pgfsys@useobject{currentmarker}{}%
\end{pgfscope}%
\begin{pgfscope}%
\pgfsys@transformshift{1.885376in}{2.245576in}%
\pgfsys@useobject{currentmarker}{}%
\end{pgfscope}%
\begin{pgfscope}%
\pgfsys@transformshift{1.906268in}{2.264301in}%
\pgfsys@useobject{currentmarker}{}%
\end{pgfscope}%
\begin{pgfscope}%
\pgfsys@transformshift{1.923873in}{2.297809in}%
\pgfsys@useobject{currentmarker}{}%
\end{pgfscope}%
\begin{pgfscope}%
\pgfsys@transformshift{1.944529in}{2.356109in}%
\pgfsys@useobject{currentmarker}{}%
\end{pgfscope}%
\begin{pgfscope}%
\pgfsys@transformshift{1.962368in}{2.487837in}%
\pgfsys@useobject{currentmarker}{}%
\end{pgfscope}%
\begin{pgfscope}%
\pgfsys@transformshift{1.984197in}{2.610197in}%
\pgfsys@useobject{currentmarker}{}%
\end{pgfscope}%
\begin{pgfscope}%
\pgfsys@transformshift{2.001333in}{2.596306in}%
\pgfsys@useobject{currentmarker}{}%
\end{pgfscope}%
\begin{pgfscope}%
\pgfsys@transformshift{2.019406in}{2.507846in}%
\pgfsys@useobject{currentmarker}{}%
\end{pgfscope}%
\begin{pgfscope}%
\pgfsys@transformshift{2.042411in}{2.364808in}%
\pgfsys@useobject{currentmarker}{}%
\end{pgfscope}%
\begin{pgfscope}%
\pgfsys@transformshift{2.056024in}{2.286726in}%
\pgfsys@useobject{currentmarker}{}%
\end{pgfscope}%
\begin{pgfscope}%
\pgfsys@transformshift{2.077620in}{2.267959in}%
\pgfsys@useobject{currentmarker}{}%
\end{pgfscope}%
\begin{pgfscope}%
\pgfsys@transformshift{2.097571in}{2.246660in}%
\pgfsys@useobject{currentmarker}{}%
\end{pgfscope}%
\begin{pgfscope}%
\pgfsys@transformshift{2.097807in}{2.244965in}%
\pgfsys@useobject{currentmarker}{}%
\end{pgfscope}%
\begin{pgfscope}%
\pgfsys@transformshift{2.118698in}{2.247921in}%
\pgfsys@useobject{currentmarker}{}%
\end{pgfscope}%
\begin{pgfscope}%
\pgfsys@transformshift{2.138415in}{2.264775in}%
\pgfsys@useobject{currentmarker}{}%
\end{pgfscope}%
\begin{pgfscope}%
\pgfsys@transformshift{2.154845in}{2.296496in}%
\pgfsys@useobject{currentmarker}{}%
\end{pgfscope}%
\begin{pgfscope}%
\pgfsys@transformshift{2.175501in}{2.376593in}%
\pgfsys@useobject{currentmarker}{}%
\end{pgfscope}%
\begin{pgfscope}%
\pgfsys@transformshift{2.193811in}{2.491680in}%
\pgfsys@useobject{currentmarker}{}%
\end{pgfscope}%
\begin{pgfscope}%
\pgfsys@transformshift{2.211651in}{2.566590in}%
\pgfsys@useobject{currentmarker}{}%
\end{pgfscope}%
\begin{pgfscope}%
\pgfsys@transformshift{2.230193in}{2.610807in}%
\pgfsys@useobject{currentmarker}{}%
\end{pgfscope}%
\begin{pgfscope}%
\pgfsys@transformshift{2.251086in}{2.536241in}%
\pgfsys@useobject{currentmarker}{}%
\end{pgfscope}%
\begin{pgfscope}%
\pgfsys@transformshift{2.271976in}{2.395365in}%
\pgfsys@useobject{currentmarker}{}%
\end{pgfscope}%
\begin{pgfscope}%
\pgfsys@transformshift{2.290755in}{2.304770in}%
\pgfsys@useobject{currentmarker}{}%
\end{pgfscope}%
\begin{pgfscope}%
\pgfsys@transformshift{2.309532in}{2.265731in}%
\pgfsys@useobject{currentmarker}{}%
\end{pgfscope}%
\begin{pgfscope}%
\pgfsys@transformshift{2.326902in}{2.246586in}%
\pgfsys@useobject{currentmarker}{}%
\end{pgfscope}%
\begin{pgfscope}%
\pgfsys@transformshift{2.348027in}{2.246082in}%
\pgfsys@useobject{currentmarker}{}%
\end{pgfscope}%
\begin{pgfscope}%
\pgfsys@transformshift{2.369389in}{2.260554in}%
\pgfsys@useobject{currentmarker}{}%
\end{pgfscope}%
\begin{pgfscope}%
\pgfsys@transformshift{2.386993in}{2.276259in}%
\pgfsys@useobject{currentmarker}{}%
\end{pgfscope}%
\begin{pgfscope}%
\pgfsys@transformshift{2.404598in}{2.311474in}%
\pgfsys@useobject{currentmarker}{}%
\end{pgfscope}%
\begin{pgfscope}%
\pgfsys@transformshift{2.422906in}{2.390475in}%
\pgfsys@useobject{currentmarker}{}%
\end{pgfscope}%
\begin{pgfscope}%
\pgfsys@transformshift{2.442625in}{2.508175in}%
\pgfsys@useobject{currentmarker}{}%
\end{pgfscope}%
\begin{pgfscope}%
\pgfsys@transformshift{2.461402in}{2.604890in}%
\pgfsys@useobject{currentmarker}{}%
\end{pgfscope}%
\begin{pgfscope}%
\pgfsys@transformshift{2.482529in}{2.574852in}%
\pgfsys@useobject{currentmarker}{}%
\end{pgfscope}%
\begin{pgfscope}%
\pgfsys@transformshift{2.500837in}{2.447832in}%
\pgfsys@useobject{currentmarker}{}%
\end{pgfscope}%
\begin{pgfscope}%
\pgfsys@transformshift{2.518207in}{2.341188in}%
\pgfsys@useobject{currentmarker}{}%
\end{pgfscope}%
\begin{pgfscope}%
\pgfsys@transformshift{2.540506in}{2.278785in}%
\pgfsys@useobject{currentmarker}{}%
\end{pgfscope}%
\begin{pgfscope}%
\pgfsys@transformshift{2.559754in}{2.252645in}%
\pgfsys@useobject{currentmarker}{}%
\end{pgfscope}%
\begin{pgfscope}%
\pgfsys@transformshift{2.577124in}{2.247037in}%
\pgfsys@useobject{currentmarker}{}%
\end{pgfscope}%
\begin{pgfscope}%
\pgfsys@transformshift{2.598483in}{2.247220in}%
\pgfsys@useobject{currentmarker}{}%
\end{pgfscope}%
\begin{pgfscope}%
\pgfsys@transformshift{2.616325in}{2.260577in}%
\pgfsys@useobject{currentmarker}{}%
\end{pgfscope}%
\begin{pgfscope}%
\pgfsys@transformshift{2.634867in}{2.289167in}%
\pgfsys@useobject{currentmarker}{}%
\end{pgfscope}%
\begin{pgfscope}%
\pgfsys@transformshift{2.654349in}{2.350766in}%
\pgfsys@useobject{currentmarker}{}%
\end{pgfscope}%
\begin{pgfscope}%
\pgfsys@transformshift{2.675005in}{2.352859in}%
\pgfsys@useobject{currentmarker}{}%
\end{pgfscope}%
\begin{pgfscope}%
\pgfsys@transformshift{2.694724in}{2.480488in}%
\pgfsys@useobject{currentmarker}{}%
\end{pgfscope}%
\begin{pgfscope}%
\pgfsys@transformshift{2.714206in}{2.594519in}%
\pgfsys@useobject{currentmarker}{}%
\end{pgfscope}%
\begin{pgfscope}%
\pgfsys@transformshift{2.732985in}{2.610498in}%
\pgfsys@useobject{currentmarker}{}%
\end{pgfscope}%
\begin{pgfscope}%
\pgfsys@transformshift{2.750824in}{2.523946in}%
\pgfsys@useobject{currentmarker}{}%
\end{pgfscope}%
\begin{pgfscope}%
\pgfsys@transformshift{2.768898in}{2.396502in}%
\pgfsys@useobject{currentmarker}{}%
\end{pgfscope}%
\begin{pgfscope}%
\pgfsys@transformshift{2.789788in}{2.302669in}%
\pgfsys@useobject{currentmarker}{}%
\end{pgfscope}%
\begin{pgfscope}%
\pgfsys@transformshift{2.807627in}{2.265508in}%
\pgfsys@useobject{currentmarker}{}%
\end{pgfscope}%
\begin{pgfscope}%
\pgfsys@transformshift{2.826406in}{2.247498in}%
\pgfsys@useobject{currentmarker}{}%
\end{pgfscope}%
\begin{pgfscope}%
\pgfsys@transformshift{2.847297in}{2.247489in}%
\pgfsys@useobject{currentmarker}{}%
\end{pgfscope}%
\begin{pgfscope}%
\pgfsys@transformshift{2.866781in}{2.265295in}%
\pgfsys@useobject{currentmarker}{}%
\end{pgfscope}%
\begin{pgfscope}%
\pgfsys@transformshift{2.884620in}{2.297404in}%
\pgfsys@useobject{currentmarker}{}%
\end{pgfscope}%
\begin{pgfscope}%
\pgfsys@transformshift{2.908562in}{2.357063in}%
\pgfsys@useobject{currentmarker}{}%
\end{pgfscope}%
\begin{pgfscope}%
\pgfsys@transformshift{2.924289in}{2.453886in}%
\pgfsys@useobject{currentmarker}{}%
\end{pgfscope}%
\begin{pgfscope}%
\pgfsys@transformshift{2.944475in}{2.565666in}%
\pgfsys@useobject{currentmarker}{}%
\end{pgfscope}%
\begin{pgfscope}%
\pgfsys@transformshift{2.962550in}{2.616986in}%
\pgfsys@useobject{currentmarker}{}%
\end{pgfscope}%
\begin{pgfscope}%
\pgfsys@transformshift{2.981562in}{2.584874in}%
\pgfsys@useobject{currentmarker}{}%
\end{pgfscope}%
\begin{pgfscope}%
\pgfsys@transformshift{3.002220in}{2.449720in}%
\pgfsys@useobject{currentmarker}{}%
\end{pgfscope}%
\begin{pgfscope}%
\pgfsys@transformshift{3.019588in}{2.366694in}%
\pgfsys@useobject{currentmarker}{}%
\end{pgfscope}%
\begin{pgfscope}%
\pgfsys@transformshift{3.037664in}{2.301289in}%
\pgfsys@useobject{currentmarker}{}%
\end{pgfscope}%
\begin{pgfscope}%
\pgfsys@transformshift{3.037898in}{2.276124in}%
\pgfsys@useobject{currentmarker}{}%
\end{pgfscope}%
\begin{pgfscope}%
\pgfsys@transformshift{3.062780in}{2.268645in}%
\pgfsys@useobject{currentmarker}{}%
\end{pgfscope}%
\begin{pgfscope}%
\pgfsys@transformshift{3.077333in}{2.253860in}%
\pgfsys@useobject{currentmarker}{}%
\end{pgfscope}%
\begin{pgfscope}%
\pgfsys@transformshift{3.095407in}{2.528753in}%
\pgfsys@useobject{currentmarker}{}%
\end{pgfscope}%
\begin{pgfscope}%
\pgfsys@transformshift{3.116532in}{2.363589in}%
\pgfsys@useobject{currentmarker}{}%
\end{pgfscope}%
\begin{pgfscope}%
\pgfsys@transformshift{3.134606in}{2.294845in}%
\pgfsys@useobject{currentmarker}{}%
\end{pgfscope}%
\begin{pgfscope}%
\pgfsys@transformshift{3.152915in}{2.259086in}%
\pgfsys@useobject{currentmarker}{}%
\end{pgfscope}%
\begin{pgfscope}%
\pgfsys@transformshift{3.171458in}{2.246071in}%
\pgfsys@useobject{currentmarker}{}%
\end{pgfscope}%
\begin{pgfscope}%
\pgfsys@transformshift{3.192114in}{2.253514in}%
\pgfsys@useobject{currentmarker}{}%
\end{pgfscope}%
\begin{pgfscope}%
\pgfsys@transformshift{3.210424in}{2.275417in}%
\pgfsys@useobject{currentmarker}{}%
\end{pgfscope}%
\begin{pgfscope}%
\pgfsys@transformshift{3.232489in}{2.330761in}%
\pgfsys@useobject{currentmarker}{}%
\end{pgfscope}%
\begin{pgfscope}%
\pgfsys@transformshift{3.252205in}{2.438759in}%
\pgfsys@useobject{currentmarker}{}%
\end{pgfscope}%
\begin{pgfscope}%
\pgfsys@transformshift{3.271453in}{2.554741in}%
\pgfsys@useobject{currentmarker}{}%
\end{pgfscope}%
\begin{pgfscope}%
\pgfsys@transformshift{3.289058in}{2.622229in}%
\pgfsys@useobject{currentmarker}{}%
\end{pgfscope}%
\begin{pgfscope}%
\pgfsys@transformshift{3.306897in}{2.596673in}%
\pgfsys@useobject{currentmarker}{}%
\end{pgfscope}%
\begin{pgfscope}%
\pgfsys@transformshift{3.328024in}{2.499581in}%
\pgfsys@useobject{currentmarker}{}%
\end{pgfscope}%
\begin{pgfscope}%
\pgfsys@transformshift{3.348680in}{2.357100in}%
\pgfsys@useobject{currentmarker}{}%
\end{pgfscope}%
\begin{pgfscope}%
\pgfsys@transformshift{3.364171in}{2.302158in}%
\pgfsys@useobject{currentmarker}{}%
\end{pgfscope}%
\begin{pgfscope}%
\pgfsys@transformshift{3.385062in}{2.263821in}%
\pgfsys@useobject{currentmarker}{}%
\end{pgfscope}%
\begin{pgfscope}%
\pgfsys@transformshift{3.405483in}{2.248122in}%
\pgfsys@useobject{currentmarker}{}%
\end{pgfscope}%
\begin{pgfscope}%
\pgfsys@transformshift{3.424731in}{2.249051in}%
\pgfsys@useobject{currentmarker}{}%
\end{pgfscope}%
\begin{pgfscope}%
\pgfsys@transformshift{3.444450in}{2.269733in}%
\pgfsys@useobject{currentmarker}{}%
\end{pgfscope}%
\begin{pgfscope}%
\pgfsys@transformshift{3.460175in}{2.294293in}%
\pgfsys@useobject{currentmarker}{}%
\end{pgfscope}%
\begin{pgfscope}%
\pgfsys@transformshift{3.480831in}{2.369214in}%
\pgfsys@useobject{currentmarker}{}%
\end{pgfscope}%
\begin{pgfscope}%
\pgfsys@transformshift{3.500784in}{2.487922in}%
\pgfsys@useobject{currentmarker}{}%
\end{pgfscope}%
\begin{pgfscope}%
\pgfsys@transformshift{3.520971in}{2.615838in}%
\pgfsys@useobject{currentmarker}{}%
\end{pgfscope}%
\begin{pgfscope}%
\pgfsys@transformshift{3.540688in}{2.629970in}%
\pgfsys@useobject{currentmarker}{}%
\end{pgfscope}%
\begin{pgfscope}%
\pgfsys@transformshift{3.558527in}{2.573862in}%
\pgfsys@useobject{currentmarker}{}%
\end{pgfscope}%
\begin{pgfscope}%
\pgfsys@transformshift{3.577540in}{2.454643in}%
\pgfsys@useobject{currentmarker}{}%
\end{pgfscope}%
\begin{pgfscope}%
\pgfsys@transformshift{3.598431in}{2.336817in}%
\pgfsys@useobject{currentmarker}{}%
\end{pgfscope}%
\begin{pgfscope}%
\pgfsys@transformshift{3.616270in}{2.299758in}%
\pgfsys@useobject{currentmarker}{}%
\end{pgfscope}%
\begin{pgfscope}%
\pgfsys@transformshift{3.634109in}{2.275243in}%
\pgfsys@useobject{currentmarker}{}%
\end{pgfscope}%
\begin{pgfscope}%
\pgfsys@transformshift{3.653123in}{2.253057in}%
\pgfsys@useobject{currentmarker}{}%
\end{pgfscope}%
\begin{pgfscope}%
\pgfsys@transformshift{3.673779in}{2.249722in}%
\pgfsys@useobject{currentmarker}{}%
\end{pgfscope}%
\begin{pgfscope}%
\pgfsys@transformshift{3.693966in}{2.265151in}%
\pgfsys@useobject{currentmarker}{}%
\end{pgfscope}%
\begin{pgfscope}%
\pgfsys@transformshift{3.713214in}{2.295635in}%
\pgfsys@useobject{currentmarker}{}%
\end{pgfscope}%
\begin{pgfscope}%
\pgfsys@transformshift{3.733636in}{2.339422in}%
\pgfsys@useobject{currentmarker}{}%
\end{pgfscope}%
\begin{pgfscope}%
\pgfsys@transformshift{3.750772in}{2.423726in}%
\pgfsys@useobject{currentmarker}{}%
\end{pgfscope}%
\begin{pgfscope}%
\pgfsys@transformshift{3.769785in}{2.545111in}%
\pgfsys@useobject{currentmarker}{}%
\end{pgfscope}%
\begin{pgfscope}%
\pgfsys@transformshift{3.789267in}{2.628799in}%
\pgfsys@useobject{currentmarker}{}%
\end{pgfscope}%
\begin{pgfscope}%
\pgfsys@transformshift{3.808280in}{2.635289in}%
\pgfsys@useobject{currentmarker}{}%
\end{pgfscope}%
\begin{pgfscope}%
\pgfsys@transformshift{3.826823in}{2.567357in}%
\pgfsys@useobject{currentmarker}{}%
\end{pgfscope}%
\begin{pgfscope}%
\pgfsys@transformshift{3.846305in}{2.471787in}%
\pgfsys@useobject{currentmarker}{}%
\end{pgfscope}%
\begin{pgfscope}%
\pgfsys@transformshift{3.866963in}{2.360495in}%
\pgfsys@useobject{currentmarker}{}%
\end{pgfscope}%
\begin{pgfscope}%
\pgfsys@transformshift{3.885505in}{2.300855in}%
\pgfsys@useobject{currentmarker}{}%
\end{pgfscope}%
\begin{pgfscope}%
\pgfsys@transformshift{3.904753in}{2.268469in}%
\pgfsys@useobject{currentmarker}{}%
\end{pgfscope}%
\begin{pgfscope}%
\pgfsys@transformshift{3.923063in}{2.257257in}%
\pgfsys@useobject{currentmarker}{}%
\end{pgfscope}%
\begin{pgfscope}%
\pgfsys@transformshift{3.942779in}{2.248873in}%
\pgfsys@useobject{currentmarker}{}%
\end{pgfscope}%
\begin{pgfscope}%
\pgfsys@transformshift{3.960384in}{2.259107in}%
\pgfsys@useobject{currentmarker}{}%
\end{pgfscope}%
\begin{pgfscope}%
\pgfsys@transformshift{3.980572in}{2.272282in}%
\pgfsys@useobject{currentmarker}{}%
\end{pgfscope}%
\begin{pgfscope}%
\pgfsys@transformshift{3.999348in}{2.310615in}%
\pgfsys@useobject{currentmarker}{}%
\end{pgfscope}%
\begin{pgfscope}%
\pgfsys@transformshift{4.019301in}{2.375462in}%
\pgfsys@useobject{currentmarker}{}%
\end{pgfscope}%
\begin{pgfscope}%
\pgfsys@transformshift{4.037375in}{2.491037in}%
\pgfsys@useobject{currentmarker}{}%
\end{pgfscope}%
\begin{pgfscope}%
\pgfsys@transformshift{4.056623in}{2.606623in}%
\pgfsys@useobject{currentmarker}{}%
\end{pgfscope}%
\begin{pgfscope}%
\pgfsys@transformshift{4.075402in}{2.650638in}%
\pgfsys@useobject{currentmarker}{}%
\end{pgfscope}%
\begin{pgfscope}%
\pgfsys@transformshift{4.097701in}{2.640046in}%
\pgfsys@useobject{currentmarker}{}%
\end{pgfscope}%
\begin{pgfscope}%
\pgfsys@transformshift{4.116245in}{2.557977in}%
\pgfsys@useobject{currentmarker}{}%
\end{pgfscope}%
\begin{pgfscope}%
\pgfsys@transformshift{4.136430in}{2.422534in}%
\pgfsys@useobject{currentmarker}{}%
\end{pgfscope}%
\begin{pgfscope}%
\pgfsys@transformshift{4.154975in}{2.349175in}%
\pgfsys@useobject{currentmarker}{}%
\end{pgfscope}%
\begin{pgfscope}%
\pgfsys@transformshift{4.174457in}{2.294729in}%
\pgfsys@useobject{currentmarker}{}%
\end{pgfscope}%
\begin{pgfscope}%
\pgfsys@transformshift{4.192531in}{2.271263in}%
\pgfsys@useobject{currentmarker}{}%
\end{pgfscope}%
\begin{pgfscope}%
\pgfsys@transformshift{4.212249in}{2.251648in}%
\pgfsys@useobject{currentmarker}{}%
\end{pgfscope}%
\begin{pgfscope}%
\pgfsys@transformshift{4.229149in}{2.515648in}%
\pgfsys@useobject{currentmarker}{}%
\end{pgfscope}%
\begin{pgfscope}%
\pgfsys@transformshift{4.250041in}{2.641239in}%
\pgfsys@useobject{currentmarker}{}%
\end{pgfscope}%
\begin{pgfscope}%
\pgfsys@transformshift{4.268584in}{2.649742in}%
\pgfsys@useobject{currentmarker}{}%
\end{pgfscope}%
\begin{pgfscope}%
\pgfsys@transformshift{4.287597in}{2.560446in}%
\pgfsys@useobject{currentmarker}{}%
\end{pgfscope}%
\begin{pgfscope}%
\pgfsys@transformshift{4.308487in}{2.393777in}%
\pgfsys@useobject{currentmarker}{}%
\end{pgfscope}%
\begin{pgfscope}%
\pgfsys@transformshift{4.327266in}{2.317759in}%
\pgfsys@useobject{currentmarker}{}%
\end{pgfscope}%
\begin{pgfscope}%
\pgfsys@transformshift{4.346983in}{2.268372in}%
\pgfsys@useobject{currentmarker}{}%
\end{pgfscope}%
\begin{pgfscope}%
\pgfsys@transformshift{4.364353in}{2.251733in}%
\pgfsys@useobject{currentmarker}{}%
\end{pgfscope}%
\begin{pgfscope}%
\pgfsys@transformshift{4.389000in}{2.264027in}%
\pgfsys@useobject{currentmarker}{}%
\end{pgfscope}%
\begin{pgfscope}%
\pgfsys@transformshift{4.402380in}{2.292349in}%
\pgfsys@useobject{currentmarker}{}%
\end{pgfscope}%
\begin{pgfscope}%
\pgfsys@transformshift{4.421158in}{2.341920in}%
\pgfsys@useobject{currentmarker}{}%
\end{pgfscope}%
\begin{pgfscope}%
\pgfsys@transformshift{4.443926in}{2.477747in}%
\pgfsys@useobject{currentmarker}{}%
\end{pgfscope}%
\begin{pgfscope}%
\pgfsys@transformshift{4.462236in}{2.619894in}%
\pgfsys@useobject{currentmarker}{}%
\end{pgfscope}%
\begin{pgfscope}%
\pgfsys@transformshift{4.481250in}{2.667047in}%
\pgfsys@useobject{currentmarker}{}%
\end{pgfscope}%
\begin{pgfscope}%
\pgfsys@transformshift{4.481013in}{2.667708in}%
\pgfsys@useobject{currentmarker}{}%
\end{pgfscope}%
\begin{pgfscope}%
\pgfsys@transformshift{4.473268in}{2.653044in}%
\pgfsys@useobject{currentmarker}{}%
\end{pgfscope}%
\begin{pgfscope}%
\pgfsys@transformshift{4.454489in}{2.510636in}%
\pgfsys@useobject{currentmarker}{}%
\end{pgfscope}%
\begin{pgfscope}%
\pgfsys@transformshift{4.433833in}{2.348663in}%
\pgfsys@useobject{currentmarker}{}%
\end{pgfscope}%
\begin{pgfscope}%
\pgfsys@transformshift{4.415993in}{2.284274in}%
\pgfsys@useobject{currentmarker}{}%
\end{pgfscope}%
\begin{pgfscope}%
\pgfsys@transformshift{4.397920in}{2.254720in}%
\pgfsys@useobject{currentmarker}{}%
\end{pgfscope}%
\begin{pgfscope}%
\pgfsys@transformshift{4.377264in}{2.262404in}%
\pgfsys@useobject{currentmarker}{}%
\end{pgfscope}%
\begin{pgfscope}%
\pgfsys@transformshift{4.359424in}{2.303268in}%
\pgfsys@useobject{currentmarker}{}%
\end{pgfscope}%
\begin{pgfscope}%
\pgfsys@transformshift{4.342054in}{2.401446in}%
\pgfsys@useobject{currentmarker}{}%
\end{pgfscope}%
\begin{pgfscope}%
\pgfsys@transformshift{4.322101in}{2.573705in}%
\pgfsys@useobject{currentmarker}{}%
\end{pgfscope}%
\begin{pgfscope}%
\pgfsys@transformshift{4.300976in}{2.654700in}%
\pgfsys@useobject{currentmarker}{}%
\end{pgfscope}%
\begin{pgfscope}%
\pgfsys@transformshift{4.282903in}{2.557836in}%
\pgfsys@useobject{currentmarker}{}%
\end{pgfscope}%
\begin{pgfscope}%
\pgfsys@transformshift{4.260604in}{2.361069in}%
\pgfsys@useobject{currentmarker}{}%
\end{pgfscope}%
\begin{pgfscope}%
\pgfsys@transformshift{4.243936in}{2.291975in}%
\pgfsys@useobject{currentmarker}{}%
\end{pgfscope}%
\begin{pgfscope}%
\pgfsys@transformshift{4.224923in}{2.255841in}%
\pgfsys@useobject{currentmarker}{}%
\end{pgfscope}%
\begin{pgfscope}%
\pgfsys@transformshift{4.203095in}{2.253474in}%
\pgfsys@useobject{currentmarker}{}%
\end{pgfscope}%
\begin{pgfscope}%
\pgfsys@transformshift{4.185256in}{2.288743in}%
\pgfsys@useobject{currentmarker}{}%
\end{pgfscope}%
\begin{pgfscope}%
\pgfsys@transformshift{4.164600in}{2.377785in}%
\pgfsys@useobject{currentmarker}{}%
\end{pgfscope}%
\begin{pgfscope}%
\pgfsys@transformshift{4.146290in}{2.542716in}%
\pgfsys@useobject{currentmarker}{}%
\end{pgfscope}%
\begin{pgfscope}%
\pgfsys@transformshift{4.127511in}{2.642952in}%
\pgfsys@useobject{currentmarker}{}%
\end{pgfscope}%
\begin{pgfscope}%
\pgfsys@transformshift{4.105212in}{2.560215in}%
\pgfsys@useobject{currentmarker}{}%
\end{pgfscope}%
\begin{pgfscope}%
\pgfsys@transformshift{4.089955in}{2.413799in}%
\pgfsys@useobject{currentmarker}{}%
\end{pgfscope}%
\begin{pgfscope}%
\pgfsys@transformshift{4.071645in}{2.307347in}%
\pgfsys@useobject{currentmarker}{}%
\end{pgfscope}%
\begin{pgfscope}%
\pgfsys@transformshift{4.053103in}{2.264166in}%
\pgfsys@useobject{currentmarker}{}%
\end{pgfscope}%
\begin{pgfscope}%
\pgfsys@transformshift{4.031272in}{2.247619in}%
\pgfsys@useobject{currentmarker}{}%
\end{pgfscope}%
\begin{pgfscope}%
\pgfsys@transformshift{4.012259in}{2.262362in}%
\pgfsys@useobject{currentmarker}{}%
\end{pgfscope}%
\begin{pgfscope}%
\pgfsys@transformshift{3.993951in}{2.308407in}%
\pgfsys@useobject{currentmarker}{}%
\end{pgfscope}%
\begin{pgfscope}%
\pgfsys@transformshift{3.974938in}{2.419741in}%
\pgfsys@useobject{currentmarker}{}%
\end{pgfscope}%
\begin{pgfscope}%
\pgfsys@transformshift{3.953342in}{2.596136in}%
\pgfsys@useobject{currentmarker}{}%
\end{pgfscope}%
\begin{pgfscope}%
\pgfsys@transformshift{3.934094in}{2.632987in}%
\pgfsys@useobject{currentmarker}{}%
\end{pgfscope}%
\begin{pgfscope}%
\pgfsys@transformshift{3.916021in}{2.521504in}%
\pgfsys@useobject{currentmarker}{}%
\end{pgfscope}%
\begin{pgfscope}%
\pgfsys@transformshift{3.897242in}{2.366420in}%
\pgfsys@useobject{currentmarker}{}%
\end{pgfscope}%
\begin{pgfscope}%
\pgfsys@transformshift{3.878934in}{2.291297in}%
\pgfsys@useobject{currentmarker}{}%
\end{pgfscope}%
\begin{pgfscope}%
\pgfsys@transformshift{3.860155in}{2.258296in}%
\pgfsys@useobject{currentmarker}{}%
\end{pgfscope}%
\begin{pgfscope}%
\pgfsys@transformshift{3.840673in}{2.245899in}%
\pgfsys@useobject{currentmarker}{}%
\end{pgfscope}%
\begin{pgfscope}%
\pgfsys@transformshift{3.822597in}{2.262129in}%
\pgfsys@useobject{currentmarker}{}%
\end{pgfscope}%
\begin{pgfscope}%
\pgfsys@transformshift{3.800769in}{2.313683in}%
\pgfsys@useobject{currentmarker}{}%
\end{pgfscope}%
\begin{pgfscope}%
\pgfsys@transformshift{3.785276in}{2.380676in}%
\pgfsys@useobject{currentmarker}{}%
\end{pgfscope}%
\begin{pgfscope}%
\pgfsys@transformshift{3.765325in}{2.530284in}%
\pgfsys@useobject{currentmarker}{}%
\end{pgfscope}%
\begin{pgfscope}%
\pgfsys@transformshift{3.740443in}{2.628593in}%
\pgfsys@useobject{currentmarker}{}%
\end{pgfscope}%
\begin{pgfscope}%
\pgfsys@transformshift{3.723776in}{2.605867in}%
\pgfsys@useobject{currentmarker}{}%
\end{pgfscope}%
\begin{pgfscope}%
\pgfsys@transformshift{3.701243in}{2.431503in}%
\pgfsys@useobject{currentmarker}{}%
\end{pgfscope}%
\begin{pgfscope}%
\pgfsys@transformshift{3.686455in}{2.346592in}%
\pgfsys@useobject{currentmarker}{}%
\end{pgfscope}%
\begin{pgfscope}%
\pgfsys@transformshift{3.667676in}{2.286019in}%
\pgfsys@useobject{currentmarker}{}%
\end{pgfscope}%
\begin{pgfscope}%
\pgfsys@transformshift{3.649368in}{2.255342in}%
\pgfsys@useobject{currentmarker}{}%
\end{pgfscope}%
\begin{pgfscope}%
\pgfsys@transformshift{3.627069in}{2.618537in}%
\pgfsys@useobject{currentmarker}{}%
\end{pgfscope}%
\begin{pgfscope}%
\pgfsys@transformshift{3.608994in}{2.506203in}%
\pgfsys@useobject{currentmarker}{}%
\end{pgfscope}%
\begin{pgfscope}%
\pgfsys@transformshift{3.589982in}{2.363030in}%
\pgfsys@useobject{currentmarker}{}%
\end{pgfscope}%
\begin{pgfscope}%
\pgfsys@transformshift{3.568621in}{2.285979in}%
\pgfsys@useobject{currentmarker}{}%
\end{pgfscope}%
\begin{pgfscope}%
\pgfsys@transformshift{3.553128in}{2.259367in}%
\pgfsys@useobject{currentmarker}{}%
\end{pgfscope}%
\begin{pgfscope}%
\pgfsys@transformshift{3.531300in}{2.244862in}%
\pgfsys@useobject{currentmarker}{}%
\end{pgfscope}%
\begin{pgfscope}%
\pgfsys@transformshift{3.515338in}{2.251876in}%
\pgfsys@useobject{currentmarker}{}%
\end{pgfscope}%
\begin{pgfscope}%
\pgfsys@transformshift{3.493742in}{2.279258in}%
\pgfsys@useobject{currentmarker}{}%
\end{pgfscope}%
\begin{pgfscope}%
\pgfsys@transformshift{3.474025in}{2.340415in}%
\pgfsys@useobject{currentmarker}{}%
\end{pgfscope}%
\begin{pgfscope}%
\pgfsys@transformshift{3.452430in}{2.487118in}%
\pgfsys@useobject{currentmarker}{}%
\end{pgfscope}%
\begin{pgfscope}%
\pgfsys@transformshift{3.436938in}{2.600870in}%
\pgfsys@useobject{currentmarker}{}%
\end{pgfscope}%
\begin{pgfscope}%
\pgfsys@transformshift{3.417925in}{2.618877in}%
\pgfsys@useobject{currentmarker}{}%
\end{pgfscope}%
\begin{pgfscope}%
\pgfsys@transformshift{3.397738in}{2.527262in}%
\pgfsys@useobject{currentmarker}{}%
\end{pgfscope}%
\begin{pgfscope}%
\pgfsys@transformshift{3.379664in}{2.370485in}%
\pgfsys@useobject{currentmarker}{}%
\end{pgfscope}%
\begin{pgfscope}%
\pgfsys@transformshift{3.357129in}{2.289778in}%
\pgfsys@useobject{currentmarker}{}%
\end{pgfscope}%
\begin{pgfscope}%
\pgfsys@transformshift{3.340933in}{2.261999in}%
\pgfsys@useobject{currentmarker}{}%
\end{pgfscope}%
\begin{pgfscope}%
\pgfsys@transformshift{3.321921in}{2.244592in}%
\pgfsys@useobject{currentmarker}{}%
\end{pgfscope}%
\begin{pgfscope}%
\pgfsys@transformshift{3.301263in}{2.254173in}%
\pgfsys@useobject{currentmarker}{}%
\end{pgfscope}%
\begin{pgfscope}%
\pgfsys@transformshift{3.282252in}{2.282186in}%
\pgfsys@useobject{currentmarker}{}%
\end{pgfscope}%
\begin{pgfscope}%
\pgfsys@transformshift{3.264411in}{2.346150in}%
\pgfsys@useobject{currentmarker}{}%
\end{pgfscope}%
\begin{pgfscope}%
\pgfsys@transformshift{3.245634in}{2.474230in}%
\pgfsys@useobject{currentmarker}{}%
\end{pgfscope}%
\begin{pgfscope}%
\pgfsys@transformshift{3.223569in}{2.604277in}%
\pgfsys@useobject{currentmarker}{}%
\end{pgfscope}%
\begin{pgfscope}%
\pgfsys@transformshift{3.205964in}{2.608423in}%
\pgfsys@useobject{currentmarker}{}%
\end{pgfscope}%
\begin{pgfscope}%
\pgfsys@transformshift{3.186951in}{2.501983in}%
\pgfsys@useobject{currentmarker}{}%
\end{pgfscope}%
\begin{pgfscope}%
\pgfsys@transformshift{3.167703in}{2.370178in}%
\pgfsys@useobject{currentmarker}{}%
\end{pgfscope}%
\begin{pgfscope}%
\pgfsys@transformshift{3.149159in}{2.297794in}%
\pgfsys@useobject{currentmarker}{}%
\end{pgfscope}%
\begin{pgfscope}%
\pgfsys@transformshift{3.129208in}{2.259268in}%
\pgfsys@useobject{currentmarker}{}%
\end{pgfscope}%
\begin{pgfscope}%
\pgfsys@transformshift{3.110429in}{2.244980in}%
\pgfsys@useobject{currentmarker}{}%
\end{pgfscope}%
\begin{pgfscope}%
\pgfsys@transformshift{3.090713in}{2.250830in}%
\pgfsys@useobject{currentmarker}{}%
\end{pgfscope}%
\begin{pgfscope}%
\pgfsys@transformshift{3.072168in}{2.270801in}%
\pgfsys@useobject{currentmarker}{}%
\end{pgfscope}%
\begin{pgfscope}%
\pgfsys@transformshift{3.050338in}{2.306986in}%
\pgfsys@useobject{currentmarker}{}%
\end{pgfscope}%
\begin{pgfscope}%
\pgfsys@transformshift{3.031796in}{2.413905in}%
\pgfsys@useobject{currentmarker}{}%
\end{pgfscope}%
\begin{pgfscope}%
\pgfsys@transformshift{3.013954in}{2.554165in}%
\pgfsys@useobject{currentmarker}{}%
\end{pgfscope}%
\begin{pgfscope}%
\pgfsys@transformshift{2.994003in}{2.615468in}%
\pgfsys@useobject{currentmarker}{}%
\end{pgfscope}%
\begin{pgfscope}%
\pgfsys@transformshift{2.975695in}{2.565702in}%
\pgfsys@useobject{currentmarker}{}%
\end{pgfscope}%
\begin{pgfscope}%
\pgfsys@transformshift{2.958091in}{2.440535in}%
\pgfsys@useobject{currentmarker}{}%
\end{pgfscope}%
\begin{pgfscope}%
\pgfsys@transformshift{2.935555in}{2.328070in}%
\pgfsys@useobject{currentmarker}{}%
\end{pgfscope}%
\begin{pgfscope}%
\pgfsys@transformshift{2.917247in}{2.275768in}%
\pgfsys@useobject{currentmarker}{}%
\end{pgfscope}%
\begin{pgfscope}%
\pgfsys@transformshift{2.898937in}{2.251509in}%
\pgfsys@useobject{currentmarker}{}%
\end{pgfscope}%
\begin{pgfscope}%
\pgfsys@transformshift{2.879926in}{2.244815in}%
\pgfsys@useobject{currentmarker}{}%
\end{pgfscope}%
\begin{pgfscope}%
\pgfsys@transformshift{2.859739in}{2.253622in}%
\pgfsys@useobject{currentmarker}{}%
\end{pgfscope}%
\begin{pgfscope}%
\pgfsys@transformshift{2.840020in}{2.270537in}%
\pgfsys@useobject{currentmarker}{}%
\end{pgfscope}%
\begin{pgfscope}%
\pgfsys@transformshift{2.821712in}{2.318356in}%
\pgfsys@useobject{currentmarker}{}%
\end{pgfscope}%
\begin{pgfscope}%
\pgfsys@transformshift{2.802464in}{2.414783in}%
\pgfsys@useobject{currentmarker}{}%
\end{pgfscope}%
\begin{pgfscope}%
\pgfsys@transformshift{2.784156in}{2.542646in}%
\pgfsys@useobject{currentmarker}{}%
\end{pgfscope}%
\begin{pgfscope}%
\pgfsys@transformshift{2.766081in}{2.612054in}%
\pgfsys@useobject{currentmarker}{}%
\end{pgfscope}%
\begin{pgfscope}%
\pgfsys@transformshift{2.743782in}{2.550352in}%
\pgfsys@useobject{currentmarker}{}%
\end{pgfscope}%
\begin{pgfscope}%
\pgfsys@transformshift{2.725003in}{2.524077in}%
\pgfsys@useobject{currentmarker}{}%
\end{pgfscope}%
\begin{pgfscope}%
\pgfsys@transformshift{2.708338in}{2.393563in}%
\pgfsys@useobject{currentmarker}{}%
\end{pgfscope}%
\begin{pgfscope}%
\pgfsys@transformshift{2.686507in}{2.299381in}%
\pgfsys@useobject{currentmarker}{}%
\end{pgfscope}%
\begin{pgfscope}%
\pgfsys@transformshift{2.667965in}{2.267771in}%
\pgfsys@useobject{currentmarker}{}%
\end{pgfscope}%
\begin{pgfscope}%
\pgfsys@transformshift{2.649186in}{2.246882in}%
\pgfsys@useobject{currentmarker}{}%
\end{pgfscope}%
\begin{pgfscope}%
\pgfsys@transformshift{2.629233in}{2.245685in}%
\pgfsys@useobject{currentmarker}{}%
\end{pgfscope}%
\begin{pgfscope}%
\pgfsys@transformshift{2.610220in}{2.260786in}%
\pgfsys@useobject{currentmarker}{}%
\end{pgfscope}%
\begin{pgfscope}%
\pgfsys@transformshift{2.589564in}{2.301090in}%
\pgfsys@useobject{currentmarker}{}%
\end{pgfscope}%
\begin{pgfscope}%
\pgfsys@transformshift{2.573604in}{2.366561in}%
\pgfsys@useobject{currentmarker}{}%
\end{pgfscope}%
\begin{pgfscope}%
\pgfsys@transformshift{2.551774in}{2.522465in}%
\pgfsys@useobject{currentmarker}{}%
\end{pgfscope}%
\begin{pgfscope}%
\pgfsys@transformshift{2.532995in}{2.596625in}%
\pgfsys@useobject{currentmarker}{}%
\end{pgfscope}%
\begin{pgfscope}%
\pgfsys@transformshift{2.514216in}{2.599043in}%
\pgfsys@useobject{currentmarker}{}%
\end{pgfscope}%
\begin{pgfscope}%
\pgfsys@transformshift{2.496377in}{2.550419in}%
\pgfsys@useobject{currentmarker}{}%
\end{pgfscope}%
\begin{pgfscope}%
\pgfsys@transformshift{2.496142in}{2.599148in}%
\pgfsys@useobject{currentmarker}{}%
\end{pgfscope}%
\begin{pgfscope}%
\pgfsys@transformshift{2.474547in}{2.612100in}%
\pgfsys@useobject{currentmarker}{}%
\end{pgfscope}%
\begin{pgfscope}%
\pgfsys@transformshift{2.459055in}{2.571591in}%
\pgfsys@useobject{currentmarker}{}%
\end{pgfscope}%
\begin{pgfscope}%
\pgfsys@transformshift{2.433234in}{2.383513in}%
\pgfsys@useobject{currentmarker}{}%
\end{pgfscope}%
\begin{pgfscope}%
\pgfsys@transformshift{2.417272in}{2.306193in}%
\pgfsys@useobject{currentmarker}{}%
\end{pgfscope}%
\begin{pgfscope}%
\pgfsys@transformshift{2.400138in}{2.270591in}%
\pgfsys@useobject{currentmarker}{}%
\end{pgfscope}%
\begin{pgfscope}%
\pgfsys@transformshift{2.379011in}{2.247519in}%
\pgfsys@useobject{currentmarker}{}%
\end{pgfscope}%
\begin{pgfscope}%
\pgfsys@transformshift{2.360469in}{2.245061in}%
\pgfsys@useobject{currentmarker}{}%
\end{pgfscope}%
\begin{pgfscope}%
\pgfsys@transformshift{2.341456in}{2.259489in}%
\pgfsys@useobject{currentmarker}{}%
\end{pgfscope}%
\begin{pgfscope}%
\pgfsys@transformshift{2.322911in}{2.282794in}%
\pgfsys@useobject{currentmarker}{}%
\end{pgfscope}%
\begin{pgfscope}%
\pgfsys@transformshift{2.302960in}{2.342208in}%
\pgfsys@useobject{currentmarker}{}%
\end{pgfscope}%
\begin{pgfscope}%
\pgfsys@transformshift{2.283007in}{2.484228in}%
\pgfsys@useobject{currentmarker}{}%
\end{pgfscope}%
\begin{pgfscope}%
\pgfsys@transformshift{2.262351in}{2.608041in}%
\pgfsys@useobject{currentmarker}{}%
\end{pgfscope}%
\begin{pgfscope}%
\pgfsys@transformshift{2.245921in}{2.613691in}%
\pgfsys@useobject{currentmarker}{}%
\end{pgfscope}%
\begin{pgfscope}%
\pgfsys@transformshift{2.224090in}{2.529688in}%
\pgfsys@useobject{currentmarker}{}%
\end{pgfscope}%
\begin{pgfscope}%
\pgfsys@transformshift{2.207660in}{2.406196in}%
\pgfsys@useobject{currentmarker}{}%
\end{pgfscope}%
\begin{pgfscope}%
\pgfsys@transformshift{2.187238in}{2.316644in}%
\pgfsys@useobject{currentmarker}{}%
\end{pgfscope}%
\begin{pgfscope}%
\pgfsys@transformshift{2.169164in}{2.271991in}%
\pgfsys@useobject{currentmarker}{}%
\end{pgfscope}%
\begin{pgfscope}%
\pgfsys@transformshift{2.150151in}{2.250814in}%
\pgfsys@useobject{currentmarker}{}%
\end{pgfscope}%
\begin{pgfscope}%
\pgfsys@transformshift{2.126678in}{2.247614in}%
\pgfsys@useobject{currentmarker}{}%
\end{pgfscope}%
\begin{pgfscope}%
\pgfsys@transformshift{2.110716in}{2.253506in}%
\pgfsys@useobject{currentmarker}{}%
\end{pgfscope}%
\begin{pgfscope}%
\pgfsys@transformshift{2.088888in}{2.276295in}%
\pgfsys@useobject{currentmarker}{}%
\end{pgfscope}%
\begin{pgfscope}%
\pgfsys@transformshift{2.073160in}{2.319097in}%
\pgfsys@useobject{currentmarker}{}%
\end{pgfscope}%
\begin{pgfscope}%
\pgfsys@transformshift{2.051330in}{2.428420in}%
\pgfsys@useobject{currentmarker}{}%
\end{pgfscope}%
\begin{pgfscope}%
\pgfsys@transformshift{2.033960in}{2.524223in}%
\pgfsys@useobject{currentmarker}{}%
\end{pgfscope}%
\begin{pgfscope}%
\pgfsys@transformshift{2.014712in}{2.612742in}%
\pgfsys@useobject{currentmarker}{}%
\end{pgfscope}%
\begin{pgfscope}%
\pgfsys@transformshift{1.996404in}{2.591065in}%
\pgfsys@useobject{currentmarker}{}%
\end{pgfscope}%
\begin{pgfscope}%
\pgfsys@transformshift{1.975043in}{2.466907in}%
\pgfsys@useobject{currentmarker}{}%
\end{pgfscope}%
\begin{pgfscope}%
\pgfsys@transformshift{1.956266in}{2.376169in}%
\pgfsys@useobject{currentmarker}{}%
\end{pgfscope}%
\begin{pgfscope}%
\pgfsys@transformshift{1.937721in}{2.305704in}%
\pgfsys@useobject{currentmarker}{}%
\end{pgfscope}%
\begin{pgfscope}%
\pgfsys@transformshift{1.915891in}{2.262898in}%
\pgfsys@useobject{currentmarker}{}%
\end{pgfscope}%
\begin{pgfscope}%
\pgfsys@transformshift{1.901808in}{2.250585in}%
\pgfsys@useobject{currentmarker}{}%
\end{pgfscope}%
\begin{pgfscope}%
\pgfsys@transformshift{1.879508in}{2.246279in}%
\pgfsys@useobject{currentmarker}{}%
\end{pgfscope}%
\begin{pgfscope}%
\pgfsys@transformshift{1.860496in}{2.258331in}%
\pgfsys@useobject{currentmarker}{}%
\end{pgfscope}%
\begin{pgfscope}%
\pgfsys@transformshift{1.840778in}{2.284436in}%
\pgfsys@useobject{currentmarker}{}%
\end{pgfscope}%
\begin{pgfscope}%
\pgfsys@transformshift{1.821999in}{2.325397in}%
\pgfsys@useobject{currentmarker}{}%
\end{pgfscope}%
\begin{pgfscope}%
\pgfsys@transformshift{1.803691in}{2.396446in}%
\pgfsys@useobject{currentmarker}{}%
\end{pgfscope}%
\begin{pgfscope}%
\pgfsys@transformshift{1.783035in}{2.524939in}%
\pgfsys@useobject{currentmarker}{}%
\end{pgfscope}%
\begin{pgfscope}%
\pgfsys@transformshift{1.765195in}{2.606809in}%
\pgfsys@useobject{currentmarker}{}%
\end{pgfscope}%
\begin{pgfscope}%
\pgfsys@transformshift{1.743365in}{2.607787in}%
\pgfsys@useobject{currentmarker}{}%
\end{pgfscope}%
\begin{pgfscope}%
\pgfsys@transformshift{1.727874in}{2.553465in}%
\pgfsys@useobject{currentmarker}{}%
\end{pgfscope}%
\begin{pgfscope}%
\pgfsys@transformshift{1.706278in}{2.415033in}%
\pgfsys@useobject{currentmarker}{}%
\end{pgfscope}%
\begin{pgfscope}%
\pgfsys@transformshift{1.687500in}{2.325948in}%
\pgfsys@useobject{currentmarker}{}%
\end{pgfscope}%
\begin{pgfscope}%
\pgfsys@transformshift{1.668957in}{2.286800in}%
\pgfsys@useobject{currentmarker}{}%
\end{pgfscope}%
\begin{pgfscope}%
\pgfsys@transformshift{1.650647in}{2.259951in}%
\pgfsys@useobject{currentmarker}{}%
\end{pgfscope}%
\begin{pgfscope}%
\pgfsys@transformshift{1.630931in}{2.248190in}%
\pgfsys@useobject{currentmarker}{}%
\end{pgfscope}%
\begin{pgfscope}%
\pgfsys@transformshift{1.613326in}{2.248784in}%
\pgfsys@useobject{currentmarker}{}%
\end{pgfscope}%
\begin{pgfscope}%
\pgfsys@transformshift{1.592670in}{2.262037in}%
\pgfsys@useobject{currentmarker}{}%
\end{pgfscope}%
\begin{pgfscope}%
\pgfsys@transformshift{1.574594in}{2.293233in}%
\pgfsys@useobject{currentmarker}{}%
\end{pgfscope}%
\begin{pgfscope}%
\pgfsys@transformshift{1.550887in}{2.358095in}%
\pgfsys@useobject{currentmarker}{}%
\end{pgfscope}%
\begin{pgfscope}%
\pgfsys@transformshift{1.533753in}{2.437965in}%
\pgfsys@useobject{currentmarker}{}%
\end{pgfscope}%
\begin{pgfscope}%
\pgfsys@transformshift{1.513096in}{2.569865in}%
\pgfsys@useobject{currentmarker}{}%
\end{pgfscope}%
\begin{pgfscope}%
\pgfsys@transformshift{1.495257in}{2.623633in}%
\pgfsys@useobject{currentmarker}{}%
\end{pgfscope}%
\begin{pgfscope}%
\pgfsys@transformshift{1.476947in}{2.620849in}%
\pgfsys@useobject{currentmarker}{}%
\end{pgfscope}%
\begin{pgfscope}%
\pgfsys@transformshift{1.459108in}{2.550618in}%
\pgfsys@useobject{currentmarker}{}%
\end{pgfscope}%
\begin{pgfscope}%
\pgfsys@transformshift{1.438921in}{2.418088in}%
\pgfsys@useobject{currentmarker}{}%
\end{pgfscope}%
\begin{pgfscope}%
\pgfsys@transformshift{1.419673in}{2.345663in}%
\pgfsys@useobject{currentmarker}{}%
\end{pgfscope}%
\begin{pgfscope}%
\pgfsys@transformshift{1.396671in}{2.409206in}%
\pgfsys@useobject{currentmarker}{}%
\end{pgfscope}%
\begin{pgfscope}%
\pgfsys@transformshift{1.379769in}{2.332683in}%
\pgfsys@useobject{currentmarker}{}%
\end{pgfscope}%
\begin{pgfscope}%
\pgfsys@transformshift{1.359113in}{2.281532in}%
\pgfsys@useobject{currentmarker}{}%
\end{pgfscope}%
\begin{pgfscope}%
\pgfsys@transformshift{1.342448in}{2.258275in}%
\pgfsys@useobject{currentmarker}{}%
\end{pgfscope}%
\begin{pgfscope}%
\pgfsys@transformshift{1.324138in}{2.247514in}%
\pgfsys@useobject{currentmarker}{}%
\end{pgfscope}%
\begin{pgfscope}%
\pgfsys@transformshift{1.304656in}{2.252239in}%
\pgfsys@useobject{currentmarker}{}%
\end{pgfscope}%
\begin{pgfscope}%
\pgfsys@transformshift{1.284234in}{2.271866in}%
\pgfsys@useobject{currentmarker}{}%
\end{pgfscope}%
\begin{pgfscope}%
\pgfsys@transformshift{1.264049in}{2.309531in}%
\pgfsys@useobject{currentmarker}{}%
\end{pgfscope}%
\begin{pgfscope}%
\pgfsys@transformshift{1.246678in}{2.344831in}%
\pgfsys@useobject{currentmarker}{}%
\end{pgfscope}%
\begin{pgfscope}%
\pgfsys@transformshift{1.225788in}{2.467525in}%
\pgfsys@useobject{currentmarker}{}%
\end{pgfscope}%
\begin{pgfscope}%
\pgfsys@transformshift{1.207243in}{2.522740in}%
\pgfsys@useobject{currentmarker}{}%
\end{pgfscope}%
\begin{pgfscope}%
\pgfsys@transformshift{1.187527in}{2.621125in}%
\pgfsys@useobject{currentmarker}{}%
\end{pgfscope}%
\begin{pgfscope}%
\pgfsys@transformshift{1.170391in}{2.638660in}%
\pgfsys@useobject{currentmarker}{}%
\end{pgfscope}%
\begin{pgfscope}%
\pgfsys@transformshift{1.149735in}{2.636918in}%
\pgfsys@useobject{currentmarker}{}%
\end{pgfscope}%
\begin{pgfscope}%
\pgfsys@transformshift{1.132130in}{2.577886in}%
\pgfsys@useobject{currentmarker}{}%
\end{pgfscope}%
\begin{pgfscope}%
\pgfsys@transformshift{1.112179in}{2.435622in}%
\pgfsys@useobject{currentmarker}{}%
\end{pgfscope}%
\begin{pgfscope}%
\pgfsys@transformshift{1.091757in}{2.335237in}%
\pgfsys@useobject{currentmarker}{}%
\end{pgfscope}%
\begin{pgfscope}%
\pgfsys@transformshift{1.071101in}{2.288567in}%
\pgfsys@useobject{currentmarker}{}%
\end{pgfscope}%
\begin{pgfscope}%
\pgfsys@transformshift{1.053025in}{2.266706in}%
\pgfsys@useobject{currentmarker}{}%
\end{pgfscope}%
\begin{pgfscope}%
\pgfsys@transformshift{1.036595in}{2.251677in}%
\pgfsys@useobject{currentmarker}{}%
\end{pgfscope}%
\begin{pgfscope}%
\pgfsys@transformshift{1.015704in}{2.250369in}%
\pgfsys@useobject{currentmarker}{}%
\end{pgfscope}%
\begin{pgfscope}%
\pgfsys@transformshift{0.995517in}{2.269524in}%
\pgfsys@useobject{currentmarker}{}%
\end{pgfscope}%
\begin{pgfscope}%
\pgfsys@transformshift{0.979321in}{2.298980in}%
\pgfsys@useobject{currentmarker}{}%
\end{pgfscope}%
\begin{pgfscope}%
\pgfsys@transformshift{0.958196in}{2.361887in}%
\pgfsys@useobject{currentmarker}{}%
\end{pgfscope}%
\begin{pgfscope}%
\pgfsys@transformshift{0.940356in}{2.460791in}%
\pgfsys@useobject{currentmarker}{}%
\end{pgfscope}%
\begin{pgfscope}%
\pgfsys@transformshift{0.918761in}{2.591982in}%
\pgfsys@useobject{currentmarker}{}%
\end{pgfscope}%
\begin{pgfscope}%
\pgfsys@transformshift{0.898339in}{2.649438in}%
\pgfsys@useobject{currentmarker}{}%
\end{pgfscope}%
\begin{pgfscope}%
\pgfsys@transformshift{0.880265in}{2.642469in}%
\pgfsys@useobject{currentmarker}{}%
\end{pgfscope}%
\begin{pgfscope}%
\pgfsys@transformshift{0.862426in}{2.564985in}%
\pgfsys@useobject{currentmarker}{}%
\end{pgfscope}%
\begin{pgfscope}%
\pgfsys@transformshift{0.842239in}{2.464481in}%
\pgfsys@useobject{currentmarker}{}%
\end{pgfscope}%
\begin{pgfscope}%
\pgfsys@transformshift{0.824870in}{2.374113in}%
\pgfsys@useobject{currentmarker}{}%
\end{pgfscope}%
\begin{pgfscope}%
\pgfsys@transformshift{0.803743in}{2.326932in}%
\pgfsys@useobject{currentmarker}{}%
\end{pgfscope}%
\begin{pgfscope}%
\pgfsys@transformshift{0.786138in}{2.287199in}%
\pgfsys@useobject{currentmarker}{}%
\end{pgfscope}%
\begin{pgfscope}%
\pgfsys@transformshift{0.763136in}{2.259667in}%
\pgfsys@useobject{currentmarker}{}%
\end{pgfscope}%
\begin{pgfscope}%
\pgfsys@transformshift{0.744826in}{2.251487in}%
\pgfsys@useobject{currentmarker}{}%
\end{pgfscope}%
\begin{pgfscope}%
\pgfsys@transformshift{0.727692in}{2.257221in}%
\pgfsys@useobject{currentmarker}{}%
\end{pgfscope}%
\begin{pgfscope}%
\pgfsys@transformshift{0.707505in}{2.280544in}%
\pgfsys@useobject{currentmarker}{}%
\end{pgfscope}%
\begin{pgfscope}%
\pgfsys@transformshift{0.689195in}{2.310909in}%
\pgfsys@useobject{currentmarker}{}%
\end{pgfscope}%
\begin{pgfscope}%
\pgfsys@transformshift{0.667835in}{2.395309in}%
\pgfsys@useobject{currentmarker}{}%
\end{pgfscope}%
\begin{pgfscope}%
\pgfsys@transformshift{0.649996in}{2.488064in}%
\pgfsys@useobject{currentmarker}{}%
\end{pgfscope}%
\begin{pgfscope}%
\pgfsys@transformshift{0.651170in}{2.483180in}%
\pgfsys@useobject{currentmarker}{}%
\end{pgfscope}%
\begin{pgfscope}%
\pgfsys@transformshift{0.657039in}{2.428515in}%
\pgfsys@useobject{currentmarker}{}%
\end{pgfscope}%
\begin{pgfscope}%
\pgfsys@transformshift{0.676755in}{2.315005in}%
\pgfsys@useobject{currentmarker}{}%
\end{pgfscope}%
\begin{pgfscope}%
\pgfsys@transformshift{0.696472in}{2.264413in}%
\pgfsys@useobject{currentmarker}{}%
\end{pgfscope}%
\begin{pgfscope}%
\pgfsys@transformshift{0.714547in}{2.251047in}%
\pgfsys@useobject{currentmarker}{}%
\end{pgfscope}%
\begin{pgfscope}%
\pgfsys@transformshift{0.732855in}{2.272342in}%
\pgfsys@useobject{currentmarker}{}%
\end{pgfscope}%
\begin{pgfscope}%
\pgfsys@transformshift{0.754685in}{2.335151in}%
\pgfsys@useobject{currentmarker}{}%
\end{pgfscope}%
\begin{pgfscope}%
\pgfsys@transformshift{0.776516in}{2.484104in}%
\pgfsys@useobject{currentmarker}{}%
\end{pgfscope}%
\begin{pgfscope}%
\pgfsys@transformshift{0.791772in}{2.617536in}%
\pgfsys@useobject{currentmarker}{}%
\end{pgfscope}%
\begin{pgfscope}%
\pgfsys@transformshift{0.810317in}{2.648360in}%
\pgfsys@useobject{currentmarker}{}%
\end{pgfscope}%
\begin{pgfscope}%
\pgfsys@transformshift{0.829564in}{2.539450in}%
\pgfsys@useobject{currentmarker}{}%
\end{pgfscope}%
\begin{pgfscope}%
\pgfsys@transformshift{0.848812in}{2.379751in}%
\pgfsys@useobject{currentmarker}{}%
\end{pgfscope}%
\begin{pgfscope}%
\pgfsys@transformshift{0.866651in}{2.292267in}%
\pgfsys@useobject{currentmarker}{}%
\end{pgfscope}%
\begin{pgfscope}%
\pgfsys@transformshift{0.888247in}{2.253534in}%
\pgfsys@useobject{currentmarker}{}%
\end{pgfscope}%
\begin{pgfscope}%
\pgfsys@transformshift{0.906555in}{2.254399in}%
\pgfsys@useobject{currentmarker}{}%
\end{pgfscope}%
\begin{pgfscope}%
\pgfsys@transformshift{0.925334in}{2.282551in}%
\pgfsys@useobject{currentmarker}{}%
\end{pgfscope}%
\begin{pgfscope}%
\pgfsys@transformshift{0.944347in}{2.341940in}%
\pgfsys@useobject{currentmarker}{}%
\end{pgfscope}%
\begin{pgfscope}%
\pgfsys@transformshift{0.961716in}{2.471104in}%
\pgfsys@useobject{currentmarker}{}%
\end{pgfscope}%
\begin{pgfscope}%
\pgfsys@transformshift{0.983546in}{2.347607in}%
\pgfsys@useobject{currentmarker}{}%
\end{pgfscope}%
\begin{pgfscope}%
\pgfsys@transformshift{1.001856in}{2.279959in}%
\pgfsys@useobject{currentmarker}{}%
\end{pgfscope}%
\begin{pgfscope}%
\pgfsys@transformshift{1.023215in}{2.248802in}%
\pgfsys@useobject{currentmarker}{}%
\end{pgfscope}%
\begin{pgfscope}%
\pgfsys@transformshift{1.041289in}{2.254851in}%
\pgfsys@useobject{currentmarker}{}%
\end{pgfscope}%
\begin{pgfscope}%
\pgfsys@transformshift{1.059130in}{2.284949in}%
\pgfsys@useobject{currentmarker}{}%
\end{pgfscope}%
\begin{pgfscope}%
\pgfsys@transformshift{1.080255in}{2.372352in}%
\pgfsys@useobject{currentmarker}{}%
\end{pgfscope}%
\begin{pgfscope}%
\pgfsys@transformshift{1.101146in}{2.544450in}%
\pgfsys@useobject{currentmarker}{}%
\end{pgfscope}%
\begin{pgfscope}%
\pgfsys@transformshift{1.116168in}{2.628914in}%
\pgfsys@useobject{currentmarker}{}%
\end{pgfscope}%
\begin{pgfscope}%
\pgfsys@transformshift{1.137059in}{2.587070in}%
\pgfsys@useobject{currentmarker}{}%
\end{pgfscope}%
\begin{pgfscope}%
\pgfsys@transformshift{1.161002in}{2.392114in}%
\pgfsys@useobject{currentmarker}{}%
\end{pgfscope}%
\begin{pgfscope}%
\pgfsys@transformshift{1.172268in}{2.313491in}%
\pgfsys@useobject{currentmarker}{}%
\end{pgfscope}%
\begin{pgfscope}%
\pgfsys@transformshift{1.193395in}{2.262389in}%
\pgfsys@useobject{currentmarker}{}%
\end{pgfscope}%
\begin{pgfscope}%
\pgfsys@transformshift{1.217337in}{2.247053in}%
\pgfsys@useobject{currentmarker}{}%
\end{pgfscope}%
\begin{pgfscope}%
\pgfsys@transformshift{1.232125in}{2.263453in}%
\pgfsys@useobject{currentmarker}{}%
\end{pgfscope}%
\begin{pgfscope}%
\pgfsys@transformshift{1.249964in}{2.298649in}%
\pgfsys@useobject{currentmarker}{}%
\end{pgfscope}%
\begin{pgfscope}%
\pgfsys@transformshift{1.270855in}{2.397440in}%
\pgfsys@useobject{currentmarker}{}%
\end{pgfscope}%
\begin{pgfscope}%
\pgfsys@transformshift{1.289399in}{2.538223in}%
\pgfsys@useobject{currentmarker}{}%
\end{pgfscope}%
\begin{pgfscope}%
\pgfsys@transformshift{1.310758in}{2.627060in}%
\pgfsys@useobject{currentmarker}{}%
\end{pgfscope}%
\begin{pgfscope}%
\pgfsys@transformshift{1.331180in}{2.554329in}%
\pgfsys@useobject{currentmarker}{}%
\end{pgfscope}%
\begin{pgfscope}%
\pgfsys@transformshift{1.348785in}{2.401265in}%
\pgfsys@useobject{currentmarker}{}%
\end{pgfscope}%
\begin{pgfscope}%
\pgfsys@transformshift{1.366859in}{2.302112in}%
\pgfsys@useobject{currentmarker}{}%
\end{pgfscope}%
\begin{pgfscope}%
\pgfsys@transformshift{1.387751in}{2.257923in}%
\pgfsys@useobject{currentmarker}{}%
\end{pgfscope}%
\begin{pgfscope}%
\pgfsys@transformshift{1.406059in}{2.245748in}%
\pgfsys@useobject{currentmarker}{}%
\end{pgfscope}%
\begin{pgfscope}%
\pgfsys@transformshift{1.424367in}{2.251586in}%
\pgfsys@useobject{currentmarker}{}%
\end{pgfscope}%
\begin{pgfscope}%
\pgfsys@transformshift{1.444554in}{2.282433in}%
\pgfsys@useobject{currentmarker}{}%
\end{pgfscope}%
\begin{pgfscope}%
\pgfsys@transformshift{1.465916in}{2.354824in}%
\pgfsys@useobject{currentmarker}{}%
\end{pgfscope}%
\begin{pgfscope}%
\pgfsys@transformshift{1.483989in}{2.488908in}%
\pgfsys@useobject{currentmarker}{}%
\end{pgfscope}%
\begin{pgfscope}%
\pgfsys@transformshift{1.501829in}{2.589713in}%
\pgfsys@useobject{currentmarker}{}%
\end{pgfscope}%
\begin{pgfscope}%
\pgfsys@transformshift{1.522485in}{2.619110in}%
\pgfsys@useobject{currentmarker}{}%
\end{pgfscope}%
\begin{pgfscope}%
\pgfsys@transformshift{1.540558in}{2.529539in}%
\pgfsys@useobject{currentmarker}{}%
\end{pgfscope}%
\begin{pgfscope}%
\pgfsys@transformshift{1.558634in}{2.383515in}%
\pgfsys@useobject{currentmarker}{}%
\end{pgfscope}%
\begin{pgfscope}%
\pgfsys@transformshift{1.579759in}{2.286155in}%
\pgfsys@useobject{currentmarker}{}%
\end{pgfscope}%
\begin{pgfscope}%
\pgfsys@transformshift{1.597833in}{2.262004in}%
\pgfsys@useobject{currentmarker}{}%
\end{pgfscope}%
\begin{pgfscope}%
\pgfsys@transformshift{1.615908in}{2.245866in}%
\pgfsys@useobject{currentmarker}{}%
\end{pgfscope}%
\begin{pgfscope}%
\pgfsys@transformshift{1.636564in}{2.251035in}%
\pgfsys@useobject{currentmarker}{}%
\end{pgfscope}%
\begin{pgfscope}%
\pgfsys@transformshift{1.657221in}{2.274528in}%
\pgfsys@useobject{currentmarker}{}%
\end{pgfscope}%
\begin{pgfscope}%
\pgfsys@transformshift{1.674823in}{2.321952in}%
\pgfsys@useobject{currentmarker}{}%
\end{pgfscope}%
\begin{pgfscope}%
\pgfsys@transformshift{1.692899in}{2.420461in}%
\pgfsys@useobject{currentmarker}{}%
\end{pgfscope}%
\begin{pgfscope}%
\pgfsys@transformshift{1.714024in}{2.548696in}%
\pgfsys@useobject{currentmarker}{}%
\end{pgfscope}%
\begin{pgfscope}%
\pgfsys@transformshift{1.731394in}{2.617416in}%
\pgfsys@useobject{currentmarker}{}%
\end{pgfscope}%
\begin{pgfscope}%
\pgfsys@transformshift{1.751111in}{2.580760in}%
\pgfsys@useobject{currentmarker}{}%
\end{pgfscope}%
\begin{pgfscope}%
\pgfsys@transformshift{1.771064in}{2.479439in}%
\pgfsys@useobject{currentmarker}{}%
\end{pgfscope}%
\begin{pgfscope}%
\pgfsys@transformshift{1.791485in}{2.330619in}%
\pgfsys@useobject{currentmarker}{}%
\end{pgfscope}%
\begin{pgfscope}%
\pgfsys@transformshift{1.810262in}{2.273386in}%
\pgfsys@useobject{currentmarker}{}%
\end{pgfscope}%
\begin{pgfscope}%
\pgfsys@transformshift{1.828104in}{2.253374in}%
\pgfsys@useobject{currentmarker}{}%
\end{pgfscope}%
\begin{pgfscope}%
\pgfsys@transformshift{1.848523in}{2.244971in}%
\pgfsys@useobject{currentmarker}{}%
\end{pgfscope}%
\begin{pgfscope}%
\pgfsys@transformshift{1.866833in}{2.253319in}%
\pgfsys@useobject{currentmarker}{}%
\end{pgfscope}%
\begin{pgfscope}%
\pgfsys@transformshift{1.887958in}{2.275495in}%
\pgfsys@useobject{currentmarker}{}%
\end{pgfscope}%
\begin{pgfscope}%
\pgfsys@transformshift{1.904155in}{2.315595in}%
\pgfsys@useobject{currentmarker}{}%
\end{pgfscope}%
\begin{pgfscope}%
\pgfsys@transformshift{1.924811in}{2.415590in}%
\pgfsys@useobject{currentmarker}{}%
\end{pgfscope}%
\begin{pgfscope}%
\pgfsys@transformshift{1.942884in}{2.554567in}%
\pgfsys@useobject{currentmarker}{}%
\end{pgfscope}%
\begin{pgfscope}%
\pgfsys@transformshift{1.963777in}{2.613385in}%
\pgfsys@useobject{currentmarker}{}%
\end{pgfscope}%
\begin{pgfscope}%
\pgfsys@transformshift{1.982554in}{2.560514in}%
\pgfsys@useobject{currentmarker}{}%
\end{pgfscope}%
\begin{pgfscope}%
\pgfsys@transformshift{2.002272in}{2.462670in}%
\pgfsys@useobject{currentmarker}{}%
\end{pgfscope}%
\begin{pgfscope}%
\pgfsys@transformshift{2.020111in}{2.340399in}%
\pgfsys@useobject{currentmarker}{}%
\end{pgfscope}%
\begin{pgfscope}%
\pgfsys@transformshift{2.039125in}{2.285458in}%
\pgfsys@useobject{currentmarker}{}%
\end{pgfscope}%
\begin{pgfscope}%
\pgfsys@transformshift{2.059076in}{2.262878in}%
\pgfsys@useobject{currentmarker}{}%
\end{pgfscope}%
\begin{pgfscope}%
\pgfsys@transformshift{2.080437in}{2.248250in}%
\pgfsys@useobject{currentmarker}{}%
\end{pgfscope}%
\begin{pgfscope}%
\pgfsys@transformshift{2.096633in}{2.244718in}%
\pgfsys@useobject{currentmarker}{}%
\end{pgfscope}%
\begin{pgfscope}%
\pgfsys@transformshift{2.117055in}{2.260706in}%
\pgfsys@useobject{currentmarker}{}%
\end{pgfscope}%
\begin{pgfscope}%
\pgfsys@transformshift{2.134189in}{2.289037in}%
\pgfsys@useobject{currentmarker}{}%
\end{pgfscope}%
\begin{pgfscope}%
\pgfsys@transformshift{2.156959in}{2.368941in}%
\pgfsys@useobject{currentmarker}{}%
\end{pgfscope}%
\begin{pgfscope}%
\pgfsys@transformshift{2.173390in}{2.476611in}%
\pgfsys@useobject{currentmarker}{}%
\end{pgfscope}%
\begin{pgfscope}%
\pgfsys@transformshift{2.194280in}{2.597086in}%
\pgfsys@useobject{currentmarker}{}%
\end{pgfscope}%
\begin{pgfscope}%
\pgfsys@transformshift{2.212119in}{2.602872in}%
\pgfsys@useobject{currentmarker}{}%
\end{pgfscope}%
\begin{pgfscope}%
\pgfsys@transformshift{2.233481in}{2.487617in}%
\pgfsys@useobject{currentmarker}{}%
\end{pgfscope}%
\begin{pgfscope}%
\pgfsys@transformshift{2.252258in}{2.365250in}%
\pgfsys@useobject{currentmarker}{}%
\end{pgfscope}%
\begin{pgfscope}%
\pgfsys@transformshift{2.270802in}{2.294843in}%
\pgfsys@useobject{currentmarker}{}%
\end{pgfscope}%
\begin{pgfscope}%
\pgfsys@transformshift{2.292163in}{2.298045in}%
\pgfsys@useobject{currentmarker}{}%
\end{pgfscope}%
\begin{pgfscope}%
\pgfsys@transformshift{2.310237in}{2.262585in}%
\pgfsys@useobject{currentmarker}{}%
\end{pgfscope}%
\begin{pgfscope}%
\pgfsys@transformshift{2.331128in}{2.244804in}%
\pgfsys@useobject{currentmarker}{}%
\end{pgfscope}%
\begin{pgfscope}%
\pgfsys@transformshift{2.351549in}{2.249654in}%
\pgfsys@useobject{currentmarker}{}%
\end{pgfscope}%
\begin{pgfscope}%
\pgfsys@transformshift{2.367041in}{2.265662in}%
\pgfsys@useobject{currentmarker}{}%
\end{pgfscope}%
\begin{pgfscope}%
\pgfsys@transformshift{2.383708in}{2.303065in}%
\pgfsys@useobject{currentmarker}{}%
\end{pgfscope}%
\begin{pgfscope}%
\pgfsys@transformshift{2.405067in}{2.366506in}%
\pgfsys@useobject{currentmarker}{}%
\end{pgfscope}%
\begin{pgfscope}%
\pgfsys@transformshift{2.425958in}{2.472410in}%
\pgfsys@useobject{currentmarker}{}%
\end{pgfscope}%
\begin{pgfscope}%
\pgfsys@transformshift{2.443797in}{2.584190in}%
\pgfsys@useobject{currentmarker}{}%
\end{pgfscope}%
\begin{pgfscope}%
\pgfsys@transformshift{2.465158in}{2.610373in}%
\pgfsys@useobject{currentmarker}{}%
\end{pgfscope}%
\begin{pgfscope}%
\pgfsys@transformshift{2.482529in}{2.574511in}%
\pgfsys@useobject{currentmarker}{}%
\end{pgfscope}%
\begin{pgfscope}%
\pgfsys@transformshift{2.500602in}{2.489757in}%
\pgfsys@useobject{currentmarker}{}%
\end{pgfscope}%
\begin{pgfscope}%
\pgfsys@transformshift{2.519147in}{2.376633in}%
\pgfsys@useobject{currentmarker}{}%
\end{pgfscope}%
\begin{pgfscope}%
\pgfsys@transformshift{2.539803in}{2.311044in}%
\pgfsys@useobject{currentmarker}{}%
\end{pgfscope}%
\begin{pgfscope}%
\pgfsys@transformshift{2.557642in}{2.265620in}%
\pgfsys@useobject{currentmarker}{}%
\end{pgfscope}%
\begin{pgfscope}%
\pgfsys@transformshift{2.582287in}{2.245580in}%
\pgfsys@useobject{currentmarker}{}%
\end{pgfscope}%
\begin{pgfscope}%
\pgfsys@transformshift{2.597780in}{2.245672in}%
\pgfsys@useobject{currentmarker}{}%
\end{pgfscope}%
\begin{pgfscope}%
\pgfsys@transformshift{2.614916in}{2.261878in}%
\pgfsys@useobject{currentmarker}{}%
\end{pgfscope}%
\begin{pgfscope}%
\pgfsys@transformshift{2.637215in}{2.291594in}%
\pgfsys@useobject{currentmarker}{}%
\end{pgfscope}%
\begin{pgfscope}%
\pgfsys@transformshift{2.654586in}{2.360959in}%
\pgfsys@useobject{currentmarker}{}%
\end{pgfscope}%
\begin{pgfscope}%
\pgfsys@transformshift{2.675711in}{2.494234in}%
\pgfsys@useobject{currentmarker}{}%
\end{pgfscope}%
\begin{pgfscope}%
\pgfsys@transformshift{2.694019in}{2.596241in}%
\pgfsys@useobject{currentmarker}{}%
\end{pgfscope}%
\begin{pgfscope}%
\pgfsys@transformshift{2.711858in}{2.604334in}%
\pgfsys@useobject{currentmarker}{}%
\end{pgfscope}%
\begin{pgfscope}%
\pgfsys@transformshift{2.732280in}{2.490121in}%
\pgfsys@useobject{currentmarker}{}%
\end{pgfscope}%
\begin{pgfscope}%
\pgfsys@transformshift{2.750119in}{2.393099in}%
\pgfsys@useobject{currentmarker}{}%
\end{pgfscope}%
\begin{pgfscope}%
\pgfsys@transformshift{2.768898in}{2.320335in}%
\pgfsys@useobject{currentmarker}{}%
\end{pgfscope}%
\begin{pgfscope}%
\pgfsys@transformshift{2.789788in}{2.272797in}%
\pgfsys@useobject{currentmarker}{}%
\end{pgfscope}%
\begin{pgfscope}%
\pgfsys@transformshift{2.808098in}{2.253232in}%
\pgfsys@useobject{currentmarker}{}%
\end{pgfscope}%
\begin{pgfscope}%
\pgfsys@transformshift{2.829223in}{2.244095in}%
\pgfsys@useobject{currentmarker}{}%
\end{pgfscope}%
\begin{pgfscope}%
\pgfsys@transformshift{2.846594in}{2.252282in}%
\pgfsys@useobject{currentmarker}{}%
\end{pgfscope}%
\begin{pgfscope}%
\pgfsys@transformshift{2.864902in}{2.266763in}%
\pgfsys@useobject{currentmarker}{}%
\end{pgfscope}%
\begin{pgfscope}%
\pgfsys@transformshift{2.886497in}{2.306044in}%
\pgfsys@useobject{currentmarker}{}%
\end{pgfscope}%
\begin{pgfscope}%
\pgfsys@transformshift{2.904102in}{2.381950in}%
\pgfsys@useobject{currentmarker}{}%
\end{pgfscope}%
\begin{pgfscope}%
\pgfsys@transformshift{2.921941in}{2.452680in}%
\pgfsys@useobject{currentmarker}{}%
\end{pgfscope}%
\begin{pgfscope}%
\pgfsys@transformshift{2.944006in}{2.590128in}%
\pgfsys@useobject{currentmarker}{}%
\end{pgfscope}%
\begin{pgfscope}%
\pgfsys@transformshift{2.964193in}{2.608736in}%
\pgfsys@useobject{currentmarker}{}%
\end{pgfscope}%
\begin{pgfscope}%
\pgfsys@transformshift{2.981798in}{2.571293in}%
\pgfsys@useobject{currentmarker}{}%
\end{pgfscope}%
\begin{pgfscope}%
\pgfsys@transformshift{3.000575in}{2.497806in}%
\pgfsys@useobject{currentmarker}{}%
\end{pgfscope}%
\begin{pgfscope}%
\pgfsys@transformshift{3.017945in}{2.377577in}%
\pgfsys@useobject{currentmarker}{}%
\end{pgfscope}%
\begin{pgfscope}%
\pgfsys@transformshift{3.040010in}{2.293633in}%
\pgfsys@useobject{currentmarker}{}%
\end{pgfscope}%
\begin{pgfscope}%
\pgfsys@transformshift{3.056675in}{2.268000in}%
\pgfsys@useobject{currentmarker}{}%
\end{pgfscope}%
\begin{pgfscope}%
\pgfsys@transformshift{3.078976in}{2.248436in}%
\pgfsys@useobject{currentmarker}{}%
\end{pgfscope}%
\begin{pgfscope}%
\pgfsys@transformshift{3.096815in}{2.246261in}%
\pgfsys@useobject{currentmarker}{}%
\end{pgfscope}%
\begin{pgfscope}%
\pgfsys@transformshift{3.117706in}{2.258810in}%
\pgfsys@useobject{currentmarker}{}%
\end{pgfscope}%
\begin{pgfscope}%
\pgfsys@transformshift{3.136485in}{2.282585in}%
\pgfsys@useobject{currentmarker}{}%
\end{pgfscope}%
\begin{pgfscope}%
\pgfsys@transformshift{3.154793in}{2.310990in}%
\pgfsys@useobject{currentmarker}{}%
\end{pgfscope}%
\begin{pgfscope}%
\pgfsys@transformshift{3.173337in}{2.397010in}%
\pgfsys@useobject{currentmarker}{}%
\end{pgfscope}%
\begin{pgfscope}%
\pgfsys@transformshift{3.191411in}{2.516568in}%
\pgfsys@useobject{currentmarker}{}%
\end{pgfscope}%
\begin{pgfscope}%
\pgfsys@transformshift{3.212536in}{2.613759in}%
\pgfsys@useobject{currentmarker}{}%
\end{pgfscope}%
\begin{pgfscope}%
\pgfsys@transformshift{3.231549in}{2.613886in}%
\pgfsys@useobject{currentmarker}{}%
\end{pgfscope}%
\begin{pgfscope}%
\pgfsys@transformshift{3.247277in}{2.522706in}%
\pgfsys@useobject{currentmarker}{}%
\end{pgfscope}%
\begin{pgfscope}%
\pgfsys@transformshift{3.272393in}{2.390964in}%
\pgfsys@useobject{currentmarker}{}%
\end{pgfscope}%
\begin{pgfscope}%
\pgfsys@transformshift{3.289058in}{2.330234in}%
\pgfsys@useobject{currentmarker}{}%
\end{pgfscope}%
\begin{pgfscope}%
\pgfsys@transformshift{3.307837in}{2.283766in}%
\pgfsys@useobject{currentmarker}{}%
\end{pgfscope}%
\begin{pgfscope}%
\pgfsys@transformshift{3.328258in}{2.264537in}%
\pgfsys@useobject{currentmarker}{}%
\end{pgfscope}%
\begin{pgfscope}%
\pgfsys@transformshift{3.346566in}{2.261821in}%
\pgfsys@useobject{currentmarker}{}%
\end{pgfscope}%
\begin{pgfscope}%
\pgfsys@transformshift{3.364171in}{2.246992in}%
\pgfsys@useobject{currentmarker}{}%
\end{pgfscope}%
\begin{pgfscope}%
\pgfsys@transformshift{3.385767in}{2.252493in}%
\pgfsys@useobject{currentmarker}{}%
\end{pgfscope}%
\begin{pgfscope}%
\pgfsys@transformshift{3.403841in}{2.271283in}%
\pgfsys@useobject{currentmarker}{}%
\end{pgfscope}%
\begin{pgfscope}%
\pgfsys@transformshift{3.427314in}{2.306373in}%
\pgfsys@useobject{currentmarker}{}%
\end{pgfscope}%
\begin{pgfscope}%
\pgfsys@transformshift{3.443510in}{2.376481in}%
\pgfsys@useobject{currentmarker}{}%
\end{pgfscope}%
\begin{pgfscope}%
\pgfsys@transformshift{3.464166in}{2.483506in}%
\pgfsys@useobject{currentmarker}{}%
\end{pgfscope}%
\begin{pgfscope}%
\pgfsys@transformshift{3.482474in}{2.595923in}%
\pgfsys@useobject{currentmarker}{}%
\end{pgfscope}%
\begin{pgfscope}%
\pgfsys@transformshift{3.500079in}{2.630456in}%
\pgfsys@useobject{currentmarker}{}%
\end{pgfscope}%
\begin{pgfscope}%
\pgfsys@transformshift{3.518623in}{2.598019in}%
\pgfsys@useobject{currentmarker}{}%
\end{pgfscope}%
\begin{pgfscope}%
\pgfsys@transformshift{3.538811in}{2.465961in}%
\pgfsys@useobject{currentmarker}{}%
\end{pgfscope}%
\begin{pgfscope}%
\pgfsys@transformshift{3.558058in}{2.356212in}%
\pgfsys@useobject{currentmarker}{}%
\end{pgfscope}%
\begin{pgfscope}%
\pgfsys@transformshift{3.578009in}{2.298282in}%
\pgfsys@useobject{currentmarker}{}%
\end{pgfscope}%
\begin{pgfscope}%
\pgfsys@transformshift{3.600310in}{2.270483in}%
\pgfsys@useobject{currentmarker}{}%
\end{pgfscope}%
\begin{pgfscope}%
\pgfsys@transformshift{3.615098in}{2.253252in}%
\pgfsys@useobject{currentmarker}{}%
\end{pgfscope}%
\begin{pgfscope}%
\pgfsys@transformshift{3.635518in}{2.253066in}%
\pgfsys@useobject{currentmarker}{}%
\end{pgfscope}%
\begin{pgfscope}%
\pgfsys@transformshift{3.653123in}{2.249558in}%
\pgfsys@useobject{currentmarker}{}%
\end{pgfscope}%
\begin{pgfscope}%
\pgfsys@transformshift{3.674484in}{2.267646in}%
\pgfsys@useobject{currentmarker}{}%
\end{pgfscope}%
\begin{pgfscope}%
\pgfsys@transformshift{3.693263in}{2.299358in}%
\pgfsys@useobject{currentmarker}{}%
\end{pgfscope}%
\begin{pgfscope}%
\pgfsys@transformshift{3.712040in}{2.362246in}%
\pgfsys@useobject{currentmarker}{}%
\end{pgfscope}%
\begin{pgfscope}%
\pgfsys@transformshift{3.729410in}{2.456950in}%
\pgfsys@useobject{currentmarker}{}%
\end{pgfscope}%
\begin{pgfscope}%
\pgfsys@transformshift{3.749127in}{2.519061in}%
\pgfsys@useobject{currentmarker}{}%
\end{pgfscope}%
\begin{pgfscope}%
\pgfsys@transformshift{3.770254in}{2.628925in}%
\pgfsys@useobject{currentmarker}{}%
\end{pgfscope}%
\begin{pgfscope}%
\pgfsys@transformshift{3.789736in}{2.632764in}%
\pgfsys@useobject{currentmarker}{}%
\end{pgfscope}%
\begin{pgfscope}%
\pgfsys@transformshift{3.809689in}{2.562416in}%
\pgfsys@useobject{currentmarker}{}%
\end{pgfscope}%
\begin{pgfscope}%
\pgfsys@transformshift{3.828702in}{2.446914in}%
\pgfsys@useobject{currentmarker}{}%
\end{pgfscope}%
\begin{pgfscope}%
\pgfsys@transformshift{3.845133in}{2.358216in}%
\pgfsys@useobject{currentmarker}{}%
\end{pgfscope}%
\begin{pgfscope}%
\pgfsys@transformshift{3.864380in}{2.301962in}%
\pgfsys@useobject{currentmarker}{}%
\end{pgfscope}%
\begin{pgfscope}%
\pgfsys@transformshift{3.889496in}{2.260399in}%
\pgfsys@useobject{currentmarker}{}%
\end{pgfscope}%
\begin{pgfscope}%
\pgfsys@transformshift{3.905222in}{2.253043in}%
\pgfsys@useobject{currentmarker}{}%
\end{pgfscope}%
\begin{pgfscope}%
\pgfsys@transformshift{3.924940in}{2.248194in}%
\pgfsys@useobject{currentmarker}{}%
\end{pgfscope}%
\begin{pgfscope}%
\pgfsys@transformshift{3.939494in}{2.257375in}%
\pgfsys@useobject{currentmarker}{}%
\end{pgfscope}%
\begin{pgfscope}%
\pgfsys@transformshift{3.962027in}{2.282528in}%
\pgfsys@useobject{currentmarker}{}%
\end{pgfscope}%
\begin{pgfscope}%
\pgfsys@transformshift{3.981040in}{2.326918in}%
\pgfsys@useobject{currentmarker}{}%
\end{pgfscope}%
\begin{pgfscope}%
\pgfsys@transformshift{4.000054in}{2.380769in}%
\pgfsys@useobject{currentmarker}{}%
\end{pgfscope}%
\begin{pgfscope}%
\pgfsys@transformshift{4.019301in}{2.498381in}%
\pgfsys@useobject{currentmarker}{}%
\end{pgfscope}%
\begin{pgfscope}%
\pgfsys@transformshift{4.038315in}{2.604185in}%
\pgfsys@useobject{currentmarker}{}%
\end{pgfscope}%
\begin{pgfscope}%
\pgfsys@transformshift{4.057562in}{2.646460in}%
\pgfsys@useobject{currentmarker}{}%
\end{pgfscope}%
\begin{pgfscope}%
\pgfsys@transformshift{4.076105in}{2.646505in}%
\pgfsys@useobject{currentmarker}{}%
\end{pgfscope}%
\begin{pgfscope}%
\pgfsys@transformshift{4.098640in}{2.591008in}%
\pgfsys@useobject{currentmarker}{}%
\end{pgfscope}%
\begin{pgfscope}%
\pgfsys@transformshift{4.116479in}{2.504904in}%
\pgfsys@useobject{currentmarker}{}%
\end{pgfscope}%
\begin{pgfscope}%
\pgfsys@transformshift{4.135024in}{2.398195in}%
\pgfsys@useobject{currentmarker}{}%
\end{pgfscope}%
\begin{pgfscope}%
\pgfsys@transformshift{4.154506in}{2.327779in}%
\pgfsys@useobject{currentmarker}{}%
\end{pgfscope}%
\begin{pgfscope}%
\pgfsys@transformshift{4.173519in}{2.300139in}%
\pgfsys@useobject{currentmarker}{}%
\end{pgfscope}%
\begin{pgfscope}%
\pgfsys@transformshift{4.189950in}{2.269859in}%
\pgfsys@useobject{currentmarker}{}%
\end{pgfscope}%
\begin{pgfscope}%
\pgfsys@transformshift{4.211544in}{2.252261in}%
\pgfsys@useobject{currentmarker}{}%
\end{pgfscope}%
\begin{pgfscope}%
\pgfsys@transformshift{4.231965in}{2.255435in}%
\pgfsys@useobject{currentmarker}{}%
\end{pgfscope}%
\begin{pgfscope}%
\pgfsys@transformshift{4.250510in}{2.279492in}%
\pgfsys@useobject{currentmarker}{}%
\end{pgfscope}%
\begin{pgfscope}%
\pgfsys@transformshift{4.268115in}{2.649082in}%
\pgfsys@useobject{currentmarker}{}%
\end{pgfscope}%
\begin{pgfscope}%
\pgfsys@transformshift{4.290414in}{2.526389in}%
\pgfsys@useobject{currentmarker}{}%
\end{pgfscope}%
\begin{pgfscope}%
\pgfsys@transformshift{4.309661in}{2.390617in}%
\pgfsys@useobject{currentmarker}{}%
\end{pgfscope}%
\begin{pgfscope}%
\pgfsys@transformshift{4.324918in}{2.314931in}%
\pgfsys@useobject{currentmarker}{}%
\end{pgfscope}%
\begin{pgfscope}%
\pgfsys@transformshift{4.350034in}{2.267908in}%
\pgfsys@useobject{currentmarker}{}%
\end{pgfscope}%
\begin{pgfscope}%
\pgfsys@transformshift{4.365293in}{2.251955in}%
\pgfsys@useobject{currentmarker}{}%
\end{pgfscope}%
\begin{pgfscope}%
\pgfsys@transformshift{4.385009in}{2.259603in}%
\pgfsys@useobject{currentmarker}{}%
\end{pgfscope}%
\begin{pgfscope}%
\pgfsys@transformshift{4.404962in}{2.291419in}%
\pgfsys@useobject{currentmarker}{}%
\end{pgfscope}%
\begin{pgfscope}%
\pgfsys@transformshift{4.422801in}{2.350081in}%
\pgfsys@useobject{currentmarker}{}%
\end{pgfscope}%
\begin{pgfscope}%
\pgfsys@transformshift{4.441344in}{2.445750in}%
\pgfsys@useobject{currentmarker}{}%
\end{pgfscope}%
\begin{pgfscope}%
\pgfsys@transformshift{4.460357in}{2.589737in}%
\pgfsys@useobject{currentmarker}{}%
\end{pgfscope}%
\begin{pgfscope}%
\pgfsys@transformshift{4.479605in}{2.665025in}%
\pgfsys@useobject{currentmarker}{}%
\end{pgfscope}%
\begin{pgfscope}%
\pgfsys@transformshift{4.474207in}{2.650943in}%
\pgfsys@useobject{currentmarker}{}%
\end{pgfscope}%
\begin{pgfscope}%
\pgfsys@transformshift{4.456837in}{2.514180in}%
\pgfsys@useobject{currentmarker}{}%
\end{pgfscope}%
\begin{pgfscope}%
\pgfsys@transformshift{4.436884in}{2.353717in}%
\pgfsys@useobject{currentmarker}{}%
\end{pgfscope}%
\begin{pgfscope}%
\pgfsys@transformshift{4.416699in}{2.283793in}%
\pgfsys@useobject{currentmarker}{}%
\end{pgfscope}%
\begin{pgfscope}%
\pgfsys@transformshift{4.396511in}{2.252431in}%
\pgfsys@useobject{currentmarker}{}%
\end{pgfscope}%
\begin{pgfscope}%
\pgfsys@transformshift{4.378203in}{2.262705in}%
\pgfsys@useobject{currentmarker}{}%
\end{pgfscope}%
\begin{pgfscope}%
\pgfsys@transformshift{4.358016in}{2.316728in}%
\pgfsys@useobject{currentmarker}{}%
\end{pgfscope}%
\begin{pgfscope}%
\pgfsys@transformshift{4.339237in}{2.426894in}%
\pgfsys@useobject{currentmarker}{}%
\end{pgfscope}%
\begin{pgfscope}%
\pgfsys@transformshift{4.320458in}{2.600898in}%
\pgfsys@useobject{currentmarker}{}%
\end{pgfscope}%
\begin{pgfscope}%
\pgfsys@transformshift{4.302619in}{2.655549in}%
\pgfsys@useobject{currentmarker}{}%
\end{pgfscope}%
\begin{pgfscope}%
\pgfsys@transformshift{4.281963in}{2.535016in}%
\pgfsys@useobject{currentmarker}{}%
\end{pgfscope}%
\begin{pgfscope}%
\pgfsys@transformshift{4.260838in}{2.356087in}%
\pgfsys@useobject{currentmarker}{}%
\end{pgfscope}%
\begin{pgfscope}%
\pgfsys@transformshift{4.242528in}{2.285039in}%
\pgfsys@useobject{currentmarker}{}%
\end{pgfscope}%
\begin{pgfscope}%
\pgfsys@transformshift{4.224689in}{2.253757in}%
\pgfsys@useobject{currentmarker}{}%
\end{pgfscope}%
\begin{pgfscope}%
\pgfsys@transformshift{4.206381in}{2.256166in}%
\pgfsys@useobject{currentmarker}{}%
\end{pgfscope}%
\begin{pgfscope}%
\pgfsys@transformshift{4.186899in}{2.291455in}%
\pgfsys@useobject{currentmarker}{}%
\end{pgfscope}%
\begin{pgfscope}%
\pgfsys@transformshift{4.166477in}{2.394806in}%
\pgfsys@useobject{currentmarker}{}%
\end{pgfscope}%
\begin{pgfscope}%
\pgfsys@transformshift{4.146758in}{2.559274in}%
\pgfsys@useobject{currentmarker}{}%
\end{pgfscope}%
\begin{pgfscope}%
\pgfsys@transformshift{4.128216in}{2.644939in}%
\pgfsys@useobject{currentmarker}{}%
\end{pgfscope}%
\begin{pgfscope}%
\pgfsys@transformshift{4.109906in}{2.572461in}%
\pgfsys@useobject{currentmarker}{}%
\end{pgfscope}%
\begin{pgfscope}%
\pgfsys@transformshift{4.093006in}{2.404984in}%
\pgfsys@useobject{currentmarker}{}%
\end{pgfscope}%
\begin{pgfscope}%
\pgfsys@transformshift{4.067656in}{2.289178in}%
\pgfsys@useobject{currentmarker}{}%
\end{pgfscope}%
\begin{pgfscope}%
\pgfsys@transformshift{4.051460in}{2.260366in}%
\pgfsys@useobject{currentmarker}{}%
\end{pgfscope}%
\begin{pgfscope}%
\pgfsys@transformshift{4.033384in}{2.248763in}%
\pgfsys@useobject{currentmarker}{}%
\end{pgfscope}%
\begin{pgfscope}%
\pgfsys@transformshift{4.013199in}{2.268770in}%
\pgfsys@useobject{currentmarker}{}%
\end{pgfscope}%
\begin{pgfscope}%
\pgfsys@transformshift{3.994185in}{2.316258in}%
\pgfsys@useobject{currentmarker}{}%
\end{pgfscope}%
\begin{pgfscope}%
\pgfsys@transformshift{3.972824in}{2.458529in}%
\pgfsys@useobject{currentmarker}{}%
\end{pgfscope}%
\begin{pgfscope}%
\pgfsys@transformshift{3.954282in}{2.589913in}%
\pgfsys@useobject{currentmarker}{}%
\end{pgfscope}%
\begin{pgfscope}%
\pgfsys@transformshift{3.938554in}{2.635389in}%
\pgfsys@useobject{currentmarker}{}%
\end{pgfscope}%
\begin{pgfscope}%
\pgfsys@transformshift{3.917195in}{2.522374in}%
\pgfsys@useobject{currentmarker}{}%
\end{pgfscope}%
\begin{pgfscope}%
\pgfsys@transformshift{3.899590in}{2.364861in}%
\pgfsys@useobject{currentmarker}{}%
\end{pgfscope}%
\begin{pgfscope}%
\pgfsys@transformshift{3.875646in}{2.286207in}%
\pgfsys@useobject{currentmarker}{}%
\end{pgfscope}%
\begin{pgfscope}%
\pgfsys@transformshift{3.860389in}{2.258515in}%
\pgfsys@useobject{currentmarker}{}%
\end{pgfscope}%
\begin{pgfscope}%
\pgfsys@transformshift{3.841142in}{2.246469in}%
\pgfsys@useobject{currentmarker}{}%
\end{pgfscope}%
\begin{pgfscope}%
\pgfsys@transformshift{3.822834in}{2.259604in}%
\pgfsys@useobject{currentmarker}{}%
\end{pgfscope}%
\begin{pgfscope}%
\pgfsys@transformshift{3.801941in}{2.285818in}%
\pgfsys@useobject{currentmarker}{}%
\end{pgfscope}%
\begin{pgfscope}%
\pgfsys@transformshift{3.779173in}{2.373505in}%
\pgfsys@useobject{currentmarker}{}%
\end{pgfscope}%
\begin{pgfscope}%
\pgfsys@transformshift{3.764854in}{2.487985in}%
\pgfsys@useobject{currentmarker}{}%
\end{pgfscope}%
\begin{pgfscope}%
\pgfsys@transformshift{3.741147in}{2.613604in}%
\pgfsys@useobject{currentmarker}{}%
\end{pgfscope}%
\begin{pgfscope}%
\pgfsys@transformshift{3.721665in}{2.608007in}%
\pgfsys@useobject{currentmarker}{}%
\end{pgfscope}%
\begin{pgfscope}%
\pgfsys@transformshift{3.707580in}{2.512784in}%
\pgfsys@useobject{currentmarker}{}%
\end{pgfscope}%
\begin{pgfscope}%
\pgfsys@transformshift{3.684578in}{2.355530in}%
\pgfsys@useobject{currentmarker}{}%
\end{pgfscope}%
\begin{pgfscope}%
\pgfsys@transformshift{3.666033in}{2.298407in}%
\pgfsys@useobject{currentmarker}{}%
\end{pgfscope}%
\begin{pgfscope}%
\pgfsys@transformshift{3.647489in}{2.600102in}%
\pgfsys@useobject{currentmarker}{}%
\end{pgfscope}%
\begin{pgfscope}%
\pgfsys@transformshift{3.628478in}{2.613217in}%
\pgfsys@useobject{currentmarker}{}%
\end{pgfscope}%
\begin{pgfscope}%
\pgfsys@transformshift{3.611107in}{2.486809in}%
\pgfsys@useobject{currentmarker}{}%
\end{pgfscope}%
\begin{pgfscope}%
\pgfsys@transformshift{3.590685in}{2.347440in}%
\pgfsys@useobject{currentmarker}{}%
\end{pgfscope}%
\begin{pgfscope}%
\pgfsys@transformshift{3.571907in}{2.282626in}%
\pgfsys@useobject{currentmarker}{}%
\end{pgfscope}%
\begin{pgfscope}%
\pgfsys@transformshift{3.549608in}{2.253053in}%
\pgfsys@useobject{currentmarker}{}%
\end{pgfscope}%
\begin{pgfscope}%
\pgfsys@transformshift{3.532237in}{2.246497in}%
\pgfsys@useobject{currentmarker}{}%
\end{pgfscope}%
\begin{pgfscope}%
\pgfsys@transformshift{3.513929in}{2.260733in}%
\pgfsys@useobject{currentmarker}{}%
\end{pgfscope}%
\begin{pgfscope}%
\pgfsys@transformshift{3.495150in}{2.298542in}%
\pgfsys@useobject{currentmarker}{}%
\end{pgfscope}%
\begin{pgfscope}%
\pgfsys@transformshift{3.473555in}{2.412116in}%
\pgfsys@useobject{currentmarker}{}%
\end{pgfscope}%
\begin{pgfscope}%
\pgfsys@transformshift{3.454778in}{2.561612in}%
\pgfsys@useobject{currentmarker}{}%
\end{pgfscope}%
\begin{pgfscope}%
\pgfsys@transformshift{3.437642in}{2.620930in}%
\pgfsys@useobject{currentmarker}{}%
\end{pgfscope}%
\begin{pgfscope}%
\pgfsys@transformshift{3.417454in}{2.556371in}%
\pgfsys@useobject{currentmarker}{}%
\end{pgfscope}%
\begin{pgfscope}%
\pgfsys@transformshift{3.400086in}{2.420925in}%
\pgfsys@useobject{currentmarker}{}%
\end{pgfscope}%
\begin{pgfscope}%
\pgfsys@transformshift{3.374265in}{2.319511in}%
\pgfsys@useobject{currentmarker}{}%
\end{pgfscope}%
\begin{pgfscope}%
\pgfsys@transformshift{3.359946in}{2.280503in}%
\pgfsys@useobject{currentmarker}{}%
\end{pgfscope}%
\begin{pgfscope}%
\pgfsys@transformshift{3.338352in}{2.251553in}%
\pgfsys@useobject{currentmarker}{}%
\end{pgfscope}%
\begin{pgfscope}%
\pgfsys@transformshift{3.320042in}{2.245758in}%
\pgfsys@useobject{currentmarker}{}%
\end{pgfscope}%
\begin{pgfscope}%
\pgfsys@transformshift{3.300794in}{2.260791in}%
\pgfsys@useobject{currentmarker}{}%
\end{pgfscope}%
\begin{pgfscope}%
\pgfsys@transformshift{3.283424in}{2.285569in}%
\pgfsys@useobject{currentmarker}{}%
\end{pgfscope}%
\begin{pgfscope}%
\pgfsys@transformshift{3.263239in}{2.348453in}%
\pgfsys@useobject{currentmarker}{}%
\end{pgfscope}%
\begin{pgfscope}%
\pgfsys@transformshift{3.246337in}{2.439810in}%
\pgfsys@useobject{currentmarker}{}%
\end{pgfscope}%
\begin{pgfscope}%
\pgfsys@transformshift{3.225212in}{2.596187in}%
\pgfsys@useobject{currentmarker}{}%
\end{pgfscope}%
\begin{pgfscope}%
\pgfsys@transformshift{3.208076in}{2.612847in}%
\pgfsys@useobject{currentmarker}{}%
\end{pgfscope}%
\begin{pgfscope}%
\pgfsys@transformshift{3.185308in}{2.488108in}%
\pgfsys@useobject{currentmarker}{}%
\end{pgfscope}%
\begin{pgfscope}%
\pgfsys@transformshift{3.166295in}{2.363737in}%
\pgfsys@useobject{currentmarker}{}%
\end{pgfscope}%
\begin{pgfscope}%
\pgfsys@transformshift{3.148221in}{2.291663in}%
\pgfsys@useobject{currentmarker}{}%
\end{pgfscope}%
\begin{pgfscope}%
\pgfsys@transformshift{3.129911in}{2.260517in}%
\pgfsys@useobject{currentmarker}{}%
\end{pgfscope}%
\begin{pgfscope}%
\pgfsys@transformshift{3.107612in}{2.244962in}%
\pgfsys@useobject{currentmarker}{}%
\end{pgfscope}%
\begin{pgfscope}%
\pgfsys@transformshift{3.092824in}{2.250445in}%
\pgfsys@useobject{currentmarker}{}%
\end{pgfscope}%
\begin{pgfscope}%
\pgfsys@transformshift{3.070056in}{2.275127in}%
\pgfsys@useobject{currentmarker}{}%
\end{pgfscope}%
\begin{pgfscope}%
\pgfsys@transformshift{3.051278in}{2.320703in}%
\pgfsys@useobject{currentmarker}{}%
\end{pgfscope}%
\begin{pgfscope}%
\pgfsys@transformshift{3.033204in}{2.429667in}%
\pgfsys@useobject{currentmarker}{}%
\end{pgfscope}%
\begin{pgfscope}%
\pgfsys@transformshift{3.014425in}{2.514290in}%
\pgfsys@useobject{currentmarker}{}%
\end{pgfscope}%
\begin{pgfscope}%
\pgfsys@transformshift{2.996115in}{2.588108in}%
\pgfsys@useobject{currentmarker}{}%
\end{pgfscope}%
\begin{pgfscope}%
\pgfsys@transformshift{2.978747in}{2.605516in}%
\pgfsys@useobject{currentmarker}{}%
\end{pgfscope}%
\begin{pgfscope}%
\pgfsys@transformshift{2.955508in}{2.479579in}%
\pgfsys@useobject{currentmarker}{}%
\end{pgfscope}%
\begin{pgfscope}%
\pgfsys@transformshift{2.938372in}{2.361871in}%
\pgfsys@useobject{currentmarker}{}%
\end{pgfscope}%
\begin{pgfscope}%
\pgfsys@transformshift{2.915604in}{2.288373in}%
\pgfsys@useobject{currentmarker}{}%
\end{pgfscope}%
\begin{pgfscope}%
\pgfsys@transformshift{2.899877in}{2.259639in}%
\pgfsys@useobject{currentmarker}{}%
\end{pgfscope}%
\begin{pgfscope}%
\pgfsys@transformshift{2.881332in}{2.244504in}%
\pgfsys@useobject{currentmarker}{}%
\end{pgfscope}%
\begin{pgfscope}%
\pgfsys@transformshift{2.859268in}{2.246581in}%
\pgfsys@useobject{currentmarker}{}%
\end{pgfscope}%
\begin{pgfscope}%
\pgfsys@transformshift{2.841194in}{2.263128in}%
\pgfsys@useobject{currentmarker}{}%
\end{pgfscope}%
\begin{pgfscope}%
\pgfsys@transformshift{2.822415in}{2.302109in}%
\pgfsys@useobject{currentmarker}{}%
\end{pgfscope}%
\begin{pgfscope}%
\pgfsys@transformshift{2.800351in}{2.406150in}%
\pgfsys@useobject{currentmarker}{}%
\end{pgfscope}%
\begin{pgfscope}%
\pgfsys@transformshift{2.785328in}{2.517035in}%
\pgfsys@useobject{currentmarker}{}%
\end{pgfscope}%
\begin{pgfscope}%
\pgfsys@transformshift{2.759744in}{2.611604in}%
\pgfsys@useobject{currentmarker}{}%
\end{pgfscope}%
\begin{pgfscope}%
\pgfsys@transformshift{2.747773in}{2.592183in}%
\pgfsys@useobject{currentmarker}{}%
\end{pgfscope}%
\begin{pgfscope}%
\pgfsys@transformshift{2.726177in}{2.474855in}%
\pgfsys@useobject{currentmarker}{}%
\end{pgfscope}%
\begin{pgfscope}%
\pgfsys@transformshift{2.706226in}{2.353794in}%
\pgfsys@useobject{currentmarker}{}%
\end{pgfscope}%
\begin{pgfscope}%
\pgfsys@transformshift{2.686273in}{2.288594in}%
\pgfsys@useobject{currentmarker}{}%
\end{pgfscope}%
\begin{pgfscope}%
\pgfsys@transformshift{2.668434in}{2.257185in}%
\pgfsys@useobject{currentmarker}{}%
\end{pgfscope}%
\begin{pgfscope}%
\pgfsys@transformshift{2.645900in}{2.244816in}%
\pgfsys@useobject{currentmarker}{}%
\end{pgfscope}%
\begin{pgfscope}%
\pgfsys@transformshift{2.626887in}{2.244798in}%
\pgfsys@useobject{currentmarker}{}%
\end{pgfscope}%
\begin{pgfscope}%
\pgfsys@transformshift{2.609282in}{2.258264in}%
\pgfsys@useobject{currentmarker}{}%
\end{pgfscope}%
\begin{pgfscope}%
\pgfsys@transformshift{2.590738in}{2.282165in}%
\pgfsys@useobject{currentmarker}{}%
\end{pgfscope}%
\begin{pgfscope}%
\pgfsys@transformshift{2.571725in}{2.344655in}%
\pgfsys@useobject{currentmarker}{}%
\end{pgfscope}%
\begin{pgfscope}%
\pgfsys@transformshift{2.553651in}{2.467934in}%
\pgfsys@useobject{currentmarker}{}%
\end{pgfscope}%
\begin{pgfscope}%
\pgfsys@transformshift{2.533934in}{2.472590in}%
\pgfsys@useobject{currentmarker}{}%
\end{pgfscope}%
\begin{pgfscope}%
\pgfsys@transformshift{2.516095in}{2.592637in}%
\pgfsys@useobject{currentmarker}{}%
\end{pgfscope}%
\begin{pgfscope}%
\pgfsys@transformshift{2.494734in}{2.599161in}%
\pgfsys@useobject{currentmarker}{}%
\end{pgfscope}%
\begin{pgfscope}%
\pgfsys@transformshift{2.475486in}{2.505657in}%
\pgfsys@useobject{currentmarker}{}%
\end{pgfscope}%
\begin{pgfscope}%
\pgfsys@transformshift{2.459055in}{2.386450in}%
\pgfsys@useobject{currentmarker}{}%
\end{pgfscope}%
\begin{pgfscope}%
\pgfsys@transformshift{2.437460in}{2.299360in}%
\pgfsys@useobject{currentmarker}{}%
\end{pgfscope}%
\begin{pgfscope}%
\pgfsys@transformshift{2.418212in}{2.272473in}%
\pgfsys@useobject{currentmarker}{}%
\end{pgfscope}%
\begin{pgfscope}%
\pgfsys@transformshift{2.399433in}{2.250295in}%
\pgfsys@useobject{currentmarker}{}%
\end{pgfscope}%
\begin{pgfscope}%
\pgfsys@transformshift{2.379717in}{2.244154in}%
\pgfsys@useobject{currentmarker}{}%
\end{pgfscope}%
\begin{pgfscope}%
\pgfsys@transformshift{2.360235in}{2.259623in}%
\pgfsys@useobject{currentmarker}{}%
\end{pgfscope}%
\begin{pgfscope}%
\pgfsys@transformshift{2.339344in}{2.294629in}%
\pgfsys@useobject{currentmarker}{}%
\end{pgfscope}%
\begin{pgfscope}%
\pgfsys@transformshift{2.321034in}{2.371776in}%
\pgfsys@useobject{currentmarker}{}%
\end{pgfscope}%
\begin{pgfscope}%
\pgfsys@transformshift{2.303664in}{2.491930in}%
\pgfsys@useobject{currentmarker}{}%
\end{pgfscope}%
\begin{pgfscope}%
\pgfsys@transformshift{2.285824in}{2.601702in}%
\pgfsys@useobject{currentmarker}{}%
\end{pgfscope}%
\begin{pgfscope}%
\pgfsys@transformshift{2.265873in}{2.609908in}%
\pgfsys@useobject{currentmarker}{}%
\end{pgfscope}%
\begin{pgfscope}%
\pgfsys@transformshift{2.244983in}{2.515419in}%
\pgfsys@useobject{currentmarker}{}%
\end{pgfscope}%
\begin{pgfscope}%
\pgfsys@transformshift{2.222448in}{2.366137in}%
\pgfsys@useobject{currentmarker}{}%
\end{pgfscope}%
\begin{pgfscope}%
\pgfsys@transformshift{2.200617in}{2.315039in}%
\pgfsys@useobject{currentmarker}{}%
\end{pgfscope}%
\begin{pgfscope}%
\pgfsys@transformshift{2.189820in}{2.284138in}%
\pgfsys@useobject{currentmarker}{}%
\end{pgfscope}%
\begin{pgfscope}%
\pgfsys@transformshift{2.166347in}{2.257778in}%
\pgfsys@useobject{currentmarker}{}%
\end{pgfscope}%
\begin{pgfscope}%
\pgfsys@transformshift{2.148039in}{2.246182in}%
\pgfsys@useobject{currentmarker}{}%
\end{pgfscope}%
\begin{pgfscope}%
\pgfsys@transformshift{2.128086in}{2.254727in}%
\pgfsys@useobject{currentmarker}{}%
\end{pgfscope}%
\begin{pgfscope}%
\pgfsys@transformshift{2.112125in}{2.269533in}%
\pgfsys@useobject{currentmarker}{}%
\end{pgfscope}%
\begin{pgfscope}%
\pgfsys@transformshift{2.090765in}{2.315278in}%
\pgfsys@useobject{currentmarker}{}%
\end{pgfscope}%
\begin{pgfscope}%
\pgfsys@transformshift{2.070578in}{2.382681in}%
\pgfsys@useobject{currentmarker}{}%
\end{pgfscope}%
\begin{pgfscope}%
\pgfsys@transformshift{2.052270in}{2.512280in}%
\pgfsys@useobject{currentmarker}{}%
\end{pgfscope}%
\begin{pgfscope}%
\pgfsys@transformshift{2.033725in}{2.594420in}%
\pgfsys@useobject{currentmarker}{}%
\end{pgfscope}%
\begin{pgfscope}%
\pgfsys@transformshift{2.015183in}{2.613439in}%
\pgfsys@useobject{currentmarker}{}%
\end{pgfscope}%
\begin{pgfscope}%
\pgfsys@transformshift{1.997342in}{2.586998in}%
\pgfsys@useobject{currentmarker}{}%
\end{pgfscope}%
\begin{pgfscope}%
\pgfsys@transformshift{1.972931in}{2.414690in}%
\pgfsys@useobject{currentmarker}{}%
\end{pgfscope}%
\begin{pgfscope}%
\pgfsys@transformshift{1.956500in}{2.383880in}%
\pgfsys@useobject{currentmarker}{}%
\end{pgfscope}%
\begin{pgfscope}%
\pgfsys@transformshift{1.937487in}{2.314339in}%
\pgfsys@useobject{currentmarker}{}%
\end{pgfscope}%
\begin{pgfscope}%
\pgfsys@transformshift{1.919177in}{2.274372in}%
\pgfsys@useobject{currentmarker}{}%
\end{pgfscope}%
\begin{pgfscope}%
\pgfsys@transformshift{1.901338in}{2.251771in}%
\pgfsys@useobject{currentmarker}{}%
\end{pgfscope}%
\begin{pgfscope}%
\pgfsys@transformshift{1.878804in}{2.246558in}%
\pgfsys@useobject{currentmarker}{}%
\end{pgfscope}%
\begin{pgfscope}%
\pgfsys@transformshift{1.860731in}{2.258009in}%
\pgfsys@useobject{currentmarker}{}%
\end{pgfscope}%
\begin{pgfscope}%
\pgfsys@transformshift{1.845238in}{2.281127in}%
\pgfsys@useobject{currentmarker}{}%
\end{pgfscope}%
\begin{pgfscope}%
\pgfsys@transformshift{1.824113in}{2.334434in}%
\pgfsys@useobject{currentmarker}{}%
\end{pgfscope}%
\begin{pgfscope}%
\pgfsys@transformshift{1.802282in}{2.463088in}%
\pgfsys@useobject{currentmarker}{}%
\end{pgfscope}%
\begin{pgfscope}%
\pgfsys@transformshift{1.783738in}{2.553618in}%
\pgfsys@useobject{currentmarker}{}%
\end{pgfscope}%
\begin{pgfscope}%
\pgfsys@transformshift{1.765664in}{2.615317in}%
\pgfsys@useobject{currentmarker}{}%
\end{pgfscope}%
\begin{pgfscope}%
\pgfsys@transformshift{1.747122in}{2.603323in}%
\pgfsys@useobject{currentmarker}{}%
\end{pgfscope}%
\begin{pgfscope}%
\pgfsys@transformshift{1.727169in}{2.512222in}%
\pgfsys@useobject{currentmarker}{}%
\end{pgfscope}%
\begin{pgfscope}%
\pgfsys@transformshift{1.707452in}{2.383667in}%
\pgfsys@useobject{currentmarker}{}%
\end{pgfscope}%
\begin{pgfscope}%
\pgfsys@transformshift{1.688908in}{2.317130in}%
\pgfsys@useobject{currentmarker}{}%
\end{pgfscope}%
\begin{pgfscope}%
\pgfsys@transformshift{1.666843in}{2.287344in}%
\pgfsys@useobject{currentmarker}{}%
\end{pgfscope}%
\begin{pgfscope}%
\pgfsys@transformshift{1.652056in}{2.264380in}%
\pgfsys@useobject{currentmarker}{}%
\end{pgfscope}%
\begin{pgfscope}%
\pgfsys@transformshift{1.629757in}{2.247773in}%
\pgfsys@useobject{currentmarker}{}%
\end{pgfscope}%
\begin{pgfscope}%
\pgfsys@transformshift{1.608395in}{2.251826in}%
\pgfsys@useobject{currentmarker}{}%
\end{pgfscope}%
\begin{pgfscope}%
\pgfsys@transformshift{1.590322in}{2.262042in}%
\pgfsys@useobject{currentmarker}{}%
\end{pgfscope}%
\begin{pgfscope}%
\pgfsys@transformshift{1.571779in}{2.290101in}%
\pgfsys@useobject{currentmarker}{}%
\end{pgfscope}%
\begin{pgfscope}%
\pgfsys@transformshift{1.553703in}{2.346216in}%
\pgfsys@useobject{currentmarker}{}%
\end{pgfscope}%
\begin{pgfscope}%
\pgfsys@transformshift{1.534690in}{2.438882in}%
\pgfsys@useobject{currentmarker}{}%
\end{pgfscope}%
\begin{pgfscope}%
\pgfsys@transformshift{1.515208in}{2.570917in}%
\pgfsys@useobject{currentmarker}{}%
\end{pgfscope}%
\begin{pgfscope}%
\pgfsys@transformshift{1.497603in}{2.612238in}%
\pgfsys@useobject{currentmarker}{}%
\end{pgfscope}%
\begin{pgfscope}%
\pgfsys@transformshift{1.477652in}{2.627376in}%
\pgfsys@useobject{currentmarker}{}%
\end{pgfscope}%
\begin{pgfscope}%
\pgfsys@transformshift{1.458168in}{2.588473in}%
\pgfsys@useobject{currentmarker}{}%
\end{pgfscope}%
\begin{pgfscope}%
\pgfsys@transformshift{1.438686in}{2.460818in}%
\pgfsys@useobject{currentmarker}{}%
\end{pgfscope}%
\begin{pgfscope}%
\pgfsys@transformshift{1.418030in}{2.369657in}%
\pgfsys@useobject{currentmarker}{}%
\end{pgfscope}%
\begin{pgfscope}%
\pgfsys@transformshift{1.399956in}{2.308456in}%
\pgfsys@useobject{currentmarker}{}%
\end{pgfscope}%
\begin{pgfscope}%
\pgfsys@transformshift{1.382586in}{2.288892in}%
\pgfsys@useobject{currentmarker}{}%
\end{pgfscope}%
\begin{pgfscope}%
\pgfsys@transformshift{1.359582in}{2.262542in}%
\pgfsys@useobject{currentmarker}{}%
\end{pgfscope}%
\begin{pgfscope}%
\pgfsys@transformshift{1.341979in}{2.249005in}%
\pgfsys@useobject{currentmarker}{}%
\end{pgfscope}%
\begin{pgfscope}%
\pgfsys@transformshift{1.324843in}{2.249875in}%
\pgfsys@useobject{currentmarker}{}%
\end{pgfscope}%
\begin{pgfscope}%
\pgfsys@transformshift{1.304421in}{2.252668in}%
\pgfsys@useobject{currentmarker}{}%
\end{pgfscope}%
\begin{pgfscope}%
\pgfsys@transformshift{1.281653in}{2.269660in}%
\pgfsys@useobject{currentmarker}{}%
\end{pgfscope}%
\begin{pgfscope}%
\pgfsys@transformshift{1.266395in}{2.297719in}%
\pgfsys@useobject{currentmarker}{}%
\end{pgfscope}%
\begin{pgfscope}%
\pgfsys@transformshift{1.246678in}{2.371108in}%
\pgfsys@useobject{currentmarker}{}%
\end{pgfscope}%
\begin{pgfscope}%
\pgfsys@transformshift{1.225082in}{2.507303in}%
\pgfsys@useobject{currentmarker}{}%
\end{pgfscope}%
\begin{pgfscope}%
\pgfsys@transformshift{1.208417in}{2.568564in}%
\pgfsys@useobject{currentmarker}{}%
\end{pgfscope}%
\begin{pgfscope}%
\pgfsys@transformshift{1.187996in}{2.636624in}%
\pgfsys@useobject{currentmarker}{}%
\end{pgfscope}%
\begin{pgfscope}%
\pgfsys@transformshift{1.164054in}{2.614005in}%
\pgfsys@useobject{currentmarker}{}%
\end{pgfscope}%
\begin{pgfscope}%
\pgfsys@transformshift{1.149266in}{2.547645in}%
\pgfsys@useobject{currentmarker}{}%
\end{pgfscope}%
\begin{pgfscope}%
\pgfsys@transformshift{1.131661in}{2.436258in}%
\pgfsys@useobject{currentmarker}{}%
\end{pgfscope}%
\begin{pgfscope}%
\pgfsys@transformshift{1.111474in}{2.336862in}%
\pgfsys@useobject{currentmarker}{}%
\end{pgfscope}%
\begin{pgfscope}%
\pgfsys@transformshift{1.090818in}{2.291537in}%
\pgfsys@useobject{currentmarker}{}%
\end{pgfscope}%
\begin{pgfscope}%
\pgfsys@transformshift{1.073918in}{2.276585in}%
\pgfsys@useobject{currentmarker}{}%
\end{pgfscope}%
\begin{pgfscope}%
\pgfsys@transformshift{1.051148in}{2.255751in}%
\pgfsys@useobject{currentmarker}{}%
\end{pgfscope}%
\begin{pgfscope}%
\pgfsys@transformshift{1.030492in}{2.250078in}%
\pgfsys@useobject{currentmarker}{}%
\end{pgfscope}%
\begin{pgfscope}%
\pgfsys@transformshift{1.010070in}{2.261804in}%
\pgfsys@useobject{currentmarker}{}%
\end{pgfscope}%
\begin{pgfscope}%
\pgfsys@transformshift{0.995048in}{2.278689in}%
\pgfsys@useobject{currentmarker}{}%
\end{pgfscope}%
\begin{pgfscope}%
\pgfsys@transformshift{0.976504in}{2.307390in}%
\pgfsys@useobject{currentmarker}{}%
\end{pgfscope}%
\begin{pgfscope}%
\pgfsys@transformshift{0.955379in}{2.361239in}%
\pgfsys@useobject{currentmarker}{}%
\end{pgfscope}%
\begin{pgfscope}%
\pgfsys@transformshift{0.938713in}{2.446478in}%
\pgfsys@useobject{currentmarker}{}%
\end{pgfscope}%
\begin{pgfscope}%
\pgfsys@transformshift{0.921109in}{2.565962in}%
\pgfsys@useobject{currentmarker}{}%
\end{pgfscope}%
\begin{pgfscope}%
\pgfsys@transformshift{0.903035in}{2.634657in}%
\pgfsys@useobject{currentmarker}{}%
\end{pgfscope}%
\begin{pgfscope}%
\pgfsys@transformshift{0.882379in}{2.650118in}%
\pgfsys@useobject{currentmarker}{}%
\end{pgfscope}%
\begin{pgfscope}%
\pgfsys@transformshift{0.861252in}{2.574592in}%
\pgfsys@useobject{currentmarker}{}%
\end{pgfscope}%
\begin{pgfscope}%
\pgfsys@transformshift{0.843178in}{2.461698in}%
\pgfsys@useobject{currentmarker}{}%
\end{pgfscope}%
\begin{pgfscope}%
\pgfsys@transformshift{0.823225in}{2.358147in}%
\pgfsys@useobject{currentmarker}{}%
\end{pgfscope}%
\begin{pgfscope}%
\pgfsys@transformshift{0.803509in}{2.300419in}%
\pgfsys@useobject{currentmarker}{}%
\end{pgfscope}%
\begin{pgfscope}%
\pgfsys@transformshift{0.784261in}{2.277393in}%
\pgfsys@useobject{currentmarker}{}%
\end{pgfscope}%
\begin{pgfscope}%
\pgfsys@transformshift{0.766188in}{2.261530in}%
\pgfsys@useobject{currentmarker}{}%
\end{pgfscope}%
\begin{pgfscope}%
\pgfsys@transformshift{0.748112in}{2.252118in}%
\pgfsys@useobject{currentmarker}{}%
\end{pgfscope}%
\begin{pgfscope}%
\pgfsys@transformshift{0.725813in}{2.262203in}%
\pgfsys@useobject{currentmarker}{}%
\end{pgfscope}%
\begin{pgfscope}%
\pgfsys@transformshift{0.707739in}{2.282794in}%
\pgfsys@useobject{currentmarker}{}%
\end{pgfscope}%
\begin{pgfscope}%
\pgfsys@transformshift{0.690840in}{2.324179in}%
\pgfsys@useobject{currentmarker}{}%
\end{pgfscope}%
\begin{pgfscope}%
\pgfsys@transformshift{0.669713in}{2.389037in}%
\pgfsys@useobject{currentmarker}{}%
\end{pgfscope}%
\begin{pgfscope}%
\pgfsys@transformshift{0.652108in}{2.472879in}%
\pgfsys@useobject{currentmarker}{}%
\end{pgfscope}%
\begin{pgfscope}%
\pgfsys@transformshift{0.651405in}{2.468013in}%
\pgfsys@useobject{currentmarker}{}%
\end{pgfscope}%
\begin{pgfscope}%
\pgfsys@transformshift{0.654456in}{2.428709in}%
\pgfsys@useobject{currentmarker}{}%
\end{pgfscope}%
\begin{pgfscope}%
\pgfsys@transformshift{0.676755in}{2.305156in}%
\pgfsys@useobject{currentmarker}{}%
\end{pgfscope}%
\begin{pgfscope}%
\pgfsys@transformshift{0.695768in}{2.262265in}%
\pgfsys@useobject{currentmarker}{}%
\end{pgfscope}%
\begin{pgfscope}%
\pgfsys@transformshift{0.714311in}{2.252900in}%
\pgfsys@useobject{currentmarker}{}%
\end{pgfscope}%
\begin{pgfscope}%
\pgfsys@transformshift{0.733795in}{2.277825in}%
\pgfsys@useobject{currentmarker}{}%
\end{pgfscope}%
\begin{pgfscope}%
\pgfsys@transformshift{0.750929in}{2.334352in}%
\pgfsys@useobject{currentmarker}{}%
\end{pgfscope}%
\begin{pgfscope}%
\pgfsys@transformshift{0.770647in}{2.455148in}%
\pgfsys@useobject{currentmarker}{}%
\end{pgfscope}%
\begin{pgfscope}%
\pgfsys@transformshift{0.795293in}{2.643358in}%
\pgfsys@useobject{currentmarker}{}%
\end{pgfscope}%
\begin{pgfscope}%
\pgfsys@transformshift{0.811020in}{2.641951in}%
\pgfsys@useobject{currentmarker}{}%
\end{pgfscope}%
\begin{pgfscope}%
\pgfsys@transformshift{0.811489in}{2.585011in}%
\pgfsys@useobject{currentmarker}{}%
\end{pgfscope}%
\begin{pgfscope}%
\pgfsys@transformshift{0.830973in}{2.521127in}%
\pgfsys@useobject{currentmarker}{}%
\end{pgfscope}%
\begin{pgfscope}%
\pgfsys@transformshift{0.848107in}{2.368202in}%
\pgfsys@useobject{currentmarker}{}%
\end{pgfscope}%
\begin{pgfscope}%
\pgfsys@transformshift{0.866886in}{2.288057in}%
\pgfsys@useobject{currentmarker}{}%
\end{pgfscope}%
\begin{pgfscope}%
\pgfsys@transformshift{0.888247in}{2.252342in}%
\pgfsys@useobject{currentmarker}{}%
\end{pgfscope}%
\begin{pgfscope}%
\pgfsys@transformshift{0.906790in}{2.255711in}%
\pgfsys@useobject{currentmarker}{}%
\end{pgfscope}%
\begin{pgfscope}%
\pgfsys@transformshift{0.925100in}{2.284735in}%
\pgfsys@useobject{currentmarker}{}%
\end{pgfscope}%
\begin{pgfscope}%
\pgfsys@transformshift{0.944347in}{2.351826in}%
\pgfsys@useobject{currentmarker}{}%
\end{pgfscope}%
\begin{pgfscope}%
\pgfsys@transformshift{0.965238in}{2.509526in}%
\pgfsys@useobject{currentmarker}{}%
\end{pgfscope}%
\begin{pgfscope}%
\pgfsys@transformshift{0.983311in}{2.632092in}%
\pgfsys@useobject{currentmarker}{}%
\end{pgfscope}%
\begin{pgfscope}%
\pgfsys@transformshift{1.001856in}{2.617859in}%
\pgfsys@useobject{currentmarker}{}%
\end{pgfscope}%
\begin{pgfscope}%
\pgfsys@transformshift{1.022512in}{2.459574in}%
\pgfsys@useobject{currentmarker}{}%
\end{pgfscope}%
\begin{pgfscope}%
\pgfsys@transformshift{1.040820in}{2.335600in}%
\pgfsys@useobject{currentmarker}{}%
\end{pgfscope}%
\begin{pgfscope}%
\pgfsys@transformshift{1.058425in}{2.275933in}%
\pgfsys@useobject{currentmarker}{}%
\end{pgfscope}%
\begin{pgfscope}%
\pgfsys@transformshift{1.079315in}{2.249007in}%
\pgfsys@useobject{currentmarker}{}%
\end{pgfscope}%
\begin{pgfscope}%
\pgfsys@transformshift{1.098094in}{2.255922in}%
\pgfsys@useobject{currentmarker}{}%
\end{pgfscope}%
\begin{pgfscope}%
\pgfsys@transformshift{1.115699in}{2.284170in}%
\pgfsys@useobject{currentmarker}{}%
\end{pgfscope}%
\begin{pgfscope}%
\pgfsys@transformshift{1.134947in}{2.361537in}%
\pgfsys@useobject{currentmarker}{}%
\end{pgfscope}%
\begin{pgfscope}%
\pgfsys@transformshift{1.157246in}{2.535916in}%
\pgfsys@useobject{currentmarker}{}%
\end{pgfscope}%
\begin{pgfscope}%
\pgfsys@transformshift{1.175085in}{2.630122in}%
\pgfsys@useobject{currentmarker}{}%
\end{pgfscope}%
\begin{pgfscope}%
\pgfsys@transformshift{1.192924in}{2.596905in}%
\pgfsys@useobject{currentmarker}{}%
\end{pgfscope}%
\begin{pgfscope}%
\pgfsys@transformshift{1.213346in}{2.427961in}%
\pgfsys@useobject{currentmarker}{}%
\end{pgfscope}%
\begin{pgfscope}%
\pgfsys@transformshift{1.233768in}{2.314425in}%
\pgfsys@useobject{currentmarker}{}%
\end{pgfscope}%
\begin{pgfscope}%
\pgfsys@transformshift{1.250198in}{2.266269in}%
\pgfsys@useobject{currentmarker}{}%
\end{pgfscope}%
\begin{pgfscope}%
\pgfsys@transformshift{1.271560in}{2.246632in}%
\pgfsys@useobject{currentmarker}{}%
\end{pgfscope}%
\begin{pgfscope}%
\pgfsys@transformshift{1.292450in}{2.258780in}%
\pgfsys@useobject{currentmarker}{}%
\end{pgfscope}%
\begin{pgfscope}%
\pgfsys@transformshift{1.311932in}{2.297130in}%
\pgfsys@useobject{currentmarker}{}%
\end{pgfscope}%
\begin{pgfscope}%
\pgfsys@transformshift{1.327894in}{2.357934in}%
\pgfsys@useobject{currentmarker}{}%
\end{pgfscope}%
\begin{pgfscope}%
\pgfsys@transformshift{1.353714in}{2.554429in}%
\pgfsys@useobject{currentmarker}{}%
\end{pgfscope}%
\begin{pgfscope}%
\pgfsys@transformshift{1.365921in}{2.611393in}%
\pgfsys@useobject{currentmarker}{}%
\end{pgfscope}%
\begin{pgfscope}%
\pgfsys@transformshift{1.388454in}{2.595833in}%
\pgfsys@useobject{currentmarker}{}%
\end{pgfscope}%
\begin{pgfscope}%
\pgfsys@transformshift{1.404651in}{2.459751in}%
\pgfsys@useobject{currentmarker}{}%
\end{pgfscope}%
\begin{pgfscope}%
\pgfsys@transformshift{1.426012in}{2.321405in}%
\pgfsys@useobject{currentmarker}{}%
\end{pgfscope}%
\begin{pgfscope}%
\pgfsys@transformshift{1.445260in}{2.267597in}%
\pgfsys@useobject{currentmarker}{}%
\end{pgfscope}%
\begin{pgfscope}%
\pgfsys@transformshift{1.462159in}{2.249659in}%
\pgfsys@useobject{currentmarker}{}%
\end{pgfscope}%
\begin{pgfscope}%
\pgfsys@transformshift{1.482347in}{2.249379in}%
\pgfsys@useobject{currentmarker}{}%
\end{pgfscope}%
\begin{pgfscope}%
\pgfsys@transformshift{1.501125in}{2.270585in}%
\pgfsys@useobject{currentmarker}{}%
\end{pgfscope}%
\begin{pgfscope}%
\pgfsys@transformshift{1.519199in}{2.315004in}%
\pgfsys@useobject{currentmarker}{}%
\end{pgfscope}%
\begin{pgfscope}%
\pgfsys@transformshift{1.540090in}{2.426696in}%
\pgfsys@useobject{currentmarker}{}%
\end{pgfscope}%
\begin{pgfscope}%
\pgfsys@transformshift{1.557929in}{2.570985in}%
\pgfsys@useobject{currentmarker}{}%
\end{pgfscope}%
\begin{pgfscope}%
\pgfsys@transformshift{1.580462in}{2.615202in}%
\pgfsys@useobject{currentmarker}{}%
\end{pgfscope}%
\begin{pgfscope}%
\pgfsys@transformshift{1.598538in}{2.522032in}%
\pgfsys@useobject{currentmarker}{}%
\end{pgfscope}%
\begin{pgfscope}%
\pgfsys@transformshift{1.619428in}{2.394885in}%
\pgfsys@useobject{currentmarker}{}%
\end{pgfscope}%
\begin{pgfscope}%
\pgfsys@transformshift{1.636564in}{2.304569in}%
\pgfsys@useobject{currentmarker}{}%
\end{pgfscope}%
\begin{pgfscope}%
\pgfsys@transformshift{1.652995in}{2.265248in}%
\pgfsys@useobject{currentmarker}{}%
\end{pgfscope}%
\begin{pgfscope}%
\pgfsys@transformshift{1.673651in}{2.246010in}%
\pgfsys@useobject{currentmarker}{}%
\end{pgfscope}%
\begin{pgfscope}%
\pgfsys@transformshift{1.694776in}{2.252797in}%
\pgfsys@useobject{currentmarker}{}%
\end{pgfscope}%
\begin{pgfscope}%
\pgfsys@transformshift{1.712850in}{2.273941in}%
\pgfsys@useobject{currentmarker}{}%
\end{pgfscope}%
\begin{pgfscope}%
\pgfsys@transformshift{1.733977in}{2.314378in}%
\pgfsys@useobject{currentmarker}{}%
\end{pgfscope}%
\begin{pgfscope}%
\pgfsys@transformshift{1.752754in}{2.403119in}%
\pgfsys@useobject{currentmarker}{}%
\end{pgfscope}%
\begin{pgfscope}%
\pgfsys@transformshift{1.769890in}{2.508644in}%
\pgfsys@useobject{currentmarker}{}%
\end{pgfscope}%
\begin{pgfscope}%
\pgfsys@transformshift{1.790780in}{2.614055in}%
\pgfsys@useobject{currentmarker}{}%
\end{pgfscope}%
\begin{pgfscope}%
\pgfsys@transformshift{1.809090in}{2.601527in}%
\pgfsys@useobject{currentmarker}{}%
\end{pgfscope}%
\begin{pgfscope}%
\pgfsys@transformshift{1.826695in}{2.493158in}%
\pgfsys@useobject{currentmarker}{}%
\end{pgfscope}%
\begin{pgfscope}%
\pgfsys@transformshift{1.847820in}{2.342848in}%
\pgfsys@useobject{currentmarker}{}%
\end{pgfscope}%
\begin{pgfscope}%
\pgfsys@transformshift{1.865894in}{2.277276in}%
\pgfsys@useobject{currentmarker}{}%
\end{pgfscope}%
\begin{pgfscope}%
\pgfsys@transformshift{1.886550in}{2.254739in}%
\pgfsys@useobject{currentmarker}{}%
\end{pgfscope}%
\begin{pgfscope}%
\pgfsys@transformshift{1.905563in}{2.245356in}%
\pgfsys@useobject{currentmarker}{}%
\end{pgfscope}%
\begin{pgfscope}%
\pgfsys@transformshift{1.925516in}{2.253193in}%
\pgfsys@useobject{currentmarker}{}%
\end{pgfscope}%
\begin{pgfscope}%
\pgfsys@transformshift{1.944529in}{2.278842in}%
\pgfsys@useobject{currentmarker}{}%
\end{pgfscope}%
\begin{pgfscope}%
\pgfsys@transformshift{1.965654in}{2.344500in}%
\pgfsys@useobject{currentmarker}{}%
\end{pgfscope}%
\begin{pgfscope}%
\pgfsys@transformshift{1.979502in}{2.447085in}%
\pgfsys@useobject{currentmarker}{}%
\end{pgfscope}%
\begin{pgfscope}%
\pgfsys@transformshift{2.001098in}{2.546144in}%
\pgfsys@useobject{currentmarker}{}%
\end{pgfscope}%
\begin{pgfscope}%
\pgfsys@transformshift{2.022223in}{2.614812in}%
\pgfsys@useobject{currentmarker}{}%
\end{pgfscope}%
\begin{pgfscope}%
\pgfsys@transformshift{2.039359in}{2.595879in}%
\pgfsys@useobject{currentmarker}{}%
\end{pgfscope}%
\begin{pgfscope}%
\pgfsys@transformshift{2.057198in}{2.469415in}%
\pgfsys@useobject{currentmarker}{}%
\end{pgfscope}%
\begin{pgfscope}%
\pgfsys@transformshift{2.079263in}{2.337738in}%
\pgfsys@useobject{currentmarker}{}%
\end{pgfscope}%
\begin{pgfscope}%
\pgfsys@transformshift{2.100154in}{2.271723in}%
\pgfsys@useobject{currentmarker}{}%
\end{pgfscope}%
\begin{pgfscope}%
\pgfsys@transformshift{2.118464in}{2.250441in}%
\pgfsys@useobject{currentmarker}{}%
\end{pgfscope}%
\begin{pgfscope}%
\pgfsys@transformshift{2.136068in}{2.244172in}%
\pgfsys@useobject{currentmarker}{}%
\end{pgfscope}%
\begin{pgfscope}%
\pgfsys@transformshift{2.154142in}{2.257835in}%
\pgfsys@useobject{currentmarker}{}%
\end{pgfscope}%
\begin{pgfscope}%
\pgfsys@transformshift{2.175501in}{2.291124in}%
\pgfsys@useobject{currentmarker}{}%
\end{pgfscope}%
\begin{pgfscope}%
\pgfsys@transformshift{2.192872in}{2.356012in}%
\pgfsys@useobject{currentmarker}{}%
\end{pgfscope}%
\begin{pgfscope}%
\pgfsys@transformshift{2.214233in}{2.498231in}%
\pgfsys@useobject{currentmarker}{}%
\end{pgfscope}%
\begin{pgfscope}%
\pgfsys@transformshift{2.231602in}{2.530280in}%
\pgfsys@useobject{currentmarker}{}%
\end{pgfscope}%
\begin{pgfscope}%
\pgfsys@transformshift{2.256249in}{2.303723in}%
\pgfsys@useobject{currentmarker}{}%
\end{pgfscope}%
\begin{pgfscope}%
\pgfsys@transformshift{2.270099in}{2.373250in}%
\pgfsys@useobject{currentmarker}{}%
\end{pgfscope}%
\begin{pgfscope}%
\pgfsys@transformshift{2.289347in}{2.457042in}%
\pgfsys@useobject{currentmarker}{}%
\end{pgfscope}%
\begin{pgfscope}%
\pgfsys@transformshift{2.309532in}{2.600520in}%
\pgfsys@useobject{currentmarker}{}%
\end{pgfscope}%
\begin{pgfscope}%
\pgfsys@transformshift{2.326433in}{2.603592in}%
\pgfsys@useobject{currentmarker}{}%
\end{pgfscope}%
\begin{pgfscope}%
\pgfsys@transformshift{2.346384in}{2.473463in}%
\pgfsys@useobject{currentmarker}{}%
\end{pgfscope}%
\begin{pgfscope}%
\pgfsys@transformshift{2.367980in}{2.329049in}%
\pgfsys@useobject{currentmarker}{}%
\end{pgfscope}%
\begin{pgfscope}%
\pgfsys@transformshift{2.385116in}{2.271839in}%
\pgfsys@useobject{currentmarker}{}%
\end{pgfscope}%
\begin{pgfscope}%
\pgfsys@transformshift{2.406007in}{2.248406in}%
\pgfsys@useobject{currentmarker}{}%
\end{pgfscope}%
\begin{pgfscope}%
\pgfsys@transformshift{2.424315in}{2.245100in}%
\pgfsys@useobject{currentmarker}{}%
\end{pgfscope}%
\begin{pgfscope}%
\pgfsys@transformshift{2.444971in}{2.262838in}%
\pgfsys@useobject{currentmarker}{}%
\end{pgfscope}%
\begin{pgfscope}%
\pgfsys@transformshift{2.462576in}{2.297783in}%
\pgfsys@useobject{currentmarker}{}%
\end{pgfscope}%
\begin{pgfscope}%
\pgfsys@transformshift{2.481120in}{2.376648in}%
\pgfsys@useobject{currentmarker}{}%
\end{pgfscope}%
\begin{pgfscope}%
\pgfsys@transformshift{2.503185in}{2.504060in}%
\pgfsys@useobject{currentmarker}{}%
\end{pgfscope}%
\begin{pgfscope}%
\pgfsys@transformshift{2.518441in}{2.593766in}%
\pgfsys@useobject{currentmarker}{}%
\end{pgfscope}%
\begin{pgfscope}%
\pgfsys@transformshift{2.538629in}{2.601012in}%
\pgfsys@useobject{currentmarker}{}%
\end{pgfscope}%
\begin{pgfscope}%
\pgfsys@transformshift{2.559519in}{2.500943in}%
\pgfsys@useobject{currentmarker}{}%
\end{pgfscope}%
\begin{pgfscope}%
\pgfsys@transformshift{2.577593in}{2.369582in}%
\pgfsys@useobject{currentmarker}{}%
\end{pgfscope}%
\begin{pgfscope}%
\pgfsys@transformshift{2.595668in}{2.294728in}%
\pgfsys@useobject{currentmarker}{}%
\end{pgfscope}%
\begin{pgfscope}%
\pgfsys@transformshift{2.616325in}{2.257684in}%
\pgfsys@useobject{currentmarker}{}%
\end{pgfscope}%
\begin{pgfscope}%
\pgfsys@transformshift{2.634164in}{2.244785in}%
\pgfsys@useobject{currentmarker}{}%
\end{pgfscope}%
\begin{pgfscope}%
\pgfsys@transformshift{2.655994in}{2.251020in}%
\pgfsys@useobject{currentmarker}{}%
\end{pgfscope}%
\begin{pgfscope}%
\pgfsys@transformshift{2.673362in}{2.273267in}%
\pgfsys@useobject{currentmarker}{}%
\end{pgfscope}%
\begin{pgfscope}%
\pgfsys@transformshift{2.691438in}{2.306120in}%
\pgfsys@useobject{currentmarker}{}%
\end{pgfscope}%
\begin{pgfscope}%
\pgfsys@transformshift{2.713032in}{2.407844in}%
\pgfsys@useobject{currentmarker}{}%
\end{pgfscope}%
\begin{pgfscope}%
\pgfsys@transformshift{2.734159in}{2.546178in}%
\pgfsys@useobject{currentmarker}{}%
\end{pgfscope}%
\begin{pgfscope}%
\pgfsys@transformshift{2.751998in}{2.607808in}%
\pgfsys@useobject{currentmarker}{}%
\end{pgfscope}%
\begin{pgfscope}%
\pgfsys@transformshift{2.770306in}{2.589841in}%
\pgfsys@useobject{currentmarker}{}%
\end{pgfscope}%
\begin{pgfscope}%
\pgfsys@transformshift{2.792371in}{2.449797in}%
\pgfsys@useobject{currentmarker}{}%
\end{pgfscope}%
\begin{pgfscope}%
\pgfsys@transformshift{2.812792in}{2.323852in}%
\pgfsys@useobject{currentmarker}{}%
\end{pgfscope}%
\begin{pgfscope}%
\pgfsys@transformshift{2.827580in}{2.282167in}%
\pgfsys@useobject{currentmarker}{}%
\end{pgfscope}%
\begin{pgfscope}%
\pgfsys@transformshift{2.848940in}{2.251829in}%
\pgfsys@useobject{currentmarker}{}%
\end{pgfscope}%
\begin{pgfscope}%
\pgfsys@transformshift{2.866781in}{2.246276in}%
\pgfsys@useobject{currentmarker}{}%
\end{pgfscope}%
\begin{pgfscope}%
\pgfsys@transformshift{2.884386in}{2.247091in}%
\pgfsys@useobject{currentmarker}{}%
\end{pgfscope}%
\begin{pgfscope}%
\pgfsys@transformshift{2.906450in}{2.270691in}%
\pgfsys@useobject{currentmarker}{}%
\end{pgfscope}%
\begin{pgfscope}%
\pgfsys@transformshift{2.923115in}{2.311965in}%
\pgfsys@useobject{currentmarker}{}%
\end{pgfscope}%
\begin{pgfscope}%
\pgfsys@transformshift{2.946354in}{2.377037in}%
\pgfsys@useobject{currentmarker}{}%
\end{pgfscope}%
\begin{pgfscope}%
\pgfsys@transformshift{2.959733in}{2.467249in}%
\pgfsys@useobject{currentmarker}{}%
\end{pgfscope}%
\begin{pgfscope}%
\pgfsys@transformshift{2.981093in}{2.580511in}%
\pgfsys@useobject{currentmarker}{}%
\end{pgfscope}%
\begin{pgfscope}%
\pgfsys@transformshift{3.002454in}{2.610765in}%
\pgfsys@useobject{currentmarker}{}%
\end{pgfscope}%
\begin{pgfscope}%
\pgfsys@transformshift{3.019354in}{2.548265in}%
\pgfsys@useobject{currentmarker}{}%
\end{pgfscope}%
\begin{pgfscope}%
\pgfsys@transformshift{3.037664in}{2.409275in}%
\pgfsys@useobject{currentmarker}{}%
\end{pgfscope}%
\begin{pgfscope}%
\pgfsys@transformshift{3.056441in}{2.315451in}%
\pgfsys@useobject{currentmarker}{}%
\end{pgfscope}%
\begin{pgfscope}%
\pgfsys@transformshift{3.076159in}{2.270837in}%
\pgfsys@useobject{currentmarker}{}%
\end{pgfscope}%
\begin{pgfscope}%
\pgfsys@transformshift{3.098458in}{2.251090in}%
\pgfsys@useobject{currentmarker}{}%
\end{pgfscope}%
\begin{pgfscope}%
\pgfsys@transformshift{3.115358in}{2.245115in}%
\pgfsys@useobject{currentmarker}{}%
\end{pgfscope}%
\begin{pgfscope}%
\pgfsys@transformshift{3.133197in}{2.253599in}%
\pgfsys@useobject{currentmarker}{}%
\end{pgfscope}%
\begin{pgfscope}%
\pgfsys@transformshift{3.154793in}{2.277130in}%
\pgfsys@useobject{currentmarker}{}%
\end{pgfscope}%
\begin{pgfscope}%
\pgfsys@transformshift{3.174980in}{2.323214in}%
\pgfsys@useobject{currentmarker}{}%
\end{pgfscope}%
\begin{pgfscope}%
\pgfsys@transformshift{3.193993in}{2.408301in}%
\pgfsys@useobject{currentmarker}{}%
\end{pgfscope}%
\begin{pgfscope}%
\pgfsys@transformshift{3.212301in}{2.540576in}%
\pgfsys@useobject{currentmarker}{}%
\end{pgfscope}%
\begin{pgfscope}%
\pgfsys@transformshift{3.232489in}{2.616570in}%
\pgfsys@useobject{currentmarker}{}%
\end{pgfscope}%
\begin{pgfscope}%
\pgfsys@transformshift{3.250797in}{2.612398in}%
\pgfsys@useobject{currentmarker}{}%
\end{pgfscope}%
\begin{pgfscope}%
\pgfsys@transformshift{3.272393in}{2.525662in}%
\pgfsys@useobject{currentmarker}{}%
\end{pgfscope}%
\begin{pgfscope}%
\pgfsys@transformshift{3.287180in}{2.424336in}%
\pgfsys@useobject{currentmarker}{}%
\end{pgfscope}%
\begin{pgfscope}%
\pgfsys@transformshift{3.307837in}{2.332272in}%
\pgfsys@useobject{currentmarker}{}%
\end{pgfscope}%
\begin{pgfscope}%
\pgfsys@transformshift{3.327084in}{2.281779in}%
\pgfsys@useobject{currentmarker}{}%
\end{pgfscope}%
\begin{pgfscope}%
\pgfsys@transformshift{3.345392in}{2.257822in}%
\pgfsys@useobject{currentmarker}{}%
\end{pgfscope}%
\begin{pgfscope}%
\pgfsys@transformshift{3.367222in}{2.247530in}%
\pgfsys@useobject{currentmarker}{}%
\end{pgfscope}%
\begin{pgfscope}%
\pgfsys@transformshift{3.384358in}{2.249022in}%
\pgfsys@useobject{currentmarker}{}%
\end{pgfscope}%
\begin{pgfscope}%
\pgfsys@transformshift{3.401729in}{2.265054in}%
\pgfsys@useobject{currentmarker}{}%
\end{pgfscope}%
\begin{pgfscope}%
\pgfsys@transformshift{3.423323in}{2.257534in}%
\pgfsys@useobject{currentmarker}{}%
\end{pgfscope}%
\begin{pgfscope}%
\pgfsys@transformshift{3.443510in}{2.288930in}%
\pgfsys@useobject{currentmarker}{}%
\end{pgfscope}%
\begin{pgfscope}%
\pgfsys@transformshift{3.462054in}{2.337004in}%
\pgfsys@useobject{currentmarker}{}%
\end{pgfscope}%
\begin{pgfscope}%
\pgfsys@transformshift{3.481771in}{2.411526in}%
\pgfsys@useobject{currentmarker}{}%
\end{pgfscope}%
\begin{pgfscope}%
\pgfsys@transformshift{3.499376in}{2.528320in}%
\pgfsys@useobject{currentmarker}{}%
\end{pgfscope}%
\begin{pgfscope}%
\pgfsys@transformshift{3.520266in}{2.624259in}%
\pgfsys@useobject{currentmarker}{}%
\end{pgfscope}%
\begin{pgfscope}%
\pgfsys@transformshift{3.537637in}{2.612909in}%
\pgfsys@useobject{currentmarker}{}%
\end{pgfscope}%
\begin{pgfscope}%
\pgfsys@transformshift{3.556415in}{2.519846in}%
\pgfsys@useobject{currentmarker}{}%
\end{pgfscope}%
\begin{pgfscope}%
\pgfsys@transformshift{3.577540in}{2.386120in}%
\pgfsys@useobject{currentmarker}{}%
\end{pgfscope}%
\begin{pgfscope}%
\pgfsys@transformshift{3.599605in}{2.306908in}%
\pgfsys@useobject{currentmarker}{}%
\end{pgfscope}%
\begin{pgfscope}%
\pgfsys@transformshift{3.616975in}{2.274464in}%
\pgfsys@useobject{currentmarker}{}%
\end{pgfscope}%
\begin{pgfscope}%
\pgfsys@transformshift{3.635049in}{2.253219in}%
\pgfsys@useobject{currentmarker}{}%
\end{pgfscope}%
\begin{pgfscope}%
\pgfsys@transformshift{3.656410in}{2.257658in}%
\pgfsys@useobject{currentmarker}{}%
\end{pgfscope}%
\begin{pgfscope}%
\pgfsys@transformshift{3.672370in}{2.247251in}%
\pgfsys@useobject{currentmarker}{}%
\end{pgfscope}%
\begin{pgfscope}%
\pgfsys@transformshift{3.691149in}{2.253066in}%
\pgfsys@useobject{currentmarker}{}%
\end{pgfscope}%
\begin{pgfscope}%
\pgfsys@transformshift{3.713214in}{2.265619in}%
\pgfsys@useobject{currentmarker}{}%
\end{pgfscope}%
\begin{pgfscope}%
\pgfsys@transformshift{3.732696in}{2.305891in}%
\pgfsys@useobject{currentmarker}{}%
\end{pgfscope}%
\begin{pgfscope}%
\pgfsys@transformshift{3.748658in}{2.352686in}%
\pgfsys@useobject{currentmarker}{}%
\end{pgfscope}%
\begin{pgfscope}%
\pgfsys@transformshift{3.766732in}{2.454609in}%
\pgfsys@useobject{currentmarker}{}%
\end{pgfscope}%
\begin{pgfscope}%
\pgfsys@transformshift{3.789032in}{2.605786in}%
\pgfsys@useobject{currentmarker}{}%
\end{pgfscope}%
\begin{pgfscope}%
\pgfsys@transformshift{3.808280in}{2.641280in}%
\pgfsys@useobject{currentmarker}{}%
\end{pgfscope}%
\begin{pgfscope}%
\pgfsys@transformshift{3.827293in}{2.615777in}%
\pgfsys@useobject{currentmarker}{}%
\end{pgfscope}%
\begin{pgfscope}%
\pgfsys@transformshift{3.846541in}{2.505388in}%
\pgfsys@useobject{currentmarker}{}%
\end{pgfscope}%
\begin{pgfscope}%
\pgfsys@transformshift{3.865554in}{2.382972in}%
\pgfsys@useobject{currentmarker}{}%
\end{pgfscope}%
\begin{pgfscope}%
\pgfsys@transformshift{3.884331in}{2.314158in}%
\pgfsys@useobject{currentmarker}{}%
\end{pgfscope}%
\begin{pgfscope}%
\pgfsys@transformshift{3.903815in}{2.273288in}%
\pgfsys@useobject{currentmarker}{}%
\end{pgfscope}%
\begin{pgfscope}%
\pgfsys@transformshift{3.922358in}{2.254527in}%
\pgfsys@useobject{currentmarker}{}%
\end{pgfscope}%
\begin{pgfscope}%
\pgfsys@transformshift{3.943483in}{2.248626in}%
\pgfsys@useobject{currentmarker}{}%
\end{pgfscope}%
\begin{pgfscope}%
\pgfsys@transformshift{3.962967in}{2.258930in}%
\pgfsys@useobject{currentmarker}{}%
\end{pgfscope}%
\begin{pgfscope}%
\pgfsys@transformshift{3.981040in}{2.282323in}%
\pgfsys@useobject{currentmarker}{}%
\end{pgfscope}%
\begin{pgfscope}%
\pgfsys@transformshift{3.999348in}{2.321759in}%
\pgfsys@useobject{currentmarker}{}%
\end{pgfscope}%
\begin{pgfscope}%
\pgfsys@transformshift{4.018596in}{2.399361in}%
\pgfsys@useobject{currentmarker}{}%
\end{pgfscope}%
\begin{pgfscope}%
\pgfsys@transformshift{4.038080in}{2.527367in}%
\pgfsys@useobject{currentmarker}{}%
\end{pgfscope}%
\begin{pgfscope}%
\pgfsys@transformshift{4.056623in}{2.616000in}%
\pgfsys@useobject{currentmarker}{}%
\end{pgfscope}%
\begin{pgfscope}%
\pgfsys@transformshift{4.077513in}{2.651204in}%
\pgfsys@useobject{currentmarker}{}%
\end{pgfscope}%
\begin{pgfscope}%
\pgfsys@transformshift{4.096527in}{2.604508in}%
\pgfsys@useobject{currentmarker}{}%
\end{pgfscope}%
\begin{pgfscope}%
\pgfsys@transformshift{4.117888in}{2.653950in}%
\pgfsys@useobject{currentmarker}{}%
\end{pgfscope}%
\begin{pgfscope}%
\pgfsys@transformshift{4.133850in}{2.631328in}%
\pgfsys@useobject{currentmarker}{}%
\end{pgfscope}%
\begin{pgfscope}%
\pgfsys@transformshift{4.151689in}{2.531450in}%
\pgfsys@useobject{currentmarker}{}%
\end{pgfscope}%
\begin{pgfscope}%
\pgfsys@transformshift{4.174691in}{2.410867in}%
\pgfsys@useobject{currentmarker}{}%
\end{pgfscope}%
\begin{pgfscope}%
\pgfsys@transformshift{4.189479in}{2.345464in}%
\pgfsys@useobject{currentmarker}{}%
\end{pgfscope}%
\begin{pgfscope}%
\pgfsys@transformshift{4.209666in}{2.292827in}%
\pgfsys@useobject{currentmarker}{}%
\end{pgfscope}%
\begin{pgfscope}%
\pgfsys@transformshift{4.231262in}{2.258512in}%
\pgfsys@useobject{currentmarker}{}%
\end{pgfscope}%
\begin{pgfscope}%
\pgfsys@transformshift{4.250041in}{2.250795in}%
\pgfsys@useobject{currentmarker}{}%
\end{pgfscope}%
\begin{pgfscope}%
\pgfsys@transformshift{4.269523in}{2.260897in}%
\pgfsys@useobject{currentmarker}{}%
\end{pgfscope}%
\begin{pgfscope}%
\pgfsys@transformshift{4.288771in}{2.286792in}%
\pgfsys@useobject{currentmarker}{}%
\end{pgfscope}%
\begin{pgfscope}%
\pgfsys@transformshift{4.307313in}{2.335495in}%
\pgfsys@useobject{currentmarker}{}%
\end{pgfscope}%
\begin{pgfscope}%
\pgfsys@transformshift{4.325858in}{2.421483in}%
\pgfsys@useobject{currentmarker}{}%
\end{pgfscope}%
\begin{pgfscope}%
\pgfsys@transformshift{4.346045in}{2.519232in}%
\pgfsys@useobject{currentmarker}{}%
\end{pgfscope}%
\begin{pgfscope}%
\pgfsys@transformshift{4.346514in}{2.567184in}%
\pgfsys@useobject{currentmarker}{}%
\end{pgfscope}%
\begin{pgfscope}%
\pgfsys@transformshift{4.363884in}{2.602413in}%
\pgfsys@useobject{currentmarker}{}%
\end{pgfscope}%
\begin{pgfscope}%
\pgfsys@transformshift{4.385244in}{2.666248in}%
\pgfsys@useobject{currentmarker}{}%
\end{pgfscope}%
\begin{pgfscope}%
\pgfsys@transformshift{4.404257in}{2.630655in}%
\pgfsys@useobject{currentmarker}{}%
\end{pgfscope}%
\begin{pgfscope}%
\pgfsys@transformshift{4.423270in}{2.539833in}%
\pgfsys@useobject{currentmarker}{}%
\end{pgfscope}%
\begin{pgfscope}%
\pgfsys@transformshift{4.441815in}{2.426766in}%
\pgfsys@useobject{currentmarker}{}%
\end{pgfscope}%
\begin{pgfscope}%
\pgfsys@transformshift{4.460828in}{2.340771in}%
\pgfsys@useobject{currentmarker}{}%
\end{pgfscope}%
\begin{pgfscope}%
\pgfsys@transformshift{4.479841in}{2.293806in}%
\pgfsys@useobject{currentmarker}{}%
\end{pgfscope}%
\begin{pgfscope}%
\pgfsys@transformshift{4.480310in}{2.293381in}%
\pgfsys@useobject{currentmarker}{}%
\end{pgfscope}%
\begin{pgfscope}%
\pgfsys@transformshift{4.473737in}{2.311205in}%
\pgfsys@useobject{currentmarker}{}%
\end{pgfscope}%
\begin{pgfscope}%
\pgfsys@transformshift{4.456837in}{2.405185in}%
\pgfsys@useobject{currentmarker}{}%
\end{pgfscope}%
\begin{pgfscope}%
\pgfsys@transformshift{4.435712in}{2.590563in}%
\pgfsys@useobject{currentmarker}{}%
\end{pgfscope}%
\begin{pgfscope}%
\pgfsys@transformshift{4.417871in}{2.269700in}%
\pgfsys@useobject{currentmarker}{}%
\end{pgfscope}%
\begin{pgfscope}%
\pgfsys@transformshift{4.396980in}{2.250008in}%
\pgfsys@useobject{currentmarker}{}%
\end{pgfscope}%
\begin{pgfscope}%
\pgfsys@transformshift{4.373273in}{2.284174in}%
\pgfsys@useobject{currentmarker}{}%
\end{pgfscope}%
\begin{pgfscope}%
\pgfsys@transformshift{4.360833in}{2.317980in}%
\pgfsys@useobject{currentmarker}{}%
\end{pgfscope}%
\begin{pgfscope}%
\pgfsys@transformshift{4.340411in}{2.421839in}%
\pgfsys@useobject{currentmarker}{}%
\end{pgfscope}%
\begin{pgfscope}%
\pgfsys@transformshift{4.322101in}{2.585451in}%
\pgfsys@useobject{currentmarker}{}%
\end{pgfscope}%
\begin{pgfscope}%
\pgfsys@transformshift{4.301916in}{2.655150in}%
\pgfsys@useobject{currentmarker}{}%
\end{pgfscope}%
\begin{pgfscope}%
\pgfsys@transformshift{4.284545in}{2.543177in}%
\pgfsys@useobject{currentmarker}{}%
\end{pgfscope}%
\begin{pgfscope}%
\pgfsys@transformshift{4.260369in}{2.357202in}%
\pgfsys@useobject{currentmarker}{}%
\end{pgfscope}%
\begin{pgfscope}%
\pgfsys@transformshift{4.245816in}{2.294394in}%
\pgfsys@useobject{currentmarker}{}%
\end{pgfscope}%
\begin{pgfscope}%
\pgfsys@transformshift{4.224923in}{2.255039in}%
\pgfsys@useobject{currentmarker}{}%
\end{pgfscope}%
\begin{pgfscope}%
\pgfsys@transformshift{4.203329in}{2.255118in}%
\pgfsys@useobject{currentmarker}{}%
\end{pgfscope}%
\begin{pgfscope}%
\pgfsys@transformshift{4.185019in}{2.286398in}%
\pgfsys@useobject{currentmarker}{}%
\end{pgfscope}%
\begin{pgfscope}%
\pgfsys@transformshift{4.167180in}{2.361444in}%
\pgfsys@useobject{currentmarker}{}%
\end{pgfscope}%
\begin{pgfscope}%
\pgfsys@transformshift{4.149575in}{2.510958in}%
\pgfsys@useobject{currentmarker}{}%
\end{pgfscope}%
\begin{pgfscope}%
\pgfsys@transformshift{4.128216in}{2.640982in}%
\pgfsys@useobject{currentmarker}{}%
\end{pgfscope}%
\begin{pgfscope}%
\pgfsys@transformshift{4.111080in}{2.612349in}%
\pgfsys@useobject{currentmarker}{}%
\end{pgfscope}%
\begin{pgfscope}%
\pgfsys@transformshift{4.088781in}{2.410668in}%
\pgfsys@useobject{currentmarker}{}%
\end{pgfscope}%
\begin{pgfscope}%
\pgfsys@transformshift{4.070942in}{2.311270in}%
\pgfsys@useobject{currentmarker}{}%
\end{pgfscope}%
\begin{pgfscope}%
\pgfsys@transformshift{4.052397in}{2.265425in}%
\pgfsys@useobject{currentmarker}{}%
\end{pgfscope}%
\begin{pgfscope}%
\pgfsys@transformshift{4.031976in}{2.247424in}%
\pgfsys@useobject{currentmarker}{}%
\end{pgfscope}%
\begin{pgfscope}%
\pgfsys@transformshift{4.013199in}{2.264349in}%
\pgfsys@useobject{currentmarker}{}%
\end{pgfscope}%
\begin{pgfscope}%
\pgfsys@transformshift{3.994420in}{2.312598in}%
\pgfsys@useobject{currentmarker}{}%
\end{pgfscope}%
\begin{pgfscope}%
\pgfsys@transformshift{3.973059in}{2.447919in}%
\pgfsys@useobject{currentmarker}{}%
\end{pgfscope}%
\begin{pgfscope}%
\pgfsys@transformshift{3.958036in}{2.578636in}%
\pgfsys@useobject{currentmarker}{}%
\end{pgfscope}%
\begin{pgfscope}%
\pgfsys@transformshift{3.935268in}{2.630570in}%
\pgfsys@useobject{currentmarker}{}%
\end{pgfscope}%
\begin{pgfscope}%
\pgfsys@transformshift{3.917195in}{2.536362in}%
\pgfsys@useobject{currentmarker}{}%
\end{pgfscope}%
\begin{pgfscope}%
\pgfsys@transformshift{3.897947in}{2.373300in}%
\pgfsys@useobject{currentmarker}{}%
\end{pgfscope}%
\begin{pgfscope}%
\pgfsys@transformshift{3.879637in}{2.294682in}%
\pgfsys@useobject{currentmarker}{}%
\end{pgfscope}%
\begin{pgfscope}%
\pgfsys@transformshift{3.857572in}{2.274351in}%
\pgfsys@useobject{currentmarker}{}%
\end{pgfscope}%
\begin{pgfscope}%
\pgfsys@transformshift{3.839733in}{2.248423in}%
\pgfsys@useobject{currentmarker}{}%
\end{pgfscope}%
\begin{pgfscope}%
\pgfsys@transformshift{3.820486in}{2.251124in}%
\pgfsys@useobject{currentmarker}{}%
\end{pgfscope}%
\begin{pgfscope}%
\pgfsys@transformshift{3.801238in}{2.279074in}%
\pgfsys@useobject{currentmarker}{}%
\end{pgfscope}%
\begin{pgfscope}%
\pgfsys@transformshift{3.783399in}{2.343543in}%
\pgfsys@useobject{currentmarker}{}%
\end{pgfscope}%
\begin{pgfscope}%
\pgfsys@transformshift{3.761568in}{2.466586in}%
\pgfsys@useobject{currentmarker}{}%
\end{pgfscope}%
\begin{pgfscope}%
\pgfsys@transformshift{3.743729in}{2.602671in}%
\pgfsys@useobject{currentmarker}{}%
\end{pgfscope}%
\begin{pgfscope}%
\pgfsys@transformshift{3.724716in}{2.622370in}%
\pgfsys@useobject{currentmarker}{}%
\end{pgfscope}%
\begin{pgfscope}%
\pgfsys@transformshift{3.708285in}{2.503917in}%
\pgfsys@useobject{currentmarker}{}%
\end{pgfscope}%
\begin{pgfscope}%
\pgfsys@transformshift{3.684812in}{2.354530in}%
\pgfsys@useobject{currentmarker}{}%
\end{pgfscope}%
\begin{pgfscope}%
\pgfsys@transformshift{3.668147in}{2.294044in}%
\pgfsys@useobject{currentmarker}{}%
\end{pgfscope}%
\begin{pgfscope}%
\pgfsys@transformshift{3.648897in}{2.265847in}%
\pgfsys@useobject{currentmarker}{}%
\end{pgfscope}%
\begin{pgfscope}%
\pgfsys@transformshift{3.629181in}{2.246825in}%
\pgfsys@useobject{currentmarker}{}%
\end{pgfscope}%
\begin{pgfscope}%
\pgfsys@transformshift{3.607116in}{2.253090in}%
\pgfsys@useobject{currentmarker}{}%
\end{pgfscope}%
\begin{pgfscope}%
\pgfsys@transformshift{3.585520in}{2.287681in}%
\pgfsys@useobject{currentmarker}{}%
\end{pgfscope}%
\begin{pgfscope}%
\pgfsys@transformshift{3.570264in}{2.324567in}%
\pgfsys@useobject{currentmarker}{}%
\end{pgfscope}%
\begin{pgfscope}%
\pgfsys@transformshift{3.549608in}{2.451929in}%
\pgfsys@useobject{currentmarker}{}%
\end{pgfscope}%
\begin{pgfscope}%
\pgfsys@transformshift{3.533177in}{2.562214in}%
\pgfsys@useobject{currentmarker}{}%
\end{pgfscope}%
\begin{pgfscope}%
\pgfsys@transformshift{3.514867in}{2.621762in}%
\pgfsys@useobject{currentmarker}{}%
\end{pgfscope}%
\begin{pgfscope}%
\pgfsys@transformshift{3.494447in}{2.612177in}%
\pgfsys@useobject{currentmarker}{}%
\end{pgfscope}%
\begin{pgfscope}%
\pgfsys@transformshift{3.474260in}{2.508786in}%
\pgfsys@useobject{currentmarker}{}%
\end{pgfscope}%
\begin{pgfscope}%
\pgfsys@transformshift{3.457358in}{2.387437in}%
\pgfsys@useobject{currentmarker}{}%
\end{pgfscope}%
\begin{pgfscope}%
\pgfsys@transformshift{3.438347in}{2.304258in}%
\pgfsys@useobject{currentmarker}{}%
\end{pgfscope}%
\begin{pgfscope}%
\pgfsys@transformshift{3.412995in}{2.258382in}%
\pgfsys@useobject{currentmarker}{}%
\end{pgfscope}%
\begin{pgfscope}%
\pgfsys@transformshift{3.396564in}{2.251473in}%
\pgfsys@useobject{currentmarker}{}%
\end{pgfscope}%
\begin{pgfscope}%
\pgfsys@transformshift{3.378021in}{2.245362in}%
\pgfsys@useobject{currentmarker}{}%
\end{pgfscope}%
\begin{pgfscope}%
\pgfsys@transformshift{3.359008in}{2.258747in}%
\pgfsys@useobject{currentmarker}{}%
\end{pgfscope}%
\begin{pgfscope}%
\pgfsys@transformshift{3.339760in}{2.291769in}%
\pgfsys@useobject{currentmarker}{}%
\end{pgfscope}%
\begin{pgfscope}%
\pgfsys@transformshift{3.320982in}{2.369240in}%
\pgfsys@useobject{currentmarker}{}%
\end{pgfscope}%
\begin{pgfscope}%
\pgfsys@transformshift{3.302203in}{2.493115in}%
\pgfsys@useobject{currentmarker}{}%
\end{pgfscope}%
\begin{pgfscope}%
\pgfsys@transformshift{3.280843in}{2.596654in}%
\pgfsys@useobject{currentmarker}{}%
\end{pgfscope}%
\begin{pgfscope}%
\pgfsys@transformshift{3.264176in}{2.612219in}%
\pgfsys@useobject{currentmarker}{}%
\end{pgfscope}%
\begin{pgfscope}%
\pgfsys@transformshift{3.243520in}{2.551449in}%
\pgfsys@useobject{currentmarker}{}%
\end{pgfscope}%
\begin{pgfscope}%
\pgfsys@transformshift{3.223569in}{2.395624in}%
\pgfsys@useobject{currentmarker}{}%
\end{pgfscope}%
\begin{pgfscope}%
\pgfsys@transformshift{3.205259in}{2.312258in}%
\pgfsys@useobject{currentmarker}{}%
\end{pgfscope}%
\begin{pgfscope}%
\pgfsys@transformshift{3.184603in}{2.266760in}%
\pgfsys@useobject{currentmarker}{}%
\end{pgfscope}%
\begin{pgfscope}%
\pgfsys@transformshift{3.165826in}{2.247332in}%
\pgfsys@useobject{currentmarker}{}%
\end{pgfscope}%
\begin{pgfscope}%
\pgfsys@transformshift{3.147516in}{2.450135in}%
\pgfsys@useobject{currentmarker}{}%
\end{pgfscope}%
\begin{pgfscope}%
\pgfsys@transformshift{3.129677in}{2.593883in}%
\pgfsys@useobject{currentmarker}{}%
\end{pgfscope}%
\begin{pgfscope}%
\pgfsys@transformshift{3.109960in}{2.605001in}%
\pgfsys@useobject{currentmarker}{}%
\end{pgfscope}%
\begin{pgfscope}%
\pgfsys@transformshift{3.091416in}{2.485287in}%
\pgfsys@useobject{currentmarker}{}%
\end{pgfscope}%
\begin{pgfscope}%
\pgfsys@transformshift{3.066769in}{2.330932in}%
\pgfsys@useobject{currentmarker}{}%
\end{pgfscope}%
\begin{pgfscope}%
\pgfsys@transformshift{3.054563in}{2.288295in}%
\pgfsys@useobject{currentmarker}{}%
\end{pgfscope}%
\begin{pgfscope}%
\pgfsys@transformshift{3.035550in}{2.254902in}%
\pgfsys@useobject{currentmarker}{}%
\end{pgfscope}%
\begin{pgfscope}%
\pgfsys@transformshift{3.013720in}{2.244523in}%
\pgfsys@useobject{currentmarker}{}%
\end{pgfscope}%
\begin{pgfscope}%
\pgfsys@transformshift{2.995412in}{2.255682in}%
\pgfsys@useobject{currentmarker}{}%
\end{pgfscope}%
\begin{pgfscope}%
\pgfsys@transformshift{2.975695in}{2.295394in}%
\pgfsys@useobject{currentmarker}{}%
\end{pgfscope}%
\begin{pgfscope}%
\pgfsys@transformshift{2.955977in}{2.378670in}%
\pgfsys@useobject{currentmarker}{}%
\end{pgfscope}%
\begin{pgfscope}%
\pgfsys@transformshift{2.936495in}{2.528959in}%
\pgfsys@useobject{currentmarker}{}%
\end{pgfscope}%
\begin{pgfscope}%
\pgfsys@transformshift{2.918656in}{2.613712in}%
\pgfsys@useobject{currentmarker}{}%
\end{pgfscope}%
\begin{pgfscope}%
\pgfsys@transformshift{2.900346in}{2.576247in}%
\pgfsys@useobject{currentmarker}{}%
\end{pgfscope}%
\begin{pgfscope}%
\pgfsys@transformshift{2.879926in}{2.495010in}%
\pgfsys@useobject{currentmarker}{}%
\end{pgfscope}%
\begin{pgfscope}%
\pgfsys@transformshift{2.858799in}{2.351635in}%
\pgfsys@useobject{currentmarker}{}%
\end{pgfscope}%
\begin{pgfscope}%
\pgfsys@transformshift{2.840491in}{2.285180in}%
\pgfsys@useobject{currentmarker}{}%
\end{pgfscope}%
\begin{pgfscope}%
\pgfsys@transformshift{2.821712in}{2.255213in}%
\pgfsys@useobject{currentmarker}{}%
\end{pgfscope}%
\begin{pgfscope}%
\pgfsys@transformshift{2.803404in}{2.244525in}%
\pgfsys@useobject{currentmarker}{}%
\end{pgfscope}%
\begin{pgfscope}%
\pgfsys@transformshift{2.783685in}{2.253812in}%
\pgfsys@useobject{currentmarker}{}%
\end{pgfscope}%
\begin{pgfscope}%
\pgfsys@transformshift{2.763264in}{2.287012in}%
\pgfsys@useobject{currentmarker}{}%
\end{pgfscope}%
\begin{pgfscope}%
\pgfsys@transformshift{2.744721in}{2.360030in}%
\pgfsys@useobject{currentmarker}{}%
\end{pgfscope}%
\begin{pgfscope}%
\pgfsys@transformshift{2.744016in}{2.441145in}%
\pgfsys@useobject{currentmarker}{}%
\end{pgfscope}%
\begin{pgfscope}%
\pgfsys@transformshift{2.725708in}{2.498901in}%
\pgfsys@useobject{currentmarker}{}%
\end{pgfscope}%
\begin{pgfscope}%
\pgfsys@transformshift{2.707398in}{2.605429in}%
\pgfsys@useobject{currentmarker}{}%
\end{pgfscope}%
\begin{pgfscope}%
\pgfsys@transformshift{2.689090in}{2.608593in}%
\pgfsys@useobject{currentmarker}{}%
\end{pgfscope}%
\begin{pgfscope}%
\pgfsys@transformshift{2.666557in}{2.491001in}%
\pgfsys@useobject{currentmarker}{}%
\end{pgfscope}%
\begin{pgfscope}%
\pgfsys@transformshift{2.644961in}{2.343987in}%
\pgfsys@useobject{currentmarker}{}%
\end{pgfscope}%
\begin{pgfscope}%
\pgfsys@transformshift{2.629704in}{2.293645in}%
\pgfsys@useobject{currentmarker}{}%
\end{pgfscope}%
\begin{pgfscope}%
\pgfsys@transformshift{2.610456in}{2.259723in}%
\pgfsys@useobject{currentmarker}{}%
\end{pgfscope}%
\begin{pgfscope}%
\pgfsys@transformshift{2.589564in}{2.244610in}%
\pgfsys@useobject{currentmarker}{}%
\end{pgfscope}%
\begin{pgfscope}%
\pgfsys@transformshift{2.571490in}{2.251306in}%
\pgfsys@useobject{currentmarker}{}%
\end{pgfscope}%
\begin{pgfscope}%
\pgfsys@transformshift{2.552008in}{2.273003in}%
\pgfsys@useobject{currentmarker}{}%
\end{pgfscope}%
\begin{pgfscope}%
\pgfsys@transformshift{2.532760in}{2.323319in}%
\pgfsys@useobject{currentmarker}{}%
\end{pgfscope}%
\begin{pgfscope}%
\pgfsys@transformshift{2.515390in}{2.436970in}%
\pgfsys@useobject{currentmarker}{}%
\end{pgfscope}%
\begin{pgfscope}%
\pgfsys@transformshift{2.493560in}{2.575993in}%
\pgfsys@useobject{currentmarker}{}%
\end{pgfscope}%
\begin{pgfscope}%
\pgfsys@transformshift{2.475721in}{2.610396in}%
\pgfsys@useobject{currentmarker}{}%
\end{pgfscope}%
\begin{pgfscope}%
\pgfsys@transformshift{2.456473in}{2.581069in}%
\pgfsys@useobject{currentmarker}{}%
\end{pgfscope}%
\begin{pgfscope}%
\pgfsys@transformshift{2.437460in}{2.479690in}%
\pgfsys@useobject{currentmarker}{}%
\end{pgfscope}%
\begin{pgfscope}%
\pgfsys@transformshift{2.418681in}{2.357280in}%
\pgfsys@useobject{currentmarker}{}%
\end{pgfscope}%
\begin{pgfscope}%
\pgfsys@transformshift{2.397556in}{2.282010in}%
\pgfsys@useobject{currentmarker}{}%
\end{pgfscope}%
\begin{pgfscope}%
\pgfsys@transformshift{2.378777in}{2.266924in}%
\pgfsys@useobject{currentmarker}{}%
\end{pgfscope}%
\begin{pgfscope}%
\pgfsys@transformshift{2.360938in}{2.247384in}%
\pgfsys@useobject{currentmarker}{}%
\end{pgfscope}%
\begin{pgfscope}%
\pgfsys@transformshift{2.341456in}{2.246169in}%
\pgfsys@useobject{currentmarker}{}%
\end{pgfscope}%
\begin{pgfscope}%
\pgfsys@transformshift{2.323617in}{2.259104in}%
\pgfsys@useobject{currentmarker}{}%
\end{pgfscope}%
\begin{pgfscope}%
\pgfsys@transformshift{2.302021in}{2.301023in}%
\pgfsys@useobject{currentmarker}{}%
\end{pgfscope}%
\begin{pgfscope}%
\pgfsys@transformshift{2.286530in}{2.367818in}%
\pgfsys@useobject{currentmarker}{}%
\end{pgfscope}%
\begin{pgfscope}%
\pgfsys@transformshift{2.263994in}{2.525214in}%
\pgfsys@useobject{currentmarker}{}%
\end{pgfscope}%
\begin{pgfscope}%
\pgfsys@transformshift{2.246626in}{2.606325in}%
\pgfsys@useobject{currentmarker}{}%
\end{pgfscope}%
\begin{pgfscope}%
\pgfsys@transformshift{2.224325in}{2.611160in}%
\pgfsys@useobject{currentmarker}{}%
\end{pgfscope}%
\begin{pgfscope}%
\pgfsys@transformshift{2.205548in}{2.533012in}%
\pgfsys@useobject{currentmarker}{}%
\end{pgfscope}%
\begin{pgfscope}%
\pgfsys@transformshift{2.187004in}{2.401886in}%
\pgfsys@useobject{currentmarker}{}%
\end{pgfscope}%
\begin{pgfscope}%
\pgfsys@transformshift{2.168930in}{2.320823in}%
\pgfsys@useobject{currentmarker}{}%
\end{pgfscope}%
\begin{pgfscope}%
\pgfsys@transformshift{2.147569in}{2.268787in}%
\pgfsys@useobject{currentmarker}{}%
\end{pgfscope}%
\begin{pgfscope}%
\pgfsys@transformshift{2.128321in}{2.252208in}%
\pgfsys@useobject{currentmarker}{}%
\end{pgfscope}%
\begin{pgfscope}%
\pgfsys@transformshift{2.112125in}{2.245331in}%
\pgfsys@useobject{currentmarker}{}%
\end{pgfscope}%
\begin{pgfscope}%
\pgfsys@transformshift{2.091939in}{2.256629in}%
\pgfsys@useobject{currentmarker}{}%
\end{pgfscope}%
\begin{pgfscope}%
\pgfsys@transformshift{2.070343in}{2.288580in}%
\pgfsys@useobject{currentmarker}{}%
\end{pgfscope}%
\begin{pgfscope}%
\pgfsys@transformshift{2.052973in}{2.355649in}%
\pgfsys@useobject{currentmarker}{}%
\end{pgfscope}%
\begin{pgfscope}%
\pgfsys@transformshift{2.033491in}{2.490363in}%
\pgfsys@useobject{currentmarker}{}%
\end{pgfscope}%
\begin{pgfscope}%
\pgfsys@transformshift{2.014009in}{2.599479in}%
\pgfsys@useobject{currentmarker}{}%
\end{pgfscope}%
\begin{pgfscope}%
\pgfsys@transformshift{1.995933in}{2.614660in}%
\pgfsys@useobject{currentmarker}{}%
\end{pgfscope}%
\begin{pgfscope}%
\pgfsys@transformshift{1.974105in}{2.593779in}%
\pgfsys@useobject{currentmarker}{}%
\end{pgfscope}%
\begin{pgfscope}%
\pgfsys@transformshift{1.955795in}{2.481402in}%
\pgfsys@useobject{currentmarker}{}%
\end{pgfscope}%
\begin{pgfscope}%
\pgfsys@transformshift{1.938425in}{2.607486in}%
\pgfsys@useobject{currentmarker}{}%
\end{pgfscope}%
\begin{pgfscope}%
\pgfsys@transformshift{1.915422in}{2.584543in}%
\pgfsys@useobject{currentmarker}{}%
\end{pgfscope}%
\begin{pgfscope}%
\pgfsys@transformshift{1.900634in}{2.498350in}%
\pgfsys@useobject{currentmarker}{}%
\end{pgfscope}%
\begin{pgfscope}%
\pgfsys@transformshift{1.881856in}{2.359548in}%
\pgfsys@useobject{currentmarker}{}%
\end{pgfscope}%
\begin{pgfscope}%
\pgfsys@transformshift{1.860496in}{2.289182in}%
\pgfsys@useobject{currentmarker}{}%
\end{pgfscope}%
\begin{pgfscope}%
\pgfsys@transformshift{1.841012in}{2.262753in}%
\pgfsys@useobject{currentmarker}{}%
\end{pgfscope}%
\begin{pgfscope}%
\pgfsys@transformshift{1.822939in}{2.247606in}%
\pgfsys@useobject{currentmarker}{}%
\end{pgfscope}%
\begin{pgfscope}%
\pgfsys@transformshift{1.804394in}{2.251010in}%
\pgfsys@useobject{currentmarker}{}%
\end{pgfscope}%
\begin{pgfscope}%
\pgfsys@transformshift{1.782330in}{2.277906in}%
\pgfsys@useobject{currentmarker}{}%
\end{pgfscope}%
\begin{pgfscope}%
\pgfsys@transformshift{1.767307in}{2.313676in}%
\pgfsys@useobject{currentmarker}{}%
\end{pgfscope}%
\begin{pgfscope}%
\pgfsys@transformshift{1.745713in}{2.421892in}%
\pgfsys@useobject{currentmarker}{}%
\end{pgfscope}%
\begin{pgfscope}%
\pgfsys@transformshift{1.726935in}{2.550065in}%
\pgfsys@useobject{currentmarker}{}%
\end{pgfscope}%
\begin{pgfscope}%
\pgfsys@transformshift{1.708861in}{2.619954in}%
\pgfsys@useobject{currentmarker}{}%
\end{pgfscope}%
\begin{pgfscope}%
\pgfsys@transformshift{1.687265in}{2.613606in}%
\pgfsys@useobject{currentmarker}{}%
\end{pgfscope}%
\begin{pgfscope}%
\pgfsys@transformshift{1.668957in}{2.522129in}%
\pgfsys@useobject{currentmarker}{}%
\end{pgfscope}%
\begin{pgfscope}%
\pgfsys@transformshift{1.650647in}{2.403202in}%
\pgfsys@useobject{currentmarker}{}%
\end{pgfscope}%
\begin{pgfscope}%
\pgfsys@transformshift{1.629522in}{2.323462in}%
\pgfsys@useobject{currentmarker}{}%
\end{pgfscope}%
\begin{pgfscope}%
\pgfsys@transformshift{1.610509in}{2.279750in}%
\pgfsys@useobject{currentmarker}{}%
\end{pgfscope}%
\begin{pgfscope}%
\pgfsys@transformshift{1.591496in}{2.254985in}%
\pgfsys@useobject{currentmarker}{}%
\end{pgfscope}%
\begin{pgfscope}%
\pgfsys@transformshift{1.573186in}{2.246266in}%
\pgfsys@useobject{currentmarker}{}%
\end{pgfscope}%
\begin{pgfscope}%
\pgfsys@transformshift{1.554878in}{2.252352in}%
\pgfsys@useobject{currentmarker}{}%
\end{pgfscope}%
\begin{pgfscope}%
\pgfsys@transformshift{1.534221in}{2.280505in}%
\pgfsys@useobject{currentmarker}{}%
\end{pgfscope}%
\begin{pgfscope}%
\pgfsys@transformshift{1.515443in}{2.330511in}%
\pgfsys@useobject{currentmarker}{}%
\end{pgfscope}%
\begin{pgfscope}%
\pgfsys@transformshift{1.496195in}{2.428825in}%
\pgfsys@useobject{currentmarker}{}%
\end{pgfscope}%
\begin{pgfscope}%
\pgfsys@transformshift{1.476478in}{2.568625in}%
\pgfsys@useobject{currentmarker}{}%
\end{pgfscope}%
\begin{pgfscope}%
\pgfsys@transformshift{1.456760in}{2.628726in}%
\pgfsys@useobject{currentmarker}{}%
\end{pgfscope}%
\begin{pgfscope}%
\pgfsys@transformshift{1.438686in}{2.619340in}%
\pgfsys@useobject{currentmarker}{}%
\end{pgfscope}%
\begin{pgfscope}%
\pgfsys@transformshift{1.420847in}{2.541199in}%
\pgfsys@useobject{currentmarker}{}%
\end{pgfscope}%
\begin{pgfscope}%
\pgfsys@transformshift{1.398314in}{2.420555in}%
\pgfsys@useobject{currentmarker}{}%
\end{pgfscope}%
\begin{pgfscope}%
\pgfsys@transformshift{1.379300in}{2.349042in}%
\pgfsys@useobject{currentmarker}{}%
\end{pgfscope}%
\begin{pgfscope}%
\pgfsys@transformshift{1.361227in}{2.299619in}%
\pgfsys@useobject{currentmarker}{}%
\end{pgfscope}%
\begin{pgfscope}%
\pgfsys@transformshift{1.340334in}{2.264592in}%
\pgfsys@useobject{currentmarker}{}%
\end{pgfscope}%
\begin{pgfscope}%
\pgfsys@transformshift{1.323200in}{2.342791in}%
\pgfsys@useobject{currentmarker}{}%
\end{pgfscope}%
\begin{pgfscope}%
\pgfsys@transformshift{1.304187in}{2.290558in}%
\pgfsys@useobject{currentmarker}{}%
\end{pgfscope}%
\begin{pgfscope}%
\pgfsys@transformshift{1.284234in}{2.261977in}%
\pgfsys@useobject{currentmarker}{}%
\end{pgfscope}%
\begin{pgfscope}%
\pgfsys@transformshift{1.267100in}{2.249074in}%
\pgfsys@useobject{currentmarker}{}%
\end{pgfscope}%
\begin{pgfscope}%
\pgfsys@transformshift{1.246209in}{2.250071in}%
\pgfsys@useobject{currentmarker}{}%
\end{pgfscope}%
\begin{pgfscope}%
\pgfsys@transformshift{1.226257in}{2.273059in}%
\pgfsys@useobject{currentmarker}{}%
\end{pgfscope}%
\begin{pgfscope}%
\pgfsys@transformshift{1.209357in}{2.310001in}%
\pgfsys@useobject{currentmarker}{}%
\end{pgfscope}%
\begin{pgfscope}%
\pgfsys@transformshift{1.188701in}{2.387540in}%
\pgfsys@useobject{currentmarker}{}%
\end{pgfscope}%
\begin{pgfscope}%
\pgfsys@transformshift{1.168279in}{2.491701in}%
\pgfsys@useobject{currentmarker}{}%
\end{pgfscope}%
\begin{pgfscope}%
\pgfsys@transformshift{1.151143in}{2.593578in}%
\pgfsys@useobject{currentmarker}{}%
\end{pgfscope}%
\begin{pgfscope}%
\pgfsys@transformshift{1.129547in}{2.641991in}%
\pgfsys@useobject{currentmarker}{}%
\end{pgfscope}%
\begin{pgfscope}%
\pgfsys@transformshift{1.109596in}{2.608571in}%
\pgfsys@useobject{currentmarker}{}%
\end{pgfscope}%
\begin{pgfscope}%
\pgfsys@transformshift{1.091757in}{2.495392in}%
\pgfsys@useobject{currentmarker}{}%
\end{pgfscope}%
\begin{pgfscope}%
\pgfsys@transformshift{1.072039in}{2.375529in}%
\pgfsys@useobject{currentmarker}{}%
\end{pgfscope}%
\begin{pgfscope}%
\pgfsys@transformshift{1.054434in}{2.315523in}%
\pgfsys@useobject{currentmarker}{}%
\end{pgfscope}%
\begin{pgfscope}%
\pgfsys@transformshift{1.034952in}{2.282228in}%
\pgfsys@useobject{currentmarker}{}%
\end{pgfscope}%
\begin{pgfscope}%
\pgfsys@transformshift{1.016878in}{2.258941in}%
\pgfsys@useobject{currentmarker}{}%
\end{pgfscope}%
\begin{pgfscope}%
\pgfsys@transformshift{0.995988in}{2.249792in}%
\pgfsys@useobject{currentmarker}{}%
\end{pgfscope}%
\begin{pgfscope}%
\pgfsys@transformshift{0.977912in}{2.259283in}%
\pgfsys@useobject{currentmarker}{}%
\end{pgfscope}%
\begin{pgfscope}%
\pgfsys@transformshift{0.956787in}{2.281018in}%
\pgfsys@useobject{currentmarker}{}%
\end{pgfscope}%
\begin{pgfscope}%
\pgfsys@transformshift{0.940122in}{2.318841in}%
\pgfsys@useobject{currentmarker}{}%
\end{pgfscope}%
\begin{pgfscope}%
\pgfsys@transformshift{0.919700in}{2.389521in}%
\pgfsys@useobject{currentmarker}{}%
\end{pgfscope}%
\begin{pgfscope}%
\pgfsys@transformshift{0.899747in}{2.526391in}%
\pgfsys@useobject{currentmarker}{}%
\end{pgfscope}%
\begin{pgfscope}%
\pgfsys@transformshift{0.881439in}{2.623729in}%
\pgfsys@useobject{currentmarker}{}%
\end{pgfscope}%
\begin{pgfscope}%
\pgfsys@transformshift{0.861017in}{2.653376in}%
\pgfsys@useobject{currentmarker}{}%
\end{pgfscope}%
\begin{pgfscope}%
\pgfsys@transformshift{0.841535in}{2.617717in}%
\pgfsys@useobject{currentmarker}{}%
\end{pgfscope}%
\begin{pgfscope}%
\pgfsys@transformshift{0.823931in}{2.502994in}%
\pgfsys@useobject{currentmarker}{}%
\end{pgfscope}%
\begin{pgfscope}%
\pgfsys@transformshift{0.802569in}{2.380791in}%
\pgfsys@useobject{currentmarker}{}%
\end{pgfscope}%
\begin{pgfscope}%
\pgfsys@transformshift{0.784027in}{2.323016in}%
\pgfsys@useobject{currentmarker}{}%
\end{pgfscope}%
\begin{pgfscope}%
\pgfsys@transformshift{0.767125in}{2.288231in}%
\pgfsys@useobject{currentmarker}{}%
\end{pgfscope}%
\begin{pgfscope}%
\pgfsys@transformshift{0.747409in}{2.264269in}%
\pgfsys@useobject{currentmarker}{}%
\end{pgfscope}%
\begin{pgfscope}%
\pgfsys@transformshift{0.727456in}{2.251369in}%
\pgfsys@useobject{currentmarker}{}%
\end{pgfscope}%
\begin{pgfscope}%
\pgfsys@transformshift{0.707739in}{2.259084in}%
\pgfsys@useobject{currentmarker}{}%
\end{pgfscope}%
\begin{pgfscope}%
\pgfsys@transformshift{0.687318in}{2.282215in}%
\pgfsys@useobject{currentmarker}{}%
\end{pgfscope}%
\begin{pgfscope}%
\pgfsys@transformshift{0.669478in}{2.323191in}%
\pgfsys@useobject{currentmarker}{}%
\end{pgfscope}%
\begin{pgfscope}%
\pgfsys@transformshift{0.652342in}{2.386083in}%
\pgfsys@useobject{currentmarker}{}%
\end{pgfscope}%
\begin{pgfscope}%
\pgfsys@transformshift{0.651874in}{2.383743in}%
\pgfsys@useobject{currentmarker}{}%
\end{pgfscope}%
\begin{pgfscope}%
\pgfsys@transformshift{0.655865in}{2.361854in}%
\pgfsys@useobject{currentmarker}{}%
\end{pgfscope}%
\begin{pgfscope}%
\pgfsys@transformshift{0.674643in}{2.285749in}%
\pgfsys@useobject{currentmarker}{}%
\end{pgfscope}%
\begin{pgfscope}%
\pgfsys@transformshift{0.699054in}{2.252183in}%
\pgfsys@useobject{currentmarker}{}%
\end{pgfscope}%
\begin{pgfscope}%
\pgfsys@transformshift{0.712434in}{2.264644in}%
\pgfsys@useobject{currentmarker}{}%
\end{pgfscope}%
\begin{pgfscope}%
\pgfsys@transformshift{0.735203in}{2.319910in}%
\pgfsys@useobject{currentmarker}{}%
\end{pgfscope}%
\begin{pgfscope}%
\pgfsys@transformshift{0.754685in}{2.419123in}%
\pgfsys@useobject{currentmarker}{}%
\end{pgfscope}%
\begin{pgfscope}%
\pgfsys@transformshift{0.772290in}{2.588410in}%
\pgfsys@useobject{currentmarker}{}%
\end{pgfscope}%
\begin{pgfscope}%
\pgfsys@transformshift{0.791303in}{2.654966in}%
\pgfsys@useobject{currentmarker}{}%
\end{pgfscope}%
\begin{pgfscope}%
\pgfsys@transformshift{0.809846in}{2.577813in}%
\pgfsys@useobject{currentmarker}{}%
\end{pgfscope}%
\begin{pgfscope}%
\pgfsys@transformshift{0.828156in}{2.414653in}%
\pgfsys@useobject{currentmarker}{}%
\end{pgfscope}%
\begin{pgfscope}%
\pgfsys@transformshift{0.850221in}{2.298296in}%
\pgfsys@useobject{currentmarker}{}%
\end{pgfscope}%
\begin{pgfscope}%
\pgfsys@transformshift{0.868294in}{2.259745in}%
\pgfsys@useobject{currentmarker}{}%
\end{pgfscope}%
\begin{pgfscope}%
\pgfsys@transformshift{0.887776in}{2.252196in}%
\pgfsys@useobject{currentmarker}{}%
\end{pgfscope}%
\begin{pgfscope}%
\pgfsys@transformshift{0.906555in}{2.277901in}%
\pgfsys@useobject{currentmarker}{}%
\end{pgfscope}%
\begin{pgfscope}%
\pgfsys@transformshift{0.925100in}{2.336738in}%
\pgfsys@useobject{currentmarker}{}%
\end{pgfscope}%
\begin{pgfscope}%
\pgfsys@transformshift{0.943876in}{2.464390in}%
\pgfsys@useobject{currentmarker}{}%
\end{pgfscope}%
\begin{pgfscope}%
\pgfsys@transformshift{0.962655in}{2.613858in}%
\pgfsys@useobject{currentmarker}{}%
\end{pgfscope}%
\begin{pgfscope}%
\pgfsys@transformshift{0.984486in}{2.626889in}%
\pgfsys@useobject{currentmarker}{}%
\end{pgfscope}%
\begin{pgfscope}%
\pgfsys@transformshift{1.004673in}{2.494811in}%
\pgfsys@useobject{currentmarker}{}%
\end{pgfscope}%
\begin{pgfscope}%
\pgfsys@transformshift{1.022278in}{2.347596in}%
\pgfsys@useobject{currentmarker}{}%
\end{pgfscope}%
\begin{pgfscope}%
\pgfsys@transformshift{1.040586in}{2.280106in}%
\pgfsys@useobject{currentmarker}{}%
\end{pgfscope}%
\begin{pgfscope}%
\pgfsys@transformshift{1.057956in}{2.251448in}%
\pgfsys@useobject{currentmarker}{}%
\end{pgfscope}%
\begin{pgfscope}%
\pgfsys@transformshift{1.077673in}{2.254341in}%
\pgfsys@useobject{currentmarker}{}%
\end{pgfscope}%
\begin{pgfscope}%
\pgfsys@transformshift{1.095746in}{2.286864in}%
\pgfsys@useobject{currentmarker}{}%
\end{pgfscope}%
\begin{pgfscope}%
\pgfsys@transformshift{1.119454in}{2.365582in}%
\pgfsys@useobject{currentmarker}{}%
\end{pgfscope}%
\begin{pgfscope}%
\pgfsys@transformshift{1.136355in}{2.513281in}%
\pgfsys@useobject{currentmarker}{}%
\end{pgfscope}%
\begin{pgfscope}%
\pgfsys@transformshift{1.155134in}{2.626950in}%
\pgfsys@useobject{currentmarker}{}%
\end{pgfscope}%
\begin{pgfscope}%
\pgfsys@transformshift{1.173911in}{2.604702in}%
\pgfsys@useobject{currentmarker}{}%
\end{pgfscope}%
\begin{pgfscope}%
\pgfsys@transformshift{1.191987in}{2.457389in}%
\pgfsys@useobject{currentmarker}{}%
\end{pgfscope}%
\begin{pgfscope}%
\pgfsys@transformshift{1.213817in}{2.321400in}%
\pgfsys@useobject{currentmarker}{}%
\end{pgfscope}%
\begin{pgfscope}%
\pgfsys@transformshift{1.233768in}{2.266467in}%
\pgfsys@useobject{currentmarker}{}%
\end{pgfscope}%
\begin{pgfscope}%
\pgfsys@transformshift{1.252547in}{2.248382in}%
\pgfsys@useobject{currentmarker}{}%
\end{pgfscope}%
\begin{pgfscope}%
\pgfsys@transformshift{1.270151in}{2.252787in}%
\pgfsys@useobject{currentmarker}{}%
\end{pgfscope}%
\begin{pgfscope}%
\pgfsys@transformshift{1.290102in}{2.284892in}%
\pgfsys@useobject{currentmarker}{}%
\end{pgfscope}%
\begin{pgfscope}%
\pgfsys@transformshift{1.308647in}{2.349392in}%
\pgfsys@useobject{currentmarker}{}%
\end{pgfscope}%
\begin{pgfscope}%
\pgfsys@transformshift{1.326486in}{2.477066in}%
\pgfsys@useobject{currentmarker}{}%
\end{pgfscope}%
\begin{pgfscope}%
\pgfsys@transformshift{1.347376in}{2.618285in}%
\pgfsys@useobject{currentmarker}{}%
\end{pgfscope}%
\begin{pgfscope}%
\pgfsys@transformshift{1.367564in}{2.594984in}%
\pgfsys@useobject{currentmarker}{}%
\end{pgfscope}%
\begin{pgfscope}%
\pgfsys@transformshift{1.386106in}{2.448751in}%
\pgfsys@useobject{currentmarker}{}%
\end{pgfscope}%
\begin{pgfscope}%
\pgfsys@transformshift{1.404416in}{2.334618in}%
\pgfsys@useobject{currentmarker}{}%
\end{pgfscope}%
\begin{pgfscope}%
\pgfsys@transformshift{1.426246in}{2.269296in}%
\pgfsys@useobject{currentmarker}{}%
\end{pgfscope}%
\begin{pgfscope}%
\pgfsys@transformshift{1.442677in}{2.248552in}%
\pgfsys@useobject{currentmarker}{}%
\end{pgfscope}%
\begin{pgfscope}%
\pgfsys@transformshift{1.464742in}{2.254202in}%
\pgfsys@useobject{currentmarker}{}%
\end{pgfscope}%
\begin{pgfscope}%
\pgfsys@transformshift{1.481173in}{2.280545in}%
\pgfsys@useobject{currentmarker}{}%
\end{pgfscope}%
\begin{pgfscope}%
\pgfsys@transformshift{1.502298in}{2.345669in}%
\pgfsys@useobject{currentmarker}{}%
\end{pgfscope}%
\begin{pgfscope}%
\pgfsys@transformshift{1.520137in}{2.456854in}%
\pgfsys@useobject{currentmarker}{}%
\end{pgfscope}%
\begin{pgfscope}%
\pgfsys@transformshift{1.542201in}{2.600810in}%
\pgfsys@useobject{currentmarker}{}%
\end{pgfscope}%
\begin{pgfscope}%
\pgfsys@transformshift{1.558163in}{2.616670in}%
\pgfsys@useobject{currentmarker}{}%
\end{pgfscope}%
\begin{pgfscope}%
\pgfsys@transformshift{1.578819in}{2.481893in}%
\pgfsys@useobject{currentmarker}{}%
\end{pgfscope}%
\begin{pgfscope}%
\pgfsys@transformshift{1.596659in}{2.605922in}%
\pgfsys@useobject{currentmarker}{}%
\end{pgfscope}%
\begin{pgfscope}%
\pgfsys@transformshift{1.618489in}{2.609996in}%
\pgfsys@useobject{currentmarker}{}%
\end{pgfscope}%
\begin{pgfscope}%
\pgfsys@transformshift{1.636799in}{2.535406in}%
\pgfsys@useobject{currentmarker}{}%
\end{pgfscope}%
\begin{pgfscope}%
\pgfsys@transformshift{1.656281in}{2.378482in}%
\pgfsys@useobject{currentmarker}{}%
\end{pgfscope}%
\begin{pgfscope}%
\pgfsys@transformshift{1.675294in}{2.285761in}%
\pgfsys@useobject{currentmarker}{}%
\end{pgfscope}%
\begin{pgfscope}%
\pgfsys@transformshift{1.693837in}{2.255263in}%
\pgfsys@useobject{currentmarker}{}%
\end{pgfscope}%
\begin{pgfscope}%
\pgfsys@transformshift{1.714493in}{2.246507in}%
\pgfsys@useobject{currentmarker}{}%
\end{pgfscope}%
\begin{pgfscope}%
\pgfsys@transformshift{1.732098in}{2.260916in}%
\pgfsys@useobject{currentmarker}{}%
\end{pgfscope}%
\begin{pgfscope}%
\pgfsys@transformshift{1.752754in}{2.296603in}%
\pgfsys@useobject{currentmarker}{}%
\end{pgfscope}%
\begin{pgfscope}%
\pgfsys@transformshift{1.771064in}{2.375577in}%
\pgfsys@useobject{currentmarker}{}%
\end{pgfscope}%
\begin{pgfscope}%
\pgfsys@transformshift{1.789606in}{2.507877in}%
\pgfsys@useobject{currentmarker}{}%
\end{pgfscope}%
\begin{pgfscope}%
\pgfsys@transformshift{1.809794in}{2.613019in}%
\pgfsys@useobject{currentmarker}{}%
\end{pgfscope}%
\begin{pgfscope}%
\pgfsys@transformshift{1.828807in}{2.589909in}%
\pgfsys@useobject{currentmarker}{}%
\end{pgfscope}%
\begin{pgfscope}%
\pgfsys@transformshift{1.849932in}{2.456467in}%
\pgfsys@useobject{currentmarker}{}%
\end{pgfscope}%
\begin{pgfscope}%
\pgfsys@transformshift{1.868242in}{2.331562in}%
\pgfsys@useobject{currentmarker}{}%
\end{pgfscope}%
\begin{pgfscope}%
\pgfsys@transformshift{1.868476in}{2.292033in}%
\pgfsys@useobject{currentmarker}{}%
\end{pgfscope}%
\begin{pgfscope}%
\pgfsys@transformshift{1.888427in}{2.273050in}%
\pgfsys@useobject{currentmarker}{}%
\end{pgfscope}%
\begin{pgfscope}%
\pgfsys@transformshift{1.907677in}{2.249746in}%
\pgfsys@useobject{currentmarker}{}%
\end{pgfscope}%
\begin{pgfscope}%
\pgfsys@transformshift{1.924576in}{2.244942in}%
\pgfsys@useobject{currentmarker}{}%
\end{pgfscope}%
\begin{pgfscope}%
\pgfsys@transformshift{1.945232in}{2.260295in}%
\pgfsys@useobject{currentmarker}{}%
\end{pgfscope}%
\begin{pgfscope}%
\pgfsys@transformshift{1.963072in}{2.289979in}%
\pgfsys@useobject{currentmarker}{}%
\end{pgfscope}%
\begin{pgfscope}%
\pgfsys@transformshift{1.981616in}{2.354705in}%
\pgfsys@useobject{currentmarker}{}%
\end{pgfscope}%
\begin{pgfscope}%
\pgfsys@transformshift{2.002507in}{2.509041in}%
\pgfsys@useobject{currentmarker}{}%
\end{pgfscope}%
\begin{pgfscope}%
\pgfsys@transformshift{2.023866in}{2.613492in}%
\pgfsys@useobject{currentmarker}{}%
\end{pgfscope}%
\begin{pgfscope}%
\pgfsys@transformshift{2.037951in}{2.595528in}%
\pgfsys@useobject{currentmarker}{}%
\end{pgfscope}%
\begin{pgfscope}%
\pgfsys@transformshift{2.059546in}{2.500690in}%
\pgfsys@useobject{currentmarker}{}%
\end{pgfscope}%
\begin{pgfscope}%
\pgfsys@transformshift{2.077386in}{2.370074in}%
\pgfsys@useobject{currentmarker}{}%
\end{pgfscope}%
\begin{pgfscope}%
\pgfsys@transformshift{2.098276in}{2.285517in}%
\pgfsys@useobject{currentmarker}{}%
\end{pgfscope}%
\begin{pgfscope}%
\pgfsys@transformshift{2.117524in}{2.255888in}%
\pgfsys@useobject{currentmarker}{}%
\end{pgfscope}%
\begin{pgfscope}%
\pgfsys@transformshift{2.140763in}{2.245304in}%
\pgfsys@useobject{currentmarker}{}%
\end{pgfscope}%
\begin{pgfscope}%
\pgfsys@transformshift{2.155550in}{2.249915in}%
\pgfsys@useobject{currentmarker}{}%
\end{pgfscope}%
\begin{pgfscope}%
\pgfsys@transformshift{2.175501in}{2.270472in}%
\pgfsys@useobject{currentmarker}{}%
\end{pgfscope}%
\begin{pgfscope}%
\pgfsys@transformshift{2.193577in}{2.314660in}%
\pgfsys@useobject{currentmarker}{}%
\end{pgfscope}%
\begin{pgfscope}%
\pgfsys@transformshift{2.214936in}{2.407980in}%
\pgfsys@useobject{currentmarker}{}%
\end{pgfscope}%
\begin{pgfscope}%
\pgfsys@transformshift{2.232776in}{2.549192in}%
\pgfsys@useobject{currentmarker}{}%
\end{pgfscope}%
\begin{pgfscope}%
\pgfsys@transformshift{2.251086in}{2.568601in}%
\pgfsys@useobject{currentmarker}{}%
\end{pgfscope}%
\begin{pgfscope}%
\pgfsys@transformshift{2.275028in}{2.601921in}%
\pgfsys@useobject{currentmarker}{}%
\end{pgfscope}%
\begin{pgfscope}%
\pgfsys@transformshift{2.290050in}{2.513887in}%
\pgfsys@useobject{currentmarker}{}%
\end{pgfscope}%
\begin{pgfscope}%
\pgfsys@transformshift{2.307889in}{2.378302in}%
\pgfsys@useobject{currentmarker}{}%
\end{pgfscope}%
\begin{pgfscope}%
\pgfsys@transformshift{2.328311in}{2.291553in}%
\pgfsys@useobject{currentmarker}{}%
\end{pgfscope}%
\begin{pgfscope}%
\pgfsys@transformshift{2.345916in}{2.267073in}%
\pgfsys@useobject{currentmarker}{}%
\end{pgfscope}%
\begin{pgfscope}%
\pgfsys@transformshift{2.367511in}{2.247260in}%
\pgfsys@useobject{currentmarker}{}%
\end{pgfscope}%
\begin{pgfscope}%
\pgfsys@transformshift{2.384880in}{2.245855in}%
\pgfsys@useobject{currentmarker}{}%
\end{pgfscope}%
\begin{pgfscope}%
\pgfsys@transformshift{2.406944in}{2.263007in}%
\pgfsys@useobject{currentmarker}{}%
\end{pgfscope}%
\begin{pgfscope}%
\pgfsys@transformshift{2.424315in}{2.297876in}%
\pgfsys@useobject{currentmarker}{}%
\end{pgfscope}%
\begin{pgfscope}%
\pgfsys@transformshift{2.444033in}{2.375889in}%
\pgfsys@useobject{currentmarker}{}%
\end{pgfscope}%
\begin{pgfscope}%
\pgfsys@transformshift{2.462341in}{2.492766in}%
\pgfsys@useobject{currentmarker}{}%
\end{pgfscope}%
\begin{pgfscope}%
\pgfsys@transformshift{2.481355in}{2.602418in}%
\pgfsys@useobject{currentmarker}{}%
\end{pgfscope}%
\begin{pgfscope}%
\pgfsys@transformshift{2.502950in}{2.600944in}%
\pgfsys@useobject{currentmarker}{}%
\end{pgfscope}%
\begin{pgfscope}%
\pgfsys@transformshift{2.521493in}{2.493902in}%
\pgfsys@useobject{currentmarker}{}%
\end{pgfscope}%
\begin{pgfscope}%
\pgfsys@transformshift{2.542149in}{2.346614in}%
\pgfsys@useobject{currentmarker}{}%
\end{pgfscope}%
\begin{pgfscope}%
\pgfsys@transformshift{2.558580in}{2.290008in}%
\pgfsys@useobject{currentmarker}{}%
\end{pgfscope}%
\begin{pgfscope}%
\pgfsys@transformshift{2.578767in}{2.255698in}%
\pgfsys@useobject{currentmarker}{}%
\end{pgfscope}%
\begin{pgfscope}%
\pgfsys@transformshift{2.596372in}{2.263896in}%
\pgfsys@useobject{currentmarker}{}%
\end{pgfscope}%
\begin{pgfscope}%
\pgfsys@transformshift{2.613977in}{2.255128in}%
\pgfsys@useobject{currentmarker}{}%
\end{pgfscope}%
\begin{pgfscope}%
\pgfsys@transformshift{2.635336in}{2.244323in}%
\pgfsys@useobject{currentmarker}{}%
\end{pgfscope}%
\begin{pgfscope}%
\pgfsys@transformshift{2.653412in}{2.253256in}%
\pgfsys@useobject{currentmarker}{}%
\end{pgfscope}%
\begin{pgfscope}%
\pgfsys@transformshift{2.674068in}{2.283589in}%
\pgfsys@useobject{currentmarker}{}%
\end{pgfscope}%
\begin{pgfscope}%
\pgfsys@transformshift{2.692376in}{2.340717in}%
\pgfsys@useobject{currentmarker}{}%
\end{pgfscope}%
\begin{pgfscope}%
\pgfsys@transformshift{2.712797in}{2.467329in}%
\pgfsys@useobject{currentmarker}{}%
\end{pgfscope}%
\begin{pgfscope}%
\pgfsys@transformshift{2.731576in}{2.574548in}%
\pgfsys@useobject{currentmarker}{}%
\end{pgfscope}%
\begin{pgfscope}%
\pgfsys@transformshift{2.753641in}{2.610131in}%
\pgfsys@useobject{currentmarker}{}%
\end{pgfscope}%
\begin{pgfscope}%
\pgfsys@transformshift{2.773592in}{2.498302in}%
\pgfsys@useobject{currentmarker}{}%
\end{pgfscope}%
\begin{pgfscope}%
\pgfsys@transformshift{2.788850in}{2.491324in}%
\pgfsys@useobject{currentmarker}{}%
\end{pgfscope}%
\begin{pgfscope}%
\pgfsys@transformshift{2.810444in}{2.339459in}%
\pgfsys@useobject{currentmarker}{}%
\end{pgfscope}%
\begin{pgfscope}%
\pgfsys@transformshift{2.829458in}{2.278299in}%
\pgfsys@useobject{currentmarker}{}%
\end{pgfscope}%
\begin{pgfscope}%
\pgfsys@transformshift{2.844951in}{2.256108in}%
\pgfsys@useobject{currentmarker}{}%
\end{pgfscope}%
\begin{pgfscope}%
\pgfsys@transformshift{2.866544in}{2.245382in}%
\pgfsys@useobject{currentmarker}{}%
\end{pgfscope}%
\begin{pgfscope}%
\pgfsys@transformshift{2.888375in}{2.256694in}%
\pgfsys@useobject{currentmarker}{}%
\end{pgfscope}%
\begin{pgfscope}%
\pgfsys@transformshift{2.903163in}{2.277255in}%
\pgfsys@useobject{currentmarker}{}%
\end{pgfscope}%
\begin{pgfscope}%
\pgfsys@transformshift{2.924524in}{2.321045in}%
\pgfsys@useobject{currentmarker}{}%
\end{pgfscope}%
\begin{pgfscope}%
\pgfsys@transformshift{2.942363in}{2.405136in}%
\pgfsys@useobject{currentmarker}{}%
\end{pgfscope}%
\begin{pgfscope}%
\pgfsys@transformshift{2.963488in}{2.549339in}%
\pgfsys@useobject{currentmarker}{}%
\end{pgfscope}%
\begin{pgfscope}%
\pgfsys@transformshift{2.981562in}{2.598635in}%
\pgfsys@useobject{currentmarker}{}%
\end{pgfscope}%
\begin{pgfscope}%
\pgfsys@transformshift{2.998698in}{2.609947in}%
\pgfsys@useobject{currentmarker}{}%
\end{pgfscope}%
\begin{pgfscope}%
\pgfsys@transformshift{3.019588in}{2.497787in}%
\pgfsys@useobject{currentmarker}{}%
\end{pgfscope}%
\begin{pgfscope}%
\pgfsys@transformshift{3.037427in}{2.375609in}%
\pgfsys@useobject{currentmarker}{}%
\end{pgfscope}%
\begin{pgfscope}%
\pgfsys@transformshift{3.056206in}{2.304526in}%
\pgfsys@useobject{currentmarker}{}%
\end{pgfscope}%
\begin{pgfscope}%
\pgfsys@transformshift{3.076862in}{2.269131in}%
\pgfsys@useobject{currentmarker}{}%
\end{pgfscope}%
\begin{pgfscope}%
\pgfsys@transformshift{3.094936in}{2.249715in}%
\pgfsys@useobject{currentmarker}{}%
\end{pgfscope}%
\begin{pgfscope}%
\pgfsys@transformshift{3.116297in}{2.469258in}%
\pgfsys@useobject{currentmarker}{}%
\end{pgfscope}%
\begin{pgfscope}%
\pgfsys@transformshift{3.135311in}{2.528157in}%
\pgfsys@useobject{currentmarker}{}%
\end{pgfscope}%
\begin{pgfscope}%
\pgfsys@transformshift{3.156201in}{2.385961in}%
\pgfsys@useobject{currentmarker}{}%
\end{pgfscope}%
\begin{pgfscope}%
\pgfsys@transformshift{3.176389in}{2.292972in}%
\pgfsys@useobject{currentmarker}{}%
\end{pgfscope}%
\begin{pgfscope}%
\pgfsys@transformshift{3.191411in}{2.269892in}%
\pgfsys@useobject{currentmarker}{}%
\end{pgfscope}%
\begin{pgfscope}%
\pgfsys@transformshift{3.214884in}{2.245715in}%
\pgfsys@useobject{currentmarker}{}%
\end{pgfscope}%
\begin{pgfscope}%
\pgfsys@transformshift{3.229906in}{2.248806in}%
\pgfsys@useobject{currentmarker}{}%
\end{pgfscope}%
\begin{pgfscope}%
\pgfsys@transformshift{3.251031in}{2.270082in}%
\pgfsys@useobject{currentmarker}{}%
\end{pgfscope}%
\begin{pgfscope}%
\pgfsys@transformshift{3.269341in}{2.311229in}%
\pgfsys@useobject{currentmarker}{}%
\end{pgfscope}%
\begin{pgfscope}%
\pgfsys@transformshift{3.293049in}{2.456423in}%
\pgfsys@useobject{currentmarker}{}%
\end{pgfscope}%
\begin{pgfscope}%
\pgfsys@transformshift{3.308071in}{2.563022in}%
\pgfsys@useobject{currentmarker}{}%
\end{pgfscope}%
\begin{pgfscope}%
\pgfsys@transformshift{3.327319in}{2.624542in}%
\pgfsys@useobject{currentmarker}{}%
\end{pgfscope}%
\begin{pgfscope}%
\pgfsys@transformshift{3.345863in}{2.596319in}%
\pgfsys@useobject{currentmarker}{}%
\end{pgfscope}%
\begin{pgfscope}%
\pgfsys@transformshift{3.366285in}{2.488652in}%
\pgfsys@useobject{currentmarker}{}%
\end{pgfscope}%
\begin{pgfscope}%
\pgfsys@transformshift{3.384124in}{2.357246in}%
\pgfsys@useobject{currentmarker}{}%
\end{pgfscope}%
\begin{pgfscope}%
\pgfsys@transformshift{3.405015in}{2.285464in}%
\pgfsys@useobject{currentmarker}{}%
\end{pgfscope}%
\begin{pgfscope}%
\pgfsys@transformshift{3.423559in}{2.259583in}%
\pgfsys@useobject{currentmarker}{}%
\end{pgfscope}%
\begin{pgfscope}%
\pgfsys@transformshift{3.441162in}{2.246931in}%
\pgfsys@useobject{currentmarker}{}%
\end{pgfscope}%
\begin{pgfscope}%
\pgfsys@transformshift{3.462758in}{2.261111in}%
\pgfsys@useobject{currentmarker}{}%
\end{pgfscope}%
\begin{pgfscope}%
\pgfsys@transformshift{3.480362in}{2.290483in}%
\pgfsys@useobject{currentmarker}{}%
\end{pgfscope}%
\begin{pgfscope}%
\pgfsys@transformshift{3.501019in}{2.356920in}%
\pgfsys@useobject{currentmarker}{}%
\end{pgfscope}%
\begin{pgfscope}%
\pgfsys@transformshift{3.517449in}{2.401118in}%
\pgfsys@useobject{currentmarker}{}%
\end{pgfscope}%
\begin{pgfscope}%
\pgfsys@transformshift{3.537168in}{2.529095in}%
\pgfsys@useobject{currentmarker}{}%
\end{pgfscope}%
\begin{pgfscope}%
\pgfsys@transformshift{3.559232in}{2.628723in}%
\pgfsys@useobject{currentmarker}{}%
\end{pgfscope}%
\begin{pgfscope}%
\pgfsys@transformshift{3.579652in}{2.597605in}%
\pgfsys@useobject{currentmarker}{}%
\end{pgfscope}%
\begin{pgfscope}%
\pgfsys@transformshift{3.598197in}{2.496175in}%
\pgfsys@useobject{currentmarker}{}%
\end{pgfscope}%
\begin{pgfscope}%
\pgfsys@transformshift{3.616036in}{2.375025in}%
\pgfsys@useobject{currentmarker}{}%
\end{pgfscope}%
\begin{pgfscope}%
\pgfsys@transformshift{3.633641in}{2.312425in}%
\pgfsys@useobject{currentmarker}{}%
\end{pgfscope}%
\begin{pgfscope}%
\pgfsys@transformshift{3.654766in}{2.271025in}%
\pgfsys@useobject{currentmarker}{}%
\end{pgfscope}%
\begin{pgfscope}%
\pgfsys@transformshift{3.675893in}{2.250531in}%
\pgfsys@useobject{currentmarker}{}%
\end{pgfscope}%
\begin{pgfscope}%
\pgfsys@transformshift{3.690446in}{2.248507in}%
\pgfsys@useobject{currentmarker}{}%
\end{pgfscope}%
\begin{pgfscope}%
\pgfsys@transformshift{3.712040in}{2.270188in}%
\pgfsys@useobject{currentmarker}{}%
\end{pgfscope}%
\begin{pgfscope}%
\pgfsys@transformshift{3.731287in}{2.297643in}%
\pgfsys@useobject{currentmarker}{}%
\end{pgfscope}%
\begin{pgfscope}%
\pgfsys@transformshift{3.751006in}{2.360360in}%
\pgfsys@useobject{currentmarker}{}%
\end{pgfscope}%
\begin{pgfscope}%
\pgfsys@transformshift{3.769785in}{2.459818in}%
\pgfsys@useobject{currentmarker}{}%
\end{pgfscope}%
\begin{pgfscope}%
\pgfsys@transformshift{3.789970in}{2.590007in}%
\pgfsys@useobject{currentmarker}{}%
\end{pgfscope}%
\begin{pgfscope}%
\pgfsys@transformshift{3.808280in}{2.641389in}%
\pgfsys@useobject{currentmarker}{}%
\end{pgfscope}%
\begin{pgfscope}%
\pgfsys@transformshift{3.827528in}{2.619436in}%
\pgfsys@useobject{currentmarker}{}%
\end{pgfscope}%
\begin{pgfscope}%
\pgfsys@transformshift{3.845601in}{2.509919in}%
\pgfsys@useobject{currentmarker}{}%
\end{pgfscope}%
\begin{pgfscope}%
\pgfsys@transformshift{3.865789in}{2.370595in}%
\pgfsys@useobject{currentmarker}{}%
\end{pgfscope}%
\begin{pgfscope}%
\pgfsys@transformshift{3.884802in}{2.316909in}%
\pgfsys@useobject{currentmarker}{}%
\end{pgfscope}%
\begin{pgfscope}%
\pgfsys@transformshift{3.903110in}{2.278095in}%
\pgfsys@useobject{currentmarker}{}%
\end{pgfscope}%
\begin{pgfscope}%
\pgfsys@transformshift{3.922358in}{2.260559in}%
\pgfsys@useobject{currentmarker}{}%
\end{pgfscope}%
\begin{pgfscope}%
\pgfsys@transformshift{3.940431in}{2.249722in}%
\pgfsys@useobject{currentmarker}{}%
\end{pgfscope}%
\begin{pgfscope}%
\pgfsys@transformshift{3.961324in}{2.254553in}%
\pgfsys@useobject{currentmarker}{}%
\end{pgfscope}%
\begin{pgfscope}%
\pgfsys@transformshift{3.980572in}{2.276045in}%
\pgfsys@useobject{currentmarker}{}%
\end{pgfscope}%
\begin{pgfscope}%
\pgfsys@transformshift{4.002400in}{2.327290in}%
\pgfsys@useobject{currentmarker}{}%
\end{pgfscope}%
\begin{pgfscope}%
\pgfsys@transformshift{4.020241in}{2.398539in}%
\pgfsys@useobject{currentmarker}{}%
\end{pgfscope}%
\begin{pgfscope}%
\pgfsys@transformshift{4.039018in}{2.507527in}%
\pgfsys@useobject{currentmarker}{}%
\end{pgfscope}%
\begin{pgfscope}%
\pgfsys@transformshift{4.057797in}{2.609434in}%
\pgfsys@useobject{currentmarker}{}%
\end{pgfscope}%
\begin{pgfscope}%
\pgfsys@transformshift{4.076576in}{2.648799in}%
\pgfsys@useobject{currentmarker}{}%
\end{pgfscope}%
\begin{pgfscope}%
\pgfsys@transformshift{4.095589in}{2.649379in}%
\pgfsys@useobject{currentmarker}{}%
\end{pgfscope}%
\begin{pgfscope}%
\pgfsys@transformshift{4.117888in}{2.582737in}%
\pgfsys@useobject{currentmarker}{}%
\end{pgfscope}%
\begin{pgfscope}%
\pgfsys@transformshift{4.135024in}{2.475367in}%
\pgfsys@useobject{currentmarker}{}%
\end{pgfscope}%
\begin{pgfscope}%
\pgfsys@transformshift{4.153332in}{2.381330in}%
\pgfsys@useobject{currentmarker}{}%
\end{pgfscope}%
\begin{pgfscope}%
\pgfsys@transformshift{4.172345in}{2.311274in}%
\pgfsys@useobject{currentmarker}{}%
\end{pgfscope}%
\begin{pgfscope}%
\pgfsys@transformshift{4.190653in}{2.286235in}%
\pgfsys@useobject{currentmarker}{}%
\end{pgfscope}%
\begin{pgfscope}%
\pgfsys@transformshift{4.213189in}{2.257080in}%
\pgfsys@useobject{currentmarker}{}%
\end{pgfscope}%
\begin{pgfscope}%
\pgfsys@transformshift{4.232671in}{2.252455in}%
\pgfsys@useobject{currentmarker}{}%
\end{pgfscope}%
\begin{pgfscope}%
\pgfsys@transformshift{4.251213in}{2.260060in}%
\pgfsys@useobject{currentmarker}{}%
\end{pgfscope}%
\begin{pgfscope}%
\pgfsys@transformshift{4.269523in}{2.287029in}%
\pgfsys@useobject{currentmarker}{}%
\end{pgfscope}%
\begin{pgfscope}%
\pgfsys@transformshift{4.293231in}{2.344671in}%
\pgfsys@useobject{currentmarker}{}%
\end{pgfscope}%
\begin{pgfscope}%
\pgfsys@transformshift{4.308487in}{2.393888in}%
\pgfsys@useobject{currentmarker}{}%
\end{pgfscope}%
\begin{pgfscope}%
\pgfsys@transformshift{4.327266in}{2.500854in}%
\pgfsys@useobject{currentmarker}{}%
\end{pgfscope}%
\begin{pgfscope}%
\pgfsys@transformshift{4.346748in}{2.608357in}%
\pgfsys@useobject{currentmarker}{}%
\end{pgfscope}%
\begin{pgfscope}%
\pgfsys@transformshift{4.365293in}{2.660720in}%
\pgfsys@useobject{currentmarker}{}%
\end{pgfscope}%
\begin{pgfscope}%
\pgfsys@transformshift{4.383835in}{2.650593in}%
\pgfsys@useobject{currentmarker}{}%
\end{pgfscope}%
\begin{pgfscope}%
\pgfsys@transformshift{4.403554in}{2.263225in}%
\pgfsys@useobject{currentmarker}{}%
\end{pgfscope}%
\begin{pgfscope}%
\pgfsys@transformshift{4.422096in}{2.295733in}%
\pgfsys@useobject{currentmarker}{}%
\end{pgfscope}%
\begin{pgfscope}%
\pgfsys@transformshift{4.440875in}{2.366121in}%
\pgfsys@useobject{currentmarker}{}%
\end{pgfscope}%
\begin{pgfscope}%
\pgfsys@transformshift{4.460357in}{2.492200in}%
\pgfsys@useobject{currentmarker}{}%
\end{pgfscope}%
\begin{pgfscope}%
\pgfsys@transformshift{4.479136in}{2.633483in}%
\pgfsys@useobject{currentmarker}{}%
\end{pgfscope}%
\begin{pgfscope}%
\pgfsys@transformshift{4.481013in}{2.638480in}%
\pgfsys@useobject{currentmarker}{}%
\end{pgfscope}%
\begin{pgfscope}%
\pgfsys@transformshift{4.475145in}{2.603682in}%
\pgfsys@useobject{currentmarker}{}%
\end{pgfscope}%
\begin{pgfscope}%
\pgfsys@transformshift{4.452143in}{2.391916in}%
\pgfsys@useobject{currentmarker}{}%
\end{pgfscope}%
\begin{pgfscope}%
\pgfsys@transformshift{4.435007in}{2.304330in}%
\pgfsys@useobject{currentmarker}{}%
\end{pgfscope}%
\begin{pgfscope}%
\pgfsys@transformshift{4.416699in}{2.261908in}%
\pgfsys@useobject{currentmarker}{}%
\end{pgfscope}%
\begin{pgfscope}%
\pgfsys@transformshift{4.398154in}{2.253373in}%
\pgfsys@useobject{currentmarker}{}%
\end{pgfscope}%
\begin{pgfscope}%
\pgfsys@transformshift{4.376795in}{2.286320in}%
\pgfsys@useobject{currentmarker}{}%
\end{pgfscope}%
\begin{pgfscope}%
\pgfsys@transformshift{4.359424in}{2.364307in}%
\pgfsys@useobject{currentmarker}{}%
\end{pgfscope}%
\begin{pgfscope}%
\pgfsys@transformshift{4.339942in}{2.532908in}%
\pgfsys@useobject{currentmarker}{}%
\end{pgfscope}%
\begin{pgfscope}%
\pgfsys@transformshift{4.317641in}{2.653920in}%
\pgfsys@useobject{currentmarker}{}%
\end{pgfscope}%
\begin{pgfscope}%
\pgfsys@transformshift{4.303090in}{2.626235in}%
\pgfsys@useobject{currentmarker}{}%
\end{pgfscope}%
\begin{pgfscope}%
\pgfsys@transformshift{4.278208in}{2.409531in}%
\pgfsys@useobject{currentmarker}{}%
\end{pgfscope}%
\begin{pgfscope}%
\pgfsys@transformshift{4.263889in}{2.322737in}%
\pgfsys@useobject{currentmarker}{}%
\end{pgfscope}%
\begin{pgfscope}%
\pgfsys@transformshift{4.244407in}{2.269488in}%
\pgfsys@useobject{currentmarker}{}%
\end{pgfscope}%
\begin{pgfscope}%
\pgfsys@transformshift{4.226097in}{2.249624in}%
\pgfsys@useobject{currentmarker}{}%
\end{pgfscope}%
\begin{pgfscope}%
\pgfsys@transformshift{4.205675in}{2.268666in}%
\pgfsys@useobject{currentmarker}{}%
\end{pgfscope}%
\begin{pgfscope}%
\pgfsys@transformshift{4.188071in}{2.319808in}%
\pgfsys@useobject{currentmarker}{}%
\end{pgfscope}%
\begin{pgfscope}%
\pgfsys@transformshift{4.166477in}{2.461349in}%
\pgfsys@useobject{currentmarker}{}%
\end{pgfscope}%
\begin{pgfscope}%
\pgfsys@transformshift{4.149812in}{2.607851in}%
\pgfsys@useobject{currentmarker}{}%
\end{pgfscope}%
\begin{pgfscope}%
\pgfsys@transformshift{4.127982in}{2.637343in}%
\pgfsys@useobject{currentmarker}{}%
\end{pgfscope}%
\begin{pgfscope}%
\pgfsys@transformshift{4.109437in}{2.505184in}%
\pgfsys@useobject{currentmarker}{}%
\end{pgfscope}%
\begin{pgfscope}%
\pgfsys@transformshift{4.088781in}{2.345664in}%
\pgfsys@useobject{currentmarker}{}%
\end{pgfscope}%
\begin{pgfscope}%
\pgfsys@transformshift{4.067890in}{2.278693in}%
\pgfsys@useobject{currentmarker}{}%
\end{pgfscope}%
\begin{pgfscope}%
\pgfsys@transformshift{4.050286in}{2.256154in}%
\pgfsys@useobject{currentmarker}{}%
\end{pgfscope}%
\begin{pgfscope}%
\pgfsys@transformshift{4.033150in}{2.249910in}%
\pgfsys@useobject{currentmarker}{}%
\end{pgfscope}%
\begin{pgfscope}%
\pgfsys@transformshift{4.014373in}{2.274826in}%
\pgfsys@useobject{currentmarker}{}%
\end{pgfscope}%
\begin{pgfscope}%
\pgfsys@transformshift{3.993951in}{2.333033in}%
\pgfsys@useobject{currentmarker}{}%
\end{pgfscope}%
\begin{pgfscope}%
\pgfsys@transformshift{3.975875in}{2.456652in}%
\pgfsys@useobject{currentmarker}{}%
\end{pgfscope}%
\begin{pgfscope}%
\pgfsys@transformshift{3.954750in}{2.615801in}%
\pgfsys@useobject{currentmarker}{}%
\end{pgfscope}%
\begin{pgfscope}%
\pgfsys@transformshift{3.933625in}{2.617307in}%
\pgfsys@useobject{currentmarker}{}%
\end{pgfscope}%
\begin{pgfscope}%
\pgfsys@transformshift{3.916255in}{2.479488in}%
\pgfsys@useobject{currentmarker}{}%
\end{pgfscope}%
\begin{pgfscope}%
\pgfsys@transformshift{3.897947in}{2.344098in}%
\pgfsys@useobject{currentmarker}{}%
\end{pgfscope}%
\begin{pgfscope}%
\pgfsys@transformshift{3.878934in}{2.281823in}%
\pgfsys@useobject{currentmarker}{}%
\end{pgfscope}%
\begin{pgfscope}%
\pgfsys@transformshift{3.854990in}{2.249229in}%
\pgfsys@useobject{currentmarker}{}%
\end{pgfscope}%
\begin{pgfscope}%
\pgfsys@transformshift{3.839030in}{2.249568in}%
\pgfsys@useobject{currentmarker}{}%
\end{pgfscope}%
\begin{pgfscope}%
\pgfsys@transformshift{3.819782in}{2.450461in}%
\pgfsys@useobject{currentmarker}{}%
\end{pgfscope}%
\begin{pgfscope}%
\pgfsys@transformshift{3.801472in}{2.328223in}%
\pgfsys@useobject{currentmarker}{}%
\end{pgfscope}%
\begin{pgfscope}%
\pgfsys@transformshift{3.782693in}{2.273181in}%
\pgfsys@useobject{currentmarker}{}%
\end{pgfscope}%
\begin{pgfscope}%
\pgfsys@transformshift{3.765559in}{2.250803in}%
\pgfsys@useobject{currentmarker}{}%
\end{pgfscope}%
\begin{pgfscope}%
\pgfsys@transformshift{3.743495in}{2.255078in}%
\pgfsys@useobject{currentmarker}{}%
\end{pgfscope}%
\begin{pgfscope}%
\pgfsys@transformshift{3.725656in}{2.285701in}%
\pgfsys@useobject{currentmarker}{}%
\end{pgfscope}%
\begin{pgfscope}%
\pgfsys@transformshift{3.705234in}{2.373668in}%
\pgfsys@useobject{currentmarker}{}%
\end{pgfscope}%
\begin{pgfscope}%
\pgfsys@transformshift{3.685986in}{2.530484in}%
\pgfsys@useobject{currentmarker}{}%
\end{pgfscope}%
\begin{pgfscope}%
\pgfsys@transformshift{3.666738in}{2.618384in}%
\pgfsys@useobject{currentmarker}{}%
\end{pgfscope}%
\begin{pgfscope}%
\pgfsys@transformshift{3.648428in}{2.600856in}%
\pgfsys@useobject{currentmarker}{}%
\end{pgfscope}%
\begin{pgfscope}%
\pgfsys@transformshift{3.629415in}{2.466920in}%
\pgfsys@useobject{currentmarker}{}%
\end{pgfscope}%
\begin{pgfscope}%
\pgfsys@transformshift{3.607351in}{2.328095in}%
\pgfsys@useobject{currentmarker}{}%
\end{pgfscope}%
\begin{pgfscope}%
\pgfsys@transformshift{3.591859in}{2.280276in}%
\pgfsys@useobject{currentmarker}{}%
\end{pgfscope}%
\begin{pgfscope}%
\pgfsys@transformshift{3.569560in}{2.250943in}%
\pgfsys@useobject{currentmarker}{}%
\end{pgfscope}%
\begin{pgfscope}%
\pgfsys@transformshift{3.551250in}{2.250335in}%
\pgfsys@useobject{currentmarker}{}%
\end{pgfscope}%
\begin{pgfscope}%
\pgfsys@transformshift{3.529420in}{2.271621in}%
\pgfsys@useobject{currentmarker}{}%
\end{pgfscope}%
\begin{pgfscope}%
\pgfsys@transformshift{3.514398in}{2.312332in}%
\pgfsys@useobject{currentmarker}{}%
\end{pgfscope}%
\begin{pgfscope}%
\pgfsys@transformshift{3.496324in}{2.396772in}%
\pgfsys@useobject{currentmarker}{}%
\end{pgfscope}%
\begin{pgfscope}%
\pgfsys@transformshift{3.474494in}{2.525920in}%
\pgfsys@useobject{currentmarker}{}%
\end{pgfscope}%
\begin{pgfscope}%
\pgfsys@transformshift{3.455481in}{2.618956in}%
\pgfsys@useobject{currentmarker}{}%
\end{pgfscope}%
\begin{pgfscope}%
\pgfsys@transformshift{3.436702in}{2.583964in}%
\pgfsys@useobject{currentmarker}{}%
\end{pgfscope}%
\begin{pgfscope}%
\pgfsys@transformshift{3.418160in}{2.499375in}%
\pgfsys@useobject{currentmarker}{}%
\end{pgfscope}%
\begin{pgfscope}%
\pgfsys@transformshift{3.398912in}{2.364299in}%
\pgfsys@useobject{currentmarker}{}%
\end{pgfscope}%
\begin{pgfscope}%
\pgfsys@transformshift{3.376847in}{2.285269in}%
\pgfsys@useobject{currentmarker}{}%
\end{pgfscope}%
\begin{pgfscope}%
\pgfsys@transformshift{3.359008in}{2.254625in}%
\pgfsys@useobject{currentmarker}{}%
\end{pgfscope}%
\begin{pgfscope}%
\pgfsys@transformshift{3.340464in}{2.245038in}%
\pgfsys@useobject{currentmarker}{}%
\end{pgfscope}%
\begin{pgfscope}%
\pgfsys@transformshift{3.321685in}{2.258016in}%
\pgfsys@useobject{currentmarker}{}%
\end{pgfscope}%
\begin{pgfscope}%
\pgfsys@transformshift{3.302672in}{2.290445in}%
\pgfsys@useobject{currentmarker}{}%
\end{pgfscope}%
\begin{pgfscope}%
\pgfsys@transformshift{3.278261in}{2.381693in}%
\pgfsys@useobject{currentmarker}{}%
\end{pgfscope}%
\begin{pgfscope}%
\pgfsys@transformshift{3.265350in}{2.477521in}%
\pgfsys@useobject{currentmarker}{}%
\end{pgfscope}%
\begin{pgfscope}%
\pgfsys@transformshift{3.243991in}{2.375478in}%
\pgfsys@useobject{currentmarker}{}%
\end{pgfscope}%
\begin{pgfscope}%
\pgfsys@transformshift{3.226620in}{2.495291in}%
\pgfsys@useobject{currentmarker}{}%
\end{pgfscope}%
\begin{pgfscope}%
\pgfsys@transformshift{3.203851in}{2.614547in}%
\pgfsys@useobject{currentmarker}{}%
\end{pgfscope}%
\begin{pgfscope}%
\pgfsys@transformshift{3.185074in}{2.581068in}%
\pgfsys@useobject{currentmarker}{}%
\end{pgfscope}%
\begin{pgfscope}%
\pgfsys@transformshift{3.169346in}{2.471990in}%
\pgfsys@useobject{currentmarker}{}%
\end{pgfscope}%
\begin{pgfscope}%
\pgfsys@transformshift{3.147985in}{2.338883in}%
\pgfsys@useobject{currentmarker}{}%
\end{pgfscope}%
\begin{pgfscope}%
\pgfsys@transformshift{3.127565in}{2.277680in}%
\pgfsys@useobject{currentmarker}{}%
\end{pgfscope}%
\begin{pgfscope}%
\pgfsys@transformshift{3.109255in}{2.252307in}%
\pgfsys@useobject{currentmarker}{}%
\end{pgfscope}%
\begin{pgfscope}%
\pgfsys@transformshift{3.091416in}{2.245155in}%
\pgfsys@useobject{currentmarker}{}%
\end{pgfscope}%
\begin{pgfscope}%
\pgfsys@transformshift{3.071699in}{2.260206in}%
\pgfsys@useobject{currentmarker}{}%
\end{pgfscope}%
\begin{pgfscope}%
\pgfsys@transformshift{3.051512in}{2.289823in}%
\pgfsys@useobject{currentmarker}{}%
\end{pgfscope}%
\begin{pgfscope}%
\pgfsys@transformshift{3.032968in}{2.363584in}%
\pgfsys@useobject{currentmarker}{}%
\end{pgfscope}%
\begin{pgfscope}%
\pgfsys@transformshift{3.014191in}{2.499479in}%
\pgfsys@useobject{currentmarker}{}%
\end{pgfscope}%
\begin{pgfscope}%
\pgfsys@transformshift{2.997055in}{2.606039in}%
\pgfsys@useobject{currentmarker}{}%
\end{pgfscope}%
\begin{pgfscope}%
\pgfsys@transformshift{2.978276in}{2.603310in}%
\pgfsys@useobject{currentmarker}{}%
\end{pgfscope}%
\begin{pgfscope}%
\pgfsys@transformshift{2.957151in}{2.509399in}%
\pgfsys@useobject{currentmarker}{}%
\end{pgfscope}%
\begin{pgfscope}%
\pgfsys@transformshift{2.936729in}{2.363673in}%
\pgfsys@useobject{currentmarker}{}%
\end{pgfscope}%
\begin{pgfscope}%
\pgfsys@transformshift{2.917482in}{2.301352in}%
\pgfsys@useobject{currentmarker}{}%
\end{pgfscope}%
\begin{pgfscope}%
\pgfsys@transformshift{2.901051in}{2.268287in}%
\pgfsys@useobject{currentmarker}{}%
\end{pgfscope}%
\begin{pgfscope}%
\pgfsys@transformshift{2.878752in}{2.246936in}%
\pgfsys@useobject{currentmarker}{}%
\end{pgfscope}%
\begin{pgfscope}%
\pgfsys@transformshift{2.857156in}{2.247673in}%
\pgfsys@useobject{currentmarker}{}%
\end{pgfscope}%
\begin{pgfscope}%
\pgfsys@transformshift{2.839082in}{2.264581in}%
\pgfsys@useobject{currentmarker}{}%
\end{pgfscope}%
\begin{pgfscope}%
\pgfsys@transformshift{2.820304in}{2.305611in}%
\pgfsys@useobject{currentmarker}{}%
\end{pgfscope}%
\begin{pgfscope}%
\pgfsys@transformshift{2.800821in}{2.390378in}%
\pgfsys@useobject{currentmarker}{}%
\end{pgfscope}%
\begin{pgfscope}%
\pgfsys@transformshift{2.786034in}{2.510489in}%
\pgfsys@useobject{currentmarker}{}%
\end{pgfscope}%
\begin{pgfscope}%
\pgfsys@transformshift{2.765377in}{2.605637in}%
\pgfsys@useobject{currentmarker}{}%
\end{pgfscope}%
\begin{pgfscope}%
\pgfsys@transformshift{2.745190in}{2.606110in}%
\pgfsys@useobject{currentmarker}{}%
\end{pgfscope}%
\begin{pgfscope}%
\pgfsys@transformshift{2.724065in}{2.499630in}%
\pgfsys@useobject{currentmarker}{}%
\end{pgfscope}%
\begin{pgfscope}%
\pgfsys@transformshift{2.702001in}{2.367242in}%
\pgfsys@useobject{currentmarker}{}%
\end{pgfscope}%
\begin{pgfscope}%
\pgfsys@transformshift{2.687213in}{2.318261in}%
\pgfsys@useobject{currentmarker}{}%
\end{pgfscope}%
\begin{pgfscope}%
\pgfsys@transformshift{2.668434in}{2.274957in}%
\pgfsys@useobject{currentmarker}{}%
\end{pgfscope}%
\begin{pgfscope}%
\pgfsys@transformshift{2.649421in}{2.258400in}%
\pgfsys@useobject{currentmarker}{}%
\end{pgfscope}%
\begin{pgfscope}%
\pgfsys@transformshift{2.628061in}{2.244598in}%
\pgfsys@useobject{currentmarker}{}%
\end{pgfscope}%
\begin{pgfscope}%
\pgfsys@transformshift{2.610925in}{2.250370in}%
\pgfsys@useobject{currentmarker}{}%
\end{pgfscope}%
\begin{pgfscope}%
\pgfsys@transformshift{2.590738in}{2.272478in}%
\pgfsys@useobject{currentmarker}{}%
\end{pgfscope}%
\begin{pgfscope}%
\pgfsys@transformshift{2.569378in}{2.303654in}%
\pgfsys@useobject{currentmarker}{}%
\end{pgfscope}%
\begin{pgfscope}%
\pgfsys@transformshift{2.550834in}{2.397499in}%
\pgfsys@useobject{currentmarker}{}%
\end{pgfscope}%
\begin{pgfscope}%
\pgfsys@transformshift{2.532526in}{2.523575in}%
\pgfsys@useobject{currentmarker}{}%
\end{pgfscope}%
\begin{pgfscope}%
\pgfsys@transformshift{2.516095in}{2.608995in}%
\pgfsys@useobject{currentmarker}{}%
\end{pgfscope}%
\begin{pgfscope}%
\pgfsys@transformshift{2.494968in}{2.576213in}%
\pgfsys@useobject{currentmarker}{}%
\end{pgfscope}%
\begin{pgfscope}%
\pgfsys@transformshift{2.475955in}{2.456663in}%
\pgfsys@useobject{currentmarker}{}%
\end{pgfscope}%
\begin{pgfscope}%
\pgfsys@transformshift{2.455533in}{2.328894in}%
\pgfsys@useobject{currentmarker}{}%
\end{pgfscope}%
\begin{pgfscope}%
\pgfsys@transformshift{2.437694in}{2.284248in}%
\pgfsys@useobject{currentmarker}{}%
\end{pgfscope}%
\begin{pgfscope}%
\pgfsys@transformshift{2.421263in}{2.259406in}%
\pgfsys@useobject{currentmarker}{}%
\end{pgfscope}%
\begin{pgfscope}%
\pgfsys@transformshift{2.398964in}{2.244539in}%
\pgfsys@useobject{currentmarker}{}%
\end{pgfscope}%
\begin{pgfscope}%
\pgfsys@transformshift{2.377369in}{2.252685in}%
\pgfsys@useobject{currentmarker}{}%
\end{pgfscope}%
\begin{pgfscope}%
\pgfsys@transformshift{2.358826in}{2.276407in}%
\pgfsys@useobject{currentmarker}{}%
\end{pgfscope}%
\begin{pgfscope}%
\pgfsys@transformshift{2.343333in}{2.308191in}%
\pgfsys@useobject{currentmarker}{}%
\end{pgfscope}%
\begin{pgfscope}%
\pgfsys@transformshift{2.321739in}{2.416893in}%
\pgfsys@useobject{currentmarker}{}%
\end{pgfscope}%
\begin{pgfscope}%
\pgfsys@transformshift{2.303664in}{2.539147in}%
\pgfsys@useobject{currentmarker}{}%
\end{pgfscope}%
\begin{pgfscope}%
\pgfsys@transformshift{2.281365in}{2.612201in}%
\pgfsys@useobject{currentmarker}{}%
\end{pgfscope}%
\begin{pgfscope}%
\pgfsys@transformshift{2.263525in}{2.579574in}%
\pgfsys@useobject{currentmarker}{}%
\end{pgfscope}%
\begin{pgfscope}%
\pgfsys@transformshift{2.244512in}{2.475615in}%
\pgfsys@useobject{currentmarker}{}%
\end{pgfscope}%
\begin{pgfscope}%
\pgfsys@transformshift{2.225970in}{2.372097in}%
\pgfsys@useobject{currentmarker}{}%
\end{pgfscope}%
\begin{pgfscope}%
\pgfsys@transformshift{2.207894in}{2.305290in}%
\pgfsys@useobject{currentmarker}{}%
\end{pgfscope}%
\begin{pgfscope}%
\pgfsys@transformshift{2.188412in}{2.269542in}%
\pgfsys@useobject{currentmarker}{}%
\end{pgfscope}%
\begin{pgfscope}%
\pgfsys@transformshift{2.167990in}{2.252141in}%
\pgfsys@useobject{currentmarker}{}%
\end{pgfscope}%
\begin{pgfscope}%
\pgfsys@transformshift{2.148743in}{2.245217in}%
\pgfsys@useobject{currentmarker}{}%
\end{pgfscope}%
\begin{pgfscope}%
\pgfsys@transformshift{2.126912in}{2.259828in}%
\pgfsys@useobject{currentmarker}{}%
\end{pgfscope}%
\begin{pgfscope}%
\pgfsys@transformshift{2.111421in}{2.279712in}%
\pgfsys@useobject{currentmarker}{}%
\end{pgfscope}%
\begin{pgfscope}%
\pgfsys@transformshift{2.089825in}{2.302262in}%
\pgfsys@useobject{currentmarker}{}%
\end{pgfscope}%
\begin{pgfscope}%
\pgfsys@transformshift{2.071283in}{2.395285in}%
\pgfsys@useobject{currentmarker}{}%
\end{pgfscope}%
\begin{pgfscope}%
\pgfsys@transformshift{2.053207in}{2.507343in}%
\pgfsys@useobject{currentmarker}{}%
\end{pgfscope}%
\begin{pgfscope}%
\pgfsys@transformshift{2.034665in}{2.559430in}%
\pgfsys@useobject{currentmarker}{}%
\end{pgfscope}%
\begin{pgfscope}%
\pgfsys@transformshift{2.015652in}{2.616667in}%
\pgfsys@useobject{currentmarker}{}%
\end{pgfscope}%
\begin{pgfscope}%
\pgfsys@transformshift{1.994290in}{2.563590in}%
\pgfsys@useobject{currentmarker}{}%
\end{pgfscope}%
\begin{pgfscope}%
\pgfsys@transformshift{1.975748in}{2.451787in}%
\pgfsys@useobject{currentmarker}{}%
\end{pgfscope}%
\begin{pgfscope}%
\pgfsys@transformshift{1.958143in}{2.345102in}%
\pgfsys@useobject{currentmarker}{}%
\end{pgfscope}%
\begin{pgfscope}%
\pgfsys@transformshift{1.937956in}{2.290315in}%
\pgfsys@useobject{currentmarker}{}%
\end{pgfscope}%
\begin{pgfscope}%
\pgfsys@transformshift{1.920820in}{2.276806in}%
\pgfsys@useobject{currentmarker}{}%
\end{pgfscope}%
\begin{pgfscope}%
\pgfsys@transformshift{1.898992in}{2.251501in}%
\pgfsys@useobject{currentmarker}{}%
\end{pgfscope}%
\begin{pgfscope}%
\pgfsys@transformshift{1.880213in}{2.245850in}%
\pgfsys@useobject{currentmarker}{}%
\end{pgfscope}%
\begin{pgfscope}%
\pgfsys@transformshift{1.858617in}{2.259749in}%
\pgfsys@useobject{currentmarker}{}%
\end{pgfscope}%
\begin{pgfscope}%
\pgfsys@transformshift{1.840074in}{2.289600in}%
\pgfsys@useobject{currentmarker}{}%
\end{pgfscope}%
\begin{pgfscope}%
\pgfsys@transformshift{1.821530in}{2.349229in}%
\pgfsys@useobject{currentmarker}{}%
\end{pgfscope}%
\begin{pgfscope}%
\pgfsys@transformshift{1.802282in}{2.456892in}%
\pgfsys@useobject{currentmarker}{}%
\end{pgfscope}%
\begin{pgfscope}%
\pgfsys@transformshift{1.783504in}{2.556293in}%
\pgfsys@useobject{currentmarker}{}%
\end{pgfscope}%
\begin{pgfscope}%
\pgfsys@transformshift{1.764961in}{2.617586in}%
\pgfsys@useobject{currentmarker}{}%
\end{pgfscope}%
\begin{pgfscope}%
\pgfsys@transformshift{1.744070in}{2.600204in}%
\pgfsys@useobject{currentmarker}{}%
\end{pgfscope}%
\begin{pgfscope}%
\pgfsys@transformshift{1.728109in}{2.512466in}%
\pgfsys@useobject{currentmarker}{}%
\end{pgfscope}%
\begin{pgfscope}%
\pgfsys@transformshift{1.710738in}{2.397599in}%
\pgfsys@useobject{currentmarker}{}%
\end{pgfscope}%
\begin{pgfscope}%
\pgfsys@transformshift{1.687734in}{2.336709in}%
\pgfsys@useobject{currentmarker}{}%
\end{pgfscope}%
\begin{pgfscope}%
\pgfsys@transformshift{1.670364in}{2.296862in}%
\pgfsys@useobject{currentmarker}{}%
\end{pgfscope}%
\begin{pgfscope}%
\pgfsys@transformshift{1.649004in}{2.265048in}%
\pgfsys@useobject{currentmarker}{}%
\end{pgfscope}%
\begin{pgfscope}%
\pgfsys@transformshift{1.629757in}{2.251982in}%
\pgfsys@useobject{currentmarker}{}%
\end{pgfscope}%
\begin{pgfscope}%
\pgfsys@transformshift{1.611447in}{2.247850in}%
\pgfsys@useobject{currentmarker}{}%
\end{pgfscope}%
\begin{pgfscope}%
\pgfsys@transformshift{1.592904in}{2.257999in}%
\pgfsys@useobject{currentmarker}{}%
\end{pgfscope}%
\begin{pgfscope}%
\pgfsys@transformshift{1.571543in}{2.289409in}%
\pgfsys@useobject{currentmarker}{}%
\end{pgfscope}%
\begin{pgfscope}%
\pgfsys@transformshift{1.554643in}{2.340595in}%
\pgfsys@useobject{currentmarker}{}%
\end{pgfscope}%
\begin{pgfscope}%
\pgfsys@transformshift{1.531170in}{2.328674in}%
\pgfsys@useobject{currentmarker}{}%
\end{pgfscope}%
\begin{pgfscope}%
\pgfsys@transformshift{1.512157in}{2.420470in}%
\pgfsys@useobject{currentmarker}{}%
\end{pgfscope}%
\begin{pgfscope}%
\pgfsys@transformshift{1.494552in}{2.540015in}%
\pgfsys@useobject{currentmarker}{}%
\end{pgfscope}%
\begin{pgfscope}%
\pgfsys@transformshift{1.479061in}{2.607942in}%
\pgfsys@useobject{currentmarker}{}%
\end{pgfscope}%
\begin{pgfscope}%
\pgfsys@transformshift{1.459577in}{2.627704in}%
\pgfsys@useobject{currentmarker}{}%
\end{pgfscope}%
\begin{pgfscope}%
\pgfsys@transformshift{1.438686in}{2.551058in}%
\pgfsys@useobject{currentmarker}{}%
\end{pgfscope}%
\begin{pgfscope}%
\pgfsys@transformshift{1.418265in}{2.419898in}%
\pgfsys@useobject{currentmarker}{}%
\end{pgfscope}%
\begin{pgfscope}%
\pgfsys@transformshift{1.395731in}{2.328226in}%
\pgfsys@useobject{currentmarker}{}%
\end{pgfscope}%
\begin{pgfscope}%
\pgfsys@transformshift{1.379066in}{2.302673in}%
\pgfsys@useobject{currentmarker}{}%
\end{pgfscope}%
\begin{pgfscope}%
\pgfsys@transformshift{1.360521in}{2.277704in}%
\pgfsys@useobject{currentmarker}{}%
\end{pgfscope}%
\begin{pgfscope}%
\pgfsys@transformshift{1.340334in}{2.252992in}%
\pgfsys@useobject{currentmarker}{}%
\end{pgfscope}%
\begin{pgfscope}%
\pgfsys@transformshift{1.322261in}{2.248218in}%
\pgfsys@useobject{currentmarker}{}%
\end{pgfscope}%
\begin{pgfscope}%
\pgfsys@transformshift{1.303482in}{2.257287in}%
\pgfsys@useobject{currentmarker}{}%
\end{pgfscope}%
\begin{pgfscope}%
\pgfsys@transformshift{1.284470in}{2.281824in}%
\pgfsys@useobject{currentmarker}{}%
\end{pgfscope}%
\begin{pgfscope}%
\pgfsys@transformshift{1.263343in}{2.325305in}%
\pgfsys@useobject{currentmarker}{}%
\end{pgfscope}%
\begin{pgfscope}%
\pgfsys@transformshift{1.245035in}{2.397061in}%
\pgfsys@useobject{currentmarker}{}%
\end{pgfscope}%
\begin{pgfscope}%
\pgfsys@transformshift{1.226960in}{2.495249in}%
\pgfsys@useobject{currentmarker}{}%
\end{pgfscope}%
\begin{pgfscope}%
\pgfsys@transformshift{1.208183in}{2.559380in}%
\pgfsys@useobject{currentmarker}{}%
\end{pgfscope}%
\begin{pgfscope}%
\pgfsys@transformshift{1.188464in}{2.631790in}%
\pgfsys@useobject{currentmarker}{}%
\end{pgfscope}%
\begin{pgfscope}%
\pgfsys@transformshift{1.168043in}{2.634113in}%
\pgfsys@useobject{currentmarker}{}%
\end{pgfscope}%
\begin{pgfscope}%
\pgfsys@transformshift{1.149500in}{2.562794in}%
\pgfsys@useobject{currentmarker}{}%
\end{pgfscope}%
\begin{pgfscope}%
\pgfsys@transformshift{1.130018in}{2.437072in}%
\pgfsys@useobject{currentmarker}{}%
\end{pgfscope}%
\begin{pgfscope}%
\pgfsys@transformshift{1.111474in}{2.349868in}%
\pgfsys@useobject{currentmarker}{}%
\end{pgfscope}%
\begin{pgfscope}%
\pgfsys@transformshift{1.091523in}{2.301002in}%
\pgfsys@useobject{currentmarker}{}%
\end{pgfscope}%
\begin{pgfscope}%
\pgfsys@transformshift{1.073682in}{2.279860in}%
\pgfsys@useobject{currentmarker}{}%
\end{pgfscope}%
\begin{pgfscope}%
\pgfsys@transformshift{1.053496in}{2.255588in}%
\pgfsys@useobject{currentmarker}{}%
\end{pgfscope}%
\begin{pgfscope}%
\pgfsys@transformshift{1.035891in}{2.249998in}%
\pgfsys@useobject{currentmarker}{}%
\end{pgfscope}%
\begin{pgfscope}%
\pgfsys@transformshift{1.016173in}{2.265812in}%
\pgfsys@useobject{currentmarker}{}%
\end{pgfscope}%
\begin{pgfscope}%
\pgfsys@transformshift{0.995517in}{2.292040in}%
\pgfsys@useobject{currentmarker}{}%
\end{pgfscope}%
\begin{pgfscope}%
\pgfsys@transformshift{0.976740in}{2.372935in}%
\pgfsys@useobject{currentmarker}{}%
\end{pgfscope}%
\begin{pgfscope}%
\pgfsys@transformshift{0.957492in}{2.307059in}%
\pgfsys@useobject{currentmarker}{}%
\end{pgfscope}%
\begin{pgfscope}%
\pgfsys@transformshift{0.940356in}{2.274625in}%
\pgfsys@useobject{currentmarker}{}%
\end{pgfscope}%
\begin{pgfscope}%
\pgfsys@transformshift{0.917823in}{2.250966in}%
\pgfsys@useobject{currentmarker}{}%
\end{pgfscope}%
\begin{pgfscope}%
\pgfsys@transformshift{0.901156in}{2.254389in}%
\pgfsys@useobject{currentmarker}{}%
\end{pgfscope}%
\begin{pgfscope}%
\pgfsys@transformshift{0.880265in}{2.283626in}%
\pgfsys@useobject{currentmarker}{}%
\end{pgfscope}%
\begin{pgfscope}%
\pgfsys@transformshift{0.860783in}{2.340220in}%
\pgfsys@useobject{currentmarker}{}%
\end{pgfscope}%
\begin{pgfscope}%
\pgfsys@transformshift{0.844118in}{2.392509in}%
\pgfsys@useobject{currentmarker}{}%
\end{pgfscope}%
\begin{pgfscope}%
\pgfsys@transformshift{0.823696in}{2.516603in}%
\pgfsys@useobject{currentmarker}{}%
\end{pgfscope}%
\begin{pgfscope}%
\pgfsys@transformshift{0.803040in}{2.611793in}%
\pgfsys@useobject{currentmarker}{}%
\end{pgfscope}%
\begin{pgfscope}%
\pgfsys@transformshift{0.784027in}{2.655772in}%
\pgfsys@useobject{currentmarker}{}%
\end{pgfscope}%
\begin{pgfscope}%
\pgfsys@transformshift{0.763371in}{2.588322in}%
\pgfsys@useobject{currentmarker}{}%
\end{pgfscope}%
\begin{pgfscope}%
\pgfsys@transformshift{0.747174in}{2.472848in}%
\pgfsys@useobject{currentmarker}{}%
\end{pgfscope}%
\begin{pgfscope}%
\pgfsys@transformshift{0.727456in}{2.354903in}%
\pgfsys@useobject{currentmarker}{}%
\end{pgfscope}%
\begin{pgfscope}%
\pgfsys@transformshift{0.707036in}{2.294094in}%
\pgfsys@useobject{currentmarker}{}%
\end{pgfscope}%
\begin{pgfscope}%
\pgfsys@transformshift{0.688960in}{2.265747in}%
\pgfsys@useobject{currentmarker}{}%
\end{pgfscope}%
\begin{pgfscope}%
\pgfsys@transformshift{0.665958in}{2.252396in}%
\pgfsys@useobject{currentmarker}{}%
\end{pgfscope}%
\begin{pgfscope}%
\pgfsys@transformshift{0.650231in}{2.273404in}%
\pgfsys@useobject{currentmarker}{}%
\end{pgfscope}%
\begin{pgfscope}%
\pgfsys@transformshift{0.650934in}{2.274600in}%
\pgfsys@useobject{currentmarker}{}%
\end{pgfscope}%
\begin{pgfscope}%
\pgfsys@transformshift{0.656099in}{2.285592in}%
\pgfsys@useobject{currentmarker}{}%
\end{pgfscope}%
\begin{pgfscope}%
\pgfsys@transformshift{0.676521in}{2.365114in}%
\pgfsys@useobject{currentmarker}{}%
\end{pgfscope}%
\begin{pgfscope}%
\pgfsys@transformshift{0.696003in}{2.513701in}%
\pgfsys@useobject{currentmarker}{}%
\end{pgfscope}%
\begin{pgfscope}%
\pgfsys@transformshift{0.712199in}{2.636779in}%
\pgfsys@useobject{currentmarker}{}%
\end{pgfscope}%
\begin{pgfscope}%
\pgfsys@transformshift{0.734498in}{2.619200in}%
\pgfsys@useobject{currentmarker}{}%
\end{pgfscope}%
\begin{pgfscope}%
\pgfsys@transformshift{0.752808in}{2.468424in}%
\pgfsys@useobject{currentmarker}{}%
\end{pgfscope}%
\begin{pgfscope}%
\pgfsys@transformshift{0.771351in}{2.362620in}%
\pgfsys@useobject{currentmarker}{}%
\end{pgfscope}%
\begin{pgfscope}%
\pgfsys@transformshift{0.790129in}{2.284151in}%
\pgfsys@useobject{currentmarker}{}%
\end{pgfscope}%
\begin{pgfscope}%
\pgfsys@transformshift{0.809612in}{2.253102in}%
\pgfsys@useobject{currentmarker}{}%
\end{pgfscope}%
\begin{pgfscope}%
\pgfsys@transformshift{0.828390in}{2.257652in}%
\pgfsys@useobject{currentmarker}{}%
\end{pgfscope}%
\begin{pgfscope}%
\pgfsys@transformshift{0.850455in}{2.305682in}%
\pgfsys@useobject{currentmarker}{}%
\end{pgfscope}%
\begin{pgfscope}%
\pgfsys@transformshift{0.868997in}{2.397505in}%
\pgfsys@useobject{currentmarker}{}%
\end{pgfscope}%
\begin{pgfscope}%
\pgfsys@transformshift{0.887776in}{2.563830in}%
\pgfsys@useobject{currentmarker}{}%
\end{pgfscope}%
\begin{pgfscope}%
\pgfsys@transformshift{0.906790in}{2.645948in}%
\pgfsys@useobject{currentmarker}{}%
\end{pgfscope}%
\begin{pgfscope}%
\pgfsys@transformshift{0.925803in}{2.574452in}%
\pgfsys@useobject{currentmarker}{}%
\end{pgfscope}%
\begin{pgfscope}%
\pgfsys@transformshift{0.946928in}{2.384104in}%
\pgfsys@useobject{currentmarker}{}%
\end{pgfscope}%
\begin{pgfscope}%
\pgfsys@transformshift{0.963595in}{2.304562in}%
\pgfsys@useobject{currentmarker}{}%
\end{pgfscope}%
\begin{pgfscope}%
\pgfsys@transformshift{0.983546in}{2.261812in}%
\pgfsys@useobject{currentmarker}{}%
\end{pgfscope}%
\begin{pgfscope}%
\pgfsys@transformshift{1.000916in}{2.249121in}%
\pgfsys@useobject{currentmarker}{}%
\end{pgfscope}%
\begin{pgfscope}%
\pgfsys@transformshift{1.026501in}{2.284377in}%
\pgfsys@useobject{currentmarker}{}%
\end{pgfscope}%
\begin{pgfscope}%
\pgfsys@transformshift{1.042934in}{2.329115in}%
\pgfsys@useobject{currentmarker}{}%
\end{pgfscope}%
\begin{pgfscope}%
\pgfsys@transformshift{1.061007in}{2.447994in}%
\pgfsys@useobject{currentmarker}{}%
\end{pgfscope}%
\begin{pgfscope}%
\pgfsys@transformshift{1.080255in}{2.605280in}%
\pgfsys@useobject{currentmarker}{}%
\end{pgfscope}%
\begin{pgfscope}%
\pgfsys@transformshift{1.098563in}{2.627734in}%
\pgfsys@useobject{currentmarker}{}%
\end{pgfscope}%
\begin{pgfscope}%
\pgfsys@transformshift{1.116873in}{2.501908in}%
\pgfsys@useobject{currentmarker}{}%
\end{pgfscope}%
\begin{pgfscope}%
\pgfsys@transformshift{1.141049in}{2.333348in}%
\pgfsys@useobject{currentmarker}{}%
\end{pgfscope}%
\begin{pgfscope}%
\pgfsys@transformshift{1.154663in}{2.281824in}%
\pgfsys@useobject{currentmarker}{}%
\end{pgfscope}%
\begin{pgfscope}%
\pgfsys@transformshift{1.173442in}{2.251169in}%
\pgfsys@useobject{currentmarker}{}%
\end{pgfscope}%
\begin{pgfscope}%
\pgfsys@transformshift{1.195507in}{2.252066in}%
\pgfsys@useobject{currentmarker}{}%
\end{pgfscope}%
\begin{pgfscope}%
\pgfsys@transformshift{1.213112in}{2.279226in}%
\pgfsys@useobject{currentmarker}{}%
\end{pgfscope}%
\begin{pgfscope}%
\pgfsys@transformshift{1.233533in}{2.334942in}%
\pgfsys@useobject{currentmarker}{}%
\end{pgfscope}%
\begin{pgfscope}%
\pgfsys@transformshift{1.253955in}{2.445659in}%
\pgfsys@useobject{currentmarker}{}%
\end{pgfscope}%
\begin{pgfscope}%
\pgfsys@transformshift{1.272497in}{2.577878in}%
\pgfsys@useobject{currentmarker}{}%
\end{pgfscope}%
\begin{pgfscope}%
\pgfsys@transformshift{1.288694in}{2.628508in}%
\pgfsys@useobject{currentmarker}{}%
\end{pgfscope}%
\begin{pgfscope}%
\pgfsys@transformshift{1.310055in}{2.524726in}%
\pgfsys@useobject{currentmarker}{}%
\end{pgfscope}%
\begin{pgfscope}%
\pgfsys@transformshift{1.328129in}{2.378497in}%
\pgfsys@useobject{currentmarker}{}%
\end{pgfscope}%
\begin{pgfscope}%
\pgfsys@transformshift{1.349725in}{2.288141in}%
\pgfsys@useobject{currentmarker}{}%
\end{pgfscope}%
\begin{pgfscope}%
\pgfsys@transformshift{1.367095in}{2.256179in}%
\pgfsys@useobject{currentmarker}{}%
\end{pgfscope}%
\begin{pgfscope}%
\pgfsys@transformshift{1.387751in}{2.246686in}%
\pgfsys@useobject{currentmarker}{}%
\end{pgfscope}%
\begin{pgfscope}%
\pgfsys@transformshift{1.406528in}{2.263174in}%
\pgfsys@useobject{currentmarker}{}%
\end{pgfscope}%
\begin{pgfscope}%
\pgfsys@transformshift{1.423664in}{2.298181in}%
\pgfsys@useobject{currentmarker}{}%
\end{pgfscope}%
\begin{pgfscope}%
\pgfsys@transformshift{1.444789in}{2.393605in}%
\pgfsys@useobject{currentmarker}{}%
\end{pgfscope}%
\begin{pgfscope}%
\pgfsys@transformshift{1.462628in}{2.508648in}%
\pgfsys@useobject{currentmarker}{}%
\end{pgfscope}%
\begin{pgfscope}%
\pgfsys@transformshift{1.487041in}{2.622717in}%
\pgfsys@useobject{currentmarker}{}%
\end{pgfscope}%
\begin{pgfscope}%
\pgfsys@transformshift{1.501829in}{2.610138in}%
\pgfsys@useobject{currentmarker}{}%
\end{pgfscope}%
\begin{pgfscope}%
\pgfsys@transformshift{1.522016in}{2.598374in}%
\pgfsys@useobject{currentmarker}{}%
\end{pgfscope}%
\begin{pgfscope}%
\pgfsys@transformshift{1.539855in}{2.466173in}%
\pgfsys@useobject{currentmarker}{}%
\end{pgfscope}%
\begin{pgfscope}%
\pgfsys@transformshift{1.559806in}{2.330138in}%
\pgfsys@useobject{currentmarker}{}%
\end{pgfscope}%
\begin{pgfscope}%
\pgfsys@transformshift{1.579525in}{2.277671in}%
\pgfsys@useobject{currentmarker}{}%
\end{pgfscope}%
\begin{pgfscope}%
\pgfsys@transformshift{1.599946in}{2.250118in}%
\pgfsys@useobject{currentmarker}{}%
\end{pgfscope}%
\begin{pgfscope}%
\pgfsys@transformshift{1.617786in}{2.246763in}%
\pgfsys@useobject{currentmarker}{}%
\end{pgfscope}%
\begin{pgfscope}%
\pgfsys@transformshift{1.634451in}{2.260185in}%
\pgfsys@useobject{currentmarker}{}%
\end{pgfscope}%
\begin{pgfscope}%
\pgfsys@transformshift{1.656515in}{2.282453in}%
\pgfsys@useobject{currentmarker}{}%
\end{pgfscope}%
\begin{pgfscope}%
\pgfsys@transformshift{1.674589in}{2.323653in}%
\pgfsys@useobject{currentmarker}{}%
\end{pgfscope}%
\begin{pgfscope}%
\pgfsys@transformshift{1.694307in}{2.431019in}%
\pgfsys@useobject{currentmarker}{}%
\end{pgfscope}%
\begin{pgfscope}%
\pgfsys@transformshift{1.714258in}{2.426199in}%
\pgfsys@useobject{currentmarker}{}%
\end{pgfscope}%
\begin{pgfscope}%
\pgfsys@transformshift{1.729986in}{2.560231in}%
\pgfsys@useobject{currentmarker}{}%
\end{pgfscope}%
\begin{pgfscope}%
\pgfsys@transformshift{1.750642in}{2.617418in}%
\pgfsys@useobject{currentmarker}{}%
\end{pgfscope}%
\begin{pgfscope}%
\pgfsys@transformshift{1.771064in}{2.534482in}%
\pgfsys@useobject{currentmarker}{}%
\end{pgfscope}%
\begin{pgfscope}%
\pgfsys@transformshift{1.789606in}{2.388435in}%
\pgfsys@useobject{currentmarker}{}%
\end{pgfscope}%
\begin{pgfscope}%
\pgfsys@transformshift{1.809559in}{2.300899in}%
\pgfsys@useobject{currentmarker}{}%
\end{pgfscope}%
\begin{pgfscope}%
\pgfsys@transformshift{1.829276in}{2.295402in}%
\pgfsys@useobject{currentmarker}{}%
\end{pgfscope}%
\begin{pgfscope}%
\pgfsys@transformshift{1.846412in}{2.565136in}%
\pgfsys@useobject{currentmarker}{}%
\end{pgfscope}%
\begin{pgfscope}%
\pgfsys@transformshift{1.867537in}{2.612057in}%
\pgfsys@useobject{currentmarker}{}%
\end{pgfscope}%
\begin{pgfscope}%
\pgfsys@transformshift{1.888898in}{2.526224in}%
\pgfsys@useobject{currentmarker}{}%
\end{pgfscope}%
\begin{pgfscope}%
\pgfsys@transformshift{1.888193in}{2.431288in}%
\pgfsys@useobject{currentmarker}{}%
\end{pgfscope}%
\begin{pgfscope}%
\pgfsys@transformshift{1.907206in}{2.376537in}%
\pgfsys@useobject{currentmarker}{}%
\end{pgfscope}%
\begin{pgfscope}%
\pgfsys@transformshift{1.924576in}{2.292945in}%
\pgfsys@useobject{currentmarker}{}%
\end{pgfscope}%
\begin{pgfscope}%
\pgfsys@transformshift{1.943121in}{2.257816in}%
\pgfsys@useobject{currentmarker}{}%
\end{pgfscope}%
\begin{pgfscope}%
\pgfsys@transformshift{1.963541in}{2.244798in}%
\pgfsys@useobject{currentmarker}{}%
\end{pgfscope}%
\begin{pgfscope}%
\pgfsys@transformshift{1.983025in}{2.256169in}%
\pgfsys@useobject{currentmarker}{}%
\end{pgfscope}%
\begin{pgfscope}%
\pgfsys@transformshift{2.002741in}{2.289819in}%
\pgfsys@useobject{currentmarker}{}%
\end{pgfscope}%
\begin{pgfscope}%
\pgfsys@transformshift{2.020111in}{2.356771in}%
\pgfsys@useobject{currentmarker}{}%
\end{pgfscope}%
\begin{pgfscope}%
\pgfsys@transformshift{2.041471in}{2.516717in}%
\pgfsys@useobject{currentmarker}{}%
\end{pgfscope}%
\begin{pgfscope}%
\pgfsys@transformshift{2.059310in}{2.606300in}%
\pgfsys@useobject{currentmarker}{}%
\end{pgfscope}%
\begin{pgfscope}%
\pgfsys@transformshift{2.077386in}{2.594247in}%
\pgfsys@useobject{currentmarker}{}%
\end{pgfscope}%
\begin{pgfscope}%
\pgfsys@transformshift{2.098276in}{2.464537in}%
\pgfsys@useobject{currentmarker}{}%
\end{pgfscope}%
\begin{pgfscope}%
\pgfsys@transformshift{2.116115in}{2.338774in}%
\pgfsys@useobject{currentmarker}{}%
\end{pgfscope}%
\begin{pgfscope}%
\pgfsys@transformshift{2.139589in}{2.269172in}%
\pgfsys@useobject{currentmarker}{}%
\end{pgfscope}%
\begin{pgfscope}%
\pgfsys@transformshift{2.158602in}{2.248035in}%
\pgfsys@useobject{currentmarker}{}%
\end{pgfscope}%
\begin{pgfscope}%
\pgfsys@transformshift{2.175972in}{2.245893in}%
\pgfsys@useobject{currentmarker}{}%
\end{pgfscope}%
\begin{pgfscope}%
\pgfsys@transformshift{2.193577in}{2.255336in}%
\pgfsys@useobject{currentmarker}{}%
\end{pgfscope}%
\begin{pgfscope}%
\pgfsys@transformshift{2.211651in}{2.282222in}%
\pgfsys@useobject{currentmarker}{}%
\end{pgfscope}%
\begin{pgfscope}%
\pgfsys@transformshift{2.233010in}{2.354832in}%
\pgfsys@useobject{currentmarker}{}%
\end{pgfscope}%
\begin{pgfscope}%
\pgfsys@transformshift{2.250849in}{2.450264in}%
\pgfsys@useobject{currentmarker}{}%
\end{pgfscope}%
\begin{pgfscope}%
\pgfsys@transformshift{2.272679in}{2.592940in}%
\pgfsys@useobject{currentmarker}{}%
\end{pgfscope}%
\begin{pgfscope}%
\pgfsys@transformshift{2.289815in}{2.612871in}%
\pgfsys@useobject{currentmarker}{}%
\end{pgfscope}%
\begin{pgfscope}%
\pgfsys@transformshift{2.308360in}{2.549917in}%
\pgfsys@useobject{currentmarker}{}%
\end{pgfscope}%
\begin{pgfscope}%
\pgfsys@transformshift{2.329485in}{2.380169in}%
\pgfsys@useobject{currentmarker}{}%
\end{pgfscope}%
\begin{pgfscope}%
\pgfsys@transformshift{2.348027in}{2.300790in}%
\pgfsys@useobject{currentmarker}{}%
\end{pgfscope}%
\begin{pgfscope}%
\pgfsys@transformshift{2.365398in}{2.264565in}%
\pgfsys@useobject{currentmarker}{}%
\end{pgfscope}%
\begin{pgfscope}%
\pgfsys@transformshift{2.387228in}{2.245301in}%
\pgfsys@useobject{currentmarker}{}%
\end{pgfscope}%
\begin{pgfscope}%
\pgfsys@transformshift{2.404129in}{2.248384in}%
\pgfsys@useobject{currentmarker}{}%
\end{pgfscope}%
\begin{pgfscope}%
\pgfsys@transformshift{2.424785in}{2.272221in}%
\pgfsys@useobject{currentmarker}{}%
\end{pgfscope}%
\begin{pgfscope}%
\pgfsys@transformshift{2.443328in}{2.317840in}%
\pgfsys@useobject{currentmarker}{}%
\end{pgfscope}%
\begin{pgfscope}%
\pgfsys@transformshift{2.462810in}{2.414097in}%
\pgfsys@useobject{currentmarker}{}%
\end{pgfscope}%
\begin{pgfscope}%
\pgfsys@transformshift{2.481823in}{2.502674in}%
\pgfsys@useobject{currentmarker}{}%
\end{pgfscope}%
\begin{pgfscope}%
\pgfsys@transformshift{2.499663in}{2.571740in}%
\pgfsys@useobject{currentmarker}{}%
\end{pgfscope}%
\begin{pgfscope}%
\pgfsys@transformshift{2.520789in}{2.610351in}%
\pgfsys@useobject{currentmarker}{}%
\end{pgfscope}%
\begin{pgfscope}%
\pgfsys@transformshift{2.539566in}{2.527140in}%
\pgfsys@useobject{currentmarker}{}%
\end{pgfscope}%
\begin{pgfscope}%
\pgfsys@transformshift{2.557642in}{2.398480in}%
\pgfsys@useobject{currentmarker}{}%
\end{pgfscope}%
\begin{pgfscope}%
\pgfsys@transformshift{2.578064in}{2.304721in}%
\pgfsys@useobject{currentmarker}{}%
\end{pgfscope}%
\begin{pgfscope}%
\pgfsys@transformshift{2.596137in}{2.264653in}%
\pgfsys@useobject{currentmarker}{}%
\end{pgfscope}%
\begin{pgfscope}%
\pgfsys@transformshift{2.617733in}{2.246854in}%
\pgfsys@useobject{currentmarker}{}%
\end{pgfscope}%
\begin{pgfscope}%
\pgfsys@transformshift{2.635572in}{2.247324in}%
\pgfsys@useobject{currentmarker}{}%
\end{pgfscope}%
\begin{pgfscope}%
\pgfsys@transformshift{2.653412in}{2.264430in}%
\pgfsys@useobject{currentmarker}{}%
\end{pgfscope}%
\begin{pgfscope}%
\pgfsys@transformshift{2.674068in}{2.302157in}%
\pgfsys@useobject{currentmarker}{}%
\end{pgfscope}%
\begin{pgfscope}%
\pgfsys@transformshift{2.691907in}{2.382110in}%
\pgfsys@useobject{currentmarker}{}%
\end{pgfscope}%
\begin{pgfscope}%
\pgfsys@transformshift{2.713971in}{2.521632in}%
\pgfsys@useobject{currentmarker}{}%
\end{pgfscope}%
\begin{pgfscope}%
\pgfsys@transformshift{2.731342in}{2.605908in}%
\pgfsys@useobject{currentmarker}{}%
\end{pgfscope}%
\begin{pgfscope}%
\pgfsys@transformshift{2.752467in}{2.593907in}%
\pgfsys@useobject{currentmarker}{}%
\end{pgfscope}%
\begin{pgfscope}%
\pgfsys@transformshift{2.770775in}{2.475884in}%
\pgfsys@useobject{currentmarker}{}%
\end{pgfscope}%
\begin{pgfscope}%
\pgfsys@transformshift{2.788616in}{2.358709in}%
\pgfsys@useobject{currentmarker}{}%
\end{pgfscope}%
\begin{pgfscope}%
\pgfsys@transformshift{2.807864in}{2.291886in}%
\pgfsys@useobject{currentmarker}{}%
\end{pgfscope}%
\begin{pgfscope}%
\pgfsys@transformshift{2.828989in}{2.258459in}%
\pgfsys@useobject{currentmarker}{}%
\end{pgfscope}%
\begin{pgfscope}%
\pgfsys@transformshift{2.846359in}{2.245729in}%
\pgfsys@useobject{currentmarker}{}%
\end{pgfscope}%
\begin{pgfscope}%
\pgfsys@transformshift{2.866781in}{2.248894in}%
\pgfsys@useobject{currentmarker}{}%
\end{pgfscope}%
\begin{pgfscope}%
\pgfsys@transformshift{2.886966in}{2.262403in}%
\pgfsys@useobject{currentmarker}{}%
\end{pgfscope}%
\begin{pgfscope}%
\pgfsys@transformshift{2.904337in}{2.290418in}%
\pgfsys@useobject{currentmarker}{}%
\end{pgfscope}%
\begin{pgfscope}%
\pgfsys@transformshift{2.924993in}{2.367759in}%
\pgfsys@useobject{currentmarker}{}%
\end{pgfscope}%
\begin{pgfscope}%
\pgfsys@transformshift{2.943066in}{2.474419in}%
\pgfsys@useobject{currentmarker}{}%
\end{pgfscope}%
\begin{pgfscope}%
\pgfsys@transformshift{2.963488in}{2.602298in}%
\pgfsys@useobject{currentmarker}{}%
\end{pgfscope}%
\begin{pgfscope}%
\pgfsys@transformshift{2.979919in}{2.616377in}%
\pgfsys@useobject{currentmarker}{}%
\end{pgfscope}%
\begin{pgfscope}%
\pgfsys@transformshift{3.001046in}{2.587354in}%
\pgfsys@useobject{currentmarker}{}%
\end{pgfscope}%
\begin{pgfscope}%
\pgfsys@transformshift{3.022640in}{2.450186in}%
\pgfsys@useobject{currentmarker}{}%
\end{pgfscope}%
\begin{pgfscope}%
\pgfsys@transformshift{3.039072in}{2.500508in}%
\pgfsys@useobject{currentmarker}{}%
\end{pgfscope}%
\begin{pgfscope}%
\pgfsys@transformshift{3.058084in}{2.374557in}%
\pgfsys@useobject{currentmarker}{}%
\end{pgfscope}%
\begin{pgfscope}%
\pgfsys@transformshift{3.078271in}{2.294679in}%
\pgfsys@useobject{currentmarker}{}%
\end{pgfscope}%
\begin{pgfscope}%
\pgfsys@transformshift{3.097284in}{2.263612in}%
\pgfsys@useobject{currentmarker}{}%
\end{pgfscope}%
\begin{pgfscope}%
\pgfsys@transformshift{3.113715in}{2.248170in}%
\pgfsys@useobject{currentmarker}{}%
\end{pgfscope}%
\begin{pgfscope}%
\pgfsys@transformshift{3.134842in}{2.251395in}%
\pgfsys@useobject{currentmarker}{}%
\end{pgfscope}%
\begin{pgfscope}%
\pgfsys@transformshift{3.156201in}{2.265098in}%
\pgfsys@useobject{currentmarker}{}%
\end{pgfscope}%
\begin{pgfscope}%
\pgfsys@transformshift{3.174509in}{2.300337in}%
\pgfsys@useobject{currentmarker}{}%
\end{pgfscope}%
\begin{pgfscope}%
\pgfsys@transformshift{3.191880in}{2.352844in}%
\pgfsys@useobject{currentmarker}{}%
\end{pgfscope}%
\begin{pgfscope}%
\pgfsys@transformshift{3.212301in}{2.453185in}%
\pgfsys@useobject{currentmarker}{}%
\end{pgfscope}%
\begin{pgfscope}%
\pgfsys@transformshift{3.232254in}{2.561086in}%
\pgfsys@useobject{currentmarker}{}%
\end{pgfscope}%
\begin{pgfscope}%
\pgfsys@transformshift{3.250562in}{2.622879in}%
\pgfsys@useobject{currentmarker}{}%
\end{pgfscope}%
\begin{pgfscope}%
\pgfsys@transformshift{3.267698in}{2.587199in}%
\pgfsys@useobject{currentmarker}{}%
\end{pgfscope}%
\begin{pgfscope}%
\pgfsys@transformshift{3.287884in}{2.476481in}%
\pgfsys@useobject{currentmarker}{}%
\end{pgfscope}%
\begin{pgfscope}%
\pgfsys@transformshift{3.309714in}{2.402507in}%
\pgfsys@useobject{currentmarker}{}%
\end{pgfscope}%
\begin{pgfscope}%
\pgfsys@transformshift{3.324736in}{2.322070in}%
\pgfsys@useobject{currentmarker}{}%
\end{pgfscope}%
\begin{pgfscope}%
\pgfsys@transformshift{3.348209in}{2.268380in}%
\pgfsys@useobject{currentmarker}{}%
\end{pgfscope}%
\begin{pgfscope}%
\pgfsys@transformshift{3.364876in}{2.253450in}%
\pgfsys@useobject{currentmarker}{}%
\end{pgfscope}%
\begin{pgfscope}%
\pgfsys@transformshift{3.384827in}{2.248081in}%
\pgfsys@useobject{currentmarker}{}%
\end{pgfscope}%
\begin{pgfscope}%
\pgfsys@transformshift{3.403606in}{2.262090in}%
\pgfsys@useobject{currentmarker}{}%
\end{pgfscope}%
\begin{pgfscope}%
\pgfsys@transformshift{3.423559in}{2.296716in}%
\pgfsys@useobject{currentmarker}{}%
\end{pgfscope}%
\begin{pgfscope}%
\pgfsys@transformshift{3.443041in}{2.342052in}%
\pgfsys@useobject{currentmarker}{}%
\end{pgfscope}%
\begin{pgfscope}%
\pgfsys@transformshift{3.463463in}{2.404025in}%
\pgfsys@useobject{currentmarker}{}%
\end{pgfscope}%
\begin{pgfscope}%
\pgfsys@transformshift{3.481302in}{2.526786in}%
\pgfsys@useobject{currentmarker}{}%
\end{pgfscope}%
\begin{pgfscope}%
\pgfsys@transformshift{3.499141in}{2.619367in}%
\pgfsys@useobject{currentmarker}{}%
\end{pgfscope}%
\begin{pgfscope}%
\pgfsys@transformshift{3.520501in}{2.621277in}%
\pgfsys@useobject{currentmarker}{}%
\end{pgfscope}%
\begin{pgfscope}%
\pgfsys@transformshift{3.537871in}{2.542989in}%
\pgfsys@useobject{currentmarker}{}%
\end{pgfscope}%
\begin{pgfscope}%
\pgfsys@transformshift{3.556650in}{2.448263in}%
\pgfsys@useobject{currentmarker}{}%
\end{pgfscope}%
\begin{pgfscope}%
\pgfsys@transformshift{3.577306in}{2.344371in}%
\pgfsys@useobject{currentmarker}{}%
\end{pgfscope}%
\begin{pgfscope}%
\pgfsys@transformshift{3.595145in}{2.296552in}%
\pgfsys@useobject{currentmarker}{}%
\end{pgfscope}%
\begin{pgfscope}%
\pgfsys@transformshift{3.616505in}{2.264135in}%
\pgfsys@useobject{currentmarker}{}%
\end{pgfscope}%
\begin{pgfscope}%
\pgfsys@transformshift{3.634346in}{2.250632in}%
\pgfsys@useobject{currentmarker}{}%
\end{pgfscope}%
\begin{pgfscope}%
\pgfsys@transformshift{3.654062in}{2.249544in}%
\pgfsys@useobject{currentmarker}{}%
\end{pgfscope}%
\begin{pgfscope}%
\pgfsys@transformshift{3.673779in}{2.268170in}%
\pgfsys@useobject{currentmarker}{}%
\end{pgfscope}%
\begin{pgfscope}%
\pgfsys@transformshift{3.692558in}{2.289775in}%
\pgfsys@useobject{currentmarker}{}%
\end{pgfscope}%
\begin{pgfscope}%
\pgfsys@transformshift{3.713919in}{2.332635in}%
\pgfsys@useobject{currentmarker}{}%
\end{pgfscope}%
\begin{pgfscope}%
\pgfsys@transformshift{3.728471in}{2.396915in}%
\pgfsys@useobject{currentmarker}{}%
\end{pgfscope}%
\begin{pgfscope}%
\pgfsys@transformshift{3.749127in}{2.517791in}%
\pgfsys@useobject{currentmarker}{}%
\end{pgfscope}%
\begin{pgfscope}%
\pgfsys@transformshift{3.770019in}{2.263645in}%
\pgfsys@useobject{currentmarker}{}%
\end{pgfscope}%
\begin{pgfscope}%
\pgfsys@transformshift{3.791613in}{2.299243in}%
\pgfsys@useobject{currentmarker}{}%
\end{pgfscope}%
\begin{pgfscope}%
\pgfsys@transformshift{3.804524in}{2.363273in}%
\pgfsys@useobject{currentmarker}{}%
\end{pgfscope}%
\begin{pgfscope}%
\pgfsys@transformshift{3.827057in}{2.464755in}%
\pgfsys@useobject{currentmarker}{}%
\end{pgfscope}%
\begin{pgfscope}%
\pgfsys@transformshift{3.846776in}{2.593960in}%
\pgfsys@useobject{currentmarker}{}%
\end{pgfscope}%
\begin{pgfscope}%
\pgfsys@transformshift{3.866492in}{2.642535in}%
\pgfsys@useobject{currentmarker}{}%
\end{pgfscope}%
\begin{pgfscope}%
\pgfsys@transformshift{3.885505in}{2.608899in}%
\pgfsys@useobject{currentmarker}{}%
\end{pgfscope}%
\begin{pgfscope}%
\pgfsys@transformshift{3.902407in}{2.518501in}%
\pgfsys@useobject{currentmarker}{}%
\end{pgfscope}%
\begin{pgfscope}%
\pgfsys@transformshift{3.924235in}{2.379734in}%
\pgfsys@useobject{currentmarker}{}%
\end{pgfscope}%
\begin{pgfscope}%
\pgfsys@transformshift{3.942074in}{2.313832in}%
\pgfsys@useobject{currentmarker}{}%
\end{pgfscope}%
\begin{pgfscope}%
\pgfsys@transformshift{3.960150in}{2.271193in}%
\pgfsys@useobject{currentmarker}{}%
\end{pgfscope}%
\begin{pgfscope}%
\pgfsys@transformshift{3.979632in}{2.252857in}%
\pgfsys@useobject{currentmarker}{}%
\end{pgfscope}%
\begin{pgfscope}%
\pgfsys@transformshift{3.997471in}{2.250438in}%
\pgfsys@useobject{currentmarker}{}%
\end{pgfscope}%
\begin{pgfscope}%
\pgfsys@transformshift{4.018362in}{2.269684in}%
\pgfsys@useobject{currentmarker}{}%
\end{pgfscope}%
\begin{pgfscope}%
\pgfsys@transformshift{4.037375in}{2.307784in}%
\pgfsys@useobject{currentmarker}{}%
\end{pgfscope}%
\begin{pgfscope}%
\pgfsys@transformshift{4.057797in}{2.367131in}%
\pgfsys@useobject{currentmarker}{}%
\end{pgfscope}%
\begin{pgfscope}%
\pgfsys@transformshift{4.079392in}{2.486704in}%
\pgfsys@useobject{currentmarker}{}%
\end{pgfscope}%
\begin{pgfscope}%
\pgfsys@transformshift{4.095118in}{2.594718in}%
\pgfsys@useobject{currentmarker}{}%
\end{pgfscope}%
\begin{pgfscope}%
\pgfsys@transformshift{4.113194in}{2.653270in}%
\pgfsys@useobject{currentmarker}{}%
\end{pgfscope}%
\begin{pgfscope}%
\pgfsys@transformshift{4.133379in}{2.618695in}%
\pgfsys@useobject{currentmarker}{}%
\end{pgfscope}%
\begin{pgfscope}%
\pgfsys@transformshift{4.152158in}{2.524799in}%
\pgfsys@useobject{currentmarker}{}%
\end{pgfscope}%
\begin{pgfscope}%
\pgfsys@transformshift{4.174691in}{2.384458in}%
\pgfsys@useobject{currentmarker}{}%
\end{pgfscope}%
\begin{pgfscope}%
\pgfsys@transformshift{4.192062in}{2.328264in}%
\pgfsys@useobject{currentmarker}{}%
\end{pgfscope}%
\begin{pgfscope}%
\pgfsys@transformshift{4.210606in}{2.282192in}%
\pgfsys@useobject{currentmarker}{}%
\end{pgfscope}%
\begin{pgfscope}%
\pgfsys@transformshift{4.229385in}{2.259638in}%
\pgfsys@useobject{currentmarker}{}%
\end{pgfscope}%
\begin{pgfscope}%
\pgfsys@transformshift{4.248867in}{2.252411in}%
\pgfsys@useobject{currentmarker}{}%
\end{pgfscope}%
\begin{pgfscope}%
\pgfsys@transformshift{4.267644in}{2.260796in}%
\pgfsys@useobject{currentmarker}{}%
\end{pgfscope}%
\begin{pgfscope}%
\pgfsys@transformshift{4.287831in}{2.285003in}%
\pgfsys@useobject{currentmarker}{}%
\end{pgfscope}%
\begin{pgfscope}%
\pgfsys@transformshift{4.308487in}{2.326437in}%
\pgfsys@useobject{currentmarker}{}%
\end{pgfscope}%
\begin{pgfscope}%
\pgfsys@transformshift{4.331021in}{2.434686in}%
\pgfsys@useobject{currentmarker}{}%
\end{pgfscope}%
\begin{pgfscope}%
\pgfsys@transformshift{4.344871in}{2.519720in}%
\pgfsys@useobject{currentmarker}{}%
\end{pgfscope}%
\begin{pgfscope}%
\pgfsys@transformshift{4.362710in}{2.616844in}%
\pgfsys@useobject{currentmarker}{}%
\end{pgfscope}%
\begin{pgfscope}%
\pgfsys@transformshift{4.385244in}{2.665577in}%
\pgfsys@useobject{currentmarker}{}%
\end{pgfscope}%
\begin{pgfscope}%
\pgfsys@transformshift{4.405900in}{2.631565in}%
\pgfsys@useobject{currentmarker}{}%
\end{pgfscope}%
\begin{pgfscope}%
\pgfsys@transformshift{4.423505in}{2.464655in}%
\pgfsys@useobject{currentmarker}{}%
\end{pgfscope}%
\begin{pgfscope}%
\pgfsys@transformshift{4.442752in}{2.593862in}%
\pgfsys@useobject{currentmarker}{}%
\end{pgfscope}%
\begin{pgfscope}%
\pgfsys@transformshift{4.461766in}{2.665758in}%
\pgfsys@useobject{currentmarker}{}%
\end{pgfscope}%
\begin{pgfscope}%
\pgfsys@transformshift{4.481484in}{2.646140in}%
\pgfsys@useobject{currentmarker}{}%
\end{pgfscope}%
\begin{pgfscope}%
\pgfsys@transformshift{4.481953in}{2.643105in}%
\pgfsys@useobject{currentmarker}{}%
\end{pgfscope}%
\begin{pgfscope}%
\pgfsys@transformshift{4.473737in}{2.666685in}%
\pgfsys@useobject{currentmarker}{}%
\end{pgfscope}%
\begin{pgfscope}%
\pgfsys@transformshift{4.455897in}{2.629842in}%
\pgfsys@useobject{currentmarker}{}%
\end{pgfscope}%
\begin{pgfscope}%
\pgfsys@transformshift{4.435241in}{2.442481in}%
\pgfsys@useobject{currentmarker}{}%
\end{pgfscope}%
\begin{pgfscope}%
\pgfsys@transformshift{4.417871in}{2.331032in}%
\pgfsys@useobject{currentmarker}{}%
\end{pgfscope}%
\begin{pgfscope}%
\pgfsys@transformshift{4.398389in}{2.273442in}%
\pgfsys@useobject{currentmarker}{}%
\end{pgfscope}%
\begin{pgfscope}%
\pgfsys@transformshift{4.377264in}{2.251359in}%
\pgfsys@useobject{currentmarker}{}%
\end{pgfscope}%
\begin{pgfscope}%
\pgfsys@transformshift{4.358954in}{2.272395in}%
\pgfsys@useobject{currentmarker}{}%
\end{pgfscope}%
\begin{pgfscope}%
\pgfsys@transformshift{4.338768in}{2.336059in}%
\pgfsys@useobject{currentmarker}{}%
\end{pgfscope}%
\begin{pgfscope}%
\pgfsys@transformshift{4.322101in}{2.462820in}%
\pgfsys@useobject{currentmarker}{}%
\end{pgfscope}%
\begin{pgfscope}%
\pgfsys@transformshift{4.300976in}{2.631840in}%
\pgfsys@useobject{currentmarker}{}%
\end{pgfscope}%
\begin{pgfscope}%
\pgfsys@transformshift{4.281963in}{2.646533in}%
\pgfsys@useobject{currentmarker}{}%
\end{pgfscope}%
\begin{pgfscope}%
\pgfsys@transformshift{4.264358in}{2.524107in}%
\pgfsys@useobject{currentmarker}{}%
\end{pgfscope}%
\begin{pgfscope}%
\pgfsys@transformshift{4.243936in}{2.355132in}%
\pgfsys@useobject{currentmarker}{}%
\end{pgfscope}%
\begin{pgfscope}%
\pgfsys@transformshift{4.223046in}{2.280348in}%
\pgfsys@useobject{currentmarker}{}%
\end{pgfscope}%
\begin{pgfscope}%
\pgfsys@transformshift{4.205207in}{2.252274in}%
\pgfsys@useobject{currentmarker}{}%
\end{pgfscope}%
\begin{pgfscope}%
\pgfsys@transformshift{4.186664in}{2.257490in}%
\pgfsys@useobject{currentmarker}{}%
\end{pgfscope}%
\begin{pgfscope}%
\pgfsys@transformshift{4.166242in}{2.297924in}%
\pgfsys@useobject{currentmarker}{}%
\end{pgfscope}%
\begin{pgfscope}%
\pgfsys@transformshift{4.148167in}{2.384136in}%
\pgfsys@useobject{currentmarker}{}%
\end{pgfscope}%
\begin{pgfscope}%
\pgfsys@transformshift{4.127982in}{2.569520in}%
\pgfsys@useobject{currentmarker}{}%
\end{pgfscope}%
\begin{pgfscope}%
\pgfsys@transformshift{4.109437in}{2.645936in}%
\pgfsys@useobject{currentmarker}{}%
\end{pgfscope}%
\begin{pgfscope}%
\pgfsys@transformshift{4.089250in}{2.568366in}%
\pgfsys@useobject{currentmarker}{}%
\end{pgfscope}%
\begin{pgfscope}%
\pgfsys@transformshift{4.070942in}{2.412771in}%
\pgfsys@useobject{currentmarker}{}%
\end{pgfscope}%
\begin{pgfscope}%
\pgfsys@transformshift{4.052868in}{2.310370in}%
\pgfsys@useobject{currentmarker}{}%
\end{pgfscope}%
\begin{pgfscope}%
\pgfsys@transformshift{4.031741in}{2.259091in}%
\pgfsys@useobject{currentmarker}{}%
\end{pgfscope}%
\begin{pgfscope}%
\pgfsys@transformshift{4.012493in}{2.248558in}%
\pgfsys@useobject{currentmarker}{}%
\end{pgfscope}%
\begin{pgfscope}%
\pgfsys@transformshift{3.993480in}{2.270190in}%
\pgfsys@useobject{currentmarker}{}%
\end{pgfscope}%
\begin{pgfscope}%
\pgfsys@transformshift{3.971181in}{2.336981in}%
\pgfsys@useobject{currentmarker}{}%
\end{pgfscope}%
\begin{pgfscope}%
\pgfsys@transformshift{3.956159in}{2.437249in}%
\pgfsys@useobject{currentmarker}{}%
\end{pgfscope}%
\begin{pgfscope}%
\pgfsys@transformshift{3.935737in}{2.609615in}%
\pgfsys@useobject{currentmarker}{}%
\end{pgfscope}%
\begin{pgfscope}%
\pgfsys@transformshift{3.919775in}{2.631719in}%
\pgfsys@useobject{currentmarker}{}%
\end{pgfscope}%
\begin{pgfscope}%
\pgfsys@transformshift{3.897476in}{2.512537in}%
\pgfsys@useobject{currentmarker}{}%
\end{pgfscope}%
\begin{pgfscope}%
\pgfsys@transformshift{3.878697in}{2.367537in}%
\pgfsys@useobject{currentmarker}{}%
\end{pgfscope}%
\begin{pgfscope}%
\pgfsys@transformshift{3.857104in}{2.282563in}%
\pgfsys@useobject{currentmarker}{}%
\end{pgfscope}%
\begin{pgfscope}%
\pgfsys@transformshift{3.841376in}{2.257895in}%
\pgfsys@useobject{currentmarker}{}%
\end{pgfscope}%
\begin{pgfscope}%
\pgfsys@transformshift{3.819311in}{2.246789in}%
\pgfsys@useobject{currentmarker}{}%
\end{pgfscope}%
\begin{pgfscope}%
\pgfsys@transformshift{3.801238in}{2.258265in}%
\pgfsys@useobject{currentmarker}{}%
\end{pgfscope}%
\begin{pgfscope}%
\pgfsys@transformshift{3.783399in}{2.288481in}%
\pgfsys@useobject{currentmarker}{}%
\end{pgfscope}%
\begin{pgfscope}%
\pgfsys@transformshift{3.764151in}{2.375954in}%
\pgfsys@useobject{currentmarker}{}%
\end{pgfscope}%
\begin{pgfscope}%
\pgfsys@transformshift{3.744432in}{2.528627in}%
\pgfsys@useobject{currentmarker}{}%
\end{pgfscope}%
\begin{pgfscope}%
\pgfsys@transformshift{3.726828in}{2.625236in}%
\pgfsys@useobject{currentmarker}{}%
\end{pgfscope}%
\begin{pgfscope}%
\pgfsys@transformshift{3.706877in}{2.593132in}%
\pgfsys@useobject{currentmarker}{}%
\end{pgfscope}%
\begin{pgfscope}%
\pgfsys@transformshift{3.688332in}{2.476116in}%
\pgfsys@useobject{currentmarker}{}%
\end{pgfscope}%
\begin{pgfscope}%
\pgfsys@transformshift{3.666738in}{2.374210in}%
\pgfsys@useobject{currentmarker}{}%
\end{pgfscope}%
\begin{pgfscope}%
\pgfsys@transformshift{3.647960in}{2.361260in}%
\pgfsys@useobject{currentmarker}{}%
\end{pgfscope}%
\begin{pgfscope}%
\pgfsys@transformshift{3.628712in}{2.287271in}%
\pgfsys@useobject{currentmarker}{}%
\end{pgfscope}%
\begin{pgfscope}%
\pgfsys@transformshift{3.610873in}{2.256353in}%
\pgfsys@useobject{currentmarker}{}%
\end{pgfscope}%
\begin{pgfscope}%
\pgfsys@transformshift{3.588808in}{2.246095in}%
\pgfsys@useobject{currentmarker}{}%
\end{pgfscope}%
\begin{pgfscope}%
\pgfsys@transformshift{3.570498in}{2.263730in}%
\pgfsys@useobject{currentmarker}{}%
\end{pgfscope}%
\begin{pgfscope}%
\pgfsys@transformshift{3.554536in}{2.304574in}%
\pgfsys@useobject{currentmarker}{}%
\end{pgfscope}%
\begin{pgfscope}%
\pgfsys@transformshift{3.533646in}{2.409601in}%
\pgfsys@useobject{currentmarker}{}%
\end{pgfscope}%
\begin{pgfscope}%
\pgfsys@transformshift{3.514632in}{2.450416in}%
\pgfsys@useobject{currentmarker}{}%
\end{pgfscope}%
\begin{pgfscope}%
\pgfsys@transformshift{3.495855in}{2.584883in}%
\pgfsys@useobject{currentmarker}{}%
\end{pgfscope}%
\begin{pgfscope}%
\pgfsys@transformshift{3.477311in}{2.277476in}%
\pgfsys@useobject{currentmarker}{}%
\end{pgfscope}%
\begin{pgfscope}%
\pgfsys@transformshift{3.454778in}{2.349598in}%
\pgfsys@useobject{currentmarker}{}%
\end{pgfscope}%
\begin{pgfscope}%
\pgfsys@transformshift{3.436938in}{2.486301in}%
\pgfsys@useobject{currentmarker}{}%
\end{pgfscope}%
\begin{pgfscope}%
\pgfsys@transformshift{3.418160in}{2.610041in}%
\pgfsys@useobject{currentmarker}{}%
\end{pgfscope}%
\begin{pgfscope}%
\pgfsys@transformshift{3.397972in}{2.589207in}%
\pgfsys@useobject{currentmarker}{}%
\end{pgfscope}%
\begin{pgfscope}%
\pgfsys@transformshift{3.378256in}{2.482500in}%
\pgfsys@useobject{currentmarker}{}%
\end{pgfscope}%
\begin{pgfscope}%
\pgfsys@transformshift{3.359477in}{2.351632in}%
\pgfsys@useobject{currentmarker}{}%
\end{pgfscope}%
\begin{pgfscope}%
\pgfsys@transformshift{3.341169in}{2.287793in}%
\pgfsys@useobject{currentmarker}{}%
\end{pgfscope}%
\begin{pgfscope}%
\pgfsys@transformshift{3.321921in}{2.255335in}%
\pgfsys@useobject{currentmarker}{}%
\end{pgfscope}%
\begin{pgfscope}%
\pgfsys@transformshift{3.300091in}{2.245291in}%
\pgfsys@useobject{currentmarker}{}%
\end{pgfscope}%
\begin{pgfscope}%
\pgfsys@transformshift{3.281078in}{2.255737in}%
\pgfsys@useobject{currentmarker}{}%
\end{pgfscope}%
\begin{pgfscope}%
\pgfsys@transformshift{3.262768in}{2.284851in}%
\pgfsys@useobject{currentmarker}{}%
\end{pgfscope}%
\begin{pgfscope}%
\pgfsys@transformshift{3.244225in}{2.353997in}%
\pgfsys@useobject{currentmarker}{}%
\end{pgfscope}%
\begin{pgfscope}%
\pgfsys@transformshift{3.227794in}{2.479425in}%
\pgfsys@useobject{currentmarker}{}%
\end{pgfscope}%
\begin{pgfscope}%
\pgfsys@transformshift{3.205259in}{2.606780in}%
\pgfsys@useobject{currentmarker}{}%
\end{pgfscope}%
\begin{pgfscope}%
\pgfsys@transformshift{3.189534in}{2.609255in}%
\pgfsys@useobject{currentmarker}{}%
\end{pgfscope}%
\begin{pgfscope}%
\pgfsys@transformshift{3.168172in}{2.491448in}%
\pgfsys@useobject{currentmarker}{}%
\end{pgfscope}%
\begin{pgfscope}%
\pgfsys@transformshift{3.148690in}{2.357472in}%
\pgfsys@useobject{currentmarker}{}%
\end{pgfscope}%
\begin{pgfscope}%
\pgfsys@transformshift{3.126626in}{2.281895in}%
\pgfsys@useobject{currentmarker}{}%
\end{pgfscope}%
\begin{pgfscope}%
\pgfsys@transformshift{3.112072in}{2.259310in}%
\pgfsys@useobject{currentmarker}{}%
\end{pgfscope}%
\begin{pgfscope}%
\pgfsys@transformshift{3.089068in}{2.244348in}%
\pgfsys@useobject{currentmarker}{}%
\end{pgfscope}%
\begin{pgfscope}%
\pgfsys@transformshift{3.073811in}{2.251923in}%
\pgfsys@useobject{currentmarker}{}%
\end{pgfscope}%
\begin{pgfscope}%
\pgfsys@transformshift{3.053155in}{2.280620in}%
\pgfsys@useobject{currentmarker}{}%
\end{pgfscope}%
\begin{pgfscope}%
\pgfsys@transformshift{3.034612in}{2.322525in}%
\pgfsys@useobject{currentmarker}{}%
\end{pgfscope}%
\begin{pgfscope}%
\pgfsys@transformshift{3.013017in}{2.425748in}%
\pgfsys@useobject{currentmarker}{}%
\end{pgfscope}%
\begin{pgfscope}%
\pgfsys@transformshift{2.993298in}{2.577844in}%
\pgfsys@useobject{currentmarker}{}%
\end{pgfscope}%
\begin{pgfscope}%
\pgfsys@transformshift{2.974756in}{2.615894in}%
\pgfsys@useobject{currentmarker}{}%
\end{pgfscope}%
\begin{pgfscope}%
\pgfsys@transformshift{2.954568in}{2.569409in}%
\pgfsys@useobject{currentmarker}{}%
\end{pgfscope}%
\begin{pgfscope}%
\pgfsys@transformshift{2.938372in}{2.469043in}%
\pgfsys@useobject{currentmarker}{}%
\end{pgfscope}%
\begin{pgfscope}%
\pgfsys@transformshift{2.918890in}{2.338123in}%
\pgfsys@useobject{currentmarker}{}%
\end{pgfscope}%
\begin{pgfscope}%
\pgfsys@transformshift{2.900111in}{2.345603in}%
\pgfsys@useobject{currentmarker}{}%
\end{pgfscope}%
\begin{pgfscope}%
\pgfsys@transformshift{2.879455in}{2.274790in}%
\pgfsys@useobject{currentmarker}{}%
\end{pgfscope}%
\begin{pgfscope}%
\pgfsys@transformshift{2.860913in}{2.251579in}%
\pgfsys@useobject{currentmarker}{}%
\end{pgfscope}%
\begin{pgfscope}%
\pgfsys@transformshift{2.839082in}{2.245886in}%
\pgfsys@useobject{currentmarker}{}%
\end{pgfscope}%
\begin{pgfscope}%
\pgfsys@transformshift{2.819364in}{2.263659in}%
\pgfsys@useobject{currentmarker}{}%
\end{pgfscope}%
\begin{pgfscope}%
\pgfsys@transformshift{2.805047in}{2.295292in}%
\pgfsys@useobject{currentmarker}{}%
\end{pgfscope}%
\begin{pgfscope}%
\pgfsys@transformshift{2.804576in}{2.339983in}%
\pgfsys@useobject{currentmarker}{}%
\end{pgfscope}%
\begin{pgfscope}%
\pgfsys@transformshift{2.783451in}{2.363404in}%
\pgfsys@useobject{currentmarker}{}%
\end{pgfscope}%
\begin{pgfscope}%
\pgfsys@transformshift{2.764438in}{2.492345in}%
\pgfsys@useobject{currentmarker}{}%
\end{pgfscope}%
\begin{pgfscope}%
\pgfsys@transformshift{2.746599in}{2.595923in}%
\pgfsys@useobject{currentmarker}{}%
\end{pgfscope}%
\begin{pgfscope}%
\pgfsys@transformshift{2.724534in}{2.597632in}%
\pgfsys@useobject{currentmarker}{}%
\end{pgfscope}%
\begin{pgfscope}%
\pgfsys@transformshift{2.705990in}{2.495170in}%
\pgfsys@useobject{currentmarker}{}%
\end{pgfscope}%
\begin{pgfscope}%
\pgfsys@transformshift{2.684161in}{2.345907in}%
\pgfsys@useobject{currentmarker}{}%
\end{pgfscope}%
\begin{pgfscope}%
\pgfsys@transformshift{2.668668in}{2.295759in}%
\pgfsys@useobject{currentmarker}{}%
\end{pgfscope}%
\begin{pgfscope}%
\pgfsys@transformshift{2.648717in}{2.256888in}%
\pgfsys@useobject{currentmarker}{}%
\end{pgfscope}%
\begin{pgfscope}%
\pgfsys@transformshift{2.627590in}{2.244538in}%
\pgfsys@useobject{currentmarker}{}%
\end{pgfscope}%
\begin{pgfscope}%
\pgfsys@transformshift{2.612099in}{2.249515in}%
\pgfsys@useobject{currentmarker}{}%
\end{pgfscope}%
\begin{pgfscope}%
\pgfsys@transformshift{2.593555in}{2.274222in}%
\pgfsys@useobject{currentmarker}{}%
\end{pgfscope}%
\begin{pgfscope}%
\pgfsys@transformshift{2.566796in}{2.347759in}%
\pgfsys@useobject{currentmarker}{}%
\end{pgfscope}%
\begin{pgfscope}%
\pgfsys@transformshift{2.549894in}{2.448879in}%
\pgfsys@useobject{currentmarker}{}%
\end{pgfscope}%
\begin{pgfscope}%
\pgfsys@transformshift{2.532055in}{2.570617in}%
\pgfsys@useobject{currentmarker}{}%
\end{pgfscope}%
\begin{pgfscope}%
\pgfsys@transformshift{2.513513in}{2.611519in}%
\pgfsys@useobject{currentmarker}{}%
\end{pgfscope}%
\begin{pgfscope}%
\pgfsys@transformshift{2.494734in}{2.550383in}%
\pgfsys@useobject{currentmarker}{}%
\end{pgfscope}%
\begin{pgfscope}%
\pgfsys@transformshift{2.476426in}{2.457061in}%
\pgfsys@useobject{currentmarker}{}%
\end{pgfscope}%
\begin{pgfscope}%
\pgfsys@transformshift{2.455533in}{2.336961in}%
\pgfsys@useobject{currentmarker}{}%
\end{pgfscope}%
\begin{pgfscope}%
\pgfsys@transformshift{2.438634in}{2.287041in}%
\pgfsys@useobject{currentmarker}{}%
\end{pgfscope}%
\begin{pgfscope}%
\pgfsys@transformshift{2.417272in}{2.259927in}%
\pgfsys@useobject{currentmarker}{}%
\end{pgfscope}%
\begin{pgfscope}%
\pgfsys@transformshift{2.399199in}{2.250351in}%
\pgfsys@useobject{currentmarker}{}%
\end{pgfscope}%
\begin{pgfscope}%
\pgfsys@transformshift{2.380656in}{2.244817in}%
\pgfsys@useobject{currentmarker}{}%
\end{pgfscope}%
\begin{pgfscope}%
\pgfsys@transformshift{2.359529in}{2.258847in}%
\pgfsys@useobject{currentmarker}{}%
\end{pgfscope}%
\begin{pgfscope}%
\pgfsys@transformshift{2.340282in}{2.289610in}%
\pgfsys@useobject{currentmarker}{}%
\end{pgfscope}%
\begin{pgfscope}%
\pgfsys@transformshift{2.321974in}{2.348395in}%
\pgfsys@useobject{currentmarker}{}%
\end{pgfscope}%
\begin{pgfscope}%
\pgfsys@transformshift{2.303429in}{2.468662in}%
\pgfsys@useobject{currentmarker}{}%
\end{pgfscope}%
\begin{pgfscope}%
\pgfsys@transformshift{2.282304in}{2.581826in}%
\pgfsys@useobject{currentmarker}{}%
\end{pgfscope}%
\begin{pgfscope}%
\pgfsys@transformshift{2.265873in}{2.614100in}%
\pgfsys@useobject{currentmarker}{}%
\end{pgfscope}%
\begin{pgfscope}%
\pgfsys@transformshift{2.246155in}{2.539176in}%
\pgfsys@useobject{currentmarker}{}%
\end{pgfscope}%
\begin{pgfscope}%
\pgfsys@transformshift{2.225970in}{2.425604in}%
\pgfsys@useobject{currentmarker}{}%
\end{pgfscope}%
\begin{pgfscope}%
\pgfsys@transformshift{2.205077in}{2.349464in}%
\pgfsys@useobject{currentmarker}{}%
\end{pgfscope}%
\begin{pgfscope}%
\pgfsys@transformshift{2.186300in}{2.290474in}%
\pgfsys@useobject{currentmarker}{}%
\end{pgfscope}%
\begin{pgfscope}%
\pgfsys@transformshift{2.167990in}{2.274060in}%
\pgfsys@useobject{currentmarker}{}%
\end{pgfscope}%
\begin{pgfscope}%
\pgfsys@transformshift{2.148274in}{2.258632in}%
\pgfsys@useobject{currentmarker}{}%
\end{pgfscope}%
\begin{pgfscope}%
\pgfsys@transformshift{2.130200in}{2.251261in}%
\pgfsys@useobject{currentmarker}{}%
\end{pgfscope}%
\begin{pgfscope}%
\pgfsys@transformshift{2.112359in}{2.246352in}%
\pgfsys@useobject{currentmarker}{}%
\end{pgfscope}%
\begin{pgfscope}%
\pgfsys@transformshift{2.090060in}{2.260576in}%
\pgfsys@useobject{currentmarker}{}%
\end{pgfscope}%
\begin{pgfscope}%
\pgfsys@transformshift{2.071283in}{2.295772in}%
\pgfsys@useobject{currentmarker}{}%
\end{pgfscope}%
\begin{pgfscope}%
\pgfsys@transformshift{2.053913in}{2.375070in}%
\pgfsys@useobject{currentmarker}{}%
\end{pgfscope}%
\begin{pgfscope}%
\pgfsys@transformshift{2.036542in}{2.528986in}%
\pgfsys@useobject{currentmarker}{}%
\end{pgfscope}%
\begin{pgfscope}%
\pgfsys@transformshift{2.016589in}{2.609055in}%
\pgfsys@useobject{currentmarker}{}%
\end{pgfscope}%
\begin{pgfscope}%
\pgfsys@transformshift{1.994056in}{2.607245in}%
\pgfsys@useobject{currentmarker}{}%
\end{pgfscope}%
\begin{pgfscope}%
\pgfsys@transformshift{1.975982in}{2.517219in}%
\pgfsys@useobject{currentmarker}{}%
\end{pgfscope}%
\begin{pgfscope}%
\pgfsys@transformshift{1.957438in}{2.409554in}%
\pgfsys@useobject{currentmarker}{}%
\end{pgfscope}%
\begin{pgfscope}%
\pgfsys@transformshift{1.935373in}{2.312886in}%
\pgfsys@useobject{currentmarker}{}%
\end{pgfscope}%
\begin{pgfscope}%
\pgfsys@transformshift{1.917768in}{2.275345in}%
\pgfsys@useobject{currentmarker}{}%
\end{pgfscope}%
\begin{pgfscope}%
\pgfsys@transformshift{1.900164in}{2.257926in}%
\pgfsys@useobject{currentmarker}{}%
\end{pgfscope}%
\begin{pgfscope}%
\pgfsys@transformshift{1.876456in}{2.247687in}%
\pgfsys@useobject{currentmarker}{}%
\end{pgfscope}%
\begin{pgfscope}%
\pgfsys@transformshift{1.859088in}{2.255441in}%
\pgfsys@useobject{currentmarker}{}%
\end{pgfscope}%
\begin{pgfscope}%
\pgfsys@transformshift{1.840778in}{2.280128in}%
\pgfsys@useobject{currentmarker}{}%
\end{pgfscope}%
\begin{pgfscope}%
\pgfsys@transformshift{1.822235in}{2.321988in}%
\pgfsys@useobject{currentmarker}{}%
\end{pgfscope}%
\begin{pgfscope}%
\pgfsys@transformshift{1.803456in}{2.378434in}%
\pgfsys@useobject{currentmarker}{}%
\end{pgfscope}%
\begin{pgfscope}%
\pgfsys@transformshift{1.782330in}{2.512453in}%
\pgfsys@useobject{currentmarker}{}%
\end{pgfscope}%
\begin{pgfscope}%
\pgfsys@transformshift{1.763787in}{2.469421in}%
\pgfsys@useobject{currentmarker}{}%
\end{pgfscope}%
\begin{pgfscope}%
\pgfsys@transformshift{1.747122in}{2.587435in}%
\pgfsys@useobject{currentmarker}{}%
\end{pgfscope}%
\begin{pgfscope}%
\pgfsys@transformshift{1.726700in}{2.626316in}%
\pgfsys@useobject{currentmarker}{}%
\end{pgfscope}%
\begin{pgfscope}%
\pgfsys@transformshift{1.707687in}{2.579844in}%
\pgfsys@useobject{currentmarker}{}%
\end{pgfscope}%
\begin{pgfscope}%
\pgfsys@transformshift{1.686560in}{2.492008in}%
\pgfsys@useobject{currentmarker}{}%
\end{pgfscope}%
\begin{pgfscope}%
\pgfsys@transformshift{1.670834in}{2.410187in}%
\pgfsys@useobject{currentmarker}{}%
\end{pgfscope}%
\begin{pgfscope}%
\pgfsys@transformshift{1.648535in}{2.612440in}%
\pgfsys@useobject{currentmarker}{}%
\end{pgfscope}%
\begin{pgfscope}%
\pgfsys@transformshift{1.630460in}{2.615505in}%
\pgfsys@useobject{currentmarker}{}%
\end{pgfscope}%
\begin{pgfscope}%
\pgfsys@transformshift{1.612152in}{2.527776in}%
\pgfsys@useobject{currentmarker}{}%
\end{pgfscope}%
\end{pgfscope}%
\begin{pgfscope}%
\pgfsetrectcap%
\pgfsetmiterjoin%
\pgfsetlinewidth{0.501875pt}%
\definecolor{currentstroke}{rgb}{0.000000,0.000000,0.000000}%
\pgfsetstrokecolor{currentstroke}%
\pgfsetdash{}{0pt}%
\pgfpathmoveto{\pgfqpoint{0.444748in}{2.222124in}}%
\pgfpathlineto{\pgfqpoint{0.444748in}{2.689374in}}%
\pgfusepath{stroke}%
\end{pgfscope}%
\begin{pgfscope}%
\pgfsetrectcap%
\pgfsetmiterjoin%
\pgfsetlinewidth{0.501875pt}%
\definecolor{currentstroke}{rgb}{0.000000,0.000000,0.000000}%
\pgfsetstrokecolor{currentstroke}%
\pgfsetdash{}{0pt}%
\pgfpathmoveto{\pgfqpoint{4.676167in}{2.222124in}}%
\pgfpathlineto{\pgfqpoint{4.676167in}{2.689374in}}%
\pgfusepath{stroke}%
\end{pgfscope}%
\begin{pgfscope}%
\pgfsetrectcap%
\pgfsetmiterjoin%
\pgfsetlinewidth{0.501875pt}%
\definecolor{currentstroke}{rgb}{0.000000,0.000000,0.000000}%
\pgfsetstrokecolor{currentstroke}%
\pgfsetdash{}{0pt}%
\pgfpathmoveto{\pgfqpoint{0.444748in}{2.222124in}}%
\pgfpathlineto{\pgfqpoint{4.676167in}{2.222124in}}%
\pgfusepath{stroke}%
\end{pgfscope}%
\begin{pgfscope}%
\pgfsetrectcap%
\pgfsetmiterjoin%
\pgfsetlinewidth{0.501875pt}%
\definecolor{currentstroke}{rgb}{0.000000,0.000000,0.000000}%
\pgfsetstrokecolor{currentstroke}%
\pgfsetdash{}{0pt}%
\pgfpathmoveto{\pgfqpoint{0.444748in}{2.689374in}}%
\pgfpathlineto{\pgfqpoint{4.676167in}{2.689374in}}%
\pgfusepath{stroke}%
\end{pgfscope}%
\begin{pgfscope}%
\definecolor{textcolor}{rgb}{0.000000,0.000000,0.000000}%
\pgfsetstrokecolor{textcolor}%
\pgfsetfillcolor{textcolor}%
\pgftext[x=2.560458in,y=2.772708in,,base]{\color{textcolor}\rmfamily\fontsize{12.000000}{14.400000}\selectfont T = \qty{3.2}{\kelvin}}%
\end{pgfscope}%
\begin{pgfscope}%
\pgfsetbuttcap%
\pgfsetmiterjoin%
\definecolor{currentfill}{rgb}{1.000000,1.000000,1.000000}%
\pgfsetfillcolor{currentfill}%
\pgfsetlinewidth{0.000000pt}%
\definecolor{currentstroke}{rgb}{0.000000,0.000000,0.000000}%
\pgfsetstrokecolor{currentstroke}%
\pgfsetstrokeopacity{0.000000}%
\pgfsetdash{}{0pt}%
\pgfpathmoveto{\pgfqpoint{0.444748in}{1.326898in}}%
\pgfpathlineto{\pgfqpoint{4.676167in}{1.326898in}}%
\pgfpathlineto{\pgfqpoint{4.676167in}{1.794149in}}%
\pgfpathlineto{\pgfqpoint{0.444748in}{1.794149in}}%
\pgfpathlineto{\pgfqpoint{0.444748in}{1.326898in}}%
\pgfpathclose%
\pgfusepath{fill}%
\end{pgfscope}%
\begin{pgfscope}%
\pgfsetbuttcap%
\pgfsetroundjoin%
\definecolor{currentfill}{rgb}{0.000000,0.000000,0.000000}%
\pgfsetfillcolor{currentfill}%
\pgfsetlinewidth{0.501875pt}%
\definecolor{currentstroke}{rgb}{0.000000,0.000000,0.000000}%
\pgfsetstrokecolor{currentstroke}%
\pgfsetdash{}{0pt}%
\pgfsys@defobject{currentmarker}{\pgfqpoint{0.000000in}{0.000000in}}{\pgfqpoint{0.000000in}{0.041667in}}{%
\pgfpathmoveto{\pgfqpoint{0.000000in}{0.000000in}}%
\pgfpathlineto{\pgfqpoint{0.000000in}{0.041667in}}%
\pgfusepath{stroke,fill}%
}%
\begin{pgfscope}%
\pgfsys@transformshift{0.643886in}{1.326898in}%
\pgfsys@useobject{currentmarker}{}%
\end{pgfscope}%
\end{pgfscope}%
\begin{pgfscope}%
\pgfsetbuttcap%
\pgfsetroundjoin%
\definecolor{currentfill}{rgb}{0.000000,0.000000,0.000000}%
\pgfsetfillcolor{currentfill}%
\pgfsetlinewidth{0.501875pt}%
\definecolor{currentstroke}{rgb}{0.000000,0.000000,0.000000}%
\pgfsetstrokecolor{currentstroke}%
\pgfsetdash{}{0pt}%
\pgfsys@defobject{currentmarker}{\pgfqpoint{0.000000in}{-0.041667in}}{\pgfqpoint{0.000000in}{0.000000in}}{%
\pgfpathmoveto{\pgfqpoint{0.000000in}{0.000000in}}%
\pgfpathlineto{\pgfqpoint{0.000000in}{-0.041667in}}%
\pgfusepath{stroke,fill}%
}%
\begin{pgfscope}%
\pgfsys@transformshift{0.643886in}{1.794149in}%
\pgfsys@useobject{currentmarker}{}%
\end{pgfscope}%
\end{pgfscope}%
\begin{pgfscope}%
\pgfsetbuttcap%
\pgfsetroundjoin%
\definecolor{currentfill}{rgb}{0.000000,0.000000,0.000000}%
\pgfsetfillcolor{currentfill}%
\pgfsetlinewidth{0.501875pt}%
\definecolor{currentstroke}{rgb}{0.000000,0.000000,0.000000}%
\pgfsetstrokecolor{currentstroke}%
\pgfsetdash{}{0pt}%
\pgfsys@defobject{currentmarker}{\pgfqpoint{0.000000in}{0.000000in}}{\pgfqpoint{0.000000in}{0.041667in}}{%
\pgfpathmoveto{\pgfqpoint{0.000000in}{0.000000in}}%
\pgfpathlineto{\pgfqpoint{0.000000in}{0.041667in}}%
\pgfusepath{stroke,fill}%
}%
\begin{pgfscope}%
\pgfsys@transformshift{1.124261in}{1.326898in}%
\pgfsys@useobject{currentmarker}{}%
\end{pgfscope}%
\end{pgfscope}%
\begin{pgfscope}%
\pgfsetbuttcap%
\pgfsetroundjoin%
\definecolor{currentfill}{rgb}{0.000000,0.000000,0.000000}%
\pgfsetfillcolor{currentfill}%
\pgfsetlinewidth{0.501875pt}%
\definecolor{currentstroke}{rgb}{0.000000,0.000000,0.000000}%
\pgfsetstrokecolor{currentstroke}%
\pgfsetdash{}{0pt}%
\pgfsys@defobject{currentmarker}{\pgfqpoint{0.000000in}{-0.041667in}}{\pgfqpoint{0.000000in}{0.000000in}}{%
\pgfpathmoveto{\pgfqpoint{0.000000in}{0.000000in}}%
\pgfpathlineto{\pgfqpoint{0.000000in}{-0.041667in}}%
\pgfusepath{stroke,fill}%
}%
\begin{pgfscope}%
\pgfsys@transformshift{1.124261in}{1.794149in}%
\pgfsys@useobject{currentmarker}{}%
\end{pgfscope}%
\end{pgfscope}%
\begin{pgfscope}%
\pgfsetbuttcap%
\pgfsetroundjoin%
\definecolor{currentfill}{rgb}{0.000000,0.000000,0.000000}%
\pgfsetfillcolor{currentfill}%
\pgfsetlinewidth{0.501875pt}%
\definecolor{currentstroke}{rgb}{0.000000,0.000000,0.000000}%
\pgfsetstrokecolor{currentstroke}%
\pgfsetdash{}{0pt}%
\pgfsys@defobject{currentmarker}{\pgfqpoint{0.000000in}{0.000000in}}{\pgfqpoint{0.000000in}{0.041667in}}{%
\pgfpathmoveto{\pgfqpoint{0.000000in}{0.000000in}}%
\pgfpathlineto{\pgfqpoint{0.000000in}{0.041667in}}%
\pgfusepath{stroke,fill}%
}%
\begin{pgfscope}%
\pgfsys@transformshift{1.604637in}{1.326898in}%
\pgfsys@useobject{currentmarker}{}%
\end{pgfscope}%
\end{pgfscope}%
\begin{pgfscope}%
\pgfsetbuttcap%
\pgfsetroundjoin%
\definecolor{currentfill}{rgb}{0.000000,0.000000,0.000000}%
\pgfsetfillcolor{currentfill}%
\pgfsetlinewidth{0.501875pt}%
\definecolor{currentstroke}{rgb}{0.000000,0.000000,0.000000}%
\pgfsetstrokecolor{currentstroke}%
\pgfsetdash{}{0pt}%
\pgfsys@defobject{currentmarker}{\pgfqpoint{0.000000in}{-0.041667in}}{\pgfqpoint{0.000000in}{0.000000in}}{%
\pgfpathmoveto{\pgfqpoint{0.000000in}{0.000000in}}%
\pgfpathlineto{\pgfqpoint{0.000000in}{-0.041667in}}%
\pgfusepath{stroke,fill}%
}%
\begin{pgfscope}%
\pgfsys@transformshift{1.604637in}{1.794149in}%
\pgfsys@useobject{currentmarker}{}%
\end{pgfscope}%
\end{pgfscope}%
\begin{pgfscope}%
\pgfsetbuttcap%
\pgfsetroundjoin%
\definecolor{currentfill}{rgb}{0.000000,0.000000,0.000000}%
\pgfsetfillcolor{currentfill}%
\pgfsetlinewidth{0.501875pt}%
\definecolor{currentstroke}{rgb}{0.000000,0.000000,0.000000}%
\pgfsetstrokecolor{currentstroke}%
\pgfsetdash{}{0pt}%
\pgfsys@defobject{currentmarker}{\pgfqpoint{0.000000in}{0.000000in}}{\pgfqpoint{0.000000in}{0.041667in}}{%
\pgfpathmoveto{\pgfqpoint{0.000000in}{0.000000in}}%
\pgfpathlineto{\pgfqpoint{0.000000in}{0.041667in}}%
\pgfusepath{stroke,fill}%
}%
\begin{pgfscope}%
\pgfsys@transformshift{2.085012in}{1.326898in}%
\pgfsys@useobject{currentmarker}{}%
\end{pgfscope}%
\end{pgfscope}%
\begin{pgfscope}%
\pgfsetbuttcap%
\pgfsetroundjoin%
\definecolor{currentfill}{rgb}{0.000000,0.000000,0.000000}%
\pgfsetfillcolor{currentfill}%
\pgfsetlinewidth{0.501875pt}%
\definecolor{currentstroke}{rgb}{0.000000,0.000000,0.000000}%
\pgfsetstrokecolor{currentstroke}%
\pgfsetdash{}{0pt}%
\pgfsys@defobject{currentmarker}{\pgfqpoint{0.000000in}{-0.041667in}}{\pgfqpoint{0.000000in}{0.000000in}}{%
\pgfpathmoveto{\pgfqpoint{0.000000in}{0.000000in}}%
\pgfpathlineto{\pgfqpoint{0.000000in}{-0.041667in}}%
\pgfusepath{stroke,fill}%
}%
\begin{pgfscope}%
\pgfsys@transformshift{2.085012in}{1.794149in}%
\pgfsys@useobject{currentmarker}{}%
\end{pgfscope}%
\end{pgfscope}%
\begin{pgfscope}%
\pgfsetbuttcap%
\pgfsetroundjoin%
\definecolor{currentfill}{rgb}{0.000000,0.000000,0.000000}%
\pgfsetfillcolor{currentfill}%
\pgfsetlinewidth{0.501875pt}%
\definecolor{currentstroke}{rgb}{0.000000,0.000000,0.000000}%
\pgfsetstrokecolor{currentstroke}%
\pgfsetdash{}{0pt}%
\pgfsys@defobject{currentmarker}{\pgfqpoint{0.000000in}{0.000000in}}{\pgfqpoint{0.000000in}{0.041667in}}{%
\pgfpathmoveto{\pgfqpoint{0.000000in}{0.000000in}}%
\pgfpathlineto{\pgfqpoint{0.000000in}{0.041667in}}%
\pgfusepath{stroke,fill}%
}%
\begin{pgfscope}%
\pgfsys@transformshift{2.565388in}{1.326898in}%
\pgfsys@useobject{currentmarker}{}%
\end{pgfscope}%
\end{pgfscope}%
\begin{pgfscope}%
\pgfsetbuttcap%
\pgfsetroundjoin%
\definecolor{currentfill}{rgb}{0.000000,0.000000,0.000000}%
\pgfsetfillcolor{currentfill}%
\pgfsetlinewidth{0.501875pt}%
\definecolor{currentstroke}{rgb}{0.000000,0.000000,0.000000}%
\pgfsetstrokecolor{currentstroke}%
\pgfsetdash{}{0pt}%
\pgfsys@defobject{currentmarker}{\pgfqpoint{0.000000in}{-0.041667in}}{\pgfqpoint{0.000000in}{0.000000in}}{%
\pgfpathmoveto{\pgfqpoint{0.000000in}{0.000000in}}%
\pgfpathlineto{\pgfqpoint{0.000000in}{-0.041667in}}%
\pgfusepath{stroke,fill}%
}%
\begin{pgfscope}%
\pgfsys@transformshift{2.565388in}{1.794149in}%
\pgfsys@useobject{currentmarker}{}%
\end{pgfscope}%
\end{pgfscope}%
\begin{pgfscope}%
\pgfsetbuttcap%
\pgfsetroundjoin%
\definecolor{currentfill}{rgb}{0.000000,0.000000,0.000000}%
\pgfsetfillcolor{currentfill}%
\pgfsetlinewidth{0.501875pt}%
\definecolor{currentstroke}{rgb}{0.000000,0.000000,0.000000}%
\pgfsetstrokecolor{currentstroke}%
\pgfsetdash{}{0pt}%
\pgfsys@defobject{currentmarker}{\pgfqpoint{0.000000in}{0.000000in}}{\pgfqpoint{0.000000in}{0.041667in}}{%
\pgfpathmoveto{\pgfqpoint{0.000000in}{0.000000in}}%
\pgfpathlineto{\pgfqpoint{0.000000in}{0.041667in}}%
\pgfusepath{stroke,fill}%
}%
\begin{pgfscope}%
\pgfsys@transformshift{3.045763in}{1.326898in}%
\pgfsys@useobject{currentmarker}{}%
\end{pgfscope}%
\end{pgfscope}%
\begin{pgfscope}%
\pgfsetbuttcap%
\pgfsetroundjoin%
\definecolor{currentfill}{rgb}{0.000000,0.000000,0.000000}%
\pgfsetfillcolor{currentfill}%
\pgfsetlinewidth{0.501875pt}%
\definecolor{currentstroke}{rgb}{0.000000,0.000000,0.000000}%
\pgfsetstrokecolor{currentstroke}%
\pgfsetdash{}{0pt}%
\pgfsys@defobject{currentmarker}{\pgfqpoint{0.000000in}{-0.041667in}}{\pgfqpoint{0.000000in}{0.000000in}}{%
\pgfpathmoveto{\pgfqpoint{0.000000in}{0.000000in}}%
\pgfpathlineto{\pgfqpoint{0.000000in}{-0.041667in}}%
\pgfusepath{stroke,fill}%
}%
\begin{pgfscope}%
\pgfsys@transformshift{3.045763in}{1.794149in}%
\pgfsys@useobject{currentmarker}{}%
\end{pgfscope}%
\end{pgfscope}%
\begin{pgfscope}%
\pgfsetbuttcap%
\pgfsetroundjoin%
\definecolor{currentfill}{rgb}{0.000000,0.000000,0.000000}%
\pgfsetfillcolor{currentfill}%
\pgfsetlinewidth{0.501875pt}%
\definecolor{currentstroke}{rgb}{0.000000,0.000000,0.000000}%
\pgfsetstrokecolor{currentstroke}%
\pgfsetdash{}{0pt}%
\pgfsys@defobject{currentmarker}{\pgfqpoint{0.000000in}{0.000000in}}{\pgfqpoint{0.000000in}{0.041667in}}{%
\pgfpathmoveto{\pgfqpoint{0.000000in}{0.000000in}}%
\pgfpathlineto{\pgfqpoint{0.000000in}{0.041667in}}%
\pgfusepath{stroke,fill}%
}%
\begin{pgfscope}%
\pgfsys@transformshift{3.526138in}{1.326898in}%
\pgfsys@useobject{currentmarker}{}%
\end{pgfscope}%
\end{pgfscope}%
\begin{pgfscope}%
\pgfsetbuttcap%
\pgfsetroundjoin%
\definecolor{currentfill}{rgb}{0.000000,0.000000,0.000000}%
\pgfsetfillcolor{currentfill}%
\pgfsetlinewidth{0.501875pt}%
\definecolor{currentstroke}{rgb}{0.000000,0.000000,0.000000}%
\pgfsetstrokecolor{currentstroke}%
\pgfsetdash{}{0pt}%
\pgfsys@defobject{currentmarker}{\pgfqpoint{0.000000in}{-0.041667in}}{\pgfqpoint{0.000000in}{0.000000in}}{%
\pgfpathmoveto{\pgfqpoint{0.000000in}{0.000000in}}%
\pgfpathlineto{\pgfqpoint{0.000000in}{-0.041667in}}%
\pgfusepath{stroke,fill}%
}%
\begin{pgfscope}%
\pgfsys@transformshift{3.526138in}{1.794149in}%
\pgfsys@useobject{currentmarker}{}%
\end{pgfscope}%
\end{pgfscope}%
\begin{pgfscope}%
\pgfsetbuttcap%
\pgfsetroundjoin%
\definecolor{currentfill}{rgb}{0.000000,0.000000,0.000000}%
\pgfsetfillcolor{currentfill}%
\pgfsetlinewidth{0.501875pt}%
\definecolor{currentstroke}{rgb}{0.000000,0.000000,0.000000}%
\pgfsetstrokecolor{currentstroke}%
\pgfsetdash{}{0pt}%
\pgfsys@defobject{currentmarker}{\pgfqpoint{0.000000in}{0.000000in}}{\pgfqpoint{0.000000in}{0.041667in}}{%
\pgfpathmoveto{\pgfqpoint{0.000000in}{0.000000in}}%
\pgfpathlineto{\pgfqpoint{0.000000in}{0.041667in}}%
\pgfusepath{stroke,fill}%
}%
\begin{pgfscope}%
\pgfsys@transformshift{4.006514in}{1.326898in}%
\pgfsys@useobject{currentmarker}{}%
\end{pgfscope}%
\end{pgfscope}%
\begin{pgfscope}%
\pgfsetbuttcap%
\pgfsetroundjoin%
\definecolor{currentfill}{rgb}{0.000000,0.000000,0.000000}%
\pgfsetfillcolor{currentfill}%
\pgfsetlinewidth{0.501875pt}%
\definecolor{currentstroke}{rgb}{0.000000,0.000000,0.000000}%
\pgfsetstrokecolor{currentstroke}%
\pgfsetdash{}{0pt}%
\pgfsys@defobject{currentmarker}{\pgfqpoint{0.000000in}{-0.041667in}}{\pgfqpoint{0.000000in}{0.000000in}}{%
\pgfpathmoveto{\pgfqpoint{0.000000in}{0.000000in}}%
\pgfpathlineto{\pgfqpoint{0.000000in}{-0.041667in}}%
\pgfusepath{stroke,fill}%
}%
\begin{pgfscope}%
\pgfsys@transformshift{4.006514in}{1.794149in}%
\pgfsys@useobject{currentmarker}{}%
\end{pgfscope}%
\end{pgfscope}%
\begin{pgfscope}%
\pgfsetbuttcap%
\pgfsetroundjoin%
\definecolor{currentfill}{rgb}{0.000000,0.000000,0.000000}%
\pgfsetfillcolor{currentfill}%
\pgfsetlinewidth{0.501875pt}%
\definecolor{currentstroke}{rgb}{0.000000,0.000000,0.000000}%
\pgfsetstrokecolor{currentstroke}%
\pgfsetdash{}{0pt}%
\pgfsys@defobject{currentmarker}{\pgfqpoint{0.000000in}{0.000000in}}{\pgfqpoint{0.000000in}{0.041667in}}{%
\pgfpathmoveto{\pgfqpoint{0.000000in}{0.000000in}}%
\pgfpathlineto{\pgfqpoint{0.000000in}{0.041667in}}%
\pgfusepath{stroke,fill}%
}%
\begin{pgfscope}%
\pgfsys@transformshift{4.486889in}{1.326898in}%
\pgfsys@useobject{currentmarker}{}%
\end{pgfscope}%
\end{pgfscope}%
\begin{pgfscope}%
\pgfsetbuttcap%
\pgfsetroundjoin%
\definecolor{currentfill}{rgb}{0.000000,0.000000,0.000000}%
\pgfsetfillcolor{currentfill}%
\pgfsetlinewidth{0.501875pt}%
\definecolor{currentstroke}{rgb}{0.000000,0.000000,0.000000}%
\pgfsetstrokecolor{currentstroke}%
\pgfsetdash{}{0pt}%
\pgfsys@defobject{currentmarker}{\pgfqpoint{0.000000in}{-0.041667in}}{\pgfqpoint{0.000000in}{0.000000in}}{%
\pgfpathmoveto{\pgfqpoint{0.000000in}{0.000000in}}%
\pgfpathlineto{\pgfqpoint{0.000000in}{-0.041667in}}%
\pgfusepath{stroke,fill}%
}%
\begin{pgfscope}%
\pgfsys@transformshift{4.486889in}{1.794149in}%
\pgfsys@useobject{currentmarker}{}%
\end{pgfscope}%
\end{pgfscope}%
\begin{pgfscope}%
\pgfsetbuttcap%
\pgfsetroundjoin%
\definecolor{currentfill}{rgb}{0.000000,0.000000,0.000000}%
\pgfsetfillcolor{currentfill}%
\pgfsetlinewidth{0.501875pt}%
\definecolor{currentstroke}{rgb}{0.000000,0.000000,0.000000}%
\pgfsetstrokecolor{currentstroke}%
\pgfsetdash{}{0pt}%
\pgfsys@defobject{currentmarker}{\pgfqpoint{0.000000in}{0.000000in}}{\pgfqpoint{0.000000in}{0.020833in}}{%
\pgfpathmoveto{\pgfqpoint{0.000000in}{0.000000in}}%
\pgfpathlineto{\pgfqpoint{0.000000in}{0.020833in}}%
\pgfusepath{stroke,fill}%
}%
\begin{pgfscope}%
\pgfsys@transformshift{0.451736in}{1.326898in}%
\pgfsys@useobject{currentmarker}{}%
\end{pgfscope}%
\end{pgfscope}%
\begin{pgfscope}%
\pgfsetbuttcap%
\pgfsetroundjoin%
\definecolor{currentfill}{rgb}{0.000000,0.000000,0.000000}%
\pgfsetfillcolor{currentfill}%
\pgfsetlinewidth{0.501875pt}%
\definecolor{currentstroke}{rgb}{0.000000,0.000000,0.000000}%
\pgfsetstrokecolor{currentstroke}%
\pgfsetdash{}{0pt}%
\pgfsys@defobject{currentmarker}{\pgfqpoint{0.000000in}{-0.020833in}}{\pgfqpoint{0.000000in}{0.000000in}}{%
\pgfpathmoveto{\pgfqpoint{0.000000in}{0.000000in}}%
\pgfpathlineto{\pgfqpoint{0.000000in}{-0.020833in}}%
\pgfusepath{stroke,fill}%
}%
\begin{pgfscope}%
\pgfsys@transformshift{0.451736in}{1.794149in}%
\pgfsys@useobject{currentmarker}{}%
\end{pgfscope}%
\end{pgfscope}%
\begin{pgfscope}%
\pgfsetbuttcap%
\pgfsetroundjoin%
\definecolor{currentfill}{rgb}{0.000000,0.000000,0.000000}%
\pgfsetfillcolor{currentfill}%
\pgfsetlinewidth{0.501875pt}%
\definecolor{currentstroke}{rgb}{0.000000,0.000000,0.000000}%
\pgfsetstrokecolor{currentstroke}%
\pgfsetdash{}{0pt}%
\pgfsys@defobject{currentmarker}{\pgfqpoint{0.000000in}{0.000000in}}{\pgfqpoint{0.000000in}{0.020833in}}{%
\pgfpathmoveto{\pgfqpoint{0.000000in}{0.000000in}}%
\pgfpathlineto{\pgfqpoint{0.000000in}{0.020833in}}%
\pgfusepath{stroke,fill}%
}%
\begin{pgfscope}%
\pgfsys@transformshift{0.547811in}{1.326898in}%
\pgfsys@useobject{currentmarker}{}%
\end{pgfscope}%
\end{pgfscope}%
\begin{pgfscope}%
\pgfsetbuttcap%
\pgfsetroundjoin%
\definecolor{currentfill}{rgb}{0.000000,0.000000,0.000000}%
\pgfsetfillcolor{currentfill}%
\pgfsetlinewidth{0.501875pt}%
\definecolor{currentstroke}{rgb}{0.000000,0.000000,0.000000}%
\pgfsetstrokecolor{currentstroke}%
\pgfsetdash{}{0pt}%
\pgfsys@defobject{currentmarker}{\pgfqpoint{0.000000in}{-0.020833in}}{\pgfqpoint{0.000000in}{0.000000in}}{%
\pgfpathmoveto{\pgfqpoint{0.000000in}{0.000000in}}%
\pgfpathlineto{\pgfqpoint{0.000000in}{-0.020833in}}%
\pgfusepath{stroke,fill}%
}%
\begin{pgfscope}%
\pgfsys@transformshift{0.547811in}{1.794149in}%
\pgfsys@useobject{currentmarker}{}%
\end{pgfscope}%
\end{pgfscope}%
\begin{pgfscope}%
\pgfsetbuttcap%
\pgfsetroundjoin%
\definecolor{currentfill}{rgb}{0.000000,0.000000,0.000000}%
\pgfsetfillcolor{currentfill}%
\pgfsetlinewidth{0.501875pt}%
\definecolor{currentstroke}{rgb}{0.000000,0.000000,0.000000}%
\pgfsetstrokecolor{currentstroke}%
\pgfsetdash{}{0pt}%
\pgfsys@defobject{currentmarker}{\pgfqpoint{0.000000in}{0.000000in}}{\pgfqpoint{0.000000in}{0.020833in}}{%
\pgfpathmoveto{\pgfqpoint{0.000000in}{0.000000in}}%
\pgfpathlineto{\pgfqpoint{0.000000in}{0.020833in}}%
\pgfusepath{stroke,fill}%
}%
\begin{pgfscope}%
\pgfsys@transformshift{0.739961in}{1.326898in}%
\pgfsys@useobject{currentmarker}{}%
\end{pgfscope}%
\end{pgfscope}%
\begin{pgfscope}%
\pgfsetbuttcap%
\pgfsetroundjoin%
\definecolor{currentfill}{rgb}{0.000000,0.000000,0.000000}%
\pgfsetfillcolor{currentfill}%
\pgfsetlinewidth{0.501875pt}%
\definecolor{currentstroke}{rgb}{0.000000,0.000000,0.000000}%
\pgfsetstrokecolor{currentstroke}%
\pgfsetdash{}{0pt}%
\pgfsys@defobject{currentmarker}{\pgfqpoint{0.000000in}{-0.020833in}}{\pgfqpoint{0.000000in}{0.000000in}}{%
\pgfpathmoveto{\pgfqpoint{0.000000in}{0.000000in}}%
\pgfpathlineto{\pgfqpoint{0.000000in}{-0.020833in}}%
\pgfusepath{stroke,fill}%
}%
\begin{pgfscope}%
\pgfsys@transformshift{0.739961in}{1.794149in}%
\pgfsys@useobject{currentmarker}{}%
\end{pgfscope}%
\end{pgfscope}%
\begin{pgfscope}%
\pgfsetbuttcap%
\pgfsetroundjoin%
\definecolor{currentfill}{rgb}{0.000000,0.000000,0.000000}%
\pgfsetfillcolor{currentfill}%
\pgfsetlinewidth{0.501875pt}%
\definecolor{currentstroke}{rgb}{0.000000,0.000000,0.000000}%
\pgfsetstrokecolor{currentstroke}%
\pgfsetdash{}{0pt}%
\pgfsys@defobject{currentmarker}{\pgfqpoint{0.000000in}{0.000000in}}{\pgfqpoint{0.000000in}{0.020833in}}{%
\pgfpathmoveto{\pgfqpoint{0.000000in}{0.000000in}}%
\pgfpathlineto{\pgfqpoint{0.000000in}{0.020833in}}%
\pgfusepath{stroke,fill}%
}%
\begin{pgfscope}%
\pgfsys@transformshift{0.836036in}{1.326898in}%
\pgfsys@useobject{currentmarker}{}%
\end{pgfscope}%
\end{pgfscope}%
\begin{pgfscope}%
\pgfsetbuttcap%
\pgfsetroundjoin%
\definecolor{currentfill}{rgb}{0.000000,0.000000,0.000000}%
\pgfsetfillcolor{currentfill}%
\pgfsetlinewidth{0.501875pt}%
\definecolor{currentstroke}{rgb}{0.000000,0.000000,0.000000}%
\pgfsetstrokecolor{currentstroke}%
\pgfsetdash{}{0pt}%
\pgfsys@defobject{currentmarker}{\pgfqpoint{0.000000in}{-0.020833in}}{\pgfqpoint{0.000000in}{0.000000in}}{%
\pgfpathmoveto{\pgfqpoint{0.000000in}{0.000000in}}%
\pgfpathlineto{\pgfqpoint{0.000000in}{-0.020833in}}%
\pgfusepath{stroke,fill}%
}%
\begin{pgfscope}%
\pgfsys@transformshift{0.836036in}{1.794149in}%
\pgfsys@useobject{currentmarker}{}%
\end{pgfscope}%
\end{pgfscope}%
\begin{pgfscope}%
\pgfsetbuttcap%
\pgfsetroundjoin%
\definecolor{currentfill}{rgb}{0.000000,0.000000,0.000000}%
\pgfsetfillcolor{currentfill}%
\pgfsetlinewidth{0.501875pt}%
\definecolor{currentstroke}{rgb}{0.000000,0.000000,0.000000}%
\pgfsetstrokecolor{currentstroke}%
\pgfsetdash{}{0pt}%
\pgfsys@defobject{currentmarker}{\pgfqpoint{0.000000in}{0.000000in}}{\pgfqpoint{0.000000in}{0.020833in}}{%
\pgfpathmoveto{\pgfqpoint{0.000000in}{0.000000in}}%
\pgfpathlineto{\pgfqpoint{0.000000in}{0.020833in}}%
\pgfusepath{stroke,fill}%
}%
\begin{pgfscope}%
\pgfsys@transformshift{0.932111in}{1.326898in}%
\pgfsys@useobject{currentmarker}{}%
\end{pgfscope}%
\end{pgfscope}%
\begin{pgfscope}%
\pgfsetbuttcap%
\pgfsetroundjoin%
\definecolor{currentfill}{rgb}{0.000000,0.000000,0.000000}%
\pgfsetfillcolor{currentfill}%
\pgfsetlinewidth{0.501875pt}%
\definecolor{currentstroke}{rgb}{0.000000,0.000000,0.000000}%
\pgfsetstrokecolor{currentstroke}%
\pgfsetdash{}{0pt}%
\pgfsys@defobject{currentmarker}{\pgfqpoint{0.000000in}{-0.020833in}}{\pgfqpoint{0.000000in}{0.000000in}}{%
\pgfpathmoveto{\pgfqpoint{0.000000in}{0.000000in}}%
\pgfpathlineto{\pgfqpoint{0.000000in}{-0.020833in}}%
\pgfusepath{stroke,fill}%
}%
\begin{pgfscope}%
\pgfsys@transformshift{0.932111in}{1.794149in}%
\pgfsys@useobject{currentmarker}{}%
\end{pgfscope}%
\end{pgfscope}%
\begin{pgfscope}%
\pgfsetbuttcap%
\pgfsetroundjoin%
\definecolor{currentfill}{rgb}{0.000000,0.000000,0.000000}%
\pgfsetfillcolor{currentfill}%
\pgfsetlinewidth{0.501875pt}%
\definecolor{currentstroke}{rgb}{0.000000,0.000000,0.000000}%
\pgfsetstrokecolor{currentstroke}%
\pgfsetdash{}{0pt}%
\pgfsys@defobject{currentmarker}{\pgfqpoint{0.000000in}{0.000000in}}{\pgfqpoint{0.000000in}{0.020833in}}{%
\pgfpathmoveto{\pgfqpoint{0.000000in}{0.000000in}}%
\pgfpathlineto{\pgfqpoint{0.000000in}{0.020833in}}%
\pgfusepath{stroke,fill}%
}%
\begin{pgfscope}%
\pgfsys@transformshift{1.028186in}{1.326898in}%
\pgfsys@useobject{currentmarker}{}%
\end{pgfscope}%
\end{pgfscope}%
\begin{pgfscope}%
\pgfsetbuttcap%
\pgfsetroundjoin%
\definecolor{currentfill}{rgb}{0.000000,0.000000,0.000000}%
\pgfsetfillcolor{currentfill}%
\pgfsetlinewidth{0.501875pt}%
\definecolor{currentstroke}{rgb}{0.000000,0.000000,0.000000}%
\pgfsetstrokecolor{currentstroke}%
\pgfsetdash{}{0pt}%
\pgfsys@defobject{currentmarker}{\pgfqpoint{0.000000in}{-0.020833in}}{\pgfqpoint{0.000000in}{0.000000in}}{%
\pgfpathmoveto{\pgfqpoint{0.000000in}{0.000000in}}%
\pgfpathlineto{\pgfqpoint{0.000000in}{-0.020833in}}%
\pgfusepath{stroke,fill}%
}%
\begin{pgfscope}%
\pgfsys@transformshift{1.028186in}{1.794149in}%
\pgfsys@useobject{currentmarker}{}%
\end{pgfscope}%
\end{pgfscope}%
\begin{pgfscope}%
\pgfsetbuttcap%
\pgfsetroundjoin%
\definecolor{currentfill}{rgb}{0.000000,0.000000,0.000000}%
\pgfsetfillcolor{currentfill}%
\pgfsetlinewidth{0.501875pt}%
\definecolor{currentstroke}{rgb}{0.000000,0.000000,0.000000}%
\pgfsetstrokecolor{currentstroke}%
\pgfsetdash{}{0pt}%
\pgfsys@defobject{currentmarker}{\pgfqpoint{0.000000in}{0.000000in}}{\pgfqpoint{0.000000in}{0.020833in}}{%
\pgfpathmoveto{\pgfqpoint{0.000000in}{0.000000in}}%
\pgfpathlineto{\pgfqpoint{0.000000in}{0.020833in}}%
\pgfusepath{stroke,fill}%
}%
\begin{pgfscope}%
\pgfsys@transformshift{1.220336in}{1.326898in}%
\pgfsys@useobject{currentmarker}{}%
\end{pgfscope}%
\end{pgfscope}%
\begin{pgfscope}%
\pgfsetbuttcap%
\pgfsetroundjoin%
\definecolor{currentfill}{rgb}{0.000000,0.000000,0.000000}%
\pgfsetfillcolor{currentfill}%
\pgfsetlinewidth{0.501875pt}%
\definecolor{currentstroke}{rgb}{0.000000,0.000000,0.000000}%
\pgfsetstrokecolor{currentstroke}%
\pgfsetdash{}{0pt}%
\pgfsys@defobject{currentmarker}{\pgfqpoint{0.000000in}{-0.020833in}}{\pgfqpoint{0.000000in}{0.000000in}}{%
\pgfpathmoveto{\pgfqpoint{0.000000in}{0.000000in}}%
\pgfpathlineto{\pgfqpoint{0.000000in}{-0.020833in}}%
\pgfusepath{stroke,fill}%
}%
\begin{pgfscope}%
\pgfsys@transformshift{1.220336in}{1.794149in}%
\pgfsys@useobject{currentmarker}{}%
\end{pgfscope}%
\end{pgfscope}%
\begin{pgfscope}%
\pgfsetbuttcap%
\pgfsetroundjoin%
\definecolor{currentfill}{rgb}{0.000000,0.000000,0.000000}%
\pgfsetfillcolor{currentfill}%
\pgfsetlinewidth{0.501875pt}%
\definecolor{currentstroke}{rgb}{0.000000,0.000000,0.000000}%
\pgfsetstrokecolor{currentstroke}%
\pgfsetdash{}{0pt}%
\pgfsys@defobject{currentmarker}{\pgfqpoint{0.000000in}{0.000000in}}{\pgfqpoint{0.000000in}{0.020833in}}{%
\pgfpathmoveto{\pgfqpoint{0.000000in}{0.000000in}}%
\pgfpathlineto{\pgfqpoint{0.000000in}{0.020833in}}%
\pgfusepath{stroke,fill}%
}%
\begin{pgfscope}%
\pgfsys@transformshift{1.316411in}{1.326898in}%
\pgfsys@useobject{currentmarker}{}%
\end{pgfscope}%
\end{pgfscope}%
\begin{pgfscope}%
\pgfsetbuttcap%
\pgfsetroundjoin%
\definecolor{currentfill}{rgb}{0.000000,0.000000,0.000000}%
\pgfsetfillcolor{currentfill}%
\pgfsetlinewidth{0.501875pt}%
\definecolor{currentstroke}{rgb}{0.000000,0.000000,0.000000}%
\pgfsetstrokecolor{currentstroke}%
\pgfsetdash{}{0pt}%
\pgfsys@defobject{currentmarker}{\pgfqpoint{0.000000in}{-0.020833in}}{\pgfqpoint{0.000000in}{0.000000in}}{%
\pgfpathmoveto{\pgfqpoint{0.000000in}{0.000000in}}%
\pgfpathlineto{\pgfqpoint{0.000000in}{-0.020833in}}%
\pgfusepath{stroke,fill}%
}%
\begin{pgfscope}%
\pgfsys@transformshift{1.316411in}{1.794149in}%
\pgfsys@useobject{currentmarker}{}%
\end{pgfscope}%
\end{pgfscope}%
\begin{pgfscope}%
\pgfsetbuttcap%
\pgfsetroundjoin%
\definecolor{currentfill}{rgb}{0.000000,0.000000,0.000000}%
\pgfsetfillcolor{currentfill}%
\pgfsetlinewidth{0.501875pt}%
\definecolor{currentstroke}{rgb}{0.000000,0.000000,0.000000}%
\pgfsetstrokecolor{currentstroke}%
\pgfsetdash{}{0pt}%
\pgfsys@defobject{currentmarker}{\pgfqpoint{0.000000in}{0.000000in}}{\pgfqpoint{0.000000in}{0.020833in}}{%
\pgfpathmoveto{\pgfqpoint{0.000000in}{0.000000in}}%
\pgfpathlineto{\pgfqpoint{0.000000in}{0.020833in}}%
\pgfusepath{stroke,fill}%
}%
\begin{pgfscope}%
\pgfsys@transformshift{1.412487in}{1.326898in}%
\pgfsys@useobject{currentmarker}{}%
\end{pgfscope}%
\end{pgfscope}%
\begin{pgfscope}%
\pgfsetbuttcap%
\pgfsetroundjoin%
\definecolor{currentfill}{rgb}{0.000000,0.000000,0.000000}%
\pgfsetfillcolor{currentfill}%
\pgfsetlinewidth{0.501875pt}%
\definecolor{currentstroke}{rgb}{0.000000,0.000000,0.000000}%
\pgfsetstrokecolor{currentstroke}%
\pgfsetdash{}{0pt}%
\pgfsys@defobject{currentmarker}{\pgfqpoint{0.000000in}{-0.020833in}}{\pgfqpoint{0.000000in}{0.000000in}}{%
\pgfpathmoveto{\pgfqpoint{0.000000in}{0.000000in}}%
\pgfpathlineto{\pgfqpoint{0.000000in}{-0.020833in}}%
\pgfusepath{stroke,fill}%
}%
\begin{pgfscope}%
\pgfsys@transformshift{1.412487in}{1.794149in}%
\pgfsys@useobject{currentmarker}{}%
\end{pgfscope}%
\end{pgfscope}%
\begin{pgfscope}%
\pgfsetbuttcap%
\pgfsetroundjoin%
\definecolor{currentfill}{rgb}{0.000000,0.000000,0.000000}%
\pgfsetfillcolor{currentfill}%
\pgfsetlinewidth{0.501875pt}%
\definecolor{currentstroke}{rgb}{0.000000,0.000000,0.000000}%
\pgfsetstrokecolor{currentstroke}%
\pgfsetdash{}{0pt}%
\pgfsys@defobject{currentmarker}{\pgfqpoint{0.000000in}{0.000000in}}{\pgfqpoint{0.000000in}{0.020833in}}{%
\pgfpathmoveto{\pgfqpoint{0.000000in}{0.000000in}}%
\pgfpathlineto{\pgfqpoint{0.000000in}{0.020833in}}%
\pgfusepath{stroke,fill}%
}%
\begin{pgfscope}%
\pgfsys@transformshift{1.508562in}{1.326898in}%
\pgfsys@useobject{currentmarker}{}%
\end{pgfscope}%
\end{pgfscope}%
\begin{pgfscope}%
\pgfsetbuttcap%
\pgfsetroundjoin%
\definecolor{currentfill}{rgb}{0.000000,0.000000,0.000000}%
\pgfsetfillcolor{currentfill}%
\pgfsetlinewidth{0.501875pt}%
\definecolor{currentstroke}{rgb}{0.000000,0.000000,0.000000}%
\pgfsetstrokecolor{currentstroke}%
\pgfsetdash{}{0pt}%
\pgfsys@defobject{currentmarker}{\pgfqpoint{0.000000in}{-0.020833in}}{\pgfqpoint{0.000000in}{0.000000in}}{%
\pgfpathmoveto{\pgfqpoint{0.000000in}{0.000000in}}%
\pgfpathlineto{\pgfqpoint{0.000000in}{-0.020833in}}%
\pgfusepath{stroke,fill}%
}%
\begin{pgfscope}%
\pgfsys@transformshift{1.508562in}{1.794149in}%
\pgfsys@useobject{currentmarker}{}%
\end{pgfscope}%
\end{pgfscope}%
\begin{pgfscope}%
\pgfsetbuttcap%
\pgfsetroundjoin%
\definecolor{currentfill}{rgb}{0.000000,0.000000,0.000000}%
\pgfsetfillcolor{currentfill}%
\pgfsetlinewidth{0.501875pt}%
\definecolor{currentstroke}{rgb}{0.000000,0.000000,0.000000}%
\pgfsetstrokecolor{currentstroke}%
\pgfsetdash{}{0pt}%
\pgfsys@defobject{currentmarker}{\pgfqpoint{0.000000in}{0.000000in}}{\pgfqpoint{0.000000in}{0.020833in}}{%
\pgfpathmoveto{\pgfqpoint{0.000000in}{0.000000in}}%
\pgfpathlineto{\pgfqpoint{0.000000in}{0.020833in}}%
\pgfusepath{stroke,fill}%
}%
\begin{pgfscope}%
\pgfsys@transformshift{1.700712in}{1.326898in}%
\pgfsys@useobject{currentmarker}{}%
\end{pgfscope}%
\end{pgfscope}%
\begin{pgfscope}%
\pgfsetbuttcap%
\pgfsetroundjoin%
\definecolor{currentfill}{rgb}{0.000000,0.000000,0.000000}%
\pgfsetfillcolor{currentfill}%
\pgfsetlinewidth{0.501875pt}%
\definecolor{currentstroke}{rgb}{0.000000,0.000000,0.000000}%
\pgfsetstrokecolor{currentstroke}%
\pgfsetdash{}{0pt}%
\pgfsys@defobject{currentmarker}{\pgfqpoint{0.000000in}{-0.020833in}}{\pgfqpoint{0.000000in}{0.000000in}}{%
\pgfpathmoveto{\pgfqpoint{0.000000in}{0.000000in}}%
\pgfpathlineto{\pgfqpoint{0.000000in}{-0.020833in}}%
\pgfusepath{stroke,fill}%
}%
\begin{pgfscope}%
\pgfsys@transformshift{1.700712in}{1.794149in}%
\pgfsys@useobject{currentmarker}{}%
\end{pgfscope}%
\end{pgfscope}%
\begin{pgfscope}%
\pgfsetbuttcap%
\pgfsetroundjoin%
\definecolor{currentfill}{rgb}{0.000000,0.000000,0.000000}%
\pgfsetfillcolor{currentfill}%
\pgfsetlinewidth{0.501875pt}%
\definecolor{currentstroke}{rgb}{0.000000,0.000000,0.000000}%
\pgfsetstrokecolor{currentstroke}%
\pgfsetdash{}{0pt}%
\pgfsys@defobject{currentmarker}{\pgfqpoint{0.000000in}{0.000000in}}{\pgfqpoint{0.000000in}{0.020833in}}{%
\pgfpathmoveto{\pgfqpoint{0.000000in}{0.000000in}}%
\pgfpathlineto{\pgfqpoint{0.000000in}{0.020833in}}%
\pgfusepath{stroke,fill}%
}%
\begin{pgfscope}%
\pgfsys@transformshift{1.796787in}{1.326898in}%
\pgfsys@useobject{currentmarker}{}%
\end{pgfscope}%
\end{pgfscope}%
\begin{pgfscope}%
\pgfsetbuttcap%
\pgfsetroundjoin%
\definecolor{currentfill}{rgb}{0.000000,0.000000,0.000000}%
\pgfsetfillcolor{currentfill}%
\pgfsetlinewidth{0.501875pt}%
\definecolor{currentstroke}{rgb}{0.000000,0.000000,0.000000}%
\pgfsetstrokecolor{currentstroke}%
\pgfsetdash{}{0pt}%
\pgfsys@defobject{currentmarker}{\pgfqpoint{0.000000in}{-0.020833in}}{\pgfqpoint{0.000000in}{0.000000in}}{%
\pgfpathmoveto{\pgfqpoint{0.000000in}{0.000000in}}%
\pgfpathlineto{\pgfqpoint{0.000000in}{-0.020833in}}%
\pgfusepath{stroke,fill}%
}%
\begin{pgfscope}%
\pgfsys@transformshift{1.796787in}{1.794149in}%
\pgfsys@useobject{currentmarker}{}%
\end{pgfscope}%
\end{pgfscope}%
\begin{pgfscope}%
\pgfsetbuttcap%
\pgfsetroundjoin%
\definecolor{currentfill}{rgb}{0.000000,0.000000,0.000000}%
\pgfsetfillcolor{currentfill}%
\pgfsetlinewidth{0.501875pt}%
\definecolor{currentstroke}{rgb}{0.000000,0.000000,0.000000}%
\pgfsetstrokecolor{currentstroke}%
\pgfsetdash{}{0pt}%
\pgfsys@defobject{currentmarker}{\pgfqpoint{0.000000in}{0.000000in}}{\pgfqpoint{0.000000in}{0.020833in}}{%
\pgfpathmoveto{\pgfqpoint{0.000000in}{0.000000in}}%
\pgfpathlineto{\pgfqpoint{0.000000in}{0.020833in}}%
\pgfusepath{stroke,fill}%
}%
\begin{pgfscope}%
\pgfsys@transformshift{1.892862in}{1.326898in}%
\pgfsys@useobject{currentmarker}{}%
\end{pgfscope}%
\end{pgfscope}%
\begin{pgfscope}%
\pgfsetbuttcap%
\pgfsetroundjoin%
\definecolor{currentfill}{rgb}{0.000000,0.000000,0.000000}%
\pgfsetfillcolor{currentfill}%
\pgfsetlinewidth{0.501875pt}%
\definecolor{currentstroke}{rgb}{0.000000,0.000000,0.000000}%
\pgfsetstrokecolor{currentstroke}%
\pgfsetdash{}{0pt}%
\pgfsys@defobject{currentmarker}{\pgfqpoint{0.000000in}{-0.020833in}}{\pgfqpoint{0.000000in}{0.000000in}}{%
\pgfpathmoveto{\pgfqpoint{0.000000in}{0.000000in}}%
\pgfpathlineto{\pgfqpoint{0.000000in}{-0.020833in}}%
\pgfusepath{stroke,fill}%
}%
\begin{pgfscope}%
\pgfsys@transformshift{1.892862in}{1.794149in}%
\pgfsys@useobject{currentmarker}{}%
\end{pgfscope}%
\end{pgfscope}%
\begin{pgfscope}%
\pgfsetbuttcap%
\pgfsetroundjoin%
\definecolor{currentfill}{rgb}{0.000000,0.000000,0.000000}%
\pgfsetfillcolor{currentfill}%
\pgfsetlinewidth{0.501875pt}%
\definecolor{currentstroke}{rgb}{0.000000,0.000000,0.000000}%
\pgfsetstrokecolor{currentstroke}%
\pgfsetdash{}{0pt}%
\pgfsys@defobject{currentmarker}{\pgfqpoint{0.000000in}{0.000000in}}{\pgfqpoint{0.000000in}{0.020833in}}{%
\pgfpathmoveto{\pgfqpoint{0.000000in}{0.000000in}}%
\pgfpathlineto{\pgfqpoint{0.000000in}{0.020833in}}%
\pgfusepath{stroke,fill}%
}%
\begin{pgfscope}%
\pgfsys@transformshift{1.988937in}{1.326898in}%
\pgfsys@useobject{currentmarker}{}%
\end{pgfscope}%
\end{pgfscope}%
\begin{pgfscope}%
\pgfsetbuttcap%
\pgfsetroundjoin%
\definecolor{currentfill}{rgb}{0.000000,0.000000,0.000000}%
\pgfsetfillcolor{currentfill}%
\pgfsetlinewidth{0.501875pt}%
\definecolor{currentstroke}{rgb}{0.000000,0.000000,0.000000}%
\pgfsetstrokecolor{currentstroke}%
\pgfsetdash{}{0pt}%
\pgfsys@defobject{currentmarker}{\pgfqpoint{0.000000in}{-0.020833in}}{\pgfqpoint{0.000000in}{0.000000in}}{%
\pgfpathmoveto{\pgfqpoint{0.000000in}{0.000000in}}%
\pgfpathlineto{\pgfqpoint{0.000000in}{-0.020833in}}%
\pgfusepath{stroke,fill}%
}%
\begin{pgfscope}%
\pgfsys@transformshift{1.988937in}{1.794149in}%
\pgfsys@useobject{currentmarker}{}%
\end{pgfscope}%
\end{pgfscope}%
\begin{pgfscope}%
\pgfsetbuttcap%
\pgfsetroundjoin%
\definecolor{currentfill}{rgb}{0.000000,0.000000,0.000000}%
\pgfsetfillcolor{currentfill}%
\pgfsetlinewidth{0.501875pt}%
\definecolor{currentstroke}{rgb}{0.000000,0.000000,0.000000}%
\pgfsetstrokecolor{currentstroke}%
\pgfsetdash{}{0pt}%
\pgfsys@defobject{currentmarker}{\pgfqpoint{0.000000in}{0.000000in}}{\pgfqpoint{0.000000in}{0.020833in}}{%
\pgfpathmoveto{\pgfqpoint{0.000000in}{0.000000in}}%
\pgfpathlineto{\pgfqpoint{0.000000in}{0.020833in}}%
\pgfusepath{stroke,fill}%
}%
\begin{pgfscope}%
\pgfsys@transformshift{2.181087in}{1.326898in}%
\pgfsys@useobject{currentmarker}{}%
\end{pgfscope}%
\end{pgfscope}%
\begin{pgfscope}%
\pgfsetbuttcap%
\pgfsetroundjoin%
\definecolor{currentfill}{rgb}{0.000000,0.000000,0.000000}%
\pgfsetfillcolor{currentfill}%
\pgfsetlinewidth{0.501875pt}%
\definecolor{currentstroke}{rgb}{0.000000,0.000000,0.000000}%
\pgfsetstrokecolor{currentstroke}%
\pgfsetdash{}{0pt}%
\pgfsys@defobject{currentmarker}{\pgfqpoint{0.000000in}{-0.020833in}}{\pgfqpoint{0.000000in}{0.000000in}}{%
\pgfpathmoveto{\pgfqpoint{0.000000in}{0.000000in}}%
\pgfpathlineto{\pgfqpoint{0.000000in}{-0.020833in}}%
\pgfusepath{stroke,fill}%
}%
\begin{pgfscope}%
\pgfsys@transformshift{2.181087in}{1.794149in}%
\pgfsys@useobject{currentmarker}{}%
\end{pgfscope}%
\end{pgfscope}%
\begin{pgfscope}%
\pgfsetbuttcap%
\pgfsetroundjoin%
\definecolor{currentfill}{rgb}{0.000000,0.000000,0.000000}%
\pgfsetfillcolor{currentfill}%
\pgfsetlinewidth{0.501875pt}%
\definecolor{currentstroke}{rgb}{0.000000,0.000000,0.000000}%
\pgfsetstrokecolor{currentstroke}%
\pgfsetdash{}{0pt}%
\pgfsys@defobject{currentmarker}{\pgfqpoint{0.000000in}{0.000000in}}{\pgfqpoint{0.000000in}{0.020833in}}{%
\pgfpathmoveto{\pgfqpoint{0.000000in}{0.000000in}}%
\pgfpathlineto{\pgfqpoint{0.000000in}{0.020833in}}%
\pgfusepath{stroke,fill}%
}%
\begin{pgfscope}%
\pgfsys@transformshift{2.277162in}{1.326898in}%
\pgfsys@useobject{currentmarker}{}%
\end{pgfscope}%
\end{pgfscope}%
\begin{pgfscope}%
\pgfsetbuttcap%
\pgfsetroundjoin%
\definecolor{currentfill}{rgb}{0.000000,0.000000,0.000000}%
\pgfsetfillcolor{currentfill}%
\pgfsetlinewidth{0.501875pt}%
\definecolor{currentstroke}{rgb}{0.000000,0.000000,0.000000}%
\pgfsetstrokecolor{currentstroke}%
\pgfsetdash{}{0pt}%
\pgfsys@defobject{currentmarker}{\pgfqpoint{0.000000in}{-0.020833in}}{\pgfqpoint{0.000000in}{0.000000in}}{%
\pgfpathmoveto{\pgfqpoint{0.000000in}{0.000000in}}%
\pgfpathlineto{\pgfqpoint{0.000000in}{-0.020833in}}%
\pgfusepath{stroke,fill}%
}%
\begin{pgfscope}%
\pgfsys@transformshift{2.277162in}{1.794149in}%
\pgfsys@useobject{currentmarker}{}%
\end{pgfscope}%
\end{pgfscope}%
\begin{pgfscope}%
\pgfsetbuttcap%
\pgfsetroundjoin%
\definecolor{currentfill}{rgb}{0.000000,0.000000,0.000000}%
\pgfsetfillcolor{currentfill}%
\pgfsetlinewidth{0.501875pt}%
\definecolor{currentstroke}{rgb}{0.000000,0.000000,0.000000}%
\pgfsetstrokecolor{currentstroke}%
\pgfsetdash{}{0pt}%
\pgfsys@defobject{currentmarker}{\pgfqpoint{0.000000in}{0.000000in}}{\pgfqpoint{0.000000in}{0.020833in}}{%
\pgfpathmoveto{\pgfqpoint{0.000000in}{0.000000in}}%
\pgfpathlineto{\pgfqpoint{0.000000in}{0.020833in}}%
\pgfusepath{stroke,fill}%
}%
\begin{pgfscope}%
\pgfsys@transformshift{2.373237in}{1.326898in}%
\pgfsys@useobject{currentmarker}{}%
\end{pgfscope}%
\end{pgfscope}%
\begin{pgfscope}%
\pgfsetbuttcap%
\pgfsetroundjoin%
\definecolor{currentfill}{rgb}{0.000000,0.000000,0.000000}%
\pgfsetfillcolor{currentfill}%
\pgfsetlinewidth{0.501875pt}%
\definecolor{currentstroke}{rgb}{0.000000,0.000000,0.000000}%
\pgfsetstrokecolor{currentstroke}%
\pgfsetdash{}{0pt}%
\pgfsys@defobject{currentmarker}{\pgfqpoint{0.000000in}{-0.020833in}}{\pgfqpoint{0.000000in}{0.000000in}}{%
\pgfpathmoveto{\pgfqpoint{0.000000in}{0.000000in}}%
\pgfpathlineto{\pgfqpoint{0.000000in}{-0.020833in}}%
\pgfusepath{stroke,fill}%
}%
\begin{pgfscope}%
\pgfsys@transformshift{2.373237in}{1.794149in}%
\pgfsys@useobject{currentmarker}{}%
\end{pgfscope}%
\end{pgfscope}%
\begin{pgfscope}%
\pgfsetbuttcap%
\pgfsetroundjoin%
\definecolor{currentfill}{rgb}{0.000000,0.000000,0.000000}%
\pgfsetfillcolor{currentfill}%
\pgfsetlinewidth{0.501875pt}%
\definecolor{currentstroke}{rgb}{0.000000,0.000000,0.000000}%
\pgfsetstrokecolor{currentstroke}%
\pgfsetdash{}{0pt}%
\pgfsys@defobject{currentmarker}{\pgfqpoint{0.000000in}{0.000000in}}{\pgfqpoint{0.000000in}{0.020833in}}{%
\pgfpathmoveto{\pgfqpoint{0.000000in}{0.000000in}}%
\pgfpathlineto{\pgfqpoint{0.000000in}{0.020833in}}%
\pgfusepath{stroke,fill}%
}%
\begin{pgfscope}%
\pgfsys@transformshift{2.469312in}{1.326898in}%
\pgfsys@useobject{currentmarker}{}%
\end{pgfscope}%
\end{pgfscope}%
\begin{pgfscope}%
\pgfsetbuttcap%
\pgfsetroundjoin%
\definecolor{currentfill}{rgb}{0.000000,0.000000,0.000000}%
\pgfsetfillcolor{currentfill}%
\pgfsetlinewidth{0.501875pt}%
\definecolor{currentstroke}{rgb}{0.000000,0.000000,0.000000}%
\pgfsetstrokecolor{currentstroke}%
\pgfsetdash{}{0pt}%
\pgfsys@defobject{currentmarker}{\pgfqpoint{0.000000in}{-0.020833in}}{\pgfqpoint{0.000000in}{0.000000in}}{%
\pgfpathmoveto{\pgfqpoint{0.000000in}{0.000000in}}%
\pgfpathlineto{\pgfqpoint{0.000000in}{-0.020833in}}%
\pgfusepath{stroke,fill}%
}%
\begin{pgfscope}%
\pgfsys@transformshift{2.469312in}{1.794149in}%
\pgfsys@useobject{currentmarker}{}%
\end{pgfscope}%
\end{pgfscope}%
\begin{pgfscope}%
\pgfsetbuttcap%
\pgfsetroundjoin%
\definecolor{currentfill}{rgb}{0.000000,0.000000,0.000000}%
\pgfsetfillcolor{currentfill}%
\pgfsetlinewidth{0.501875pt}%
\definecolor{currentstroke}{rgb}{0.000000,0.000000,0.000000}%
\pgfsetstrokecolor{currentstroke}%
\pgfsetdash{}{0pt}%
\pgfsys@defobject{currentmarker}{\pgfqpoint{0.000000in}{0.000000in}}{\pgfqpoint{0.000000in}{0.020833in}}{%
\pgfpathmoveto{\pgfqpoint{0.000000in}{0.000000in}}%
\pgfpathlineto{\pgfqpoint{0.000000in}{0.020833in}}%
\pgfusepath{stroke,fill}%
}%
\begin{pgfscope}%
\pgfsys@transformshift{2.661463in}{1.326898in}%
\pgfsys@useobject{currentmarker}{}%
\end{pgfscope}%
\end{pgfscope}%
\begin{pgfscope}%
\pgfsetbuttcap%
\pgfsetroundjoin%
\definecolor{currentfill}{rgb}{0.000000,0.000000,0.000000}%
\pgfsetfillcolor{currentfill}%
\pgfsetlinewidth{0.501875pt}%
\definecolor{currentstroke}{rgb}{0.000000,0.000000,0.000000}%
\pgfsetstrokecolor{currentstroke}%
\pgfsetdash{}{0pt}%
\pgfsys@defobject{currentmarker}{\pgfqpoint{0.000000in}{-0.020833in}}{\pgfqpoint{0.000000in}{0.000000in}}{%
\pgfpathmoveto{\pgfqpoint{0.000000in}{0.000000in}}%
\pgfpathlineto{\pgfqpoint{0.000000in}{-0.020833in}}%
\pgfusepath{stroke,fill}%
}%
\begin{pgfscope}%
\pgfsys@transformshift{2.661463in}{1.794149in}%
\pgfsys@useobject{currentmarker}{}%
\end{pgfscope}%
\end{pgfscope}%
\begin{pgfscope}%
\pgfsetbuttcap%
\pgfsetroundjoin%
\definecolor{currentfill}{rgb}{0.000000,0.000000,0.000000}%
\pgfsetfillcolor{currentfill}%
\pgfsetlinewidth{0.501875pt}%
\definecolor{currentstroke}{rgb}{0.000000,0.000000,0.000000}%
\pgfsetstrokecolor{currentstroke}%
\pgfsetdash{}{0pt}%
\pgfsys@defobject{currentmarker}{\pgfqpoint{0.000000in}{0.000000in}}{\pgfqpoint{0.000000in}{0.020833in}}{%
\pgfpathmoveto{\pgfqpoint{0.000000in}{0.000000in}}%
\pgfpathlineto{\pgfqpoint{0.000000in}{0.020833in}}%
\pgfusepath{stroke,fill}%
}%
\begin{pgfscope}%
\pgfsys@transformshift{2.757538in}{1.326898in}%
\pgfsys@useobject{currentmarker}{}%
\end{pgfscope}%
\end{pgfscope}%
\begin{pgfscope}%
\pgfsetbuttcap%
\pgfsetroundjoin%
\definecolor{currentfill}{rgb}{0.000000,0.000000,0.000000}%
\pgfsetfillcolor{currentfill}%
\pgfsetlinewidth{0.501875pt}%
\definecolor{currentstroke}{rgb}{0.000000,0.000000,0.000000}%
\pgfsetstrokecolor{currentstroke}%
\pgfsetdash{}{0pt}%
\pgfsys@defobject{currentmarker}{\pgfqpoint{0.000000in}{-0.020833in}}{\pgfqpoint{0.000000in}{0.000000in}}{%
\pgfpathmoveto{\pgfqpoint{0.000000in}{0.000000in}}%
\pgfpathlineto{\pgfqpoint{0.000000in}{-0.020833in}}%
\pgfusepath{stroke,fill}%
}%
\begin{pgfscope}%
\pgfsys@transformshift{2.757538in}{1.794149in}%
\pgfsys@useobject{currentmarker}{}%
\end{pgfscope}%
\end{pgfscope}%
\begin{pgfscope}%
\pgfsetbuttcap%
\pgfsetroundjoin%
\definecolor{currentfill}{rgb}{0.000000,0.000000,0.000000}%
\pgfsetfillcolor{currentfill}%
\pgfsetlinewidth{0.501875pt}%
\definecolor{currentstroke}{rgb}{0.000000,0.000000,0.000000}%
\pgfsetstrokecolor{currentstroke}%
\pgfsetdash{}{0pt}%
\pgfsys@defobject{currentmarker}{\pgfqpoint{0.000000in}{0.000000in}}{\pgfqpoint{0.000000in}{0.020833in}}{%
\pgfpathmoveto{\pgfqpoint{0.000000in}{0.000000in}}%
\pgfpathlineto{\pgfqpoint{0.000000in}{0.020833in}}%
\pgfusepath{stroke,fill}%
}%
\begin{pgfscope}%
\pgfsys@transformshift{2.853613in}{1.326898in}%
\pgfsys@useobject{currentmarker}{}%
\end{pgfscope}%
\end{pgfscope}%
\begin{pgfscope}%
\pgfsetbuttcap%
\pgfsetroundjoin%
\definecolor{currentfill}{rgb}{0.000000,0.000000,0.000000}%
\pgfsetfillcolor{currentfill}%
\pgfsetlinewidth{0.501875pt}%
\definecolor{currentstroke}{rgb}{0.000000,0.000000,0.000000}%
\pgfsetstrokecolor{currentstroke}%
\pgfsetdash{}{0pt}%
\pgfsys@defobject{currentmarker}{\pgfqpoint{0.000000in}{-0.020833in}}{\pgfqpoint{0.000000in}{0.000000in}}{%
\pgfpathmoveto{\pgfqpoint{0.000000in}{0.000000in}}%
\pgfpathlineto{\pgfqpoint{0.000000in}{-0.020833in}}%
\pgfusepath{stroke,fill}%
}%
\begin{pgfscope}%
\pgfsys@transformshift{2.853613in}{1.794149in}%
\pgfsys@useobject{currentmarker}{}%
\end{pgfscope}%
\end{pgfscope}%
\begin{pgfscope}%
\pgfsetbuttcap%
\pgfsetroundjoin%
\definecolor{currentfill}{rgb}{0.000000,0.000000,0.000000}%
\pgfsetfillcolor{currentfill}%
\pgfsetlinewidth{0.501875pt}%
\definecolor{currentstroke}{rgb}{0.000000,0.000000,0.000000}%
\pgfsetstrokecolor{currentstroke}%
\pgfsetdash{}{0pt}%
\pgfsys@defobject{currentmarker}{\pgfqpoint{0.000000in}{0.000000in}}{\pgfqpoint{0.000000in}{0.020833in}}{%
\pgfpathmoveto{\pgfqpoint{0.000000in}{0.000000in}}%
\pgfpathlineto{\pgfqpoint{0.000000in}{0.020833in}}%
\pgfusepath{stroke,fill}%
}%
\begin{pgfscope}%
\pgfsys@transformshift{2.949688in}{1.326898in}%
\pgfsys@useobject{currentmarker}{}%
\end{pgfscope}%
\end{pgfscope}%
\begin{pgfscope}%
\pgfsetbuttcap%
\pgfsetroundjoin%
\definecolor{currentfill}{rgb}{0.000000,0.000000,0.000000}%
\pgfsetfillcolor{currentfill}%
\pgfsetlinewidth{0.501875pt}%
\definecolor{currentstroke}{rgb}{0.000000,0.000000,0.000000}%
\pgfsetstrokecolor{currentstroke}%
\pgfsetdash{}{0pt}%
\pgfsys@defobject{currentmarker}{\pgfqpoint{0.000000in}{-0.020833in}}{\pgfqpoint{0.000000in}{0.000000in}}{%
\pgfpathmoveto{\pgfqpoint{0.000000in}{0.000000in}}%
\pgfpathlineto{\pgfqpoint{0.000000in}{-0.020833in}}%
\pgfusepath{stroke,fill}%
}%
\begin{pgfscope}%
\pgfsys@transformshift{2.949688in}{1.794149in}%
\pgfsys@useobject{currentmarker}{}%
\end{pgfscope}%
\end{pgfscope}%
\begin{pgfscope}%
\pgfsetbuttcap%
\pgfsetroundjoin%
\definecolor{currentfill}{rgb}{0.000000,0.000000,0.000000}%
\pgfsetfillcolor{currentfill}%
\pgfsetlinewidth{0.501875pt}%
\definecolor{currentstroke}{rgb}{0.000000,0.000000,0.000000}%
\pgfsetstrokecolor{currentstroke}%
\pgfsetdash{}{0pt}%
\pgfsys@defobject{currentmarker}{\pgfqpoint{0.000000in}{0.000000in}}{\pgfqpoint{0.000000in}{0.020833in}}{%
\pgfpathmoveto{\pgfqpoint{0.000000in}{0.000000in}}%
\pgfpathlineto{\pgfqpoint{0.000000in}{0.020833in}}%
\pgfusepath{stroke,fill}%
}%
\begin{pgfscope}%
\pgfsys@transformshift{3.141838in}{1.326898in}%
\pgfsys@useobject{currentmarker}{}%
\end{pgfscope}%
\end{pgfscope}%
\begin{pgfscope}%
\pgfsetbuttcap%
\pgfsetroundjoin%
\definecolor{currentfill}{rgb}{0.000000,0.000000,0.000000}%
\pgfsetfillcolor{currentfill}%
\pgfsetlinewidth{0.501875pt}%
\definecolor{currentstroke}{rgb}{0.000000,0.000000,0.000000}%
\pgfsetstrokecolor{currentstroke}%
\pgfsetdash{}{0pt}%
\pgfsys@defobject{currentmarker}{\pgfqpoint{0.000000in}{-0.020833in}}{\pgfqpoint{0.000000in}{0.000000in}}{%
\pgfpathmoveto{\pgfqpoint{0.000000in}{0.000000in}}%
\pgfpathlineto{\pgfqpoint{0.000000in}{-0.020833in}}%
\pgfusepath{stroke,fill}%
}%
\begin{pgfscope}%
\pgfsys@transformshift{3.141838in}{1.794149in}%
\pgfsys@useobject{currentmarker}{}%
\end{pgfscope}%
\end{pgfscope}%
\begin{pgfscope}%
\pgfsetbuttcap%
\pgfsetroundjoin%
\definecolor{currentfill}{rgb}{0.000000,0.000000,0.000000}%
\pgfsetfillcolor{currentfill}%
\pgfsetlinewidth{0.501875pt}%
\definecolor{currentstroke}{rgb}{0.000000,0.000000,0.000000}%
\pgfsetstrokecolor{currentstroke}%
\pgfsetdash{}{0pt}%
\pgfsys@defobject{currentmarker}{\pgfqpoint{0.000000in}{0.000000in}}{\pgfqpoint{0.000000in}{0.020833in}}{%
\pgfpathmoveto{\pgfqpoint{0.000000in}{0.000000in}}%
\pgfpathlineto{\pgfqpoint{0.000000in}{0.020833in}}%
\pgfusepath{stroke,fill}%
}%
\begin{pgfscope}%
\pgfsys@transformshift{3.237913in}{1.326898in}%
\pgfsys@useobject{currentmarker}{}%
\end{pgfscope}%
\end{pgfscope}%
\begin{pgfscope}%
\pgfsetbuttcap%
\pgfsetroundjoin%
\definecolor{currentfill}{rgb}{0.000000,0.000000,0.000000}%
\pgfsetfillcolor{currentfill}%
\pgfsetlinewidth{0.501875pt}%
\definecolor{currentstroke}{rgb}{0.000000,0.000000,0.000000}%
\pgfsetstrokecolor{currentstroke}%
\pgfsetdash{}{0pt}%
\pgfsys@defobject{currentmarker}{\pgfqpoint{0.000000in}{-0.020833in}}{\pgfqpoint{0.000000in}{0.000000in}}{%
\pgfpathmoveto{\pgfqpoint{0.000000in}{0.000000in}}%
\pgfpathlineto{\pgfqpoint{0.000000in}{-0.020833in}}%
\pgfusepath{stroke,fill}%
}%
\begin{pgfscope}%
\pgfsys@transformshift{3.237913in}{1.794149in}%
\pgfsys@useobject{currentmarker}{}%
\end{pgfscope}%
\end{pgfscope}%
\begin{pgfscope}%
\pgfsetbuttcap%
\pgfsetroundjoin%
\definecolor{currentfill}{rgb}{0.000000,0.000000,0.000000}%
\pgfsetfillcolor{currentfill}%
\pgfsetlinewidth{0.501875pt}%
\definecolor{currentstroke}{rgb}{0.000000,0.000000,0.000000}%
\pgfsetstrokecolor{currentstroke}%
\pgfsetdash{}{0pt}%
\pgfsys@defobject{currentmarker}{\pgfqpoint{0.000000in}{0.000000in}}{\pgfqpoint{0.000000in}{0.020833in}}{%
\pgfpathmoveto{\pgfqpoint{0.000000in}{0.000000in}}%
\pgfpathlineto{\pgfqpoint{0.000000in}{0.020833in}}%
\pgfusepath{stroke,fill}%
}%
\begin{pgfscope}%
\pgfsys@transformshift{3.333988in}{1.326898in}%
\pgfsys@useobject{currentmarker}{}%
\end{pgfscope}%
\end{pgfscope}%
\begin{pgfscope}%
\pgfsetbuttcap%
\pgfsetroundjoin%
\definecolor{currentfill}{rgb}{0.000000,0.000000,0.000000}%
\pgfsetfillcolor{currentfill}%
\pgfsetlinewidth{0.501875pt}%
\definecolor{currentstroke}{rgb}{0.000000,0.000000,0.000000}%
\pgfsetstrokecolor{currentstroke}%
\pgfsetdash{}{0pt}%
\pgfsys@defobject{currentmarker}{\pgfqpoint{0.000000in}{-0.020833in}}{\pgfqpoint{0.000000in}{0.000000in}}{%
\pgfpathmoveto{\pgfqpoint{0.000000in}{0.000000in}}%
\pgfpathlineto{\pgfqpoint{0.000000in}{-0.020833in}}%
\pgfusepath{stroke,fill}%
}%
\begin{pgfscope}%
\pgfsys@transformshift{3.333988in}{1.794149in}%
\pgfsys@useobject{currentmarker}{}%
\end{pgfscope}%
\end{pgfscope}%
\begin{pgfscope}%
\pgfsetbuttcap%
\pgfsetroundjoin%
\definecolor{currentfill}{rgb}{0.000000,0.000000,0.000000}%
\pgfsetfillcolor{currentfill}%
\pgfsetlinewidth{0.501875pt}%
\definecolor{currentstroke}{rgb}{0.000000,0.000000,0.000000}%
\pgfsetstrokecolor{currentstroke}%
\pgfsetdash{}{0pt}%
\pgfsys@defobject{currentmarker}{\pgfqpoint{0.000000in}{0.000000in}}{\pgfqpoint{0.000000in}{0.020833in}}{%
\pgfpathmoveto{\pgfqpoint{0.000000in}{0.000000in}}%
\pgfpathlineto{\pgfqpoint{0.000000in}{0.020833in}}%
\pgfusepath{stroke,fill}%
}%
\begin{pgfscope}%
\pgfsys@transformshift{3.430063in}{1.326898in}%
\pgfsys@useobject{currentmarker}{}%
\end{pgfscope}%
\end{pgfscope}%
\begin{pgfscope}%
\pgfsetbuttcap%
\pgfsetroundjoin%
\definecolor{currentfill}{rgb}{0.000000,0.000000,0.000000}%
\pgfsetfillcolor{currentfill}%
\pgfsetlinewidth{0.501875pt}%
\definecolor{currentstroke}{rgb}{0.000000,0.000000,0.000000}%
\pgfsetstrokecolor{currentstroke}%
\pgfsetdash{}{0pt}%
\pgfsys@defobject{currentmarker}{\pgfqpoint{0.000000in}{-0.020833in}}{\pgfqpoint{0.000000in}{0.000000in}}{%
\pgfpathmoveto{\pgfqpoint{0.000000in}{0.000000in}}%
\pgfpathlineto{\pgfqpoint{0.000000in}{-0.020833in}}%
\pgfusepath{stroke,fill}%
}%
\begin{pgfscope}%
\pgfsys@transformshift{3.430063in}{1.794149in}%
\pgfsys@useobject{currentmarker}{}%
\end{pgfscope}%
\end{pgfscope}%
\begin{pgfscope}%
\pgfsetbuttcap%
\pgfsetroundjoin%
\definecolor{currentfill}{rgb}{0.000000,0.000000,0.000000}%
\pgfsetfillcolor{currentfill}%
\pgfsetlinewidth{0.501875pt}%
\definecolor{currentstroke}{rgb}{0.000000,0.000000,0.000000}%
\pgfsetstrokecolor{currentstroke}%
\pgfsetdash{}{0pt}%
\pgfsys@defobject{currentmarker}{\pgfqpoint{0.000000in}{0.000000in}}{\pgfqpoint{0.000000in}{0.020833in}}{%
\pgfpathmoveto{\pgfqpoint{0.000000in}{0.000000in}}%
\pgfpathlineto{\pgfqpoint{0.000000in}{0.020833in}}%
\pgfusepath{stroke,fill}%
}%
\begin{pgfscope}%
\pgfsys@transformshift{3.622213in}{1.326898in}%
\pgfsys@useobject{currentmarker}{}%
\end{pgfscope}%
\end{pgfscope}%
\begin{pgfscope}%
\pgfsetbuttcap%
\pgfsetroundjoin%
\definecolor{currentfill}{rgb}{0.000000,0.000000,0.000000}%
\pgfsetfillcolor{currentfill}%
\pgfsetlinewidth{0.501875pt}%
\definecolor{currentstroke}{rgb}{0.000000,0.000000,0.000000}%
\pgfsetstrokecolor{currentstroke}%
\pgfsetdash{}{0pt}%
\pgfsys@defobject{currentmarker}{\pgfqpoint{0.000000in}{-0.020833in}}{\pgfqpoint{0.000000in}{0.000000in}}{%
\pgfpathmoveto{\pgfqpoint{0.000000in}{0.000000in}}%
\pgfpathlineto{\pgfqpoint{0.000000in}{-0.020833in}}%
\pgfusepath{stroke,fill}%
}%
\begin{pgfscope}%
\pgfsys@transformshift{3.622213in}{1.794149in}%
\pgfsys@useobject{currentmarker}{}%
\end{pgfscope}%
\end{pgfscope}%
\begin{pgfscope}%
\pgfsetbuttcap%
\pgfsetroundjoin%
\definecolor{currentfill}{rgb}{0.000000,0.000000,0.000000}%
\pgfsetfillcolor{currentfill}%
\pgfsetlinewidth{0.501875pt}%
\definecolor{currentstroke}{rgb}{0.000000,0.000000,0.000000}%
\pgfsetstrokecolor{currentstroke}%
\pgfsetdash{}{0pt}%
\pgfsys@defobject{currentmarker}{\pgfqpoint{0.000000in}{0.000000in}}{\pgfqpoint{0.000000in}{0.020833in}}{%
\pgfpathmoveto{\pgfqpoint{0.000000in}{0.000000in}}%
\pgfpathlineto{\pgfqpoint{0.000000in}{0.020833in}}%
\pgfusepath{stroke,fill}%
}%
\begin{pgfscope}%
\pgfsys@transformshift{3.718289in}{1.326898in}%
\pgfsys@useobject{currentmarker}{}%
\end{pgfscope}%
\end{pgfscope}%
\begin{pgfscope}%
\pgfsetbuttcap%
\pgfsetroundjoin%
\definecolor{currentfill}{rgb}{0.000000,0.000000,0.000000}%
\pgfsetfillcolor{currentfill}%
\pgfsetlinewidth{0.501875pt}%
\definecolor{currentstroke}{rgb}{0.000000,0.000000,0.000000}%
\pgfsetstrokecolor{currentstroke}%
\pgfsetdash{}{0pt}%
\pgfsys@defobject{currentmarker}{\pgfqpoint{0.000000in}{-0.020833in}}{\pgfqpoint{0.000000in}{0.000000in}}{%
\pgfpathmoveto{\pgfqpoint{0.000000in}{0.000000in}}%
\pgfpathlineto{\pgfqpoint{0.000000in}{-0.020833in}}%
\pgfusepath{stroke,fill}%
}%
\begin{pgfscope}%
\pgfsys@transformshift{3.718289in}{1.794149in}%
\pgfsys@useobject{currentmarker}{}%
\end{pgfscope}%
\end{pgfscope}%
\begin{pgfscope}%
\pgfsetbuttcap%
\pgfsetroundjoin%
\definecolor{currentfill}{rgb}{0.000000,0.000000,0.000000}%
\pgfsetfillcolor{currentfill}%
\pgfsetlinewidth{0.501875pt}%
\definecolor{currentstroke}{rgb}{0.000000,0.000000,0.000000}%
\pgfsetstrokecolor{currentstroke}%
\pgfsetdash{}{0pt}%
\pgfsys@defobject{currentmarker}{\pgfqpoint{0.000000in}{0.000000in}}{\pgfqpoint{0.000000in}{0.020833in}}{%
\pgfpathmoveto{\pgfqpoint{0.000000in}{0.000000in}}%
\pgfpathlineto{\pgfqpoint{0.000000in}{0.020833in}}%
\pgfusepath{stroke,fill}%
}%
\begin{pgfscope}%
\pgfsys@transformshift{3.814364in}{1.326898in}%
\pgfsys@useobject{currentmarker}{}%
\end{pgfscope}%
\end{pgfscope}%
\begin{pgfscope}%
\pgfsetbuttcap%
\pgfsetroundjoin%
\definecolor{currentfill}{rgb}{0.000000,0.000000,0.000000}%
\pgfsetfillcolor{currentfill}%
\pgfsetlinewidth{0.501875pt}%
\definecolor{currentstroke}{rgb}{0.000000,0.000000,0.000000}%
\pgfsetstrokecolor{currentstroke}%
\pgfsetdash{}{0pt}%
\pgfsys@defobject{currentmarker}{\pgfqpoint{0.000000in}{-0.020833in}}{\pgfqpoint{0.000000in}{0.000000in}}{%
\pgfpathmoveto{\pgfqpoint{0.000000in}{0.000000in}}%
\pgfpathlineto{\pgfqpoint{0.000000in}{-0.020833in}}%
\pgfusepath{stroke,fill}%
}%
\begin{pgfscope}%
\pgfsys@transformshift{3.814364in}{1.794149in}%
\pgfsys@useobject{currentmarker}{}%
\end{pgfscope}%
\end{pgfscope}%
\begin{pgfscope}%
\pgfsetbuttcap%
\pgfsetroundjoin%
\definecolor{currentfill}{rgb}{0.000000,0.000000,0.000000}%
\pgfsetfillcolor{currentfill}%
\pgfsetlinewidth{0.501875pt}%
\definecolor{currentstroke}{rgb}{0.000000,0.000000,0.000000}%
\pgfsetstrokecolor{currentstroke}%
\pgfsetdash{}{0pt}%
\pgfsys@defobject{currentmarker}{\pgfqpoint{0.000000in}{0.000000in}}{\pgfqpoint{0.000000in}{0.020833in}}{%
\pgfpathmoveto{\pgfqpoint{0.000000in}{0.000000in}}%
\pgfpathlineto{\pgfqpoint{0.000000in}{0.020833in}}%
\pgfusepath{stroke,fill}%
}%
\begin{pgfscope}%
\pgfsys@transformshift{3.910439in}{1.326898in}%
\pgfsys@useobject{currentmarker}{}%
\end{pgfscope}%
\end{pgfscope}%
\begin{pgfscope}%
\pgfsetbuttcap%
\pgfsetroundjoin%
\definecolor{currentfill}{rgb}{0.000000,0.000000,0.000000}%
\pgfsetfillcolor{currentfill}%
\pgfsetlinewidth{0.501875pt}%
\definecolor{currentstroke}{rgb}{0.000000,0.000000,0.000000}%
\pgfsetstrokecolor{currentstroke}%
\pgfsetdash{}{0pt}%
\pgfsys@defobject{currentmarker}{\pgfqpoint{0.000000in}{-0.020833in}}{\pgfqpoint{0.000000in}{0.000000in}}{%
\pgfpathmoveto{\pgfqpoint{0.000000in}{0.000000in}}%
\pgfpathlineto{\pgfqpoint{0.000000in}{-0.020833in}}%
\pgfusepath{stroke,fill}%
}%
\begin{pgfscope}%
\pgfsys@transformshift{3.910439in}{1.794149in}%
\pgfsys@useobject{currentmarker}{}%
\end{pgfscope}%
\end{pgfscope}%
\begin{pgfscope}%
\pgfsetbuttcap%
\pgfsetroundjoin%
\definecolor{currentfill}{rgb}{0.000000,0.000000,0.000000}%
\pgfsetfillcolor{currentfill}%
\pgfsetlinewidth{0.501875pt}%
\definecolor{currentstroke}{rgb}{0.000000,0.000000,0.000000}%
\pgfsetstrokecolor{currentstroke}%
\pgfsetdash{}{0pt}%
\pgfsys@defobject{currentmarker}{\pgfqpoint{0.000000in}{0.000000in}}{\pgfqpoint{0.000000in}{0.020833in}}{%
\pgfpathmoveto{\pgfqpoint{0.000000in}{0.000000in}}%
\pgfpathlineto{\pgfqpoint{0.000000in}{0.020833in}}%
\pgfusepath{stroke,fill}%
}%
\begin{pgfscope}%
\pgfsys@transformshift{4.102589in}{1.326898in}%
\pgfsys@useobject{currentmarker}{}%
\end{pgfscope}%
\end{pgfscope}%
\begin{pgfscope}%
\pgfsetbuttcap%
\pgfsetroundjoin%
\definecolor{currentfill}{rgb}{0.000000,0.000000,0.000000}%
\pgfsetfillcolor{currentfill}%
\pgfsetlinewidth{0.501875pt}%
\definecolor{currentstroke}{rgb}{0.000000,0.000000,0.000000}%
\pgfsetstrokecolor{currentstroke}%
\pgfsetdash{}{0pt}%
\pgfsys@defobject{currentmarker}{\pgfqpoint{0.000000in}{-0.020833in}}{\pgfqpoint{0.000000in}{0.000000in}}{%
\pgfpathmoveto{\pgfqpoint{0.000000in}{0.000000in}}%
\pgfpathlineto{\pgfqpoint{0.000000in}{-0.020833in}}%
\pgfusepath{stroke,fill}%
}%
\begin{pgfscope}%
\pgfsys@transformshift{4.102589in}{1.794149in}%
\pgfsys@useobject{currentmarker}{}%
\end{pgfscope}%
\end{pgfscope}%
\begin{pgfscope}%
\pgfsetbuttcap%
\pgfsetroundjoin%
\definecolor{currentfill}{rgb}{0.000000,0.000000,0.000000}%
\pgfsetfillcolor{currentfill}%
\pgfsetlinewidth{0.501875pt}%
\definecolor{currentstroke}{rgb}{0.000000,0.000000,0.000000}%
\pgfsetstrokecolor{currentstroke}%
\pgfsetdash{}{0pt}%
\pgfsys@defobject{currentmarker}{\pgfqpoint{0.000000in}{0.000000in}}{\pgfqpoint{0.000000in}{0.020833in}}{%
\pgfpathmoveto{\pgfqpoint{0.000000in}{0.000000in}}%
\pgfpathlineto{\pgfqpoint{0.000000in}{0.020833in}}%
\pgfusepath{stroke,fill}%
}%
\begin{pgfscope}%
\pgfsys@transformshift{4.198664in}{1.326898in}%
\pgfsys@useobject{currentmarker}{}%
\end{pgfscope}%
\end{pgfscope}%
\begin{pgfscope}%
\pgfsetbuttcap%
\pgfsetroundjoin%
\definecolor{currentfill}{rgb}{0.000000,0.000000,0.000000}%
\pgfsetfillcolor{currentfill}%
\pgfsetlinewidth{0.501875pt}%
\definecolor{currentstroke}{rgb}{0.000000,0.000000,0.000000}%
\pgfsetstrokecolor{currentstroke}%
\pgfsetdash{}{0pt}%
\pgfsys@defobject{currentmarker}{\pgfqpoint{0.000000in}{-0.020833in}}{\pgfqpoint{0.000000in}{0.000000in}}{%
\pgfpathmoveto{\pgfqpoint{0.000000in}{0.000000in}}%
\pgfpathlineto{\pgfqpoint{0.000000in}{-0.020833in}}%
\pgfusepath{stroke,fill}%
}%
\begin{pgfscope}%
\pgfsys@transformshift{4.198664in}{1.794149in}%
\pgfsys@useobject{currentmarker}{}%
\end{pgfscope}%
\end{pgfscope}%
\begin{pgfscope}%
\pgfsetbuttcap%
\pgfsetroundjoin%
\definecolor{currentfill}{rgb}{0.000000,0.000000,0.000000}%
\pgfsetfillcolor{currentfill}%
\pgfsetlinewidth{0.501875pt}%
\definecolor{currentstroke}{rgb}{0.000000,0.000000,0.000000}%
\pgfsetstrokecolor{currentstroke}%
\pgfsetdash{}{0pt}%
\pgfsys@defobject{currentmarker}{\pgfqpoint{0.000000in}{0.000000in}}{\pgfqpoint{0.000000in}{0.020833in}}{%
\pgfpathmoveto{\pgfqpoint{0.000000in}{0.000000in}}%
\pgfpathlineto{\pgfqpoint{0.000000in}{0.020833in}}%
\pgfusepath{stroke,fill}%
}%
\begin{pgfscope}%
\pgfsys@transformshift{4.294739in}{1.326898in}%
\pgfsys@useobject{currentmarker}{}%
\end{pgfscope}%
\end{pgfscope}%
\begin{pgfscope}%
\pgfsetbuttcap%
\pgfsetroundjoin%
\definecolor{currentfill}{rgb}{0.000000,0.000000,0.000000}%
\pgfsetfillcolor{currentfill}%
\pgfsetlinewidth{0.501875pt}%
\definecolor{currentstroke}{rgb}{0.000000,0.000000,0.000000}%
\pgfsetstrokecolor{currentstroke}%
\pgfsetdash{}{0pt}%
\pgfsys@defobject{currentmarker}{\pgfqpoint{0.000000in}{-0.020833in}}{\pgfqpoint{0.000000in}{0.000000in}}{%
\pgfpathmoveto{\pgfqpoint{0.000000in}{0.000000in}}%
\pgfpathlineto{\pgfqpoint{0.000000in}{-0.020833in}}%
\pgfusepath{stroke,fill}%
}%
\begin{pgfscope}%
\pgfsys@transformshift{4.294739in}{1.794149in}%
\pgfsys@useobject{currentmarker}{}%
\end{pgfscope}%
\end{pgfscope}%
\begin{pgfscope}%
\pgfsetbuttcap%
\pgfsetroundjoin%
\definecolor{currentfill}{rgb}{0.000000,0.000000,0.000000}%
\pgfsetfillcolor{currentfill}%
\pgfsetlinewidth{0.501875pt}%
\definecolor{currentstroke}{rgb}{0.000000,0.000000,0.000000}%
\pgfsetstrokecolor{currentstroke}%
\pgfsetdash{}{0pt}%
\pgfsys@defobject{currentmarker}{\pgfqpoint{0.000000in}{0.000000in}}{\pgfqpoint{0.000000in}{0.020833in}}{%
\pgfpathmoveto{\pgfqpoint{0.000000in}{0.000000in}}%
\pgfpathlineto{\pgfqpoint{0.000000in}{0.020833in}}%
\pgfusepath{stroke,fill}%
}%
\begin{pgfscope}%
\pgfsys@transformshift{4.390814in}{1.326898in}%
\pgfsys@useobject{currentmarker}{}%
\end{pgfscope}%
\end{pgfscope}%
\begin{pgfscope}%
\pgfsetbuttcap%
\pgfsetroundjoin%
\definecolor{currentfill}{rgb}{0.000000,0.000000,0.000000}%
\pgfsetfillcolor{currentfill}%
\pgfsetlinewidth{0.501875pt}%
\definecolor{currentstroke}{rgb}{0.000000,0.000000,0.000000}%
\pgfsetstrokecolor{currentstroke}%
\pgfsetdash{}{0pt}%
\pgfsys@defobject{currentmarker}{\pgfqpoint{0.000000in}{-0.020833in}}{\pgfqpoint{0.000000in}{0.000000in}}{%
\pgfpathmoveto{\pgfqpoint{0.000000in}{0.000000in}}%
\pgfpathlineto{\pgfqpoint{0.000000in}{-0.020833in}}%
\pgfusepath{stroke,fill}%
}%
\begin{pgfscope}%
\pgfsys@transformshift{4.390814in}{1.794149in}%
\pgfsys@useobject{currentmarker}{}%
\end{pgfscope}%
\end{pgfscope}%
\begin{pgfscope}%
\pgfsetbuttcap%
\pgfsetroundjoin%
\definecolor{currentfill}{rgb}{0.000000,0.000000,0.000000}%
\pgfsetfillcolor{currentfill}%
\pgfsetlinewidth{0.501875pt}%
\definecolor{currentstroke}{rgb}{0.000000,0.000000,0.000000}%
\pgfsetstrokecolor{currentstroke}%
\pgfsetdash{}{0pt}%
\pgfsys@defobject{currentmarker}{\pgfqpoint{0.000000in}{0.000000in}}{\pgfqpoint{0.000000in}{0.020833in}}{%
\pgfpathmoveto{\pgfqpoint{0.000000in}{0.000000in}}%
\pgfpathlineto{\pgfqpoint{0.000000in}{0.020833in}}%
\pgfusepath{stroke,fill}%
}%
\begin{pgfscope}%
\pgfsys@transformshift{4.582964in}{1.326898in}%
\pgfsys@useobject{currentmarker}{}%
\end{pgfscope}%
\end{pgfscope}%
\begin{pgfscope}%
\pgfsetbuttcap%
\pgfsetroundjoin%
\definecolor{currentfill}{rgb}{0.000000,0.000000,0.000000}%
\pgfsetfillcolor{currentfill}%
\pgfsetlinewidth{0.501875pt}%
\definecolor{currentstroke}{rgb}{0.000000,0.000000,0.000000}%
\pgfsetstrokecolor{currentstroke}%
\pgfsetdash{}{0pt}%
\pgfsys@defobject{currentmarker}{\pgfqpoint{0.000000in}{-0.020833in}}{\pgfqpoint{0.000000in}{0.000000in}}{%
\pgfpathmoveto{\pgfqpoint{0.000000in}{0.000000in}}%
\pgfpathlineto{\pgfqpoint{0.000000in}{-0.020833in}}%
\pgfusepath{stroke,fill}%
}%
\begin{pgfscope}%
\pgfsys@transformshift{4.582964in}{1.794149in}%
\pgfsys@useobject{currentmarker}{}%
\end{pgfscope}%
\end{pgfscope}%
\begin{pgfscope}%
\pgfsetbuttcap%
\pgfsetroundjoin%
\definecolor{currentfill}{rgb}{0.000000,0.000000,0.000000}%
\pgfsetfillcolor{currentfill}%
\pgfsetlinewidth{0.501875pt}%
\definecolor{currentstroke}{rgb}{0.000000,0.000000,0.000000}%
\pgfsetstrokecolor{currentstroke}%
\pgfsetdash{}{0pt}%
\pgfsys@defobject{currentmarker}{\pgfqpoint{0.000000in}{0.000000in}}{\pgfqpoint{0.041667in}{0.000000in}}{%
\pgfpathmoveto{\pgfqpoint{0.000000in}{0.000000in}}%
\pgfpathlineto{\pgfqpoint{0.041667in}{0.000000in}}%
\pgfusepath{stroke,fill}%
}%
\begin{pgfscope}%
\pgfsys@transformshift{0.444748in}{1.462853in}%
\pgfsys@useobject{currentmarker}{}%
\end{pgfscope}%
\end{pgfscope}%
\begin{pgfscope}%
\pgfsetbuttcap%
\pgfsetroundjoin%
\definecolor{currentfill}{rgb}{0.000000,0.000000,0.000000}%
\pgfsetfillcolor{currentfill}%
\pgfsetlinewidth{0.501875pt}%
\definecolor{currentstroke}{rgb}{0.000000,0.000000,0.000000}%
\pgfsetstrokecolor{currentstroke}%
\pgfsetdash{}{0pt}%
\pgfsys@defobject{currentmarker}{\pgfqpoint{-0.041667in}{0.000000in}}{\pgfqpoint{-0.000000in}{0.000000in}}{%
\pgfpathmoveto{\pgfqpoint{-0.000000in}{0.000000in}}%
\pgfpathlineto{\pgfqpoint{-0.041667in}{0.000000in}}%
\pgfusepath{stroke,fill}%
}%
\begin{pgfscope}%
\pgfsys@transformshift{4.676167in}{1.462853in}%
\pgfsys@useobject{currentmarker}{}%
\end{pgfscope}%
\end{pgfscope}%
\begin{pgfscope}%
\definecolor{textcolor}{rgb}{0.000000,0.000000,0.000000}%
\pgfsetstrokecolor{textcolor}%
\pgfsetfillcolor{textcolor}%
\pgftext[x=0.257248in, y=1.414635in, left, base]{\color{textcolor}\rmfamily\fontsize{10.000000}{12.000000}\selectfont \(\displaystyle {25}\)}%
\end{pgfscope}%
\begin{pgfscope}%
\pgfsetbuttcap%
\pgfsetroundjoin%
\definecolor{currentfill}{rgb}{0.000000,0.000000,0.000000}%
\pgfsetfillcolor{currentfill}%
\pgfsetlinewidth{0.501875pt}%
\definecolor{currentstroke}{rgb}{0.000000,0.000000,0.000000}%
\pgfsetstrokecolor{currentstroke}%
\pgfsetdash{}{0pt}%
\pgfsys@defobject{currentmarker}{\pgfqpoint{0.000000in}{0.000000in}}{\pgfqpoint{0.041667in}{0.000000in}}{%
\pgfpathmoveto{\pgfqpoint{0.000000in}{0.000000in}}%
\pgfpathlineto{\pgfqpoint{0.041667in}{0.000000in}}%
\pgfusepath{stroke,fill}%
}%
\begin{pgfscope}%
\pgfsys@transformshift{0.444748in}{1.609576in}%
\pgfsys@useobject{currentmarker}{}%
\end{pgfscope}%
\end{pgfscope}%
\begin{pgfscope}%
\pgfsetbuttcap%
\pgfsetroundjoin%
\definecolor{currentfill}{rgb}{0.000000,0.000000,0.000000}%
\pgfsetfillcolor{currentfill}%
\pgfsetlinewidth{0.501875pt}%
\definecolor{currentstroke}{rgb}{0.000000,0.000000,0.000000}%
\pgfsetstrokecolor{currentstroke}%
\pgfsetdash{}{0pt}%
\pgfsys@defobject{currentmarker}{\pgfqpoint{-0.041667in}{0.000000in}}{\pgfqpoint{-0.000000in}{0.000000in}}{%
\pgfpathmoveto{\pgfqpoint{-0.000000in}{0.000000in}}%
\pgfpathlineto{\pgfqpoint{-0.041667in}{0.000000in}}%
\pgfusepath{stroke,fill}%
}%
\begin{pgfscope}%
\pgfsys@transformshift{4.676167in}{1.609576in}%
\pgfsys@useobject{currentmarker}{}%
\end{pgfscope}%
\end{pgfscope}%
\begin{pgfscope}%
\definecolor{textcolor}{rgb}{0.000000,0.000000,0.000000}%
\pgfsetstrokecolor{textcolor}%
\pgfsetfillcolor{textcolor}%
\pgftext[x=0.257248in, y=1.561358in, left, base]{\color{textcolor}\rmfamily\fontsize{10.000000}{12.000000}\selectfont \(\displaystyle {50}\)}%
\end{pgfscope}%
\begin{pgfscope}%
\pgfsetbuttcap%
\pgfsetroundjoin%
\definecolor{currentfill}{rgb}{0.000000,0.000000,0.000000}%
\pgfsetfillcolor{currentfill}%
\pgfsetlinewidth{0.501875pt}%
\definecolor{currentstroke}{rgb}{0.000000,0.000000,0.000000}%
\pgfsetstrokecolor{currentstroke}%
\pgfsetdash{}{0pt}%
\pgfsys@defobject{currentmarker}{\pgfqpoint{0.000000in}{0.000000in}}{\pgfqpoint{0.041667in}{0.000000in}}{%
\pgfpathmoveto{\pgfqpoint{0.000000in}{0.000000in}}%
\pgfpathlineto{\pgfqpoint{0.041667in}{0.000000in}}%
\pgfusepath{stroke,fill}%
}%
\begin{pgfscope}%
\pgfsys@transformshift{0.444748in}{1.756299in}%
\pgfsys@useobject{currentmarker}{}%
\end{pgfscope}%
\end{pgfscope}%
\begin{pgfscope}%
\pgfsetbuttcap%
\pgfsetroundjoin%
\definecolor{currentfill}{rgb}{0.000000,0.000000,0.000000}%
\pgfsetfillcolor{currentfill}%
\pgfsetlinewidth{0.501875pt}%
\definecolor{currentstroke}{rgb}{0.000000,0.000000,0.000000}%
\pgfsetstrokecolor{currentstroke}%
\pgfsetdash{}{0pt}%
\pgfsys@defobject{currentmarker}{\pgfqpoint{-0.041667in}{0.000000in}}{\pgfqpoint{-0.000000in}{0.000000in}}{%
\pgfpathmoveto{\pgfqpoint{-0.000000in}{0.000000in}}%
\pgfpathlineto{\pgfqpoint{-0.041667in}{0.000000in}}%
\pgfusepath{stroke,fill}%
}%
\begin{pgfscope}%
\pgfsys@transformshift{4.676167in}{1.756299in}%
\pgfsys@useobject{currentmarker}{}%
\end{pgfscope}%
\end{pgfscope}%
\begin{pgfscope}%
\definecolor{textcolor}{rgb}{0.000000,0.000000,0.000000}%
\pgfsetstrokecolor{textcolor}%
\pgfsetfillcolor{textcolor}%
\pgftext[x=0.257248in, y=1.708081in, left, base]{\color{textcolor}\rmfamily\fontsize{10.000000}{12.000000}\selectfont \(\displaystyle {75}\)}%
\end{pgfscope}%
\begin{pgfscope}%
\pgfsetbuttcap%
\pgfsetroundjoin%
\definecolor{currentfill}{rgb}{0.000000,0.000000,0.000000}%
\pgfsetfillcolor{currentfill}%
\pgfsetlinewidth{0.501875pt}%
\definecolor{currentstroke}{rgb}{0.000000,0.000000,0.000000}%
\pgfsetstrokecolor{currentstroke}%
\pgfsetdash{}{0pt}%
\pgfsys@defobject{currentmarker}{\pgfqpoint{0.000000in}{0.000000in}}{\pgfqpoint{0.020833in}{0.000000in}}{%
\pgfpathmoveto{\pgfqpoint{0.000000in}{0.000000in}}%
\pgfpathlineto{\pgfqpoint{0.020833in}{0.000000in}}%
\pgfusepath{stroke,fill}%
}%
\begin{pgfscope}%
\pgfsys@transformshift{0.444748in}{1.345475in}%
\pgfsys@useobject{currentmarker}{}%
\end{pgfscope}%
\end{pgfscope}%
\begin{pgfscope}%
\pgfsetbuttcap%
\pgfsetroundjoin%
\definecolor{currentfill}{rgb}{0.000000,0.000000,0.000000}%
\pgfsetfillcolor{currentfill}%
\pgfsetlinewidth{0.501875pt}%
\definecolor{currentstroke}{rgb}{0.000000,0.000000,0.000000}%
\pgfsetstrokecolor{currentstroke}%
\pgfsetdash{}{0pt}%
\pgfsys@defobject{currentmarker}{\pgfqpoint{-0.020833in}{0.000000in}}{\pgfqpoint{-0.000000in}{0.000000in}}{%
\pgfpathmoveto{\pgfqpoint{-0.000000in}{0.000000in}}%
\pgfpathlineto{\pgfqpoint{-0.020833in}{0.000000in}}%
\pgfusepath{stroke,fill}%
}%
\begin{pgfscope}%
\pgfsys@transformshift{4.676167in}{1.345475in}%
\pgfsys@useobject{currentmarker}{}%
\end{pgfscope}%
\end{pgfscope}%
\begin{pgfscope}%
\pgfsetbuttcap%
\pgfsetroundjoin%
\definecolor{currentfill}{rgb}{0.000000,0.000000,0.000000}%
\pgfsetfillcolor{currentfill}%
\pgfsetlinewidth{0.501875pt}%
\definecolor{currentstroke}{rgb}{0.000000,0.000000,0.000000}%
\pgfsetstrokecolor{currentstroke}%
\pgfsetdash{}{0pt}%
\pgfsys@defobject{currentmarker}{\pgfqpoint{0.000000in}{0.000000in}}{\pgfqpoint{0.020833in}{0.000000in}}{%
\pgfpathmoveto{\pgfqpoint{0.000000in}{0.000000in}}%
\pgfpathlineto{\pgfqpoint{0.020833in}{0.000000in}}%
\pgfusepath{stroke,fill}%
}%
\begin{pgfscope}%
\pgfsys@transformshift{0.444748in}{1.374819in}%
\pgfsys@useobject{currentmarker}{}%
\end{pgfscope}%
\end{pgfscope}%
\begin{pgfscope}%
\pgfsetbuttcap%
\pgfsetroundjoin%
\definecolor{currentfill}{rgb}{0.000000,0.000000,0.000000}%
\pgfsetfillcolor{currentfill}%
\pgfsetlinewidth{0.501875pt}%
\definecolor{currentstroke}{rgb}{0.000000,0.000000,0.000000}%
\pgfsetstrokecolor{currentstroke}%
\pgfsetdash{}{0pt}%
\pgfsys@defobject{currentmarker}{\pgfqpoint{-0.020833in}{0.000000in}}{\pgfqpoint{-0.000000in}{0.000000in}}{%
\pgfpathmoveto{\pgfqpoint{-0.000000in}{0.000000in}}%
\pgfpathlineto{\pgfqpoint{-0.020833in}{0.000000in}}%
\pgfusepath{stroke,fill}%
}%
\begin{pgfscope}%
\pgfsys@transformshift{4.676167in}{1.374819in}%
\pgfsys@useobject{currentmarker}{}%
\end{pgfscope}%
\end{pgfscope}%
\begin{pgfscope}%
\pgfsetbuttcap%
\pgfsetroundjoin%
\definecolor{currentfill}{rgb}{0.000000,0.000000,0.000000}%
\pgfsetfillcolor{currentfill}%
\pgfsetlinewidth{0.501875pt}%
\definecolor{currentstroke}{rgb}{0.000000,0.000000,0.000000}%
\pgfsetstrokecolor{currentstroke}%
\pgfsetdash{}{0pt}%
\pgfsys@defobject{currentmarker}{\pgfqpoint{0.000000in}{0.000000in}}{\pgfqpoint{0.020833in}{0.000000in}}{%
\pgfpathmoveto{\pgfqpoint{0.000000in}{0.000000in}}%
\pgfpathlineto{\pgfqpoint{0.020833in}{0.000000in}}%
\pgfusepath{stroke,fill}%
}%
\begin{pgfscope}%
\pgfsys@transformshift{0.444748in}{1.404164in}%
\pgfsys@useobject{currentmarker}{}%
\end{pgfscope}%
\end{pgfscope}%
\begin{pgfscope}%
\pgfsetbuttcap%
\pgfsetroundjoin%
\definecolor{currentfill}{rgb}{0.000000,0.000000,0.000000}%
\pgfsetfillcolor{currentfill}%
\pgfsetlinewidth{0.501875pt}%
\definecolor{currentstroke}{rgb}{0.000000,0.000000,0.000000}%
\pgfsetstrokecolor{currentstroke}%
\pgfsetdash{}{0pt}%
\pgfsys@defobject{currentmarker}{\pgfqpoint{-0.020833in}{0.000000in}}{\pgfqpoint{-0.000000in}{0.000000in}}{%
\pgfpathmoveto{\pgfqpoint{-0.000000in}{0.000000in}}%
\pgfpathlineto{\pgfqpoint{-0.020833in}{0.000000in}}%
\pgfusepath{stroke,fill}%
}%
\begin{pgfscope}%
\pgfsys@transformshift{4.676167in}{1.404164in}%
\pgfsys@useobject{currentmarker}{}%
\end{pgfscope}%
\end{pgfscope}%
\begin{pgfscope}%
\pgfsetbuttcap%
\pgfsetroundjoin%
\definecolor{currentfill}{rgb}{0.000000,0.000000,0.000000}%
\pgfsetfillcolor{currentfill}%
\pgfsetlinewidth{0.501875pt}%
\definecolor{currentstroke}{rgb}{0.000000,0.000000,0.000000}%
\pgfsetstrokecolor{currentstroke}%
\pgfsetdash{}{0pt}%
\pgfsys@defobject{currentmarker}{\pgfqpoint{0.000000in}{0.000000in}}{\pgfqpoint{0.020833in}{0.000000in}}{%
\pgfpathmoveto{\pgfqpoint{0.000000in}{0.000000in}}%
\pgfpathlineto{\pgfqpoint{0.020833in}{0.000000in}}%
\pgfusepath{stroke,fill}%
}%
\begin{pgfscope}%
\pgfsys@transformshift{0.444748in}{1.433508in}%
\pgfsys@useobject{currentmarker}{}%
\end{pgfscope}%
\end{pgfscope}%
\begin{pgfscope}%
\pgfsetbuttcap%
\pgfsetroundjoin%
\definecolor{currentfill}{rgb}{0.000000,0.000000,0.000000}%
\pgfsetfillcolor{currentfill}%
\pgfsetlinewidth{0.501875pt}%
\definecolor{currentstroke}{rgb}{0.000000,0.000000,0.000000}%
\pgfsetstrokecolor{currentstroke}%
\pgfsetdash{}{0pt}%
\pgfsys@defobject{currentmarker}{\pgfqpoint{-0.020833in}{0.000000in}}{\pgfqpoint{-0.000000in}{0.000000in}}{%
\pgfpathmoveto{\pgfqpoint{-0.000000in}{0.000000in}}%
\pgfpathlineto{\pgfqpoint{-0.020833in}{0.000000in}}%
\pgfusepath{stroke,fill}%
}%
\begin{pgfscope}%
\pgfsys@transformshift{4.676167in}{1.433508in}%
\pgfsys@useobject{currentmarker}{}%
\end{pgfscope}%
\end{pgfscope}%
\begin{pgfscope}%
\pgfsetbuttcap%
\pgfsetroundjoin%
\definecolor{currentfill}{rgb}{0.000000,0.000000,0.000000}%
\pgfsetfillcolor{currentfill}%
\pgfsetlinewidth{0.501875pt}%
\definecolor{currentstroke}{rgb}{0.000000,0.000000,0.000000}%
\pgfsetstrokecolor{currentstroke}%
\pgfsetdash{}{0pt}%
\pgfsys@defobject{currentmarker}{\pgfqpoint{0.000000in}{0.000000in}}{\pgfqpoint{0.020833in}{0.000000in}}{%
\pgfpathmoveto{\pgfqpoint{0.000000in}{0.000000in}}%
\pgfpathlineto{\pgfqpoint{0.020833in}{0.000000in}}%
\pgfusepath{stroke,fill}%
}%
\begin{pgfscope}%
\pgfsys@transformshift{0.444748in}{1.492198in}%
\pgfsys@useobject{currentmarker}{}%
\end{pgfscope}%
\end{pgfscope}%
\begin{pgfscope}%
\pgfsetbuttcap%
\pgfsetroundjoin%
\definecolor{currentfill}{rgb}{0.000000,0.000000,0.000000}%
\pgfsetfillcolor{currentfill}%
\pgfsetlinewidth{0.501875pt}%
\definecolor{currentstroke}{rgb}{0.000000,0.000000,0.000000}%
\pgfsetstrokecolor{currentstroke}%
\pgfsetdash{}{0pt}%
\pgfsys@defobject{currentmarker}{\pgfqpoint{-0.020833in}{0.000000in}}{\pgfqpoint{-0.000000in}{0.000000in}}{%
\pgfpathmoveto{\pgfqpoint{-0.000000in}{0.000000in}}%
\pgfpathlineto{\pgfqpoint{-0.020833in}{0.000000in}}%
\pgfusepath{stroke,fill}%
}%
\begin{pgfscope}%
\pgfsys@transformshift{4.676167in}{1.492198in}%
\pgfsys@useobject{currentmarker}{}%
\end{pgfscope}%
\end{pgfscope}%
\begin{pgfscope}%
\pgfsetbuttcap%
\pgfsetroundjoin%
\definecolor{currentfill}{rgb}{0.000000,0.000000,0.000000}%
\pgfsetfillcolor{currentfill}%
\pgfsetlinewidth{0.501875pt}%
\definecolor{currentstroke}{rgb}{0.000000,0.000000,0.000000}%
\pgfsetstrokecolor{currentstroke}%
\pgfsetdash{}{0pt}%
\pgfsys@defobject{currentmarker}{\pgfqpoint{0.000000in}{0.000000in}}{\pgfqpoint{0.020833in}{0.000000in}}{%
\pgfpathmoveto{\pgfqpoint{0.000000in}{0.000000in}}%
\pgfpathlineto{\pgfqpoint{0.020833in}{0.000000in}}%
\pgfusepath{stroke,fill}%
}%
\begin{pgfscope}%
\pgfsys@transformshift{0.444748in}{1.521542in}%
\pgfsys@useobject{currentmarker}{}%
\end{pgfscope}%
\end{pgfscope}%
\begin{pgfscope}%
\pgfsetbuttcap%
\pgfsetroundjoin%
\definecolor{currentfill}{rgb}{0.000000,0.000000,0.000000}%
\pgfsetfillcolor{currentfill}%
\pgfsetlinewidth{0.501875pt}%
\definecolor{currentstroke}{rgb}{0.000000,0.000000,0.000000}%
\pgfsetstrokecolor{currentstroke}%
\pgfsetdash{}{0pt}%
\pgfsys@defobject{currentmarker}{\pgfqpoint{-0.020833in}{0.000000in}}{\pgfqpoint{-0.000000in}{0.000000in}}{%
\pgfpathmoveto{\pgfqpoint{-0.000000in}{0.000000in}}%
\pgfpathlineto{\pgfqpoint{-0.020833in}{0.000000in}}%
\pgfusepath{stroke,fill}%
}%
\begin{pgfscope}%
\pgfsys@transformshift{4.676167in}{1.521542in}%
\pgfsys@useobject{currentmarker}{}%
\end{pgfscope}%
\end{pgfscope}%
\begin{pgfscope}%
\pgfsetbuttcap%
\pgfsetroundjoin%
\definecolor{currentfill}{rgb}{0.000000,0.000000,0.000000}%
\pgfsetfillcolor{currentfill}%
\pgfsetlinewidth{0.501875pt}%
\definecolor{currentstroke}{rgb}{0.000000,0.000000,0.000000}%
\pgfsetstrokecolor{currentstroke}%
\pgfsetdash{}{0pt}%
\pgfsys@defobject{currentmarker}{\pgfqpoint{0.000000in}{0.000000in}}{\pgfqpoint{0.020833in}{0.000000in}}{%
\pgfpathmoveto{\pgfqpoint{0.000000in}{0.000000in}}%
\pgfpathlineto{\pgfqpoint{0.020833in}{0.000000in}}%
\pgfusepath{stroke,fill}%
}%
\begin{pgfscope}%
\pgfsys@transformshift{0.444748in}{1.550887in}%
\pgfsys@useobject{currentmarker}{}%
\end{pgfscope}%
\end{pgfscope}%
\begin{pgfscope}%
\pgfsetbuttcap%
\pgfsetroundjoin%
\definecolor{currentfill}{rgb}{0.000000,0.000000,0.000000}%
\pgfsetfillcolor{currentfill}%
\pgfsetlinewidth{0.501875pt}%
\definecolor{currentstroke}{rgb}{0.000000,0.000000,0.000000}%
\pgfsetstrokecolor{currentstroke}%
\pgfsetdash{}{0pt}%
\pgfsys@defobject{currentmarker}{\pgfqpoint{-0.020833in}{0.000000in}}{\pgfqpoint{-0.000000in}{0.000000in}}{%
\pgfpathmoveto{\pgfqpoint{-0.000000in}{0.000000in}}%
\pgfpathlineto{\pgfqpoint{-0.020833in}{0.000000in}}%
\pgfusepath{stroke,fill}%
}%
\begin{pgfscope}%
\pgfsys@transformshift{4.676167in}{1.550887in}%
\pgfsys@useobject{currentmarker}{}%
\end{pgfscope}%
\end{pgfscope}%
\begin{pgfscope}%
\pgfsetbuttcap%
\pgfsetroundjoin%
\definecolor{currentfill}{rgb}{0.000000,0.000000,0.000000}%
\pgfsetfillcolor{currentfill}%
\pgfsetlinewidth{0.501875pt}%
\definecolor{currentstroke}{rgb}{0.000000,0.000000,0.000000}%
\pgfsetstrokecolor{currentstroke}%
\pgfsetdash{}{0pt}%
\pgfsys@defobject{currentmarker}{\pgfqpoint{0.000000in}{0.000000in}}{\pgfqpoint{0.020833in}{0.000000in}}{%
\pgfpathmoveto{\pgfqpoint{0.000000in}{0.000000in}}%
\pgfpathlineto{\pgfqpoint{0.020833in}{0.000000in}}%
\pgfusepath{stroke,fill}%
}%
\begin{pgfscope}%
\pgfsys@transformshift{0.444748in}{1.580231in}%
\pgfsys@useobject{currentmarker}{}%
\end{pgfscope}%
\end{pgfscope}%
\begin{pgfscope}%
\pgfsetbuttcap%
\pgfsetroundjoin%
\definecolor{currentfill}{rgb}{0.000000,0.000000,0.000000}%
\pgfsetfillcolor{currentfill}%
\pgfsetlinewidth{0.501875pt}%
\definecolor{currentstroke}{rgb}{0.000000,0.000000,0.000000}%
\pgfsetstrokecolor{currentstroke}%
\pgfsetdash{}{0pt}%
\pgfsys@defobject{currentmarker}{\pgfqpoint{-0.020833in}{0.000000in}}{\pgfqpoint{-0.000000in}{0.000000in}}{%
\pgfpathmoveto{\pgfqpoint{-0.000000in}{0.000000in}}%
\pgfpathlineto{\pgfqpoint{-0.020833in}{0.000000in}}%
\pgfusepath{stroke,fill}%
}%
\begin{pgfscope}%
\pgfsys@transformshift{4.676167in}{1.580231in}%
\pgfsys@useobject{currentmarker}{}%
\end{pgfscope}%
\end{pgfscope}%
\begin{pgfscope}%
\pgfsetbuttcap%
\pgfsetroundjoin%
\definecolor{currentfill}{rgb}{0.000000,0.000000,0.000000}%
\pgfsetfillcolor{currentfill}%
\pgfsetlinewidth{0.501875pt}%
\definecolor{currentstroke}{rgb}{0.000000,0.000000,0.000000}%
\pgfsetstrokecolor{currentstroke}%
\pgfsetdash{}{0pt}%
\pgfsys@defobject{currentmarker}{\pgfqpoint{0.000000in}{0.000000in}}{\pgfqpoint{0.020833in}{0.000000in}}{%
\pgfpathmoveto{\pgfqpoint{0.000000in}{0.000000in}}%
\pgfpathlineto{\pgfqpoint{0.020833in}{0.000000in}}%
\pgfusepath{stroke,fill}%
}%
\begin{pgfscope}%
\pgfsys@transformshift{0.444748in}{1.638921in}%
\pgfsys@useobject{currentmarker}{}%
\end{pgfscope}%
\end{pgfscope}%
\begin{pgfscope}%
\pgfsetbuttcap%
\pgfsetroundjoin%
\definecolor{currentfill}{rgb}{0.000000,0.000000,0.000000}%
\pgfsetfillcolor{currentfill}%
\pgfsetlinewidth{0.501875pt}%
\definecolor{currentstroke}{rgb}{0.000000,0.000000,0.000000}%
\pgfsetstrokecolor{currentstroke}%
\pgfsetdash{}{0pt}%
\pgfsys@defobject{currentmarker}{\pgfqpoint{-0.020833in}{0.000000in}}{\pgfqpoint{-0.000000in}{0.000000in}}{%
\pgfpathmoveto{\pgfqpoint{-0.000000in}{0.000000in}}%
\pgfpathlineto{\pgfqpoint{-0.020833in}{0.000000in}}%
\pgfusepath{stroke,fill}%
}%
\begin{pgfscope}%
\pgfsys@transformshift{4.676167in}{1.638921in}%
\pgfsys@useobject{currentmarker}{}%
\end{pgfscope}%
\end{pgfscope}%
\begin{pgfscope}%
\pgfsetbuttcap%
\pgfsetroundjoin%
\definecolor{currentfill}{rgb}{0.000000,0.000000,0.000000}%
\pgfsetfillcolor{currentfill}%
\pgfsetlinewidth{0.501875pt}%
\definecolor{currentstroke}{rgb}{0.000000,0.000000,0.000000}%
\pgfsetstrokecolor{currentstroke}%
\pgfsetdash{}{0pt}%
\pgfsys@defobject{currentmarker}{\pgfqpoint{0.000000in}{0.000000in}}{\pgfqpoint{0.020833in}{0.000000in}}{%
\pgfpathmoveto{\pgfqpoint{0.000000in}{0.000000in}}%
\pgfpathlineto{\pgfqpoint{0.020833in}{0.000000in}}%
\pgfusepath{stroke,fill}%
}%
\begin{pgfscope}%
\pgfsys@transformshift{0.444748in}{1.668265in}%
\pgfsys@useobject{currentmarker}{}%
\end{pgfscope}%
\end{pgfscope}%
\begin{pgfscope}%
\pgfsetbuttcap%
\pgfsetroundjoin%
\definecolor{currentfill}{rgb}{0.000000,0.000000,0.000000}%
\pgfsetfillcolor{currentfill}%
\pgfsetlinewidth{0.501875pt}%
\definecolor{currentstroke}{rgb}{0.000000,0.000000,0.000000}%
\pgfsetstrokecolor{currentstroke}%
\pgfsetdash{}{0pt}%
\pgfsys@defobject{currentmarker}{\pgfqpoint{-0.020833in}{0.000000in}}{\pgfqpoint{-0.000000in}{0.000000in}}{%
\pgfpathmoveto{\pgfqpoint{-0.000000in}{0.000000in}}%
\pgfpathlineto{\pgfqpoint{-0.020833in}{0.000000in}}%
\pgfusepath{stroke,fill}%
}%
\begin{pgfscope}%
\pgfsys@transformshift{4.676167in}{1.668265in}%
\pgfsys@useobject{currentmarker}{}%
\end{pgfscope}%
\end{pgfscope}%
\begin{pgfscope}%
\pgfsetbuttcap%
\pgfsetroundjoin%
\definecolor{currentfill}{rgb}{0.000000,0.000000,0.000000}%
\pgfsetfillcolor{currentfill}%
\pgfsetlinewidth{0.501875pt}%
\definecolor{currentstroke}{rgb}{0.000000,0.000000,0.000000}%
\pgfsetstrokecolor{currentstroke}%
\pgfsetdash{}{0pt}%
\pgfsys@defobject{currentmarker}{\pgfqpoint{0.000000in}{0.000000in}}{\pgfqpoint{0.020833in}{0.000000in}}{%
\pgfpathmoveto{\pgfqpoint{0.000000in}{0.000000in}}%
\pgfpathlineto{\pgfqpoint{0.020833in}{0.000000in}}%
\pgfusepath{stroke,fill}%
}%
\begin{pgfscope}%
\pgfsys@transformshift{0.444748in}{1.697610in}%
\pgfsys@useobject{currentmarker}{}%
\end{pgfscope}%
\end{pgfscope}%
\begin{pgfscope}%
\pgfsetbuttcap%
\pgfsetroundjoin%
\definecolor{currentfill}{rgb}{0.000000,0.000000,0.000000}%
\pgfsetfillcolor{currentfill}%
\pgfsetlinewidth{0.501875pt}%
\definecolor{currentstroke}{rgb}{0.000000,0.000000,0.000000}%
\pgfsetstrokecolor{currentstroke}%
\pgfsetdash{}{0pt}%
\pgfsys@defobject{currentmarker}{\pgfqpoint{-0.020833in}{0.000000in}}{\pgfqpoint{-0.000000in}{0.000000in}}{%
\pgfpathmoveto{\pgfqpoint{-0.000000in}{0.000000in}}%
\pgfpathlineto{\pgfqpoint{-0.020833in}{0.000000in}}%
\pgfusepath{stroke,fill}%
}%
\begin{pgfscope}%
\pgfsys@transformshift{4.676167in}{1.697610in}%
\pgfsys@useobject{currentmarker}{}%
\end{pgfscope}%
\end{pgfscope}%
\begin{pgfscope}%
\pgfsetbuttcap%
\pgfsetroundjoin%
\definecolor{currentfill}{rgb}{0.000000,0.000000,0.000000}%
\pgfsetfillcolor{currentfill}%
\pgfsetlinewidth{0.501875pt}%
\definecolor{currentstroke}{rgb}{0.000000,0.000000,0.000000}%
\pgfsetstrokecolor{currentstroke}%
\pgfsetdash{}{0pt}%
\pgfsys@defobject{currentmarker}{\pgfqpoint{0.000000in}{0.000000in}}{\pgfqpoint{0.020833in}{0.000000in}}{%
\pgfpathmoveto{\pgfqpoint{0.000000in}{0.000000in}}%
\pgfpathlineto{\pgfqpoint{0.020833in}{0.000000in}}%
\pgfusepath{stroke,fill}%
}%
\begin{pgfscope}%
\pgfsys@transformshift{0.444748in}{1.726954in}%
\pgfsys@useobject{currentmarker}{}%
\end{pgfscope}%
\end{pgfscope}%
\begin{pgfscope}%
\pgfsetbuttcap%
\pgfsetroundjoin%
\definecolor{currentfill}{rgb}{0.000000,0.000000,0.000000}%
\pgfsetfillcolor{currentfill}%
\pgfsetlinewidth{0.501875pt}%
\definecolor{currentstroke}{rgb}{0.000000,0.000000,0.000000}%
\pgfsetstrokecolor{currentstroke}%
\pgfsetdash{}{0pt}%
\pgfsys@defobject{currentmarker}{\pgfqpoint{-0.020833in}{0.000000in}}{\pgfqpoint{-0.000000in}{0.000000in}}{%
\pgfpathmoveto{\pgfqpoint{-0.000000in}{0.000000in}}%
\pgfpathlineto{\pgfqpoint{-0.020833in}{0.000000in}}%
\pgfusepath{stroke,fill}%
}%
\begin{pgfscope}%
\pgfsys@transformshift{4.676167in}{1.726954in}%
\pgfsys@useobject{currentmarker}{}%
\end{pgfscope}%
\end{pgfscope}%
\begin{pgfscope}%
\pgfsetbuttcap%
\pgfsetroundjoin%
\definecolor{currentfill}{rgb}{0.000000,0.000000,0.000000}%
\pgfsetfillcolor{currentfill}%
\pgfsetlinewidth{0.501875pt}%
\definecolor{currentstroke}{rgb}{0.000000,0.000000,0.000000}%
\pgfsetstrokecolor{currentstroke}%
\pgfsetdash{}{0pt}%
\pgfsys@defobject{currentmarker}{\pgfqpoint{0.000000in}{0.000000in}}{\pgfqpoint{0.020833in}{0.000000in}}{%
\pgfpathmoveto{\pgfqpoint{0.000000in}{0.000000in}}%
\pgfpathlineto{\pgfqpoint{0.020833in}{0.000000in}}%
\pgfusepath{stroke,fill}%
}%
\begin{pgfscope}%
\pgfsys@transformshift{0.444748in}{1.785644in}%
\pgfsys@useobject{currentmarker}{}%
\end{pgfscope}%
\end{pgfscope}%
\begin{pgfscope}%
\pgfsetbuttcap%
\pgfsetroundjoin%
\definecolor{currentfill}{rgb}{0.000000,0.000000,0.000000}%
\pgfsetfillcolor{currentfill}%
\pgfsetlinewidth{0.501875pt}%
\definecolor{currentstroke}{rgb}{0.000000,0.000000,0.000000}%
\pgfsetstrokecolor{currentstroke}%
\pgfsetdash{}{0pt}%
\pgfsys@defobject{currentmarker}{\pgfqpoint{-0.020833in}{0.000000in}}{\pgfqpoint{-0.000000in}{0.000000in}}{%
\pgfpathmoveto{\pgfqpoint{-0.000000in}{0.000000in}}%
\pgfpathlineto{\pgfqpoint{-0.020833in}{0.000000in}}%
\pgfusepath{stroke,fill}%
}%
\begin{pgfscope}%
\pgfsys@transformshift{4.676167in}{1.785644in}%
\pgfsys@useobject{currentmarker}{}%
\end{pgfscope}%
\end{pgfscope}%
\begin{pgfscope}%
\definecolor{textcolor}{rgb}{0.000000,0.000000,0.000000}%
\pgfsetstrokecolor{textcolor}%
\pgfsetfillcolor{textcolor}%
\pgftext[x=0.201692in,y=1.560524in,,bottom,rotate=90.000000]{\color{textcolor}\rmfamily\fontsize{12.000000}{14.400000}\selectfont \(\displaystyle V_s\) (\unit{\micro\volt})}%
\end{pgfscope}%
\begin{pgfscope}%
\pgfpathrectangle{\pgfqpoint{0.444748in}{1.326898in}}{\pgfqpoint{4.231419in}{0.467251in}}%
\pgfusepath{clip}%
\pgfsetbuttcap%
\pgfsetroundjoin%
\pgfsetlinewidth{1.003750pt}%
\definecolor{currentstroke}{rgb}{0.047059,0.364706,0.647059}%
\pgfsetstrokecolor{currentstroke}%
\pgfsetdash{{3.700000pt}{1.600000pt}}{0.000000pt}%
\pgfpathmoveto{\pgfqpoint{0.652108in}{1.498748in}}%
\pgfpathlineto{\pgfqpoint{0.657273in}{1.450784in}}%
\pgfpathlineto{\pgfqpoint{0.677929in}{1.378001in}}%
\pgfpathlineto{\pgfqpoint{0.694594in}{1.377324in}}%
\pgfpathlineto{\pgfqpoint{0.712668in}{1.437962in}}%
\pgfpathlineto{\pgfqpoint{0.731681in}{1.524905in}}%
\pgfpathlineto{\pgfqpoint{0.753980in}{1.686005in}}%
\pgfpathlineto{\pgfqpoint{0.773228in}{1.761981in}}%
\pgfpathlineto{\pgfqpoint{0.791538in}{1.752397in}}%
\pgfpathlineto{\pgfqpoint{0.810082in}{1.659671in}}%
\pgfpathlineto{\pgfqpoint{0.830033in}{1.511780in}}%
\pgfpathlineto{\pgfqpoint{0.848107in}{1.422598in}}%
\pgfpathlineto{\pgfqpoint{0.866886in}{1.367133in}}%
\pgfpathlineto{\pgfqpoint{0.885665in}{1.385040in}}%
\pgfpathlineto{\pgfqpoint{0.907024in}{1.453870in}}%
\pgfpathlineto{\pgfqpoint{0.924629in}{1.556132in}}%
\pgfpathlineto{\pgfqpoint{0.943876in}{1.690691in}}%
\pgfpathlineto{\pgfqpoint{0.964767in}{1.758165in}}%
\pgfpathlineto{\pgfqpoint{0.983077in}{1.724567in}}%
\pgfpathlineto{\pgfqpoint{1.019930in}{1.475797in}}%
\pgfpathlineto{\pgfqpoint{1.043168in}{1.385311in}}%
\pgfpathlineto{\pgfqpoint{1.060773in}{1.360999in}}%
\pgfpathlineto{\pgfqpoint{1.079786in}{1.401421in}}%
\pgfpathlineto{\pgfqpoint{1.102554in}{1.500475in}}%
\pgfpathlineto{\pgfqpoint{1.115699in}{1.581227in}}%
\pgfpathlineto{\pgfqpoint{1.133304in}{1.732099in}}%
\pgfpathlineto{\pgfqpoint{1.155603in}{1.752655in}}%
\pgfpathlineto{\pgfqpoint{1.176259in}{1.688576in}}%
\pgfpathlineto{\pgfqpoint{1.195272in}{1.560838in}}%
\pgfpathlineto{\pgfqpoint{1.214051in}{1.444090in}}%
\pgfpathlineto{\pgfqpoint{1.230716in}{1.380713in}}%
\pgfpathlineto{\pgfqpoint{1.251607in}{1.357151in}}%
\pgfpathlineto{\pgfqpoint{1.272263in}{1.397419in}}%
\pgfpathlineto{\pgfqpoint{1.292450in}{1.474657in}}%
\pgfpathlineto{\pgfqpoint{1.310290in}{1.578340in}}%
\pgfpathlineto{\pgfqpoint{1.328834in}{1.697052in}}%
\pgfpathlineto{\pgfqpoint{1.349019in}{1.747644in}}%
\pgfpathlineto{\pgfqpoint{1.366390in}{1.717955in}}%
\pgfpathlineto{\pgfqpoint{1.387280in}{1.599496in}}%
\pgfpathlineto{\pgfqpoint{1.405825in}{1.475322in}}%
\pgfpathlineto{\pgfqpoint{1.424367in}{1.396755in}}%
\pgfpathlineto{\pgfqpoint{1.446432in}{1.355327in}}%
\pgfpathlineto{\pgfqpoint{1.464976in}{1.374839in}}%
\pgfpathlineto{\pgfqpoint{1.482112in}{1.419054in}}%
\pgfpathlineto{\pgfqpoint{1.504411in}{1.427425in}}%
\pgfpathlineto{\pgfqpoint{1.520842in}{1.500140in}}%
\pgfpathlineto{\pgfqpoint{1.540324in}{1.626767in}}%
\pgfpathlineto{\pgfqpoint{1.561685in}{1.731621in}}%
\pgfpathlineto{\pgfqpoint{1.579525in}{1.739485in}}%
\pgfpathlineto{\pgfqpoint{1.597364in}{1.692577in}}%
\pgfpathlineto{\pgfqpoint{1.620602in}{1.519136in}}%
\pgfpathlineto{\pgfqpoint{1.634685in}{1.606888in}}%
\pgfpathlineto{\pgfqpoint{1.662618in}{1.448569in}}%
\pgfpathlineto{\pgfqpoint{1.676937in}{1.393823in}}%
\pgfpathlineto{\pgfqpoint{1.694542in}{1.357091in}}%
\pgfpathlineto{\pgfqpoint{1.712381in}{1.357672in}}%
\pgfpathlineto{\pgfqpoint{1.732568in}{1.409396in}}%
\pgfpathlineto{\pgfqpoint{1.752285in}{1.496036in}}%
\pgfpathlineto{\pgfqpoint{1.771767in}{1.612154in}}%
\pgfpathlineto{\pgfqpoint{1.790311in}{1.711801in}}%
\pgfpathlineto{\pgfqpoint{1.810028in}{1.740513in}}%
\pgfpathlineto{\pgfqpoint{1.828104in}{1.716876in}}%
\pgfpathlineto{\pgfqpoint{1.845943in}{1.614540in}}%
\pgfpathlineto{\pgfqpoint{1.866364in}{1.478878in}}%
\pgfpathlineto{\pgfqpoint{1.887255in}{1.411165in}}%
\pgfpathlineto{\pgfqpoint{1.904623in}{1.366373in}}%
\pgfpathlineto{\pgfqpoint{1.925985in}{1.353803in}}%
\pgfpathlineto{\pgfqpoint{1.943590in}{1.388758in}}%
\pgfpathlineto{\pgfqpoint{1.960960in}{1.453121in}}%
\pgfpathlineto{\pgfqpoint{1.982790in}{1.540406in}}%
\pgfpathlineto{\pgfqpoint{2.000629in}{1.662950in}}%
\pgfpathlineto{\pgfqpoint{2.021754in}{1.737150in}}%
\pgfpathlineto{\pgfqpoint{2.039359in}{1.733002in}}%
\pgfpathlineto{\pgfqpoint{2.056964in}{1.696799in}}%
\pgfpathlineto{\pgfqpoint{2.080203in}{1.559762in}}%
\pgfpathlineto{\pgfqpoint{2.096163in}{1.455384in}}%
\pgfpathlineto{\pgfqpoint{2.117290in}{1.389355in}}%
\pgfpathlineto{\pgfqpoint{2.135598in}{1.357606in}}%
\pgfpathlineto{\pgfqpoint{2.156724in}{1.360355in}}%
\pgfpathlineto{\pgfqpoint{2.173624in}{1.399646in}}%
\pgfpathlineto{\pgfqpoint{2.194280in}{1.463433in}}%
\pgfpathlineto{\pgfqpoint{2.210945in}{1.564607in}}%
\pgfpathlineto{\pgfqpoint{2.230429in}{1.670074in}}%
\pgfpathlineto{\pgfqpoint{2.252258in}{1.732803in}}%
\pgfpathlineto{\pgfqpoint{2.269863in}{1.728156in}}%
\pgfpathlineto{\pgfqpoint{2.290519in}{1.637707in}}%
\pgfpathlineto{\pgfqpoint{2.308594in}{1.518613in}}%
\pgfpathlineto{\pgfqpoint{2.326668in}{1.433769in}}%
\pgfpathlineto{\pgfqpoint{2.347793in}{1.376094in}}%
\pgfpathlineto{\pgfqpoint{2.365632in}{1.350593in}}%
\pgfpathlineto{\pgfqpoint{2.385819in}{1.367910in}}%
\pgfpathlineto{\pgfqpoint{2.404598in}{1.409429in}}%
\pgfpathlineto{\pgfqpoint{2.426663in}{1.468345in}}%
\pgfpathlineto{\pgfqpoint{2.442625in}{1.560035in}}%
\pgfpathlineto{\pgfqpoint{2.466098in}{1.686958in}}%
\pgfpathlineto{\pgfqpoint{2.482058in}{1.734337in}}%
\pgfpathlineto{\pgfqpoint{2.500602in}{1.727641in}}%
\pgfpathlineto{\pgfqpoint{2.518441in}{1.650735in}}%
\pgfpathlineto{\pgfqpoint{2.539332in}{1.513818in}}%
\pgfpathlineto{\pgfqpoint{2.560928in}{1.428058in}}%
\pgfpathlineto{\pgfqpoint{2.578767in}{1.376332in}}%
\pgfpathlineto{\pgfqpoint{2.596606in}{1.353091in}}%
\pgfpathlineto{\pgfqpoint{2.620548in}{1.371774in}}%
\pgfpathlineto{\pgfqpoint{2.635336in}{1.408273in}}%
\pgfpathlineto{\pgfqpoint{2.654586in}{1.469957in}}%
\pgfpathlineto{\pgfqpoint{2.674068in}{1.590909in}}%
\pgfpathlineto{\pgfqpoint{2.691672in}{1.676350in}}%
\pgfpathlineto{\pgfqpoint{2.712329in}{1.710999in}}%
\pgfpathlineto{\pgfqpoint{2.731576in}{1.738086in}}%
\pgfpathlineto{\pgfqpoint{2.749416in}{1.727149in}}%
\pgfpathlineto{\pgfqpoint{2.770541in}{1.678663in}}%
\pgfpathlineto{\pgfqpoint{2.788380in}{1.602729in}}%
\pgfpathlineto{\pgfqpoint{2.810444in}{1.499133in}}%
\pgfpathlineto{\pgfqpoint{2.828989in}{1.433443in}}%
\pgfpathlineto{\pgfqpoint{2.845420in}{1.380631in}}%
\pgfpathlineto{\pgfqpoint{2.866076in}{1.353371in}}%
\pgfpathlineto{\pgfqpoint{2.884620in}{1.374077in}}%
\pgfpathlineto{\pgfqpoint{2.905042in}{1.433541in}}%
\pgfpathlineto{\pgfqpoint{2.925227in}{1.515262in}}%
\pgfpathlineto{\pgfqpoint{2.943066in}{1.625728in}}%
\pgfpathlineto{\pgfqpoint{2.962080in}{1.711984in}}%
\pgfpathlineto{\pgfqpoint{2.979216in}{1.741176in}}%
\pgfpathlineto{\pgfqpoint{3.002689in}{1.720853in}}%
\pgfpathlineto{\pgfqpoint{3.022171in}{1.629774in}}%
\pgfpathlineto{\pgfqpoint{3.040715in}{1.526577in}}%
\pgfpathlineto{\pgfqpoint{3.058084in}{1.556063in}}%
\pgfpathlineto{\pgfqpoint{3.078976in}{1.441126in}}%
\pgfpathlineto{\pgfqpoint{3.097753in}{1.390628in}}%
\pgfpathlineto{\pgfqpoint{3.115123in}{1.358157in}}%
\pgfpathlineto{\pgfqpoint{3.136250in}{1.361284in}}%
\pgfpathlineto{\pgfqpoint{3.153853in}{1.380539in}}%
\pgfpathlineto{\pgfqpoint{3.172632in}{1.431342in}}%
\pgfpathlineto{\pgfqpoint{3.193288in}{1.511871in}}%
\pgfpathlineto{\pgfqpoint{3.211362in}{1.619598in}}%
\pgfpathlineto{\pgfqpoint{3.232254in}{1.696985in}}%
\pgfpathlineto{\pgfqpoint{3.250328in}{1.742014in}}%
\pgfpathlineto{\pgfqpoint{3.267464in}{1.736090in}}%
\pgfpathlineto{\pgfqpoint{3.288823in}{1.674909in}}%
\pgfpathlineto{\pgfqpoint{3.306897in}{1.569507in}}%
\pgfpathlineto{\pgfqpoint{3.325676in}{1.476208in}}%
\pgfpathlineto{\pgfqpoint{3.347975in}{1.414032in}}%
\pgfpathlineto{\pgfqpoint{3.367693in}{1.369223in}}%
\pgfpathlineto{\pgfqpoint{3.382950in}{1.357151in}}%
\pgfpathlineto{\pgfqpoint{3.403606in}{1.372505in}}%
\pgfpathlineto{\pgfqpoint{3.423088in}{1.409723in}}%
\pgfpathlineto{\pgfqpoint{3.443041in}{1.482599in}}%
\pgfpathlineto{\pgfqpoint{3.481771in}{1.679915in}}%
\pgfpathlineto{\pgfqpoint{3.500784in}{1.736457in}}%
\pgfpathlineto{\pgfqpoint{3.517449in}{1.747291in}}%
\pgfpathlineto{\pgfqpoint{3.535759in}{1.747853in}}%
\pgfpathlineto{\pgfqpoint{3.557119in}{1.697837in}}%
\pgfpathlineto{\pgfqpoint{3.596319in}{1.481198in}}%
\pgfpathlineto{\pgfqpoint{3.615801in}{1.427530in}}%
\pgfpathlineto{\pgfqpoint{3.635989in}{1.385574in}}%
\pgfpathlineto{\pgfqpoint{3.653828in}{1.359084in}}%
\pgfpathlineto{\pgfqpoint{3.674718in}{1.375239in}}%
\pgfpathlineto{\pgfqpoint{3.692792in}{1.415044in}}%
\pgfpathlineto{\pgfqpoint{3.711102in}{1.472434in}}%
\pgfpathlineto{\pgfqpoint{3.770957in}{1.723744in}}%
\pgfpathlineto{\pgfqpoint{3.788327in}{1.751444in}}%
\pgfpathlineto{\pgfqpoint{3.809218in}{1.754472in}}%
\pgfpathlineto{\pgfqpoint{3.828936in}{1.715286in}}%
\pgfpathlineto{\pgfqpoint{3.866963in}{1.521469in}}%
\pgfpathlineto{\pgfqpoint{3.884566in}{1.453300in}}%
\pgfpathlineto{\pgfqpoint{3.903579in}{1.401274in}}%
\pgfpathlineto{\pgfqpoint{3.922123in}{1.417930in}}%
\pgfpathlineto{\pgfqpoint{3.939259in}{1.375561in}}%
\pgfpathlineto{\pgfqpoint{3.960853in}{1.362914in}}%
\pgfpathlineto{\pgfqpoint{3.980806in}{1.387415in}}%
\pgfpathlineto{\pgfqpoint{3.999819in}{1.428021in}}%
\pgfpathlineto{\pgfqpoint{4.038783in}{1.573214in}}%
\pgfpathlineto{\pgfqpoint{4.056623in}{1.674071in}}%
\pgfpathlineto{\pgfqpoint{4.077513in}{1.743822in}}%
\pgfpathlineto{\pgfqpoint{4.094180in}{1.764063in}}%
\pgfpathlineto{\pgfqpoint{4.116011in}{1.747660in}}%
\pgfpathlineto{\pgfqpoint{4.135024in}{1.695956in}}%
\pgfpathlineto{\pgfqpoint{4.154740in}{1.612986in}}%
\pgfpathlineto{\pgfqpoint{4.177274in}{1.497353in}}%
\pgfpathlineto{\pgfqpoint{4.193470in}{1.440253in}}%
\pgfpathlineto{\pgfqpoint{4.212249in}{1.409294in}}%
\pgfpathlineto{\pgfqpoint{4.228211in}{1.375667in}}%
\pgfpathlineto{\pgfqpoint{4.250041in}{1.379644in}}%
\pgfpathlineto{\pgfqpoint{4.288536in}{1.473028in}}%
\pgfpathlineto{\pgfqpoint{4.306610in}{1.550058in}}%
\pgfpathlineto{\pgfqpoint{4.325858in}{1.622555in}}%
\pgfpathlineto{\pgfqpoint{4.345340in}{1.681941in}}%
\pgfpathlineto{\pgfqpoint{4.364588in}{1.754713in}}%
\pgfpathlineto{\pgfqpoint{4.383366in}{1.771814in}}%
\pgfpathlineto{\pgfqpoint{4.402145in}{1.762704in}}%
\pgfpathlineto{\pgfqpoint{4.421627in}{1.705530in}}%
\pgfpathlineto{\pgfqpoint{4.441109in}{1.678461in}}%
\pgfpathlineto{\pgfqpoint{4.459419in}{1.571454in}}%
\pgfpathlineto{\pgfqpoint{4.479841in}{1.471158in}}%
\pgfpathlineto{\pgfqpoint{4.473973in}{1.521698in}}%
\pgfpathlineto{\pgfqpoint{4.438998in}{1.772559in}}%
\pgfpathlineto{\pgfqpoint{4.416228in}{1.595322in}}%
\pgfpathlineto{\pgfqpoint{4.397451in}{1.413626in}}%
\pgfpathlineto{\pgfqpoint{4.376558in}{1.368033in}}%
\pgfpathlineto{\pgfqpoint{4.358250in}{1.404170in}}%
\pgfpathlineto{\pgfqpoint{4.340646in}{1.482804in}}%
\pgfpathlineto{\pgfqpoint{4.316938in}{1.657163in}}%
\pgfpathlineto{\pgfqpoint{4.303090in}{1.740471in}}%
\pgfpathlineto{\pgfqpoint{4.283371in}{1.763238in}}%
\pgfpathlineto{\pgfqpoint{4.265298in}{1.706864in}}%
\pgfpathlineto{\pgfqpoint{4.242764in}{1.551594in}}%
\pgfpathlineto{\pgfqpoint{4.224923in}{1.437570in}}%
\pgfpathlineto{\pgfqpoint{4.204267in}{1.379248in}}%
\pgfpathlineto{\pgfqpoint{4.185019in}{1.375463in}}%
\pgfpathlineto{\pgfqpoint{4.166711in}{1.430991in}}%
\pgfpathlineto{\pgfqpoint{4.149341in}{1.528136in}}%
\pgfpathlineto{\pgfqpoint{4.129154in}{1.685655in}}%
\pgfpathlineto{\pgfqpoint{4.109906in}{1.753757in}}%
\pgfpathlineto{\pgfqpoint{4.088078in}{1.733658in}}%
\pgfpathlineto{\pgfqpoint{4.052397in}{1.506526in}}%
\pgfpathlineto{\pgfqpoint{4.034793in}{1.419422in}}%
\pgfpathlineto{\pgfqpoint{4.012493in}{1.370045in}}%
\pgfpathlineto{\pgfqpoint{3.991134in}{1.371419in}}%
\pgfpathlineto{\pgfqpoint{3.973529in}{1.427959in}}%
\pgfpathlineto{\pgfqpoint{3.952873in}{1.536026in}}%
\pgfpathlineto{\pgfqpoint{3.938554in}{1.656155in}}%
\pgfpathlineto{\pgfqpoint{3.914143in}{1.748821in}}%
\pgfpathlineto{\pgfqpoint{3.897947in}{1.742667in}}%
\pgfpathlineto{\pgfqpoint{3.880342in}{1.661727in}}%
\pgfpathlineto{\pgfqpoint{3.858981in}{1.505989in}}%
\pgfpathlineto{\pgfqpoint{3.836447in}{1.404221in}}%
\pgfpathlineto{\pgfqpoint{3.822363in}{1.368157in}}%
\pgfpathlineto{\pgfqpoint{3.800533in}{1.361263in}}%
\pgfpathlineto{\pgfqpoint{3.783164in}{1.384338in}}%
\pgfpathlineto{\pgfqpoint{3.764620in}{1.453746in}}%
\pgfpathlineto{\pgfqpoint{3.724011in}{1.701122in}}%
\pgfpathlineto{\pgfqpoint{3.704999in}{1.747103in}}%
\pgfpathlineto{\pgfqpoint{3.686689in}{1.724352in}}%
\pgfpathlineto{\pgfqpoint{3.665799in}{1.634178in}}%
\pgfpathlineto{\pgfqpoint{3.648428in}{1.526066in}}%
\pgfpathlineto{\pgfqpoint{3.608290in}{1.388701in}}%
\pgfpathlineto{\pgfqpoint{3.588103in}{1.354821in}}%
\pgfpathlineto{\pgfqpoint{3.571438in}{1.376784in}}%
\pgfpathlineto{\pgfqpoint{3.551721in}{1.433440in}}%
\pgfpathlineto{\pgfqpoint{3.532708in}{1.500922in}}%
\pgfpathlineto{\pgfqpoint{3.512990in}{1.633814in}}%
\pgfpathlineto{\pgfqpoint{3.493273in}{1.716250in}}%
\pgfpathlineto{\pgfqpoint{3.474963in}{1.737138in}}%
\pgfpathlineto{\pgfqpoint{3.456421in}{1.733555in}}%
\pgfpathlineto{\pgfqpoint{3.437173in}{1.645917in}}%
\pgfpathlineto{\pgfqpoint{3.417691in}{1.510198in}}%
\pgfpathlineto{\pgfqpoint{3.398207in}{1.426343in}}%
\pgfpathlineto{\pgfqpoint{3.376142in}{1.376199in}}%
\pgfpathlineto{\pgfqpoint{3.357834in}{1.352716in}}%
\pgfpathlineto{\pgfqpoint{3.338586in}{1.374829in}}%
\pgfpathlineto{\pgfqpoint{3.322156in}{1.427902in}}%
\pgfpathlineto{\pgfqpoint{3.299857in}{1.508326in}}%
\pgfpathlineto{\pgfqpoint{3.282721in}{1.434265in}}%
\pgfpathlineto{\pgfqpoint{3.263004in}{1.547517in}}%
\pgfpathlineto{\pgfqpoint{3.242112in}{1.680258in}}%
\pgfpathlineto{\pgfqpoint{3.224978in}{1.737717in}}%
\pgfpathlineto{\pgfqpoint{3.206433in}{1.737258in}}%
\pgfpathlineto{\pgfqpoint{3.183665in}{1.663085in}}%
\pgfpathlineto{\pgfqpoint{3.165355in}{1.537480in}}%
\pgfpathlineto{\pgfqpoint{3.150099in}{1.458750in}}%
\pgfpathlineto{\pgfqpoint{3.128974in}{1.394052in}}%
\pgfpathlineto{\pgfqpoint{3.109255in}{1.355061in}}%
\pgfpathlineto{\pgfqpoint{3.090242in}{1.364199in}}%
\pgfpathlineto{\pgfqpoint{3.070994in}{1.408701in}}%
\pgfpathlineto{\pgfqpoint{3.053624in}{1.475178in}}%
\pgfpathlineto{\pgfqpoint{3.034142in}{1.580312in}}%
\pgfpathlineto{\pgfqpoint{3.014191in}{1.364164in}}%
\pgfpathlineto{\pgfqpoint{2.993769in}{1.409297in}}%
\pgfpathlineto{\pgfqpoint{2.975695in}{1.459467in}}%
\pgfpathlineto{\pgfqpoint{2.956917in}{1.570708in}}%
\pgfpathlineto{\pgfqpoint{2.934852in}{1.709016in}}%
\pgfpathlineto{\pgfqpoint{2.917013in}{1.739321in}}%
\pgfpathlineto{\pgfqpoint{2.898234in}{1.697258in}}%
\pgfpathlineto{\pgfqpoint{2.879926in}{1.583734in}}%
\pgfpathlineto{\pgfqpoint{2.860442in}{1.473177in}}%
\pgfpathlineto{\pgfqpoint{2.840960in}{1.400406in}}%
\pgfpathlineto{\pgfqpoint{2.820304in}{1.355662in}}%
\pgfpathlineto{\pgfqpoint{2.801759in}{1.359590in}}%
\pgfpathlineto{\pgfqpoint{2.786268in}{1.381057in}}%
\pgfpathlineto{\pgfqpoint{2.764203in}{1.452206in}}%
\pgfpathlineto{\pgfqpoint{2.724300in}{1.692035in}}%
\pgfpathlineto{\pgfqpoint{2.706460in}{1.738009in}}%
\pgfpathlineto{\pgfqpoint{2.688150in}{1.725318in}}%
\pgfpathlineto{\pgfqpoint{2.666791in}{1.632723in}}%
\pgfpathlineto{\pgfqpoint{2.645430in}{1.489693in}}%
\pgfpathlineto{\pgfqpoint{2.625713in}{1.414437in}}%
\pgfpathlineto{\pgfqpoint{2.607405in}{1.365983in}}%
\pgfpathlineto{\pgfqpoint{2.591912in}{1.350536in}}%
\pgfpathlineto{\pgfqpoint{2.571725in}{1.357925in}}%
\pgfpathlineto{\pgfqpoint{2.553417in}{1.398071in}}%
\pgfpathlineto{\pgfqpoint{2.533229in}{1.465381in}}%
\pgfpathlineto{\pgfqpoint{2.513042in}{1.557949in}}%
\pgfpathlineto{\pgfqpoint{2.494499in}{1.685357in}}%
\pgfpathlineto{\pgfqpoint{2.475955in}{1.737143in}}%
\pgfpathlineto{\pgfqpoint{2.457178in}{1.722000in}}%
\pgfpathlineto{\pgfqpoint{2.435114in}{1.616860in}}%
\pgfpathlineto{\pgfqpoint{2.420560in}{1.567675in}}%
\pgfpathlineto{\pgfqpoint{2.398730in}{1.453777in}}%
\pgfpathlineto{\pgfqpoint{2.379717in}{1.394719in}}%
\pgfpathlineto{\pgfqpoint{2.360703in}{1.358320in}}%
\pgfpathlineto{\pgfqpoint{2.343099in}{1.357101in}}%
\pgfpathlineto{\pgfqpoint{2.320800in}{1.398445in}}%
\pgfpathlineto{\pgfqpoint{2.301552in}{1.458788in}}%
\pgfpathlineto{\pgfqpoint{2.264934in}{1.689612in}}%
\pgfpathlineto{\pgfqpoint{2.245921in}{1.735876in}}%
\pgfpathlineto{\pgfqpoint{2.227847in}{1.731481in}}%
\pgfpathlineto{\pgfqpoint{2.209537in}{1.664930in}}%
\pgfpathlineto{\pgfqpoint{2.187943in}{1.581168in}}%
\pgfpathlineto{\pgfqpoint{2.166113in}{1.462202in}}%
\pgfpathlineto{\pgfqpoint{2.149448in}{1.409098in}}%
\pgfpathlineto{\pgfqpoint{2.129026in}{1.365876in}}%
\pgfpathlineto{\pgfqpoint{2.110013in}{1.351877in}}%
\pgfpathlineto{\pgfqpoint{2.091468in}{1.373872in}}%
\pgfpathlineto{\pgfqpoint{2.072691in}{1.418762in}}%
\pgfpathlineto{\pgfqpoint{2.054147in}{1.498759in}}%
\pgfpathlineto{\pgfqpoint{2.033022in}{1.628101in}}%
\pgfpathlineto{\pgfqpoint{2.014712in}{1.718070in}}%
\pgfpathlineto{\pgfqpoint{1.996873in}{1.741781in}}%
\pgfpathlineto{\pgfqpoint{1.975982in}{1.711256in}}%
\pgfpathlineto{\pgfqpoint{1.956969in}{1.656287in}}%
\pgfpathlineto{\pgfqpoint{1.937487in}{1.538964in}}%
\pgfpathlineto{\pgfqpoint{1.919648in}{1.472181in}}%
\pgfpathlineto{\pgfqpoint{1.899929in}{1.407479in}}%
\pgfpathlineto{\pgfqpoint{1.877630in}{1.367207in}}%
\pgfpathlineto{\pgfqpoint{1.863311in}{1.356801in}}%
\pgfpathlineto{\pgfqpoint{1.841012in}{1.378763in}}%
\pgfpathlineto{\pgfqpoint{1.824113in}{1.415971in}}%
\pgfpathlineto{\pgfqpoint{1.802751in}{1.497906in}}%
\pgfpathlineto{\pgfqpoint{1.782095in}{1.619086in}}%
\pgfpathlineto{\pgfqpoint{1.764490in}{1.703174in}}%
\pgfpathlineto{\pgfqpoint{1.745948in}{1.741161in}}%
\pgfpathlineto{\pgfqpoint{1.728109in}{1.741431in}}%
\pgfpathlineto{\pgfqpoint{1.705104in}{1.694251in}}%
\pgfpathlineto{\pgfqpoint{1.689848in}{1.643368in}}%
\pgfpathlineto{\pgfqpoint{1.666140in}{1.509749in}}%
\pgfpathlineto{\pgfqpoint{1.650178in}{1.462846in}}%
\pgfpathlineto{\pgfqpoint{1.630931in}{1.413545in}}%
\pgfpathlineto{\pgfqpoint{1.609335in}{1.367274in}}%
\pgfpathlineto{\pgfqpoint{1.591730in}{1.358904in}}%
\pgfpathlineto{\pgfqpoint{1.572951in}{1.378258in}}%
\pgfpathlineto{\pgfqpoint{1.555112in}{1.411880in}}%
\pgfpathlineto{\pgfqpoint{1.534690in}{1.486032in}}%
\pgfpathlineto{\pgfqpoint{1.515913in}{1.541148in}}%
\pgfpathlineto{\pgfqpoint{1.496195in}{1.582261in}}%
\pgfpathlineto{\pgfqpoint{1.475304in}{1.698929in}}%
\pgfpathlineto{\pgfqpoint{1.455353in}{1.747670in}}%
\pgfpathlineto{\pgfqpoint{1.437043in}{1.747349in}}%
\pgfpathlineto{\pgfqpoint{1.416856in}{1.710836in}}%
\pgfpathlineto{\pgfqpoint{1.400896in}{1.629975in}}%
\pgfpathlineto{\pgfqpoint{1.382586in}{1.559560in}}%
\pgfpathlineto{\pgfqpoint{1.360053in}{1.466411in}}%
\pgfpathlineto{\pgfqpoint{1.342213in}{1.408867in}}%
\pgfpathlineto{\pgfqpoint{1.323435in}{1.375469in}}%
\pgfpathlineto{\pgfqpoint{1.304421in}{1.360512in}}%
\pgfpathlineto{\pgfqpoint{1.283531in}{1.390283in}}%
\pgfpathlineto{\pgfqpoint{1.267569in}{1.387221in}}%
\pgfpathlineto{\pgfqpoint{1.245973in}{1.390635in}}%
\pgfpathlineto{\pgfqpoint{1.224145in}{1.461801in}}%
\pgfpathlineto{\pgfqpoint{1.205835in}{1.548993in}}%
\pgfpathlineto{\pgfqpoint{1.187761in}{1.438051in}}%
\pgfpathlineto{\pgfqpoint{1.167339in}{1.512525in}}%
\pgfpathlineto{\pgfqpoint{1.149500in}{1.596712in}}%
\pgfpathlineto{\pgfqpoint{1.130956in}{1.707451in}}%
\pgfpathlineto{\pgfqpoint{1.111005in}{1.755983in}}%
\pgfpathlineto{\pgfqpoint{1.091992in}{1.753161in}}%
\pgfpathlineto{\pgfqpoint{1.072978in}{1.708421in}}%
\pgfpathlineto{\pgfqpoint{1.052557in}{1.599574in}}%
\pgfpathlineto{\pgfqpoint{1.034249in}{1.509029in}}%
\pgfpathlineto{\pgfqpoint{1.012887in}{1.434085in}}%
\pgfpathlineto{\pgfqpoint{0.996925in}{1.396771in}}%
\pgfpathlineto{\pgfqpoint{0.974626in}{1.368304in}}%
\pgfpathlineto{\pgfqpoint{0.956318in}{1.375816in}}%
\pgfpathlineto{\pgfqpoint{0.938479in}{1.415911in}}%
\pgfpathlineto{\pgfqpoint{0.920403in}{1.479790in}}%
\pgfpathlineto{\pgfqpoint{0.899513in}{1.571715in}}%
\pgfpathlineto{\pgfqpoint{0.881674in}{1.664525in}}%
\pgfpathlineto{\pgfqpoint{0.859843in}{1.747699in}}%
\pgfpathlineto{\pgfqpoint{0.843178in}{1.766333in}}%
\pgfpathlineto{\pgfqpoint{0.822053in}{1.744832in}}%
\pgfpathlineto{\pgfqpoint{0.803978in}{1.699546in}}%
\pgfpathlineto{\pgfqpoint{0.787078in}{1.635306in}}%
\pgfpathlineto{\pgfqpoint{0.766656in}{1.547249in}}%
\pgfpathlineto{\pgfqpoint{0.746469in}{1.474817in}}%
\pgfpathlineto{\pgfqpoint{0.729569in}{1.420767in}}%
\pgfpathlineto{\pgfqpoint{0.708679in}{1.376728in}}%
\pgfpathlineto{\pgfqpoint{0.691074in}{1.371447in}}%
\pgfpathlineto{\pgfqpoint{0.669478in}{1.402284in}}%
\pgfpathlineto{\pgfqpoint{0.652579in}{1.447899in}}%
\pgfpathlineto{\pgfqpoint{0.655865in}{1.423102in}}%
\pgfpathlineto{\pgfqpoint{0.674878in}{1.722320in}}%
\pgfpathlineto{\pgfqpoint{0.696003in}{1.767184in}}%
\pgfpathlineto{\pgfqpoint{0.714782in}{1.719953in}}%
\pgfpathlineto{\pgfqpoint{0.733090in}{1.579540in}}%
\pgfpathlineto{\pgfqpoint{0.752572in}{1.456497in}}%
\pgfpathlineto{\pgfqpoint{0.772056in}{1.382793in}}%
\pgfpathlineto{\pgfqpoint{0.790129in}{1.372957in}}%
\pgfpathlineto{\pgfqpoint{0.811020in}{1.430785in}}%
\pgfpathlineto{\pgfqpoint{0.830973in}{1.545749in}}%
\pgfpathlineto{\pgfqpoint{0.849750in}{1.687001in}}%
\pgfpathlineto{\pgfqpoint{0.869234in}{1.757408in}}%
\pgfpathlineto{\pgfqpoint{0.888482in}{1.745941in}}%
\pgfpathlineto{\pgfqpoint{0.905852in}{1.644430in}}%
\pgfpathlineto{\pgfqpoint{0.926272in}{1.499417in}}%
\pgfpathlineto{\pgfqpoint{0.945519in}{1.409365in}}%
\pgfpathlineto{\pgfqpoint{0.964533in}{1.363055in}}%
\pgfpathlineto{\pgfqpoint{0.983546in}{1.388502in}}%
\pgfpathlineto{\pgfqpoint{1.003030in}{1.456379in}}%
\pgfpathlineto{\pgfqpoint{1.021807in}{1.564063in}}%
\pgfpathlineto{\pgfqpoint{1.039882in}{1.696932in}}%
\pgfpathlineto{\pgfqpoint{1.059365in}{1.753278in}}%
\pgfpathlineto{\pgfqpoint{1.081429in}{1.714581in}}%
\pgfpathlineto{\pgfqpoint{1.100442in}{1.752911in}}%
\pgfpathlineto{\pgfqpoint{1.118047in}{1.711924in}}%
\pgfpathlineto{\pgfqpoint{1.155368in}{1.455691in}}%
\pgfpathlineto{\pgfqpoint{1.177199in}{1.377745in}}%
\pgfpathlineto{\pgfqpoint{1.192924in}{1.357235in}}%
\pgfpathlineto{\pgfqpoint{1.214520in}{1.396359in}}%
\pgfpathlineto{\pgfqpoint{1.233768in}{1.465298in}}%
\pgfpathlineto{\pgfqpoint{1.252781in}{1.568728in}}%
\pgfpathlineto{\pgfqpoint{1.271560in}{1.695692in}}%
\pgfpathlineto{\pgfqpoint{1.289633in}{1.747144in}}%
\pgfpathlineto{\pgfqpoint{1.311698in}{1.717198in}}%
\pgfpathlineto{\pgfqpoint{1.330477in}{1.600285in}}%
\pgfpathlineto{\pgfqpoint{1.349490in}{1.474472in}}%
\pgfpathlineto{\pgfqpoint{1.368738in}{1.397647in}}%
\pgfpathlineto{\pgfqpoint{1.388454in}{1.362418in}}%
\pgfpathlineto{\pgfqpoint{1.406059in}{1.362968in}}%
\pgfpathlineto{\pgfqpoint{1.425541in}{1.406394in}}%
\pgfpathlineto{\pgfqpoint{1.442912in}{1.463490in}}%
\pgfpathlineto{\pgfqpoint{1.465681in}{1.585845in}}%
\pgfpathlineto{\pgfqpoint{1.483286in}{1.697338in}}%
\pgfpathlineto{\pgfqpoint{1.501829in}{1.743555in}}%
\pgfpathlineto{\pgfqpoint{1.523659in}{1.728641in}}%
\pgfpathlineto{\pgfqpoint{1.543141in}{1.637383in}}%
\pgfpathlineto{\pgfqpoint{1.560746in}{1.502248in}}%
\pgfpathlineto{\pgfqpoint{1.579054in}{1.416839in}}%
\pgfpathlineto{\pgfqpoint{1.600884in}{1.364711in}}%
\pgfpathlineto{\pgfqpoint{1.618960in}{1.356100in}}%
\pgfpathlineto{\pgfqpoint{1.634685in}{1.371416in}}%
\pgfpathlineto{\pgfqpoint{1.657924in}{1.435273in}}%
\pgfpathlineto{\pgfqpoint{1.673886in}{1.507355in}}%
\pgfpathlineto{\pgfqpoint{1.695011in}{1.643942in}}%
\pgfpathlineto{\pgfqpoint{1.713790in}{1.727425in}}%
\pgfpathlineto{\pgfqpoint{1.737028in}{1.739335in}}%
\pgfpathlineto{\pgfqpoint{1.753928in}{1.701853in}}%
\pgfpathlineto{\pgfqpoint{1.771064in}{1.595115in}}%
\pgfpathlineto{\pgfqpoint{1.788903in}{1.545574in}}%
\pgfpathlineto{\pgfqpoint{1.809794in}{1.457308in}}%
\pgfpathlineto{\pgfqpoint{1.827867in}{1.406675in}}%
\pgfpathlineto{\pgfqpoint{1.846177in}{1.361808in}}%
\pgfpathlineto{\pgfqpoint{1.868242in}{1.355890in}}%
\pgfpathlineto{\pgfqpoint{1.884907in}{1.383074in}}%
\pgfpathlineto{\pgfqpoint{1.905094in}{1.435438in}}%
\pgfpathlineto{\pgfqpoint{1.923637in}{1.491018in}}%
\pgfpathlineto{\pgfqpoint{1.944998in}{1.625316in}}%
\pgfpathlineto{\pgfqpoint{1.961898in}{1.665500in}}%
\pgfpathlineto{\pgfqpoint{1.983025in}{1.736561in}}%
\pgfpathlineto{\pgfqpoint{2.001098in}{1.726282in}}%
\pgfpathlineto{\pgfqpoint{2.021520in}{1.631503in}}%
\pgfpathlineto{\pgfqpoint{2.039359in}{1.525247in}}%
\pgfpathlineto{\pgfqpoint{2.058138in}{1.433640in}}%
\pgfpathlineto{\pgfqpoint{2.078794in}{1.374651in}}%
\pgfpathlineto{\pgfqpoint{2.097571in}{1.351398in}}%
\pgfpathlineto{\pgfqpoint{2.117993in}{1.716236in}}%
\pgfpathlineto{\pgfqpoint{2.137240in}{1.636746in}}%
\pgfpathlineto{\pgfqpoint{2.156724in}{1.498125in}}%
\pgfpathlineto{\pgfqpoint{2.174564in}{1.414386in}}%
\pgfpathlineto{\pgfqpoint{2.195689in}{1.363369in}}%
\pgfpathlineto{\pgfqpoint{2.214233in}{1.354974in}}%
\pgfpathlineto{\pgfqpoint{2.230898in}{1.390960in}}%
\pgfpathlineto{\pgfqpoint{2.252023in}{1.464162in}}%
\pgfpathlineto{\pgfqpoint{2.294275in}{1.713168in}}%
\pgfpathlineto{\pgfqpoint{2.309766in}{1.738911in}}%
\pgfpathlineto{\pgfqpoint{2.328780in}{1.709612in}}%
\pgfpathlineto{\pgfqpoint{2.349672in}{1.645237in}}%
\pgfpathlineto{\pgfqpoint{2.368215in}{1.522676in}}%
\pgfpathlineto{\pgfqpoint{2.388871in}{1.434629in}}%
\pgfpathlineto{\pgfqpoint{2.407415in}{1.375046in}}%
\pgfpathlineto{\pgfqpoint{2.423377in}{1.351533in}}%
\pgfpathlineto{\pgfqpoint{2.444502in}{1.369034in}}%
\pgfpathlineto{\pgfqpoint{2.462107in}{1.420643in}}%
\pgfpathlineto{\pgfqpoint{2.481120in}{1.494254in}}%
\pgfpathlineto{\pgfqpoint{2.504359in}{1.644920in}}%
\pgfpathlineto{\pgfqpoint{2.522901in}{1.723852in}}%
\pgfpathlineto{\pgfqpoint{2.521964in}{1.736117in}}%
\pgfpathlineto{\pgfqpoint{2.539566in}{1.738109in}}%
\pgfpathlineto{\pgfqpoint{2.560459in}{1.679095in}}%
\pgfpathlineto{\pgfqpoint{2.578533in}{1.557031in}}%
\pgfpathlineto{\pgfqpoint{2.596137in}{1.511581in}}%
\pgfpathlineto{\pgfqpoint{2.617262in}{1.414760in}}%
\pgfpathlineto{\pgfqpoint{2.634633in}{1.370837in}}%
\pgfpathlineto{\pgfqpoint{2.652941in}{1.350821in}}%
\pgfpathlineto{\pgfqpoint{2.673833in}{1.382846in}}%
\pgfpathlineto{\pgfqpoint{2.692610in}{1.435495in}}%
\pgfpathlineto{\pgfqpoint{2.713032in}{1.528876in}}%
\pgfpathlineto{\pgfqpoint{2.731576in}{1.637793in}}%
\pgfpathlineto{\pgfqpoint{2.749416in}{1.716335in}}%
\pgfpathlineto{\pgfqpoint{2.769603in}{1.736656in}}%
\pgfpathlineto{\pgfqpoint{2.791431in}{1.670757in}}%
\pgfpathlineto{\pgfqpoint{2.809036in}{1.572135in}}%
\pgfpathlineto{\pgfqpoint{2.827111in}{1.458845in}}%
\pgfpathlineto{\pgfqpoint{2.847533in}{1.395004in}}%
\pgfpathlineto{\pgfqpoint{2.865607in}{1.365437in}}%
\pgfpathlineto{\pgfqpoint{2.884854in}{1.353960in}}%
\pgfpathlineto{\pgfqpoint{2.904805in}{1.389284in}}%
\pgfpathlineto{\pgfqpoint{2.923115in}{1.438995in}}%
\pgfpathlineto{\pgfqpoint{2.943537in}{1.520518in}}%
\pgfpathlineto{\pgfqpoint{2.964428in}{1.609796in}}%
\pgfpathlineto{\pgfqpoint{2.979684in}{1.698320in}}%
\pgfpathlineto{\pgfqpoint{3.001046in}{1.740894in}}%
\pgfpathlineto{\pgfqpoint{3.021702in}{1.721085in}}%
\pgfpathlineto{\pgfqpoint{3.039776in}{1.636727in}}%
\pgfpathlineto{\pgfqpoint{3.057615in}{1.526164in}}%
\pgfpathlineto{\pgfqpoint{3.078271in}{1.475859in}}%
\pgfpathlineto{\pgfqpoint{3.097519in}{1.407000in}}%
\pgfpathlineto{\pgfqpoint{3.117237in}{1.365195in}}%
\pgfpathlineto{\pgfqpoint{3.136014in}{1.357109in}}%
\pgfpathlineto{\pgfqpoint{3.153853in}{1.389808in}}%
\pgfpathlineto{\pgfqpoint{3.174980in}{1.452655in}}%
\pgfpathlineto{\pgfqpoint{3.193759in}{1.491508in}}%
\pgfpathlineto{\pgfqpoint{3.211362in}{1.594291in}}%
\pgfpathlineto{\pgfqpoint{3.232018in}{1.704982in}}%
\pgfpathlineto{\pgfqpoint{3.250328in}{1.740896in}}%
\pgfpathlineto{\pgfqpoint{3.270279in}{1.730211in}}%
\pgfpathlineto{\pgfqpoint{3.286712in}{1.655387in}}%
\pgfpathlineto{\pgfqpoint{3.308776in}{1.568823in}}%
\pgfpathlineto{\pgfqpoint{3.324973in}{1.492383in}}%
\pgfpathlineto{\pgfqpoint{3.345629in}{1.412422in}}%
\pgfpathlineto{\pgfqpoint{3.366988in}{1.366557in}}%
\pgfpathlineto{\pgfqpoint{3.386705in}{1.357043in}}%
\pgfpathlineto{\pgfqpoint{3.403137in}{1.386318in}}%
\pgfpathlineto{\pgfqpoint{3.424262in}{1.430960in}}%
\pgfpathlineto{\pgfqpoint{3.441162in}{1.482936in}}%
\pgfpathlineto{\pgfqpoint{3.461818in}{1.568924in}}%
\pgfpathlineto{\pgfqpoint{3.481068in}{1.674702in}}%
\pgfpathlineto{\pgfqpoint{3.498907in}{1.734052in}}%
\pgfpathlineto{\pgfqpoint{3.519329in}{1.749731in}}%
\pgfpathlineto{\pgfqpoint{3.538576in}{1.726438in}}%
\pgfpathlineto{\pgfqpoint{3.597023in}{1.465559in}}%
\pgfpathlineto{\pgfqpoint{3.615567in}{1.747054in}}%
\pgfpathlineto{\pgfqpoint{3.634346in}{1.745251in}}%
\pgfpathlineto{\pgfqpoint{3.652888in}{1.689739in}}%
\pgfpathlineto{\pgfqpoint{3.674250in}{1.552981in}}%
\pgfpathlineto{\pgfqpoint{3.694906in}{1.463067in}}%
\pgfpathlineto{\pgfqpoint{3.712040in}{1.406259in}}%
\pgfpathlineto{\pgfqpoint{3.729879in}{1.366177in}}%
\pgfpathlineto{\pgfqpoint{3.749832in}{1.369754in}}%
\pgfpathlineto{\pgfqpoint{3.768140in}{1.414049in}}%
\pgfpathlineto{\pgfqpoint{3.787153in}{1.467051in}}%
\pgfpathlineto{\pgfqpoint{3.807575in}{1.564983in}}%
\pgfpathlineto{\pgfqpoint{3.825180in}{1.668584in}}%
\pgfpathlineto{\pgfqpoint{3.846776in}{1.738433in}}%
\pgfpathlineto{\pgfqpoint{3.864849in}{1.756936in}}%
\pgfpathlineto{\pgfqpoint{3.885271in}{1.723436in}}%
\pgfpathlineto{\pgfqpoint{3.904284in}{1.644607in}}%
\pgfpathlineto{\pgfqpoint{3.942779in}{1.462247in}}%
\pgfpathlineto{\pgfqpoint{3.962967in}{1.400795in}}%
\pgfpathlineto{\pgfqpoint{3.980806in}{1.371230in}}%
\pgfpathlineto{\pgfqpoint{4.000054in}{1.365598in}}%
\pgfpathlineto{\pgfqpoint{4.016953in}{1.392387in}}%
\pgfpathlineto{\pgfqpoint{4.058266in}{1.513939in}}%
\pgfpathlineto{\pgfqpoint{4.081270in}{1.641916in}}%
\pgfpathlineto{\pgfqpoint{4.098640in}{1.719048in}}%
\pgfpathlineto{\pgfqpoint{4.114602in}{1.758622in}}%
\pgfpathlineto{\pgfqpoint{4.135493in}{1.710596in}}%
\pgfpathlineto{\pgfqpoint{4.154506in}{1.732117in}}%
\pgfpathlineto{\pgfqpoint{4.174691in}{1.764861in}}%
\pgfpathlineto{\pgfqpoint{4.195347in}{1.738658in}}%
\pgfpathlineto{\pgfqpoint{4.211075in}{1.684213in}}%
\pgfpathlineto{\pgfqpoint{4.228914in}{1.575876in}}%
\pgfpathlineto{\pgfqpoint{4.249805in}{1.484101in}}%
\pgfpathlineto{\pgfqpoint{4.267644in}{1.423675in}}%
\pgfpathlineto{\pgfqpoint{4.292057in}{1.373985in}}%
\pgfpathlineto{\pgfqpoint{4.307313in}{1.370559in}}%
\pgfpathlineto{\pgfqpoint{4.326797in}{1.412710in}}%
\pgfpathlineto{\pgfqpoint{4.343228in}{1.467083in}}%
\pgfpathlineto{\pgfqpoint{4.383835in}{1.641553in}}%
\pgfpathlineto{\pgfqpoint{4.401440in}{1.733864in}}%
\pgfpathlineto{\pgfqpoint{4.421393in}{1.769087in}}%
\pgfpathlineto{\pgfqpoint{4.444632in}{1.760931in}}%
\pgfpathlineto{\pgfqpoint{4.462236in}{1.701833in}}%
\pgfpathlineto{\pgfqpoint{4.481013in}{1.629748in}}%
\pgfpathlineto{\pgfqpoint{4.482422in}{1.635679in}}%
\pgfpathlineto{\pgfqpoint{4.473268in}{1.694530in}}%
\pgfpathlineto{\pgfqpoint{4.455897in}{1.763841in}}%
\pgfpathlineto{\pgfqpoint{4.436884in}{1.759788in}}%
\pgfpathlineto{\pgfqpoint{4.419279in}{1.641074in}}%
\pgfpathlineto{\pgfqpoint{4.396511in}{1.513902in}}%
\pgfpathlineto{\pgfqpoint{4.376558in}{1.417607in}}%
\pgfpathlineto{\pgfqpoint{4.360362in}{1.375373in}}%
\pgfpathlineto{\pgfqpoint{4.340177in}{1.393345in}}%
\pgfpathlineto{\pgfqpoint{4.321867in}{1.458046in}}%
\pgfpathlineto{\pgfqpoint{4.282197in}{1.726486in}}%
\pgfpathlineto{\pgfqpoint{4.261307in}{1.764726in}}%
\pgfpathlineto{\pgfqpoint{4.242999in}{1.721493in}}%
\pgfpathlineto{\pgfqpoint{4.223280in}{1.572188in}}%
\pgfpathlineto{\pgfqpoint{4.204972in}{1.467014in}}%
\pgfpathlineto{\pgfqpoint{4.188307in}{1.398895in}}%
\pgfpathlineto{\pgfqpoint{4.167180in}{1.363090in}}%
\pgfpathlineto{\pgfqpoint{4.148872in}{1.410367in}}%
\pgfpathlineto{\pgfqpoint{4.127042in}{1.504013in}}%
\pgfpathlineto{\pgfqpoint{4.109906in}{1.632580in}}%
\pgfpathlineto{\pgfqpoint{4.088547in}{1.742920in}}%
\pgfpathlineto{\pgfqpoint{4.070473in}{1.754687in}}%
\pgfpathlineto{\pgfqpoint{4.052163in}{1.691721in}}%
\pgfpathlineto{\pgfqpoint{4.030333in}{1.520805in}}%
\pgfpathlineto{\pgfqpoint{4.014607in}{1.431138in}}%
\pgfpathlineto{\pgfqpoint{3.992308in}{1.369665in}}%
\pgfpathlineto{\pgfqpoint{3.973764in}{1.369278in}}%
\pgfpathlineto{\pgfqpoint{3.955924in}{1.423493in}}%
\pgfpathlineto{\pgfqpoint{3.933860in}{1.527481in}}%
\pgfpathlineto{\pgfqpoint{3.916021in}{1.664110in}}%
\pgfpathlineto{\pgfqpoint{3.895833in}{1.747241in}}%
\pgfpathlineto{\pgfqpoint{3.882220in}{1.743163in}}%
\pgfpathlineto{\pgfqpoint{3.860155in}{1.664404in}}%
\pgfpathlineto{\pgfqpoint{3.839733in}{1.513618in}}%
\pgfpathlineto{\pgfqpoint{3.821660in}{1.424235in}}%
\pgfpathlineto{\pgfqpoint{3.801941in}{1.368926in}}%
\pgfpathlineto{\pgfqpoint{3.782928in}{1.359750in}}%
\pgfpathlineto{\pgfqpoint{3.762977in}{1.412345in}}%
\pgfpathlineto{\pgfqpoint{3.745607in}{1.493737in}}%
\pgfpathlineto{\pgfqpoint{3.725185in}{1.632325in}}%
\pgfpathlineto{\pgfqpoint{3.703591in}{1.708701in}}%
\pgfpathlineto{\pgfqpoint{3.686221in}{1.747631in}}%
\pgfpathlineto{\pgfqpoint{3.667676in}{1.713867in}}%
\pgfpathlineto{\pgfqpoint{3.646786in}{1.607229in}}%
\pgfpathlineto{\pgfqpoint{3.629650in}{1.490936in}}%
\pgfpathlineto{\pgfqpoint{3.612984in}{1.416067in}}%
\pgfpathlineto{\pgfqpoint{3.588808in}{1.362944in}}%
\pgfpathlineto{\pgfqpoint{3.570733in}{1.359951in}}%
\pgfpathlineto{\pgfqpoint{3.551721in}{1.402413in}}%
\pgfpathlineto{\pgfqpoint{3.533411in}{1.470682in}}%
\pgfpathlineto{\pgfqpoint{3.512755in}{1.597131in}}%
\pgfpathlineto{\pgfqpoint{3.495150in}{1.672788in}}%
\pgfpathlineto{\pgfqpoint{3.473791in}{1.742045in}}%
\pgfpathlineto{\pgfqpoint{3.457595in}{1.738832in}}%
\pgfpathlineto{\pgfqpoint{3.436233in}{1.660897in}}%
\pgfpathlineto{\pgfqpoint{3.417691in}{1.543658in}}%
\pgfpathlineto{\pgfqpoint{3.398912in}{1.449444in}}%
\pgfpathlineto{\pgfqpoint{3.379664in}{1.392796in}}%
\pgfpathlineto{\pgfqpoint{3.358303in}{1.353529in}}%
\pgfpathlineto{\pgfqpoint{3.340933in}{1.678020in}}%
\pgfpathlineto{\pgfqpoint{3.323093in}{1.552917in}}%
\pgfpathlineto{\pgfqpoint{3.301263in}{1.436136in}}%
\pgfpathlineto{\pgfqpoint{3.281547in}{1.377607in}}%
\pgfpathlineto{\pgfqpoint{3.263004in}{1.354286in}}%
\pgfpathlineto{\pgfqpoint{3.244694in}{1.370746in}}%
\pgfpathlineto{\pgfqpoint{3.224743in}{1.434999in}}%
\pgfpathlineto{\pgfqpoint{3.207607in}{1.516610in}}%
\pgfpathlineto{\pgfqpoint{3.188594in}{1.656091in}}%
\pgfpathlineto{\pgfqpoint{3.170050in}{1.734072in}}%
\pgfpathlineto{\pgfqpoint{3.147985in}{1.722879in}}%
\pgfpathlineto{\pgfqpoint{3.128974in}{1.641123in}}%
\pgfpathlineto{\pgfqpoint{3.110898in}{1.519919in}}%
\pgfpathlineto{\pgfqpoint{3.091650in}{1.441061in}}%
\pgfpathlineto{\pgfqpoint{3.070760in}{1.373455in}}%
\pgfpathlineto{\pgfqpoint{3.052215in}{1.352662in}}%
\pgfpathlineto{\pgfqpoint{3.030622in}{1.383527in}}%
\pgfpathlineto{\pgfqpoint{3.015599in}{1.426824in}}%
\pgfpathlineto{\pgfqpoint{2.993064in}{1.539702in}}%
\pgfpathlineto{\pgfqpoint{2.976633in}{1.659321in}}%
\pgfpathlineto{\pgfqpoint{2.957385in}{1.730732in}}%
\pgfpathlineto{\pgfqpoint{2.935790in}{1.727150in}}%
\pgfpathlineto{\pgfqpoint{2.917716in}{1.653809in}}%
\pgfpathlineto{\pgfqpoint{2.896825in}{1.519665in}}%
\pgfpathlineto{\pgfqpoint{2.859973in}{1.428062in}}%
\pgfpathlineto{\pgfqpoint{2.841194in}{1.374940in}}%
\pgfpathlineto{\pgfqpoint{2.823120in}{1.350123in}}%
\pgfpathlineto{\pgfqpoint{2.803638in}{1.377192in}}%
\pgfpathlineto{\pgfqpoint{2.785799in}{1.426925in}}%
\pgfpathlineto{\pgfqpoint{2.764203in}{1.540989in}}%
\pgfpathlineto{\pgfqpoint{2.745190in}{1.672866in}}%
\pgfpathlineto{\pgfqpoint{2.726411in}{1.734636in}}%
\pgfpathlineto{\pgfqpoint{2.706929in}{1.606592in}}%
\pgfpathlineto{\pgfqpoint{2.689090in}{1.716702in}}%
\pgfpathlineto{\pgfqpoint{2.668199in}{1.738672in}}%
\pgfpathlineto{\pgfqpoint{2.649186in}{1.709588in}}%
\pgfpathlineto{\pgfqpoint{2.610691in}{1.471961in}}%
\pgfpathlineto{\pgfqpoint{2.591443in}{1.399922in}}%
\pgfpathlineto{\pgfqpoint{2.570787in}{1.355722in}}%
\pgfpathlineto{\pgfqpoint{2.550834in}{1.359272in}}%
\pgfpathlineto{\pgfqpoint{2.532055in}{1.402308in}}%
\pgfpathlineto{\pgfqpoint{2.513747in}{1.473753in}}%
\pgfpathlineto{\pgfqpoint{2.495203in}{1.592332in}}%
\pgfpathlineto{\pgfqpoint{2.475955in}{1.705119in}}%
\pgfpathlineto{\pgfqpoint{2.455770in}{1.737927in}}%
\pgfpathlineto{\pgfqpoint{2.436522in}{1.700857in}}%
\pgfpathlineto{\pgfqpoint{2.419152in}{1.598921in}}%
\pgfpathlineto{\pgfqpoint{2.399433in}{1.496744in}}%
\pgfpathlineto{\pgfqpoint{2.377134in}{1.411290in}}%
\pgfpathlineto{\pgfqpoint{2.359061in}{1.366861in}}%
\pgfpathlineto{\pgfqpoint{2.341456in}{1.350539in}}%
\pgfpathlineto{\pgfqpoint{2.322442in}{1.373401in}}%
\pgfpathlineto{\pgfqpoint{2.303429in}{1.421128in}}%
\pgfpathlineto{\pgfqpoint{2.278782in}{1.541362in}}%
\pgfpathlineto{\pgfqpoint{2.263057in}{1.641403in}}%
\pgfpathlineto{\pgfqpoint{2.247798in}{1.715516in}}%
\pgfpathlineto{\pgfqpoint{2.225970in}{1.737577in}}%
\pgfpathlineto{\pgfqpoint{2.206956in}{1.698640in}}%
\pgfpathlineto{\pgfqpoint{2.185829in}{1.605275in}}%
\pgfpathlineto{\pgfqpoint{2.170104in}{1.509048in}}%
\pgfpathlineto{\pgfqpoint{2.151794in}{1.436168in}}%
\pgfpathlineto{\pgfqpoint{2.130200in}{1.391679in}}%
\pgfpathlineto{\pgfqpoint{2.111890in}{1.357207in}}%
\pgfpathlineto{\pgfqpoint{2.090531in}{1.367908in}}%
\pgfpathlineto{\pgfqpoint{2.071752in}{1.404092in}}%
\pgfpathlineto{\pgfqpoint{2.053207in}{1.474056in}}%
\pgfpathlineto{\pgfqpoint{2.034431in}{1.584047in}}%
\pgfpathlineto{\pgfqpoint{2.017060in}{1.692434in}}%
\pgfpathlineto{\pgfqpoint{1.992882in}{1.741308in}}%
\pgfpathlineto{\pgfqpoint{1.977625in}{1.714762in}}%
\pgfpathlineto{\pgfqpoint{1.957438in}{1.730585in}}%
\pgfpathlineto{\pgfqpoint{1.936313in}{1.735893in}}%
\pgfpathlineto{\pgfqpoint{1.920117in}{1.692662in}}%
\pgfpathlineto{\pgfqpoint{1.899460in}{1.570213in}}%
\pgfpathlineto{\pgfqpoint{1.881387in}{1.471074in}}%
\pgfpathlineto{\pgfqpoint{1.860260in}{1.398007in}}%
\pgfpathlineto{\pgfqpoint{1.840309in}{1.364058in}}%
\pgfpathlineto{\pgfqpoint{1.821530in}{1.356948in}}%
\pgfpathlineto{\pgfqpoint{1.802517in}{1.385608in}}%
\pgfpathlineto{\pgfqpoint{1.784209in}{1.431654in}}%
\pgfpathlineto{\pgfqpoint{1.765664in}{1.501640in}}%
\pgfpathlineto{\pgfqpoint{1.743600in}{1.634598in}}%
\pgfpathlineto{\pgfqpoint{1.725292in}{1.696857in}}%
\pgfpathlineto{\pgfqpoint{1.706278in}{1.742551in}}%
\pgfpathlineto{\pgfqpoint{1.689142in}{1.743856in}}%
\pgfpathlineto{\pgfqpoint{1.668721in}{1.697521in}}%
\pgfpathlineto{\pgfqpoint{1.628817in}{1.486781in}}%
\pgfpathlineto{\pgfqpoint{1.611212in}{1.424665in}}%
\pgfpathlineto{\pgfqpoint{1.592904in}{1.384315in}}%
\pgfpathlineto{\pgfqpoint{1.571074in}{1.358025in}}%
\pgfpathlineto{\pgfqpoint{1.556286in}{1.365443in}}%
\pgfpathlineto{\pgfqpoint{1.531873in}{1.413162in}}%
\pgfpathlineto{\pgfqpoint{1.515913in}{1.464345in}}%
\pgfpathlineto{\pgfqpoint{1.497369in}{1.554325in}}%
\pgfpathlineto{\pgfqpoint{1.478590in}{1.665865in}}%
\pgfpathlineto{\pgfqpoint{1.458639in}{1.739034in}}%
\pgfpathlineto{\pgfqpoint{1.438217in}{1.750912in}}%
\pgfpathlineto{\pgfqpoint{1.416856in}{1.728373in}}%
\pgfpathlineto{\pgfqpoint{1.398314in}{1.676916in}}%
\pgfpathlineto{\pgfqpoint{1.379066in}{1.601756in}}%
\pgfpathlineto{\pgfqpoint{1.361695in}{1.522741in}}%
\pgfpathlineto{\pgfqpoint{1.336580in}{1.421305in}}%
\pgfpathlineto{\pgfqpoint{1.324138in}{1.391577in}}%
\pgfpathlineto{\pgfqpoint{1.302544in}{1.362954in}}%
\pgfpathlineto{\pgfqpoint{1.284705in}{1.370830in}}%
\pgfpathlineto{\pgfqpoint{1.265457in}{1.396044in}}%
\pgfpathlineto{\pgfqpoint{1.225317in}{1.526997in}}%
\pgfpathlineto{\pgfqpoint{1.207948in}{1.638181in}}%
\pgfpathlineto{\pgfqpoint{1.188701in}{1.722754in}}%
\pgfpathlineto{\pgfqpoint{1.169922in}{1.753233in}}%
\pgfpathlineto{\pgfqpoint{1.148561in}{1.756521in}}%
\pgfpathlineto{\pgfqpoint{1.133069in}{1.736911in}}%
\pgfpathlineto{\pgfqpoint{1.111943in}{1.694215in}}%
\pgfpathlineto{\pgfqpoint{1.093166in}{1.599610in}}%
\pgfpathlineto{\pgfqpoint{1.070867in}{1.499968in}}%
\pgfpathlineto{\pgfqpoint{1.052322in}{1.434989in}}%
\pgfpathlineto{\pgfqpoint{1.034014in}{1.396173in}}%
\pgfpathlineto{\pgfqpoint{1.015470in}{1.374369in}}%
\pgfpathlineto{\pgfqpoint{0.994579in}{1.367731in}}%
\pgfpathlineto{\pgfqpoint{0.975331in}{1.403450in}}%
\pgfpathlineto{\pgfqpoint{0.957256in}{1.454950in}}%
\pgfpathlineto{\pgfqpoint{0.938713in}{1.471170in}}%
\pgfpathlineto{\pgfqpoint{0.901390in}{1.672985in}}%
\pgfpathlineto{\pgfqpoint{0.882613in}{1.739341in}}%
\pgfpathlineto{\pgfqpoint{0.860549in}{1.765575in}}%
\pgfpathlineto{\pgfqpoint{0.842709in}{1.750081in}}%
\pgfpathlineto{\pgfqpoint{0.821348in}{1.686504in}}%
\pgfpathlineto{\pgfqpoint{0.802806in}{1.592956in}}%
\pgfpathlineto{\pgfqpoint{0.784730in}{1.511604in}}%
\pgfpathlineto{\pgfqpoint{0.768065in}{1.494140in}}%
\pgfpathlineto{\pgfqpoint{0.747878in}{1.424207in}}%
\pgfpathlineto{\pgfqpoint{0.727456in}{1.390793in}}%
\pgfpathlineto{\pgfqpoint{0.707036in}{1.370455in}}%
\pgfpathlineto{\pgfqpoint{0.686144in}{1.395782in}}%
\pgfpathlineto{\pgfqpoint{0.651170in}{1.493367in}}%
\pgfpathlineto{\pgfqpoint{0.652579in}{1.485101in}}%
\pgfpathlineto{\pgfqpoint{0.657742in}{1.461606in}}%
\pgfpathlineto{\pgfqpoint{0.675347in}{1.404951in}}%
\pgfpathlineto{\pgfqpoint{0.697646in}{1.502606in}}%
\pgfpathlineto{\pgfqpoint{0.715250in}{1.629307in}}%
\pgfpathlineto{\pgfqpoint{0.732621in}{1.731012in}}%
\pgfpathlineto{\pgfqpoint{0.751400in}{1.766961in}}%
\pgfpathlineto{\pgfqpoint{0.773699in}{1.702016in}}%
\pgfpathlineto{\pgfqpoint{0.791538in}{1.556851in}}%
\pgfpathlineto{\pgfqpoint{0.811020in}{1.447742in}}%
\pgfpathlineto{\pgfqpoint{0.829799in}{1.379461in}}%
\pgfpathlineto{\pgfqpoint{0.847169in}{1.370690in}}%
\pgfpathlineto{\pgfqpoint{0.870406in}{1.440649in}}%
\pgfpathlineto{\pgfqpoint{0.887307in}{1.688526in}}%
\pgfpathlineto{\pgfqpoint{0.906321in}{1.758158in}}%
\pgfpathlineto{\pgfqpoint{0.926272in}{1.735777in}}%
\pgfpathlineto{\pgfqpoint{0.945285in}{1.622502in}}%
\pgfpathlineto{\pgfqpoint{0.963595in}{1.480958in}}%
\pgfpathlineto{\pgfqpoint{0.987302in}{1.383188in}}%
\pgfpathlineto{\pgfqpoint{1.003733in}{1.361108in}}%
\pgfpathlineto{\pgfqpoint{1.021104in}{1.399808in}}%
\pgfpathlineto{\pgfqpoint{1.039882in}{1.474669in}}%
\pgfpathlineto{\pgfqpoint{1.061945in}{1.614743in}}%
\pgfpathlineto{\pgfqpoint{1.081664in}{1.726064in}}%
\pgfpathlineto{\pgfqpoint{1.098799in}{1.752776in}}%
\pgfpathlineto{\pgfqpoint{1.115465in}{1.710596in}}%
\pgfpathlineto{\pgfqpoint{1.139172in}{1.542124in}}%
\pgfpathlineto{\pgfqpoint{1.158889in}{1.434160in}}%
\pgfpathlineto{\pgfqpoint{1.176964in}{1.371348in}}%
\pgfpathlineto{\pgfqpoint{1.196446in}{1.363660in}}%
\pgfpathlineto{\pgfqpoint{1.215223in}{1.408978in}}%
\pgfpathlineto{\pgfqpoint{1.233768in}{1.485142in}}%
\pgfpathlineto{\pgfqpoint{1.250669in}{1.573960in}}%
\pgfpathlineto{\pgfqpoint{1.271794in}{1.716942in}}%
\pgfpathlineto{\pgfqpoint{1.290807in}{1.748724in}}%
\pgfpathlineto{\pgfqpoint{1.310758in}{1.721410in}}%
\pgfpathlineto{\pgfqpoint{1.331180in}{1.581925in}}%
\pgfpathlineto{\pgfqpoint{1.346908in}{1.488287in}}%
\pgfpathlineto{\pgfqpoint{1.368972in}{1.398738in}}%
\pgfpathlineto{\pgfqpoint{1.390332in}{1.356411in}}%
\pgfpathlineto{\pgfqpoint{1.405120in}{1.362482in}}%
\pgfpathlineto{\pgfqpoint{1.427889in}{1.427511in}}%
\pgfpathlineto{\pgfqpoint{1.444554in}{1.502541in}}%
\pgfpathlineto{\pgfqpoint{1.462159in}{1.618979in}}%
\pgfpathlineto{\pgfqpoint{1.482112in}{1.362216in}}%
\pgfpathlineto{\pgfqpoint{1.503003in}{1.364639in}}%
\pgfpathlineto{\pgfqpoint{1.522016in}{1.414824in}}%
\pgfpathlineto{\pgfqpoint{1.541029in}{1.497321in}}%
\pgfpathlineto{\pgfqpoint{1.560043in}{1.637311in}}%
\pgfpathlineto{\pgfqpoint{1.578585in}{1.725709in}}%
\pgfpathlineto{\pgfqpoint{1.596190in}{1.741629in}}%
\pgfpathlineto{\pgfqpoint{1.617786in}{1.666144in}}%
\pgfpathlineto{\pgfqpoint{1.636799in}{1.565934in}}%
\pgfpathlineto{\pgfqpoint{1.655107in}{1.448946in}}%
\pgfpathlineto{\pgfqpoint{1.674120in}{1.381353in}}%
\pgfpathlineto{\pgfqpoint{1.695481in}{1.352955in}}%
\pgfpathlineto{\pgfqpoint{1.713084in}{1.374686in}}%
\pgfpathlineto{\pgfqpoint{1.733977in}{1.449944in}}%
\pgfpathlineto{\pgfqpoint{1.752990in}{1.553851in}}%
\pgfpathlineto{\pgfqpoint{1.771064in}{1.678806in}}%
\pgfpathlineto{\pgfqpoint{1.791485in}{1.739263in}}%
\pgfpathlineto{\pgfqpoint{1.809325in}{1.729358in}}%
\pgfpathlineto{\pgfqpoint{1.830450in}{1.622440in}}%
\pgfpathlineto{\pgfqpoint{1.865190in}{1.421130in}}%
\pgfpathlineto{\pgfqpoint{1.887724in}{1.370199in}}%
\pgfpathlineto{\pgfqpoint{1.905798in}{1.352554in}}%
\pgfpathlineto{\pgfqpoint{1.925516in}{1.371507in}}%
\pgfpathlineto{\pgfqpoint{1.945232in}{1.429496in}}%
\pgfpathlineto{\pgfqpoint{1.963777in}{1.504666in}}%
\pgfpathlineto{\pgfqpoint{1.981145in}{1.629457in}}%
\pgfpathlineto{\pgfqpoint{1.999924in}{1.704126in}}%
\pgfpathlineto{\pgfqpoint{2.020111in}{1.739419in}}%
\pgfpathlineto{\pgfqpoint{2.042176in}{1.700077in}}%
\pgfpathlineto{\pgfqpoint{2.059781in}{1.739403in}}%
\pgfpathlineto{\pgfqpoint{2.077620in}{1.711669in}}%
\pgfpathlineto{\pgfqpoint{2.101562in}{1.564351in}}%
\pgfpathlineto{\pgfqpoint{2.113299in}{1.469193in}}%
\pgfpathlineto{\pgfqpoint{2.137711in}{1.393639in}}%
\pgfpathlineto{\pgfqpoint{2.155316in}{1.363228in}}%
\pgfpathlineto{\pgfqpoint{2.175972in}{1.357928in}}%
\pgfpathlineto{\pgfqpoint{2.193341in}{1.394456in}}%
\pgfpathlineto{\pgfqpoint{2.211885in}{1.456592in}}%
\pgfpathlineto{\pgfqpoint{2.232307in}{1.567897in}}%
\pgfpathlineto{\pgfqpoint{2.254137in}{1.693296in}}%
\pgfpathlineto{\pgfqpoint{2.269394in}{1.732710in}}%
\pgfpathlineto{\pgfqpoint{2.289347in}{1.723349in}}%
\pgfpathlineto{\pgfqpoint{2.309063in}{1.640054in}}%
\pgfpathlineto{\pgfqpoint{2.327137in}{1.514475in}}%
\pgfpathlineto{\pgfqpoint{2.346619in}{1.425201in}}%
\pgfpathlineto{\pgfqpoint{2.367277in}{1.369675in}}%
\pgfpathlineto{\pgfqpoint{2.384176in}{1.350277in}}%
\pgfpathlineto{\pgfqpoint{2.406241in}{1.375885in}}%
\pgfpathlineto{\pgfqpoint{2.426428in}{1.426845in}}%
\pgfpathlineto{\pgfqpoint{2.443562in}{1.499752in}}%
\pgfpathlineto{\pgfqpoint{2.463281in}{1.572163in}}%
\pgfpathlineto{\pgfqpoint{2.482058in}{1.633912in}}%
\pgfpathlineto{\pgfqpoint{2.499663in}{1.721845in}}%
\pgfpathlineto{\pgfqpoint{2.520789in}{1.737259in}}%
\pgfpathlineto{\pgfqpoint{2.539332in}{1.696289in}}%
\pgfpathlineto{\pgfqpoint{2.557408in}{1.590372in}}%
\pgfpathlineto{\pgfqpoint{2.577827in}{1.455067in}}%
\pgfpathlineto{\pgfqpoint{2.598954in}{1.387082in}}%
\pgfpathlineto{\pgfqpoint{2.616559in}{1.366526in}}%
\pgfpathlineto{\pgfqpoint{2.634867in}{1.351979in}}%
\pgfpathlineto{\pgfqpoint{2.655523in}{1.384016in}}%
\pgfpathlineto{\pgfqpoint{2.673597in}{1.425125in}}%
\pgfpathlineto{\pgfqpoint{2.692141in}{1.507594in}}%
\pgfpathlineto{\pgfqpoint{2.715146in}{1.632918in}}%
\pgfpathlineto{\pgfqpoint{2.732514in}{1.682654in}}%
\pgfpathlineto{\pgfqpoint{2.750119in}{1.723368in}}%
\pgfpathlineto{\pgfqpoint{2.770072in}{1.734701in}}%
\pgfpathlineto{\pgfqpoint{2.790728in}{1.657029in}}%
\pgfpathlineto{\pgfqpoint{2.808333in}{1.531853in}}%
\pgfpathlineto{\pgfqpoint{2.826406in}{1.445967in}}%
\pgfpathlineto{\pgfqpoint{2.847533in}{1.379383in}}%
\pgfpathlineto{\pgfqpoint{2.864433in}{1.355985in}}%
\pgfpathlineto{\pgfqpoint{2.883680in}{1.364269in}}%
\pgfpathlineto{\pgfqpoint{2.901754in}{1.411803in}}%
\pgfpathlineto{\pgfqpoint{2.923115in}{1.484911in}}%
\pgfpathlineto{\pgfqpoint{2.945180in}{1.602409in}}%
\pgfpathlineto{\pgfqpoint{2.964662in}{1.671259in}}%
\pgfpathlineto{\pgfqpoint{2.980390in}{1.722448in}}%
\pgfpathlineto{\pgfqpoint{3.002220in}{1.737880in}}%
\pgfpathlineto{\pgfqpoint{3.019823in}{1.681759in}}%
\pgfpathlineto{\pgfqpoint{3.043767in}{1.544448in}}%
\pgfpathlineto{\pgfqpoint{3.057380in}{1.474182in}}%
\pgfpathlineto{\pgfqpoint{3.078271in}{1.399168in}}%
\pgfpathlineto{\pgfqpoint{3.097519in}{1.368047in}}%
\pgfpathlineto{\pgfqpoint{3.115123in}{1.354725in}}%
\pgfpathlineto{\pgfqpoint{3.137422in}{1.390643in}}%
\pgfpathlineto{\pgfqpoint{3.152681in}{1.428699in}}%
\pgfpathlineto{\pgfqpoint{3.172398in}{1.486253in}}%
\pgfpathlineto{\pgfqpoint{3.192114in}{1.589747in}}%
\pgfpathlineto{\pgfqpoint{3.215118in}{1.703249in}}%
\pgfpathlineto{\pgfqpoint{3.230375in}{1.727692in}}%
\pgfpathlineto{\pgfqpoint{3.249623in}{1.744786in}}%
\pgfpathlineto{\pgfqpoint{3.270279in}{1.708341in}}%
\pgfpathlineto{\pgfqpoint{3.288354in}{1.625597in}}%
\pgfpathlineto{\pgfqpoint{3.309714in}{1.509466in}}%
\pgfpathlineto{\pgfqpoint{3.326850in}{1.440639in}}%
\pgfpathlineto{\pgfqpoint{3.347272in}{1.391374in}}%
\pgfpathlineto{\pgfqpoint{3.367222in}{1.358288in}}%
\pgfpathlineto{\pgfqpoint{3.386001in}{1.360857in}}%
\pgfpathlineto{\pgfqpoint{3.404075in}{1.384767in}}%
\pgfpathlineto{\pgfqpoint{3.424028in}{1.448909in}}%
\pgfpathlineto{\pgfqpoint{3.442101in}{1.519925in}}%
\pgfpathlineto{\pgfqpoint{3.460411in}{1.585270in}}%
\pgfpathlineto{\pgfqpoint{3.483179in}{1.695704in}}%
\pgfpathlineto{\pgfqpoint{3.500550in}{1.449145in}}%
\pgfpathlineto{\pgfqpoint{3.523788in}{1.582301in}}%
\pgfpathlineto{\pgfqpoint{3.558996in}{1.737203in}}%
\pgfpathlineto{\pgfqpoint{3.577072in}{1.750110in}}%
\pgfpathlineto{\pgfqpoint{3.594676in}{1.720050in}}%
\pgfpathlineto{\pgfqpoint{3.615801in}{1.623586in}}%
\pgfpathlineto{\pgfqpoint{3.632937in}{1.531160in}}%
\pgfpathlineto{\pgfqpoint{3.655002in}{1.445543in}}%
\pgfpathlineto{\pgfqpoint{3.672370in}{1.390387in}}%
\pgfpathlineto{\pgfqpoint{3.690680in}{1.361162in}}%
\pgfpathlineto{\pgfqpoint{3.712040in}{1.376046in}}%
\pgfpathlineto{\pgfqpoint{3.733167in}{1.413578in}}%
\pgfpathlineto{\pgfqpoint{3.751475in}{1.442412in}}%
\pgfpathlineto{\pgfqpoint{3.769314in}{1.507962in}}%
\pgfpathlineto{\pgfqpoint{3.789267in}{1.627234in}}%
\pgfpathlineto{\pgfqpoint{3.808515in}{1.717199in}}%
\pgfpathlineto{\pgfqpoint{3.825885in}{1.754361in}}%
\pgfpathlineto{\pgfqpoint{3.846776in}{1.749647in}}%
\pgfpathlineto{\pgfqpoint{3.865084in}{1.709938in}}%
\pgfpathlineto{\pgfqpoint{3.886445in}{1.593728in}}%
\pgfpathlineto{\pgfqpoint{3.920715in}{1.444973in}}%
\pgfpathlineto{\pgfqpoint{3.943248in}{1.386329in}}%
\pgfpathlineto{\pgfqpoint{3.961793in}{1.366108in}}%
\pgfpathlineto{\pgfqpoint{3.979398in}{1.367116in}}%
\pgfpathlineto{\pgfqpoint{4.000288in}{1.404687in}}%
\pgfpathlineto{\pgfqpoint{4.018362in}{1.461545in}}%
\pgfpathlineto{\pgfqpoint{4.039723in}{1.521074in}}%
\pgfpathlineto{\pgfqpoint{4.057093in}{1.620124in}}%
\pgfpathlineto{\pgfqpoint{4.077984in}{1.701692in}}%
\pgfpathlineto{\pgfqpoint{4.096761in}{1.755233in}}%
\pgfpathlineto{\pgfqpoint{4.114837in}{1.763176in}}%
\pgfpathlineto{\pgfqpoint{4.134084in}{1.737371in}}%
\pgfpathlineto{\pgfqpoint{4.154271in}{1.677502in}}%
\pgfpathlineto{\pgfqpoint{4.171406in}{1.570843in}}%
\pgfpathlineto{\pgfqpoint{4.193001in}{1.471958in}}%
\pgfpathlineto{\pgfqpoint{4.211544in}{1.450194in}}%
\pgfpathlineto{\pgfqpoint{4.228680in}{1.392432in}}%
\pgfpathlineto{\pgfqpoint{4.250041in}{1.368217in}}%
\pgfpathlineto{\pgfqpoint{4.267644in}{1.393233in}}%
\pgfpathlineto{\pgfqpoint{4.288771in}{1.450301in}}%
\pgfpathlineto{\pgfqpoint{4.307550in}{1.520077in}}%
\pgfpathlineto{\pgfqpoint{4.327032in}{1.612767in}}%
\pgfpathlineto{\pgfqpoint{4.345574in}{1.693457in}}%
\pgfpathlineto{\pgfqpoint{4.366936in}{1.759257in}}%
\pgfpathlineto{\pgfqpoint{4.388295in}{1.772221in}}%
\pgfpathlineto{\pgfqpoint{4.406839in}{1.757757in}}%
\pgfpathlineto{\pgfqpoint{4.421393in}{1.713770in}}%
\pgfpathlineto{\pgfqpoint{4.460592in}{1.524423in}}%
\pgfpathlineto{\pgfqpoint{4.478196in}{1.457766in}}%
\pgfpathlineto{\pgfqpoint{4.474442in}{1.475992in}}%
\pgfpathlineto{\pgfqpoint{4.456837in}{1.597045in}}%
\pgfpathlineto{\pgfqpoint{4.435476in}{1.727949in}}%
\pgfpathlineto{\pgfqpoint{4.417168in}{1.771631in}}%
\pgfpathlineto{\pgfqpoint{4.400268in}{1.724970in}}%
\pgfpathlineto{\pgfqpoint{4.378907in}{1.626950in}}%
\pgfpathlineto{\pgfqpoint{4.357782in}{1.489052in}}%
\pgfpathlineto{\pgfqpoint{4.340880in}{1.414891in}}%
\pgfpathlineto{\pgfqpoint{4.319989in}{1.367923in}}%
\pgfpathlineto{\pgfqpoint{4.300507in}{1.403878in}}%
\pgfpathlineto{\pgfqpoint{4.282903in}{1.480562in}}%
\pgfpathlineto{\pgfqpoint{4.261541in}{1.642050in}}%
\pgfpathlineto{\pgfqpoint{4.243702in}{1.739679in}}%
\pgfpathlineto{\pgfqpoint{4.225160in}{1.763508in}}%
\pgfpathlineto{\pgfqpoint{4.205912in}{1.704111in}}%
\pgfpathlineto{\pgfqpoint{4.184551in}{1.553338in}}%
\pgfpathlineto{\pgfqpoint{4.168120in}{1.454218in}}%
\pgfpathlineto{\pgfqpoint{4.146524in}{1.383632in}}%
\pgfpathlineto{\pgfqpoint{4.126339in}{1.370526in}}%
\pgfpathlineto{\pgfqpoint{4.109203in}{1.420113in}}%
\pgfpathlineto{\pgfqpoint{4.090189in}{1.516109in}}%
\pgfpathlineto{\pgfqpoint{4.071176in}{1.662020in}}%
\pgfpathlineto{\pgfqpoint{4.050286in}{1.751266in}}%
\pgfpathlineto{\pgfqpoint{4.033384in}{1.748528in}}%
\pgfpathlineto{\pgfqpoint{4.012493in}{1.655215in}}%
\pgfpathlineto{\pgfqpoint{3.992308in}{1.509968in}}%
\pgfpathlineto{\pgfqpoint{3.974467in}{1.424396in}}%
\pgfpathlineto{\pgfqpoint{3.954047in}{1.366000in}}%
\pgfpathlineto{\pgfqpoint{3.936911in}{1.368487in}}%
\pgfpathlineto{\pgfqpoint{3.916255in}{1.434072in}}%
\pgfpathlineto{\pgfqpoint{3.898181in}{1.524457in}}%
\pgfpathlineto{\pgfqpoint{3.874003in}{1.695076in}}%
\pgfpathlineto{\pgfqpoint{3.859450in}{1.745813in}}%
\pgfpathlineto{\pgfqpoint{3.837856in}{1.730585in}}%
\pgfpathlineto{\pgfqpoint{3.823302in}{1.643484in}}%
\pgfpathlineto{\pgfqpoint{3.800769in}{1.496544in}}%
\pgfpathlineto{\pgfqpoint{3.782693in}{1.417852in}}%
\pgfpathlineto{\pgfqpoint{3.761334in}{1.360858in}}%
\pgfpathlineto{\pgfqpoint{3.745607in}{1.363865in}}%
\pgfpathlineto{\pgfqpoint{3.724247in}{1.418438in}}%
\pgfpathlineto{\pgfqpoint{3.700774in}{1.538181in}}%
\pgfpathlineto{\pgfqpoint{3.686455in}{1.647012in}}%
\pgfpathlineto{\pgfqpoint{3.663216in}{1.736204in}}%
\pgfpathlineto{\pgfqpoint{3.647960in}{1.745378in}}%
\pgfpathlineto{\pgfqpoint{3.627069in}{1.720272in}}%
\pgfpathlineto{\pgfqpoint{3.610168in}{1.616787in}}%
\pgfpathlineto{\pgfqpoint{3.591859in}{1.496813in}}%
\pgfpathlineto{\pgfqpoint{3.566978in}{1.389786in}}%
\pgfpathlineto{\pgfqpoint{3.552893in}{1.368014in}}%
\pgfpathlineto{\pgfqpoint{3.532003in}{1.359908in}}%
\pgfpathlineto{\pgfqpoint{3.515103in}{1.389337in}}%
\pgfpathlineto{\pgfqpoint{3.493742in}{1.468366in}}%
\pgfpathlineto{\pgfqpoint{3.476608in}{1.547705in}}%
\pgfpathlineto{\pgfqpoint{3.455012in}{1.689149in}}%
\pgfpathlineto{\pgfqpoint{3.437173in}{1.738914in}}%
\pgfpathlineto{\pgfqpoint{3.416046in}{1.719935in}}%
\pgfpathlineto{\pgfqpoint{3.378725in}{1.526406in}}%
\pgfpathlineto{\pgfqpoint{3.359477in}{1.442242in}}%
\pgfpathlineto{\pgfqpoint{3.338821in}{1.376611in}}%
\pgfpathlineto{\pgfqpoint{3.321450in}{1.353026in}}%
\pgfpathlineto{\pgfqpoint{3.302203in}{1.368774in}}%
\pgfpathlineto{\pgfqpoint{3.281781in}{1.414324in}}%
\pgfpathlineto{\pgfqpoint{3.263473in}{1.482720in}}%
\pgfpathlineto{\pgfqpoint{3.245634in}{1.597606in}}%
\pgfpathlineto{\pgfqpoint{3.225446in}{1.713375in}}%
\pgfpathlineto{\pgfqpoint{3.204087in}{1.739602in}}%
\pgfpathlineto{\pgfqpoint{3.186482in}{1.695274in}}%
\pgfpathlineto{\pgfqpoint{3.169112in}{1.594896in}}%
\pgfpathlineto{\pgfqpoint{3.147282in}{1.504560in}}%
\pgfpathlineto{\pgfqpoint{3.129442in}{1.420887in}}%
\pgfpathlineto{\pgfqpoint{3.110898in}{1.370305in}}%
\pgfpathlineto{\pgfqpoint{3.093059in}{1.352697in}}%
\pgfpathlineto{\pgfqpoint{3.070291in}{1.377935in}}%
\pgfpathlineto{\pgfqpoint{3.052686in}{1.435685in}}%
\pgfpathlineto{\pgfqpoint{3.034612in}{1.526888in}}%
\pgfpathlineto{\pgfqpoint{3.012782in}{1.647759in}}%
\pgfpathlineto{\pgfqpoint{2.994003in}{1.727274in}}%
\pgfpathlineto{\pgfqpoint{2.975225in}{1.736541in}}%
\pgfpathlineto{\pgfqpoint{2.956682in}{1.705079in}}%
\pgfpathlineto{\pgfqpoint{2.936260in}{1.584935in}}%
\pgfpathlineto{\pgfqpoint{2.919359in}{1.522396in}}%
\pgfpathlineto{\pgfqpoint{2.898234in}{1.421857in}}%
\pgfpathlineto{\pgfqpoint{2.878986in}{1.381018in}}%
\pgfpathlineto{\pgfqpoint{2.859739in}{1.359934in}}%
\pgfpathlineto{\pgfqpoint{2.841665in}{1.358103in}}%
\pgfpathlineto{\pgfqpoint{2.822415in}{1.429287in}}%
\pgfpathlineto{\pgfqpoint{2.801290in}{1.474203in}}%
\pgfpathlineto{\pgfqpoint{2.764203in}{1.697948in}}%
\pgfpathlineto{\pgfqpoint{2.746130in}{1.737637in}}%
\pgfpathlineto{\pgfqpoint{2.726411in}{1.713344in}}%
\pgfpathlineto{\pgfqpoint{2.704817in}{1.651783in}}%
\pgfpathlineto{\pgfqpoint{2.686978in}{1.549464in}}%
\pgfpathlineto{\pgfqpoint{2.628296in}{1.353461in}}%
\pgfpathlineto{\pgfqpoint{2.612099in}{1.357151in}}%
\pgfpathlineto{\pgfqpoint{2.593086in}{1.391092in}}%
\pgfpathlineto{\pgfqpoint{2.574073in}{1.451065in}}%
\pgfpathlineto{\pgfqpoint{2.551774in}{1.543617in}}%
\pgfpathlineto{\pgfqpoint{2.532292in}{1.660941in}}%
\pgfpathlineto{\pgfqpoint{2.513513in}{1.729085in}}%
\pgfpathlineto{\pgfqpoint{2.496142in}{1.734194in}}%
\pgfpathlineto{\pgfqpoint{2.474312in}{1.662598in}}%
\pgfpathlineto{\pgfqpoint{2.456004in}{1.552070in}}%
\pgfpathlineto{\pgfqpoint{2.436286in}{1.454590in}}%
\pgfpathlineto{\pgfqpoint{2.415161in}{1.738431in}}%
\pgfpathlineto{\pgfqpoint{2.400138in}{1.722504in}}%
\pgfpathlineto{\pgfqpoint{2.376900in}{1.602389in}}%
\pgfpathlineto{\pgfqpoint{2.359764in}{1.490614in}}%
\pgfpathlineto{\pgfqpoint{2.341221in}{1.418372in}}%
\pgfpathlineto{\pgfqpoint{2.321739in}{1.368608in}}%
\pgfpathlineto{\pgfqpoint{2.300612in}{1.353751in}}%
\pgfpathlineto{\pgfqpoint{2.286295in}{1.379011in}}%
\pgfpathlineto{\pgfqpoint{2.264231in}{1.446700in}}%
\pgfpathlineto{\pgfqpoint{2.244747in}{1.537295in}}%
\pgfpathlineto{\pgfqpoint{2.226438in}{1.636316in}}%
\pgfpathlineto{\pgfqpoint{2.206722in}{1.725605in}}%
\pgfpathlineto{\pgfqpoint{2.187709in}{1.738613in}}%
\pgfpathlineto{\pgfqpoint{2.168695in}{1.699461in}}%
\pgfpathlineto{\pgfqpoint{2.146865in}{1.575010in}}%
\pgfpathlineto{\pgfqpoint{2.129729in}{1.469791in}}%
\pgfpathlineto{\pgfqpoint{2.110013in}{1.406602in}}%
\pgfpathlineto{\pgfqpoint{2.088417in}{1.363562in}}%
\pgfpathlineto{\pgfqpoint{2.073160in}{1.399227in}}%
\pgfpathlineto{\pgfqpoint{2.051096in}{1.371397in}}%
\pgfpathlineto{\pgfqpoint{2.033491in}{1.352963in}}%
\pgfpathlineto{\pgfqpoint{2.014478in}{1.376021in}}%
\pgfpathlineto{\pgfqpoint{1.995464in}{1.405651in}}%
\pgfpathlineto{\pgfqpoint{1.974808in}{1.489508in}}%
\pgfpathlineto{\pgfqpoint{1.955326in}{1.622405in}}%
\pgfpathlineto{\pgfqpoint{1.937956in}{1.704490in}}%
\pgfpathlineto{\pgfqpoint{1.919882in}{1.742377in}}%
\pgfpathlineto{\pgfqpoint{1.898992in}{1.717082in}}%
\pgfpathlineto{\pgfqpoint{1.880916in}{1.622373in}}%
\pgfpathlineto{\pgfqpoint{1.863311in}{1.504118in}}%
\pgfpathlineto{\pgfqpoint{1.843595in}{1.432035in}}%
\pgfpathlineto{\pgfqpoint{1.824347in}{1.387174in}}%
\pgfpathlineto{\pgfqpoint{1.800874in}{1.354850in}}%
\pgfpathlineto{\pgfqpoint{1.784912in}{1.369047in}}%
\pgfpathlineto{\pgfqpoint{1.762847in}{1.421632in}}%
\pgfpathlineto{\pgfqpoint{1.744774in}{1.508061in}}%
\pgfpathlineto{\pgfqpoint{1.725292in}{1.629740in}}%
\pgfpathlineto{\pgfqpoint{1.706747in}{1.711188in}}%
\pgfpathlineto{\pgfqpoint{1.688674in}{1.746217in}}%
\pgfpathlineto{\pgfqpoint{1.667078in}{1.713736in}}%
\pgfpathlineto{\pgfqpoint{1.648065in}{1.629738in}}%
\pgfpathlineto{\pgfqpoint{1.629757in}{1.556198in}}%
\pgfpathlineto{\pgfqpoint{1.611917in}{1.474391in}}%
\pgfpathlineto{\pgfqpoint{1.589853in}{1.402109in}}%
\pgfpathlineto{\pgfqpoint{1.572482in}{1.365968in}}%
\pgfpathlineto{\pgfqpoint{1.553000in}{1.360179in}}%
\pgfpathlineto{\pgfqpoint{1.535395in}{1.390472in}}%
\pgfpathlineto{\pgfqpoint{1.515443in}{1.436579in}}%
\pgfpathlineto{\pgfqpoint{1.496900in}{1.516995in}}%
\pgfpathlineto{\pgfqpoint{1.474130in}{1.644060in}}%
\pgfpathlineto{\pgfqpoint{1.456996in}{1.435523in}}%
\pgfpathlineto{\pgfqpoint{1.436809in}{1.418559in}}%
\pgfpathlineto{\pgfqpoint{1.417561in}{1.506426in}}%
\pgfpathlineto{\pgfqpoint{1.401130in}{1.606237in}}%
\pgfpathlineto{\pgfqpoint{1.384229in}{1.694587in}}%
\pgfpathlineto{\pgfqpoint{1.359113in}{1.751515in}}%
\pgfpathlineto{\pgfqpoint{1.343856in}{1.741226in}}%
\pgfpathlineto{\pgfqpoint{1.324138in}{1.700004in}}%
\pgfpathlineto{\pgfqpoint{1.305595in}{1.589602in}}%
\pgfpathlineto{\pgfqpoint{1.284470in}{1.478711in}}%
\pgfpathlineto{\pgfqpoint{1.261935in}{1.406483in}}%
\pgfpathlineto{\pgfqpoint{1.247147in}{1.371336in}}%
\pgfpathlineto{\pgfqpoint{1.228368in}{1.358588in}}%
\pgfpathlineto{\pgfqpoint{1.207478in}{1.389082in}}%
\pgfpathlineto{\pgfqpoint{1.187761in}{1.445062in}}%
\pgfpathlineto{\pgfqpoint{1.169217in}{1.531528in}}%
\pgfpathlineto{\pgfqpoint{1.150674in}{1.646846in}}%
\pgfpathlineto{\pgfqpoint{1.129079in}{1.729093in}}%
\pgfpathlineto{\pgfqpoint{1.112882in}{1.756321in}}%
\pgfpathlineto{\pgfqpoint{1.092226in}{1.743373in}}%
\pgfpathlineto{\pgfqpoint{1.074152in}{1.675131in}}%
\pgfpathlineto{\pgfqpoint{1.054670in}{1.575142in}}%
\pgfpathlineto{\pgfqpoint{1.036126in}{1.489608in}}%
\pgfpathlineto{\pgfqpoint{1.011479in}{1.412225in}}%
\pgfpathlineto{\pgfqpoint{0.995988in}{1.588707in}}%
\pgfpathlineto{\pgfqpoint{0.976974in}{1.491233in}}%
\pgfpathlineto{\pgfqpoint{0.958196in}{1.426855in}}%
\pgfpathlineto{\pgfqpoint{0.937071in}{1.374750in}}%
\pgfpathlineto{\pgfqpoint{0.921577in}{1.365782in}}%
\pgfpathlineto{\pgfqpoint{0.901627in}{1.402169in}}%
\pgfpathlineto{\pgfqpoint{0.883082in}{1.441271in}}%
\pgfpathlineto{\pgfqpoint{0.863366in}{1.527543in}}%
\pgfpathlineto{\pgfqpoint{0.840830in}{1.632688in}}%
\pgfpathlineto{\pgfqpoint{0.824399in}{1.724359in}}%
\pgfpathlineto{\pgfqpoint{0.804683in}{1.764068in}}%
\pgfpathlineto{\pgfqpoint{0.786609in}{1.758853in}}%
\pgfpathlineto{\pgfqpoint{0.762665in}{1.675761in}}%
\pgfpathlineto{\pgfqpoint{0.746469in}{1.600841in}}%
\pgfpathlineto{\pgfqpoint{0.728161in}{1.504393in}}%
\pgfpathlineto{\pgfqpoint{0.709148in}{1.440211in}}%
\pgfpathlineto{\pgfqpoint{0.686614in}{1.386189in}}%
\pgfpathlineto{\pgfqpoint{0.669244in}{1.369783in}}%
\pgfpathlineto{\pgfqpoint{0.651170in}{1.392321in}}%
\pgfpathlineto{\pgfqpoint{0.648119in}{1.392648in}}%
\pgfpathlineto{\pgfqpoint{0.654925in}{1.377038in}}%
\pgfpathlineto{\pgfqpoint{0.676286in}{1.394012in}}%
\pgfpathlineto{\pgfqpoint{0.695063in}{1.462600in}}%
\pgfpathlineto{\pgfqpoint{0.713139in}{1.565875in}}%
\pgfpathlineto{\pgfqpoint{0.734969in}{1.715793in}}%
\pgfpathlineto{\pgfqpoint{0.751165in}{1.764213in}}%
\pgfpathlineto{\pgfqpoint{0.772290in}{1.734615in}}%
\pgfpathlineto{\pgfqpoint{0.791303in}{1.617525in}}%
\pgfpathlineto{\pgfqpoint{0.810786in}{1.473784in}}%
\pgfpathlineto{\pgfqpoint{0.829094in}{1.393117in}}%
\pgfpathlineto{\pgfqpoint{0.847638in}{1.364833in}}%
\pgfpathlineto{\pgfqpoint{0.867120in}{1.414218in}}%
\pgfpathlineto{\pgfqpoint{0.886602in}{1.497064in}}%
\pgfpathlineto{\pgfqpoint{0.905852in}{1.643157in}}%
\pgfpathlineto{\pgfqpoint{0.924863in}{1.743361in}}%
\pgfpathlineto{\pgfqpoint{0.945285in}{1.753170in}}%
\pgfpathlineto{\pgfqpoint{0.963361in}{1.667696in}}%
\pgfpathlineto{\pgfqpoint{0.983077in}{1.528721in}}%
\pgfpathlineto{\pgfqpoint{1.003030in}{1.425651in}}%
\pgfpathlineto{\pgfqpoint{1.021572in}{1.367817in}}%
\pgfpathlineto{\pgfqpoint{1.039646in}{1.374288in}}%
\pgfpathlineto{\pgfqpoint{1.059365in}{1.435951in}}%
\pgfpathlineto{\pgfqpoint{1.080489in}{1.550570in}}%
\pgfpathlineto{\pgfqpoint{1.100911in}{1.692480in}}%
\pgfpathlineto{\pgfqpoint{1.116168in}{1.663326in}}%
\pgfpathlineto{\pgfqpoint{1.138233in}{1.749953in}}%
\pgfpathlineto{\pgfqpoint{1.154900in}{1.739871in}}%
\pgfpathlineto{\pgfqpoint{1.174147in}{1.641934in}}%
\pgfpathlineto{\pgfqpoint{1.195507in}{1.483780in}}%
\pgfpathlineto{\pgfqpoint{1.215460in}{1.397961in}}%
\pgfpathlineto{\pgfqpoint{1.231890in}{1.359670in}}%
\pgfpathlineto{\pgfqpoint{1.250669in}{1.374436in}}%
\pgfpathlineto{\pgfqpoint{1.275314in}{1.456141in}}%
\pgfpathlineto{\pgfqpoint{1.294562in}{1.574153in}}%
\pgfpathlineto{\pgfqpoint{1.310993in}{1.686393in}}%
\pgfpathlineto{\pgfqpoint{1.326486in}{1.742936in}}%
\pgfpathlineto{\pgfqpoint{1.348316in}{1.728768in}}%
\pgfpathlineto{\pgfqpoint{1.367564in}{1.623345in}}%
\pgfpathlineto{\pgfqpoint{1.386106in}{1.490023in}}%
\pgfpathlineto{\pgfqpoint{1.405825in}{1.401891in}}%
\pgfpathlineto{\pgfqpoint{1.425776in}{1.357561in}}%
\pgfpathlineto{\pgfqpoint{1.447606in}{1.379968in}}%
\pgfpathlineto{\pgfqpoint{1.464273in}{1.421369in}}%
\pgfpathlineto{\pgfqpoint{1.481407in}{1.500585in}}%
\pgfpathlineto{\pgfqpoint{1.500889in}{1.619709in}}%
\pgfpathlineto{\pgfqpoint{1.523190in}{1.728732in}}%
\pgfpathlineto{\pgfqpoint{1.538447in}{1.742741in}}%
\pgfpathlineto{\pgfqpoint{1.563328in}{1.660264in}}%
\pgfpathlineto{\pgfqpoint{1.579054in}{1.544271in}}%
\pgfpathlineto{\pgfqpoint{1.598538in}{1.439414in}}%
\pgfpathlineto{\pgfqpoint{1.617551in}{1.393065in}}%
\pgfpathlineto{\pgfqpoint{1.636799in}{1.522943in}}%
\pgfpathlineto{\pgfqpoint{1.655107in}{1.436930in}}%
\pgfpathlineto{\pgfqpoint{1.673181in}{1.378246in}}%
\pgfpathlineto{\pgfqpoint{1.695481in}{1.353649in}}%
\pgfpathlineto{\pgfqpoint{1.715901in}{1.376931in}}%
\pgfpathlineto{\pgfqpoint{1.733740in}{1.432521in}}%
\pgfpathlineto{\pgfqpoint{1.751816in}{1.523834in}}%
\pgfpathlineto{\pgfqpoint{1.771064in}{1.639544in}}%
\pgfpathlineto{\pgfqpoint{1.790780in}{1.727932in}}%
\pgfpathlineto{\pgfqpoint{1.808619in}{1.740560in}}%
\pgfpathlineto{\pgfqpoint{1.829981in}{1.680955in}}%
\pgfpathlineto{\pgfqpoint{1.848760in}{1.549180in}}%
\pgfpathlineto{\pgfqpoint{1.867771in}{1.445079in}}%
\pgfpathlineto{\pgfqpoint{1.886315in}{1.385058in}}%
\pgfpathlineto{\pgfqpoint{1.905094in}{1.355648in}}%
\pgfpathlineto{\pgfqpoint{1.923402in}{1.362913in}}%
\pgfpathlineto{\pgfqpoint{1.945467in}{1.415095in}}%
\pgfpathlineto{\pgfqpoint{1.964011in}{1.491521in}}%
\pgfpathlineto{\pgfqpoint{1.980911in}{1.586179in}}%
\pgfpathlineto{\pgfqpoint{2.001333in}{1.709565in}}%
\pgfpathlineto{\pgfqpoint{2.021754in}{1.739656in}}%
\pgfpathlineto{\pgfqpoint{2.038890in}{1.708906in}}%
\pgfpathlineto{\pgfqpoint{2.057198in}{1.599930in}}%
\pgfpathlineto{\pgfqpoint{2.079732in}{1.475021in}}%
\pgfpathlineto{\pgfqpoint{2.096633in}{1.412427in}}%
\pgfpathlineto{\pgfqpoint{2.116350in}{1.364220in}}%
\pgfpathlineto{\pgfqpoint{2.136772in}{1.353236in}}%
\pgfpathlineto{\pgfqpoint{2.155316in}{1.386752in}}%
\pgfpathlineto{\pgfqpoint{2.172684in}{1.423673in}}%
\pgfpathlineto{\pgfqpoint{2.193811in}{1.510131in}}%
\pgfpathlineto{\pgfqpoint{2.211651in}{1.631100in}}%
\pgfpathlineto{\pgfqpoint{2.233950in}{1.724352in}}%
\pgfpathlineto{\pgfqpoint{2.252728in}{1.737006in}}%
\pgfpathlineto{\pgfqpoint{2.271976in}{1.686156in}}%
\pgfpathlineto{\pgfqpoint{2.292163in}{1.552435in}}%
\pgfpathlineto{\pgfqpoint{2.308123in}{1.470598in}}%
\pgfpathlineto{\pgfqpoint{2.328780in}{1.396687in}}%
\pgfpathlineto{\pgfqpoint{2.350375in}{1.354606in}}%
\pgfpathlineto{\pgfqpoint{2.367746in}{1.356166in}}%
\pgfpathlineto{\pgfqpoint{2.385351in}{1.390749in}}%
\pgfpathlineto{\pgfqpoint{2.404129in}{1.440371in}}%
\pgfpathlineto{\pgfqpoint{2.424785in}{1.543296in}}%
\pgfpathlineto{\pgfqpoint{2.442390in}{1.654615in}}%
\pgfpathlineto{\pgfqpoint{2.459993in}{1.725343in}}%
\pgfpathlineto{\pgfqpoint{2.485814in}{1.727110in}}%
\pgfpathlineto{\pgfqpoint{2.502245in}{1.695998in}}%
\pgfpathlineto{\pgfqpoint{2.541446in}{1.474292in}}%
\pgfpathlineto{\pgfqpoint{2.555999in}{1.413547in}}%
\pgfpathlineto{\pgfqpoint{2.576655in}{1.365337in}}%
\pgfpathlineto{\pgfqpoint{2.597780in}{1.356060in}}%
\pgfpathlineto{\pgfqpoint{2.615151in}{1.351818in}}%
\pgfpathlineto{\pgfqpoint{2.633693in}{1.362355in}}%
\pgfpathlineto{\pgfqpoint{2.654820in}{1.414744in}}%
\pgfpathlineto{\pgfqpoint{2.672659in}{1.482292in}}%
\pgfpathlineto{\pgfqpoint{2.713032in}{1.699826in}}%
\pgfpathlineto{\pgfqpoint{2.730402in}{1.737241in}}%
\pgfpathlineto{\pgfqpoint{2.751058in}{1.714501in}}%
\pgfpathlineto{\pgfqpoint{2.769603in}{1.671840in}}%
\pgfpathlineto{\pgfqpoint{2.790023in}{1.535399in}}%
\pgfpathlineto{\pgfqpoint{2.808801in}{1.451171in}}%
\pgfpathlineto{\pgfqpoint{2.827815in}{1.387806in}}%
\pgfpathlineto{\pgfqpoint{2.847297in}{1.358146in}}%
\pgfpathlineto{\pgfqpoint{2.864667in}{1.359945in}}%
\pgfpathlineto{\pgfqpoint{2.889080in}{1.398757in}}%
\pgfpathlineto{\pgfqpoint{2.904337in}{1.446087in}}%
\pgfpathlineto{\pgfqpoint{2.926870in}{1.539208in}}%
\pgfpathlineto{\pgfqpoint{2.944946in}{1.643341in}}%
\pgfpathlineto{\pgfqpoint{2.962550in}{1.717844in}}%
\pgfpathlineto{\pgfqpoint{2.978510in}{1.426901in}}%
\pgfpathlineto{\pgfqpoint{3.020997in}{1.597907in}}%
\pgfpathlineto{\pgfqpoint{3.038367in}{1.704613in}}%
\pgfpathlineto{\pgfqpoint{3.058554in}{1.741786in}}%
\pgfpathlineto{\pgfqpoint{3.074985in}{1.694093in}}%
\pgfpathlineto{\pgfqpoint{3.097284in}{1.620294in}}%
\pgfpathlineto{\pgfqpoint{3.113949in}{1.504819in}}%
\pgfpathlineto{\pgfqpoint{3.134606in}{1.415737in}}%
\pgfpathlineto{\pgfqpoint{3.155262in}{1.370781in}}%
\pgfpathlineto{\pgfqpoint{3.174040in}{1.352561in}}%
\pgfpathlineto{\pgfqpoint{3.194931in}{1.385384in}}%
\pgfpathlineto{\pgfqpoint{3.213241in}{1.425356in}}%
\pgfpathlineto{\pgfqpoint{3.230611in}{1.483994in}}%
\pgfpathlineto{\pgfqpoint{3.254083in}{1.628370in}}%
\pgfpathlineto{\pgfqpoint{3.269107in}{1.702542in}}%
\pgfpathlineto{\pgfqpoint{3.290466in}{1.744913in}}%
\pgfpathlineto{\pgfqpoint{3.307837in}{1.723489in}}%
\pgfpathlineto{\pgfqpoint{3.325676in}{1.670971in}}%
\pgfpathlineto{\pgfqpoint{3.347975in}{1.531397in}}%
\pgfpathlineto{\pgfqpoint{3.365814in}{1.442689in}}%
\pgfpathlineto{\pgfqpoint{3.383653in}{1.397663in}}%
\pgfpathlineto{\pgfqpoint{3.403841in}{1.358415in}}%
\pgfpathlineto{\pgfqpoint{3.421914in}{1.361623in}}%
\pgfpathlineto{\pgfqpoint{3.443744in}{1.404610in}}%
\pgfpathlineto{\pgfqpoint{3.461349in}{1.462838in}}%
\pgfpathlineto{\pgfqpoint{3.480128in}{1.547200in}}%
\pgfpathlineto{\pgfqpoint{3.500079in}{1.661350in}}%
\pgfpathlineto{\pgfqpoint{3.519329in}{1.721299in}}%
\pgfpathlineto{\pgfqpoint{3.539748in}{1.750243in}}%
\pgfpathlineto{\pgfqpoint{3.558293in}{1.726323in}}%
\pgfpathlineto{\pgfqpoint{3.578480in}{1.635952in}}%
\pgfpathlineto{\pgfqpoint{3.596319in}{1.539796in}}%
\pgfpathlineto{\pgfqpoint{3.614159in}{1.469017in}}%
\pgfpathlineto{\pgfqpoint{3.635754in}{1.405245in}}%
\pgfpathlineto{\pgfqpoint{3.653593in}{1.368275in}}%
\pgfpathlineto{\pgfqpoint{3.671433in}{1.358445in}}%
\pgfpathlineto{\pgfqpoint{3.693263in}{1.387553in}}%
\pgfpathlineto{\pgfqpoint{3.710868in}{1.422282in}}%
\pgfpathlineto{\pgfqpoint{3.731993in}{1.499649in}}%
\pgfpathlineto{\pgfqpoint{3.753588in}{1.598739in}}%
\pgfpathlineto{\pgfqpoint{3.771428in}{1.697419in}}%
\pgfpathlineto{\pgfqpoint{3.788562in}{1.728995in}}%
\pgfpathlineto{\pgfqpoint{3.806637in}{1.753279in}}%
\pgfpathlineto{\pgfqpoint{3.828702in}{1.745781in}}%
\pgfpathlineto{\pgfqpoint{3.845836in}{1.700716in}}%
\pgfpathlineto{\pgfqpoint{3.885036in}{1.490123in}}%
\pgfpathlineto{\pgfqpoint{3.903345in}{1.427317in}}%
\pgfpathlineto{\pgfqpoint{3.920949in}{1.388886in}}%
\pgfpathlineto{\pgfqpoint{3.945128in}{1.361807in}}%
\pgfpathlineto{\pgfqpoint{3.960150in}{1.372245in}}%
\pgfpathlineto{\pgfqpoint{3.999585in}{1.453197in}}%
\pgfpathlineto{\pgfqpoint{4.016953in}{1.516305in}}%
\pgfpathlineto{\pgfqpoint{4.038549in}{1.624892in}}%
\pgfpathlineto{\pgfqpoint{4.059908in}{1.722412in}}%
\pgfpathlineto{\pgfqpoint{4.077279in}{1.757286in}}%
\pgfpathlineto{\pgfqpoint{4.096527in}{1.763805in}}%
\pgfpathlineto{\pgfqpoint{4.119765in}{1.718170in}}%
\pgfpathlineto{\pgfqpoint{4.134787in}{1.679571in}}%
\pgfpathlineto{\pgfqpoint{4.153097in}{1.573789in}}%
\pgfpathlineto{\pgfqpoint{4.174222in}{1.482138in}}%
\pgfpathlineto{\pgfqpoint{4.191593in}{1.445200in}}%
\pgfpathlineto{\pgfqpoint{4.209666in}{1.392320in}}%
\pgfpathlineto{\pgfqpoint{4.231028in}{1.369536in}}%
\pgfpathlineto{\pgfqpoint{4.249101in}{1.368353in}}%
\pgfpathlineto{\pgfqpoint{4.269523in}{1.409052in}}%
\pgfpathlineto{\pgfqpoint{4.287362in}{1.459396in}}%
\pgfpathlineto{\pgfqpoint{4.305905in}{1.664351in}}%
\pgfpathlineto{\pgfqpoint{4.325389in}{1.537988in}}%
\pgfpathlineto{\pgfqpoint{4.349096in}{1.423411in}}%
\pgfpathlineto{\pgfqpoint{4.366230in}{1.377261in}}%
\pgfpathlineto{\pgfqpoint{4.384072in}{1.372047in}}%
\pgfpathlineto{\pgfqpoint{4.401440in}{1.415836in}}%
\pgfpathlineto{\pgfqpoint{4.420453in}{1.455289in}}%
\pgfpathlineto{\pgfqpoint{4.442283in}{1.565253in}}%
\pgfpathlineto{\pgfqpoint{4.460592in}{1.681920in}}%
\pgfpathlineto{\pgfqpoint{4.479605in}{1.755073in}}%
\pgfpathlineto{\pgfqpoint{4.474911in}{1.736174in}}%
\pgfpathlineto{\pgfqpoint{4.434303in}{1.470562in}}%
\pgfpathlineto{\pgfqpoint{4.412473in}{1.385698in}}%
\pgfpathlineto{\pgfqpoint{4.396980in}{1.367850in}}%
\pgfpathlineto{\pgfqpoint{4.378438in}{1.419045in}}%
\pgfpathlineto{\pgfqpoint{4.357076in}{1.532977in}}%
\pgfpathlineto{\pgfqpoint{4.339003in}{1.676036in}}%
\pgfpathlineto{\pgfqpoint{4.321164in}{1.754547in}}%
\pgfpathlineto{\pgfqpoint{4.298628in}{1.745726in}}%
\pgfpathlineto{\pgfqpoint{4.282903in}{1.659448in}}%
\pgfpathlineto{\pgfqpoint{4.264358in}{1.526747in}}%
\pgfpathlineto{\pgfqpoint{4.243702in}{1.421973in}}%
\pgfpathlineto{\pgfqpoint{4.221872in}{1.363981in}}%
\pgfpathlineto{\pgfqpoint{4.204267in}{1.386350in}}%
\pgfpathlineto{\pgfqpoint{4.187133in}{1.453325in}}%
\pgfpathlineto{\pgfqpoint{4.165772in}{1.582698in}}%
\pgfpathlineto{\pgfqpoint{4.148638in}{1.710532in}}%
\pgfpathlineto{\pgfqpoint{4.127511in}{1.758813in}}%
\pgfpathlineto{\pgfqpoint{4.108029in}{1.706625in}}%
\pgfpathlineto{\pgfqpoint{4.072350in}{1.451460in}}%
\pgfpathlineto{\pgfqpoint{4.050051in}{1.386835in}}%
\pgfpathlineto{\pgfqpoint{4.033620in}{1.359226in}}%
\pgfpathlineto{\pgfqpoint{4.011556in}{1.402409in}}%
\pgfpathlineto{\pgfqpoint{3.994420in}{1.476683in}}%
\pgfpathlineto{\pgfqpoint{3.973764in}{1.621795in}}%
\pgfpathlineto{\pgfqpoint{3.956159in}{1.722521in}}%
\pgfpathlineto{\pgfqpoint{3.935503in}{1.751118in}}%
\pgfpathlineto{\pgfqpoint{3.915786in}{1.717108in}}%
\pgfpathlineto{\pgfqpoint{3.897947in}{1.752209in}}%
\pgfpathlineto{\pgfqpoint{3.877525in}{1.702430in}}%
\pgfpathlineto{\pgfqpoint{3.861095in}{1.585224in}}%
\pgfpathlineto{\pgfqpoint{3.839030in}{1.460601in}}%
\pgfpathlineto{\pgfqpoint{3.821189in}{1.393394in}}%
\pgfpathlineto{\pgfqpoint{3.797247in}{1.356900in}}%
\pgfpathlineto{\pgfqpoint{3.783164in}{1.380704in}}%
\pgfpathlineto{\pgfqpoint{3.763917in}{1.449093in}}%
\pgfpathlineto{\pgfqpoint{3.722368in}{1.702457in}}%
\pgfpathlineto{\pgfqpoint{3.705703in}{1.744935in}}%
\pgfpathlineto{\pgfqpoint{3.686924in}{1.730888in}}%
\pgfpathlineto{\pgfqpoint{3.667442in}{1.625178in}}%
\pgfpathlineto{\pgfqpoint{3.649134in}{1.510026in}}%
\pgfpathlineto{\pgfqpoint{3.627772in}{1.427400in}}%
\pgfpathlineto{\pgfqpoint{3.611342in}{1.377739in}}%
\pgfpathlineto{\pgfqpoint{3.592797in}{1.354324in}}%
\pgfpathlineto{\pgfqpoint{3.568855in}{1.385374in}}%
\pgfpathlineto{\pgfqpoint{3.552190in}{1.421205in}}%
\pgfpathlineto{\pgfqpoint{3.530594in}{1.512474in}}%
\pgfpathlineto{\pgfqpoint{3.513695in}{1.612565in}}%
\pgfpathlineto{\pgfqpoint{3.497498in}{1.693108in}}%
\pgfpathlineto{\pgfqpoint{3.474963in}{1.744088in}}%
\pgfpathlineto{\pgfqpoint{3.451726in}{1.712365in}}%
\pgfpathlineto{\pgfqpoint{3.415812in}{1.514442in}}%
\pgfpathlineto{\pgfqpoint{3.398441in}{1.430963in}}%
\pgfpathlineto{\pgfqpoint{3.377551in}{1.371166in}}%
\pgfpathlineto{\pgfqpoint{3.356191in}{1.394365in}}%
\pgfpathlineto{\pgfqpoint{3.341403in}{1.361051in}}%
\pgfpathlineto{\pgfqpoint{3.320982in}{1.361462in}}%
\pgfpathlineto{\pgfqpoint{3.304080in}{1.397569in}}%
\pgfpathlineto{\pgfqpoint{3.286241in}{1.493184in}}%
\pgfpathlineto{\pgfqpoint{3.264176in}{1.603310in}}%
\pgfpathlineto{\pgfqpoint{3.245163in}{1.714370in}}%
\pgfpathlineto{\pgfqpoint{3.226855in}{1.739601in}}%
\pgfpathlineto{\pgfqpoint{3.207842in}{1.694961in}}%
\pgfpathlineto{\pgfqpoint{3.188125in}{1.573259in}}%
\pgfpathlineto{\pgfqpoint{3.168877in}{1.462079in}}%
\pgfpathlineto{\pgfqpoint{3.147516in}{1.393351in}}%
\pgfpathlineto{\pgfqpoint{3.130851in}{1.357538in}}%
\pgfpathlineto{\pgfqpoint{3.108786in}{1.354336in}}%
\pgfpathlineto{\pgfqpoint{3.091181in}{1.385681in}}%
\pgfpathlineto{\pgfqpoint{3.069820in}{1.456109in}}%
\pgfpathlineto{\pgfqpoint{3.051981in}{1.553590in}}%
\pgfpathlineto{\pgfqpoint{3.030622in}{1.682871in}}%
\pgfpathlineto{\pgfqpoint{3.015128in}{1.731569in}}%
\pgfpathlineto{\pgfqpoint{2.996820in}{1.732417in}}%
\pgfpathlineto{\pgfqpoint{2.974287in}{1.647028in}}%
\pgfpathlineto{\pgfqpoint{2.955743in}{1.543922in}}%
\pgfpathlineto{\pgfqpoint{2.936964in}{1.452951in}}%
\pgfpathlineto{\pgfqpoint{2.918421in}{1.392952in}}%
\pgfpathlineto{\pgfqpoint{2.897060in}{1.352050in}}%
\pgfpathlineto{\pgfqpoint{2.878986in}{1.361209in}}%
\pgfpathlineto{\pgfqpoint{2.859739in}{1.394142in}}%
\pgfpathlineto{\pgfqpoint{2.841665in}{1.456057in}}%
\pgfpathlineto{\pgfqpoint{2.821946in}{1.570217in}}%
\pgfpathlineto{\pgfqpoint{2.803404in}{1.630672in}}%
\pgfpathlineto{\pgfqpoint{2.778288in}{1.732604in}}%
\pgfpathlineto{\pgfqpoint{2.743782in}{1.737966in}}%
\pgfpathlineto{\pgfqpoint{2.726882in}{1.701561in}}%
\pgfpathlineto{\pgfqpoint{2.690264in}{1.504011in}}%
\pgfpathlineto{\pgfqpoint{2.670077in}{1.415133in}}%
\pgfpathlineto{\pgfqpoint{2.647072in}{1.362524in}}%
\pgfpathlineto{\pgfqpoint{2.629233in}{1.349482in}}%
\pgfpathlineto{\pgfqpoint{2.611160in}{1.376386in}}%
\pgfpathlineto{\pgfqpoint{2.593320in}{1.419857in}}%
\pgfpathlineto{\pgfqpoint{2.571256in}{1.520426in}}%
\pgfpathlineto{\pgfqpoint{2.555294in}{1.619083in}}%
\pgfpathlineto{\pgfqpoint{2.534872in}{1.719243in}}%
\pgfpathlineto{\pgfqpoint{2.514450in}{1.736847in}}%
\pgfpathlineto{\pgfqpoint{2.495908in}{1.697831in}}%
\pgfpathlineto{\pgfqpoint{2.477364in}{1.598666in}}%
\pgfpathlineto{\pgfqpoint{2.455065in}{1.494789in}}%
\pgfpathlineto{\pgfqpoint{2.437225in}{1.417367in}}%
\pgfpathlineto{\pgfqpoint{2.420089in}{1.372106in}}%
\pgfpathlineto{\pgfqpoint{2.396853in}{1.350010in}}%
\pgfpathlineto{\pgfqpoint{2.380420in}{1.366567in}}%
\pgfpathlineto{\pgfqpoint{2.360703in}{1.401933in}}%
\pgfpathlineto{\pgfqpoint{2.342864in}{1.453787in}}%
\pgfpathlineto{\pgfqpoint{2.302255in}{1.685118in}}%
\pgfpathlineto{\pgfqpoint{2.283007in}{1.722626in}}%
\pgfpathlineto{\pgfqpoint{2.264465in}{1.736516in}}%
\pgfpathlineto{\pgfqpoint{2.245452in}{1.720473in}}%
\pgfpathlineto{\pgfqpoint{2.226907in}{1.643411in}}%
\pgfpathlineto{\pgfqpoint{2.206017in}{1.537887in}}%
\pgfpathlineto{\pgfqpoint{2.186535in}{1.455298in}}%
\pgfpathlineto{\pgfqpoint{2.149448in}{1.370518in}}%
\pgfpathlineto{\pgfqpoint{2.129729in}{1.351946in}}%
\pgfpathlineto{\pgfqpoint{2.109544in}{1.376896in}}%
\pgfpathlineto{\pgfqpoint{2.090296in}{1.429967in}}%
\pgfpathlineto{\pgfqpoint{2.071517in}{1.504201in}}%
\pgfpathlineto{\pgfqpoint{2.052270in}{1.528399in}}%
\pgfpathlineto{\pgfqpoint{2.033256in}{1.578883in}}%
\pgfpathlineto{\pgfqpoint{2.014009in}{1.638537in}}%
\pgfpathlineto{\pgfqpoint{1.995933in}{1.725243in}}%
\pgfpathlineto{\pgfqpoint{1.977391in}{1.740284in}}%
\pgfpathlineto{\pgfqpoint{1.959081in}{1.699165in}}%
\pgfpathlineto{\pgfqpoint{1.936313in}{1.622727in}}%
\pgfpathlineto{\pgfqpoint{1.918005in}{1.508897in}}%
\pgfpathlineto{\pgfqpoint{1.900164in}{1.447593in}}%
\pgfpathlineto{\pgfqpoint{1.878099in}{1.385250in}}%
\pgfpathlineto{\pgfqpoint{1.860260in}{1.355695in}}%
\pgfpathlineto{\pgfqpoint{1.841483in}{1.362369in}}%
\pgfpathlineto{\pgfqpoint{1.822939in}{1.395840in}}%
\pgfpathlineto{\pgfqpoint{1.803456in}{1.453040in}}%
\pgfpathlineto{\pgfqpoint{1.783974in}{1.560242in}}%
\pgfpathlineto{\pgfqpoint{1.765899in}{1.635809in}}%
\pgfpathlineto{\pgfqpoint{1.744539in}{1.718172in}}%
\pgfpathlineto{\pgfqpoint{1.726229in}{1.746070in}}%
\pgfpathlineto{\pgfqpoint{1.707687in}{1.721726in}}%
\pgfpathlineto{\pgfqpoint{1.689142in}{1.662228in}}%
\pgfpathlineto{\pgfqpoint{1.668252in}{1.536482in}}%
\pgfpathlineto{\pgfqpoint{1.649239in}{1.461746in}}%
\pgfpathlineto{\pgfqpoint{1.630225in}{1.413568in}}%
\pgfpathlineto{\pgfqpoint{1.611917in}{1.373770in}}%
\pgfpathlineto{\pgfqpoint{1.590556in}{1.356268in}}%
\pgfpathlineto{\pgfqpoint{1.571543in}{1.364397in}}%
\pgfpathlineto{\pgfqpoint{1.553703in}{1.402634in}}%
\pgfpathlineto{\pgfqpoint{1.532578in}{1.464973in}}%
\pgfpathlineto{\pgfqpoint{1.516382in}{1.456325in}}%
\pgfpathlineto{\pgfqpoint{1.494318in}{1.572148in}}%
\pgfpathlineto{\pgfqpoint{1.472487in}{1.697628in}}%
\pgfpathlineto{\pgfqpoint{1.456760in}{1.708012in}}%
\pgfpathlineto{\pgfqpoint{1.438452in}{1.707750in}}%
\pgfpathlineto{\pgfqpoint{1.420144in}{1.750402in}}%
\pgfpathlineto{\pgfqpoint{1.398314in}{1.734748in}}%
\pgfpathlineto{\pgfqpoint{1.379300in}{1.669137in}}%
\pgfpathlineto{\pgfqpoint{1.361461in}{1.556001in}}%
\pgfpathlineto{\pgfqpoint{1.342917in}{1.475231in}}%
\pgfpathlineto{\pgfqpoint{1.321557in}{1.412139in}}%
\pgfpathlineto{\pgfqpoint{1.302778in}{1.378019in}}%
\pgfpathlineto{\pgfqpoint{1.284470in}{1.358394in}}%
\pgfpathlineto{\pgfqpoint{1.265221in}{1.376534in}}%
\pgfpathlineto{\pgfqpoint{1.247147in}{1.419120in}}%
\pgfpathlineto{\pgfqpoint{1.225553in}{1.475084in}}%
\pgfpathlineto{\pgfqpoint{1.188464in}{1.673322in}}%
\pgfpathlineto{\pgfqpoint{1.170156in}{1.738487in}}%
\pgfpathlineto{\pgfqpoint{1.151612in}{1.758497in}}%
\pgfpathlineto{\pgfqpoint{1.129313in}{1.744583in}}%
\pgfpathlineto{\pgfqpoint{1.111943in}{1.686403in}}%
\pgfpathlineto{\pgfqpoint{1.092460in}{1.579294in}}%
\pgfpathlineto{\pgfqpoint{1.072509in}{1.500316in}}%
\pgfpathlineto{\pgfqpoint{1.053262in}{1.436121in}}%
\pgfpathlineto{\pgfqpoint{1.033543in}{1.402810in}}%
\pgfpathlineto{\pgfqpoint{1.015470in}{1.381280in}}%
\pgfpathlineto{\pgfqpoint{0.996456in}{1.364751in}}%
\pgfpathlineto{\pgfqpoint{0.979086in}{1.393143in}}%
\pgfpathlineto{\pgfqpoint{0.957961in}{1.443598in}}%
\pgfpathlineto{\pgfqpoint{0.938245in}{1.505421in}}%
\pgfpathlineto{\pgfqpoint{0.919700in}{1.553332in}}%
\pgfpathlineto{\pgfqpoint{0.900452in}{1.661893in}}%
\pgfpathlineto{\pgfqpoint{0.882142in}{1.737587in}}%
\pgfpathlineto{\pgfqpoint{0.861252in}{1.765089in}}%
\pgfpathlineto{\pgfqpoint{0.842944in}{1.753089in}}%
\pgfpathlineto{\pgfqpoint{0.823931in}{1.694590in}}%
\pgfpathlineto{\pgfqpoint{0.803040in}{1.573209in}}%
\pgfpathlineto{\pgfqpoint{0.767596in}{1.444789in}}%
\pgfpathlineto{\pgfqpoint{0.748348in}{1.395813in}}%
\pgfpathlineto{\pgfqpoint{0.726518in}{1.367683in}}%
\pgfpathlineto{\pgfqpoint{0.707974in}{1.370433in}}%
\pgfpathlineto{\pgfqpoint{0.688492in}{1.376374in}}%
\pgfpathlineto{\pgfqpoint{0.669478in}{1.479123in}}%
\pgfpathlineto{\pgfqpoint{0.648353in}{1.416895in}}%
\pgfpathlineto{\pgfqpoint{0.651639in}{1.421720in}}%
\pgfpathlineto{\pgfqpoint{0.659150in}{1.458849in}}%
\pgfpathlineto{\pgfqpoint{0.676990in}{1.562713in}}%
\pgfpathlineto{\pgfqpoint{0.695063in}{1.700913in}}%
\pgfpathlineto{\pgfqpoint{0.713373in}{1.763612in}}%
\pgfpathlineto{\pgfqpoint{0.734498in}{1.741477in}}%
\pgfpathlineto{\pgfqpoint{0.753043in}{1.629450in}}%
\pgfpathlineto{\pgfqpoint{0.770177in}{1.496137in}}%
\pgfpathlineto{\pgfqpoint{0.795058in}{1.388284in}}%
\pgfpathlineto{\pgfqpoint{0.810317in}{1.364980in}}%
\pgfpathlineto{\pgfqpoint{0.827922in}{1.408316in}}%
\pgfpathlineto{\pgfqpoint{0.849047in}{1.503790in}}%
\pgfpathlineto{\pgfqpoint{0.867591in}{1.638682in}}%
\pgfpathlineto{\pgfqpoint{0.886602in}{1.744276in}}%
\pgfpathlineto{\pgfqpoint{0.908432in}{1.749318in}}%
\pgfpathlineto{\pgfqpoint{0.924394in}{1.675930in}}%
\pgfpathlineto{\pgfqpoint{0.944816in}{1.510812in}}%
\pgfpathlineto{\pgfqpoint{0.964298in}{1.415929in}}%
\pgfpathlineto{\pgfqpoint{0.981434in}{1.366402in}}%
\pgfpathlineto{\pgfqpoint{1.001151in}{1.371084in}}%
\pgfpathlineto{\pgfqpoint{1.022746in}{1.442503in}}%
\pgfpathlineto{\pgfqpoint{1.041994in}{1.543230in}}%
\pgfpathlineto{\pgfqpoint{1.061711in}{1.680286in}}%
\pgfpathlineto{\pgfqpoint{1.080255in}{1.750117in}}%
\pgfpathlineto{\pgfqpoint{1.097625in}{1.744656in}}%
\pgfpathlineto{\pgfqpoint{1.118516in}{1.627409in}}%
\pgfpathlineto{\pgfqpoint{1.137529in}{1.486140in}}%
\pgfpathlineto{\pgfqpoint{1.157246in}{1.401111in}}%
\pgfpathlineto{\pgfqpoint{1.176964in}{1.359200in}}%
\pgfpathlineto{\pgfqpoint{1.195741in}{1.377722in}}%
\pgfpathlineto{\pgfqpoint{1.214520in}{1.438724in}}%
\pgfpathlineto{\pgfqpoint{1.234707in}{1.539171in}}%
\pgfpathlineto{\pgfqpoint{1.254424in}{1.672934in}}%
\pgfpathlineto{\pgfqpoint{1.271560in}{1.744318in}}%
\pgfpathlineto{\pgfqpoint{1.291745in}{1.743157in}}%
\pgfpathlineto{\pgfqpoint{1.310055in}{1.665600in}}%
\pgfpathlineto{\pgfqpoint{1.349256in}{1.434032in}}%
\pgfpathlineto{\pgfqpoint{1.366390in}{1.374436in}}%
\pgfpathlineto{\pgfqpoint{1.386577in}{1.356470in}}%
\pgfpathlineto{\pgfqpoint{1.408173in}{1.402720in}}%
\pgfpathlineto{\pgfqpoint{1.425776in}{1.464689in}}%
\pgfpathlineto{\pgfqpoint{1.443380in}{1.541776in}}%
\pgfpathlineto{\pgfqpoint{1.465681in}{1.656193in}}%
\pgfpathlineto{\pgfqpoint{1.482112in}{1.732826in}}%
\pgfpathlineto{\pgfqpoint{1.502298in}{1.738614in}}%
\pgfpathlineto{\pgfqpoint{1.520373in}{1.687924in}}%
\pgfpathlineto{\pgfqpoint{1.542907in}{1.547648in}}%
\pgfpathlineto{\pgfqpoint{1.561685in}{1.497584in}}%
\pgfpathlineto{\pgfqpoint{1.578819in}{1.408849in}}%
\pgfpathlineto{\pgfqpoint{1.598538in}{1.365460in}}%
\pgfpathlineto{\pgfqpoint{1.617080in}{1.354045in}}%
\pgfpathlineto{\pgfqpoint{1.636799in}{1.386189in}}%
\pgfpathlineto{\pgfqpoint{1.654167in}{1.430758in}}%
\pgfpathlineto{\pgfqpoint{1.676703in}{1.501109in}}%
\pgfpathlineto{\pgfqpoint{1.695716in}{1.575896in}}%
\pgfpathlineto{\pgfqpoint{1.713790in}{1.680118in}}%
\pgfpathlineto{\pgfqpoint{1.729281in}{1.739741in}}%
\pgfpathlineto{\pgfqpoint{1.750408in}{1.730985in}}%
\pgfpathlineto{\pgfqpoint{1.770359in}{1.641094in}}%
\pgfpathlineto{\pgfqpoint{1.788903in}{1.513354in}}%
\pgfpathlineto{\pgfqpoint{1.808151in}{1.427701in}}%
\pgfpathlineto{\pgfqpoint{1.830215in}{1.378210in}}%
\pgfpathlineto{\pgfqpoint{1.849229in}{1.355057in}}%
\pgfpathlineto{\pgfqpoint{1.867771in}{1.363591in}}%
\pgfpathlineto{\pgfqpoint{1.885612in}{1.405289in}}%
\pgfpathlineto{\pgfqpoint{1.904860in}{1.457554in}}%
\pgfpathlineto{\pgfqpoint{1.922699in}{1.415598in}}%
\pgfpathlineto{\pgfqpoint{1.941712in}{1.365787in}}%
\pgfpathlineto{\pgfqpoint{1.964480in}{1.360673in}}%
\pgfpathlineto{\pgfqpoint{1.982554in}{1.405795in}}%
\pgfpathlineto{\pgfqpoint{2.002038in}{1.483417in}}%
\pgfpathlineto{\pgfqpoint{2.020580in}{1.594237in}}%
\pgfpathlineto{\pgfqpoint{2.038890in}{1.685545in}}%
\pgfpathlineto{\pgfqpoint{2.062127in}{1.739575in}}%
\pgfpathlineto{\pgfqpoint{2.078560in}{1.712571in}}%
\pgfpathlineto{\pgfqpoint{2.103676in}{1.568520in}}%
\pgfpathlineto{\pgfqpoint{2.117290in}{1.474250in}}%
\pgfpathlineto{\pgfqpoint{2.134894in}{1.402146in}}%
\pgfpathlineto{\pgfqpoint{2.152265in}{1.365695in}}%
\pgfpathlineto{\pgfqpoint{2.173390in}{1.354713in}}%
\pgfpathlineto{\pgfqpoint{2.194280in}{1.396627in}}%
\pgfpathlineto{\pgfqpoint{2.212354in}{1.456280in}}%
\pgfpathlineto{\pgfqpoint{2.236532in}{1.545260in}}%
\pgfpathlineto{\pgfqpoint{2.252023in}{1.675915in}}%
\pgfpathlineto{\pgfqpoint{2.271271in}{1.736372in}}%
\pgfpathlineto{\pgfqpoint{2.290989in}{1.722721in}}%
\pgfpathlineto{\pgfqpoint{2.310003in}{1.654386in}}%
\pgfpathlineto{\pgfqpoint{2.327607in}{1.519626in}}%
\pgfpathlineto{\pgfqpoint{2.346855in}{1.434883in}}%
\pgfpathlineto{\pgfqpoint{2.365398in}{1.376950in}}%
\pgfpathlineto{\pgfqpoint{2.386759in}{1.350269in}}%
\pgfpathlineto{\pgfqpoint{2.406241in}{1.370842in}}%
\pgfpathlineto{\pgfqpoint{2.422203in}{1.412574in}}%
\pgfpathlineto{\pgfqpoint{2.442625in}{1.452862in}}%
\pgfpathlineto{\pgfqpoint{2.463515in}{1.562542in}}%
\pgfpathlineto{\pgfqpoint{2.482529in}{1.672659in}}%
\pgfpathlineto{\pgfqpoint{2.502714in}{1.736332in}}%
\pgfpathlineto{\pgfqpoint{2.519615in}{1.729165in}}%
\pgfpathlineto{\pgfqpoint{2.539566in}{1.658234in}}%
\pgfpathlineto{\pgfqpoint{2.560459in}{1.523471in}}%
\pgfpathlineto{\pgfqpoint{2.578767in}{1.473732in}}%
\pgfpathlineto{\pgfqpoint{2.597077in}{1.407791in}}%
\pgfpathlineto{\pgfqpoint{2.614680in}{1.362966in}}%
\pgfpathlineto{\pgfqpoint{2.637215in}{1.357250in}}%
\pgfpathlineto{\pgfqpoint{2.656463in}{1.400796in}}%
\pgfpathlineto{\pgfqpoint{2.673597in}{1.457723in}}%
\pgfpathlineto{\pgfqpoint{2.692847in}{1.546577in}}%
\pgfpathlineto{\pgfqpoint{2.713503in}{1.676599in}}%
\pgfpathlineto{\pgfqpoint{2.730871in}{1.728399in}}%
\pgfpathlineto{\pgfqpoint{2.751998in}{1.732281in}}%
\pgfpathlineto{\pgfqpoint{2.770541in}{1.687470in}}%
\pgfpathlineto{\pgfqpoint{2.788380in}{1.571117in}}%
\pgfpathlineto{\pgfqpoint{2.810915in}{1.452597in}}%
\pgfpathlineto{\pgfqpoint{2.828049in}{1.402046in}}%
\pgfpathlineto{\pgfqpoint{2.846125in}{1.375627in}}%
\pgfpathlineto{\pgfqpoint{2.866076in}{1.352031in}}%
\pgfpathlineto{\pgfqpoint{2.884620in}{1.371382in}}%
\pgfpathlineto{\pgfqpoint{2.905979in}{1.422699in}}%
\pgfpathlineto{\pgfqpoint{2.923115in}{1.501495in}}%
\pgfpathlineto{\pgfqpoint{2.962314in}{1.704759in}}%
\pgfpathlineto{\pgfqpoint{2.980624in}{1.738109in}}%
\pgfpathlineto{\pgfqpoint{3.000811in}{1.726991in}}%
\pgfpathlineto{\pgfqpoint{3.018416in}{1.674676in}}%
\pgfpathlineto{\pgfqpoint{3.057849in}{1.459315in}}%
\pgfpathlineto{\pgfqpoint{3.076862in}{1.403288in}}%
\pgfpathlineto{\pgfqpoint{3.096581in}{1.364612in}}%
\pgfpathlineto{\pgfqpoint{3.115592in}{1.358460in}}%
\pgfpathlineto{\pgfqpoint{3.132259in}{1.380205in}}%
\pgfpathlineto{\pgfqpoint{3.154090in}{1.438916in}}%
\pgfpathlineto{\pgfqpoint{3.175215in}{1.535744in}}%
\pgfpathlineto{\pgfqpoint{3.189534in}{1.628120in}}%
\pgfpathlineto{\pgfqpoint{3.210893in}{1.427971in}}%
\pgfpathlineto{\pgfqpoint{3.251031in}{1.593272in}}%
\pgfpathlineto{\pgfqpoint{3.270515in}{1.711172in}}%
\pgfpathlineto{\pgfqpoint{3.287649in}{1.744931in}}%
\pgfpathlineto{\pgfqpoint{3.306428in}{1.728849in}}%
\pgfpathlineto{\pgfqpoint{3.328024in}{1.665123in}}%
\pgfpathlineto{\pgfqpoint{3.345629in}{1.557178in}}%
\pgfpathlineto{\pgfqpoint{3.366285in}{1.465701in}}%
\pgfpathlineto{\pgfqpoint{3.384593in}{1.400990in}}%
\pgfpathlineto{\pgfqpoint{3.404780in}{1.362525in}}%
\pgfpathlineto{\pgfqpoint{3.425671in}{1.356208in}}%
\pgfpathlineto{\pgfqpoint{3.443979in}{1.386273in}}%
\pgfpathlineto{\pgfqpoint{3.478720in}{1.504821in}}%
\pgfpathlineto{\pgfqpoint{3.521440in}{1.722847in}}%
\pgfpathlineto{\pgfqpoint{3.537871in}{1.749726in}}%
\pgfpathlineto{\pgfqpoint{3.559232in}{1.731200in}}%
\pgfpathlineto{\pgfqpoint{3.576837in}{1.669911in}}%
\pgfpathlineto{\pgfqpoint{3.597728in}{1.541277in}}%
\pgfpathlineto{\pgfqpoint{3.614627in}{1.486113in}}%
\pgfpathlineto{\pgfqpoint{3.639509in}{1.414805in}}%
\pgfpathlineto{\pgfqpoint{3.654531in}{1.380759in}}%
\pgfpathlineto{\pgfqpoint{3.674484in}{1.358594in}}%
\pgfpathlineto{\pgfqpoint{3.693966in}{1.372716in}}%
\pgfpathlineto{\pgfqpoint{3.711571in}{1.415147in}}%
\pgfpathlineto{\pgfqpoint{3.729879in}{1.473066in}}%
\pgfpathlineto{\pgfqpoint{3.749363in}{1.552549in}}%
\pgfpathlineto{\pgfqpoint{3.770254in}{1.652666in}}%
\pgfpathlineto{\pgfqpoint{3.789501in}{1.718590in}}%
\pgfpathlineto{\pgfqpoint{3.805932in}{1.754870in}}%
\pgfpathlineto{\pgfqpoint{3.827528in}{1.635780in}}%
\pgfpathlineto{\pgfqpoint{3.845133in}{1.713945in}}%
\pgfpathlineto{\pgfqpoint{3.866963in}{1.755047in}}%
\pgfpathlineto{\pgfqpoint{3.884331in}{1.749294in}}%
\pgfpathlineto{\pgfqpoint{3.902407in}{1.686220in}}%
\pgfpathlineto{\pgfqpoint{3.923297in}{1.566138in}}%
\pgfpathlineto{\pgfqpoint{3.941137in}{1.481178in}}%
\pgfpathlineto{\pgfqpoint{3.960384in}{1.422827in}}%
\pgfpathlineto{\pgfqpoint{3.979398in}{1.376847in}}%
\pgfpathlineto{\pgfqpoint{3.999585in}{1.363645in}}%
\pgfpathlineto{\pgfqpoint{4.018127in}{1.390072in}}%
\pgfpathlineto{\pgfqpoint{4.039723in}{1.443504in}}%
\pgfpathlineto{\pgfqpoint{4.058266in}{1.531804in}}%
\pgfpathlineto{\pgfqpoint{4.094649in}{1.713911in}}%
\pgfpathlineto{\pgfqpoint{4.117419in}{1.759825in}}%
\pgfpathlineto{\pgfqpoint{4.134084in}{1.759373in}}%
\pgfpathlineto{\pgfqpoint{4.151923in}{1.731657in}}%
\pgfpathlineto{\pgfqpoint{4.175631in}{1.631861in}}%
\pgfpathlineto{\pgfqpoint{4.190184in}{1.532052in}}%
\pgfpathlineto{\pgfqpoint{4.230323in}{1.406412in}}%
\pgfpathlineto{\pgfqpoint{4.250275in}{1.380319in}}%
\pgfpathlineto{\pgfqpoint{4.268115in}{1.367043in}}%
\pgfpathlineto{\pgfqpoint{4.308019in}{1.432602in}}%
\pgfpathlineto{\pgfqpoint{4.327032in}{1.493674in}}%
\pgfpathlineto{\pgfqpoint{4.345340in}{1.577342in}}%
\pgfpathlineto{\pgfqpoint{4.366701in}{1.686167in}}%
\pgfpathlineto{\pgfqpoint{4.384306in}{1.751429in}}%
\pgfpathlineto{\pgfqpoint{4.402380in}{1.772697in}}%
\pgfpathlineto{\pgfqpoint{4.423505in}{1.754703in}}%
\pgfpathlineto{\pgfqpoint{4.441580in}{1.693771in}}%
\pgfpathlineto{\pgfqpoint{4.459654in}{1.600178in}}%
\pgfpathlineto{\pgfqpoint{4.478902in}{1.684167in}}%
\pgfpathlineto{\pgfqpoint{4.480310in}{1.682109in}}%
\pgfpathlineto{\pgfqpoint{4.474207in}{1.721656in}}%
\pgfpathlineto{\pgfqpoint{4.456132in}{1.772204in}}%
\pgfpathlineto{\pgfqpoint{4.438058in}{1.750196in}}%
\pgfpathlineto{\pgfqpoint{4.415290in}{1.615349in}}%
\pgfpathlineto{\pgfqpoint{4.397685in}{1.492111in}}%
\pgfpathlineto{\pgfqpoint{4.378438in}{1.406837in}}%
\pgfpathlineto{\pgfqpoint{4.354259in}{1.370407in}}%
\pgfpathlineto{\pgfqpoint{4.339942in}{1.412625in}}%
\pgfpathlineto{\pgfqpoint{4.323275in}{1.493272in}}%
\pgfpathlineto{\pgfqpoint{4.300976in}{1.635436in}}%
\pgfpathlineto{\pgfqpoint{4.279380in}{1.752122in}}%
\pgfpathlineto{\pgfqpoint{4.262012in}{1.760759in}}%
\pgfpathlineto{\pgfqpoint{4.243233in}{1.695285in}}%
\pgfpathlineto{\pgfqpoint{4.225160in}{1.559815in}}%
\pgfpathlineto{\pgfqpoint{4.203564in}{1.435993in}}%
\pgfpathlineto{\pgfqpoint{4.185725in}{1.379120in}}%
\pgfpathlineto{\pgfqpoint{4.166008in}{1.372288in}}%
\pgfpathlineto{\pgfqpoint{4.147698in}{1.390629in}}%
\pgfpathlineto{\pgfqpoint{4.127745in}{1.475260in}}%
\pgfpathlineto{\pgfqpoint{4.110377in}{1.597797in}}%
\pgfpathlineto{\pgfqpoint{4.086904in}{1.739820in}}%
\pgfpathlineto{\pgfqpoint{4.072819in}{1.758665in}}%
\pgfpathlineto{\pgfqpoint{4.053103in}{1.703179in}}%
\pgfpathlineto{\pgfqpoint{4.028221in}{1.529541in}}%
\pgfpathlineto{\pgfqpoint{4.010851in}{1.443001in}}%
\pgfpathlineto{\pgfqpoint{3.994185in}{1.383738in}}%
\pgfpathlineto{\pgfqpoint{3.974938in}{1.360160in}}%
\pgfpathlineto{\pgfqpoint{3.954516in}{1.412115in}}%
\pgfpathlineto{\pgfqpoint{3.940197in}{1.480652in}}%
\pgfpathlineto{\pgfqpoint{3.918132in}{1.632780in}}%
\pgfpathlineto{\pgfqpoint{3.896302in}{1.742689in}}%
\pgfpathlineto{\pgfqpoint{3.876820in}{1.745877in}}%
\pgfpathlineto{\pgfqpoint{3.856869in}{1.662203in}}%
\pgfpathlineto{\pgfqpoint{3.839264in}{1.529218in}}%
\pgfpathlineto{\pgfqpoint{3.821894in}{1.444456in}}%
\pgfpathlineto{\pgfqpoint{3.802881in}{1.376259in}}%
\pgfpathlineto{\pgfqpoint{3.783399in}{1.355788in}}%
\pgfpathlineto{\pgfqpoint{3.764620in}{1.394115in}}%
\pgfpathlineto{\pgfqpoint{3.742555in}{1.482070in}}%
\pgfpathlineto{\pgfqpoint{3.724950in}{1.604752in}}%
\pgfpathlineto{\pgfqpoint{3.704999in}{1.724197in}}%
\pgfpathlineto{\pgfqpoint{3.687158in}{1.747431in}}%
\pgfpathlineto{\pgfqpoint{3.663687in}{1.678982in}}%
\pgfpathlineto{\pgfqpoint{3.648663in}{1.577557in}}%
\pgfpathlineto{\pgfqpoint{3.627304in}{1.462477in}}%
\pgfpathlineto{\pgfqpoint{3.611342in}{1.400554in}}%
\pgfpathlineto{\pgfqpoint{3.590451in}{1.358387in}}%
\pgfpathlineto{\pgfqpoint{3.569324in}{1.366661in}}%
\pgfpathlineto{\pgfqpoint{3.551721in}{1.412987in}}%
\pgfpathlineto{\pgfqpoint{3.534351in}{1.484813in}}%
\pgfpathlineto{\pgfqpoint{3.513695in}{1.557321in}}%
\pgfpathlineto{\pgfqpoint{3.493039in}{1.685295in}}%
\pgfpathlineto{\pgfqpoint{3.475668in}{1.741686in}}%
\pgfpathlineto{\pgfqpoint{3.454072in}{1.736391in}}%
\pgfpathlineto{\pgfqpoint{3.436702in}{1.659780in}}%
\pgfpathlineto{\pgfqpoint{3.415812in}{1.531576in}}%
\pgfpathlineto{\pgfqpoint{3.397738in}{1.469757in}}%
\pgfpathlineto{\pgfqpoint{3.360885in}{1.364517in}}%
\pgfpathlineto{\pgfqpoint{3.338586in}{1.356508in}}%
\pgfpathlineto{\pgfqpoint{3.321450in}{1.391077in}}%
\pgfpathlineto{\pgfqpoint{3.299857in}{1.460230in}}%
\pgfpathlineto{\pgfqpoint{3.279200in}{1.587335in}}%
\pgfpathlineto{\pgfqpoint{3.264647in}{1.415771in}}%
\pgfpathlineto{\pgfqpoint{3.243520in}{1.360698in}}%
\pgfpathlineto{\pgfqpoint{3.223098in}{1.362873in}}%
\pgfpathlineto{\pgfqpoint{3.208310in}{1.401291in}}%
\pgfpathlineto{\pgfqpoint{3.188125in}{1.482358in}}%
\pgfpathlineto{\pgfqpoint{3.166295in}{1.623670in}}%
\pgfpathlineto{\pgfqpoint{3.148690in}{1.695979in}}%
\pgfpathlineto{\pgfqpoint{3.128737in}{1.740084in}}%
\pgfpathlineto{\pgfqpoint{3.110664in}{1.702889in}}%
\pgfpathlineto{\pgfqpoint{3.070525in}{1.466026in}}%
\pgfpathlineto{\pgfqpoint{3.052921in}{1.398081in}}%
\pgfpathlineto{\pgfqpoint{3.033204in}{1.354119in}}%
\pgfpathlineto{\pgfqpoint{3.012077in}{1.365114in}}%
\pgfpathlineto{\pgfqpoint{2.974990in}{1.465151in}}%
\pgfpathlineto{\pgfqpoint{2.958091in}{1.590368in}}%
\pgfpathlineto{\pgfqpoint{2.935321in}{1.711023in}}%
\pgfpathlineto{\pgfqpoint{2.914430in}{1.738270in}}%
\pgfpathlineto{\pgfqpoint{2.899174in}{1.708657in}}%
\pgfpathlineto{\pgfqpoint{2.879221in}{1.587112in}}%
\pgfpathlineto{\pgfqpoint{2.854808in}{1.449527in}}%
\pgfpathlineto{\pgfqpoint{2.841665in}{1.404104in}}%
\pgfpathlineto{\pgfqpoint{2.820304in}{1.359035in}}%
\pgfpathlineto{\pgfqpoint{2.805047in}{1.353469in}}%
\pgfpathlineto{\pgfqpoint{2.783217in}{1.393249in}}%
\pgfpathlineto{\pgfqpoint{2.763029in}{1.459775in}}%
\pgfpathlineto{\pgfqpoint{2.744956in}{1.576472in}}%
\pgfpathlineto{\pgfqpoint{2.722422in}{1.709015in}}%
\pgfpathlineto{\pgfqpoint{2.705990in}{1.739290in}}%
\pgfpathlineto{\pgfqpoint{2.688385in}{1.712172in}}%
\pgfpathlineto{\pgfqpoint{2.667025in}{1.671148in}}%
\pgfpathlineto{\pgfqpoint{2.648012in}{1.540814in}}%
\pgfpathlineto{\pgfqpoint{2.627825in}{1.440239in}}%
\pgfpathlineto{\pgfqpoint{2.610456in}{1.385801in}}%
\pgfpathlineto{\pgfqpoint{2.590972in}{1.352356in}}%
\pgfpathlineto{\pgfqpoint{2.571490in}{1.359606in}}%
\pgfpathlineto{\pgfqpoint{2.551069in}{1.410473in}}%
\pgfpathlineto{\pgfqpoint{2.534169in}{1.468597in}}%
\pgfpathlineto{\pgfqpoint{2.515859in}{1.584293in}}%
\pgfpathlineto{\pgfqpoint{2.494031in}{1.710558in}}%
\pgfpathlineto{\pgfqpoint{2.475252in}{1.734290in}}%
\pgfpathlineto{\pgfqpoint{2.458587in}{1.729320in}}%
\pgfpathlineto{\pgfqpoint{2.438399in}{1.655115in}}%
\pgfpathlineto{\pgfqpoint{2.438634in}{1.564835in}}%
\pgfpathlineto{\pgfqpoint{2.418212in}{1.517491in}}%
\pgfpathlineto{\pgfqpoint{2.400842in}{1.444012in}}%
\pgfpathlineto{\pgfqpoint{2.382299in}{1.388980in}}%
\pgfpathlineto{\pgfqpoint{2.360000in}{1.352827in}}%
\pgfpathlineto{\pgfqpoint{2.341456in}{1.348765in}}%
\pgfpathlineto{\pgfqpoint{2.319860in}{1.372888in}}%
\pgfpathlineto{\pgfqpoint{2.302255in}{1.425164in}}%
\pgfpathlineto{\pgfqpoint{2.263760in}{1.616232in}}%
\pgfpathlineto{\pgfqpoint{2.244278in}{1.724288in}}%
\pgfpathlineto{\pgfqpoint{2.226204in}{1.739110in}}%
\pgfpathlineto{\pgfqpoint{2.207894in}{1.702038in}}%
\pgfpathlineto{\pgfqpoint{2.186769in}{1.585001in}}%
\pgfpathlineto{\pgfqpoint{2.168225in}{1.475269in}}%
\pgfpathlineto{\pgfqpoint{2.149448in}{1.413008in}}%
\pgfpathlineto{\pgfqpoint{2.130435in}{1.369441in}}%
\pgfpathlineto{\pgfqpoint{2.112359in}{1.352790in}}%
\pgfpathlineto{\pgfqpoint{2.090765in}{1.372622in}}%
\pgfpathlineto{\pgfqpoint{2.071517in}{1.405418in}}%
\pgfpathlineto{\pgfqpoint{2.052973in}{1.457428in}}%
\pgfpathlineto{\pgfqpoint{2.013774in}{1.693772in}}%
\pgfpathlineto{\pgfqpoint{1.994290in}{1.740984in}}%
\pgfpathlineto{\pgfqpoint{1.976217in}{1.722214in}}%
\pgfpathlineto{\pgfqpoint{1.958143in}{1.660446in}}%
\pgfpathlineto{\pgfqpoint{1.938190in}{1.560053in}}%
\pgfpathlineto{\pgfqpoint{1.920351in}{1.482010in}}%
\pgfpathlineto{\pgfqpoint{1.898757in}{1.404018in}}%
\pgfpathlineto{\pgfqpoint{1.879978in}{1.373323in}}%
\pgfpathlineto{\pgfqpoint{1.862373in}{1.359123in}}%
\pgfpathlineto{\pgfqpoint{1.842891in}{1.354552in}}%
\pgfpathlineto{\pgfqpoint{1.821061in}{1.388208in}}%
\pgfpathlineto{\pgfqpoint{1.804865in}{1.437818in}}%
\pgfpathlineto{\pgfqpoint{1.785852in}{1.526822in}}%
\pgfpathlineto{\pgfqpoint{1.765430in}{1.658847in}}%
\pgfpathlineto{\pgfqpoint{1.746885in}{1.733075in}}%
\pgfpathlineto{\pgfqpoint{1.724821in}{1.744291in}}%
\pgfpathlineto{\pgfqpoint{1.687265in}{1.637804in}}%
\pgfpathlineto{\pgfqpoint{1.669191in}{1.540818in}}%
\pgfpathlineto{\pgfqpoint{1.647361in}{1.457986in}}%
\pgfpathlineto{\pgfqpoint{1.631634in}{1.412962in}}%
\pgfpathlineto{\pgfqpoint{1.607223in}{1.367201in}}%
\pgfpathlineto{\pgfqpoint{1.591027in}{1.355799in}}%
\pgfpathlineto{\pgfqpoint{1.574594in}{1.370821in}}%
\pgfpathlineto{\pgfqpoint{1.554643in}{1.407544in}}%
\pgfpathlineto{\pgfqpoint{1.535395in}{1.482944in}}%
\pgfpathlineto{\pgfqpoint{1.511922in}{1.555303in}}%
\pgfpathlineto{\pgfqpoint{1.495726in}{1.646930in}}%
\pgfpathlineto{\pgfqpoint{1.477416in}{1.728477in}}%
\pgfpathlineto{\pgfqpoint{1.454883in}{1.751220in}}%
\pgfpathlineto{\pgfqpoint{1.437043in}{1.739787in}}%
\pgfpathlineto{\pgfqpoint{1.421787in}{1.684414in}}%
\pgfpathlineto{\pgfqpoint{1.400660in}{1.617272in}}%
\pgfpathlineto{\pgfqpoint{1.380943in}{1.679689in}}%
\pgfpathlineto{\pgfqpoint{1.362870in}{1.573367in}}%
\pgfpathlineto{\pgfqpoint{1.342213in}{1.534149in}}%
\pgfpathlineto{\pgfqpoint{1.319914in}{1.457997in}}%
\pgfpathlineto{\pgfqpoint{1.301370in}{1.400417in}}%
\pgfpathlineto{\pgfqpoint{1.286582in}{1.369934in}}%
\pgfpathlineto{\pgfqpoint{1.264517in}{1.360674in}}%
\pgfpathlineto{\pgfqpoint{1.245739in}{1.379898in}}%
\pgfpathlineto{\pgfqpoint{1.226960in}{1.417413in}}%
\pgfpathlineto{\pgfqpoint{1.208417in}{1.467898in}}%
\pgfpathlineto{\pgfqpoint{1.189873in}{1.566976in}}%
\pgfpathlineto{\pgfqpoint{1.167105in}{1.699796in}}%
\pgfpathlineto{\pgfqpoint{1.152317in}{1.744495in}}%
\pgfpathlineto{\pgfqpoint{1.129784in}{1.758132in}}%
\pgfpathlineto{\pgfqpoint{1.112413in}{1.747633in}}%
\pgfpathlineto{\pgfqpoint{1.090583in}{1.670135in}}%
\pgfpathlineto{\pgfqpoint{1.072039in}{1.558608in}}%
\pgfpathlineto{\pgfqpoint{1.053965in}{1.488141in}}%
\pgfpathlineto{\pgfqpoint{1.034952in}{1.430094in}}%
\pgfpathlineto{\pgfqpoint{1.016878in}{1.385931in}}%
\pgfpathlineto{\pgfqpoint{0.995282in}{1.364432in}}%
\pgfpathlineto{\pgfqpoint{0.978617in}{1.384664in}}%
\pgfpathlineto{\pgfqpoint{0.957727in}{1.435580in}}%
\pgfpathlineto{\pgfqpoint{0.939417in}{1.500761in}}%
\pgfpathlineto{\pgfqpoint{0.918057in}{1.595976in}}%
\pgfpathlineto{\pgfqpoint{0.899513in}{1.697235in}}%
\pgfpathlineto{\pgfqpoint{0.880734in}{1.754000in}}%
\pgfpathlineto{\pgfqpoint{0.863131in}{1.766067in}}%
\pgfpathlineto{\pgfqpoint{0.844118in}{1.749618in}}%
\pgfpathlineto{\pgfqpoint{0.822522in}{1.422965in}}%
\pgfpathlineto{\pgfqpoint{0.803978in}{1.504203in}}%
\pgfpathlineto{\pgfqpoint{0.785201in}{1.624983in}}%
\pgfpathlineto{\pgfqpoint{0.763839in}{1.734853in}}%
\pgfpathlineto{\pgfqpoint{0.748112in}{1.765599in}}%
\pgfpathlineto{\pgfqpoint{0.726284in}{1.749994in}}%
\pgfpathlineto{\pgfqpoint{0.708208in}{1.687150in}}%
\pgfpathlineto{\pgfqpoint{0.689431in}{1.578530in}}%
\pgfpathlineto{\pgfqpoint{0.667367in}{1.469104in}}%
\pgfpathlineto{\pgfqpoint{0.649525in}{1.410910in}}%
\pgfpathlineto{\pgfqpoint{0.650934in}{1.413632in}}%
\pgfpathlineto{\pgfqpoint{0.677224in}{1.544734in}}%
\pgfpathlineto{\pgfqpoint{0.696237in}{1.692589in}}%
\pgfpathlineto{\pgfqpoint{0.713842in}{1.761706in}}%
\pgfpathlineto{\pgfqpoint{0.731916in}{1.751946in}}%
\pgfpathlineto{\pgfqpoint{0.751634in}{1.646499in}}%
\pgfpathlineto{\pgfqpoint{0.772759in}{1.489598in}}%
\pgfpathlineto{\pgfqpoint{0.790833in}{1.403391in}}%
\pgfpathlineto{\pgfqpoint{0.808203in}{1.427716in}}%
\pgfpathlineto{\pgfqpoint{0.828156in}{1.542286in}}%
\pgfpathlineto{\pgfqpoint{0.848107in}{1.682724in}}%
\pgfpathlineto{\pgfqpoint{0.869234in}{1.758529in}}%
\pgfpathlineto{\pgfqpoint{0.887307in}{1.734275in}}%
\pgfpathlineto{\pgfqpoint{0.906086in}{1.614398in}}%
\pgfpathlineto{\pgfqpoint{0.926037in}{1.452906in}}%
\pgfpathlineto{\pgfqpoint{0.944816in}{1.381172in}}%
\pgfpathlineto{\pgfqpoint{0.963124in}{1.361322in}}%
\pgfpathlineto{\pgfqpoint{0.984486in}{1.417791in}}%
\pgfpathlineto{\pgfqpoint{1.003968in}{1.508112in}}%
\pgfpathlineto{\pgfqpoint{1.023215in}{1.649807in}}%
\pgfpathlineto{\pgfqpoint{1.038708in}{1.734705in}}%
\pgfpathlineto{\pgfqpoint{1.060537in}{1.744735in}}%
\pgfpathlineto{\pgfqpoint{1.080021in}{1.658762in}}%
\pgfpathlineto{\pgfqpoint{1.097860in}{1.519810in}}%
\pgfpathlineto{\pgfqpoint{1.118282in}{1.412436in}}%
\pgfpathlineto{\pgfqpoint{1.137059in}{1.363145in}}%
\pgfpathlineto{\pgfqpoint{1.156777in}{1.365690in}}%
\pgfpathlineto{\pgfqpoint{1.175790in}{1.414704in}}%
\pgfpathlineto{\pgfqpoint{1.196212in}{1.488247in}}%
\pgfpathlineto{\pgfqpoint{1.214520in}{1.611356in}}%
\pgfpathlineto{\pgfqpoint{1.230716in}{1.711160in}}%
\pgfpathlineto{\pgfqpoint{1.252312in}{1.748761in}}%
\pgfpathlineto{\pgfqpoint{1.271560in}{1.728353in}}%
\pgfpathlineto{\pgfqpoint{1.291042in}{1.626783in}}%
\pgfpathlineto{\pgfqpoint{1.310055in}{1.485872in}}%
\pgfpathlineto{\pgfqpoint{1.328834in}{1.408967in}}%
\pgfpathlineto{\pgfqpoint{1.347847in}{1.360893in}}%
\pgfpathlineto{\pgfqpoint{1.366390in}{1.354998in}}%
\pgfpathlineto{\pgfqpoint{1.384934in}{1.392245in}}%
\pgfpathlineto{\pgfqpoint{1.405356in}{1.455286in}}%
\pgfpathlineto{\pgfqpoint{1.427420in}{1.563329in}}%
\pgfpathlineto{\pgfqpoint{1.445963in}{1.690121in}}%
\pgfpathlineto{\pgfqpoint{1.462864in}{1.737562in}}%
\pgfpathlineto{\pgfqpoint{1.481407in}{1.736203in}}%
\pgfpathlineto{\pgfqpoint{1.502063in}{1.654453in}}%
\pgfpathlineto{\pgfqpoint{1.520608in}{1.541195in}}%
\pgfpathlineto{\pgfqpoint{1.541264in}{1.434397in}}%
\pgfpathlineto{\pgfqpoint{1.560043in}{1.375674in}}%
\pgfpathlineto{\pgfqpoint{1.579290in}{1.356013in}}%
\pgfpathlineto{\pgfqpoint{1.598067in}{1.359593in}}%
\pgfpathlineto{\pgfqpoint{1.617551in}{1.396772in}}%
\pgfpathlineto{\pgfqpoint{1.635625in}{1.460664in}}%
\pgfpathlineto{\pgfqpoint{1.659332in}{1.555167in}}%
\pgfpathlineto{\pgfqpoint{1.675060in}{1.594660in}}%
\pgfpathlineto{\pgfqpoint{1.696654in}{1.448539in}}%
\pgfpathlineto{\pgfqpoint{1.712381in}{1.391180in}}%
\pgfpathlineto{\pgfqpoint{1.733037in}{1.351840in}}%
\pgfpathlineto{\pgfqpoint{1.751582in}{1.370881in}}%
\pgfpathlineto{\pgfqpoint{1.770359in}{1.429280in}}%
\pgfpathlineto{\pgfqpoint{1.789843in}{1.509822in}}%
\pgfpathlineto{\pgfqpoint{1.808151in}{1.635976in}}%
\pgfpathlineto{\pgfqpoint{1.831155in}{1.732433in}}%
\pgfpathlineto{\pgfqpoint{1.848994in}{1.733223in}}%
\pgfpathlineto{\pgfqpoint{1.868476in}{1.650643in}}%
\pgfpathlineto{\pgfqpoint{1.887021in}{1.528695in}}%
\pgfpathlineto{\pgfqpoint{1.908614in}{1.414092in}}%
\pgfpathlineto{\pgfqpoint{1.925280in}{1.367156in}}%
\pgfpathlineto{\pgfqpoint{1.944293in}{1.349621in}}%
\pgfpathlineto{\pgfqpoint{1.962368in}{1.360818in}}%
\pgfpathlineto{\pgfqpoint{1.981145in}{1.407863in}}%
\pgfpathlineto{\pgfqpoint{2.003681in}{1.505324in}}%
\pgfpathlineto{\pgfqpoint{2.020346in}{1.625137in}}%
\pgfpathlineto{\pgfqpoint{2.039359in}{1.705414in}}%
\pgfpathlineto{\pgfqpoint{2.061658in}{1.738319in}}%
\pgfpathlineto{\pgfqpoint{2.077620in}{1.700418in}}%
\pgfpathlineto{\pgfqpoint{2.098745in}{1.561225in}}%
\pgfpathlineto{\pgfqpoint{2.113533in}{1.453307in}}%
\pgfpathlineto{\pgfqpoint{2.136303in}{1.388356in}}%
\pgfpathlineto{\pgfqpoint{2.157662in}{1.350798in}}%
\pgfpathlineto{\pgfqpoint{2.175736in}{1.362770in}}%
\pgfpathlineto{\pgfqpoint{2.194046in}{1.407326in}}%
\pgfpathlineto{\pgfqpoint{2.212354in}{1.475676in}}%
\pgfpathlineto{\pgfqpoint{2.233010in}{1.600611in}}%
\pgfpathlineto{\pgfqpoint{2.251086in}{1.702835in}}%
\pgfpathlineto{\pgfqpoint{2.268690in}{1.738802in}}%
\pgfpathlineto{\pgfqpoint{2.286530in}{1.724600in}}%
\pgfpathlineto{\pgfqpoint{2.308360in}{1.645537in}}%
\pgfpathlineto{\pgfqpoint{2.329485in}{1.498654in}}%
\pgfpathlineto{\pgfqpoint{2.347324in}{1.424984in}}%
\pgfpathlineto{\pgfqpoint{2.364929in}{1.373811in}}%
\pgfpathlineto{\pgfqpoint{2.385585in}{1.348137in}}%
\pgfpathlineto{\pgfqpoint{2.404364in}{1.366679in}}%
\pgfpathlineto{\pgfqpoint{2.442859in}{1.466160in}}%
\pgfpathlineto{\pgfqpoint{2.463515in}{1.601593in}}%
\pgfpathlineto{\pgfqpoint{2.481589in}{1.705585in}}%
\pgfpathlineto{\pgfqpoint{2.501776in}{1.736443in}}%
\pgfpathlineto{\pgfqpoint{2.521493in}{1.723430in}}%
\pgfpathlineto{\pgfqpoint{2.538629in}{1.640522in}}%
\pgfpathlineto{\pgfqpoint{2.559754in}{1.484084in}}%
\pgfpathlineto{\pgfqpoint{2.578298in}{1.417689in}}%
\pgfpathlineto{\pgfqpoint{2.599189in}{1.364382in}}%
\pgfpathlineto{\pgfqpoint{2.635336in}{1.352571in}}%
\pgfpathlineto{\pgfqpoint{2.655523in}{1.389612in}}%
\pgfpathlineto{\pgfqpoint{2.673362in}{1.435950in}}%
\pgfpathlineto{\pgfqpoint{2.694489in}{1.538175in}}%
\pgfpathlineto{\pgfqpoint{2.715849in}{1.669498in}}%
\pgfpathlineto{\pgfqpoint{2.731342in}{1.728002in}}%
\pgfpathlineto{\pgfqpoint{2.749181in}{1.736389in}}%
\pgfpathlineto{\pgfqpoint{2.769132in}{1.692881in}}%
\pgfpathlineto{\pgfqpoint{2.790962in}{1.556575in}}%
\pgfpathlineto{\pgfqpoint{2.808567in}{1.490319in}}%
\pgfpathlineto{\pgfqpoint{2.825703in}{1.413595in}}%
\pgfpathlineto{\pgfqpoint{2.847533in}{1.364888in}}%
\pgfpathlineto{\pgfqpoint{2.865607in}{1.350327in}}%
\pgfpathlineto{\pgfqpoint{2.886497in}{1.381982in}}%
\pgfpathlineto{\pgfqpoint{2.903397in}{1.404567in}}%
\pgfpathlineto{\pgfqpoint{2.925227in}{1.470104in}}%
\pgfpathlineto{\pgfqpoint{2.943066in}{1.569167in}}%
\pgfpathlineto{\pgfqpoint{2.960437in}{1.678627in}}%
\pgfpathlineto{\pgfqpoint{2.981562in}{1.738291in}}%
\pgfpathlineto{\pgfqpoint{2.999872in}{1.737939in}}%
\pgfpathlineto{\pgfqpoint{3.020762in}{1.678812in}}%
\pgfpathlineto{\pgfqpoint{3.039307in}{1.558553in}}%
\pgfpathlineto{\pgfqpoint{3.059963in}{1.445108in}}%
\pgfpathlineto{\pgfqpoint{3.078271in}{1.394278in}}%
\pgfpathlineto{\pgfqpoint{3.096345in}{1.363451in}}%
\pgfpathlineto{\pgfqpoint{3.116766in}{1.354866in}}%
\pgfpathlineto{\pgfqpoint{3.135076in}{1.383439in}}%
\pgfpathlineto{\pgfqpoint{3.152915in}{1.425452in}}%
\pgfpathlineto{\pgfqpoint{3.173572in}{1.516268in}}%
\pgfpathlineto{\pgfqpoint{3.212770in}{1.704070in}}%
\pgfpathlineto{\pgfqpoint{3.229672in}{1.740274in}}%
\pgfpathlineto{\pgfqpoint{3.252205in}{1.737359in}}%
\pgfpathlineto{\pgfqpoint{3.268167in}{1.705652in}}%
\pgfpathlineto{\pgfqpoint{3.287649in}{1.593941in}}%
\pgfpathlineto{\pgfqpoint{3.308305in}{1.500901in}}%
\pgfpathlineto{\pgfqpoint{3.326381in}{1.433147in}}%
\pgfpathlineto{\pgfqpoint{3.347272in}{1.377831in}}%
\pgfpathlineto{\pgfqpoint{3.365111in}{1.355938in}}%
\pgfpathlineto{\pgfqpoint{3.384358in}{1.363825in}}%
\pgfpathlineto{\pgfqpoint{3.402901in}{1.401399in}}%
\pgfpathlineto{\pgfqpoint{3.425436in}{1.468561in}}%
\pgfpathlineto{\pgfqpoint{3.442336in}{1.545264in}}%
\pgfpathlineto{\pgfqpoint{3.463697in}{1.655184in}}%
\pgfpathlineto{\pgfqpoint{3.481771in}{1.718523in}}%
\pgfpathlineto{\pgfqpoint{3.500315in}{1.749360in}}%
\pgfpathlineto{\pgfqpoint{3.520971in}{1.736669in}}%
\pgfpathlineto{\pgfqpoint{3.538340in}{1.677542in}}%
\pgfpathlineto{\pgfqpoint{3.556415in}{1.599179in}}%
\pgfpathlineto{\pgfqpoint{3.575898in}{1.501949in}}%
\pgfpathlineto{\pgfqpoint{3.595614in}{1.442393in}}%
\pgfpathlineto{\pgfqpoint{3.616036in}{1.387449in}}%
\pgfpathlineto{\pgfqpoint{3.634580in}{1.359478in}}%
\pgfpathlineto{\pgfqpoint{3.656174in}{1.366868in}}%
\pgfpathlineto{\pgfqpoint{3.673544in}{1.399890in}}%
\pgfpathlineto{\pgfqpoint{3.693027in}{1.417338in}}%
\pgfpathlineto{\pgfqpoint{3.712745in}{1.470034in}}%
\pgfpathlineto{\pgfqpoint{3.731287in}{1.529823in}}%
\pgfpathlineto{\pgfqpoint{3.768845in}{1.717869in}}%
\pgfpathlineto{\pgfqpoint{3.790441in}{1.752590in}}%
\pgfpathlineto{\pgfqpoint{3.808983in}{1.756370in}}%
\pgfpathlineto{\pgfqpoint{3.826354in}{1.752489in}}%
\pgfpathlineto{\pgfqpoint{3.845133in}{1.732976in}}%
\pgfpathlineto{\pgfqpoint{3.865318in}{1.668777in}}%
\pgfpathlineto{\pgfqpoint{3.886445in}{1.531953in}}%
\pgfpathlineto{\pgfqpoint{3.904050in}{1.451570in}}%
\pgfpathlineto{\pgfqpoint{3.923063in}{1.397920in}}%
\pgfpathlineto{\pgfqpoint{3.944422in}{1.367620in}}%
\pgfpathlineto{\pgfqpoint{3.960384in}{1.362351in}}%
\pgfpathlineto{\pgfqpoint{3.981980in}{1.394311in}}%
\pgfpathlineto{\pgfqpoint{3.998880in}{1.431174in}}%
\pgfpathlineto{\pgfqpoint{4.022118in}{1.508405in}}%
\pgfpathlineto{\pgfqpoint{4.039489in}{1.608210in}}%
\pgfpathlineto{\pgfqpoint{4.058266in}{1.695743in}}%
\pgfpathlineto{\pgfqpoint{4.075167in}{1.747957in}}%
\pgfpathlineto{\pgfqpoint{4.096997in}{1.763995in}}%
\pgfpathlineto{\pgfqpoint{4.115071in}{1.757196in}}%
\pgfpathlineto{\pgfqpoint{4.135024in}{1.713158in}}%
\pgfpathlineto{\pgfqpoint{4.154740in}{1.577591in}}%
\pgfpathlineto{\pgfqpoint{4.172580in}{1.701798in}}%
\pgfpathlineto{\pgfqpoint{4.193705in}{1.762824in}}%
\pgfpathlineto{\pgfqpoint{4.211780in}{1.755664in}}%
\pgfpathlineto{\pgfqpoint{4.229854in}{1.698858in}}%
\pgfpathlineto{\pgfqpoint{4.249805in}{1.573159in}}%
\pgfpathlineto{\pgfqpoint{4.267880in}{1.486688in}}%
\pgfpathlineto{\pgfqpoint{4.289005in}{1.410648in}}%
\pgfpathlineto{\pgfqpoint{4.307079in}{1.375551in}}%
\pgfpathlineto{\pgfqpoint{4.328440in}{1.377539in}}%
\pgfpathlineto{\pgfqpoint{4.344871in}{1.408635in}}%
\pgfpathlineto{\pgfqpoint{4.365762in}{1.464329in}}%
\pgfpathlineto{\pgfqpoint{4.384540in}{1.542459in}}%
\pgfpathlineto{\pgfqpoint{4.404023in}{1.653118in}}%
\pgfpathlineto{\pgfqpoint{4.423505in}{1.741998in}}%
\pgfpathlineto{\pgfqpoint{4.443223in}{1.772910in}}%
\pgfpathlineto{\pgfqpoint{4.462705in}{1.760711in}}%
\pgfpathlineto{\pgfqpoint{4.481718in}{1.696425in}}%
\pgfpathlineto{\pgfqpoint{4.482187in}{1.699363in}}%
\pgfpathlineto{\pgfqpoint{4.472564in}{1.744488in}}%
\pgfpathlineto{\pgfqpoint{4.453551in}{1.772896in}}%
\pgfpathlineto{\pgfqpoint{4.435476in}{1.727785in}}%
\pgfpathlineto{\pgfqpoint{4.416933in}{1.594706in}}%
\pgfpathlineto{\pgfqpoint{4.399094in}{1.474852in}}%
\pgfpathlineto{\pgfqpoint{4.377029in}{1.390158in}}%
\pgfpathlineto{\pgfqpoint{4.358485in}{1.369295in}}%
\pgfpathlineto{\pgfqpoint{4.338298in}{1.452728in}}%
\pgfpathlineto{\pgfqpoint{4.318347in}{1.581939in}}%
\pgfpathlineto{\pgfqpoint{4.302150in}{1.701595in}}%
\pgfpathlineto{\pgfqpoint{4.281260in}{1.765153in}}%
\pgfpathlineto{\pgfqpoint{4.262715in}{1.735757in}}%
\pgfpathlineto{\pgfqpoint{4.226097in}{1.487841in}}%
\pgfpathlineto{\pgfqpoint{4.204738in}{1.396259in}}%
\pgfpathlineto{\pgfqpoint{4.186193in}{1.363739in}}%
\pgfpathlineto{\pgfqpoint{4.167415in}{1.390929in}}%
\pgfpathlineto{\pgfqpoint{4.146290in}{1.482364in}}%
\pgfpathlineto{\pgfqpoint{4.128216in}{1.614653in}}%
\pgfpathlineto{\pgfqpoint{4.110611in}{1.727371in}}%
\pgfpathlineto{\pgfqpoint{4.089955in}{1.732580in}}%
\pgfpathlineto{\pgfqpoint{4.072585in}{1.757365in}}%
\pgfpathlineto{\pgfqpoint{4.050051in}{1.679618in}}%
\pgfpathlineto{\pgfqpoint{4.032915in}{1.547738in}}%
\pgfpathlineto{\pgfqpoint{4.013668in}{1.442785in}}%
\pgfpathlineto{\pgfqpoint{3.993011in}{1.374853in}}%
\pgfpathlineto{\pgfqpoint{3.973764in}{1.363617in}}%
\pgfpathlineto{\pgfqpoint{3.955219in}{1.410058in}}%
\pgfpathlineto{\pgfqpoint{3.938554in}{1.490900in}}%
\pgfpathlineto{\pgfqpoint{3.917429in}{1.640454in}}%
\pgfpathlineto{\pgfqpoint{3.897007in}{1.739918in}}%
\pgfpathlineto{\pgfqpoint{3.879168in}{1.746344in}}%
\pgfpathlineto{\pgfqpoint{3.859215in}{1.661485in}}%
\pgfpathlineto{\pgfqpoint{3.838794in}{1.512750in}}%
\pgfpathlineto{\pgfqpoint{3.822363in}{1.432842in}}%
\pgfpathlineto{\pgfqpoint{3.801707in}{1.365940in}}%
\pgfpathlineto{\pgfqpoint{3.783399in}{1.355377in}}%
\pgfpathlineto{\pgfqpoint{3.763917in}{1.394910in}}%
\pgfpathlineto{\pgfqpoint{3.742790in}{1.483582in}}%
\pgfpathlineto{\pgfqpoint{3.725185in}{1.607146in}}%
\pgfpathlineto{\pgfqpoint{3.705234in}{1.726154in}}%
\pgfpathlineto{\pgfqpoint{3.686689in}{1.747973in}}%
\pgfpathlineto{\pgfqpoint{3.666502in}{1.699089in}}%
\pgfpathlineto{\pgfqpoint{3.649134in}{1.582932in}}%
\pgfpathlineto{\pgfqpoint{3.628241in}{1.453030in}}%
\pgfpathlineto{\pgfqpoint{3.607351in}{1.388548in}}%
\pgfpathlineto{\pgfqpoint{3.590217in}{1.710874in}}%
\pgfpathlineto{\pgfqpoint{3.551250in}{1.469683in}}%
\pgfpathlineto{\pgfqpoint{3.534585in}{1.401927in}}%
\pgfpathlineto{\pgfqpoint{3.514398in}{1.355799in}}%
\pgfpathlineto{\pgfqpoint{3.493039in}{1.368143in}}%
\pgfpathlineto{\pgfqpoint{3.476137in}{1.395837in}}%
\pgfpathlineto{\pgfqpoint{3.457829in}{1.471375in}}%
\pgfpathlineto{\pgfqpoint{3.437173in}{1.612406in}}%
\pgfpathlineto{\pgfqpoint{3.417691in}{1.722870in}}%
\pgfpathlineto{\pgfqpoint{3.398677in}{1.740062in}}%
\pgfpathlineto{\pgfqpoint{3.379193in}{1.671165in}}%
\pgfpathlineto{\pgfqpoint{3.360417in}{1.555832in}}%
\pgfpathlineto{\pgfqpoint{3.340464in}{1.456389in}}%
\pgfpathlineto{\pgfqpoint{3.317930in}{1.385567in}}%
\pgfpathlineto{\pgfqpoint{3.301968in}{1.351835in}}%
\pgfpathlineto{\pgfqpoint{3.282015in}{1.365398in}}%
\pgfpathlineto{\pgfqpoint{3.264647in}{1.407052in}}%
\pgfpathlineto{\pgfqpoint{3.246337in}{1.454953in}}%
\pgfpathlineto{\pgfqpoint{3.205730in}{1.710315in}}%
\pgfpathlineto{\pgfqpoint{3.188594in}{1.741105in}}%
\pgfpathlineto{\pgfqpoint{3.166060in}{1.699975in}}%
\pgfpathlineto{\pgfqpoint{3.148690in}{1.597290in}}%
\pgfpathlineto{\pgfqpoint{3.131085in}{1.484135in}}%
\pgfpathlineto{\pgfqpoint{3.110898in}{1.405956in}}%
\pgfpathlineto{\pgfqpoint{3.090713in}{1.355996in}}%
\pgfpathlineto{\pgfqpoint{3.070994in}{1.357778in}}%
\pgfpathlineto{\pgfqpoint{3.053155in}{1.394872in}}%
\pgfpathlineto{\pgfqpoint{3.035316in}{1.463791in}}%
\pgfpathlineto{\pgfqpoint{2.994003in}{1.692795in}}%
\pgfpathlineto{\pgfqpoint{2.976399in}{1.737701in}}%
\pgfpathlineto{\pgfqpoint{2.954803in}{1.716895in}}%
\pgfpathlineto{\pgfqpoint{2.937669in}{1.665612in}}%
\pgfpathlineto{\pgfqpoint{2.918421in}{1.544428in}}%
\pgfpathlineto{\pgfqpoint{2.900816in}{1.457874in}}%
\pgfpathlineto{\pgfqpoint{2.878986in}{1.385904in}}%
\pgfpathlineto{\pgfqpoint{2.860207in}{1.354155in}}%
\pgfpathlineto{\pgfqpoint{2.841194in}{1.362469in}}%
\pgfpathlineto{\pgfqpoint{2.821009in}{1.410689in}}%
\pgfpathlineto{\pgfqpoint{2.781808in}{1.574492in}}%
\pgfpathlineto{\pgfqpoint{2.763498in}{1.692630in}}%
\pgfpathlineto{\pgfqpoint{2.743313in}{1.737850in}}%
\pgfpathlineto{\pgfqpoint{2.726646in}{1.727088in}}%
\pgfpathlineto{\pgfqpoint{2.707869in}{1.665816in}}%
\pgfpathlineto{\pgfqpoint{2.685570in}{1.534696in}}%
\pgfpathlineto{\pgfqpoint{2.667494in}{1.446254in}}%
\pgfpathlineto{\pgfqpoint{2.651769in}{1.398595in}}%
\pgfpathlineto{\pgfqpoint{2.630173in}{1.354270in}}%
\pgfpathlineto{\pgfqpoint{2.611160in}{1.356612in}}%
\pgfpathlineto{\pgfqpoint{2.570553in}{1.428025in}}%
\pgfpathlineto{\pgfqpoint{2.552243in}{1.519052in}}%
\pgfpathlineto{\pgfqpoint{2.533229in}{1.640996in}}%
\pgfpathlineto{\pgfqpoint{2.514450in}{1.722920in}}%
\pgfpathlineto{\pgfqpoint{2.496142in}{1.735241in}}%
\pgfpathlineto{\pgfqpoint{2.474312in}{1.673570in}}%
\pgfpathlineto{\pgfqpoint{2.455299in}{1.587253in}}%
\pgfpathlineto{\pgfqpoint{2.436756in}{1.486521in}}%
\pgfpathlineto{\pgfqpoint{2.399904in}{1.386340in}}%
\pgfpathlineto{\pgfqpoint{2.375257in}{1.676378in}}%
\pgfpathlineto{\pgfqpoint{2.358826in}{1.568555in}}%
\pgfpathlineto{\pgfqpoint{2.340987in}{1.459249in}}%
\pgfpathlineto{\pgfqpoint{2.319626in}{1.387710in}}%
\pgfpathlineto{\pgfqpoint{2.303664in}{1.354878in}}%
\pgfpathlineto{\pgfqpoint{2.285121in}{1.355825in}}%
\pgfpathlineto{\pgfqpoint{2.264231in}{1.393991in}}%
\pgfpathlineto{\pgfqpoint{2.244512in}{1.465390in}}%
\pgfpathlineto{\pgfqpoint{2.203669in}{1.713902in}}%
\pgfpathlineto{\pgfqpoint{2.188646in}{1.739822in}}%
\pgfpathlineto{\pgfqpoint{2.170104in}{1.715139in}}%
\pgfpathlineto{\pgfqpoint{2.148743in}{1.609521in}}%
\pgfpathlineto{\pgfqpoint{2.132781in}{1.523021in}}%
\pgfpathlineto{\pgfqpoint{2.110482in}{1.430349in}}%
\pgfpathlineto{\pgfqpoint{2.088417in}{1.370719in}}%
\pgfpathlineto{\pgfqpoint{2.070812in}{1.353197in}}%
\pgfpathlineto{\pgfqpoint{2.052035in}{1.367090in}}%
\pgfpathlineto{\pgfqpoint{2.033491in}{1.415513in}}%
\pgfpathlineto{\pgfqpoint{2.014946in}{1.455136in}}%
\pgfpathlineto{\pgfqpoint{1.977391in}{1.687593in}}%
\pgfpathlineto{\pgfqpoint{1.956029in}{1.741488in}}%
\pgfpathlineto{\pgfqpoint{1.939364in}{1.733338in}}%
\pgfpathlineto{\pgfqpoint{1.917534in}{1.642496in}}%
\pgfpathlineto{\pgfqpoint{1.899695in}{1.542346in}}%
\pgfpathlineto{\pgfqpoint{1.879039in}{1.443413in}}%
\pgfpathlineto{\pgfqpoint{1.860496in}{1.386480in}}%
\pgfpathlineto{\pgfqpoint{1.841717in}{1.354285in}}%
\pgfpathlineto{\pgfqpoint{1.822704in}{1.364322in}}%
\pgfpathlineto{\pgfqpoint{1.803925in}{1.401192in}}%
\pgfpathlineto{\pgfqpoint{1.784912in}{1.458602in}}%
\pgfpathlineto{\pgfqpoint{1.763787in}{1.545646in}}%
\pgfpathlineto{\pgfqpoint{1.745713in}{1.660581in}}%
\pgfpathlineto{\pgfqpoint{1.726935in}{1.728571in}}%
\pgfpathlineto{\pgfqpoint{1.708390in}{1.744578in}}%
\pgfpathlineto{\pgfqpoint{1.689613in}{1.727196in}}%
\pgfpathlineto{\pgfqpoint{1.667783in}{1.636767in}}%
\pgfpathlineto{\pgfqpoint{1.649239in}{1.541710in}}%
\pgfpathlineto{\pgfqpoint{1.631165in}{1.459918in}}%
\pgfpathlineto{\pgfqpoint{1.610274in}{1.398457in}}%
\pgfpathlineto{\pgfqpoint{1.592670in}{1.362813in}}%
\pgfpathlineto{\pgfqpoint{1.572482in}{1.361589in}}%
\pgfpathlineto{\pgfqpoint{1.553938in}{1.391914in}}%
\pgfpathlineto{\pgfqpoint{1.535161in}{1.436954in}}%
\pgfpathlineto{\pgfqpoint{1.515208in}{1.491050in}}%
\pgfpathlineto{\pgfqpoint{1.495021in}{1.579464in}}%
\pgfpathlineto{\pgfqpoint{1.476478in}{1.693935in}}%
\pgfpathlineto{\pgfqpoint{1.457934in}{1.743131in}}%
\pgfpathlineto{\pgfqpoint{1.437043in}{1.747606in}}%
\pgfpathlineto{\pgfqpoint{1.415684in}{1.695103in}}%
\pgfpathlineto{\pgfqpoint{1.399251in}{1.605765in}}%
\pgfpathlineto{\pgfqpoint{1.379066in}{1.502658in}}%
\pgfpathlineto{\pgfqpoint{1.360756in}{1.467984in}}%
\pgfpathlineto{\pgfqpoint{1.343387in}{1.425444in}}%
\pgfpathlineto{\pgfqpoint{1.321792in}{1.378103in}}%
\pgfpathlineto{\pgfqpoint{1.303482in}{1.357954in}}%
\pgfpathlineto{\pgfqpoint{1.284939in}{1.367247in}}%
\pgfpathlineto{\pgfqpoint{1.264049in}{1.410974in}}%
\pgfpathlineto{\pgfqpoint{1.245739in}{1.469417in}}%
\pgfpathlineto{\pgfqpoint{1.227899in}{1.532283in}}%
\pgfpathlineto{\pgfqpoint{1.209357in}{1.642685in}}%
\pgfpathlineto{\pgfqpoint{1.187056in}{1.733378in}}%
\pgfpathlineto{\pgfqpoint{1.168279in}{1.755611in}}%
\pgfpathlineto{\pgfqpoint{1.148795in}{1.752259in}}%
\pgfpathlineto{\pgfqpoint{1.129313in}{1.710448in}}%
\pgfpathlineto{\pgfqpoint{1.110534in}{1.754630in}}%
\pgfpathlineto{\pgfqpoint{1.088940in}{1.697197in}}%
\pgfpathlineto{\pgfqpoint{1.054670in}{1.522615in}}%
\pgfpathlineto{\pgfqpoint{1.032840in}{1.438617in}}%
\pgfpathlineto{\pgfqpoint{1.014530in}{1.406555in}}%
\pgfpathlineto{\pgfqpoint{0.995988in}{1.369050in}}%
\pgfpathlineto{\pgfqpoint{0.977443in}{1.368099in}}%
\pgfpathlineto{\pgfqpoint{0.955144in}{1.409451in}}%
\pgfpathlineto{\pgfqpoint{0.936365in}{1.474925in}}%
\pgfpathlineto{\pgfqpoint{0.917823in}{1.567289in}}%
\pgfpathlineto{\pgfqpoint{0.899747in}{1.665284in}}%
\pgfpathlineto{\pgfqpoint{0.881674in}{1.737941in}}%
\pgfpathlineto{\pgfqpoint{0.862192in}{1.763619in}}%
\pgfpathlineto{\pgfqpoint{0.840596in}{1.752036in}}%
\pgfpathlineto{\pgfqpoint{0.822757in}{1.716037in}}%
\pgfpathlineto{\pgfqpoint{0.804212in}{1.613590in}}%
\pgfpathlineto{\pgfqpoint{0.786373in}{1.530603in}}%
\pgfpathlineto{\pgfqpoint{0.764308in}{1.453481in}}%
\pgfpathlineto{\pgfqpoint{0.745766in}{1.410077in}}%
\pgfpathlineto{\pgfqpoint{0.727456in}{1.392370in}}%
\pgfpathlineto{\pgfqpoint{0.710087in}{1.371119in}}%
\pgfpathlineto{\pgfqpoint{0.688726in}{1.376640in}}%
\pgfpathlineto{\pgfqpoint{0.665253in}{1.433448in}}%
\pgfpathlineto{\pgfqpoint{0.649762in}{1.495206in}}%
\pgfpathlineto{\pgfqpoint{0.654456in}{1.461072in}}%
\pgfpathlineto{\pgfqpoint{0.676990in}{1.382714in}}%
\pgfpathlineto{\pgfqpoint{0.695063in}{1.375295in}}%
\pgfpathlineto{\pgfqpoint{0.712904in}{1.426433in}}%
\pgfpathlineto{\pgfqpoint{0.734029in}{1.528475in}}%
\pgfpathlineto{\pgfqpoint{0.752572in}{1.659759in}}%
\pgfpathlineto{\pgfqpoint{0.773228in}{1.759083in}}%
\pgfpathlineto{\pgfqpoint{0.792007in}{1.755847in}}%
\pgfpathlineto{\pgfqpoint{0.809846in}{1.674321in}}%
\pgfpathlineto{\pgfqpoint{0.830738in}{1.518007in}}%
\pgfpathlineto{\pgfqpoint{0.830973in}{1.454949in}}%
\pgfpathlineto{\pgfqpoint{0.847169in}{1.429654in}}%
\pgfpathlineto{\pgfqpoint{0.867120in}{1.370053in}}%
\pgfpathlineto{\pgfqpoint{0.888482in}{1.384935in}}%
\pgfpathlineto{\pgfqpoint{0.906790in}{1.448096in}}%
\pgfpathlineto{\pgfqpoint{0.924629in}{1.543185in}}%
\pgfpathlineto{\pgfqpoint{0.945285in}{1.698349in}}%
\pgfpathlineto{\pgfqpoint{0.963595in}{1.757091in}}%
\pgfpathlineto{\pgfqpoint{0.984954in}{1.724948in}}%
\pgfpathlineto{\pgfqpoint{1.005142in}{1.575447in}}%
\pgfpathlineto{\pgfqpoint{1.020633in}{1.470323in}}%
\pgfpathlineto{\pgfqpoint{1.043637in}{1.380444in}}%
\pgfpathlineto{\pgfqpoint{1.060068in}{1.360029in}}%
\pgfpathlineto{\pgfqpoint{1.080021in}{1.402997in}}%
\pgfpathlineto{\pgfqpoint{1.095983in}{1.466344in}}%
\pgfpathlineto{\pgfqpoint{1.118047in}{1.604883in}}%
\pgfpathlineto{\pgfqpoint{1.137529in}{1.709807in}}%
\pgfpathlineto{\pgfqpoint{1.156308in}{1.752990in}}%
\pgfpathlineto{\pgfqpoint{1.172973in}{1.720654in}}%
\pgfpathlineto{\pgfqpoint{1.195272in}{1.567912in}}%
\pgfpathlineto{\pgfqpoint{1.214989in}{1.447386in}}%
\pgfpathlineto{\pgfqpoint{1.234237in}{1.381720in}}%
\pgfpathlineto{\pgfqpoint{1.254424in}{1.354203in}}%
\pgfpathlineto{\pgfqpoint{1.270620in}{1.380252in}}%
\pgfpathlineto{\pgfqpoint{1.291276in}{1.441035in}}%
\pgfpathlineto{\pgfqpoint{1.312167in}{1.548033in}}%
\pgfpathlineto{\pgfqpoint{1.329303in}{1.686516in}}%
\pgfpathlineto{\pgfqpoint{1.347847in}{1.746346in}}%
\pgfpathlineto{\pgfqpoint{1.368738in}{1.717273in}}%
\pgfpathlineto{\pgfqpoint{1.386106in}{1.732843in}}%
\pgfpathlineto{\pgfqpoint{1.405825in}{1.694811in}}%
\pgfpathlineto{\pgfqpoint{1.424604in}{1.590765in}}%
\pgfpathlineto{\pgfqpoint{1.443617in}{1.460006in}}%
\pgfpathlineto{\pgfqpoint{1.462394in}{1.385940in}}%
\pgfpathlineto{\pgfqpoint{1.482581in}{1.352351in}}%
\pgfpathlineto{\pgfqpoint{1.502534in}{1.378768in}}%
\pgfpathlineto{\pgfqpoint{1.523190in}{1.443972in}}%
\pgfpathlineto{\pgfqpoint{1.542438in}{1.550361in}}%
\pgfpathlineto{\pgfqpoint{1.564971in}{1.687108in}}%
\pgfpathlineto{\pgfqpoint{1.580228in}{1.739678in}}%
\pgfpathlineto{\pgfqpoint{1.599241in}{1.727092in}}%
\pgfpathlineto{\pgfqpoint{1.621071in}{1.601105in}}%
\pgfpathlineto{\pgfqpoint{1.636328in}{1.492594in}}%
\pgfpathlineto{\pgfqpoint{1.658627in}{1.394678in}}%
\pgfpathlineto{\pgfqpoint{1.674823in}{1.358851in}}%
\pgfpathlineto{\pgfqpoint{1.693602in}{1.350461in}}%
\pgfpathlineto{\pgfqpoint{1.712381in}{1.382422in}}%
\pgfpathlineto{\pgfqpoint{1.731160in}{1.420790in}}%
\pgfpathlineto{\pgfqpoint{1.749937in}{1.516290in}}%
\pgfpathlineto{\pgfqpoint{1.772707in}{1.642778in}}%
\pgfpathlineto{\pgfqpoint{1.791249in}{1.731057in}}%
\pgfpathlineto{\pgfqpoint{1.810733in}{1.734936in}}%
\pgfpathlineto{\pgfqpoint{1.828807in}{1.670289in}}%
\pgfpathlineto{\pgfqpoint{1.848523in}{1.527894in}}%
\pgfpathlineto{\pgfqpoint{1.867771in}{1.425884in}}%
\pgfpathlineto{\pgfqpoint{1.885612in}{1.366353in}}%
\pgfpathlineto{\pgfqpoint{1.905563in}{1.350252in}}%
\pgfpathlineto{\pgfqpoint{1.924342in}{1.370276in}}%
\pgfpathlineto{\pgfqpoint{1.944764in}{1.417017in}}%
\pgfpathlineto{\pgfqpoint{1.963777in}{1.500853in}}%
\pgfpathlineto{\pgfqpoint{1.982319in}{1.623102in}}%
\pgfpathlineto{\pgfqpoint{2.001567in}{1.719190in}}%
\pgfpathlineto{\pgfqpoint{2.020111in}{1.738452in}}%
\pgfpathlineto{\pgfqpoint{2.042176in}{1.679537in}}%
\pgfpathlineto{\pgfqpoint{2.078560in}{1.446454in}}%
\pgfpathlineto{\pgfqpoint{2.098276in}{1.386551in}}%
\pgfpathlineto{\pgfqpoint{2.116821in}{1.352590in}}%
\pgfpathlineto{\pgfqpoint{2.136537in}{1.354908in}}%
\pgfpathlineto{\pgfqpoint{2.157428in}{1.398217in}}%
\pgfpathlineto{\pgfqpoint{2.175972in}{1.449854in}}%
\pgfpathlineto{\pgfqpoint{2.194280in}{1.542035in}}%
\pgfpathlineto{\pgfqpoint{2.212825in}{1.664219in}}%
\pgfpathlineto{\pgfqpoint{2.231367in}{1.729563in}}%
\pgfpathlineto{\pgfqpoint{2.252023in}{1.731328in}}%
\pgfpathlineto{\pgfqpoint{2.271976in}{1.650945in}}%
\pgfpathlineto{\pgfqpoint{2.290519in}{1.522241in}}%
\pgfpathlineto{\pgfqpoint{2.311411in}{1.426144in}}%
\pgfpathlineto{\pgfqpoint{2.330188in}{1.379292in}}%
\pgfpathlineto{\pgfqpoint{2.346619in}{1.351386in}}%
\pgfpathlineto{\pgfqpoint{2.368215in}{1.427866in}}%
\pgfpathlineto{\pgfqpoint{2.385819in}{1.376440in}}%
\pgfpathlineto{\pgfqpoint{2.385585in}{1.356194in}}%
\pgfpathlineto{\pgfqpoint{2.407415in}{1.348746in}}%
\pgfpathlineto{\pgfqpoint{2.425489in}{1.369618in}}%
\pgfpathlineto{\pgfqpoint{2.442390in}{1.415812in}}%
\pgfpathlineto{\pgfqpoint{2.460698in}{1.482124in}}%
\pgfpathlineto{\pgfqpoint{2.502479in}{1.713976in}}%
\pgfpathlineto{\pgfqpoint{2.520555in}{1.738890in}}%
\pgfpathlineto{\pgfqpoint{2.538863in}{1.702286in}}%
\pgfpathlineto{\pgfqpoint{2.558816in}{1.572592in}}%
\pgfpathlineto{\pgfqpoint{2.577358in}{1.464798in}}%
\pgfpathlineto{\pgfqpoint{2.598720in}{1.392043in}}%
\pgfpathlineto{\pgfqpoint{2.616325in}{1.357547in}}%
\pgfpathlineto{\pgfqpoint{2.634164in}{1.351786in}}%
\pgfpathlineto{\pgfqpoint{2.655054in}{1.387157in}}%
\pgfpathlineto{\pgfqpoint{2.672425in}{1.437301in}}%
\pgfpathlineto{\pgfqpoint{2.692610in}{1.487558in}}%
\pgfpathlineto{\pgfqpoint{2.711623in}{1.579105in}}%
\pgfpathlineto{\pgfqpoint{2.732514in}{1.688306in}}%
\pgfpathlineto{\pgfqpoint{2.750355in}{1.734945in}}%
\pgfpathlineto{\pgfqpoint{2.771480in}{1.721237in}}%
\pgfpathlineto{\pgfqpoint{2.788616in}{1.635645in}}%
\pgfpathlineto{\pgfqpoint{2.806219in}{1.513192in}}%
\pgfpathlineto{\pgfqpoint{2.849879in}{1.383453in}}%
\pgfpathlineto{\pgfqpoint{2.866781in}{1.352693in}}%
\pgfpathlineto{\pgfqpoint{2.885089in}{1.359489in}}%
\pgfpathlineto{\pgfqpoint{2.906214in}{1.399345in}}%
\pgfpathlineto{\pgfqpoint{2.923819in}{1.430078in}}%
\pgfpathlineto{\pgfqpoint{2.941894in}{1.513118in}}%
\pgfpathlineto{\pgfqpoint{2.961845in}{1.648926in}}%
\pgfpathlineto{\pgfqpoint{2.980390in}{1.725247in}}%
\pgfpathlineto{\pgfqpoint{3.001046in}{1.741166in}}%
\pgfpathlineto{\pgfqpoint{3.019354in}{1.712709in}}%
\pgfpathlineto{\pgfqpoint{3.056206in}{1.485808in}}%
\pgfpathlineto{\pgfqpoint{3.076862in}{1.428187in}}%
\pgfpathlineto{\pgfqpoint{3.099161in}{1.373643in}}%
\pgfpathlineto{\pgfqpoint{3.115592in}{1.351707in}}%
\pgfpathlineto{\pgfqpoint{3.136954in}{1.373061in}}%
\pgfpathlineto{\pgfqpoint{3.154090in}{1.410490in}}%
\pgfpathlineto{\pgfqpoint{3.173337in}{1.467504in}}%
\pgfpathlineto{\pgfqpoint{3.193288in}{1.553962in}}%
\pgfpathlineto{\pgfqpoint{3.211833in}{1.653716in}}%
\pgfpathlineto{\pgfqpoint{3.231549in}{1.734879in}}%
\pgfpathlineto{\pgfqpoint{3.250562in}{1.744806in}}%
\pgfpathlineto{\pgfqpoint{3.267933in}{1.715462in}}%
\pgfpathlineto{\pgfqpoint{3.288589in}{1.612879in}}%
\pgfpathlineto{\pgfqpoint{3.309245in}{1.486287in}}%
\pgfpathlineto{\pgfqpoint{3.327789in}{1.431864in}}%
\pgfpathlineto{\pgfqpoint{3.346098in}{1.383924in}}%
\pgfpathlineto{\pgfqpoint{3.363937in}{1.357703in}}%
\pgfpathlineto{\pgfqpoint{3.385298in}{1.362313in}}%
\pgfpathlineto{\pgfqpoint{3.408769in}{1.395655in}}%
\pgfpathlineto{\pgfqpoint{3.422854in}{1.432513in}}%
\pgfpathlineto{\pgfqpoint{3.443041in}{1.496275in}}%
\pgfpathlineto{\pgfqpoint{3.462523in}{1.582590in}}%
\pgfpathlineto{\pgfqpoint{3.481536in}{1.686016in}}%
\pgfpathlineto{\pgfqpoint{3.499610in}{1.741272in}}%
\pgfpathlineto{\pgfqpoint{3.519563in}{1.745729in}}%
\pgfpathlineto{\pgfqpoint{3.539514in}{1.718260in}}%
\pgfpathlineto{\pgfqpoint{3.555476in}{1.644101in}}%
\pgfpathlineto{\pgfqpoint{3.579418in}{1.525199in}}%
\pgfpathlineto{\pgfqpoint{3.597962in}{1.462599in}}%
\pgfpathlineto{\pgfqpoint{3.616505in}{1.415926in}}%
\pgfpathlineto{\pgfqpoint{3.634109in}{1.385596in}}%
\pgfpathlineto{\pgfqpoint{3.656174in}{1.358545in}}%
\pgfpathlineto{\pgfqpoint{3.673310in}{1.380144in}}%
\pgfpathlineto{\pgfqpoint{3.690680in}{1.418415in}}%
\pgfpathlineto{\pgfqpoint{3.712511in}{1.487098in}}%
\pgfpathlineto{\pgfqpoint{3.729879in}{1.551716in}}%
\pgfpathlineto{\pgfqpoint{3.754526in}{1.680370in}}%
\pgfpathlineto{\pgfqpoint{3.771193in}{1.736227in}}%
\pgfpathlineto{\pgfqpoint{3.790205in}{1.745095in}}%
\pgfpathlineto{\pgfqpoint{3.807341in}{1.754923in}}%
\pgfpathlineto{\pgfqpoint{3.826354in}{1.722092in}}%
\pgfpathlineto{\pgfqpoint{3.864615in}{1.528064in}}%
\pgfpathlineto{\pgfqpoint{3.884097in}{1.485313in}}%
\pgfpathlineto{\pgfqpoint{3.901233in}{1.415359in}}%
\pgfpathlineto{\pgfqpoint{3.926349in}{1.445880in}}%
\pgfpathlineto{\pgfqpoint{3.942545in}{1.391058in}}%
\pgfpathlineto{\pgfqpoint{3.961324in}{1.361897in}}%
\pgfpathlineto{\pgfqpoint{3.983152in}{1.364388in}}%
\pgfpathlineto{\pgfqpoint{3.999585in}{1.399998in}}%
\pgfpathlineto{\pgfqpoint{4.018596in}{1.452764in}}%
\pgfpathlineto{\pgfqpoint{4.036672in}{1.514832in}}%
\pgfpathlineto{\pgfqpoint{4.058502in}{1.635279in}}%
\pgfpathlineto{\pgfqpoint{4.076576in}{1.697305in}}%
\pgfpathlineto{\pgfqpoint{4.096527in}{1.755907in}}%
\pgfpathlineto{\pgfqpoint{4.114366in}{1.761774in}}%
\pgfpathlineto{\pgfqpoint{4.135493in}{1.716235in}}%
\pgfpathlineto{\pgfqpoint{4.154506in}{1.622870in}}%
\pgfpathlineto{\pgfqpoint{4.192767in}{1.470190in}}%
\pgfpathlineto{\pgfqpoint{4.209901in}{1.408149in}}%
\pgfpathlineto{\pgfqpoint{4.231731in}{1.373723in}}%
\pgfpathlineto{\pgfqpoint{4.250510in}{1.368377in}}%
\pgfpathlineto{\pgfqpoint{4.267880in}{1.390001in}}%
\pgfpathlineto{\pgfqpoint{4.285014in}{1.433505in}}%
\pgfpathlineto{\pgfqpoint{4.306141in}{1.502516in}}%
\pgfpathlineto{\pgfqpoint{4.324918in}{1.579851in}}%
\pgfpathlineto{\pgfqpoint{4.346279in}{1.696701in}}%
\pgfpathlineto{\pgfqpoint{4.364119in}{1.741828in}}%
\pgfpathlineto{\pgfqpoint{4.384775in}{1.771689in}}%
\pgfpathlineto{\pgfqpoint{4.402614in}{1.758694in}}%
\pgfpathlineto{\pgfqpoint{4.421393in}{1.707903in}}%
\pgfpathlineto{\pgfqpoint{4.441580in}{1.627474in}}%
\pgfpathlineto{\pgfqpoint{4.460828in}{1.535315in}}%
\pgfpathlineto{\pgfqpoint{4.481953in}{1.448117in}}%
\pgfpathlineto{\pgfqpoint{4.475145in}{1.476522in}}%
\pgfpathlineto{\pgfqpoint{4.453786in}{1.636499in}}%
\pgfpathlineto{\pgfqpoint{4.435007in}{1.742639in}}%
\pgfpathlineto{\pgfqpoint{4.416464in}{1.772407in}}%
\pgfpathlineto{\pgfqpoint{4.397215in}{1.725619in}}%
\pgfpathlineto{\pgfqpoint{4.377029in}{1.577432in}}%
\pgfpathlineto{\pgfqpoint{4.357547in}{1.462801in}}%
\pgfpathlineto{\pgfqpoint{4.338298in}{1.393604in}}%
\pgfpathlineto{\pgfqpoint{4.317173in}{1.370233in}}%
\pgfpathlineto{\pgfqpoint{4.302150in}{1.413755in}}%
\pgfpathlineto{\pgfqpoint{4.281025in}{1.519238in}}%
\pgfpathlineto{\pgfqpoint{4.262715in}{1.662002in}}%
\pgfpathlineto{\pgfqpoint{4.243468in}{1.751734in}}%
\pgfpathlineto{\pgfqpoint{4.224454in}{1.758553in}}%
\pgfpathlineto{\pgfqpoint{4.206850in}{1.684702in}}%
\pgfpathlineto{\pgfqpoint{4.185490in}{1.532766in}}%
\pgfpathlineto{\pgfqpoint{4.166711in}{1.427908in}}%
\pgfpathlineto{\pgfqpoint{4.149575in}{1.378277in}}%
\pgfpathlineto{\pgfqpoint{4.128216in}{1.375739in}}%
\pgfpathlineto{\pgfqpoint{4.110846in}{1.428294in}}%
\pgfpathlineto{\pgfqpoint{4.088781in}{1.533225in}}%
\pgfpathlineto{\pgfqpoint{4.071881in}{1.671298in}}%
\pgfpathlineto{\pgfqpoint{4.070942in}{1.733856in}}%
\pgfpathlineto{\pgfqpoint{4.050286in}{1.753385in}}%
\pgfpathlineto{\pgfqpoint{4.033150in}{1.745760in}}%
\pgfpathlineto{\pgfqpoint{4.011085in}{1.622116in}}%
\pgfpathlineto{\pgfqpoint{3.994889in}{1.508174in}}%
\pgfpathlineto{\pgfqpoint{3.974467in}{1.412925in}}%
\pgfpathlineto{\pgfqpoint{3.954282in}{1.361711in}}%
\pgfpathlineto{\pgfqpoint{3.936677in}{1.366860in}}%
\pgfpathlineto{\pgfqpoint{3.912499in}{1.441931in}}%
\pgfpathlineto{\pgfqpoint{3.899354in}{1.525945in}}%
\pgfpathlineto{\pgfqpoint{3.878934in}{1.674476in}}%
\pgfpathlineto{\pgfqpoint{3.858041in}{1.749532in}}%
\pgfpathlineto{\pgfqpoint{3.840673in}{1.751836in}}%
\pgfpathlineto{\pgfqpoint{3.816729in}{1.680008in}}%
\pgfpathlineto{\pgfqpoint{3.802412in}{1.570434in}}%
\pgfpathlineto{\pgfqpoint{3.783164in}{1.456529in}}%
\pgfpathlineto{\pgfqpoint{3.765794in}{1.385522in}}%
\pgfpathlineto{\pgfqpoint{3.744903in}{1.354974in}}%
\pgfpathlineto{\pgfqpoint{3.725656in}{1.387765in}}%
\pgfpathlineto{\pgfqpoint{3.704294in}{1.470002in}}%
\pgfpathlineto{\pgfqpoint{3.687395in}{1.562741in}}%
\pgfpathlineto{\pgfqpoint{3.667676in}{1.682021in}}%
\pgfpathlineto{\pgfqpoint{3.647960in}{1.744976in}}%
\pgfpathlineto{\pgfqpoint{3.630824in}{1.729921in}}%
\pgfpathlineto{\pgfqpoint{3.609230in}{1.617809in}}%
\pgfpathlineto{\pgfqpoint{3.589511in}{1.479597in}}%
\pgfpathlineto{\pgfqpoint{3.568855in}{1.400313in}}%
\pgfpathlineto{\pgfqpoint{3.550547in}{1.359557in}}%
\pgfpathlineto{\pgfqpoint{3.532942in}{1.356921in}}%
\pgfpathlineto{\pgfqpoint{3.512286in}{1.403899in}}%
\pgfpathlineto{\pgfqpoint{3.495619in}{1.477930in}}%
\pgfpathlineto{\pgfqpoint{3.475434in}{1.620794in}}%
\pgfpathlineto{\pgfqpoint{3.457358in}{1.720222in}}%
\pgfpathlineto{\pgfqpoint{3.436702in}{1.743872in}}%
\pgfpathlineto{\pgfqpoint{3.419097in}{1.708233in}}%
\pgfpathlineto{\pgfqpoint{3.380838in}{1.484034in}}%
\pgfpathlineto{\pgfqpoint{3.359242in}{1.409160in}}%
\pgfpathlineto{\pgfqpoint{3.338352in}{1.364708in}}%
\pgfpathlineto{\pgfqpoint{3.321685in}{1.354132in}}%
\pgfpathlineto{\pgfqpoint{3.301029in}{1.394683in}}%
\pgfpathlineto{\pgfqpoint{3.282486in}{1.448399in}}%
\pgfpathlineto{\pgfqpoint{3.264176in}{1.553572in}}%
\pgfpathlineto{\pgfqpoint{3.243754in}{1.697084in}}%
\pgfpathlineto{\pgfqpoint{3.224507in}{1.740961in}}%
\pgfpathlineto{\pgfqpoint{3.205259in}{1.720761in}}%
\pgfpathlineto{\pgfqpoint{3.188594in}{1.668841in}}%
\pgfpathlineto{\pgfqpoint{3.166764in}{1.536712in}}%
\pgfpathlineto{\pgfqpoint{3.149159in}{1.558944in}}%
\pgfpathlineto{\pgfqpoint{3.128268in}{1.439369in}}%
\pgfpathlineto{\pgfqpoint{3.110429in}{1.401775in}}%
\pgfpathlineto{\pgfqpoint{3.110429in}{1.401775in}}%
\pgfusepath{stroke}%
\end{pgfscope}%
\begin{pgfscope}%
\pgfpathrectangle{\pgfqpoint{0.444748in}{1.326898in}}{\pgfqpoint{4.231419in}{0.467251in}}%
\pgfusepath{clip}%
\pgfsetbuttcap%
\pgfsetroundjoin%
\definecolor{currentfill}{rgb}{0.047059,0.364706,0.647059}%
\pgfsetfillcolor{currentfill}%
\pgfsetlinewidth{1.003750pt}%
\definecolor{currentstroke}{rgb}{0.047059,0.364706,0.647059}%
\pgfsetstrokecolor{currentstroke}%
\pgfsetdash{}{0pt}%
\pgfsys@defobject{currentmarker}{\pgfqpoint{-0.010417in}{-0.010417in}}{\pgfqpoint{0.010417in}{0.010417in}}{%
\pgfpathmoveto{\pgfqpoint{0.000000in}{-0.010417in}}%
\pgfpathcurveto{\pgfqpoint{0.002763in}{-0.010417in}}{\pgfqpoint{0.005412in}{-0.009319in}}{\pgfqpoint{0.007366in}{-0.007366in}}%
\pgfpathcurveto{\pgfqpoint{0.009319in}{-0.005412in}}{\pgfqpoint{0.010417in}{-0.002763in}}{\pgfqpoint{0.010417in}{0.000000in}}%
\pgfpathcurveto{\pgfqpoint{0.010417in}{0.002763in}}{\pgfqpoint{0.009319in}{0.005412in}}{\pgfqpoint{0.007366in}{0.007366in}}%
\pgfpathcurveto{\pgfqpoint{0.005412in}{0.009319in}}{\pgfqpoint{0.002763in}{0.010417in}}{\pgfqpoint{0.000000in}{0.010417in}}%
\pgfpathcurveto{\pgfqpoint{-0.002763in}{0.010417in}}{\pgfqpoint{-0.005412in}{0.009319in}}{\pgfqpoint{-0.007366in}{0.007366in}}%
\pgfpathcurveto{\pgfqpoint{-0.009319in}{0.005412in}}{\pgfqpoint{-0.010417in}{0.002763in}}{\pgfqpoint{-0.010417in}{0.000000in}}%
\pgfpathcurveto{\pgfqpoint{-0.010417in}{-0.002763in}}{\pgfqpoint{-0.009319in}{-0.005412in}}{\pgfqpoint{-0.007366in}{-0.007366in}}%
\pgfpathcurveto{\pgfqpoint{-0.005412in}{-0.009319in}}{\pgfqpoint{-0.002763in}{-0.010417in}}{\pgfqpoint{0.000000in}{-0.010417in}}%
\pgfpathlineto{\pgfqpoint{0.000000in}{-0.010417in}}%
\pgfpathclose%
\pgfusepath{stroke,fill}%
}%
\begin{pgfscope}%
\pgfsys@transformshift{0.652108in}{1.498748in}%
\pgfsys@useobject{currentmarker}{}%
\end{pgfscope}%
\begin{pgfscope}%
\pgfsys@transformshift{0.657273in}{1.450784in}%
\pgfsys@useobject{currentmarker}{}%
\end{pgfscope}%
\begin{pgfscope}%
\pgfsys@transformshift{0.677929in}{1.378001in}%
\pgfsys@useobject{currentmarker}{}%
\end{pgfscope}%
\begin{pgfscope}%
\pgfsys@transformshift{0.694594in}{1.377324in}%
\pgfsys@useobject{currentmarker}{}%
\end{pgfscope}%
\begin{pgfscope}%
\pgfsys@transformshift{0.712668in}{1.437962in}%
\pgfsys@useobject{currentmarker}{}%
\end{pgfscope}%
\begin{pgfscope}%
\pgfsys@transformshift{0.731681in}{1.524905in}%
\pgfsys@useobject{currentmarker}{}%
\end{pgfscope}%
\begin{pgfscope}%
\pgfsys@transformshift{0.753980in}{1.686005in}%
\pgfsys@useobject{currentmarker}{}%
\end{pgfscope}%
\begin{pgfscope}%
\pgfsys@transformshift{0.773228in}{1.761981in}%
\pgfsys@useobject{currentmarker}{}%
\end{pgfscope}%
\begin{pgfscope}%
\pgfsys@transformshift{0.791538in}{1.752397in}%
\pgfsys@useobject{currentmarker}{}%
\end{pgfscope}%
\begin{pgfscope}%
\pgfsys@transformshift{0.810082in}{1.659671in}%
\pgfsys@useobject{currentmarker}{}%
\end{pgfscope}%
\begin{pgfscope}%
\pgfsys@transformshift{0.830033in}{1.511780in}%
\pgfsys@useobject{currentmarker}{}%
\end{pgfscope}%
\begin{pgfscope}%
\pgfsys@transformshift{0.848107in}{1.422598in}%
\pgfsys@useobject{currentmarker}{}%
\end{pgfscope}%
\begin{pgfscope}%
\pgfsys@transformshift{0.866886in}{1.367133in}%
\pgfsys@useobject{currentmarker}{}%
\end{pgfscope}%
\begin{pgfscope}%
\pgfsys@transformshift{0.885665in}{1.385040in}%
\pgfsys@useobject{currentmarker}{}%
\end{pgfscope}%
\begin{pgfscope}%
\pgfsys@transformshift{0.907024in}{1.453870in}%
\pgfsys@useobject{currentmarker}{}%
\end{pgfscope}%
\begin{pgfscope}%
\pgfsys@transformshift{0.924629in}{1.556132in}%
\pgfsys@useobject{currentmarker}{}%
\end{pgfscope}%
\begin{pgfscope}%
\pgfsys@transformshift{0.943876in}{1.690691in}%
\pgfsys@useobject{currentmarker}{}%
\end{pgfscope}%
\begin{pgfscope}%
\pgfsys@transformshift{0.964767in}{1.758165in}%
\pgfsys@useobject{currentmarker}{}%
\end{pgfscope}%
\begin{pgfscope}%
\pgfsys@transformshift{0.983077in}{1.724567in}%
\pgfsys@useobject{currentmarker}{}%
\end{pgfscope}%
\begin{pgfscope}%
\pgfsys@transformshift{1.004436in}{1.581003in}%
\pgfsys@useobject{currentmarker}{}%
\end{pgfscope}%
\begin{pgfscope}%
\pgfsys@transformshift{1.019930in}{1.475797in}%
\pgfsys@useobject{currentmarker}{}%
\end{pgfscope}%
\begin{pgfscope}%
\pgfsys@transformshift{1.043168in}{1.385311in}%
\pgfsys@useobject{currentmarker}{}%
\end{pgfscope}%
\begin{pgfscope}%
\pgfsys@transformshift{1.060773in}{1.360999in}%
\pgfsys@useobject{currentmarker}{}%
\end{pgfscope}%
\begin{pgfscope}%
\pgfsys@transformshift{1.079786in}{1.401421in}%
\pgfsys@useobject{currentmarker}{}%
\end{pgfscope}%
\begin{pgfscope}%
\pgfsys@transformshift{1.102554in}{1.500475in}%
\pgfsys@useobject{currentmarker}{}%
\end{pgfscope}%
\begin{pgfscope}%
\pgfsys@transformshift{1.115699in}{1.581227in}%
\pgfsys@useobject{currentmarker}{}%
\end{pgfscope}%
\begin{pgfscope}%
\pgfsys@transformshift{1.133304in}{1.732099in}%
\pgfsys@useobject{currentmarker}{}%
\end{pgfscope}%
\begin{pgfscope}%
\pgfsys@transformshift{1.155603in}{1.752655in}%
\pgfsys@useobject{currentmarker}{}%
\end{pgfscope}%
\begin{pgfscope}%
\pgfsys@transformshift{1.176259in}{1.688576in}%
\pgfsys@useobject{currentmarker}{}%
\end{pgfscope}%
\begin{pgfscope}%
\pgfsys@transformshift{1.195272in}{1.560838in}%
\pgfsys@useobject{currentmarker}{}%
\end{pgfscope}%
\begin{pgfscope}%
\pgfsys@transformshift{1.214051in}{1.444090in}%
\pgfsys@useobject{currentmarker}{}%
\end{pgfscope}%
\begin{pgfscope}%
\pgfsys@transformshift{1.230716in}{1.380713in}%
\pgfsys@useobject{currentmarker}{}%
\end{pgfscope}%
\begin{pgfscope}%
\pgfsys@transformshift{1.251607in}{1.357151in}%
\pgfsys@useobject{currentmarker}{}%
\end{pgfscope}%
\begin{pgfscope}%
\pgfsys@transformshift{1.272263in}{1.397419in}%
\pgfsys@useobject{currentmarker}{}%
\end{pgfscope}%
\begin{pgfscope}%
\pgfsys@transformshift{1.292450in}{1.474657in}%
\pgfsys@useobject{currentmarker}{}%
\end{pgfscope}%
\begin{pgfscope}%
\pgfsys@transformshift{1.310290in}{1.578340in}%
\pgfsys@useobject{currentmarker}{}%
\end{pgfscope}%
\begin{pgfscope}%
\pgfsys@transformshift{1.328834in}{1.697052in}%
\pgfsys@useobject{currentmarker}{}%
\end{pgfscope}%
\begin{pgfscope}%
\pgfsys@transformshift{1.349019in}{1.747644in}%
\pgfsys@useobject{currentmarker}{}%
\end{pgfscope}%
\begin{pgfscope}%
\pgfsys@transformshift{1.366390in}{1.717955in}%
\pgfsys@useobject{currentmarker}{}%
\end{pgfscope}%
\begin{pgfscope}%
\pgfsys@transformshift{1.387280in}{1.599496in}%
\pgfsys@useobject{currentmarker}{}%
\end{pgfscope}%
\begin{pgfscope}%
\pgfsys@transformshift{1.405825in}{1.475322in}%
\pgfsys@useobject{currentmarker}{}%
\end{pgfscope}%
\begin{pgfscope}%
\pgfsys@transformshift{1.424367in}{1.396755in}%
\pgfsys@useobject{currentmarker}{}%
\end{pgfscope}%
\begin{pgfscope}%
\pgfsys@transformshift{1.446432in}{1.355327in}%
\pgfsys@useobject{currentmarker}{}%
\end{pgfscope}%
\begin{pgfscope}%
\pgfsys@transformshift{1.464976in}{1.374839in}%
\pgfsys@useobject{currentmarker}{}%
\end{pgfscope}%
\begin{pgfscope}%
\pgfsys@transformshift{1.482112in}{1.419054in}%
\pgfsys@useobject{currentmarker}{}%
\end{pgfscope}%
\begin{pgfscope}%
\pgfsys@transformshift{1.504411in}{1.427425in}%
\pgfsys@useobject{currentmarker}{}%
\end{pgfscope}%
\begin{pgfscope}%
\pgfsys@transformshift{1.520842in}{1.500140in}%
\pgfsys@useobject{currentmarker}{}%
\end{pgfscope}%
\begin{pgfscope}%
\pgfsys@transformshift{1.540324in}{1.626767in}%
\pgfsys@useobject{currentmarker}{}%
\end{pgfscope}%
\begin{pgfscope}%
\pgfsys@transformshift{1.561685in}{1.731621in}%
\pgfsys@useobject{currentmarker}{}%
\end{pgfscope}%
\begin{pgfscope}%
\pgfsys@transformshift{1.579525in}{1.739485in}%
\pgfsys@useobject{currentmarker}{}%
\end{pgfscope}%
\begin{pgfscope}%
\pgfsys@transformshift{1.597364in}{1.692577in}%
\pgfsys@useobject{currentmarker}{}%
\end{pgfscope}%
\begin{pgfscope}%
\pgfsys@transformshift{1.620602in}{1.519136in}%
\pgfsys@useobject{currentmarker}{}%
\end{pgfscope}%
\begin{pgfscope}%
\pgfsys@transformshift{1.634685in}{1.606888in}%
\pgfsys@useobject{currentmarker}{}%
\end{pgfscope}%
\begin{pgfscope}%
\pgfsys@transformshift{1.662618in}{1.448569in}%
\pgfsys@useobject{currentmarker}{}%
\end{pgfscope}%
\begin{pgfscope}%
\pgfsys@transformshift{1.676937in}{1.393823in}%
\pgfsys@useobject{currentmarker}{}%
\end{pgfscope}%
\begin{pgfscope}%
\pgfsys@transformshift{1.694542in}{1.357091in}%
\pgfsys@useobject{currentmarker}{}%
\end{pgfscope}%
\begin{pgfscope}%
\pgfsys@transformshift{1.712381in}{1.357672in}%
\pgfsys@useobject{currentmarker}{}%
\end{pgfscope}%
\begin{pgfscope}%
\pgfsys@transformshift{1.732568in}{1.409396in}%
\pgfsys@useobject{currentmarker}{}%
\end{pgfscope}%
\begin{pgfscope}%
\pgfsys@transformshift{1.752285in}{1.496036in}%
\pgfsys@useobject{currentmarker}{}%
\end{pgfscope}%
\begin{pgfscope}%
\pgfsys@transformshift{1.771767in}{1.612154in}%
\pgfsys@useobject{currentmarker}{}%
\end{pgfscope}%
\begin{pgfscope}%
\pgfsys@transformshift{1.790311in}{1.711801in}%
\pgfsys@useobject{currentmarker}{}%
\end{pgfscope}%
\begin{pgfscope}%
\pgfsys@transformshift{1.810028in}{1.740513in}%
\pgfsys@useobject{currentmarker}{}%
\end{pgfscope}%
\begin{pgfscope}%
\pgfsys@transformshift{1.828104in}{1.716876in}%
\pgfsys@useobject{currentmarker}{}%
\end{pgfscope}%
\begin{pgfscope}%
\pgfsys@transformshift{1.845943in}{1.614540in}%
\pgfsys@useobject{currentmarker}{}%
\end{pgfscope}%
\begin{pgfscope}%
\pgfsys@transformshift{1.866364in}{1.478878in}%
\pgfsys@useobject{currentmarker}{}%
\end{pgfscope}%
\begin{pgfscope}%
\pgfsys@transformshift{1.887255in}{1.411165in}%
\pgfsys@useobject{currentmarker}{}%
\end{pgfscope}%
\begin{pgfscope}%
\pgfsys@transformshift{1.904623in}{1.366373in}%
\pgfsys@useobject{currentmarker}{}%
\end{pgfscope}%
\begin{pgfscope}%
\pgfsys@transformshift{1.925985in}{1.353803in}%
\pgfsys@useobject{currentmarker}{}%
\end{pgfscope}%
\begin{pgfscope}%
\pgfsys@transformshift{1.943590in}{1.388758in}%
\pgfsys@useobject{currentmarker}{}%
\end{pgfscope}%
\begin{pgfscope}%
\pgfsys@transformshift{1.960960in}{1.453121in}%
\pgfsys@useobject{currentmarker}{}%
\end{pgfscope}%
\begin{pgfscope}%
\pgfsys@transformshift{1.982790in}{1.540406in}%
\pgfsys@useobject{currentmarker}{}%
\end{pgfscope}%
\begin{pgfscope}%
\pgfsys@transformshift{2.000629in}{1.662950in}%
\pgfsys@useobject{currentmarker}{}%
\end{pgfscope}%
\begin{pgfscope}%
\pgfsys@transformshift{2.021754in}{1.737150in}%
\pgfsys@useobject{currentmarker}{}%
\end{pgfscope}%
\begin{pgfscope}%
\pgfsys@transformshift{2.039359in}{1.733002in}%
\pgfsys@useobject{currentmarker}{}%
\end{pgfscope}%
\begin{pgfscope}%
\pgfsys@transformshift{2.056964in}{1.696799in}%
\pgfsys@useobject{currentmarker}{}%
\end{pgfscope}%
\begin{pgfscope}%
\pgfsys@transformshift{2.080203in}{1.559762in}%
\pgfsys@useobject{currentmarker}{}%
\end{pgfscope}%
\begin{pgfscope}%
\pgfsys@transformshift{2.096163in}{1.455384in}%
\pgfsys@useobject{currentmarker}{}%
\end{pgfscope}%
\begin{pgfscope}%
\pgfsys@transformshift{2.117290in}{1.389355in}%
\pgfsys@useobject{currentmarker}{}%
\end{pgfscope}%
\begin{pgfscope}%
\pgfsys@transformshift{2.135598in}{1.357606in}%
\pgfsys@useobject{currentmarker}{}%
\end{pgfscope}%
\begin{pgfscope}%
\pgfsys@transformshift{2.156724in}{1.360355in}%
\pgfsys@useobject{currentmarker}{}%
\end{pgfscope}%
\begin{pgfscope}%
\pgfsys@transformshift{2.173624in}{1.399646in}%
\pgfsys@useobject{currentmarker}{}%
\end{pgfscope}%
\begin{pgfscope}%
\pgfsys@transformshift{2.194280in}{1.463433in}%
\pgfsys@useobject{currentmarker}{}%
\end{pgfscope}%
\begin{pgfscope}%
\pgfsys@transformshift{2.210945in}{1.564607in}%
\pgfsys@useobject{currentmarker}{}%
\end{pgfscope}%
\begin{pgfscope}%
\pgfsys@transformshift{2.230429in}{1.670074in}%
\pgfsys@useobject{currentmarker}{}%
\end{pgfscope}%
\begin{pgfscope}%
\pgfsys@transformshift{2.252258in}{1.732803in}%
\pgfsys@useobject{currentmarker}{}%
\end{pgfscope}%
\begin{pgfscope}%
\pgfsys@transformshift{2.269863in}{1.728156in}%
\pgfsys@useobject{currentmarker}{}%
\end{pgfscope}%
\begin{pgfscope}%
\pgfsys@transformshift{2.290519in}{1.637707in}%
\pgfsys@useobject{currentmarker}{}%
\end{pgfscope}%
\begin{pgfscope}%
\pgfsys@transformshift{2.308594in}{1.518613in}%
\pgfsys@useobject{currentmarker}{}%
\end{pgfscope}%
\begin{pgfscope}%
\pgfsys@transformshift{2.326668in}{1.433769in}%
\pgfsys@useobject{currentmarker}{}%
\end{pgfscope}%
\begin{pgfscope}%
\pgfsys@transformshift{2.347793in}{1.376094in}%
\pgfsys@useobject{currentmarker}{}%
\end{pgfscope}%
\begin{pgfscope}%
\pgfsys@transformshift{2.365632in}{1.350593in}%
\pgfsys@useobject{currentmarker}{}%
\end{pgfscope}%
\begin{pgfscope}%
\pgfsys@transformshift{2.385819in}{1.367910in}%
\pgfsys@useobject{currentmarker}{}%
\end{pgfscope}%
\begin{pgfscope}%
\pgfsys@transformshift{2.404598in}{1.409429in}%
\pgfsys@useobject{currentmarker}{}%
\end{pgfscope}%
\begin{pgfscope}%
\pgfsys@transformshift{2.426663in}{1.468345in}%
\pgfsys@useobject{currentmarker}{}%
\end{pgfscope}%
\begin{pgfscope}%
\pgfsys@transformshift{2.442625in}{1.560035in}%
\pgfsys@useobject{currentmarker}{}%
\end{pgfscope}%
\begin{pgfscope}%
\pgfsys@transformshift{2.466098in}{1.686958in}%
\pgfsys@useobject{currentmarker}{}%
\end{pgfscope}%
\begin{pgfscope}%
\pgfsys@transformshift{2.482058in}{1.734337in}%
\pgfsys@useobject{currentmarker}{}%
\end{pgfscope}%
\begin{pgfscope}%
\pgfsys@transformshift{2.500602in}{1.727641in}%
\pgfsys@useobject{currentmarker}{}%
\end{pgfscope}%
\begin{pgfscope}%
\pgfsys@transformshift{2.518441in}{1.650735in}%
\pgfsys@useobject{currentmarker}{}%
\end{pgfscope}%
\begin{pgfscope}%
\pgfsys@transformshift{2.539332in}{1.513818in}%
\pgfsys@useobject{currentmarker}{}%
\end{pgfscope}%
\begin{pgfscope}%
\pgfsys@transformshift{2.560928in}{1.428058in}%
\pgfsys@useobject{currentmarker}{}%
\end{pgfscope}%
\begin{pgfscope}%
\pgfsys@transformshift{2.578767in}{1.376332in}%
\pgfsys@useobject{currentmarker}{}%
\end{pgfscope}%
\begin{pgfscope}%
\pgfsys@transformshift{2.596606in}{1.353091in}%
\pgfsys@useobject{currentmarker}{}%
\end{pgfscope}%
\begin{pgfscope}%
\pgfsys@transformshift{2.620548in}{1.371774in}%
\pgfsys@useobject{currentmarker}{}%
\end{pgfscope}%
\begin{pgfscope}%
\pgfsys@transformshift{2.635336in}{1.408273in}%
\pgfsys@useobject{currentmarker}{}%
\end{pgfscope}%
\begin{pgfscope}%
\pgfsys@transformshift{2.654586in}{1.469957in}%
\pgfsys@useobject{currentmarker}{}%
\end{pgfscope}%
\begin{pgfscope}%
\pgfsys@transformshift{2.674068in}{1.590909in}%
\pgfsys@useobject{currentmarker}{}%
\end{pgfscope}%
\begin{pgfscope}%
\pgfsys@transformshift{2.691672in}{1.676350in}%
\pgfsys@useobject{currentmarker}{}%
\end{pgfscope}%
\begin{pgfscope}%
\pgfsys@transformshift{2.712329in}{1.710999in}%
\pgfsys@useobject{currentmarker}{}%
\end{pgfscope}%
\begin{pgfscope}%
\pgfsys@transformshift{2.731576in}{1.738086in}%
\pgfsys@useobject{currentmarker}{}%
\end{pgfscope}%
\begin{pgfscope}%
\pgfsys@transformshift{2.749416in}{1.727149in}%
\pgfsys@useobject{currentmarker}{}%
\end{pgfscope}%
\begin{pgfscope}%
\pgfsys@transformshift{2.770541in}{1.678663in}%
\pgfsys@useobject{currentmarker}{}%
\end{pgfscope}%
\begin{pgfscope}%
\pgfsys@transformshift{2.788380in}{1.602729in}%
\pgfsys@useobject{currentmarker}{}%
\end{pgfscope}%
\begin{pgfscope}%
\pgfsys@transformshift{2.810444in}{1.499133in}%
\pgfsys@useobject{currentmarker}{}%
\end{pgfscope}%
\begin{pgfscope}%
\pgfsys@transformshift{2.828989in}{1.433443in}%
\pgfsys@useobject{currentmarker}{}%
\end{pgfscope}%
\begin{pgfscope}%
\pgfsys@transformshift{2.845420in}{1.380631in}%
\pgfsys@useobject{currentmarker}{}%
\end{pgfscope}%
\begin{pgfscope}%
\pgfsys@transformshift{2.866076in}{1.353371in}%
\pgfsys@useobject{currentmarker}{}%
\end{pgfscope}%
\begin{pgfscope}%
\pgfsys@transformshift{2.884620in}{1.374077in}%
\pgfsys@useobject{currentmarker}{}%
\end{pgfscope}%
\begin{pgfscope}%
\pgfsys@transformshift{2.905042in}{1.433541in}%
\pgfsys@useobject{currentmarker}{}%
\end{pgfscope}%
\begin{pgfscope}%
\pgfsys@transformshift{2.925227in}{1.515262in}%
\pgfsys@useobject{currentmarker}{}%
\end{pgfscope}%
\begin{pgfscope}%
\pgfsys@transformshift{2.943066in}{1.625728in}%
\pgfsys@useobject{currentmarker}{}%
\end{pgfscope}%
\begin{pgfscope}%
\pgfsys@transformshift{2.962080in}{1.711984in}%
\pgfsys@useobject{currentmarker}{}%
\end{pgfscope}%
\begin{pgfscope}%
\pgfsys@transformshift{2.979216in}{1.741176in}%
\pgfsys@useobject{currentmarker}{}%
\end{pgfscope}%
\begin{pgfscope}%
\pgfsys@transformshift{3.002689in}{1.720853in}%
\pgfsys@useobject{currentmarker}{}%
\end{pgfscope}%
\begin{pgfscope}%
\pgfsys@transformshift{3.022171in}{1.629774in}%
\pgfsys@useobject{currentmarker}{}%
\end{pgfscope}%
\begin{pgfscope}%
\pgfsys@transformshift{3.040715in}{1.526577in}%
\pgfsys@useobject{currentmarker}{}%
\end{pgfscope}%
\begin{pgfscope}%
\pgfsys@transformshift{3.058084in}{1.556063in}%
\pgfsys@useobject{currentmarker}{}%
\end{pgfscope}%
\begin{pgfscope}%
\pgfsys@transformshift{3.078976in}{1.441126in}%
\pgfsys@useobject{currentmarker}{}%
\end{pgfscope}%
\begin{pgfscope}%
\pgfsys@transformshift{3.097753in}{1.390628in}%
\pgfsys@useobject{currentmarker}{}%
\end{pgfscope}%
\begin{pgfscope}%
\pgfsys@transformshift{3.115123in}{1.358157in}%
\pgfsys@useobject{currentmarker}{}%
\end{pgfscope}%
\begin{pgfscope}%
\pgfsys@transformshift{3.136250in}{1.361284in}%
\pgfsys@useobject{currentmarker}{}%
\end{pgfscope}%
\begin{pgfscope}%
\pgfsys@transformshift{3.153853in}{1.380539in}%
\pgfsys@useobject{currentmarker}{}%
\end{pgfscope}%
\begin{pgfscope}%
\pgfsys@transformshift{3.172632in}{1.431342in}%
\pgfsys@useobject{currentmarker}{}%
\end{pgfscope}%
\begin{pgfscope}%
\pgfsys@transformshift{3.193288in}{1.511871in}%
\pgfsys@useobject{currentmarker}{}%
\end{pgfscope}%
\begin{pgfscope}%
\pgfsys@transformshift{3.211362in}{1.619598in}%
\pgfsys@useobject{currentmarker}{}%
\end{pgfscope}%
\begin{pgfscope}%
\pgfsys@transformshift{3.232254in}{1.696985in}%
\pgfsys@useobject{currentmarker}{}%
\end{pgfscope}%
\begin{pgfscope}%
\pgfsys@transformshift{3.250328in}{1.742014in}%
\pgfsys@useobject{currentmarker}{}%
\end{pgfscope}%
\begin{pgfscope}%
\pgfsys@transformshift{3.267464in}{1.736090in}%
\pgfsys@useobject{currentmarker}{}%
\end{pgfscope}%
\begin{pgfscope}%
\pgfsys@transformshift{3.288823in}{1.674909in}%
\pgfsys@useobject{currentmarker}{}%
\end{pgfscope}%
\begin{pgfscope}%
\pgfsys@transformshift{3.306897in}{1.569507in}%
\pgfsys@useobject{currentmarker}{}%
\end{pgfscope}%
\begin{pgfscope}%
\pgfsys@transformshift{3.325676in}{1.476208in}%
\pgfsys@useobject{currentmarker}{}%
\end{pgfscope}%
\begin{pgfscope}%
\pgfsys@transformshift{3.347975in}{1.414032in}%
\pgfsys@useobject{currentmarker}{}%
\end{pgfscope}%
\begin{pgfscope}%
\pgfsys@transformshift{3.367693in}{1.369223in}%
\pgfsys@useobject{currentmarker}{}%
\end{pgfscope}%
\begin{pgfscope}%
\pgfsys@transformshift{3.382950in}{1.357151in}%
\pgfsys@useobject{currentmarker}{}%
\end{pgfscope}%
\begin{pgfscope}%
\pgfsys@transformshift{3.403606in}{1.372505in}%
\pgfsys@useobject{currentmarker}{}%
\end{pgfscope}%
\begin{pgfscope}%
\pgfsys@transformshift{3.423088in}{1.409723in}%
\pgfsys@useobject{currentmarker}{}%
\end{pgfscope}%
\begin{pgfscope}%
\pgfsys@transformshift{3.443041in}{1.482599in}%
\pgfsys@useobject{currentmarker}{}%
\end{pgfscope}%
\begin{pgfscope}%
\pgfsys@transformshift{3.461349in}{1.574761in}%
\pgfsys@useobject{currentmarker}{}%
\end{pgfscope}%
\begin{pgfscope}%
\pgfsys@transformshift{3.481771in}{1.679915in}%
\pgfsys@useobject{currentmarker}{}%
\end{pgfscope}%
\begin{pgfscope}%
\pgfsys@transformshift{3.500784in}{1.736457in}%
\pgfsys@useobject{currentmarker}{}%
\end{pgfscope}%
\begin{pgfscope}%
\pgfsys@transformshift{3.517449in}{1.747291in}%
\pgfsys@useobject{currentmarker}{}%
\end{pgfscope}%
\begin{pgfscope}%
\pgfsys@transformshift{3.535759in}{1.747853in}%
\pgfsys@useobject{currentmarker}{}%
\end{pgfscope}%
\begin{pgfscope}%
\pgfsys@transformshift{3.557119in}{1.697837in}%
\pgfsys@useobject{currentmarker}{}%
\end{pgfscope}%
\begin{pgfscope}%
\pgfsys@transformshift{3.578480in}{1.579246in}%
\pgfsys@useobject{currentmarker}{}%
\end{pgfscope}%
\begin{pgfscope}%
\pgfsys@transformshift{3.596319in}{1.481198in}%
\pgfsys@useobject{currentmarker}{}%
\end{pgfscope}%
\begin{pgfscope}%
\pgfsys@transformshift{3.615801in}{1.427530in}%
\pgfsys@useobject{currentmarker}{}%
\end{pgfscope}%
\begin{pgfscope}%
\pgfsys@transformshift{3.635989in}{1.385574in}%
\pgfsys@useobject{currentmarker}{}%
\end{pgfscope}%
\begin{pgfscope}%
\pgfsys@transformshift{3.653828in}{1.359084in}%
\pgfsys@useobject{currentmarker}{}%
\end{pgfscope}%
\begin{pgfscope}%
\pgfsys@transformshift{3.674718in}{1.375239in}%
\pgfsys@useobject{currentmarker}{}%
\end{pgfscope}%
\begin{pgfscope}%
\pgfsys@transformshift{3.692792in}{1.415044in}%
\pgfsys@useobject{currentmarker}{}%
\end{pgfscope}%
\begin{pgfscope}%
\pgfsys@transformshift{3.711102in}{1.472434in}%
\pgfsys@useobject{currentmarker}{}%
\end{pgfscope}%
\begin{pgfscope}%
\pgfsys@transformshift{3.732696in}{1.564336in}%
\pgfsys@useobject{currentmarker}{}%
\end{pgfscope}%
\begin{pgfscope}%
\pgfsys@transformshift{3.748892in}{1.629602in}%
\pgfsys@useobject{currentmarker}{}%
\end{pgfscope}%
\begin{pgfscope}%
\pgfsys@transformshift{3.770957in}{1.723744in}%
\pgfsys@useobject{currentmarker}{}%
\end{pgfscope}%
\begin{pgfscope}%
\pgfsys@transformshift{3.788327in}{1.751444in}%
\pgfsys@useobject{currentmarker}{}%
\end{pgfscope}%
\begin{pgfscope}%
\pgfsys@transformshift{3.809218in}{1.754472in}%
\pgfsys@useobject{currentmarker}{}%
\end{pgfscope}%
\begin{pgfscope}%
\pgfsys@transformshift{3.828936in}{1.715286in}%
\pgfsys@useobject{currentmarker}{}%
\end{pgfscope}%
\begin{pgfscope}%
\pgfsys@transformshift{3.850296in}{1.605939in}%
\pgfsys@useobject{currentmarker}{}%
\end{pgfscope}%
\begin{pgfscope}%
\pgfsys@transformshift{3.866963in}{1.521469in}%
\pgfsys@useobject{currentmarker}{}%
\end{pgfscope}%
\begin{pgfscope}%
\pgfsys@transformshift{3.884566in}{1.453300in}%
\pgfsys@useobject{currentmarker}{}%
\end{pgfscope}%
\begin{pgfscope}%
\pgfsys@transformshift{3.903579in}{1.401274in}%
\pgfsys@useobject{currentmarker}{}%
\end{pgfscope}%
\begin{pgfscope}%
\pgfsys@transformshift{3.922123in}{1.417930in}%
\pgfsys@useobject{currentmarker}{}%
\end{pgfscope}%
\begin{pgfscope}%
\pgfsys@transformshift{3.939259in}{1.375561in}%
\pgfsys@useobject{currentmarker}{}%
\end{pgfscope}%
\begin{pgfscope}%
\pgfsys@transformshift{3.960853in}{1.362914in}%
\pgfsys@useobject{currentmarker}{}%
\end{pgfscope}%
\begin{pgfscope}%
\pgfsys@transformshift{3.980806in}{1.387415in}%
\pgfsys@useobject{currentmarker}{}%
\end{pgfscope}%
\begin{pgfscope}%
\pgfsys@transformshift{3.999819in}{1.428021in}%
\pgfsys@useobject{currentmarker}{}%
\end{pgfscope}%
\begin{pgfscope}%
\pgfsys@transformshift{4.021179in}{1.509028in}%
\pgfsys@useobject{currentmarker}{}%
\end{pgfscope}%
\begin{pgfscope}%
\pgfsys@transformshift{4.038783in}{1.573214in}%
\pgfsys@useobject{currentmarker}{}%
\end{pgfscope}%
\begin{pgfscope}%
\pgfsys@transformshift{4.056623in}{1.674071in}%
\pgfsys@useobject{currentmarker}{}%
\end{pgfscope}%
\begin{pgfscope}%
\pgfsys@transformshift{4.077513in}{1.743822in}%
\pgfsys@useobject{currentmarker}{}%
\end{pgfscope}%
\begin{pgfscope}%
\pgfsys@transformshift{4.094180in}{1.764063in}%
\pgfsys@useobject{currentmarker}{}%
\end{pgfscope}%
\begin{pgfscope}%
\pgfsys@transformshift{4.116011in}{1.747660in}%
\pgfsys@useobject{currentmarker}{}%
\end{pgfscope}%
\begin{pgfscope}%
\pgfsys@transformshift{4.135024in}{1.695956in}%
\pgfsys@useobject{currentmarker}{}%
\end{pgfscope}%
\begin{pgfscope}%
\pgfsys@transformshift{4.154740in}{1.612986in}%
\pgfsys@useobject{currentmarker}{}%
\end{pgfscope}%
\begin{pgfscope}%
\pgfsys@transformshift{4.177274in}{1.497353in}%
\pgfsys@useobject{currentmarker}{}%
\end{pgfscope}%
\begin{pgfscope}%
\pgfsys@transformshift{4.193470in}{1.440253in}%
\pgfsys@useobject{currentmarker}{}%
\end{pgfscope}%
\begin{pgfscope}%
\pgfsys@transformshift{4.212249in}{1.409294in}%
\pgfsys@useobject{currentmarker}{}%
\end{pgfscope}%
\begin{pgfscope}%
\pgfsys@transformshift{4.228211in}{1.375667in}%
\pgfsys@useobject{currentmarker}{}%
\end{pgfscope}%
\begin{pgfscope}%
\pgfsys@transformshift{4.250041in}{1.379644in}%
\pgfsys@useobject{currentmarker}{}%
\end{pgfscope}%
\begin{pgfscope}%
\pgfsys@transformshift{4.268584in}{1.423315in}%
\pgfsys@useobject{currentmarker}{}%
\end{pgfscope}%
\begin{pgfscope}%
\pgfsys@transformshift{4.288536in}{1.473028in}%
\pgfsys@useobject{currentmarker}{}%
\end{pgfscope}%
\begin{pgfscope}%
\pgfsys@transformshift{4.306610in}{1.550058in}%
\pgfsys@useobject{currentmarker}{}%
\end{pgfscope}%
\begin{pgfscope}%
\pgfsys@transformshift{4.325858in}{1.622555in}%
\pgfsys@useobject{currentmarker}{}%
\end{pgfscope}%
\begin{pgfscope}%
\pgfsys@transformshift{4.345340in}{1.681941in}%
\pgfsys@useobject{currentmarker}{}%
\end{pgfscope}%
\begin{pgfscope}%
\pgfsys@transformshift{4.364588in}{1.754713in}%
\pgfsys@useobject{currentmarker}{}%
\end{pgfscope}%
\begin{pgfscope}%
\pgfsys@transformshift{4.383366in}{1.771814in}%
\pgfsys@useobject{currentmarker}{}%
\end{pgfscope}%
\begin{pgfscope}%
\pgfsys@transformshift{4.402145in}{1.762704in}%
\pgfsys@useobject{currentmarker}{}%
\end{pgfscope}%
\begin{pgfscope}%
\pgfsys@transformshift{4.421627in}{1.705530in}%
\pgfsys@useobject{currentmarker}{}%
\end{pgfscope}%
\begin{pgfscope}%
\pgfsys@transformshift{4.441109in}{1.678461in}%
\pgfsys@useobject{currentmarker}{}%
\end{pgfscope}%
\begin{pgfscope}%
\pgfsys@transformshift{4.459419in}{1.571454in}%
\pgfsys@useobject{currentmarker}{}%
\end{pgfscope}%
\begin{pgfscope}%
\pgfsys@transformshift{4.478433in}{1.477477in}%
\pgfsys@useobject{currentmarker}{}%
\end{pgfscope}%
\begin{pgfscope}%
\pgfsys@transformshift{4.479841in}{1.471158in}%
\pgfsys@useobject{currentmarker}{}%
\end{pgfscope}%
\begin{pgfscope}%
\pgfsys@transformshift{4.473973in}{1.521698in}%
\pgfsys@useobject{currentmarker}{}%
\end{pgfscope}%
\begin{pgfscope}%
\pgfsys@transformshift{4.456132in}{1.650321in}%
\pgfsys@useobject{currentmarker}{}%
\end{pgfscope}%
\begin{pgfscope}%
\pgfsys@transformshift{4.438998in}{1.772559in}%
\pgfsys@useobject{currentmarker}{}%
\end{pgfscope}%
\begin{pgfscope}%
\pgfsys@transformshift{4.416228in}{1.595322in}%
\pgfsys@useobject{currentmarker}{}%
\end{pgfscope}%
\begin{pgfscope}%
\pgfsys@transformshift{4.397451in}{1.413626in}%
\pgfsys@useobject{currentmarker}{}%
\end{pgfscope}%
\begin{pgfscope}%
\pgfsys@transformshift{4.376558in}{1.368033in}%
\pgfsys@useobject{currentmarker}{}%
\end{pgfscope}%
\begin{pgfscope}%
\pgfsys@transformshift{4.358250in}{1.404170in}%
\pgfsys@useobject{currentmarker}{}%
\end{pgfscope}%
\begin{pgfscope}%
\pgfsys@transformshift{4.340646in}{1.482804in}%
\pgfsys@useobject{currentmarker}{}%
\end{pgfscope}%
\begin{pgfscope}%
\pgfsys@transformshift{4.316938in}{1.657163in}%
\pgfsys@useobject{currentmarker}{}%
\end{pgfscope}%
\begin{pgfscope}%
\pgfsys@transformshift{4.303090in}{1.740471in}%
\pgfsys@useobject{currentmarker}{}%
\end{pgfscope}%
\begin{pgfscope}%
\pgfsys@transformshift{4.283371in}{1.763238in}%
\pgfsys@useobject{currentmarker}{}%
\end{pgfscope}%
\begin{pgfscope}%
\pgfsys@transformshift{4.265298in}{1.706864in}%
\pgfsys@useobject{currentmarker}{}%
\end{pgfscope}%
\begin{pgfscope}%
\pgfsys@transformshift{4.242764in}{1.551594in}%
\pgfsys@useobject{currentmarker}{}%
\end{pgfscope}%
\begin{pgfscope}%
\pgfsys@transformshift{4.224923in}{1.437570in}%
\pgfsys@useobject{currentmarker}{}%
\end{pgfscope}%
\begin{pgfscope}%
\pgfsys@transformshift{4.204267in}{1.379248in}%
\pgfsys@useobject{currentmarker}{}%
\end{pgfscope}%
\begin{pgfscope}%
\pgfsys@transformshift{4.185019in}{1.375463in}%
\pgfsys@useobject{currentmarker}{}%
\end{pgfscope}%
\begin{pgfscope}%
\pgfsys@transformshift{4.166711in}{1.430991in}%
\pgfsys@useobject{currentmarker}{}%
\end{pgfscope}%
\begin{pgfscope}%
\pgfsys@transformshift{4.149341in}{1.528136in}%
\pgfsys@useobject{currentmarker}{}%
\end{pgfscope}%
\begin{pgfscope}%
\pgfsys@transformshift{4.129154in}{1.685655in}%
\pgfsys@useobject{currentmarker}{}%
\end{pgfscope}%
\begin{pgfscope}%
\pgfsys@transformshift{4.109906in}{1.753757in}%
\pgfsys@useobject{currentmarker}{}%
\end{pgfscope}%
\begin{pgfscope}%
\pgfsys@transformshift{4.088078in}{1.733658in}%
\pgfsys@useobject{currentmarker}{}%
\end{pgfscope}%
\begin{pgfscope}%
\pgfsys@transformshift{4.070002in}{1.620416in}%
\pgfsys@useobject{currentmarker}{}%
\end{pgfscope}%
\begin{pgfscope}%
\pgfsys@transformshift{4.052397in}{1.506526in}%
\pgfsys@useobject{currentmarker}{}%
\end{pgfscope}%
\begin{pgfscope}%
\pgfsys@transformshift{4.034793in}{1.419422in}%
\pgfsys@useobject{currentmarker}{}%
\end{pgfscope}%
\begin{pgfscope}%
\pgfsys@transformshift{4.012493in}{1.370045in}%
\pgfsys@useobject{currentmarker}{}%
\end{pgfscope}%
\begin{pgfscope}%
\pgfsys@transformshift{3.991134in}{1.371419in}%
\pgfsys@useobject{currentmarker}{}%
\end{pgfscope}%
\begin{pgfscope}%
\pgfsys@transformshift{3.973529in}{1.427959in}%
\pgfsys@useobject{currentmarker}{}%
\end{pgfscope}%
\begin{pgfscope}%
\pgfsys@transformshift{3.952873in}{1.536026in}%
\pgfsys@useobject{currentmarker}{}%
\end{pgfscope}%
\begin{pgfscope}%
\pgfsys@transformshift{3.938554in}{1.656155in}%
\pgfsys@useobject{currentmarker}{}%
\end{pgfscope}%
\begin{pgfscope}%
\pgfsys@transformshift{3.914143in}{1.748821in}%
\pgfsys@useobject{currentmarker}{}%
\end{pgfscope}%
\begin{pgfscope}%
\pgfsys@transformshift{3.897947in}{1.742667in}%
\pgfsys@useobject{currentmarker}{}%
\end{pgfscope}%
\begin{pgfscope}%
\pgfsys@transformshift{3.880342in}{1.661727in}%
\pgfsys@useobject{currentmarker}{}%
\end{pgfscope}%
\begin{pgfscope}%
\pgfsys@transformshift{3.858981in}{1.505989in}%
\pgfsys@useobject{currentmarker}{}%
\end{pgfscope}%
\begin{pgfscope}%
\pgfsys@transformshift{3.836447in}{1.404221in}%
\pgfsys@useobject{currentmarker}{}%
\end{pgfscope}%
\begin{pgfscope}%
\pgfsys@transformshift{3.822363in}{1.368157in}%
\pgfsys@useobject{currentmarker}{}%
\end{pgfscope}%
\begin{pgfscope}%
\pgfsys@transformshift{3.800533in}{1.361263in}%
\pgfsys@useobject{currentmarker}{}%
\end{pgfscope}%
\begin{pgfscope}%
\pgfsys@transformshift{3.783164in}{1.384338in}%
\pgfsys@useobject{currentmarker}{}%
\end{pgfscope}%
\begin{pgfscope}%
\pgfsys@transformshift{3.764620in}{1.453746in}%
\pgfsys@useobject{currentmarker}{}%
\end{pgfscope}%
\begin{pgfscope}%
\pgfsys@transformshift{3.742790in}{1.590327in}%
\pgfsys@useobject{currentmarker}{}%
\end{pgfscope}%
\begin{pgfscope}%
\pgfsys@transformshift{3.724011in}{1.701122in}%
\pgfsys@useobject{currentmarker}{}%
\end{pgfscope}%
\begin{pgfscope}%
\pgfsys@transformshift{3.704999in}{1.747103in}%
\pgfsys@useobject{currentmarker}{}%
\end{pgfscope}%
\begin{pgfscope}%
\pgfsys@transformshift{3.686689in}{1.724352in}%
\pgfsys@useobject{currentmarker}{}%
\end{pgfscope}%
\begin{pgfscope}%
\pgfsys@transformshift{3.665799in}{1.634178in}%
\pgfsys@useobject{currentmarker}{}%
\end{pgfscope}%
\begin{pgfscope}%
\pgfsys@transformshift{3.648428in}{1.526066in}%
\pgfsys@useobject{currentmarker}{}%
\end{pgfscope}%
\begin{pgfscope}%
\pgfsys@transformshift{3.627069in}{1.453445in}%
\pgfsys@useobject{currentmarker}{}%
\end{pgfscope}%
\begin{pgfscope}%
\pgfsys@transformshift{3.608290in}{1.388701in}%
\pgfsys@useobject{currentmarker}{}%
\end{pgfscope}%
\begin{pgfscope}%
\pgfsys@transformshift{3.588103in}{1.354821in}%
\pgfsys@useobject{currentmarker}{}%
\end{pgfscope}%
\begin{pgfscope}%
\pgfsys@transformshift{3.571438in}{1.376784in}%
\pgfsys@useobject{currentmarker}{}%
\end{pgfscope}%
\begin{pgfscope}%
\pgfsys@transformshift{3.551721in}{1.433440in}%
\pgfsys@useobject{currentmarker}{}%
\end{pgfscope}%
\begin{pgfscope}%
\pgfsys@transformshift{3.532708in}{1.500922in}%
\pgfsys@useobject{currentmarker}{}%
\end{pgfscope}%
\begin{pgfscope}%
\pgfsys@transformshift{3.512990in}{1.633814in}%
\pgfsys@useobject{currentmarker}{}%
\end{pgfscope}%
\begin{pgfscope}%
\pgfsys@transformshift{3.493273in}{1.716250in}%
\pgfsys@useobject{currentmarker}{}%
\end{pgfscope}%
\begin{pgfscope}%
\pgfsys@transformshift{3.474963in}{1.737138in}%
\pgfsys@useobject{currentmarker}{}%
\end{pgfscope}%
\begin{pgfscope}%
\pgfsys@transformshift{3.456421in}{1.733555in}%
\pgfsys@useobject{currentmarker}{}%
\end{pgfscope}%
\begin{pgfscope}%
\pgfsys@transformshift{3.437173in}{1.645917in}%
\pgfsys@useobject{currentmarker}{}%
\end{pgfscope}%
\begin{pgfscope}%
\pgfsys@transformshift{3.417691in}{1.510198in}%
\pgfsys@useobject{currentmarker}{}%
\end{pgfscope}%
\begin{pgfscope}%
\pgfsys@transformshift{3.398207in}{1.426343in}%
\pgfsys@useobject{currentmarker}{}%
\end{pgfscope}%
\begin{pgfscope}%
\pgfsys@transformshift{3.376142in}{1.376199in}%
\pgfsys@useobject{currentmarker}{}%
\end{pgfscope}%
\begin{pgfscope}%
\pgfsys@transformshift{3.357834in}{1.352716in}%
\pgfsys@useobject{currentmarker}{}%
\end{pgfscope}%
\begin{pgfscope}%
\pgfsys@transformshift{3.338586in}{1.374829in}%
\pgfsys@useobject{currentmarker}{}%
\end{pgfscope}%
\begin{pgfscope}%
\pgfsys@transformshift{3.322156in}{1.427902in}%
\pgfsys@useobject{currentmarker}{}%
\end{pgfscope}%
\begin{pgfscope}%
\pgfsys@transformshift{3.299857in}{1.508326in}%
\pgfsys@useobject{currentmarker}{}%
\end{pgfscope}%
\begin{pgfscope}%
\pgfsys@transformshift{3.282721in}{1.434265in}%
\pgfsys@useobject{currentmarker}{}%
\end{pgfscope}%
\begin{pgfscope}%
\pgfsys@transformshift{3.263004in}{1.547517in}%
\pgfsys@useobject{currentmarker}{}%
\end{pgfscope}%
\begin{pgfscope}%
\pgfsys@transformshift{3.242112in}{1.680258in}%
\pgfsys@useobject{currentmarker}{}%
\end{pgfscope}%
\begin{pgfscope}%
\pgfsys@transformshift{3.224978in}{1.737717in}%
\pgfsys@useobject{currentmarker}{}%
\end{pgfscope}%
\begin{pgfscope}%
\pgfsys@transformshift{3.206433in}{1.737258in}%
\pgfsys@useobject{currentmarker}{}%
\end{pgfscope}%
\begin{pgfscope}%
\pgfsys@transformshift{3.183665in}{1.663085in}%
\pgfsys@useobject{currentmarker}{}%
\end{pgfscope}%
\begin{pgfscope}%
\pgfsys@transformshift{3.165355in}{1.537480in}%
\pgfsys@useobject{currentmarker}{}%
\end{pgfscope}%
\begin{pgfscope}%
\pgfsys@transformshift{3.150099in}{1.458750in}%
\pgfsys@useobject{currentmarker}{}%
\end{pgfscope}%
\begin{pgfscope}%
\pgfsys@transformshift{3.128974in}{1.394052in}%
\pgfsys@useobject{currentmarker}{}%
\end{pgfscope}%
\begin{pgfscope}%
\pgfsys@transformshift{3.109255in}{1.355061in}%
\pgfsys@useobject{currentmarker}{}%
\end{pgfscope}%
\begin{pgfscope}%
\pgfsys@transformshift{3.090242in}{1.364199in}%
\pgfsys@useobject{currentmarker}{}%
\end{pgfscope}%
\begin{pgfscope}%
\pgfsys@transformshift{3.070994in}{1.408701in}%
\pgfsys@useobject{currentmarker}{}%
\end{pgfscope}%
\begin{pgfscope}%
\pgfsys@transformshift{3.053624in}{1.475178in}%
\pgfsys@useobject{currentmarker}{}%
\end{pgfscope}%
\begin{pgfscope}%
\pgfsys@transformshift{3.034142in}{1.580312in}%
\pgfsys@useobject{currentmarker}{}%
\end{pgfscope}%
\begin{pgfscope}%
\pgfsys@transformshift{3.014191in}{1.364164in}%
\pgfsys@useobject{currentmarker}{}%
\end{pgfscope}%
\begin{pgfscope}%
\pgfsys@transformshift{2.993769in}{1.409297in}%
\pgfsys@useobject{currentmarker}{}%
\end{pgfscope}%
\begin{pgfscope}%
\pgfsys@transformshift{2.975695in}{1.459467in}%
\pgfsys@useobject{currentmarker}{}%
\end{pgfscope}%
\begin{pgfscope}%
\pgfsys@transformshift{2.956917in}{1.570708in}%
\pgfsys@useobject{currentmarker}{}%
\end{pgfscope}%
\begin{pgfscope}%
\pgfsys@transformshift{2.934852in}{1.709016in}%
\pgfsys@useobject{currentmarker}{}%
\end{pgfscope}%
\begin{pgfscope}%
\pgfsys@transformshift{2.917013in}{1.739321in}%
\pgfsys@useobject{currentmarker}{}%
\end{pgfscope}%
\begin{pgfscope}%
\pgfsys@transformshift{2.898234in}{1.697258in}%
\pgfsys@useobject{currentmarker}{}%
\end{pgfscope}%
\begin{pgfscope}%
\pgfsys@transformshift{2.879926in}{1.583734in}%
\pgfsys@useobject{currentmarker}{}%
\end{pgfscope}%
\begin{pgfscope}%
\pgfsys@transformshift{2.860442in}{1.473177in}%
\pgfsys@useobject{currentmarker}{}%
\end{pgfscope}%
\begin{pgfscope}%
\pgfsys@transformshift{2.840960in}{1.400406in}%
\pgfsys@useobject{currentmarker}{}%
\end{pgfscope}%
\begin{pgfscope}%
\pgfsys@transformshift{2.820304in}{1.355662in}%
\pgfsys@useobject{currentmarker}{}%
\end{pgfscope}%
\begin{pgfscope}%
\pgfsys@transformshift{2.801759in}{1.359590in}%
\pgfsys@useobject{currentmarker}{}%
\end{pgfscope}%
\begin{pgfscope}%
\pgfsys@transformshift{2.786268in}{1.381057in}%
\pgfsys@useobject{currentmarker}{}%
\end{pgfscope}%
\begin{pgfscope}%
\pgfsys@transformshift{2.764203in}{1.452206in}%
\pgfsys@useobject{currentmarker}{}%
\end{pgfscope}%
\begin{pgfscope}%
\pgfsys@transformshift{2.746364in}{1.561486in}%
\pgfsys@useobject{currentmarker}{}%
\end{pgfscope}%
\begin{pgfscope}%
\pgfsys@transformshift{2.724300in}{1.692035in}%
\pgfsys@useobject{currentmarker}{}%
\end{pgfscope}%
\begin{pgfscope}%
\pgfsys@transformshift{2.706460in}{1.738009in}%
\pgfsys@useobject{currentmarker}{}%
\end{pgfscope}%
\begin{pgfscope}%
\pgfsys@transformshift{2.688150in}{1.725318in}%
\pgfsys@useobject{currentmarker}{}%
\end{pgfscope}%
\begin{pgfscope}%
\pgfsys@transformshift{2.666791in}{1.632723in}%
\pgfsys@useobject{currentmarker}{}%
\end{pgfscope}%
\begin{pgfscope}%
\pgfsys@transformshift{2.645430in}{1.489693in}%
\pgfsys@useobject{currentmarker}{}%
\end{pgfscope}%
\begin{pgfscope}%
\pgfsys@transformshift{2.625713in}{1.414437in}%
\pgfsys@useobject{currentmarker}{}%
\end{pgfscope}%
\begin{pgfscope}%
\pgfsys@transformshift{2.607405in}{1.365983in}%
\pgfsys@useobject{currentmarker}{}%
\end{pgfscope}%
\begin{pgfscope}%
\pgfsys@transformshift{2.591912in}{1.350536in}%
\pgfsys@useobject{currentmarker}{}%
\end{pgfscope}%
\begin{pgfscope}%
\pgfsys@transformshift{2.571725in}{1.357925in}%
\pgfsys@useobject{currentmarker}{}%
\end{pgfscope}%
\begin{pgfscope}%
\pgfsys@transformshift{2.553417in}{1.398071in}%
\pgfsys@useobject{currentmarker}{}%
\end{pgfscope}%
\begin{pgfscope}%
\pgfsys@transformshift{2.533229in}{1.465381in}%
\pgfsys@useobject{currentmarker}{}%
\end{pgfscope}%
\begin{pgfscope}%
\pgfsys@transformshift{2.513042in}{1.557949in}%
\pgfsys@useobject{currentmarker}{}%
\end{pgfscope}%
\begin{pgfscope}%
\pgfsys@transformshift{2.494499in}{1.685357in}%
\pgfsys@useobject{currentmarker}{}%
\end{pgfscope}%
\begin{pgfscope}%
\pgfsys@transformshift{2.475955in}{1.737143in}%
\pgfsys@useobject{currentmarker}{}%
\end{pgfscope}%
\begin{pgfscope}%
\pgfsys@transformshift{2.457178in}{1.722000in}%
\pgfsys@useobject{currentmarker}{}%
\end{pgfscope}%
\begin{pgfscope}%
\pgfsys@transformshift{2.435114in}{1.616860in}%
\pgfsys@useobject{currentmarker}{}%
\end{pgfscope}%
\begin{pgfscope}%
\pgfsys@transformshift{2.420560in}{1.567675in}%
\pgfsys@useobject{currentmarker}{}%
\end{pgfscope}%
\begin{pgfscope}%
\pgfsys@transformshift{2.398730in}{1.453777in}%
\pgfsys@useobject{currentmarker}{}%
\end{pgfscope}%
\begin{pgfscope}%
\pgfsys@transformshift{2.379717in}{1.394719in}%
\pgfsys@useobject{currentmarker}{}%
\end{pgfscope}%
\begin{pgfscope}%
\pgfsys@transformshift{2.360703in}{1.358320in}%
\pgfsys@useobject{currentmarker}{}%
\end{pgfscope}%
\begin{pgfscope}%
\pgfsys@transformshift{2.343099in}{1.357101in}%
\pgfsys@useobject{currentmarker}{}%
\end{pgfscope}%
\begin{pgfscope}%
\pgfsys@transformshift{2.320800in}{1.398445in}%
\pgfsys@useobject{currentmarker}{}%
\end{pgfscope}%
\begin{pgfscope}%
\pgfsys@transformshift{2.301552in}{1.458788in}%
\pgfsys@useobject{currentmarker}{}%
\end{pgfscope}%
\begin{pgfscope}%
\pgfsys@transformshift{2.283007in}{1.573986in}%
\pgfsys@useobject{currentmarker}{}%
\end{pgfscope}%
\begin{pgfscope}%
\pgfsys@transformshift{2.264934in}{1.689612in}%
\pgfsys@useobject{currentmarker}{}%
\end{pgfscope}%
\begin{pgfscope}%
\pgfsys@transformshift{2.245921in}{1.735876in}%
\pgfsys@useobject{currentmarker}{}%
\end{pgfscope}%
\begin{pgfscope}%
\pgfsys@transformshift{2.227847in}{1.731481in}%
\pgfsys@useobject{currentmarker}{}%
\end{pgfscope}%
\begin{pgfscope}%
\pgfsys@transformshift{2.209537in}{1.664930in}%
\pgfsys@useobject{currentmarker}{}%
\end{pgfscope}%
\begin{pgfscope}%
\pgfsys@transformshift{2.187943in}{1.581168in}%
\pgfsys@useobject{currentmarker}{}%
\end{pgfscope}%
\begin{pgfscope}%
\pgfsys@transformshift{2.166113in}{1.462202in}%
\pgfsys@useobject{currentmarker}{}%
\end{pgfscope}%
\begin{pgfscope}%
\pgfsys@transformshift{2.149448in}{1.409098in}%
\pgfsys@useobject{currentmarker}{}%
\end{pgfscope}%
\begin{pgfscope}%
\pgfsys@transformshift{2.129026in}{1.365876in}%
\pgfsys@useobject{currentmarker}{}%
\end{pgfscope}%
\begin{pgfscope}%
\pgfsys@transformshift{2.110013in}{1.351877in}%
\pgfsys@useobject{currentmarker}{}%
\end{pgfscope}%
\begin{pgfscope}%
\pgfsys@transformshift{2.091468in}{1.373872in}%
\pgfsys@useobject{currentmarker}{}%
\end{pgfscope}%
\begin{pgfscope}%
\pgfsys@transformshift{2.072691in}{1.418762in}%
\pgfsys@useobject{currentmarker}{}%
\end{pgfscope}%
\begin{pgfscope}%
\pgfsys@transformshift{2.054147in}{1.498759in}%
\pgfsys@useobject{currentmarker}{}%
\end{pgfscope}%
\begin{pgfscope}%
\pgfsys@transformshift{2.033022in}{1.628101in}%
\pgfsys@useobject{currentmarker}{}%
\end{pgfscope}%
\begin{pgfscope}%
\pgfsys@transformshift{2.014712in}{1.718070in}%
\pgfsys@useobject{currentmarker}{}%
\end{pgfscope}%
\begin{pgfscope}%
\pgfsys@transformshift{1.996873in}{1.741781in}%
\pgfsys@useobject{currentmarker}{}%
\end{pgfscope}%
\begin{pgfscope}%
\pgfsys@transformshift{1.975982in}{1.711256in}%
\pgfsys@useobject{currentmarker}{}%
\end{pgfscope}%
\begin{pgfscope}%
\pgfsys@transformshift{1.956969in}{1.656287in}%
\pgfsys@useobject{currentmarker}{}%
\end{pgfscope}%
\begin{pgfscope}%
\pgfsys@transformshift{1.937487in}{1.538964in}%
\pgfsys@useobject{currentmarker}{}%
\end{pgfscope}%
\begin{pgfscope}%
\pgfsys@transformshift{1.919648in}{1.472181in}%
\pgfsys@useobject{currentmarker}{}%
\end{pgfscope}%
\begin{pgfscope}%
\pgfsys@transformshift{1.899929in}{1.407479in}%
\pgfsys@useobject{currentmarker}{}%
\end{pgfscope}%
\begin{pgfscope}%
\pgfsys@transformshift{1.877630in}{1.367207in}%
\pgfsys@useobject{currentmarker}{}%
\end{pgfscope}%
\begin{pgfscope}%
\pgfsys@transformshift{1.863311in}{1.356801in}%
\pgfsys@useobject{currentmarker}{}%
\end{pgfscope}%
\begin{pgfscope}%
\pgfsys@transformshift{1.841012in}{1.378763in}%
\pgfsys@useobject{currentmarker}{}%
\end{pgfscope}%
\begin{pgfscope}%
\pgfsys@transformshift{1.824113in}{1.415971in}%
\pgfsys@useobject{currentmarker}{}%
\end{pgfscope}%
\begin{pgfscope}%
\pgfsys@transformshift{1.802751in}{1.497906in}%
\pgfsys@useobject{currentmarker}{}%
\end{pgfscope}%
\begin{pgfscope}%
\pgfsys@transformshift{1.782095in}{1.619086in}%
\pgfsys@useobject{currentmarker}{}%
\end{pgfscope}%
\begin{pgfscope}%
\pgfsys@transformshift{1.764490in}{1.703174in}%
\pgfsys@useobject{currentmarker}{}%
\end{pgfscope}%
\begin{pgfscope}%
\pgfsys@transformshift{1.745948in}{1.741161in}%
\pgfsys@useobject{currentmarker}{}%
\end{pgfscope}%
\begin{pgfscope}%
\pgfsys@transformshift{1.728109in}{1.741431in}%
\pgfsys@useobject{currentmarker}{}%
\end{pgfscope}%
\begin{pgfscope}%
\pgfsys@transformshift{1.705104in}{1.694251in}%
\pgfsys@useobject{currentmarker}{}%
\end{pgfscope}%
\begin{pgfscope}%
\pgfsys@transformshift{1.689848in}{1.643368in}%
\pgfsys@useobject{currentmarker}{}%
\end{pgfscope}%
\begin{pgfscope}%
\pgfsys@transformshift{1.666140in}{1.509749in}%
\pgfsys@useobject{currentmarker}{}%
\end{pgfscope}%
\begin{pgfscope}%
\pgfsys@transformshift{1.650178in}{1.462846in}%
\pgfsys@useobject{currentmarker}{}%
\end{pgfscope}%
\begin{pgfscope}%
\pgfsys@transformshift{1.630931in}{1.413545in}%
\pgfsys@useobject{currentmarker}{}%
\end{pgfscope}%
\begin{pgfscope}%
\pgfsys@transformshift{1.609335in}{1.367274in}%
\pgfsys@useobject{currentmarker}{}%
\end{pgfscope}%
\begin{pgfscope}%
\pgfsys@transformshift{1.591730in}{1.358904in}%
\pgfsys@useobject{currentmarker}{}%
\end{pgfscope}%
\begin{pgfscope}%
\pgfsys@transformshift{1.572951in}{1.378258in}%
\pgfsys@useobject{currentmarker}{}%
\end{pgfscope}%
\begin{pgfscope}%
\pgfsys@transformshift{1.555112in}{1.411880in}%
\pgfsys@useobject{currentmarker}{}%
\end{pgfscope}%
\begin{pgfscope}%
\pgfsys@transformshift{1.534690in}{1.486032in}%
\pgfsys@useobject{currentmarker}{}%
\end{pgfscope}%
\begin{pgfscope}%
\pgfsys@transformshift{1.515913in}{1.541148in}%
\pgfsys@useobject{currentmarker}{}%
\end{pgfscope}%
\begin{pgfscope}%
\pgfsys@transformshift{1.496195in}{1.582261in}%
\pgfsys@useobject{currentmarker}{}%
\end{pgfscope}%
\begin{pgfscope}%
\pgfsys@transformshift{1.475304in}{1.698929in}%
\pgfsys@useobject{currentmarker}{}%
\end{pgfscope}%
\begin{pgfscope}%
\pgfsys@transformshift{1.455353in}{1.747670in}%
\pgfsys@useobject{currentmarker}{}%
\end{pgfscope}%
\begin{pgfscope}%
\pgfsys@transformshift{1.437043in}{1.747349in}%
\pgfsys@useobject{currentmarker}{}%
\end{pgfscope}%
\begin{pgfscope}%
\pgfsys@transformshift{1.416856in}{1.710836in}%
\pgfsys@useobject{currentmarker}{}%
\end{pgfscope}%
\begin{pgfscope}%
\pgfsys@transformshift{1.400896in}{1.629975in}%
\pgfsys@useobject{currentmarker}{}%
\end{pgfscope}%
\begin{pgfscope}%
\pgfsys@transformshift{1.382586in}{1.559560in}%
\pgfsys@useobject{currentmarker}{}%
\end{pgfscope}%
\begin{pgfscope}%
\pgfsys@transformshift{1.360053in}{1.466411in}%
\pgfsys@useobject{currentmarker}{}%
\end{pgfscope}%
\begin{pgfscope}%
\pgfsys@transformshift{1.342213in}{1.408867in}%
\pgfsys@useobject{currentmarker}{}%
\end{pgfscope}%
\begin{pgfscope}%
\pgfsys@transformshift{1.323435in}{1.375469in}%
\pgfsys@useobject{currentmarker}{}%
\end{pgfscope}%
\begin{pgfscope}%
\pgfsys@transformshift{1.304421in}{1.360512in}%
\pgfsys@useobject{currentmarker}{}%
\end{pgfscope}%
\begin{pgfscope}%
\pgfsys@transformshift{1.283531in}{1.390283in}%
\pgfsys@useobject{currentmarker}{}%
\end{pgfscope}%
\begin{pgfscope}%
\pgfsys@transformshift{1.267569in}{1.387221in}%
\pgfsys@useobject{currentmarker}{}%
\end{pgfscope}%
\begin{pgfscope}%
\pgfsys@transformshift{1.245973in}{1.390635in}%
\pgfsys@useobject{currentmarker}{}%
\end{pgfscope}%
\begin{pgfscope}%
\pgfsys@transformshift{1.224145in}{1.461801in}%
\pgfsys@useobject{currentmarker}{}%
\end{pgfscope}%
\begin{pgfscope}%
\pgfsys@transformshift{1.205835in}{1.548993in}%
\pgfsys@useobject{currentmarker}{}%
\end{pgfscope}%
\begin{pgfscope}%
\pgfsys@transformshift{1.187761in}{1.438051in}%
\pgfsys@useobject{currentmarker}{}%
\end{pgfscope}%
\begin{pgfscope}%
\pgfsys@transformshift{1.167339in}{1.512525in}%
\pgfsys@useobject{currentmarker}{}%
\end{pgfscope}%
\begin{pgfscope}%
\pgfsys@transformshift{1.149500in}{1.596712in}%
\pgfsys@useobject{currentmarker}{}%
\end{pgfscope}%
\begin{pgfscope}%
\pgfsys@transformshift{1.130956in}{1.707451in}%
\pgfsys@useobject{currentmarker}{}%
\end{pgfscope}%
\begin{pgfscope}%
\pgfsys@transformshift{1.111005in}{1.755983in}%
\pgfsys@useobject{currentmarker}{}%
\end{pgfscope}%
\begin{pgfscope}%
\pgfsys@transformshift{1.091992in}{1.753161in}%
\pgfsys@useobject{currentmarker}{}%
\end{pgfscope}%
\begin{pgfscope}%
\pgfsys@transformshift{1.072978in}{1.708421in}%
\pgfsys@useobject{currentmarker}{}%
\end{pgfscope}%
\begin{pgfscope}%
\pgfsys@transformshift{1.052557in}{1.599574in}%
\pgfsys@useobject{currentmarker}{}%
\end{pgfscope}%
\begin{pgfscope}%
\pgfsys@transformshift{1.034249in}{1.509029in}%
\pgfsys@useobject{currentmarker}{}%
\end{pgfscope}%
\begin{pgfscope}%
\pgfsys@transformshift{1.012887in}{1.434085in}%
\pgfsys@useobject{currentmarker}{}%
\end{pgfscope}%
\begin{pgfscope}%
\pgfsys@transformshift{0.996925in}{1.396771in}%
\pgfsys@useobject{currentmarker}{}%
\end{pgfscope}%
\begin{pgfscope}%
\pgfsys@transformshift{0.974626in}{1.368304in}%
\pgfsys@useobject{currentmarker}{}%
\end{pgfscope}%
\begin{pgfscope}%
\pgfsys@transformshift{0.956318in}{1.375816in}%
\pgfsys@useobject{currentmarker}{}%
\end{pgfscope}%
\begin{pgfscope}%
\pgfsys@transformshift{0.938479in}{1.415911in}%
\pgfsys@useobject{currentmarker}{}%
\end{pgfscope}%
\begin{pgfscope}%
\pgfsys@transformshift{0.920403in}{1.479790in}%
\pgfsys@useobject{currentmarker}{}%
\end{pgfscope}%
\begin{pgfscope}%
\pgfsys@transformshift{0.899513in}{1.571715in}%
\pgfsys@useobject{currentmarker}{}%
\end{pgfscope}%
\begin{pgfscope}%
\pgfsys@transformshift{0.881674in}{1.664525in}%
\pgfsys@useobject{currentmarker}{}%
\end{pgfscope}%
\begin{pgfscope}%
\pgfsys@transformshift{0.859843in}{1.747699in}%
\pgfsys@useobject{currentmarker}{}%
\end{pgfscope}%
\begin{pgfscope}%
\pgfsys@transformshift{0.843178in}{1.766333in}%
\pgfsys@useobject{currentmarker}{}%
\end{pgfscope}%
\begin{pgfscope}%
\pgfsys@transformshift{0.822053in}{1.744832in}%
\pgfsys@useobject{currentmarker}{}%
\end{pgfscope}%
\begin{pgfscope}%
\pgfsys@transformshift{0.803978in}{1.699546in}%
\pgfsys@useobject{currentmarker}{}%
\end{pgfscope}%
\begin{pgfscope}%
\pgfsys@transformshift{0.787078in}{1.635306in}%
\pgfsys@useobject{currentmarker}{}%
\end{pgfscope}%
\begin{pgfscope}%
\pgfsys@transformshift{0.766656in}{1.547249in}%
\pgfsys@useobject{currentmarker}{}%
\end{pgfscope}%
\begin{pgfscope}%
\pgfsys@transformshift{0.746469in}{1.474817in}%
\pgfsys@useobject{currentmarker}{}%
\end{pgfscope}%
\begin{pgfscope}%
\pgfsys@transformshift{0.729569in}{1.420767in}%
\pgfsys@useobject{currentmarker}{}%
\end{pgfscope}%
\begin{pgfscope}%
\pgfsys@transformshift{0.708679in}{1.376728in}%
\pgfsys@useobject{currentmarker}{}%
\end{pgfscope}%
\begin{pgfscope}%
\pgfsys@transformshift{0.691074in}{1.371447in}%
\pgfsys@useobject{currentmarker}{}%
\end{pgfscope}%
\begin{pgfscope}%
\pgfsys@transformshift{0.669478in}{1.402284in}%
\pgfsys@useobject{currentmarker}{}%
\end{pgfscope}%
\begin{pgfscope}%
\pgfsys@transformshift{0.652579in}{1.447899in}%
\pgfsys@useobject{currentmarker}{}%
\end{pgfscope}%
\begin{pgfscope}%
\pgfsys@transformshift{0.655865in}{1.423102in}%
\pgfsys@useobject{currentmarker}{}%
\end{pgfscope}%
\begin{pgfscope}%
\pgfsys@transformshift{0.674878in}{1.722320in}%
\pgfsys@useobject{currentmarker}{}%
\end{pgfscope}%
\begin{pgfscope}%
\pgfsys@transformshift{0.696003in}{1.767184in}%
\pgfsys@useobject{currentmarker}{}%
\end{pgfscope}%
\begin{pgfscope}%
\pgfsys@transformshift{0.714782in}{1.719953in}%
\pgfsys@useobject{currentmarker}{}%
\end{pgfscope}%
\begin{pgfscope}%
\pgfsys@transformshift{0.733090in}{1.579540in}%
\pgfsys@useobject{currentmarker}{}%
\end{pgfscope}%
\begin{pgfscope}%
\pgfsys@transformshift{0.752572in}{1.456497in}%
\pgfsys@useobject{currentmarker}{}%
\end{pgfscope}%
\begin{pgfscope}%
\pgfsys@transformshift{0.772056in}{1.382793in}%
\pgfsys@useobject{currentmarker}{}%
\end{pgfscope}%
\begin{pgfscope}%
\pgfsys@transformshift{0.790129in}{1.372957in}%
\pgfsys@useobject{currentmarker}{}%
\end{pgfscope}%
\begin{pgfscope}%
\pgfsys@transformshift{0.811020in}{1.430785in}%
\pgfsys@useobject{currentmarker}{}%
\end{pgfscope}%
\begin{pgfscope}%
\pgfsys@transformshift{0.830973in}{1.545749in}%
\pgfsys@useobject{currentmarker}{}%
\end{pgfscope}%
\begin{pgfscope}%
\pgfsys@transformshift{0.849750in}{1.687001in}%
\pgfsys@useobject{currentmarker}{}%
\end{pgfscope}%
\begin{pgfscope}%
\pgfsys@transformshift{0.869234in}{1.757408in}%
\pgfsys@useobject{currentmarker}{}%
\end{pgfscope}%
\begin{pgfscope}%
\pgfsys@transformshift{0.888482in}{1.745941in}%
\pgfsys@useobject{currentmarker}{}%
\end{pgfscope}%
\begin{pgfscope}%
\pgfsys@transformshift{0.905852in}{1.644430in}%
\pgfsys@useobject{currentmarker}{}%
\end{pgfscope}%
\begin{pgfscope}%
\pgfsys@transformshift{0.926272in}{1.499417in}%
\pgfsys@useobject{currentmarker}{}%
\end{pgfscope}%
\begin{pgfscope}%
\pgfsys@transformshift{0.945519in}{1.409365in}%
\pgfsys@useobject{currentmarker}{}%
\end{pgfscope}%
\begin{pgfscope}%
\pgfsys@transformshift{0.964533in}{1.363055in}%
\pgfsys@useobject{currentmarker}{}%
\end{pgfscope}%
\begin{pgfscope}%
\pgfsys@transformshift{0.983546in}{1.388502in}%
\pgfsys@useobject{currentmarker}{}%
\end{pgfscope}%
\begin{pgfscope}%
\pgfsys@transformshift{1.003030in}{1.456379in}%
\pgfsys@useobject{currentmarker}{}%
\end{pgfscope}%
\begin{pgfscope}%
\pgfsys@transformshift{1.021807in}{1.564063in}%
\pgfsys@useobject{currentmarker}{}%
\end{pgfscope}%
\begin{pgfscope}%
\pgfsys@transformshift{1.039882in}{1.696932in}%
\pgfsys@useobject{currentmarker}{}%
\end{pgfscope}%
\begin{pgfscope}%
\pgfsys@transformshift{1.059365in}{1.753278in}%
\pgfsys@useobject{currentmarker}{}%
\end{pgfscope}%
\begin{pgfscope}%
\pgfsys@transformshift{1.081429in}{1.714581in}%
\pgfsys@useobject{currentmarker}{}%
\end{pgfscope}%
\begin{pgfscope}%
\pgfsys@transformshift{1.100442in}{1.752911in}%
\pgfsys@useobject{currentmarker}{}%
\end{pgfscope}%
\begin{pgfscope}%
\pgfsys@transformshift{1.118047in}{1.711924in}%
\pgfsys@useobject{currentmarker}{}%
\end{pgfscope}%
\begin{pgfscope}%
\pgfsys@transformshift{1.137295in}{1.581422in}%
\pgfsys@useobject{currentmarker}{}%
\end{pgfscope}%
\begin{pgfscope}%
\pgfsys@transformshift{1.155368in}{1.455691in}%
\pgfsys@useobject{currentmarker}{}%
\end{pgfscope}%
\begin{pgfscope}%
\pgfsys@transformshift{1.177199in}{1.377745in}%
\pgfsys@useobject{currentmarker}{}%
\end{pgfscope}%
\begin{pgfscope}%
\pgfsys@transformshift{1.192924in}{1.357235in}%
\pgfsys@useobject{currentmarker}{}%
\end{pgfscope}%
\begin{pgfscope}%
\pgfsys@transformshift{1.214520in}{1.396359in}%
\pgfsys@useobject{currentmarker}{}%
\end{pgfscope}%
\begin{pgfscope}%
\pgfsys@transformshift{1.233768in}{1.465298in}%
\pgfsys@useobject{currentmarker}{}%
\end{pgfscope}%
\begin{pgfscope}%
\pgfsys@transformshift{1.252781in}{1.568728in}%
\pgfsys@useobject{currentmarker}{}%
\end{pgfscope}%
\begin{pgfscope}%
\pgfsys@transformshift{1.271560in}{1.695692in}%
\pgfsys@useobject{currentmarker}{}%
\end{pgfscope}%
\begin{pgfscope}%
\pgfsys@transformshift{1.289633in}{1.747144in}%
\pgfsys@useobject{currentmarker}{}%
\end{pgfscope}%
\begin{pgfscope}%
\pgfsys@transformshift{1.311698in}{1.717198in}%
\pgfsys@useobject{currentmarker}{}%
\end{pgfscope}%
\begin{pgfscope}%
\pgfsys@transformshift{1.330477in}{1.600285in}%
\pgfsys@useobject{currentmarker}{}%
\end{pgfscope}%
\begin{pgfscope}%
\pgfsys@transformshift{1.349490in}{1.474472in}%
\pgfsys@useobject{currentmarker}{}%
\end{pgfscope}%
\begin{pgfscope}%
\pgfsys@transformshift{1.368738in}{1.397647in}%
\pgfsys@useobject{currentmarker}{}%
\end{pgfscope}%
\begin{pgfscope}%
\pgfsys@transformshift{1.388454in}{1.362418in}%
\pgfsys@useobject{currentmarker}{}%
\end{pgfscope}%
\begin{pgfscope}%
\pgfsys@transformshift{1.406059in}{1.362968in}%
\pgfsys@useobject{currentmarker}{}%
\end{pgfscope}%
\begin{pgfscope}%
\pgfsys@transformshift{1.425541in}{1.406394in}%
\pgfsys@useobject{currentmarker}{}%
\end{pgfscope}%
\begin{pgfscope}%
\pgfsys@transformshift{1.442912in}{1.463490in}%
\pgfsys@useobject{currentmarker}{}%
\end{pgfscope}%
\begin{pgfscope}%
\pgfsys@transformshift{1.465681in}{1.585845in}%
\pgfsys@useobject{currentmarker}{}%
\end{pgfscope}%
\begin{pgfscope}%
\pgfsys@transformshift{1.483286in}{1.697338in}%
\pgfsys@useobject{currentmarker}{}%
\end{pgfscope}%
\begin{pgfscope}%
\pgfsys@transformshift{1.501829in}{1.743555in}%
\pgfsys@useobject{currentmarker}{}%
\end{pgfscope}%
\begin{pgfscope}%
\pgfsys@transformshift{1.523659in}{1.728641in}%
\pgfsys@useobject{currentmarker}{}%
\end{pgfscope}%
\begin{pgfscope}%
\pgfsys@transformshift{1.543141in}{1.637383in}%
\pgfsys@useobject{currentmarker}{}%
\end{pgfscope}%
\begin{pgfscope}%
\pgfsys@transformshift{1.560746in}{1.502248in}%
\pgfsys@useobject{currentmarker}{}%
\end{pgfscope}%
\begin{pgfscope}%
\pgfsys@transformshift{1.579054in}{1.416839in}%
\pgfsys@useobject{currentmarker}{}%
\end{pgfscope}%
\begin{pgfscope}%
\pgfsys@transformshift{1.600884in}{1.364711in}%
\pgfsys@useobject{currentmarker}{}%
\end{pgfscope}%
\begin{pgfscope}%
\pgfsys@transformshift{1.618960in}{1.356100in}%
\pgfsys@useobject{currentmarker}{}%
\end{pgfscope}%
\begin{pgfscope}%
\pgfsys@transformshift{1.634685in}{1.371416in}%
\pgfsys@useobject{currentmarker}{}%
\end{pgfscope}%
\begin{pgfscope}%
\pgfsys@transformshift{1.657924in}{1.435273in}%
\pgfsys@useobject{currentmarker}{}%
\end{pgfscope}%
\begin{pgfscope}%
\pgfsys@transformshift{1.673886in}{1.507355in}%
\pgfsys@useobject{currentmarker}{}%
\end{pgfscope}%
\begin{pgfscope}%
\pgfsys@transformshift{1.695011in}{1.643942in}%
\pgfsys@useobject{currentmarker}{}%
\end{pgfscope}%
\begin{pgfscope}%
\pgfsys@transformshift{1.713790in}{1.727425in}%
\pgfsys@useobject{currentmarker}{}%
\end{pgfscope}%
\begin{pgfscope}%
\pgfsys@transformshift{1.737028in}{1.739335in}%
\pgfsys@useobject{currentmarker}{}%
\end{pgfscope}%
\begin{pgfscope}%
\pgfsys@transformshift{1.753928in}{1.701853in}%
\pgfsys@useobject{currentmarker}{}%
\end{pgfscope}%
\begin{pgfscope}%
\pgfsys@transformshift{1.771064in}{1.595115in}%
\pgfsys@useobject{currentmarker}{}%
\end{pgfscope}%
\begin{pgfscope}%
\pgfsys@transformshift{1.788903in}{1.545574in}%
\pgfsys@useobject{currentmarker}{}%
\end{pgfscope}%
\begin{pgfscope}%
\pgfsys@transformshift{1.809794in}{1.457308in}%
\pgfsys@useobject{currentmarker}{}%
\end{pgfscope}%
\begin{pgfscope}%
\pgfsys@transformshift{1.827867in}{1.406675in}%
\pgfsys@useobject{currentmarker}{}%
\end{pgfscope}%
\begin{pgfscope}%
\pgfsys@transformshift{1.846177in}{1.361808in}%
\pgfsys@useobject{currentmarker}{}%
\end{pgfscope}%
\begin{pgfscope}%
\pgfsys@transformshift{1.868242in}{1.355890in}%
\pgfsys@useobject{currentmarker}{}%
\end{pgfscope}%
\begin{pgfscope}%
\pgfsys@transformshift{1.884907in}{1.383074in}%
\pgfsys@useobject{currentmarker}{}%
\end{pgfscope}%
\begin{pgfscope}%
\pgfsys@transformshift{1.905094in}{1.435438in}%
\pgfsys@useobject{currentmarker}{}%
\end{pgfscope}%
\begin{pgfscope}%
\pgfsys@transformshift{1.923637in}{1.491018in}%
\pgfsys@useobject{currentmarker}{}%
\end{pgfscope}%
\begin{pgfscope}%
\pgfsys@transformshift{1.944998in}{1.625316in}%
\pgfsys@useobject{currentmarker}{}%
\end{pgfscope}%
\begin{pgfscope}%
\pgfsys@transformshift{1.961898in}{1.665500in}%
\pgfsys@useobject{currentmarker}{}%
\end{pgfscope}%
\begin{pgfscope}%
\pgfsys@transformshift{1.983025in}{1.736561in}%
\pgfsys@useobject{currentmarker}{}%
\end{pgfscope}%
\begin{pgfscope}%
\pgfsys@transformshift{2.001098in}{1.726282in}%
\pgfsys@useobject{currentmarker}{}%
\end{pgfscope}%
\begin{pgfscope}%
\pgfsys@transformshift{2.021520in}{1.631503in}%
\pgfsys@useobject{currentmarker}{}%
\end{pgfscope}%
\begin{pgfscope}%
\pgfsys@transformshift{2.039359in}{1.525247in}%
\pgfsys@useobject{currentmarker}{}%
\end{pgfscope}%
\begin{pgfscope}%
\pgfsys@transformshift{2.058138in}{1.433640in}%
\pgfsys@useobject{currentmarker}{}%
\end{pgfscope}%
\begin{pgfscope}%
\pgfsys@transformshift{2.078794in}{1.374651in}%
\pgfsys@useobject{currentmarker}{}%
\end{pgfscope}%
\begin{pgfscope}%
\pgfsys@transformshift{2.097571in}{1.351398in}%
\pgfsys@useobject{currentmarker}{}%
\end{pgfscope}%
\begin{pgfscope}%
\pgfsys@transformshift{2.117993in}{1.716236in}%
\pgfsys@useobject{currentmarker}{}%
\end{pgfscope}%
\begin{pgfscope}%
\pgfsys@transformshift{2.137240in}{1.636746in}%
\pgfsys@useobject{currentmarker}{}%
\end{pgfscope}%
\begin{pgfscope}%
\pgfsys@transformshift{2.156724in}{1.498125in}%
\pgfsys@useobject{currentmarker}{}%
\end{pgfscope}%
\begin{pgfscope}%
\pgfsys@transformshift{2.174564in}{1.414386in}%
\pgfsys@useobject{currentmarker}{}%
\end{pgfscope}%
\begin{pgfscope}%
\pgfsys@transformshift{2.195689in}{1.363369in}%
\pgfsys@useobject{currentmarker}{}%
\end{pgfscope}%
\begin{pgfscope}%
\pgfsys@transformshift{2.214233in}{1.354974in}%
\pgfsys@useobject{currentmarker}{}%
\end{pgfscope}%
\begin{pgfscope}%
\pgfsys@transformshift{2.230898in}{1.390960in}%
\pgfsys@useobject{currentmarker}{}%
\end{pgfscope}%
\begin{pgfscope}%
\pgfsys@transformshift{2.252023in}{1.464162in}%
\pgfsys@useobject{currentmarker}{}%
\end{pgfscope}%
\begin{pgfscope}%
\pgfsys@transformshift{2.269628in}{1.566605in}%
\pgfsys@useobject{currentmarker}{}%
\end{pgfscope}%
\begin{pgfscope}%
\pgfsys@transformshift{2.294275in}{1.713168in}%
\pgfsys@useobject{currentmarker}{}%
\end{pgfscope}%
\begin{pgfscope}%
\pgfsys@transformshift{2.309766in}{1.738911in}%
\pgfsys@useobject{currentmarker}{}%
\end{pgfscope}%
\begin{pgfscope}%
\pgfsys@transformshift{2.328780in}{1.709612in}%
\pgfsys@useobject{currentmarker}{}%
\end{pgfscope}%
\begin{pgfscope}%
\pgfsys@transformshift{2.349672in}{1.645237in}%
\pgfsys@useobject{currentmarker}{}%
\end{pgfscope}%
\begin{pgfscope}%
\pgfsys@transformshift{2.368215in}{1.522676in}%
\pgfsys@useobject{currentmarker}{}%
\end{pgfscope}%
\begin{pgfscope}%
\pgfsys@transformshift{2.388871in}{1.434629in}%
\pgfsys@useobject{currentmarker}{}%
\end{pgfscope}%
\begin{pgfscope}%
\pgfsys@transformshift{2.407415in}{1.375046in}%
\pgfsys@useobject{currentmarker}{}%
\end{pgfscope}%
\begin{pgfscope}%
\pgfsys@transformshift{2.423377in}{1.351533in}%
\pgfsys@useobject{currentmarker}{}%
\end{pgfscope}%
\begin{pgfscope}%
\pgfsys@transformshift{2.444502in}{1.369034in}%
\pgfsys@useobject{currentmarker}{}%
\end{pgfscope}%
\begin{pgfscope}%
\pgfsys@transformshift{2.462107in}{1.420643in}%
\pgfsys@useobject{currentmarker}{}%
\end{pgfscope}%
\begin{pgfscope}%
\pgfsys@transformshift{2.481120in}{1.494254in}%
\pgfsys@useobject{currentmarker}{}%
\end{pgfscope}%
\begin{pgfscope}%
\pgfsys@transformshift{2.504359in}{1.644920in}%
\pgfsys@useobject{currentmarker}{}%
\end{pgfscope}%
\begin{pgfscope}%
\pgfsys@transformshift{2.522901in}{1.723852in}%
\pgfsys@useobject{currentmarker}{}%
\end{pgfscope}%
\begin{pgfscope}%
\pgfsys@transformshift{2.521964in}{1.736117in}%
\pgfsys@useobject{currentmarker}{}%
\end{pgfscope}%
\begin{pgfscope}%
\pgfsys@transformshift{2.539566in}{1.738109in}%
\pgfsys@useobject{currentmarker}{}%
\end{pgfscope}%
\begin{pgfscope}%
\pgfsys@transformshift{2.560459in}{1.679095in}%
\pgfsys@useobject{currentmarker}{}%
\end{pgfscope}%
\begin{pgfscope}%
\pgfsys@transformshift{2.578533in}{1.557031in}%
\pgfsys@useobject{currentmarker}{}%
\end{pgfscope}%
\begin{pgfscope}%
\pgfsys@transformshift{2.596137in}{1.511581in}%
\pgfsys@useobject{currentmarker}{}%
\end{pgfscope}%
\begin{pgfscope}%
\pgfsys@transformshift{2.617262in}{1.414760in}%
\pgfsys@useobject{currentmarker}{}%
\end{pgfscope}%
\begin{pgfscope}%
\pgfsys@transformshift{2.634633in}{1.370837in}%
\pgfsys@useobject{currentmarker}{}%
\end{pgfscope}%
\begin{pgfscope}%
\pgfsys@transformshift{2.652941in}{1.350821in}%
\pgfsys@useobject{currentmarker}{}%
\end{pgfscope}%
\begin{pgfscope}%
\pgfsys@transformshift{2.673833in}{1.382846in}%
\pgfsys@useobject{currentmarker}{}%
\end{pgfscope}%
\begin{pgfscope}%
\pgfsys@transformshift{2.692610in}{1.435495in}%
\pgfsys@useobject{currentmarker}{}%
\end{pgfscope}%
\begin{pgfscope}%
\pgfsys@transformshift{2.713032in}{1.528876in}%
\pgfsys@useobject{currentmarker}{}%
\end{pgfscope}%
\begin{pgfscope}%
\pgfsys@transformshift{2.731576in}{1.637793in}%
\pgfsys@useobject{currentmarker}{}%
\end{pgfscope}%
\begin{pgfscope}%
\pgfsys@transformshift{2.749416in}{1.716335in}%
\pgfsys@useobject{currentmarker}{}%
\end{pgfscope}%
\begin{pgfscope}%
\pgfsys@transformshift{2.769603in}{1.736656in}%
\pgfsys@useobject{currentmarker}{}%
\end{pgfscope}%
\begin{pgfscope}%
\pgfsys@transformshift{2.791431in}{1.670757in}%
\pgfsys@useobject{currentmarker}{}%
\end{pgfscope}%
\begin{pgfscope}%
\pgfsys@transformshift{2.809036in}{1.572135in}%
\pgfsys@useobject{currentmarker}{}%
\end{pgfscope}%
\begin{pgfscope}%
\pgfsys@transformshift{2.827111in}{1.458845in}%
\pgfsys@useobject{currentmarker}{}%
\end{pgfscope}%
\begin{pgfscope}%
\pgfsys@transformshift{2.847533in}{1.395004in}%
\pgfsys@useobject{currentmarker}{}%
\end{pgfscope}%
\begin{pgfscope}%
\pgfsys@transformshift{2.865607in}{1.365437in}%
\pgfsys@useobject{currentmarker}{}%
\end{pgfscope}%
\begin{pgfscope}%
\pgfsys@transformshift{2.884854in}{1.353960in}%
\pgfsys@useobject{currentmarker}{}%
\end{pgfscope}%
\begin{pgfscope}%
\pgfsys@transformshift{2.904805in}{1.389284in}%
\pgfsys@useobject{currentmarker}{}%
\end{pgfscope}%
\begin{pgfscope}%
\pgfsys@transformshift{2.923115in}{1.438995in}%
\pgfsys@useobject{currentmarker}{}%
\end{pgfscope}%
\begin{pgfscope}%
\pgfsys@transformshift{2.943537in}{1.520518in}%
\pgfsys@useobject{currentmarker}{}%
\end{pgfscope}%
\begin{pgfscope}%
\pgfsys@transformshift{2.964428in}{1.609796in}%
\pgfsys@useobject{currentmarker}{}%
\end{pgfscope}%
\begin{pgfscope}%
\pgfsys@transformshift{2.979684in}{1.698320in}%
\pgfsys@useobject{currentmarker}{}%
\end{pgfscope}%
\begin{pgfscope}%
\pgfsys@transformshift{3.001046in}{1.740894in}%
\pgfsys@useobject{currentmarker}{}%
\end{pgfscope}%
\begin{pgfscope}%
\pgfsys@transformshift{3.021702in}{1.721085in}%
\pgfsys@useobject{currentmarker}{}%
\end{pgfscope}%
\begin{pgfscope}%
\pgfsys@transformshift{3.039776in}{1.636727in}%
\pgfsys@useobject{currentmarker}{}%
\end{pgfscope}%
\begin{pgfscope}%
\pgfsys@transformshift{3.057615in}{1.526164in}%
\pgfsys@useobject{currentmarker}{}%
\end{pgfscope}%
\begin{pgfscope}%
\pgfsys@transformshift{3.078271in}{1.475859in}%
\pgfsys@useobject{currentmarker}{}%
\end{pgfscope}%
\begin{pgfscope}%
\pgfsys@transformshift{3.097519in}{1.407000in}%
\pgfsys@useobject{currentmarker}{}%
\end{pgfscope}%
\begin{pgfscope}%
\pgfsys@transformshift{3.117237in}{1.365195in}%
\pgfsys@useobject{currentmarker}{}%
\end{pgfscope}%
\begin{pgfscope}%
\pgfsys@transformshift{3.136014in}{1.357109in}%
\pgfsys@useobject{currentmarker}{}%
\end{pgfscope}%
\begin{pgfscope}%
\pgfsys@transformshift{3.153853in}{1.389808in}%
\pgfsys@useobject{currentmarker}{}%
\end{pgfscope}%
\begin{pgfscope}%
\pgfsys@transformshift{3.174980in}{1.452655in}%
\pgfsys@useobject{currentmarker}{}%
\end{pgfscope}%
\begin{pgfscope}%
\pgfsys@transformshift{3.193759in}{1.491508in}%
\pgfsys@useobject{currentmarker}{}%
\end{pgfscope}%
\begin{pgfscope}%
\pgfsys@transformshift{3.211362in}{1.594291in}%
\pgfsys@useobject{currentmarker}{}%
\end{pgfscope}%
\begin{pgfscope}%
\pgfsys@transformshift{3.232018in}{1.704982in}%
\pgfsys@useobject{currentmarker}{}%
\end{pgfscope}%
\begin{pgfscope}%
\pgfsys@transformshift{3.250328in}{1.740896in}%
\pgfsys@useobject{currentmarker}{}%
\end{pgfscope}%
\begin{pgfscope}%
\pgfsys@transformshift{3.270279in}{1.730211in}%
\pgfsys@useobject{currentmarker}{}%
\end{pgfscope}%
\begin{pgfscope}%
\pgfsys@transformshift{3.286712in}{1.655387in}%
\pgfsys@useobject{currentmarker}{}%
\end{pgfscope}%
\begin{pgfscope}%
\pgfsys@transformshift{3.308776in}{1.568823in}%
\pgfsys@useobject{currentmarker}{}%
\end{pgfscope}%
\begin{pgfscope}%
\pgfsys@transformshift{3.324973in}{1.492383in}%
\pgfsys@useobject{currentmarker}{}%
\end{pgfscope}%
\begin{pgfscope}%
\pgfsys@transformshift{3.345629in}{1.412422in}%
\pgfsys@useobject{currentmarker}{}%
\end{pgfscope}%
\begin{pgfscope}%
\pgfsys@transformshift{3.366988in}{1.366557in}%
\pgfsys@useobject{currentmarker}{}%
\end{pgfscope}%
\begin{pgfscope}%
\pgfsys@transformshift{3.386705in}{1.357043in}%
\pgfsys@useobject{currentmarker}{}%
\end{pgfscope}%
\begin{pgfscope}%
\pgfsys@transformshift{3.403137in}{1.386318in}%
\pgfsys@useobject{currentmarker}{}%
\end{pgfscope}%
\begin{pgfscope}%
\pgfsys@transformshift{3.424262in}{1.430960in}%
\pgfsys@useobject{currentmarker}{}%
\end{pgfscope}%
\begin{pgfscope}%
\pgfsys@transformshift{3.441162in}{1.482936in}%
\pgfsys@useobject{currentmarker}{}%
\end{pgfscope}%
\begin{pgfscope}%
\pgfsys@transformshift{3.461818in}{1.568924in}%
\pgfsys@useobject{currentmarker}{}%
\end{pgfscope}%
\begin{pgfscope}%
\pgfsys@transformshift{3.481068in}{1.674702in}%
\pgfsys@useobject{currentmarker}{}%
\end{pgfscope}%
\begin{pgfscope}%
\pgfsys@transformshift{3.498907in}{1.734052in}%
\pgfsys@useobject{currentmarker}{}%
\end{pgfscope}%
\begin{pgfscope}%
\pgfsys@transformshift{3.519329in}{1.749731in}%
\pgfsys@useobject{currentmarker}{}%
\end{pgfscope}%
\begin{pgfscope}%
\pgfsys@transformshift{3.538576in}{1.726438in}%
\pgfsys@useobject{currentmarker}{}%
\end{pgfscope}%
\begin{pgfscope}%
\pgfsys@transformshift{3.560405in}{1.630597in}%
\pgfsys@useobject{currentmarker}{}%
\end{pgfscope}%
\begin{pgfscope}%
\pgfsys@transformshift{3.576132in}{1.559564in}%
\pgfsys@useobject{currentmarker}{}%
\end{pgfscope}%
\begin{pgfscope}%
\pgfsys@transformshift{3.597023in}{1.465559in}%
\pgfsys@useobject{currentmarker}{}%
\end{pgfscope}%
\begin{pgfscope}%
\pgfsys@transformshift{3.615567in}{1.747054in}%
\pgfsys@useobject{currentmarker}{}%
\end{pgfscope}%
\begin{pgfscope}%
\pgfsys@transformshift{3.634346in}{1.745251in}%
\pgfsys@useobject{currentmarker}{}%
\end{pgfscope}%
\begin{pgfscope}%
\pgfsys@transformshift{3.652888in}{1.689739in}%
\pgfsys@useobject{currentmarker}{}%
\end{pgfscope}%
\begin{pgfscope}%
\pgfsys@transformshift{3.674250in}{1.552981in}%
\pgfsys@useobject{currentmarker}{}%
\end{pgfscope}%
\begin{pgfscope}%
\pgfsys@transformshift{3.694906in}{1.463067in}%
\pgfsys@useobject{currentmarker}{}%
\end{pgfscope}%
\begin{pgfscope}%
\pgfsys@transformshift{3.712040in}{1.406259in}%
\pgfsys@useobject{currentmarker}{}%
\end{pgfscope}%
\begin{pgfscope}%
\pgfsys@transformshift{3.729879in}{1.366177in}%
\pgfsys@useobject{currentmarker}{}%
\end{pgfscope}%
\begin{pgfscope}%
\pgfsys@transformshift{3.749832in}{1.369754in}%
\pgfsys@useobject{currentmarker}{}%
\end{pgfscope}%
\begin{pgfscope}%
\pgfsys@transformshift{3.768140in}{1.414049in}%
\pgfsys@useobject{currentmarker}{}%
\end{pgfscope}%
\begin{pgfscope}%
\pgfsys@transformshift{3.787153in}{1.467051in}%
\pgfsys@useobject{currentmarker}{}%
\end{pgfscope}%
\begin{pgfscope}%
\pgfsys@transformshift{3.807575in}{1.564983in}%
\pgfsys@useobject{currentmarker}{}%
\end{pgfscope}%
\begin{pgfscope}%
\pgfsys@transformshift{3.825180in}{1.668584in}%
\pgfsys@useobject{currentmarker}{}%
\end{pgfscope}%
\begin{pgfscope}%
\pgfsys@transformshift{3.846776in}{1.738433in}%
\pgfsys@useobject{currentmarker}{}%
\end{pgfscope}%
\begin{pgfscope}%
\pgfsys@transformshift{3.864849in}{1.756936in}%
\pgfsys@useobject{currentmarker}{}%
\end{pgfscope}%
\begin{pgfscope}%
\pgfsys@transformshift{3.885271in}{1.723436in}%
\pgfsys@useobject{currentmarker}{}%
\end{pgfscope}%
\begin{pgfscope}%
\pgfsys@transformshift{3.904284in}{1.644607in}%
\pgfsys@useobject{currentmarker}{}%
\end{pgfscope}%
\begin{pgfscope}%
\pgfsys@transformshift{3.925175in}{1.547949in}%
\pgfsys@useobject{currentmarker}{}%
\end{pgfscope}%
\begin{pgfscope}%
\pgfsys@transformshift{3.942779in}{1.462247in}%
\pgfsys@useobject{currentmarker}{}%
\end{pgfscope}%
\begin{pgfscope}%
\pgfsys@transformshift{3.962967in}{1.400795in}%
\pgfsys@useobject{currentmarker}{}%
\end{pgfscope}%
\begin{pgfscope}%
\pgfsys@transformshift{3.980806in}{1.371230in}%
\pgfsys@useobject{currentmarker}{}%
\end{pgfscope}%
\begin{pgfscope}%
\pgfsys@transformshift{4.000054in}{1.365598in}%
\pgfsys@useobject{currentmarker}{}%
\end{pgfscope}%
\begin{pgfscope}%
\pgfsys@transformshift{4.016953in}{1.392387in}%
\pgfsys@useobject{currentmarker}{}%
\end{pgfscope}%
\begin{pgfscope}%
\pgfsys@transformshift{4.040426in}{1.459831in}%
\pgfsys@useobject{currentmarker}{}%
\end{pgfscope}%
\begin{pgfscope}%
\pgfsys@transformshift{4.058266in}{1.513939in}%
\pgfsys@useobject{currentmarker}{}%
\end{pgfscope}%
\begin{pgfscope}%
\pgfsys@transformshift{4.081270in}{1.641916in}%
\pgfsys@useobject{currentmarker}{}%
\end{pgfscope}%
\begin{pgfscope}%
\pgfsys@transformshift{4.098640in}{1.719048in}%
\pgfsys@useobject{currentmarker}{}%
\end{pgfscope}%
\begin{pgfscope}%
\pgfsys@transformshift{4.114602in}{1.758622in}%
\pgfsys@useobject{currentmarker}{}%
\end{pgfscope}%
\begin{pgfscope}%
\pgfsys@transformshift{4.135493in}{1.710596in}%
\pgfsys@useobject{currentmarker}{}%
\end{pgfscope}%
\begin{pgfscope}%
\pgfsys@transformshift{4.154506in}{1.732117in}%
\pgfsys@useobject{currentmarker}{}%
\end{pgfscope}%
\begin{pgfscope}%
\pgfsys@transformshift{4.174691in}{1.764861in}%
\pgfsys@useobject{currentmarker}{}%
\end{pgfscope}%
\begin{pgfscope}%
\pgfsys@transformshift{4.195347in}{1.738658in}%
\pgfsys@useobject{currentmarker}{}%
\end{pgfscope}%
\begin{pgfscope}%
\pgfsys@transformshift{4.211075in}{1.684213in}%
\pgfsys@useobject{currentmarker}{}%
\end{pgfscope}%
\begin{pgfscope}%
\pgfsys@transformshift{4.228914in}{1.575876in}%
\pgfsys@useobject{currentmarker}{}%
\end{pgfscope}%
\begin{pgfscope}%
\pgfsys@transformshift{4.249805in}{1.484101in}%
\pgfsys@useobject{currentmarker}{}%
\end{pgfscope}%
\begin{pgfscope}%
\pgfsys@transformshift{4.267644in}{1.423675in}%
\pgfsys@useobject{currentmarker}{}%
\end{pgfscope}%
\begin{pgfscope}%
\pgfsys@transformshift{4.292057in}{1.373985in}%
\pgfsys@useobject{currentmarker}{}%
\end{pgfscope}%
\begin{pgfscope}%
\pgfsys@transformshift{4.307313in}{1.370559in}%
\pgfsys@useobject{currentmarker}{}%
\end{pgfscope}%
\begin{pgfscope}%
\pgfsys@transformshift{4.326797in}{1.412710in}%
\pgfsys@useobject{currentmarker}{}%
\end{pgfscope}%
\begin{pgfscope}%
\pgfsys@transformshift{4.343228in}{1.467083in}%
\pgfsys@useobject{currentmarker}{}%
\end{pgfscope}%
\begin{pgfscope}%
\pgfsys@transformshift{4.366701in}{1.566486in}%
\pgfsys@useobject{currentmarker}{}%
\end{pgfscope}%
\begin{pgfscope}%
\pgfsys@transformshift{4.383835in}{1.641553in}%
\pgfsys@useobject{currentmarker}{}%
\end{pgfscope}%
\begin{pgfscope}%
\pgfsys@transformshift{4.401440in}{1.733864in}%
\pgfsys@useobject{currentmarker}{}%
\end{pgfscope}%
\begin{pgfscope}%
\pgfsys@transformshift{4.421393in}{1.769087in}%
\pgfsys@useobject{currentmarker}{}%
\end{pgfscope}%
\begin{pgfscope}%
\pgfsys@transformshift{4.444632in}{1.760931in}%
\pgfsys@useobject{currentmarker}{}%
\end{pgfscope}%
\begin{pgfscope}%
\pgfsys@transformshift{4.462236in}{1.701833in}%
\pgfsys@useobject{currentmarker}{}%
\end{pgfscope}%
\begin{pgfscope}%
\pgfsys@transformshift{4.481013in}{1.629748in}%
\pgfsys@useobject{currentmarker}{}%
\end{pgfscope}%
\begin{pgfscope}%
\pgfsys@transformshift{4.482422in}{1.635679in}%
\pgfsys@useobject{currentmarker}{}%
\end{pgfscope}%
\begin{pgfscope}%
\pgfsys@transformshift{4.473268in}{1.694530in}%
\pgfsys@useobject{currentmarker}{}%
\end{pgfscope}%
\begin{pgfscope}%
\pgfsys@transformshift{4.455897in}{1.763841in}%
\pgfsys@useobject{currentmarker}{}%
\end{pgfscope}%
\begin{pgfscope}%
\pgfsys@transformshift{4.436884in}{1.759788in}%
\pgfsys@useobject{currentmarker}{}%
\end{pgfscope}%
\begin{pgfscope}%
\pgfsys@transformshift{4.419279in}{1.641074in}%
\pgfsys@useobject{currentmarker}{}%
\end{pgfscope}%
\begin{pgfscope}%
\pgfsys@transformshift{4.396511in}{1.513902in}%
\pgfsys@useobject{currentmarker}{}%
\end{pgfscope}%
\begin{pgfscope}%
\pgfsys@transformshift{4.376558in}{1.417607in}%
\pgfsys@useobject{currentmarker}{}%
\end{pgfscope}%
\begin{pgfscope}%
\pgfsys@transformshift{4.360362in}{1.375373in}%
\pgfsys@useobject{currentmarker}{}%
\end{pgfscope}%
\begin{pgfscope}%
\pgfsys@transformshift{4.340177in}{1.393345in}%
\pgfsys@useobject{currentmarker}{}%
\end{pgfscope}%
\begin{pgfscope}%
\pgfsys@transformshift{4.321867in}{1.458046in}%
\pgfsys@useobject{currentmarker}{}%
\end{pgfscope}%
\begin{pgfscope}%
\pgfsys@transformshift{4.301916in}{1.592799in}%
\pgfsys@useobject{currentmarker}{}%
\end{pgfscope}%
\begin{pgfscope}%
\pgfsys@transformshift{4.282197in}{1.726486in}%
\pgfsys@useobject{currentmarker}{}%
\end{pgfscope}%
\begin{pgfscope}%
\pgfsys@transformshift{4.261307in}{1.764726in}%
\pgfsys@useobject{currentmarker}{}%
\end{pgfscope}%
\begin{pgfscope}%
\pgfsys@transformshift{4.242999in}{1.721493in}%
\pgfsys@useobject{currentmarker}{}%
\end{pgfscope}%
\begin{pgfscope}%
\pgfsys@transformshift{4.223280in}{1.572188in}%
\pgfsys@useobject{currentmarker}{}%
\end{pgfscope}%
\begin{pgfscope}%
\pgfsys@transformshift{4.204972in}{1.467014in}%
\pgfsys@useobject{currentmarker}{}%
\end{pgfscope}%
\begin{pgfscope}%
\pgfsys@transformshift{4.188307in}{1.398895in}%
\pgfsys@useobject{currentmarker}{}%
\end{pgfscope}%
\begin{pgfscope}%
\pgfsys@transformshift{4.167180in}{1.363090in}%
\pgfsys@useobject{currentmarker}{}%
\end{pgfscope}%
\begin{pgfscope}%
\pgfsys@transformshift{4.148872in}{1.410367in}%
\pgfsys@useobject{currentmarker}{}%
\end{pgfscope}%
\begin{pgfscope}%
\pgfsys@transformshift{4.127042in}{1.504013in}%
\pgfsys@useobject{currentmarker}{}%
\end{pgfscope}%
\begin{pgfscope}%
\pgfsys@transformshift{4.109906in}{1.632580in}%
\pgfsys@useobject{currentmarker}{}%
\end{pgfscope}%
\begin{pgfscope}%
\pgfsys@transformshift{4.088547in}{1.742920in}%
\pgfsys@useobject{currentmarker}{}%
\end{pgfscope}%
\begin{pgfscope}%
\pgfsys@transformshift{4.070473in}{1.754687in}%
\pgfsys@useobject{currentmarker}{}%
\end{pgfscope}%
\begin{pgfscope}%
\pgfsys@transformshift{4.052163in}{1.691721in}%
\pgfsys@useobject{currentmarker}{}%
\end{pgfscope}%
\begin{pgfscope}%
\pgfsys@transformshift{4.030333in}{1.520805in}%
\pgfsys@useobject{currentmarker}{}%
\end{pgfscope}%
\begin{pgfscope}%
\pgfsys@transformshift{4.014607in}{1.431138in}%
\pgfsys@useobject{currentmarker}{}%
\end{pgfscope}%
\begin{pgfscope}%
\pgfsys@transformshift{3.992308in}{1.369665in}%
\pgfsys@useobject{currentmarker}{}%
\end{pgfscope}%
\begin{pgfscope}%
\pgfsys@transformshift{3.973764in}{1.369278in}%
\pgfsys@useobject{currentmarker}{}%
\end{pgfscope}%
\begin{pgfscope}%
\pgfsys@transformshift{3.955924in}{1.423493in}%
\pgfsys@useobject{currentmarker}{}%
\end{pgfscope}%
\begin{pgfscope}%
\pgfsys@transformshift{3.933860in}{1.527481in}%
\pgfsys@useobject{currentmarker}{}%
\end{pgfscope}%
\begin{pgfscope}%
\pgfsys@transformshift{3.916021in}{1.664110in}%
\pgfsys@useobject{currentmarker}{}%
\end{pgfscope}%
\begin{pgfscope}%
\pgfsys@transformshift{3.895833in}{1.747241in}%
\pgfsys@useobject{currentmarker}{}%
\end{pgfscope}%
\begin{pgfscope}%
\pgfsys@transformshift{3.882220in}{1.743163in}%
\pgfsys@useobject{currentmarker}{}%
\end{pgfscope}%
\begin{pgfscope}%
\pgfsys@transformshift{3.860155in}{1.664404in}%
\pgfsys@useobject{currentmarker}{}%
\end{pgfscope}%
\begin{pgfscope}%
\pgfsys@transformshift{3.839733in}{1.513618in}%
\pgfsys@useobject{currentmarker}{}%
\end{pgfscope}%
\begin{pgfscope}%
\pgfsys@transformshift{3.821660in}{1.424235in}%
\pgfsys@useobject{currentmarker}{}%
\end{pgfscope}%
\begin{pgfscope}%
\pgfsys@transformshift{3.801941in}{1.368926in}%
\pgfsys@useobject{currentmarker}{}%
\end{pgfscope}%
\begin{pgfscope}%
\pgfsys@transformshift{3.782928in}{1.359750in}%
\pgfsys@useobject{currentmarker}{}%
\end{pgfscope}%
\begin{pgfscope}%
\pgfsys@transformshift{3.762977in}{1.412345in}%
\pgfsys@useobject{currentmarker}{}%
\end{pgfscope}%
\begin{pgfscope}%
\pgfsys@transformshift{3.745607in}{1.493737in}%
\pgfsys@useobject{currentmarker}{}%
\end{pgfscope}%
\begin{pgfscope}%
\pgfsys@transformshift{3.725185in}{1.632325in}%
\pgfsys@useobject{currentmarker}{}%
\end{pgfscope}%
\begin{pgfscope}%
\pgfsys@transformshift{3.703591in}{1.708701in}%
\pgfsys@useobject{currentmarker}{}%
\end{pgfscope}%
\begin{pgfscope}%
\pgfsys@transformshift{3.686221in}{1.747631in}%
\pgfsys@useobject{currentmarker}{}%
\end{pgfscope}%
\begin{pgfscope}%
\pgfsys@transformshift{3.667676in}{1.713867in}%
\pgfsys@useobject{currentmarker}{}%
\end{pgfscope}%
\begin{pgfscope}%
\pgfsys@transformshift{3.646786in}{1.607229in}%
\pgfsys@useobject{currentmarker}{}%
\end{pgfscope}%
\begin{pgfscope}%
\pgfsys@transformshift{3.629650in}{1.490936in}%
\pgfsys@useobject{currentmarker}{}%
\end{pgfscope}%
\begin{pgfscope}%
\pgfsys@transformshift{3.612984in}{1.416067in}%
\pgfsys@useobject{currentmarker}{}%
\end{pgfscope}%
\begin{pgfscope}%
\pgfsys@transformshift{3.588808in}{1.362944in}%
\pgfsys@useobject{currentmarker}{}%
\end{pgfscope}%
\begin{pgfscope}%
\pgfsys@transformshift{3.570733in}{1.359951in}%
\pgfsys@useobject{currentmarker}{}%
\end{pgfscope}%
\begin{pgfscope}%
\pgfsys@transformshift{3.551721in}{1.402413in}%
\pgfsys@useobject{currentmarker}{}%
\end{pgfscope}%
\begin{pgfscope}%
\pgfsys@transformshift{3.533411in}{1.470682in}%
\pgfsys@useobject{currentmarker}{}%
\end{pgfscope}%
\begin{pgfscope}%
\pgfsys@transformshift{3.512755in}{1.597131in}%
\pgfsys@useobject{currentmarker}{}%
\end{pgfscope}%
\begin{pgfscope}%
\pgfsys@transformshift{3.495150in}{1.672788in}%
\pgfsys@useobject{currentmarker}{}%
\end{pgfscope}%
\begin{pgfscope}%
\pgfsys@transformshift{3.473791in}{1.742045in}%
\pgfsys@useobject{currentmarker}{}%
\end{pgfscope}%
\begin{pgfscope}%
\pgfsys@transformshift{3.457595in}{1.738832in}%
\pgfsys@useobject{currentmarker}{}%
\end{pgfscope}%
\begin{pgfscope}%
\pgfsys@transformshift{3.436233in}{1.660897in}%
\pgfsys@useobject{currentmarker}{}%
\end{pgfscope}%
\begin{pgfscope}%
\pgfsys@transformshift{3.417691in}{1.543658in}%
\pgfsys@useobject{currentmarker}{}%
\end{pgfscope}%
\begin{pgfscope}%
\pgfsys@transformshift{3.398912in}{1.449444in}%
\pgfsys@useobject{currentmarker}{}%
\end{pgfscope}%
\begin{pgfscope}%
\pgfsys@transformshift{3.379664in}{1.392796in}%
\pgfsys@useobject{currentmarker}{}%
\end{pgfscope}%
\begin{pgfscope}%
\pgfsys@transformshift{3.358303in}{1.353529in}%
\pgfsys@useobject{currentmarker}{}%
\end{pgfscope}%
\begin{pgfscope}%
\pgfsys@transformshift{3.340933in}{1.678020in}%
\pgfsys@useobject{currentmarker}{}%
\end{pgfscope}%
\begin{pgfscope}%
\pgfsys@transformshift{3.323093in}{1.552917in}%
\pgfsys@useobject{currentmarker}{}%
\end{pgfscope}%
\begin{pgfscope}%
\pgfsys@transformshift{3.301263in}{1.436136in}%
\pgfsys@useobject{currentmarker}{}%
\end{pgfscope}%
\begin{pgfscope}%
\pgfsys@transformshift{3.281547in}{1.377607in}%
\pgfsys@useobject{currentmarker}{}%
\end{pgfscope}%
\begin{pgfscope}%
\pgfsys@transformshift{3.263004in}{1.354286in}%
\pgfsys@useobject{currentmarker}{}%
\end{pgfscope}%
\begin{pgfscope}%
\pgfsys@transformshift{3.244694in}{1.370746in}%
\pgfsys@useobject{currentmarker}{}%
\end{pgfscope}%
\begin{pgfscope}%
\pgfsys@transformshift{3.224743in}{1.434999in}%
\pgfsys@useobject{currentmarker}{}%
\end{pgfscope}%
\begin{pgfscope}%
\pgfsys@transformshift{3.207607in}{1.516610in}%
\pgfsys@useobject{currentmarker}{}%
\end{pgfscope}%
\begin{pgfscope}%
\pgfsys@transformshift{3.188594in}{1.656091in}%
\pgfsys@useobject{currentmarker}{}%
\end{pgfscope}%
\begin{pgfscope}%
\pgfsys@transformshift{3.170050in}{1.734072in}%
\pgfsys@useobject{currentmarker}{}%
\end{pgfscope}%
\begin{pgfscope}%
\pgfsys@transformshift{3.147985in}{1.722879in}%
\pgfsys@useobject{currentmarker}{}%
\end{pgfscope}%
\begin{pgfscope}%
\pgfsys@transformshift{3.128974in}{1.641123in}%
\pgfsys@useobject{currentmarker}{}%
\end{pgfscope}%
\begin{pgfscope}%
\pgfsys@transformshift{3.110898in}{1.519919in}%
\pgfsys@useobject{currentmarker}{}%
\end{pgfscope}%
\begin{pgfscope}%
\pgfsys@transformshift{3.091650in}{1.441061in}%
\pgfsys@useobject{currentmarker}{}%
\end{pgfscope}%
\begin{pgfscope}%
\pgfsys@transformshift{3.070760in}{1.373455in}%
\pgfsys@useobject{currentmarker}{}%
\end{pgfscope}%
\begin{pgfscope}%
\pgfsys@transformshift{3.052215in}{1.352662in}%
\pgfsys@useobject{currentmarker}{}%
\end{pgfscope}%
\begin{pgfscope}%
\pgfsys@transformshift{3.030622in}{1.383527in}%
\pgfsys@useobject{currentmarker}{}%
\end{pgfscope}%
\begin{pgfscope}%
\pgfsys@transformshift{3.015599in}{1.426824in}%
\pgfsys@useobject{currentmarker}{}%
\end{pgfscope}%
\begin{pgfscope}%
\pgfsys@transformshift{2.993064in}{1.539702in}%
\pgfsys@useobject{currentmarker}{}%
\end{pgfscope}%
\begin{pgfscope}%
\pgfsys@transformshift{2.976633in}{1.659321in}%
\pgfsys@useobject{currentmarker}{}%
\end{pgfscope}%
\begin{pgfscope}%
\pgfsys@transformshift{2.957385in}{1.730732in}%
\pgfsys@useobject{currentmarker}{}%
\end{pgfscope}%
\begin{pgfscope}%
\pgfsys@transformshift{2.935790in}{1.727150in}%
\pgfsys@useobject{currentmarker}{}%
\end{pgfscope}%
\begin{pgfscope}%
\pgfsys@transformshift{2.917716in}{1.653809in}%
\pgfsys@useobject{currentmarker}{}%
\end{pgfscope}%
\begin{pgfscope}%
\pgfsys@transformshift{2.896825in}{1.519665in}%
\pgfsys@useobject{currentmarker}{}%
\end{pgfscope}%
\begin{pgfscope}%
\pgfsys@transformshift{2.878517in}{1.474563in}%
\pgfsys@useobject{currentmarker}{}%
\end{pgfscope}%
\begin{pgfscope}%
\pgfsys@transformshift{2.859973in}{1.428062in}%
\pgfsys@useobject{currentmarker}{}%
\end{pgfscope}%
\begin{pgfscope}%
\pgfsys@transformshift{2.841194in}{1.374940in}%
\pgfsys@useobject{currentmarker}{}%
\end{pgfscope}%
\begin{pgfscope}%
\pgfsys@transformshift{2.823120in}{1.350123in}%
\pgfsys@useobject{currentmarker}{}%
\end{pgfscope}%
\begin{pgfscope}%
\pgfsys@transformshift{2.803638in}{1.377192in}%
\pgfsys@useobject{currentmarker}{}%
\end{pgfscope}%
\begin{pgfscope}%
\pgfsys@transformshift{2.785799in}{1.426925in}%
\pgfsys@useobject{currentmarker}{}%
\end{pgfscope}%
\begin{pgfscope}%
\pgfsys@transformshift{2.764203in}{1.540989in}%
\pgfsys@useobject{currentmarker}{}%
\end{pgfscope}%
\begin{pgfscope}%
\pgfsys@transformshift{2.745190in}{1.672866in}%
\pgfsys@useobject{currentmarker}{}%
\end{pgfscope}%
\begin{pgfscope}%
\pgfsys@transformshift{2.726411in}{1.734636in}%
\pgfsys@useobject{currentmarker}{}%
\end{pgfscope}%
\begin{pgfscope}%
\pgfsys@transformshift{2.706929in}{1.606592in}%
\pgfsys@useobject{currentmarker}{}%
\end{pgfscope}%
\begin{pgfscope}%
\pgfsys@transformshift{2.689090in}{1.716702in}%
\pgfsys@useobject{currentmarker}{}%
\end{pgfscope}%
\begin{pgfscope}%
\pgfsys@transformshift{2.668199in}{1.738672in}%
\pgfsys@useobject{currentmarker}{}%
\end{pgfscope}%
\begin{pgfscope}%
\pgfsys@transformshift{2.649186in}{1.709588in}%
\pgfsys@useobject{currentmarker}{}%
\end{pgfscope}%
\begin{pgfscope}%
\pgfsys@transformshift{2.629704in}{1.588036in}%
\pgfsys@useobject{currentmarker}{}%
\end{pgfscope}%
\begin{pgfscope}%
\pgfsys@transformshift{2.610691in}{1.471961in}%
\pgfsys@useobject{currentmarker}{}%
\end{pgfscope}%
\begin{pgfscope}%
\pgfsys@transformshift{2.591443in}{1.399922in}%
\pgfsys@useobject{currentmarker}{}%
\end{pgfscope}%
\begin{pgfscope}%
\pgfsys@transformshift{2.570787in}{1.355722in}%
\pgfsys@useobject{currentmarker}{}%
\end{pgfscope}%
\begin{pgfscope}%
\pgfsys@transformshift{2.550834in}{1.359272in}%
\pgfsys@useobject{currentmarker}{}%
\end{pgfscope}%
\begin{pgfscope}%
\pgfsys@transformshift{2.532055in}{1.402308in}%
\pgfsys@useobject{currentmarker}{}%
\end{pgfscope}%
\begin{pgfscope}%
\pgfsys@transformshift{2.513747in}{1.473753in}%
\pgfsys@useobject{currentmarker}{}%
\end{pgfscope}%
\begin{pgfscope}%
\pgfsys@transformshift{2.495203in}{1.592332in}%
\pgfsys@useobject{currentmarker}{}%
\end{pgfscope}%
\begin{pgfscope}%
\pgfsys@transformshift{2.475955in}{1.705119in}%
\pgfsys@useobject{currentmarker}{}%
\end{pgfscope}%
\begin{pgfscope}%
\pgfsys@transformshift{2.455770in}{1.737927in}%
\pgfsys@useobject{currentmarker}{}%
\end{pgfscope}%
\begin{pgfscope}%
\pgfsys@transformshift{2.436522in}{1.700857in}%
\pgfsys@useobject{currentmarker}{}%
\end{pgfscope}%
\begin{pgfscope}%
\pgfsys@transformshift{2.419152in}{1.598921in}%
\pgfsys@useobject{currentmarker}{}%
\end{pgfscope}%
\begin{pgfscope}%
\pgfsys@transformshift{2.399433in}{1.496744in}%
\pgfsys@useobject{currentmarker}{}%
\end{pgfscope}%
\begin{pgfscope}%
\pgfsys@transformshift{2.377134in}{1.411290in}%
\pgfsys@useobject{currentmarker}{}%
\end{pgfscope}%
\begin{pgfscope}%
\pgfsys@transformshift{2.359061in}{1.366861in}%
\pgfsys@useobject{currentmarker}{}%
\end{pgfscope}%
\begin{pgfscope}%
\pgfsys@transformshift{2.341456in}{1.350539in}%
\pgfsys@useobject{currentmarker}{}%
\end{pgfscope}%
\begin{pgfscope}%
\pgfsys@transformshift{2.322442in}{1.373401in}%
\pgfsys@useobject{currentmarker}{}%
\end{pgfscope}%
\begin{pgfscope}%
\pgfsys@transformshift{2.303429in}{1.421128in}%
\pgfsys@useobject{currentmarker}{}%
\end{pgfscope}%
\begin{pgfscope}%
\pgfsys@transformshift{2.278782in}{1.541362in}%
\pgfsys@useobject{currentmarker}{}%
\end{pgfscope}%
\begin{pgfscope}%
\pgfsys@transformshift{2.263057in}{1.641403in}%
\pgfsys@useobject{currentmarker}{}%
\end{pgfscope}%
\begin{pgfscope}%
\pgfsys@transformshift{2.247798in}{1.715516in}%
\pgfsys@useobject{currentmarker}{}%
\end{pgfscope}%
\begin{pgfscope}%
\pgfsys@transformshift{2.225970in}{1.737577in}%
\pgfsys@useobject{currentmarker}{}%
\end{pgfscope}%
\begin{pgfscope}%
\pgfsys@transformshift{2.206956in}{1.698640in}%
\pgfsys@useobject{currentmarker}{}%
\end{pgfscope}%
\begin{pgfscope}%
\pgfsys@transformshift{2.185829in}{1.605275in}%
\pgfsys@useobject{currentmarker}{}%
\end{pgfscope}%
\begin{pgfscope}%
\pgfsys@transformshift{2.170104in}{1.509048in}%
\pgfsys@useobject{currentmarker}{}%
\end{pgfscope}%
\begin{pgfscope}%
\pgfsys@transformshift{2.151794in}{1.436168in}%
\pgfsys@useobject{currentmarker}{}%
\end{pgfscope}%
\begin{pgfscope}%
\pgfsys@transformshift{2.130200in}{1.391679in}%
\pgfsys@useobject{currentmarker}{}%
\end{pgfscope}%
\begin{pgfscope}%
\pgfsys@transformshift{2.111890in}{1.357207in}%
\pgfsys@useobject{currentmarker}{}%
\end{pgfscope}%
\begin{pgfscope}%
\pgfsys@transformshift{2.090531in}{1.367908in}%
\pgfsys@useobject{currentmarker}{}%
\end{pgfscope}%
\begin{pgfscope}%
\pgfsys@transformshift{2.071752in}{1.404092in}%
\pgfsys@useobject{currentmarker}{}%
\end{pgfscope}%
\begin{pgfscope}%
\pgfsys@transformshift{2.053207in}{1.474056in}%
\pgfsys@useobject{currentmarker}{}%
\end{pgfscope}%
\begin{pgfscope}%
\pgfsys@transformshift{2.034431in}{1.584047in}%
\pgfsys@useobject{currentmarker}{}%
\end{pgfscope}%
\begin{pgfscope}%
\pgfsys@transformshift{2.017060in}{1.692434in}%
\pgfsys@useobject{currentmarker}{}%
\end{pgfscope}%
\begin{pgfscope}%
\pgfsys@transformshift{1.992882in}{1.741308in}%
\pgfsys@useobject{currentmarker}{}%
\end{pgfscope}%
\begin{pgfscope}%
\pgfsys@transformshift{1.977625in}{1.714762in}%
\pgfsys@useobject{currentmarker}{}%
\end{pgfscope}%
\begin{pgfscope}%
\pgfsys@transformshift{1.957438in}{1.730585in}%
\pgfsys@useobject{currentmarker}{}%
\end{pgfscope}%
\begin{pgfscope}%
\pgfsys@transformshift{1.936313in}{1.735893in}%
\pgfsys@useobject{currentmarker}{}%
\end{pgfscope}%
\begin{pgfscope}%
\pgfsys@transformshift{1.920117in}{1.692662in}%
\pgfsys@useobject{currentmarker}{}%
\end{pgfscope}%
\begin{pgfscope}%
\pgfsys@transformshift{1.899460in}{1.570213in}%
\pgfsys@useobject{currentmarker}{}%
\end{pgfscope}%
\begin{pgfscope}%
\pgfsys@transformshift{1.881387in}{1.471074in}%
\pgfsys@useobject{currentmarker}{}%
\end{pgfscope}%
\begin{pgfscope}%
\pgfsys@transformshift{1.860260in}{1.398007in}%
\pgfsys@useobject{currentmarker}{}%
\end{pgfscope}%
\begin{pgfscope}%
\pgfsys@transformshift{1.840309in}{1.364058in}%
\pgfsys@useobject{currentmarker}{}%
\end{pgfscope}%
\begin{pgfscope}%
\pgfsys@transformshift{1.821530in}{1.356948in}%
\pgfsys@useobject{currentmarker}{}%
\end{pgfscope}%
\begin{pgfscope}%
\pgfsys@transformshift{1.802517in}{1.385608in}%
\pgfsys@useobject{currentmarker}{}%
\end{pgfscope}%
\begin{pgfscope}%
\pgfsys@transformshift{1.784209in}{1.431654in}%
\pgfsys@useobject{currentmarker}{}%
\end{pgfscope}%
\begin{pgfscope}%
\pgfsys@transformshift{1.765664in}{1.501640in}%
\pgfsys@useobject{currentmarker}{}%
\end{pgfscope}%
\begin{pgfscope}%
\pgfsys@transformshift{1.743600in}{1.634598in}%
\pgfsys@useobject{currentmarker}{}%
\end{pgfscope}%
\begin{pgfscope}%
\pgfsys@transformshift{1.725292in}{1.696857in}%
\pgfsys@useobject{currentmarker}{}%
\end{pgfscope}%
\begin{pgfscope}%
\pgfsys@transformshift{1.706278in}{1.742551in}%
\pgfsys@useobject{currentmarker}{}%
\end{pgfscope}%
\begin{pgfscope}%
\pgfsys@transformshift{1.689142in}{1.743856in}%
\pgfsys@useobject{currentmarker}{}%
\end{pgfscope}%
\begin{pgfscope}%
\pgfsys@transformshift{1.668721in}{1.697521in}%
\pgfsys@useobject{currentmarker}{}%
\end{pgfscope}%
\begin{pgfscope}%
\pgfsys@transformshift{1.647596in}{1.588964in}%
\pgfsys@useobject{currentmarker}{}%
\end{pgfscope}%
\begin{pgfscope}%
\pgfsys@transformshift{1.628817in}{1.486781in}%
\pgfsys@useobject{currentmarker}{}%
\end{pgfscope}%
\begin{pgfscope}%
\pgfsys@transformshift{1.611212in}{1.424665in}%
\pgfsys@useobject{currentmarker}{}%
\end{pgfscope}%
\begin{pgfscope}%
\pgfsys@transformshift{1.592904in}{1.384315in}%
\pgfsys@useobject{currentmarker}{}%
\end{pgfscope}%
\begin{pgfscope}%
\pgfsys@transformshift{1.571074in}{1.358025in}%
\pgfsys@useobject{currentmarker}{}%
\end{pgfscope}%
\begin{pgfscope}%
\pgfsys@transformshift{1.556286in}{1.365443in}%
\pgfsys@useobject{currentmarker}{}%
\end{pgfscope}%
\begin{pgfscope}%
\pgfsys@transformshift{1.531873in}{1.413162in}%
\pgfsys@useobject{currentmarker}{}%
\end{pgfscope}%
\begin{pgfscope}%
\pgfsys@transformshift{1.515913in}{1.464345in}%
\pgfsys@useobject{currentmarker}{}%
\end{pgfscope}%
\begin{pgfscope}%
\pgfsys@transformshift{1.497369in}{1.554325in}%
\pgfsys@useobject{currentmarker}{}%
\end{pgfscope}%
\begin{pgfscope}%
\pgfsys@transformshift{1.478590in}{1.665865in}%
\pgfsys@useobject{currentmarker}{}%
\end{pgfscope}%
\begin{pgfscope}%
\pgfsys@transformshift{1.458639in}{1.739034in}%
\pgfsys@useobject{currentmarker}{}%
\end{pgfscope}%
\begin{pgfscope}%
\pgfsys@transformshift{1.438217in}{1.750912in}%
\pgfsys@useobject{currentmarker}{}%
\end{pgfscope}%
\begin{pgfscope}%
\pgfsys@transformshift{1.416856in}{1.728373in}%
\pgfsys@useobject{currentmarker}{}%
\end{pgfscope}%
\begin{pgfscope}%
\pgfsys@transformshift{1.398314in}{1.676916in}%
\pgfsys@useobject{currentmarker}{}%
\end{pgfscope}%
\begin{pgfscope}%
\pgfsys@transformshift{1.379066in}{1.601756in}%
\pgfsys@useobject{currentmarker}{}%
\end{pgfscope}%
\begin{pgfscope}%
\pgfsys@transformshift{1.361695in}{1.522741in}%
\pgfsys@useobject{currentmarker}{}%
\end{pgfscope}%
\begin{pgfscope}%
\pgfsys@transformshift{1.336580in}{1.421305in}%
\pgfsys@useobject{currentmarker}{}%
\end{pgfscope}%
\begin{pgfscope}%
\pgfsys@transformshift{1.324138in}{1.391577in}%
\pgfsys@useobject{currentmarker}{}%
\end{pgfscope}%
\begin{pgfscope}%
\pgfsys@transformshift{1.302544in}{1.362954in}%
\pgfsys@useobject{currentmarker}{}%
\end{pgfscope}%
\begin{pgfscope}%
\pgfsys@transformshift{1.284705in}{1.370830in}%
\pgfsys@useobject{currentmarker}{}%
\end{pgfscope}%
\begin{pgfscope}%
\pgfsys@transformshift{1.265457in}{1.396044in}%
\pgfsys@useobject{currentmarker}{}%
\end{pgfscope}%
\begin{pgfscope}%
\pgfsys@transformshift{1.245973in}{1.460104in}%
\pgfsys@useobject{currentmarker}{}%
\end{pgfscope}%
\begin{pgfscope}%
\pgfsys@transformshift{1.225317in}{1.526997in}%
\pgfsys@useobject{currentmarker}{}%
\end{pgfscope}%
\begin{pgfscope}%
\pgfsys@transformshift{1.207948in}{1.638181in}%
\pgfsys@useobject{currentmarker}{}%
\end{pgfscope}%
\begin{pgfscope}%
\pgfsys@transformshift{1.188701in}{1.722754in}%
\pgfsys@useobject{currentmarker}{}%
\end{pgfscope}%
\begin{pgfscope}%
\pgfsys@transformshift{1.169922in}{1.753233in}%
\pgfsys@useobject{currentmarker}{}%
\end{pgfscope}%
\begin{pgfscope}%
\pgfsys@transformshift{1.148561in}{1.756521in}%
\pgfsys@useobject{currentmarker}{}%
\end{pgfscope}%
\begin{pgfscope}%
\pgfsys@transformshift{1.133069in}{1.736911in}%
\pgfsys@useobject{currentmarker}{}%
\end{pgfscope}%
\begin{pgfscope}%
\pgfsys@transformshift{1.111943in}{1.694215in}%
\pgfsys@useobject{currentmarker}{}%
\end{pgfscope}%
\begin{pgfscope}%
\pgfsys@transformshift{1.093166in}{1.599610in}%
\pgfsys@useobject{currentmarker}{}%
\end{pgfscope}%
\begin{pgfscope}%
\pgfsys@transformshift{1.070867in}{1.499968in}%
\pgfsys@useobject{currentmarker}{}%
\end{pgfscope}%
\begin{pgfscope}%
\pgfsys@transformshift{1.052322in}{1.434989in}%
\pgfsys@useobject{currentmarker}{}%
\end{pgfscope}%
\begin{pgfscope}%
\pgfsys@transformshift{1.034014in}{1.396173in}%
\pgfsys@useobject{currentmarker}{}%
\end{pgfscope}%
\begin{pgfscope}%
\pgfsys@transformshift{1.015470in}{1.374369in}%
\pgfsys@useobject{currentmarker}{}%
\end{pgfscope}%
\begin{pgfscope}%
\pgfsys@transformshift{0.994579in}{1.367731in}%
\pgfsys@useobject{currentmarker}{}%
\end{pgfscope}%
\begin{pgfscope}%
\pgfsys@transformshift{0.975331in}{1.403450in}%
\pgfsys@useobject{currentmarker}{}%
\end{pgfscope}%
\begin{pgfscope}%
\pgfsys@transformshift{0.957256in}{1.454950in}%
\pgfsys@useobject{currentmarker}{}%
\end{pgfscope}%
\begin{pgfscope}%
\pgfsys@transformshift{0.938713in}{1.471170in}%
\pgfsys@useobject{currentmarker}{}%
\end{pgfscope}%
\begin{pgfscope}%
\pgfsys@transformshift{0.919935in}{1.572143in}%
\pgfsys@useobject{currentmarker}{}%
\end{pgfscope}%
\begin{pgfscope}%
\pgfsys@transformshift{0.901390in}{1.672985in}%
\pgfsys@useobject{currentmarker}{}%
\end{pgfscope}%
\begin{pgfscope}%
\pgfsys@transformshift{0.882613in}{1.739341in}%
\pgfsys@useobject{currentmarker}{}%
\end{pgfscope}%
\begin{pgfscope}%
\pgfsys@transformshift{0.860549in}{1.765575in}%
\pgfsys@useobject{currentmarker}{}%
\end{pgfscope}%
\begin{pgfscope}%
\pgfsys@transformshift{0.842709in}{1.750081in}%
\pgfsys@useobject{currentmarker}{}%
\end{pgfscope}%
\begin{pgfscope}%
\pgfsys@transformshift{0.821348in}{1.686504in}%
\pgfsys@useobject{currentmarker}{}%
\end{pgfscope}%
\begin{pgfscope}%
\pgfsys@transformshift{0.802806in}{1.592956in}%
\pgfsys@useobject{currentmarker}{}%
\end{pgfscope}%
\begin{pgfscope}%
\pgfsys@transformshift{0.784730in}{1.511604in}%
\pgfsys@useobject{currentmarker}{}%
\end{pgfscope}%
\begin{pgfscope}%
\pgfsys@transformshift{0.768065in}{1.494140in}%
\pgfsys@useobject{currentmarker}{}%
\end{pgfscope}%
\begin{pgfscope}%
\pgfsys@transformshift{0.747878in}{1.424207in}%
\pgfsys@useobject{currentmarker}{}%
\end{pgfscope}%
\begin{pgfscope}%
\pgfsys@transformshift{0.727456in}{1.390793in}%
\pgfsys@useobject{currentmarker}{}%
\end{pgfscope}%
\begin{pgfscope}%
\pgfsys@transformshift{0.707036in}{1.370455in}%
\pgfsys@useobject{currentmarker}{}%
\end{pgfscope}%
\begin{pgfscope}%
\pgfsys@transformshift{0.686144in}{1.395782in}%
\pgfsys@useobject{currentmarker}{}%
\end{pgfscope}%
\begin{pgfscope}%
\pgfsys@transformshift{0.669244in}{1.442164in}%
\pgfsys@useobject{currentmarker}{}%
\end{pgfscope}%
\begin{pgfscope}%
\pgfsys@transformshift{0.651170in}{1.493367in}%
\pgfsys@useobject{currentmarker}{}%
\end{pgfscope}%
\begin{pgfscope}%
\pgfsys@transformshift{0.652579in}{1.485101in}%
\pgfsys@useobject{currentmarker}{}%
\end{pgfscope}%
\begin{pgfscope}%
\pgfsys@transformshift{0.657742in}{1.461606in}%
\pgfsys@useobject{currentmarker}{}%
\end{pgfscope}%
\begin{pgfscope}%
\pgfsys@transformshift{0.675347in}{1.404951in}%
\pgfsys@useobject{currentmarker}{}%
\end{pgfscope}%
\begin{pgfscope}%
\pgfsys@transformshift{0.697646in}{1.502606in}%
\pgfsys@useobject{currentmarker}{}%
\end{pgfscope}%
\begin{pgfscope}%
\pgfsys@transformshift{0.715250in}{1.629307in}%
\pgfsys@useobject{currentmarker}{}%
\end{pgfscope}%
\begin{pgfscope}%
\pgfsys@transformshift{0.732621in}{1.731012in}%
\pgfsys@useobject{currentmarker}{}%
\end{pgfscope}%
\begin{pgfscope}%
\pgfsys@transformshift{0.751400in}{1.766961in}%
\pgfsys@useobject{currentmarker}{}%
\end{pgfscope}%
\begin{pgfscope}%
\pgfsys@transformshift{0.773699in}{1.702016in}%
\pgfsys@useobject{currentmarker}{}%
\end{pgfscope}%
\begin{pgfscope}%
\pgfsys@transformshift{0.791538in}{1.556851in}%
\pgfsys@useobject{currentmarker}{}%
\end{pgfscope}%
\begin{pgfscope}%
\pgfsys@transformshift{0.811020in}{1.447742in}%
\pgfsys@useobject{currentmarker}{}%
\end{pgfscope}%
\begin{pgfscope}%
\pgfsys@transformshift{0.829799in}{1.379461in}%
\pgfsys@useobject{currentmarker}{}%
\end{pgfscope}%
\begin{pgfscope}%
\pgfsys@transformshift{0.847169in}{1.370690in}%
\pgfsys@useobject{currentmarker}{}%
\end{pgfscope}%
\begin{pgfscope}%
\pgfsys@transformshift{0.870406in}{1.440649in}%
\pgfsys@useobject{currentmarker}{}%
\end{pgfscope}%
\begin{pgfscope}%
\pgfsys@transformshift{0.887307in}{1.688526in}%
\pgfsys@useobject{currentmarker}{}%
\end{pgfscope}%
\begin{pgfscope}%
\pgfsys@transformshift{0.906321in}{1.758158in}%
\pgfsys@useobject{currentmarker}{}%
\end{pgfscope}%
\begin{pgfscope}%
\pgfsys@transformshift{0.926272in}{1.735777in}%
\pgfsys@useobject{currentmarker}{}%
\end{pgfscope}%
\begin{pgfscope}%
\pgfsys@transformshift{0.945285in}{1.622502in}%
\pgfsys@useobject{currentmarker}{}%
\end{pgfscope}%
\begin{pgfscope}%
\pgfsys@transformshift{0.963595in}{1.480958in}%
\pgfsys@useobject{currentmarker}{}%
\end{pgfscope}%
\begin{pgfscope}%
\pgfsys@transformshift{0.987302in}{1.383188in}%
\pgfsys@useobject{currentmarker}{}%
\end{pgfscope}%
\begin{pgfscope}%
\pgfsys@transformshift{1.003733in}{1.361108in}%
\pgfsys@useobject{currentmarker}{}%
\end{pgfscope}%
\begin{pgfscope}%
\pgfsys@transformshift{1.021104in}{1.399808in}%
\pgfsys@useobject{currentmarker}{}%
\end{pgfscope}%
\begin{pgfscope}%
\pgfsys@transformshift{1.039882in}{1.474669in}%
\pgfsys@useobject{currentmarker}{}%
\end{pgfscope}%
\begin{pgfscope}%
\pgfsys@transformshift{1.061945in}{1.614743in}%
\pgfsys@useobject{currentmarker}{}%
\end{pgfscope}%
\begin{pgfscope}%
\pgfsys@transformshift{1.081664in}{1.726064in}%
\pgfsys@useobject{currentmarker}{}%
\end{pgfscope}%
\begin{pgfscope}%
\pgfsys@transformshift{1.098799in}{1.752776in}%
\pgfsys@useobject{currentmarker}{}%
\end{pgfscope}%
\begin{pgfscope}%
\pgfsys@transformshift{1.115465in}{1.710596in}%
\pgfsys@useobject{currentmarker}{}%
\end{pgfscope}%
\begin{pgfscope}%
\pgfsys@transformshift{1.139172in}{1.542124in}%
\pgfsys@useobject{currentmarker}{}%
\end{pgfscope}%
\begin{pgfscope}%
\pgfsys@transformshift{1.158889in}{1.434160in}%
\pgfsys@useobject{currentmarker}{}%
\end{pgfscope}%
\begin{pgfscope}%
\pgfsys@transformshift{1.176964in}{1.371348in}%
\pgfsys@useobject{currentmarker}{}%
\end{pgfscope}%
\begin{pgfscope}%
\pgfsys@transformshift{1.196446in}{1.363660in}%
\pgfsys@useobject{currentmarker}{}%
\end{pgfscope}%
\begin{pgfscope}%
\pgfsys@transformshift{1.215223in}{1.408978in}%
\pgfsys@useobject{currentmarker}{}%
\end{pgfscope}%
\begin{pgfscope}%
\pgfsys@transformshift{1.233768in}{1.485142in}%
\pgfsys@useobject{currentmarker}{}%
\end{pgfscope}%
\begin{pgfscope}%
\pgfsys@transformshift{1.250669in}{1.573960in}%
\pgfsys@useobject{currentmarker}{}%
\end{pgfscope}%
\begin{pgfscope}%
\pgfsys@transformshift{1.271794in}{1.716942in}%
\pgfsys@useobject{currentmarker}{}%
\end{pgfscope}%
\begin{pgfscope}%
\pgfsys@transformshift{1.290807in}{1.748724in}%
\pgfsys@useobject{currentmarker}{}%
\end{pgfscope}%
\begin{pgfscope}%
\pgfsys@transformshift{1.310758in}{1.721410in}%
\pgfsys@useobject{currentmarker}{}%
\end{pgfscope}%
\begin{pgfscope}%
\pgfsys@transformshift{1.331180in}{1.581925in}%
\pgfsys@useobject{currentmarker}{}%
\end{pgfscope}%
\begin{pgfscope}%
\pgfsys@transformshift{1.346908in}{1.488287in}%
\pgfsys@useobject{currentmarker}{}%
\end{pgfscope}%
\begin{pgfscope}%
\pgfsys@transformshift{1.368972in}{1.398738in}%
\pgfsys@useobject{currentmarker}{}%
\end{pgfscope}%
\begin{pgfscope}%
\pgfsys@transformshift{1.390332in}{1.356411in}%
\pgfsys@useobject{currentmarker}{}%
\end{pgfscope}%
\begin{pgfscope}%
\pgfsys@transformshift{1.405120in}{1.362482in}%
\pgfsys@useobject{currentmarker}{}%
\end{pgfscope}%
\begin{pgfscope}%
\pgfsys@transformshift{1.427889in}{1.427511in}%
\pgfsys@useobject{currentmarker}{}%
\end{pgfscope}%
\begin{pgfscope}%
\pgfsys@transformshift{1.444554in}{1.502541in}%
\pgfsys@useobject{currentmarker}{}%
\end{pgfscope}%
\begin{pgfscope}%
\pgfsys@transformshift{1.462159in}{1.618979in}%
\pgfsys@useobject{currentmarker}{}%
\end{pgfscope}%
\begin{pgfscope}%
\pgfsys@transformshift{1.482112in}{1.362216in}%
\pgfsys@useobject{currentmarker}{}%
\end{pgfscope}%
\begin{pgfscope}%
\pgfsys@transformshift{1.503003in}{1.364639in}%
\pgfsys@useobject{currentmarker}{}%
\end{pgfscope}%
\begin{pgfscope}%
\pgfsys@transformshift{1.522016in}{1.414824in}%
\pgfsys@useobject{currentmarker}{}%
\end{pgfscope}%
\begin{pgfscope}%
\pgfsys@transformshift{1.541029in}{1.497321in}%
\pgfsys@useobject{currentmarker}{}%
\end{pgfscope}%
\begin{pgfscope}%
\pgfsys@transformshift{1.560043in}{1.637311in}%
\pgfsys@useobject{currentmarker}{}%
\end{pgfscope}%
\begin{pgfscope}%
\pgfsys@transformshift{1.578585in}{1.725709in}%
\pgfsys@useobject{currentmarker}{}%
\end{pgfscope}%
\begin{pgfscope}%
\pgfsys@transformshift{1.596190in}{1.741629in}%
\pgfsys@useobject{currentmarker}{}%
\end{pgfscope}%
\begin{pgfscope}%
\pgfsys@transformshift{1.617786in}{1.666144in}%
\pgfsys@useobject{currentmarker}{}%
\end{pgfscope}%
\begin{pgfscope}%
\pgfsys@transformshift{1.636799in}{1.565934in}%
\pgfsys@useobject{currentmarker}{}%
\end{pgfscope}%
\begin{pgfscope}%
\pgfsys@transformshift{1.655107in}{1.448946in}%
\pgfsys@useobject{currentmarker}{}%
\end{pgfscope}%
\begin{pgfscope}%
\pgfsys@transformshift{1.674120in}{1.381353in}%
\pgfsys@useobject{currentmarker}{}%
\end{pgfscope}%
\begin{pgfscope}%
\pgfsys@transformshift{1.695481in}{1.352955in}%
\pgfsys@useobject{currentmarker}{}%
\end{pgfscope}%
\begin{pgfscope}%
\pgfsys@transformshift{1.713084in}{1.374686in}%
\pgfsys@useobject{currentmarker}{}%
\end{pgfscope}%
\begin{pgfscope}%
\pgfsys@transformshift{1.733977in}{1.449944in}%
\pgfsys@useobject{currentmarker}{}%
\end{pgfscope}%
\begin{pgfscope}%
\pgfsys@transformshift{1.752990in}{1.553851in}%
\pgfsys@useobject{currentmarker}{}%
\end{pgfscope}%
\begin{pgfscope}%
\pgfsys@transformshift{1.771064in}{1.678806in}%
\pgfsys@useobject{currentmarker}{}%
\end{pgfscope}%
\begin{pgfscope}%
\pgfsys@transformshift{1.791485in}{1.739263in}%
\pgfsys@useobject{currentmarker}{}%
\end{pgfscope}%
\begin{pgfscope}%
\pgfsys@transformshift{1.809325in}{1.729358in}%
\pgfsys@useobject{currentmarker}{}%
\end{pgfscope}%
\begin{pgfscope}%
\pgfsys@transformshift{1.830450in}{1.622440in}%
\pgfsys@useobject{currentmarker}{}%
\end{pgfscope}%
\begin{pgfscope}%
\pgfsys@transformshift{1.849463in}{1.508600in}%
\pgfsys@useobject{currentmarker}{}%
\end{pgfscope}%
\begin{pgfscope}%
\pgfsys@transformshift{1.865190in}{1.421130in}%
\pgfsys@useobject{currentmarker}{}%
\end{pgfscope}%
\begin{pgfscope}%
\pgfsys@transformshift{1.887724in}{1.370199in}%
\pgfsys@useobject{currentmarker}{}%
\end{pgfscope}%
\begin{pgfscope}%
\pgfsys@transformshift{1.905798in}{1.352554in}%
\pgfsys@useobject{currentmarker}{}%
\end{pgfscope}%
\begin{pgfscope}%
\pgfsys@transformshift{1.925516in}{1.371507in}%
\pgfsys@useobject{currentmarker}{}%
\end{pgfscope}%
\begin{pgfscope}%
\pgfsys@transformshift{1.945232in}{1.429496in}%
\pgfsys@useobject{currentmarker}{}%
\end{pgfscope}%
\begin{pgfscope}%
\pgfsys@transformshift{1.963777in}{1.504666in}%
\pgfsys@useobject{currentmarker}{}%
\end{pgfscope}%
\begin{pgfscope}%
\pgfsys@transformshift{1.981145in}{1.629457in}%
\pgfsys@useobject{currentmarker}{}%
\end{pgfscope}%
\begin{pgfscope}%
\pgfsys@transformshift{1.999924in}{1.704126in}%
\pgfsys@useobject{currentmarker}{}%
\end{pgfscope}%
\begin{pgfscope}%
\pgfsys@transformshift{2.020111in}{1.739419in}%
\pgfsys@useobject{currentmarker}{}%
\end{pgfscope}%
\begin{pgfscope}%
\pgfsys@transformshift{2.042176in}{1.700077in}%
\pgfsys@useobject{currentmarker}{}%
\end{pgfscope}%
\begin{pgfscope}%
\pgfsys@transformshift{2.059781in}{1.739403in}%
\pgfsys@useobject{currentmarker}{}%
\end{pgfscope}%
\begin{pgfscope}%
\pgfsys@transformshift{2.077620in}{1.711669in}%
\pgfsys@useobject{currentmarker}{}%
\end{pgfscope}%
\begin{pgfscope}%
\pgfsys@transformshift{2.101562in}{1.564351in}%
\pgfsys@useobject{currentmarker}{}%
\end{pgfscope}%
\begin{pgfscope}%
\pgfsys@transformshift{2.113299in}{1.469193in}%
\pgfsys@useobject{currentmarker}{}%
\end{pgfscope}%
\begin{pgfscope}%
\pgfsys@transformshift{2.137711in}{1.393639in}%
\pgfsys@useobject{currentmarker}{}%
\end{pgfscope}%
\begin{pgfscope}%
\pgfsys@transformshift{2.155316in}{1.363228in}%
\pgfsys@useobject{currentmarker}{}%
\end{pgfscope}%
\begin{pgfscope}%
\pgfsys@transformshift{2.175972in}{1.357928in}%
\pgfsys@useobject{currentmarker}{}%
\end{pgfscope}%
\begin{pgfscope}%
\pgfsys@transformshift{2.193341in}{1.394456in}%
\pgfsys@useobject{currentmarker}{}%
\end{pgfscope}%
\begin{pgfscope}%
\pgfsys@transformshift{2.211885in}{1.456592in}%
\pgfsys@useobject{currentmarker}{}%
\end{pgfscope}%
\begin{pgfscope}%
\pgfsys@transformshift{2.232307in}{1.567897in}%
\pgfsys@useobject{currentmarker}{}%
\end{pgfscope}%
\begin{pgfscope}%
\pgfsys@transformshift{2.254137in}{1.693296in}%
\pgfsys@useobject{currentmarker}{}%
\end{pgfscope}%
\begin{pgfscope}%
\pgfsys@transformshift{2.269394in}{1.732710in}%
\pgfsys@useobject{currentmarker}{}%
\end{pgfscope}%
\begin{pgfscope}%
\pgfsys@transformshift{2.289347in}{1.723349in}%
\pgfsys@useobject{currentmarker}{}%
\end{pgfscope}%
\begin{pgfscope}%
\pgfsys@transformshift{2.309063in}{1.640054in}%
\pgfsys@useobject{currentmarker}{}%
\end{pgfscope}%
\begin{pgfscope}%
\pgfsys@transformshift{2.327137in}{1.514475in}%
\pgfsys@useobject{currentmarker}{}%
\end{pgfscope}%
\begin{pgfscope}%
\pgfsys@transformshift{2.346619in}{1.425201in}%
\pgfsys@useobject{currentmarker}{}%
\end{pgfscope}%
\begin{pgfscope}%
\pgfsys@transformshift{2.367277in}{1.369675in}%
\pgfsys@useobject{currentmarker}{}%
\end{pgfscope}%
\begin{pgfscope}%
\pgfsys@transformshift{2.384176in}{1.350277in}%
\pgfsys@useobject{currentmarker}{}%
\end{pgfscope}%
\begin{pgfscope}%
\pgfsys@transformshift{2.406241in}{1.375885in}%
\pgfsys@useobject{currentmarker}{}%
\end{pgfscope}%
\begin{pgfscope}%
\pgfsys@transformshift{2.426428in}{1.426845in}%
\pgfsys@useobject{currentmarker}{}%
\end{pgfscope}%
\begin{pgfscope}%
\pgfsys@transformshift{2.443562in}{1.499752in}%
\pgfsys@useobject{currentmarker}{}%
\end{pgfscope}%
\begin{pgfscope}%
\pgfsys@transformshift{2.463281in}{1.572163in}%
\pgfsys@useobject{currentmarker}{}%
\end{pgfscope}%
\begin{pgfscope}%
\pgfsys@transformshift{2.482058in}{1.633912in}%
\pgfsys@useobject{currentmarker}{}%
\end{pgfscope}%
\begin{pgfscope}%
\pgfsys@transformshift{2.499663in}{1.721845in}%
\pgfsys@useobject{currentmarker}{}%
\end{pgfscope}%
\begin{pgfscope}%
\pgfsys@transformshift{2.520789in}{1.737259in}%
\pgfsys@useobject{currentmarker}{}%
\end{pgfscope}%
\begin{pgfscope}%
\pgfsys@transformshift{2.539332in}{1.696289in}%
\pgfsys@useobject{currentmarker}{}%
\end{pgfscope}%
\begin{pgfscope}%
\pgfsys@transformshift{2.557408in}{1.590372in}%
\pgfsys@useobject{currentmarker}{}%
\end{pgfscope}%
\begin{pgfscope}%
\pgfsys@transformshift{2.577827in}{1.455067in}%
\pgfsys@useobject{currentmarker}{}%
\end{pgfscope}%
\begin{pgfscope}%
\pgfsys@transformshift{2.598954in}{1.387082in}%
\pgfsys@useobject{currentmarker}{}%
\end{pgfscope}%
\begin{pgfscope}%
\pgfsys@transformshift{2.616559in}{1.366526in}%
\pgfsys@useobject{currentmarker}{}%
\end{pgfscope}%
\begin{pgfscope}%
\pgfsys@transformshift{2.634867in}{1.351979in}%
\pgfsys@useobject{currentmarker}{}%
\end{pgfscope}%
\begin{pgfscope}%
\pgfsys@transformshift{2.655523in}{1.384016in}%
\pgfsys@useobject{currentmarker}{}%
\end{pgfscope}%
\begin{pgfscope}%
\pgfsys@transformshift{2.673597in}{1.425125in}%
\pgfsys@useobject{currentmarker}{}%
\end{pgfscope}%
\begin{pgfscope}%
\pgfsys@transformshift{2.692141in}{1.507594in}%
\pgfsys@useobject{currentmarker}{}%
\end{pgfscope}%
\begin{pgfscope}%
\pgfsys@transformshift{2.715146in}{1.632918in}%
\pgfsys@useobject{currentmarker}{}%
\end{pgfscope}%
\begin{pgfscope}%
\pgfsys@transformshift{2.732514in}{1.682654in}%
\pgfsys@useobject{currentmarker}{}%
\end{pgfscope}%
\begin{pgfscope}%
\pgfsys@transformshift{2.750119in}{1.723368in}%
\pgfsys@useobject{currentmarker}{}%
\end{pgfscope}%
\begin{pgfscope}%
\pgfsys@transformshift{2.770072in}{1.734701in}%
\pgfsys@useobject{currentmarker}{}%
\end{pgfscope}%
\begin{pgfscope}%
\pgfsys@transformshift{2.790728in}{1.657029in}%
\pgfsys@useobject{currentmarker}{}%
\end{pgfscope}%
\begin{pgfscope}%
\pgfsys@transformshift{2.808333in}{1.531853in}%
\pgfsys@useobject{currentmarker}{}%
\end{pgfscope}%
\begin{pgfscope}%
\pgfsys@transformshift{2.826406in}{1.445967in}%
\pgfsys@useobject{currentmarker}{}%
\end{pgfscope}%
\begin{pgfscope}%
\pgfsys@transformshift{2.847533in}{1.379383in}%
\pgfsys@useobject{currentmarker}{}%
\end{pgfscope}%
\begin{pgfscope}%
\pgfsys@transformshift{2.864433in}{1.355985in}%
\pgfsys@useobject{currentmarker}{}%
\end{pgfscope}%
\begin{pgfscope}%
\pgfsys@transformshift{2.883680in}{1.364269in}%
\pgfsys@useobject{currentmarker}{}%
\end{pgfscope}%
\begin{pgfscope}%
\pgfsys@transformshift{2.901754in}{1.411803in}%
\pgfsys@useobject{currentmarker}{}%
\end{pgfscope}%
\begin{pgfscope}%
\pgfsys@transformshift{2.923115in}{1.484911in}%
\pgfsys@useobject{currentmarker}{}%
\end{pgfscope}%
\begin{pgfscope}%
\pgfsys@transformshift{2.945180in}{1.602409in}%
\pgfsys@useobject{currentmarker}{}%
\end{pgfscope}%
\begin{pgfscope}%
\pgfsys@transformshift{2.964662in}{1.671259in}%
\pgfsys@useobject{currentmarker}{}%
\end{pgfscope}%
\begin{pgfscope}%
\pgfsys@transformshift{2.980390in}{1.722448in}%
\pgfsys@useobject{currentmarker}{}%
\end{pgfscope}%
\begin{pgfscope}%
\pgfsys@transformshift{3.002220in}{1.737880in}%
\pgfsys@useobject{currentmarker}{}%
\end{pgfscope}%
\begin{pgfscope}%
\pgfsys@transformshift{3.019823in}{1.681759in}%
\pgfsys@useobject{currentmarker}{}%
\end{pgfscope}%
\begin{pgfscope}%
\pgfsys@transformshift{3.043767in}{1.544448in}%
\pgfsys@useobject{currentmarker}{}%
\end{pgfscope}%
\begin{pgfscope}%
\pgfsys@transformshift{3.057380in}{1.474182in}%
\pgfsys@useobject{currentmarker}{}%
\end{pgfscope}%
\begin{pgfscope}%
\pgfsys@transformshift{3.078271in}{1.399168in}%
\pgfsys@useobject{currentmarker}{}%
\end{pgfscope}%
\begin{pgfscope}%
\pgfsys@transformshift{3.097519in}{1.368047in}%
\pgfsys@useobject{currentmarker}{}%
\end{pgfscope}%
\begin{pgfscope}%
\pgfsys@transformshift{3.115123in}{1.354725in}%
\pgfsys@useobject{currentmarker}{}%
\end{pgfscope}%
\begin{pgfscope}%
\pgfsys@transformshift{3.137422in}{1.390643in}%
\pgfsys@useobject{currentmarker}{}%
\end{pgfscope}%
\begin{pgfscope}%
\pgfsys@transformshift{3.152681in}{1.428699in}%
\pgfsys@useobject{currentmarker}{}%
\end{pgfscope}%
\begin{pgfscope}%
\pgfsys@transformshift{3.172398in}{1.486253in}%
\pgfsys@useobject{currentmarker}{}%
\end{pgfscope}%
\begin{pgfscope}%
\pgfsys@transformshift{3.192114in}{1.589747in}%
\pgfsys@useobject{currentmarker}{}%
\end{pgfscope}%
\begin{pgfscope}%
\pgfsys@transformshift{3.215118in}{1.703249in}%
\pgfsys@useobject{currentmarker}{}%
\end{pgfscope}%
\begin{pgfscope}%
\pgfsys@transformshift{3.230375in}{1.727692in}%
\pgfsys@useobject{currentmarker}{}%
\end{pgfscope}%
\begin{pgfscope}%
\pgfsys@transformshift{3.249623in}{1.744786in}%
\pgfsys@useobject{currentmarker}{}%
\end{pgfscope}%
\begin{pgfscope}%
\pgfsys@transformshift{3.270279in}{1.708341in}%
\pgfsys@useobject{currentmarker}{}%
\end{pgfscope}%
\begin{pgfscope}%
\pgfsys@transformshift{3.288354in}{1.625597in}%
\pgfsys@useobject{currentmarker}{}%
\end{pgfscope}%
\begin{pgfscope}%
\pgfsys@transformshift{3.309714in}{1.509466in}%
\pgfsys@useobject{currentmarker}{}%
\end{pgfscope}%
\begin{pgfscope}%
\pgfsys@transformshift{3.326850in}{1.440639in}%
\pgfsys@useobject{currentmarker}{}%
\end{pgfscope}%
\begin{pgfscope}%
\pgfsys@transformshift{3.347272in}{1.391374in}%
\pgfsys@useobject{currentmarker}{}%
\end{pgfscope}%
\begin{pgfscope}%
\pgfsys@transformshift{3.367222in}{1.358288in}%
\pgfsys@useobject{currentmarker}{}%
\end{pgfscope}%
\begin{pgfscope}%
\pgfsys@transformshift{3.386001in}{1.360857in}%
\pgfsys@useobject{currentmarker}{}%
\end{pgfscope}%
\begin{pgfscope}%
\pgfsys@transformshift{3.404075in}{1.384767in}%
\pgfsys@useobject{currentmarker}{}%
\end{pgfscope}%
\begin{pgfscope}%
\pgfsys@transformshift{3.424028in}{1.448909in}%
\pgfsys@useobject{currentmarker}{}%
\end{pgfscope}%
\begin{pgfscope}%
\pgfsys@transformshift{3.442101in}{1.519925in}%
\pgfsys@useobject{currentmarker}{}%
\end{pgfscope}%
\begin{pgfscope}%
\pgfsys@transformshift{3.460411in}{1.585270in}%
\pgfsys@useobject{currentmarker}{}%
\end{pgfscope}%
\begin{pgfscope}%
\pgfsys@transformshift{3.483179in}{1.695704in}%
\pgfsys@useobject{currentmarker}{}%
\end{pgfscope}%
\begin{pgfscope}%
\pgfsys@transformshift{3.500550in}{1.449145in}%
\pgfsys@useobject{currentmarker}{}%
\end{pgfscope}%
\begin{pgfscope}%
\pgfsys@transformshift{3.523788in}{1.582301in}%
\pgfsys@useobject{currentmarker}{}%
\end{pgfscope}%
\begin{pgfscope}%
\pgfsys@transformshift{3.537402in}{1.640906in}%
\pgfsys@useobject{currentmarker}{}%
\end{pgfscope}%
\begin{pgfscope}%
\pgfsys@transformshift{3.558996in}{1.737203in}%
\pgfsys@useobject{currentmarker}{}%
\end{pgfscope}%
\begin{pgfscope}%
\pgfsys@transformshift{3.577072in}{1.750110in}%
\pgfsys@useobject{currentmarker}{}%
\end{pgfscope}%
\begin{pgfscope}%
\pgfsys@transformshift{3.594676in}{1.720050in}%
\pgfsys@useobject{currentmarker}{}%
\end{pgfscope}%
\begin{pgfscope}%
\pgfsys@transformshift{3.615801in}{1.623586in}%
\pgfsys@useobject{currentmarker}{}%
\end{pgfscope}%
\begin{pgfscope}%
\pgfsys@transformshift{3.632937in}{1.531160in}%
\pgfsys@useobject{currentmarker}{}%
\end{pgfscope}%
\begin{pgfscope}%
\pgfsys@transformshift{3.655002in}{1.445543in}%
\pgfsys@useobject{currentmarker}{}%
\end{pgfscope}%
\begin{pgfscope}%
\pgfsys@transformshift{3.672370in}{1.390387in}%
\pgfsys@useobject{currentmarker}{}%
\end{pgfscope}%
\begin{pgfscope}%
\pgfsys@transformshift{3.690680in}{1.361162in}%
\pgfsys@useobject{currentmarker}{}%
\end{pgfscope}%
\begin{pgfscope}%
\pgfsys@transformshift{3.712040in}{1.376046in}%
\pgfsys@useobject{currentmarker}{}%
\end{pgfscope}%
\begin{pgfscope}%
\pgfsys@transformshift{3.733167in}{1.413578in}%
\pgfsys@useobject{currentmarker}{}%
\end{pgfscope}%
\begin{pgfscope}%
\pgfsys@transformshift{3.751475in}{1.442412in}%
\pgfsys@useobject{currentmarker}{}%
\end{pgfscope}%
\begin{pgfscope}%
\pgfsys@transformshift{3.769314in}{1.507962in}%
\pgfsys@useobject{currentmarker}{}%
\end{pgfscope}%
\begin{pgfscope}%
\pgfsys@transformshift{3.789267in}{1.627234in}%
\pgfsys@useobject{currentmarker}{}%
\end{pgfscope}%
\begin{pgfscope}%
\pgfsys@transformshift{3.808515in}{1.717199in}%
\pgfsys@useobject{currentmarker}{}%
\end{pgfscope}%
\begin{pgfscope}%
\pgfsys@transformshift{3.825885in}{1.754361in}%
\pgfsys@useobject{currentmarker}{}%
\end{pgfscope}%
\begin{pgfscope}%
\pgfsys@transformshift{3.846776in}{1.749647in}%
\pgfsys@useobject{currentmarker}{}%
\end{pgfscope}%
\begin{pgfscope}%
\pgfsys@transformshift{3.865084in}{1.709938in}%
\pgfsys@useobject{currentmarker}{}%
\end{pgfscope}%
\begin{pgfscope}%
\pgfsys@transformshift{3.886445in}{1.593728in}%
\pgfsys@useobject{currentmarker}{}%
\end{pgfscope}%
\begin{pgfscope}%
\pgfsys@transformshift{3.903815in}{1.520484in}%
\pgfsys@useobject{currentmarker}{}%
\end{pgfscope}%
\begin{pgfscope}%
\pgfsys@transformshift{3.920715in}{1.444973in}%
\pgfsys@useobject{currentmarker}{}%
\end{pgfscope}%
\begin{pgfscope}%
\pgfsys@transformshift{3.943248in}{1.386329in}%
\pgfsys@useobject{currentmarker}{}%
\end{pgfscope}%
\begin{pgfscope}%
\pgfsys@transformshift{3.961793in}{1.366108in}%
\pgfsys@useobject{currentmarker}{}%
\end{pgfscope}%
\begin{pgfscope}%
\pgfsys@transformshift{3.979398in}{1.367116in}%
\pgfsys@useobject{currentmarker}{}%
\end{pgfscope}%
\begin{pgfscope}%
\pgfsys@transformshift{4.000288in}{1.404687in}%
\pgfsys@useobject{currentmarker}{}%
\end{pgfscope}%
\begin{pgfscope}%
\pgfsys@transformshift{4.018362in}{1.461545in}%
\pgfsys@useobject{currentmarker}{}%
\end{pgfscope}%
\begin{pgfscope}%
\pgfsys@transformshift{4.039723in}{1.521074in}%
\pgfsys@useobject{currentmarker}{}%
\end{pgfscope}%
\begin{pgfscope}%
\pgfsys@transformshift{4.057093in}{1.620124in}%
\pgfsys@useobject{currentmarker}{}%
\end{pgfscope}%
\begin{pgfscope}%
\pgfsys@transformshift{4.077984in}{1.701692in}%
\pgfsys@useobject{currentmarker}{}%
\end{pgfscope}%
\begin{pgfscope}%
\pgfsys@transformshift{4.096761in}{1.755233in}%
\pgfsys@useobject{currentmarker}{}%
\end{pgfscope}%
\begin{pgfscope}%
\pgfsys@transformshift{4.114837in}{1.763176in}%
\pgfsys@useobject{currentmarker}{}%
\end{pgfscope}%
\begin{pgfscope}%
\pgfsys@transformshift{4.134084in}{1.737371in}%
\pgfsys@useobject{currentmarker}{}%
\end{pgfscope}%
\begin{pgfscope}%
\pgfsys@transformshift{4.154271in}{1.677502in}%
\pgfsys@useobject{currentmarker}{}%
\end{pgfscope}%
\begin{pgfscope}%
\pgfsys@transformshift{4.171406in}{1.570843in}%
\pgfsys@useobject{currentmarker}{}%
\end{pgfscope}%
\begin{pgfscope}%
\pgfsys@transformshift{4.193001in}{1.471958in}%
\pgfsys@useobject{currentmarker}{}%
\end{pgfscope}%
\begin{pgfscope}%
\pgfsys@transformshift{4.211544in}{1.450194in}%
\pgfsys@useobject{currentmarker}{}%
\end{pgfscope}%
\begin{pgfscope}%
\pgfsys@transformshift{4.228680in}{1.392432in}%
\pgfsys@useobject{currentmarker}{}%
\end{pgfscope}%
\begin{pgfscope}%
\pgfsys@transformshift{4.250041in}{1.368217in}%
\pgfsys@useobject{currentmarker}{}%
\end{pgfscope}%
\begin{pgfscope}%
\pgfsys@transformshift{4.267644in}{1.393233in}%
\pgfsys@useobject{currentmarker}{}%
\end{pgfscope}%
\begin{pgfscope}%
\pgfsys@transformshift{4.288771in}{1.450301in}%
\pgfsys@useobject{currentmarker}{}%
\end{pgfscope}%
\begin{pgfscope}%
\pgfsys@transformshift{4.307550in}{1.520077in}%
\pgfsys@useobject{currentmarker}{}%
\end{pgfscope}%
\begin{pgfscope}%
\pgfsys@transformshift{4.327032in}{1.612767in}%
\pgfsys@useobject{currentmarker}{}%
\end{pgfscope}%
\begin{pgfscope}%
\pgfsys@transformshift{4.345574in}{1.693457in}%
\pgfsys@useobject{currentmarker}{}%
\end{pgfscope}%
\begin{pgfscope}%
\pgfsys@transformshift{4.366936in}{1.759257in}%
\pgfsys@useobject{currentmarker}{}%
\end{pgfscope}%
\begin{pgfscope}%
\pgfsys@transformshift{4.388295in}{1.772221in}%
\pgfsys@useobject{currentmarker}{}%
\end{pgfscope}%
\begin{pgfscope}%
\pgfsys@transformshift{4.406839in}{1.757757in}%
\pgfsys@useobject{currentmarker}{}%
\end{pgfscope}%
\begin{pgfscope}%
\pgfsys@transformshift{4.421393in}{1.713770in}%
\pgfsys@useobject{currentmarker}{}%
\end{pgfscope}%
\begin{pgfscope}%
\pgfsys@transformshift{4.445569in}{1.596035in}%
\pgfsys@useobject{currentmarker}{}%
\end{pgfscope}%
\begin{pgfscope}%
\pgfsys@transformshift{4.460592in}{1.524423in}%
\pgfsys@useobject{currentmarker}{}%
\end{pgfscope}%
\begin{pgfscope}%
\pgfsys@transformshift{4.478196in}{1.457766in}%
\pgfsys@useobject{currentmarker}{}%
\end{pgfscope}%
\begin{pgfscope}%
\pgfsys@transformshift{4.478196in}{1.459874in}%
\pgfsys@useobject{currentmarker}{}%
\end{pgfscope}%
\begin{pgfscope}%
\pgfsys@transformshift{4.474442in}{1.475992in}%
\pgfsys@useobject{currentmarker}{}%
\end{pgfscope}%
\begin{pgfscope}%
\pgfsys@transformshift{4.456837in}{1.597045in}%
\pgfsys@useobject{currentmarker}{}%
\end{pgfscope}%
\begin{pgfscope}%
\pgfsys@transformshift{4.435476in}{1.727949in}%
\pgfsys@useobject{currentmarker}{}%
\end{pgfscope}%
\begin{pgfscope}%
\pgfsys@transformshift{4.417168in}{1.771631in}%
\pgfsys@useobject{currentmarker}{}%
\end{pgfscope}%
\begin{pgfscope}%
\pgfsys@transformshift{4.400268in}{1.724970in}%
\pgfsys@useobject{currentmarker}{}%
\end{pgfscope}%
\begin{pgfscope}%
\pgfsys@transformshift{4.378907in}{1.626950in}%
\pgfsys@useobject{currentmarker}{}%
\end{pgfscope}%
\begin{pgfscope}%
\pgfsys@transformshift{4.357782in}{1.489052in}%
\pgfsys@useobject{currentmarker}{}%
\end{pgfscope}%
\begin{pgfscope}%
\pgfsys@transformshift{4.340880in}{1.414891in}%
\pgfsys@useobject{currentmarker}{}%
\end{pgfscope}%
\begin{pgfscope}%
\pgfsys@transformshift{4.319989in}{1.367923in}%
\pgfsys@useobject{currentmarker}{}%
\end{pgfscope}%
\begin{pgfscope}%
\pgfsys@transformshift{4.300507in}{1.403878in}%
\pgfsys@useobject{currentmarker}{}%
\end{pgfscope}%
\begin{pgfscope}%
\pgfsys@transformshift{4.282903in}{1.480562in}%
\pgfsys@useobject{currentmarker}{}%
\end{pgfscope}%
\begin{pgfscope}%
\pgfsys@transformshift{4.261541in}{1.642050in}%
\pgfsys@useobject{currentmarker}{}%
\end{pgfscope}%
\begin{pgfscope}%
\pgfsys@transformshift{4.243702in}{1.739679in}%
\pgfsys@useobject{currentmarker}{}%
\end{pgfscope}%
\begin{pgfscope}%
\pgfsys@transformshift{4.225160in}{1.763508in}%
\pgfsys@useobject{currentmarker}{}%
\end{pgfscope}%
\begin{pgfscope}%
\pgfsys@transformshift{4.205912in}{1.704111in}%
\pgfsys@useobject{currentmarker}{}%
\end{pgfscope}%
\begin{pgfscope}%
\pgfsys@transformshift{4.184551in}{1.553338in}%
\pgfsys@useobject{currentmarker}{}%
\end{pgfscope}%
\begin{pgfscope}%
\pgfsys@transformshift{4.168120in}{1.454218in}%
\pgfsys@useobject{currentmarker}{}%
\end{pgfscope}%
\begin{pgfscope}%
\pgfsys@transformshift{4.146524in}{1.383632in}%
\pgfsys@useobject{currentmarker}{}%
\end{pgfscope}%
\begin{pgfscope}%
\pgfsys@transformshift{4.126339in}{1.370526in}%
\pgfsys@useobject{currentmarker}{}%
\end{pgfscope}%
\begin{pgfscope}%
\pgfsys@transformshift{4.109203in}{1.420113in}%
\pgfsys@useobject{currentmarker}{}%
\end{pgfscope}%
\begin{pgfscope}%
\pgfsys@transformshift{4.090189in}{1.516109in}%
\pgfsys@useobject{currentmarker}{}%
\end{pgfscope}%
\begin{pgfscope}%
\pgfsys@transformshift{4.071176in}{1.662020in}%
\pgfsys@useobject{currentmarker}{}%
\end{pgfscope}%
\begin{pgfscope}%
\pgfsys@transformshift{4.050286in}{1.751266in}%
\pgfsys@useobject{currentmarker}{}%
\end{pgfscope}%
\begin{pgfscope}%
\pgfsys@transformshift{4.033384in}{1.748528in}%
\pgfsys@useobject{currentmarker}{}%
\end{pgfscope}%
\begin{pgfscope}%
\pgfsys@transformshift{4.012493in}{1.655215in}%
\pgfsys@useobject{currentmarker}{}%
\end{pgfscope}%
\begin{pgfscope}%
\pgfsys@transformshift{3.992308in}{1.509968in}%
\pgfsys@useobject{currentmarker}{}%
\end{pgfscope}%
\begin{pgfscope}%
\pgfsys@transformshift{3.974467in}{1.424396in}%
\pgfsys@useobject{currentmarker}{}%
\end{pgfscope}%
\begin{pgfscope}%
\pgfsys@transformshift{3.954047in}{1.366000in}%
\pgfsys@useobject{currentmarker}{}%
\end{pgfscope}%
\begin{pgfscope}%
\pgfsys@transformshift{3.936911in}{1.368487in}%
\pgfsys@useobject{currentmarker}{}%
\end{pgfscope}%
\begin{pgfscope}%
\pgfsys@transformshift{3.916255in}{1.434072in}%
\pgfsys@useobject{currentmarker}{}%
\end{pgfscope}%
\begin{pgfscope}%
\pgfsys@transformshift{3.898181in}{1.524457in}%
\pgfsys@useobject{currentmarker}{}%
\end{pgfscope}%
\begin{pgfscope}%
\pgfsys@transformshift{3.874003in}{1.695076in}%
\pgfsys@useobject{currentmarker}{}%
\end{pgfscope}%
\begin{pgfscope}%
\pgfsys@transformshift{3.859450in}{1.745813in}%
\pgfsys@useobject{currentmarker}{}%
\end{pgfscope}%
\begin{pgfscope}%
\pgfsys@transformshift{3.837856in}{1.730585in}%
\pgfsys@useobject{currentmarker}{}%
\end{pgfscope}%
\begin{pgfscope}%
\pgfsys@transformshift{3.823302in}{1.643484in}%
\pgfsys@useobject{currentmarker}{}%
\end{pgfscope}%
\begin{pgfscope}%
\pgfsys@transformshift{3.800769in}{1.496544in}%
\pgfsys@useobject{currentmarker}{}%
\end{pgfscope}%
\begin{pgfscope}%
\pgfsys@transformshift{3.782693in}{1.417852in}%
\pgfsys@useobject{currentmarker}{}%
\end{pgfscope}%
\begin{pgfscope}%
\pgfsys@transformshift{3.761334in}{1.360858in}%
\pgfsys@useobject{currentmarker}{}%
\end{pgfscope}%
\begin{pgfscope}%
\pgfsys@transformshift{3.745607in}{1.363865in}%
\pgfsys@useobject{currentmarker}{}%
\end{pgfscope}%
\begin{pgfscope}%
\pgfsys@transformshift{3.724247in}{1.418438in}%
\pgfsys@useobject{currentmarker}{}%
\end{pgfscope}%
\begin{pgfscope}%
\pgfsys@transformshift{3.700774in}{1.538181in}%
\pgfsys@useobject{currentmarker}{}%
\end{pgfscope}%
\begin{pgfscope}%
\pgfsys@transformshift{3.686455in}{1.647012in}%
\pgfsys@useobject{currentmarker}{}%
\end{pgfscope}%
\begin{pgfscope}%
\pgfsys@transformshift{3.663216in}{1.736204in}%
\pgfsys@useobject{currentmarker}{}%
\end{pgfscope}%
\begin{pgfscope}%
\pgfsys@transformshift{3.647960in}{1.745378in}%
\pgfsys@useobject{currentmarker}{}%
\end{pgfscope}%
\begin{pgfscope}%
\pgfsys@transformshift{3.627069in}{1.720272in}%
\pgfsys@useobject{currentmarker}{}%
\end{pgfscope}%
\begin{pgfscope}%
\pgfsys@transformshift{3.610168in}{1.616787in}%
\pgfsys@useobject{currentmarker}{}%
\end{pgfscope}%
\begin{pgfscope}%
\pgfsys@transformshift{3.591859in}{1.496813in}%
\pgfsys@useobject{currentmarker}{}%
\end{pgfscope}%
\begin{pgfscope}%
\pgfsys@transformshift{3.566978in}{1.389786in}%
\pgfsys@useobject{currentmarker}{}%
\end{pgfscope}%
\begin{pgfscope}%
\pgfsys@transformshift{3.552893in}{1.368014in}%
\pgfsys@useobject{currentmarker}{}%
\end{pgfscope}%
\begin{pgfscope}%
\pgfsys@transformshift{3.532003in}{1.359908in}%
\pgfsys@useobject{currentmarker}{}%
\end{pgfscope}%
\begin{pgfscope}%
\pgfsys@transformshift{3.515103in}{1.389337in}%
\pgfsys@useobject{currentmarker}{}%
\end{pgfscope}%
\begin{pgfscope}%
\pgfsys@transformshift{3.493742in}{1.468366in}%
\pgfsys@useobject{currentmarker}{}%
\end{pgfscope}%
\begin{pgfscope}%
\pgfsys@transformshift{3.476608in}{1.547705in}%
\pgfsys@useobject{currentmarker}{}%
\end{pgfscope}%
\begin{pgfscope}%
\pgfsys@transformshift{3.455012in}{1.689149in}%
\pgfsys@useobject{currentmarker}{}%
\end{pgfscope}%
\begin{pgfscope}%
\pgfsys@transformshift{3.437173in}{1.738914in}%
\pgfsys@useobject{currentmarker}{}%
\end{pgfscope}%
\begin{pgfscope}%
\pgfsys@transformshift{3.416046in}{1.719935in}%
\pgfsys@useobject{currentmarker}{}%
\end{pgfscope}%
\begin{pgfscope}%
\pgfsys@transformshift{3.399146in}{1.634476in}%
\pgfsys@useobject{currentmarker}{}%
\end{pgfscope}%
\begin{pgfscope}%
\pgfsys@transformshift{3.378725in}{1.526406in}%
\pgfsys@useobject{currentmarker}{}%
\end{pgfscope}%
\begin{pgfscope}%
\pgfsys@transformshift{3.359477in}{1.442242in}%
\pgfsys@useobject{currentmarker}{}%
\end{pgfscope}%
\begin{pgfscope}%
\pgfsys@transformshift{3.338821in}{1.376611in}%
\pgfsys@useobject{currentmarker}{}%
\end{pgfscope}%
\begin{pgfscope}%
\pgfsys@transformshift{3.321450in}{1.353026in}%
\pgfsys@useobject{currentmarker}{}%
\end{pgfscope}%
\begin{pgfscope}%
\pgfsys@transformshift{3.302203in}{1.368774in}%
\pgfsys@useobject{currentmarker}{}%
\end{pgfscope}%
\begin{pgfscope}%
\pgfsys@transformshift{3.281781in}{1.414324in}%
\pgfsys@useobject{currentmarker}{}%
\end{pgfscope}%
\begin{pgfscope}%
\pgfsys@transformshift{3.263473in}{1.482720in}%
\pgfsys@useobject{currentmarker}{}%
\end{pgfscope}%
\begin{pgfscope}%
\pgfsys@transformshift{3.245634in}{1.597606in}%
\pgfsys@useobject{currentmarker}{}%
\end{pgfscope}%
\begin{pgfscope}%
\pgfsys@transformshift{3.225446in}{1.713375in}%
\pgfsys@useobject{currentmarker}{}%
\end{pgfscope}%
\begin{pgfscope}%
\pgfsys@transformshift{3.204087in}{1.739602in}%
\pgfsys@useobject{currentmarker}{}%
\end{pgfscope}%
\begin{pgfscope}%
\pgfsys@transformshift{3.186482in}{1.695274in}%
\pgfsys@useobject{currentmarker}{}%
\end{pgfscope}%
\begin{pgfscope}%
\pgfsys@transformshift{3.169112in}{1.594896in}%
\pgfsys@useobject{currentmarker}{}%
\end{pgfscope}%
\begin{pgfscope}%
\pgfsys@transformshift{3.147282in}{1.504560in}%
\pgfsys@useobject{currentmarker}{}%
\end{pgfscope}%
\begin{pgfscope}%
\pgfsys@transformshift{3.129442in}{1.420887in}%
\pgfsys@useobject{currentmarker}{}%
\end{pgfscope}%
\begin{pgfscope}%
\pgfsys@transformshift{3.110898in}{1.370305in}%
\pgfsys@useobject{currentmarker}{}%
\end{pgfscope}%
\begin{pgfscope}%
\pgfsys@transformshift{3.093059in}{1.352697in}%
\pgfsys@useobject{currentmarker}{}%
\end{pgfscope}%
\begin{pgfscope}%
\pgfsys@transformshift{3.070291in}{1.377935in}%
\pgfsys@useobject{currentmarker}{}%
\end{pgfscope}%
\begin{pgfscope}%
\pgfsys@transformshift{3.052686in}{1.435685in}%
\pgfsys@useobject{currentmarker}{}%
\end{pgfscope}%
\begin{pgfscope}%
\pgfsys@transformshift{3.034612in}{1.526888in}%
\pgfsys@useobject{currentmarker}{}%
\end{pgfscope}%
\begin{pgfscope}%
\pgfsys@transformshift{3.012782in}{1.647759in}%
\pgfsys@useobject{currentmarker}{}%
\end{pgfscope}%
\begin{pgfscope}%
\pgfsys@transformshift{2.994003in}{1.727274in}%
\pgfsys@useobject{currentmarker}{}%
\end{pgfscope}%
\begin{pgfscope}%
\pgfsys@transformshift{2.975225in}{1.736541in}%
\pgfsys@useobject{currentmarker}{}%
\end{pgfscope}%
\begin{pgfscope}%
\pgfsys@transformshift{2.956682in}{1.705079in}%
\pgfsys@useobject{currentmarker}{}%
\end{pgfscope}%
\begin{pgfscope}%
\pgfsys@transformshift{2.936260in}{1.584935in}%
\pgfsys@useobject{currentmarker}{}%
\end{pgfscope}%
\begin{pgfscope}%
\pgfsys@transformshift{2.919359in}{1.522396in}%
\pgfsys@useobject{currentmarker}{}%
\end{pgfscope}%
\begin{pgfscope}%
\pgfsys@transformshift{2.898234in}{1.421857in}%
\pgfsys@useobject{currentmarker}{}%
\end{pgfscope}%
\begin{pgfscope}%
\pgfsys@transformshift{2.878986in}{1.381018in}%
\pgfsys@useobject{currentmarker}{}%
\end{pgfscope}%
\begin{pgfscope}%
\pgfsys@transformshift{2.859739in}{1.359934in}%
\pgfsys@useobject{currentmarker}{}%
\end{pgfscope}%
\begin{pgfscope}%
\pgfsys@transformshift{2.841665in}{1.358103in}%
\pgfsys@useobject{currentmarker}{}%
\end{pgfscope}%
\begin{pgfscope}%
\pgfsys@transformshift{2.822415in}{1.429287in}%
\pgfsys@useobject{currentmarker}{}%
\end{pgfscope}%
\begin{pgfscope}%
\pgfsys@transformshift{2.801290in}{1.474203in}%
\pgfsys@useobject{currentmarker}{}%
\end{pgfscope}%
\begin{pgfscope}%
\pgfsys@transformshift{2.764203in}{1.697948in}%
\pgfsys@useobject{currentmarker}{}%
\end{pgfscope}%
\begin{pgfscope}%
\pgfsys@transformshift{2.746130in}{1.737637in}%
\pgfsys@useobject{currentmarker}{}%
\end{pgfscope}%
\begin{pgfscope}%
\pgfsys@transformshift{2.726411in}{1.713344in}%
\pgfsys@useobject{currentmarker}{}%
\end{pgfscope}%
\begin{pgfscope}%
\pgfsys@transformshift{2.704817in}{1.651783in}%
\pgfsys@useobject{currentmarker}{}%
\end{pgfscope}%
\begin{pgfscope}%
\pgfsys@transformshift{2.686978in}{1.549464in}%
\pgfsys@useobject{currentmarker}{}%
\end{pgfscope}%
\begin{pgfscope}%
\pgfsys@transformshift{2.628296in}{1.353461in}%
\pgfsys@useobject{currentmarker}{}%
\end{pgfscope}%
\begin{pgfscope}%
\pgfsys@transformshift{2.612099in}{1.357151in}%
\pgfsys@useobject{currentmarker}{}%
\end{pgfscope}%
\begin{pgfscope}%
\pgfsys@transformshift{2.593086in}{1.391092in}%
\pgfsys@useobject{currentmarker}{}%
\end{pgfscope}%
\begin{pgfscope}%
\pgfsys@transformshift{2.574073in}{1.451065in}%
\pgfsys@useobject{currentmarker}{}%
\end{pgfscope}%
\begin{pgfscope}%
\pgfsys@transformshift{2.551774in}{1.543617in}%
\pgfsys@useobject{currentmarker}{}%
\end{pgfscope}%
\begin{pgfscope}%
\pgfsys@transformshift{2.532292in}{1.660941in}%
\pgfsys@useobject{currentmarker}{}%
\end{pgfscope}%
\begin{pgfscope}%
\pgfsys@transformshift{2.513513in}{1.729085in}%
\pgfsys@useobject{currentmarker}{}%
\end{pgfscope}%
\begin{pgfscope}%
\pgfsys@transformshift{2.496142in}{1.734194in}%
\pgfsys@useobject{currentmarker}{}%
\end{pgfscope}%
\begin{pgfscope}%
\pgfsys@transformshift{2.474312in}{1.662598in}%
\pgfsys@useobject{currentmarker}{}%
\end{pgfscope}%
\begin{pgfscope}%
\pgfsys@transformshift{2.456004in}{1.552070in}%
\pgfsys@useobject{currentmarker}{}%
\end{pgfscope}%
\begin{pgfscope}%
\pgfsys@transformshift{2.436286in}{1.454590in}%
\pgfsys@useobject{currentmarker}{}%
\end{pgfscope}%
\begin{pgfscope}%
\pgfsys@transformshift{2.415161in}{1.738431in}%
\pgfsys@useobject{currentmarker}{}%
\end{pgfscope}%
\begin{pgfscope}%
\pgfsys@transformshift{2.400138in}{1.722504in}%
\pgfsys@useobject{currentmarker}{}%
\end{pgfscope}%
\begin{pgfscope}%
\pgfsys@transformshift{2.376900in}{1.602389in}%
\pgfsys@useobject{currentmarker}{}%
\end{pgfscope}%
\begin{pgfscope}%
\pgfsys@transformshift{2.359764in}{1.490614in}%
\pgfsys@useobject{currentmarker}{}%
\end{pgfscope}%
\begin{pgfscope}%
\pgfsys@transformshift{2.341221in}{1.418372in}%
\pgfsys@useobject{currentmarker}{}%
\end{pgfscope}%
\begin{pgfscope}%
\pgfsys@transformshift{2.321739in}{1.368608in}%
\pgfsys@useobject{currentmarker}{}%
\end{pgfscope}%
\begin{pgfscope}%
\pgfsys@transformshift{2.300612in}{1.353751in}%
\pgfsys@useobject{currentmarker}{}%
\end{pgfscope}%
\begin{pgfscope}%
\pgfsys@transformshift{2.286295in}{1.379011in}%
\pgfsys@useobject{currentmarker}{}%
\end{pgfscope}%
\begin{pgfscope}%
\pgfsys@transformshift{2.264231in}{1.446700in}%
\pgfsys@useobject{currentmarker}{}%
\end{pgfscope}%
\begin{pgfscope}%
\pgfsys@transformshift{2.244747in}{1.537295in}%
\pgfsys@useobject{currentmarker}{}%
\end{pgfscope}%
\begin{pgfscope}%
\pgfsys@transformshift{2.226438in}{1.636316in}%
\pgfsys@useobject{currentmarker}{}%
\end{pgfscope}%
\begin{pgfscope}%
\pgfsys@transformshift{2.206722in}{1.725605in}%
\pgfsys@useobject{currentmarker}{}%
\end{pgfscope}%
\begin{pgfscope}%
\pgfsys@transformshift{2.187709in}{1.738613in}%
\pgfsys@useobject{currentmarker}{}%
\end{pgfscope}%
\begin{pgfscope}%
\pgfsys@transformshift{2.168695in}{1.699461in}%
\pgfsys@useobject{currentmarker}{}%
\end{pgfscope}%
\begin{pgfscope}%
\pgfsys@transformshift{2.146865in}{1.575010in}%
\pgfsys@useobject{currentmarker}{}%
\end{pgfscope}%
\begin{pgfscope}%
\pgfsys@transformshift{2.129729in}{1.469791in}%
\pgfsys@useobject{currentmarker}{}%
\end{pgfscope}%
\begin{pgfscope}%
\pgfsys@transformshift{2.110013in}{1.406602in}%
\pgfsys@useobject{currentmarker}{}%
\end{pgfscope}%
\begin{pgfscope}%
\pgfsys@transformshift{2.088417in}{1.363562in}%
\pgfsys@useobject{currentmarker}{}%
\end{pgfscope}%
\begin{pgfscope}%
\pgfsys@transformshift{2.073160in}{1.399227in}%
\pgfsys@useobject{currentmarker}{}%
\end{pgfscope}%
\begin{pgfscope}%
\pgfsys@transformshift{2.051096in}{1.371397in}%
\pgfsys@useobject{currentmarker}{}%
\end{pgfscope}%
\begin{pgfscope}%
\pgfsys@transformshift{2.033491in}{1.352963in}%
\pgfsys@useobject{currentmarker}{}%
\end{pgfscope}%
\begin{pgfscope}%
\pgfsys@transformshift{2.014478in}{1.376021in}%
\pgfsys@useobject{currentmarker}{}%
\end{pgfscope}%
\begin{pgfscope}%
\pgfsys@transformshift{1.995464in}{1.405651in}%
\pgfsys@useobject{currentmarker}{}%
\end{pgfscope}%
\begin{pgfscope}%
\pgfsys@transformshift{1.974808in}{1.489508in}%
\pgfsys@useobject{currentmarker}{}%
\end{pgfscope}%
\begin{pgfscope}%
\pgfsys@transformshift{1.955326in}{1.622405in}%
\pgfsys@useobject{currentmarker}{}%
\end{pgfscope}%
\begin{pgfscope}%
\pgfsys@transformshift{1.937956in}{1.704490in}%
\pgfsys@useobject{currentmarker}{}%
\end{pgfscope}%
\begin{pgfscope}%
\pgfsys@transformshift{1.919882in}{1.742377in}%
\pgfsys@useobject{currentmarker}{}%
\end{pgfscope}%
\begin{pgfscope}%
\pgfsys@transformshift{1.898992in}{1.717082in}%
\pgfsys@useobject{currentmarker}{}%
\end{pgfscope}%
\begin{pgfscope}%
\pgfsys@transformshift{1.880916in}{1.622373in}%
\pgfsys@useobject{currentmarker}{}%
\end{pgfscope}%
\begin{pgfscope}%
\pgfsys@transformshift{1.863311in}{1.504118in}%
\pgfsys@useobject{currentmarker}{}%
\end{pgfscope}%
\begin{pgfscope}%
\pgfsys@transformshift{1.843595in}{1.432035in}%
\pgfsys@useobject{currentmarker}{}%
\end{pgfscope}%
\begin{pgfscope}%
\pgfsys@transformshift{1.824347in}{1.387174in}%
\pgfsys@useobject{currentmarker}{}%
\end{pgfscope}%
\begin{pgfscope}%
\pgfsys@transformshift{1.800874in}{1.354850in}%
\pgfsys@useobject{currentmarker}{}%
\end{pgfscope}%
\begin{pgfscope}%
\pgfsys@transformshift{1.784912in}{1.369047in}%
\pgfsys@useobject{currentmarker}{}%
\end{pgfscope}%
\begin{pgfscope}%
\pgfsys@transformshift{1.762847in}{1.421632in}%
\pgfsys@useobject{currentmarker}{}%
\end{pgfscope}%
\begin{pgfscope}%
\pgfsys@transformshift{1.744774in}{1.508061in}%
\pgfsys@useobject{currentmarker}{}%
\end{pgfscope}%
\begin{pgfscope}%
\pgfsys@transformshift{1.725292in}{1.629740in}%
\pgfsys@useobject{currentmarker}{}%
\end{pgfscope}%
\begin{pgfscope}%
\pgfsys@transformshift{1.706747in}{1.711188in}%
\pgfsys@useobject{currentmarker}{}%
\end{pgfscope}%
\begin{pgfscope}%
\pgfsys@transformshift{1.688674in}{1.746217in}%
\pgfsys@useobject{currentmarker}{}%
\end{pgfscope}%
\begin{pgfscope}%
\pgfsys@transformshift{1.667078in}{1.713736in}%
\pgfsys@useobject{currentmarker}{}%
\end{pgfscope}%
\begin{pgfscope}%
\pgfsys@transformshift{1.648065in}{1.629738in}%
\pgfsys@useobject{currentmarker}{}%
\end{pgfscope}%
\begin{pgfscope}%
\pgfsys@transformshift{1.629757in}{1.556198in}%
\pgfsys@useobject{currentmarker}{}%
\end{pgfscope}%
\begin{pgfscope}%
\pgfsys@transformshift{1.611917in}{1.474391in}%
\pgfsys@useobject{currentmarker}{}%
\end{pgfscope}%
\begin{pgfscope}%
\pgfsys@transformshift{1.589853in}{1.402109in}%
\pgfsys@useobject{currentmarker}{}%
\end{pgfscope}%
\begin{pgfscope}%
\pgfsys@transformshift{1.572482in}{1.365968in}%
\pgfsys@useobject{currentmarker}{}%
\end{pgfscope}%
\begin{pgfscope}%
\pgfsys@transformshift{1.553000in}{1.360179in}%
\pgfsys@useobject{currentmarker}{}%
\end{pgfscope}%
\begin{pgfscope}%
\pgfsys@transformshift{1.535395in}{1.390472in}%
\pgfsys@useobject{currentmarker}{}%
\end{pgfscope}%
\begin{pgfscope}%
\pgfsys@transformshift{1.515443in}{1.436579in}%
\pgfsys@useobject{currentmarker}{}%
\end{pgfscope}%
\begin{pgfscope}%
\pgfsys@transformshift{1.496900in}{1.516995in}%
\pgfsys@useobject{currentmarker}{}%
\end{pgfscope}%
\begin{pgfscope}%
\pgfsys@transformshift{1.474130in}{1.644060in}%
\pgfsys@useobject{currentmarker}{}%
\end{pgfscope}%
\begin{pgfscope}%
\pgfsys@transformshift{1.456996in}{1.435523in}%
\pgfsys@useobject{currentmarker}{}%
\end{pgfscope}%
\begin{pgfscope}%
\pgfsys@transformshift{1.436809in}{1.418559in}%
\pgfsys@useobject{currentmarker}{}%
\end{pgfscope}%
\begin{pgfscope}%
\pgfsys@transformshift{1.417561in}{1.506426in}%
\pgfsys@useobject{currentmarker}{}%
\end{pgfscope}%
\begin{pgfscope}%
\pgfsys@transformshift{1.401130in}{1.606237in}%
\pgfsys@useobject{currentmarker}{}%
\end{pgfscope}%
\begin{pgfscope}%
\pgfsys@transformshift{1.384229in}{1.694587in}%
\pgfsys@useobject{currentmarker}{}%
\end{pgfscope}%
\begin{pgfscope}%
\pgfsys@transformshift{1.359113in}{1.751515in}%
\pgfsys@useobject{currentmarker}{}%
\end{pgfscope}%
\begin{pgfscope}%
\pgfsys@transformshift{1.343856in}{1.741226in}%
\pgfsys@useobject{currentmarker}{}%
\end{pgfscope}%
\begin{pgfscope}%
\pgfsys@transformshift{1.324138in}{1.700004in}%
\pgfsys@useobject{currentmarker}{}%
\end{pgfscope}%
\begin{pgfscope}%
\pgfsys@transformshift{1.305595in}{1.589602in}%
\pgfsys@useobject{currentmarker}{}%
\end{pgfscope}%
\begin{pgfscope}%
\pgfsys@transformshift{1.284470in}{1.478711in}%
\pgfsys@useobject{currentmarker}{}%
\end{pgfscope}%
\begin{pgfscope}%
\pgfsys@transformshift{1.261935in}{1.406483in}%
\pgfsys@useobject{currentmarker}{}%
\end{pgfscope}%
\begin{pgfscope}%
\pgfsys@transformshift{1.247147in}{1.371336in}%
\pgfsys@useobject{currentmarker}{}%
\end{pgfscope}%
\begin{pgfscope}%
\pgfsys@transformshift{1.228368in}{1.358588in}%
\pgfsys@useobject{currentmarker}{}%
\end{pgfscope}%
\begin{pgfscope}%
\pgfsys@transformshift{1.207478in}{1.389082in}%
\pgfsys@useobject{currentmarker}{}%
\end{pgfscope}%
\begin{pgfscope}%
\pgfsys@transformshift{1.187761in}{1.445062in}%
\pgfsys@useobject{currentmarker}{}%
\end{pgfscope}%
\begin{pgfscope}%
\pgfsys@transformshift{1.169217in}{1.531528in}%
\pgfsys@useobject{currentmarker}{}%
\end{pgfscope}%
\begin{pgfscope}%
\pgfsys@transformshift{1.150674in}{1.646846in}%
\pgfsys@useobject{currentmarker}{}%
\end{pgfscope}%
\begin{pgfscope}%
\pgfsys@transformshift{1.129079in}{1.729093in}%
\pgfsys@useobject{currentmarker}{}%
\end{pgfscope}%
\begin{pgfscope}%
\pgfsys@transformshift{1.112882in}{1.756321in}%
\pgfsys@useobject{currentmarker}{}%
\end{pgfscope}%
\begin{pgfscope}%
\pgfsys@transformshift{1.092226in}{1.743373in}%
\pgfsys@useobject{currentmarker}{}%
\end{pgfscope}%
\begin{pgfscope}%
\pgfsys@transformshift{1.074152in}{1.675131in}%
\pgfsys@useobject{currentmarker}{}%
\end{pgfscope}%
\begin{pgfscope}%
\pgfsys@transformshift{1.054670in}{1.575142in}%
\pgfsys@useobject{currentmarker}{}%
\end{pgfscope}%
\begin{pgfscope}%
\pgfsys@transformshift{1.036126in}{1.489608in}%
\pgfsys@useobject{currentmarker}{}%
\end{pgfscope}%
\begin{pgfscope}%
\pgfsys@transformshift{1.011479in}{1.412225in}%
\pgfsys@useobject{currentmarker}{}%
\end{pgfscope}%
\begin{pgfscope}%
\pgfsys@transformshift{0.995988in}{1.588707in}%
\pgfsys@useobject{currentmarker}{}%
\end{pgfscope}%
\begin{pgfscope}%
\pgfsys@transformshift{0.976974in}{1.491233in}%
\pgfsys@useobject{currentmarker}{}%
\end{pgfscope}%
\begin{pgfscope}%
\pgfsys@transformshift{0.958196in}{1.426855in}%
\pgfsys@useobject{currentmarker}{}%
\end{pgfscope}%
\begin{pgfscope}%
\pgfsys@transformshift{0.937071in}{1.374750in}%
\pgfsys@useobject{currentmarker}{}%
\end{pgfscope}%
\begin{pgfscope}%
\pgfsys@transformshift{0.921577in}{1.365782in}%
\pgfsys@useobject{currentmarker}{}%
\end{pgfscope}%
\begin{pgfscope}%
\pgfsys@transformshift{0.901627in}{1.402169in}%
\pgfsys@useobject{currentmarker}{}%
\end{pgfscope}%
\begin{pgfscope}%
\pgfsys@transformshift{0.883082in}{1.441271in}%
\pgfsys@useobject{currentmarker}{}%
\end{pgfscope}%
\begin{pgfscope}%
\pgfsys@transformshift{0.863366in}{1.527543in}%
\pgfsys@useobject{currentmarker}{}%
\end{pgfscope}%
\begin{pgfscope}%
\pgfsys@transformshift{0.840830in}{1.632688in}%
\pgfsys@useobject{currentmarker}{}%
\end{pgfscope}%
\begin{pgfscope}%
\pgfsys@transformshift{0.824399in}{1.724359in}%
\pgfsys@useobject{currentmarker}{}%
\end{pgfscope}%
\begin{pgfscope}%
\pgfsys@transformshift{0.804683in}{1.764068in}%
\pgfsys@useobject{currentmarker}{}%
\end{pgfscope}%
\begin{pgfscope}%
\pgfsys@transformshift{0.786609in}{1.758853in}%
\pgfsys@useobject{currentmarker}{}%
\end{pgfscope}%
\begin{pgfscope}%
\pgfsys@transformshift{0.762665in}{1.675761in}%
\pgfsys@useobject{currentmarker}{}%
\end{pgfscope}%
\begin{pgfscope}%
\pgfsys@transformshift{0.746469in}{1.600841in}%
\pgfsys@useobject{currentmarker}{}%
\end{pgfscope}%
\begin{pgfscope}%
\pgfsys@transformshift{0.728161in}{1.504393in}%
\pgfsys@useobject{currentmarker}{}%
\end{pgfscope}%
\begin{pgfscope}%
\pgfsys@transformshift{0.709148in}{1.440211in}%
\pgfsys@useobject{currentmarker}{}%
\end{pgfscope}%
\begin{pgfscope}%
\pgfsys@transformshift{0.686614in}{1.386189in}%
\pgfsys@useobject{currentmarker}{}%
\end{pgfscope}%
\begin{pgfscope}%
\pgfsys@transformshift{0.669244in}{1.369783in}%
\pgfsys@useobject{currentmarker}{}%
\end{pgfscope}%
\begin{pgfscope}%
\pgfsys@transformshift{0.651170in}{1.392321in}%
\pgfsys@useobject{currentmarker}{}%
\end{pgfscope}%
\begin{pgfscope}%
\pgfsys@transformshift{0.648119in}{1.392648in}%
\pgfsys@useobject{currentmarker}{}%
\end{pgfscope}%
\begin{pgfscope}%
\pgfsys@transformshift{0.654925in}{1.377038in}%
\pgfsys@useobject{currentmarker}{}%
\end{pgfscope}%
\begin{pgfscope}%
\pgfsys@transformshift{0.676286in}{1.394012in}%
\pgfsys@useobject{currentmarker}{}%
\end{pgfscope}%
\begin{pgfscope}%
\pgfsys@transformshift{0.695063in}{1.462600in}%
\pgfsys@useobject{currentmarker}{}%
\end{pgfscope}%
\begin{pgfscope}%
\pgfsys@transformshift{0.713139in}{1.565875in}%
\pgfsys@useobject{currentmarker}{}%
\end{pgfscope}%
\begin{pgfscope}%
\pgfsys@transformshift{0.734969in}{1.715793in}%
\pgfsys@useobject{currentmarker}{}%
\end{pgfscope}%
\begin{pgfscope}%
\pgfsys@transformshift{0.751165in}{1.764213in}%
\pgfsys@useobject{currentmarker}{}%
\end{pgfscope}%
\begin{pgfscope}%
\pgfsys@transformshift{0.772290in}{1.734615in}%
\pgfsys@useobject{currentmarker}{}%
\end{pgfscope}%
\begin{pgfscope}%
\pgfsys@transformshift{0.791303in}{1.617525in}%
\pgfsys@useobject{currentmarker}{}%
\end{pgfscope}%
\begin{pgfscope}%
\pgfsys@transformshift{0.810786in}{1.473784in}%
\pgfsys@useobject{currentmarker}{}%
\end{pgfscope}%
\begin{pgfscope}%
\pgfsys@transformshift{0.829094in}{1.393117in}%
\pgfsys@useobject{currentmarker}{}%
\end{pgfscope}%
\begin{pgfscope}%
\pgfsys@transformshift{0.847638in}{1.364833in}%
\pgfsys@useobject{currentmarker}{}%
\end{pgfscope}%
\begin{pgfscope}%
\pgfsys@transformshift{0.867120in}{1.414218in}%
\pgfsys@useobject{currentmarker}{}%
\end{pgfscope}%
\begin{pgfscope}%
\pgfsys@transformshift{0.886602in}{1.497064in}%
\pgfsys@useobject{currentmarker}{}%
\end{pgfscope}%
\begin{pgfscope}%
\pgfsys@transformshift{0.905852in}{1.643157in}%
\pgfsys@useobject{currentmarker}{}%
\end{pgfscope}%
\begin{pgfscope}%
\pgfsys@transformshift{0.924863in}{1.743361in}%
\pgfsys@useobject{currentmarker}{}%
\end{pgfscope}%
\begin{pgfscope}%
\pgfsys@transformshift{0.945285in}{1.753170in}%
\pgfsys@useobject{currentmarker}{}%
\end{pgfscope}%
\begin{pgfscope}%
\pgfsys@transformshift{0.963361in}{1.667696in}%
\pgfsys@useobject{currentmarker}{}%
\end{pgfscope}%
\begin{pgfscope}%
\pgfsys@transformshift{0.983077in}{1.528721in}%
\pgfsys@useobject{currentmarker}{}%
\end{pgfscope}%
\begin{pgfscope}%
\pgfsys@transformshift{1.003030in}{1.425651in}%
\pgfsys@useobject{currentmarker}{}%
\end{pgfscope}%
\begin{pgfscope}%
\pgfsys@transformshift{1.021572in}{1.367817in}%
\pgfsys@useobject{currentmarker}{}%
\end{pgfscope}%
\begin{pgfscope}%
\pgfsys@transformshift{1.039646in}{1.374288in}%
\pgfsys@useobject{currentmarker}{}%
\end{pgfscope}%
\begin{pgfscope}%
\pgfsys@transformshift{1.059365in}{1.435951in}%
\pgfsys@useobject{currentmarker}{}%
\end{pgfscope}%
\begin{pgfscope}%
\pgfsys@transformshift{1.080489in}{1.550570in}%
\pgfsys@useobject{currentmarker}{}%
\end{pgfscope}%
\begin{pgfscope}%
\pgfsys@transformshift{1.100911in}{1.692480in}%
\pgfsys@useobject{currentmarker}{}%
\end{pgfscope}%
\begin{pgfscope}%
\pgfsys@transformshift{1.116168in}{1.663326in}%
\pgfsys@useobject{currentmarker}{}%
\end{pgfscope}%
\begin{pgfscope}%
\pgfsys@transformshift{1.138233in}{1.749953in}%
\pgfsys@useobject{currentmarker}{}%
\end{pgfscope}%
\begin{pgfscope}%
\pgfsys@transformshift{1.154900in}{1.739871in}%
\pgfsys@useobject{currentmarker}{}%
\end{pgfscope}%
\begin{pgfscope}%
\pgfsys@transformshift{1.174147in}{1.641934in}%
\pgfsys@useobject{currentmarker}{}%
\end{pgfscope}%
\begin{pgfscope}%
\pgfsys@transformshift{1.195507in}{1.483780in}%
\pgfsys@useobject{currentmarker}{}%
\end{pgfscope}%
\begin{pgfscope}%
\pgfsys@transformshift{1.215460in}{1.397961in}%
\pgfsys@useobject{currentmarker}{}%
\end{pgfscope}%
\begin{pgfscope}%
\pgfsys@transformshift{1.231890in}{1.359670in}%
\pgfsys@useobject{currentmarker}{}%
\end{pgfscope}%
\begin{pgfscope}%
\pgfsys@transformshift{1.250669in}{1.374436in}%
\pgfsys@useobject{currentmarker}{}%
\end{pgfscope}%
\begin{pgfscope}%
\pgfsys@transformshift{1.275314in}{1.456141in}%
\pgfsys@useobject{currentmarker}{}%
\end{pgfscope}%
\begin{pgfscope}%
\pgfsys@transformshift{1.294562in}{1.574153in}%
\pgfsys@useobject{currentmarker}{}%
\end{pgfscope}%
\begin{pgfscope}%
\pgfsys@transformshift{1.310993in}{1.686393in}%
\pgfsys@useobject{currentmarker}{}%
\end{pgfscope}%
\begin{pgfscope}%
\pgfsys@transformshift{1.326486in}{1.742936in}%
\pgfsys@useobject{currentmarker}{}%
\end{pgfscope}%
\begin{pgfscope}%
\pgfsys@transformshift{1.348316in}{1.728768in}%
\pgfsys@useobject{currentmarker}{}%
\end{pgfscope}%
\begin{pgfscope}%
\pgfsys@transformshift{1.367564in}{1.623345in}%
\pgfsys@useobject{currentmarker}{}%
\end{pgfscope}%
\begin{pgfscope}%
\pgfsys@transformshift{1.386106in}{1.490023in}%
\pgfsys@useobject{currentmarker}{}%
\end{pgfscope}%
\begin{pgfscope}%
\pgfsys@transformshift{1.405825in}{1.401891in}%
\pgfsys@useobject{currentmarker}{}%
\end{pgfscope}%
\begin{pgfscope}%
\pgfsys@transformshift{1.425776in}{1.357561in}%
\pgfsys@useobject{currentmarker}{}%
\end{pgfscope}%
\begin{pgfscope}%
\pgfsys@transformshift{1.447606in}{1.379968in}%
\pgfsys@useobject{currentmarker}{}%
\end{pgfscope}%
\begin{pgfscope}%
\pgfsys@transformshift{1.464273in}{1.421369in}%
\pgfsys@useobject{currentmarker}{}%
\end{pgfscope}%
\begin{pgfscope}%
\pgfsys@transformshift{1.481407in}{1.500585in}%
\pgfsys@useobject{currentmarker}{}%
\end{pgfscope}%
\begin{pgfscope}%
\pgfsys@transformshift{1.500889in}{1.619709in}%
\pgfsys@useobject{currentmarker}{}%
\end{pgfscope}%
\begin{pgfscope}%
\pgfsys@transformshift{1.523190in}{1.728732in}%
\pgfsys@useobject{currentmarker}{}%
\end{pgfscope}%
\begin{pgfscope}%
\pgfsys@transformshift{1.538447in}{1.742741in}%
\pgfsys@useobject{currentmarker}{}%
\end{pgfscope}%
\begin{pgfscope}%
\pgfsys@transformshift{1.563328in}{1.660264in}%
\pgfsys@useobject{currentmarker}{}%
\end{pgfscope}%
\begin{pgfscope}%
\pgfsys@transformshift{1.579054in}{1.544271in}%
\pgfsys@useobject{currentmarker}{}%
\end{pgfscope}%
\begin{pgfscope}%
\pgfsys@transformshift{1.598538in}{1.439414in}%
\pgfsys@useobject{currentmarker}{}%
\end{pgfscope}%
\begin{pgfscope}%
\pgfsys@transformshift{1.617551in}{1.393065in}%
\pgfsys@useobject{currentmarker}{}%
\end{pgfscope}%
\begin{pgfscope}%
\pgfsys@transformshift{1.636799in}{1.522943in}%
\pgfsys@useobject{currentmarker}{}%
\end{pgfscope}%
\begin{pgfscope}%
\pgfsys@transformshift{1.655107in}{1.436930in}%
\pgfsys@useobject{currentmarker}{}%
\end{pgfscope}%
\begin{pgfscope}%
\pgfsys@transformshift{1.673181in}{1.378246in}%
\pgfsys@useobject{currentmarker}{}%
\end{pgfscope}%
\begin{pgfscope}%
\pgfsys@transformshift{1.695481in}{1.353649in}%
\pgfsys@useobject{currentmarker}{}%
\end{pgfscope}%
\begin{pgfscope}%
\pgfsys@transformshift{1.715901in}{1.376931in}%
\pgfsys@useobject{currentmarker}{}%
\end{pgfscope}%
\begin{pgfscope}%
\pgfsys@transformshift{1.733740in}{1.432521in}%
\pgfsys@useobject{currentmarker}{}%
\end{pgfscope}%
\begin{pgfscope}%
\pgfsys@transformshift{1.751816in}{1.523834in}%
\pgfsys@useobject{currentmarker}{}%
\end{pgfscope}%
\begin{pgfscope}%
\pgfsys@transformshift{1.771064in}{1.639544in}%
\pgfsys@useobject{currentmarker}{}%
\end{pgfscope}%
\begin{pgfscope}%
\pgfsys@transformshift{1.790780in}{1.727932in}%
\pgfsys@useobject{currentmarker}{}%
\end{pgfscope}%
\begin{pgfscope}%
\pgfsys@transformshift{1.808619in}{1.740560in}%
\pgfsys@useobject{currentmarker}{}%
\end{pgfscope}%
\begin{pgfscope}%
\pgfsys@transformshift{1.829981in}{1.680955in}%
\pgfsys@useobject{currentmarker}{}%
\end{pgfscope}%
\begin{pgfscope}%
\pgfsys@transformshift{1.848760in}{1.549180in}%
\pgfsys@useobject{currentmarker}{}%
\end{pgfscope}%
\begin{pgfscope}%
\pgfsys@transformshift{1.867771in}{1.445079in}%
\pgfsys@useobject{currentmarker}{}%
\end{pgfscope}%
\begin{pgfscope}%
\pgfsys@transformshift{1.886315in}{1.385058in}%
\pgfsys@useobject{currentmarker}{}%
\end{pgfscope}%
\begin{pgfscope}%
\pgfsys@transformshift{1.905094in}{1.355648in}%
\pgfsys@useobject{currentmarker}{}%
\end{pgfscope}%
\begin{pgfscope}%
\pgfsys@transformshift{1.923402in}{1.362913in}%
\pgfsys@useobject{currentmarker}{}%
\end{pgfscope}%
\begin{pgfscope}%
\pgfsys@transformshift{1.945467in}{1.415095in}%
\pgfsys@useobject{currentmarker}{}%
\end{pgfscope}%
\begin{pgfscope}%
\pgfsys@transformshift{1.964011in}{1.491521in}%
\pgfsys@useobject{currentmarker}{}%
\end{pgfscope}%
\begin{pgfscope}%
\pgfsys@transformshift{1.980911in}{1.586179in}%
\pgfsys@useobject{currentmarker}{}%
\end{pgfscope}%
\begin{pgfscope}%
\pgfsys@transformshift{2.001333in}{1.709565in}%
\pgfsys@useobject{currentmarker}{}%
\end{pgfscope}%
\begin{pgfscope}%
\pgfsys@transformshift{2.021754in}{1.739656in}%
\pgfsys@useobject{currentmarker}{}%
\end{pgfscope}%
\begin{pgfscope}%
\pgfsys@transformshift{2.038890in}{1.708906in}%
\pgfsys@useobject{currentmarker}{}%
\end{pgfscope}%
\begin{pgfscope}%
\pgfsys@transformshift{2.057198in}{1.599930in}%
\pgfsys@useobject{currentmarker}{}%
\end{pgfscope}%
\begin{pgfscope}%
\pgfsys@transformshift{2.079732in}{1.475021in}%
\pgfsys@useobject{currentmarker}{}%
\end{pgfscope}%
\begin{pgfscope}%
\pgfsys@transformshift{2.096633in}{1.412427in}%
\pgfsys@useobject{currentmarker}{}%
\end{pgfscope}%
\begin{pgfscope}%
\pgfsys@transformshift{2.116350in}{1.364220in}%
\pgfsys@useobject{currentmarker}{}%
\end{pgfscope}%
\begin{pgfscope}%
\pgfsys@transformshift{2.136772in}{1.353236in}%
\pgfsys@useobject{currentmarker}{}%
\end{pgfscope}%
\begin{pgfscope}%
\pgfsys@transformshift{2.155316in}{1.386752in}%
\pgfsys@useobject{currentmarker}{}%
\end{pgfscope}%
\begin{pgfscope}%
\pgfsys@transformshift{2.172684in}{1.423673in}%
\pgfsys@useobject{currentmarker}{}%
\end{pgfscope}%
\begin{pgfscope}%
\pgfsys@transformshift{2.193811in}{1.510131in}%
\pgfsys@useobject{currentmarker}{}%
\end{pgfscope}%
\begin{pgfscope}%
\pgfsys@transformshift{2.211651in}{1.631100in}%
\pgfsys@useobject{currentmarker}{}%
\end{pgfscope}%
\begin{pgfscope}%
\pgfsys@transformshift{2.233950in}{1.724352in}%
\pgfsys@useobject{currentmarker}{}%
\end{pgfscope}%
\begin{pgfscope}%
\pgfsys@transformshift{2.252728in}{1.737006in}%
\pgfsys@useobject{currentmarker}{}%
\end{pgfscope}%
\begin{pgfscope}%
\pgfsys@transformshift{2.271976in}{1.686156in}%
\pgfsys@useobject{currentmarker}{}%
\end{pgfscope}%
\begin{pgfscope}%
\pgfsys@transformshift{2.292163in}{1.552435in}%
\pgfsys@useobject{currentmarker}{}%
\end{pgfscope}%
\begin{pgfscope}%
\pgfsys@transformshift{2.308123in}{1.470598in}%
\pgfsys@useobject{currentmarker}{}%
\end{pgfscope}%
\begin{pgfscope}%
\pgfsys@transformshift{2.328780in}{1.396687in}%
\pgfsys@useobject{currentmarker}{}%
\end{pgfscope}%
\begin{pgfscope}%
\pgfsys@transformshift{2.350375in}{1.354606in}%
\pgfsys@useobject{currentmarker}{}%
\end{pgfscope}%
\begin{pgfscope}%
\pgfsys@transformshift{2.367746in}{1.356166in}%
\pgfsys@useobject{currentmarker}{}%
\end{pgfscope}%
\begin{pgfscope}%
\pgfsys@transformshift{2.385351in}{1.390749in}%
\pgfsys@useobject{currentmarker}{}%
\end{pgfscope}%
\begin{pgfscope}%
\pgfsys@transformshift{2.404129in}{1.440371in}%
\pgfsys@useobject{currentmarker}{}%
\end{pgfscope}%
\begin{pgfscope}%
\pgfsys@transformshift{2.424785in}{1.543296in}%
\pgfsys@useobject{currentmarker}{}%
\end{pgfscope}%
\begin{pgfscope}%
\pgfsys@transformshift{2.442390in}{1.654615in}%
\pgfsys@useobject{currentmarker}{}%
\end{pgfscope}%
\begin{pgfscope}%
\pgfsys@transformshift{2.459993in}{1.725343in}%
\pgfsys@useobject{currentmarker}{}%
\end{pgfscope}%
\begin{pgfscope}%
\pgfsys@transformshift{2.485814in}{1.727110in}%
\pgfsys@useobject{currentmarker}{}%
\end{pgfscope}%
\begin{pgfscope}%
\pgfsys@transformshift{2.502245in}{1.695998in}%
\pgfsys@useobject{currentmarker}{}%
\end{pgfscope}%
\begin{pgfscope}%
\pgfsys@transformshift{2.520319in}{1.594581in}%
\pgfsys@useobject{currentmarker}{}%
\end{pgfscope}%
\begin{pgfscope}%
\pgfsys@transformshift{2.541446in}{1.474292in}%
\pgfsys@useobject{currentmarker}{}%
\end{pgfscope}%
\begin{pgfscope}%
\pgfsys@transformshift{2.555999in}{1.413547in}%
\pgfsys@useobject{currentmarker}{}%
\end{pgfscope}%
\begin{pgfscope}%
\pgfsys@transformshift{2.576655in}{1.365337in}%
\pgfsys@useobject{currentmarker}{}%
\end{pgfscope}%
\begin{pgfscope}%
\pgfsys@transformshift{2.597780in}{1.356060in}%
\pgfsys@useobject{currentmarker}{}%
\end{pgfscope}%
\begin{pgfscope}%
\pgfsys@transformshift{2.615151in}{1.351818in}%
\pgfsys@useobject{currentmarker}{}%
\end{pgfscope}%
\begin{pgfscope}%
\pgfsys@transformshift{2.633693in}{1.362355in}%
\pgfsys@useobject{currentmarker}{}%
\end{pgfscope}%
\begin{pgfscope}%
\pgfsys@transformshift{2.654820in}{1.414744in}%
\pgfsys@useobject{currentmarker}{}%
\end{pgfscope}%
\begin{pgfscope}%
\pgfsys@transformshift{2.672659in}{1.482292in}%
\pgfsys@useobject{currentmarker}{}%
\end{pgfscope}%
\begin{pgfscope}%
\pgfsys@transformshift{2.694489in}{1.597472in}%
\pgfsys@useobject{currentmarker}{}%
\end{pgfscope}%
\begin{pgfscope}%
\pgfsys@transformshift{2.713032in}{1.699826in}%
\pgfsys@useobject{currentmarker}{}%
\end{pgfscope}%
\begin{pgfscope}%
\pgfsys@transformshift{2.730402in}{1.737241in}%
\pgfsys@useobject{currentmarker}{}%
\end{pgfscope}%
\begin{pgfscope}%
\pgfsys@transformshift{2.751058in}{1.714501in}%
\pgfsys@useobject{currentmarker}{}%
\end{pgfscope}%
\begin{pgfscope}%
\pgfsys@transformshift{2.769603in}{1.671840in}%
\pgfsys@useobject{currentmarker}{}%
\end{pgfscope}%
\begin{pgfscope}%
\pgfsys@transformshift{2.790023in}{1.535399in}%
\pgfsys@useobject{currentmarker}{}%
\end{pgfscope}%
\begin{pgfscope}%
\pgfsys@transformshift{2.808801in}{1.451171in}%
\pgfsys@useobject{currentmarker}{}%
\end{pgfscope}%
\begin{pgfscope}%
\pgfsys@transformshift{2.827815in}{1.387806in}%
\pgfsys@useobject{currentmarker}{}%
\end{pgfscope}%
\begin{pgfscope}%
\pgfsys@transformshift{2.847297in}{1.358146in}%
\pgfsys@useobject{currentmarker}{}%
\end{pgfscope}%
\begin{pgfscope}%
\pgfsys@transformshift{2.864667in}{1.359945in}%
\pgfsys@useobject{currentmarker}{}%
\end{pgfscope}%
\begin{pgfscope}%
\pgfsys@transformshift{2.889080in}{1.398757in}%
\pgfsys@useobject{currentmarker}{}%
\end{pgfscope}%
\begin{pgfscope}%
\pgfsys@transformshift{2.904337in}{1.446087in}%
\pgfsys@useobject{currentmarker}{}%
\end{pgfscope}%
\begin{pgfscope}%
\pgfsys@transformshift{2.926870in}{1.539208in}%
\pgfsys@useobject{currentmarker}{}%
\end{pgfscope}%
\begin{pgfscope}%
\pgfsys@transformshift{2.944946in}{1.643341in}%
\pgfsys@useobject{currentmarker}{}%
\end{pgfscope}%
\begin{pgfscope}%
\pgfsys@transformshift{2.962550in}{1.717844in}%
\pgfsys@useobject{currentmarker}{}%
\end{pgfscope}%
\begin{pgfscope}%
\pgfsys@transformshift{2.978510in}{1.426901in}%
\pgfsys@useobject{currentmarker}{}%
\end{pgfscope}%
\begin{pgfscope}%
\pgfsys@transformshift{2.999637in}{1.512848in}%
\pgfsys@useobject{currentmarker}{}%
\end{pgfscope}%
\begin{pgfscope}%
\pgfsys@transformshift{3.020997in}{1.597907in}%
\pgfsys@useobject{currentmarker}{}%
\end{pgfscope}%
\begin{pgfscope}%
\pgfsys@transformshift{3.038367in}{1.704613in}%
\pgfsys@useobject{currentmarker}{}%
\end{pgfscope}%
\begin{pgfscope}%
\pgfsys@transformshift{3.058554in}{1.741786in}%
\pgfsys@useobject{currentmarker}{}%
\end{pgfscope}%
\begin{pgfscope}%
\pgfsys@transformshift{3.074985in}{1.694093in}%
\pgfsys@useobject{currentmarker}{}%
\end{pgfscope}%
\begin{pgfscope}%
\pgfsys@transformshift{3.097284in}{1.620294in}%
\pgfsys@useobject{currentmarker}{}%
\end{pgfscope}%
\begin{pgfscope}%
\pgfsys@transformshift{3.113949in}{1.504819in}%
\pgfsys@useobject{currentmarker}{}%
\end{pgfscope}%
\begin{pgfscope}%
\pgfsys@transformshift{3.134606in}{1.415737in}%
\pgfsys@useobject{currentmarker}{}%
\end{pgfscope}%
\begin{pgfscope}%
\pgfsys@transformshift{3.155262in}{1.370781in}%
\pgfsys@useobject{currentmarker}{}%
\end{pgfscope}%
\begin{pgfscope}%
\pgfsys@transformshift{3.174040in}{1.352561in}%
\pgfsys@useobject{currentmarker}{}%
\end{pgfscope}%
\begin{pgfscope}%
\pgfsys@transformshift{3.194931in}{1.385384in}%
\pgfsys@useobject{currentmarker}{}%
\end{pgfscope}%
\begin{pgfscope}%
\pgfsys@transformshift{3.213241in}{1.425356in}%
\pgfsys@useobject{currentmarker}{}%
\end{pgfscope}%
\begin{pgfscope}%
\pgfsys@transformshift{3.230611in}{1.483994in}%
\pgfsys@useobject{currentmarker}{}%
\end{pgfscope}%
\begin{pgfscope}%
\pgfsys@transformshift{3.254083in}{1.628370in}%
\pgfsys@useobject{currentmarker}{}%
\end{pgfscope}%
\begin{pgfscope}%
\pgfsys@transformshift{3.269107in}{1.702542in}%
\pgfsys@useobject{currentmarker}{}%
\end{pgfscope}%
\begin{pgfscope}%
\pgfsys@transformshift{3.290466in}{1.744913in}%
\pgfsys@useobject{currentmarker}{}%
\end{pgfscope}%
\begin{pgfscope}%
\pgfsys@transformshift{3.307837in}{1.723489in}%
\pgfsys@useobject{currentmarker}{}%
\end{pgfscope}%
\begin{pgfscope}%
\pgfsys@transformshift{3.325676in}{1.670971in}%
\pgfsys@useobject{currentmarker}{}%
\end{pgfscope}%
\begin{pgfscope}%
\pgfsys@transformshift{3.347975in}{1.531397in}%
\pgfsys@useobject{currentmarker}{}%
\end{pgfscope}%
\begin{pgfscope}%
\pgfsys@transformshift{3.365814in}{1.442689in}%
\pgfsys@useobject{currentmarker}{}%
\end{pgfscope}%
\begin{pgfscope}%
\pgfsys@transformshift{3.383653in}{1.397663in}%
\pgfsys@useobject{currentmarker}{}%
\end{pgfscope}%
\begin{pgfscope}%
\pgfsys@transformshift{3.403841in}{1.358415in}%
\pgfsys@useobject{currentmarker}{}%
\end{pgfscope}%
\begin{pgfscope}%
\pgfsys@transformshift{3.421914in}{1.361623in}%
\pgfsys@useobject{currentmarker}{}%
\end{pgfscope}%
\begin{pgfscope}%
\pgfsys@transformshift{3.443744in}{1.404610in}%
\pgfsys@useobject{currentmarker}{}%
\end{pgfscope}%
\begin{pgfscope}%
\pgfsys@transformshift{3.461349in}{1.462838in}%
\pgfsys@useobject{currentmarker}{}%
\end{pgfscope}%
\begin{pgfscope}%
\pgfsys@transformshift{3.480128in}{1.547200in}%
\pgfsys@useobject{currentmarker}{}%
\end{pgfscope}%
\begin{pgfscope}%
\pgfsys@transformshift{3.500079in}{1.661350in}%
\pgfsys@useobject{currentmarker}{}%
\end{pgfscope}%
\begin{pgfscope}%
\pgfsys@transformshift{3.519329in}{1.721299in}%
\pgfsys@useobject{currentmarker}{}%
\end{pgfscope}%
\begin{pgfscope}%
\pgfsys@transformshift{3.539748in}{1.750243in}%
\pgfsys@useobject{currentmarker}{}%
\end{pgfscope}%
\begin{pgfscope}%
\pgfsys@transformshift{3.558293in}{1.726323in}%
\pgfsys@useobject{currentmarker}{}%
\end{pgfscope}%
\begin{pgfscope}%
\pgfsys@transformshift{3.578480in}{1.635952in}%
\pgfsys@useobject{currentmarker}{}%
\end{pgfscope}%
\begin{pgfscope}%
\pgfsys@transformshift{3.596319in}{1.539796in}%
\pgfsys@useobject{currentmarker}{}%
\end{pgfscope}%
\begin{pgfscope}%
\pgfsys@transformshift{3.614159in}{1.469017in}%
\pgfsys@useobject{currentmarker}{}%
\end{pgfscope}%
\begin{pgfscope}%
\pgfsys@transformshift{3.635754in}{1.405245in}%
\pgfsys@useobject{currentmarker}{}%
\end{pgfscope}%
\begin{pgfscope}%
\pgfsys@transformshift{3.653593in}{1.368275in}%
\pgfsys@useobject{currentmarker}{}%
\end{pgfscope}%
\begin{pgfscope}%
\pgfsys@transformshift{3.671433in}{1.358445in}%
\pgfsys@useobject{currentmarker}{}%
\end{pgfscope}%
\begin{pgfscope}%
\pgfsys@transformshift{3.693263in}{1.387553in}%
\pgfsys@useobject{currentmarker}{}%
\end{pgfscope}%
\begin{pgfscope}%
\pgfsys@transformshift{3.710868in}{1.422282in}%
\pgfsys@useobject{currentmarker}{}%
\end{pgfscope}%
\begin{pgfscope}%
\pgfsys@transformshift{3.731993in}{1.499649in}%
\pgfsys@useobject{currentmarker}{}%
\end{pgfscope}%
\begin{pgfscope}%
\pgfsys@transformshift{3.753588in}{1.598739in}%
\pgfsys@useobject{currentmarker}{}%
\end{pgfscope}%
\begin{pgfscope}%
\pgfsys@transformshift{3.771428in}{1.697419in}%
\pgfsys@useobject{currentmarker}{}%
\end{pgfscope}%
\begin{pgfscope}%
\pgfsys@transformshift{3.788562in}{1.728995in}%
\pgfsys@useobject{currentmarker}{}%
\end{pgfscope}%
\begin{pgfscope}%
\pgfsys@transformshift{3.806637in}{1.753279in}%
\pgfsys@useobject{currentmarker}{}%
\end{pgfscope}%
\begin{pgfscope}%
\pgfsys@transformshift{3.828702in}{1.745781in}%
\pgfsys@useobject{currentmarker}{}%
\end{pgfscope}%
\begin{pgfscope}%
\pgfsys@transformshift{3.845836in}{1.700716in}%
\pgfsys@useobject{currentmarker}{}%
\end{pgfscope}%
\begin{pgfscope}%
\pgfsys@transformshift{3.866726in}{1.585496in}%
\pgfsys@useobject{currentmarker}{}%
\end{pgfscope}%
\begin{pgfscope}%
\pgfsys@transformshift{3.885036in}{1.490123in}%
\pgfsys@useobject{currentmarker}{}%
\end{pgfscope}%
\begin{pgfscope}%
\pgfsys@transformshift{3.903345in}{1.427317in}%
\pgfsys@useobject{currentmarker}{}%
\end{pgfscope}%
\begin{pgfscope}%
\pgfsys@transformshift{3.920949in}{1.388886in}%
\pgfsys@useobject{currentmarker}{}%
\end{pgfscope}%
\begin{pgfscope}%
\pgfsys@transformshift{3.945128in}{1.361807in}%
\pgfsys@useobject{currentmarker}{}%
\end{pgfscope}%
\begin{pgfscope}%
\pgfsys@transformshift{3.960150in}{1.372245in}%
\pgfsys@useobject{currentmarker}{}%
\end{pgfscope}%
\begin{pgfscope}%
\pgfsys@transformshift{3.982449in}{1.416979in}%
\pgfsys@useobject{currentmarker}{}%
\end{pgfscope}%
\begin{pgfscope}%
\pgfsys@transformshift{3.999585in}{1.453197in}%
\pgfsys@useobject{currentmarker}{}%
\end{pgfscope}%
\begin{pgfscope}%
\pgfsys@transformshift{4.016953in}{1.516305in}%
\pgfsys@useobject{currentmarker}{}%
\end{pgfscope}%
\begin{pgfscope}%
\pgfsys@transformshift{4.038549in}{1.624892in}%
\pgfsys@useobject{currentmarker}{}%
\end{pgfscope}%
\begin{pgfscope}%
\pgfsys@transformshift{4.059908in}{1.722412in}%
\pgfsys@useobject{currentmarker}{}%
\end{pgfscope}%
\begin{pgfscope}%
\pgfsys@transformshift{4.077279in}{1.757286in}%
\pgfsys@useobject{currentmarker}{}%
\end{pgfscope}%
\begin{pgfscope}%
\pgfsys@transformshift{4.096527in}{1.763805in}%
\pgfsys@useobject{currentmarker}{}%
\end{pgfscope}%
\begin{pgfscope}%
\pgfsys@transformshift{4.119765in}{1.718170in}%
\pgfsys@useobject{currentmarker}{}%
\end{pgfscope}%
\begin{pgfscope}%
\pgfsys@transformshift{4.134787in}{1.679571in}%
\pgfsys@useobject{currentmarker}{}%
\end{pgfscope}%
\begin{pgfscope}%
\pgfsys@transformshift{4.153097in}{1.573789in}%
\pgfsys@useobject{currentmarker}{}%
\end{pgfscope}%
\begin{pgfscope}%
\pgfsys@transformshift{4.174222in}{1.482138in}%
\pgfsys@useobject{currentmarker}{}%
\end{pgfscope}%
\begin{pgfscope}%
\pgfsys@transformshift{4.191593in}{1.445200in}%
\pgfsys@useobject{currentmarker}{}%
\end{pgfscope}%
\begin{pgfscope}%
\pgfsys@transformshift{4.209666in}{1.392320in}%
\pgfsys@useobject{currentmarker}{}%
\end{pgfscope}%
\begin{pgfscope}%
\pgfsys@transformshift{4.231028in}{1.369536in}%
\pgfsys@useobject{currentmarker}{}%
\end{pgfscope}%
\begin{pgfscope}%
\pgfsys@transformshift{4.249101in}{1.368353in}%
\pgfsys@useobject{currentmarker}{}%
\end{pgfscope}%
\begin{pgfscope}%
\pgfsys@transformshift{4.269523in}{1.409052in}%
\pgfsys@useobject{currentmarker}{}%
\end{pgfscope}%
\begin{pgfscope}%
\pgfsys@transformshift{4.287362in}{1.459396in}%
\pgfsys@useobject{currentmarker}{}%
\end{pgfscope}%
\begin{pgfscope}%
\pgfsys@transformshift{4.305905in}{1.664351in}%
\pgfsys@useobject{currentmarker}{}%
\end{pgfscope}%
\begin{pgfscope}%
\pgfsys@transformshift{4.325389in}{1.537988in}%
\pgfsys@useobject{currentmarker}{}%
\end{pgfscope}%
\begin{pgfscope}%
\pgfsys@transformshift{4.349096in}{1.423411in}%
\pgfsys@useobject{currentmarker}{}%
\end{pgfscope}%
\begin{pgfscope}%
\pgfsys@transformshift{4.366230in}{1.377261in}%
\pgfsys@useobject{currentmarker}{}%
\end{pgfscope}%
\begin{pgfscope}%
\pgfsys@transformshift{4.384072in}{1.372047in}%
\pgfsys@useobject{currentmarker}{}%
\end{pgfscope}%
\begin{pgfscope}%
\pgfsys@transformshift{4.401440in}{1.415836in}%
\pgfsys@useobject{currentmarker}{}%
\end{pgfscope}%
\begin{pgfscope}%
\pgfsys@transformshift{4.420453in}{1.455289in}%
\pgfsys@useobject{currentmarker}{}%
\end{pgfscope}%
\begin{pgfscope}%
\pgfsys@transformshift{4.442283in}{1.565253in}%
\pgfsys@useobject{currentmarker}{}%
\end{pgfscope}%
\begin{pgfscope}%
\pgfsys@transformshift{4.460592in}{1.681920in}%
\pgfsys@useobject{currentmarker}{}%
\end{pgfscope}%
\begin{pgfscope}%
\pgfsys@transformshift{4.479605in}{1.755073in}%
\pgfsys@useobject{currentmarker}{}%
\end{pgfscope}%
\begin{pgfscope}%
\pgfsys@transformshift{4.477727in}{1.752130in}%
\pgfsys@useobject{currentmarker}{}%
\end{pgfscope}%
\begin{pgfscope}%
\pgfsys@transformshift{4.474911in}{1.736174in}%
\pgfsys@useobject{currentmarker}{}%
\end{pgfscope}%
\begin{pgfscope}%
\pgfsys@transformshift{4.455428in}{1.612066in}%
\pgfsys@useobject{currentmarker}{}%
\end{pgfscope}%
\begin{pgfscope}%
\pgfsys@transformshift{4.434303in}{1.470562in}%
\pgfsys@useobject{currentmarker}{}%
\end{pgfscope}%
\begin{pgfscope}%
\pgfsys@transformshift{4.412473in}{1.385698in}%
\pgfsys@useobject{currentmarker}{}%
\end{pgfscope}%
\begin{pgfscope}%
\pgfsys@transformshift{4.396980in}{1.367850in}%
\pgfsys@useobject{currentmarker}{}%
\end{pgfscope}%
\begin{pgfscope}%
\pgfsys@transformshift{4.378438in}{1.419045in}%
\pgfsys@useobject{currentmarker}{}%
\end{pgfscope}%
\begin{pgfscope}%
\pgfsys@transformshift{4.357076in}{1.532977in}%
\pgfsys@useobject{currentmarker}{}%
\end{pgfscope}%
\begin{pgfscope}%
\pgfsys@transformshift{4.339003in}{1.676036in}%
\pgfsys@useobject{currentmarker}{}%
\end{pgfscope}%
\begin{pgfscope}%
\pgfsys@transformshift{4.321164in}{1.754547in}%
\pgfsys@useobject{currentmarker}{}%
\end{pgfscope}%
\begin{pgfscope}%
\pgfsys@transformshift{4.298628in}{1.745726in}%
\pgfsys@useobject{currentmarker}{}%
\end{pgfscope}%
\begin{pgfscope}%
\pgfsys@transformshift{4.282903in}{1.659448in}%
\pgfsys@useobject{currentmarker}{}%
\end{pgfscope}%
\begin{pgfscope}%
\pgfsys@transformshift{4.264358in}{1.526747in}%
\pgfsys@useobject{currentmarker}{}%
\end{pgfscope}%
\begin{pgfscope}%
\pgfsys@transformshift{4.243702in}{1.421973in}%
\pgfsys@useobject{currentmarker}{}%
\end{pgfscope}%
\begin{pgfscope}%
\pgfsys@transformshift{4.221872in}{1.363981in}%
\pgfsys@useobject{currentmarker}{}%
\end{pgfscope}%
\begin{pgfscope}%
\pgfsys@transformshift{4.204267in}{1.386350in}%
\pgfsys@useobject{currentmarker}{}%
\end{pgfscope}%
\begin{pgfscope}%
\pgfsys@transformshift{4.187133in}{1.453325in}%
\pgfsys@useobject{currentmarker}{}%
\end{pgfscope}%
\begin{pgfscope}%
\pgfsys@transformshift{4.165772in}{1.582698in}%
\pgfsys@useobject{currentmarker}{}%
\end{pgfscope}%
\begin{pgfscope}%
\pgfsys@transformshift{4.148638in}{1.710532in}%
\pgfsys@useobject{currentmarker}{}%
\end{pgfscope}%
\begin{pgfscope}%
\pgfsys@transformshift{4.127511in}{1.758813in}%
\pgfsys@useobject{currentmarker}{}%
\end{pgfscope}%
\begin{pgfscope}%
\pgfsys@transformshift{4.108029in}{1.706625in}%
\pgfsys@useobject{currentmarker}{}%
\end{pgfscope}%
\begin{pgfscope}%
\pgfsys@transformshift{4.089015in}{1.569502in}%
\pgfsys@useobject{currentmarker}{}%
\end{pgfscope}%
\begin{pgfscope}%
\pgfsys@transformshift{4.072350in}{1.451460in}%
\pgfsys@useobject{currentmarker}{}%
\end{pgfscope}%
\begin{pgfscope}%
\pgfsys@transformshift{4.050051in}{1.386835in}%
\pgfsys@useobject{currentmarker}{}%
\end{pgfscope}%
\begin{pgfscope}%
\pgfsys@transformshift{4.033620in}{1.359226in}%
\pgfsys@useobject{currentmarker}{}%
\end{pgfscope}%
\begin{pgfscope}%
\pgfsys@transformshift{4.011556in}{1.402409in}%
\pgfsys@useobject{currentmarker}{}%
\end{pgfscope}%
\begin{pgfscope}%
\pgfsys@transformshift{3.994420in}{1.476683in}%
\pgfsys@useobject{currentmarker}{}%
\end{pgfscope}%
\begin{pgfscope}%
\pgfsys@transformshift{3.973764in}{1.621795in}%
\pgfsys@useobject{currentmarker}{}%
\end{pgfscope}%
\begin{pgfscope}%
\pgfsys@transformshift{3.956159in}{1.722521in}%
\pgfsys@useobject{currentmarker}{}%
\end{pgfscope}%
\begin{pgfscope}%
\pgfsys@transformshift{3.935503in}{1.751118in}%
\pgfsys@useobject{currentmarker}{}%
\end{pgfscope}%
\begin{pgfscope}%
\pgfsys@transformshift{3.915786in}{1.717108in}%
\pgfsys@useobject{currentmarker}{}%
\end{pgfscope}%
\begin{pgfscope}%
\pgfsys@transformshift{3.897947in}{1.752209in}%
\pgfsys@useobject{currentmarker}{}%
\end{pgfscope}%
\begin{pgfscope}%
\pgfsys@transformshift{3.877525in}{1.702430in}%
\pgfsys@useobject{currentmarker}{}%
\end{pgfscope}%
\begin{pgfscope}%
\pgfsys@transformshift{3.861095in}{1.585224in}%
\pgfsys@useobject{currentmarker}{}%
\end{pgfscope}%
\begin{pgfscope}%
\pgfsys@transformshift{3.839030in}{1.460601in}%
\pgfsys@useobject{currentmarker}{}%
\end{pgfscope}%
\begin{pgfscope}%
\pgfsys@transformshift{3.821189in}{1.393394in}%
\pgfsys@useobject{currentmarker}{}%
\end{pgfscope}%
\begin{pgfscope}%
\pgfsys@transformshift{3.797247in}{1.356900in}%
\pgfsys@useobject{currentmarker}{}%
\end{pgfscope}%
\begin{pgfscope}%
\pgfsys@transformshift{3.783164in}{1.380704in}%
\pgfsys@useobject{currentmarker}{}%
\end{pgfscope}%
\begin{pgfscope}%
\pgfsys@transformshift{3.763917in}{1.449093in}%
\pgfsys@useobject{currentmarker}{}%
\end{pgfscope}%
\begin{pgfscope}%
\pgfsys@transformshift{3.744903in}{1.563891in}%
\pgfsys@useobject{currentmarker}{}%
\end{pgfscope}%
\begin{pgfscope}%
\pgfsys@transformshift{3.722368in}{1.702457in}%
\pgfsys@useobject{currentmarker}{}%
\end{pgfscope}%
\begin{pgfscope}%
\pgfsys@transformshift{3.705703in}{1.744935in}%
\pgfsys@useobject{currentmarker}{}%
\end{pgfscope}%
\begin{pgfscope}%
\pgfsys@transformshift{3.686924in}{1.730888in}%
\pgfsys@useobject{currentmarker}{}%
\end{pgfscope}%
\begin{pgfscope}%
\pgfsys@transformshift{3.667442in}{1.625178in}%
\pgfsys@useobject{currentmarker}{}%
\end{pgfscope}%
\begin{pgfscope}%
\pgfsys@transformshift{3.649134in}{1.510026in}%
\pgfsys@useobject{currentmarker}{}%
\end{pgfscope}%
\begin{pgfscope}%
\pgfsys@transformshift{3.627772in}{1.427400in}%
\pgfsys@useobject{currentmarker}{}%
\end{pgfscope}%
\begin{pgfscope}%
\pgfsys@transformshift{3.611342in}{1.377739in}%
\pgfsys@useobject{currentmarker}{}%
\end{pgfscope}%
\begin{pgfscope}%
\pgfsys@transformshift{3.592797in}{1.354324in}%
\pgfsys@useobject{currentmarker}{}%
\end{pgfscope}%
\begin{pgfscope}%
\pgfsys@transformshift{3.568855in}{1.385374in}%
\pgfsys@useobject{currentmarker}{}%
\end{pgfscope}%
\begin{pgfscope}%
\pgfsys@transformshift{3.552190in}{1.421205in}%
\pgfsys@useobject{currentmarker}{}%
\end{pgfscope}%
\begin{pgfscope}%
\pgfsys@transformshift{3.530594in}{1.512474in}%
\pgfsys@useobject{currentmarker}{}%
\end{pgfscope}%
\begin{pgfscope}%
\pgfsys@transformshift{3.513695in}{1.612565in}%
\pgfsys@useobject{currentmarker}{}%
\end{pgfscope}%
\begin{pgfscope}%
\pgfsys@transformshift{3.497498in}{1.693108in}%
\pgfsys@useobject{currentmarker}{}%
\end{pgfscope}%
\begin{pgfscope}%
\pgfsys@transformshift{3.474963in}{1.744088in}%
\pgfsys@useobject{currentmarker}{}%
\end{pgfscope}%
\begin{pgfscope}%
\pgfsys@transformshift{3.451726in}{1.712365in}%
\pgfsys@useobject{currentmarker}{}%
\end{pgfscope}%
\begin{pgfscope}%
\pgfsys@transformshift{3.437173in}{1.632096in}%
\pgfsys@useobject{currentmarker}{}%
\end{pgfscope}%
\begin{pgfscope}%
\pgfsys@transformshift{3.415812in}{1.514442in}%
\pgfsys@useobject{currentmarker}{}%
\end{pgfscope}%
\begin{pgfscope}%
\pgfsys@transformshift{3.398441in}{1.430963in}%
\pgfsys@useobject{currentmarker}{}%
\end{pgfscope}%
\begin{pgfscope}%
\pgfsys@transformshift{3.377551in}{1.371166in}%
\pgfsys@useobject{currentmarker}{}%
\end{pgfscope}%
\begin{pgfscope}%
\pgfsys@transformshift{3.356191in}{1.394365in}%
\pgfsys@useobject{currentmarker}{}%
\end{pgfscope}%
\begin{pgfscope}%
\pgfsys@transformshift{3.341403in}{1.361051in}%
\pgfsys@useobject{currentmarker}{}%
\end{pgfscope}%
\begin{pgfscope}%
\pgfsys@transformshift{3.320982in}{1.361462in}%
\pgfsys@useobject{currentmarker}{}%
\end{pgfscope}%
\begin{pgfscope}%
\pgfsys@transformshift{3.304080in}{1.397569in}%
\pgfsys@useobject{currentmarker}{}%
\end{pgfscope}%
\begin{pgfscope}%
\pgfsys@transformshift{3.286241in}{1.493184in}%
\pgfsys@useobject{currentmarker}{}%
\end{pgfscope}%
\begin{pgfscope}%
\pgfsys@transformshift{3.264176in}{1.603310in}%
\pgfsys@useobject{currentmarker}{}%
\end{pgfscope}%
\begin{pgfscope}%
\pgfsys@transformshift{3.245163in}{1.714370in}%
\pgfsys@useobject{currentmarker}{}%
\end{pgfscope}%
\begin{pgfscope}%
\pgfsys@transformshift{3.226855in}{1.739601in}%
\pgfsys@useobject{currentmarker}{}%
\end{pgfscope}%
\begin{pgfscope}%
\pgfsys@transformshift{3.207842in}{1.694961in}%
\pgfsys@useobject{currentmarker}{}%
\end{pgfscope}%
\begin{pgfscope}%
\pgfsys@transformshift{3.188125in}{1.573259in}%
\pgfsys@useobject{currentmarker}{}%
\end{pgfscope}%
\begin{pgfscope}%
\pgfsys@transformshift{3.168877in}{1.462079in}%
\pgfsys@useobject{currentmarker}{}%
\end{pgfscope}%
\begin{pgfscope}%
\pgfsys@transformshift{3.147516in}{1.393351in}%
\pgfsys@useobject{currentmarker}{}%
\end{pgfscope}%
\begin{pgfscope}%
\pgfsys@transformshift{3.130851in}{1.357538in}%
\pgfsys@useobject{currentmarker}{}%
\end{pgfscope}%
\begin{pgfscope}%
\pgfsys@transformshift{3.108786in}{1.354336in}%
\pgfsys@useobject{currentmarker}{}%
\end{pgfscope}%
\begin{pgfscope}%
\pgfsys@transformshift{3.091181in}{1.385681in}%
\pgfsys@useobject{currentmarker}{}%
\end{pgfscope}%
\begin{pgfscope}%
\pgfsys@transformshift{3.069820in}{1.456109in}%
\pgfsys@useobject{currentmarker}{}%
\end{pgfscope}%
\begin{pgfscope}%
\pgfsys@transformshift{3.051981in}{1.553590in}%
\pgfsys@useobject{currentmarker}{}%
\end{pgfscope}%
\begin{pgfscope}%
\pgfsys@transformshift{3.030622in}{1.682871in}%
\pgfsys@useobject{currentmarker}{}%
\end{pgfscope}%
\begin{pgfscope}%
\pgfsys@transformshift{3.015128in}{1.731569in}%
\pgfsys@useobject{currentmarker}{}%
\end{pgfscope}%
\begin{pgfscope}%
\pgfsys@transformshift{2.996820in}{1.732417in}%
\pgfsys@useobject{currentmarker}{}%
\end{pgfscope}%
\begin{pgfscope}%
\pgfsys@transformshift{2.974287in}{1.647028in}%
\pgfsys@useobject{currentmarker}{}%
\end{pgfscope}%
\begin{pgfscope}%
\pgfsys@transformshift{2.955743in}{1.543922in}%
\pgfsys@useobject{currentmarker}{}%
\end{pgfscope}%
\begin{pgfscope}%
\pgfsys@transformshift{2.936964in}{1.452951in}%
\pgfsys@useobject{currentmarker}{}%
\end{pgfscope}%
\begin{pgfscope}%
\pgfsys@transformshift{2.918421in}{1.392952in}%
\pgfsys@useobject{currentmarker}{}%
\end{pgfscope}%
\begin{pgfscope}%
\pgfsys@transformshift{2.897060in}{1.352050in}%
\pgfsys@useobject{currentmarker}{}%
\end{pgfscope}%
\begin{pgfscope}%
\pgfsys@transformshift{2.878986in}{1.361209in}%
\pgfsys@useobject{currentmarker}{}%
\end{pgfscope}%
\begin{pgfscope}%
\pgfsys@transformshift{2.859739in}{1.394142in}%
\pgfsys@useobject{currentmarker}{}%
\end{pgfscope}%
\begin{pgfscope}%
\pgfsys@transformshift{2.841665in}{1.456057in}%
\pgfsys@useobject{currentmarker}{}%
\end{pgfscope}%
\begin{pgfscope}%
\pgfsys@transformshift{2.821946in}{1.570217in}%
\pgfsys@useobject{currentmarker}{}%
\end{pgfscope}%
\begin{pgfscope}%
\pgfsys@transformshift{2.803404in}{1.630672in}%
\pgfsys@useobject{currentmarker}{}%
\end{pgfscope}%
\begin{pgfscope}%
\pgfsys@transformshift{2.778288in}{1.732604in}%
\pgfsys@useobject{currentmarker}{}%
\end{pgfscope}%
\begin{pgfscope}%
\pgfsys@transformshift{2.763498in}{1.734657in}%
\pgfsys@useobject{currentmarker}{}%
\end{pgfscope}%
\begin{pgfscope}%
\pgfsys@transformshift{2.743782in}{1.737966in}%
\pgfsys@useobject{currentmarker}{}%
\end{pgfscope}%
\begin{pgfscope}%
\pgfsys@transformshift{2.726882in}{1.701561in}%
\pgfsys@useobject{currentmarker}{}%
\end{pgfscope}%
\begin{pgfscope}%
\pgfsys@transformshift{2.703643in}{1.576902in}%
\pgfsys@useobject{currentmarker}{}%
\end{pgfscope}%
\begin{pgfscope}%
\pgfsys@transformshift{2.690264in}{1.504011in}%
\pgfsys@useobject{currentmarker}{}%
\end{pgfscope}%
\begin{pgfscope}%
\pgfsys@transformshift{2.670077in}{1.415133in}%
\pgfsys@useobject{currentmarker}{}%
\end{pgfscope}%
\begin{pgfscope}%
\pgfsys@transformshift{2.647072in}{1.362524in}%
\pgfsys@useobject{currentmarker}{}%
\end{pgfscope}%
\begin{pgfscope}%
\pgfsys@transformshift{2.629233in}{1.349482in}%
\pgfsys@useobject{currentmarker}{}%
\end{pgfscope}%
\begin{pgfscope}%
\pgfsys@transformshift{2.611160in}{1.376386in}%
\pgfsys@useobject{currentmarker}{}%
\end{pgfscope}%
\begin{pgfscope}%
\pgfsys@transformshift{2.593320in}{1.419857in}%
\pgfsys@useobject{currentmarker}{}%
\end{pgfscope}%
\begin{pgfscope}%
\pgfsys@transformshift{2.571256in}{1.520426in}%
\pgfsys@useobject{currentmarker}{}%
\end{pgfscope}%
\begin{pgfscope}%
\pgfsys@transformshift{2.555294in}{1.619083in}%
\pgfsys@useobject{currentmarker}{}%
\end{pgfscope}%
\begin{pgfscope}%
\pgfsys@transformshift{2.534872in}{1.719243in}%
\pgfsys@useobject{currentmarker}{}%
\end{pgfscope}%
\begin{pgfscope}%
\pgfsys@transformshift{2.514450in}{1.736847in}%
\pgfsys@useobject{currentmarker}{}%
\end{pgfscope}%
\begin{pgfscope}%
\pgfsys@transformshift{2.495908in}{1.697831in}%
\pgfsys@useobject{currentmarker}{}%
\end{pgfscope}%
\begin{pgfscope}%
\pgfsys@transformshift{2.477364in}{1.598666in}%
\pgfsys@useobject{currentmarker}{}%
\end{pgfscope}%
\begin{pgfscope}%
\pgfsys@transformshift{2.455065in}{1.494789in}%
\pgfsys@useobject{currentmarker}{}%
\end{pgfscope}%
\begin{pgfscope}%
\pgfsys@transformshift{2.437225in}{1.417367in}%
\pgfsys@useobject{currentmarker}{}%
\end{pgfscope}%
\begin{pgfscope}%
\pgfsys@transformshift{2.420089in}{1.372106in}%
\pgfsys@useobject{currentmarker}{}%
\end{pgfscope}%
\begin{pgfscope}%
\pgfsys@transformshift{2.396853in}{1.350010in}%
\pgfsys@useobject{currentmarker}{}%
\end{pgfscope}%
\begin{pgfscope}%
\pgfsys@transformshift{2.380420in}{1.366567in}%
\pgfsys@useobject{currentmarker}{}%
\end{pgfscope}%
\begin{pgfscope}%
\pgfsys@transformshift{2.360703in}{1.401933in}%
\pgfsys@useobject{currentmarker}{}%
\end{pgfscope}%
\begin{pgfscope}%
\pgfsys@transformshift{2.342864in}{1.453787in}%
\pgfsys@useobject{currentmarker}{}%
\end{pgfscope}%
\begin{pgfscope}%
\pgfsys@transformshift{2.321034in}{1.575928in}%
\pgfsys@useobject{currentmarker}{}%
\end{pgfscope}%
\begin{pgfscope}%
\pgfsys@transformshift{2.302255in}{1.685118in}%
\pgfsys@useobject{currentmarker}{}%
\end{pgfscope}%
\begin{pgfscope}%
\pgfsys@transformshift{2.283007in}{1.722626in}%
\pgfsys@useobject{currentmarker}{}%
\end{pgfscope}%
\begin{pgfscope}%
\pgfsys@transformshift{2.264465in}{1.736516in}%
\pgfsys@useobject{currentmarker}{}%
\end{pgfscope}%
\begin{pgfscope}%
\pgfsys@transformshift{2.245452in}{1.720473in}%
\pgfsys@useobject{currentmarker}{}%
\end{pgfscope}%
\begin{pgfscope}%
\pgfsys@transformshift{2.226907in}{1.643411in}%
\pgfsys@useobject{currentmarker}{}%
\end{pgfscope}%
\begin{pgfscope}%
\pgfsys@transformshift{2.206017in}{1.537887in}%
\pgfsys@useobject{currentmarker}{}%
\end{pgfscope}%
\begin{pgfscope}%
\pgfsys@transformshift{2.186535in}{1.455298in}%
\pgfsys@useobject{currentmarker}{}%
\end{pgfscope}%
\begin{pgfscope}%
\pgfsys@transformshift{2.168225in}{1.412894in}%
\pgfsys@useobject{currentmarker}{}%
\end{pgfscope}%
\begin{pgfscope}%
\pgfsys@transformshift{2.149448in}{1.370518in}%
\pgfsys@useobject{currentmarker}{}%
\end{pgfscope}%
\begin{pgfscope}%
\pgfsys@transformshift{2.129729in}{1.351946in}%
\pgfsys@useobject{currentmarker}{}%
\end{pgfscope}%
\begin{pgfscope}%
\pgfsys@transformshift{2.109544in}{1.376896in}%
\pgfsys@useobject{currentmarker}{}%
\end{pgfscope}%
\begin{pgfscope}%
\pgfsys@transformshift{2.090296in}{1.429967in}%
\pgfsys@useobject{currentmarker}{}%
\end{pgfscope}%
\begin{pgfscope}%
\pgfsys@transformshift{2.071517in}{1.504201in}%
\pgfsys@useobject{currentmarker}{}%
\end{pgfscope}%
\begin{pgfscope}%
\pgfsys@transformshift{2.052270in}{1.528399in}%
\pgfsys@useobject{currentmarker}{}%
\end{pgfscope}%
\begin{pgfscope}%
\pgfsys@transformshift{2.033256in}{1.578883in}%
\pgfsys@useobject{currentmarker}{}%
\end{pgfscope}%
\begin{pgfscope}%
\pgfsys@transformshift{2.014009in}{1.638537in}%
\pgfsys@useobject{currentmarker}{}%
\end{pgfscope}%
\begin{pgfscope}%
\pgfsys@transformshift{1.995933in}{1.725243in}%
\pgfsys@useobject{currentmarker}{}%
\end{pgfscope}%
\begin{pgfscope}%
\pgfsys@transformshift{1.977391in}{1.740284in}%
\pgfsys@useobject{currentmarker}{}%
\end{pgfscope}%
\begin{pgfscope}%
\pgfsys@transformshift{1.959081in}{1.699165in}%
\pgfsys@useobject{currentmarker}{}%
\end{pgfscope}%
\begin{pgfscope}%
\pgfsys@transformshift{1.936313in}{1.622727in}%
\pgfsys@useobject{currentmarker}{}%
\end{pgfscope}%
\begin{pgfscope}%
\pgfsys@transformshift{1.918005in}{1.508897in}%
\pgfsys@useobject{currentmarker}{}%
\end{pgfscope}%
\begin{pgfscope}%
\pgfsys@transformshift{1.900164in}{1.447593in}%
\pgfsys@useobject{currentmarker}{}%
\end{pgfscope}%
\begin{pgfscope}%
\pgfsys@transformshift{1.878099in}{1.385250in}%
\pgfsys@useobject{currentmarker}{}%
\end{pgfscope}%
\begin{pgfscope}%
\pgfsys@transformshift{1.860260in}{1.355695in}%
\pgfsys@useobject{currentmarker}{}%
\end{pgfscope}%
\begin{pgfscope}%
\pgfsys@transformshift{1.841483in}{1.362369in}%
\pgfsys@useobject{currentmarker}{}%
\end{pgfscope}%
\begin{pgfscope}%
\pgfsys@transformshift{1.822939in}{1.395840in}%
\pgfsys@useobject{currentmarker}{}%
\end{pgfscope}%
\begin{pgfscope}%
\pgfsys@transformshift{1.803456in}{1.453040in}%
\pgfsys@useobject{currentmarker}{}%
\end{pgfscope}%
\begin{pgfscope}%
\pgfsys@transformshift{1.783974in}{1.560242in}%
\pgfsys@useobject{currentmarker}{}%
\end{pgfscope}%
\begin{pgfscope}%
\pgfsys@transformshift{1.765899in}{1.635809in}%
\pgfsys@useobject{currentmarker}{}%
\end{pgfscope}%
\begin{pgfscope}%
\pgfsys@transformshift{1.744539in}{1.718172in}%
\pgfsys@useobject{currentmarker}{}%
\end{pgfscope}%
\begin{pgfscope}%
\pgfsys@transformshift{1.726229in}{1.746070in}%
\pgfsys@useobject{currentmarker}{}%
\end{pgfscope}%
\begin{pgfscope}%
\pgfsys@transformshift{1.707687in}{1.721726in}%
\pgfsys@useobject{currentmarker}{}%
\end{pgfscope}%
\begin{pgfscope}%
\pgfsys@transformshift{1.689142in}{1.662228in}%
\pgfsys@useobject{currentmarker}{}%
\end{pgfscope}%
\begin{pgfscope}%
\pgfsys@transformshift{1.668252in}{1.536482in}%
\pgfsys@useobject{currentmarker}{}%
\end{pgfscope}%
\begin{pgfscope}%
\pgfsys@transformshift{1.649239in}{1.461746in}%
\pgfsys@useobject{currentmarker}{}%
\end{pgfscope}%
\begin{pgfscope}%
\pgfsys@transformshift{1.630225in}{1.413568in}%
\pgfsys@useobject{currentmarker}{}%
\end{pgfscope}%
\begin{pgfscope}%
\pgfsys@transformshift{1.611917in}{1.373770in}%
\pgfsys@useobject{currentmarker}{}%
\end{pgfscope}%
\begin{pgfscope}%
\pgfsys@transformshift{1.590556in}{1.356268in}%
\pgfsys@useobject{currentmarker}{}%
\end{pgfscope}%
\begin{pgfscope}%
\pgfsys@transformshift{1.571543in}{1.364397in}%
\pgfsys@useobject{currentmarker}{}%
\end{pgfscope}%
\begin{pgfscope}%
\pgfsys@transformshift{1.553703in}{1.402634in}%
\pgfsys@useobject{currentmarker}{}%
\end{pgfscope}%
\begin{pgfscope}%
\pgfsys@transformshift{1.532578in}{1.464973in}%
\pgfsys@useobject{currentmarker}{}%
\end{pgfscope}%
\begin{pgfscope}%
\pgfsys@transformshift{1.516382in}{1.456325in}%
\pgfsys@useobject{currentmarker}{}%
\end{pgfscope}%
\begin{pgfscope}%
\pgfsys@transformshift{1.494318in}{1.572148in}%
\pgfsys@useobject{currentmarker}{}%
\end{pgfscope}%
\begin{pgfscope}%
\pgfsys@transformshift{1.472487in}{1.697628in}%
\pgfsys@useobject{currentmarker}{}%
\end{pgfscope}%
\begin{pgfscope}%
\pgfsys@transformshift{1.456760in}{1.708012in}%
\pgfsys@useobject{currentmarker}{}%
\end{pgfscope}%
\begin{pgfscope}%
\pgfsys@transformshift{1.438452in}{1.707750in}%
\pgfsys@useobject{currentmarker}{}%
\end{pgfscope}%
\begin{pgfscope}%
\pgfsys@transformshift{1.420144in}{1.750402in}%
\pgfsys@useobject{currentmarker}{}%
\end{pgfscope}%
\begin{pgfscope}%
\pgfsys@transformshift{1.398314in}{1.734748in}%
\pgfsys@useobject{currentmarker}{}%
\end{pgfscope}%
\begin{pgfscope}%
\pgfsys@transformshift{1.379300in}{1.669137in}%
\pgfsys@useobject{currentmarker}{}%
\end{pgfscope}%
\begin{pgfscope}%
\pgfsys@transformshift{1.361461in}{1.556001in}%
\pgfsys@useobject{currentmarker}{}%
\end{pgfscope}%
\begin{pgfscope}%
\pgfsys@transformshift{1.342917in}{1.475231in}%
\pgfsys@useobject{currentmarker}{}%
\end{pgfscope}%
\begin{pgfscope}%
\pgfsys@transformshift{1.321557in}{1.412139in}%
\pgfsys@useobject{currentmarker}{}%
\end{pgfscope}%
\begin{pgfscope}%
\pgfsys@transformshift{1.302778in}{1.378019in}%
\pgfsys@useobject{currentmarker}{}%
\end{pgfscope}%
\begin{pgfscope}%
\pgfsys@transformshift{1.284470in}{1.358394in}%
\pgfsys@useobject{currentmarker}{}%
\end{pgfscope}%
\begin{pgfscope}%
\pgfsys@transformshift{1.265221in}{1.376534in}%
\pgfsys@useobject{currentmarker}{}%
\end{pgfscope}%
\begin{pgfscope}%
\pgfsys@transformshift{1.247147in}{1.419120in}%
\pgfsys@useobject{currentmarker}{}%
\end{pgfscope}%
\begin{pgfscope}%
\pgfsys@transformshift{1.225553in}{1.475084in}%
\pgfsys@useobject{currentmarker}{}%
\end{pgfscope}%
\begin{pgfscope}%
\pgfsys@transformshift{1.207009in}{1.573068in}%
\pgfsys@useobject{currentmarker}{}%
\end{pgfscope}%
\begin{pgfscope}%
\pgfsys@transformshift{1.188464in}{1.673322in}%
\pgfsys@useobject{currentmarker}{}%
\end{pgfscope}%
\begin{pgfscope}%
\pgfsys@transformshift{1.170156in}{1.738487in}%
\pgfsys@useobject{currentmarker}{}%
\end{pgfscope}%
\begin{pgfscope}%
\pgfsys@transformshift{1.151612in}{1.758497in}%
\pgfsys@useobject{currentmarker}{}%
\end{pgfscope}%
\begin{pgfscope}%
\pgfsys@transformshift{1.129313in}{1.744583in}%
\pgfsys@useobject{currentmarker}{}%
\end{pgfscope}%
\begin{pgfscope}%
\pgfsys@transformshift{1.111943in}{1.686403in}%
\pgfsys@useobject{currentmarker}{}%
\end{pgfscope}%
\begin{pgfscope}%
\pgfsys@transformshift{1.092460in}{1.579294in}%
\pgfsys@useobject{currentmarker}{}%
\end{pgfscope}%
\begin{pgfscope}%
\pgfsys@transformshift{1.072509in}{1.500316in}%
\pgfsys@useobject{currentmarker}{}%
\end{pgfscope}%
\begin{pgfscope}%
\pgfsys@transformshift{1.053262in}{1.436121in}%
\pgfsys@useobject{currentmarker}{}%
\end{pgfscope}%
\begin{pgfscope}%
\pgfsys@transformshift{1.033543in}{1.402810in}%
\pgfsys@useobject{currentmarker}{}%
\end{pgfscope}%
\begin{pgfscope}%
\pgfsys@transformshift{1.015470in}{1.381280in}%
\pgfsys@useobject{currentmarker}{}%
\end{pgfscope}%
\begin{pgfscope}%
\pgfsys@transformshift{0.996456in}{1.364751in}%
\pgfsys@useobject{currentmarker}{}%
\end{pgfscope}%
\begin{pgfscope}%
\pgfsys@transformshift{0.979086in}{1.393143in}%
\pgfsys@useobject{currentmarker}{}%
\end{pgfscope}%
\begin{pgfscope}%
\pgfsys@transformshift{0.957961in}{1.443598in}%
\pgfsys@useobject{currentmarker}{}%
\end{pgfscope}%
\begin{pgfscope}%
\pgfsys@transformshift{0.938245in}{1.505421in}%
\pgfsys@useobject{currentmarker}{}%
\end{pgfscope}%
\begin{pgfscope}%
\pgfsys@transformshift{0.919700in}{1.553332in}%
\pgfsys@useobject{currentmarker}{}%
\end{pgfscope}%
\begin{pgfscope}%
\pgfsys@transformshift{0.900452in}{1.661893in}%
\pgfsys@useobject{currentmarker}{}%
\end{pgfscope}%
\begin{pgfscope}%
\pgfsys@transformshift{0.882142in}{1.737587in}%
\pgfsys@useobject{currentmarker}{}%
\end{pgfscope}%
\begin{pgfscope}%
\pgfsys@transformshift{0.861252in}{1.765089in}%
\pgfsys@useobject{currentmarker}{}%
\end{pgfscope}%
\begin{pgfscope}%
\pgfsys@transformshift{0.842944in}{1.753089in}%
\pgfsys@useobject{currentmarker}{}%
\end{pgfscope}%
\begin{pgfscope}%
\pgfsys@transformshift{0.823931in}{1.694590in}%
\pgfsys@useobject{currentmarker}{}%
\end{pgfscope}%
\begin{pgfscope}%
\pgfsys@transformshift{0.803040in}{1.573209in}%
\pgfsys@useobject{currentmarker}{}%
\end{pgfscope}%
\begin{pgfscope}%
\pgfsys@transformshift{0.784027in}{1.503655in}%
\pgfsys@useobject{currentmarker}{}%
\end{pgfscope}%
\begin{pgfscope}%
\pgfsys@transformshift{0.767596in}{1.444789in}%
\pgfsys@useobject{currentmarker}{}%
\end{pgfscope}%
\begin{pgfscope}%
\pgfsys@transformshift{0.748348in}{1.395813in}%
\pgfsys@useobject{currentmarker}{}%
\end{pgfscope}%
\begin{pgfscope}%
\pgfsys@transformshift{0.726518in}{1.367683in}%
\pgfsys@useobject{currentmarker}{}%
\end{pgfscope}%
\begin{pgfscope}%
\pgfsys@transformshift{0.707974in}{1.370433in}%
\pgfsys@useobject{currentmarker}{}%
\end{pgfscope}%
\begin{pgfscope}%
\pgfsys@transformshift{0.688492in}{1.376374in}%
\pgfsys@useobject{currentmarker}{}%
\end{pgfscope}%
\begin{pgfscope}%
\pgfsys@transformshift{0.669478in}{1.479123in}%
\pgfsys@useobject{currentmarker}{}%
\end{pgfscope}%
\begin{pgfscope}%
\pgfsys@transformshift{0.648353in}{1.416895in}%
\pgfsys@useobject{currentmarker}{}%
\end{pgfscope}%
\begin{pgfscope}%
\pgfsys@transformshift{0.651639in}{1.421720in}%
\pgfsys@useobject{currentmarker}{}%
\end{pgfscope}%
\begin{pgfscope}%
\pgfsys@transformshift{0.659150in}{1.458849in}%
\pgfsys@useobject{currentmarker}{}%
\end{pgfscope}%
\begin{pgfscope}%
\pgfsys@transformshift{0.676990in}{1.562713in}%
\pgfsys@useobject{currentmarker}{}%
\end{pgfscope}%
\begin{pgfscope}%
\pgfsys@transformshift{0.695063in}{1.700913in}%
\pgfsys@useobject{currentmarker}{}%
\end{pgfscope}%
\begin{pgfscope}%
\pgfsys@transformshift{0.713373in}{1.763612in}%
\pgfsys@useobject{currentmarker}{}%
\end{pgfscope}%
\begin{pgfscope}%
\pgfsys@transformshift{0.734498in}{1.741477in}%
\pgfsys@useobject{currentmarker}{}%
\end{pgfscope}%
\begin{pgfscope}%
\pgfsys@transformshift{0.753043in}{1.629450in}%
\pgfsys@useobject{currentmarker}{}%
\end{pgfscope}%
\begin{pgfscope}%
\pgfsys@transformshift{0.770177in}{1.496137in}%
\pgfsys@useobject{currentmarker}{}%
\end{pgfscope}%
\begin{pgfscope}%
\pgfsys@transformshift{0.795058in}{1.388284in}%
\pgfsys@useobject{currentmarker}{}%
\end{pgfscope}%
\begin{pgfscope}%
\pgfsys@transformshift{0.810317in}{1.364980in}%
\pgfsys@useobject{currentmarker}{}%
\end{pgfscope}%
\begin{pgfscope}%
\pgfsys@transformshift{0.827922in}{1.408316in}%
\pgfsys@useobject{currentmarker}{}%
\end{pgfscope}%
\begin{pgfscope}%
\pgfsys@transformshift{0.849047in}{1.503790in}%
\pgfsys@useobject{currentmarker}{}%
\end{pgfscope}%
\begin{pgfscope}%
\pgfsys@transformshift{0.867591in}{1.638682in}%
\pgfsys@useobject{currentmarker}{}%
\end{pgfscope}%
\begin{pgfscope}%
\pgfsys@transformshift{0.886602in}{1.744276in}%
\pgfsys@useobject{currentmarker}{}%
\end{pgfscope}%
\begin{pgfscope}%
\pgfsys@transformshift{0.908432in}{1.749318in}%
\pgfsys@useobject{currentmarker}{}%
\end{pgfscope}%
\begin{pgfscope}%
\pgfsys@transformshift{0.924394in}{1.675930in}%
\pgfsys@useobject{currentmarker}{}%
\end{pgfscope}%
\begin{pgfscope}%
\pgfsys@transformshift{0.944816in}{1.510812in}%
\pgfsys@useobject{currentmarker}{}%
\end{pgfscope}%
\begin{pgfscope}%
\pgfsys@transformshift{0.964298in}{1.415929in}%
\pgfsys@useobject{currentmarker}{}%
\end{pgfscope}%
\begin{pgfscope}%
\pgfsys@transformshift{0.981434in}{1.366402in}%
\pgfsys@useobject{currentmarker}{}%
\end{pgfscope}%
\begin{pgfscope}%
\pgfsys@transformshift{1.001151in}{1.371084in}%
\pgfsys@useobject{currentmarker}{}%
\end{pgfscope}%
\begin{pgfscope}%
\pgfsys@transformshift{1.022746in}{1.442503in}%
\pgfsys@useobject{currentmarker}{}%
\end{pgfscope}%
\begin{pgfscope}%
\pgfsys@transformshift{1.041994in}{1.543230in}%
\pgfsys@useobject{currentmarker}{}%
\end{pgfscope}%
\begin{pgfscope}%
\pgfsys@transformshift{1.061711in}{1.680286in}%
\pgfsys@useobject{currentmarker}{}%
\end{pgfscope}%
\begin{pgfscope}%
\pgfsys@transformshift{1.080255in}{1.750117in}%
\pgfsys@useobject{currentmarker}{}%
\end{pgfscope}%
\begin{pgfscope}%
\pgfsys@transformshift{1.097625in}{1.744656in}%
\pgfsys@useobject{currentmarker}{}%
\end{pgfscope}%
\begin{pgfscope}%
\pgfsys@transformshift{1.118516in}{1.627409in}%
\pgfsys@useobject{currentmarker}{}%
\end{pgfscope}%
\begin{pgfscope}%
\pgfsys@transformshift{1.137529in}{1.486140in}%
\pgfsys@useobject{currentmarker}{}%
\end{pgfscope}%
\begin{pgfscope}%
\pgfsys@transformshift{1.157246in}{1.401111in}%
\pgfsys@useobject{currentmarker}{}%
\end{pgfscope}%
\begin{pgfscope}%
\pgfsys@transformshift{1.176964in}{1.359200in}%
\pgfsys@useobject{currentmarker}{}%
\end{pgfscope}%
\begin{pgfscope}%
\pgfsys@transformshift{1.195741in}{1.377722in}%
\pgfsys@useobject{currentmarker}{}%
\end{pgfscope}%
\begin{pgfscope}%
\pgfsys@transformshift{1.214520in}{1.438724in}%
\pgfsys@useobject{currentmarker}{}%
\end{pgfscope}%
\begin{pgfscope}%
\pgfsys@transformshift{1.234707in}{1.539171in}%
\pgfsys@useobject{currentmarker}{}%
\end{pgfscope}%
\begin{pgfscope}%
\pgfsys@transformshift{1.254424in}{1.672934in}%
\pgfsys@useobject{currentmarker}{}%
\end{pgfscope}%
\begin{pgfscope}%
\pgfsys@transformshift{1.271560in}{1.744318in}%
\pgfsys@useobject{currentmarker}{}%
\end{pgfscope}%
\begin{pgfscope}%
\pgfsys@transformshift{1.291745in}{1.743157in}%
\pgfsys@useobject{currentmarker}{}%
\end{pgfscope}%
\begin{pgfscope}%
\pgfsys@transformshift{1.310055in}{1.665600in}%
\pgfsys@useobject{currentmarker}{}%
\end{pgfscope}%
\begin{pgfscope}%
\pgfsys@transformshift{1.330946in}{1.538648in}%
\pgfsys@useobject{currentmarker}{}%
\end{pgfscope}%
\begin{pgfscope}%
\pgfsys@transformshift{1.349256in}{1.434032in}%
\pgfsys@useobject{currentmarker}{}%
\end{pgfscope}%
\begin{pgfscope}%
\pgfsys@transformshift{1.366390in}{1.374436in}%
\pgfsys@useobject{currentmarker}{}%
\end{pgfscope}%
\begin{pgfscope}%
\pgfsys@transformshift{1.386577in}{1.356470in}%
\pgfsys@useobject{currentmarker}{}%
\end{pgfscope}%
\begin{pgfscope}%
\pgfsys@transformshift{1.408173in}{1.402720in}%
\pgfsys@useobject{currentmarker}{}%
\end{pgfscope}%
\begin{pgfscope}%
\pgfsys@transformshift{1.425776in}{1.464689in}%
\pgfsys@useobject{currentmarker}{}%
\end{pgfscope}%
\begin{pgfscope}%
\pgfsys@transformshift{1.443380in}{1.541776in}%
\pgfsys@useobject{currentmarker}{}%
\end{pgfscope}%
\begin{pgfscope}%
\pgfsys@transformshift{1.465681in}{1.656193in}%
\pgfsys@useobject{currentmarker}{}%
\end{pgfscope}%
\begin{pgfscope}%
\pgfsys@transformshift{1.482112in}{1.732826in}%
\pgfsys@useobject{currentmarker}{}%
\end{pgfscope}%
\begin{pgfscope}%
\pgfsys@transformshift{1.502298in}{1.738614in}%
\pgfsys@useobject{currentmarker}{}%
\end{pgfscope}%
\begin{pgfscope}%
\pgfsys@transformshift{1.520373in}{1.687924in}%
\pgfsys@useobject{currentmarker}{}%
\end{pgfscope}%
\begin{pgfscope}%
\pgfsys@transformshift{1.542907in}{1.547648in}%
\pgfsys@useobject{currentmarker}{}%
\end{pgfscope}%
\begin{pgfscope}%
\pgfsys@transformshift{1.561685in}{1.497584in}%
\pgfsys@useobject{currentmarker}{}%
\end{pgfscope}%
\begin{pgfscope}%
\pgfsys@transformshift{1.578819in}{1.408849in}%
\pgfsys@useobject{currentmarker}{}%
\end{pgfscope}%
\begin{pgfscope}%
\pgfsys@transformshift{1.598538in}{1.365460in}%
\pgfsys@useobject{currentmarker}{}%
\end{pgfscope}%
\begin{pgfscope}%
\pgfsys@transformshift{1.617080in}{1.354045in}%
\pgfsys@useobject{currentmarker}{}%
\end{pgfscope}%
\begin{pgfscope}%
\pgfsys@transformshift{1.636799in}{1.386189in}%
\pgfsys@useobject{currentmarker}{}%
\end{pgfscope}%
\begin{pgfscope}%
\pgfsys@transformshift{1.654167in}{1.430758in}%
\pgfsys@useobject{currentmarker}{}%
\end{pgfscope}%
\begin{pgfscope}%
\pgfsys@transformshift{1.676703in}{1.501109in}%
\pgfsys@useobject{currentmarker}{}%
\end{pgfscope}%
\begin{pgfscope}%
\pgfsys@transformshift{1.695716in}{1.575896in}%
\pgfsys@useobject{currentmarker}{}%
\end{pgfscope}%
\begin{pgfscope}%
\pgfsys@transformshift{1.713790in}{1.680118in}%
\pgfsys@useobject{currentmarker}{}%
\end{pgfscope}%
\begin{pgfscope}%
\pgfsys@transformshift{1.729281in}{1.739741in}%
\pgfsys@useobject{currentmarker}{}%
\end{pgfscope}%
\begin{pgfscope}%
\pgfsys@transformshift{1.750408in}{1.730985in}%
\pgfsys@useobject{currentmarker}{}%
\end{pgfscope}%
\begin{pgfscope}%
\pgfsys@transformshift{1.770359in}{1.641094in}%
\pgfsys@useobject{currentmarker}{}%
\end{pgfscope}%
\begin{pgfscope}%
\pgfsys@transformshift{1.788903in}{1.513354in}%
\pgfsys@useobject{currentmarker}{}%
\end{pgfscope}%
\begin{pgfscope}%
\pgfsys@transformshift{1.808151in}{1.427701in}%
\pgfsys@useobject{currentmarker}{}%
\end{pgfscope}%
\begin{pgfscope}%
\pgfsys@transformshift{1.830215in}{1.378210in}%
\pgfsys@useobject{currentmarker}{}%
\end{pgfscope}%
\begin{pgfscope}%
\pgfsys@transformshift{1.849229in}{1.355057in}%
\pgfsys@useobject{currentmarker}{}%
\end{pgfscope}%
\begin{pgfscope}%
\pgfsys@transformshift{1.867771in}{1.363591in}%
\pgfsys@useobject{currentmarker}{}%
\end{pgfscope}%
\begin{pgfscope}%
\pgfsys@transformshift{1.885612in}{1.405289in}%
\pgfsys@useobject{currentmarker}{}%
\end{pgfscope}%
\begin{pgfscope}%
\pgfsys@transformshift{1.904860in}{1.457554in}%
\pgfsys@useobject{currentmarker}{}%
\end{pgfscope}%
\begin{pgfscope}%
\pgfsys@transformshift{1.922699in}{1.415598in}%
\pgfsys@useobject{currentmarker}{}%
\end{pgfscope}%
\begin{pgfscope}%
\pgfsys@transformshift{1.941712in}{1.365787in}%
\pgfsys@useobject{currentmarker}{}%
\end{pgfscope}%
\begin{pgfscope}%
\pgfsys@transformshift{1.964480in}{1.360673in}%
\pgfsys@useobject{currentmarker}{}%
\end{pgfscope}%
\begin{pgfscope}%
\pgfsys@transformshift{1.982554in}{1.405795in}%
\pgfsys@useobject{currentmarker}{}%
\end{pgfscope}%
\begin{pgfscope}%
\pgfsys@transformshift{2.002038in}{1.483417in}%
\pgfsys@useobject{currentmarker}{}%
\end{pgfscope}%
\begin{pgfscope}%
\pgfsys@transformshift{2.020580in}{1.594237in}%
\pgfsys@useobject{currentmarker}{}%
\end{pgfscope}%
\begin{pgfscope}%
\pgfsys@transformshift{2.038890in}{1.685545in}%
\pgfsys@useobject{currentmarker}{}%
\end{pgfscope}%
\begin{pgfscope}%
\pgfsys@transformshift{2.062127in}{1.739575in}%
\pgfsys@useobject{currentmarker}{}%
\end{pgfscope}%
\begin{pgfscope}%
\pgfsys@transformshift{2.078560in}{1.712571in}%
\pgfsys@useobject{currentmarker}{}%
\end{pgfscope}%
\begin{pgfscope}%
\pgfsys@transformshift{2.103676in}{1.568520in}%
\pgfsys@useobject{currentmarker}{}%
\end{pgfscope}%
\begin{pgfscope}%
\pgfsys@transformshift{2.117290in}{1.474250in}%
\pgfsys@useobject{currentmarker}{}%
\end{pgfscope}%
\begin{pgfscope}%
\pgfsys@transformshift{2.134894in}{1.402146in}%
\pgfsys@useobject{currentmarker}{}%
\end{pgfscope}%
\begin{pgfscope}%
\pgfsys@transformshift{2.152265in}{1.365695in}%
\pgfsys@useobject{currentmarker}{}%
\end{pgfscope}%
\begin{pgfscope}%
\pgfsys@transformshift{2.173390in}{1.354713in}%
\pgfsys@useobject{currentmarker}{}%
\end{pgfscope}%
\begin{pgfscope}%
\pgfsys@transformshift{2.194280in}{1.396627in}%
\pgfsys@useobject{currentmarker}{}%
\end{pgfscope}%
\begin{pgfscope}%
\pgfsys@transformshift{2.212354in}{1.456280in}%
\pgfsys@useobject{currentmarker}{}%
\end{pgfscope}%
\begin{pgfscope}%
\pgfsys@transformshift{2.236532in}{1.545260in}%
\pgfsys@useobject{currentmarker}{}%
\end{pgfscope}%
\begin{pgfscope}%
\pgfsys@transformshift{2.252023in}{1.675915in}%
\pgfsys@useobject{currentmarker}{}%
\end{pgfscope}%
\begin{pgfscope}%
\pgfsys@transformshift{2.271271in}{1.736372in}%
\pgfsys@useobject{currentmarker}{}%
\end{pgfscope}%
\begin{pgfscope}%
\pgfsys@transformshift{2.290989in}{1.722721in}%
\pgfsys@useobject{currentmarker}{}%
\end{pgfscope}%
\begin{pgfscope}%
\pgfsys@transformshift{2.310003in}{1.654386in}%
\pgfsys@useobject{currentmarker}{}%
\end{pgfscope}%
\begin{pgfscope}%
\pgfsys@transformshift{2.327607in}{1.519626in}%
\pgfsys@useobject{currentmarker}{}%
\end{pgfscope}%
\begin{pgfscope}%
\pgfsys@transformshift{2.346855in}{1.434883in}%
\pgfsys@useobject{currentmarker}{}%
\end{pgfscope}%
\begin{pgfscope}%
\pgfsys@transformshift{2.365398in}{1.376950in}%
\pgfsys@useobject{currentmarker}{}%
\end{pgfscope}%
\begin{pgfscope}%
\pgfsys@transformshift{2.386759in}{1.350269in}%
\pgfsys@useobject{currentmarker}{}%
\end{pgfscope}%
\begin{pgfscope}%
\pgfsys@transformshift{2.406241in}{1.370842in}%
\pgfsys@useobject{currentmarker}{}%
\end{pgfscope}%
\begin{pgfscope}%
\pgfsys@transformshift{2.422203in}{1.412574in}%
\pgfsys@useobject{currentmarker}{}%
\end{pgfscope}%
\begin{pgfscope}%
\pgfsys@transformshift{2.442625in}{1.452862in}%
\pgfsys@useobject{currentmarker}{}%
\end{pgfscope}%
\begin{pgfscope}%
\pgfsys@transformshift{2.463515in}{1.562542in}%
\pgfsys@useobject{currentmarker}{}%
\end{pgfscope}%
\begin{pgfscope}%
\pgfsys@transformshift{2.482529in}{1.672659in}%
\pgfsys@useobject{currentmarker}{}%
\end{pgfscope}%
\begin{pgfscope}%
\pgfsys@transformshift{2.502714in}{1.736332in}%
\pgfsys@useobject{currentmarker}{}%
\end{pgfscope}%
\begin{pgfscope}%
\pgfsys@transformshift{2.519615in}{1.729165in}%
\pgfsys@useobject{currentmarker}{}%
\end{pgfscope}%
\begin{pgfscope}%
\pgfsys@transformshift{2.539566in}{1.658234in}%
\pgfsys@useobject{currentmarker}{}%
\end{pgfscope}%
\begin{pgfscope}%
\pgfsys@transformshift{2.560459in}{1.523471in}%
\pgfsys@useobject{currentmarker}{}%
\end{pgfscope}%
\begin{pgfscope}%
\pgfsys@transformshift{2.578767in}{1.473732in}%
\pgfsys@useobject{currentmarker}{}%
\end{pgfscope}%
\begin{pgfscope}%
\pgfsys@transformshift{2.597077in}{1.407791in}%
\pgfsys@useobject{currentmarker}{}%
\end{pgfscope}%
\begin{pgfscope}%
\pgfsys@transformshift{2.614680in}{1.362966in}%
\pgfsys@useobject{currentmarker}{}%
\end{pgfscope}%
\begin{pgfscope}%
\pgfsys@transformshift{2.637215in}{1.357250in}%
\pgfsys@useobject{currentmarker}{}%
\end{pgfscope}%
\begin{pgfscope}%
\pgfsys@transformshift{2.656463in}{1.400796in}%
\pgfsys@useobject{currentmarker}{}%
\end{pgfscope}%
\begin{pgfscope}%
\pgfsys@transformshift{2.673597in}{1.457723in}%
\pgfsys@useobject{currentmarker}{}%
\end{pgfscope}%
\begin{pgfscope}%
\pgfsys@transformshift{2.692847in}{1.546577in}%
\pgfsys@useobject{currentmarker}{}%
\end{pgfscope}%
\begin{pgfscope}%
\pgfsys@transformshift{2.713503in}{1.676599in}%
\pgfsys@useobject{currentmarker}{}%
\end{pgfscope}%
\begin{pgfscope}%
\pgfsys@transformshift{2.730871in}{1.728399in}%
\pgfsys@useobject{currentmarker}{}%
\end{pgfscope}%
\begin{pgfscope}%
\pgfsys@transformshift{2.751998in}{1.732281in}%
\pgfsys@useobject{currentmarker}{}%
\end{pgfscope}%
\begin{pgfscope}%
\pgfsys@transformshift{2.770541in}{1.687470in}%
\pgfsys@useobject{currentmarker}{}%
\end{pgfscope}%
\begin{pgfscope}%
\pgfsys@transformshift{2.788380in}{1.571117in}%
\pgfsys@useobject{currentmarker}{}%
\end{pgfscope}%
\begin{pgfscope}%
\pgfsys@transformshift{2.810915in}{1.452597in}%
\pgfsys@useobject{currentmarker}{}%
\end{pgfscope}%
\begin{pgfscope}%
\pgfsys@transformshift{2.828049in}{1.402046in}%
\pgfsys@useobject{currentmarker}{}%
\end{pgfscope}%
\begin{pgfscope}%
\pgfsys@transformshift{2.846125in}{1.375627in}%
\pgfsys@useobject{currentmarker}{}%
\end{pgfscope}%
\begin{pgfscope}%
\pgfsys@transformshift{2.866076in}{1.352031in}%
\pgfsys@useobject{currentmarker}{}%
\end{pgfscope}%
\begin{pgfscope}%
\pgfsys@transformshift{2.884620in}{1.371382in}%
\pgfsys@useobject{currentmarker}{}%
\end{pgfscope}%
\begin{pgfscope}%
\pgfsys@transformshift{2.905979in}{1.422699in}%
\pgfsys@useobject{currentmarker}{}%
\end{pgfscope}%
\begin{pgfscope}%
\pgfsys@transformshift{2.923115in}{1.501495in}%
\pgfsys@useobject{currentmarker}{}%
\end{pgfscope}%
\begin{pgfscope}%
\pgfsys@transformshift{2.945180in}{1.615094in}%
\pgfsys@useobject{currentmarker}{}%
\end{pgfscope}%
\begin{pgfscope}%
\pgfsys@transformshift{2.962314in}{1.704759in}%
\pgfsys@useobject{currentmarker}{}%
\end{pgfscope}%
\begin{pgfscope}%
\pgfsys@transformshift{2.980624in}{1.738109in}%
\pgfsys@useobject{currentmarker}{}%
\end{pgfscope}%
\begin{pgfscope}%
\pgfsys@transformshift{3.000811in}{1.726991in}%
\pgfsys@useobject{currentmarker}{}%
\end{pgfscope}%
\begin{pgfscope}%
\pgfsys@transformshift{3.018416in}{1.674676in}%
\pgfsys@useobject{currentmarker}{}%
\end{pgfscope}%
\begin{pgfscope}%
\pgfsys@transformshift{3.038367in}{1.567661in}%
\pgfsys@useobject{currentmarker}{}%
\end{pgfscope}%
\begin{pgfscope}%
\pgfsys@transformshift{3.057849in}{1.459315in}%
\pgfsys@useobject{currentmarker}{}%
\end{pgfscope}%
\begin{pgfscope}%
\pgfsys@transformshift{3.076862in}{1.403288in}%
\pgfsys@useobject{currentmarker}{}%
\end{pgfscope}%
\begin{pgfscope}%
\pgfsys@transformshift{3.096581in}{1.364612in}%
\pgfsys@useobject{currentmarker}{}%
\end{pgfscope}%
\begin{pgfscope}%
\pgfsys@transformshift{3.115592in}{1.358460in}%
\pgfsys@useobject{currentmarker}{}%
\end{pgfscope}%
\begin{pgfscope}%
\pgfsys@transformshift{3.132259in}{1.380205in}%
\pgfsys@useobject{currentmarker}{}%
\end{pgfscope}%
\begin{pgfscope}%
\pgfsys@transformshift{3.154090in}{1.438916in}%
\pgfsys@useobject{currentmarker}{}%
\end{pgfscope}%
\begin{pgfscope}%
\pgfsys@transformshift{3.175215in}{1.535744in}%
\pgfsys@useobject{currentmarker}{}%
\end{pgfscope}%
\begin{pgfscope}%
\pgfsys@transformshift{3.189534in}{1.628120in}%
\pgfsys@useobject{currentmarker}{}%
\end{pgfscope}%
\begin{pgfscope}%
\pgfsys@transformshift{3.210893in}{1.427971in}%
\pgfsys@useobject{currentmarker}{}%
\end{pgfscope}%
\begin{pgfscope}%
\pgfsys@transformshift{3.231315in}{1.510179in}%
\pgfsys@useobject{currentmarker}{}%
\end{pgfscope}%
\begin{pgfscope}%
\pgfsys@transformshift{3.251031in}{1.593272in}%
\pgfsys@useobject{currentmarker}{}%
\end{pgfscope}%
\begin{pgfscope}%
\pgfsys@transformshift{3.270515in}{1.711172in}%
\pgfsys@useobject{currentmarker}{}%
\end{pgfscope}%
\begin{pgfscope}%
\pgfsys@transformshift{3.287649in}{1.744931in}%
\pgfsys@useobject{currentmarker}{}%
\end{pgfscope}%
\begin{pgfscope}%
\pgfsys@transformshift{3.306428in}{1.728849in}%
\pgfsys@useobject{currentmarker}{}%
\end{pgfscope}%
\begin{pgfscope}%
\pgfsys@transformshift{3.328024in}{1.665123in}%
\pgfsys@useobject{currentmarker}{}%
\end{pgfscope}%
\begin{pgfscope}%
\pgfsys@transformshift{3.345629in}{1.557178in}%
\pgfsys@useobject{currentmarker}{}%
\end{pgfscope}%
\begin{pgfscope}%
\pgfsys@transformshift{3.366285in}{1.465701in}%
\pgfsys@useobject{currentmarker}{}%
\end{pgfscope}%
\begin{pgfscope}%
\pgfsys@transformshift{3.384593in}{1.400990in}%
\pgfsys@useobject{currentmarker}{}%
\end{pgfscope}%
\begin{pgfscope}%
\pgfsys@transformshift{3.404780in}{1.362525in}%
\pgfsys@useobject{currentmarker}{}%
\end{pgfscope}%
\begin{pgfscope}%
\pgfsys@transformshift{3.425671in}{1.356208in}%
\pgfsys@useobject{currentmarker}{}%
\end{pgfscope}%
\begin{pgfscope}%
\pgfsys@transformshift{3.443979in}{1.386273in}%
\pgfsys@useobject{currentmarker}{}%
\end{pgfscope}%
\begin{pgfscope}%
\pgfsys@transformshift{3.461349in}{1.444856in}%
\pgfsys@useobject{currentmarker}{}%
\end{pgfscope}%
\begin{pgfscope}%
\pgfsys@transformshift{3.478720in}{1.504821in}%
\pgfsys@useobject{currentmarker}{}%
\end{pgfscope}%
\begin{pgfscope}%
\pgfsys@transformshift{3.500784in}{1.617654in}%
\pgfsys@useobject{currentmarker}{}%
\end{pgfscope}%
\begin{pgfscope}%
\pgfsys@transformshift{3.521440in}{1.722847in}%
\pgfsys@useobject{currentmarker}{}%
\end{pgfscope}%
\begin{pgfscope}%
\pgfsys@transformshift{3.537871in}{1.749726in}%
\pgfsys@useobject{currentmarker}{}%
\end{pgfscope}%
\begin{pgfscope}%
\pgfsys@transformshift{3.559232in}{1.731200in}%
\pgfsys@useobject{currentmarker}{}%
\end{pgfscope}%
\begin{pgfscope}%
\pgfsys@transformshift{3.576837in}{1.669911in}%
\pgfsys@useobject{currentmarker}{}%
\end{pgfscope}%
\begin{pgfscope}%
\pgfsys@transformshift{3.597728in}{1.541277in}%
\pgfsys@useobject{currentmarker}{}%
\end{pgfscope}%
\begin{pgfscope}%
\pgfsys@transformshift{3.614627in}{1.486113in}%
\pgfsys@useobject{currentmarker}{}%
\end{pgfscope}%
\begin{pgfscope}%
\pgfsys@transformshift{3.639509in}{1.414805in}%
\pgfsys@useobject{currentmarker}{}%
\end{pgfscope}%
\begin{pgfscope}%
\pgfsys@transformshift{3.654531in}{1.380759in}%
\pgfsys@useobject{currentmarker}{}%
\end{pgfscope}%
\begin{pgfscope}%
\pgfsys@transformshift{3.674484in}{1.358594in}%
\pgfsys@useobject{currentmarker}{}%
\end{pgfscope}%
\begin{pgfscope}%
\pgfsys@transformshift{3.693966in}{1.372716in}%
\pgfsys@useobject{currentmarker}{}%
\end{pgfscope}%
\begin{pgfscope}%
\pgfsys@transformshift{3.711571in}{1.415147in}%
\pgfsys@useobject{currentmarker}{}%
\end{pgfscope}%
\begin{pgfscope}%
\pgfsys@transformshift{3.729879in}{1.473066in}%
\pgfsys@useobject{currentmarker}{}%
\end{pgfscope}%
\begin{pgfscope}%
\pgfsys@transformshift{3.749363in}{1.552549in}%
\pgfsys@useobject{currentmarker}{}%
\end{pgfscope}%
\begin{pgfscope}%
\pgfsys@transformshift{3.770254in}{1.652666in}%
\pgfsys@useobject{currentmarker}{}%
\end{pgfscope}%
\begin{pgfscope}%
\pgfsys@transformshift{3.789501in}{1.718590in}%
\pgfsys@useobject{currentmarker}{}%
\end{pgfscope}%
\begin{pgfscope}%
\pgfsys@transformshift{3.805932in}{1.754870in}%
\pgfsys@useobject{currentmarker}{}%
\end{pgfscope}%
\begin{pgfscope}%
\pgfsys@transformshift{3.827528in}{1.635780in}%
\pgfsys@useobject{currentmarker}{}%
\end{pgfscope}%
\begin{pgfscope}%
\pgfsys@transformshift{3.845133in}{1.713945in}%
\pgfsys@useobject{currentmarker}{}%
\end{pgfscope}%
\begin{pgfscope}%
\pgfsys@transformshift{3.866963in}{1.755047in}%
\pgfsys@useobject{currentmarker}{}%
\end{pgfscope}%
\begin{pgfscope}%
\pgfsys@transformshift{3.884331in}{1.749294in}%
\pgfsys@useobject{currentmarker}{}%
\end{pgfscope}%
\begin{pgfscope}%
\pgfsys@transformshift{3.902407in}{1.686220in}%
\pgfsys@useobject{currentmarker}{}%
\end{pgfscope}%
\begin{pgfscope}%
\pgfsys@transformshift{3.923297in}{1.566138in}%
\pgfsys@useobject{currentmarker}{}%
\end{pgfscope}%
\begin{pgfscope}%
\pgfsys@transformshift{3.941137in}{1.481178in}%
\pgfsys@useobject{currentmarker}{}%
\end{pgfscope}%
\begin{pgfscope}%
\pgfsys@transformshift{3.960384in}{1.422827in}%
\pgfsys@useobject{currentmarker}{}%
\end{pgfscope}%
\begin{pgfscope}%
\pgfsys@transformshift{3.979398in}{1.376847in}%
\pgfsys@useobject{currentmarker}{}%
\end{pgfscope}%
\begin{pgfscope}%
\pgfsys@transformshift{3.999585in}{1.363645in}%
\pgfsys@useobject{currentmarker}{}%
\end{pgfscope}%
\begin{pgfscope}%
\pgfsys@transformshift{4.018127in}{1.390072in}%
\pgfsys@useobject{currentmarker}{}%
\end{pgfscope}%
\begin{pgfscope}%
\pgfsys@transformshift{4.039723in}{1.443504in}%
\pgfsys@useobject{currentmarker}{}%
\end{pgfscope}%
\begin{pgfscope}%
\pgfsys@transformshift{4.058266in}{1.531804in}%
\pgfsys@useobject{currentmarker}{}%
\end{pgfscope}%
\begin{pgfscope}%
\pgfsys@transformshift{4.076576in}{1.625457in}%
\pgfsys@useobject{currentmarker}{}%
\end{pgfscope}%
\begin{pgfscope}%
\pgfsys@transformshift{4.094649in}{1.713911in}%
\pgfsys@useobject{currentmarker}{}%
\end{pgfscope}%
\begin{pgfscope}%
\pgfsys@transformshift{4.117419in}{1.759825in}%
\pgfsys@useobject{currentmarker}{}%
\end{pgfscope}%
\begin{pgfscope}%
\pgfsys@transformshift{4.134084in}{1.759373in}%
\pgfsys@useobject{currentmarker}{}%
\end{pgfscope}%
\begin{pgfscope}%
\pgfsys@transformshift{4.151923in}{1.731657in}%
\pgfsys@useobject{currentmarker}{}%
\end{pgfscope}%
\begin{pgfscope}%
\pgfsys@transformshift{4.175631in}{1.631861in}%
\pgfsys@useobject{currentmarker}{}%
\end{pgfscope}%
\begin{pgfscope}%
\pgfsys@transformshift{4.190184in}{1.532052in}%
\pgfsys@useobject{currentmarker}{}%
\end{pgfscope}%
\begin{pgfscope}%
\pgfsys@transformshift{4.212718in}{1.463305in}%
\pgfsys@useobject{currentmarker}{}%
\end{pgfscope}%
\begin{pgfscope}%
\pgfsys@transformshift{4.230323in}{1.406412in}%
\pgfsys@useobject{currentmarker}{}%
\end{pgfscope}%
\begin{pgfscope}%
\pgfsys@transformshift{4.250275in}{1.380319in}%
\pgfsys@useobject{currentmarker}{}%
\end{pgfscope}%
\begin{pgfscope}%
\pgfsys@transformshift{4.268115in}{1.367043in}%
\pgfsys@useobject{currentmarker}{}%
\end{pgfscope}%
\begin{pgfscope}%
\pgfsys@transformshift{4.290179in}{1.402910in}%
\pgfsys@useobject{currentmarker}{}%
\end{pgfscope}%
\begin{pgfscope}%
\pgfsys@transformshift{4.308019in}{1.432602in}%
\pgfsys@useobject{currentmarker}{}%
\end{pgfscope}%
\begin{pgfscope}%
\pgfsys@transformshift{4.327032in}{1.493674in}%
\pgfsys@useobject{currentmarker}{}%
\end{pgfscope}%
\begin{pgfscope}%
\pgfsys@transformshift{4.345340in}{1.577342in}%
\pgfsys@useobject{currentmarker}{}%
\end{pgfscope}%
\begin{pgfscope}%
\pgfsys@transformshift{4.366701in}{1.686167in}%
\pgfsys@useobject{currentmarker}{}%
\end{pgfscope}%
\begin{pgfscope}%
\pgfsys@transformshift{4.384306in}{1.751429in}%
\pgfsys@useobject{currentmarker}{}%
\end{pgfscope}%
\begin{pgfscope}%
\pgfsys@transformshift{4.402380in}{1.772697in}%
\pgfsys@useobject{currentmarker}{}%
\end{pgfscope}%
\begin{pgfscope}%
\pgfsys@transformshift{4.423505in}{1.754703in}%
\pgfsys@useobject{currentmarker}{}%
\end{pgfscope}%
\begin{pgfscope}%
\pgfsys@transformshift{4.441580in}{1.693771in}%
\pgfsys@useobject{currentmarker}{}%
\end{pgfscope}%
\begin{pgfscope}%
\pgfsys@transformshift{4.459654in}{1.600178in}%
\pgfsys@useobject{currentmarker}{}%
\end{pgfscope}%
\begin{pgfscope}%
\pgfsys@transformshift{4.478902in}{1.684167in}%
\pgfsys@useobject{currentmarker}{}%
\end{pgfscope}%
\begin{pgfscope}%
\pgfsys@transformshift{4.480310in}{1.682109in}%
\pgfsys@useobject{currentmarker}{}%
\end{pgfscope}%
\begin{pgfscope}%
\pgfsys@transformshift{4.474207in}{1.721656in}%
\pgfsys@useobject{currentmarker}{}%
\end{pgfscope}%
\begin{pgfscope}%
\pgfsys@transformshift{4.456132in}{1.772204in}%
\pgfsys@useobject{currentmarker}{}%
\end{pgfscope}%
\begin{pgfscope}%
\pgfsys@transformshift{4.438058in}{1.750196in}%
\pgfsys@useobject{currentmarker}{}%
\end{pgfscope}%
\begin{pgfscope}%
\pgfsys@transformshift{4.415290in}{1.615349in}%
\pgfsys@useobject{currentmarker}{}%
\end{pgfscope}%
\begin{pgfscope}%
\pgfsys@transformshift{4.397685in}{1.492111in}%
\pgfsys@useobject{currentmarker}{}%
\end{pgfscope}%
\begin{pgfscope}%
\pgfsys@transformshift{4.378438in}{1.406837in}%
\pgfsys@useobject{currentmarker}{}%
\end{pgfscope}%
\begin{pgfscope}%
\pgfsys@transformshift{4.354259in}{1.370407in}%
\pgfsys@useobject{currentmarker}{}%
\end{pgfscope}%
\begin{pgfscope}%
\pgfsys@transformshift{4.339942in}{1.412625in}%
\pgfsys@useobject{currentmarker}{}%
\end{pgfscope}%
\begin{pgfscope}%
\pgfsys@transformshift{4.323275in}{1.493272in}%
\pgfsys@useobject{currentmarker}{}%
\end{pgfscope}%
\begin{pgfscope}%
\pgfsys@transformshift{4.300976in}{1.635436in}%
\pgfsys@useobject{currentmarker}{}%
\end{pgfscope}%
\begin{pgfscope}%
\pgfsys@transformshift{4.279380in}{1.752122in}%
\pgfsys@useobject{currentmarker}{}%
\end{pgfscope}%
\begin{pgfscope}%
\pgfsys@transformshift{4.262012in}{1.760759in}%
\pgfsys@useobject{currentmarker}{}%
\end{pgfscope}%
\begin{pgfscope}%
\pgfsys@transformshift{4.243233in}{1.695285in}%
\pgfsys@useobject{currentmarker}{}%
\end{pgfscope}%
\begin{pgfscope}%
\pgfsys@transformshift{4.225160in}{1.559815in}%
\pgfsys@useobject{currentmarker}{}%
\end{pgfscope}%
\begin{pgfscope}%
\pgfsys@transformshift{4.203564in}{1.435993in}%
\pgfsys@useobject{currentmarker}{}%
\end{pgfscope}%
\begin{pgfscope}%
\pgfsys@transformshift{4.185725in}{1.379120in}%
\pgfsys@useobject{currentmarker}{}%
\end{pgfscope}%
\begin{pgfscope}%
\pgfsys@transformshift{4.166008in}{1.372288in}%
\pgfsys@useobject{currentmarker}{}%
\end{pgfscope}%
\begin{pgfscope}%
\pgfsys@transformshift{4.147698in}{1.390629in}%
\pgfsys@useobject{currentmarker}{}%
\end{pgfscope}%
\begin{pgfscope}%
\pgfsys@transformshift{4.127745in}{1.475260in}%
\pgfsys@useobject{currentmarker}{}%
\end{pgfscope}%
\begin{pgfscope}%
\pgfsys@transformshift{4.110377in}{1.597797in}%
\pgfsys@useobject{currentmarker}{}%
\end{pgfscope}%
\begin{pgfscope}%
\pgfsys@transformshift{4.086904in}{1.739820in}%
\pgfsys@useobject{currentmarker}{}%
\end{pgfscope}%
\begin{pgfscope}%
\pgfsys@transformshift{4.072819in}{1.758665in}%
\pgfsys@useobject{currentmarker}{}%
\end{pgfscope}%
\begin{pgfscope}%
\pgfsys@transformshift{4.053103in}{1.703179in}%
\pgfsys@useobject{currentmarker}{}%
\end{pgfscope}%
\begin{pgfscope}%
\pgfsys@transformshift{4.028221in}{1.529541in}%
\pgfsys@useobject{currentmarker}{}%
\end{pgfscope}%
\begin{pgfscope}%
\pgfsys@transformshift{4.010851in}{1.443001in}%
\pgfsys@useobject{currentmarker}{}%
\end{pgfscope}%
\begin{pgfscope}%
\pgfsys@transformshift{3.994185in}{1.383738in}%
\pgfsys@useobject{currentmarker}{}%
\end{pgfscope}%
\begin{pgfscope}%
\pgfsys@transformshift{3.974938in}{1.360160in}%
\pgfsys@useobject{currentmarker}{}%
\end{pgfscope}%
\begin{pgfscope}%
\pgfsys@transformshift{3.954516in}{1.412115in}%
\pgfsys@useobject{currentmarker}{}%
\end{pgfscope}%
\begin{pgfscope}%
\pgfsys@transformshift{3.940197in}{1.480652in}%
\pgfsys@useobject{currentmarker}{}%
\end{pgfscope}%
\begin{pgfscope}%
\pgfsys@transformshift{3.918132in}{1.632780in}%
\pgfsys@useobject{currentmarker}{}%
\end{pgfscope}%
\begin{pgfscope}%
\pgfsys@transformshift{3.896302in}{1.742689in}%
\pgfsys@useobject{currentmarker}{}%
\end{pgfscope}%
\begin{pgfscope}%
\pgfsys@transformshift{3.876820in}{1.745877in}%
\pgfsys@useobject{currentmarker}{}%
\end{pgfscope}%
\begin{pgfscope}%
\pgfsys@transformshift{3.856869in}{1.662203in}%
\pgfsys@useobject{currentmarker}{}%
\end{pgfscope}%
\begin{pgfscope}%
\pgfsys@transformshift{3.839264in}{1.529218in}%
\pgfsys@useobject{currentmarker}{}%
\end{pgfscope}%
\begin{pgfscope}%
\pgfsys@transformshift{3.821894in}{1.444456in}%
\pgfsys@useobject{currentmarker}{}%
\end{pgfscope}%
\begin{pgfscope}%
\pgfsys@transformshift{3.802881in}{1.376259in}%
\pgfsys@useobject{currentmarker}{}%
\end{pgfscope}%
\begin{pgfscope}%
\pgfsys@transformshift{3.783399in}{1.355788in}%
\pgfsys@useobject{currentmarker}{}%
\end{pgfscope}%
\begin{pgfscope}%
\pgfsys@transformshift{3.764620in}{1.394115in}%
\pgfsys@useobject{currentmarker}{}%
\end{pgfscope}%
\begin{pgfscope}%
\pgfsys@transformshift{3.742555in}{1.482070in}%
\pgfsys@useobject{currentmarker}{}%
\end{pgfscope}%
\begin{pgfscope}%
\pgfsys@transformshift{3.724950in}{1.604752in}%
\pgfsys@useobject{currentmarker}{}%
\end{pgfscope}%
\begin{pgfscope}%
\pgfsys@transformshift{3.704999in}{1.724197in}%
\pgfsys@useobject{currentmarker}{}%
\end{pgfscope}%
\begin{pgfscope}%
\pgfsys@transformshift{3.687158in}{1.747431in}%
\pgfsys@useobject{currentmarker}{}%
\end{pgfscope}%
\begin{pgfscope}%
\pgfsys@transformshift{3.663687in}{1.678982in}%
\pgfsys@useobject{currentmarker}{}%
\end{pgfscope}%
\begin{pgfscope}%
\pgfsys@transformshift{3.648663in}{1.577557in}%
\pgfsys@useobject{currentmarker}{}%
\end{pgfscope}%
\begin{pgfscope}%
\pgfsys@transformshift{3.627304in}{1.462477in}%
\pgfsys@useobject{currentmarker}{}%
\end{pgfscope}%
\begin{pgfscope}%
\pgfsys@transformshift{3.611342in}{1.400554in}%
\pgfsys@useobject{currentmarker}{}%
\end{pgfscope}%
\begin{pgfscope}%
\pgfsys@transformshift{3.590451in}{1.358387in}%
\pgfsys@useobject{currentmarker}{}%
\end{pgfscope}%
\begin{pgfscope}%
\pgfsys@transformshift{3.569324in}{1.366661in}%
\pgfsys@useobject{currentmarker}{}%
\end{pgfscope}%
\begin{pgfscope}%
\pgfsys@transformshift{3.551721in}{1.412987in}%
\pgfsys@useobject{currentmarker}{}%
\end{pgfscope}%
\begin{pgfscope}%
\pgfsys@transformshift{3.534351in}{1.484813in}%
\pgfsys@useobject{currentmarker}{}%
\end{pgfscope}%
\begin{pgfscope}%
\pgfsys@transformshift{3.513695in}{1.557321in}%
\pgfsys@useobject{currentmarker}{}%
\end{pgfscope}%
\begin{pgfscope}%
\pgfsys@transformshift{3.493039in}{1.685295in}%
\pgfsys@useobject{currentmarker}{}%
\end{pgfscope}%
\begin{pgfscope}%
\pgfsys@transformshift{3.475668in}{1.741686in}%
\pgfsys@useobject{currentmarker}{}%
\end{pgfscope}%
\begin{pgfscope}%
\pgfsys@transformshift{3.454072in}{1.736391in}%
\pgfsys@useobject{currentmarker}{}%
\end{pgfscope}%
\begin{pgfscope}%
\pgfsys@transformshift{3.436702in}{1.659780in}%
\pgfsys@useobject{currentmarker}{}%
\end{pgfscope}%
\begin{pgfscope}%
\pgfsys@transformshift{3.415812in}{1.531576in}%
\pgfsys@useobject{currentmarker}{}%
\end{pgfscope}%
\begin{pgfscope}%
\pgfsys@transformshift{3.397738in}{1.469757in}%
\pgfsys@useobject{currentmarker}{}%
\end{pgfscope}%
\begin{pgfscope}%
\pgfsys@transformshift{3.377082in}{1.412130in}%
\pgfsys@useobject{currentmarker}{}%
\end{pgfscope}%
\begin{pgfscope}%
\pgfsys@transformshift{3.360885in}{1.364517in}%
\pgfsys@useobject{currentmarker}{}%
\end{pgfscope}%
\begin{pgfscope}%
\pgfsys@transformshift{3.338586in}{1.356508in}%
\pgfsys@useobject{currentmarker}{}%
\end{pgfscope}%
\begin{pgfscope}%
\pgfsys@transformshift{3.321450in}{1.391077in}%
\pgfsys@useobject{currentmarker}{}%
\end{pgfscope}%
\begin{pgfscope}%
\pgfsys@transformshift{3.299857in}{1.460230in}%
\pgfsys@useobject{currentmarker}{}%
\end{pgfscope}%
\begin{pgfscope}%
\pgfsys@transformshift{3.279200in}{1.587335in}%
\pgfsys@useobject{currentmarker}{}%
\end{pgfscope}%
\begin{pgfscope}%
\pgfsys@transformshift{3.264647in}{1.415771in}%
\pgfsys@useobject{currentmarker}{}%
\end{pgfscope}%
\begin{pgfscope}%
\pgfsys@transformshift{3.243520in}{1.360698in}%
\pgfsys@useobject{currentmarker}{}%
\end{pgfscope}%
\begin{pgfscope}%
\pgfsys@transformshift{3.223098in}{1.362873in}%
\pgfsys@useobject{currentmarker}{}%
\end{pgfscope}%
\begin{pgfscope}%
\pgfsys@transformshift{3.208310in}{1.401291in}%
\pgfsys@useobject{currentmarker}{}%
\end{pgfscope}%
\begin{pgfscope}%
\pgfsys@transformshift{3.188125in}{1.482358in}%
\pgfsys@useobject{currentmarker}{}%
\end{pgfscope}%
\begin{pgfscope}%
\pgfsys@transformshift{3.166295in}{1.623670in}%
\pgfsys@useobject{currentmarker}{}%
\end{pgfscope}%
\begin{pgfscope}%
\pgfsys@transformshift{3.148690in}{1.695979in}%
\pgfsys@useobject{currentmarker}{}%
\end{pgfscope}%
\begin{pgfscope}%
\pgfsys@transformshift{3.128737in}{1.740084in}%
\pgfsys@useobject{currentmarker}{}%
\end{pgfscope}%
\begin{pgfscope}%
\pgfsys@transformshift{3.110664in}{1.702889in}%
\pgfsys@useobject{currentmarker}{}%
\end{pgfscope}%
\begin{pgfscope}%
\pgfsys@transformshift{3.091650in}{1.593737in}%
\pgfsys@useobject{currentmarker}{}%
\end{pgfscope}%
\begin{pgfscope}%
\pgfsys@transformshift{3.070525in}{1.466026in}%
\pgfsys@useobject{currentmarker}{}%
\end{pgfscope}%
\begin{pgfscope}%
\pgfsys@transformshift{3.052921in}{1.398081in}%
\pgfsys@useobject{currentmarker}{}%
\end{pgfscope}%
\begin{pgfscope}%
\pgfsys@transformshift{3.033204in}{1.354119in}%
\pgfsys@useobject{currentmarker}{}%
\end{pgfscope}%
\begin{pgfscope}%
\pgfsys@transformshift{3.012077in}{1.365114in}%
\pgfsys@useobject{currentmarker}{}%
\end{pgfscope}%
\begin{pgfscope}%
\pgfsys@transformshift{2.994472in}{1.413361in}%
\pgfsys@useobject{currentmarker}{}%
\end{pgfscope}%
\begin{pgfscope}%
\pgfsys@transformshift{2.974990in}{1.465151in}%
\pgfsys@useobject{currentmarker}{}%
\end{pgfscope}%
\begin{pgfscope}%
\pgfsys@transformshift{2.958091in}{1.590368in}%
\pgfsys@useobject{currentmarker}{}%
\end{pgfscope}%
\begin{pgfscope}%
\pgfsys@transformshift{2.935321in}{1.711023in}%
\pgfsys@useobject{currentmarker}{}%
\end{pgfscope}%
\begin{pgfscope}%
\pgfsys@transformshift{2.914430in}{1.738270in}%
\pgfsys@useobject{currentmarker}{}%
\end{pgfscope}%
\begin{pgfscope}%
\pgfsys@transformshift{2.899174in}{1.708657in}%
\pgfsys@useobject{currentmarker}{}%
\end{pgfscope}%
\begin{pgfscope}%
\pgfsys@transformshift{2.879221in}{1.587112in}%
\pgfsys@useobject{currentmarker}{}%
\end{pgfscope}%
\begin{pgfscope}%
\pgfsys@transformshift{2.854808in}{1.449527in}%
\pgfsys@useobject{currentmarker}{}%
\end{pgfscope}%
\begin{pgfscope}%
\pgfsys@transformshift{2.841665in}{1.404104in}%
\pgfsys@useobject{currentmarker}{}%
\end{pgfscope}%
\begin{pgfscope}%
\pgfsys@transformshift{2.820304in}{1.359035in}%
\pgfsys@useobject{currentmarker}{}%
\end{pgfscope}%
\begin{pgfscope}%
\pgfsys@transformshift{2.805047in}{1.353469in}%
\pgfsys@useobject{currentmarker}{}%
\end{pgfscope}%
\begin{pgfscope}%
\pgfsys@transformshift{2.783217in}{1.393249in}%
\pgfsys@useobject{currentmarker}{}%
\end{pgfscope}%
\begin{pgfscope}%
\pgfsys@transformshift{2.763029in}{1.459775in}%
\pgfsys@useobject{currentmarker}{}%
\end{pgfscope}%
\begin{pgfscope}%
\pgfsys@transformshift{2.744956in}{1.576472in}%
\pgfsys@useobject{currentmarker}{}%
\end{pgfscope}%
\begin{pgfscope}%
\pgfsys@transformshift{2.722422in}{1.709015in}%
\pgfsys@useobject{currentmarker}{}%
\end{pgfscope}%
\begin{pgfscope}%
\pgfsys@transformshift{2.705990in}{1.739290in}%
\pgfsys@useobject{currentmarker}{}%
\end{pgfscope}%
\begin{pgfscope}%
\pgfsys@transformshift{2.688385in}{1.712172in}%
\pgfsys@useobject{currentmarker}{}%
\end{pgfscope}%
\begin{pgfscope}%
\pgfsys@transformshift{2.667025in}{1.671148in}%
\pgfsys@useobject{currentmarker}{}%
\end{pgfscope}%
\begin{pgfscope}%
\pgfsys@transformshift{2.648012in}{1.540814in}%
\pgfsys@useobject{currentmarker}{}%
\end{pgfscope}%
\begin{pgfscope}%
\pgfsys@transformshift{2.627825in}{1.440239in}%
\pgfsys@useobject{currentmarker}{}%
\end{pgfscope}%
\begin{pgfscope}%
\pgfsys@transformshift{2.610456in}{1.385801in}%
\pgfsys@useobject{currentmarker}{}%
\end{pgfscope}%
\begin{pgfscope}%
\pgfsys@transformshift{2.590972in}{1.352356in}%
\pgfsys@useobject{currentmarker}{}%
\end{pgfscope}%
\begin{pgfscope}%
\pgfsys@transformshift{2.571490in}{1.359606in}%
\pgfsys@useobject{currentmarker}{}%
\end{pgfscope}%
\begin{pgfscope}%
\pgfsys@transformshift{2.551069in}{1.410473in}%
\pgfsys@useobject{currentmarker}{}%
\end{pgfscope}%
\begin{pgfscope}%
\pgfsys@transformshift{2.534169in}{1.468597in}%
\pgfsys@useobject{currentmarker}{}%
\end{pgfscope}%
\begin{pgfscope}%
\pgfsys@transformshift{2.515859in}{1.584293in}%
\pgfsys@useobject{currentmarker}{}%
\end{pgfscope}%
\begin{pgfscope}%
\pgfsys@transformshift{2.494031in}{1.710558in}%
\pgfsys@useobject{currentmarker}{}%
\end{pgfscope}%
\begin{pgfscope}%
\pgfsys@transformshift{2.475252in}{1.734290in}%
\pgfsys@useobject{currentmarker}{}%
\end{pgfscope}%
\begin{pgfscope}%
\pgfsys@transformshift{2.458587in}{1.729320in}%
\pgfsys@useobject{currentmarker}{}%
\end{pgfscope}%
\begin{pgfscope}%
\pgfsys@transformshift{2.438399in}{1.655115in}%
\pgfsys@useobject{currentmarker}{}%
\end{pgfscope}%
\begin{pgfscope}%
\pgfsys@transformshift{2.438634in}{1.564835in}%
\pgfsys@useobject{currentmarker}{}%
\end{pgfscope}%
\begin{pgfscope}%
\pgfsys@transformshift{2.418212in}{1.517491in}%
\pgfsys@useobject{currentmarker}{}%
\end{pgfscope}%
\begin{pgfscope}%
\pgfsys@transformshift{2.400842in}{1.444012in}%
\pgfsys@useobject{currentmarker}{}%
\end{pgfscope}%
\begin{pgfscope}%
\pgfsys@transformshift{2.382299in}{1.388980in}%
\pgfsys@useobject{currentmarker}{}%
\end{pgfscope}%
\begin{pgfscope}%
\pgfsys@transformshift{2.360000in}{1.352827in}%
\pgfsys@useobject{currentmarker}{}%
\end{pgfscope}%
\begin{pgfscope}%
\pgfsys@transformshift{2.341456in}{1.348765in}%
\pgfsys@useobject{currentmarker}{}%
\end{pgfscope}%
\begin{pgfscope}%
\pgfsys@transformshift{2.319860in}{1.372888in}%
\pgfsys@useobject{currentmarker}{}%
\end{pgfscope}%
\begin{pgfscope}%
\pgfsys@transformshift{2.302255in}{1.425164in}%
\pgfsys@useobject{currentmarker}{}%
\end{pgfscope}%
\begin{pgfscope}%
\pgfsys@transformshift{2.282773in}{1.520112in}%
\pgfsys@useobject{currentmarker}{}%
\end{pgfscope}%
\begin{pgfscope}%
\pgfsys@transformshift{2.263760in}{1.616232in}%
\pgfsys@useobject{currentmarker}{}%
\end{pgfscope}%
\begin{pgfscope}%
\pgfsys@transformshift{2.244278in}{1.724288in}%
\pgfsys@useobject{currentmarker}{}%
\end{pgfscope}%
\begin{pgfscope}%
\pgfsys@transformshift{2.226204in}{1.739110in}%
\pgfsys@useobject{currentmarker}{}%
\end{pgfscope}%
\begin{pgfscope}%
\pgfsys@transformshift{2.207894in}{1.702038in}%
\pgfsys@useobject{currentmarker}{}%
\end{pgfscope}%
\begin{pgfscope}%
\pgfsys@transformshift{2.186769in}{1.585001in}%
\pgfsys@useobject{currentmarker}{}%
\end{pgfscope}%
\begin{pgfscope}%
\pgfsys@transformshift{2.168225in}{1.475269in}%
\pgfsys@useobject{currentmarker}{}%
\end{pgfscope}%
\begin{pgfscope}%
\pgfsys@transformshift{2.149448in}{1.413008in}%
\pgfsys@useobject{currentmarker}{}%
\end{pgfscope}%
\begin{pgfscope}%
\pgfsys@transformshift{2.130435in}{1.369441in}%
\pgfsys@useobject{currentmarker}{}%
\end{pgfscope}%
\begin{pgfscope}%
\pgfsys@transformshift{2.112359in}{1.352790in}%
\pgfsys@useobject{currentmarker}{}%
\end{pgfscope}%
\begin{pgfscope}%
\pgfsys@transformshift{2.090765in}{1.372622in}%
\pgfsys@useobject{currentmarker}{}%
\end{pgfscope}%
\begin{pgfscope}%
\pgfsys@transformshift{2.071517in}{1.405418in}%
\pgfsys@useobject{currentmarker}{}%
\end{pgfscope}%
\begin{pgfscope}%
\pgfsys@transformshift{2.052973in}{1.457428in}%
\pgfsys@useobject{currentmarker}{}%
\end{pgfscope}%
\begin{pgfscope}%
\pgfsys@transformshift{2.034899in}{1.567520in}%
\pgfsys@useobject{currentmarker}{}%
\end{pgfscope}%
\begin{pgfscope}%
\pgfsys@transformshift{2.013774in}{1.693772in}%
\pgfsys@useobject{currentmarker}{}%
\end{pgfscope}%
\begin{pgfscope}%
\pgfsys@transformshift{1.994290in}{1.740984in}%
\pgfsys@useobject{currentmarker}{}%
\end{pgfscope}%
\begin{pgfscope}%
\pgfsys@transformshift{1.976217in}{1.722214in}%
\pgfsys@useobject{currentmarker}{}%
\end{pgfscope}%
\begin{pgfscope}%
\pgfsys@transformshift{1.958143in}{1.660446in}%
\pgfsys@useobject{currentmarker}{}%
\end{pgfscope}%
\begin{pgfscope}%
\pgfsys@transformshift{1.938190in}{1.560053in}%
\pgfsys@useobject{currentmarker}{}%
\end{pgfscope}%
\begin{pgfscope}%
\pgfsys@transformshift{1.920351in}{1.482010in}%
\pgfsys@useobject{currentmarker}{}%
\end{pgfscope}%
\begin{pgfscope}%
\pgfsys@transformshift{1.898757in}{1.404018in}%
\pgfsys@useobject{currentmarker}{}%
\end{pgfscope}%
\begin{pgfscope}%
\pgfsys@transformshift{1.879978in}{1.373323in}%
\pgfsys@useobject{currentmarker}{}%
\end{pgfscope}%
\begin{pgfscope}%
\pgfsys@transformshift{1.862373in}{1.359123in}%
\pgfsys@useobject{currentmarker}{}%
\end{pgfscope}%
\begin{pgfscope}%
\pgfsys@transformshift{1.842891in}{1.354552in}%
\pgfsys@useobject{currentmarker}{}%
\end{pgfscope}%
\begin{pgfscope}%
\pgfsys@transformshift{1.821061in}{1.388208in}%
\pgfsys@useobject{currentmarker}{}%
\end{pgfscope}%
\begin{pgfscope}%
\pgfsys@transformshift{1.804865in}{1.437818in}%
\pgfsys@useobject{currentmarker}{}%
\end{pgfscope}%
\begin{pgfscope}%
\pgfsys@transformshift{1.785852in}{1.526822in}%
\pgfsys@useobject{currentmarker}{}%
\end{pgfscope}%
\begin{pgfscope}%
\pgfsys@transformshift{1.765430in}{1.658847in}%
\pgfsys@useobject{currentmarker}{}%
\end{pgfscope}%
\begin{pgfscope}%
\pgfsys@transformshift{1.746885in}{1.733075in}%
\pgfsys@useobject{currentmarker}{}%
\end{pgfscope}%
\begin{pgfscope}%
\pgfsys@transformshift{1.724821in}{1.744291in}%
\pgfsys@useobject{currentmarker}{}%
\end{pgfscope}%
\begin{pgfscope}%
\pgfsys@transformshift{1.710738in}{1.705332in}%
\pgfsys@useobject{currentmarker}{}%
\end{pgfscope}%
\begin{pgfscope}%
\pgfsys@transformshift{1.687265in}{1.637804in}%
\pgfsys@useobject{currentmarker}{}%
\end{pgfscope}%
\begin{pgfscope}%
\pgfsys@transformshift{1.669191in}{1.540818in}%
\pgfsys@useobject{currentmarker}{}%
\end{pgfscope}%
\begin{pgfscope}%
\pgfsys@transformshift{1.647361in}{1.457986in}%
\pgfsys@useobject{currentmarker}{}%
\end{pgfscope}%
\begin{pgfscope}%
\pgfsys@transformshift{1.631634in}{1.412962in}%
\pgfsys@useobject{currentmarker}{}%
\end{pgfscope}%
\begin{pgfscope}%
\pgfsys@transformshift{1.607223in}{1.367201in}%
\pgfsys@useobject{currentmarker}{}%
\end{pgfscope}%
\begin{pgfscope}%
\pgfsys@transformshift{1.591027in}{1.355799in}%
\pgfsys@useobject{currentmarker}{}%
\end{pgfscope}%
\begin{pgfscope}%
\pgfsys@transformshift{1.574594in}{1.370821in}%
\pgfsys@useobject{currentmarker}{}%
\end{pgfscope}%
\begin{pgfscope}%
\pgfsys@transformshift{1.554643in}{1.407544in}%
\pgfsys@useobject{currentmarker}{}%
\end{pgfscope}%
\begin{pgfscope}%
\pgfsys@transformshift{1.535395in}{1.482944in}%
\pgfsys@useobject{currentmarker}{}%
\end{pgfscope}%
\begin{pgfscope}%
\pgfsys@transformshift{1.511922in}{1.555303in}%
\pgfsys@useobject{currentmarker}{}%
\end{pgfscope}%
\begin{pgfscope}%
\pgfsys@transformshift{1.495726in}{1.646930in}%
\pgfsys@useobject{currentmarker}{}%
\end{pgfscope}%
\begin{pgfscope}%
\pgfsys@transformshift{1.477416in}{1.728477in}%
\pgfsys@useobject{currentmarker}{}%
\end{pgfscope}%
\begin{pgfscope}%
\pgfsys@transformshift{1.454883in}{1.751220in}%
\pgfsys@useobject{currentmarker}{}%
\end{pgfscope}%
\begin{pgfscope}%
\pgfsys@transformshift{1.437043in}{1.739787in}%
\pgfsys@useobject{currentmarker}{}%
\end{pgfscope}%
\begin{pgfscope}%
\pgfsys@transformshift{1.421787in}{1.684414in}%
\pgfsys@useobject{currentmarker}{}%
\end{pgfscope}%
\begin{pgfscope}%
\pgfsys@transformshift{1.400660in}{1.617272in}%
\pgfsys@useobject{currentmarker}{}%
\end{pgfscope}%
\begin{pgfscope}%
\pgfsys@transformshift{1.380943in}{1.679689in}%
\pgfsys@useobject{currentmarker}{}%
\end{pgfscope}%
\begin{pgfscope}%
\pgfsys@transformshift{1.362870in}{1.573367in}%
\pgfsys@useobject{currentmarker}{}%
\end{pgfscope}%
\begin{pgfscope}%
\pgfsys@transformshift{1.342213in}{1.534149in}%
\pgfsys@useobject{currentmarker}{}%
\end{pgfscope}%
\begin{pgfscope}%
\pgfsys@transformshift{1.319914in}{1.457997in}%
\pgfsys@useobject{currentmarker}{}%
\end{pgfscope}%
\begin{pgfscope}%
\pgfsys@transformshift{1.301370in}{1.400417in}%
\pgfsys@useobject{currentmarker}{}%
\end{pgfscope}%
\begin{pgfscope}%
\pgfsys@transformshift{1.286582in}{1.369934in}%
\pgfsys@useobject{currentmarker}{}%
\end{pgfscope}%
\begin{pgfscope}%
\pgfsys@transformshift{1.264517in}{1.360674in}%
\pgfsys@useobject{currentmarker}{}%
\end{pgfscope}%
\begin{pgfscope}%
\pgfsys@transformshift{1.245739in}{1.379898in}%
\pgfsys@useobject{currentmarker}{}%
\end{pgfscope}%
\begin{pgfscope}%
\pgfsys@transformshift{1.226960in}{1.417413in}%
\pgfsys@useobject{currentmarker}{}%
\end{pgfscope}%
\begin{pgfscope}%
\pgfsys@transformshift{1.208417in}{1.467898in}%
\pgfsys@useobject{currentmarker}{}%
\end{pgfscope}%
\begin{pgfscope}%
\pgfsys@transformshift{1.189873in}{1.566976in}%
\pgfsys@useobject{currentmarker}{}%
\end{pgfscope}%
\begin{pgfscope}%
\pgfsys@transformshift{1.167105in}{1.699796in}%
\pgfsys@useobject{currentmarker}{}%
\end{pgfscope}%
\begin{pgfscope}%
\pgfsys@transformshift{1.152317in}{1.744495in}%
\pgfsys@useobject{currentmarker}{}%
\end{pgfscope}%
\begin{pgfscope}%
\pgfsys@transformshift{1.129784in}{1.758132in}%
\pgfsys@useobject{currentmarker}{}%
\end{pgfscope}%
\begin{pgfscope}%
\pgfsys@transformshift{1.112413in}{1.747633in}%
\pgfsys@useobject{currentmarker}{}%
\end{pgfscope}%
\begin{pgfscope}%
\pgfsys@transformshift{1.090583in}{1.670135in}%
\pgfsys@useobject{currentmarker}{}%
\end{pgfscope}%
\begin{pgfscope}%
\pgfsys@transformshift{1.072039in}{1.558608in}%
\pgfsys@useobject{currentmarker}{}%
\end{pgfscope}%
\begin{pgfscope}%
\pgfsys@transformshift{1.053965in}{1.488141in}%
\pgfsys@useobject{currentmarker}{}%
\end{pgfscope}%
\begin{pgfscope}%
\pgfsys@transformshift{1.034952in}{1.430094in}%
\pgfsys@useobject{currentmarker}{}%
\end{pgfscope}%
\begin{pgfscope}%
\pgfsys@transformshift{1.016878in}{1.385931in}%
\pgfsys@useobject{currentmarker}{}%
\end{pgfscope}%
\begin{pgfscope}%
\pgfsys@transformshift{0.995282in}{1.364432in}%
\pgfsys@useobject{currentmarker}{}%
\end{pgfscope}%
\begin{pgfscope}%
\pgfsys@transformshift{0.978617in}{1.384664in}%
\pgfsys@useobject{currentmarker}{}%
\end{pgfscope}%
\begin{pgfscope}%
\pgfsys@transformshift{0.957727in}{1.435580in}%
\pgfsys@useobject{currentmarker}{}%
\end{pgfscope}%
\begin{pgfscope}%
\pgfsys@transformshift{0.939417in}{1.500761in}%
\pgfsys@useobject{currentmarker}{}%
\end{pgfscope}%
\begin{pgfscope}%
\pgfsys@transformshift{0.918057in}{1.595976in}%
\pgfsys@useobject{currentmarker}{}%
\end{pgfscope}%
\begin{pgfscope}%
\pgfsys@transformshift{0.899513in}{1.697235in}%
\pgfsys@useobject{currentmarker}{}%
\end{pgfscope}%
\begin{pgfscope}%
\pgfsys@transformshift{0.880734in}{1.754000in}%
\pgfsys@useobject{currentmarker}{}%
\end{pgfscope}%
\begin{pgfscope}%
\pgfsys@transformshift{0.863131in}{1.766067in}%
\pgfsys@useobject{currentmarker}{}%
\end{pgfscope}%
\begin{pgfscope}%
\pgfsys@transformshift{0.844118in}{1.749618in}%
\pgfsys@useobject{currentmarker}{}%
\end{pgfscope}%
\begin{pgfscope}%
\pgfsys@transformshift{0.822522in}{1.422965in}%
\pgfsys@useobject{currentmarker}{}%
\end{pgfscope}%
\begin{pgfscope}%
\pgfsys@transformshift{0.803978in}{1.504203in}%
\pgfsys@useobject{currentmarker}{}%
\end{pgfscope}%
\begin{pgfscope}%
\pgfsys@transformshift{0.785201in}{1.624983in}%
\pgfsys@useobject{currentmarker}{}%
\end{pgfscope}%
\begin{pgfscope}%
\pgfsys@transformshift{0.763839in}{1.734853in}%
\pgfsys@useobject{currentmarker}{}%
\end{pgfscope}%
\begin{pgfscope}%
\pgfsys@transformshift{0.748112in}{1.765599in}%
\pgfsys@useobject{currentmarker}{}%
\end{pgfscope}%
\begin{pgfscope}%
\pgfsys@transformshift{0.726284in}{1.749994in}%
\pgfsys@useobject{currentmarker}{}%
\end{pgfscope}%
\begin{pgfscope}%
\pgfsys@transformshift{0.708208in}{1.687150in}%
\pgfsys@useobject{currentmarker}{}%
\end{pgfscope}%
\begin{pgfscope}%
\pgfsys@transformshift{0.689431in}{1.578530in}%
\pgfsys@useobject{currentmarker}{}%
\end{pgfscope}%
\begin{pgfscope}%
\pgfsys@transformshift{0.667367in}{1.469104in}%
\pgfsys@useobject{currentmarker}{}%
\end{pgfscope}%
\begin{pgfscope}%
\pgfsys@transformshift{0.649525in}{1.410910in}%
\pgfsys@useobject{currentmarker}{}%
\end{pgfscope}%
\begin{pgfscope}%
\pgfsys@transformshift{0.650934in}{1.413632in}%
\pgfsys@useobject{currentmarker}{}%
\end{pgfscope}%
\begin{pgfscope}%
\pgfsys@transformshift{0.657976in}{1.449249in}%
\pgfsys@useobject{currentmarker}{}%
\end{pgfscope}%
\begin{pgfscope}%
\pgfsys@transformshift{0.677224in}{1.544734in}%
\pgfsys@useobject{currentmarker}{}%
\end{pgfscope}%
\begin{pgfscope}%
\pgfsys@transformshift{0.696237in}{1.692589in}%
\pgfsys@useobject{currentmarker}{}%
\end{pgfscope}%
\begin{pgfscope}%
\pgfsys@transformshift{0.713842in}{1.761706in}%
\pgfsys@useobject{currentmarker}{}%
\end{pgfscope}%
\begin{pgfscope}%
\pgfsys@transformshift{0.731916in}{1.751946in}%
\pgfsys@useobject{currentmarker}{}%
\end{pgfscope}%
\begin{pgfscope}%
\pgfsys@transformshift{0.751634in}{1.646499in}%
\pgfsys@useobject{currentmarker}{}%
\end{pgfscope}%
\begin{pgfscope}%
\pgfsys@transformshift{0.772759in}{1.489598in}%
\pgfsys@useobject{currentmarker}{}%
\end{pgfscope}%
\begin{pgfscope}%
\pgfsys@transformshift{0.790833in}{1.403391in}%
\pgfsys@useobject{currentmarker}{}%
\end{pgfscope}%
\begin{pgfscope}%
\pgfsys@transformshift{0.808203in}{1.427716in}%
\pgfsys@useobject{currentmarker}{}%
\end{pgfscope}%
\begin{pgfscope}%
\pgfsys@transformshift{0.828156in}{1.542286in}%
\pgfsys@useobject{currentmarker}{}%
\end{pgfscope}%
\begin{pgfscope}%
\pgfsys@transformshift{0.848107in}{1.682724in}%
\pgfsys@useobject{currentmarker}{}%
\end{pgfscope}%
\begin{pgfscope}%
\pgfsys@transformshift{0.869234in}{1.758529in}%
\pgfsys@useobject{currentmarker}{}%
\end{pgfscope}%
\begin{pgfscope}%
\pgfsys@transformshift{0.887307in}{1.734275in}%
\pgfsys@useobject{currentmarker}{}%
\end{pgfscope}%
\begin{pgfscope}%
\pgfsys@transformshift{0.906086in}{1.614398in}%
\pgfsys@useobject{currentmarker}{}%
\end{pgfscope}%
\begin{pgfscope}%
\pgfsys@transformshift{0.926037in}{1.452906in}%
\pgfsys@useobject{currentmarker}{}%
\end{pgfscope}%
\begin{pgfscope}%
\pgfsys@transformshift{0.944816in}{1.381172in}%
\pgfsys@useobject{currentmarker}{}%
\end{pgfscope}%
\begin{pgfscope}%
\pgfsys@transformshift{0.963124in}{1.361322in}%
\pgfsys@useobject{currentmarker}{}%
\end{pgfscope}%
\begin{pgfscope}%
\pgfsys@transformshift{0.984486in}{1.417791in}%
\pgfsys@useobject{currentmarker}{}%
\end{pgfscope}%
\begin{pgfscope}%
\pgfsys@transformshift{1.003968in}{1.508112in}%
\pgfsys@useobject{currentmarker}{}%
\end{pgfscope}%
\begin{pgfscope}%
\pgfsys@transformshift{1.023215in}{1.649807in}%
\pgfsys@useobject{currentmarker}{}%
\end{pgfscope}%
\begin{pgfscope}%
\pgfsys@transformshift{1.038708in}{1.734705in}%
\pgfsys@useobject{currentmarker}{}%
\end{pgfscope}%
\begin{pgfscope}%
\pgfsys@transformshift{1.060537in}{1.744735in}%
\pgfsys@useobject{currentmarker}{}%
\end{pgfscope}%
\begin{pgfscope}%
\pgfsys@transformshift{1.080021in}{1.658762in}%
\pgfsys@useobject{currentmarker}{}%
\end{pgfscope}%
\begin{pgfscope}%
\pgfsys@transformshift{1.097860in}{1.519810in}%
\pgfsys@useobject{currentmarker}{}%
\end{pgfscope}%
\begin{pgfscope}%
\pgfsys@transformshift{1.118282in}{1.412436in}%
\pgfsys@useobject{currentmarker}{}%
\end{pgfscope}%
\begin{pgfscope}%
\pgfsys@transformshift{1.137059in}{1.363145in}%
\pgfsys@useobject{currentmarker}{}%
\end{pgfscope}%
\begin{pgfscope}%
\pgfsys@transformshift{1.156777in}{1.365690in}%
\pgfsys@useobject{currentmarker}{}%
\end{pgfscope}%
\begin{pgfscope}%
\pgfsys@transformshift{1.175790in}{1.414704in}%
\pgfsys@useobject{currentmarker}{}%
\end{pgfscope}%
\begin{pgfscope}%
\pgfsys@transformshift{1.196212in}{1.488247in}%
\pgfsys@useobject{currentmarker}{}%
\end{pgfscope}%
\begin{pgfscope}%
\pgfsys@transformshift{1.214520in}{1.611356in}%
\pgfsys@useobject{currentmarker}{}%
\end{pgfscope}%
\begin{pgfscope}%
\pgfsys@transformshift{1.230716in}{1.711160in}%
\pgfsys@useobject{currentmarker}{}%
\end{pgfscope}%
\begin{pgfscope}%
\pgfsys@transformshift{1.252312in}{1.748761in}%
\pgfsys@useobject{currentmarker}{}%
\end{pgfscope}%
\begin{pgfscope}%
\pgfsys@transformshift{1.271560in}{1.728353in}%
\pgfsys@useobject{currentmarker}{}%
\end{pgfscope}%
\begin{pgfscope}%
\pgfsys@transformshift{1.291042in}{1.626783in}%
\pgfsys@useobject{currentmarker}{}%
\end{pgfscope}%
\begin{pgfscope}%
\pgfsys@transformshift{1.310055in}{1.485872in}%
\pgfsys@useobject{currentmarker}{}%
\end{pgfscope}%
\begin{pgfscope}%
\pgfsys@transformshift{1.328834in}{1.408967in}%
\pgfsys@useobject{currentmarker}{}%
\end{pgfscope}%
\begin{pgfscope}%
\pgfsys@transformshift{1.347847in}{1.360893in}%
\pgfsys@useobject{currentmarker}{}%
\end{pgfscope}%
\begin{pgfscope}%
\pgfsys@transformshift{1.366390in}{1.354998in}%
\pgfsys@useobject{currentmarker}{}%
\end{pgfscope}%
\begin{pgfscope}%
\pgfsys@transformshift{1.384934in}{1.392245in}%
\pgfsys@useobject{currentmarker}{}%
\end{pgfscope}%
\begin{pgfscope}%
\pgfsys@transformshift{1.405356in}{1.455286in}%
\pgfsys@useobject{currentmarker}{}%
\end{pgfscope}%
\begin{pgfscope}%
\pgfsys@transformshift{1.427420in}{1.563329in}%
\pgfsys@useobject{currentmarker}{}%
\end{pgfscope}%
\begin{pgfscope}%
\pgfsys@transformshift{1.445963in}{1.690121in}%
\pgfsys@useobject{currentmarker}{}%
\end{pgfscope}%
\begin{pgfscope}%
\pgfsys@transformshift{1.462864in}{1.737562in}%
\pgfsys@useobject{currentmarker}{}%
\end{pgfscope}%
\begin{pgfscope}%
\pgfsys@transformshift{1.481407in}{1.736203in}%
\pgfsys@useobject{currentmarker}{}%
\end{pgfscope}%
\begin{pgfscope}%
\pgfsys@transformshift{1.502063in}{1.654453in}%
\pgfsys@useobject{currentmarker}{}%
\end{pgfscope}%
\begin{pgfscope}%
\pgfsys@transformshift{1.520608in}{1.541195in}%
\pgfsys@useobject{currentmarker}{}%
\end{pgfscope}%
\begin{pgfscope}%
\pgfsys@transformshift{1.541264in}{1.434397in}%
\pgfsys@useobject{currentmarker}{}%
\end{pgfscope}%
\begin{pgfscope}%
\pgfsys@transformshift{1.560043in}{1.375674in}%
\pgfsys@useobject{currentmarker}{}%
\end{pgfscope}%
\begin{pgfscope}%
\pgfsys@transformshift{1.579290in}{1.356013in}%
\pgfsys@useobject{currentmarker}{}%
\end{pgfscope}%
\begin{pgfscope}%
\pgfsys@transformshift{1.598067in}{1.359593in}%
\pgfsys@useobject{currentmarker}{}%
\end{pgfscope}%
\begin{pgfscope}%
\pgfsys@transformshift{1.617551in}{1.396772in}%
\pgfsys@useobject{currentmarker}{}%
\end{pgfscope}%
\begin{pgfscope}%
\pgfsys@transformshift{1.635625in}{1.460664in}%
\pgfsys@useobject{currentmarker}{}%
\end{pgfscope}%
\begin{pgfscope}%
\pgfsys@transformshift{1.659332in}{1.555167in}%
\pgfsys@useobject{currentmarker}{}%
\end{pgfscope}%
\begin{pgfscope}%
\pgfsys@transformshift{1.675060in}{1.594660in}%
\pgfsys@useobject{currentmarker}{}%
\end{pgfscope}%
\begin{pgfscope}%
\pgfsys@transformshift{1.696654in}{1.448539in}%
\pgfsys@useobject{currentmarker}{}%
\end{pgfscope}%
\begin{pgfscope}%
\pgfsys@transformshift{1.712381in}{1.391180in}%
\pgfsys@useobject{currentmarker}{}%
\end{pgfscope}%
\begin{pgfscope}%
\pgfsys@transformshift{1.733037in}{1.351840in}%
\pgfsys@useobject{currentmarker}{}%
\end{pgfscope}%
\begin{pgfscope}%
\pgfsys@transformshift{1.751582in}{1.370881in}%
\pgfsys@useobject{currentmarker}{}%
\end{pgfscope}%
\begin{pgfscope}%
\pgfsys@transformshift{1.770359in}{1.429280in}%
\pgfsys@useobject{currentmarker}{}%
\end{pgfscope}%
\begin{pgfscope}%
\pgfsys@transformshift{1.789843in}{1.509822in}%
\pgfsys@useobject{currentmarker}{}%
\end{pgfscope}%
\begin{pgfscope}%
\pgfsys@transformshift{1.808151in}{1.635976in}%
\pgfsys@useobject{currentmarker}{}%
\end{pgfscope}%
\begin{pgfscope}%
\pgfsys@transformshift{1.831155in}{1.732433in}%
\pgfsys@useobject{currentmarker}{}%
\end{pgfscope}%
\begin{pgfscope}%
\pgfsys@transformshift{1.848994in}{1.733223in}%
\pgfsys@useobject{currentmarker}{}%
\end{pgfscope}%
\begin{pgfscope}%
\pgfsys@transformshift{1.868476in}{1.650643in}%
\pgfsys@useobject{currentmarker}{}%
\end{pgfscope}%
\begin{pgfscope}%
\pgfsys@transformshift{1.887021in}{1.528695in}%
\pgfsys@useobject{currentmarker}{}%
\end{pgfscope}%
\begin{pgfscope}%
\pgfsys@transformshift{1.908614in}{1.414092in}%
\pgfsys@useobject{currentmarker}{}%
\end{pgfscope}%
\begin{pgfscope}%
\pgfsys@transformshift{1.925280in}{1.367156in}%
\pgfsys@useobject{currentmarker}{}%
\end{pgfscope}%
\begin{pgfscope}%
\pgfsys@transformshift{1.944293in}{1.349621in}%
\pgfsys@useobject{currentmarker}{}%
\end{pgfscope}%
\begin{pgfscope}%
\pgfsys@transformshift{1.962368in}{1.360818in}%
\pgfsys@useobject{currentmarker}{}%
\end{pgfscope}%
\begin{pgfscope}%
\pgfsys@transformshift{1.981145in}{1.407863in}%
\pgfsys@useobject{currentmarker}{}%
\end{pgfscope}%
\begin{pgfscope}%
\pgfsys@transformshift{2.003681in}{1.505324in}%
\pgfsys@useobject{currentmarker}{}%
\end{pgfscope}%
\begin{pgfscope}%
\pgfsys@transformshift{2.020346in}{1.625137in}%
\pgfsys@useobject{currentmarker}{}%
\end{pgfscope}%
\begin{pgfscope}%
\pgfsys@transformshift{2.039359in}{1.705414in}%
\pgfsys@useobject{currentmarker}{}%
\end{pgfscope}%
\begin{pgfscope}%
\pgfsys@transformshift{2.061658in}{1.738319in}%
\pgfsys@useobject{currentmarker}{}%
\end{pgfscope}%
\begin{pgfscope}%
\pgfsys@transformshift{2.077620in}{1.700418in}%
\pgfsys@useobject{currentmarker}{}%
\end{pgfscope}%
\begin{pgfscope}%
\pgfsys@transformshift{2.098745in}{1.561225in}%
\pgfsys@useobject{currentmarker}{}%
\end{pgfscope}%
\begin{pgfscope}%
\pgfsys@transformshift{2.113533in}{1.453307in}%
\pgfsys@useobject{currentmarker}{}%
\end{pgfscope}%
\begin{pgfscope}%
\pgfsys@transformshift{2.136303in}{1.388356in}%
\pgfsys@useobject{currentmarker}{}%
\end{pgfscope}%
\begin{pgfscope}%
\pgfsys@transformshift{2.157662in}{1.350798in}%
\pgfsys@useobject{currentmarker}{}%
\end{pgfscope}%
\begin{pgfscope}%
\pgfsys@transformshift{2.175736in}{1.362770in}%
\pgfsys@useobject{currentmarker}{}%
\end{pgfscope}%
\begin{pgfscope}%
\pgfsys@transformshift{2.194046in}{1.407326in}%
\pgfsys@useobject{currentmarker}{}%
\end{pgfscope}%
\begin{pgfscope}%
\pgfsys@transformshift{2.212354in}{1.475676in}%
\pgfsys@useobject{currentmarker}{}%
\end{pgfscope}%
\begin{pgfscope}%
\pgfsys@transformshift{2.233010in}{1.600611in}%
\pgfsys@useobject{currentmarker}{}%
\end{pgfscope}%
\begin{pgfscope}%
\pgfsys@transformshift{2.251086in}{1.702835in}%
\pgfsys@useobject{currentmarker}{}%
\end{pgfscope}%
\begin{pgfscope}%
\pgfsys@transformshift{2.268690in}{1.738802in}%
\pgfsys@useobject{currentmarker}{}%
\end{pgfscope}%
\begin{pgfscope}%
\pgfsys@transformshift{2.286530in}{1.724600in}%
\pgfsys@useobject{currentmarker}{}%
\end{pgfscope}%
\begin{pgfscope}%
\pgfsys@transformshift{2.308360in}{1.645537in}%
\pgfsys@useobject{currentmarker}{}%
\end{pgfscope}%
\begin{pgfscope}%
\pgfsys@transformshift{2.329485in}{1.498654in}%
\pgfsys@useobject{currentmarker}{}%
\end{pgfscope}%
\begin{pgfscope}%
\pgfsys@transformshift{2.347324in}{1.424984in}%
\pgfsys@useobject{currentmarker}{}%
\end{pgfscope}%
\begin{pgfscope}%
\pgfsys@transformshift{2.364929in}{1.373811in}%
\pgfsys@useobject{currentmarker}{}%
\end{pgfscope}%
\begin{pgfscope}%
\pgfsys@transformshift{2.385585in}{1.348137in}%
\pgfsys@useobject{currentmarker}{}%
\end{pgfscope}%
\begin{pgfscope}%
\pgfsys@transformshift{2.404364in}{1.366679in}%
\pgfsys@useobject{currentmarker}{}%
\end{pgfscope}%
\begin{pgfscope}%
\pgfsys@transformshift{2.424785in}{1.418606in}%
\pgfsys@useobject{currentmarker}{}%
\end{pgfscope}%
\begin{pgfscope}%
\pgfsys@transformshift{2.442859in}{1.466160in}%
\pgfsys@useobject{currentmarker}{}%
\end{pgfscope}%
\begin{pgfscope}%
\pgfsys@transformshift{2.463515in}{1.601593in}%
\pgfsys@useobject{currentmarker}{}%
\end{pgfscope}%
\begin{pgfscope}%
\pgfsys@transformshift{2.481589in}{1.705585in}%
\pgfsys@useobject{currentmarker}{}%
\end{pgfscope}%
\begin{pgfscope}%
\pgfsys@transformshift{2.501776in}{1.736443in}%
\pgfsys@useobject{currentmarker}{}%
\end{pgfscope}%
\begin{pgfscope}%
\pgfsys@transformshift{2.521493in}{1.723430in}%
\pgfsys@useobject{currentmarker}{}%
\end{pgfscope}%
\begin{pgfscope}%
\pgfsys@transformshift{2.538629in}{1.640522in}%
\pgfsys@useobject{currentmarker}{}%
\end{pgfscope}%
\begin{pgfscope}%
\pgfsys@transformshift{2.559754in}{1.484084in}%
\pgfsys@useobject{currentmarker}{}%
\end{pgfscope}%
\begin{pgfscope}%
\pgfsys@transformshift{2.578298in}{1.417689in}%
\pgfsys@useobject{currentmarker}{}%
\end{pgfscope}%
\begin{pgfscope}%
\pgfsys@transformshift{2.599189in}{1.364382in}%
\pgfsys@useobject{currentmarker}{}%
\end{pgfscope}%
\begin{pgfscope}%
\pgfsys@transformshift{2.617028in}{1.358426in}%
\pgfsys@useobject{currentmarker}{}%
\end{pgfscope}%
\begin{pgfscope}%
\pgfsys@transformshift{2.635336in}{1.352571in}%
\pgfsys@useobject{currentmarker}{}%
\end{pgfscope}%
\begin{pgfscope}%
\pgfsys@transformshift{2.655523in}{1.389612in}%
\pgfsys@useobject{currentmarker}{}%
\end{pgfscope}%
\begin{pgfscope}%
\pgfsys@transformshift{2.673362in}{1.435950in}%
\pgfsys@useobject{currentmarker}{}%
\end{pgfscope}%
\begin{pgfscope}%
\pgfsys@transformshift{2.694489in}{1.538175in}%
\pgfsys@useobject{currentmarker}{}%
\end{pgfscope}%
\begin{pgfscope}%
\pgfsys@transformshift{2.715849in}{1.669498in}%
\pgfsys@useobject{currentmarker}{}%
\end{pgfscope}%
\begin{pgfscope}%
\pgfsys@transformshift{2.731342in}{1.728002in}%
\pgfsys@useobject{currentmarker}{}%
\end{pgfscope}%
\begin{pgfscope}%
\pgfsys@transformshift{2.749181in}{1.736389in}%
\pgfsys@useobject{currentmarker}{}%
\end{pgfscope}%
\begin{pgfscope}%
\pgfsys@transformshift{2.769132in}{1.692881in}%
\pgfsys@useobject{currentmarker}{}%
\end{pgfscope}%
\begin{pgfscope}%
\pgfsys@transformshift{2.790962in}{1.556575in}%
\pgfsys@useobject{currentmarker}{}%
\end{pgfscope}%
\begin{pgfscope}%
\pgfsys@transformshift{2.808567in}{1.490319in}%
\pgfsys@useobject{currentmarker}{}%
\end{pgfscope}%
\begin{pgfscope}%
\pgfsys@transformshift{2.825703in}{1.413595in}%
\pgfsys@useobject{currentmarker}{}%
\end{pgfscope}%
\begin{pgfscope}%
\pgfsys@transformshift{2.847533in}{1.364888in}%
\pgfsys@useobject{currentmarker}{}%
\end{pgfscope}%
\begin{pgfscope}%
\pgfsys@transformshift{2.865607in}{1.350327in}%
\pgfsys@useobject{currentmarker}{}%
\end{pgfscope}%
\begin{pgfscope}%
\pgfsys@transformshift{2.886497in}{1.381982in}%
\pgfsys@useobject{currentmarker}{}%
\end{pgfscope}%
\begin{pgfscope}%
\pgfsys@transformshift{2.903397in}{1.404567in}%
\pgfsys@useobject{currentmarker}{}%
\end{pgfscope}%
\begin{pgfscope}%
\pgfsys@transformshift{2.925227in}{1.470104in}%
\pgfsys@useobject{currentmarker}{}%
\end{pgfscope}%
\begin{pgfscope}%
\pgfsys@transformshift{2.943066in}{1.569167in}%
\pgfsys@useobject{currentmarker}{}%
\end{pgfscope}%
\begin{pgfscope}%
\pgfsys@transformshift{2.960437in}{1.678627in}%
\pgfsys@useobject{currentmarker}{}%
\end{pgfscope}%
\begin{pgfscope}%
\pgfsys@transformshift{2.981562in}{1.738291in}%
\pgfsys@useobject{currentmarker}{}%
\end{pgfscope}%
\begin{pgfscope}%
\pgfsys@transformshift{2.999872in}{1.737939in}%
\pgfsys@useobject{currentmarker}{}%
\end{pgfscope}%
\begin{pgfscope}%
\pgfsys@transformshift{3.020762in}{1.678812in}%
\pgfsys@useobject{currentmarker}{}%
\end{pgfscope}%
\begin{pgfscope}%
\pgfsys@transformshift{3.039307in}{1.558553in}%
\pgfsys@useobject{currentmarker}{}%
\end{pgfscope}%
\begin{pgfscope}%
\pgfsys@transformshift{3.059963in}{1.445108in}%
\pgfsys@useobject{currentmarker}{}%
\end{pgfscope}%
\begin{pgfscope}%
\pgfsys@transformshift{3.078271in}{1.394278in}%
\pgfsys@useobject{currentmarker}{}%
\end{pgfscope}%
\begin{pgfscope}%
\pgfsys@transformshift{3.096345in}{1.363451in}%
\pgfsys@useobject{currentmarker}{}%
\end{pgfscope}%
\begin{pgfscope}%
\pgfsys@transformshift{3.116766in}{1.354866in}%
\pgfsys@useobject{currentmarker}{}%
\end{pgfscope}%
\begin{pgfscope}%
\pgfsys@transformshift{3.135076in}{1.383439in}%
\pgfsys@useobject{currentmarker}{}%
\end{pgfscope}%
\begin{pgfscope}%
\pgfsys@transformshift{3.152915in}{1.425452in}%
\pgfsys@useobject{currentmarker}{}%
\end{pgfscope}%
\begin{pgfscope}%
\pgfsys@transformshift{3.173572in}{1.516268in}%
\pgfsys@useobject{currentmarker}{}%
\end{pgfscope}%
\begin{pgfscope}%
\pgfsys@transformshift{3.192114in}{1.606014in}%
\pgfsys@useobject{currentmarker}{}%
\end{pgfscope}%
\begin{pgfscope}%
\pgfsys@transformshift{3.212770in}{1.704070in}%
\pgfsys@useobject{currentmarker}{}%
\end{pgfscope}%
\begin{pgfscope}%
\pgfsys@transformshift{3.229672in}{1.740274in}%
\pgfsys@useobject{currentmarker}{}%
\end{pgfscope}%
\begin{pgfscope}%
\pgfsys@transformshift{3.252205in}{1.737359in}%
\pgfsys@useobject{currentmarker}{}%
\end{pgfscope}%
\begin{pgfscope}%
\pgfsys@transformshift{3.268167in}{1.705652in}%
\pgfsys@useobject{currentmarker}{}%
\end{pgfscope}%
\begin{pgfscope}%
\pgfsys@transformshift{3.287649in}{1.593941in}%
\pgfsys@useobject{currentmarker}{}%
\end{pgfscope}%
\begin{pgfscope}%
\pgfsys@transformshift{3.308305in}{1.500901in}%
\pgfsys@useobject{currentmarker}{}%
\end{pgfscope}%
\begin{pgfscope}%
\pgfsys@transformshift{3.326381in}{1.433147in}%
\pgfsys@useobject{currentmarker}{}%
\end{pgfscope}%
\begin{pgfscope}%
\pgfsys@transformshift{3.347272in}{1.377831in}%
\pgfsys@useobject{currentmarker}{}%
\end{pgfscope}%
\begin{pgfscope}%
\pgfsys@transformshift{3.365111in}{1.355938in}%
\pgfsys@useobject{currentmarker}{}%
\end{pgfscope}%
\begin{pgfscope}%
\pgfsys@transformshift{3.384358in}{1.363825in}%
\pgfsys@useobject{currentmarker}{}%
\end{pgfscope}%
\begin{pgfscope}%
\pgfsys@transformshift{3.402901in}{1.401399in}%
\pgfsys@useobject{currentmarker}{}%
\end{pgfscope}%
\begin{pgfscope}%
\pgfsys@transformshift{3.425436in}{1.468561in}%
\pgfsys@useobject{currentmarker}{}%
\end{pgfscope}%
\begin{pgfscope}%
\pgfsys@transformshift{3.442336in}{1.545264in}%
\pgfsys@useobject{currentmarker}{}%
\end{pgfscope}%
\begin{pgfscope}%
\pgfsys@transformshift{3.463697in}{1.655184in}%
\pgfsys@useobject{currentmarker}{}%
\end{pgfscope}%
\begin{pgfscope}%
\pgfsys@transformshift{3.481771in}{1.718523in}%
\pgfsys@useobject{currentmarker}{}%
\end{pgfscope}%
\begin{pgfscope}%
\pgfsys@transformshift{3.500315in}{1.749360in}%
\pgfsys@useobject{currentmarker}{}%
\end{pgfscope}%
\begin{pgfscope}%
\pgfsys@transformshift{3.520971in}{1.736669in}%
\pgfsys@useobject{currentmarker}{}%
\end{pgfscope}%
\begin{pgfscope}%
\pgfsys@transformshift{3.538340in}{1.677542in}%
\pgfsys@useobject{currentmarker}{}%
\end{pgfscope}%
\begin{pgfscope}%
\pgfsys@transformshift{3.556415in}{1.599179in}%
\pgfsys@useobject{currentmarker}{}%
\end{pgfscope}%
\begin{pgfscope}%
\pgfsys@transformshift{3.575898in}{1.501949in}%
\pgfsys@useobject{currentmarker}{}%
\end{pgfscope}%
\begin{pgfscope}%
\pgfsys@transformshift{3.595614in}{1.442393in}%
\pgfsys@useobject{currentmarker}{}%
\end{pgfscope}%
\begin{pgfscope}%
\pgfsys@transformshift{3.616036in}{1.387449in}%
\pgfsys@useobject{currentmarker}{}%
\end{pgfscope}%
\begin{pgfscope}%
\pgfsys@transformshift{3.634580in}{1.359478in}%
\pgfsys@useobject{currentmarker}{}%
\end{pgfscope}%
\begin{pgfscope}%
\pgfsys@transformshift{3.656174in}{1.366868in}%
\pgfsys@useobject{currentmarker}{}%
\end{pgfscope}%
\begin{pgfscope}%
\pgfsys@transformshift{3.673544in}{1.399890in}%
\pgfsys@useobject{currentmarker}{}%
\end{pgfscope}%
\begin{pgfscope}%
\pgfsys@transformshift{3.693027in}{1.417338in}%
\pgfsys@useobject{currentmarker}{}%
\end{pgfscope}%
\begin{pgfscope}%
\pgfsys@transformshift{3.712745in}{1.470034in}%
\pgfsys@useobject{currentmarker}{}%
\end{pgfscope}%
\begin{pgfscope}%
\pgfsys@transformshift{3.731287in}{1.529823in}%
\pgfsys@useobject{currentmarker}{}%
\end{pgfscope}%
\begin{pgfscope}%
\pgfsys@transformshift{3.748892in}{1.618971in}%
\pgfsys@useobject{currentmarker}{}%
\end{pgfscope}%
\begin{pgfscope}%
\pgfsys@transformshift{3.768845in}{1.717869in}%
\pgfsys@useobject{currentmarker}{}%
\end{pgfscope}%
\begin{pgfscope}%
\pgfsys@transformshift{3.790441in}{1.752590in}%
\pgfsys@useobject{currentmarker}{}%
\end{pgfscope}%
\begin{pgfscope}%
\pgfsys@transformshift{3.808983in}{1.756370in}%
\pgfsys@useobject{currentmarker}{}%
\end{pgfscope}%
\begin{pgfscope}%
\pgfsys@transformshift{3.826354in}{1.752489in}%
\pgfsys@useobject{currentmarker}{}%
\end{pgfscope}%
\begin{pgfscope}%
\pgfsys@transformshift{3.845133in}{1.732976in}%
\pgfsys@useobject{currentmarker}{}%
\end{pgfscope}%
\begin{pgfscope}%
\pgfsys@transformshift{3.865318in}{1.668777in}%
\pgfsys@useobject{currentmarker}{}%
\end{pgfscope}%
\begin{pgfscope}%
\pgfsys@transformshift{3.886445in}{1.531953in}%
\pgfsys@useobject{currentmarker}{}%
\end{pgfscope}%
\begin{pgfscope}%
\pgfsys@transformshift{3.904050in}{1.451570in}%
\pgfsys@useobject{currentmarker}{}%
\end{pgfscope}%
\begin{pgfscope}%
\pgfsys@transformshift{3.923063in}{1.397920in}%
\pgfsys@useobject{currentmarker}{}%
\end{pgfscope}%
\begin{pgfscope}%
\pgfsys@transformshift{3.944422in}{1.367620in}%
\pgfsys@useobject{currentmarker}{}%
\end{pgfscope}%
\begin{pgfscope}%
\pgfsys@transformshift{3.960384in}{1.362351in}%
\pgfsys@useobject{currentmarker}{}%
\end{pgfscope}%
\begin{pgfscope}%
\pgfsys@transformshift{3.981980in}{1.394311in}%
\pgfsys@useobject{currentmarker}{}%
\end{pgfscope}%
\begin{pgfscope}%
\pgfsys@transformshift{3.998880in}{1.431174in}%
\pgfsys@useobject{currentmarker}{}%
\end{pgfscope}%
\begin{pgfscope}%
\pgfsys@transformshift{4.022118in}{1.508405in}%
\pgfsys@useobject{currentmarker}{}%
\end{pgfscope}%
\begin{pgfscope}%
\pgfsys@transformshift{4.039489in}{1.608210in}%
\pgfsys@useobject{currentmarker}{}%
\end{pgfscope}%
\begin{pgfscope}%
\pgfsys@transformshift{4.058266in}{1.695743in}%
\pgfsys@useobject{currentmarker}{}%
\end{pgfscope}%
\begin{pgfscope}%
\pgfsys@transformshift{4.075167in}{1.747957in}%
\pgfsys@useobject{currentmarker}{}%
\end{pgfscope}%
\begin{pgfscope}%
\pgfsys@transformshift{4.096997in}{1.763995in}%
\pgfsys@useobject{currentmarker}{}%
\end{pgfscope}%
\begin{pgfscope}%
\pgfsys@transformshift{4.115071in}{1.757196in}%
\pgfsys@useobject{currentmarker}{}%
\end{pgfscope}%
\begin{pgfscope}%
\pgfsys@transformshift{4.135024in}{1.713158in}%
\pgfsys@useobject{currentmarker}{}%
\end{pgfscope}%
\begin{pgfscope}%
\pgfsys@transformshift{4.154740in}{1.577591in}%
\pgfsys@useobject{currentmarker}{}%
\end{pgfscope}%
\begin{pgfscope}%
\pgfsys@transformshift{4.172580in}{1.701798in}%
\pgfsys@useobject{currentmarker}{}%
\end{pgfscope}%
\begin{pgfscope}%
\pgfsys@transformshift{4.193705in}{1.762824in}%
\pgfsys@useobject{currentmarker}{}%
\end{pgfscope}%
\begin{pgfscope}%
\pgfsys@transformshift{4.211780in}{1.755664in}%
\pgfsys@useobject{currentmarker}{}%
\end{pgfscope}%
\begin{pgfscope}%
\pgfsys@transformshift{4.229854in}{1.698858in}%
\pgfsys@useobject{currentmarker}{}%
\end{pgfscope}%
\begin{pgfscope}%
\pgfsys@transformshift{4.249805in}{1.573159in}%
\pgfsys@useobject{currentmarker}{}%
\end{pgfscope}%
\begin{pgfscope}%
\pgfsys@transformshift{4.267880in}{1.486688in}%
\pgfsys@useobject{currentmarker}{}%
\end{pgfscope}%
\begin{pgfscope}%
\pgfsys@transformshift{4.289005in}{1.410648in}%
\pgfsys@useobject{currentmarker}{}%
\end{pgfscope}%
\begin{pgfscope}%
\pgfsys@transformshift{4.307079in}{1.375551in}%
\pgfsys@useobject{currentmarker}{}%
\end{pgfscope}%
\begin{pgfscope}%
\pgfsys@transformshift{4.328440in}{1.377539in}%
\pgfsys@useobject{currentmarker}{}%
\end{pgfscope}%
\begin{pgfscope}%
\pgfsys@transformshift{4.344871in}{1.408635in}%
\pgfsys@useobject{currentmarker}{}%
\end{pgfscope}%
\begin{pgfscope}%
\pgfsys@transformshift{4.365762in}{1.464329in}%
\pgfsys@useobject{currentmarker}{}%
\end{pgfscope}%
\begin{pgfscope}%
\pgfsys@transformshift{4.384540in}{1.542459in}%
\pgfsys@useobject{currentmarker}{}%
\end{pgfscope}%
\begin{pgfscope}%
\pgfsys@transformshift{4.404023in}{1.653118in}%
\pgfsys@useobject{currentmarker}{}%
\end{pgfscope}%
\begin{pgfscope}%
\pgfsys@transformshift{4.423505in}{1.741998in}%
\pgfsys@useobject{currentmarker}{}%
\end{pgfscope}%
\begin{pgfscope}%
\pgfsys@transformshift{4.443223in}{1.772910in}%
\pgfsys@useobject{currentmarker}{}%
\end{pgfscope}%
\begin{pgfscope}%
\pgfsys@transformshift{4.462705in}{1.760711in}%
\pgfsys@useobject{currentmarker}{}%
\end{pgfscope}%
\begin{pgfscope}%
\pgfsys@transformshift{4.481718in}{1.696425in}%
\pgfsys@useobject{currentmarker}{}%
\end{pgfscope}%
\begin{pgfscope}%
\pgfsys@transformshift{4.482187in}{1.699363in}%
\pgfsys@useobject{currentmarker}{}%
\end{pgfscope}%
\begin{pgfscope}%
\pgfsys@transformshift{4.472564in}{1.744488in}%
\pgfsys@useobject{currentmarker}{}%
\end{pgfscope}%
\begin{pgfscope}%
\pgfsys@transformshift{4.453551in}{1.772896in}%
\pgfsys@useobject{currentmarker}{}%
\end{pgfscope}%
\begin{pgfscope}%
\pgfsys@transformshift{4.435476in}{1.727785in}%
\pgfsys@useobject{currentmarker}{}%
\end{pgfscope}%
\begin{pgfscope}%
\pgfsys@transformshift{4.416933in}{1.594706in}%
\pgfsys@useobject{currentmarker}{}%
\end{pgfscope}%
\begin{pgfscope}%
\pgfsys@transformshift{4.399094in}{1.474852in}%
\pgfsys@useobject{currentmarker}{}%
\end{pgfscope}%
\begin{pgfscope}%
\pgfsys@transformshift{4.377029in}{1.390158in}%
\pgfsys@useobject{currentmarker}{}%
\end{pgfscope}%
\begin{pgfscope}%
\pgfsys@transformshift{4.358485in}{1.369295in}%
\pgfsys@useobject{currentmarker}{}%
\end{pgfscope}%
\begin{pgfscope}%
\pgfsys@transformshift{4.338298in}{1.452728in}%
\pgfsys@useobject{currentmarker}{}%
\end{pgfscope}%
\begin{pgfscope}%
\pgfsys@transformshift{4.318347in}{1.581939in}%
\pgfsys@useobject{currentmarker}{}%
\end{pgfscope}%
\begin{pgfscope}%
\pgfsys@transformshift{4.302150in}{1.701595in}%
\pgfsys@useobject{currentmarker}{}%
\end{pgfscope}%
\begin{pgfscope}%
\pgfsys@transformshift{4.281260in}{1.765153in}%
\pgfsys@useobject{currentmarker}{}%
\end{pgfscope}%
\begin{pgfscope}%
\pgfsys@transformshift{4.262715in}{1.735757in}%
\pgfsys@useobject{currentmarker}{}%
\end{pgfscope}%
\begin{pgfscope}%
\pgfsys@transformshift{4.242059in}{1.592458in}%
\pgfsys@useobject{currentmarker}{}%
\end{pgfscope}%
\begin{pgfscope}%
\pgfsys@transformshift{4.226097in}{1.487841in}%
\pgfsys@useobject{currentmarker}{}%
\end{pgfscope}%
\begin{pgfscope}%
\pgfsys@transformshift{4.204738in}{1.396259in}%
\pgfsys@useobject{currentmarker}{}%
\end{pgfscope}%
\begin{pgfscope}%
\pgfsys@transformshift{4.186193in}{1.363739in}%
\pgfsys@useobject{currentmarker}{}%
\end{pgfscope}%
\begin{pgfscope}%
\pgfsys@transformshift{4.167415in}{1.390929in}%
\pgfsys@useobject{currentmarker}{}%
\end{pgfscope}%
\begin{pgfscope}%
\pgfsys@transformshift{4.146290in}{1.482364in}%
\pgfsys@useobject{currentmarker}{}%
\end{pgfscope}%
\begin{pgfscope}%
\pgfsys@transformshift{4.128216in}{1.614653in}%
\pgfsys@useobject{currentmarker}{}%
\end{pgfscope}%
\begin{pgfscope}%
\pgfsys@transformshift{4.110611in}{1.727371in}%
\pgfsys@useobject{currentmarker}{}%
\end{pgfscope}%
\begin{pgfscope}%
\pgfsys@transformshift{4.089955in}{1.732580in}%
\pgfsys@useobject{currentmarker}{}%
\end{pgfscope}%
\begin{pgfscope}%
\pgfsys@transformshift{4.072585in}{1.757365in}%
\pgfsys@useobject{currentmarker}{}%
\end{pgfscope}%
\begin{pgfscope}%
\pgfsys@transformshift{4.050051in}{1.679618in}%
\pgfsys@useobject{currentmarker}{}%
\end{pgfscope}%
\begin{pgfscope}%
\pgfsys@transformshift{4.032915in}{1.547738in}%
\pgfsys@useobject{currentmarker}{}%
\end{pgfscope}%
\begin{pgfscope}%
\pgfsys@transformshift{4.013668in}{1.442785in}%
\pgfsys@useobject{currentmarker}{}%
\end{pgfscope}%
\begin{pgfscope}%
\pgfsys@transformshift{3.993011in}{1.374853in}%
\pgfsys@useobject{currentmarker}{}%
\end{pgfscope}%
\begin{pgfscope}%
\pgfsys@transformshift{3.973764in}{1.363617in}%
\pgfsys@useobject{currentmarker}{}%
\end{pgfscope}%
\begin{pgfscope}%
\pgfsys@transformshift{3.955219in}{1.410058in}%
\pgfsys@useobject{currentmarker}{}%
\end{pgfscope}%
\begin{pgfscope}%
\pgfsys@transformshift{3.938554in}{1.490900in}%
\pgfsys@useobject{currentmarker}{}%
\end{pgfscope}%
\begin{pgfscope}%
\pgfsys@transformshift{3.917429in}{1.640454in}%
\pgfsys@useobject{currentmarker}{}%
\end{pgfscope}%
\begin{pgfscope}%
\pgfsys@transformshift{3.897007in}{1.739918in}%
\pgfsys@useobject{currentmarker}{}%
\end{pgfscope}%
\begin{pgfscope}%
\pgfsys@transformshift{3.879168in}{1.746344in}%
\pgfsys@useobject{currentmarker}{}%
\end{pgfscope}%
\begin{pgfscope}%
\pgfsys@transformshift{3.859215in}{1.661485in}%
\pgfsys@useobject{currentmarker}{}%
\end{pgfscope}%
\begin{pgfscope}%
\pgfsys@transformshift{3.838794in}{1.512750in}%
\pgfsys@useobject{currentmarker}{}%
\end{pgfscope}%
\begin{pgfscope}%
\pgfsys@transformshift{3.822363in}{1.432842in}%
\pgfsys@useobject{currentmarker}{}%
\end{pgfscope}%
\begin{pgfscope}%
\pgfsys@transformshift{3.801707in}{1.365940in}%
\pgfsys@useobject{currentmarker}{}%
\end{pgfscope}%
\begin{pgfscope}%
\pgfsys@transformshift{3.783399in}{1.355377in}%
\pgfsys@useobject{currentmarker}{}%
\end{pgfscope}%
\begin{pgfscope}%
\pgfsys@transformshift{3.763917in}{1.394910in}%
\pgfsys@useobject{currentmarker}{}%
\end{pgfscope}%
\begin{pgfscope}%
\pgfsys@transformshift{3.742790in}{1.483582in}%
\pgfsys@useobject{currentmarker}{}%
\end{pgfscope}%
\begin{pgfscope}%
\pgfsys@transformshift{3.725185in}{1.607146in}%
\pgfsys@useobject{currentmarker}{}%
\end{pgfscope}%
\begin{pgfscope}%
\pgfsys@transformshift{3.705234in}{1.726154in}%
\pgfsys@useobject{currentmarker}{}%
\end{pgfscope}%
\begin{pgfscope}%
\pgfsys@transformshift{3.686689in}{1.747973in}%
\pgfsys@useobject{currentmarker}{}%
\end{pgfscope}%
\begin{pgfscope}%
\pgfsys@transformshift{3.666502in}{1.699089in}%
\pgfsys@useobject{currentmarker}{}%
\end{pgfscope}%
\begin{pgfscope}%
\pgfsys@transformshift{3.649134in}{1.582932in}%
\pgfsys@useobject{currentmarker}{}%
\end{pgfscope}%
\begin{pgfscope}%
\pgfsys@transformshift{3.628241in}{1.453030in}%
\pgfsys@useobject{currentmarker}{}%
\end{pgfscope}%
\begin{pgfscope}%
\pgfsys@transformshift{3.607351in}{1.388548in}%
\pgfsys@useobject{currentmarker}{}%
\end{pgfscope}%
\begin{pgfscope}%
\pgfsys@transformshift{3.590217in}{1.710874in}%
\pgfsys@useobject{currentmarker}{}%
\end{pgfscope}%
\begin{pgfscope}%
\pgfsys@transformshift{3.572612in}{1.604340in}%
\pgfsys@useobject{currentmarker}{}%
\end{pgfscope}%
\begin{pgfscope}%
\pgfsys@transformshift{3.551250in}{1.469683in}%
\pgfsys@useobject{currentmarker}{}%
\end{pgfscope}%
\begin{pgfscope}%
\pgfsys@transformshift{3.534585in}{1.401927in}%
\pgfsys@useobject{currentmarker}{}%
\end{pgfscope}%
\begin{pgfscope}%
\pgfsys@transformshift{3.514398in}{1.355799in}%
\pgfsys@useobject{currentmarker}{}%
\end{pgfscope}%
\begin{pgfscope}%
\pgfsys@transformshift{3.493039in}{1.368143in}%
\pgfsys@useobject{currentmarker}{}%
\end{pgfscope}%
\begin{pgfscope}%
\pgfsys@transformshift{3.476137in}{1.395837in}%
\pgfsys@useobject{currentmarker}{}%
\end{pgfscope}%
\begin{pgfscope}%
\pgfsys@transformshift{3.457829in}{1.471375in}%
\pgfsys@useobject{currentmarker}{}%
\end{pgfscope}%
\begin{pgfscope}%
\pgfsys@transformshift{3.437173in}{1.612406in}%
\pgfsys@useobject{currentmarker}{}%
\end{pgfscope}%
\begin{pgfscope}%
\pgfsys@transformshift{3.417691in}{1.722870in}%
\pgfsys@useobject{currentmarker}{}%
\end{pgfscope}%
\begin{pgfscope}%
\pgfsys@transformshift{3.398677in}{1.740062in}%
\pgfsys@useobject{currentmarker}{}%
\end{pgfscope}%
\begin{pgfscope}%
\pgfsys@transformshift{3.379193in}{1.671165in}%
\pgfsys@useobject{currentmarker}{}%
\end{pgfscope}%
\begin{pgfscope}%
\pgfsys@transformshift{3.360417in}{1.555832in}%
\pgfsys@useobject{currentmarker}{}%
\end{pgfscope}%
\begin{pgfscope}%
\pgfsys@transformshift{3.340464in}{1.456389in}%
\pgfsys@useobject{currentmarker}{}%
\end{pgfscope}%
\begin{pgfscope}%
\pgfsys@transformshift{3.317930in}{1.385567in}%
\pgfsys@useobject{currentmarker}{}%
\end{pgfscope}%
\begin{pgfscope}%
\pgfsys@transformshift{3.301968in}{1.351835in}%
\pgfsys@useobject{currentmarker}{}%
\end{pgfscope}%
\begin{pgfscope}%
\pgfsys@transformshift{3.282015in}{1.365398in}%
\pgfsys@useobject{currentmarker}{}%
\end{pgfscope}%
\begin{pgfscope}%
\pgfsys@transformshift{3.264647in}{1.407052in}%
\pgfsys@useobject{currentmarker}{}%
\end{pgfscope}%
\begin{pgfscope}%
\pgfsys@transformshift{3.246337in}{1.454953in}%
\pgfsys@useobject{currentmarker}{}%
\end{pgfscope}%
\begin{pgfscope}%
\pgfsys@transformshift{3.227089in}{1.577553in}%
\pgfsys@useobject{currentmarker}{}%
\end{pgfscope}%
\begin{pgfscope}%
\pgfsys@transformshift{3.205730in}{1.710315in}%
\pgfsys@useobject{currentmarker}{}%
\end{pgfscope}%
\begin{pgfscope}%
\pgfsys@transformshift{3.188594in}{1.741105in}%
\pgfsys@useobject{currentmarker}{}%
\end{pgfscope}%
\begin{pgfscope}%
\pgfsys@transformshift{3.166060in}{1.699975in}%
\pgfsys@useobject{currentmarker}{}%
\end{pgfscope}%
\begin{pgfscope}%
\pgfsys@transformshift{3.148690in}{1.597290in}%
\pgfsys@useobject{currentmarker}{}%
\end{pgfscope}%
\begin{pgfscope}%
\pgfsys@transformshift{3.131085in}{1.484135in}%
\pgfsys@useobject{currentmarker}{}%
\end{pgfscope}%
\begin{pgfscope}%
\pgfsys@transformshift{3.110898in}{1.405956in}%
\pgfsys@useobject{currentmarker}{}%
\end{pgfscope}%
\begin{pgfscope}%
\pgfsys@transformshift{3.090713in}{1.355996in}%
\pgfsys@useobject{currentmarker}{}%
\end{pgfscope}%
\begin{pgfscope}%
\pgfsys@transformshift{3.070994in}{1.357778in}%
\pgfsys@useobject{currentmarker}{}%
\end{pgfscope}%
\begin{pgfscope}%
\pgfsys@transformshift{3.053155in}{1.394872in}%
\pgfsys@useobject{currentmarker}{}%
\end{pgfscope}%
\begin{pgfscope}%
\pgfsys@transformshift{3.035316in}{1.463791in}%
\pgfsys@useobject{currentmarker}{}%
\end{pgfscope}%
\begin{pgfscope}%
\pgfsys@transformshift{3.014660in}{1.578357in}%
\pgfsys@useobject{currentmarker}{}%
\end{pgfscope}%
\begin{pgfscope}%
\pgfsys@transformshift{2.994003in}{1.692795in}%
\pgfsys@useobject{currentmarker}{}%
\end{pgfscope}%
\begin{pgfscope}%
\pgfsys@transformshift{2.976399in}{1.737701in}%
\pgfsys@useobject{currentmarker}{}%
\end{pgfscope}%
\begin{pgfscope}%
\pgfsys@transformshift{2.954803in}{1.716895in}%
\pgfsys@useobject{currentmarker}{}%
\end{pgfscope}%
\begin{pgfscope}%
\pgfsys@transformshift{2.937669in}{1.665612in}%
\pgfsys@useobject{currentmarker}{}%
\end{pgfscope}%
\begin{pgfscope}%
\pgfsys@transformshift{2.918421in}{1.544428in}%
\pgfsys@useobject{currentmarker}{}%
\end{pgfscope}%
\begin{pgfscope}%
\pgfsys@transformshift{2.900816in}{1.457874in}%
\pgfsys@useobject{currentmarker}{}%
\end{pgfscope}%
\begin{pgfscope}%
\pgfsys@transformshift{2.878986in}{1.385904in}%
\pgfsys@useobject{currentmarker}{}%
\end{pgfscope}%
\begin{pgfscope}%
\pgfsys@transformshift{2.860207in}{1.354155in}%
\pgfsys@useobject{currentmarker}{}%
\end{pgfscope}%
\begin{pgfscope}%
\pgfsys@transformshift{2.841194in}{1.362469in}%
\pgfsys@useobject{currentmarker}{}%
\end{pgfscope}%
\begin{pgfscope}%
\pgfsys@transformshift{2.821009in}{1.410689in}%
\pgfsys@useobject{currentmarker}{}%
\end{pgfscope}%
\begin{pgfscope}%
\pgfsys@transformshift{2.801525in}{1.493465in}%
\pgfsys@useobject{currentmarker}{}%
\end{pgfscope}%
\begin{pgfscope}%
\pgfsys@transformshift{2.781808in}{1.574492in}%
\pgfsys@useobject{currentmarker}{}%
\end{pgfscope}%
\begin{pgfscope}%
\pgfsys@transformshift{2.763498in}{1.692630in}%
\pgfsys@useobject{currentmarker}{}%
\end{pgfscope}%
\begin{pgfscope}%
\pgfsys@transformshift{2.743313in}{1.737850in}%
\pgfsys@useobject{currentmarker}{}%
\end{pgfscope}%
\begin{pgfscope}%
\pgfsys@transformshift{2.726646in}{1.727088in}%
\pgfsys@useobject{currentmarker}{}%
\end{pgfscope}%
\begin{pgfscope}%
\pgfsys@transformshift{2.707869in}{1.665816in}%
\pgfsys@useobject{currentmarker}{}%
\end{pgfscope}%
\begin{pgfscope}%
\pgfsys@transformshift{2.685570in}{1.534696in}%
\pgfsys@useobject{currentmarker}{}%
\end{pgfscope}%
\begin{pgfscope}%
\pgfsys@transformshift{2.667494in}{1.446254in}%
\pgfsys@useobject{currentmarker}{}%
\end{pgfscope}%
\begin{pgfscope}%
\pgfsys@transformshift{2.651769in}{1.398595in}%
\pgfsys@useobject{currentmarker}{}%
\end{pgfscope}%
\begin{pgfscope}%
\pgfsys@transformshift{2.630173in}{1.354270in}%
\pgfsys@useobject{currentmarker}{}%
\end{pgfscope}%
\begin{pgfscope}%
\pgfsys@transformshift{2.611160in}{1.356612in}%
\pgfsys@useobject{currentmarker}{}%
\end{pgfscope}%
\begin{pgfscope}%
\pgfsys@transformshift{2.592615in}{1.389771in}%
\pgfsys@useobject{currentmarker}{}%
\end{pgfscope}%
\begin{pgfscope}%
\pgfsys@transformshift{2.570553in}{1.428025in}%
\pgfsys@useobject{currentmarker}{}%
\end{pgfscope}%
\begin{pgfscope}%
\pgfsys@transformshift{2.552243in}{1.519052in}%
\pgfsys@useobject{currentmarker}{}%
\end{pgfscope}%
\begin{pgfscope}%
\pgfsys@transformshift{2.533229in}{1.640996in}%
\pgfsys@useobject{currentmarker}{}%
\end{pgfscope}%
\begin{pgfscope}%
\pgfsys@transformshift{2.514450in}{1.722920in}%
\pgfsys@useobject{currentmarker}{}%
\end{pgfscope}%
\begin{pgfscope}%
\pgfsys@transformshift{2.496142in}{1.735241in}%
\pgfsys@useobject{currentmarker}{}%
\end{pgfscope}%
\begin{pgfscope}%
\pgfsys@transformshift{2.474312in}{1.673570in}%
\pgfsys@useobject{currentmarker}{}%
\end{pgfscope}%
\begin{pgfscope}%
\pgfsys@transformshift{2.455299in}{1.587253in}%
\pgfsys@useobject{currentmarker}{}%
\end{pgfscope}%
\begin{pgfscope}%
\pgfsys@transformshift{2.436756in}{1.486521in}%
\pgfsys@useobject{currentmarker}{}%
\end{pgfscope}%
\begin{pgfscope}%
\pgfsys@transformshift{2.415630in}{1.430319in}%
\pgfsys@useobject{currentmarker}{}%
\end{pgfscope}%
\begin{pgfscope}%
\pgfsys@transformshift{2.399904in}{1.386340in}%
\pgfsys@useobject{currentmarker}{}%
\end{pgfscope}%
\begin{pgfscope}%
\pgfsys@transformshift{2.375257in}{1.676378in}%
\pgfsys@useobject{currentmarker}{}%
\end{pgfscope}%
\begin{pgfscope}%
\pgfsys@transformshift{2.358826in}{1.568555in}%
\pgfsys@useobject{currentmarker}{}%
\end{pgfscope}%
\begin{pgfscope}%
\pgfsys@transformshift{2.340987in}{1.459249in}%
\pgfsys@useobject{currentmarker}{}%
\end{pgfscope}%
\begin{pgfscope}%
\pgfsys@transformshift{2.319626in}{1.387710in}%
\pgfsys@useobject{currentmarker}{}%
\end{pgfscope}%
\begin{pgfscope}%
\pgfsys@transformshift{2.303664in}{1.354878in}%
\pgfsys@useobject{currentmarker}{}%
\end{pgfscope}%
\begin{pgfscope}%
\pgfsys@transformshift{2.285121in}{1.355825in}%
\pgfsys@useobject{currentmarker}{}%
\end{pgfscope}%
\begin{pgfscope}%
\pgfsys@transformshift{2.264231in}{1.393991in}%
\pgfsys@useobject{currentmarker}{}%
\end{pgfscope}%
\begin{pgfscope}%
\pgfsys@transformshift{2.244512in}{1.465390in}%
\pgfsys@useobject{currentmarker}{}%
\end{pgfscope}%
\begin{pgfscope}%
\pgfsys@transformshift{2.226673in}{1.574104in}%
\pgfsys@useobject{currentmarker}{}%
\end{pgfscope}%
\begin{pgfscope}%
\pgfsys@transformshift{2.203669in}{1.713902in}%
\pgfsys@useobject{currentmarker}{}%
\end{pgfscope}%
\begin{pgfscope}%
\pgfsys@transformshift{2.188646in}{1.739822in}%
\pgfsys@useobject{currentmarker}{}%
\end{pgfscope}%
\begin{pgfscope}%
\pgfsys@transformshift{2.170104in}{1.715139in}%
\pgfsys@useobject{currentmarker}{}%
\end{pgfscope}%
\begin{pgfscope}%
\pgfsys@transformshift{2.148743in}{1.609521in}%
\pgfsys@useobject{currentmarker}{}%
\end{pgfscope}%
\begin{pgfscope}%
\pgfsys@transformshift{2.132781in}{1.523021in}%
\pgfsys@useobject{currentmarker}{}%
\end{pgfscope}%
\begin{pgfscope}%
\pgfsys@transformshift{2.110482in}{1.430349in}%
\pgfsys@useobject{currentmarker}{}%
\end{pgfscope}%
\begin{pgfscope}%
\pgfsys@transformshift{2.088417in}{1.370719in}%
\pgfsys@useobject{currentmarker}{}%
\end{pgfscope}%
\begin{pgfscope}%
\pgfsys@transformshift{2.070812in}{1.353197in}%
\pgfsys@useobject{currentmarker}{}%
\end{pgfscope}%
\begin{pgfscope}%
\pgfsys@transformshift{2.052035in}{1.367090in}%
\pgfsys@useobject{currentmarker}{}%
\end{pgfscope}%
\begin{pgfscope}%
\pgfsys@transformshift{2.033491in}{1.415513in}%
\pgfsys@useobject{currentmarker}{}%
\end{pgfscope}%
\begin{pgfscope}%
\pgfsys@transformshift{2.014946in}{1.455136in}%
\pgfsys@useobject{currentmarker}{}%
\end{pgfscope}%
\begin{pgfscope}%
\pgfsys@transformshift{1.996170in}{1.571083in}%
\pgfsys@useobject{currentmarker}{}%
\end{pgfscope}%
\begin{pgfscope}%
\pgfsys@transformshift{1.977391in}{1.687593in}%
\pgfsys@useobject{currentmarker}{}%
\end{pgfscope}%
\begin{pgfscope}%
\pgfsys@transformshift{1.956029in}{1.741488in}%
\pgfsys@useobject{currentmarker}{}%
\end{pgfscope}%
\begin{pgfscope}%
\pgfsys@transformshift{1.939364in}{1.733338in}%
\pgfsys@useobject{currentmarker}{}%
\end{pgfscope}%
\begin{pgfscope}%
\pgfsys@transformshift{1.917534in}{1.642496in}%
\pgfsys@useobject{currentmarker}{}%
\end{pgfscope}%
\begin{pgfscope}%
\pgfsys@transformshift{1.899695in}{1.542346in}%
\pgfsys@useobject{currentmarker}{}%
\end{pgfscope}%
\begin{pgfscope}%
\pgfsys@transformshift{1.879039in}{1.443413in}%
\pgfsys@useobject{currentmarker}{}%
\end{pgfscope}%
\begin{pgfscope}%
\pgfsys@transformshift{1.860496in}{1.386480in}%
\pgfsys@useobject{currentmarker}{}%
\end{pgfscope}%
\begin{pgfscope}%
\pgfsys@transformshift{1.841717in}{1.354285in}%
\pgfsys@useobject{currentmarker}{}%
\end{pgfscope}%
\begin{pgfscope}%
\pgfsys@transformshift{1.822704in}{1.364322in}%
\pgfsys@useobject{currentmarker}{}%
\end{pgfscope}%
\begin{pgfscope}%
\pgfsys@transformshift{1.803925in}{1.401192in}%
\pgfsys@useobject{currentmarker}{}%
\end{pgfscope}%
\begin{pgfscope}%
\pgfsys@transformshift{1.784912in}{1.458602in}%
\pgfsys@useobject{currentmarker}{}%
\end{pgfscope}%
\begin{pgfscope}%
\pgfsys@transformshift{1.763787in}{1.545646in}%
\pgfsys@useobject{currentmarker}{}%
\end{pgfscope}%
\begin{pgfscope}%
\pgfsys@transformshift{1.745713in}{1.660581in}%
\pgfsys@useobject{currentmarker}{}%
\end{pgfscope}%
\begin{pgfscope}%
\pgfsys@transformshift{1.726935in}{1.728571in}%
\pgfsys@useobject{currentmarker}{}%
\end{pgfscope}%
\begin{pgfscope}%
\pgfsys@transformshift{1.708390in}{1.744578in}%
\pgfsys@useobject{currentmarker}{}%
\end{pgfscope}%
\begin{pgfscope}%
\pgfsys@transformshift{1.689613in}{1.727196in}%
\pgfsys@useobject{currentmarker}{}%
\end{pgfscope}%
\begin{pgfscope}%
\pgfsys@transformshift{1.667783in}{1.636767in}%
\pgfsys@useobject{currentmarker}{}%
\end{pgfscope}%
\begin{pgfscope}%
\pgfsys@transformshift{1.649239in}{1.541710in}%
\pgfsys@useobject{currentmarker}{}%
\end{pgfscope}%
\begin{pgfscope}%
\pgfsys@transformshift{1.631165in}{1.459918in}%
\pgfsys@useobject{currentmarker}{}%
\end{pgfscope}%
\begin{pgfscope}%
\pgfsys@transformshift{1.610274in}{1.398457in}%
\pgfsys@useobject{currentmarker}{}%
\end{pgfscope}%
\begin{pgfscope}%
\pgfsys@transformshift{1.592670in}{1.362813in}%
\pgfsys@useobject{currentmarker}{}%
\end{pgfscope}%
\begin{pgfscope}%
\pgfsys@transformshift{1.572482in}{1.361589in}%
\pgfsys@useobject{currentmarker}{}%
\end{pgfscope}%
\begin{pgfscope}%
\pgfsys@transformshift{1.553938in}{1.391914in}%
\pgfsys@useobject{currentmarker}{}%
\end{pgfscope}%
\begin{pgfscope}%
\pgfsys@transformshift{1.535161in}{1.436954in}%
\pgfsys@useobject{currentmarker}{}%
\end{pgfscope}%
\begin{pgfscope}%
\pgfsys@transformshift{1.515208in}{1.491050in}%
\pgfsys@useobject{currentmarker}{}%
\end{pgfscope}%
\begin{pgfscope}%
\pgfsys@transformshift{1.495021in}{1.579464in}%
\pgfsys@useobject{currentmarker}{}%
\end{pgfscope}%
\begin{pgfscope}%
\pgfsys@transformshift{1.476478in}{1.693935in}%
\pgfsys@useobject{currentmarker}{}%
\end{pgfscope}%
\begin{pgfscope}%
\pgfsys@transformshift{1.457934in}{1.743131in}%
\pgfsys@useobject{currentmarker}{}%
\end{pgfscope}%
\begin{pgfscope}%
\pgfsys@transformshift{1.437043in}{1.747606in}%
\pgfsys@useobject{currentmarker}{}%
\end{pgfscope}%
\begin{pgfscope}%
\pgfsys@transformshift{1.415684in}{1.695103in}%
\pgfsys@useobject{currentmarker}{}%
\end{pgfscope}%
\begin{pgfscope}%
\pgfsys@transformshift{1.399251in}{1.605765in}%
\pgfsys@useobject{currentmarker}{}%
\end{pgfscope}%
\begin{pgfscope}%
\pgfsys@transformshift{1.379066in}{1.502658in}%
\pgfsys@useobject{currentmarker}{}%
\end{pgfscope}%
\begin{pgfscope}%
\pgfsys@transformshift{1.360756in}{1.467984in}%
\pgfsys@useobject{currentmarker}{}%
\end{pgfscope}%
\begin{pgfscope}%
\pgfsys@transformshift{1.343387in}{1.425444in}%
\pgfsys@useobject{currentmarker}{}%
\end{pgfscope}%
\begin{pgfscope}%
\pgfsys@transformshift{1.321792in}{1.378103in}%
\pgfsys@useobject{currentmarker}{}%
\end{pgfscope}%
\begin{pgfscope}%
\pgfsys@transformshift{1.303482in}{1.357954in}%
\pgfsys@useobject{currentmarker}{}%
\end{pgfscope}%
\begin{pgfscope}%
\pgfsys@transformshift{1.284939in}{1.367247in}%
\pgfsys@useobject{currentmarker}{}%
\end{pgfscope}%
\begin{pgfscope}%
\pgfsys@transformshift{1.264049in}{1.410974in}%
\pgfsys@useobject{currentmarker}{}%
\end{pgfscope}%
\begin{pgfscope}%
\pgfsys@transformshift{1.245739in}{1.469417in}%
\pgfsys@useobject{currentmarker}{}%
\end{pgfscope}%
\begin{pgfscope}%
\pgfsys@transformshift{1.227899in}{1.532283in}%
\pgfsys@useobject{currentmarker}{}%
\end{pgfscope}%
\begin{pgfscope}%
\pgfsys@transformshift{1.209357in}{1.642685in}%
\pgfsys@useobject{currentmarker}{}%
\end{pgfscope}%
\begin{pgfscope}%
\pgfsys@transformshift{1.187056in}{1.733378in}%
\pgfsys@useobject{currentmarker}{}%
\end{pgfscope}%
\begin{pgfscope}%
\pgfsys@transformshift{1.168279in}{1.755611in}%
\pgfsys@useobject{currentmarker}{}%
\end{pgfscope}%
\begin{pgfscope}%
\pgfsys@transformshift{1.148795in}{1.752259in}%
\pgfsys@useobject{currentmarker}{}%
\end{pgfscope}%
\begin{pgfscope}%
\pgfsys@transformshift{1.129313in}{1.710448in}%
\pgfsys@useobject{currentmarker}{}%
\end{pgfscope}%
\begin{pgfscope}%
\pgfsys@transformshift{1.110534in}{1.754630in}%
\pgfsys@useobject{currentmarker}{}%
\end{pgfscope}%
\begin{pgfscope}%
\pgfsys@transformshift{1.088940in}{1.697197in}%
\pgfsys@useobject{currentmarker}{}%
\end{pgfscope}%
\begin{pgfscope}%
\pgfsys@transformshift{1.072744in}{1.616376in}%
\pgfsys@useobject{currentmarker}{}%
\end{pgfscope}%
\begin{pgfscope}%
\pgfsys@transformshift{1.054670in}{1.522615in}%
\pgfsys@useobject{currentmarker}{}%
\end{pgfscope}%
\begin{pgfscope}%
\pgfsys@transformshift{1.032840in}{1.438617in}%
\pgfsys@useobject{currentmarker}{}%
\end{pgfscope}%
\begin{pgfscope}%
\pgfsys@transformshift{1.014530in}{1.406555in}%
\pgfsys@useobject{currentmarker}{}%
\end{pgfscope}%
\begin{pgfscope}%
\pgfsys@transformshift{0.995988in}{1.369050in}%
\pgfsys@useobject{currentmarker}{}%
\end{pgfscope}%
\begin{pgfscope}%
\pgfsys@transformshift{0.977443in}{1.368099in}%
\pgfsys@useobject{currentmarker}{}%
\end{pgfscope}%
\begin{pgfscope}%
\pgfsys@transformshift{0.955144in}{1.409451in}%
\pgfsys@useobject{currentmarker}{}%
\end{pgfscope}%
\begin{pgfscope}%
\pgfsys@transformshift{0.936365in}{1.474925in}%
\pgfsys@useobject{currentmarker}{}%
\end{pgfscope}%
\begin{pgfscope}%
\pgfsys@transformshift{0.917823in}{1.567289in}%
\pgfsys@useobject{currentmarker}{}%
\end{pgfscope}%
\begin{pgfscope}%
\pgfsys@transformshift{0.899747in}{1.665284in}%
\pgfsys@useobject{currentmarker}{}%
\end{pgfscope}%
\begin{pgfscope}%
\pgfsys@transformshift{0.881674in}{1.737941in}%
\pgfsys@useobject{currentmarker}{}%
\end{pgfscope}%
\begin{pgfscope}%
\pgfsys@transformshift{0.862192in}{1.763619in}%
\pgfsys@useobject{currentmarker}{}%
\end{pgfscope}%
\begin{pgfscope}%
\pgfsys@transformshift{0.840596in}{1.752036in}%
\pgfsys@useobject{currentmarker}{}%
\end{pgfscope}%
\begin{pgfscope}%
\pgfsys@transformshift{0.822757in}{1.716037in}%
\pgfsys@useobject{currentmarker}{}%
\end{pgfscope}%
\begin{pgfscope}%
\pgfsys@transformshift{0.804212in}{1.613590in}%
\pgfsys@useobject{currentmarker}{}%
\end{pgfscope}%
\begin{pgfscope}%
\pgfsys@transformshift{0.786373in}{1.530603in}%
\pgfsys@useobject{currentmarker}{}%
\end{pgfscope}%
\begin{pgfscope}%
\pgfsys@transformshift{0.764308in}{1.453481in}%
\pgfsys@useobject{currentmarker}{}%
\end{pgfscope}%
\begin{pgfscope}%
\pgfsys@transformshift{0.745766in}{1.410077in}%
\pgfsys@useobject{currentmarker}{}%
\end{pgfscope}%
\begin{pgfscope}%
\pgfsys@transformshift{0.727456in}{1.392370in}%
\pgfsys@useobject{currentmarker}{}%
\end{pgfscope}%
\begin{pgfscope}%
\pgfsys@transformshift{0.710087in}{1.371119in}%
\pgfsys@useobject{currentmarker}{}%
\end{pgfscope}%
\begin{pgfscope}%
\pgfsys@transformshift{0.688726in}{1.376640in}%
\pgfsys@useobject{currentmarker}{}%
\end{pgfscope}%
\begin{pgfscope}%
\pgfsys@transformshift{0.665253in}{1.433448in}%
\pgfsys@useobject{currentmarker}{}%
\end{pgfscope}%
\begin{pgfscope}%
\pgfsys@transformshift{0.650934in}{1.487539in}%
\pgfsys@useobject{currentmarker}{}%
\end{pgfscope}%
\begin{pgfscope}%
\pgfsys@transformshift{0.649762in}{1.495206in}%
\pgfsys@useobject{currentmarker}{}%
\end{pgfscope}%
\begin{pgfscope}%
\pgfsys@transformshift{0.654456in}{1.461072in}%
\pgfsys@useobject{currentmarker}{}%
\end{pgfscope}%
\begin{pgfscope}%
\pgfsys@transformshift{0.676990in}{1.382714in}%
\pgfsys@useobject{currentmarker}{}%
\end{pgfscope}%
\begin{pgfscope}%
\pgfsys@transformshift{0.695063in}{1.375295in}%
\pgfsys@useobject{currentmarker}{}%
\end{pgfscope}%
\begin{pgfscope}%
\pgfsys@transformshift{0.712904in}{1.426433in}%
\pgfsys@useobject{currentmarker}{}%
\end{pgfscope}%
\begin{pgfscope}%
\pgfsys@transformshift{0.734029in}{1.528475in}%
\pgfsys@useobject{currentmarker}{}%
\end{pgfscope}%
\begin{pgfscope}%
\pgfsys@transformshift{0.752572in}{1.659759in}%
\pgfsys@useobject{currentmarker}{}%
\end{pgfscope}%
\begin{pgfscope}%
\pgfsys@transformshift{0.773228in}{1.759083in}%
\pgfsys@useobject{currentmarker}{}%
\end{pgfscope}%
\begin{pgfscope}%
\pgfsys@transformshift{0.792007in}{1.755847in}%
\pgfsys@useobject{currentmarker}{}%
\end{pgfscope}%
\begin{pgfscope}%
\pgfsys@transformshift{0.809846in}{1.674321in}%
\pgfsys@useobject{currentmarker}{}%
\end{pgfscope}%
\begin{pgfscope}%
\pgfsys@transformshift{0.830738in}{1.518007in}%
\pgfsys@useobject{currentmarker}{}%
\end{pgfscope}%
\begin{pgfscope}%
\pgfsys@transformshift{0.830973in}{1.454949in}%
\pgfsys@useobject{currentmarker}{}%
\end{pgfscope}%
\begin{pgfscope}%
\pgfsys@transformshift{0.847169in}{1.429654in}%
\pgfsys@useobject{currentmarker}{}%
\end{pgfscope}%
\begin{pgfscope}%
\pgfsys@transformshift{0.867120in}{1.370053in}%
\pgfsys@useobject{currentmarker}{}%
\end{pgfscope}%
\begin{pgfscope}%
\pgfsys@transformshift{0.888482in}{1.384935in}%
\pgfsys@useobject{currentmarker}{}%
\end{pgfscope}%
\begin{pgfscope}%
\pgfsys@transformshift{0.906790in}{1.448096in}%
\pgfsys@useobject{currentmarker}{}%
\end{pgfscope}%
\begin{pgfscope}%
\pgfsys@transformshift{0.924629in}{1.543185in}%
\pgfsys@useobject{currentmarker}{}%
\end{pgfscope}%
\begin{pgfscope}%
\pgfsys@transformshift{0.945285in}{1.698349in}%
\pgfsys@useobject{currentmarker}{}%
\end{pgfscope}%
\begin{pgfscope}%
\pgfsys@transformshift{0.963595in}{1.757091in}%
\pgfsys@useobject{currentmarker}{}%
\end{pgfscope}%
\begin{pgfscope}%
\pgfsys@transformshift{0.984954in}{1.724948in}%
\pgfsys@useobject{currentmarker}{}%
\end{pgfscope}%
\begin{pgfscope}%
\pgfsys@transformshift{1.005142in}{1.575447in}%
\pgfsys@useobject{currentmarker}{}%
\end{pgfscope}%
\begin{pgfscope}%
\pgfsys@transformshift{1.020633in}{1.470323in}%
\pgfsys@useobject{currentmarker}{}%
\end{pgfscope}%
\begin{pgfscope}%
\pgfsys@transformshift{1.043637in}{1.380444in}%
\pgfsys@useobject{currentmarker}{}%
\end{pgfscope}%
\begin{pgfscope}%
\pgfsys@transformshift{1.060068in}{1.360029in}%
\pgfsys@useobject{currentmarker}{}%
\end{pgfscope}%
\begin{pgfscope}%
\pgfsys@transformshift{1.080021in}{1.402997in}%
\pgfsys@useobject{currentmarker}{}%
\end{pgfscope}%
\begin{pgfscope}%
\pgfsys@transformshift{1.095983in}{1.466344in}%
\pgfsys@useobject{currentmarker}{}%
\end{pgfscope}%
\begin{pgfscope}%
\pgfsys@transformshift{1.118047in}{1.604883in}%
\pgfsys@useobject{currentmarker}{}%
\end{pgfscope}%
\begin{pgfscope}%
\pgfsys@transformshift{1.137529in}{1.709807in}%
\pgfsys@useobject{currentmarker}{}%
\end{pgfscope}%
\begin{pgfscope}%
\pgfsys@transformshift{1.156308in}{1.752990in}%
\pgfsys@useobject{currentmarker}{}%
\end{pgfscope}%
\begin{pgfscope}%
\pgfsys@transformshift{1.172973in}{1.720654in}%
\pgfsys@useobject{currentmarker}{}%
\end{pgfscope}%
\begin{pgfscope}%
\pgfsys@transformshift{1.195272in}{1.567912in}%
\pgfsys@useobject{currentmarker}{}%
\end{pgfscope}%
\begin{pgfscope}%
\pgfsys@transformshift{1.214989in}{1.447386in}%
\pgfsys@useobject{currentmarker}{}%
\end{pgfscope}%
\begin{pgfscope}%
\pgfsys@transformshift{1.234237in}{1.381720in}%
\pgfsys@useobject{currentmarker}{}%
\end{pgfscope}%
\begin{pgfscope}%
\pgfsys@transformshift{1.254424in}{1.354203in}%
\pgfsys@useobject{currentmarker}{}%
\end{pgfscope}%
\begin{pgfscope}%
\pgfsys@transformshift{1.270620in}{1.380252in}%
\pgfsys@useobject{currentmarker}{}%
\end{pgfscope}%
\begin{pgfscope}%
\pgfsys@transformshift{1.291276in}{1.441035in}%
\pgfsys@useobject{currentmarker}{}%
\end{pgfscope}%
\begin{pgfscope}%
\pgfsys@transformshift{1.312167in}{1.548033in}%
\pgfsys@useobject{currentmarker}{}%
\end{pgfscope}%
\begin{pgfscope}%
\pgfsys@transformshift{1.329303in}{1.686516in}%
\pgfsys@useobject{currentmarker}{}%
\end{pgfscope}%
\begin{pgfscope}%
\pgfsys@transformshift{1.347847in}{1.746346in}%
\pgfsys@useobject{currentmarker}{}%
\end{pgfscope}%
\begin{pgfscope}%
\pgfsys@transformshift{1.368738in}{1.717273in}%
\pgfsys@useobject{currentmarker}{}%
\end{pgfscope}%
\begin{pgfscope}%
\pgfsys@transformshift{1.386106in}{1.732843in}%
\pgfsys@useobject{currentmarker}{}%
\end{pgfscope}%
\begin{pgfscope}%
\pgfsys@transformshift{1.405825in}{1.694811in}%
\pgfsys@useobject{currentmarker}{}%
\end{pgfscope}%
\begin{pgfscope}%
\pgfsys@transformshift{1.424604in}{1.590765in}%
\pgfsys@useobject{currentmarker}{}%
\end{pgfscope}%
\begin{pgfscope}%
\pgfsys@transformshift{1.443617in}{1.460006in}%
\pgfsys@useobject{currentmarker}{}%
\end{pgfscope}%
\begin{pgfscope}%
\pgfsys@transformshift{1.462394in}{1.385940in}%
\pgfsys@useobject{currentmarker}{}%
\end{pgfscope}%
\begin{pgfscope}%
\pgfsys@transformshift{1.482581in}{1.352351in}%
\pgfsys@useobject{currentmarker}{}%
\end{pgfscope}%
\begin{pgfscope}%
\pgfsys@transformshift{1.502534in}{1.378768in}%
\pgfsys@useobject{currentmarker}{}%
\end{pgfscope}%
\begin{pgfscope}%
\pgfsys@transformshift{1.523190in}{1.443972in}%
\pgfsys@useobject{currentmarker}{}%
\end{pgfscope}%
\begin{pgfscope}%
\pgfsys@transformshift{1.542438in}{1.550361in}%
\pgfsys@useobject{currentmarker}{}%
\end{pgfscope}%
\begin{pgfscope}%
\pgfsys@transformshift{1.564971in}{1.687108in}%
\pgfsys@useobject{currentmarker}{}%
\end{pgfscope}%
\begin{pgfscope}%
\pgfsys@transformshift{1.580228in}{1.739678in}%
\pgfsys@useobject{currentmarker}{}%
\end{pgfscope}%
\begin{pgfscope}%
\pgfsys@transformshift{1.599241in}{1.727092in}%
\pgfsys@useobject{currentmarker}{}%
\end{pgfscope}%
\begin{pgfscope}%
\pgfsys@transformshift{1.621071in}{1.601105in}%
\pgfsys@useobject{currentmarker}{}%
\end{pgfscope}%
\begin{pgfscope}%
\pgfsys@transformshift{1.636328in}{1.492594in}%
\pgfsys@useobject{currentmarker}{}%
\end{pgfscope}%
\begin{pgfscope}%
\pgfsys@transformshift{1.658627in}{1.394678in}%
\pgfsys@useobject{currentmarker}{}%
\end{pgfscope}%
\begin{pgfscope}%
\pgfsys@transformshift{1.674823in}{1.358851in}%
\pgfsys@useobject{currentmarker}{}%
\end{pgfscope}%
\begin{pgfscope}%
\pgfsys@transformshift{1.693602in}{1.350461in}%
\pgfsys@useobject{currentmarker}{}%
\end{pgfscope}%
\begin{pgfscope}%
\pgfsys@transformshift{1.712381in}{1.382422in}%
\pgfsys@useobject{currentmarker}{}%
\end{pgfscope}%
\begin{pgfscope}%
\pgfsys@transformshift{1.731160in}{1.420790in}%
\pgfsys@useobject{currentmarker}{}%
\end{pgfscope}%
\begin{pgfscope}%
\pgfsys@transformshift{1.749937in}{1.516290in}%
\pgfsys@useobject{currentmarker}{}%
\end{pgfscope}%
\begin{pgfscope}%
\pgfsys@transformshift{1.772707in}{1.642778in}%
\pgfsys@useobject{currentmarker}{}%
\end{pgfscope}%
\begin{pgfscope}%
\pgfsys@transformshift{1.791249in}{1.731057in}%
\pgfsys@useobject{currentmarker}{}%
\end{pgfscope}%
\begin{pgfscope}%
\pgfsys@transformshift{1.810733in}{1.734936in}%
\pgfsys@useobject{currentmarker}{}%
\end{pgfscope}%
\begin{pgfscope}%
\pgfsys@transformshift{1.828807in}{1.670289in}%
\pgfsys@useobject{currentmarker}{}%
\end{pgfscope}%
\begin{pgfscope}%
\pgfsys@transformshift{1.848523in}{1.527894in}%
\pgfsys@useobject{currentmarker}{}%
\end{pgfscope}%
\begin{pgfscope}%
\pgfsys@transformshift{1.867771in}{1.425884in}%
\pgfsys@useobject{currentmarker}{}%
\end{pgfscope}%
\begin{pgfscope}%
\pgfsys@transformshift{1.885612in}{1.366353in}%
\pgfsys@useobject{currentmarker}{}%
\end{pgfscope}%
\begin{pgfscope}%
\pgfsys@transformshift{1.905563in}{1.350252in}%
\pgfsys@useobject{currentmarker}{}%
\end{pgfscope}%
\begin{pgfscope}%
\pgfsys@transformshift{1.924342in}{1.370276in}%
\pgfsys@useobject{currentmarker}{}%
\end{pgfscope}%
\begin{pgfscope}%
\pgfsys@transformshift{1.944764in}{1.417017in}%
\pgfsys@useobject{currentmarker}{}%
\end{pgfscope}%
\begin{pgfscope}%
\pgfsys@transformshift{1.963777in}{1.500853in}%
\pgfsys@useobject{currentmarker}{}%
\end{pgfscope}%
\begin{pgfscope}%
\pgfsys@transformshift{1.982319in}{1.623102in}%
\pgfsys@useobject{currentmarker}{}%
\end{pgfscope}%
\begin{pgfscope}%
\pgfsys@transformshift{2.001567in}{1.719190in}%
\pgfsys@useobject{currentmarker}{}%
\end{pgfscope}%
\begin{pgfscope}%
\pgfsys@transformshift{2.020111in}{1.738452in}%
\pgfsys@useobject{currentmarker}{}%
\end{pgfscope}%
\begin{pgfscope}%
\pgfsys@transformshift{2.042176in}{1.679537in}%
\pgfsys@useobject{currentmarker}{}%
\end{pgfscope}%
\begin{pgfscope}%
\pgfsys@transformshift{2.058138in}{1.578386in}%
\pgfsys@useobject{currentmarker}{}%
\end{pgfscope}%
\begin{pgfscope}%
\pgfsys@transformshift{2.078560in}{1.446454in}%
\pgfsys@useobject{currentmarker}{}%
\end{pgfscope}%
\begin{pgfscope}%
\pgfsys@transformshift{2.098276in}{1.386551in}%
\pgfsys@useobject{currentmarker}{}%
\end{pgfscope}%
\begin{pgfscope}%
\pgfsys@transformshift{2.116821in}{1.352590in}%
\pgfsys@useobject{currentmarker}{}%
\end{pgfscope}%
\begin{pgfscope}%
\pgfsys@transformshift{2.136537in}{1.354908in}%
\pgfsys@useobject{currentmarker}{}%
\end{pgfscope}%
\begin{pgfscope}%
\pgfsys@transformshift{2.157428in}{1.398217in}%
\pgfsys@useobject{currentmarker}{}%
\end{pgfscope}%
\begin{pgfscope}%
\pgfsys@transformshift{2.175972in}{1.449854in}%
\pgfsys@useobject{currentmarker}{}%
\end{pgfscope}%
\begin{pgfscope}%
\pgfsys@transformshift{2.194280in}{1.542035in}%
\pgfsys@useobject{currentmarker}{}%
\end{pgfscope}%
\begin{pgfscope}%
\pgfsys@transformshift{2.212825in}{1.664219in}%
\pgfsys@useobject{currentmarker}{}%
\end{pgfscope}%
\begin{pgfscope}%
\pgfsys@transformshift{2.231367in}{1.729563in}%
\pgfsys@useobject{currentmarker}{}%
\end{pgfscope}%
\begin{pgfscope}%
\pgfsys@transformshift{2.252023in}{1.731328in}%
\pgfsys@useobject{currentmarker}{}%
\end{pgfscope}%
\begin{pgfscope}%
\pgfsys@transformshift{2.271976in}{1.650945in}%
\pgfsys@useobject{currentmarker}{}%
\end{pgfscope}%
\begin{pgfscope}%
\pgfsys@transformshift{2.290519in}{1.522241in}%
\pgfsys@useobject{currentmarker}{}%
\end{pgfscope}%
\begin{pgfscope}%
\pgfsys@transformshift{2.311411in}{1.426144in}%
\pgfsys@useobject{currentmarker}{}%
\end{pgfscope}%
\begin{pgfscope}%
\pgfsys@transformshift{2.330188in}{1.379292in}%
\pgfsys@useobject{currentmarker}{}%
\end{pgfscope}%
\begin{pgfscope}%
\pgfsys@transformshift{2.346619in}{1.351386in}%
\pgfsys@useobject{currentmarker}{}%
\end{pgfscope}%
\begin{pgfscope}%
\pgfsys@transformshift{2.368215in}{1.427866in}%
\pgfsys@useobject{currentmarker}{}%
\end{pgfscope}%
\begin{pgfscope}%
\pgfsys@transformshift{2.385819in}{1.376440in}%
\pgfsys@useobject{currentmarker}{}%
\end{pgfscope}%
\begin{pgfscope}%
\pgfsys@transformshift{2.385585in}{1.356194in}%
\pgfsys@useobject{currentmarker}{}%
\end{pgfscope}%
\begin{pgfscope}%
\pgfsys@transformshift{2.407415in}{1.348746in}%
\pgfsys@useobject{currentmarker}{}%
\end{pgfscope}%
\begin{pgfscope}%
\pgfsys@transformshift{2.425489in}{1.369618in}%
\pgfsys@useobject{currentmarker}{}%
\end{pgfscope}%
\begin{pgfscope}%
\pgfsys@transformshift{2.442390in}{1.415812in}%
\pgfsys@useobject{currentmarker}{}%
\end{pgfscope}%
\begin{pgfscope}%
\pgfsys@transformshift{2.460698in}{1.482124in}%
\pgfsys@useobject{currentmarker}{}%
\end{pgfscope}%
\begin{pgfscope}%
\pgfsys@transformshift{2.482529in}{1.604619in}%
\pgfsys@useobject{currentmarker}{}%
\end{pgfscope}%
\begin{pgfscope}%
\pgfsys@transformshift{2.502479in}{1.713976in}%
\pgfsys@useobject{currentmarker}{}%
\end{pgfscope}%
\begin{pgfscope}%
\pgfsys@transformshift{2.520555in}{1.738890in}%
\pgfsys@useobject{currentmarker}{}%
\end{pgfscope}%
\begin{pgfscope}%
\pgfsys@transformshift{2.538863in}{1.702286in}%
\pgfsys@useobject{currentmarker}{}%
\end{pgfscope}%
\begin{pgfscope}%
\pgfsys@transformshift{2.558816in}{1.572592in}%
\pgfsys@useobject{currentmarker}{}%
\end{pgfscope}%
\begin{pgfscope}%
\pgfsys@transformshift{2.577358in}{1.464798in}%
\pgfsys@useobject{currentmarker}{}%
\end{pgfscope}%
\begin{pgfscope}%
\pgfsys@transformshift{2.598720in}{1.392043in}%
\pgfsys@useobject{currentmarker}{}%
\end{pgfscope}%
\begin{pgfscope}%
\pgfsys@transformshift{2.616325in}{1.357547in}%
\pgfsys@useobject{currentmarker}{}%
\end{pgfscope}%
\begin{pgfscope}%
\pgfsys@transformshift{2.634164in}{1.351786in}%
\pgfsys@useobject{currentmarker}{}%
\end{pgfscope}%
\begin{pgfscope}%
\pgfsys@transformshift{2.655054in}{1.387157in}%
\pgfsys@useobject{currentmarker}{}%
\end{pgfscope}%
\begin{pgfscope}%
\pgfsys@transformshift{2.672425in}{1.437301in}%
\pgfsys@useobject{currentmarker}{}%
\end{pgfscope}%
\begin{pgfscope}%
\pgfsys@transformshift{2.692610in}{1.487558in}%
\pgfsys@useobject{currentmarker}{}%
\end{pgfscope}%
\begin{pgfscope}%
\pgfsys@transformshift{2.711623in}{1.579105in}%
\pgfsys@useobject{currentmarker}{}%
\end{pgfscope}%
\begin{pgfscope}%
\pgfsys@transformshift{2.732514in}{1.688306in}%
\pgfsys@useobject{currentmarker}{}%
\end{pgfscope}%
\begin{pgfscope}%
\pgfsys@transformshift{2.750355in}{1.734945in}%
\pgfsys@useobject{currentmarker}{}%
\end{pgfscope}%
\begin{pgfscope}%
\pgfsys@transformshift{2.771480in}{1.721237in}%
\pgfsys@useobject{currentmarker}{}%
\end{pgfscope}%
\begin{pgfscope}%
\pgfsys@transformshift{2.788616in}{1.635645in}%
\pgfsys@useobject{currentmarker}{}%
\end{pgfscope}%
\begin{pgfscope}%
\pgfsys@transformshift{2.806219in}{1.513192in}%
\pgfsys@useobject{currentmarker}{}%
\end{pgfscope}%
\begin{pgfscope}%
\pgfsys@transformshift{2.828049in}{1.447008in}%
\pgfsys@useobject{currentmarker}{}%
\end{pgfscope}%
\begin{pgfscope}%
\pgfsys@transformshift{2.849879in}{1.383453in}%
\pgfsys@useobject{currentmarker}{}%
\end{pgfscope}%
\begin{pgfscope}%
\pgfsys@transformshift{2.866781in}{1.352693in}%
\pgfsys@useobject{currentmarker}{}%
\end{pgfscope}%
\begin{pgfscope}%
\pgfsys@transformshift{2.885089in}{1.359489in}%
\pgfsys@useobject{currentmarker}{}%
\end{pgfscope}%
\begin{pgfscope}%
\pgfsys@transformshift{2.906214in}{1.399345in}%
\pgfsys@useobject{currentmarker}{}%
\end{pgfscope}%
\begin{pgfscope}%
\pgfsys@transformshift{2.923819in}{1.430078in}%
\pgfsys@useobject{currentmarker}{}%
\end{pgfscope}%
\begin{pgfscope}%
\pgfsys@transformshift{2.941894in}{1.513118in}%
\pgfsys@useobject{currentmarker}{}%
\end{pgfscope}%
\begin{pgfscope}%
\pgfsys@transformshift{2.961845in}{1.648926in}%
\pgfsys@useobject{currentmarker}{}%
\end{pgfscope}%
\begin{pgfscope}%
\pgfsys@transformshift{2.980390in}{1.725247in}%
\pgfsys@useobject{currentmarker}{}%
\end{pgfscope}%
\begin{pgfscope}%
\pgfsys@transformshift{3.001046in}{1.741166in}%
\pgfsys@useobject{currentmarker}{}%
\end{pgfscope}%
\begin{pgfscope}%
\pgfsys@transformshift{3.019354in}{1.712709in}%
\pgfsys@useobject{currentmarker}{}%
\end{pgfscope}%
\begin{pgfscope}%
\pgfsys@transformshift{3.041184in}{1.577852in}%
\pgfsys@useobject{currentmarker}{}%
\end{pgfscope}%
\begin{pgfscope}%
\pgfsys@transformshift{3.056206in}{1.485808in}%
\pgfsys@useobject{currentmarker}{}%
\end{pgfscope}%
\begin{pgfscope}%
\pgfsys@transformshift{3.076862in}{1.428187in}%
\pgfsys@useobject{currentmarker}{}%
\end{pgfscope}%
\begin{pgfscope}%
\pgfsys@transformshift{3.099161in}{1.373643in}%
\pgfsys@useobject{currentmarker}{}%
\end{pgfscope}%
\begin{pgfscope}%
\pgfsys@transformshift{3.115592in}{1.351707in}%
\pgfsys@useobject{currentmarker}{}%
\end{pgfscope}%
\begin{pgfscope}%
\pgfsys@transformshift{3.136954in}{1.373061in}%
\pgfsys@useobject{currentmarker}{}%
\end{pgfscope}%
\begin{pgfscope}%
\pgfsys@transformshift{3.154090in}{1.410490in}%
\pgfsys@useobject{currentmarker}{}%
\end{pgfscope}%
\begin{pgfscope}%
\pgfsys@transformshift{3.173337in}{1.467504in}%
\pgfsys@useobject{currentmarker}{}%
\end{pgfscope}%
\begin{pgfscope}%
\pgfsys@transformshift{3.193288in}{1.553962in}%
\pgfsys@useobject{currentmarker}{}%
\end{pgfscope}%
\begin{pgfscope}%
\pgfsys@transformshift{3.211833in}{1.653716in}%
\pgfsys@useobject{currentmarker}{}%
\end{pgfscope}%
\begin{pgfscope}%
\pgfsys@transformshift{3.231549in}{1.734879in}%
\pgfsys@useobject{currentmarker}{}%
\end{pgfscope}%
\begin{pgfscope}%
\pgfsys@transformshift{3.250562in}{1.744806in}%
\pgfsys@useobject{currentmarker}{}%
\end{pgfscope}%
\begin{pgfscope}%
\pgfsys@transformshift{3.267933in}{1.715462in}%
\pgfsys@useobject{currentmarker}{}%
\end{pgfscope}%
\begin{pgfscope}%
\pgfsys@transformshift{3.288589in}{1.612879in}%
\pgfsys@useobject{currentmarker}{}%
\end{pgfscope}%
\begin{pgfscope}%
\pgfsys@transformshift{3.309245in}{1.486287in}%
\pgfsys@useobject{currentmarker}{}%
\end{pgfscope}%
\begin{pgfscope}%
\pgfsys@transformshift{3.327789in}{1.431864in}%
\pgfsys@useobject{currentmarker}{}%
\end{pgfscope}%
\begin{pgfscope}%
\pgfsys@transformshift{3.346098in}{1.383924in}%
\pgfsys@useobject{currentmarker}{}%
\end{pgfscope}%
\begin{pgfscope}%
\pgfsys@transformshift{3.363937in}{1.357703in}%
\pgfsys@useobject{currentmarker}{}%
\end{pgfscope}%
\begin{pgfscope}%
\pgfsys@transformshift{3.385298in}{1.362313in}%
\pgfsys@useobject{currentmarker}{}%
\end{pgfscope}%
\begin{pgfscope}%
\pgfsys@transformshift{3.408769in}{1.395655in}%
\pgfsys@useobject{currentmarker}{}%
\end{pgfscope}%
\begin{pgfscope}%
\pgfsys@transformshift{3.422854in}{1.432513in}%
\pgfsys@useobject{currentmarker}{}%
\end{pgfscope}%
\begin{pgfscope}%
\pgfsys@transformshift{3.443041in}{1.496275in}%
\pgfsys@useobject{currentmarker}{}%
\end{pgfscope}%
\begin{pgfscope}%
\pgfsys@transformshift{3.462523in}{1.582590in}%
\pgfsys@useobject{currentmarker}{}%
\end{pgfscope}%
\begin{pgfscope}%
\pgfsys@transformshift{3.481536in}{1.686016in}%
\pgfsys@useobject{currentmarker}{}%
\end{pgfscope}%
\begin{pgfscope}%
\pgfsys@transformshift{3.499610in}{1.741272in}%
\pgfsys@useobject{currentmarker}{}%
\end{pgfscope}%
\begin{pgfscope}%
\pgfsys@transformshift{3.519563in}{1.745729in}%
\pgfsys@useobject{currentmarker}{}%
\end{pgfscope}%
\begin{pgfscope}%
\pgfsys@transformshift{3.539514in}{1.718260in}%
\pgfsys@useobject{currentmarker}{}%
\end{pgfscope}%
\begin{pgfscope}%
\pgfsys@transformshift{3.555476in}{1.644101in}%
\pgfsys@useobject{currentmarker}{}%
\end{pgfscope}%
\begin{pgfscope}%
\pgfsys@transformshift{3.579418in}{1.525199in}%
\pgfsys@useobject{currentmarker}{}%
\end{pgfscope}%
\begin{pgfscope}%
\pgfsys@transformshift{3.597962in}{1.462599in}%
\pgfsys@useobject{currentmarker}{}%
\end{pgfscope}%
\begin{pgfscope}%
\pgfsys@transformshift{3.616505in}{1.415926in}%
\pgfsys@useobject{currentmarker}{}%
\end{pgfscope}%
\begin{pgfscope}%
\pgfsys@transformshift{3.634109in}{1.385596in}%
\pgfsys@useobject{currentmarker}{}%
\end{pgfscope}%
\begin{pgfscope}%
\pgfsys@transformshift{3.656174in}{1.358545in}%
\pgfsys@useobject{currentmarker}{}%
\end{pgfscope}%
\begin{pgfscope}%
\pgfsys@transformshift{3.673310in}{1.380144in}%
\pgfsys@useobject{currentmarker}{}%
\end{pgfscope}%
\begin{pgfscope}%
\pgfsys@transformshift{3.690680in}{1.418415in}%
\pgfsys@useobject{currentmarker}{}%
\end{pgfscope}%
\begin{pgfscope}%
\pgfsys@transformshift{3.712511in}{1.487098in}%
\pgfsys@useobject{currentmarker}{}%
\end{pgfscope}%
\begin{pgfscope}%
\pgfsys@transformshift{3.729879in}{1.551716in}%
\pgfsys@useobject{currentmarker}{}%
\end{pgfscope}%
\begin{pgfscope}%
\pgfsys@transformshift{3.754526in}{1.680370in}%
\pgfsys@useobject{currentmarker}{}%
\end{pgfscope}%
\begin{pgfscope}%
\pgfsys@transformshift{3.771193in}{1.736227in}%
\pgfsys@useobject{currentmarker}{}%
\end{pgfscope}%
\begin{pgfscope}%
\pgfsys@transformshift{3.790205in}{1.745095in}%
\pgfsys@useobject{currentmarker}{}%
\end{pgfscope}%
\begin{pgfscope}%
\pgfsys@transformshift{3.807341in}{1.754923in}%
\pgfsys@useobject{currentmarker}{}%
\end{pgfscope}%
\begin{pgfscope}%
\pgfsys@transformshift{3.826354in}{1.722092in}%
\pgfsys@useobject{currentmarker}{}%
\end{pgfscope}%
\begin{pgfscope}%
\pgfsys@transformshift{3.846776in}{1.619245in}%
\pgfsys@useobject{currentmarker}{}%
\end{pgfscope}%
\begin{pgfscope}%
\pgfsys@transformshift{3.864615in}{1.528064in}%
\pgfsys@useobject{currentmarker}{}%
\end{pgfscope}%
\begin{pgfscope}%
\pgfsys@transformshift{3.884097in}{1.485313in}%
\pgfsys@useobject{currentmarker}{}%
\end{pgfscope}%
\begin{pgfscope}%
\pgfsys@transformshift{3.901233in}{1.415359in}%
\pgfsys@useobject{currentmarker}{}%
\end{pgfscope}%
\begin{pgfscope}%
\pgfsys@transformshift{3.926349in}{1.445880in}%
\pgfsys@useobject{currentmarker}{}%
\end{pgfscope}%
\begin{pgfscope}%
\pgfsys@transformshift{3.942545in}{1.391058in}%
\pgfsys@useobject{currentmarker}{}%
\end{pgfscope}%
\begin{pgfscope}%
\pgfsys@transformshift{3.961324in}{1.361897in}%
\pgfsys@useobject{currentmarker}{}%
\end{pgfscope}%
\begin{pgfscope}%
\pgfsys@transformshift{3.983152in}{1.364388in}%
\pgfsys@useobject{currentmarker}{}%
\end{pgfscope}%
\begin{pgfscope}%
\pgfsys@transformshift{3.999585in}{1.399998in}%
\pgfsys@useobject{currentmarker}{}%
\end{pgfscope}%
\begin{pgfscope}%
\pgfsys@transformshift{4.018596in}{1.452764in}%
\pgfsys@useobject{currentmarker}{}%
\end{pgfscope}%
\begin{pgfscope}%
\pgfsys@transformshift{4.036672in}{1.514832in}%
\pgfsys@useobject{currentmarker}{}%
\end{pgfscope}%
\begin{pgfscope}%
\pgfsys@transformshift{4.058502in}{1.635279in}%
\pgfsys@useobject{currentmarker}{}%
\end{pgfscope}%
\begin{pgfscope}%
\pgfsys@transformshift{4.076576in}{1.697305in}%
\pgfsys@useobject{currentmarker}{}%
\end{pgfscope}%
\begin{pgfscope}%
\pgfsys@transformshift{4.096527in}{1.755907in}%
\pgfsys@useobject{currentmarker}{}%
\end{pgfscope}%
\begin{pgfscope}%
\pgfsys@transformshift{4.114366in}{1.761774in}%
\pgfsys@useobject{currentmarker}{}%
\end{pgfscope}%
\begin{pgfscope}%
\pgfsys@transformshift{4.135493in}{1.716235in}%
\pgfsys@useobject{currentmarker}{}%
\end{pgfscope}%
\begin{pgfscope}%
\pgfsys@transformshift{4.154506in}{1.622870in}%
\pgfsys@useobject{currentmarker}{}%
\end{pgfscope}%
\begin{pgfscope}%
\pgfsys@transformshift{4.171171in}{1.555294in}%
\pgfsys@useobject{currentmarker}{}%
\end{pgfscope}%
\begin{pgfscope}%
\pgfsys@transformshift{4.192767in}{1.470190in}%
\pgfsys@useobject{currentmarker}{}%
\end{pgfscope}%
\begin{pgfscope}%
\pgfsys@transformshift{4.209901in}{1.408149in}%
\pgfsys@useobject{currentmarker}{}%
\end{pgfscope}%
\begin{pgfscope}%
\pgfsys@transformshift{4.231731in}{1.373723in}%
\pgfsys@useobject{currentmarker}{}%
\end{pgfscope}%
\begin{pgfscope}%
\pgfsys@transformshift{4.250510in}{1.368377in}%
\pgfsys@useobject{currentmarker}{}%
\end{pgfscope}%
\begin{pgfscope}%
\pgfsys@transformshift{4.267880in}{1.390001in}%
\pgfsys@useobject{currentmarker}{}%
\end{pgfscope}%
\begin{pgfscope}%
\pgfsys@transformshift{4.285014in}{1.433505in}%
\pgfsys@useobject{currentmarker}{}%
\end{pgfscope}%
\begin{pgfscope}%
\pgfsys@transformshift{4.306141in}{1.502516in}%
\pgfsys@useobject{currentmarker}{}%
\end{pgfscope}%
\begin{pgfscope}%
\pgfsys@transformshift{4.324918in}{1.579851in}%
\pgfsys@useobject{currentmarker}{}%
\end{pgfscope}%
\begin{pgfscope}%
\pgfsys@transformshift{4.346279in}{1.696701in}%
\pgfsys@useobject{currentmarker}{}%
\end{pgfscope}%
\begin{pgfscope}%
\pgfsys@transformshift{4.364119in}{1.741828in}%
\pgfsys@useobject{currentmarker}{}%
\end{pgfscope}%
\begin{pgfscope}%
\pgfsys@transformshift{4.384775in}{1.771689in}%
\pgfsys@useobject{currentmarker}{}%
\end{pgfscope}%
\begin{pgfscope}%
\pgfsys@transformshift{4.402614in}{1.758694in}%
\pgfsys@useobject{currentmarker}{}%
\end{pgfscope}%
\begin{pgfscope}%
\pgfsys@transformshift{4.421393in}{1.707903in}%
\pgfsys@useobject{currentmarker}{}%
\end{pgfscope}%
\begin{pgfscope}%
\pgfsys@transformshift{4.441580in}{1.627474in}%
\pgfsys@useobject{currentmarker}{}%
\end{pgfscope}%
\begin{pgfscope}%
\pgfsys@transformshift{4.460828in}{1.535315in}%
\pgfsys@useobject{currentmarker}{}%
\end{pgfscope}%
\begin{pgfscope}%
\pgfsys@transformshift{4.481953in}{1.448117in}%
\pgfsys@useobject{currentmarker}{}%
\end{pgfscope}%
\begin{pgfscope}%
\pgfsys@transformshift{4.482422in}{1.450703in}%
\pgfsys@useobject{currentmarker}{}%
\end{pgfscope}%
\begin{pgfscope}%
\pgfsys@transformshift{4.475145in}{1.476522in}%
\pgfsys@useobject{currentmarker}{}%
\end{pgfscope}%
\begin{pgfscope}%
\pgfsys@transformshift{4.453786in}{1.636499in}%
\pgfsys@useobject{currentmarker}{}%
\end{pgfscope}%
\begin{pgfscope}%
\pgfsys@transformshift{4.435007in}{1.742639in}%
\pgfsys@useobject{currentmarker}{}%
\end{pgfscope}%
\begin{pgfscope}%
\pgfsys@transformshift{4.416464in}{1.772407in}%
\pgfsys@useobject{currentmarker}{}%
\end{pgfscope}%
\begin{pgfscope}%
\pgfsys@transformshift{4.397215in}{1.725619in}%
\pgfsys@useobject{currentmarker}{}%
\end{pgfscope}%
\begin{pgfscope}%
\pgfsys@transformshift{4.377029in}{1.577432in}%
\pgfsys@useobject{currentmarker}{}%
\end{pgfscope}%
\begin{pgfscope}%
\pgfsys@transformshift{4.357547in}{1.462801in}%
\pgfsys@useobject{currentmarker}{}%
\end{pgfscope}%
\begin{pgfscope}%
\pgfsys@transformshift{4.338298in}{1.393604in}%
\pgfsys@useobject{currentmarker}{}%
\end{pgfscope}%
\begin{pgfscope}%
\pgfsys@transformshift{4.317173in}{1.370233in}%
\pgfsys@useobject{currentmarker}{}%
\end{pgfscope}%
\begin{pgfscope}%
\pgfsys@transformshift{4.302150in}{1.413755in}%
\pgfsys@useobject{currentmarker}{}%
\end{pgfscope}%
\begin{pgfscope}%
\pgfsys@transformshift{4.281025in}{1.519238in}%
\pgfsys@useobject{currentmarker}{}%
\end{pgfscope}%
\begin{pgfscope}%
\pgfsys@transformshift{4.262715in}{1.662002in}%
\pgfsys@useobject{currentmarker}{}%
\end{pgfscope}%
\begin{pgfscope}%
\pgfsys@transformshift{4.243468in}{1.751734in}%
\pgfsys@useobject{currentmarker}{}%
\end{pgfscope}%
\begin{pgfscope}%
\pgfsys@transformshift{4.224454in}{1.758553in}%
\pgfsys@useobject{currentmarker}{}%
\end{pgfscope}%
\begin{pgfscope}%
\pgfsys@transformshift{4.206850in}{1.684702in}%
\pgfsys@useobject{currentmarker}{}%
\end{pgfscope}%
\begin{pgfscope}%
\pgfsys@transformshift{4.185490in}{1.532766in}%
\pgfsys@useobject{currentmarker}{}%
\end{pgfscope}%
\begin{pgfscope}%
\pgfsys@transformshift{4.166711in}{1.427908in}%
\pgfsys@useobject{currentmarker}{}%
\end{pgfscope}%
\begin{pgfscope}%
\pgfsys@transformshift{4.149575in}{1.378277in}%
\pgfsys@useobject{currentmarker}{}%
\end{pgfscope}%
\begin{pgfscope}%
\pgfsys@transformshift{4.128216in}{1.375739in}%
\pgfsys@useobject{currentmarker}{}%
\end{pgfscope}%
\begin{pgfscope}%
\pgfsys@transformshift{4.110846in}{1.428294in}%
\pgfsys@useobject{currentmarker}{}%
\end{pgfscope}%
\begin{pgfscope}%
\pgfsys@transformshift{4.088781in}{1.533225in}%
\pgfsys@useobject{currentmarker}{}%
\end{pgfscope}%
\begin{pgfscope}%
\pgfsys@transformshift{4.071881in}{1.671298in}%
\pgfsys@useobject{currentmarker}{}%
\end{pgfscope}%
\begin{pgfscope}%
\pgfsys@transformshift{4.070942in}{1.733856in}%
\pgfsys@useobject{currentmarker}{}%
\end{pgfscope}%
\begin{pgfscope}%
\pgfsys@transformshift{4.050286in}{1.753385in}%
\pgfsys@useobject{currentmarker}{}%
\end{pgfscope}%
\begin{pgfscope}%
\pgfsys@transformshift{4.033150in}{1.745760in}%
\pgfsys@useobject{currentmarker}{}%
\end{pgfscope}%
\begin{pgfscope}%
\pgfsys@transformshift{4.011085in}{1.622116in}%
\pgfsys@useobject{currentmarker}{}%
\end{pgfscope}%
\begin{pgfscope}%
\pgfsys@transformshift{3.994889in}{1.508174in}%
\pgfsys@useobject{currentmarker}{}%
\end{pgfscope}%
\begin{pgfscope}%
\pgfsys@transformshift{3.974467in}{1.412925in}%
\pgfsys@useobject{currentmarker}{}%
\end{pgfscope}%
\begin{pgfscope}%
\pgfsys@transformshift{3.954282in}{1.361711in}%
\pgfsys@useobject{currentmarker}{}%
\end{pgfscope}%
\begin{pgfscope}%
\pgfsys@transformshift{3.936677in}{1.366860in}%
\pgfsys@useobject{currentmarker}{}%
\end{pgfscope}%
\begin{pgfscope}%
\pgfsys@transformshift{3.912499in}{1.441931in}%
\pgfsys@useobject{currentmarker}{}%
\end{pgfscope}%
\begin{pgfscope}%
\pgfsys@transformshift{3.899354in}{1.525945in}%
\pgfsys@useobject{currentmarker}{}%
\end{pgfscope}%
\begin{pgfscope}%
\pgfsys@transformshift{3.878934in}{1.674476in}%
\pgfsys@useobject{currentmarker}{}%
\end{pgfscope}%
\begin{pgfscope}%
\pgfsys@transformshift{3.858041in}{1.749532in}%
\pgfsys@useobject{currentmarker}{}%
\end{pgfscope}%
\begin{pgfscope}%
\pgfsys@transformshift{3.840673in}{1.751836in}%
\pgfsys@useobject{currentmarker}{}%
\end{pgfscope}%
\begin{pgfscope}%
\pgfsys@transformshift{3.816729in}{1.680008in}%
\pgfsys@useobject{currentmarker}{}%
\end{pgfscope}%
\begin{pgfscope}%
\pgfsys@transformshift{3.802412in}{1.570434in}%
\pgfsys@useobject{currentmarker}{}%
\end{pgfscope}%
\begin{pgfscope}%
\pgfsys@transformshift{3.783164in}{1.456529in}%
\pgfsys@useobject{currentmarker}{}%
\end{pgfscope}%
\begin{pgfscope}%
\pgfsys@transformshift{3.765794in}{1.385522in}%
\pgfsys@useobject{currentmarker}{}%
\end{pgfscope}%
\begin{pgfscope}%
\pgfsys@transformshift{3.744903in}{1.354974in}%
\pgfsys@useobject{currentmarker}{}%
\end{pgfscope}%
\begin{pgfscope}%
\pgfsys@transformshift{3.725656in}{1.387765in}%
\pgfsys@useobject{currentmarker}{}%
\end{pgfscope}%
\begin{pgfscope}%
\pgfsys@transformshift{3.704294in}{1.470002in}%
\pgfsys@useobject{currentmarker}{}%
\end{pgfscope}%
\begin{pgfscope}%
\pgfsys@transformshift{3.687395in}{1.562741in}%
\pgfsys@useobject{currentmarker}{}%
\end{pgfscope}%
\begin{pgfscope}%
\pgfsys@transformshift{3.667676in}{1.682021in}%
\pgfsys@useobject{currentmarker}{}%
\end{pgfscope}%
\begin{pgfscope}%
\pgfsys@transformshift{3.647960in}{1.744976in}%
\pgfsys@useobject{currentmarker}{}%
\end{pgfscope}%
\begin{pgfscope}%
\pgfsys@transformshift{3.630824in}{1.729921in}%
\pgfsys@useobject{currentmarker}{}%
\end{pgfscope}%
\begin{pgfscope}%
\pgfsys@transformshift{3.609230in}{1.617809in}%
\pgfsys@useobject{currentmarker}{}%
\end{pgfscope}%
\begin{pgfscope}%
\pgfsys@transformshift{3.589511in}{1.479597in}%
\pgfsys@useobject{currentmarker}{}%
\end{pgfscope}%
\begin{pgfscope}%
\pgfsys@transformshift{3.568855in}{1.400313in}%
\pgfsys@useobject{currentmarker}{}%
\end{pgfscope}%
\begin{pgfscope}%
\pgfsys@transformshift{3.550547in}{1.359557in}%
\pgfsys@useobject{currentmarker}{}%
\end{pgfscope}%
\begin{pgfscope}%
\pgfsys@transformshift{3.532942in}{1.356921in}%
\pgfsys@useobject{currentmarker}{}%
\end{pgfscope}%
\begin{pgfscope}%
\pgfsys@transformshift{3.512286in}{1.403899in}%
\pgfsys@useobject{currentmarker}{}%
\end{pgfscope}%
\begin{pgfscope}%
\pgfsys@transformshift{3.495619in}{1.477930in}%
\pgfsys@useobject{currentmarker}{}%
\end{pgfscope}%
\begin{pgfscope}%
\pgfsys@transformshift{3.475434in}{1.620794in}%
\pgfsys@useobject{currentmarker}{}%
\end{pgfscope}%
\begin{pgfscope}%
\pgfsys@transformshift{3.457358in}{1.720222in}%
\pgfsys@useobject{currentmarker}{}%
\end{pgfscope}%
\begin{pgfscope}%
\pgfsys@transformshift{3.436702in}{1.743872in}%
\pgfsys@useobject{currentmarker}{}%
\end{pgfscope}%
\begin{pgfscope}%
\pgfsys@transformshift{3.419097in}{1.708233in}%
\pgfsys@useobject{currentmarker}{}%
\end{pgfscope}%
\begin{pgfscope}%
\pgfsys@transformshift{3.397503in}{1.584755in}%
\pgfsys@useobject{currentmarker}{}%
\end{pgfscope}%
\begin{pgfscope}%
\pgfsys@transformshift{3.380838in}{1.484034in}%
\pgfsys@useobject{currentmarker}{}%
\end{pgfscope}%
\begin{pgfscope}%
\pgfsys@transformshift{3.359242in}{1.409160in}%
\pgfsys@useobject{currentmarker}{}%
\end{pgfscope}%
\begin{pgfscope}%
\pgfsys@transformshift{3.338352in}{1.364708in}%
\pgfsys@useobject{currentmarker}{}%
\end{pgfscope}%
\begin{pgfscope}%
\pgfsys@transformshift{3.321685in}{1.354132in}%
\pgfsys@useobject{currentmarker}{}%
\end{pgfscope}%
\begin{pgfscope}%
\pgfsys@transformshift{3.301029in}{1.394683in}%
\pgfsys@useobject{currentmarker}{}%
\end{pgfscope}%
\begin{pgfscope}%
\pgfsys@transformshift{3.282486in}{1.448399in}%
\pgfsys@useobject{currentmarker}{}%
\end{pgfscope}%
\begin{pgfscope}%
\pgfsys@transformshift{3.264176in}{1.553572in}%
\pgfsys@useobject{currentmarker}{}%
\end{pgfscope}%
\begin{pgfscope}%
\pgfsys@transformshift{3.243754in}{1.697084in}%
\pgfsys@useobject{currentmarker}{}%
\end{pgfscope}%
\begin{pgfscope}%
\pgfsys@transformshift{3.224507in}{1.740961in}%
\pgfsys@useobject{currentmarker}{}%
\end{pgfscope}%
\begin{pgfscope}%
\pgfsys@transformshift{3.205259in}{1.720761in}%
\pgfsys@useobject{currentmarker}{}%
\end{pgfscope}%
\begin{pgfscope}%
\pgfsys@transformshift{3.188594in}{1.668841in}%
\pgfsys@useobject{currentmarker}{}%
\end{pgfscope}%
\begin{pgfscope}%
\pgfsys@transformshift{3.166764in}{1.536712in}%
\pgfsys@useobject{currentmarker}{}%
\end{pgfscope}%
\begin{pgfscope}%
\pgfsys@transformshift{3.149159in}{1.558944in}%
\pgfsys@useobject{currentmarker}{}%
\end{pgfscope}%
\begin{pgfscope}%
\pgfsys@transformshift{3.128268in}{1.439369in}%
\pgfsys@useobject{currentmarker}{}%
\end{pgfscope}%
\begin{pgfscope}%
\pgfsys@transformshift{3.110429in}{1.401775in}%
\pgfsys@useobject{currentmarker}{}%
\end{pgfscope}%
\end{pgfscope}%
\begin{pgfscope}%
\pgfsetrectcap%
\pgfsetmiterjoin%
\pgfsetlinewidth{0.501875pt}%
\definecolor{currentstroke}{rgb}{0.000000,0.000000,0.000000}%
\pgfsetstrokecolor{currentstroke}%
\pgfsetdash{}{0pt}%
\pgfpathmoveto{\pgfqpoint{0.444748in}{1.326898in}}%
\pgfpathlineto{\pgfqpoint{0.444748in}{1.794149in}}%
\pgfusepath{stroke}%
\end{pgfscope}%
\begin{pgfscope}%
\pgfsetrectcap%
\pgfsetmiterjoin%
\pgfsetlinewidth{0.501875pt}%
\definecolor{currentstroke}{rgb}{0.000000,0.000000,0.000000}%
\pgfsetstrokecolor{currentstroke}%
\pgfsetdash{}{0pt}%
\pgfpathmoveto{\pgfqpoint{4.676167in}{1.326898in}}%
\pgfpathlineto{\pgfqpoint{4.676167in}{1.794149in}}%
\pgfusepath{stroke}%
\end{pgfscope}%
\begin{pgfscope}%
\pgfsetrectcap%
\pgfsetmiterjoin%
\pgfsetlinewidth{0.501875pt}%
\definecolor{currentstroke}{rgb}{0.000000,0.000000,0.000000}%
\pgfsetstrokecolor{currentstroke}%
\pgfsetdash{}{0pt}%
\pgfpathmoveto{\pgfqpoint{0.444748in}{1.326898in}}%
\pgfpathlineto{\pgfqpoint{4.676167in}{1.326898in}}%
\pgfusepath{stroke}%
\end{pgfscope}%
\begin{pgfscope}%
\pgfsetrectcap%
\pgfsetmiterjoin%
\pgfsetlinewidth{0.501875pt}%
\definecolor{currentstroke}{rgb}{0.000000,0.000000,0.000000}%
\pgfsetstrokecolor{currentstroke}%
\pgfsetdash{}{0pt}%
\pgfpathmoveto{\pgfqpoint{0.444748in}{1.794149in}}%
\pgfpathlineto{\pgfqpoint{4.676167in}{1.794149in}}%
\pgfusepath{stroke}%
\end{pgfscope}%
\begin{pgfscope}%
\definecolor{textcolor}{rgb}{0.000000,0.000000,0.000000}%
\pgfsetstrokecolor{textcolor}%
\pgfsetfillcolor{textcolor}%
\pgftext[x=2.560458in,y=1.877482in,,base]{\color{textcolor}\rmfamily\fontsize{12.000000}{14.400000}\selectfont T = \qty{3.4}{\kelvin}}%
\end{pgfscope}%
\begin{pgfscope}%
\pgfsetbuttcap%
\pgfsetmiterjoin%
\definecolor{currentfill}{rgb}{1.000000,1.000000,1.000000}%
\pgfsetfillcolor{currentfill}%
\pgfsetlinewidth{0.000000pt}%
\definecolor{currentstroke}{rgb}{0.000000,0.000000,0.000000}%
\pgfsetstrokecolor{currentstroke}%
\pgfsetstrokeopacity{0.000000}%
\pgfsetdash{}{0pt}%
\pgfpathmoveto{\pgfqpoint{0.444748in}{0.431673in}}%
\pgfpathlineto{\pgfqpoint{4.676167in}{0.431673in}}%
\pgfpathlineto{\pgfqpoint{4.676167in}{0.898923in}}%
\pgfpathlineto{\pgfqpoint{0.444748in}{0.898923in}}%
\pgfpathlineto{\pgfqpoint{0.444748in}{0.431673in}}%
\pgfpathclose%
\pgfusepath{fill}%
\end{pgfscope}%
\begin{pgfscope}%
\pgfsetbuttcap%
\pgfsetroundjoin%
\definecolor{currentfill}{rgb}{0.000000,0.000000,0.000000}%
\pgfsetfillcolor{currentfill}%
\pgfsetlinewidth{0.501875pt}%
\definecolor{currentstroke}{rgb}{0.000000,0.000000,0.000000}%
\pgfsetstrokecolor{currentstroke}%
\pgfsetdash{}{0pt}%
\pgfsys@defobject{currentmarker}{\pgfqpoint{0.000000in}{0.000000in}}{\pgfqpoint{0.000000in}{0.041667in}}{%
\pgfpathmoveto{\pgfqpoint{0.000000in}{0.000000in}}%
\pgfpathlineto{\pgfqpoint{0.000000in}{0.041667in}}%
\pgfusepath{stroke,fill}%
}%
\begin{pgfscope}%
\pgfsys@transformshift{0.643886in}{0.431673in}%
\pgfsys@useobject{currentmarker}{}%
\end{pgfscope}%
\end{pgfscope}%
\begin{pgfscope}%
\pgfsetbuttcap%
\pgfsetroundjoin%
\definecolor{currentfill}{rgb}{0.000000,0.000000,0.000000}%
\pgfsetfillcolor{currentfill}%
\pgfsetlinewidth{0.501875pt}%
\definecolor{currentstroke}{rgb}{0.000000,0.000000,0.000000}%
\pgfsetstrokecolor{currentstroke}%
\pgfsetdash{}{0pt}%
\pgfsys@defobject{currentmarker}{\pgfqpoint{0.000000in}{-0.041667in}}{\pgfqpoint{0.000000in}{0.000000in}}{%
\pgfpathmoveto{\pgfqpoint{0.000000in}{0.000000in}}%
\pgfpathlineto{\pgfqpoint{0.000000in}{-0.041667in}}%
\pgfusepath{stroke,fill}%
}%
\begin{pgfscope}%
\pgfsys@transformshift{0.643886in}{0.898923in}%
\pgfsys@useobject{currentmarker}{}%
\end{pgfscope}%
\end{pgfscope}%
\begin{pgfscope}%
\definecolor{textcolor}{rgb}{0.000000,0.000000,0.000000}%
\pgfsetstrokecolor{textcolor}%
\pgfsetfillcolor{textcolor}%
\pgftext[x=0.643886in,y=0.383062in,,top]{\color{textcolor}\rmfamily\fontsize{10.000000}{12.000000}\selectfont \(\displaystyle {\ensuremath{-}10.0}\)}%
\end{pgfscope}%
\begin{pgfscope}%
\pgfsetbuttcap%
\pgfsetroundjoin%
\definecolor{currentfill}{rgb}{0.000000,0.000000,0.000000}%
\pgfsetfillcolor{currentfill}%
\pgfsetlinewidth{0.501875pt}%
\definecolor{currentstroke}{rgb}{0.000000,0.000000,0.000000}%
\pgfsetstrokecolor{currentstroke}%
\pgfsetdash{}{0pt}%
\pgfsys@defobject{currentmarker}{\pgfqpoint{0.000000in}{0.000000in}}{\pgfqpoint{0.000000in}{0.041667in}}{%
\pgfpathmoveto{\pgfqpoint{0.000000in}{0.000000in}}%
\pgfpathlineto{\pgfqpoint{0.000000in}{0.041667in}}%
\pgfusepath{stroke,fill}%
}%
\begin{pgfscope}%
\pgfsys@transformshift{1.124261in}{0.431673in}%
\pgfsys@useobject{currentmarker}{}%
\end{pgfscope}%
\end{pgfscope}%
\begin{pgfscope}%
\pgfsetbuttcap%
\pgfsetroundjoin%
\definecolor{currentfill}{rgb}{0.000000,0.000000,0.000000}%
\pgfsetfillcolor{currentfill}%
\pgfsetlinewidth{0.501875pt}%
\definecolor{currentstroke}{rgb}{0.000000,0.000000,0.000000}%
\pgfsetstrokecolor{currentstroke}%
\pgfsetdash{}{0pt}%
\pgfsys@defobject{currentmarker}{\pgfqpoint{0.000000in}{-0.041667in}}{\pgfqpoint{0.000000in}{0.000000in}}{%
\pgfpathmoveto{\pgfqpoint{0.000000in}{0.000000in}}%
\pgfpathlineto{\pgfqpoint{0.000000in}{-0.041667in}}%
\pgfusepath{stroke,fill}%
}%
\begin{pgfscope}%
\pgfsys@transformshift{1.124261in}{0.898923in}%
\pgfsys@useobject{currentmarker}{}%
\end{pgfscope}%
\end{pgfscope}%
\begin{pgfscope}%
\definecolor{textcolor}{rgb}{0.000000,0.000000,0.000000}%
\pgfsetstrokecolor{textcolor}%
\pgfsetfillcolor{textcolor}%
\pgftext[x=1.124261in,y=0.383062in,,top]{\color{textcolor}\rmfamily\fontsize{10.000000}{12.000000}\selectfont \(\displaystyle {\ensuremath{-}7.5}\)}%
\end{pgfscope}%
\begin{pgfscope}%
\pgfsetbuttcap%
\pgfsetroundjoin%
\definecolor{currentfill}{rgb}{0.000000,0.000000,0.000000}%
\pgfsetfillcolor{currentfill}%
\pgfsetlinewidth{0.501875pt}%
\definecolor{currentstroke}{rgb}{0.000000,0.000000,0.000000}%
\pgfsetstrokecolor{currentstroke}%
\pgfsetdash{}{0pt}%
\pgfsys@defobject{currentmarker}{\pgfqpoint{0.000000in}{0.000000in}}{\pgfqpoint{0.000000in}{0.041667in}}{%
\pgfpathmoveto{\pgfqpoint{0.000000in}{0.000000in}}%
\pgfpathlineto{\pgfqpoint{0.000000in}{0.041667in}}%
\pgfusepath{stroke,fill}%
}%
\begin{pgfscope}%
\pgfsys@transformshift{1.604637in}{0.431673in}%
\pgfsys@useobject{currentmarker}{}%
\end{pgfscope}%
\end{pgfscope}%
\begin{pgfscope}%
\pgfsetbuttcap%
\pgfsetroundjoin%
\definecolor{currentfill}{rgb}{0.000000,0.000000,0.000000}%
\pgfsetfillcolor{currentfill}%
\pgfsetlinewidth{0.501875pt}%
\definecolor{currentstroke}{rgb}{0.000000,0.000000,0.000000}%
\pgfsetstrokecolor{currentstroke}%
\pgfsetdash{}{0pt}%
\pgfsys@defobject{currentmarker}{\pgfqpoint{0.000000in}{-0.041667in}}{\pgfqpoint{0.000000in}{0.000000in}}{%
\pgfpathmoveto{\pgfqpoint{0.000000in}{0.000000in}}%
\pgfpathlineto{\pgfqpoint{0.000000in}{-0.041667in}}%
\pgfusepath{stroke,fill}%
}%
\begin{pgfscope}%
\pgfsys@transformshift{1.604637in}{0.898923in}%
\pgfsys@useobject{currentmarker}{}%
\end{pgfscope}%
\end{pgfscope}%
\begin{pgfscope}%
\definecolor{textcolor}{rgb}{0.000000,0.000000,0.000000}%
\pgfsetstrokecolor{textcolor}%
\pgfsetfillcolor{textcolor}%
\pgftext[x=1.604637in,y=0.383062in,,top]{\color{textcolor}\rmfamily\fontsize{10.000000}{12.000000}\selectfont \(\displaystyle {\ensuremath{-}5.0}\)}%
\end{pgfscope}%
\begin{pgfscope}%
\pgfsetbuttcap%
\pgfsetroundjoin%
\definecolor{currentfill}{rgb}{0.000000,0.000000,0.000000}%
\pgfsetfillcolor{currentfill}%
\pgfsetlinewidth{0.501875pt}%
\definecolor{currentstroke}{rgb}{0.000000,0.000000,0.000000}%
\pgfsetstrokecolor{currentstroke}%
\pgfsetdash{}{0pt}%
\pgfsys@defobject{currentmarker}{\pgfqpoint{0.000000in}{0.000000in}}{\pgfqpoint{0.000000in}{0.041667in}}{%
\pgfpathmoveto{\pgfqpoint{0.000000in}{0.000000in}}%
\pgfpathlineto{\pgfqpoint{0.000000in}{0.041667in}}%
\pgfusepath{stroke,fill}%
}%
\begin{pgfscope}%
\pgfsys@transformshift{2.085012in}{0.431673in}%
\pgfsys@useobject{currentmarker}{}%
\end{pgfscope}%
\end{pgfscope}%
\begin{pgfscope}%
\pgfsetbuttcap%
\pgfsetroundjoin%
\definecolor{currentfill}{rgb}{0.000000,0.000000,0.000000}%
\pgfsetfillcolor{currentfill}%
\pgfsetlinewidth{0.501875pt}%
\definecolor{currentstroke}{rgb}{0.000000,0.000000,0.000000}%
\pgfsetstrokecolor{currentstroke}%
\pgfsetdash{}{0pt}%
\pgfsys@defobject{currentmarker}{\pgfqpoint{0.000000in}{-0.041667in}}{\pgfqpoint{0.000000in}{0.000000in}}{%
\pgfpathmoveto{\pgfqpoint{0.000000in}{0.000000in}}%
\pgfpathlineto{\pgfqpoint{0.000000in}{-0.041667in}}%
\pgfusepath{stroke,fill}%
}%
\begin{pgfscope}%
\pgfsys@transformshift{2.085012in}{0.898923in}%
\pgfsys@useobject{currentmarker}{}%
\end{pgfscope}%
\end{pgfscope}%
\begin{pgfscope}%
\definecolor{textcolor}{rgb}{0.000000,0.000000,0.000000}%
\pgfsetstrokecolor{textcolor}%
\pgfsetfillcolor{textcolor}%
\pgftext[x=2.085012in,y=0.383062in,,top]{\color{textcolor}\rmfamily\fontsize{10.000000}{12.000000}\selectfont \(\displaystyle {\ensuremath{-}2.5}\)}%
\end{pgfscope}%
\begin{pgfscope}%
\pgfsetbuttcap%
\pgfsetroundjoin%
\definecolor{currentfill}{rgb}{0.000000,0.000000,0.000000}%
\pgfsetfillcolor{currentfill}%
\pgfsetlinewidth{0.501875pt}%
\definecolor{currentstroke}{rgb}{0.000000,0.000000,0.000000}%
\pgfsetstrokecolor{currentstroke}%
\pgfsetdash{}{0pt}%
\pgfsys@defobject{currentmarker}{\pgfqpoint{0.000000in}{0.000000in}}{\pgfqpoint{0.000000in}{0.041667in}}{%
\pgfpathmoveto{\pgfqpoint{0.000000in}{0.000000in}}%
\pgfpathlineto{\pgfqpoint{0.000000in}{0.041667in}}%
\pgfusepath{stroke,fill}%
}%
\begin{pgfscope}%
\pgfsys@transformshift{2.565388in}{0.431673in}%
\pgfsys@useobject{currentmarker}{}%
\end{pgfscope}%
\end{pgfscope}%
\begin{pgfscope}%
\pgfsetbuttcap%
\pgfsetroundjoin%
\definecolor{currentfill}{rgb}{0.000000,0.000000,0.000000}%
\pgfsetfillcolor{currentfill}%
\pgfsetlinewidth{0.501875pt}%
\definecolor{currentstroke}{rgb}{0.000000,0.000000,0.000000}%
\pgfsetstrokecolor{currentstroke}%
\pgfsetdash{}{0pt}%
\pgfsys@defobject{currentmarker}{\pgfqpoint{0.000000in}{-0.041667in}}{\pgfqpoint{0.000000in}{0.000000in}}{%
\pgfpathmoveto{\pgfqpoint{0.000000in}{0.000000in}}%
\pgfpathlineto{\pgfqpoint{0.000000in}{-0.041667in}}%
\pgfusepath{stroke,fill}%
}%
\begin{pgfscope}%
\pgfsys@transformshift{2.565388in}{0.898923in}%
\pgfsys@useobject{currentmarker}{}%
\end{pgfscope}%
\end{pgfscope}%
\begin{pgfscope}%
\definecolor{textcolor}{rgb}{0.000000,0.000000,0.000000}%
\pgfsetstrokecolor{textcolor}%
\pgfsetfillcolor{textcolor}%
\pgftext[x=2.565388in,y=0.383062in,,top]{\color{textcolor}\rmfamily\fontsize{10.000000}{12.000000}\selectfont \(\displaystyle {0.0}\)}%
\end{pgfscope}%
\begin{pgfscope}%
\pgfsetbuttcap%
\pgfsetroundjoin%
\definecolor{currentfill}{rgb}{0.000000,0.000000,0.000000}%
\pgfsetfillcolor{currentfill}%
\pgfsetlinewidth{0.501875pt}%
\definecolor{currentstroke}{rgb}{0.000000,0.000000,0.000000}%
\pgfsetstrokecolor{currentstroke}%
\pgfsetdash{}{0pt}%
\pgfsys@defobject{currentmarker}{\pgfqpoint{0.000000in}{0.000000in}}{\pgfqpoint{0.000000in}{0.041667in}}{%
\pgfpathmoveto{\pgfqpoint{0.000000in}{0.000000in}}%
\pgfpathlineto{\pgfqpoint{0.000000in}{0.041667in}}%
\pgfusepath{stroke,fill}%
}%
\begin{pgfscope}%
\pgfsys@transformshift{3.045763in}{0.431673in}%
\pgfsys@useobject{currentmarker}{}%
\end{pgfscope}%
\end{pgfscope}%
\begin{pgfscope}%
\pgfsetbuttcap%
\pgfsetroundjoin%
\definecolor{currentfill}{rgb}{0.000000,0.000000,0.000000}%
\pgfsetfillcolor{currentfill}%
\pgfsetlinewidth{0.501875pt}%
\definecolor{currentstroke}{rgb}{0.000000,0.000000,0.000000}%
\pgfsetstrokecolor{currentstroke}%
\pgfsetdash{}{0pt}%
\pgfsys@defobject{currentmarker}{\pgfqpoint{0.000000in}{-0.041667in}}{\pgfqpoint{0.000000in}{0.000000in}}{%
\pgfpathmoveto{\pgfqpoint{0.000000in}{0.000000in}}%
\pgfpathlineto{\pgfqpoint{0.000000in}{-0.041667in}}%
\pgfusepath{stroke,fill}%
}%
\begin{pgfscope}%
\pgfsys@transformshift{3.045763in}{0.898923in}%
\pgfsys@useobject{currentmarker}{}%
\end{pgfscope}%
\end{pgfscope}%
\begin{pgfscope}%
\definecolor{textcolor}{rgb}{0.000000,0.000000,0.000000}%
\pgfsetstrokecolor{textcolor}%
\pgfsetfillcolor{textcolor}%
\pgftext[x=3.045763in,y=0.383062in,,top]{\color{textcolor}\rmfamily\fontsize{10.000000}{12.000000}\selectfont \(\displaystyle {2.5}\)}%
\end{pgfscope}%
\begin{pgfscope}%
\pgfsetbuttcap%
\pgfsetroundjoin%
\definecolor{currentfill}{rgb}{0.000000,0.000000,0.000000}%
\pgfsetfillcolor{currentfill}%
\pgfsetlinewidth{0.501875pt}%
\definecolor{currentstroke}{rgb}{0.000000,0.000000,0.000000}%
\pgfsetstrokecolor{currentstroke}%
\pgfsetdash{}{0pt}%
\pgfsys@defobject{currentmarker}{\pgfqpoint{0.000000in}{0.000000in}}{\pgfqpoint{0.000000in}{0.041667in}}{%
\pgfpathmoveto{\pgfqpoint{0.000000in}{0.000000in}}%
\pgfpathlineto{\pgfqpoint{0.000000in}{0.041667in}}%
\pgfusepath{stroke,fill}%
}%
\begin{pgfscope}%
\pgfsys@transformshift{3.526138in}{0.431673in}%
\pgfsys@useobject{currentmarker}{}%
\end{pgfscope}%
\end{pgfscope}%
\begin{pgfscope}%
\pgfsetbuttcap%
\pgfsetroundjoin%
\definecolor{currentfill}{rgb}{0.000000,0.000000,0.000000}%
\pgfsetfillcolor{currentfill}%
\pgfsetlinewidth{0.501875pt}%
\definecolor{currentstroke}{rgb}{0.000000,0.000000,0.000000}%
\pgfsetstrokecolor{currentstroke}%
\pgfsetdash{}{0pt}%
\pgfsys@defobject{currentmarker}{\pgfqpoint{0.000000in}{-0.041667in}}{\pgfqpoint{0.000000in}{0.000000in}}{%
\pgfpathmoveto{\pgfqpoint{0.000000in}{0.000000in}}%
\pgfpathlineto{\pgfqpoint{0.000000in}{-0.041667in}}%
\pgfusepath{stroke,fill}%
}%
\begin{pgfscope}%
\pgfsys@transformshift{3.526138in}{0.898923in}%
\pgfsys@useobject{currentmarker}{}%
\end{pgfscope}%
\end{pgfscope}%
\begin{pgfscope}%
\definecolor{textcolor}{rgb}{0.000000,0.000000,0.000000}%
\pgfsetstrokecolor{textcolor}%
\pgfsetfillcolor{textcolor}%
\pgftext[x=3.526138in,y=0.383062in,,top]{\color{textcolor}\rmfamily\fontsize{10.000000}{12.000000}\selectfont \(\displaystyle {5.0}\)}%
\end{pgfscope}%
\begin{pgfscope}%
\pgfsetbuttcap%
\pgfsetroundjoin%
\definecolor{currentfill}{rgb}{0.000000,0.000000,0.000000}%
\pgfsetfillcolor{currentfill}%
\pgfsetlinewidth{0.501875pt}%
\definecolor{currentstroke}{rgb}{0.000000,0.000000,0.000000}%
\pgfsetstrokecolor{currentstroke}%
\pgfsetdash{}{0pt}%
\pgfsys@defobject{currentmarker}{\pgfqpoint{0.000000in}{0.000000in}}{\pgfqpoint{0.000000in}{0.041667in}}{%
\pgfpathmoveto{\pgfqpoint{0.000000in}{0.000000in}}%
\pgfpathlineto{\pgfqpoint{0.000000in}{0.041667in}}%
\pgfusepath{stroke,fill}%
}%
\begin{pgfscope}%
\pgfsys@transformshift{4.006514in}{0.431673in}%
\pgfsys@useobject{currentmarker}{}%
\end{pgfscope}%
\end{pgfscope}%
\begin{pgfscope}%
\pgfsetbuttcap%
\pgfsetroundjoin%
\definecolor{currentfill}{rgb}{0.000000,0.000000,0.000000}%
\pgfsetfillcolor{currentfill}%
\pgfsetlinewidth{0.501875pt}%
\definecolor{currentstroke}{rgb}{0.000000,0.000000,0.000000}%
\pgfsetstrokecolor{currentstroke}%
\pgfsetdash{}{0pt}%
\pgfsys@defobject{currentmarker}{\pgfqpoint{0.000000in}{-0.041667in}}{\pgfqpoint{0.000000in}{0.000000in}}{%
\pgfpathmoveto{\pgfqpoint{0.000000in}{0.000000in}}%
\pgfpathlineto{\pgfqpoint{0.000000in}{-0.041667in}}%
\pgfusepath{stroke,fill}%
}%
\begin{pgfscope}%
\pgfsys@transformshift{4.006514in}{0.898923in}%
\pgfsys@useobject{currentmarker}{}%
\end{pgfscope}%
\end{pgfscope}%
\begin{pgfscope}%
\definecolor{textcolor}{rgb}{0.000000,0.000000,0.000000}%
\pgfsetstrokecolor{textcolor}%
\pgfsetfillcolor{textcolor}%
\pgftext[x=4.006514in,y=0.383062in,,top]{\color{textcolor}\rmfamily\fontsize{10.000000}{12.000000}\selectfont \(\displaystyle {7.5}\)}%
\end{pgfscope}%
\begin{pgfscope}%
\pgfsetbuttcap%
\pgfsetroundjoin%
\definecolor{currentfill}{rgb}{0.000000,0.000000,0.000000}%
\pgfsetfillcolor{currentfill}%
\pgfsetlinewidth{0.501875pt}%
\definecolor{currentstroke}{rgb}{0.000000,0.000000,0.000000}%
\pgfsetstrokecolor{currentstroke}%
\pgfsetdash{}{0pt}%
\pgfsys@defobject{currentmarker}{\pgfqpoint{0.000000in}{0.000000in}}{\pgfqpoint{0.000000in}{0.041667in}}{%
\pgfpathmoveto{\pgfqpoint{0.000000in}{0.000000in}}%
\pgfpathlineto{\pgfqpoint{0.000000in}{0.041667in}}%
\pgfusepath{stroke,fill}%
}%
\begin{pgfscope}%
\pgfsys@transformshift{4.486889in}{0.431673in}%
\pgfsys@useobject{currentmarker}{}%
\end{pgfscope}%
\end{pgfscope}%
\begin{pgfscope}%
\pgfsetbuttcap%
\pgfsetroundjoin%
\definecolor{currentfill}{rgb}{0.000000,0.000000,0.000000}%
\pgfsetfillcolor{currentfill}%
\pgfsetlinewidth{0.501875pt}%
\definecolor{currentstroke}{rgb}{0.000000,0.000000,0.000000}%
\pgfsetstrokecolor{currentstroke}%
\pgfsetdash{}{0pt}%
\pgfsys@defobject{currentmarker}{\pgfqpoint{0.000000in}{-0.041667in}}{\pgfqpoint{0.000000in}{0.000000in}}{%
\pgfpathmoveto{\pgfqpoint{0.000000in}{0.000000in}}%
\pgfpathlineto{\pgfqpoint{0.000000in}{-0.041667in}}%
\pgfusepath{stroke,fill}%
}%
\begin{pgfscope}%
\pgfsys@transformshift{4.486889in}{0.898923in}%
\pgfsys@useobject{currentmarker}{}%
\end{pgfscope}%
\end{pgfscope}%
\begin{pgfscope}%
\definecolor{textcolor}{rgb}{0.000000,0.000000,0.000000}%
\pgfsetstrokecolor{textcolor}%
\pgfsetfillcolor{textcolor}%
\pgftext[x=4.486889in,y=0.383062in,,top]{\color{textcolor}\rmfamily\fontsize{10.000000}{12.000000}\selectfont \(\displaystyle {10.0}\)}%
\end{pgfscope}%
\begin{pgfscope}%
\pgfsetbuttcap%
\pgfsetroundjoin%
\definecolor{currentfill}{rgb}{0.000000,0.000000,0.000000}%
\pgfsetfillcolor{currentfill}%
\pgfsetlinewidth{0.501875pt}%
\definecolor{currentstroke}{rgb}{0.000000,0.000000,0.000000}%
\pgfsetstrokecolor{currentstroke}%
\pgfsetdash{}{0pt}%
\pgfsys@defobject{currentmarker}{\pgfqpoint{0.000000in}{0.000000in}}{\pgfqpoint{0.000000in}{0.020833in}}{%
\pgfpathmoveto{\pgfqpoint{0.000000in}{0.000000in}}%
\pgfpathlineto{\pgfqpoint{0.000000in}{0.020833in}}%
\pgfusepath{stroke,fill}%
}%
\begin{pgfscope}%
\pgfsys@transformshift{0.451736in}{0.431673in}%
\pgfsys@useobject{currentmarker}{}%
\end{pgfscope}%
\end{pgfscope}%
\begin{pgfscope}%
\pgfsetbuttcap%
\pgfsetroundjoin%
\definecolor{currentfill}{rgb}{0.000000,0.000000,0.000000}%
\pgfsetfillcolor{currentfill}%
\pgfsetlinewidth{0.501875pt}%
\definecolor{currentstroke}{rgb}{0.000000,0.000000,0.000000}%
\pgfsetstrokecolor{currentstroke}%
\pgfsetdash{}{0pt}%
\pgfsys@defobject{currentmarker}{\pgfqpoint{0.000000in}{-0.020833in}}{\pgfqpoint{0.000000in}{0.000000in}}{%
\pgfpathmoveto{\pgfqpoint{0.000000in}{0.000000in}}%
\pgfpathlineto{\pgfqpoint{0.000000in}{-0.020833in}}%
\pgfusepath{stroke,fill}%
}%
\begin{pgfscope}%
\pgfsys@transformshift{0.451736in}{0.898923in}%
\pgfsys@useobject{currentmarker}{}%
\end{pgfscope}%
\end{pgfscope}%
\begin{pgfscope}%
\pgfsetbuttcap%
\pgfsetroundjoin%
\definecolor{currentfill}{rgb}{0.000000,0.000000,0.000000}%
\pgfsetfillcolor{currentfill}%
\pgfsetlinewidth{0.501875pt}%
\definecolor{currentstroke}{rgb}{0.000000,0.000000,0.000000}%
\pgfsetstrokecolor{currentstroke}%
\pgfsetdash{}{0pt}%
\pgfsys@defobject{currentmarker}{\pgfqpoint{0.000000in}{0.000000in}}{\pgfqpoint{0.000000in}{0.020833in}}{%
\pgfpathmoveto{\pgfqpoint{0.000000in}{0.000000in}}%
\pgfpathlineto{\pgfqpoint{0.000000in}{0.020833in}}%
\pgfusepath{stroke,fill}%
}%
\begin{pgfscope}%
\pgfsys@transformshift{0.547811in}{0.431673in}%
\pgfsys@useobject{currentmarker}{}%
\end{pgfscope}%
\end{pgfscope}%
\begin{pgfscope}%
\pgfsetbuttcap%
\pgfsetroundjoin%
\definecolor{currentfill}{rgb}{0.000000,0.000000,0.000000}%
\pgfsetfillcolor{currentfill}%
\pgfsetlinewidth{0.501875pt}%
\definecolor{currentstroke}{rgb}{0.000000,0.000000,0.000000}%
\pgfsetstrokecolor{currentstroke}%
\pgfsetdash{}{0pt}%
\pgfsys@defobject{currentmarker}{\pgfqpoint{0.000000in}{-0.020833in}}{\pgfqpoint{0.000000in}{0.000000in}}{%
\pgfpathmoveto{\pgfqpoint{0.000000in}{0.000000in}}%
\pgfpathlineto{\pgfqpoint{0.000000in}{-0.020833in}}%
\pgfusepath{stroke,fill}%
}%
\begin{pgfscope}%
\pgfsys@transformshift{0.547811in}{0.898923in}%
\pgfsys@useobject{currentmarker}{}%
\end{pgfscope}%
\end{pgfscope}%
\begin{pgfscope}%
\pgfsetbuttcap%
\pgfsetroundjoin%
\definecolor{currentfill}{rgb}{0.000000,0.000000,0.000000}%
\pgfsetfillcolor{currentfill}%
\pgfsetlinewidth{0.501875pt}%
\definecolor{currentstroke}{rgb}{0.000000,0.000000,0.000000}%
\pgfsetstrokecolor{currentstroke}%
\pgfsetdash{}{0pt}%
\pgfsys@defobject{currentmarker}{\pgfqpoint{0.000000in}{0.000000in}}{\pgfqpoint{0.000000in}{0.020833in}}{%
\pgfpathmoveto{\pgfqpoint{0.000000in}{0.000000in}}%
\pgfpathlineto{\pgfqpoint{0.000000in}{0.020833in}}%
\pgfusepath{stroke,fill}%
}%
\begin{pgfscope}%
\pgfsys@transformshift{0.739961in}{0.431673in}%
\pgfsys@useobject{currentmarker}{}%
\end{pgfscope}%
\end{pgfscope}%
\begin{pgfscope}%
\pgfsetbuttcap%
\pgfsetroundjoin%
\definecolor{currentfill}{rgb}{0.000000,0.000000,0.000000}%
\pgfsetfillcolor{currentfill}%
\pgfsetlinewidth{0.501875pt}%
\definecolor{currentstroke}{rgb}{0.000000,0.000000,0.000000}%
\pgfsetstrokecolor{currentstroke}%
\pgfsetdash{}{0pt}%
\pgfsys@defobject{currentmarker}{\pgfqpoint{0.000000in}{-0.020833in}}{\pgfqpoint{0.000000in}{0.000000in}}{%
\pgfpathmoveto{\pgfqpoint{0.000000in}{0.000000in}}%
\pgfpathlineto{\pgfqpoint{0.000000in}{-0.020833in}}%
\pgfusepath{stroke,fill}%
}%
\begin{pgfscope}%
\pgfsys@transformshift{0.739961in}{0.898923in}%
\pgfsys@useobject{currentmarker}{}%
\end{pgfscope}%
\end{pgfscope}%
\begin{pgfscope}%
\pgfsetbuttcap%
\pgfsetroundjoin%
\definecolor{currentfill}{rgb}{0.000000,0.000000,0.000000}%
\pgfsetfillcolor{currentfill}%
\pgfsetlinewidth{0.501875pt}%
\definecolor{currentstroke}{rgb}{0.000000,0.000000,0.000000}%
\pgfsetstrokecolor{currentstroke}%
\pgfsetdash{}{0pt}%
\pgfsys@defobject{currentmarker}{\pgfqpoint{0.000000in}{0.000000in}}{\pgfqpoint{0.000000in}{0.020833in}}{%
\pgfpathmoveto{\pgfqpoint{0.000000in}{0.000000in}}%
\pgfpathlineto{\pgfqpoint{0.000000in}{0.020833in}}%
\pgfusepath{stroke,fill}%
}%
\begin{pgfscope}%
\pgfsys@transformshift{0.836036in}{0.431673in}%
\pgfsys@useobject{currentmarker}{}%
\end{pgfscope}%
\end{pgfscope}%
\begin{pgfscope}%
\pgfsetbuttcap%
\pgfsetroundjoin%
\definecolor{currentfill}{rgb}{0.000000,0.000000,0.000000}%
\pgfsetfillcolor{currentfill}%
\pgfsetlinewidth{0.501875pt}%
\definecolor{currentstroke}{rgb}{0.000000,0.000000,0.000000}%
\pgfsetstrokecolor{currentstroke}%
\pgfsetdash{}{0pt}%
\pgfsys@defobject{currentmarker}{\pgfqpoint{0.000000in}{-0.020833in}}{\pgfqpoint{0.000000in}{0.000000in}}{%
\pgfpathmoveto{\pgfqpoint{0.000000in}{0.000000in}}%
\pgfpathlineto{\pgfqpoint{0.000000in}{-0.020833in}}%
\pgfusepath{stroke,fill}%
}%
\begin{pgfscope}%
\pgfsys@transformshift{0.836036in}{0.898923in}%
\pgfsys@useobject{currentmarker}{}%
\end{pgfscope}%
\end{pgfscope}%
\begin{pgfscope}%
\pgfsetbuttcap%
\pgfsetroundjoin%
\definecolor{currentfill}{rgb}{0.000000,0.000000,0.000000}%
\pgfsetfillcolor{currentfill}%
\pgfsetlinewidth{0.501875pt}%
\definecolor{currentstroke}{rgb}{0.000000,0.000000,0.000000}%
\pgfsetstrokecolor{currentstroke}%
\pgfsetdash{}{0pt}%
\pgfsys@defobject{currentmarker}{\pgfqpoint{0.000000in}{0.000000in}}{\pgfqpoint{0.000000in}{0.020833in}}{%
\pgfpathmoveto{\pgfqpoint{0.000000in}{0.000000in}}%
\pgfpathlineto{\pgfqpoint{0.000000in}{0.020833in}}%
\pgfusepath{stroke,fill}%
}%
\begin{pgfscope}%
\pgfsys@transformshift{0.932111in}{0.431673in}%
\pgfsys@useobject{currentmarker}{}%
\end{pgfscope}%
\end{pgfscope}%
\begin{pgfscope}%
\pgfsetbuttcap%
\pgfsetroundjoin%
\definecolor{currentfill}{rgb}{0.000000,0.000000,0.000000}%
\pgfsetfillcolor{currentfill}%
\pgfsetlinewidth{0.501875pt}%
\definecolor{currentstroke}{rgb}{0.000000,0.000000,0.000000}%
\pgfsetstrokecolor{currentstroke}%
\pgfsetdash{}{0pt}%
\pgfsys@defobject{currentmarker}{\pgfqpoint{0.000000in}{-0.020833in}}{\pgfqpoint{0.000000in}{0.000000in}}{%
\pgfpathmoveto{\pgfqpoint{0.000000in}{0.000000in}}%
\pgfpathlineto{\pgfqpoint{0.000000in}{-0.020833in}}%
\pgfusepath{stroke,fill}%
}%
\begin{pgfscope}%
\pgfsys@transformshift{0.932111in}{0.898923in}%
\pgfsys@useobject{currentmarker}{}%
\end{pgfscope}%
\end{pgfscope}%
\begin{pgfscope}%
\pgfsetbuttcap%
\pgfsetroundjoin%
\definecolor{currentfill}{rgb}{0.000000,0.000000,0.000000}%
\pgfsetfillcolor{currentfill}%
\pgfsetlinewidth{0.501875pt}%
\definecolor{currentstroke}{rgb}{0.000000,0.000000,0.000000}%
\pgfsetstrokecolor{currentstroke}%
\pgfsetdash{}{0pt}%
\pgfsys@defobject{currentmarker}{\pgfqpoint{0.000000in}{0.000000in}}{\pgfqpoint{0.000000in}{0.020833in}}{%
\pgfpathmoveto{\pgfqpoint{0.000000in}{0.000000in}}%
\pgfpathlineto{\pgfqpoint{0.000000in}{0.020833in}}%
\pgfusepath{stroke,fill}%
}%
\begin{pgfscope}%
\pgfsys@transformshift{1.028186in}{0.431673in}%
\pgfsys@useobject{currentmarker}{}%
\end{pgfscope}%
\end{pgfscope}%
\begin{pgfscope}%
\pgfsetbuttcap%
\pgfsetroundjoin%
\definecolor{currentfill}{rgb}{0.000000,0.000000,0.000000}%
\pgfsetfillcolor{currentfill}%
\pgfsetlinewidth{0.501875pt}%
\definecolor{currentstroke}{rgb}{0.000000,0.000000,0.000000}%
\pgfsetstrokecolor{currentstroke}%
\pgfsetdash{}{0pt}%
\pgfsys@defobject{currentmarker}{\pgfqpoint{0.000000in}{-0.020833in}}{\pgfqpoint{0.000000in}{0.000000in}}{%
\pgfpathmoveto{\pgfqpoint{0.000000in}{0.000000in}}%
\pgfpathlineto{\pgfqpoint{0.000000in}{-0.020833in}}%
\pgfusepath{stroke,fill}%
}%
\begin{pgfscope}%
\pgfsys@transformshift{1.028186in}{0.898923in}%
\pgfsys@useobject{currentmarker}{}%
\end{pgfscope}%
\end{pgfscope}%
\begin{pgfscope}%
\pgfsetbuttcap%
\pgfsetroundjoin%
\definecolor{currentfill}{rgb}{0.000000,0.000000,0.000000}%
\pgfsetfillcolor{currentfill}%
\pgfsetlinewidth{0.501875pt}%
\definecolor{currentstroke}{rgb}{0.000000,0.000000,0.000000}%
\pgfsetstrokecolor{currentstroke}%
\pgfsetdash{}{0pt}%
\pgfsys@defobject{currentmarker}{\pgfqpoint{0.000000in}{0.000000in}}{\pgfqpoint{0.000000in}{0.020833in}}{%
\pgfpathmoveto{\pgfqpoint{0.000000in}{0.000000in}}%
\pgfpathlineto{\pgfqpoint{0.000000in}{0.020833in}}%
\pgfusepath{stroke,fill}%
}%
\begin{pgfscope}%
\pgfsys@transformshift{1.220336in}{0.431673in}%
\pgfsys@useobject{currentmarker}{}%
\end{pgfscope}%
\end{pgfscope}%
\begin{pgfscope}%
\pgfsetbuttcap%
\pgfsetroundjoin%
\definecolor{currentfill}{rgb}{0.000000,0.000000,0.000000}%
\pgfsetfillcolor{currentfill}%
\pgfsetlinewidth{0.501875pt}%
\definecolor{currentstroke}{rgb}{0.000000,0.000000,0.000000}%
\pgfsetstrokecolor{currentstroke}%
\pgfsetdash{}{0pt}%
\pgfsys@defobject{currentmarker}{\pgfqpoint{0.000000in}{-0.020833in}}{\pgfqpoint{0.000000in}{0.000000in}}{%
\pgfpathmoveto{\pgfqpoint{0.000000in}{0.000000in}}%
\pgfpathlineto{\pgfqpoint{0.000000in}{-0.020833in}}%
\pgfusepath{stroke,fill}%
}%
\begin{pgfscope}%
\pgfsys@transformshift{1.220336in}{0.898923in}%
\pgfsys@useobject{currentmarker}{}%
\end{pgfscope}%
\end{pgfscope}%
\begin{pgfscope}%
\pgfsetbuttcap%
\pgfsetroundjoin%
\definecolor{currentfill}{rgb}{0.000000,0.000000,0.000000}%
\pgfsetfillcolor{currentfill}%
\pgfsetlinewidth{0.501875pt}%
\definecolor{currentstroke}{rgb}{0.000000,0.000000,0.000000}%
\pgfsetstrokecolor{currentstroke}%
\pgfsetdash{}{0pt}%
\pgfsys@defobject{currentmarker}{\pgfqpoint{0.000000in}{0.000000in}}{\pgfqpoint{0.000000in}{0.020833in}}{%
\pgfpathmoveto{\pgfqpoint{0.000000in}{0.000000in}}%
\pgfpathlineto{\pgfqpoint{0.000000in}{0.020833in}}%
\pgfusepath{stroke,fill}%
}%
\begin{pgfscope}%
\pgfsys@transformshift{1.316411in}{0.431673in}%
\pgfsys@useobject{currentmarker}{}%
\end{pgfscope}%
\end{pgfscope}%
\begin{pgfscope}%
\pgfsetbuttcap%
\pgfsetroundjoin%
\definecolor{currentfill}{rgb}{0.000000,0.000000,0.000000}%
\pgfsetfillcolor{currentfill}%
\pgfsetlinewidth{0.501875pt}%
\definecolor{currentstroke}{rgb}{0.000000,0.000000,0.000000}%
\pgfsetstrokecolor{currentstroke}%
\pgfsetdash{}{0pt}%
\pgfsys@defobject{currentmarker}{\pgfqpoint{0.000000in}{-0.020833in}}{\pgfqpoint{0.000000in}{0.000000in}}{%
\pgfpathmoveto{\pgfqpoint{0.000000in}{0.000000in}}%
\pgfpathlineto{\pgfqpoint{0.000000in}{-0.020833in}}%
\pgfusepath{stroke,fill}%
}%
\begin{pgfscope}%
\pgfsys@transformshift{1.316411in}{0.898923in}%
\pgfsys@useobject{currentmarker}{}%
\end{pgfscope}%
\end{pgfscope}%
\begin{pgfscope}%
\pgfsetbuttcap%
\pgfsetroundjoin%
\definecolor{currentfill}{rgb}{0.000000,0.000000,0.000000}%
\pgfsetfillcolor{currentfill}%
\pgfsetlinewidth{0.501875pt}%
\definecolor{currentstroke}{rgb}{0.000000,0.000000,0.000000}%
\pgfsetstrokecolor{currentstroke}%
\pgfsetdash{}{0pt}%
\pgfsys@defobject{currentmarker}{\pgfqpoint{0.000000in}{0.000000in}}{\pgfqpoint{0.000000in}{0.020833in}}{%
\pgfpathmoveto{\pgfqpoint{0.000000in}{0.000000in}}%
\pgfpathlineto{\pgfqpoint{0.000000in}{0.020833in}}%
\pgfusepath{stroke,fill}%
}%
\begin{pgfscope}%
\pgfsys@transformshift{1.412487in}{0.431673in}%
\pgfsys@useobject{currentmarker}{}%
\end{pgfscope}%
\end{pgfscope}%
\begin{pgfscope}%
\pgfsetbuttcap%
\pgfsetroundjoin%
\definecolor{currentfill}{rgb}{0.000000,0.000000,0.000000}%
\pgfsetfillcolor{currentfill}%
\pgfsetlinewidth{0.501875pt}%
\definecolor{currentstroke}{rgb}{0.000000,0.000000,0.000000}%
\pgfsetstrokecolor{currentstroke}%
\pgfsetdash{}{0pt}%
\pgfsys@defobject{currentmarker}{\pgfqpoint{0.000000in}{-0.020833in}}{\pgfqpoint{0.000000in}{0.000000in}}{%
\pgfpathmoveto{\pgfqpoint{0.000000in}{0.000000in}}%
\pgfpathlineto{\pgfqpoint{0.000000in}{-0.020833in}}%
\pgfusepath{stroke,fill}%
}%
\begin{pgfscope}%
\pgfsys@transformshift{1.412487in}{0.898923in}%
\pgfsys@useobject{currentmarker}{}%
\end{pgfscope}%
\end{pgfscope}%
\begin{pgfscope}%
\pgfsetbuttcap%
\pgfsetroundjoin%
\definecolor{currentfill}{rgb}{0.000000,0.000000,0.000000}%
\pgfsetfillcolor{currentfill}%
\pgfsetlinewidth{0.501875pt}%
\definecolor{currentstroke}{rgb}{0.000000,0.000000,0.000000}%
\pgfsetstrokecolor{currentstroke}%
\pgfsetdash{}{0pt}%
\pgfsys@defobject{currentmarker}{\pgfqpoint{0.000000in}{0.000000in}}{\pgfqpoint{0.000000in}{0.020833in}}{%
\pgfpathmoveto{\pgfqpoint{0.000000in}{0.000000in}}%
\pgfpathlineto{\pgfqpoint{0.000000in}{0.020833in}}%
\pgfusepath{stroke,fill}%
}%
\begin{pgfscope}%
\pgfsys@transformshift{1.508562in}{0.431673in}%
\pgfsys@useobject{currentmarker}{}%
\end{pgfscope}%
\end{pgfscope}%
\begin{pgfscope}%
\pgfsetbuttcap%
\pgfsetroundjoin%
\definecolor{currentfill}{rgb}{0.000000,0.000000,0.000000}%
\pgfsetfillcolor{currentfill}%
\pgfsetlinewidth{0.501875pt}%
\definecolor{currentstroke}{rgb}{0.000000,0.000000,0.000000}%
\pgfsetstrokecolor{currentstroke}%
\pgfsetdash{}{0pt}%
\pgfsys@defobject{currentmarker}{\pgfqpoint{0.000000in}{-0.020833in}}{\pgfqpoint{0.000000in}{0.000000in}}{%
\pgfpathmoveto{\pgfqpoint{0.000000in}{0.000000in}}%
\pgfpathlineto{\pgfqpoint{0.000000in}{-0.020833in}}%
\pgfusepath{stroke,fill}%
}%
\begin{pgfscope}%
\pgfsys@transformshift{1.508562in}{0.898923in}%
\pgfsys@useobject{currentmarker}{}%
\end{pgfscope}%
\end{pgfscope}%
\begin{pgfscope}%
\pgfsetbuttcap%
\pgfsetroundjoin%
\definecolor{currentfill}{rgb}{0.000000,0.000000,0.000000}%
\pgfsetfillcolor{currentfill}%
\pgfsetlinewidth{0.501875pt}%
\definecolor{currentstroke}{rgb}{0.000000,0.000000,0.000000}%
\pgfsetstrokecolor{currentstroke}%
\pgfsetdash{}{0pt}%
\pgfsys@defobject{currentmarker}{\pgfqpoint{0.000000in}{0.000000in}}{\pgfqpoint{0.000000in}{0.020833in}}{%
\pgfpathmoveto{\pgfqpoint{0.000000in}{0.000000in}}%
\pgfpathlineto{\pgfqpoint{0.000000in}{0.020833in}}%
\pgfusepath{stroke,fill}%
}%
\begin{pgfscope}%
\pgfsys@transformshift{1.700712in}{0.431673in}%
\pgfsys@useobject{currentmarker}{}%
\end{pgfscope}%
\end{pgfscope}%
\begin{pgfscope}%
\pgfsetbuttcap%
\pgfsetroundjoin%
\definecolor{currentfill}{rgb}{0.000000,0.000000,0.000000}%
\pgfsetfillcolor{currentfill}%
\pgfsetlinewidth{0.501875pt}%
\definecolor{currentstroke}{rgb}{0.000000,0.000000,0.000000}%
\pgfsetstrokecolor{currentstroke}%
\pgfsetdash{}{0pt}%
\pgfsys@defobject{currentmarker}{\pgfqpoint{0.000000in}{-0.020833in}}{\pgfqpoint{0.000000in}{0.000000in}}{%
\pgfpathmoveto{\pgfqpoint{0.000000in}{0.000000in}}%
\pgfpathlineto{\pgfqpoint{0.000000in}{-0.020833in}}%
\pgfusepath{stroke,fill}%
}%
\begin{pgfscope}%
\pgfsys@transformshift{1.700712in}{0.898923in}%
\pgfsys@useobject{currentmarker}{}%
\end{pgfscope}%
\end{pgfscope}%
\begin{pgfscope}%
\pgfsetbuttcap%
\pgfsetroundjoin%
\definecolor{currentfill}{rgb}{0.000000,0.000000,0.000000}%
\pgfsetfillcolor{currentfill}%
\pgfsetlinewidth{0.501875pt}%
\definecolor{currentstroke}{rgb}{0.000000,0.000000,0.000000}%
\pgfsetstrokecolor{currentstroke}%
\pgfsetdash{}{0pt}%
\pgfsys@defobject{currentmarker}{\pgfqpoint{0.000000in}{0.000000in}}{\pgfqpoint{0.000000in}{0.020833in}}{%
\pgfpathmoveto{\pgfqpoint{0.000000in}{0.000000in}}%
\pgfpathlineto{\pgfqpoint{0.000000in}{0.020833in}}%
\pgfusepath{stroke,fill}%
}%
\begin{pgfscope}%
\pgfsys@transformshift{1.796787in}{0.431673in}%
\pgfsys@useobject{currentmarker}{}%
\end{pgfscope}%
\end{pgfscope}%
\begin{pgfscope}%
\pgfsetbuttcap%
\pgfsetroundjoin%
\definecolor{currentfill}{rgb}{0.000000,0.000000,0.000000}%
\pgfsetfillcolor{currentfill}%
\pgfsetlinewidth{0.501875pt}%
\definecolor{currentstroke}{rgb}{0.000000,0.000000,0.000000}%
\pgfsetstrokecolor{currentstroke}%
\pgfsetdash{}{0pt}%
\pgfsys@defobject{currentmarker}{\pgfqpoint{0.000000in}{-0.020833in}}{\pgfqpoint{0.000000in}{0.000000in}}{%
\pgfpathmoveto{\pgfqpoint{0.000000in}{0.000000in}}%
\pgfpathlineto{\pgfqpoint{0.000000in}{-0.020833in}}%
\pgfusepath{stroke,fill}%
}%
\begin{pgfscope}%
\pgfsys@transformshift{1.796787in}{0.898923in}%
\pgfsys@useobject{currentmarker}{}%
\end{pgfscope}%
\end{pgfscope}%
\begin{pgfscope}%
\pgfsetbuttcap%
\pgfsetroundjoin%
\definecolor{currentfill}{rgb}{0.000000,0.000000,0.000000}%
\pgfsetfillcolor{currentfill}%
\pgfsetlinewidth{0.501875pt}%
\definecolor{currentstroke}{rgb}{0.000000,0.000000,0.000000}%
\pgfsetstrokecolor{currentstroke}%
\pgfsetdash{}{0pt}%
\pgfsys@defobject{currentmarker}{\pgfqpoint{0.000000in}{0.000000in}}{\pgfqpoint{0.000000in}{0.020833in}}{%
\pgfpathmoveto{\pgfqpoint{0.000000in}{0.000000in}}%
\pgfpathlineto{\pgfqpoint{0.000000in}{0.020833in}}%
\pgfusepath{stroke,fill}%
}%
\begin{pgfscope}%
\pgfsys@transformshift{1.892862in}{0.431673in}%
\pgfsys@useobject{currentmarker}{}%
\end{pgfscope}%
\end{pgfscope}%
\begin{pgfscope}%
\pgfsetbuttcap%
\pgfsetroundjoin%
\definecolor{currentfill}{rgb}{0.000000,0.000000,0.000000}%
\pgfsetfillcolor{currentfill}%
\pgfsetlinewidth{0.501875pt}%
\definecolor{currentstroke}{rgb}{0.000000,0.000000,0.000000}%
\pgfsetstrokecolor{currentstroke}%
\pgfsetdash{}{0pt}%
\pgfsys@defobject{currentmarker}{\pgfqpoint{0.000000in}{-0.020833in}}{\pgfqpoint{0.000000in}{0.000000in}}{%
\pgfpathmoveto{\pgfqpoint{0.000000in}{0.000000in}}%
\pgfpathlineto{\pgfqpoint{0.000000in}{-0.020833in}}%
\pgfusepath{stroke,fill}%
}%
\begin{pgfscope}%
\pgfsys@transformshift{1.892862in}{0.898923in}%
\pgfsys@useobject{currentmarker}{}%
\end{pgfscope}%
\end{pgfscope}%
\begin{pgfscope}%
\pgfsetbuttcap%
\pgfsetroundjoin%
\definecolor{currentfill}{rgb}{0.000000,0.000000,0.000000}%
\pgfsetfillcolor{currentfill}%
\pgfsetlinewidth{0.501875pt}%
\definecolor{currentstroke}{rgb}{0.000000,0.000000,0.000000}%
\pgfsetstrokecolor{currentstroke}%
\pgfsetdash{}{0pt}%
\pgfsys@defobject{currentmarker}{\pgfqpoint{0.000000in}{0.000000in}}{\pgfqpoint{0.000000in}{0.020833in}}{%
\pgfpathmoveto{\pgfqpoint{0.000000in}{0.000000in}}%
\pgfpathlineto{\pgfqpoint{0.000000in}{0.020833in}}%
\pgfusepath{stroke,fill}%
}%
\begin{pgfscope}%
\pgfsys@transformshift{1.988937in}{0.431673in}%
\pgfsys@useobject{currentmarker}{}%
\end{pgfscope}%
\end{pgfscope}%
\begin{pgfscope}%
\pgfsetbuttcap%
\pgfsetroundjoin%
\definecolor{currentfill}{rgb}{0.000000,0.000000,0.000000}%
\pgfsetfillcolor{currentfill}%
\pgfsetlinewidth{0.501875pt}%
\definecolor{currentstroke}{rgb}{0.000000,0.000000,0.000000}%
\pgfsetstrokecolor{currentstroke}%
\pgfsetdash{}{0pt}%
\pgfsys@defobject{currentmarker}{\pgfqpoint{0.000000in}{-0.020833in}}{\pgfqpoint{0.000000in}{0.000000in}}{%
\pgfpathmoveto{\pgfqpoint{0.000000in}{0.000000in}}%
\pgfpathlineto{\pgfqpoint{0.000000in}{-0.020833in}}%
\pgfusepath{stroke,fill}%
}%
\begin{pgfscope}%
\pgfsys@transformshift{1.988937in}{0.898923in}%
\pgfsys@useobject{currentmarker}{}%
\end{pgfscope}%
\end{pgfscope}%
\begin{pgfscope}%
\pgfsetbuttcap%
\pgfsetroundjoin%
\definecolor{currentfill}{rgb}{0.000000,0.000000,0.000000}%
\pgfsetfillcolor{currentfill}%
\pgfsetlinewidth{0.501875pt}%
\definecolor{currentstroke}{rgb}{0.000000,0.000000,0.000000}%
\pgfsetstrokecolor{currentstroke}%
\pgfsetdash{}{0pt}%
\pgfsys@defobject{currentmarker}{\pgfqpoint{0.000000in}{0.000000in}}{\pgfqpoint{0.000000in}{0.020833in}}{%
\pgfpathmoveto{\pgfqpoint{0.000000in}{0.000000in}}%
\pgfpathlineto{\pgfqpoint{0.000000in}{0.020833in}}%
\pgfusepath{stroke,fill}%
}%
\begin{pgfscope}%
\pgfsys@transformshift{2.181087in}{0.431673in}%
\pgfsys@useobject{currentmarker}{}%
\end{pgfscope}%
\end{pgfscope}%
\begin{pgfscope}%
\pgfsetbuttcap%
\pgfsetroundjoin%
\definecolor{currentfill}{rgb}{0.000000,0.000000,0.000000}%
\pgfsetfillcolor{currentfill}%
\pgfsetlinewidth{0.501875pt}%
\definecolor{currentstroke}{rgb}{0.000000,0.000000,0.000000}%
\pgfsetstrokecolor{currentstroke}%
\pgfsetdash{}{0pt}%
\pgfsys@defobject{currentmarker}{\pgfqpoint{0.000000in}{-0.020833in}}{\pgfqpoint{0.000000in}{0.000000in}}{%
\pgfpathmoveto{\pgfqpoint{0.000000in}{0.000000in}}%
\pgfpathlineto{\pgfqpoint{0.000000in}{-0.020833in}}%
\pgfusepath{stroke,fill}%
}%
\begin{pgfscope}%
\pgfsys@transformshift{2.181087in}{0.898923in}%
\pgfsys@useobject{currentmarker}{}%
\end{pgfscope}%
\end{pgfscope}%
\begin{pgfscope}%
\pgfsetbuttcap%
\pgfsetroundjoin%
\definecolor{currentfill}{rgb}{0.000000,0.000000,0.000000}%
\pgfsetfillcolor{currentfill}%
\pgfsetlinewidth{0.501875pt}%
\definecolor{currentstroke}{rgb}{0.000000,0.000000,0.000000}%
\pgfsetstrokecolor{currentstroke}%
\pgfsetdash{}{0pt}%
\pgfsys@defobject{currentmarker}{\pgfqpoint{0.000000in}{0.000000in}}{\pgfqpoint{0.000000in}{0.020833in}}{%
\pgfpathmoveto{\pgfqpoint{0.000000in}{0.000000in}}%
\pgfpathlineto{\pgfqpoint{0.000000in}{0.020833in}}%
\pgfusepath{stroke,fill}%
}%
\begin{pgfscope}%
\pgfsys@transformshift{2.277162in}{0.431673in}%
\pgfsys@useobject{currentmarker}{}%
\end{pgfscope}%
\end{pgfscope}%
\begin{pgfscope}%
\pgfsetbuttcap%
\pgfsetroundjoin%
\definecolor{currentfill}{rgb}{0.000000,0.000000,0.000000}%
\pgfsetfillcolor{currentfill}%
\pgfsetlinewidth{0.501875pt}%
\definecolor{currentstroke}{rgb}{0.000000,0.000000,0.000000}%
\pgfsetstrokecolor{currentstroke}%
\pgfsetdash{}{0pt}%
\pgfsys@defobject{currentmarker}{\pgfqpoint{0.000000in}{-0.020833in}}{\pgfqpoint{0.000000in}{0.000000in}}{%
\pgfpathmoveto{\pgfqpoint{0.000000in}{0.000000in}}%
\pgfpathlineto{\pgfqpoint{0.000000in}{-0.020833in}}%
\pgfusepath{stroke,fill}%
}%
\begin{pgfscope}%
\pgfsys@transformshift{2.277162in}{0.898923in}%
\pgfsys@useobject{currentmarker}{}%
\end{pgfscope}%
\end{pgfscope}%
\begin{pgfscope}%
\pgfsetbuttcap%
\pgfsetroundjoin%
\definecolor{currentfill}{rgb}{0.000000,0.000000,0.000000}%
\pgfsetfillcolor{currentfill}%
\pgfsetlinewidth{0.501875pt}%
\definecolor{currentstroke}{rgb}{0.000000,0.000000,0.000000}%
\pgfsetstrokecolor{currentstroke}%
\pgfsetdash{}{0pt}%
\pgfsys@defobject{currentmarker}{\pgfqpoint{0.000000in}{0.000000in}}{\pgfqpoint{0.000000in}{0.020833in}}{%
\pgfpathmoveto{\pgfqpoint{0.000000in}{0.000000in}}%
\pgfpathlineto{\pgfqpoint{0.000000in}{0.020833in}}%
\pgfusepath{stroke,fill}%
}%
\begin{pgfscope}%
\pgfsys@transformshift{2.373237in}{0.431673in}%
\pgfsys@useobject{currentmarker}{}%
\end{pgfscope}%
\end{pgfscope}%
\begin{pgfscope}%
\pgfsetbuttcap%
\pgfsetroundjoin%
\definecolor{currentfill}{rgb}{0.000000,0.000000,0.000000}%
\pgfsetfillcolor{currentfill}%
\pgfsetlinewidth{0.501875pt}%
\definecolor{currentstroke}{rgb}{0.000000,0.000000,0.000000}%
\pgfsetstrokecolor{currentstroke}%
\pgfsetdash{}{0pt}%
\pgfsys@defobject{currentmarker}{\pgfqpoint{0.000000in}{-0.020833in}}{\pgfqpoint{0.000000in}{0.000000in}}{%
\pgfpathmoveto{\pgfqpoint{0.000000in}{0.000000in}}%
\pgfpathlineto{\pgfqpoint{0.000000in}{-0.020833in}}%
\pgfusepath{stroke,fill}%
}%
\begin{pgfscope}%
\pgfsys@transformshift{2.373237in}{0.898923in}%
\pgfsys@useobject{currentmarker}{}%
\end{pgfscope}%
\end{pgfscope}%
\begin{pgfscope}%
\pgfsetbuttcap%
\pgfsetroundjoin%
\definecolor{currentfill}{rgb}{0.000000,0.000000,0.000000}%
\pgfsetfillcolor{currentfill}%
\pgfsetlinewidth{0.501875pt}%
\definecolor{currentstroke}{rgb}{0.000000,0.000000,0.000000}%
\pgfsetstrokecolor{currentstroke}%
\pgfsetdash{}{0pt}%
\pgfsys@defobject{currentmarker}{\pgfqpoint{0.000000in}{0.000000in}}{\pgfqpoint{0.000000in}{0.020833in}}{%
\pgfpathmoveto{\pgfqpoint{0.000000in}{0.000000in}}%
\pgfpathlineto{\pgfqpoint{0.000000in}{0.020833in}}%
\pgfusepath{stroke,fill}%
}%
\begin{pgfscope}%
\pgfsys@transformshift{2.469312in}{0.431673in}%
\pgfsys@useobject{currentmarker}{}%
\end{pgfscope}%
\end{pgfscope}%
\begin{pgfscope}%
\pgfsetbuttcap%
\pgfsetroundjoin%
\definecolor{currentfill}{rgb}{0.000000,0.000000,0.000000}%
\pgfsetfillcolor{currentfill}%
\pgfsetlinewidth{0.501875pt}%
\definecolor{currentstroke}{rgb}{0.000000,0.000000,0.000000}%
\pgfsetstrokecolor{currentstroke}%
\pgfsetdash{}{0pt}%
\pgfsys@defobject{currentmarker}{\pgfqpoint{0.000000in}{-0.020833in}}{\pgfqpoint{0.000000in}{0.000000in}}{%
\pgfpathmoveto{\pgfqpoint{0.000000in}{0.000000in}}%
\pgfpathlineto{\pgfqpoint{0.000000in}{-0.020833in}}%
\pgfusepath{stroke,fill}%
}%
\begin{pgfscope}%
\pgfsys@transformshift{2.469312in}{0.898923in}%
\pgfsys@useobject{currentmarker}{}%
\end{pgfscope}%
\end{pgfscope}%
\begin{pgfscope}%
\pgfsetbuttcap%
\pgfsetroundjoin%
\definecolor{currentfill}{rgb}{0.000000,0.000000,0.000000}%
\pgfsetfillcolor{currentfill}%
\pgfsetlinewidth{0.501875pt}%
\definecolor{currentstroke}{rgb}{0.000000,0.000000,0.000000}%
\pgfsetstrokecolor{currentstroke}%
\pgfsetdash{}{0pt}%
\pgfsys@defobject{currentmarker}{\pgfqpoint{0.000000in}{0.000000in}}{\pgfqpoint{0.000000in}{0.020833in}}{%
\pgfpathmoveto{\pgfqpoint{0.000000in}{0.000000in}}%
\pgfpathlineto{\pgfqpoint{0.000000in}{0.020833in}}%
\pgfusepath{stroke,fill}%
}%
\begin{pgfscope}%
\pgfsys@transformshift{2.661463in}{0.431673in}%
\pgfsys@useobject{currentmarker}{}%
\end{pgfscope}%
\end{pgfscope}%
\begin{pgfscope}%
\pgfsetbuttcap%
\pgfsetroundjoin%
\definecolor{currentfill}{rgb}{0.000000,0.000000,0.000000}%
\pgfsetfillcolor{currentfill}%
\pgfsetlinewidth{0.501875pt}%
\definecolor{currentstroke}{rgb}{0.000000,0.000000,0.000000}%
\pgfsetstrokecolor{currentstroke}%
\pgfsetdash{}{0pt}%
\pgfsys@defobject{currentmarker}{\pgfqpoint{0.000000in}{-0.020833in}}{\pgfqpoint{0.000000in}{0.000000in}}{%
\pgfpathmoveto{\pgfqpoint{0.000000in}{0.000000in}}%
\pgfpathlineto{\pgfqpoint{0.000000in}{-0.020833in}}%
\pgfusepath{stroke,fill}%
}%
\begin{pgfscope}%
\pgfsys@transformshift{2.661463in}{0.898923in}%
\pgfsys@useobject{currentmarker}{}%
\end{pgfscope}%
\end{pgfscope}%
\begin{pgfscope}%
\pgfsetbuttcap%
\pgfsetroundjoin%
\definecolor{currentfill}{rgb}{0.000000,0.000000,0.000000}%
\pgfsetfillcolor{currentfill}%
\pgfsetlinewidth{0.501875pt}%
\definecolor{currentstroke}{rgb}{0.000000,0.000000,0.000000}%
\pgfsetstrokecolor{currentstroke}%
\pgfsetdash{}{0pt}%
\pgfsys@defobject{currentmarker}{\pgfqpoint{0.000000in}{0.000000in}}{\pgfqpoint{0.000000in}{0.020833in}}{%
\pgfpathmoveto{\pgfqpoint{0.000000in}{0.000000in}}%
\pgfpathlineto{\pgfqpoint{0.000000in}{0.020833in}}%
\pgfusepath{stroke,fill}%
}%
\begin{pgfscope}%
\pgfsys@transformshift{2.757538in}{0.431673in}%
\pgfsys@useobject{currentmarker}{}%
\end{pgfscope}%
\end{pgfscope}%
\begin{pgfscope}%
\pgfsetbuttcap%
\pgfsetroundjoin%
\definecolor{currentfill}{rgb}{0.000000,0.000000,0.000000}%
\pgfsetfillcolor{currentfill}%
\pgfsetlinewidth{0.501875pt}%
\definecolor{currentstroke}{rgb}{0.000000,0.000000,0.000000}%
\pgfsetstrokecolor{currentstroke}%
\pgfsetdash{}{0pt}%
\pgfsys@defobject{currentmarker}{\pgfqpoint{0.000000in}{-0.020833in}}{\pgfqpoint{0.000000in}{0.000000in}}{%
\pgfpathmoveto{\pgfqpoint{0.000000in}{0.000000in}}%
\pgfpathlineto{\pgfqpoint{0.000000in}{-0.020833in}}%
\pgfusepath{stroke,fill}%
}%
\begin{pgfscope}%
\pgfsys@transformshift{2.757538in}{0.898923in}%
\pgfsys@useobject{currentmarker}{}%
\end{pgfscope}%
\end{pgfscope}%
\begin{pgfscope}%
\pgfsetbuttcap%
\pgfsetroundjoin%
\definecolor{currentfill}{rgb}{0.000000,0.000000,0.000000}%
\pgfsetfillcolor{currentfill}%
\pgfsetlinewidth{0.501875pt}%
\definecolor{currentstroke}{rgb}{0.000000,0.000000,0.000000}%
\pgfsetstrokecolor{currentstroke}%
\pgfsetdash{}{0pt}%
\pgfsys@defobject{currentmarker}{\pgfqpoint{0.000000in}{0.000000in}}{\pgfqpoint{0.000000in}{0.020833in}}{%
\pgfpathmoveto{\pgfqpoint{0.000000in}{0.000000in}}%
\pgfpathlineto{\pgfqpoint{0.000000in}{0.020833in}}%
\pgfusepath{stroke,fill}%
}%
\begin{pgfscope}%
\pgfsys@transformshift{2.853613in}{0.431673in}%
\pgfsys@useobject{currentmarker}{}%
\end{pgfscope}%
\end{pgfscope}%
\begin{pgfscope}%
\pgfsetbuttcap%
\pgfsetroundjoin%
\definecolor{currentfill}{rgb}{0.000000,0.000000,0.000000}%
\pgfsetfillcolor{currentfill}%
\pgfsetlinewidth{0.501875pt}%
\definecolor{currentstroke}{rgb}{0.000000,0.000000,0.000000}%
\pgfsetstrokecolor{currentstroke}%
\pgfsetdash{}{0pt}%
\pgfsys@defobject{currentmarker}{\pgfqpoint{0.000000in}{-0.020833in}}{\pgfqpoint{0.000000in}{0.000000in}}{%
\pgfpathmoveto{\pgfqpoint{0.000000in}{0.000000in}}%
\pgfpathlineto{\pgfqpoint{0.000000in}{-0.020833in}}%
\pgfusepath{stroke,fill}%
}%
\begin{pgfscope}%
\pgfsys@transformshift{2.853613in}{0.898923in}%
\pgfsys@useobject{currentmarker}{}%
\end{pgfscope}%
\end{pgfscope}%
\begin{pgfscope}%
\pgfsetbuttcap%
\pgfsetroundjoin%
\definecolor{currentfill}{rgb}{0.000000,0.000000,0.000000}%
\pgfsetfillcolor{currentfill}%
\pgfsetlinewidth{0.501875pt}%
\definecolor{currentstroke}{rgb}{0.000000,0.000000,0.000000}%
\pgfsetstrokecolor{currentstroke}%
\pgfsetdash{}{0pt}%
\pgfsys@defobject{currentmarker}{\pgfqpoint{0.000000in}{0.000000in}}{\pgfqpoint{0.000000in}{0.020833in}}{%
\pgfpathmoveto{\pgfqpoint{0.000000in}{0.000000in}}%
\pgfpathlineto{\pgfqpoint{0.000000in}{0.020833in}}%
\pgfusepath{stroke,fill}%
}%
\begin{pgfscope}%
\pgfsys@transformshift{2.949688in}{0.431673in}%
\pgfsys@useobject{currentmarker}{}%
\end{pgfscope}%
\end{pgfscope}%
\begin{pgfscope}%
\pgfsetbuttcap%
\pgfsetroundjoin%
\definecolor{currentfill}{rgb}{0.000000,0.000000,0.000000}%
\pgfsetfillcolor{currentfill}%
\pgfsetlinewidth{0.501875pt}%
\definecolor{currentstroke}{rgb}{0.000000,0.000000,0.000000}%
\pgfsetstrokecolor{currentstroke}%
\pgfsetdash{}{0pt}%
\pgfsys@defobject{currentmarker}{\pgfqpoint{0.000000in}{-0.020833in}}{\pgfqpoint{0.000000in}{0.000000in}}{%
\pgfpathmoveto{\pgfqpoint{0.000000in}{0.000000in}}%
\pgfpathlineto{\pgfqpoint{0.000000in}{-0.020833in}}%
\pgfusepath{stroke,fill}%
}%
\begin{pgfscope}%
\pgfsys@transformshift{2.949688in}{0.898923in}%
\pgfsys@useobject{currentmarker}{}%
\end{pgfscope}%
\end{pgfscope}%
\begin{pgfscope}%
\pgfsetbuttcap%
\pgfsetroundjoin%
\definecolor{currentfill}{rgb}{0.000000,0.000000,0.000000}%
\pgfsetfillcolor{currentfill}%
\pgfsetlinewidth{0.501875pt}%
\definecolor{currentstroke}{rgb}{0.000000,0.000000,0.000000}%
\pgfsetstrokecolor{currentstroke}%
\pgfsetdash{}{0pt}%
\pgfsys@defobject{currentmarker}{\pgfqpoint{0.000000in}{0.000000in}}{\pgfqpoint{0.000000in}{0.020833in}}{%
\pgfpathmoveto{\pgfqpoint{0.000000in}{0.000000in}}%
\pgfpathlineto{\pgfqpoint{0.000000in}{0.020833in}}%
\pgfusepath{stroke,fill}%
}%
\begin{pgfscope}%
\pgfsys@transformshift{3.141838in}{0.431673in}%
\pgfsys@useobject{currentmarker}{}%
\end{pgfscope}%
\end{pgfscope}%
\begin{pgfscope}%
\pgfsetbuttcap%
\pgfsetroundjoin%
\definecolor{currentfill}{rgb}{0.000000,0.000000,0.000000}%
\pgfsetfillcolor{currentfill}%
\pgfsetlinewidth{0.501875pt}%
\definecolor{currentstroke}{rgb}{0.000000,0.000000,0.000000}%
\pgfsetstrokecolor{currentstroke}%
\pgfsetdash{}{0pt}%
\pgfsys@defobject{currentmarker}{\pgfqpoint{0.000000in}{-0.020833in}}{\pgfqpoint{0.000000in}{0.000000in}}{%
\pgfpathmoveto{\pgfqpoint{0.000000in}{0.000000in}}%
\pgfpathlineto{\pgfqpoint{0.000000in}{-0.020833in}}%
\pgfusepath{stroke,fill}%
}%
\begin{pgfscope}%
\pgfsys@transformshift{3.141838in}{0.898923in}%
\pgfsys@useobject{currentmarker}{}%
\end{pgfscope}%
\end{pgfscope}%
\begin{pgfscope}%
\pgfsetbuttcap%
\pgfsetroundjoin%
\definecolor{currentfill}{rgb}{0.000000,0.000000,0.000000}%
\pgfsetfillcolor{currentfill}%
\pgfsetlinewidth{0.501875pt}%
\definecolor{currentstroke}{rgb}{0.000000,0.000000,0.000000}%
\pgfsetstrokecolor{currentstroke}%
\pgfsetdash{}{0pt}%
\pgfsys@defobject{currentmarker}{\pgfqpoint{0.000000in}{0.000000in}}{\pgfqpoint{0.000000in}{0.020833in}}{%
\pgfpathmoveto{\pgfqpoint{0.000000in}{0.000000in}}%
\pgfpathlineto{\pgfqpoint{0.000000in}{0.020833in}}%
\pgfusepath{stroke,fill}%
}%
\begin{pgfscope}%
\pgfsys@transformshift{3.237913in}{0.431673in}%
\pgfsys@useobject{currentmarker}{}%
\end{pgfscope}%
\end{pgfscope}%
\begin{pgfscope}%
\pgfsetbuttcap%
\pgfsetroundjoin%
\definecolor{currentfill}{rgb}{0.000000,0.000000,0.000000}%
\pgfsetfillcolor{currentfill}%
\pgfsetlinewidth{0.501875pt}%
\definecolor{currentstroke}{rgb}{0.000000,0.000000,0.000000}%
\pgfsetstrokecolor{currentstroke}%
\pgfsetdash{}{0pt}%
\pgfsys@defobject{currentmarker}{\pgfqpoint{0.000000in}{-0.020833in}}{\pgfqpoint{0.000000in}{0.000000in}}{%
\pgfpathmoveto{\pgfqpoint{0.000000in}{0.000000in}}%
\pgfpathlineto{\pgfqpoint{0.000000in}{-0.020833in}}%
\pgfusepath{stroke,fill}%
}%
\begin{pgfscope}%
\pgfsys@transformshift{3.237913in}{0.898923in}%
\pgfsys@useobject{currentmarker}{}%
\end{pgfscope}%
\end{pgfscope}%
\begin{pgfscope}%
\pgfsetbuttcap%
\pgfsetroundjoin%
\definecolor{currentfill}{rgb}{0.000000,0.000000,0.000000}%
\pgfsetfillcolor{currentfill}%
\pgfsetlinewidth{0.501875pt}%
\definecolor{currentstroke}{rgb}{0.000000,0.000000,0.000000}%
\pgfsetstrokecolor{currentstroke}%
\pgfsetdash{}{0pt}%
\pgfsys@defobject{currentmarker}{\pgfqpoint{0.000000in}{0.000000in}}{\pgfqpoint{0.000000in}{0.020833in}}{%
\pgfpathmoveto{\pgfqpoint{0.000000in}{0.000000in}}%
\pgfpathlineto{\pgfqpoint{0.000000in}{0.020833in}}%
\pgfusepath{stroke,fill}%
}%
\begin{pgfscope}%
\pgfsys@transformshift{3.333988in}{0.431673in}%
\pgfsys@useobject{currentmarker}{}%
\end{pgfscope}%
\end{pgfscope}%
\begin{pgfscope}%
\pgfsetbuttcap%
\pgfsetroundjoin%
\definecolor{currentfill}{rgb}{0.000000,0.000000,0.000000}%
\pgfsetfillcolor{currentfill}%
\pgfsetlinewidth{0.501875pt}%
\definecolor{currentstroke}{rgb}{0.000000,0.000000,0.000000}%
\pgfsetstrokecolor{currentstroke}%
\pgfsetdash{}{0pt}%
\pgfsys@defobject{currentmarker}{\pgfqpoint{0.000000in}{-0.020833in}}{\pgfqpoint{0.000000in}{0.000000in}}{%
\pgfpathmoveto{\pgfqpoint{0.000000in}{0.000000in}}%
\pgfpathlineto{\pgfqpoint{0.000000in}{-0.020833in}}%
\pgfusepath{stroke,fill}%
}%
\begin{pgfscope}%
\pgfsys@transformshift{3.333988in}{0.898923in}%
\pgfsys@useobject{currentmarker}{}%
\end{pgfscope}%
\end{pgfscope}%
\begin{pgfscope}%
\pgfsetbuttcap%
\pgfsetroundjoin%
\definecolor{currentfill}{rgb}{0.000000,0.000000,0.000000}%
\pgfsetfillcolor{currentfill}%
\pgfsetlinewidth{0.501875pt}%
\definecolor{currentstroke}{rgb}{0.000000,0.000000,0.000000}%
\pgfsetstrokecolor{currentstroke}%
\pgfsetdash{}{0pt}%
\pgfsys@defobject{currentmarker}{\pgfqpoint{0.000000in}{0.000000in}}{\pgfqpoint{0.000000in}{0.020833in}}{%
\pgfpathmoveto{\pgfqpoint{0.000000in}{0.000000in}}%
\pgfpathlineto{\pgfqpoint{0.000000in}{0.020833in}}%
\pgfusepath{stroke,fill}%
}%
\begin{pgfscope}%
\pgfsys@transformshift{3.430063in}{0.431673in}%
\pgfsys@useobject{currentmarker}{}%
\end{pgfscope}%
\end{pgfscope}%
\begin{pgfscope}%
\pgfsetbuttcap%
\pgfsetroundjoin%
\definecolor{currentfill}{rgb}{0.000000,0.000000,0.000000}%
\pgfsetfillcolor{currentfill}%
\pgfsetlinewidth{0.501875pt}%
\definecolor{currentstroke}{rgb}{0.000000,0.000000,0.000000}%
\pgfsetstrokecolor{currentstroke}%
\pgfsetdash{}{0pt}%
\pgfsys@defobject{currentmarker}{\pgfqpoint{0.000000in}{-0.020833in}}{\pgfqpoint{0.000000in}{0.000000in}}{%
\pgfpathmoveto{\pgfqpoint{0.000000in}{0.000000in}}%
\pgfpathlineto{\pgfqpoint{0.000000in}{-0.020833in}}%
\pgfusepath{stroke,fill}%
}%
\begin{pgfscope}%
\pgfsys@transformshift{3.430063in}{0.898923in}%
\pgfsys@useobject{currentmarker}{}%
\end{pgfscope}%
\end{pgfscope}%
\begin{pgfscope}%
\pgfsetbuttcap%
\pgfsetroundjoin%
\definecolor{currentfill}{rgb}{0.000000,0.000000,0.000000}%
\pgfsetfillcolor{currentfill}%
\pgfsetlinewidth{0.501875pt}%
\definecolor{currentstroke}{rgb}{0.000000,0.000000,0.000000}%
\pgfsetstrokecolor{currentstroke}%
\pgfsetdash{}{0pt}%
\pgfsys@defobject{currentmarker}{\pgfqpoint{0.000000in}{0.000000in}}{\pgfqpoint{0.000000in}{0.020833in}}{%
\pgfpathmoveto{\pgfqpoint{0.000000in}{0.000000in}}%
\pgfpathlineto{\pgfqpoint{0.000000in}{0.020833in}}%
\pgfusepath{stroke,fill}%
}%
\begin{pgfscope}%
\pgfsys@transformshift{3.622213in}{0.431673in}%
\pgfsys@useobject{currentmarker}{}%
\end{pgfscope}%
\end{pgfscope}%
\begin{pgfscope}%
\pgfsetbuttcap%
\pgfsetroundjoin%
\definecolor{currentfill}{rgb}{0.000000,0.000000,0.000000}%
\pgfsetfillcolor{currentfill}%
\pgfsetlinewidth{0.501875pt}%
\definecolor{currentstroke}{rgb}{0.000000,0.000000,0.000000}%
\pgfsetstrokecolor{currentstroke}%
\pgfsetdash{}{0pt}%
\pgfsys@defobject{currentmarker}{\pgfqpoint{0.000000in}{-0.020833in}}{\pgfqpoint{0.000000in}{0.000000in}}{%
\pgfpathmoveto{\pgfqpoint{0.000000in}{0.000000in}}%
\pgfpathlineto{\pgfqpoint{0.000000in}{-0.020833in}}%
\pgfusepath{stroke,fill}%
}%
\begin{pgfscope}%
\pgfsys@transformshift{3.622213in}{0.898923in}%
\pgfsys@useobject{currentmarker}{}%
\end{pgfscope}%
\end{pgfscope}%
\begin{pgfscope}%
\pgfsetbuttcap%
\pgfsetroundjoin%
\definecolor{currentfill}{rgb}{0.000000,0.000000,0.000000}%
\pgfsetfillcolor{currentfill}%
\pgfsetlinewidth{0.501875pt}%
\definecolor{currentstroke}{rgb}{0.000000,0.000000,0.000000}%
\pgfsetstrokecolor{currentstroke}%
\pgfsetdash{}{0pt}%
\pgfsys@defobject{currentmarker}{\pgfqpoint{0.000000in}{0.000000in}}{\pgfqpoint{0.000000in}{0.020833in}}{%
\pgfpathmoveto{\pgfqpoint{0.000000in}{0.000000in}}%
\pgfpathlineto{\pgfqpoint{0.000000in}{0.020833in}}%
\pgfusepath{stroke,fill}%
}%
\begin{pgfscope}%
\pgfsys@transformshift{3.718289in}{0.431673in}%
\pgfsys@useobject{currentmarker}{}%
\end{pgfscope}%
\end{pgfscope}%
\begin{pgfscope}%
\pgfsetbuttcap%
\pgfsetroundjoin%
\definecolor{currentfill}{rgb}{0.000000,0.000000,0.000000}%
\pgfsetfillcolor{currentfill}%
\pgfsetlinewidth{0.501875pt}%
\definecolor{currentstroke}{rgb}{0.000000,0.000000,0.000000}%
\pgfsetstrokecolor{currentstroke}%
\pgfsetdash{}{0pt}%
\pgfsys@defobject{currentmarker}{\pgfqpoint{0.000000in}{-0.020833in}}{\pgfqpoint{0.000000in}{0.000000in}}{%
\pgfpathmoveto{\pgfqpoint{0.000000in}{0.000000in}}%
\pgfpathlineto{\pgfqpoint{0.000000in}{-0.020833in}}%
\pgfusepath{stroke,fill}%
}%
\begin{pgfscope}%
\pgfsys@transformshift{3.718289in}{0.898923in}%
\pgfsys@useobject{currentmarker}{}%
\end{pgfscope}%
\end{pgfscope}%
\begin{pgfscope}%
\pgfsetbuttcap%
\pgfsetroundjoin%
\definecolor{currentfill}{rgb}{0.000000,0.000000,0.000000}%
\pgfsetfillcolor{currentfill}%
\pgfsetlinewidth{0.501875pt}%
\definecolor{currentstroke}{rgb}{0.000000,0.000000,0.000000}%
\pgfsetstrokecolor{currentstroke}%
\pgfsetdash{}{0pt}%
\pgfsys@defobject{currentmarker}{\pgfqpoint{0.000000in}{0.000000in}}{\pgfqpoint{0.000000in}{0.020833in}}{%
\pgfpathmoveto{\pgfqpoint{0.000000in}{0.000000in}}%
\pgfpathlineto{\pgfqpoint{0.000000in}{0.020833in}}%
\pgfusepath{stroke,fill}%
}%
\begin{pgfscope}%
\pgfsys@transformshift{3.814364in}{0.431673in}%
\pgfsys@useobject{currentmarker}{}%
\end{pgfscope}%
\end{pgfscope}%
\begin{pgfscope}%
\pgfsetbuttcap%
\pgfsetroundjoin%
\definecolor{currentfill}{rgb}{0.000000,0.000000,0.000000}%
\pgfsetfillcolor{currentfill}%
\pgfsetlinewidth{0.501875pt}%
\definecolor{currentstroke}{rgb}{0.000000,0.000000,0.000000}%
\pgfsetstrokecolor{currentstroke}%
\pgfsetdash{}{0pt}%
\pgfsys@defobject{currentmarker}{\pgfqpoint{0.000000in}{-0.020833in}}{\pgfqpoint{0.000000in}{0.000000in}}{%
\pgfpathmoveto{\pgfqpoint{0.000000in}{0.000000in}}%
\pgfpathlineto{\pgfqpoint{0.000000in}{-0.020833in}}%
\pgfusepath{stroke,fill}%
}%
\begin{pgfscope}%
\pgfsys@transformshift{3.814364in}{0.898923in}%
\pgfsys@useobject{currentmarker}{}%
\end{pgfscope}%
\end{pgfscope}%
\begin{pgfscope}%
\pgfsetbuttcap%
\pgfsetroundjoin%
\definecolor{currentfill}{rgb}{0.000000,0.000000,0.000000}%
\pgfsetfillcolor{currentfill}%
\pgfsetlinewidth{0.501875pt}%
\definecolor{currentstroke}{rgb}{0.000000,0.000000,0.000000}%
\pgfsetstrokecolor{currentstroke}%
\pgfsetdash{}{0pt}%
\pgfsys@defobject{currentmarker}{\pgfqpoint{0.000000in}{0.000000in}}{\pgfqpoint{0.000000in}{0.020833in}}{%
\pgfpathmoveto{\pgfqpoint{0.000000in}{0.000000in}}%
\pgfpathlineto{\pgfqpoint{0.000000in}{0.020833in}}%
\pgfusepath{stroke,fill}%
}%
\begin{pgfscope}%
\pgfsys@transformshift{3.910439in}{0.431673in}%
\pgfsys@useobject{currentmarker}{}%
\end{pgfscope}%
\end{pgfscope}%
\begin{pgfscope}%
\pgfsetbuttcap%
\pgfsetroundjoin%
\definecolor{currentfill}{rgb}{0.000000,0.000000,0.000000}%
\pgfsetfillcolor{currentfill}%
\pgfsetlinewidth{0.501875pt}%
\definecolor{currentstroke}{rgb}{0.000000,0.000000,0.000000}%
\pgfsetstrokecolor{currentstroke}%
\pgfsetdash{}{0pt}%
\pgfsys@defobject{currentmarker}{\pgfqpoint{0.000000in}{-0.020833in}}{\pgfqpoint{0.000000in}{0.000000in}}{%
\pgfpathmoveto{\pgfqpoint{0.000000in}{0.000000in}}%
\pgfpathlineto{\pgfqpoint{0.000000in}{-0.020833in}}%
\pgfusepath{stroke,fill}%
}%
\begin{pgfscope}%
\pgfsys@transformshift{3.910439in}{0.898923in}%
\pgfsys@useobject{currentmarker}{}%
\end{pgfscope}%
\end{pgfscope}%
\begin{pgfscope}%
\pgfsetbuttcap%
\pgfsetroundjoin%
\definecolor{currentfill}{rgb}{0.000000,0.000000,0.000000}%
\pgfsetfillcolor{currentfill}%
\pgfsetlinewidth{0.501875pt}%
\definecolor{currentstroke}{rgb}{0.000000,0.000000,0.000000}%
\pgfsetstrokecolor{currentstroke}%
\pgfsetdash{}{0pt}%
\pgfsys@defobject{currentmarker}{\pgfqpoint{0.000000in}{0.000000in}}{\pgfqpoint{0.000000in}{0.020833in}}{%
\pgfpathmoveto{\pgfqpoint{0.000000in}{0.000000in}}%
\pgfpathlineto{\pgfqpoint{0.000000in}{0.020833in}}%
\pgfusepath{stroke,fill}%
}%
\begin{pgfscope}%
\pgfsys@transformshift{4.102589in}{0.431673in}%
\pgfsys@useobject{currentmarker}{}%
\end{pgfscope}%
\end{pgfscope}%
\begin{pgfscope}%
\pgfsetbuttcap%
\pgfsetroundjoin%
\definecolor{currentfill}{rgb}{0.000000,0.000000,0.000000}%
\pgfsetfillcolor{currentfill}%
\pgfsetlinewidth{0.501875pt}%
\definecolor{currentstroke}{rgb}{0.000000,0.000000,0.000000}%
\pgfsetstrokecolor{currentstroke}%
\pgfsetdash{}{0pt}%
\pgfsys@defobject{currentmarker}{\pgfqpoint{0.000000in}{-0.020833in}}{\pgfqpoint{0.000000in}{0.000000in}}{%
\pgfpathmoveto{\pgfqpoint{0.000000in}{0.000000in}}%
\pgfpathlineto{\pgfqpoint{0.000000in}{-0.020833in}}%
\pgfusepath{stroke,fill}%
}%
\begin{pgfscope}%
\pgfsys@transformshift{4.102589in}{0.898923in}%
\pgfsys@useobject{currentmarker}{}%
\end{pgfscope}%
\end{pgfscope}%
\begin{pgfscope}%
\pgfsetbuttcap%
\pgfsetroundjoin%
\definecolor{currentfill}{rgb}{0.000000,0.000000,0.000000}%
\pgfsetfillcolor{currentfill}%
\pgfsetlinewidth{0.501875pt}%
\definecolor{currentstroke}{rgb}{0.000000,0.000000,0.000000}%
\pgfsetstrokecolor{currentstroke}%
\pgfsetdash{}{0pt}%
\pgfsys@defobject{currentmarker}{\pgfqpoint{0.000000in}{0.000000in}}{\pgfqpoint{0.000000in}{0.020833in}}{%
\pgfpathmoveto{\pgfqpoint{0.000000in}{0.000000in}}%
\pgfpathlineto{\pgfqpoint{0.000000in}{0.020833in}}%
\pgfusepath{stroke,fill}%
}%
\begin{pgfscope}%
\pgfsys@transformshift{4.198664in}{0.431673in}%
\pgfsys@useobject{currentmarker}{}%
\end{pgfscope}%
\end{pgfscope}%
\begin{pgfscope}%
\pgfsetbuttcap%
\pgfsetroundjoin%
\definecolor{currentfill}{rgb}{0.000000,0.000000,0.000000}%
\pgfsetfillcolor{currentfill}%
\pgfsetlinewidth{0.501875pt}%
\definecolor{currentstroke}{rgb}{0.000000,0.000000,0.000000}%
\pgfsetstrokecolor{currentstroke}%
\pgfsetdash{}{0pt}%
\pgfsys@defobject{currentmarker}{\pgfqpoint{0.000000in}{-0.020833in}}{\pgfqpoint{0.000000in}{0.000000in}}{%
\pgfpathmoveto{\pgfqpoint{0.000000in}{0.000000in}}%
\pgfpathlineto{\pgfqpoint{0.000000in}{-0.020833in}}%
\pgfusepath{stroke,fill}%
}%
\begin{pgfscope}%
\pgfsys@transformshift{4.198664in}{0.898923in}%
\pgfsys@useobject{currentmarker}{}%
\end{pgfscope}%
\end{pgfscope}%
\begin{pgfscope}%
\pgfsetbuttcap%
\pgfsetroundjoin%
\definecolor{currentfill}{rgb}{0.000000,0.000000,0.000000}%
\pgfsetfillcolor{currentfill}%
\pgfsetlinewidth{0.501875pt}%
\definecolor{currentstroke}{rgb}{0.000000,0.000000,0.000000}%
\pgfsetstrokecolor{currentstroke}%
\pgfsetdash{}{0pt}%
\pgfsys@defobject{currentmarker}{\pgfqpoint{0.000000in}{0.000000in}}{\pgfqpoint{0.000000in}{0.020833in}}{%
\pgfpathmoveto{\pgfqpoint{0.000000in}{0.000000in}}%
\pgfpathlineto{\pgfqpoint{0.000000in}{0.020833in}}%
\pgfusepath{stroke,fill}%
}%
\begin{pgfscope}%
\pgfsys@transformshift{4.294739in}{0.431673in}%
\pgfsys@useobject{currentmarker}{}%
\end{pgfscope}%
\end{pgfscope}%
\begin{pgfscope}%
\pgfsetbuttcap%
\pgfsetroundjoin%
\definecolor{currentfill}{rgb}{0.000000,0.000000,0.000000}%
\pgfsetfillcolor{currentfill}%
\pgfsetlinewidth{0.501875pt}%
\definecolor{currentstroke}{rgb}{0.000000,0.000000,0.000000}%
\pgfsetstrokecolor{currentstroke}%
\pgfsetdash{}{0pt}%
\pgfsys@defobject{currentmarker}{\pgfqpoint{0.000000in}{-0.020833in}}{\pgfqpoint{0.000000in}{0.000000in}}{%
\pgfpathmoveto{\pgfqpoint{0.000000in}{0.000000in}}%
\pgfpathlineto{\pgfqpoint{0.000000in}{-0.020833in}}%
\pgfusepath{stroke,fill}%
}%
\begin{pgfscope}%
\pgfsys@transformshift{4.294739in}{0.898923in}%
\pgfsys@useobject{currentmarker}{}%
\end{pgfscope}%
\end{pgfscope}%
\begin{pgfscope}%
\pgfsetbuttcap%
\pgfsetroundjoin%
\definecolor{currentfill}{rgb}{0.000000,0.000000,0.000000}%
\pgfsetfillcolor{currentfill}%
\pgfsetlinewidth{0.501875pt}%
\definecolor{currentstroke}{rgb}{0.000000,0.000000,0.000000}%
\pgfsetstrokecolor{currentstroke}%
\pgfsetdash{}{0pt}%
\pgfsys@defobject{currentmarker}{\pgfqpoint{0.000000in}{0.000000in}}{\pgfqpoint{0.000000in}{0.020833in}}{%
\pgfpathmoveto{\pgfqpoint{0.000000in}{0.000000in}}%
\pgfpathlineto{\pgfqpoint{0.000000in}{0.020833in}}%
\pgfusepath{stroke,fill}%
}%
\begin{pgfscope}%
\pgfsys@transformshift{4.390814in}{0.431673in}%
\pgfsys@useobject{currentmarker}{}%
\end{pgfscope}%
\end{pgfscope}%
\begin{pgfscope}%
\pgfsetbuttcap%
\pgfsetroundjoin%
\definecolor{currentfill}{rgb}{0.000000,0.000000,0.000000}%
\pgfsetfillcolor{currentfill}%
\pgfsetlinewidth{0.501875pt}%
\definecolor{currentstroke}{rgb}{0.000000,0.000000,0.000000}%
\pgfsetstrokecolor{currentstroke}%
\pgfsetdash{}{0pt}%
\pgfsys@defobject{currentmarker}{\pgfqpoint{0.000000in}{-0.020833in}}{\pgfqpoint{0.000000in}{0.000000in}}{%
\pgfpathmoveto{\pgfqpoint{0.000000in}{0.000000in}}%
\pgfpathlineto{\pgfqpoint{0.000000in}{-0.020833in}}%
\pgfusepath{stroke,fill}%
}%
\begin{pgfscope}%
\pgfsys@transformshift{4.390814in}{0.898923in}%
\pgfsys@useobject{currentmarker}{}%
\end{pgfscope}%
\end{pgfscope}%
\begin{pgfscope}%
\pgfsetbuttcap%
\pgfsetroundjoin%
\definecolor{currentfill}{rgb}{0.000000,0.000000,0.000000}%
\pgfsetfillcolor{currentfill}%
\pgfsetlinewidth{0.501875pt}%
\definecolor{currentstroke}{rgb}{0.000000,0.000000,0.000000}%
\pgfsetstrokecolor{currentstroke}%
\pgfsetdash{}{0pt}%
\pgfsys@defobject{currentmarker}{\pgfqpoint{0.000000in}{0.000000in}}{\pgfqpoint{0.000000in}{0.020833in}}{%
\pgfpathmoveto{\pgfqpoint{0.000000in}{0.000000in}}%
\pgfpathlineto{\pgfqpoint{0.000000in}{0.020833in}}%
\pgfusepath{stroke,fill}%
}%
\begin{pgfscope}%
\pgfsys@transformshift{4.582964in}{0.431673in}%
\pgfsys@useobject{currentmarker}{}%
\end{pgfscope}%
\end{pgfscope}%
\begin{pgfscope}%
\pgfsetbuttcap%
\pgfsetroundjoin%
\definecolor{currentfill}{rgb}{0.000000,0.000000,0.000000}%
\pgfsetfillcolor{currentfill}%
\pgfsetlinewidth{0.501875pt}%
\definecolor{currentstroke}{rgb}{0.000000,0.000000,0.000000}%
\pgfsetstrokecolor{currentstroke}%
\pgfsetdash{}{0pt}%
\pgfsys@defobject{currentmarker}{\pgfqpoint{0.000000in}{-0.020833in}}{\pgfqpoint{0.000000in}{0.000000in}}{%
\pgfpathmoveto{\pgfqpoint{0.000000in}{0.000000in}}%
\pgfpathlineto{\pgfqpoint{0.000000in}{-0.020833in}}%
\pgfusepath{stroke,fill}%
}%
\begin{pgfscope}%
\pgfsys@transformshift{4.582964in}{0.898923in}%
\pgfsys@useobject{currentmarker}{}%
\end{pgfscope}%
\end{pgfscope}%
\begin{pgfscope}%
\definecolor{textcolor}{rgb}{0.000000,0.000000,0.000000}%
\pgfsetstrokecolor{textcolor}%
\pgfsetfillcolor{textcolor}%
\pgftext[x=2.560458in,y=0.201367in,,top]{\color{textcolor}\rmfamily\fontsize{12.000000}{14.400000}\selectfont \(\displaystyle \mu_0 H\) (\unit{mT})}%
\end{pgfscope}%
\begin{pgfscope}%
\pgfsetbuttcap%
\pgfsetroundjoin%
\definecolor{currentfill}{rgb}{0.000000,0.000000,0.000000}%
\pgfsetfillcolor{currentfill}%
\pgfsetlinewidth{0.501875pt}%
\definecolor{currentstroke}{rgb}{0.000000,0.000000,0.000000}%
\pgfsetstrokecolor{currentstroke}%
\pgfsetdash{}{0pt}%
\pgfsys@defobject{currentmarker}{\pgfqpoint{0.000000in}{0.000000in}}{\pgfqpoint{0.041667in}{0.000000in}}{%
\pgfpathmoveto{\pgfqpoint{0.000000in}{0.000000in}}%
\pgfpathlineto{\pgfqpoint{0.041667in}{0.000000in}}%
\pgfusepath{stroke,fill}%
}%
\begin{pgfscope}%
\pgfsys@transformshift{0.444748in}{0.474955in}%
\pgfsys@useobject{currentmarker}{}%
\end{pgfscope}%
\end{pgfscope}%
\begin{pgfscope}%
\pgfsetbuttcap%
\pgfsetroundjoin%
\definecolor{currentfill}{rgb}{0.000000,0.000000,0.000000}%
\pgfsetfillcolor{currentfill}%
\pgfsetlinewidth{0.501875pt}%
\definecolor{currentstroke}{rgb}{0.000000,0.000000,0.000000}%
\pgfsetstrokecolor{currentstroke}%
\pgfsetdash{}{0pt}%
\pgfsys@defobject{currentmarker}{\pgfqpoint{-0.041667in}{0.000000in}}{\pgfqpoint{-0.000000in}{0.000000in}}{%
\pgfpathmoveto{\pgfqpoint{-0.000000in}{0.000000in}}%
\pgfpathlineto{\pgfqpoint{-0.041667in}{0.000000in}}%
\pgfusepath{stroke,fill}%
}%
\begin{pgfscope}%
\pgfsys@transformshift{4.676167in}{0.474955in}%
\pgfsys@useobject{currentmarker}{}%
\end{pgfscope}%
\end{pgfscope}%
\begin{pgfscope}%
\definecolor{textcolor}{rgb}{0.000000,0.000000,0.000000}%
\pgfsetstrokecolor{textcolor}%
\pgfsetfillcolor{textcolor}%
\pgftext[x=0.257248in, y=0.426737in, left, base]{\color{textcolor}\rmfamily\fontsize{10.000000}{12.000000}\selectfont \(\displaystyle {25}\)}%
\end{pgfscope}%
\begin{pgfscope}%
\pgfsetbuttcap%
\pgfsetroundjoin%
\definecolor{currentfill}{rgb}{0.000000,0.000000,0.000000}%
\pgfsetfillcolor{currentfill}%
\pgfsetlinewidth{0.501875pt}%
\definecolor{currentstroke}{rgb}{0.000000,0.000000,0.000000}%
\pgfsetstrokecolor{currentstroke}%
\pgfsetdash{}{0pt}%
\pgfsys@defobject{currentmarker}{\pgfqpoint{0.000000in}{0.000000in}}{\pgfqpoint{0.041667in}{0.000000in}}{%
\pgfpathmoveto{\pgfqpoint{0.000000in}{0.000000in}}%
\pgfpathlineto{\pgfqpoint{0.041667in}{0.000000in}}%
\pgfusepath{stroke,fill}%
}%
\begin{pgfscope}%
\pgfsys@transformshift{0.444748in}{0.618491in}%
\pgfsys@useobject{currentmarker}{}%
\end{pgfscope}%
\end{pgfscope}%
\begin{pgfscope}%
\pgfsetbuttcap%
\pgfsetroundjoin%
\definecolor{currentfill}{rgb}{0.000000,0.000000,0.000000}%
\pgfsetfillcolor{currentfill}%
\pgfsetlinewidth{0.501875pt}%
\definecolor{currentstroke}{rgb}{0.000000,0.000000,0.000000}%
\pgfsetstrokecolor{currentstroke}%
\pgfsetdash{}{0pt}%
\pgfsys@defobject{currentmarker}{\pgfqpoint{-0.041667in}{0.000000in}}{\pgfqpoint{-0.000000in}{0.000000in}}{%
\pgfpathmoveto{\pgfqpoint{-0.000000in}{0.000000in}}%
\pgfpathlineto{\pgfqpoint{-0.041667in}{0.000000in}}%
\pgfusepath{stroke,fill}%
}%
\begin{pgfscope}%
\pgfsys@transformshift{4.676167in}{0.618491in}%
\pgfsys@useobject{currentmarker}{}%
\end{pgfscope}%
\end{pgfscope}%
\begin{pgfscope}%
\definecolor{textcolor}{rgb}{0.000000,0.000000,0.000000}%
\pgfsetstrokecolor{textcolor}%
\pgfsetfillcolor{textcolor}%
\pgftext[x=0.257248in, y=0.570274in, left, base]{\color{textcolor}\rmfamily\fontsize{10.000000}{12.000000}\selectfont \(\displaystyle {50}\)}%
\end{pgfscope}%
\begin{pgfscope}%
\pgfsetbuttcap%
\pgfsetroundjoin%
\definecolor{currentfill}{rgb}{0.000000,0.000000,0.000000}%
\pgfsetfillcolor{currentfill}%
\pgfsetlinewidth{0.501875pt}%
\definecolor{currentstroke}{rgb}{0.000000,0.000000,0.000000}%
\pgfsetstrokecolor{currentstroke}%
\pgfsetdash{}{0pt}%
\pgfsys@defobject{currentmarker}{\pgfqpoint{0.000000in}{0.000000in}}{\pgfqpoint{0.041667in}{0.000000in}}{%
\pgfpathmoveto{\pgfqpoint{0.000000in}{0.000000in}}%
\pgfpathlineto{\pgfqpoint{0.041667in}{0.000000in}}%
\pgfusepath{stroke,fill}%
}%
\begin{pgfscope}%
\pgfsys@transformshift{0.444748in}{0.762028in}%
\pgfsys@useobject{currentmarker}{}%
\end{pgfscope}%
\end{pgfscope}%
\begin{pgfscope}%
\pgfsetbuttcap%
\pgfsetroundjoin%
\definecolor{currentfill}{rgb}{0.000000,0.000000,0.000000}%
\pgfsetfillcolor{currentfill}%
\pgfsetlinewidth{0.501875pt}%
\definecolor{currentstroke}{rgb}{0.000000,0.000000,0.000000}%
\pgfsetstrokecolor{currentstroke}%
\pgfsetdash{}{0pt}%
\pgfsys@defobject{currentmarker}{\pgfqpoint{-0.041667in}{0.000000in}}{\pgfqpoint{-0.000000in}{0.000000in}}{%
\pgfpathmoveto{\pgfqpoint{-0.000000in}{0.000000in}}%
\pgfpathlineto{\pgfqpoint{-0.041667in}{0.000000in}}%
\pgfusepath{stroke,fill}%
}%
\begin{pgfscope}%
\pgfsys@transformshift{4.676167in}{0.762028in}%
\pgfsys@useobject{currentmarker}{}%
\end{pgfscope}%
\end{pgfscope}%
\begin{pgfscope}%
\definecolor{textcolor}{rgb}{0.000000,0.000000,0.000000}%
\pgfsetstrokecolor{textcolor}%
\pgfsetfillcolor{textcolor}%
\pgftext[x=0.257248in, y=0.713810in, left, base]{\color{textcolor}\rmfamily\fontsize{10.000000}{12.000000}\selectfont \(\displaystyle {75}\)}%
\end{pgfscope}%
\begin{pgfscope}%
\pgfsetbuttcap%
\pgfsetroundjoin%
\definecolor{currentfill}{rgb}{0.000000,0.000000,0.000000}%
\pgfsetfillcolor{currentfill}%
\pgfsetlinewidth{0.501875pt}%
\definecolor{currentstroke}{rgb}{0.000000,0.000000,0.000000}%
\pgfsetstrokecolor{currentstroke}%
\pgfsetdash{}{0pt}%
\pgfsys@defobject{currentmarker}{\pgfqpoint{0.000000in}{0.000000in}}{\pgfqpoint{0.020833in}{0.000000in}}{%
\pgfpathmoveto{\pgfqpoint{0.000000in}{0.000000in}}%
\pgfpathlineto{\pgfqpoint{0.020833in}{0.000000in}}%
\pgfusepath{stroke,fill}%
}%
\begin{pgfscope}%
\pgfsys@transformshift{0.444748in}{0.446248in}%
\pgfsys@useobject{currentmarker}{}%
\end{pgfscope}%
\end{pgfscope}%
\begin{pgfscope}%
\pgfsetbuttcap%
\pgfsetroundjoin%
\definecolor{currentfill}{rgb}{0.000000,0.000000,0.000000}%
\pgfsetfillcolor{currentfill}%
\pgfsetlinewidth{0.501875pt}%
\definecolor{currentstroke}{rgb}{0.000000,0.000000,0.000000}%
\pgfsetstrokecolor{currentstroke}%
\pgfsetdash{}{0pt}%
\pgfsys@defobject{currentmarker}{\pgfqpoint{-0.020833in}{0.000000in}}{\pgfqpoint{-0.000000in}{0.000000in}}{%
\pgfpathmoveto{\pgfqpoint{-0.000000in}{0.000000in}}%
\pgfpathlineto{\pgfqpoint{-0.020833in}{0.000000in}}%
\pgfusepath{stroke,fill}%
}%
\begin{pgfscope}%
\pgfsys@transformshift{4.676167in}{0.446248in}%
\pgfsys@useobject{currentmarker}{}%
\end{pgfscope}%
\end{pgfscope}%
\begin{pgfscope}%
\pgfsetbuttcap%
\pgfsetroundjoin%
\definecolor{currentfill}{rgb}{0.000000,0.000000,0.000000}%
\pgfsetfillcolor{currentfill}%
\pgfsetlinewidth{0.501875pt}%
\definecolor{currentstroke}{rgb}{0.000000,0.000000,0.000000}%
\pgfsetstrokecolor{currentstroke}%
\pgfsetdash{}{0pt}%
\pgfsys@defobject{currentmarker}{\pgfqpoint{0.000000in}{0.000000in}}{\pgfqpoint{0.020833in}{0.000000in}}{%
\pgfpathmoveto{\pgfqpoint{0.000000in}{0.000000in}}%
\pgfpathlineto{\pgfqpoint{0.020833in}{0.000000in}}%
\pgfusepath{stroke,fill}%
}%
\begin{pgfscope}%
\pgfsys@transformshift{0.444748in}{0.503662in}%
\pgfsys@useobject{currentmarker}{}%
\end{pgfscope}%
\end{pgfscope}%
\begin{pgfscope}%
\pgfsetbuttcap%
\pgfsetroundjoin%
\definecolor{currentfill}{rgb}{0.000000,0.000000,0.000000}%
\pgfsetfillcolor{currentfill}%
\pgfsetlinewidth{0.501875pt}%
\definecolor{currentstroke}{rgb}{0.000000,0.000000,0.000000}%
\pgfsetstrokecolor{currentstroke}%
\pgfsetdash{}{0pt}%
\pgfsys@defobject{currentmarker}{\pgfqpoint{-0.020833in}{0.000000in}}{\pgfqpoint{-0.000000in}{0.000000in}}{%
\pgfpathmoveto{\pgfqpoint{-0.000000in}{0.000000in}}%
\pgfpathlineto{\pgfqpoint{-0.020833in}{0.000000in}}%
\pgfusepath{stroke,fill}%
}%
\begin{pgfscope}%
\pgfsys@transformshift{4.676167in}{0.503662in}%
\pgfsys@useobject{currentmarker}{}%
\end{pgfscope}%
\end{pgfscope}%
\begin{pgfscope}%
\pgfsetbuttcap%
\pgfsetroundjoin%
\definecolor{currentfill}{rgb}{0.000000,0.000000,0.000000}%
\pgfsetfillcolor{currentfill}%
\pgfsetlinewidth{0.501875pt}%
\definecolor{currentstroke}{rgb}{0.000000,0.000000,0.000000}%
\pgfsetstrokecolor{currentstroke}%
\pgfsetdash{}{0pt}%
\pgfsys@defobject{currentmarker}{\pgfqpoint{0.000000in}{0.000000in}}{\pgfqpoint{0.020833in}{0.000000in}}{%
\pgfpathmoveto{\pgfqpoint{0.000000in}{0.000000in}}%
\pgfpathlineto{\pgfqpoint{0.020833in}{0.000000in}}%
\pgfusepath{stroke,fill}%
}%
\begin{pgfscope}%
\pgfsys@transformshift{0.444748in}{0.532370in}%
\pgfsys@useobject{currentmarker}{}%
\end{pgfscope}%
\end{pgfscope}%
\begin{pgfscope}%
\pgfsetbuttcap%
\pgfsetroundjoin%
\definecolor{currentfill}{rgb}{0.000000,0.000000,0.000000}%
\pgfsetfillcolor{currentfill}%
\pgfsetlinewidth{0.501875pt}%
\definecolor{currentstroke}{rgb}{0.000000,0.000000,0.000000}%
\pgfsetstrokecolor{currentstroke}%
\pgfsetdash{}{0pt}%
\pgfsys@defobject{currentmarker}{\pgfqpoint{-0.020833in}{0.000000in}}{\pgfqpoint{-0.000000in}{0.000000in}}{%
\pgfpathmoveto{\pgfqpoint{-0.000000in}{0.000000in}}%
\pgfpathlineto{\pgfqpoint{-0.020833in}{0.000000in}}%
\pgfusepath{stroke,fill}%
}%
\begin{pgfscope}%
\pgfsys@transformshift{4.676167in}{0.532370in}%
\pgfsys@useobject{currentmarker}{}%
\end{pgfscope}%
\end{pgfscope}%
\begin{pgfscope}%
\pgfsetbuttcap%
\pgfsetroundjoin%
\definecolor{currentfill}{rgb}{0.000000,0.000000,0.000000}%
\pgfsetfillcolor{currentfill}%
\pgfsetlinewidth{0.501875pt}%
\definecolor{currentstroke}{rgb}{0.000000,0.000000,0.000000}%
\pgfsetstrokecolor{currentstroke}%
\pgfsetdash{}{0pt}%
\pgfsys@defobject{currentmarker}{\pgfqpoint{0.000000in}{0.000000in}}{\pgfqpoint{0.020833in}{0.000000in}}{%
\pgfpathmoveto{\pgfqpoint{0.000000in}{0.000000in}}%
\pgfpathlineto{\pgfqpoint{0.020833in}{0.000000in}}%
\pgfusepath{stroke,fill}%
}%
\begin{pgfscope}%
\pgfsys@transformshift{0.444748in}{0.561077in}%
\pgfsys@useobject{currentmarker}{}%
\end{pgfscope}%
\end{pgfscope}%
\begin{pgfscope}%
\pgfsetbuttcap%
\pgfsetroundjoin%
\definecolor{currentfill}{rgb}{0.000000,0.000000,0.000000}%
\pgfsetfillcolor{currentfill}%
\pgfsetlinewidth{0.501875pt}%
\definecolor{currentstroke}{rgb}{0.000000,0.000000,0.000000}%
\pgfsetstrokecolor{currentstroke}%
\pgfsetdash{}{0pt}%
\pgfsys@defobject{currentmarker}{\pgfqpoint{-0.020833in}{0.000000in}}{\pgfqpoint{-0.000000in}{0.000000in}}{%
\pgfpathmoveto{\pgfqpoint{-0.000000in}{0.000000in}}%
\pgfpathlineto{\pgfqpoint{-0.020833in}{0.000000in}}%
\pgfusepath{stroke,fill}%
}%
\begin{pgfscope}%
\pgfsys@transformshift{4.676167in}{0.561077in}%
\pgfsys@useobject{currentmarker}{}%
\end{pgfscope}%
\end{pgfscope}%
\begin{pgfscope}%
\pgfsetbuttcap%
\pgfsetroundjoin%
\definecolor{currentfill}{rgb}{0.000000,0.000000,0.000000}%
\pgfsetfillcolor{currentfill}%
\pgfsetlinewidth{0.501875pt}%
\definecolor{currentstroke}{rgb}{0.000000,0.000000,0.000000}%
\pgfsetstrokecolor{currentstroke}%
\pgfsetdash{}{0pt}%
\pgfsys@defobject{currentmarker}{\pgfqpoint{0.000000in}{0.000000in}}{\pgfqpoint{0.020833in}{0.000000in}}{%
\pgfpathmoveto{\pgfqpoint{0.000000in}{0.000000in}}%
\pgfpathlineto{\pgfqpoint{0.020833in}{0.000000in}}%
\pgfusepath{stroke,fill}%
}%
\begin{pgfscope}%
\pgfsys@transformshift{0.444748in}{0.589784in}%
\pgfsys@useobject{currentmarker}{}%
\end{pgfscope}%
\end{pgfscope}%
\begin{pgfscope}%
\pgfsetbuttcap%
\pgfsetroundjoin%
\definecolor{currentfill}{rgb}{0.000000,0.000000,0.000000}%
\pgfsetfillcolor{currentfill}%
\pgfsetlinewidth{0.501875pt}%
\definecolor{currentstroke}{rgb}{0.000000,0.000000,0.000000}%
\pgfsetstrokecolor{currentstroke}%
\pgfsetdash{}{0pt}%
\pgfsys@defobject{currentmarker}{\pgfqpoint{-0.020833in}{0.000000in}}{\pgfqpoint{-0.000000in}{0.000000in}}{%
\pgfpathmoveto{\pgfqpoint{-0.000000in}{0.000000in}}%
\pgfpathlineto{\pgfqpoint{-0.020833in}{0.000000in}}%
\pgfusepath{stroke,fill}%
}%
\begin{pgfscope}%
\pgfsys@transformshift{4.676167in}{0.589784in}%
\pgfsys@useobject{currentmarker}{}%
\end{pgfscope}%
\end{pgfscope}%
\begin{pgfscope}%
\pgfsetbuttcap%
\pgfsetroundjoin%
\definecolor{currentfill}{rgb}{0.000000,0.000000,0.000000}%
\pgfsetfillcolor{currentfill}%
\pgfsetlinewidth{0.501875pt}%
\definecolor{currentstroke}{rgb}{0.000000,0.000000,0.000000}%
\pgfsetstrokecolor{currentstroke}%
\pgfsetdash{}{0pt}%
\pgfsys@defobject{currentmarker}{\pgfqpoint{0.000000in}{0.000000in}}{\pgfqpoint{0.020833in}{0.000000in}}{%
\pgfpathmoveto{\pgfqpoint{0.000000in}{0.000000in}}%
\pgfpathlineto{\pgfqpoint{0.020833in}{0.000000in}}%
\pgfusepath{stroke,fill}%
}%
\begin{pgfscope}%
\pgfsys@transformshift{0.444748in}{0.647199in}%
\pgfsys@useobject{currentmarker}{}%
\end{pgfscope}%
\end{pgfscope}%
\begin{pgfscope}%
\pgfsetbuttcap%
\pgfsetroundjoin%
\definecolor{currentfill}{rgb}{0.000000,0.000000,0.000000}%
\pgfsetfillcolor{currentfill}%
\pgfsetlinewidth{0.501875pt}%
\definecolor{currentstroke}{rgb}{0.000000,0.000000,0.000000}%
\pgfsetstrokecolor{currentstroke}%
\pgfsetdash{}{0pt}%
\pgfsys@defobject{currentmarker}{\pgfqpoint{-0.020833in}{0.000000in}}{\pgfqpoint{-0.000000in}{0.000000in}}{%
\pgfpathmoveto{\pgfqpoint{-0.000000in}{0.000000in}}%
\pgfpathlineto{\pgfqpoint{-0.020833in}{0.000000in}}%
\pgfusepath{stroke,fill}%
}%
\begin{pgfscope}%
\pgfsys@transformshift{4.676167in}{0.647199in}%
\pgfsys@useobject{currentmarker}{}%
\end{pgfscope}%
\end{pgfscope}%
\begin{pgfscope}%
\pgfsetbuttcap%
\pgfsetroundjoin%
\definecolor{currentfill}{rgb}{0.000000,0.000000,0.000000}%
\pgfsetfillcolor{currentfill}%
\pgfsetlinewidth{0.501875pt}%
\definecolor{currentstroke}{rgb}{0.000000,0.000000,0.000000}%
\pgfsetstrokecolor{currentstroke}%
\pgfsetdash{}{0pt}%
\pgfsys@defobject{currentmarker}{\pgfqpoint{0.000000in}{0.000000in}}{\pgfqpoint{0.020833in}{0.000000in}}{%
\pgfpathmoveto{\pgfqpoint{0.000000in}{0.000000in}}%
\pgfpathlineto{\pgfqpoint{0.020833in}{0.000000in}}%
\pgfusepath{stroke,fill}%
}%
\begin{pgfscope}%
\pgfsys@transformshift{0.444748in}{0.675906in}%
\pgfsys@useobject{currentmarker}{}%
\end{pgfscope}%
\end{pgfscope}%
\begin{pgfscope}%
\pgfsetbuttcap%
\pgfsetroundjoin%
\definecolor{currentfill}{rgb}{0.000000,0.000000,0.000000}%
\pgfsetfillcolor{currentfill}%
\pgfsetlinewidth{0.501875pt}%
\definecolor{currentstroke}{rgb}{0.000000,0.000000,0.000000}%
\pgfsetstrokecolor{currentstroke}%
\pgfsetdash{}{0pt}%
\pgfsys@defobject{currentmarker}{\pgfqpoint{-0.020833in}{0.000000in}}{\pgfqpoint{-0.000000in}{0.000000in}}{%
\pgfpathmoveto{\pgfqpoint{-0.000000in}{0.000000in}}%
\pgfpathlineto{\pgfqpoint{-0.020833in}{0.000000in}}%
\pgfusepath{stroke,fill}%
}%
\begin{pgfscope}%
\pgfsys@transformshift{4.676167in}{0.675906in}%
\pgfsys@useobject{currentmarker}{}%
\end{pgfscope}%
\end{pgfscope}%
\begin{pgfscope}%
\pgfsetbuttcap%
\pgfsetroundjoin%
\definecolor{currentfill}{rgb}{0.000000,0.000000,0.000000}%
\pgfsetfillcolor{currentfill}%
\pgfsetlinewidth{0.501875pt}%
\definecolor{currentstroke}{rgb}{0.000000,0.000000,0.000000}%
\pgfsetstrokecolor{currentstroke}%
\pgfsetdash{}{0pt}%
\pgfsys@defobject{currentmarker}{\pgfqpoint{0.000000in}{0.000000in}}{\pgfqpoint{0.020833in}{0.000000in}}{%
\pgfpathmoveto{\pgfqpoint{0.000000in}{0.000000in}}%
\pgfpathlineto{\pgfqpoint{0.020833in}{0.000000in}}%
\pgfusepath{stroke,fill}%
}%
\begin{pgfscope}%
\pgfsys@transformshift{0.444748in}{0.704613in}%
\pgfsys@useobject{currentmarker}{}%
\end{pgfscope}%
\end{pgfscope}%
\begin{pgfscope}%
\pgfsetbuttcap%
\pgfsetroundjoin%
\definecolor{currentfill}{rgb}{0.000000,0.000000,0.000000}%
\pgfsetfillcolor{currentfill}%
\pgfsetlinewidth{0.501875pt}%
\definecolor{currentstroke}{rgb}{0.000000,0.000000,0.000000}%
\pgfsetstrokecolor{currentstroke}%
\pgfsetdash{}{0pt}%
\pgfsys@defobject{currentmarker}{\pgfqpoint{-0.020833in}{0.000000in}}{\pgfqpoint{-0.000000in}{0.000000in}}{%
\pgfpathmoveto{\pgfqpoint{-0.000000in}{0.000000in}}%
\pgfpathlineto{\pgfqpoint{-0.020833in}{0.000000in}}%
\pgfusepath{stroke,fill}%
}%
\begin{pgfscope}%
\pgfsys@transformshift{4.676167in}{0.704613in}%
\pgfsys@useobject{currentmarker}{}%
\end{pgfscope}%
\end{pgfscope}%
\begin{pgfscope}%
\pgfsetbuttcap%
\pgfsetroundjoin%
\definecolor{currentfill}{rgb}{0.000000,0.000000,0.000000}%
\pgfsetfillcolor{currentfill}%
\pgfsetlinewidth{0.501875pt}%
\definecolor{currentstroke}{rgb}{0.000000,0.000000,0.000000}%
\pgfsetstrokecolor{currentstroke}%
\pgfsetdash{}{0pt}%
\pgfsys@defobject{currentmarker}{\pgfqpoint{0.000000in}{0.000000in}}{\pgfqpoint{0.020833in}{0.000000in}}{%
\pgfpathmoveto{\pgfqpoint{0.000000in}{0.000000in}}%
\pgfpathlineto{\pgfqpoint{0.020833in}{0.000000in}}%
\pgfusepath{stroke,fill}%
}%
\begin{pgfscope}%
\pgfsys@transformshift{0.444748in}{0.733320in}%
\pgfsys@useobject{currentmarker}{}%
\end{pgfscope}%
\end{pgfscope}%
\begin{pgfscope}%
\pgfsetbuttcap%
\pgfsetroundjoin%
\definecolor{currentfill}{rgb}{0.000000,0.000000,0.000000}%
\pgfsetfillcolor{currentfill}%
\pgfsetlinewidth{0.501875pt}%
\definecolor{currentstroke}{rgb}{0.000000,0.000000,0.000000}%
\pgfsetstrokecolor{currentstroke}%
\pgfsetdash{}{0pt}%
\pgfsys@defobject{currentmarker}{\pgfqpoint{-0.020833in}{0.000000in}}{\pgfqpoint{-0.000000in}{0.000000in}}{%
\pgfpathmoveto{\pgfqpoint{-0.000000in}{0.000000in}}%
\pgfpathlineto{\pgfqpoint{-0.020833in}{0.000000in}}%
\pgfusepath{stroke,fill}%
}%
\begin{pgfscope}%
\pgfsys@transformshift{4.676167in}{0.733320in}%
\pgfsys@useobject{currentmarker}{}%
\end{pgfscope}%
\end{pgfscope}%
\begin{pgfscope}%
\pgfsetbuttcap%
\pgfsetroundjoin%
\definecolor{currentfill}{rgb}{0.000000,0.000000,0.000000}%
\pgfsetfillcolor{currentfill}%
\pgfsetlinewidth{0.501875pt}%
\definecolor{currentstroke}{rgb}{0.000000,0.000000,0.000000}%
\pgfsetstrokecolor{currentstroke}%
\pgfsetdash{}{0pt}%
\pgfsys@defobject{currentmarker}{\pgfqpoint{0.000000in}{0.000000in}}{\pgfqpoint{0.020833in}{0.000000in}}{%
\pgfpathmoveto{\pgfqpoint{0.000000in}{0.000000in}}%
\pgfpathlineto{\pgfqpoint{0.020833in}{0.000000in}}%
\pgfusepath{stroke,fill}%
}%
\begin{pgfscope}%
\pgfsys@transformshift{0.444748in}{0.790735in}%
\pgfsys@useobject{currentmarker}{}%
\end{pgfscope}%
\end{pgfscope}%
\begin{pgfscope}%
\pgfsetbuttcap%
\pgfsetroundjoin%
\definecolor{currentfill}{rgb}{0.000000,0.000000,0.000000}%
\pgfsetfillcolor{currentfill}%
\pgfsetlinewidth{0.501875pt}%
\definecolor{currentstroke}{rgb}{0.000000,0.000000,0.000000}%
\pgfsetstrokecolor{currentstroke}%
\pgfsetdash{}{0pt}%
\pgfsys@defobject{currentmarker}{\pgfqpoint{-0.020833in}{0.000000in}}{\pgfqpoint{-0.000000in}{0.000000in}}{%
\pgfpathmoveto{\pgfqpoint{-0.000000in}{0.000000in}}%
\pgfpathlineto{\pgfqpoint{-0.020833in}{0.000000in}}%
\pgfusepath{stroke,fill}%
}%
\begin{pgfscope}%
\pgfsys@transformshift{4.676167in}{0.790735in}%
\pgfsys@useobject{currentmarker}{}%
\end{pgfscope}%
\end{pgfscope}%
\begin{pgfscope}%
\pgfsetbuttcap%
\pgfsetroundjoin%
\definecolor{currentfill}{rgb}{0.000000,0.000000,0.000000}%
\pgfsetfillcolor{currentfill}%
\pgfsetlinewidth{0.501875pt}%
\definecolor{currentstroke}{rgb}{0.000000,0.000000,0.000000}%
\pgfsetstrokecolor{currentstroke}%
\pgfsetdash{}{0pt}%
\pgfsys@defobject{currentmarker}{\pgfqpoint{0.000000in}{0.000000in}}{\pgfqpoint{0.020833in}{0.000000in}}{%
\pgfpathmoveto{\pgfqpoint{0.000000in}{0.000000in}}%
\pgfpathlineto{\pgfqpoint{0.020833in}{0.000000in}}%
\pgfusepath{stroke,fill}%
}%
\begin{pgfscope}%
\pgfsys@transformshift{0.444748in}{0.819442in}%
\pgfsys@useobject{currentmarker}{}%
\end{pgfscope}%
\end{pgfscope}%
\begin{pgfscope}%
\pgfsetbuttcap%
\pgfsetroundjoin%
\definecolor{currentfill}{rgb}{0.000000,0.000000,0.000000}%
\pgfsetfillcolor{currentfill}%
\pgfsetlinewidth{0.501875pt}%
\definecolor{currentstroke}{rgb}{0.000000,0.000000,0.000000}%
\pgfsetstrokecolor{currentstroke}%
\pgfsetdash{}{0pt}%
\pgfsys@defobject{currentmarker}{\pgfqpoint{-0.020833in}{0.000000in}}{\pgfqpoint{-0.000000in}{0.000000in}}{%
\pgfpathmoveto{\pgfqpoint{-0.000000in}{0.000000in}}%
\pgfpathlineto{\pgfqpoint{-0.020833in}{0.000000in}}%
\pgfusepath{stroke,fill}%
}%
\begin{pgfscope}%
\pgfsys@transformshift{4.676167in}{0.819442in}%
\pgfsys@useobject{currentmarker}{}%
\end{pgfscope}%
\end{pgfscope}%
\begin{pgfscope}%
\pgfsetbuttcap%
\pgfsetroundjoin%
\definecolor{currentfill}{rgb}{0.000000,0.000000,0.000000}%
\pgfsetfillcolor{currentfill}%
\pgfsetlinewidth{0.501875pt}%
\definecolor{currentstroke}{rgb}{0.000000,0.000000,0.000000}%
\pgfsetstrokecolor{currentstroke}%
\pgfsetdash{}{0pt}%
\pgfsys@defobject{currentmarker}{\pgfqpoint{0.000000in}{0.000000in}}{\pgfqpoint{0.020833in}{0.000000in}}{%
\pgfpathmoveto{\pgfqpoint{0.000000in}{0.000000in}}%
\pgfpathlineto{\pgfqpoint{0.020833in}{0.000000in}}%
\pgfusepath{stroke,fill}%
}%
\begin{pgfscope}%
\pgfsys@transformshift{0.444748in}{0.848150in}%
\pgfsys@useobject{currentmarker}{}%
\end{pgfscope}%
\end{pgfscope}%
\begin{pgfscope}%
\pgfsetbuttcap%
\pgfsetroundjoin%
\definecolor{currentfill}{rgb}{0.000000,0.000000,0.000000}%
\pgfsetfillcolor{currentfill}%
\pgfsetlinewidth{0.501875pt}%
\definecolor{currentstroke}{rgb}{0.000000,0.000000,0.000000}%
\pgfsetstrokecolor{currentstroke}%
\pgfsetdash{}{0pt}%
\pgfsys@defobject{currentmarker}{\pgfqpoint{-0.020833in}{0.000000in}}{\pgfqpoint{-0.000000in}{0.000000in}}{%
\pgfpathmoveto{\pgfqpoint{-0.000000in}{0.000000in}}%
\pgfpathlineto{\pgfqpoint{-0.020833in}{0.000000in}}%
\pgfusepath{stroke,fill}%
}%
\begin{pgfscope}%
\pgfsys@transformshift{4.676167in}{0.848150in}%
\pgfsys@useobject{currentmarker}{}%
\end{pgfscope}%
\end{pgfscope}%
\begin{pgfscope}%
\pgfsetbuttcap%
\pgfsetroundjoin%
\definecolor{currentfill}{rgb}{0.000000,0.000000,0.000000}%
\pgfsetfillcolor{currentfill}%
\pgfsetlinewidth{0.501875pt}%
\definecolor{currentstroke}{rgb}{0.000000,0.000000,0.000000}%
\pgfsetstrokecolor{currentstroke}%
\pgfsetdash{}{0pt}%
\pgfsys@defobject{currentmarker}{\pgfqpoint{0.000000in}{0.000000in}}{\pgfqpoint{0.020833in}{0.000000in}}{%
\pgfpathmoveto{\pgfqpoint{0.000000in}{0.000000in}}%
\pgfpathlineto{\pgfqpoint{0.020833in}{0.000000in}}%
\pgfusepath{stroke,fill}%
}%
\begin{pgfscope}%
\pgfsys@transformshift{0.444748in}{0.876857in}%
\pgfsys@useobject{currentmarker}{}%
\end{pgfscope}%
\end{pgfscope}%
\begin{pgfscope}%
\pgfsetbuttcap%
\pgfsetroundjoin%
\definecolor{currentfill}{rgb}{0.000000,0.000000,0.000000}%
\pgfsetfillcolor{currentfill}%
\pgfsetlinewidth{0.501875pt}%
\definecolor{currentstroke}{rgb}{0.000000,0.000000,0.000000}%
\pgfsetstrokecolor{currentstroke}%
\pgfsetdash{}{0pt}%
\pgfsys@defobject{currentmarker}{\pgfqpoint{-0.020833in}{0.000000in}}{\pgfqpoint{-0.000000in}{0.000000in}}{%
\pgfpathmoveto{\pgfqpoint{-0.000000in}{0.000000in}}%
\pgfpathlineto{\pgfqpoint{-0.020833in}{0.000000in}}%
\pgfusepath{stroke,fill}%
}%
\begin{pgfscope}%
\pgfsys@transformshift{4.676167in}{0.876857in}%
\pgfsys@useobject{currentmarker}{}%
\end{pgfscope}%
\end{pgfscope}%
\begin{pgfscope}%
\definecolor{textcolor}{rgb}{0.000000,0.000000,0.000000}%
\pgfsetstrokecolor{textcolor}%
\pgfsetfillcolor{textcolor}%
\pgftext[x=0.201692in,y=0.665298in,,bottom,rotate=90.000000]{\color{textcolor}\rmfamily\fontsize{12.000000}{14.400000}\selectfont \(\displaystyle V_s\) (\unit{\micro\volt})}%
\end{pgfscope}%
\begin{pgfscope}%
\pgfpathrectangle{\pgfqpoint{0.444748in}{0.431673in}}{\pgfqpoint{4.231419in}{0.467251in}}%
\pgfusepath{clip}%
\pgfsetbuttcap%
\pgfsetroundjoin%
\pgfsetlinewidth{1.003750pt}%
\definecolor{currentstroke}{rgb}{0.047059,0.364706,0.647059}%
\pgfsetstrokecolor{currentstroke}%
\pgfsetdash{{3.700000pt}{1.600000pt}}{0.000000pt}%
\pgfpathmoveto{\pgfqpoint{0.642720in}{0.608542in}}%
\pgfpathlineto{\pgfqpoint{0.655865in}{0.690812in}}%
\pgfpathlineto{\pgfqpoint{0.677460in}{0.803479in}}%
\pgfpathlineto{\pgfqpoint{0.695534in}{0.857900in}}%
\pgfpathlineto{\pgfqpoint{0.713608in}{0.874267in}}%
\pgfpathlineto{\pgfqpoint{0.734733in}{0.844316in}}%
\pgfpathlineto{\pgfqpoint{0.753277in}{0.761166in}}%
\pgfpathlineto{\pgfqpoint{0.773699in}{0.624162in}}%
\pgfpathlineto{\pgfqpoint{0.791772in}{0.515613in}}%
\pgfpathlineto{\pgfqpoint{0.809846in}{0.503616in}}%
\pgfpathlineto{\pgfqpoint{0.830738in}{0.622273in}}%
\pgfpathlineto{\pgfqpoint{0.848107in}{0.749145in}}%
\pgfpathlineto{\pgfqpoint{0.866651in}{0.827577in}}%
\pgfpathlineto{\pgfqpoint{0.887073in}{0.867777in}}%
\pgfpathlineto{\pgfqpoint{0.906790in}{0.857801in}}%
\pgfpathlineto{\pgfqpoint{0.925334in}{0.795234in}}%
\pgfpathlineto{\pgfqpoint{0.944582in}{0.677988in}}%
\pgfpathlineto{\pgfqpoint{0.960073in}{0.558076in}}%
\pgfpathlineto{\pgfqpoint{0.982137in}{0.477381in}}%
\pgfpathlineto{\pgfqpoint{1.003968in}{0.555799in}}%
\pgfpathlineto{\pgfqpoint{1.039646in}{0.758239in}}%
\pgfpathlineto{\pgfqpoint{1.058894in}{0.830227in}}%
\pgfpathlineto{\pgfqpoint{1.081898in}{0.864481in}}%
\pgfpathlineto{\pgfqpoint{1.099737in}{0.849947in}}%
\pgfpathlineto{\pgfqpoint{1.116873in}{0.794242in}}%
\pgfpathlineto{\pgfqpoint{1.135652in}{0.676155in}}%
\pgfpathlineto{\pgfqpoint{1.157717in}{0.549179in}}%
\pgfpathlineto{\pgfqpoint{1.176964in}{0.469474in}}%
\pgfpathlineto{\pgfqpoint{1.195272in}{0.517186in}}%
\pgfpathlineto{\pgfqpoint{1.216163in}{0.634063in}}%
\pgfpathlineto{\pgfqpoint{1.230247in}{0.727777in}}%
\pgfpathlineto{\pgfqpoint{1.252312in}{0.826053in}}%
\pgfpathlineto{\pgfqpoint{1.272968in}{0.852693in}}%
\pgfpathlineto{\pgfqpoint{1.289868in}{0.857661in}}%
\pgfpathlineto{\pgfqpoint{1.310055in}{0.819156in}}%
\pgfpathlineto{\pgfqpoint{1.329303in}{0.735011in}}%
\pgfpathlineto{\pgfqpoint{1.347376in}{0.616282in}}%
\pgfpathlineto{\pgfqpoint{1.366859in}{0.515072in}}%
\pgfpathlineto{\pgfqpoint{1.388923in}{0.468935in}}%
\pgfpathlineto{\pgfqpoint{1.405120in}{0.529967in}}%
\pgfpathlineto{\pgfqpoint{1.424133in}{0.611327in}}%
\pgfpathlineto{\pgfqpoint{1.443380in}{0.727969in}}%
\pgfpathlineto{\pgfqpoint{1.465211in}{0.819494in}}%
\pgfpathlineto{\pgfqpoint{1.480938in}{0.849710in}}%
\pgfpathlineto{\pgfqpoint{1.503237in}{0.852103in}}%
\pgfpathlineto{\pgfqpoint{1.520608in}{0.826423in}}%
\pgfpathlineto{\pgfqpoint{1.539855in}{0.742554in}}%
\pgfpathlineto{\pgfqpoint{1.561685in}{0.615008in}}%
\pgfpathlineto{\pgfqpoint{1.576708in}{0.545907in}}%
\pgfpathlineto{\pgfqpoint{1.596895in}{0.461987in}}%
\pgfpathlineto{\pgfqpoint{1.617315in}{0.498862in}}%
\pgfpathlineto{\pgfqpoint{1.636094in}{0.538375in}}%
\pgfpathlineto{\pgfqpoint{1.660741in}{0.640570in}}%
\pgfpathlineto{\pgfqpoint{1.676232in}{0.730583in}}%
\pgfpathlineto{\pgfqpoint{1.695011in}{0.807562in}}%
\pgfpathlineto{\pgfqpoint{1.713790in}{0.842027in}}%
\pgfpathlineto{\pgfqpoint{1.731863in}{0.482608in}}%
\pgfpathlineto{\pgfqpoint{1.751111in}{0.578138in}}%
\pgfpathlineto{\pgfqpoint{1.773410in}{0.710339in}}%
\pgfpathlineto{\pgfqpoint{1.794537in}{0.812633in}}%
\pgfpathlineto{\pgfqpoint{1.808854in}{0.833494in}}%
\pgfpathlineto{\pgfqpoint{1.825755in}{0.854256in}}%
\pgfpathlineto{\pgfqpoint{1.848994in}{0.828715in}}%
\pgfpathlineto{\pgfqpoint{1.866364in}{0.761561in}}%
\pgfpathlineto{\pgfqpoint{1.888663in}{0.620490in}}%
\pgfpathlineto{\pgfqpoint{1.904155in}{0.539251in}}%
\pgfpathlineto{\pgfqpoint{1.926219in}{0.457446in}}%
\pgfpathlineto{\pgfqpoint{1.944293in}{0.493475in}}%
\pgfpathlineto{\pgfqpoint{1.964011in}{0.581817in}}%
\pgfpathlineto{\pgfqpoint{1.981382in}{0.677458in}}%
\pgfpathlineto{\pgfqpoint{2.001098in}{0.766642in}}%
\pgfpathlineto{\pgfqpoint{2.022223in}{0.834290in}}%
\pgfpathlineto{\pgfqpoint{2.041236in}{0.853431in}}%
\pgfpathlineto{\pgfqpoint{2.059781in}{0.838260in}}%
\pgfpathlineto{\pgfqpoint{2.078089in}{0.777105in}}%
\pgfpathlineto{\pgfqpoint{2.096868in}{0.665077in}}%
\pgfpathlineto{\pgfqpoint{2.116115in}{0.564326in}}%
\pgfpathlineto{\pgfqpoint{2.136303in}{0.483846in}}%
\pgfpathlineto{\pgfqpoint{2.155785in}{0.460101in}}%
\pgfpathlineto{\pgfqpoint{2.175972in}{0.530363in}}%
\pgfpathlineto{\pgfqpoint{2.191932in}{0.611963in}}%
\pgfpathlineto{\pgfqpoint{2.213293in}{0.727811in}}%
\pgfpathlineto{\pgfqpoint{2.230898in}{0.804131in}}%
\pgfpathlineto{\pgfqpoint{2.252728in}{0.845227in}}%
\pgfpathlineto{\pgfqpoint{2.275967in}{0.850929in}}%
\pgfpathlineto{\pgfqpoint{2.289581in}{0.852806in}}%
\pgfpathlineto{\pgfqpoint{2.310940in}{0.834549in}}%
\pgfpathlineto{\pgfqpoint{2.329250in}{0.768237in}}%
\pgfpathlineto{\pgfqpoint{2.347090in}{0.663066in}}%
\pgfpathlineto{\pgfqpoint{2.368449in}{0.544253in}}%
\pgfpathlineto{\pgfqpoint{2.387697in}{0.466896in}}%
\pgfpathlineto{\pgfqpoint{2.404364in}{0.462256in}}%
\pgfpathlineto{\pgfqpoint{2.423846in}{0.527879in}}%
\pgfpathlineto{\pgfqpoint{2.442154in}{0.638704in}}%
\pgfpathlineto{\pgfqpoint{2.463750in}{0.746382in}}%
\pgfpathlineto{\pgfqpoint{2.481355in}{0.814911in}}%
\pgfpathlineto{\pgfqpoint{2.500133in}{0.846687in}}%
\pgfpathlineto{\pgfqpoint{2.520319in}{0.849788in}}%
\pgfpathlineto{\pgfqpoint{2.538863in}{0.830246in}}%
\pgfpathlineto{\pgfqpoint{2.559285in}{0.748141in}}%
\pgfpathlineto{\pgfqpoint{2.594729in}{0.548213in}}%
\pgfpathlineto{\pgfqpoint{2.614916in}{0.467644in}}%
\pgfpathlineto{\pgfqpoint{2.637450in}{0.476028in}}%
\pgfpathlineto{\pgfqpoint{2.654349in}{0.551356in}}%
\pgfpathlineto{\pgfqpoint{2.673128in}{0.601190in}}%
\pgfpathlineto{\pgfqpoint{2.694019in}{0.717255in}}%
\pgfpathlineto{\pgfqpoint{2.712329in}{0.762705in}}%
\pgfpathlineto{\pgfqpoint{2.732750in}{0.829297in}}%
\pgfpathlineto{\pgfqpoint{2.750590in}{0.852438in}}%
\pgfpathlineto{\pgfqpoint{2.767960in}{0.849981in}}%
\pgfpathlineto{\pgfqpoint{2.789319in}{0.810694in}}%
\pgfpathlineto{\pgfqpoint{2.810210in}{0.714648in}}%
\pgfpathlineto{\pgfqpoint{2.828049in}{0.611271in}}%
\pgfpathlineto{\pgfqpoint{2.846594in}{0.518971in}}%
\pgfpathlineto{\pgfqpoint{2.867484in}{0.463681in}}%
\pgfpathlineto{\pgfqpoint{2.884149in}{0.460146in}}%
\pgfpathlineto{\pgfqpoint{2.906919in}{0.545945in}}%
\pgfpathlineto{\pgfqpoint{2.924758in}{0.629440in}}%
\pgfpathlineto{\pgfqpoint{2.942363in}{0.482651in}}%
\pgfpathlineto{\pgfqpoint{2.963254in}{0.473268in}}%
\pgfpathlineto{\pgfqpoint{2.981798in}{0.551102in}}%
\pgfpathlineto{\pgfqpoint{3.020762in}{0.765406in}}%
\pgfpathlineto{\pgfqpoint{3.038602in}{0.828147in}}%
\pgfpathlineto{\pgfqpoint{3.058554in}{0.854664in}}%
\pgfpathlineto{\pgfqpoint{3.077097in}{0.846574in}}%
\pgfpathlineto{\pgfqpoint{3.094467in}{0.805108in}}%
\pgfpathlineto{\pgfqpoint{3.117940in}{0.692308in}}%
\pgfpathlineto{\pgfqpoint{3.136719in}{0.613345in}}%
\pgfpathlineto{\pgfqpoint{3.153853in}{0.527609in}}%
\pgfpathlineto{\pgfqpoint{3.172398in}{0.480625in}}%
\pgfpathlineto{\pgfqpoint{3.193523in}{0.469096in}}%
\pgfpathlineto{\pgfqpoint{3.212067in}{0.535101in}}%
\pgfpathlineto{\pgfqpoint{3.231315in}{0.537979in}}%
\pgfpathlineto{\pgfqpoint{3.249623in}{0.634342in}}%
\pgfpathlineto{\pgfqpoint{3.272158in}{0.738338in}}%
\pgfpathlineto{\pgfqpoint{3.289763in}{0.801914in}}%
\pgfpathlineto{\pgfqpoint{3.307837in}{0.762391in}}%
\pgfpathlineto{\pgfqpoint{3.328024in}{0.837232in}}%
\pgfpathlineto{\pgfqpoint{3.346566in}{0.857524in}}%
\pgfpathlineto{\pgfqpoint{3.364171in}{0.849217in}}%
\pgfpathlineto{\pgfqpoint{3.386470in}{0.791669in}}%
\pgfpathlineto{\pgfqpoint{3.407126in}{0.691584in}}%
\pgfpathlineto{\pgfqpoint{3.424731in}{0.594635in}}%
\pgfpathlineto{\pgfqpoint{3.442101in}{0.509062in}}%
\pgfpathlineto{\pgfqpoint{3.463697in}{0.467878in}}%
\pgfpathlineto{\pgfqpoint{3.482240in}{0.527022in}}%
\pgfpathlineto{\pgfqpoint{3.501253in}{0.618113in}}%
\pgfpathlineto{\pgfqpoint{3.521440in}{0.720542in}}%
\pgfpathlineto{\pgfqpoint{3.538340in}{0.784327in}}%
\pgfpathlineto{\pgfqpoint{3.559232in}{0.780593in}}%
\pgfpathlineto{\pgfqpoint{3.577072in}{0.834547in}}%
\pgfpathlineto{\pgfqpoint{3.598197in}{0.861311in}}%
\pgfpathlineto{\pgfqpoint{3.616270in}{0.854967in}}%
\pgfpathlineto{\pgfqpoint{3.633172in}{0.824548in}}%
\pgfpathlineto{\pgfqpoint{3.652185in}{0.757343in}}%
\pgfpathlineto{\pgfqpoint{3.673779in}{0.676636in}}%
\pgfpathlineto{\pgfqpoint{3.691149in}{0.605058in}}%
\pgfpathlineto{\pgfqpoint{3.712511in}{0.506963in}}%
\pgfpathlineto{\pgfqpoint{3.730115in}{0.474652in}}%
\pgfpathlineto{\pgfqpoint{3.749832in}{0.536603in}}%
\pgfpathlineto{\pgfqpoint{3.769080in}{0.627057in}}%
\pgfpathlineto{\pgfqpoint{3.807341in}{0.781940in}}%
\pgfpathlineto{\pgfqpoint{3.825180in}{0.837140in}}%
\pgfpathlineto{\pgfqpoint{3.846305in}{0.863822in}}%
\pgfpathlineto{\pgfqpoint{3.867432in}{0.862647in}}%
\pgfpathlineto{\pgfqpoint{3.885271in}{0.843188in}}%
\pgfpathlineto{\pgfqpoint{3.902876in}{0.793388in}}%
\pgfpathlineto{\pgfqpoint{3.924235in}{0.817913in}}%
\pgfpathlineto{\pgfqpoint{3.942074in}{0.857386in}}%
\pgfpathlineto{\pgfqpoint{3.960619in}{0.866326in}}%
\pgfpathlineto{\pgfqpoint{3.981744in}{0.841332in}}%
\pgfpathlineto{\pgfqpoint{3.999348in}{0.778424in}}%
\pgfpathlineto{\pgfqpoint{4.039252in}{0.578580in}}%
\pgfpathlineto{\pgfqpoint{4.056623in}{0.508630in}}%
\pgfpathlineto{\pgfqpoint{4.077279in}{0.498399in}}%
\pgfpathlineto{\pgfqpoint{4.099109in}{0.594779in}}%
\pgfpathlineto{\pgfqpoint{4.114131in}{0.675319in}}%
\pgfpathlineto{\pgfqpoint{4.138310in}{0.777645in}}%
\pgfpathlineto{\pgfqpoint{4.153332in}{0.831033in}}%
\pgfpathlineto{\pgfqpoint{4.173988in}{0.866287in}}%
\pgfpathlineto{\pgfqpoint{4.191827in}{0.870722in}}%
\pgfpathlineto{\pgfqpoint{4.210135in}{0.855854in}}%
\pgfpathlineto{\pgfqpoint{4.231965in}{0.807519in}}%
\pgfpathlineto{\pgfqpoint{4.250041in}{0.736468in}}%
\pgfpathlineto{\pgfqpoint{4.270697in}{0.624856in}}%
\pgfpathlineto{\pgfqpoint{4.287597in}{0.546416in}}%
\pgfpathlineto{\pgfqpoint{4.307079in}{0.494339in}}%
\pgfpathlineto{\pgfqpoint{4.327501in}{0.541837in}}%
\pgfpathlineto{\pgfqpoint{4.345574in}{0.634968in}}%
\pgfpathlineto{\pgfqpoint{4.364353in}{0.720277in}}%
\pgfpathlineto{\pgfqpoint{4.384540in}{0.790684in}}%
\pgfpathlineto{\pgfqpoint{4.401911in}{0.845713in}}%
\pgfpathlineto{\pgfqpoint{4.421158in}{0.872149in}}%
\pgfpathlineto{\pgfqpoint{4.438763in}{0.877643in}}%
\pgfpathlineto{\pgfqpoint{4.462236in}{0.862080in}}%
\pgfpathlineto{\pgfqpoint{4.479136in}{0.829599in}}%
\pgfpathlineto{\pgfqpoint{4.477962in}{0.828004in}}%
\pgfpathlineto{\pgfqpoint{4.475850in}{0.839171in}}%
\pgfpathlineto{\pgfqpoint{4.453786in}{0.876889in}}%
\pgfpathlineto{\pgfqpoint{4.433833in}{0.864119in}}%
\pgfpathlineto{\pgfqpoint{4.415290in}{0.810821in}}%
\pgfpathlineto{\pgfqpoint{4.395337in}{0.705533in}}%
\pgfpathlineto{\pgfqpoint{4.379141in}{0.597410in}}%
\pgfpathlineto{\pgfqpoint{4.359190in}{0.498584in}}%
\pgfpathlineto{\pgfqpoint{4.341114in}{0.548947in}}%
\pgfpathlineto{\pgfqpoint{4.298864in}{0.789158in}}%
\pgfpathlineto{\pgfqpoint{4.281025in}{0.853092in}}%
\pgfpathlineto{\pgfqpoint{4.263889in}{0.871920in}}%
\pgfpathlineto{\pgfqpoint{4.242294in}{0.850254in}}%
\pgfpathlineto{\pgfqpoint{4.225394in}{0.785317in}}%
\pgfpathlineto{\pgfqpoint{4.204738in}{0.666979in}}%
\pgfpathlineto{\pgfqpoint{4.188071in}{0.558398in}}%
\pgfpathlineto{\pgfqpoint{4.166946in}{0.484555in}}%
\pgfpathlineto{\pgfqpoint{4.146995in}{0.573633in}}%
\pgfpathlineto{\pgfqpoint{4.129154in}{0.687715in}}%
\pgfpathlineto{\pgfqpoint{4.107794in}{0.807526in}}%
\pgfpathlineto{\pgfqpoint{4.090189in}{0.857707in}}%
\pgfpathlineto{\pgfqpoint{4.070707in}{0.864576in}}%
\pgfpathlineto{\pgfqpoint{4.052634in}{0.830583in}}%
\pgfpathlineto{\pgfqpoint{4.031507in}{0.733449in}}%
\pgfpathlineto{\pgfqpoint{3.994889in}{0.511521in}}%
\pgfpathlineto{\pgfqpoint{3.976581in}{0.478823in}}%
\pgfpathlineto{\pgfqpoint{3.953342in}{0.587364in}}%
\pgfpathlineto{\pgfqpoint{3.935972in}{0.704144in}}%
\pgfpathlineto{\pgfqpoint{3.917898in}{0.799627in}}%
\pgfpathlineto{\pgfqpoint{3.897476in}{0.856326in}}%
\pgfpathlineto{\pgfqpoint{3.877760in}{0.857229in}}%
\pgfpathlineto{\pgfqpoint{3.859920in}{0.815159in}}%
\pgfpathlineto{\pgfqpoint{3.821425in}{0.625810in}}%
\pgfpathlineto{\pgfqpoint{3.802412in}{0.519865in}}%
\pgfpathlineto{\pgfqpoint{3.783867in}{0.468540in}}%
\pgfpathlineto{\pgfqpoint{3.762742in}{0.552944in}}%
\pgfpathlineto{\pgfqpoint{3.725185in}{0.778492in}}%
\pgfpathlineto{\pgfqpoint{3.707111in}{0.841945in}}%
\pgfpathlineto{\pgfqpoint{3.685986in}{0.858343in}}%
\pgfpathlineto{\pgfqpoint{3.665799in}{0.843387in}}%
\pgfpathlineto{\pgfqpoint{3.648897in}{0.788587in}}%
\pgfpathlineto{\pgfqpoint{3.628478in}{0.812497in}}%
\pgfpathlineto{\pgfqpoint{3.609933in}{0.731802in}}%
\pgfpathlineto{\pgfqpoint{3.589043in}{0.607193in}}%
\pgfpathlineto{\pgfqpoint{3.571203in}{0.511847in}}%
\pgfpathlineto{\pgfqpoint{3.553128in}{0.460752in}}%
\pgfpathlineto{\pgfqpoint{3.531534in}{0.529299in}}%
\pgfpathlineto{\pgfqpoint{3.494211in}{0.742789in}}%
\pgfpathlineto{\pgfqpoint{3.476137in}{0.822223in}}%
\pgfpathlineto{\pgfqpoint{3.455481in}{0.854377in}}%
\pgfpathlineto{\pgfqpoint{3.435999in}{0.849889in}}%
\pgfpathlineto{\pgfqpoint{3.418394in}{0.817504in}}%
\pgfpathlineto{\pgfqpoint{3.397738in}{0.745585in}}%
\pgfpathlineto{\pgfqpoint{3.358068in}{0.528298in}}%
\pgfpathlineto{\pgfqpoint{3.339526in}{0.470703in}}%
\pgfpathlineto{\pgfqpoint{3.321450in}{0.471890in}}%
\pgfpathlineto{\pgfqpoint{3.300560in}{0.511348in}}%
\pgfpathlineto{\pgfqpoint{3.282252in}{0.611700in}}%
\pgfpathlineto{\pgfqpoint{3.264881in}{0.723462in}}%
\pgfpathlineto{\pgfqpoint{3.243520in}{0.818741in}}%
\pgfpathlineto{\pgfqpoint{3.225915in}{0.845946in}}%
\pgfpathlineto{\pgfqpoint{3.205730in}{0.851732in}}%
\pgfpathlineto{\pgfqpoint{3.186951in}{0.822169in}}%
\pgfpathlineto{\pgfqpoint{3.170286in}{0.763772in}}%
\pgfpathlineto{\pgfqpoint{3.148456in}{0.653424in}}%
\pgfpathlineto{\pgfqpoint{3.130851in}{0.558558in}}%
\pgfpathlineto{\pgfqpoint{3.110664in}{0.468038in}}%
\pgfpathlineto{\pgfqpoint{3.091885in}{0.488011in}}%
\pgfpathlineto{\pgfqpoint{3.071699in}{0.457509in}}%
\pgfpathlineto{\pgfqpoint{3.049400in}{0.527885in}}%
\pgfpathlineto{\pgfqpoint{3.014425in}{0.737056in}}%
\pgfpathlineto{\pgfqpoint{2.993064in}{0.820847in}}%
\pgfpathlineto{\pgfqpoint{2.971939in}{0.852414in}}%
\pgfpathlineto{\pgfqpoint{2.957151in}{0.846747in}}%
\pgfpathlineto{\pgfqpoint{2.937198in}{0.805206in}}%
\pgfpathlineto{\pgfqpoint{2.917950in}{0.736600in}}%
\pgfpathlineto{\pgfqpoint{2.897060in}{0.618924in}}%
\pgfpathlineto{\pgfqpoint{2.878986in}{0.532073in}}%
\pgfpathlineto{\pgfqpoint{2.862321in}{0.467326in}}%
\pgfpathlineto{\pgfqpoint{2.839317in}{0.477604in}}%
\pgfpathlineto{\pgfqpoint{2.823120in}{0.456332in}}%
\pgfpathlineto{\pgfqpoint{2.800587in}{0.540800in}}%
\pgfpathlineto{\pgfqpoint{2.763735in}{0.729769in}}%
\pgfpathlineto{\pgfqpoint{2.746833in}{0.812494in}}%
\pgfpathlineto{\pgfqpoint{2.724300in}{0.850690in}}%
\pgfpathlineto{\pgfqpoint{2.705521in}{0.844418in}}%
\pgfpathlineto{\pgfqpoint{2.687213in}{0.804380in}}%
\pgfpathlineto{\pgfqpoint{2.668434in}{0.720749in}}%
\pgfpathlineto{\pgfqpoint{2.645900in}{0.599331in}}%
\pgfpathlineto{\pgfqpoint{2.630876in}{0.527457in}}%
\pgfpathlineto{\pgfqpoint{2.613273in}{0.459767in}}%
\pgfpathlineto{\pgfqpoint{2.592615in}{0.470910in}}%
\pgfpathlineto{\pgfqpoint{2.572195in}{0.563762in}}%
\pgfpathlineto{\pgfqpoint{2.552948in}{0.659592in}}%
\pgfpathlineto{\pgfqpoint{2.533698in}{0.768143in}}%
\pgfpathlineto{\pgfqpoint{2.515624in}{0.831692in}}%
\pgfpathlineto{\pgfqpoint{2.493325in}{0.851533in}}%
\pgfpathlineto{\pgfqpoint{2.477834in}{0.848076in}}%
\pgfpathlineto{\pgfqpoint{2.459524in}{0.821338in}}%
\pgfpathlineto{\pgfqpoint{2.437225in}{0.744062in}}%
\pgfpathlineto{\pgfqpoint{2.418446in}{0.695982in}}%
\pgfpathlineto{\pgfqpoint{2.399904in}{0.585482in}}%
\pgfpathlineto{\pgfqpoint{2.378777in}{0.501396in}}%
\pgfpathlineto{\pgfqpoint{2.360000in}{0.452964in}}%
\pgfpathlineto{\pgfqpoint{2.340282in}{0.498357in}}%
\pgfpathlineto{\pgfqpoint{2.318452in}{0.592492in}}%
\pgfpathlineto{\pgfqpoint{2.303429in}{0.678257in}}%
\pgfpathlineto{\pgfqpoint{2.283713in}{0.776934in}}%
\pgfpathlineto{\pgfqpoint{2.265873in}{0.830664in}}%
\pgfpathlineto{\pgfqpoint{2.244043in}{0.852024in}}%
\pgfpathlineto{\pgfqpoint{2.228081in}{0.851691in}}%
\pgfpathlineto{\pgfqpoint{2.203905in}{0.818115in}}%
\pgfpathlineto{\pgfqpoint{2.167053in}{0.702400in}}%
\pgfpathlineto{\pgfqpoint{2.148508in}{0.607984in}}%
\pgfpathlineto{\pgfqpoint{2.132312in}{0.512771in}}%
\pgfpathlineto{\pgfqpoint{2.111421in}{0.457214in}}%
\pgfpathlineto{\pgfqpoint{2.092408in}{0.488227in}}%
\pgfpathlineto{\pgfqpoint{2.072221in}{0.573968in}}%
\pgfpathlineto{\pgfqpoint{2.051565in}{0.682071in}}%
\pgfpathlineto{\pgfqpoint{2.032551in}{0.762576in}}%
\pgfpathlineto{\pgfqpoint{2.014243in}{0.828366in}}%
\pgfpathlineto{\pgfqpoint{1.995230in}{0.852204in}}%
\pgfpathlineto{\pgfqpoint{1.976922in}{0.854209in}}%
\pgfpathlineto{\pgfqpoint{1.937016in}{0.809576in}}%
\pgfpathlineto{\pgfqpoint{1.916831in}{0.729056in}}%
\pgfpathlineto{\pgfqpoint{1.898757in}{0.652149in}}%
\pgfpathlineto{\pgfqpoint{1.881152in}{0.563404in}}%
\pgfpathlineto{\pgfqpoint{1.858148in}{0.480596in}}%
\pgfpathlineto{\pgfqpoint{1.843595in}{0.463210in}}%
\pgfpathlineto{\pgfqpoint{1.824581in}{0.513694in}}%
\pgfpathlineto{\pgfqpoint{1.804160in}{0.596117in}}%
\pgfpathlineto{\pgfqpoint{1.785852in}{0.676691in}}%
\pgfpathlineto{\pgfqpoint{1.766133in}{0.781212in}}%
\pgfpathlineto{\pgfqpoint{1.747825in}{0.824018in}}%
\pgfpathlineto{\pgfqpoint{1.725526in}{0.855764in}}%
\pgfpathlineto{\pgfqpoint{1.710504in}{0.857577in}}%
\pgfpathlineto{\pgfqpoint{1.688205in}{0.850582in}}%
\pgfpathlineto{\pgfqpoint{1.648770in}{0.768338in}}%
\pgfpathlineto{\pgfqpoint{1.625766in}{0.686555in}}%
\pgfpathlineto{\pgfqpoint{1.610978in}{0.854412in}}%
\pgfpathlineto{\pgfqpoint{1.590790in}{0.816683in}}%
\pgfpathlineto{\pgfqpoint{1.572717in}{0.831666in}}%
\pgfpathlineto{\pgfqpoint{1.551592in}{0.758152in}}%
\pgfpathlineto{\pgfqpoint{1.537273in}{0.660484in}}%
\pgfpathlineto{\pgfqpoint{1.493144in}{0.482240in}}%
\pgfpathlineto{\pgfqpoint{1.479061in}{0.475926in}}%
\pgfpathlineto{\pgfqpoint{1.459577in}{0.546906in}}%
\pgfpathlineto{\pgfqpoint{1.419439in}{0.753137in}}%
\pgfpathlineto{\pgfqpoint{1.401599in}{0.814718in}}%
\pgfpathlineto{\pgfqpoint{1.381648in}{0.854268in}}%
\pgfpathlineto{\pgfqpoint{1.359582in}{0.862213in}}%
\pgfpathlineto{\pgfqpoint{1.341508in}{0.850211in}}%
\pgfpathlineto{\pgfqpoint{1.320149in}{0.803967in}}%
\pgfpathlineto{\pgfqpoint{1.306299in}{0.751096in}}%
\pgfpathlineto{\pgfqpoint{1.264049in}{0.567510in}}%
\pgfpathlineto{\pgfqpoint{1.245504in}{0.505638in}}%
\pgfpathlineto{\pgfqpoint{1.227196in}{0.477557in}}%
\pgfpathlineto{\pgfqpoint{1.208652in}{0.528991in}}%
\pgfpathlineto{\pgfqpoint{1.171330in}{0.717348in}}%
\pgfpathlineto{\pgfqpoint{1.150203in}{0.807951in}}%
\pgfpathlineto{\pgfqpoint{1.130721in}{0.850452in}}%
\pgfpathlineto{\pgfqpoint{1.112413in}{0.866157in}}%
\pgfpathlineto{\pgfqpoint{1.087766in}{0.860171in}}%
\pgfpathlineto{\pgfqpoint{1.072039in}{0.838800in}}%
\pgfpathlineto{\pgfqpoint{1.053965in}{0.788501in}}%
\pgfpathlineto{\pgfqpoint{0.975566in}{0.516661in}}%
\pgfpathlineto{\pgfqpoint{0.957492in}{0.488679in}}%
\pgfpathlineto{\pgfqpoint{0.939182in}{0.549476in}}%
\pgfpathlineto{\pgfqpoint{0.920169in}{0.643111in}}%
\pgfpathlineto{\pgfqpoint{0.899513in}{0.728222in}}%
\pgfpathlineto{\pgfqpoint{0.883787in}{0.777818in}}%
\pgfpathlineto{\pgfqpoint{0.862192in}{0.785867in}}%
\pgfpathlineto{\pgfqpoint{0.843647in}{0.844117in}}%
\pgfpathlineto{\pgfqpoint{0.825105in}{0.868750in}}%
\pgfpathlineto{\pgfqpoint{0.804212in}{0.871468in}}%
\pgfpathlineto{\pgfqpoint{0.785201in}{0.851034in}}%
\pgfpathlineto{\pgfqpoint{0.764308in}{0.800049in}}%
\pgfpathlineto{\pgfqpoint{0.745295in}{0.819772in}}%
\pgfpathlineto{\pgfqpoint{0.726284in}{0.750577in}}%
\pgfpathlineto{\pgfqpoint{0.688960in}{0.584224in}}%
\pgfpathlineto{\pgfqpoint{0.669010in}{0.529981in}}%
\pgfpathlineto{\pgfqpoint{0.650934in}{0.502949in}}%
\pgfpathlineto{\pgfqpoint{0.647883in}{0.501889in}}%
\pgfpathlineto{\pgfqpoint{0.661264in}{0.519266in}}%
\pgfpathlineto{\pgfqpoint{0.696003in}{0.709022in}}%
\pgfpathlineto{\pgfqpoint{0.714782in}{0.810390in}}%
\pgfpathlineto{\pgfqpoint{0.732855in}{0.861022in}}%
\pgfpathlineto{\pgfqpoint{0.754451in}{0.871244in}}%
\pgfpathlineto{\pgfqpoint{0.772056in}{0.836837in}}%
\pgfpathlineto{\pgfqpoint{0.790364in}{0.753414in}}%
\pgfpathlineto{\pgfqpoint{0.811960in}{0.609668in}}%
\pgfpathlineto{\pgfqpoint{0.831207in}{0.500079in}}%
\pgfpathlineto{\pgfqpoint{0.849281in}{0.514569in}}%
\pgfpathlineto{\pgfqpoint{0.887542in}{0.737758in}}%
\pgfpathlineto{\pgfqpoint{0.904678in}{0.822314in}}%
\pgfpathlineto{\pgfqpoint{0.926037in}{0.865900in}}%
\pgfpathlineto{\pgfqpoint{0.944111in}{0.858611in}}%
\pgfpathlineto{\pgfqpoint{0.963595in}{0.808082in}}%
\pgfpathlineto{\pgfqpoint{0.982843in}{0.698375in}}%
\pgfpathlineto{\pgfqpoint{1.001621in}{0.574652in}}%
\pgfpathlineto{\pgfqpoint{1.022981in}{0.478916in}}%
\pgfpathlineto{\pgfqpoint{1.041525in}{0.535114in}}%
\pgfpathlineto{\pgfqpoint{1.059599in}{0.648623in}}%
\pgfpathlineto{\pgfqpoint{1.079315in}{0.755924in}}%
\pgfpathlineto{\pgfqpoint{1.098329in}{0.834777in}}%
\pgfpathlineto{\pgfqpoint{1.119924in}{0.863833in}}%
\pgfpathlineto{\pgfqpoint{1.139407in}{0.842877in}}%
\pgfpathlineto{\pgfqpoint{1.156777in}{0.774757in}}%
\pgfpathlineto{\pgfqpoint{1.196212in}{0.549870in}}%
\pgfpathlineto{\pgfqpoint{1.212643in}{0.469936in}}%
\pgfpathlineto{\pgfqpoint{1.232594in}{0.535343in}}%
\pgfpathlineto{\pgfqpoint{1.271325in}{0.761242in}}%
\pgfpathlineto{\pgfqpoint{1.309586in}{0.846509in}}%
\pgfpathlineto{\pgfqpoint{1.328598in}{0.859315in}}%
\pgfpathlineto{\pgfqpoint{1.348082in}{0.828798in}}%
\pgfpathlineto{\pgfqpoint{1.367798in}{0.739661in}}%
\pgfpathlineto{\pgfqpoint{1.404651in}{0.518508in}}%
\pgfpathlineto{\pgfqpoint{1.427420in}{0.464155in}}%
\pgfpathlineto{\pgfqpoint{1.446197in}{0.536422in}}%
\pgfpathlineto{\pgfqpoint{1.484224in}{0.737753in}}%
\pgfpathlineto{\pgfqpoint{1.503472in}{0.818186in}}%
\pgfpathlineto{\pgfqpoint{1.522719in}{0.852763in}}%
\pgfpathlineto{\pgfqpoint{1.541498in}{0.851332in}}%
\pgfpathlineto{\pgfqpoint{1.558163in}{0.816069in}}%
\pgfpathlineto{\pgfqpoint{1.579759in}{0.747520in}}%
\pgfpathlineto{\pgfqpoint{1.598772in}{0.646493in}}%
\pgfpathlineto{\pgfqpoint{1.622245in}{0.571498in}}%
\pgfpathlineto{\pgfqpoint{1.638442in}{0.487776in}}%
\pgfpathlineto{\pgfqpoint{1.654638in}{0.459776in}}%
\pgfpathlineto{\pgfqpoint{1.677171in}{0.542877in}}%
\pgfpathlineto{\pgfqpoint{1.699471in}{0.651456in}}%
\pgfpathlineto{\pgfqpoint{1.713555in}{0.744119in}}%
\pgfpathlineto{\pgfqpoint{1.731863in}{0.819405in}}%
\pgfpathlineto{\pgfqpoint{1.751345in}{0.845797in}}%
\pgfpathlineto{\pgfqpoint{1.774818in}{0.850038in}}%
\pgfpathlineto{\pgfqpoint{1.787963in}{0.827266in}}%
\pgfpathlineto{\pgfqpoint{1.809325in}{0.737978in}}%
\pgfpathlineto{\pgfqpoint{1.828104in}{0.623221in}}%
\pgfpathlineto{\pgfqpoint{1.847586in}{0.531059in}}%
\pgfpathlineto{\pgfqpoint{1.867302in}{0.455068in}}%
\pgfpathlineto{\pgfqpoint{1.886315in}{0.490693in}}%
\pgfpathlineto{\pgfqpoint{1.905329in}{0.547059in}}%
\pgfpathlineto{\pgfqpoint{1.927393in}{0.513604in}}%
\pgfpathlineto{\pgfqpoint{1.944998in}{0.606519in}}%
\pgfpathlineto{\pgfqpoint{1.963777in}{0.720086in}}%
\pgfpathlineto{\pgfqpoint{1.984197in}{0.805843in}}%
\pgfpathlineto{\pgfqpoint{2.001567in}{0.845404in}}%
\pgfpathlineto{\pgfqpoint{2.020580in}{0.851407in}}%
\pgfpathlineto{\pgfqpoint{2.042645in}{0.809259in}}%
\pgfpathlineto{\pgfqpoint{2.057667in}{0.741742in}}%
\pgfpathlineto{\pgfqpoint{2.079263in}{0.608537in}}%
\pgfpathlineto{\pgfqpoint{2.097337in}{0.511355in}}%
\pgfpathlineto{\pgfqpoint{2.116584in}{0.508338in}}%
\pgfpathlineto{\pgfqpoint{2.137006in}{0.453541in}}%
\pgfpathlineto{\pgfqpoint{2.155550in}{0.498104in}}%
\pgfpathlineto{\pgfqpoint{2.175972in}{0.596510in}}%
\pgfpathlineto{\pgfqpoint{2.212119in}{0.787930in}}%
\pgfpathlineto{\pgfqpoint{2.230898in}{0.838134in}}%
\pgfpathlineto{\pgfqpoint{2.249677in}{0.852251in}}%
\pgfpathlineto{\pgfqpoint{2.271976in}{0.832869in}}%
\pgfpathlineto{\pgfqpoint{2.290519in}{0.778874in}}%
\pgfpathlineto{\pgfqpoint{2.327607in}{0.565048in}}%
\pgfpathlineto{\pgfqpoint{2.348967in}{0.471012in}}%
\pgfpathlineto{\pgfqpoint{2.369154in}{0.459052in}}%
\pgfpathlineto{\pgfqpoint{2.386759in}{0.530233in}}%
\pgfpathlineto{\pgfqpoint{2.402955in}{0.605125in}}%
\pgfpathlineto{\pgfqpoint{2.424315in}{0.723057in}}%
\pgfpathlineto{\pgfqpoint{2.442859in}{0.785704in}}%
\pgfpathlineto{\pgfqpoint{2.464689in}{0.835989in}}%
\pgfpathlineto{\pgfqpoint{2.482529in}{0.851683in}}%
\pgfpathlineto{\pgfqpoint{2.500133in}{0.838817in}}%
\pgfpathlineto{\pgfqpoint{2.519381in}{0.787277in}}%
\pgfpathlineto{\pgfqpoint{2.559285in}{0.563648in}}%
\pgfpathlineto{\pgfqpoint{2.577827in}{0.482729in}}%
\pgfpathlineto{\pgfqpoint{2.599892in}{0.453520in}}%
\pgfpathlineto{\pgfqpoint{2.616325in}{0.487954in}}%
\pgfpathlineto{\pgfqpoint{2.653177in}{0.663031in}}%
\pgfpathlineto{\pgfqpoint{2.673362in}{0.758215in}}%
\pgfpathlineto{\pgfqpoint{2.694253in}{0.824539in}}%
\pgfpathlineto{\pgfqpoint{2.712094in}{0.848996in}}%
\pgfpathlineto{\pgfqpoint{2.729463in}{0.852474in}}%
\pgfpathlineto{\pgfqpoint{2.750590in}{0.841849in}}%
\pgfpathlineto{\pgfqpoint{2.768898in}{0.797823in}}%
\pgfpathlineto{\pgfqpoint{2.787208in}{0.700479in}}%
\pgfpathlineto{\pgfqpoint{2.808333in}{0.609791in}}%
\pgfpathlineto{\pgfqpoint{2.832275in}{0.489217in}}%
\pgfpathlineto{\pgfqpoint{2.848705in}{0.456304in}}%
\pgfpathlineto{\pgfqpoint{2.867484in}{0.473835in}}%
\pgfpathlineto{\pgfqpoint{2.885558in}{0.550572in}}%
\pgfpathlineto{\pgfqpoint{2.901989in}{0.640725in}}%
\pgfpathlineto{\pgfqpoint{2.942363in}{0.812208in}}%
\pgfpathlineto{\pgfqpoint{2.961611in}{0.847746in}}%
\pgfpathlineto{\pgfqpoint{2.985787in}{0.850906in}}%
\pgfpathlineto{\pgfqpoint{3.000575in}{0.843431in}}%
\pgfpathlineto{\pgfqpoint{3.017711in}{0.808285in}}%
\pgfpathlineto{\pgfqpoint{3.040950in}{0.782590in}}%
\pgfpathlineto{\pgfqpoint{3.055737in}{0.834836in}}%
\pgfpathlineto{\pgfqpoint{3.076628in}{0.854818in}}%
\pgfpathlineto{\pgfqpoint{3.097753in}{0.836162in}}%
\pgfpathlineto{\pgfqpoint{3.115358in}{0.771426in}}%
\pgfpathlineto{\pgfqpoint{3.134371in}{0.662929in}}%
\pgfpathlineto{\pgfqpoint{3.154324in}{0.558069in}}%
\pgfpathlineto{\pgfqpoint{3.172866in}{0.497432in}}%
\pgfpathlineto{\pgfqpoint{3.194697in}{0.468417in}}%
\pgfpathlineto{\pgfqpoint{3.208547in}{0.546554in}}%
\pgfpathlineto{\pgfqpoint{3.232489in}{0.638831in}}%
\pgfpathlineto{\pgfqpoint{3.269341in}{0.799044in}}%
\pgfpathlineto{\pgfqpoint{3.290466in}{0.847949in}}%
\pgfpathlineto{\pgfqpoint{3.307133in}{0.856861in}}%
\pgfpathlineto{\pgfqpoint{3.328493in}{0.831419in}}%
\pgfpathlineto{\pgfqpoint{3.346332in}{0.783353in}}%
\pgfpathlineto{\pgfqpoint{3.366988in}{0.716870in}}%
\pgfpathlineto{\pgfqpoint{3.385298in}{0.614552in}}%
\pgfpathlineto{\pgfqpoint{3.405015in}{0.557079in}}%
\pgfpathlineto{\pgfqpoint{3.424731in}{0.475801in}}%
\pgfpathlineto{\pgfqpoint{3.442570in}{0.470519in}}%
\pgfpathlineto{\pgfqpoint{3.464635in}{0.552871in}}%
\pgfpathlineto{\pgfqpoint{3.478954in}{0.634706in}}%
\pgfpathlineto{\pgfqpoint{3.498670in}{0.735700in}}%
\pgfpathlineto{\pgfqpoint{3.519563in}{0.813309in}}%
\pgfpathlineto{\pgfqpoint{3.537871in}{0.851840in}}%
\pgfpathlineto{\pgfqpoint{3.559701in}{0.613628in}}%
\pgfpathlineto{\pgfqpoint{3.576132in}{0.674002in}}%
\pgfpathlineto{\pgfqpoint{3.596319in}{0.770228in}}%
\pgfpathlineto{\pgfqpoint{3.620027in}{0.844907in}}%
\pgfpathlineto{\pgfqpoint{3.636692in}{0.861143in}}%
\pgfpathlineto{\pgfqpoint{3.651951in}{0.856006in}}%
\pgfpathlineto{\pgfqpoint{3.671902in}{0.817442in}}%
\pgfpathlineto{\pgfqpoint{3.693497in}{0.725791in}}%
\pgfpathlineto{\pgfqpoint{3.711337in}{0.635833in}}%
\pgfpathlineto{\pgfqpoint{3.730350in}{0.551285in}}%
\pgfpathlineto{\pgfqpoint{3.750066in}{0.477823in}}%
\pgfpathlineto{\pgfqpoint{3.772365in}{0.509981in}}%
\pgfpathlineto{\pgfqpoint{3.788562in}{0.563407in}}%
\pgfpathlineto{\pgfqpoint{3.807341in}{0.523476in}}%
\pgfpathlineto{\pgfqpoint{3.863675in}{0.799054in}}%
\pgfpathlineto{\pgfqpoint{3.885271in}{0.852058in}}%
\pgfpathlineto{\pgfqpoint{3.904050in}{0.866244in}}%
\pgfpathlineto{\pgfqpoint{3.924471in}{0.853736in}}%
\pgfpathlineto{\pgfqpoint{3.942545in}{0.812779in}}%
\pgfpathlineto{\pgfqpoint{3.962967in}{0.725165in}}%
\pgfpathlineto{\pgfqpoint{3.998411in}{0.551977in}}%
\pgfpathlineto{\pgfqpoint{4.019536in}{0.490475in}}%
\pgfpathlineto{\pgfqpoint{4.039723in}{0.511328in}}%
\pgfpathlineto{\pgfqpoint{4.077513in}{0.685816in}}%
\pgfpathlineto{\pgfqpoint{4.095118in}{0.756409in}}%
\pgfpathlineto{\pgfqpoint{4.113897in}{0.822566in}}%
\pgfpathlineto{\pgfqpoint{4.133615in}{0.861322in}}%
\pgfpathlineto{\pgfqpoint{4.151689in}{0.871333in}}%
\pgfpathlineto{\pgfqpoint{4.173048in}{0.858087in}}%
\pgfpathlineto{\pgfqpoint{4.191827in}{0.824835in}}%
\pgfpathlineto{\pgfqpoint{4.212483in}{0.838939in}}%
\pgfpathlineto{\pgfqpoint{4.231028in}{0.830798in}}%
\pgfpathlineto{\pgfqpoint{4.249101in}{0.757637in}}%
\pgfpathlineto{\pgfqpoint{4.269992in}{0.643963in}}%
\pgfpathlineto{\pgfqpoint{4.290179in}{0.552479in}}%
\pgfpathlineto{\pgfqpoint{4.306610in}{0.496839in}}%
\pgfpathlineto{\pgfqpoint{4.327501in}{0.525463in}}%
\pgfpathlineto{\pgfqpoint{4.363884in}{0.681666in}}%
\pgfpathlineto{\pgfqpoint{4.382898in}{0.775870in}}%
\pgfpathlineto{\pgfqpoint{4.401674in}{0.835957in}}%
\pgfpathlineto{\pgfqpoint{4.423036in}{0.868248in}}%
\pgfpathlineto{\pgfqpoint{4.440875in}{0.877685in}}%
\pgfpathlineto{\pgfqpoint{4.462236in}{0.861766in}}%
\pgfpathlineto{\pgfqpoint{4.480310in}{0.828266in}}%
\pgfpathlineto{\pgfqpoint{4.481718in}{0.830028in}}%
\pgfpathlineto{\pgfqpoint{4.474442in}{0.850070in}}%
\pgfpathlineto{\pgfqpoint{4.453786in}{0.877495in}}%
\pgfpathlineto{\pgfqpoint{4.436181in}{0.738331in}}%
\pgfpathlineto{\pgfqpoint{4.417168in}{0.619666in}}%
\pgfpathlineto{\pgfqpoint{4.395103in}{0.504642in}}%
\pgfpathlineto{\pgfqpoint{4.378203in}{0.535452in}}%
\pgfpathlineto{\pgfqpoint{4.356373in}{0.671616in}}%
\pgfpathlineto{\pgfqpoint{4.341820in}{0.775410in}}%
\pgfpathlineto{\pgfqpoint{4.321632in}{0.849030in}}%
\pgfpathlineto{\pgfqpoint{4.301916in}{0.872311in}}%
\pgfpathlineto{\pgfqpoint{4.282903in}{0.848080in}}%
\pgfpathlineto{\pgfqpoint{4.265063in}{0.787162in}}%
\pgfpathlineto{\pgfqpoint{4.242999in}{0.657574in}}%
\pgfpathlineto{\pgfqpoint{4.223751in}{0.551345in}}%
\pgfpathlineto{\pgfqpoint{4.202155in}{0.483801in}}%
\pgfpathlineto{\pgfqpoint{4.183847in}{0.566211in}}%
\pgfpathlineto{\pgfqpoint{4.149341in}{0.787357in}}%
\pgfpathlineto{\pgfqpoint{4.128216in}{0.856466in}}%
\pgfpathlineto{\pgfqpoint{4.109672in}{0.866287in}}%
\pgfpathlineto{\pgfqpoint{4.090893in}{0.838193in}}%
\pgfpathlineto{\pgfqpoint{4.068830in}{0.746870in}}%
\pgfpathlineto{\pgfqpoint{4.052163in}{0.652368in}}%
\pgfpathlineto{\pgfqpoint{4.034089in}{0.540133in}}%
\pgfpathlineto{\pgfqpoint{4.013668in}{0.475625in}}%
\pgfpathlineto{\pgfqpoint{3.993011in}{0.556731in}}%
\pgfpathlineto{\pgfqpoint{3.975875in}{0.675338in}}%
\pgfpathlineto{\pgfqpoint{3.951465in}{0.801180in}}%
\pgfpathlineto{\pgfqpoint{3.937146in}{0.847087in}}%
\pgfpathlineto{\pgfqpoint{3.918838in}{0.862790in}}%
\pgfpathlineto{\pgfqpoint{3.897947in}{0.835186in}}%
\pgfpathlineto{\pgfqpoint{3.879871in}{0.765632in}}%
\pgfpathlineto{\pgfqpoint{3.861798in}{0.672953in}}%
\pgfpathlineto{\pgfqpoint{3.839733in}{0.551522in}}%
\pgfpathlineto{\pgfqpoint{3.821660in}{0.474128in}}%
\pgfpathlineto{\pgfqpoint{3.802412in}{0.498848in}}%
\pgfpathlineto{\pgfqpoint{3.784336in}{0.585277in}}%
\pgfpathlineto{\pgfqpoint{3.764620in}{0.702899in}}%
\pgfpathlineto{\pgfqpoint{3.740912in}{0.817623in}}%
\pgfpathlineto{\pgfqpoint{3.724247in}{0.840387in}}%
\pgfpathlineto{\pgfqpoint{3.706408in}{0.858923in}}%
\pgfpathlineto{\pgfqpoint{3.683169in}{0.835175in}}%
\pgfpathlineto{\pgfqpoint{3.666973in}{0.779497in}}%
\pgfpathlineto{\pgfqpoint{3.649368in}{0.681550in}}%
\pgfpathlineto{\pgfqpoint{3.610168in}{0.510699in}}%
\pgfpathlineto{\pgfqpoint{3.589277in}{0.464102in}}%
\pgfpathlineto{\pgfqpoint{3.572141in}{0.529763in}}%
\pgfpathlineto{\pgfqpoint{3.551721in}{0.630613in}}%
\pgfpathlineto{\pgfqpoint{3.532003in}{0.662862in}}%
\pgfpathlineto{\pgfqpoint{3.513695in}{0.737402in}}%
\pgfpathlineto{\pgfqpoint{3.495619in}{0.816857in}}%
\pgfpathlineto{\pgfqpoint{3.474963in}{0.854743in}}%
\pgfpathlineto{\pgfqpoint{3.451490in}{0.846957in}}%
\pgfpathlineto{\pgfqpoint{3.436938in}{0.816265in}}%
\pgfpathlineto{\pgfqpoint{3.416046in}{0.762288in}}%
\pgfpathlineto{\pgfqpoint{3.398677in}{0.699489in}}%
\pgfpathlineto{\pgfqpoint{3.378021in}{0.612077in}}%
\pgfpathlineto{\pgfqpoint{3.359477in}{0.523070in}}%
\pgfpathlineto{\pgfqpoint{3.341638in}{0.459562in}}%
\pgfpathlineto{\pgfqpoint{3.321685in}{0.504748in}}%
\pgfpathlineto{\pgfqpoint{3.303611in}{0.593194in}}%
\pgfpathlineto{\pgfqpoint{3.282486in}{0.710914in}}%
\pgfpathlineto{\pgfqpoint{3.264411in}{0.787317in}}%
\pgfpathlineto{\pgfqpoint{3.246808in}{0.841357in}}%
\pgfpathlineto{\pgfqpoint{3.227324in}{0.853542in}}%
\pgfpathlineto{\pgfqpoint{3.206433in}{0.834340in}}%
\pgfpathlineto{\pgfqpoint{3.182257in}{0.755814in}}%
\pgfpathlineto{\pgfqpoint{3.131085in}{0.502034in}}%
\pgfpathlineto{\pgfqpoint{3.110664in}{0.454070in}}%
\pgfpathlineto{\pgfqpoint{3.090476in}{0.505844in}}%
\pgfpathlineto{\pgfqpoint{3.072637in}{0.501032in}}%
\pgfpathlineto{\pgfqpoint{3.051512in}{0.603433in}}%
\pgfpathlineto{\pgfqpoint{3.016068in}{0.805003in}}%
\pgfpathlineto{\pgfqpoint{2.994003in}{0.846404in}}%
\pgfpathlineto{\pgfqpoint{2.973582in}{0.848253in}}%
\pgfpathlineto{\pgfqpoint{2.953394in}{0.812291in}}%
\pgfpathlineto{\pgfqpoint{2.934852in}{0.744259in}}%
\pgfpathlineto{\pgfqpoint{2.917716in}{0.693638in}}%
\pgfpathlineto{\pgfqpoint{2.898703in}{0.586558in}}%
\pgfpathlineto{\pgfqpoint{2.880629in}{0.494466in}}%
\pgfpathlineto{\pgfqpoint{2.861147in}{0.452911in}}%
\pgfpathlineto{\pgfqpoint{2.840256in}{0.501555in}}%
\pgfpathlineto{\pgfqpoint{2.821478in}{0.564193in}}%
\pgfpathlineto{\pgfqpoint{2.804576in}{0.665257in}}%
\pgfpathlineto{\pgfqpoint{2.786268in}{0.727843in}}%
\pgfpathlineto{\pgfqpoint{2.764672in}{0.818947in}}%
\pgfpathlineto{\pgfqpoint{2.744016in}{0.850825in}}%
\pgfpathlineto{\pgfqpoint{2.725942in}{0.844375in}}%
\pgfpathlineto{\pgfqpoint{2.703878in}{0.794589in}}%
\pgfpathlineto{\pgfqpoint{2.688150in}{0.765048in}}%
\pgfpathlineto{\pgfqpoint{2.650595in}{0.572686in}}%
\pgfpathlineto{\pgfqpoint{2.630173in}{0.478759in}}%
\pgfpathlineto{\pgfqpoint{2.611865in}{0.455099in}}%
\pgfpathlineto{\pgfqpoint{2.595198in}{0.535035in}}%
\pgfpathlineto{\pgfqpoint{2.569613in}{0.596933in}}%
\pgfpathlineto{\pgfqpoint{2.554120in}{0.686570in}}%
\pgfpathlineto{\pgfqpoint{2.533464in}{0.775935in}}%
\pgfpathlineto{\pgfqpoint{2.514921in}{0.823258in}}%
\pgfpathlineto{\pgfqpoint{2.493325in}{0.851552in}}%
\pgfpathlineto{\pgfqpoint{2.473843in}{0.833791in}}%
\pgfpathlineto{\pgfqpoint{2.454596in}{0.788040in}}%
\pgfpathlineto{\pgfqpoint{2.436286in}{0.711994in}}%
\pgfpathlineto{\pgfqpoint{2.397321in}{0.528099in}}%
\pgfpathlineto{\pgfqpoint{2.376431in}{0.452912in}}%
\pgfpathlineto{\pgfqpoint{2.358826in}{0.488267in}}%
\pgfpathlineto{\pgfqpoint{2.340516in}{0.538888in}}%
\pgfpathlineto{\pgfqpoint{2.304369in}{0.737828in}}%
\pgfpathlineto{\pgfqpoint{2.283478in}{0.814208in}}%
\pgfpathlineto{\pgfqpoint{2.263760in}{0.849820in}}%
\pgfpathlineto{\pgfqpoint{2.247798in}{0.851328in}}%
\pgfpathlineto{\pgfqpoint{2.226673in}{0.830718in}}%
\pgfpathlineto{\pgfqpoint{2.204139in}{0.766018in}}%
\pgfpathlineto{\pgfqpoint{2.189352in}{0.692079in}}%
\pgfpathlineto{\pgfqpoint{2.167287in}{0.612670in}}%
\pgfpathlineto{\pgfqpoint{2.148977in}{0.528157in}}%
\pgfpathlineto{\pgfqpoint{2.129964in}{0.492716in}}%
\pgfpathlineto{\pgfqpoint{2.110716in}{0.459463in}}%
\pgfpathlineto{\pgfqpoint{2.092174in}{0.507115in}}%
\pgfpathlineto{\pgfqpoint{2.067761in}{0.628672in}}%
\pgfpathlineto{\pgfqpoint{2.051565in}{0.638239in}}%
\pgfpathlineto{\pgfqpoint{2.033256in}{0.741877in}}%
\pgfpathlineto{\pgfqpoint{2.014946in}{0.720647in}}%
\pgfpathlineto{\pgfqpoint{1.994996in}{0.815563in}}%
\pgfpathlineto{\pgfqpoint{1.975513in}{0.851263in}}%
\pgfpathlineto{\pgfqpoint{1.955561in}{0.512348in}}%
\pgfpathlineto{\pgfqpoint{1.937956in}{0.611870in}}%
\pgfpathlineto{\pgfqpoint{1.915657in}{0.747950in}}%
\pgfpathlineto{\pgfqpoint{1.896409in}{0.821776in}}%
\pgfpathlineto{\pgfqpoint{1.879039in}{0.853282in}}%
\pgfpathlineto{\pgfqpoint{1.859557in}{0.854577in}}%
\pgfpathlineto{\pgfqpoint{1.841952in}{0.828381in}}%
\pgfpathlineto{\pgfqpoint{1.824816in}{0.776672in}}%
\pgfpathlineto{\pgfqpoint{1.781392in}{0.589649in}}%
\pgfpathlineto{\pgfqpoint{1.761910in}{0.500527in}}%
\pgfpathlineto{\pgfqpoint{1.743834in}{0.465193in}}%
\pgfpathlineto{\pgfqpoint{1.726466in}{0.523870in}}%
\pgfpathlineto{\pgfqpoint{1.704870in}{0.611076in}}%
\pgfpathlineto{\pgfqpoint{1.690082in}{0.704092in}}%
\pgfpathlineto{\pgfqpoint{1.669426in}{0.806279in}}%
\pgfpathlineto{\pgfqpoint{1.650413in}{0.848756in}}%
\pgfpathlineto{\pgfqpoint{1.630460in}{0.857784in}}%
\pgfpathlineto{\pgfqpoint{1.609335in}{0.834137in}}%
\pgfpathlineto{\pgfqpoint{1.593844in}{0.853849in}}%
\pgfpathlineto{\pgfqpoint{1.571308in}{0.811981in}}%
\pgfpathlineto{\pgfqpoint{1.553469in}{0.742593in}}%
\pgfpathlineto{\pgfqpoint{1.536335in}{0.662508in}}%
\pgfpathlineto{\pgfqpoint{1.515443in}{0.557749in}}%
\pgfpathlineto{\pgfqpoint{1.496195in}{0.484167in}}%
\pgfpathlineto{\pgfqpoint{1.471550in}{0.494805in}}%
\pgfpathlineto{\pgfqpoint{1.458168in}{0.546965in}}%
\pgfpathlineto{\pgfqpoint{1.420378in}{0.763378in}}%
\pgfpathlineto{\pgfqpoint{1.399017in}{0.827656in}}%
\pgfpathlineto{\pgfqpoint{1.381412in}{0.856977in}}%
\pgfpathlineto{\pgfqpoint{1.363573in}{0.862013in}}%
\pgfpathlineto{\pgfqpoint{1.336580in}{0.830257in}}%
\pgfpathlineto{\pgfqpoint{1.321323in}{0.784902in}}%
\pgfpathlineto{\pgfqpoint{1.303718in}{0.721454in}}%
\pgfpathlineto{\pgfqpoint{1.263578in}{0.537680in}}%
\pgfpathlineto{\pgfqpoint{1.246444in}{0.480386in}}%
\pgfpathlineto{\pgfqpoint{1.225317in}{0.489544in}}%
\pgfpathlineto{\pgfqpoint{1.187761in}{0.646024in}}%
\pgfpathlineto{\pgfqpoint{1.150203in}{0.802988in}}%
\pgfpathlineto{\pgfqpoint{1.131427in}{0.850642in}}%
\pgfpathlineto{\pgfqpoint{1.111708in}{0.867177in}}%
\pgfpathlineto{\pgfqpoint{1.091757in}{0.728731in}}%
\pgfpathlineto{\pgfqpoint{1.072744in}{0.814019in}}%
\pgfpathlineto{\pgfqpoint{1.051383in}{0.861007in}}%
\pgfpathlineto{\pgfqpoint{1.032369in}{0.867551in}}%
\pgfpathlineto{\pgfqpoint{1.016878in}{0.852918in}}%
\pgfpathlineto{\pgfqpoint{0.996456in}{0.812288in}}%
\pgfpathlineto{\pgfqpoint{0.977209in}{0.742091in}}%
\pgfpathlineto{\pgfqpoint{0.959604in}{0.658430in}}%
\pgfpathlineto{\pgfqpoint{0.937071in}{0.557689in}}%
\pgfpathlineto{\pgfqpoint{0.918761in}{0.494176in}}%
\pgfpathlineto{\pgfqpoint{0.901861in}{0.509104in}}%
\pgfpathlineto{\pgfqpoint{0.882379in}{0.594266in}}%
\pgfpathlineto{\pgfqpoint{0.859609in}{0.705587in}}%
\pgfpathlineto{\pgfqpoint{0.841535in}{0.766209in}}%
\pgfpathlineto{\pgfqpoint{0.819940in}{0.828598in}}%
\pgfpathlineto{\pgfqpoint{0.804448in}{0.542162in}}%
\pgfpathlineto{\pgfqpoint{0.761257in}{0.781208in}}%
\pgfpathlineto{\pgfqpoint{0.747878in}{0.827274in}}%
\pgfpathlineto{\pgfqpoint{0.726753in}{0.869461in}}%
\pgfpathlineto{\pgfqpoint{0.707974in}{0.870998in}}%
\pgfpathlineto{\pgfqpoint{0.689431in}{0.847980in}}%
\pgfpathlineto{\pgfqpoint{0.670418in}{0.800739in}}%
\pgfpathlineto{\pgfqpoint{0.648588in}{0.703651in}}%
\pgfpathlineto{\pgfqpoint{0.648353in}{0.704665in}}%
\pgfpathlineto{\pgfqpoint{0.658916in}{0.758697in}}%
\pgfpathlineto{\pgfqpoint{0.675815in}{0.835977in}}%
\pgfpathlineto{\pgfqpoint{0.694829in}{0.870416in}}%
\pgfpathlineto{\pgfqpoint{0.716893in}{0.861157in}}%
\pgfpathlineto{\pgfqpoint{0.734498in}{0.803521in}}%
\pgfpathlineto{\pgfqpoint{0.752572in}{0.699291in}}%
\pgfpathlineto{\pgfqpoint{0.772759in}{0.557627in}}%
\pgfpathlineto{\pgfqpoint{0.790364in}{0.487225in}}%
\pgfpathlineto{\pgfqpoint{0.811725in}{0.576373in}}%
\pgfpathlineto{\pgfqpoint{0.828859in}{0.686723in}}%
\pgfpathlineto{\pgfqpoint{0.848578in}{0.793667in}}%
\pgfpathlineto{\pgfqpoint{0.867591in}{0.850396in}}%
\pgfpathlineto{\pgfqpoint{0.885899in}{0.867498in}}%
\pgfpathlineto{\pgfqpoint{0.910546in}{0.817529in}}%
\pgfpathlineto{\pgfqpoint{0.925568in}{0.741035in}}%
\pgfpathlineto{\pgfqpoint{0.943642in}{0.617559in}}%
\pgfpathlineto{\pgfqpoint{0.961716in}{0.508318in}}%
\pgfpathlineto{\pgfqpoint{0.983077in}{0.500883in}}%
\pgfpathlineto{\pgfqpoint{1.000916in}{0.596965in}}%
\pgfpathlineto{\pgfqpoint{1.021807in}{0.721529in}}%
\pgfpathlineto{\pgfqpoint{1.039412in}{0.814150in}}%
\pgfpathlineto{\pgfqpoint{1.060068in}{0.860761in}}%
\pgfpathlineto{\pgfqpoint{1.078378in}{0.857079in}}%
\pgfpathlineto{\pgfqpoint{1.099503in}{0.800730in}}%
\pgfpathlineto{\pgfqpoint{1.138467in}{0.569545in}}%
\pgfpathlineto{\pgfqpoint{1.155837in}{0.482355in}}%
\pgfpathlineto{\pgfqpoint{1.173911in}{0.493976in}}%
\pgfpathlineto{\pgfqpoint{1.191987in}{0.584964in}}%
\pgfpathlineto{\pgfqpoint{1.214051in}{0.705732in}}%
\pgfpathlineto{\pgfqpoint{1.233299in}{0.795039in}}%
\pgfpathlineto{\pgfqpoint{1.254189in}{0.847558in}}%
\pgfpathlineto{\pgfqpoint{1.272968in}{0.859462in}}%
\pgfpathlineto{\pgfqpoint{1.290573in}{0.836334in}}%
\pgfpathlineto{\pgfqpoint{1.312872in}{0.744858in}}%
\pgfpathlineto{\pgfqpoint{1.329772in}{0.653675in}}%
\pgfpathlineto{\pgfqpoint{1.348316in}{0.541579in}}%
\pgfpathlineto{\pgfqpoint{1.368503in}{0.463920in}}%
\pgfpathlineto{\pgfqpoint{1.385872in}{0.507308in}}%
\pgfpathlineto{\pgfqpoint{1.405120in}{0.561637in}}%
\pgfpathlineto{\pgfqpoint{1.444789in}{0.785149in}}%
\pgfpathlineto{\pgfqpoint{1.464273in}{0.832080in}}%
\pgfpathlineto{\pgfqpoint{1.482581in}{0.855867in}}%
\pgfpathlineto{\pgfqpoint{1.503472in}{0.842484in}}%
\pgfpathlineto{\pgfqpoint{1.520137in}{0.798610in}}%
\pgfpathlineto{\pgfqpoint{1.538681in}{0.696001in}}%
\pgfpathlineto{\pgfqpoint{1.557929in}{0.607607in}}%
\pgfpathlineto{\pgfqpoint{1.581167in}{0.493649in}}%
\pgfpathlineto{\pgfqpoint{1.597598in}{0.457377in}}%
\pgfpathlineto{\pgfqpoint{1.618960in}{0.496494in}}%
\pgfpathlineto{\pgfqpoint{1.634451in}{0.570225in}}%
\pgfpathlineto{\pgfqpoint{1.655576in}{0.659459in}}%
\pgfpathlineto{\pgfqpoint{1.676232in}{0.762755in}}%
\pgfpathlineto{\pgfqpoint{1.691959in}{0.809367in}}%
\pgfpathlineto{\pgfqpoint{1.713321in}{0.845387in}}%
\pgfpathlineto{\pgfqpoint{1.732568in}{0.847506in}}%
\pgfpathlineto{\pgfqpoint{1.753693in}{0.794738in}}%
\pgfpathlineto{\pgfqpoint{1.773175in}{0.844716in}}%
\pgfpathlineto{\pgfqpoint{1.786789in}{0.854018in}}%
\pgfpathlineto{\pgfqpoint{1.814722in}{0.804545in}}%
\pgfpathlineto{\pgfqpoint{1.828338in}{0.739999in}}%
\pgfpathlineto{\pgfqpoint{1.849697in}{0.608782in}}%
\pgfpathlineto{\pgfqpoint{1.868242in}{0.516077in}}%
\pgfpathlineto{\pgfqpoint{1.887255in}{0.456632in}}%
\pgfpathlineto{\pgfqpoint{1.906268in}{0.479551in}}%
\pgfpathlineto{\pgfqpoint{1.925750in}{0.564258in}}%
\pgfpathlineto{\pgfqpoint{1.944529in}{0.673782in}}%
\pgfpathlineto{\pgfqpoint{1.982319in}{0.812882in}}%
\pgfpathlineto{\pgfqpoint{2.001098in}{0.848438in}}%
\pgfpathlineto{\pgfqpoint{2.019406in}{0.849586in}}%
\pgfpathlineto{\pgfqpoint{2.038185in}{0.815617in}}%
\pgfpathlineto{\pgfqpoint{2.060015in}{0.721089in}}%
\pgfpathlineto{\pgfqpoint{2.079497in}{0.612608in}}%
\pgfpathlineto{\pgfqpoint{2.098511in}{0.514723in}}%
\pgfpathlineto{\pgfqpoint{2.117055in}{0.457201in}}%
\pgfpathlineto{\pgfqpoint{2.135832in}{0.482959in}}%
\pgfpathlineto{\pgfqpoint{2.153908in}{0.570853in}}%
\pgfpathlineto{\pgfqpoint{2.173155in}{0.654479in}}%
\pgfpathlineto{\pgfqpoint{2.193577in}{0.767618in}}%
\pgfpathlineto{\pgfqpoint{2.215876in}{0.829122in}}%
\pgfpathlineto{\pgfqpoint{2.233010in}{0.851141in}}%
\pgfpathlineto{\pgfqpoint{2.248269in}{0.844276in}}%
\pgfpathlineto{\pgfqpoint{2.269863in}{0.809414in}}%
\pgfpathlineto{\pgfqpoint{2.289581in}{0.723049in}}%
\pgfpathlineto{\pgfqpoint{2.312114in}{0.591266in}}%
\pgfpathlineto{\pgfqpoint{2.329485in}{0.514515in}}%
\pgfpathlineto{\pgfqpoint{2.348967in}{0.454821in}}%
\pgfpathlineto{\pgfqpoint{2.367511in}{0.478501in}}%
\pgfpathlineto{\pgfqpoint{2.385585in}{0.547567in}}%
\pgfpathlineto{\pgfqpoint{2.423377in}{0.745630in}}%
\pgfpathlineto{\pgfqpoint{2.444502in}{0.822924in}}%
\pgfpathlineto{\pgfqpoint{2.464453in}{0.849936in}}%
\pgfpathlineto{\pgfqpoint{2.481120in}{0.848804in}}%
\pgfpathlineto{\pgfqpoint{2.502011in}{0.818998in}}%
\pgfpathlineto{\pgfqpoint{2.520789in}{0.753317in}}%
\pgfpathlineto{\pgfqpoint{2.541914in}{0.848669in}}%
\pgfpathlineto{\pgfqpoint{2.559754in}{0.809489in}}%
\pgfpathlineto{\pgfqpoint{2.575716in}{0.737239in}}%
\pgfpathlineto{\pgfqpoint{2.601771in}{0.588021in}}%
\pgfpathlineto{\pgfqpoint{2.616793in}{0.516357in}}%
\pgfpathlineto{\pgfqpoint{2.635572in}{0.452920in}}%
\pgfpathlineto{\pgfqpoint{2.654586in}{0.492721in}}%
\pgfpathlineto{\pgfqpoint{2.677822in}{0.601608in}}%
\pgfpathlineto{\pgfqpoint{2.693315in}{0.682308in}}%
\pgfpathlineto{\pgfqpoint{2.711858in}{0.772055in}}%
\pgfpathlineto{\pgfqpoint{2.733454in}{0.836297in}}%
\pgfpathlineto{\pgfqpoint{2.751998in}{0.632489in}}%
\pgfpathlineto{\pgfqpoint{2.770775in}{0.739797in}}%
\pgfpathlineto{\pgfqpoint{2.788850in}{0.800738in}}%
\pgfpathlineto{\pgfqpoint{2.807627in}{0.845224in}}%
\pgfpathlineto{\pgfqpoint{2.829692in}{0.852867in}}%
\pgfpathlineto{\pgfqpoint{2.846125in}{0.831168in}}%
\pgfpathlineto{\pgfqpoint{2.866076in}{0.751215in}}%
\pgfpathlineto{\pgfqpoint{2.885089in}{0.631511in}}%
\pgfpathlineto{\pgfqpoint{2.903868in}{0.529148in}}%
\pgfpathlineto{\pgfqpoint{2.922176in}{0.459753in}}%
\pgfpathlineto{\pgfqpoint{2.943772in}{0.500367in}}%
\pgfpathlineto{\pgfqpoint{2.962080in}{0.586594in}}%
\pgfpathlineto{\pgfqpoint{2.997994in}{0.788201in}}%
\pgfpathlineto{\pgfqpoint{3.018651in}{0.836934in}}%
\pgfpathlineto{\pgfqpoint{3.036959in}{0.854651in}}%
\pgfpathlineto{\pgfqpoint{3.059258in}{0.838039in}}%
\pgfpathlineto{\pgfqpoint{3.078740in}{0.770129in}}%
\pgfpathlineto{\pgfqpoint{3.097050in}{0.701887in}}%
\pgfpathlineto{\pgfqpoint{3.114889in}{0.647677in}}%
\pgfpathlineto{\pgfqpoint{3.135545in}{0.536345in}}%
\pgfpathlineto{\pgfqpoint{3.154324in}{0.468002in}}%
\pgfpathlineto{\pgfqpoint{3.175215in}{0.491268in}}%
\pgfpathlineto{\pgfqpoint{3.192114in}{0.553360in}}%
\pgfpathlineto{\pgfqpoint{3.231080in}{0.741201in}}%
\pgfpathlineto{\pgfqpoint{3.250797in}{0.817009in}}%
\pgfpathlineto{\pgfqpoint{3.267228in}{0.851585in}}%
\pgfpathlineto{\pgfqpoint{3.287415in}{0.856824in}}%
\pgfpathlineto{\pgfqpoint{3.308540in}{0.844475in}}%
\pgfpathlineto{\pgfqpoint{3.326145in}{0.798755in}}%
\pgfpathlineto{\pgfqpoint{3.365580in}{0.598253in}}%
\pgfpathlineto{\pgfqpoint{3.386236in}{0.503121in}}%
\pgfpathlineto{\pgfqpoint{3.403841in}{0.465478in}}%
\pgfpathlineto{\pgfqpoint{3.422385in}{0.510598in}}%
\pgfpathlineto{\pgfqpoint{3.443276in}{0.599003in}}%
\pgfpathlineto{\pgfqpoint{3.461349in}{0.662534in}}%
\pgfpathlineto{\pgfqpoint{3.481771in}{0.757424in}}%
\pgfpathlineto{\pgfqpoint{3.503132in}{0.470372in}}%
\pgfpathlineto{\pgfqpoint{3.521909in}{0.539311in}}%
\pgfpathlineto{\pgfqpoint{3.538340in}{0.611483in}}%
\pgfpathlineto{\pgfqpoint{3.559701in}{0.721344in}}%
\pgfpathlineto{\pgfqpoint{3.576837in}{0.796680in}}%
\pgfpathlineto{\pgfqpoint{3.594911in}{0.846084in}}%
\pgfpathlineto{\pgfqpoint{3.619558in}{0.860961in}}%
\pgfpathlineto{\pgfqpoint{3.634346in}{0.849138in}}%
\pgfpathlineto{\pgfqpoint{3.655236in}{0.799406in}}%
\pgfpathlineto{\pgfqpoint{3.673076in}{0.724130in}}%
\pgfpathlineto{\pgfqpoint{3.691149in}{0.627338in}}%
\pgfpathlineto{\pgfqpoint{3.708754in}{0.544519in}}%
\pgfpathlineto{\pgfqpoint{3.730115in}{0.525633in}}%
\pgfpathlineto{\pgfqpoint{3.750535in}{0.476224in}}%
\pgfpathlineto{\pgfqpoint{3.769785in}{0.528930in}}%
\pgfpathlineto{\pgfqpoint{3.789501in}{0.614667in}}%
\pgfpathlineto{\pgfqpoint{3.807106in}{0.707270in}}%
\pgfpathlineto{\pgfqpoint{3.825414in}{0.779463in}}%
\pgfpathlineto{\pgfqpoint{3.844193in}{0.838782in}}%
\pgfpathlineto{\pgfqpoint{3.863441in}{0.859665in}}%
\pgfpathlineto{\pgfqpoint{3.885036in}{0.863363in}}%
\pgfpathlineto{\pgfqpoint{3.903345in}{0.845873in}}%
\pgfpathlineto{\pgfqpoint{3.920949in}{0.809869in}}%
\pgfpathlineto{\pgfqpoint{3.942074in}{0.719871in}}%
\pgfpathlineto{\pgfqpoint{3.963436in}{0.645849in}}%
\pgfpathlineto{\pgfqpoint{3.981275in}{0.562878in}}%
\pgfpathlineto{\pgfqpoint{3.998880in}{0.495430in}}%
\pgfpathlineto{\pgfqpoint{4.021413in}{0.509847in}}%
\pgfpathlineto{\pgfqpoint{4.040661in}{0.583025in}}%
\pgfpathlineto{\pgfqpoint{4.057562in}{0.654382in}}%
\pgfpathlineto{\pgfqpoint{4.076576in}{0.745912in}}%
\pgfpathlineto{\pgfqpoint{4.097701in}{0.827202in}}%
\pgfpathlineto{\pgfqpoint{4.114602in}{0.858393in}}%
\pgfpathlineto{\pgfqpoint{4.135727in}{0.871180in}}%
\pgfpathlineto{\pgfqpoint{4.154035in}{0.862993in}}%
\pgfpathlineto{\pgfqpoint{4.174222in}{0.828774in}}%
\pgfpathlineto{\pgfqpoint{4.192531in}{0.773016in}}%
\pgfpathlineto{\pgfqpoint{4.210135in}{0.741771in}}%
\pgfpathlineto{\pgfqpoint{4.230793in}{0.663353in}}%
\pgfpathlineto{\pgfqpoint{4.248396in}{0.869890in}}%
\pgfpathlineto{\pgfqpoint{4.269052in}{0.837931in}}%
\pgfpathlineto{\pgfqpoint{4.286894in}{0.791660in}}%
\pgfpathlineto{\pgfqpoint{4.305905in}{0.697406in}}%
\pgfpathlineto{\pgfqpoint{4.326327in}{0.579443in}}%
\pgfpathlineto{\pgfqpoint{4.346279in}{0.498448in}}%
\pgfpathlineto{\pgfqpoint{4.364822in}{0.531574in}}%
\pgfpathlineto{\pgfqpoint{4.383132in}{0.612005in}}%
\pgfpathlineto{\pgfqpoint{4.400268in}{0.714704in}}%
\pgfpathlineto{\pgfqpoint{4.422567in}{0.785074in}}%
\pgfpathlineto{\pgfqpoint{4.446509in}{0.851052in}}%
\pgfpathlineto{\pgfqpoint{4.460828in}{0.871734in}}%
\pgfpathlineto{\pgfqpoint{4.483127in}{0.876872in}}%
\pgfpathlineto{\pgfqpoint{4.472799in}{0.876400in}}%
\pgfpathlineto{\pgfqpoint{4.454020in}{0.846520in}}%
\pgfpathlineto{\pgfqpoint{4.413647in}{0.637519in}}%
\pgfpathlineto{\pgfqpoint{4.394634in}{0.526174in}}%
\pgfpathlineto{\pgfqpoint{4.379610in}{0.502319in}}%
\pgfpathlineto{\pgfqpoint{4.354730in}{0.639905in}}%
\pgfpathlineto{\pgfqpoint{4.341585in}{0.725832in}}%
\pgfpathlineto{\pgfqpoint{4.320224in}{0.831486in}}%
\pgfpathlineto{\pgfqpoint{4.302150in}{0.868906in}}%
\pgfpathlineto{\pgfqpoint{4.281025in}{0.861490in}}%
\pgfpathlineto{\pgfqpoint{4.262481in}{0.808292in}}%
\pgfpathlineto{\pgfqpoint{4.244173in}{0.708411in}}%
\pgfpathlineto{\pgfqpoint{4.225863in}{0.585362in}}%
\pgfpathlineto{\pgfqpoint{4.204503in}{0.491828in}}%
\pgfpathlineto{\pgfqpoint{4.184551in}{0.538134in}}%
\pgfpathlineto{\pgfqpoint{4.165068in}{0.666690in}}%
\pgfpathlineto{\pgfqpoint{4.165068in}{0.666690in}}%
\pgfusepath{stroke}%
\end{pgfscope}%
\begin{pgfscope}%
\pgfpathrectangle{\pgfqpoint{0.444748in}{0.431673in}}{\pgfqpoint{4.231419in}{0.467251in}}%
\pgfusepath{clip}%
\pgfsetbuttcap%
\pgfsetroundjoin%
\definecolor{currentfill}{rgb}{0.047059,0.364706,0.647059}%
\pgfsetfillcolor{currentfill}%
\pgfsetlinewidth{1.003750pt}%
\definecolor{currentstroke}{rgb}{0.047059,0.364706,0.647059}%
\pgfsetstrokecolor{currentstroke}%
\pgfsetdash{}{0pt}%
\pgfsys@defobject{currentmarker}{\pgfqpoint{-0.010417in}{-0.010417in}}{\pgfqpoint{0.010417in}{0.010417in}}{%
\pgfpathmoveto{\pgfqpoint{0.000000in}{-0.010417in}}%
\pgfpathcurveto{\pgfqpoint{0.002763in}{-0.010417in}}{\pgfqpoint{0.005412in}{-0.009319in}}{\pgfqpoint{0.007366in}{-0.007366in}}%
\pgfpathcurveto{\pgfqpoint{0.009319in}{-0.005412in}}{\pgfqpoint{0.010417in}{-0.002763in}}{\pgfqpoint{0.010417in}{0.000000in}}%
\pgfpathcurveto{\pgfqpoint{0.010417in}{0.002763in}}{\pgfqpoint{0.009319in}{0.005412in}}{\pgfqpoint{0.007366in}{0.007366in}}%
\pgfpathcurveto{\pgfqpoint{0.005412in}{0.009319in}}{\pgfqpoint{0.002763in}{0.010417in}}{\pgfqpoint{0.000000in}{0.010417in}}%
\pgfpathcurveto{\pgfqpoint{-0.002763in}{0.010417in}}{\pgfqpoint{-0.005412in}{0.009319in}}{\pgfqpoint{-0.007366in}{0.007366in}}%
\pgfpathcurveto{\pgfqpoint{-0.009319in}{0.005412in}}{\pgfqpoint{-0.010417in}{0.002763in}}{\pgfqpoint{-0.010417in}{0.000000in}}%
\pgfpathcurveto{\pgfqpoint{-0.010417in}{-0.002763in}}{\pgfqpoint{-0.009319in}{-0.005412in}}{\pgfqpoint{-0.007366in}{-0.007366in}}%
\pgfpathcurveto{\pgfqpoint{-0.005412in}{-0.009319in}}{\pgfqpoint{-0.002763in}{-0.010417in}}{\pgfqpoint{0.000000in}{-0.010417in}}%
\pgfpathlineto{\pgfqpoint{0.000000in}{-0.010417in}}%
\pgfpathclose%
\pgfusepath{stroke,fill}%
}%
\begin{pgfscope}%
\pgfsys@transformshift{0.642720in}{0.608542in}%
\pgfsys@useobject{currentmarker}{}%
\end{pgfscope}%
\begin{pgfscope}%
\pgfsys@transformshift{0.655865in}{0.690812in}%
\pgfsys@useobject{currentmarker}{}%
\end{pgfscope}%
\begin{pgfscope}%
\pgfsys@transformshift{0.677460in}{0.803479in}%
\pgfsys@useobject{currentmarker}{}%
\end{pgfscope}%
\begin{pgfscope}%
\pgfsys@transformshift{0.695534in}{0.857900in}%
\pgfsys@useobject{currentmarker}{}%
\end{pgfscope}%
\begin{pgfscope}%
\pgfsys@transformshift{0.713608in}{0.874267in}%
\pgfsys@useobject{currentmarker}{}%
\end{pgfscope}%
\begin{pgfscope}%
\pgfsys@transformshift{0.734733in}{0.844316in}%
\pgfsys@useobject{currentmarker}{}%
\end{pgfscope}%
\begin{pgfscope}%
\pgfsys@transformshift{0.753277in}{0.761166in}%
\pgfsys@useobject{currentmarker}{}%
\end{pgfscope}%
\begin{pgfscope}%
\pgfsys@transformshift{0.773699in}{0.624162in}%
\pgfsys@useobject{currentmarker}{}%
\end{pgfscope}%
\begin{pgfscope}%
\pgfsys@transformshift{0.791772in}{0.515613in}%
\pgfsys@useobject{currentmarker}{}%
\end{pgfscope}%
\begin{pgfscope}%
\pgfsys@transformshift{0.809846in}{0.503616in}%
\pgfsys@useobject{currentmarker}{}%
\end{pgfscope}%
\begin{pgfscope}%
\pgfsys@transformshift{0.830738in}{0.622273in}%
\pgfsys@useobject{currentmarker}{}%
\end{pgfscope}%
\begin{pgfscope}%
\pgfsys@transformshift{0.848107in}{0.749145in}%
\pgfsys@useobject{currentmarker}{}%
\end{pgfscope}%
\begin{pgfscope}%
\pgfsys@transformshift{0.866651in}{0.827577in}%
\pgfsys@useobject{currentmarker}{}%
\end{pgfscope}%
\begin{pgfscope}%
\pgfsys@transformshift{0.887073in}{0.867777in}%
\pgfsys@useobject{currentmarker}{}%
\end{pgfscope}%
\begin{pgfscope}%
\pgfsys@transformshift{0.906790in}{0.857801in}%
\pgfsys@useobject{currentmarker}{}%
\end{pgfscope}%
\begin{pgfscope}%
\pgfsys@transformshift{0.925334in}{0.795234in}%
\pgfsys@useobject{currentmarker}{}%
\end{pgfscope}%
\begin{pgfscope}%
\pgfsys@transformshift{0.944582in}{0.677988in}%
\pgfsys@useobject{currentmarker}{}%
\end{pgfscope}%
\begin{pgfscope}%
\pgfsys@transformshift{0.960073in}{0.558076in}%
\pgfsys@useobject{currentmarker}{}%
\end{pgfscope}%
\begin{pgfscope}%
\pgfsys@transformshift{0.982137in}{0.477381in}%
\pgfsys@useobject{currentmarker}{}%
\end{pgfscope}%
\begin{pgfscope}%
\pgfsys@transformshift{1.003968in}{0.555799in}%
\pgfsys@useobject{currentmarker}{}%
\end{pgfscope}%
\begin{pgfscope}%
\pgfsys@transformshift{1.020869in}{0.653976in}%
\pgfsys@useobject{currentmarker}{}%
\end{pgfscope}%
\begin{pgfscope}%
\pgfsys@transformshift{1.039646in}{0.758239in}%
\pgfsys@useobject{currentmarker}{}%
\end{pgfscope}%
\begin{pgfscope}%
\pgfsys@transformshift{1.058894in}{0.830227in}%
\pgfsys@useobject{currentmarker}{}%
\end{pgfscope}%
\begin{pgfscope}%
\pgfsys@transformshift{1.081898in}{0.864481in}%
\pgfsys@useobject{currentmarker}{}%
\end{pgfscope}%
\begin{pgfscope}%
\pgfsys@transformshift{1.099737in}{0.849947in}%
\pgfsys@useobject{currentmarker}{}%
\end{pgfscope}%
\begin{pgfscope}%
\pgfsys@transformshift{1.116873in}{0.794242in}%
\pgfsys@useobject{currentmarker}{}%
\end{pgfscope}%
\begin{pgfscope}%
\pgfsys@transformshift{1.135652in}{0.676155in}%
\pgfsys@useobject{currentmarker}{}%
\end{pgfscope}%
\begin{pgfscope}%
\pgfsys@transformshift{1.157717in}{0.549179in}%
\pgfsys@useobject{currentmarker}{}%
\end{pgfscope}%
\begin{pgfscope}%
\pgfsys@transformshift{1.176964in}{0.469474in}%
\pgfsys@useobject{currentmarker}{}%
\end{pgfscope}%
\begin{pgfscope}%
\pgfsys@transformshift{1.195272in}{0.517186in}%
\pgfsys@useobject{currentmarker}{}%
\end{pgfscope}%
\begin{pgfscope}%
\pgfsys@transformshift{1.216163in}{0.634063in}%
\pgfsys@useobject{currentmarker}{}%
\end{pgfscope}%
\begin{pgfscope}%
\pgfsys@transformshift{1.230247in}{0.727777in}%
\pgfsys@useobject{currentmarker}{}%
\end{pgfscope}%
\begin{pgfscope}%
\pgfsys@transformshift{1.252312in}{0.826053in}%
\pgfsys@useobject{currentmarker}{}%
\end{pgfscope}%
\begin{pgfscope}%
\pgfsys@transformshift{1.272968in}{0.852693in}%
\pgfsys@useobject{currentmarker}{}%
\end{pgfscope}%
\begin{pgfscope}%
\pgfsys@transformshift{1.289868in}{0.857661in}%
\pgfsys@useobject{currentmarker}{}%
\end{pgfscope}%
\begin{pgfscope}%
\pgfsys@transformshift{1.310055in}{0.819156in}%
\pgfsys@useobject{currentmarker}{}%
\end{pgfscope}%
\begin{pgfscope}%
\pgfsys@transformshift{1.329303in}{0.735011in}%
\pgfsys@useobject{currentmarker}{}%
\end{pgfscope}%
\begin{pgfscope}%
\pgfsys@transformshift{1.347376in}{0.616282in}%
\pgfsys@useobject{currentmarker}{}%
\end{pgfscope}%
\begin{pgfscope}%
\pgfsys@transformshift{1.366859in}{0.515072in}%
\pgfsys@useobject{currentmarker}{}%
\end{pgfscope}%
\begin{pgfscope}%
\pgfsys@transformshift{1.388923in}{0.468935in}%
\pgfsys@useobject{currentmarker}{}%
\end{pgfscope}%
\begin{pgfscope}%
\pgfsys@transformshift{1.405120in}{0.529967in}%
\pgfsys@useobject{currentmarker}{}%
\end{pgfscope}%
\begin{pgfscope}%
\pgfsys@transformshift{1.424133in}{0.611327in}%
\pgfsys@useobject{currentmarker}{}%
\end{pgfscope}%
\begin{pgfscope}%
\pgfsys@transformshift{1.443380in}{0.727969in}%
\pgfsys@useobject{currentmarker}{}%
\end{pgfscope}%
\begin{pgfscope}%
\pgfsys@transformshift{1.465211in}{0.819494in}%
\pgfsys@useobject{currentmarker}{}%
\end{pgfscope}%
\begin{pgfscope}%
\pgfsys@transformshift{1.480938in}{0.849710in}%
\pgfsys@useobject{currentmarker}{}%
\end{pgfscope}%
\begin{pgfscope}%
\pgfsys@transformshift{1.503237in}{0.852103in}%
\pgfsys@useobject{currentmarker}{}%
\end{pgfscope}%
\begin{pgfscope}%
\pgfsys@transformshift{1.520608in}{0.826423in}%
\pgfsys@useobject{currentmarker}{}%
\end{pgfscope}%
\begin{pgfscope}%
\pgfsys@transformshift{1.539855in}{0.742554in}%
\pgfsys@useobject{currentmarker}{}%
\end{pgfscope}%
\begin{pgfscope}%
\pgfsys@transformshift{1.561685in}{0.615008in}%
\pgfsys@useobject{currentmarker}{}%
\end{pgfscope}%
\begin{pgfscope}%
\pgfsys@transformshift{1.576708in}{0.545907in}%
\pgfsys@useobject{currentmarker}{}%
\end{pgfscope}%
\begin{pgfscope}%
\pgfsys@transformshift{1.596895in}{0.461987in}%
\pgfsys@useobject{currentmarker}{}%
\end{pgfscope}%
\begin{pgfscope}%
\pgfsys@transformshift{1.617315in}{0.498862in}%
\pgfsys@useobject{currentmarker}{}%
\end{pgfscope}%
\begin{pgfscope}%
\pgfsys@transformshift{1.636094in}{0.538375in}%
\pgfsys@useobject{currentmarker}{}%
\end{pgfscope}%
\begin{pgfscope}%
\pgfsys@transformshift{1.660741in}{0.640570in}%
\pgfsys@useobject{currentmarker}{}%
\end{pgfscope}%
\begin{pgfscope}%
\pgfsys@transformshift{1.676232in}{0.730583in}%
\pgfsys@useobject{currentmarker}{}%
\end{pgfscope}%
\begin{pgfscope}%
\pgfsys@transformshift{1.695011in}{0.807562in}%
\pgfsys@useobject{currentmarker}{}%
\end{pgfscope}%
\begin{pgfscope}%
\pgfsys@transformshift{1.713790in}{0.842027in}%
\pgfsys@useobject{currentmarker}{}%
\end{pgfscope}%
\begin{pgfscope}%
\pgfsys@transformshift{1.731863in}{0.482608in}%
\pgfsys@useobject{currentmarker}{}%
\end{pgfscope}%
\begin{pgfscope}%
\pgfsys@transformshift{1.751111in}{0.578138in}%
\pgfsys@useobject{currentmarker}{}%
\end{pgfscope}%
\begin{pgfscope}%
\pgfsys@transformshift{1.773410in}{0.710339in}%
\pgfsys@useobject{currentmarker}{}%
\end{pgfscope}%
\begin{pgfscope}%
\pgfsys@transformshift{1.794537in}{0.812633in}%
\pgfsys@useobject{currentmarker}{}%
\end{pgfscope}%
\begin{pgfscope}%
\pgfsys@transformshift{1.808854in}{0.833494in}%
\pgfsys@useobject{currentmarker}{}%
\end{pgfscope}%
\begin{pgfscope}%
\pgfsys@transformshift{1.825755in}{0.854256in}%
\pgfsys@useobject{currentmarker}{}%
\end{pgfscope}%
\begin{pgfscope}%
\pgfsys@transformshift{1.848994in}{0.828715in}%
\pgfsys@useobject{currentmarker}{}%
\end{pgfscope}%
\begin{pgfscope}%
\pgfsys@transformshift{1.866364in}{0.761561in}%
\pgfsys@useobject{currentmarker}{}%
\end{pgfscope}%
\begin{pgfscope}%
\pgfsys@transformshift{1.888663in}{0.620490in}%
\pgfsys@useobject{currentmarker}{}%
\end{pgfscope}%
\begin{pgfscope}%
\pgfsys@transformshift{1.904155in}{0.539251in}%
\pgfsys@useobject{currentmarker}{}%
\end{pgfscope}%
\begin{pgfscope}%
\pgfsys@transformshift{1.926219in}{0.457446in}%
\pgfsys@useobject{currentmarker}{}%
\end{pgfscope}%
\begin{pgfscope}%
\pgfsys@transformshift{1.944293in}{0.493475in}%
\pgfsys@useobject{currentmarker}{}%
\end{pgfscope}%
\begin{pgfscope}%
\pgfsys@transformshift{1.964011in}{0.581817in}%
\pgfsys@useobject{currentmarker}{}%
\end{pgfscope}%
\begin{pgfscope}%
\pgfsys@transformshift{1.981382in}{0.677458in}%
\pgfsys@useobject{currentmarker}{}%
\end{pgfscope}%
\begin{pgfscope}%
\pgfsys@transformshift{2.001098in}{0.766642in}%
\pgfsys@useobject{currentmarker}{}%
\end{pgfscope}%
\begin{pgfscope}%
\pgfsys@transformshift{2.022223in}{0.834290in}%
\pgfsys@useobject{currentmarker}{}%
\end{pgfscope}%
\begin{pgfscope}%
\pgfsys@transformshift{2.041236in}{0.853431in}%
\pgfsys@useobject{currentmarker}{}%
\end{pgfscope}%
\begin{pgfscope}%
\pgfsys@transformshift{2.059781in}{0.838260in}%
\pgfsys@useobject{currentmarker}{}%
\end{pgfscope}%
\begin{pgfscope}%
\pgfsys@transformshift{2.078089in}{0.777105in}%
\pgfsys@useobject{currentmarker}{}%
\end{pgfscope}%
\begin{pgfscope}%
\pgfsys@transformshift{2.096868in}{0.665077in}%
\pgfsys@useobject{currentmarker}{}%
\end{pgfscope}%
\begin{pgfscope}%
\pgfsys@transformshift{2.116115in}{0.564326in}%
\pgfsys@useobject{currentmarker}{}%
\end{pgfscope}%
\begin{pgfscope}%
\pgfsys@transformshift{2.136303in}{0.483846in}%
\pgfsys@useobject{currentmarker}{}%
\end{pgfscope}%
\begin{pgfscope}%
\pgfsys@transformshift{2.155785in}{0.460101in}%
\pgfsys@useobject{currentmarker}{}%
\end{pgfscope}%
\begin{pgfscope}%
\pgfsys@transformshift{2.175972in}{0.530363in}%
\pgfsys@useobject{currentmarker}{}%
\end{pgfscope}%
\begin{pgfscope}%
\pgfsys@transformshift{2.191932in}{0.611963in}%
\pgfsys@useobject{currentmarker}{}%
\end{pgfscope}%
\begin{pgfscope}%
\pgfsys@transformshift{2.213293in}{0.727811in}%
\pgfsys@useobject{currentmarker}{}%
\end{pgfscope}%
\begin{pgfscope}%
\pgfsys@transformshift{2.230898in}{0.804131in}%
\pgfsys@useobject{currentmarker}{}%
\end{pgfscope}%
\begin{pgfscope}%
\pgfsys@transformshift{2.252728in}{0.845227in}%
\pgfsys@useobject{currentmarker}{}%
\end{pgfscope}%
\begin{pgfscope}%
\pgfsys@transformshift{2.275967in}{0.850929in}%
\pgfsys@useobject{currentmarker}{}%
\end{pgfscope}%
\begin{pgfscope}%
\pgfsys@transformshift{2.289581in}{0.852806in}%
\pgfsys@useobject{currentmarker}{}%
\end{pgfscope}%
\begin{pgfscope}%
\pgfsys@transformshift{2.310940in}{0.834549in}%
\pgfsys@useobject{currentmarker}{}%
\end{pgfscope}%
\begin{pgfscope}%
\pgfsys@transformshift{2.329250in}{0.768237in}%
\pgfsys@useobject{currentmarker}{}%
\end{pgfscope}%
\begin{pgfscope}%
\pgfsys@transformshift{2.347090in}{0.663066in}%
\pgfsys@useobject{currentmarker}{}%
\end{pgfscope}%
\begin{pgfscope}%
\pgfsys@transformshift{2.368449in}{0.544253in}%
\pgfsys@useobject{currentmarker}{}%
\end{pgfscope}%
\begin{pgfscope}%
\pgfsys@transformshift{2.387697in}{0.466896in}%
\pgfsys@useobject{currentmarker}{}%
\end{pgfscope}%
\begin{pgfscope}%
\pgfsys@transformshift{2.404364in}{0.462256in}%
\pgfsys@useobject{currentmarker}{}%
\end{pgfscope}%
\begin{pgfscope}%
\pgfsys@transformshift{2.423846in}{0.527879in}%
\pgfsys@useobject{currentmarker}{}%
\end{pgfscope}%
\begin{pgfscope}%
\pgfsys@transformshift{2.442154in}{0.638704in}%
\pgfsys@useobject{currentmarker}{}%
\end{pgfscope}%
\begin{pgfscope}%
\pgfsys@transformshift{2.463750in}{0.746382in}%
\pgfsys@useobject{currentmarker}{}%
\end{pgfscope}%
\begin{pgfscope}%
\pgfsys@transformshift{2.481355in}{0.814911in}%
\pgfsys@useobject{currentmarker}{}%
\end{pgfscope}%
\begin{pgfscope}%
\pgfsys@transformshift{2.500133in}{0.846687in}%
\pgfsys@useobject{currentmarker}{}%
\end{pgfscope}%
\begin{pgfscope}%
\pgfsys@transformshift{2.520319in}{0.849788in}%
\pgfsys@useobject{currentmarker}{}%
\end{pgfscope}%
\begin{pgfscope}%
\pgfsys@transformshift{2.538863in}{0.830246in}%
\pgfsys@useobject{currentmarker}{}%
\end{pgfscope}%
\begin{pgfscope}%
\pgfsys@transformshift{2.559285in}{0.748141in}%
\pgfsys@useobject{currentmarker}{}%
\end{pgfscope}%
\begin{pgfscope}%
\pgfsys@transformshift{2.580175in}{0.632793in}%
\pgfsys@useobject{currentmarker}{}%
\end{pgfscope}%
\begin{pgfscope}%
\pgfsys@transformshift{2.594729in}{0.548213in}%
\pgfsys@useobject{currentmarker}{}%
\end{pgfscope}%
\begin{pgfscope}%
\pgfsys@transformshift{2.614916in}{0.467644in}%
\pgfsys@useobject{currentmarker}{}%
\end{pgfscope}%
\begin{pgfscope}%
\pgfsys@transformshift{2.637450in}{0.476028in}%
\pgfsys@useobject{currentmarker}{}%
\end{pgfscope}%
\begin{pgfscope}%
\pgfsys@transformshift{2.654349in}{0.551356in}%
\pgfsys@useobject{currentmarker}{}%
\end{pgfscope}%
\begin{pgfscope}%
\pgfsys@transformshift{2.673128in}{0.601190in}%
\pgfsys@useobject{currentmarker}{}%
\end{pgfscope}%
\begin{pgfscope}%
\pgfsys@transformshift{2.694019in}{0.717255in}%
\pgfsys@useobject{currentmarker}{}%
\end{pgfscope}%
\begin{pgfscope}%
\pgfsys@transformshift{2.712329in}{0.762705in}%
\pgfsys@useobject{currentmarker}{}%
\end{pgfscope}%
\begin{pgfscope}%
\pgfsys@transformshift{2.732750in}{0.829297in}%
\pgfsys@useobject{currentmarker}{}%
\end{pgfscope}%
\begin{pgfscope}%
\pgfsys@transformshift{2.750590in}{0.852438in}%
\pgfsys@useobject{currentmarker}{}%
\end{pgfscope}%
\begin{pgfscope}%
\pgfsys@transformshift{2.767960in}{0.849981in}%
\pgfsys@useobject{currentmarker}{}%
\end{pgfscope}%
\begin{pgfscope}%
\pgfsys@transformshift{2.789319in}{0.810694in}%
\pgfsys@useobject{currentmarker}{}%
\end{pgfscope}%
\begin{pgfscope}%
\pgfsys@transformshift{2.810210in}{0.714648in}%
\pgfsys@useobject{currentmarker}{}%
\end{pgfscope}%
\begin{pgfscope}%
\pgfsys@transformshift{2.828049in}{0.611271in}%
\pgfsys@useobject{currentmarker}{}%
\end{pgfscope}%
\begin{pgfscope}%
\pgfsys@transformshift{2.846594in}{0.518971in}%
\pgfsys@useobject{currentmarker}{}%
\end{pgfscope}%
\begin{pgfscope}%
\pgfsys@transformshift{2.867484in}{0.463681in}%
\pgfsys@useobject{currentmarker}{}%
\end{pgfscope}%
\begin{pgfscope}%
\pgfsys@transformshift{2.884149in}{0.460146in}%
\pgfsys@useobject{currentmarker}{}%
\end{pgfscope}%
\begin{pgfscope}%
\pgfsys@transformshift{2.906919in}{0.545945in}%
\pgfsys@useobject{currentmarker}{}%
\end{pgfscope}%
\begin{pgfscope}%
\pgfsys@transformshift{2.924758in}{0.629440in}%
\pgfsys@useobject{currentmarker}{}%
\end{pgfscope}%
\begin{pgfscope}%
\pgfsys@transformshift{2.942363in}{0.482651in}%
\pgfsys@useobject{currentmarker}{}%
\end{pgfscope}%
\begin{pgfscope}%
\pgfsys@transformshift{2.963254in}{0.473268in}%
\pgfsys@useobject{currentmarker}{}%
\end{pgfscope}%
\begin{pgfscope}%
\pgfsys@transformshift{2.981798in}{0.551102in}%
\pgfsys@useobject{currentmarker}{}%
\end{pgfscope}%
\begin{pgfscope}%
\pgfsys@transformshift{2.999167in}{0.649257in}%
\pgfsys@useobject{currentmarker}{}%
\end{pgfscope}%
\begin{pgfscope}%
\pgfsys@transformshift{3.020762in}{0.765406in}%
\pgfsys@useobject{currentmarker}{}%
\end{pgfscope}%
\begin{pgfscope}%
\pgfsys@transformshift{3.038602in}{0.828147in}%
\pgfsys@useobject{currentmarker}{}%
\end{pgfscope}%
\begin{pgfscope}%
\pgfsys@transformshift{3.058554in}{0.854664in}%
\pgfsys@useobject{currentmarker}{}%
\end{pgfscope}%
\begin{pgfscope}%
\pgfsys@transformshift{3.077097in}{0.846574in}%
\pgfsys@useobject{currentmarker}{}%
\end{pgfscope}%
\begin{pgfscope}%
\pgfsys@transformshift{3.094467in}{0.805108in}%
\pgfsys@useobject{currentmarker}{}%
\end{pgfscope}%
\begin{pgfscope}%
\pgfsys@transformshift{3.117940in}{0.692308in}%
\pgfsys@useobject{currentmarker}{}%
\end{pgfscope}%
\begin{pgfscope}%
\pgfsys@transformshift{3.136719in}{0.613345in}%
\pgfsys@useobject{currentmarker}{}%
\end{pgfscope}%
\begin{pgfscope}%
\pgfsys@transformshift{3.153853in}{0.527609in}%
\pgfsys@useobject{currentmarker}{}%
\end{pgfscope}%
\begin{pgfscope}%
\pgfsys@transformshift{3.172398in}{0.480625in}%
\pgfsys@useobject{currentmarker}{}%
\end{pgfscope}%
\begin{pgfscope}%
\pgfsys@transformshift{3.193523in}{0.469096in}%
\pgfsys@useobject{currentmarker}{}%
\end{pgfscope}%
\begin{pgfscope}%
\pgfsys@transformshift{3.212067in}{0.535101in}%
\pgfsys@useobject{currentmarker}{}%
\end{pgfscope}%
\begin{pgfscope}%
\pgfsys@transformshift{3.231315in}{0.537979in}%
\pgfsys@useobject{currentmarker}{}%
\end{pgfscope}%
\begin{pgfscope}%
\pgfsys@transformshift{3.249623in}{0.634342in}%
\pgfsys@useobject{currentmarker}{}%
\end{pgfscope}%
\begin{pgfscope}%
\pgfsys@transformshift{3.272158in}{0.738338in}%
\pgfsys@useobject{currentmarker}{}%
\end{pgfscope}%
\begin{pgfscope}%
\pgfsys@transformshift{3.289763in}{0.801914in}%
\pgfsys@useobject{currentmarker}{}%
\end{pgfscope}%
\begin{pgfscope}%
\pgfsys@transformshift{3.307837in}{0.762391in}%
\pgfsys@useobject{currentmarker}{}%
\end{pgfscope}%
\begin{pgfscope}%
\pgfsys@transformshift{3.328024in}{0.837232in}%
\pgfsys@useobject{currentmarker}{}%
\end{pgfscope}%
\begin{pgfscope}%
\pgfsys@transformshift{3.346566in}{0.857524in}%
\pgfsys@useobject{currentmarker}{}%
\end{pgfscope}%
\begin{pgfscope}%
\pgfsys@transformshift{3.364171in}{0.849217in}%
\pgfsys@useobject{currentmarker}{}%
\end{pgfscope}%
\begin{pgfscope}%
\pgfsys@transformshift{3.386470in}{0.791669in}%
\pgfsys@useobject{currentmarker}{}%
\end{pgfscope}%
\begin{pgfscope}%
\pgfsys@transformshift{3.407126in}{0.691584in}%
\pgfsys@useobject{currentmarker}{}%
\end{pgfscope}%
\begin{pgfscope}%
\pgfsys@transformshift{3.424731in}{0.594635in}%
\pgfsys@useobject{currentmarker}{}%
\end{pgfscope}%
\begin{pgfscope}%
\pgfsys@transformshift{3.442101in}{0.509062in}%
\pgfsys@useobject{currentmarker}{}%
\end{pgfscope}%
\begin{pgfscope}%
\pgfsys@transformshift{3.463697in}{0.467878in}%
\pgfsys@useobject{currentmarker}{}%
\end{pgfscope}%
\begin{pgfscope}%
\pgfsys@transformshift{3.482240in}{0.527022in}%
\pgfsys@useobject{currentmarker}{}%
\end{pgfscope}%
\begin{pgfscope}%
\pgfsys@transformshift{3.501253in}{0.618113in}%
\pgfsys@useobject{currentmarker}{}%
\end{pgfscope}%
\begin{pgfscope}%
\pgfsys@transformshift{3.521440in}{0.720542in}%
\pgfsys@useobject{currentmarker}{}%
\end{pgfscope}%
\begin{pgfscope}%
\pgfsys@transformshift{3.538340in}{0.784327in}%
\pgfsys@useobject{currentmarker}{}%
\end{pgfscope}%
\begin{pgfscope}%
\pgfsys@transformshift{3.559232in}{0.780593in}%
\pgfsys@useobject{currentmarker}{}%
\end{pgfscope}%
\begin{pgfscope}%
\pgfsys@transformshift{3.577072in}{0.834547in}%
\pgfsys@useobject{currentmarker}{}%
\end{pgfscope}%
\begin{pgfscope}%
\pgfsys@transformshift{3.598197in}{0.861311in}%
\pgfsys@useobject{currentmarker}{}%
\end{pgfscope}%
\begin{pgfscope}%
\pgfsys@transformshift{3.616270in}{0.854967in}%
\pgfsys@useobject{currentmarker}{}%
\end{pgfscope}%
\begin{pgfscope}%
\pgfsys@transformshift{3.633172in}{0.824548in}%
\pgfsys@useobject{currentmarker}{}%
\end{pgfscope}%
\begin{pgfscope}%
\pgfsys@transformshift{3.652185in}{0.757343in}%
\pgfsys@useobject{currentmarker}{}%
\end{pgfscope}%
\begin{pgfscope}%
\pgfsys@transformshift{3.673779in}{0.676636in}%
\pgfsys@useobject{currentmarker}{}%
\end{pgfscope}%
\begin{pgfscope}%
\pgfsys@transformshift{3.691149in}{0.605058in}%
\pgfsys@useobject{currentmarker}{}%
\end{pgfscope}%
\begin{pgfscope}%
\pgfsys@transformshift{3.712511in}{0.506963in}%
\pgfsys@useobject{currentmarker}{}%
\end{pgfscope}%
\begin{pgfscope}%
\pgfsys@transformshift{3.730115in}{0.474652in}%
\pgfsys@useobject{currentmarker}{}%
\end{pgfscope}%
\begin{pgfscope}%
\pgfsys@transformshift{3.749832in}{0.536603in}%
\pgfsys@useobject{currentmarker}{}%
\end{pgfscope}%
\begin{pgfscope}%
\pgfsys@transformshift{3.769080in}{0.627057in}%
\pgfsys@useobject{currentmarker}{}%
\end{pgfscope}%
\begin{pgfscope}%
\pgfsys@transformshift{3.792553in}{0.719522in}%
\pgfsys@useobject{currentmarker}{}%
\end{pgfscope}%
\begin{pgfscope}%
\pgfsys@transformshift{3.807341in}{0.781940in}%
\pgfsys@useobject{currentmarker}{}%
\end{pgfscope}%
\begin{pgfscope}%
\pgfsys@transformshift{3.825180in}{0.837140in}%
\pgfsys@useobject{currentmarker}{}%
\end{pgfscope}%
\begin{pgfscope}%
\pgfsys@transformshift{3.846305in}{0.863822in}%
\pgfsys@useobject{currentmarker}{}%
\end{pgfscope}%
\begin{pgfscope}%
\pgfsys@transformshift{3.867432in}{0.862647in}%
\pgfsys@useobject{currentmarker}{}%
\end{pgfscope}%
\begin{pgfscope}%
\pgfsys@transformshift{3.885271in}{0.843188in}%
\pgfsys@useobject{currentmarker}{}%
\end{pgfscope}%
\begin{pgfscope}%
\pgfsys@transformshift{3.902876in}{0.793388in}%
\pgfsys@useobject{currentmarker}{}%
\end{pgfscope}%
\begin{pgfscope}%
\pgfsys@transformshift{3.924235in}{0.817913in}%
\pgfsys@useobject{currentmarker}{}%
\end{pgfscope}%
\begin{pgfscope}%
\pgfsys@transformshift{3.942074in}{0.857386in}%
\pgfsys@useobject{currentmarker}{}%
\end{pgfscope}%
\begin{pgfscope}%
\pgfsys@transformshift{3.960619in}{0.866326in}%
\pgfsys@useobject{currentmarker}{}%
\end{pgfscope}%
\begin{pgfscope}%
\pgfsys@transformshift{3.981744in}{0.841332in}%
\pgfsys@useobject{currentmarker}{}%
\end{pgfscope}%
\begin{pgfscope}%
\pgfsys@transformshift{3.999348in}{0.778424in}%
\pgfsys@useobject{currentmarker}{}%
\end{pgfscope}%
\begin{pgfscope}%
\pgfsys@transformshift{4.017188in}{0.691536in}%
\pgfsys@useobject{currentmarker}{}%
\end{pgfscope}%
\begin{pgfscope}%
\pgfsys@transformshift{4.039252in}{0.578580in}%
\pgfsys@useobject{currentmarker}{}%
\end{pgfscope}%
\begin{pgfscope}%
\pgfsys@transformshift{4.056623in}{0.508630in}%
\pgfsys@useobject{currentmarker}{}%
\end{pgfscope}%
\begin{pgfscope}%
\pgfsys@transformshift{4.077279in}{0.498399in}%
\pgfsys@useobject{currentmarker}{}%
\end{pgfscope}%
\begin{pgfscope}%
\pgfsys@transformshift{4.099109in}{0.594779in}%
\pgfsys@useobject{currentmarker}{}%
\end{pgfscope}%
\begin{pgfscope}%
\pgfsys@transformshift{4.114131in}{0.675319in}%
\pgfsys@useobject{currentmarker}{}%
\end{pgfscope}%
\begin{pgfscope}%
\pgfsys@transformshift{4.138310in}{0.777645in}%
\pgfsys@useobject{currentmarker}{}%
\end{pgfscope}%
\begin{pgfscope}%
\pgfsys@transformshift{4.153332in}{0.831033in}%
\pgfsys@useobject{currentmarker}{}%
\end{pgfscope}%
\begin{pgfscope}%
\pgfsys@transformshift{4.173988in}{0.866287in}%
\pgfsys@useobject{currentmarker}{}%
\end{pgfscope}%
\begin{pgfscope}%
\pgfsys@transformshift{4.191827in}{0.870722in}%
\pgfsys@useobject{currentmarker}{}%
\end{pgfscope}%
\begin{pgfscope}%
\pgfsys@transformshift{4.210135in}{0.855854in}%
\pgfsys@useobject{currentmarker}{}%
\end{pgfscope}%
\begin{pgfscope}%
\pgfsys@transformshift{4.231965in}{0.807519in}%
\pgfsys@useobject{currentmarker}{}%
\end{pgfscope}%
\begin{pgfscope}%
\pgfsys@transformshift{4.250041in}{0.736468in}%
\pgfsys@useobject{currentmarker}{}%
\end{pgfscope}%
\begin{pgfscope}%
\pgfsys@transformshift{4.270697in}{0.624856in}%
\pgfsys@useobject{currentmarker}{}%
\end{pgfscope}%
\begin{pgfscope}%
\pgfsys@transformshift{4.287597in}{0.546416in}%
\pgfsys@useobject{currentmarker}{}%
\end{pgfscope}%
\begin{pgfscope}%
\pgfsys@transformshift{4.307079in}{0.494339in}%
\pgfsys@useobject{currentmarker}{}%
\end{pgfscope}%
\begin{pgfscope}%
\pgfsys@transformshift{4.327501in}{0.541837in}%
\pgfsys@useobject{currentmarker}{}%
\end{pgfscope}%
\begin{pgfscope}%
\pgfsys@transformshift{4.345574in}{0.634968in}%
\pgfsys@useobject{currentmarker}{}%
\end{pgfscope}%
\begin{pgfscope}%
\pgfsys@transformshift{4.364353in}{0.720277in}%
\pgfsys@useobject{currentmarker}{}%
\end{pgfscope}%
\begin{pgfscope}%
\pgfsys@transformshift{4.384540in}{0.790684in}%
\pgfsys@useobject{currentmarker}{}%
\end{pgfscope}%
\begin{pgfscope}%
\pgfsys@transformshift{4.401911in}{0.845713in}%
\pgfsys@useobject{currentmarker}{}%
\end{pgfscope}%
\begin{pgfscope}%
\pgfsys@transformshift{4.421158in}{0.872149in}%
\pgfsys@useobject{currentmarker}{}%
\end{pgfscope}%
\begin{pgfscope}%
\pgfsys@transformshift{4.438763in}{0.877643in}%
\pgfsys@useobject{currentmarker}{}%
\end{pgfscope}%
\begin{pgfscope}%
\pgfsys@transformshift{4.462236in}{0.862080in}%
\pgfsys@useobject{currentmarker}{}%
\end{pgfscope}%
\begin{pgfscope}%
\pgfsys@transformshift{4.479136in}{0.829599in}%
\pgfsys@useobject{currentmarker}{}%
\end{pgfscope}%
\begin{pgfscope}%
\pgfsys@transformshift{4.477962in}{0.828004in}%
\pgfsys@useobject{currentmarker}{}%
\end{pgfscope}%
\begin{pgfscope}%
\pgfsys@transformshift{4.475850in}{0.839171in}%
\pgfsys@useobject{currentmarker}{}%
\end{pgfscope}%
\begin{pgfscope}%
\pgfsys@transformshift{4.453786in}{0.876889in}%
\pgfsys@useobject{currentmarker}{}%
\end{pgfscope}%
\begin{pgfscope}%
\pgfsys@transformshift{4.433833in}{0.864119in}%
\pgfsys@useobject{currentmarker}{}%
\end{pgfscope}%
\begin{pgfscope}%
\pgfsys@transformshift{4.415290in}{0.810821in}%
\pgfsys@useobject{currentmarker}{}%
\end{pgfscope}%
\begin{pgfscope}%
\pgfsys@transformshift{4.395337in}{0.705533in}%
\pgfsys@useobject{currentmarker}{}%
\end{pgfscope}%
\begin{pgfscope}%
\pgfsys@transformshift{4.379141in}{0.597410in}%
\pgfsys@useobject{currentmarker}{}%
\end{pgfscope}%
\begin{pgfscope}%
\pgfsys@transformshift{4.359190in}{0.498584in}%
\pgfsys@useobject{currentmarker}{}%
\end{pgfscope}%
\begin{pgfscope}%
\pgfsys@transformshift{4.341114in}{0.548947in}%
\pgfsys@useobject{currentmarker}{}%
\end{pgfscope}%
\begin{pgfscope}%
\pgfsys@transformshift{4.321867in}{0.658791in}%
\pgfsys@useobject{currentmarker}{}%
\end{pgfscope}%
\begin{pgfscope}%
\pgfsys@transformshift{4.298864in}{0.789158in}%
\pgfsys@useobject{currentmarker}{}%
\end{pgfscope}%
\begin{pgfscope}%
\pgfsys@transformshift{4.281025in}{0.853092in}%
\pgfsys@useobject{currentmarker}{}%
\end{pgfscope}%
\begin{pgfscope}%
\pgfsys@transformshift{4.263889in}{0.871920in}%
\pgfsys@useobject{currentmarker}{}%
\end{pgfscope}%
\begin{pgfscope}%
\pgfsys@transformshift{4.242294in}{0.850254in}%
\pgfsys@useobject{currentmarker}{}%
\end{pgfscope}%
\begin{pgfscope}%
\pgfsys@transformshift{4.225394in}{0.785317in}%
\pgfsys@useobject{currentmarker}{}%
\end{pgfscope}%
\begin{pgfscope}%
\pgfsys@transformshift{4.204738in}{0.666979in}%
\pgfsys@useobject{currentmarker}{}%
\end{pgfscope}%
\begin{pgfscope}%
\pgfsys@transformshift{4.188071in}{0.558398in}%
\pgfsys@useobject{currentmarker}{}%
\end{pgfscope}%
\begin{pgfscope}%
\pgfsys@transformshift{4.166946in}{0.484555in}%
\pgfsys@useobject{currentmarker}{}%
\end{pgfscope}%
\begin{pgfscope}%
\pgfsys@transformshift{4.146995in}{0.573633in}%
\pgfsys@useobject{currentmarker}{}%
\end{pgfscope}%
\begin{pgfscope}%
\pgfsys@transformshift{4.129154in}{0.687715in}%
\pgfsys@useobject{currentmarker}{}%
\end{pgfscope}%
\begin{pgfscope}%
\pgfsys@transformshift{4.107794in}{0.807526in}%
\pgfsys@useobject{currentmarker}{}%
\end{pgfscope}%
\begin{pgfscope}%
\pgfsys@transformshift{4.090189in}{0.857707in}%
\pgfsys@useobject{currentmarker}{}%
\end{pgfscope}%
\begin{pgfscope}%
\pgfsys@transformshift{4.070707in}{0.864576in}%
\pgfsys@useobject{currentmarker}{}%
\end{pgfscope}%
\begin{pgfscope}%
\pgfsys@transformshift{4.052634in}{0.830583in}%
\pgfsys@useobject{currentmarker}{}%
\end{pgfscope}%
\begin{pgfscope}%
\pgfsys@transformshift{4.031507in}{0.733449in}%
\pgfsys@useobject{currentmarker}{}%
\end{pgfscope}%
\begin{pgfscope}%
\pgfsys@transformshift{4.014607in}{0.627838in}%
\pgfsys@useobject{currentmarker}{}%
\end{pgfscope}%
\begin{pgfscope}%
\pgfsys@transformshift{3.994889in}{0.511521in}%
\pgfsys@useobject{currentmarker}{}%
\end{pgfscope}%
\begin{pgfscope}%
\pgfsys@transformshift{3.976581in}{0.478823in}%
\pgfsys@useobject{currentmarker}{}%
\end{pgfscope}%
\begin{pgfscope}%
\pgfsys@transformshift{3.953342in}{0.587364in}%
\pgfsys@useobject{currentmarker}{}%
\end{pgfscope}%
\begin{pgfscope}%
\pgfsys@transformshift{3.935972in}{0.704144in}%
\pgfsys@useobject{currentmarker}{}%
\end{pgfscope}%
\begin{pgfscope}%
\pgfsys@transformshift{3.917898in}{0.799627in}%
\pgfsys@useobject{currentmarker}{}%
\end{pgfscope}%
\begin{pgfscope}%
\pgfsys@transformshift{3.897476in}{0.856326in}%
\pgfsys@useobject{currentmarker}{}%
\end{pgfscope}%
\begin{pgfscope}%
\pgfsys@transformshift{3.877760in}{0.857229in}%
\pgfsys@useobject{currentmarker}{}%
\end{pgfscope}%
\begin{pgfscope}%
\pgfsys@transformshift{3.859920in}{0.815159in}%
\pgfsys@useobject{currentmarker}{}%
\end{pgfscope}%
\begin{pgfscope}%
\pgfsys@transformshift{3.839030in}{0.713273in}%
\pgfsys@useobject{currentmarker}{}%
\end{pgfscope}%
\begin{pgfscope}%
\pgfsys@transformshift{3.821425in}{0.625810in}%
\pgfsys@useobject{currentmarker}{}%
\end{pgfscope}%
\begin{pgfscope}%
\pgfsys@transformshift{3.802412in}{0.519865in}%
\pgfsys@useobject{currentmarker}{}%
\end{pgfscope}%
\begin{pgfscope}%
\pgfsys@transformshift{3.783867in}{0.468540in}%
\pgfsys@useobject{currentmarker}{}%
\end{pgfscope}%
\begin{pgfscope}%
\pgfsys@transformshift{3.762742in}{0.552944in}%
\pgfsys@useobject{currentmarker}{}%
\end{pgfscope}%
\begin{pgfscope}%
\pgfsys@transformshift{3.742321in}{0.678473in}%
\pgfsys@useobject{currentmarker}{}%
\end{pgfscope}%
\begin{pgfscope}%
\pgfsys@transformshift{3.725185in}{0.778492in}%
\pgfsys@useobject{currentmarker}{}%
\end{pgfscope}%
\begin{pgfscope}%
\pgfsys@transformshift{3.707111in}{0.841945in}%
\pgfsys@useobject{currentmarker}{}%
\end{pgfscope}%
\begin{pgfscope}%
\pgfsys@transformshift{3.685986in}{0.858343in}%
\pgfsys@useobject{currentmarker}{}%
\end{pgfscope}%
\begin{pgfscope}%
\pgfsys@transformshift{3.665799in}{0.843387in}%
\pgfsys@useobject{currentmarker}{}%
\end{pgfscope}%
\begin{pgfscope}%
\pgfsys@transformshift{3.648897in}{0.788587in}%
\pgfsys@useobject{currentmarker}{}%
\end{pgfscope}%
\begin{pgfscope}%
\pgfsys@transformshift{3.628478in}{0.812497in}%
\pgfsys@useobject{currentmarker}{}%
\end{pgfscope}%
\begin{pgfscope}%
\pgfsys@transformshift{3.609933in}{0.731802in}%
\pgfsys@useobject{currentmarker}{}%
\end{pgfscope}%
\begin{pgfscope}%
\pgfsys@transformshift{3.589043in}{0.607193in}%
\pgfsys@useobject{currentmarker}{}%
\end{pgfscope}%
\begin{pgfscope}%
\pgfsys@transformshift{3.571203in}{0.511847in}%
\pgfsys@useobject{currentmarker}{}%
\end{pgfscope}%
\begin{pgfscope}%
\pgfsys@transformshift{3.553128in}{0.460752in}%
\pgfsys@useobject{currentmarker}{}%
\end{pgfscope}%
\begin{pgfscope}%
\pgfsys@transformshift{3.531534in}{0.529299in}%
\pgfsys@useobject{currentmarker}{}%
\end{pgfscope}%
\begin{pgfscope}%
\pgfsys@transformshift{3.514632in}{0.624706in}%
\pgfsys@useobject{currentmarker}{}%
\end{pgfscope}%
\begin{pgfscope}%
\pgfsys@transformshift{3.494211in}{0.742789in}%
\pgfsys@useobject{currentmarker}{}%
\end{pgfscope}%
\begin{pgfscope}%
\pgfsys@transformshift{3.476137in}{0.822223in}%
\pgfsys@useobject{currentmarker}{}%
\end{pgfscope}%
\begin{pgfscope}%
\pgfsys@transformshift{3.455481in}{0.854377in}%
\pgfsys@useobject{currentmarker}{}%
\end{pgfscope}%
\begin{pgfscope}%
\pgfsys@transformshift{3.435999in}{0.849889in}%
\pgfsys@useobject{currentmarker}{}%
\end{pgfscope}%
\begin{pgfscope}%
\pgfsys@transformshift{3.418394in}{0.817504in}%
\pgfsys@useobject{currentmarker}{}%
\end{pgfscope}%
\begin{pgfscope}%
\pgfsys@transformshift{3.397738in}{0.745585in}%
\pgfsys@useobject{currentmarker}{}%
\end{pgfscope}%
\begin{pgfscope}%
\pgfsys@transformshift{3.376613in}{0.627106in}%
\pgfsys@useobject{currentmarker}{}%
\end{pgfscope}%
\begin{pgfscope}%
\pgfsys@transformshift{3.358068in}{0.528298in}%
\pgfsys@useobject{currentmarker}{}%
\end{pgfscope}%
\begin{pgfscope}%
\pgfsys@transformshift{3.339526in}{0.470703in}%
\pgfsys@useobject{currentmarker}{}%
\end{pgfscope}%
\begin{pgfscope}%
\pgfsys@transformshift{3.321450in}{0.471890in}%
\pgfsys@useobject{currentmarker}{}%
\end{pgfscope}%
\begin{pgfscope}%
\pgfsys@transformshift{3.300560in}{0.511348in}%
\pgfsys@useobject{currentmarker}{}%
\end{pgfscope}%
\begin{pgfscope}%
\pgfsys@transformshift{3.282252in}{0.611700in}%
\pgfsys@useobject{currentmarker}{}%
\end{pgfscope}%
\begin{pgfscope}%
\pgfsys@transformshift{3.264881in}{0.723462in}%
\pgfsys@useobject{currentmarker}{}%
\end{pgfscope}%
\begin{pgfscope}%
\pgfsys@transformshift{3.243520in}{0.818741in}%
\pgfsys@useobject{currentmarker}{}%
\end{pgfscope}%
\begin{pgfscope}%
\pgfsys@transformshift{3.225915in}{0.845946in}%
\pgfsys@useobject{currentmarker}{}%
\end{pgfscope}%
\begin{pgfscope}%
\pgfsys@transformshift{3.205730in}{0.851732in}%
\pgfsys@useobject{currentmarker}{}%
\end{pgfscope}%
\begin{pgfscope}%
\pgfsys@transformshift{3.186951in}{0.822169in}%
\pgfsys@useobject{currentmarker}{}%
\end{pgfscope}%
\begin{pgfscope}%
\pgfsys@transformshift{3.170286in}{0.763772in}%
\pgfsys@useobject{currentmarker}{}%
\end{pgfscope}%
\begin{pgfscope}%
\pgfsys@transformshift{3.148456in}{0.653424in}%
\pgfsys@useobject{currentmarker}{}%
\end{pgfscope}%
\begin{pgfscope}%
\pgfsys@transformshift{3.130851in}{0.558558in}%
\pgfsys@useobject{currentmarker}{}%
\end{pgfscope}%
\begin{pgfscope}%
\pgfsys@transformshift{3.110664in}{0.468038in}%
\pgfsys@useobject{currentmarker}{}%
\end{pgfscope}%
\begin{pgfscope}%
\pgfsys@transformshift{3.091885in}{0.488011in}%
\pgfsys@useobject{currentmarker}{}%
\end{pgfscope}%
\begin{pgfscope}%
\pgfsys@transformshift{3.071699in}{0.457509in}%
\pgfsys@useobject{currentmarker}{}%
\end{pgfscope}%
\begin{pgfscope}%
\pgfsys@transformshift{3.049400in}{0.527885in}%
\pgfsys@useobject{currentmarker}{}%
\end{pgfscope}%
\begin{pgfscope}%
\pgfsys@transformshift{3.035081in}{0.612332in}%
\pgfsys@useobject{currentmarker}{}%
\end{pgfscope}%
\begin{pgfscope}%
\pgfsys@transformshift{3.014425in}{0.737056in}%
\pgfsys@useobject{currentmarker}{}%
\end{pgfscope}%
\begin{pgfscope}%
\pgfsys@transformshift{2.993064in}{0.820847in}%
\pgfsys@useobject{currentmarker}{}%
\end{pgfscope}%
\begin{pgfscope}%
\pgfsys@transformshift{2.971939in}{0.852414in}%
\pgfsys@useobject{currentmarker}{}%
\end{pgfscope}%
\begin{pgfscope}%
\pgfsys@transformshift{2.957151in}{0.846747in}%
\pgfsys@useobject{currentmarker}{}%
\end{pgfscope}%
\begin{pgfscope}%
\pgfsys@transformshift{2.937198in}{0.805206in}%
\pgfsys@useobject{currentmarker}{}%
\end{pgfscope}%
\begin{pgfscope}%
\pgfsys@transformshift{2.917950in}{0.736600in}%
\pgfsys@useobject{currentmarker}{}%
\end{pgfscope}%
\begin{pgfscope}%
\pgfsys@transformshift{2.897060in}{0.618924in}%
\pgfsys@useobject{currentmarker}{}%
\end{pgfscope}%
\begin{pgfscope}%
\pgfsys@transformshift{2.878986in}{0.532073in}%
\pgfsys@useobject{currentmarker}{}%
\end{pgfscope}%
\begin{pgfscope}%
\pgfsys@transformshift{2.862321in}{0.467326in}%
\pgfsys@useobject{currentmarker}{}%
\end{pgfscope}%
\begin{pgfscope}%
\pgfsys@transformshift{2.839317in}{0.477604in}%
\pgfsys@useobject{currentmarker}{}%
\end{pgfscope}%
\begin{pgfscope}%
\pgfsys@transformshift{2.823120in}{0.456332in}%
\pgfsys@useobject{currentmarker}{}%
\end{pgfscope}%
\begin{pgfscope}%
\pgfsys@transformshift{2.800587in}{0.540800in}%
\pgfsys@useobject{currentmarker}{}%
\end{pgfscope}%
\begin{pgfscope}%
\pgfsys@transformshift{2.782748in}{0.634378in}%
\pgfsys@useobject{currentmarker}{}%
\end{pgfscope}%
\begin{pgfscope}%
\pgfsys@transformshift{2.763735in}{0.729769in}%
\pgfsys@useobject{currentmarker}{}%
\end{pgfscope}%
\begin{pgfscope}%
\pgfsys@transformshift{2.746833in}{0.812494in}%
\pgfsys@useobject{currentmarker}{}%
\end{pgfscope}%
\begin{pgfscope}%
\pgfsys@transformshift{2.724300in}{0.850690in}%
\pgfsys@useobject{currentmarker}{}%
\end{pgfscope}%
\begin{pgfscope}%
\pgfsys@transformshift{2.705521in}{0.844418in}%
\pgfsys@useobject{currentmarker}{}%
\end{pgfscope}%
\begin{pgfscope}%
\pgfsys@transformshift{2.687213in}{0.804380in}%
\pgfsys@useobject{currentmarker}{}%
\end{pgfscope}%
\begin{pgfscope}%
\pgfsys@transformshift{2.668434in}{0.720749in}%
\pgfsys@useobject{currentmarker}{}%
\end{pgfscope}%
\begin{pgfscope}%
\pgfsys@transformshift{2.645900in}{0.599331in}%
\pgfsys@useobject{currentmarker}{}%
\end{pgfscope}%
\begin{pgfscope}%
\pgfsys@transformshift{2.630876in}{0.527457in}%
\pgfsys@useobject{currentmarker}{}%
\end{pgfscope}%
\begin{pgfscope}%
\pgfsys@transformshift{2.613273in}{0.459767in}%
\pgfsys@useobject{currentmarker}{}%
\end{pgfscope}%
\begin{pgfscope}%
\pgfsys@transformshift{2.592615in}{0.470910in}%
\pgfsys@useobject{currentmarker}{}%
\end{pgfscope}%
\begin{pgfscope}%
\pgfsys@transformshift{2.572195in}{0.563762in}%
\pgfsys@useobject{currentmarker}{}%
\end{pgfscope}%
\begin{pgfscope}%
\pgfsys@transformshift{2.552948in}{0.659592in}%
\pgfsys@useobject{currentmarker}{}%
\end{pgfscope}%
\begin{pgfscope}%
\pgfsys@transformshift{2.533698in}{0.768143in}%
\pgfsys@useobject{currentmarker}{}%
\end{pgfscope}%
\begin{pgfscope}%
\pgfsys@transformshift{2.515624in}{0.831692in}%
\pgfsys@useobject{currentmarker}{}%
\end{pgfscope}%
\begin{pgfscope}%
\pgfsys@transformshift{2.493325in}{0.851533in}%
\pgfsys@useobject{currentmarker}{}%
\end{pgfscope}%
\begin{pgfscope}%
\pgfsys@transformshift{2.477834in}{0.848076in}%
\pgfsys@useobject{currentmarker}{}%
\end{pgfscope}%
\begin{pgfscope}%
\pgfsys@transformshift{2.459524in}{0.821338in}%
\pgfsys@useobject{currentmarker}{}%
\end{pgfscope}%
\begin{pgfscope}%
\pgfsys@transformshift{2.437225in}{0.744062in}%
\pgfsys@useobject{currentmarker}{}%
\end{pgfscope}%
\begin{pgfscope}%
\pgfsys@transformshift{2.418446in}{0.695982in}%
\pgfsys@useobject{currentmarker}{}%
\end{pgfscope}%
\begin{pgfscope}%
\pgfsys@transformshift{2.399904in}{0.585482in}%
\pgfsys@useobject{currentmarker}{}%
\end{pgfscope}%
\begin{pgfscope}%
\pgfsys@transformshift{2.378777in}{0.501396in}%
\pgfsys@useobject{currentmarker}{}%
\end{pgfscope}%
\begin{pgfscope}%
\pgfsys@transformshift{2.360000in}{0.452964in}%
\pgfsys@useobject{currentmarker}{}%
\end{pgfscope}%
\begin{pgfscope}%
\pgfsys@transformshift{2.340282in}{0.498357in}%
\pgfsys@useobject{currentmarker}{}%
\end{pgfscope}%
\begin{pgfscope}%
\pgfsys@transformshift{2.318452in}{0.592492in}%
\pgfsys@useobject{currentmarker}{}%
\end{pgfscope}%
\begin{pgfscope}%
\pgfsys@transformshift{2.303429in}{0.678257in}%
\pgfsys@useobject{currentmarker}{}%
\end{pgfscope}%
\begin{pgfscope}%
\pgfsys@transformshift{2.283713in}{0.776934in}%
\pgfsys@useobject{currentmarker}{}%
\end{pgfscope}%
\begin{pgfscope}%
\pgfsys@transformshift{2.265873in}{0.830664in}%
\pgfsys@useobject{currentmarker}{}%
\end{pgfscope}%
\begin{pgfscope}%
\pgfsys@transformshift{2.244043in}{0.852024in}%
\pgfsys@useobject{currentmarker}{}%
\end{pgfscope}%
\begin{pgfscope}%
\pgfsys@transformshift{2.228081in}{0.851691in}%
\pgfsys@useobject{currentmarker}{}%
\end{pgfscope}%
\begin{pgfscope}%
\pgfsys@transformshift{2.203905in}{0.818115in}%
\pgfsys@useobject{currentmarker}{}%
\end{pgfscope}%
\begin{pgfscope}%
\pgfsys@transformshift{2.187943in}{0.766408in}%
\pgfsys@useobject{currentmarker}{}%
\end{pgfscope}%
\begin{pgfscope}%
\pgfsys@transformshift{2.167053in}{0.702400in}%
\pgfsys@useobject{currentmarker}{}%
\end{pgfscope}%
\begin{pgfscope}%
\pgfsys@transformshift{2.148508in}{0.607984in}%
\pgfsys@useobject{currentmarker}{}%
\end{pgfscope}%
\begin{pgfscope}%
\pgfsys@transformshift{2.132312in}{0.512771in}%
\pgfsys@useobject{currentmarker}{}%
\end{pgfscope}%
\begin{pgfscope}%
\pgfsys@transformshift{2.111421in}{0.457214in}%
\pgfsys@useobject{currentmarker}{}%
\end{pgfscope}%
\begin{pgfscope}%
\pgfsys@transformshift{2.092408in}{0.488227in}%
\pgfsys@useobject{currentmarker}{}%
\end{pgfscope}%
\begin{pgfscope}%
\pgfsys@transformshift{2.072221in}{0.573968in}%
\pgfsys@useobject{currentmarker}{}%
\end{pgfscope}%
\begin{pgfscope}%
\pgfsys@transformshift{2.051565in}{0.682071in}%
\pgfsys@useobject{currentmarker}{}%
\end{pgfscope}%
\begin{pgfscope}%
\pgfsys@transformshift{2.032551in}{0.762576in}%
\pgfsys@useobject{currentmarker}{}%
\end{pgfscope}%
\begin{pgfscope}%
\pgfsys@transformshift{2.014243in}{0.828366in}%
\pgfsys@useobject{currentmarker}{}%
\end{pgfscope}%
\begin{pgfscope}%
\pgfsys@transformshift{1.995230in}{0.852204in}%
\pgfsys@useobject{currentmarker}{}%
\end{pgfscope}%
\begin{pgfscope}%
\pgfsys@transformshift{1.976922in}{0.854209in}%
\pgfsys@useobject{currentmarker}{}%
\end{pgfscope}%
\begin{pgfscope}%
\pgfsys@transformshift{1.956266in}{0.831227in}%
\pgfsys@useobject{currentmarker}{}%
\end{pgfscope}%
\begin{pgfscope}%
\pgfsys@transformshift{1.937016in}{0.809576in}%
\pgfsys@useobject{currentmarker}{}%
\end{pgfscope}%
\begin{pgfscope}%
\pgfsys@transformshift{1.916831in}{0.729056in}%
\pgfsys@useobject{currentmarker}{}%
\end{pgfscope}%
\begin{pgfscope}%
\pgfsys@transformshift{1.898757in}{0.652149in}%
\pgfsys@useobject{currentmarker}{}%
\end{pgfscope}%
\begin{pgfscope}%
\pgfsys@transformshift{1.881152in}{0.563404in}%
\pgfsys@useobject{currentmarker}{}%
\end{pgfscope}%
\begin{pgfscope}%
\pgfsys@transformshift{1.858148in}{0.480596in}%
\pgfsys@useobject{currentmarker}{}%
\end{pgfscope}%
\begin{pgfscope}%
\pgfsys@transformshift{1.843595in}{0.463210in}%
\pgfsys@useobject{currentmarker}{}%
\end{pgfscope}%
\begin{pgfscope}%
\pgfsys@transformshift{1.824581in}{0.513694in}%
\pgfsys@useobject{currentmarker}{}%
\end{pgfscope}%
\begin{pgfscope}%
\pgfsys@transformshift{1.804160in}{0.596117in}%
\pgfsys@useobject{currentmarker}{}%
\end{pgfscope}%
\begin{pgfscope}%
\pgfsys@transformshift{1.785852in}{0.676691in}%
\pgfsys@useobject{currentmarker}{}%
\end{pgfscope}%
\begin{pgfscope}%
\pgfsys@transformshift{1.766133in}{0.781212in}%
\pgfsys@useobject{currentmarker}{}%
\end{pgfscope}%
\begin{pgfscope}%
\pgfsys@transformshift{1.747825in}{0.824018in}%
\pgfsys@useobject{currentmarker}{}%
\end{pgfscope}%
\begin{pgfscope}%
\pgfsys@transformshift{1.725526in}{0.855764in}%
\pgfsys@useobject{currentmarker}{}%
\end{pgfscope}%
\begin{pgfscope}%
\pgfsys@transformshift{1.710504in}{0.857577in}%
\pgfsys@useobject{currentmarker}{}%
\end{pgfscope}%
\begin{pgfscope}%
\pgfsys@transformshift{1.688205in}{0.850582in}%
\pgfsys@useobject{currentmarker}{}%
\end{pgfscope}%
\begin{pgfscope}%
\pgfsys@transformshift{1.671772in}{0.816396in}%
\pgfsys@useobject{currentmarker}{}%
\end{pgfscope}%
\begin{pgfscope}%
\pgfsys@transformshift{1.648770in}{0.768338in}%
\pgfsys@useobject{currentmarker}{}%
\end{pgfscope}%
\begin{pgfscope}%
\pgfsys@transformshift{1.625766in}{0.686555in}%
\pgfsys@useobject{currentmarker}{}%
\end{pgfscope}%
\begin{pgfscope}%
\pgfsys@transformshift{1.610978in}{0.854412in}%
\pgfsys@useobject{currentmarker}{}%
\end{pgfscope}%
\begin{pgfscope}%
\pgfsys@transformshift{1.590790in}{0.816683in}%
\pgfsys@useobject{currentmarker}{}%
\end{pgfscope}%
\begin{pgfscope}%
\pgfsys@transformshift{1.572717in}{0.831666in}%
\pgfsys@useobject{currentmarker}{}%
\end{pgfscope}%
\begin{pgfscope}%
\pgfsys@transformshift{1.551592in}{0.758152in}%
\pgfsys@useobject{currentmarker}{}%
\end{pgfscope}%
\begin{pgfscope}%
\pgfsys@transformshift{1.537273in}{0.660484in}%
\pgfsys@useobject{currentmarker}{}%
\end{pgfscope}%
\begin{pgfscope}%
\pgfsys@transformshift{1.514739in}{0.567072in}%
\pgfsys@useobject{currentmarker}{}%
\end{pgfscope}%
\begin{pgfscope}%
\pgfsys@transformshift{1.493144in}{0.482240in}%
\pgfsys@useobject{currentmarker}{}%
\end{pgfscope}%
\begin{pgfscope}%
\pgfsys@transformshift{1.479061in}{0.475926in}%
\pgfsys@useobject{currentmarker}{}%
\end{pgfscope}%
\begin{pgfscope}%
\pgfsys@transformshift{1.459577in}{0.546906in}%
\pgfsys@useobject{currentmarker}{}%
\end{pgfscope}%
\begin{pgfscope}%
\pgfsys@transformshift{1.439157in}{0.654568in}%
\pgfsys@useobject{currentmarker}{}%
\end{pgfscope}%
\begin{pgfscope}%
\pgfsys@transformshift{1.419439in}{0.753137in}%
\pgfsys@useobject{currentmarker}{}%
\end{pgfscope}%
\begin{pgfscope}%
\pgfsys@transformshift{1.401599in}{0.814718in}%
\pgfsys@useobject{currentmarker}{}%
\end{pgfscope}%
\begin{pgfscope}%
\pgfsys@transformshift{1.381648in}{0.854268in}%
\pgfsys@useobject{currentmarker}{}%
\end{pgfscope}%
\begin{pgfscope}%
\pgfsys@transformshift{1.359582in}{0.862213in}%
\pgfsys@useobject{currentmarker}{}%
\end{pgfscope}%
\begin{pgfscope}%
\pgfsys@transformshift{1.341508in}{0.850211in}%
\pgfsys@useobject{currentmarker}{}%
\end{pgfscope}%
\begin{pgfscope}%
\pgfsys@transformshift{1.320149in}{0.803967in}%
\pgfsys@useobject{currentmarker}{}%
\end{pgfscope}%
\begin{pgfscope}%
\pgfsys@transformshift{1.306299in}{0.751096in}%
\pgfsys@useobject{currentmarker}{}%
\end{pgfscope}%
\begin{pgfscope}%
\pgfsys@transformshift{1.283062in}{0.650931in}%
\pgfsys@useobject{currentmarker}{}%
\end{pgfscope}%
\begin{pgfscope}%
\pgfsys@transformshift{1.264049in}{0.567510in}%
\pgfsys@useobject{currentmarker}{}%
\end{pgfscope}%
\begin{pgfscope}%
\pgfsys@transformshift{1.245504in}{0.505638in}%
\pgfsys@useobject{currentmarker}{}%
\end{pgfscope}%
\begin{pgfscope}%
\pgfsys@transformshift{1.227196in}{0.477557in}%
\pgfsys@useobject{currentmarker}{}%
\end{pgfscope}%
\begin{pgfscope}%
\pgfsys@transformshift{1.208652in}{0.528991in}%
\pgfsys@useobject{currentmarker}{}%
\end{pgfscope}%
\begin{pgfscope}%
\pgfsys@transformshift{1.186822in}{0.639058in}%
\pgfsys@useobject{currentmarker}{}%
\end{pgfscope}%
\begin{pgfscope}%
\pgfsys@transformshift{1.171330in}{0.717348in}%
\pgfsys@useobject{currentmarker}{}%
\end{pgfscope}%
\begin{pgfscope}%
\pgfsys@transformshift{1.150203in}{0.807951in}%
\pgfsys@useobject{currentmarker}{}%
\end{pgfscope}%
\begin{pgfscope}%
\pgfsys@transformshift{1.130721in}{0.850452in}%
\pgfsys@useobject{currentmarker}{}%
\end{pgfscope}%
\begin{pgfscope}%
\pgfsys@transformshift{1.112413in}{0.866157in}%
\pgfsys@useobject{currentmarker}{}%
\end{pgfscope}%
\begin{pgfscope}%
\pgfsys@transformshift{1.087766in}{0.860171in}%
\pgfsys@useobject{currentmarker}{}%
\end{pgfscope}%
\begin{pgfscope}%
\pgfsys@transformshift{1.072039in}{0.838800in}%
\pgfsys@useobject{currentmarker}{}%
\end{pgfscope}%
\begin{pgfscope}%
\pgfsys@transformshift{1.053965in}{0.788501in}%
\pgfsys@useobject{currentmarker}{}%
\end{pgfscope}%
\begin{pgfscope}%
\pgfsys@transformshift{1.034952in}{0.721953in}%
\pgfsys@useobject{currentmarker}{}%
\end{pgfscope}%
\begin{pgfscope}%
\pgfsys@transformshift{1.017581in}{0.662997in}%
\pgfsys@useobject{currentmarker}{}%
\end{pgfscope}%
\begin{pgfscope}%
\pgfsys@transformshift{0.995282in}{0.580000in}%
\pgfsys@useobject{currentmarker}{}%
\end{pgfscope}%
\begin{pgfscope}%
\pgfsys@transformshift{0.975566in}{0.516661in}%
\pgfsys@useobject{currentmarker}{}%
\end{pgfscope}%
\begin{pgfscope}%
\pgfsys@transformshift{0.957492in}{0.488679in}%
\pgfsys@useobject{currentmarker}{}%
\end{pgfscope}%
\begin{pgfscope}%
\pgfsys@transformshift{0.939182in}{0.549476in}%
\pgfsys@useobject{currentmarker}{}%
\end{pgfscope}%
\begin{pgfscope}%
\pgfsys@transformshift{0.920169in}{0.643111in}%
\pgfsys@useobject{currentmarker}{}%
\end{pgfscope}%
\begin{pgfscope}%
\pgfsys@transformshift{0.899513in}{0.728222in}%
\pgfsys@useobject{currentmarker}{}%
\end{pgfscope}%
\begin{pgfscope}%
\pgfsys@transformshift{0.883787in}{0.777818in}%
\pgfsys@useobject{currentmarker}{}%
\end{pgfscope}%
\begin{pgfscope}%
\pgfsys@transformshift{0.862192in}{0.785867in}%
\pgfsys@useobject{currentmarker}{}%
\end{pgfscope}%
\begin{pgfscope}%
\pgfsys@transformshift{0.843647in}{0.844117in}%
\pgfsys@useobject{currentmarker}{}%
\end{pgfscope}%
\begin{pgfscope}%
\pgfsys@transformshift{0.825105in}{0.868750in}%
\pgfsys@useobject{currentmarker}{}%
\end{pgfscope}%
\begin{pgfscope}%
\pgfsys@transformshift{0.804212in}{0.871468in}%
\pgfsys@useobject{currentmarker}{}%
\end{pgfscope}%
\begin{pgfscope}%
\pgfsys@transformshift{0.785201in}{0.851034in}%
\pgfsys@useobject{currentmarker}{}%
\end{pgfscope}%
\begin{pgfscope}%
\pgfsys@transformshift{0.764308in}{0.800049in}%
\pgfsys@useobject{currentmarker}{}%
\end{pgfscope}%
\begin{pgfscope}%
\pgfsys@transformshift{0.745295in}{0.819772in}%
\pgfsys@useobject{currentmarker}{}%
\end{pgfscope}%
\begin{pgfscope}%
\pgfsys@transformshift{0.726284in}{0.750577in}%
\pgfsys@useobject{currentmarker}{}%
\end{pgfscope}%
\begin{pgfscope}%
\pgfsys@transformshift{0.708445in}{0.669000in}%
\pgfsys@useobject{currentmarker}{}%
\end{pgfscope}%
\begin{pgfscope}%
\pgfsys@transformshift{0.688960in}{0.584224in}%
\pgfsys@useobject{currentmarker}{}%
\end{pgfscope}%
\begin{pgfscope}%
\pgfsys@transformshift{0.669010in}{0.529981in}%
\pgfsys@useobject{currentmarker}{}%
\end{pgfscope}%
\begin{pgfscope}%
\pgfsys@transformshift{0.650934in}{0.502949in}%
\pgfsys@useobject{currentmarker}{}%
\end{pgfscope}%
\begin{pgfscope}%
\pgfsys@transformshift{0.647883in}{0.501889in}%
\pgfsys@useobject{currentmarker}{}%
\end{pgfscope}%
\begin{pgfscope}%
\pgfsys@transformshift{0.661264in}{0.519266in}%
\pgfsys@useobject{currentmarker}{}%
\end{pgfscope}%
\begin{pgfscope}%
\pgfsys@transformshift{0.675347in}{0.598071in}%
\pgfsys@useobject{currentmarker}{}%
\end{pgfscope}%
\begin{pgfscope}%
\pgfsys@transformshift{0.696003in}{0.709022in}%
\pgfsys@useobject{currentmarker}{}%
\end{pgfscope}%
\begin{pgfscope}%
\pgfsys@transformshift{0.714782in}{0.810390in}%
\pgfsys@useobject{currentmarker}{}%
\end{pgfscope}%
\begin{pgfscope}%
\pgfsys@transformshift{0.732855in}{0.861022in}%
\pgfsys@useobject{currentmarker}{}%
\end{pgfscope}%
\begin{pgfscope}%
\pgfsys@transformshift{0.754451in}{0.871244in}%
\pgfsys@useobject{currentmarker}{}%
\end{pgfscope}%
\begin{pgfscope}%
\pgfsys@transformshift{0.772056in}{0.836837in}%
\pgfsys@useobject{currentmarker}{}%
\end{pgfscope}%
\begin{pgfscope}%
\pgfsys@transformshift{0.790364in}{0.753414in}%
\pgfsys@useobject{currentmarker}{}%
\end{pgfscope}%
\begin{pgfscope}%
\pgfsys@transformshift{0.811960in}{0.609668in}%
\pgfsys@useobject{currentmarker}{}%
\end{pgfscope}%
\begin{pgfscope}%
\pgfsys@transformshift{0.831207in}{0.500079in}%
\pgfsys@useobject{currentmarker}{}%
\end{pgfscope}%
\begin{pgfscope}%
\pgfsys@transformshift{0.849281in}{0.514569in}%
\pgfsys@useobject{currentmarker}{}%
\end{pgfscope}%
\begin{pgfscope}%
\pgfsys@transformshift{0.869468in}{0.630101in}%
\pgfsys@useobject{currentmarker}{}%
\end{pgfscope}%
\begin{pgfscope}%
\pgfsys@transformshift{0.887542in}{0.737758in}%
\pgfsys@useobject{currentmarker}{}%
\end{pgfscope}%
\begin{pgfscope}%
\pgfsys@transformshift{0.904678in}{0.822314in}%
\pgfsys@useobject{currentmarker}{}%
\end{pgfscope}%
\begin{pgfscope}%
\pgfsys@transformshift{0.926037in}{0.865900in}%
\pgfsys@useobject{currentmarker}{}%
\end{pgfscope}%
\begin{pgfscope}%
\pgfsys@transformshift{0.944111in}{0.858611in}%
\pgfsys@useobject{currentmarker}{}%
\end{pgfscope}%
\begin{pgfscope}%
\pgfsys@transformshift{0.963595in}{0.808082in}%
\pgfsys@useobject{currentmarker}{}%
\end{pgfscope}%
\begin{pgfscope}%
\pgfsys@transformshift{0.982843in}{0.698375in}%
\pgfsys@useobject{currentmarker}{}%
\end{pgfscope}%
\begin{pgfscope}%
\pgfsys@transformshift{1.001621in}{0.574652in}%
\pgfsys@useobject{currentmarker}{}%
\end{pgfscope}%
\begin{pgfscope}%
\pgfsys@transformshift{1.022981in}{0.478916in}%
\pgfsys@useobject{currentmarker}{}%
\end{pgfscope}%
\begin{pgfscope}%
\pgfsys@transformshift{1.041525in}{0.535114in}%
\pgfsys@useobject{currentmarker}{}%
\end{pgfscope}%
\begin{pgfscope}%
\pgfsys@transformshift{1.059599in}{0.648623in}%
\pgfsys@useobject{currentmarker}{}%
\end{pgfscope}%
\begin{pgfscope}%
\pgfsys@transformshift{1.079315in}{0.755924in}%
\pgfsys@useobject{currentmarker}{}%
\end{pgfscope}%
\begin{pgfscope}%
\pgfsys@transformshift{1.098329in}{0.834777in}%
\pgfsys@useobject{currentmarker}{}%
\end{pgfscope}%
\begin{pgfscope}%
\pgfsys@transformshift{1.119924in}{0.863833in}%
\pgfsys@useobject{currentmarker}{}%
\end{pgfscope}%
\begin{pgfscope}%
\pgfsys@transformshift{1.139407in}{0.842877in}%
\pgfsys@useobject{currentmarker}{}%
\end{pgfscope}%
\begin{pgfscope}%
\pgfsys@transformshift{1.156777in}{0.774757in}%
\pgfsys@useobject{currentmarker}{}%
\end{pgfscope}%
\begin{pgfscope}%
\pgfsys@transformshift{1.175790in}{0.663539in}%
\pgfsys@useobject{currentmarker}{}%
\end{pgfscope}%
\begin{pgfscope}%
\pgfsys@transformshift{1.196212in}{0.549870in}%
\pgfsys@useobject{currentmarker}{}%
\end{pgfscope}%
\begin{pgfscope}%
\pgfsys@transformshift{1.212643in}{0.469936in}%
\pgfsys@useobject{currentmarker}{}%
\end{pgfscope}%
\begin{pgfscope}%
\pgfsys@transformshift{1.232594in}{0.535343in}%
\pgfsys@useobject{currentmarker}{}%
\end{pgfscope}%
\begin{pgfscope}%
\pgfsys@transformshift{1.251841in}{0.647507in}%
\pgfsys@useobject{currentmarker}{}%
\end{pgfscope}%
\begin{pgfscope}%
\pgfsys@transformshift{1.271325in}{0.761242in}%
\pgfsys@useobject{currentmarker}{}%
\end{pgfscope}%
\begin{pgfscope}%
\pgfsys@transformshift{1.291981in}{0.808377in}%
\pgfsys@useobject{currentmarker}{}%
\end{pgfscope}%
\begin{pgfscope}%
\pgfsys@transformshift{1.309586in}{0.846509in}%
\pgfsys@useobject{currentmarker}{}%
\end{pgfscope}%
\begin{pgfscope}%
\pgfsys@transformshift{1.328598in}{0.859315in}%
\pgfsys@useobject{currentmarker}{}%
\end{pgfscope}%
\begin{pgfscope}%
\pgfsys@transformshift{1.348082in}{0.828798in}%
\pgfsys@useobject{currentmarker}{}%
\end{pgfscope}%
\begin{pgfscope}%
\pgfsys@transformshift{1.367798in}{0.739661in}%
\pgfsys@useobject{currentmarker}{}%
\end{pgfscope}%
\begin{pgfscope}%
\pgfsys@transformshift{1.385872in}{0.633437in}%
\pgfsys@useobject{currentmarker}{}%
\end{pgfscope}%
\begin{pgfscope}%
\pgfsys@transformshift{1.404651in}{0.518508in}%
\pgfsys@useobject{currentmarker}{}%
\end{pgfscope}%
\begin{pgfscope}%
\pgfsys@transformshift{1.427420in}{0.464155in}%
\pgfsys@useobject{currentmarker}{}%
\end{pgfscope}%
\begin{pgfscope}%
\pgfsys@transformshift{1.446197in}{0.536422in}%
\pgfsys@useobject{currentmarker}{}%
\end{pgfscope}%
\begin{pgfscope}%
\pgfsys@transformshift{1.464037in}{0.631715in}%
\pgfsys@useobject{currentmarker}{}%
\end{pgfscope}%
\begin{pgfscope}%
\pgfsys@transformshift{1.484224in}{0.737753in}%
\pgfsys@useobject{currentmarker}{}%
\end{pgfscope}%
\begin{pgfscope}%
\pgfsys@transformshift{1.503472in}{0.818186in}%
\pgfsys@useobject{currentmarker}{}%
\end{pgfscope}%
\begin{pgfscope}%
\pgfsys@transformshift{1.522719in}{0.852763in}%
\pgfsys@useobject{currentmarker}{}%
\end{pgfscope}%
\begin{pgfscope}%
\pgfsys@transformshift{1.541498in}{0.851332in}%
\pgfsys@useobject{currentmarker}{}%
\end{pgfscope}%
\begin{pgfscope}%
\pgfsys@transformshift{1.558163in}{0.816069in}%
\pgfsys@useobject{currentmarker}{}%
\end{pgfscope}%
\begin{pgfscope}%
\pgfsys@transformshift{1.579759in}{0.747520in}%
\pgfsys@useobject{currentmarker}{}%
\end{pgfscope}%
\begin{pgfscope}%
\pgfsys@transformshift{1.598772in}{0.646493in}%
\pgfsys@useobject{currentmarker}{}%
\end{pgfscope}%
\begin{pgfscope}%
\pgfsys@transformshift{1.622245in}{0.571498in}%
\pgfsys@useobject{currentmarker}{}%
\end{pgfscope}%
\begin{pgfscope}%
\pgfsys@transformshift{1.638442in}{0.487776in}%
\pgfsys@useobject{currentmarker}{}%
\end{pgfscope}%
\begin{pgfscope}%
\pgfsys@transformshift{1.654638in}{0.459776in}%
\pgfsys@useobject{currentmarker}{}%
\end{pgfscope}%
\begin{pgfscope}%
\pgfsys@transformshift{1.677171in}{0.542877in}%
\pgfsys@useobject{currentmarker}{}%
\end{pgfscope}%
\begin{pgfscope}%
\pgfsys@transformshift{1.699471in}{0.651456in}%
\pgfsys@useobject{currentmarker}{}%
\end{pgfscope}%
\begin{pgfscope}%
\pgfsys@transformshift{1.713555in}{0.744119in}%
\pgfsys@useobject{currentmarker}{}%
\end{pgfscope}%
\begin{pgfscope}%
\pgfsys@transformshift{1.731863in}{0.819405in}%
\pgfsys@useobject{currentmarker}{}%
\end{pgfscope}%
\begin{pgfscope}%
\pgfsys@transformshift{1.751345in}{0.845797in}%
\pgfsys@useobject{currentmarker}{}%
\end{pgfscope}%
\begin{pgfscope}%
\pgfsys@transformshift{1.774818in}{0.850038in}%
\pgfsys@useobject{currentmarker}{}%
\end{pgfscope}%
\begin{pgfscope}%
\pgfsys@transformshift{1.787963in}{0.827266in}%
\pgfsys@useobject{currentmarker}{}%
\end{pgfscope}%
\begin{pgfscope}%
\pgfsys@transformshift{1.809325in}{0.737978in}%
\pgfsys@useobject{currentmarker}{}%
\end{pgfscope}%
\begin{pgfscope}%
\pgfsys@transformshift{1.828104in}{0.623221in}%
\pgfsys@useobject{currentmarker}{}%
\end{pgfscope}%
\begin{pgfscope}%
\pgfsys@transformshift{1.847586in}{0.531059in}%
\pgfsys@useobject{currentmarker}{}%
\end{pgfscope}%
\begin{pgfscope}%
\pgfsys@transformshift{1.867302in}{0.455068in}%
\pgfsys@useobject{currentmarker}{}%
\end{pgfscope}%
\begin{pgfscope}%
\pgfsys@transformshift{1.886315in}{0.490693in}%
\pgfsys@useobject{currentmarker}{}%
\end{pgfscope}%
\begin{pgfscope}%
\pgfsys@transformshift{1.905329in}{0.547059in}%
\pgfsys@useobject{currentmarker}{}%
\end{pgfscope}%
\begin{pgfscope}%
\pgfsys@transformshift{1.927393in}{0.513604in}%
\pgfsys@useobject{currentmarker}{}%
\end{pgfscope}%
\begin{pgfscope}%
\pgfsys@transformshift{1.944998in}{0.606519in}%
\pgfsys@useobject{currentmarker}{}%
\end{pgfscope}%
\begin{pgfscope}%
\pgfsys@transformshift{1.963777in}{0.720086in}%
\pgfsys@useobject{currentmarker}{}%
\end{pgfscope}%
\begin{pgfscope}%
\pgfsys@transformshift{1.984197in}{0.805843in}%
\pgfsys@useobject{currentmarker}{}%
\end{pgfscope}%
\begin{pgfscope}%
\pgfsys@transformshift{2.001567in}{0.845404in}%
\pgfsys@useobject{currentmarker}{}%
\end{pgfscope}%
\begin{pgfscope}%
\pgfsys@transformshift{2.020580in}{0.851407in}%
\pgfsys@useobject{currentmarker}{}%
\end{pgfscope}%
\begin{pgfscope}%
\pgfsys@transformshift{2.042645in}{0.809259in}%
\pgfsys@useobject{currentmarker}{}%
\end{pgfscope}%
\begin{pgfscope}%
\pgfsys@transformshift{2.057667in}{0.741742in}%
\pgfsys@useobject{currentmarker}{}%
\end{pgfscope}%
\begin{pgfscope}%
\pgfsys@transformshift{2.079263in}{0.608537in}%
\pgfsys@useobject{currentmarker}{}%
\end{pgfscope}%
\begin{pgfscope}%
\pgfsys@transformshift{2.097337in}{0.511355in}%
\pgfsys@useobject{currentmarker}{}%
\end{pgfscope}%
\begin{pgfscope}%
\pgfsys@transformshift{2.116584in}{0.508338in}%
\pgfsys@useobject{currentmarker}{}%
\end{pgfscope}%
\begin{pgfscope}%
\pgfsys@transformshift{2.137006in}{0.453541in}%
\pgfsys@useobject{currentmarker}{}%
\end{pgfscope}%
\begin{pgfscope}%
\pgfsys@transformshift{2.155550in}{0.498104in}%
\pgfsys@useobject{currentmarker}{}%
\end{pgfscope}%
\begin{pgfscope}%
\pgfsys@transformshift{2.175972in}{0.596510in}%
\pgfsys@useobject{currentmarker}{}%
\end{pgfscope}%
\begin{pgfscope}%
\pgfsys@transformshift{2.191463in}{0.678843in}%
\pgfsys@useobject{currentmarker}{}%
\end{pgfscope}%
\begin{pgfscope}%
\pgfsys@transformshift{2.212119in}{0.787930in}%
\pgfsys@useobject{currentmarker}{}%
\end{pgfscope}%
\begin{pgfscope}%
\pgfsys@transformshift{2.230898in}{0.838134in}%
\pgfsys@useobject{currentmarker}{}%
\end{pgfscope}%
\begin{pgfscope}%
\pgfsys@transformshift{2.249677in}{0.852251in}%
\pgfsys@useobject{currentmarker}{}%
\end{pgfscope}%
\begin{pgfscope}%
\pgfsys@transformshift{2.271976in}{0.832869in}%
\pgfsys@useobject{currentmarker}{}%
\end{pgfscope}%
\begin{pgfscope}%
\pgfsys@transformshift{2.290519in}{0.778874in}%
\pgfsys@useobject{currentmarker}{}%
\end{pgfscope}%
\begin{pgfscope}%
\pgfsys@transformshift{2.308123in}{0.676286in}%
\pgfsys@useobject{currentmarker}{}%
\end{pgfscope}%
\begin{pgfscope}%
\pgfsys@transformshift{2.327607in}{0.565048in}%
\pgfsys@useobject{currentmarker}{}%
\end{pgfscope}%
\begin{pgfscope}%
\pgfsys@transformshift{2.348967in}{0.471012in}%
\pgfsys@useobject{currentmarker}{}%
\end{pgfscope}%
\begin{pgfscope}%
\pgfsys@transformshift{2.369154in}{0.459052in}%
\pgfsys@useobject{currentmarker}{}%
\end{pgfscope}%
\begin{pgfscope}%
\pgfsys@transformshift{2.386759in}{0.530233in}%
\pgfsys@useobject{currentmarker}{}%
\end{pgfscope}%
\begin{pgfscope}%
\pgfsys@transformshift{2.402955in}{0.605125in}%
\pgfsys@useobject{currentmarker}{}%
\end{pgfscope}%
\begin{pgfscope}%
\pgfsys@transformshift{2.424315in}{0.723057in}%
\pgfsys@useobject{currentmarker}{}%
\end{pgfscope}%
\begin{pgfscope}%
\pgfsys@transformshift{2.442859in}{0.785704in}%
\pgfsys@useobject{currentmarker}{}%
\end{pgfscope}%
\begin{pgfscope}%
\pgfsys@transformshift{2.464689in}{0.835989in}%
\pgfsys@useobject{currentmarker}{}%
\end{pgfscope}%
\begin{pgfscope}%
\pgfsys@transformshift{2.482529in}{0.851683in}%
\pgfsys@useobject{currentmarker}{}%
\end{pgfscope}%
\begin{pgfscope}%
\pgfsys@transformshift{2.500133in}{0.838817in}%
\pgfsys@useobject{currentmarker}{}%
\end{pgfscope}%
\begin{pgfscope}%
\pgfsys@transformshift{2.519381in}{0.787277in}%
\pgfsys@useobject{currentmarker}{}%
\end{pgfscope}%
\begin{pgfscope}%
\pgfsys@transformshift{2.543792in}{0.648031in}%
\pgfsys@useobject{currentmarker}{}%
\end{pgfscope}%
\begin{pgfscope}%
\pgfsys@transformshift{2.559285in}{0.563648in}%
\pgfsys@useobject{currentmarker}{}%
\end{pgfscope}%
\begin{pgfscope}%
\pgfsys@transformshift{2.577827in}{0.482729in}%
\pgfsys@useobject{currentmarker}{}%
\end{pgfscope}%
\begin{pgfscope}%
\pgfsys@transformshift{2.599892in}{0.453520in}%
\pgfsys@useobject{currentmarker}{}%
\end{pgfscope}%
\begin{pgfscope}%
\pgfsys@transformshift{2.616325in}{0.487954in}%
\pgfsys@useobject{currentmarker}{}%
\end{pgfscope}%
\begin{pgfscope}%
\pgfsys@transformshift{2.638624in}{0.590926in}%
\pgfsys@useobject{currentmarker}{}%
\end{pgfscope}%
\begin{pgfscope}%
\pgfsys@transformshift{2.653177in}{0.663031in}%
\pgfsys@useobject{currentmarker}{}%
\end{pgfscope}%
\begin{pgfscope}%
\pgfsys@transformshift{2.673362in}{0.758215in}%
\pgfsys@useobject{currentmarker}{}%
\end{pgfscope}%
\begin{pgfscope}%
\pgfsys@transformshift{2.694253in}{0.824539in}%
\pgfsys@useobject{currentmarker}{}%
\end{pgfscope}%
\begin{pgfscope}%
\pgfsys@transformshift{2.712094in}{0.848996in}%
\pgfsys@useobject{currentmarker}{}%
\end{pgfscope}%
\begin{pgfscope}%
\pgfsys@transformshift{2.729463in}{0.852474in}%
\pgfsys@useobject{currentmarker}{}%
\end{pgfscope}%
\begin{pgfscope}%
\pgfsys@transformshift{2.750590in}{0.841849in}%
\pgfsys@useobject{currentmarker}{}%
\end{pgfscope}%
\begin{pgfscope}%
\pgfsys@transformshift{2.768898in}{0.797823in}%
\pgfsys@useobject{currentmarker}{}%
\end{pgfscope}%
\begin{pgfscope}%
\pgfsys@transformshift{2.787208in}{0.700479in}%
\pgfsys@useobject{currentmarker}{}%
\end{pgfscope}%
\begin{pgfscope}%
\pgfsys@transformshift{2.808333in}{0.609791in}%
\pgfsys@useobject{currentmarker}{}%
\end{pgfscope}%
\begin{pgfscope}%
\pgfsys@transformshift{2.832275in}{0.489217in}%
\pgfsys@useobject{currentmarker}{}%
\end{pgfscope}%
\begin{pgfscope}%
\pgfsys@transformshift{2.848705in}{0.456304in}%
\pgfsys@useobject{currentmarker}{}%
\end{pgfscope}%
\begin{pgfscope}%
\pgfsys@transformshift{2.867484in}{0.473835in}%
\pgfsys@useobject{currentmarker}{}%
\end{pgfscope}%
\begin{pgfscope}%
\pgfsys@transformshift{2.885558in}{0.550572in}%
\pgfsys@useobject{currentmarker}{}%
\end{pgfscope}%
\begin{pgfscope}%
\pgfsys@transformshift{2.901989in}{0.640725in}%
\pgfsys@useobject{currentmarker}{}%
\end{pgfscope}%
\begin{pgfscope}%
\pgfsys@transformshift{2.922881in}{0.728508in}%
\pgfsys@useobject{currentmarker}{}%
\end{pgfscope}%
\begin{pgfscope}%
\pgfsys@transformshift{2.942363in}{0.812208in}%
\pgfsys@useobject{currentmarker}{}%
\end{pgfscope}%
\begin{pgfscope}%
\pgfsys@transformshift{2.961611in}{0.847746in}%
\pgfsys@useobject{currentmarker}{}%
\end{pgfscope}%
\begin{pgfscope}%
\pgfsys@transformshift{2.985787in}{0.850906in}%
\pgfsys@useobject{currentmarker}{}%
\end{pgfscope}%
\begin{pgfscope}%
\pgfsys@transformshift{3.000575in}{0.843431in}%
\pgfsys@useobject{currentmarker}{}%
\end{pgfscope}%
\begin{pgfscope}%
\pgfsys@transformshift{3.017711in}{0.808285in}%
\pgfsys@useobject{currentmarker}{}%
\end{pgfscope}%
\begin{pgfscope}%
\pgfsys@transformshift{3.040950in}{0.782590in}%
\pgfsys@useobject{currentmarker}{}%
\end{pgfscope}%
\begin{pgfscope}%
\pgfsys@transformshift{3.055737in}{0.834836in}%
\pgfsys@useobject{currentmarker}{}%
\end{pgfscope}%
\begin{pgfscope}%
\pgfsys@transformshift{3.076628in}{0.854818in}%
\pgfsys@useobject{currentmarker}{}%
\end{pgfscope}%
\begin{pgfscope}%
\pgfsys@transformshift{3.097753in}{0.836162in}%
\pgfsys@useobject{currentmarker}{}%
\end{pgfscope}%
\begin{pgfscope}%
\pgfsys@transformshift{3.115358in}{0.771426in}%
\pgfsys@useobject{currentmarker}{}%
\end{pgfscope}%
\begin{pgfscope}%
\pgfsys@transformshift{3.134371in}{0.662929in}%
\pgfsys@useobject{currentmarker}{}%
\end{pgfscope}%
\begin{pgfscope}%
\pgfsys@transformshift{3.154324in}{0.558069in}%
\pgfsys@useobject{currentmarker}{}%
\end{pgfscope}%
\begin{pgfscope}%
\pgfsys@transformshift{3.172866in}{0.497432in}%
\pgfsys@useobject{currentmarker}{}%
\end{pgfscope}%
\begin{pgfscope}%
\pgfsys@transformshift{3.194697in}{0.468417in}%
\pgfsys@useobject{currentmarker}{}%
\end{pgfscope}%
\begin{pgfscope}%
\pgfsys@transformshift{3.208547in}{0.546554in}%
\pgfsys@useobject{currentmarker}{}%
\end{pgfscope}%
\begin{pgfscope}%
\pgfsys@transformshift{3.232489in}{0.638831in}%
\pgfsys@useobject{currentmarker}{}%
\end{pgfscope}%
\begin{pgfscope}%
\pgfsys@transformshift{3.255257in}{0.736773in}%
\pgfsys@useobject{currentmarker}{}%
\end{pgfscope}%
\begin{pgfscope}%
\pgfsys@transformshift{3.269341in}{0.799044in}%
\pgfsys@useobject{currentmarker}{}%
\end{pgfscope}%
\begin{pgfscope}%
\pgfsys@transformshift{3.290466in}{0.847949in}%
\pgfsys@useobject{currentmarker}{}%
\end{pgfscope}%
\begin{pgfscope}%
\pgfsys@transformshift{3.307133in}{0.856861in}%
\pgfsys@useobject{currentmarker}{}%
\end{pgfscope}%
\begin{pgfscope}%
\pgfsys@transformshift{3.328493in}{0.831419in}%
\pgfsys@useobject{currentmarker}{}%
\end{pgfscope}%
\begin{pgfscope}%
\pgfsys@transformshift{3.346332in}{0.783353in}%
\pgfsys@useobject{currentmarker}{}%
\end{pgfscope}%
\begin{pgfscope}%
\pgfsys@transformshift{3.366988in}{0.716870in}%
\pgfsys@useobject{currentmarker}{}%
\end{pgfscope}%
\begin{pgfscope}%
\pgfsys@transformshift{3.385298in}{0.614552in}%
\pgfsys@useobject{currentmarker}{}%
\end{pgfscope}%
\begin{pgfscope}%
\pgfsys@transformshift{3.405015in}{0.557079in}%
\pgfsys@useobject{currentmarker}{}%
\end{pgfscope}%
\begin{pgfscope}%
\pgfsys@transformshift{3.424731in}{0.475801in}%
\pgfsys@useobject{currentmarker}{}%
\end{pgfscope}%
\begin{pgfscope}%
\pgfsys@transformshift{3.442570in}{0.470519in}%
\pgfsys@useobject{currentmarker}{}%
\end{pgfscope}%
\begin{pgfscope}%
\pgfsys@transformshift{3.464635in}{0.552871in}%
\pgfsys@useobject{currentmarker}{}%
\end{pgfscope}%
\begin{pgfscope}%
\pgfsys@transformshift{3.478954in}{0.634706in}%
\pgfsys@useobject{currentmarker}{}%
\end{pgfscope}%
\begin{pgfscope}%
\pgfsys@transformshift{3.498670in}{0.735700in}%
\pgfsys@useobject{currentmarker}{}%
\end{pgfscope}%
\begin{pgfscope}%
\pgfsys@transformshift{3.519563in}{0.813309in}%
\pgfsys@useobject{currentmarker}{}%
\end{pgfscope}%
\begin{pgfscope}%
\pgfsys@transformshift{3.537871in}{0.851840in}%
\pgfsys@useobject{currentmarker}{}%
\end{pgfscope}%
\begin{pgfscope}%
\pgfsys@transformshift{3.559701in}{0.613628in}%
\pgfsys@useobject{currentmarker}{}%
\end{pgfscope}%
\begin{pgfscope}%
\pgfsys@transformshift{3.576132in}{0.674002in}%
\pgfsys@useobject{currentmarker}{}%
\end{pgfscope}%
\begin{pgfscope}%
\pgfsys@transformshift{3.596319in}{0.770228in}%
\pgfsys@useobject{currentmarker}{}%
\end{pgfscope}%
\begin{pgfscope}%
\pgfsys@transformshift{3.620027in}{0.844907in}%
\pgfsys@useobject{currentmarker}{}%
\end{pgfscope}%
\begin{pgfscope}%
\pgfsys@transformshift{3.636692in}{0.861143in}%
\pgfsys@useobject{currentmarker}{}%
\end{pgfscope}%
\begin{pgfscope}%
\pgfsys@transformshift{3.651951in}{0.856006in}%
\pgfsys@useobject{currentmarker}{}%
\end{pgfscope}%
\begin{pgfscope}%
\pgfsys@transformshift{3.671902in}{0.817442in}%
\pgfsys@useobject{currentmarker}{}%
\end{pgfscope}%
\begin{pgfscope}%
\pgfsys@transformshift{3.693497in}{0.725791in}%
\pgfsys@useobject{currentmarker}{}%
\end{pgfscope}%
\begin{pgfscope}%
\pgfsys@transformshift{3.711337in}{0.635833in}%
\pgfsys@useobject{currentmarker}{}%
\end{pgfscope}%
\begin{pgfscope}%
\pgfsys@transformshift{3.730350in}{0.551285in}%
\pgfsys@useobject{currentmarker}{}%
\end{pgfscope}%
\begin{pgfscope}%
\pgfsys@transformshift{3.750066in}{0.477823in}%
\pgfsys@useobject{currentmarker}{}%
\end{pgfscope}%
\begin{pgfscope}%
\pgfsys@transformshift{3.772365in}{0.509981in}%
\pgfsys@useobject{currentmarker}{}%
\end{pgfscope}%
\begin{pgfscope}%
\pgfsys@transformshift{3.788562in}{0.563407in}%
\pgfsys@useobject{currentmarker}{}%
\end{pgfscope}%
\begin{pgfscope}%
\pgfsys@transformshift{3.807341in}{0.523476in}%
\pgfsys@useobject{currentmarker}{}%
\end{pgfscope}%
\begin{pgfscope}%
\pgfsys@transformshift{3.828466in}{0.626547in}%
\pgfsys@useobject{currentmarker}{}%
\end{pgfscope}%
\begin{pgfscope}%
\pgfsys@transformshift{3.847244in}{0.720632in}%
\pgfsys@useobject{currentmarker}{}%
\end{pgfscope}%
\begin{pgfscope}%
\pgfsys@transformshift{3.863675in}{0.799054in}%
\pgfsys@useobject{currentmarker}{}%
\end{pgfscope}%
\begin{pgfscope}%
\pgfsys@transformshift{3.885271in}{0.852058in}%
\pgfsys@useobject{currentmarker}{}%
\end{pgfscope}%
\begin{pgfscope}%
\pgfsys@transformshift{3.904050in}{0.866244in}%
\pgfsys@useobject{currentmarker}{}%
\end{pgfscope}%
\begin{pgfscope}%
\pgfsys@transformshift{3.924471in}{0.853736in}%
\pgfsys@useobject{currentmarker}{}%
\end{pgfscope}%
\begin{pgfscope}%
\pgfsys@transformshift{3.942545in}{0.812779in}%
\pgfsys@useobject{currentmarker}{}%
\end{pgfscope}%
\begin{pgfscope}%
\pgfsys@transformshift{3.962967in}{0.725165in}%
\pgfsys@useobject{currentmarker}{}%
\end{pgfscope}%
\begin{pgfscope}%
\pgfsys@transformshift{3.981275in}{0.636503in}%
\pgfsys@useobject{currentmarker}{}%
\end{pgfscope}%
\begin{pgfscope}%
\pgfsys@transformshift{3.998411in}{0.551977in}%
\pgfsys@useobject{currentmarker}{}%
\end{pgfscope}%
\begin{pgfscope}%
\pgfsys@transformshift{4.019536in}{0.490475in}%
\pgfsys@useobject{currentmarker}{}%
\end{pgfscope}%
\begin{pgfscope}%
\pgfsys@transformshift{4.039723in}{0.511328in}%
\pgfsys@useobject{currentmarker}{}%
\end{pgfscope}%
\begin{pgfscope}%
\pgfsys@transformshift{4.056623in}{0.589062in}%
\pgfsys@useobject{currentmarker}{}%
\end{pgfscope}%
\begin{pgfscope}%
\pgfsys@transformshift{4.077513in}{0.685816in}%
\pgfsys@useobject{currentmarker}{}%
\end{pgfscope}%
\begin{pgfscope}%
\pgfsys@transformshift{4.095118in}{0.756409in}%
\pgfsys@useobject{currentmarker}{}%
\end{pgfscope}%
\begin{pgfscope}%
\pgfsys@transformshift{4.113897in}{0.822566in}%
\pgfsys@useobject{currentmarker}{}%
\end{pgfscope}%
\begin{pgfscope}%
\pgfsys@transformshift{4.133615in}{0.861322in}%
\pgfsys@useobject{currentmarker}{}%
\end{pgfscope}%
\begin{pgfscope}%
\pgfsys@transformshift{4.151689in}{0.871333in}%
\pgfsys@useobject{currentmarker}{}%
\end{pgfscope}%
\begin{pgfscope}%
\pgfsys@transformshift{4.173048in}{0.858087in}%
\pgfsys@useobject{currentmarker}{}%
\end{pgfscope}%
\begin{pgfscope}%
\pgfsys@transformshift{4.191827in}{0.824835in}%
\pgfsys@useobject{currentmarker}{}%
\end{pgfscope}%
\begin{pgfscope}%
\pgfsys@transformshift{4.212483in}{0.838939in}%
\pgfsys@useobject{currentmarker}{}%
\end{pgfscope}%
\begin{pgfscope}%
\pgfsys@transformshift{4.231028in}{0.830798in}%
\pgfsys@useobject{currentmarker}{}%
\end{pgfscope}%
\begin{pgfscope}%
\pgfsys@transformshift{4.249101in}{0.757637in}%
\pgfsys@useobject{currentmarker}{}%
\end{pgfscope}%
\begin{pgfscope}%
\pgfsys@transformshift{4.269992in}{0.643963in}%
\pgfsys@useobject{currentmarker}{}%
\end{pgfscope}%
\begin{pgfscope}%
\pgfsys@transformshift{4.290179in}{0.552479in}%
\pgfsys@useobject{currentmarker}{}%
\end{pgfscope}%
\begin{pgfscope}%
\pgfsys@transformshift{4.306610in}{0.496839in}%
\pgfsys@useobject{currentmarker}{}%
\end{pgfscope}%
\begin{pgfscope}%
\pgfsys@transformshift{4.327501in}{0.525463in}%
\pgfsys@useobject{currentmarker}{}%
\end{pgfscope}%
\begin{pgfscope}%
\pgfsys@transformshift{4.345340in}{0.602332in}%
\pgfsys@useobject{currentmarker}{}%
\end{pgfscope}%
\begin{pgfscope}%
\pgfsys@transformshift{4.363884in}{0.681666in}%
\pgfsys@useobject{currentmarker}{}%
\end{pgfscope}%
\begin{pgfscope}%
\pgfsys@transformshift{4.382898in}{0.775870in}%
\pgfsys@useobject{currentmarker}{}%
\end{pgfscope}%
\begin{pgfscope}%
\pgfsys@transformshift{4.401674in}{0.835957in}%
\pgfsys@useobject{currentmarker}{}%
\end{pgfscope}%
\begin{pgfscope}%
\pgfsys@transformshift{4.423036in}{0.868248in}%
\pgfsys@useobject{currentmarker}{}%
\end{pgfscope}%
\begin{pgfscope}%
\pgfsys@transformshift{4.440875in}{0.877685in}%
\pgfsys@useobject{currentmarker}{}%
\end{pgfscope}%
\begin{pgfscope}%
\pgfsys@transformshift{4.462236in}{0.861766in}%
\pgfsys@useobject{currentmarker}{}%
\end{pgfscope}%
\begin{pgfscope}%
\pgfsys@transformshift{4.480310in}{0.828266in}%
\pgfsys@useobject{currentmarker}{}%
\end{pgfscope}%
\begin{pgfscope}%
\pgfsys@transformshift{4.481718in}{0.830028in}%
\pgfsys@useobject{currentmarker}{}%
\end{pgfscope}%
\begin{pgfscope}%
\pgfsys@transformshift{4.474442in}{0.850070in}%
\pgfsys@useobject{currentmarker}{}%
\end{pgfscope}%
\begin{pgfscope}%
\pgfsys@transformshift{4.453786in}{0.877495in}%
\pgfsys@useobject{currentmarker}{}%
\end{pgfscope}%
\begin{pgfscope}%
\pgfsys@transformshift{4.436181in}{0.738331in}%
\pgfsys@useobject{currentmarker}{}%
\end{pgfscope}%
\begin{pgfscope}%
\pgfsys@transformshift{4.417168in}{0.619666in}%
\pgfsys@useobject{currentmarker}{}%
\end{pgfscope}%
\begin{pgfscope}%
\pgfsys@transformshift{4.395103in}{0.504642in}%
\pgfsys@useobject{currentmarker}{}%
\end{pgfscope}%
\begin{pgfscope}%
\pgfsys@transformshift{4.378203in}{0.535452in}%
\pgfsys@useobject{currentmarker}{}%
\end{pgfscope}%
\begin{pgfscope}%
\pgfsys@transformshift{4.356373in}{0.671616in}%
\pgfsys@useobject{currentmarker}{}%
\end{pgfscope}%
\begin{pgfscope}%
\pgfsys@transformshift{4.341820in}{0.775410in}%
\pgfsys@useobject{currentmarker}{}%
\end{pgfscope}%
\begin{pgfscope}%
\pgfsys@transformshift{4.321632in}{0.849030in}%
\pgfsys@useobject{currentmarker}{}%
\end{pgfscope}%
\begin{pgfscope}%
\pgfsys@transformshift{4.301916in}{0.872311in}%
\pgfsys@useobject{currentmarker}{}%
\end{pgfscope}%
\begin{pgfscope}%
\pgfsys@transformshift{4.282903in}{0.848080in}%
\pgfsys@useobject{currentmarker}{}%
\end{pgfscope}%
\begin{pgfscope}%
\pgfsys@transformshift{4.265063in}{0.787162in}%
\pgfsys@useobject{currentmarker}{}%
\end{pgfscope}%
\begin{pgfscope}%
\pgfsys@transformshift{4.242999in}{0.657574in}%
\pgfsys@useobject{currentmarker}{}%
\end{pgfscope}%
\begin{pgfscope}%
\pgfsys@transformshift{4.223751in}{0.551345in}%
\pgfsys@useobject{currentmarker}{}%
\end{pgfscope}%
\begin{pgfscope}%
\pgfsys@transformshift{4.202155in}{0.483801in}%
\pgfsys@useobject{currentmarker}{}%
\end{pgfscope}%
\begin{pgfscope}%
\pgfsys@transformshift{4.183847in}{0.566211in}%
\pgfsys@useobject{currentmarker}{}%
\end{pgfscope}%
\begin{pgfscope}%
\pgfsys@transformshift{4.167651in}{0.668699in}%
\pgfsys@useobject{currentmarker}{}%
\end{pgfscope}%
\begin{pgfscope}%
\pgfsys@transformshift{4.149341in}{0.787357in}%
\pgfsys@useobject{currentmarker}{}%
\end{pgfscope}%
\begin{pgfscope}%
\pgfsys@transformshift{4.128216in}{0.856466in}%
\pgfsys@useobject{currentmarker}{}%
\end{pgfscope}%
\begin{pgfscope}%
\pgfsys@transformshift{4.109672in}{0.866287in}%
\pgfsys@useobject{currentmarker}{}%
\end{pgfscope}%
\begin{pgfscope}%
\pgfsys@transformshift{4.090893in}{0.838193in}%
\pgfsys@useobject{currentmarker}{}%
\end{pgfscope}%
\begin{pgfscope}%
\pgfsys@transformshift{4.068830in}{0.746870in}%
\pgfsys@useobject{currentmarker}{}%
\end{pgfscope}%
\begin{pgfscope}%
\pgfsys@transformshift{4.052163in}{0.652368in}%
\pgfsys@useobject{currentmarker}{}%
\end{pgfscope}%
\begin{pgfscope}%
\pgfsys@transformshift{4.034089in}{0.540133in}%
\pgfsys@useobject{currentmarker}{}%
\end{pgfscope}%
\begin{pgfscope}%
\pgfsys@transformshift{4.013668in}{0.475625in}%
\pgfsys@useobject{currentmarker}{}%
\end{pgfscope}%
\begin{pgfscope}%
\pgfsys@transformshift{3.993011in}{0.556731in}%
\pgfsys@useobject{currentmarker}{}%
\end{pgfscope}%
\begin{pgfscope}%
\pgfsys@transformshift{3.975875in}{0.675338in}%
\pgfsys@useobject{currentmarker}{}%
\end{pgfscope}%
\begin{pgfscope}%
\pgfsys@transformshift{3.951465in}{0.801180in}%
\pgfsys@useobject{currentmarker}{}%
\end{pgfscope}%
\begin{pgfscope}%
\pgfsys@transformshift{3.937146in}{0.847087in}%
\pgfsys@useobject{currentmarker}{}%
\end{pgfscope}%
\begin{pgfscope}%
\pgfsys@transformshift{3.918838in}{0.862790in}%
\pgfsys@useobject{currentmarker}{}%
\end{pgfscope}%
\begin{pgfscope}%
\pgfsys@transformshift{3.897947in}{0.835186in}%
\pgfsys@useobject{currentmarker}{}%
\end{pgfscope}%
\begin{pgfscope}%
\pgfsys@transformshift{3.879871in}{0.765632in}%
\pgfsys@useobject{currentmarker}{}%
\end{pgfscope}%
\begin{pgfscope}%
\pgfsys@transformshift{3.861798in}{0.672953in}%
\pgfsys@useobject{currentmarker}{}%
\end{pgfscope}%
\begin{pgfscope}%
\pgfsys@transformshift{3.839733in}{0.551522in}%
\pgfsys@useobject{currentmarker}{}%
\end{pgfscope}%
\begin{pgfscope}%
\pgfsys@transformshift{3.821660in}{0.474128in}%
\pgfsys@useobject{currentmarker}{}%
\end{pgfscope}%
\begin{pgfscope}%
\pgfsys@transformshift{3.802412in}{0.498848in}%
\pgfsys@useobject{currentmarker}{}%
\end{pgfscope}%
\begin{pgfscope}%
\pgfsys@transformshift{3.784336in}{0.585277in}%
\pgfsys@useobject{currentmarker}{}%
\end{pgfscope}%
\begin{pgfscope}%
\pgfsys@transformshift{3.764620in}{0.702899in}%
\pgfsys@useobject{currentmarker}{}%
\end{pgfscope}%
\begin{pgfscope}%
\pgfsys@transformshift{3.740912in}{0.817623in}%
\pgfsys@useobject{currentmarker}{}%
\end{pgfscope}%
\begin{pgfscope}%
\pgfsys@transformshift{3.724247in}{0.840387in}%
\pgfsys@useobject{currentmarker}{}%
\end{pgfscope}%
\begin{pgfscope}%
\pgfsys@transformshift{3.706408in}{0.858923in}%
\pgfsys@useobject{currentmarker}{}%
\end{pgfscope}%
\begin{pgfscope}%
\pgfsys@transformshift{3.683169in}{0.835175in}%
\pgfsys@useobject{currentmarker}{}%
\end{pgfscope}%
\begin{pgfscope}%
\pgfsys@transformshift{3.666973in}{0.779497in}%
\pgfsys@useobject{currentmarker}{}%
\end{pgfscope}%
\begin{pgfscope}%
\pgfsys@transformshift{3.649368in}{0.681550in}%
\pgfsys@useobject{currentmarker}{}%
\end{pgfscope}%
\begin{pgfscope}%
\pgfsys@transformshift{3.629181in}{0.594844in}%
\pgfsys@useobject{currentmarker}{}%
\end{pgfscope}%
\begin{pgfscope}%
\pgfsys@transformshift{3.610168in}{0.510699in}%
\pgfsys@useobject{currentmarker}{}%
\end{pgfscope}%
\begin{pgfscope}%
\pgfsys@transformshift{3.589277in}{0.464102in}%
\pgfsys@useobject{currentmarker}{}%
\end{pgfscope}%
\begin{pgfscope}%
\pgfsys@transformshift{3.572141in}{0.529763in}%
\pgfsys@useobject{currentmarker}{}%
\end{pgfscope}%
\begin{pgfscope}%
\pgfsys@transformshift{3.551721in}{0.630613in}%
\pgfsys@useobject{currentmarker}{}%
\end{pgfscope}%
\begin{pgfscope}%
\pgfsys@transformshift{3.532003in}{0.662862in}%
\pgfsys@useobject{currentmarker}{}%
\end{pgfscope}%
\begin{pgfscope}%
\pgfsys@transformshift{3.513695in}{0.737402in}%
\pgfsys@useobject{currentmarker}{}%
\end{pgfscope}%
\begin{pgfscope}%
\pgfsys@transformshift{3.495619in}{0.816857in}%
\pgfsys@useobject{currentmarker}{}%
\end{pgfscope}%
\begin{pgfscope}%
\pgfsys@transformshift{3.474963in}{0.854743in}%
\pgfsys@useobject{currentmarker}{}%
\end{pgfscope}%
\begin{pgfscope}%
\pgfsys@transformshift{3.451490in}{0.846957in}%
\pgfsys@useobject{currentmarker}{}%
\end{pgfscope}%
\begin{pgfscope}%
\pgfsys@transformshift{3.436938in}{0.816265in}%
\pgfsys@useobject{currentmarker}{}%
\end{pgfscope}%
\begin{pgfscope}%
\pgfsys@transformshift{3.416046in}{0.762288in}%
\pgfsys@useobject{currentmarker}{}%
\end{pgfscope}%
\begin{pgfscope}%
\pgfsys@transformshift{3.398677in}{0.699489in}%
\pgfsys@useobject{currentmarker}{}%
\end{pgfscope}%
\begin{pgfscope}%
\pgfsys@transformshift{3.378021in}{0.612077in}%
\pgfsys@useobject{currentmarker}{}%
\end{pgfscope}%
\begin{pgfscope}%
\pgfsys@transformshift{3.359477in}{0.523070in}%
\pgfsys@useobject{currentmarker}{}%
\end{pgfscope}%
\begin{pgfscope}%
\pgfsys@transformshift{3.341638in}{0.459562in}%
\pgfsys@useobject{currentmarker}{}%
\end{pgfscope}%
\begin{pgfscope}%
\pgfsys@transformshift{3.321685in}{0.504748in}%
\pgfsys@useobject{currentmarker}{}%
\end{pgfscope}%
\begin{pgfscope}%
\pgfsys@transformshift{3.303611in}{0.593194in}%
\pgfsys@useobject{currentmarker}{}%
\end{pgfscope}%
\begin{pgfscope}%
\pgfsys@transformshift{3.282486in}{0.710914in}%
\pgfsys@useobject{currentmarker}{}%
\end{pgfscope}%
\begin{pgfscope}%
\pgfsys@transformshift{3.264411in}{0.787317in}%
\pgfsys@useobject{currentmarker}{}%
\end{pgfscope}%
\begin{pgfscope}%
\pgfsys@transformshift{3.246808in}{0.841357in}%
\pgfsys@useobject{currentmarker}{}%
\end{pgfscope}%
\begin{pgfscope}%
\pgfsys@transformshift{3.227324in}{0.853542in}%
\pgfsys@useobject{currentmarker}{}%
\end{pgfscope}%
\begin{pgfscope}%
\pgfsys@transformshift{3.206433in}{0.834340in}%
\pgfsys@useobject{currentmarker}{}%
\end{pgfscope}%
\begin{pgfscope}%
\pgfsys@transformshift{3.182257in}{0.755814in}%
\pgfsys@useobject{currentmarker}{}%
\end{pgfscope}%
\begin{pgfscope}%
\pgfsys@transformshift{3.167938in}{0.683646in}%
\pgfsys@useobject{currentmarker}{}%
\end{pgfscope}%
\begin{pgfscope}%
\pgfsys@transformshift{3.146578in}{0.576800in}%
\pgfsys@useobject{currentmarker}{}%
\end{pgfscope}%
\begin{pgfscope}%
\pgfsys@transformshift{3.131085in}{0.502034in}%
\pgfsys@useobject{currentmarker}{}%
\end{pgfscope}%
\begin{pgfscope}%
\pgfsys@transformshift{3.110664in}{0.454070in}%
\pgfsys@useobject{currentmarker}{}%
\end{pgfscope}%
\begin{pgfscope}%
\pgfsys@transformshift{3.090476in}{0.505844in}%
\pgfsys@useobject{currentmarker}{}%
\end{pgfscope}%
\begin{pgfscope}%
\pgfsys@transformshift{3.072637in}{0.501032in}%
\pgfsys@useobject{currentmarker}{}%
\end{pgfscope}%
\begin{pgfscope}%
\pgfsys@transformshift{3.051512in}{0.603433in}%
\pgfsys@useobject{currentmarker}{}%
\end{pgfscope}%
\begin{pgfscope}%
\pgfsys@transformshift{3.033907in}{0.703680in}%
\pgfsys@useobject{currentmarker}{}%
\end{pgfscope}%
\begin{pgfscope}%
\pgfsys@transformshift{3.016068in}{0.805003in}%
\pgfsys@useobject{currentmarker}{}%
\end{pgfscope}%
\begin{pgfscope}%
\pgfsys@transformshift{2.994003in}{0.846404in}%
\pgfsys@useobject{currentmarker}{}%
\end{pgfscope}%
\begin{pgfscope}%
\pgfsys@transformshift{2.973582in}{0.848253in}%
\pgfsys@useobject{currentmarker}{}%
\end{pgfscope}%
\begin{pgfscope}%
\pgfsys@transformshift{2.953394in}{0.812291in}%
\pgfsys@useobject{currentmarker}{}%
\end{pgfscope}%
\begin{pgfscope}%
\pgfsys@transformshift{2.934852in}{0.744259in}%
\pgfsys@useobject{currentmarker}{}%
\end{pgfscope}%
\begin{pgfscope}%
\pgfsys@transformshift{2.917716in}{0.693638in}%
\pgfsys@useobject{currentmarker}{}%
\end{pgfscope}%
\begin{pgfscope}%
\pgfsys@transformshift{2.898703in}{0.586558in}%
\pgfsys@useobject{currentmarker}{}%
\end{pgfscope}%
\begin{pgfscope}%
\pgfsys@transformshift{2.880629in}{0.494466in}%
\pgfsys@useobject{currentmarker}{}%
\end{pgfscope}%
\begin{pgfscope}%
\pgfsys@transformshift{2.861147in}{0.452911in}%
\pgfsys@useobject{currentmarker}{}%
\end{pgfscope}%
\begin{pgfscope}%
\pgfsys@transformshift{2.840256in}{0.501555in}%
\pgfsys@useobject{currentmarker}{}%
\end{pgfscope}%
\begin{pgfscope}%
\pgfsys@transformshift{2.821478in}{0.564193in}%
\pgfsys@useobject{currentmarker}{}%
\end{pgfscope}%
\begin{pgfscope}%
\pgfsys@transformshift{2.804576in}{0.665257in}%
\pgfsys@useobject{currentmarker}{}%
\end{pgfscope}%
\begin{pgfscope}%
\pgfsys@transformshift{2.786268in}{0.727843in}%
\pgfsys@useobject{currentmarker}{}%
\end{pgfscope}%
\begin{pgfscope}%
\pgfsys@transformshift{2.764672in}{0.818947in}%
\pgfsys@useobject{currentmarker}{}%
\end{pgfscope}%
\begin{pgfscope}%
\pgfsys@transformshift{2.744016in}{0.850825in}%
\pgfsys@useobject{currentmarker}{}%
\end{pgfscope}%
\begin{pgfscope}%
\pgfsys@transformshift{2.725942in}{0.844375in}%
\pgfsys@useobject{currentmarker}{}%
\end{pgfscope}%
\begin{pgfscope}%
\pgfsys@transformshift{2.703878in}{0.794589in}%
\pgfsys@useobject{currentmarker}{}%
\end{pgfscope}%
\begin{pgfscope}%
\pgfsys@transformshift{2.688150in}{0.765048in}%
\pgfsys@useobject{currentmarker}{}%
\end{pgfscope}%
\begin{pgfscope}%
\pgfsys@transformshift{2.669373in}{0.670182in}%
\pgfsys@useobject{currentmarker}{}%
\end{pgfscope}%
\begin{pgfscope}%
\pgfsys@transformshift{2.650595in}{0.572686in}%
\pgfsys@useobject{currentmarker}{}%
\end{pgfscope}%
\begin{pgfscope}%
\pgfsys@transformshift{2.630173in}{0.478759in}%
\pgfsys@useobject{currentmarker}{}%
\end{pgfscope}%
\begin{pgfscope}%
\pgfsys@transformshift{2.611865in}{0.455099in}%
\pgfsys@useobject{currentmarker}{}%
\end{pgfscope}%
\begin{pgfscope}%
\pgfsys@transformshift{2.595198in}{0.535035in}%
\pgfsys@useobject{currentmarker}{}%
\end{pgfscope}%
\begin{pgfscope}%
\pgfsys@transformshift{2.569613in}{0.596933in}%
\pgfsys@useobject{currentmarker}{}%
\end{pgfscope}%
\begin{pgfscope}%
\pgfsys@transformshift{2.554120in}{0.686570in}%
\pgfsys@useobject{currentmarker}{}%
\end{pgfscope}%
\begin{pgfscope}%
\pgfsys@transformshift{2.533464in}{0.775935in}%
\pgfsys@useobject{currentmarker}{}%
\end{pgfscope}%
\begin{pgfscope}%
\pgfsys@transformshift{2.514921in}{0.823258in}%
\pgfsys@useobject{currentmarker}{}%
\end{pgfscope}%
\begin{pgfscope}%
\pgfsys@transformshift{2.493325in}{0.851552in}%
\pgfsys@useobject{currentmarker}{}%
\end{pgfscope}%
\begin{pgfscope}%
\pgfsys@transformshift{2.473843in}{0.833791in}%
\pgfsys@useobject{currentmarker}{}%
\end{pgfscope}%
\begin{pgfscope}%
\pgfsys@transformshift{2.454596in}{0.788040in}%
\pgfsys@useobject{currentmarker}{}%
\end{pgfscope}%
\begin{pgfscope}%
\pgfsys@transformshift{2.436286in}{0.711994in}%
\pgfsys@useobject{currentmarker}{}%
\end{pgfscope}%
\begin{pgfscope}%
\pgfsys@transformshift{2.419152in}{0.631818in}%
\pgfsys@useobject{currentmarker}{}%
\end{pgfscope}%
\begin{pgfscope}%
\pgfsys@transformshift{2.397321in}{0.528099in}%
\pgfsys@useobject{currentmarker}{}%
\end{pgfscope}%
\begin{pgfscope}%
\pgfsys@transformshift{2.376431in}{0.452912in}%
\pgfsys@useobject{currentmarker}{}%
\end{pgfscope}%
\begin{pgfscope}%
\pgfsys@transformshift{2.358826in}{0.488267in}%
\pgfsys@useobject{currentmarker}{}%
\end{pgfscope}%
\begin{pgfscope}%
\pgfsys@transformshift{2.340516in}{0.538888in}%
\pgfsys@useobject{currentmarker}{}%
\end{pgfscope}%
\begin{pgfscope}%
\pgfsys@transformshift{2.322208in}{0.636689in}%
\pgfsys@useobject{currentmarker}{}%
\end{pgfscope}%
\begin{pgfscope}%
\pgfsys@transformshift{2.304369in}{0.737828in}%
\pgfsys@useobject{currentmarker}{}%
\end{pgfscope}%
\begin{pgfscope}%
\pgfsys@transformshift{2.283478in}{0.814208in}%
\pgfsys@useobject{currentmarker}{}%
\end{pgfscope}%
\begin{pgfscope}%
\pgfsys@transformshift{2.263760in}{0.849820in}%
\pgfsys@useobject{currentmarker}{}%
\end{pgfscope}%
\begin{pgfscope}%
\pgfsys@transformshift{2.247798in}{0.851328in}%
\pgfsys@useobject{currentmarker}{}%
\end{pgfscope}%
\begin{pgfscope}%
\pgfsys@transformshift{2.226673in}{0.830718in}%
\pgfsys@useobject{currentmarker}{}%
\end{pgfscope}%
\begin{pgfscope}%
\pgfsys@transformshift{2.204139in}{0.766018in}%
\pgfsys@useobject{currentmarker}{}%
\end{pgfscope}%
\begin{pgfscope}%
\pgfsys@transformshift{2.189352in}{0.692079in}%
\pgfsys@useobject{currentmarker}{}%
\end{pgfscope}%
\begin{pgfscope}%
\pgfsys@transformshift{2.167287in}{0.612670in}%
\pgfsys@useobject{currentmarker}{}%
\end{pgfscope}%
\begin{pgfscope}%
\pgfsys@transformshift{2.148977in}{0.528157in}%
\pgfsys@useobject{currentmarker}{}%
\end{pgfscope}%
\begin{pgfscope}%
\pgfsys@transformshift{2.129964in}{0.492716in}%
\pgfsys@useobject{currentmarker}{}%
\end{pgfscope}%
\begin{pgfscope}%
\pgfsys@transformshift{2.110716in}{0.459463in}%
\pgfsys@useobject{currentmarker}{}%
\end{pgfscope}%
\begin{pgfscope}%
\pgfsys@transformshift{2.092174in}{0.507115in}%
\pgfsys@useobject{currentmarker}{}%
\end{pgfscope}%
\begin{pgfscope}%
\pgfsys@transformshift{2.067761in}{0.628672in}%
\pgfsys@useobject{currentmarker}{}%
\end{pgfscope}%
\begin{pgfscope}%
\pgfsys@transformshift{2.051565in}{0.638239in}%
\pgfsys@useobject{currentmarker}{}%
\end{pgfscope}%
\begin{pgfscope}%
\pgfsys@transformshift{2.033256in}{0.741877in}%
\pgfsys@useobject{currentmarker}{}%
\end{pgfscope}%
\begin{pgfscope}%
\pgfsys@transformshift{2.014946in}{0.720647in}%
\pgfsys@useobject{currentmarker}{}%
\end{pgfscope}%
\begin{pgfscope}%
\pgfsys@transformshift{1.994996in}{0.815563in}%
\pgfsys@useobject{currentmarker}{}%
\end{pgfscope}%
\begin{pgfscope}%
\pgfsys@transformshift{1.975513in}{0.851263in}%
\pgfsys@useobject{currentmarker}{}%
\end{pgfscope}%
\begin{pgfscope}%
\pgfsys@transformshift{1.955561in}{0.512348in}%
\pgfsys@useobject{currentmarker}{}%
\end{pgfscope}%
\begin{pgfscope}%
\pgfsys@transformshift{1.937956in}{0.611870in}%
\pgfsys@useobject{currentmarker}{}%
\end{pgfscope}%
\begin{pgfscope}%
\pgfsys@transformshift{1.915657in}{0.747950in}%
\pgfsys@useobject{currentmarker}{}%
\end{pgfscope}%
\begin{pgfscope}%
\pgfsys@transformshift{1.896409in}{0.821776in}%
\pgfsys@useobject{currentmarker}{}%
\end{pgfscope}%
\begin{pgfscope}%
\pgfsys@transformshift{1.879039in}{0.853282in}%
\pgfsys@useobject{currentmarker}{}%
\end{pgfscope}%
\begin{pgfscope}%
\pgfsys@transformshift{1.859557in}{0.854577in}%
\pgfsys@useobject{currentmarker}{}%
\end{pgfscope}%
\begin{pgfscope}%
\pgfsys@transformshift{1.841952in}{0.828381in}%
\pgfsys@useobject{currentmarker}{}%
\end{pgfscope}%
\begin{pgfscope}%
\pgfsys@transformshift{1.824816in}{0.776672in}%
\pgfsys@useobject{currentmarker}{}%
\end{pgfscope}%
\begin{pgfscope}%
\pgfsys@transformshift{1.803222in}{0.685710in}%
\pgfsys@useobject{currentmarker}{}%
\end{pgfscope}%
\begin{pgfscope}%
\pgfsys@transformshift{1.781392in}{0.589649in}%
\pgfsys@useobject{currentmarker}{}%
\end{pgfscope}%
\begin{pgfscope}%
\pgfsys@transformshift{1.761910in}{0.500527in}%
\pgfsys@useobject{currentmarker}{}%
\end{pgfscope}%
\begin{pgfscope}%
\pgfsys@transformshift{1.743834in}{0.465193in}%
\pgfsys@useobject{currentmarker}{}%
\end{pgfscope}%
\begin{pgfscope}%
\pgfsys@transformshift{1.726466in}{0.523870in}%
\pgfsys@useobject{currentmarker}{}%
\end{pgfscope}%
\begin{pgfscope}%
\pgfsys@transformshift{1.704870in}{0.611076in}%
\pgfsys@useobject{currentmarker}{}%
\end{pgfscope}%
\begin{pgfscope}%
\pgfsys@transformshift{1.690082in}{0.704092in}%
\pgfsys@useobject{currentmarker}{}%
\end{pgfscope}%
\begin{pgfscope}%
\pgfsys@transformshift{1.669426in}{0.806279in}%
\pgfsys@useobject{currentmarker}{}%
\end{pgfscope}%
\begin{pgfscope}%
\pgfsys@transformshift{1.650413in}{0.848756in}%
\pgfsys@useobject{currentmarker}{}%
\end{pgfscope}%
\begin{pgfscope}%
\pgfsys@transformshift{1.630460in}{0.857784in}%
\pgfsys@useobject{currentmarker}{}%
\end{pgfscope}%
\begin{pgfscope}%
\pgfsys@transformshift{1.609335in}{0.834137in}%
\pgfsys@useobject{currentmarker}{}%
\end{pgfscope}%
\begin{pgfscope}%
\pgfsys@transformshift{1.593844in}{0.853849in}%
\pgfsys@useobject{currentmarker}{}%
\end{pgfscope}%
\begin{pgfscope}%
\pgfsys@transformshift{1.571308in}{0.811981in}%
\pgfsys@useobject{currentmarker}{}%
\end{pgfscope}%
\begin{pgfscope}%
\pgfsys@transformshift{1.553469in}{0.742593in}%
\pgfsys@useobject{currentmarker}{}%
\end{pgfscope}%
\begin{pgfscope}%
\pgfsys@transformshift{1.536335in}{0.662508in}%
\pgfsys@useobject{currentmarker}{}%
\end{pgfscope}%
\begin{pgfscope}%
\pgfsys@transformshift{1.515443in}{0.557749in}%
\pgfsys@useobject{currentmarker}{}%
\end{pgfscope}%
\begin{pgfscope}%
\pgfsys@transformshift{1.496195in}{0.484167in}%
\pgfsys@useobject{currentmarker}{}%
\end{pgfscope}%
\begin{pgfscope}%
\pgfsys@transformshift{1.471550in}{0.494805in}%
\pgfsys@useobject{currentmarker}{}%
\end{pgfscope}%
\begin{pgfscope}%
\pgfsys@transformshift{1.458168in}{0.546965in}%
\pgfsys@useobject{currentmarker}{}%
\end{pgfscope}%
\begin{pgfscope}%
\pgfsys@transformshift{1.437278in}{0.664073in}%
\pgfsys@useobject{currentmarker}{}%
\end{pgfscope}%
\begin{pgfscope}%
\pgfsys@transformshift{1.420378in}{0.763378in}%
\pgfsys@useobject{currentmarker}{}%
\end{pgfscope}%
\begin{pgfscope}%
\pgfsys@transformshift{1.399017in}{0.827656in}%
\pgfsys@useobject{currentmarker}{}%
\end{pgfscope}%
\begin{pgfscope}%
\pgfsys@transformshift{1.381412in}{0.856977in}%
\pgfsys@useobject{currentmarker}{}%
\end{pgfscope}%
\begin{pgfscope}%
\pgfsys@transformshift{1.363573in}{0.862013in}%
\pgfsys@useobject{currentmarker}{}%
\end{pgfscope}%
\begin{pgfscope}%
\pgfsys@transformshift{1.336580in}{0.830257in}%
\pgfsys@useobject{currentmarker}{}%
\end{pgfscope}%
\begin{pgfscope}%
\pgfsys@transformshift{1.321323in}{0.784902in}%
\pgfsys@useobject{currentmarker}{}%
\end{pgfscope}%
\begin{pgfscope}%
\pgfsys@transformshift{1.303718in}{0.721454in}%
\pgfsys@useobject{currentmarker}{}%
\end{pgfscope}%
\begin{pgfscope}%
\pgfsys@transformshift{1.279774in}{0.608993in}%
\pgfsys@useobject{currentmarker}{}%
\end{pgfscope}%
\begin{pgfscope}%
\pgfsys@transformshift{1.263578in}{0.537680in}%
\pgfsys@useobject{currentmarker}{}%
\end{pgfscope}%
\begin{pgfscope}%
\pgfsys@transformshift{1.246444in}{0.480386in}%
\pgfsys@useobject{currentmarker}{}%
\end{pgfscope}%
\begin{pgfscope}%
\pgfsys@transformshift{1.225317in}{0.489544in}%
\pgfsys@useobject{currentmarker}{}%
\end{pgfscope}%
\begin{pgfscope}%
\pgfsys@transformshift{1.203018in}{0.583876in}%
\pgfsys@useobject{currentmarker}{}%
\end{pgfscope}%
\begin{pgfscope}%
\pgfsys@transformshift{1.187761in}{0.646024in}%
\pgfsys@useobject{currentmarker}{}%
\end{pgfscope}%
\begin{pgfscope}%
\pgfsys@transformshift{1.168982in}{0.722933in}%
\pgfsys@useobject{currentmarker}{}%
\end{pgfscope}%
\begin{pgfscope}%
\pgfsys@transformshift{1.150203in}{0.802988in}%
\pgfsys@useobject{currentmarker}{}%
\end{pgfscope}%
\begin{pgfscope}%
\pgfsys@transformshift{1.131427in}{0.850642in}%
\pgfsys@useobject{currentmarker}{}%
\end{pgfscope}%
\begin{pgfscope}%
\pgfsys@transformshift{1.111708in}{0.867177in}%
\pgfsys@useobject{currentmarker}{}%
\end{pgfscope}%
\begin{pgfscope}%
\pgfsys@transformshift{1.091757in}{0.728731in}%
\pgfsys@useobject{currentmarker}{}%
\end{pgfscope}%
\begin{pgfscope}%
\pgfsys@transformshift{1.072744in}{0.814019in}%
\pgfsys@useobject{currentmarker}{}%
\end{pgfscope}%
\begin{pgfscope}%
\pgfsys@transformshift{1.051383in}{0.861007in}%
\pgfsys@useobject{currentmarker}{}%
\end{pgfscope}%
\begin{pgfscope}%
\pgfsys@transformshift{1.032369in}{0.867551in}%
\pgfsys@useobject{currentmarker}{}%
\end{pgfscope}%
\begin{pgfscope}%
\pgfsys@transformshift{1.016878in}{0.852918in}%
\pgfsys@useobject{currentmarker}{}%
\end{pgfscope}%
\begin{pgfscope}%
\pgfsys@transformshift{0.996456in}{0.812288in}%
\pgfsys@useobject{currentmarker}{}%
\end{pgfscope}%
\begin{pgfscope}%
\pgfsys@transformshift{0.977209in}{0.742091in}%
\pgfsys@useobject{currentmarker}{}%
\end{pgfscope}%
\begin{pgfscope}%
\pgfsys@transformshift{0.959604in}{0.658430in}%
\pgfsys@useobject{currentmarker}{}%
\end{pgfscope}%
\begin{pgfscope}%
\pgfsys@transformshift{0.937071in}{0.557689in}%
\pgfsys@useobject{currentmarker}{}%
\end{pgfscope}%
\begin{pgfscope}%
\pgfsys@transformshift{0.918761in}{0.494176in}%
\pgfsys@useobject{currentmarker}{}%
\end{pgfscope}%
\begin{pgfscope}%
\pgfsys@transformshift{0.901861in}{0.509104in}%
\pgfsys@useobject{currentmarker}{}%
\end{pgfscope}%
\begin{pgfscope}%
\pgfsys@transformshift{0.882379in}{0.594266in}%
\pgfsys@useobject{currentmarker}{}%
\end{pgfscope}%
\begin{pgfscope}%
\pgfsys@transformshift{0.859609in}{0.705587in}%
\pgfsys@useobject{currentmarker}{}%
\end{pgfscope}%
\begin{pgfscope}%
\pgfsys@transformshift{0.841535in}{0.766209in}%
\pgfsys@useobject{currentmarker}{}%
\end{pgfscope}%
\begin{pgfscope}%
\pgfsys@transformshift{0.819940in}{0.828598in}%
\pgfsys@useobject{currentmarker}{}%
\end{pgfscope}%
\begin{pgfscope}%
\pgfsys@transformshift{0.804448in}{0.542162in}%
\pgfsys@useobject{currentmarker}{}%
\end{pgfscope}%
\begin{pgfscope}%
\pgfsys@transformshift{0.785670in}{0.646946in}%
\pgfsys@useobject{currentmarker}{}%
\end{pgfscope}%
\begin{pgfscope}%
\pgfsys@transformshift{0.761257in}{0.781208in}%
\pgfsys@useobject{currentmarker}{}%
\end{pgfscope}%
\begin{pgfscope}%
\pgfsys@transformshift{0.747878in}{0.827274in}%
\pgfsys@useobject{currentmarker}{}%
\end{pgfscope}%
\begin{pgfscope}%
\pgfsys@transformshift{0.726753in}{0.869461in}%
\pgfsys@useobject{currentmarker}{}%
\end{pgfscope}%
\begin{pgfscope}%
\pgfsys@transformshift{0.707974in}{0.870998in}%
\pgfsys@useobject{currentmarker}{}%
\end{pgfscope}%
\begin{pgfscope}%
\pgfsys@transformshift{0.689431in}{0.847980in}%
\pgfsys@useobject{currentmarker}{}%
\end{pgfscope}%
\begin{pgfscope}%
\pgfsys@transformshift{0.670418in}{0.800739in}%
\pgfsys@useobject{currentmarker}{}%
\end{pgfscope}%
\begin{pgfscope}%
\pgfsys@transformshift{0.648588in}{0.703651in}%
\pgfsys@useobject{currentmarker}{}%
\end{pgfscope}%
\begin{pgfscope}%
\pgfsys@transformshift{0.648353in}{0.704665in}%
\pgfsys@useobject{currentmarker}{}%
\end{pgfscope}%
\begin{pgfscope}%
\pgfsys@transformshift{0.658916in}{0.758697in}%
\pgfsys@useobject{currentmarker}{}%
\end{pgfscope}%
\begin{pgfscope}%
\pgfsys@transformshift{0.675815in}{0.835977in}%
\pgfsys@useobject{currentmarker}{}%
\end{pgfscope}%
\begin{pgfscope}%
\pgfsys@transformshift{0.694829in}{0.870416in}%
\pgfsys@useobject{currentmarker}{}%
\end{pgfscope}%
\begin{pgfscope}%
\pgfsys@transformshift{0.716893in}{0.861157in}%
\pgfsys@useobject{currentmarker}{}%
\end{pgfscope}%
\begin{pgfscope}%
\pgfsys@transformshift{0.734498in}{0.803521in}%
\pgfsys@useobject{currentmarker}{}%
\end{pgfscope}%
\begin{pgfscope}%
\pgfsys@transformshift{0.752572in}{0.699291in}%
\pgfsys@useobject{currentmarker}{}%
\end{pgfscope}%
\begin{pgfscope}%
\pgfsys@transformshift{0.772759in}{0.557627in}%
\pgfsys@useobject{currentmarker}{}%
\end{pgfscope}%
\begin{pgfscope}%
\pgfsys@transformshift{0.790364in}{0.487225in}%
\pgfsys@useobject{currentmarker}{}%
\end{pgfscope}%
\begin{pgfscope}%
\pgfsys@transformshift{0.811725in}{0.576373in}%
\pgfsys@useobject{currentmarker}{}%
\end{pgfscope}%
\begin{pgfscope}%
\pgfsys@transformshift{0.828859in}{0.686723in}%
\pgfsys@useobject{currentmarker}{}%
\end{pgfscope}%
\begin{pgfscope}%
\pgfsys@transformshift{0.848578in}{0.793667in}%
\pgfsys@useobject{currentmarker}{}%
\end{pgfscope}%
\begin{pgfscope}%
\pgfsys@transformshift{0.867591in}{0.850396in}%
\pgfsys@useobject{currentmarker}{}%
\end{pgfscope}%
\begin{pgfscope}%
\pgfsys@transformshift{0.885899in}{0.867498in}%
\pgfsys@useobject{currentmarker}{}%
\end{pgfscope}%
\begin{pgfscope}%
\pgfsys@transformshift{0.910546in}{0.817529in}%
\pgfsys@useobject{currentmarker}{}%
\end{pgfscope}%
\begin{pgfscope}%
\pgfsys@transformshift{0.925568in}{0.741035in}%
\pgfsys@useobject{currentmarker}{}%
\end{pgfscope}%
\begin{pgfscope}%
\pgfsys@transformshift{0.943642in}{0.617559in}%
\pgfsys@useobject{currentmarker}{}%
\end{pgfscope}%
\begin{pgfscope}%
\pgfsys@transformshift{0.961716in}{0.508318in}%
\pgfsys@useobject{currentmarker}{}%
\end{pgfscope}%
\begin{pgfscope}%
\pgfsys@transformshift{0.983077in}{0.500883in}%
\pgfsys@useobject{currentmarker}{}%
\end{pgfscope}%
\begin{pgfscope}%
\pgfsys@transformshift{1.000916in}{0.596965in}%
\pgfsys@useobject{currentmarker}{}%
\end{pgfscope}%
\begin{pgfscope}%
\pgfsys@transformshift{1.021807in}{0.721529in}%
\pgfsys@useobject{currentmarker}{}%
\end{pgfscope}%
\begin{pgfscope}%
\pgfsys@transformshift{1.039412in}{0.814150in}%
\pgfsys@useobject{currentmarker}{}%
\end{pgfscope}%
\begin{pgfscope}%
\pgfsys@transformshift{1.060068in}{0.860761in}%
\pgfsys@useobject{currentmarker}{}%
\end{pgfscope}%
\begin{pgfscope}%
\pgfsys@transformshift{1.078378in}{0.857079in}%
\pgfsys@useobject{currentmarker}{}%
\end{pgfscope}%
\begin{pgfscope}%
\pgfsys@transformshift{1.099503in}{0.800730in}%
\pgfsys@useobject{currentmarker}{}%
\end{pgfscope}%
\begin{pgfscope}%
\pgfsys@transformshift{1.117342in}{0.694396in}%
\pgfsys@useobject{currentmarker}{}%
\end{pgfscope}%
\begin{pgfscope}%
\pgfsys@transformshift{1.138467in}{0.569545in}%
\pgfsys@useobject{currentmarker}{}%
\end{pgfscope}%
\begin{pgfscope}%
\pgfsys@transformshift{1.155837in}{0.482355in}%
\pgfsys@useobject{currentmarker}{}%
\end{pgfscope}%
\begin{pgfscope}%
\pgfsys@transformshift{1.173911in}{0.493976in}%
\pgfsys@useobject{currentmarker}{}%
\end{pgfscope}%
\begin{pgfscope}%
\pgfsys@transformshift{1.191987in}{0.584964in}%
\pgfsys@useobject{currentmarker}{}%
\end{pgfscope}%
\begin{pgfscope}%
\pgfsys@transformshift{1.214051in}{0.705732in}%
\pgfsys@useobject{currentmarker}{}%
\end{pgfscope}%
\begin{pgfscope}%
\pgfsys@transformshift{1.233299in}{0.795039in}%
\pgfsys@useobject{currentmarker}{}%
\end{pgfscope}%
\begin{pgfscope}%
\pgfsys@transformshift{1.254189in}{0.847558in}%
\pgfsys@useobject{currentmarker}{}%
\end{pgfscope}%
\begin{pgfscope}%
\pgfsys@transformshift{1.272968in}{0.859462in}%
\pgfsys@useobject{currentmarker}{}%
\end{pgfscope}%
\begin{pgfscope}%
\pgfsys@transformshift{1.290573in}{0.836334in}%
\pgfsys@useobject{currentmarker}{}%
\end{pgfscope}%
\begin{pgfscope}%
\pgfsys@transformshift{1.312872in}{0.744858in}%
\pgfsys@useobject{currentmarker}{}%
\end{pgfscope}%
\begin{pgfscope}%
\pgfsys@transformshift{1.329772in}{0.653675in}%
\pgfsys@useobject{currentmarker}{}%
\end{pgfscope}%
\begin{pgfscope}%
\pgfsys@transformshift{1.348316in}{0.541579in}%
\pgfsys@useobject{currentmarker}{}%
\end{pgfscope}%
\begin{pgfscope}%
\pgfsys@transformshift{1.368503in}{0.463920in}%
\pgfsys@useobject{currentmarker}{}%
\end{pgfscope}%
\begin{pgfscope}%
\pgfsys@transformshift{1.385872in}{0.507308in}%
\pgfsys@useobject{currentmarker}{}%
\end{pgfscope}%
\begin{pgfscope}%
\pgfsys@transformshift{1.405120in}{0.561637in}%
\pgfsys@useobject{currentmarker}{}%
\end{pgfscope}%
\begin{pgfscope}%
\pgfsys@transformshift{1.428124in}{0.689477in}%
\pgfsys@useobject{currentmarker}{}%
\end{pgfscope}%
\begin{pgfscope}%
\pgfsys@transformshift{1.444789in}{0.785149in}%
\pgfsys@useobject{currentmarker}{}%
\end{pgfscope}%
\begin{pgfscope}%
\pgfsys@transformshift{1.464273in}{0.832080in}%
\pgfsys@useobject{currentmarker}{}%
\end{pgfscope}%
\begin{pgfscope}%
\pgfsys@transformshift{1.482581in}{0.855867in}%
\pgfsys@useobject{currentmarker}{}%
\end{pgfscope}%
\begin{pgfscope}%
\pgfsys@transformshift{1.503472in}{0.842484in}%
\pgfsys@useobject{currentmarker}{}%
\end{pgfscope}%
\begin{pgfscope}%
\pgfsys@transformshift{1.520137in}{0.798610in}%
\pgfsys@useobject{currentmarker}{}%
\end{pgfscope}%
\begin{pgfscope}%
\pgfsys@transformshift{1.538681in}{0.696001in}%
\pgfsys@useobject{currentmarker}{}%
\end{pgfscope}%
\begin{pgfscope}%
\pgfsys@transformshift{1.557929in}{0.607607in}%
\pgfsys@useobject{currentmarker}{}%
\end{pgfscope}%
\begin{pgfscope}%
\pgfsys@transformshift{1.581167in}{0.493649in}%
\pgfsys@useobject{currentmarker}{}%
\end{pgfscope}%
\begin{pgfscope}%
\pgfsys@transformshift{1.597598in}{0.457377in}%
\pgfsys@useobject{currentmarker}{}%
\end{pgfscope}%
\begin{pgfscope}%
\pgfsys@transformshift{1.618960in}{0.496494in}%
\pgfsys@useobject{currentmarker}{}%
\end{pgfscope}%
\begin{pgfscope}%
\pgfsys@transformshift{1.634451in}{0.570225in}%
\pgfsys@useobject{currentmarker}{}%
\end{pgfscope}%
\begin{pgfscope}%
\pgfsys@transformshift{1.655576in}{0.659459in}%
\pgfsys@useobject{currentmarker}{}%
\end{pgfscope}%
\begin{pgfscope}%
\pgfsys@transformshift{1.676232in}{0.762755in}%
\pgfsys@useobject{currentmarker}{}%
\end{pgfscope}%
\begin{pgfscope}%
\pgfsys@transformshift{1.691959in}{0.809367in}%
\pgfsys@useobject{currentmarker}{}%
\end{pgfscope}%
\begin{pgfscope}%
\pgfsys@transformshift{1.713321in}{0.845387in}%
\pgfsys@useobject{currentmarker}{}%
\end{pgfscope}%
\begin{pgfscope}%
\pgfsys@transformshift{1.732568in}{0.847506in}%
\pgfsys@useobject{currentmarker}{}%
\end{pgfscope}%
\begin{pgfscope}%
\pgfsys@transformshift{1.753693in}{0.794738in}%
\pgfsys@useobject{currentmarker}{}%
\end{pgfscope}%
\begin{pgfscope}%
\pgfsys@transformshift{1.773175in}{0.844716in}%
\pgfsys@useobject{currentmarker}{}%
\end{pgfscope}%
\begin{pgfscope}%
\pgfsys@transformshift{1.786789in}{0.854018in}%
\pgfsys@useobject{currentmarker}{}%
\end{pgfscope}%
\begin{pgfscope}%
\pgfsys@transformshift{1.814722in}{0.804545in}%
\pgfsys@useobject{currentmarker}{}%
\end{pgfscope}%
\begin{pgfscope}%
\pgfsys@transformshift{1.828338in}{0.739999in}%
\pgfsys@useobject{currentmarker}{}%
\end{pgfscope}%
\begin{pgfscope}%
\pgfsys@transformshift{1.849697in}{0.608782in}%
\pgfsys@useobject{currentmarker}{}%
\end{pgfscope}%
\begin{pgfscope}%
\pgfsys@transformshift{1.868242in}{0.516077in}%
\pgfsys@useobject{currentmarker}{}%
\end{pgfscope}%
\begin{pgfscope}%
\pgfsys@transformshift{1.887255in}{0.456632in}%
\pgfsys@useobject{currentmarker}{}%
\end{pgfscope}%
\begin{pgfscope}%
\pgfsys@transformshift{1.906268in}{0.479551in}%
\pgfsys@useobject{currentmarker}{}%
\end{pgfscope}%
\begin{pgfscope}%
\pgfsys@transformshift{1.925750in}{0.564258in}%
\pgfsys@useobject{currentmarker}{}%
\end{pgfscope}%
\begin{pgfscope}%
\pgfsys@transformshift{1.944529in}{0.673782in}%
\pgfsys@useobject{currentmarker}{}%
\end{pgfscope}%
\begin{pgfscope}%
\pgfsys@transformshift{1.963306in}{0.741273in}%
\pgfsys@useobject{currentmarker}{}%
\end{pgfscope}%
\begin{pgfscope}%
\pgfsys@transformshift{1.982319in}{0.812882in}%
\pgfsys@useobject{currentmarker}{}%
\end{pgfscope}%
\begin{pgfscope}%
\pgfsys@transformshift{2.001098in}{0.848438in}%
\pgfsys@useobject{currentmarker}{}%
\end{pgfscope}%
\begin{pgfscope}%
\pgfsys@transformshift{2.019406in}{0.849586in}%
\pgfsys@useobject{currentmarker}{}%
\end{pgfscope}%
\begin{pgfscope}%
\pgfsys@transformshift{2.038185in}{0.815617in}%
\pgfsys@useobject{currentmarker}{}%
\end{pgfscope}%
\begin{pgfscope}%
\pgfsys@transformshift{2.060015in}{0.721089in}%
\pgfsys@useobject{currentmarker}{}%
\end{pgfscope}%
\begin{pgfscope}%
\pgfsys@transformshift{2.079497in}{0.612608in}%
\pgfsys@useobject{currentmarker}{}%
\end{pgfscope}%
\begin{pgfscope}%
\pgfsys@transformshift{2.098511in}{0.514723in}%
\pgfsys@useobject{currentmarker}{}%
\end{pgfscope}%
\begin{pgfscope}%
\pgfsys@transformshift{2.117055in}{0.457201in}%
\pgfsys@useobject{currentmarker}{}%
\end{pgfscope}%
\begin{pgfscope}%
\pgfsys@transformshift{2.135832in}{0.482959in}%
\pgfsys@useobject{currentmarker}{}%
\end{pgfscope}%
\begin{pgfscope}%
\pgfsys@transformshift{2.153908in}{0.570853in}%
\pgfsys@useobject{currentmarker}{}%
\end{pgfscope}%
\begin{pgfscope}%
\pgfsys@transformshift{2.173155in}{0.654479in}%
\pgfsys@useobject{currentmarker}{}%
\end{pgfscope}%
\begin{pgfscope}%
\pgfsys@transformshift{2.193577in}{0.767618in}%
\pgfsys@useobject{currentmarker}{}%
\end{pgfscope}%
\begin{pgfscope}%
\pgfsys@transformshift{2.215876in}{0.829122in}%
\pgfsys@useobject{currentmarker}{}%
\end{pgfscope}%
\begin{pgfscope}%
\pgfsys@transformshift{2.233010in}{0.851141in}%
\pgfsys@useobject{currentmarker}{}%
\end{pgfscope}%
\begin{pgfscope}%
\pgfsys@transformshift{2.248269in}{0.844276in}%
\pgfsys@useobject{currentmarker}{}%
\end{pgfscope}%
\begin{pgfscope}%
\pgfsys@transformshift{2.269863in}{0.809414in}%
\pgfsys@useobject{currentmarker}{}%
\end{pgfscope}%
\begin{pgfscope}%
\pgfsys@transformshift{2.289581in}{0.723049in}%
\pgfsys@useobject{currentmarker}{}%
\end{pgfscope}%
\begin{pgfscope}%
\pgfsys@transformshift{2.312114in}{0.591266in}%
\pgfsys@useobject{currentmarker}{}%
\end{pgfscope}%
\begin{pgfscope}%
\pgfsys@transformshift{2.329485in}{0.514515in}%
\pgfsys@useobject{currentmarker}{}%
\end{pgfscope}%
\begin{pgfscope}%
\pgfsys@transformshift{2.348967in}{0.454821in}%
\pgfsys@useobject{currentmarker}{}%
\end{pgfscope}%
\begin{pgfscope}%
\pgfsys@transformshift{2.367511in}{0.478501in}%
\pgfsys@useobject{currentmarker}{}%
\end{pgfscope}%
\begin{pgfscope}%
\pgfsys@transformshift{2.385585in}{0.547567in}%
\pgfsys@useobject{currentmarker}{}%
\end{pgfscope}%
\begin{pgfscope}%
\pgfsys@transformshift{2.404129in}{0.646557in}%
\pgfsys@useobject{currentmarker}{}%
\end{pgfscope}%
\begin{pgfscope}%
\pgfsys@transformshift{2.423377in}{0.745630in}%
\pgfsys@useobject{currentmarker}{}%
\end{pgfscope}%
\begin{pgfscope}%
\pgfsys@transformshift{2.444502in}{0.822924in}%
\pgfsys@useobject{currentmarker}{}%
\end{pgfscope}%
\begin{pgfscope}%
\pgfsys@transformshift{2.464453in}{0.849936in}%
\pgfsys@useobject{currentmarker}{}%
\end{pgfscope}%
\begin{pgfscope}%
\pgfsys@transformshift{2.481120in}{0.848804in}%
\pgfsys@useobject{currentmarker}{}%
\end{pgfscope}%
\begin{pgfscope}%
\pgfsys@transformshift{2.502011in}{0.818998in}%
\pgfsys@useobject{currentmarker}{}%
\end{pgfscope}%
\begin{pgfscope}%
\pgfsys@transformshift{2.520789in}{0.753317in}%
\pgfsys@useobject{currentmarker}{}%
\end{pgfscope}%
\begin{pgfscope}%
\pgfsys@transformshift{2.541914in}{0.848669in}%
\pgfsys@useobject{currentmarker}{}%
\end{pgfscope}%
\begin{pgfscope}%
\pgfsys@transformshift{2.559754in}{0.809489in}%
\pgfsys@useobject{currentmarker}{}%
\end{pgfscope}%
\begin{pgfscope}%
\pgfsys@transformshift{2.575716in}{0.737239in}%
\pgfsys@useobject{currentmarker}{}%
\end{pgfscope}%
\begin{pgfscope}%
\pgfsys@transformshift{2.601771in}{0.588021in}%
\pgfsys@useobject{currentmarker}{}%
\end{pgfscope}%
\begin{pgfscope}%
\pgfsys@transformshift{2.616793in}{0.516357in}%
\pgfsys@useobject{currentmarker}{}%
\end{pgfscope}%
\begin{pgfscope}%
\pgfsys@transformshift{2.635572in}{0.452920in}%
\pgfsys@useobject{currentmarker}{}%
\end{pgfscope}%
\begin{pgfscope}%
\pgfsys@transformshift{2.654586in}{0.492721in}%
\pgfsys@useobject{currentmarker}{}%
\end{pgfscope}%
\begin{pgfscope}%
\pgfsys@transformshift{2.677822in}{0.601608in}%
\pgfsys@useobject{currentmarker}{}%
\end{pgfscope}%
\begin{pgfscope}%
\pgfsys@transformshift{2.693315in}{0.682308in}%
\pgfsys@useobject{currentmarker}{}%
\end{pgfscope}%
\begin{pgfscope}%
\pgfsys@transformshift{2.711858in}{0.772055in}%
\pgfsys@useobject{currentmarker}{}%
\end{pgfscope}%
\begin{pgfscope}%
\pgfsys@transformshift{2.733454in}{0.836297in}%
\pgfsys@useobject{currentmarker}{}%
\end{pgfscope}%
\begin{pgfscope}%
\pgfsys@transformshift{2.751998in}{0.632489in}%
\pgfsys@useobject{currentmarker}{}%
\end{pgfscope}%
\begin{pgfscope}%
\pgfsys@transformshift{2.770775in}{0.739797in}%
\pgfsys@useobject{currentmarker}{}%
\end{pgfscope}%
\begin{pgfscope}%
\pgfsys@transformshift{2.788850in}{0.800738in}%
\pgfsys@useobject{currentmarker}{}%
\end{pgfscope}%
\begin{pgfscope}%
\pgfsys@transformshift{2.807627in}{0.845224in}%
\pgfsys@useobject{currentmarker}{}%
\end{pgfscope}%
\begin{pgfscope}%
\pgfsys@transformshift{2.829692in}{0.852867in}%
\pgfsys@useobject{currentmarker}{}%
\end{pgfscope}%
\begin{pgfscope}%
\pgfsys@transformshift{2.846125in}{0.831168in}%
\pgfsys@useobject{currentmarker}{}%
\end{pgfscope}%
\begin{pgfscope}%
\pgfsys@transformshift{2.866076in}{0.751215in}%
\pgfsys@useobject{currentmarker}{}%
\end{pgfscope}%
\begin{pgfscope}%
\pgfsys@transformshift{2.885089in}{0.631511in}%
\pgfsys@useobject{currentmarker}{}%
\end{pgfscope}%
\begin{pgfscope}%
\pgfsys@transformshift{2.903868in}{0.529148in}%
\pgfsys@useobject{currentmarker}{}%
\end{pgfscope}%
\begin{pgfscope}%
\pgfsys@transformshift{2.922176in}{0.459753in}%
\pgfsys@useobject{currentmarker}{}%
\end{pgfscope}%
\begin{pgfscope}%
\pgfsys@transformshift{2.943772in}{0.500367in}%
\pgfsys@useobject{currentmarker}{}%
\end{pgfscope}%
\begin{pgfscope}%
\pgfsys@transformshift{2.962080in}{0.586594in}%
\pgfsys@useobject{currentmarker}{}%
\end{pgfscope}%
\begin{pgfscope}%
\pgfsys@transformshift{2.983207in}{0.705550in}%
\pgfsys@useobject{currentmarker}{}%
\end{pgfscope}%
\begin{pgfscope}%
\pgfsys@transformshift{2.997994in}{0.788201in}%
\pgfsys@useobject{currentmarker}{}%
\end{pgfscope}%
\begin{pgfscope}%
\pgfsys@transformshift{3.018651in}{0.836934in}%
\pgfsys@useobject{currentmarker}{}%
\end{pgfscope}%
\begin{pgfscope}%
\pgfsys@transformshift{3.036959in}{0.854651in}%
\pgfsys@useobject{currentmarker}{}%
\end{pgfscope}%
\begin{pgfscope}%
\pgfsys@transformshift{3.059258in}{0.838039in}%
\pgfsys@useobject{currentmarker}{}%
\end{pgfscope}%
\begin{pgfscope}%
\pgfsys@transformshift{3.078740in}{0.770129in}%
\pgfsys@useobject{currentmarker}{}%
\end{pgfscope}%
\begin{pgfscope}%
\pgfsys@transformshift{3.097050in}{0.701887in}%
\pgfsys@useobject{currentmarker}{}%
\end{pgfscope}%
\begin{pgfscope}%
\pgfsys@transformshift{3.114889in}{0.647677in}%
\pgfsys@useobject{currentmarker}{}%
\end{pgfscope}%
\begin{pgfscope}%
\pgfsys@transformshift{3.135545in}{0.536345in}%
\pgfsys@useobject{currentmarker}{}%
\end{pgfscope}%
\begin{pgfscope}%
\pgfsys@transformshift{3.154324in}{0.468002in}%
\pgfsys@useobject{currentmarker}{}%
\end{pgfscope}%
\begin{pgfscope}%
\pgfsys@transformshift{3.175215in}{0.491268in}%
\pgfsys@useobject{currentmarker}{}%
\end{pgfscope}%
\begin{pgfscope}%
\pgfsys@transformshift{3.192114in}{0.553360in}%
\pgfsys@useobject{currentmarker}{}%
\end{pgfscope}%
\begin{pgfscope}%
\pgfsys@transformshift{3.211598in}{0.644788in}%
\pgfsys@useobject{currentmarker}{}%
\end{pgfscope}%
\begin{pgfscope}%
\pgfsys@transformshift{3.231080in}{0.741201in}%
\pgfsys@useobject{currentmarker}{}%
\end{pgfscope}%
\begin{pgfscope}%
\pgfsys@transformshift{3.250797in}{0.817009in}%
\pgfsys@useobject{currentmarker}{}%
\end{pgfscope}%
\begin{pgfscope}%
\pgfsys@transformshift{3.267228in}{0.851585in}%
\pgfsys@useobject{currentmarker}{}%
\end{pgfscope}%
\begin{pgfscope}%
\pgfsys@transformshift{3.287415in}{0.856824in}%
\pgfsys@useobject{currentmarker}{}%
\end{pgfscope}%
\begin{pgfscope}%
\pgfsys@transformshift{3.308540in}{0.844475in}%
\pgfsys@useobject{currentmarker}{}%
\end{pgfscope}%
\begin{pgfscope}%
\pgfsys@transformshift{3.326145in}{0.798755in}%
\pgfsys@useobject{currentmarker}{}%
\end{pgfscope}%
\begin{pgfscope}%
\pgfsys@transformshift{3.346566in}{0.696274in}%
\pgfsys@useobject{currentmarker}{}%
\end{pgfscope}%
\begin{pgfscope}%
\pgfsys@transformshift{3.365580in}{0.598253in}%
\pgfsys@useobject{currentmarker}{}%
\end{pgfscope}%
\begin{pgfscope}%
\pgfsys@transformshift{3.386236in}{0.503121in}%
\pgfsys@useobject{currentmarker}{}%
\end{pgfscope}%
\begin{pgfscope}%
\pgfsys@transformshift{3.403841in}{0.465478in}%
\pgfsys@useobject{currentmarker}{}%
\end{pgfscope}%
\begin{pgfscope}%
\pgfsys@transformshift{3.422385in}{0.510598in}%
\pgfsys@useobject{currentmarker}{}%
\end{pgfscope}%
\begin{pgfscope}%
\pgfsys@transformshift{3.443276in}{0.599003in}%
\pgfsys@useobject{currentmarker}{}%
\end{pgfscope}%
\begin{pgfscope}%
\pgfsys@transformshift{3.461349in}{0.662534in}%
\pgfsys@useobject{currentmarker}{}%
\end{pgfscope}%
\begin{pgfscope}%
\pgfsys@transformshift{3.481771in}{0.757424in}%
\pgfsys@useobject{currentmarker}{}%
\end{pgfscope}%
\begin{pgfscope}%
\pgfsys@transformshift{3.503132in}{0.470372in}%
\pgfsys@useobject{currentmarker}{}%
\end{pgfscope}%
\begin{pgfscope}%
\pgfsys@transformshift{3.521909in}{0.539311in}%
\pgfsys@useobject{currentmarker}{}%
\end{pgfscope}%
\begin{pgfscope}%
\pgfsys@transformshift{3.538340in}{0.611483in}%
\pgfsys@useobject{currentmarker}{}%
\end{pgfscope}%
\begin{pgfscope}%
\pgfsys@transformshift{3.559701in}{0.721344in}%
\pgfsys@useobject{currentmarker}{}%
\end{pgfscope}%
\begin{pgfscope}%
\pgfsys@transformshift{3.576837in}{0.796680in}%
\pgfsys@useobject{currentmarker}{}%
\end{pgfscope}%
\begin{pgfscope}%
\pgfsys@transformshift{3.594911in}{0.846084in}%
\pgfsys@useobject{currentmarker}{}%
\end{pgfscope}%
\begin{pgfscope}%
\pgfsys@transformshift{3.619558in}{0.860961in}%
\pgfsys@useobject{currentmarker}{}%
\end{pgfscope}%
\begin{pgfscope}%
\pgfsys@transformshift{3.634346in}{0.849138in}%
\pgfsys@useobject{currentmarker}{}%
\end{pgfscope}%
\begin{pgfscope}%
\pgfsys@transformshift{3.655236in}{0.799406in}%
\pgfsys@useobject{currentmarker}{}%
\end{pgfscope}%
\begin{pgfscope}%
\pgfsys@transformshift{3.673076in}{0.724130in}%
\pgfsys@useobject{currentmarker}{}%
\end{pgfscope}%
\begin{pgfscope}%
\pgfsys@transformshift{3.691149in}{0.627338in}%
\pgfsys@useobject{currentmarker}{}%
\end{pgfscope}%
\begin{pgfscope}%
\pgfsys@transformshift{3.708754in}{0.544519in}%
\pgfsys@useobject{currentmarker}{}%
\end{pgfscope}%
\begin{pgfscope}%
\pgfsys@transformshift{3.730115in}{0.525633in}%
\pgfsys@useobject{currentmarker}{}%
\end{pgfscope}%
\begin{pgfscope}%
\pgfsys@transformshift{3.750535in}{0.476224in}%
\pgfsys@useobject{currentmarker}{}%
\end{pgfscope}%
\begin{pgfscope}%
\pgfsys@transformshift{3.769785in}{0.528930in}%
\pgfsys@useobject{currentmarker}{}%
\end{pgfscope}%
\begin{pgfscope}%
\pgfsys@transformshift{3.789501in}{0.614667in}%
\pgfsys@useobject{currentmarker}{}%
\end{pgfscope}%
\begin{pgfscope}%
\pgfsys@transformshift{3.807106in}{0.707270in}%
\pgfsys@useobject{currentmarker}{}%
\end{pgfscope}%
\begin{pgfscope}%
\pgfsys@transformshift{3.825414in}{0.779463in}%
\pgfsys@useobject{currentmarker}{}%
\end{pgfscope}%
\begin{pgfscope}%
\pgfsys@transformshift{3.844193in}{0.838782in}%
\pgfsys@useobject{currentmarker}{}%
\end{pgfscope}%
\begin{pgfscope}%
\pgfsys@transformshift{3.863441in}{0.859665in}%
\pgfsys@useobject{currentmarker}{}%
\end{pgfscope}%
\begin{pgfscope}%
\pgfsys@transformshift{3.885036in}{0.863363in}%
\pgfsys@useobject{currentmarker}{}%
\end{pgfscope}%
\begin{pgfscope}%
\pgfsys@transformshift{3.903345in}{0.845873in}%
\pgfsys@useobject{currentmarker}{}%
\end{pgfscope}%
\begin{pgfscope}%
\pgfsys@transformshift{3.920949in}{0.809869in}%
\pgfsys@useobject{currentmarker}{}%
\end{pgfscope}%
\begin{pgfscope}%
\pgfsys@transformshift{3.942074in}{0.719871in}%
\pgfsys@useobject{currentmarker}{}%
\end{pgfscope}%
\begin{pgfscope}%
\pgfsys@transformshift{3.963436in}{0.645849in}%
\pgfsys@useobject{currentmarker}{}%
\end{pgfscope}%
\begin{pgfscope}%
\pgfsys@transformshift{3.981275in}{0.562878in}%
\pgfsys@useobject{currentmarker}{}%
\end{pgfscope}%
\begin{pgfscope}%
\pgfsys@transformshift{3.998880in}{0.495430in}%
\pgfsys@useobject{currentmarker}{}%
\end{pgfscope}%
\begin{pgfscope}%
\pgfsys@transformshift{4.021413in}{0.509847in}%
\pgfsys@useobject{currentmarker}{}%
\end{pgfscope}%
\begin{pgfscope}%
\pgfsys@transformshift{4.040661in}{0.583025in}%
\pgfsys@useobject{currentmarker}{}%
\end{pgfscope}%
\begin{pgfscope}%
\pgfsys@transformshift{4.057562in}{0.654382in}%
\pgfsys@useobject{currentmarker}{}%
\end{pgfscope}%
\begin{pgfscope}%
\pgfsys@transformshift{4.076576in}{0.745912in}%
\pgfsys@useobject{currentmarker}{}%
\end{pgfscope}%
\begin{pgfscope}%
\pgfsys@transformshift{4.097701in}{0.827202in}%
\pgfsys@useobject{currentmarker}{}%
\end{pgfscope}%
\begin{pgfscope}%
\pgfsys@transformshift{4.114602in}{0.858393in}%
\pgfsys@useobject{currentmarker}{}%
\end{pgfscope}%
\begin{pgfscope}%
\pgfsys@transformshift{4.135727in}{0.871180in}%
\pgfsys@useobject{currentmarker}{}%
\end{pgfscope}%
\begin{pgfscope}%
\pgfsys@transformshift{4.154035in}{0.862993in}%
\pgfsys@useobject{currentmarker}{}%
\end{pgfscope}%
\begin{pgfscope}%
\pgfsys@transformshift{4.174222in}{0.828774in}%
\pgfsys@useobject{currentmarker}{}%
\end{pgfscope}%
\begin{pgfscope}%
\pgfsys@transformshift{4.192531in}{0.773016in}%
\pgfsys@useobject{currentmarker}{}%
\end{pgfscope}%
\begin{pgfscope}%
\pgfsys@transformshift{4.210135in}{0.741771in}%
\pgfsys@useobject{currentmarker}{}%
\end{pgfscope}%
\begin{pgfscope}%
\pgfsys@transformshift{4.230793in}{0.663353in}%
\pgfsys@useobject{currentmarker}{}%
\end{pgfscope}%
\begin{pgfscope}%
\pgfsys@transformshift{4.248396in}{0.869890in}%
\pgfsys@useobject{currentmarker}{}%
\end{pgfscope}%
\begin{pgfscope}%
\pgfsys@transformshift{4.269052in}{0.837931in}%
\pgfsys@useobject{currentmarker}{}%
\end{pgfscope}%
\begin{pgfscope}%
\pgfsys@transformshift{4.286894in}{0.791660in}%
\pgfsys@useobject{currentmarker}{}%
\end{pgfscope}%
\begin{pgfscope}%
\pgfsys@transformshift{4.305905in}{0.697406in}%
\pgfsys@useobject{currentmarker}{}%
\end{pgfscope}%
\begin{pgfscope}%
\pgfsys@transformshift{4.326327in}{0.579443in}%
\pgfsys@useobject{currentmarker}{}%
\end{pgfscope}%
\begin{pgfscope}%
\pgfsys@transformshift{4.346279in}{0.498448in}%
\pgfsys@useobject{currentmarker}{}%
\end{pgfscope}%
\begin{pgfscope}%
\pgfsys@transformshift{4.364822in}{0.531574in}%
\pgfsys@useobject{currentmarker}{}%
\end{pgfscope}%
\begin{pgfscope}%
\pgfsys@transformshift{4.383132in}{0.612005in}%
\pgfsys@useobject{currentmarker}{}%
\end{pgfscope}%
\begin{pgfscope}%
\pgfsys@transformshift{4.400268in}{0.714704in}%
\pgfsys@useobject{currentmarker}{}%
\end{pgfscope}%
\begin{pgfscope}%
\pgfsys@transformshift{4.422567in}{0.785074in}%
\pgfsys@useobject{currentmarker}{}%
\end{pgfscope}%
\begin{pgfscope}%
\pgfsys@transformshift{4.446509in}{0.851052in}%
\pgfsys@useobject{currentmarker}{}%
\end{pgfscope}%
\begin{pgfscope}%
\pgfsys@transformshift{4.460828in}{0.871734in}%
\pgfsys@useobject{currentmarker}{}%
\end{pgfscope}%
\begin{pgfscope}%
\pgfsys@transformshift{4.482187in}{0.876987in}%
\pgfsys@useobject{currentmarker}{}%
\end{pgfscope}%
\begin{pgfscope}%
\pgfsys@transformshift{4.483127in}{0.876872in}%
\pgfsys@useobject{currentmarker}{}%
\end{pgfscope}%
\begin{pgfscope}%
\pgfsys@transformshift{4.472799in}{0.876400in}%
\pgfsys@useobject{currentmarker}{}%
\end{pgfscope}%
\begin{pgfscope}%
\pgfsys@transformshift{4.454020in}{0.846520in}%
\pgfsys@useobject{currentmarker}{}%
\end{pgfscope}%
\begin{pgfscope}%
\pgfsys@transformshift{4.439232in}{0.769958in}%
\pgfsys@useobject{currentmarker}{}%
\end{pgfscope}%
\begin{pgfscope}%
\pgfsys@transformshift{4.413647in}{0.637519in}%
\pgfsys@useobject{currentmarker}{}%
\end{pgfscope}%
\begin{pgfscope}%
\pgfsys@transformshift{4.394634in}{0.526174in}%
\pgfsys@useobject{currentmarker}{}%
\end{pgfscope}%
\begin{pgfscope}%
\pgfsys@transformshift{4.379610in}{0.502319in}%
\pgfsys@useobject{currentmarker}{}%
\end{pgfscope}%
\begin{pgfscope}%
\pgfsys@transformshift{4.354730in}{0.639905in}%
\pgfsys@useobject{currentmarker}{}%
\end{pgfscope}%
\begin{pgfscope}%
\pgfsys@transformshift{4.341585in}{0.725832in}%
\pgfsys@useobject{currentmarker}{}%
\end{pgfscope}%
\begin{pgfscope}%
\pgfsys@transformshift{4.320224in}{0.831486in}%
\pgfsys@useobject{currentmarker}{}%
\end{pgfscope}%
\begin{pgfscope}%
\pgfsys@transformshift{4.302150in}{0.868906in}%
\pgfsys@useobject{currentmarker}{}%
\end{pgfscope}%
\begin{pgfscope}%
\pgfsys@transformshift{4.281025in}{0.861490in}%
\pgfsys@useobject{currentmarker}{}%
\end{pgfscope}%
\begin{pgfscope}%
\pgfsys@transformshift{4.262481in}{0.808292in}%
\pgfsys@useobject{currentmarker}{}%
\end{pgfscope}%
\begin{pgfscope}%
\pgfsys@transformshift{4.244173in}{0.708411in}%
\pgfsys@useobject{currentmarker}{}%
\end{pgfscope}%
\begin{pgfscope}%
\pgfsys@transformshift{4.225863in}{0.585362in}%
\pgfsys@useobject{currentmarker}{}%
\end{pgfscope}%
\begin{pgfscope}%
\pgfsys@transformshift{4.204503in}{0.491828in}%
\pgfsys@useobject{currentmarker}{}%
\end{pgfscope}%
\begin{pgfscope}%
\pgfsys@transformshift{4.184551in}{0.538134in}%
\pgfsys@useobject{currentmarker}{}%
\end{pgfscope}%
\begin{pgfscope}%
\pgfsys@transformshift{4.165068in}{0.666690in}%
\pgfsys@useobject{currentmarker}{}%
\end{pgfscope}%
\end{pgfscope}%
\begin{pgfscope}%
\pgfsetrectcap%
\pgfsetmiterjoin%
\pgfsetlinewidth{0.501875pt}%
\definecolor{currentstroke}{rgb}{0.000000,0.000000,0.000000}%
\pgfsetstrokecolor{currentstroke}%
\pgfsetdash{}{0pt}%
\pgfpathmoveto{\pgfqpoint{0.444748in}{0.431673in}}%
\pgfpathlineto{\pgfqpoint{0.444748in}{0.898923in}}%
\pgfusepath{stroke}%
\end{pgfscope}%
\begin{pgfscope}%
\pgfsetrectcap%
\pgfsetmiterjoin%
\pgfsetlinewidth{0.501875pt}%
\definecolor{currentstroke}{rgb}{0.000000,0.000000,0.000000}%
\pgfsetstrokecolor{currentstroke}%
\pgfsetdash{}{0pt}%
\pgfpathmoveto{\pgfqpoint{4.676167in}{0.431673in}}%
\pgfpathlineto{\pgfqpoint{4.676167in}{0.898923in}}%
\pgfusepath{stroke}%
\end{pgfscope}%
\begin{pgfscope}%
\pgfsetrectcap%
\pgfsetmiterjoin%
\pgfsetlinewidth{0.501875pt}%
\definecolor{currentstroke}{rgb}{0.000000,0.000000,0.000000}%
\pgfsetstrokecolor{currentstroke}%
\pgfsetdash{}{0pt}%
\pgfpathmoveto{\pgfqpoint{0.444748in}{0.431673in}}%
\pgfpathlineto{\pgfqpoint{4.676167in}{0.431673in}}%
\pgfusepath{stroke}%
\end{pgfscope}%
\begin{pgfscope}%
\pgfsetrectcap%
\pgfsetmiterjoin%
\pgfsetlinewidth{0.501875pt}%
\definecolor{currentstroke}{rgb}{0.000000,0.000000,0.000000}%
\pgfsetstrokecolor{currentstroke}%
\pgfsetdash{}{0pt}%
\pgfpathmoveto{\pgfqpoint{0.444748in}{0.898923in}}%
\pgfpathlineto{\pgfqpoint{4.676167in}{0.898923in}}%
\pgfusepath{stroke}%
\end{pgfscope}%
\begin{pgfscope}%
\definecolor{textcolor}{rgb}{0.000000,0.000000,0.000000}%
\pgfsetstrokecolor{textcolor}%
\pgfsetfillcolor{textcolor}%
\pgftext[x=2.560458in,y=0.982257in,,base]{\color{textcolor}\rmfamily\fontsize{12.000000}{14.400000}\selectfont T = \qty{3.6}{\kelvin}}%
\end{pgfscope}%
\end{pgfpicture}%
\makeatother%
\endgroup%

	\caption{SQIs at \qtylist{3.0;3.2;3.4}{\kelvin} of sample CP2.6B. We note that they appear to have a slightly curved background. We used a bias current of \qty{300}{\micro\ampere}. Near zero field the patterns seem quite stable but further out become more noisy.}
	\label{fig:CP2.6B_revisited_SQIs}
\end{figure}

Whilst the SQIs are far from perfect, they are usable. Near zero field the periodicity as well as the horizontal offset is reproducible. The amplitude of the oscillations is stable as well. Since the junction loop will only create a small flux this should be sufficient. Around zero field we can approximate the oscillations as sinusoidal. After fitting it gives a sensitivity of \qtylist{8.16;24.97;79.35}{\micro\volt\per\fluxquantum} for \qtylist{3.0;3.2;3.4}{\kelvin} respectively. These are significantly lower than what we achieved earlier. This is expected however since the temperatures are higher and the bias current is lower.
