% !TEX root = ../../../thesis.tex
The measurements were performed in a \qty{1.6}{\kelvin} cryostat equipped with a \qty{7}{\tesla} magnetic field. While cooling down, we used a simple 4-point measurement to determine the resistance of the dc-SQUID as a function of temperature. The RT-curve is shown in Figure~\ref{fig:CP1.2H-SQUID-RT}. We note that the sample becomes superconducting at \qty{8}{\kelvin} and has a quite sharp transition. The deviation from pure \ce{Nb} ($T_c=\qty{9.2}{\kelvin}$\cite{maxfieldSuperconductingPenetrationDepth1965}) is due to impurities deposited during sputtering. \ce{Nb} in particular is very sensitive to this.

\begin{figure}[ht!]
	\centering
	\import{figures/samples/CP1}{CP1.2H_critical_temperature.pgf}
	\caption{
		RT-curve of sample CP1.2H. The main figure provides a detailed view of the superconducting transition at \qty{8}{\kelvin}. The inset provides an overview of resistance between \qtyrange{2}{300}{\kelvin} which is mostly linear.
	}
	\label{fig:CP1.2H-SQUID-RT}
\end{figure}

Additionally we determined the temperature dependence of the critical current. This is shown in Figure~\ref{fig:CP1.2H-SQUID-critical-current-temperature-dependence}. We note that, similar to what we see in Figure~\ref{fig:CP1.2H-SQUID-RT}, that there are two transitions starting from \qty{7.2}{\kelvin}. The obvious one is a very sharp transition and the second one a more elongated tail. The sharp transition is for the bulk of the superconductor (leads, contact pads, etc.) whilst the more elongated one is from the dc-SQUID's junctions. This defines the range in which we can use our dc-SQUID as a magnetometer. The temperature dependence of the junction behaviour is attributed to the temperature dependence of the coherence length (Section~\ref{sec:characteristic-length-scales}). When the coherence length is relatively small compared to the junction size, then the Cooper pair density can change fast enough to be unbothered by the junction. As such at \qty{7.2}{\kelvin} the junction behaviour disappears. 

\begin{figure}[ht!]
	\centering
	\import{figures/samples/CP1}{CP1.2H_critical_current.pgf}
	\caption{Temperature dependence of the critical current of the dc-SQUID. We note that near \qty{7.2}{\kelvin} an additional region becomes visible before the big transition where the resistance spikes up to above \qty{20}{\ohm}.}
	\label{fig:CP1.2H-SQUID-critical-current-temperature-dependence}
\end{figure}

Figure~\ref{fig:CP1.2H-SQUID-SQI} shows the interference pattern of the dc-SQUID at \qty{7.6}{\kelvin}. The periodicity of the interference pattern is between \qtyrange{3}{3.5}{\milli\tesla}. This means the effective area of our dc-SQUID should be between \qtyrange{0.6}{0.7}{\square\micro\meter} which corresponds to a diameter between \qtyrange{0.87}{0.94}{\micro\meter}. However, we know from the SEM images (Figure~\ref{fig:CP1.2H-SEM-images}) that the diameter of our effective area must be between \qtyrange{1.2}{1.6}{\micro\meter} (a periodicity between \qtyrange{1}{1.8}{\milli\tesla}). This larger periodicity suggests that less flux is present in the dc-SQUID.

\begin{figure}[ht!]
	\centering
	\import{figures/samples/CP1}{CP1.2H_SQI.pgf}
	\caption{The interference pattern measured over the dc-SQUID at \qty{7.6}{\kelvin}. The black lines are equipotential lines spaced \qty{5}{\micro\volt} apart.}
	\label{fig:CP1.2H-SQUID-SQI}
\end{figure}

Flux lensing\cite{prigozhin3DSimulationSuperconducting2018} does not provide an explanation. It would cause a higher flux in the dc-SQUID by focussing the magnetic field and thus a smaller periodicity. We furthermore trust the scale in the SEM images to be correct. In our view this leaves an incorrect field readout as explanation. During our use of the cryostat we later noticed that one of the two power supplies of the magnet was turned off. The current supplies operate in parallel, as such only half the current could be delivered effectively. If the magnet does not independently measure how much current is delivered then this would explain the factor two difference. As such we believe that this caused the incorrect field (readout). Due to time constraints however we have not been able to definitively test this hypothesis.

Furthermore, the highest sensitivity of the dc-SQUID (in the linear regime) is \qtyrange{18}{21}{\micro\volt\per\milli\tesla} (or \qtyrange{36}{42}{\micro\volt\per\milli\tesla} if the field readout was indeed off by a factor two). This is not very good compared to dc-SQUIDs produced in our group earlier. Par example \citeauthor{rogSQUIDontipMagneticMicroscopy2022} \citeyear{rogSQUIDontipMagneticMicroscopy2022} reported a dc-SQUID with similar geometries\footnote{The junctions were however SNS junctions and not constriction junctions as a layer of \ce{Ag} was present.} with a sensitivity around \qty{105}{\micro\volt\per\milli\tesla}\cite{rogSQUIDontipMagneticMicroscopy2022}. Improving the sensitivity for our current device further would be difficult because we are already near the critical current where we go through the bulk transition. Changing the temperature would be an option, however since taking a SQUID interference pattern takes a long time it was not viable due to the limited measurement time. Measurements at a constant bias current were attempted but did not work due to software limitations.

Finally, we attempted to measure the CPR at \qty{7.6}{\kelvin}\footnote{Due to the poor dc-SQUID performance we did not expect this to be extremely fruitful. However it is a quick measurement and would not hurt.}. To do so we biased the dc-SQUID at \qty{200}{\micro\ampere}. We then did a current sweep through the junction loop from \qtyrange{-250}{250}{\micro\ampere} and simultaneously measured the voltage over the dc-SQUID. The IV-curve we measured looked suspiciously much like the IV-curve of the dc-SQUID at \qty{7.6}{\kelvin}. This does not make sense and the measurement probably failed for the following two reasons. Firstly, the bias current through the dc-SQUID was too high, pushing it into its normal regime. This can be seen in Figure~\ref{fig:CP1.2H-SQUID-critical-current-temperature-dependence}. Secondly, there most likely was a short between some of the connections. The first issue could have been fixed easily, but was only noted too late. The second issue was not investigated further due to time constraints. Even if we had done so we likely would not have been able to fix it in a timely manner. More people in our group reported issues with this specific batch of \ce{Si} wafers. As such it appears to be a likely culprit.