% !TEX root = ../../thesis.tex
\chapter{Conclusion}
In this thesis a method to measure the current-phase relation is demonstrated. It utilises a dc-SQUID inductively coupled to a superconducting loop with a single junction. By passing a current through the junction's loop it is possible to modulate the phase of the junction. The phase is proportional to the flux in the loop which is measured by the dc-SQUID. From the total current and the measured flux the current-phase relation can be extracted.

The temperature dependence of the current-phase relation of a SNS junction was qualitatively determined using the above method. Starting from \qtyrange{2.8}{3.6}{\kelvin} the $2\pi$-periodic current-phase relation becomes more sinusoidal. Furthermore between \qtyrange{2.8}{3.4}{\kelvin} we determined the critical current to decrease from \qtyrange{90}{45}{\micro\ampere}. Both these observations are in line with theoretical models.

Whilst successful, further improvements are possible. Most notably the implementation of a flux-locked loop. This will allow biasing the dc-SQUID at its working point. This means that the measurements will become more sensitive. Furthermore, a flux-locked loop allows integrating the results over a longer period improving the accuracy. An attempt was made to implement the flux-locked loop using a current modulation line. However, that sample was unfortunately destroyed. Future projects can use its design.

\section{Outlook}
The potential of the method has been demonstrated. Future research should finetune the method. Primarily using the flux-locked loop. In order to measure the current-phase relation of \ce{Sr2RuO4} minor adjustments to the method are needed. It is impractical to make both the junction's loop and the dc-SQUID from the same crystal. As such it is experimentally advantageous to make the dc-SQUID from a different material. Furthermore, we currently lack the knowledge if we can pin a single chiral domain wall. Which is a requirement to making the junction's loop. Alternatively, the ring of \ce{Sr2RuO4} can be used as the junction and incorporated into a loop. In that case no additional weak links must be formed, or their behaviour negligible compared to the \ce{Sr2RuO4} ring.