% !TEX root = ../../thesis.tex
\chapter{Conclusion}
In this thesis a method to measure the current-phase relation of a Josephson junction was developed and its performance demonstrated. Our method utilises a dc-SQUID inductively coupled to a superconducting loop with a single junction. By passing a current through the junction's loop it is possible to modulate the phase of the junction. This phase is proportional to the flux in the loop which is measured by the dc-SQUID. From the total current and the measured flux the current-phase relation can be extracted.

The temperature dependence of the current-phase relation of a SNS junction was qualitatively determined using the above method. Starting from \qtyrange{2.8}{3.6}{\kelvin} the $2\pi$-periodic current-phase relation becomes more sinusoidal as the critical temperature is approached. Furthermore between \qtyrange{2.8}{3.4}{\kelvin} we determined the critical current to decrease from \qtyrange{90}{45}{\micro\ampere}. Both these observations are in line with theoretical models. However, the data does not yet allow us to distinguish between diffusive and ballistic behaviour.

Whilst successful, further improvements are possible. Most notably the implementation of a flux-locked loop. This will allow biasing the dc-SQUID at its working point. This means that the measurements will become more sensitive. Furthermore, a flux-locked loop allows integrating the results over a longer period improving the accuracy. An attempt was made to implement the flux-locked loop using a current modulation line. However, that sample was unfortunately destroyed. Future projects can use its design.

\section{Outlook}
The potential of the method has been demonstrated. Future research should improve the method. The most promising improvement is the addition of a flux-locked loop dc-SQUID readout scheme that improves the sensitivity and decreases unwanted background interference.  In order to measure the current-phase relation of \ce{Sr2RuO4} minor adjustments to the method are needed. It is impractical to make both the junction's loop and the dc-SQUID from the same crystal. But we can use our electron-beam-induced deposition facilities to directly deposit superconducting \ce{WC} next to the junction under study. Furthermore, we currently lack the knowledge on how to pin a single chiral domain wall. As such it is difficult to study a single chiral domain wall. Alternatively, it might be possible to incorporate a ring of \ce{Sr2RuO4} with two chiral domain walls instead of a single junction. In that case no additional weak links must be formed, or their behaviour negligible compared to the \ce{Sr2RuO4} ring.