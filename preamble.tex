% !TEX root = thesis.tex

% Use siunitx to write out units and quantities, use special formatting for the units.
\usepackage{siunitx}
\sisetup{separate-uncertainty = true, multi-part-units = single, inter-unit-product = \ensuremath { { } \cdot { } }}

% Setup bibliography (instead of relying on the way lion-msc does it using natbib).
\usepackage[backend=biber, style=phys, biblabel=brackets]{biblatex}
\addbibresource{sources.bib}

% Use circuitikz for drawing circuits.
\usepackage{circuitikz}
% Define a symbol for Josephson junctions based on https://github.com/circuitikz/circuitikz/blob/ff0ccf13b25d9fd73fb8715fb3124018f0bce1e2/tex/pgfcircbipoles.tex#L6915-L6942.
\makeatletter
\pgfcircdeclarebipolescaled{misc}
{}
{\ctikzvalof{bipoles/barrier/height}}
{josephsonjunction}
{\ctikzvalof{bipoles/barrier/height}}
{\ctikzvalof{bipoles/barrier/width}}
{
    % this is set with normal wire linewidth
    \pgfpathmoveto{\pgfpoint{\pgf@circ@res@left}{0pt}}
    \pgfpathlineto{\pgfpoint{0*\pgf@circ@res@left}{0pt}}
    \pgfpathmoveto{\pgfpoint{\pgf@circ@res@right}{0pt}}
    \pgfpathlineto{\pgfpoint{0*\pgf@circ@res@right}{0pt}}
    \pgfusepath{draw}

    % do the cross part
    \pgf@circ@setlinewidth{bipoles}{\pgfstartlinewidth}

    \pgfpathmoveto{\pgfpoint{0.35*\pgf@circ@res@left}{0.35*\pgf@circ@res@up}}
    \pgfpathlineto{\pgfpoint{0.35*\pgf@circ@res@right}{0.35*\pgf@circ@res@down}}
    \pgfpathmoveto{\pgfpoint{0.35*\pgf@circ@res@left}{0.35*\pgf@circ@res@down}}
    \pgfpathlineto{\pgfpoint{0.35*\pgf@circ@res@right}{0.35*\pgf@circ@res@up}}

    \pgfusepath{draw}
}

\def\pgf@circ@josephsonjunction@path#1{\pgf@circ@bipole@path{josephsonjunction}{#1}}
\compattikzset{josephsonjunction/.style = {\circuitikzbasekey, /tikz/to path=\pgf@circ@josephsonjunction@path, label=#1}}
\makeatother

% Let footnotes use numbers instead of symbols.
\renewcommand{\thefootnote}{\arabic{footnote}}

% For double closed integrals among others.
\usepackage{esint}